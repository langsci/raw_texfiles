\documentclass[output=paper]{langscibook}
\ChapterDOI{10.5281/zenodo.7096306}
\author{Lorenzo Cigana\orcid{}\affiliation{University of Copenhagen} and James McElvenny\affiliation{University of Siegen}}
\title{The Copenhagen Circle}
\abstract{\noabstract}

\begin{document}
\maketitle


\paragraph*{JMc:} Today we’re talking about Louis Hjelmslev and the Copenhagen Linguistic Circle. To guide us through this topic, we’re joined by Lorenzo Cigana, who is a researcher in the Department of Nordic Studies and Linguistics at the University of Copenhagen and currently undertaking a major project on the history of the Copenhagen Linguistic Circle.

So, Lorenzo, can you tell us, what was the Copenhagen Linguistic Circle? When was it around, who were the main figures involved, and what sort of scholarship did they pursue?

\paragraph*{LC:} Dear James, I guess the best way to put it is to say that the Copenhagen Linguistic Circle was among the most important and active centres in twentieth{}-century linguistic structuralism and language sciences, along with, of course, the Circles of Paris, Geneva, Prague and, on the other side of the Atlantic, New York. It has also been referred to as the Copenhagen School, but the suitability of this label is somewhat debatable.

Not just the existence of the Copenhagen Linguistic Circle, but its very structure, was actually tied to the structure of those similar organisations. It was founded by Louis Hjelmslev and Viggo Brøndal on the 24\textsuperscript{th} of  September 1931: that’s almost exactly one month after the Second International Congress of Linguists, which was held between the 25\textsuperscript{th} and the 29\textsuperscript{th} of  August 1931, in Geneva – a city that, of course, had symbolic value since it was the city in which Ferdinand de Saussure was born. And actually, if you check the pictures that were taken during the congress, you can see a lovely, merry company of linguists all queuing to visit Ferdinand de Saussure’s mansion on the outskirts of the old part of the city. A very nice picture!

The Copenhagen Linguistic Circle also printed two series of proceedings. So we had \textit{Bulletins}, the \textit{Bulletin du Cercle linguistique de Copenhague}. The other was the \textit{Travaux} \textit{du Cercle linguistique de Copenhague}, which was a way to match what the Société linguistique de Paris and the Linguistic Circle of Prague were already doing at that time. 

What about the internal organization, you asked. The circle was divided into scientific committees, each of them devoted to the discussion of specific topics. There was a glossematic committee, for instance, which was formed by Hjelmslev and Hans Jørgen Uldall, and tasked with the development of the theory called glossematics. Then there was a phonematics committee devoted to phonological analysis, and a grammatical committee which was focused on general grammar and morphology, which had a lot of momentum. 

Now, you might have the impression here that it was Louis Hjelmslev who shaped the Circle, and you would be quite right: Louis Hjelmslev was definitely the leading figure. He was in many senses the engine behind the Circle’s activity, something that he was actually reproached for in the following years. 

At first, Hjelmslev got along very well with the other founder of the Circle, Viggo Brøndal. Hjelmslev was a comparative linguist and Indo-Europeanist, while Viggo Brøndal was a Romance philologist and philosopher, so they did complement each other. Moreover, they both were the descendants, so to speak, of two important academic traditions – something I’d really like to draw attention to, as in fact it is important to bear in mind that the Circle didn’t come out of the blue. The sprout had deep roots. 

Hjelmslev had been a student of Otto Jespersen and Holger Pedersen. Now, the first, Otto Jespersen, was an internationally renowned and influential linguist. He was said to be one of the greatest language scholars of the nineteenth and twentieth centuries, and his research was focused on grammar and the English language. He wrote a number of important works in syntax, like the theory of the three ranks. He also wrote wide-ranging contributions on the philosophy of linguistics, such ‘The Logic of Language’ (\textit{Sprogets logik}) in 1913 and the \textit{Philosophy of Grammar} in 1924. This is coincidentally what I like to call my domain of research, philosophy of grammar. Holger Pedersen, in turn, was a pure Indo-Europeanist and was in the same generation of Vilhelm Thomsen, Karl Verner – who is often mistakenly taken as German – and Hermann Möller, who corresponded with Ferdinand de Saussure and offered his version of the laryngeal theory. Although less interested in general linguistics, Pedersen worked on Albanian, Celtic, Tocharian, and Hittite, and postulated the existence of a Nostratic macro-family, linking the Indo-European family to others, like Finno-Ugric and Altaic. 

Let us take a look at Viggo Brøndal: what was his background? He was a pupil of Harald Høffding, one of the most important Danish philosophers, who worked extensively on the notion of analogy and analogical thinking, which was a topic of great importance in the epistemology of that time. Moreover, he read and commented on the \textit{Course in general linguistics}, the \textit{Cours de linguistique générale}, of Saussure, as soon as it was published, and was particularly receptive to all that came from Wilhelm von Humboldt, Gottfried Wilhelm von Leibniz, and from the phenomenological tradition of Franz Brentano and Edmund Husserl. 

However, if we look even further back, we see that the scholars I have just mentioned – Jespersen, Høffding and Pedersen – were in turn standing on the shoulders of other giants. And in fact they all had knowledge, in one way or another, of the work of their predecessors, notably Johan Madvig and Rasmus Rask, who both lived in the early nineteenth century. 

Let us just focus on Rask, who is rightly considered as the pioneer or the founding father of multiple linguistic disciplines, like Indo-European linguistics and Iranian philology, among others. Yet Rask didn’t just make factual contributions to language comparison, but also insightful theoretical and methodological advances. These advances can be found in his lectures on the philosophy of language and were especially dear to Louis Hjelmslev, who saw in them an anticipation of his own approach, and it’s no wonder why. Rask distinguished between two complementary stances in linguistics: the mechanical perspective, which provides a collection of facts, and a philosophical perspective, which tries to find the system or the link between all these facts. Rask described the difference between these two perspectives in the following way. The mechanical view deals with the process of making the materials required to paint a portrait: the colors, the canvas, etc. But only the philosophical perspective deals with the process of painting and the study of portraits themselves. This distinction is quite important. 

The reason why I decided to give this glimpse into the background of the Circle is that it is important to bear in mind that the influence of those figures lingered on: they were still present in the mind of the Circle’s members as a tradition they all came from. Rask in particular was dear to many linguists of the Copenhagen School. Jespersen wrote a biography of Rask, Hjelmslev collected his diaries and tried to make him a structuralist \textit{ante litteram}, and Diderichsen tried to reframe Hjelmslev’s own interpretations. It was on such fertile ground that the Copenhagen Linguistic Circle built its own scholarship. 

Let’s now return to the Copenhagen School itself. You asked, James, who its members were and what kind of contribution they made. Well, despite the claim that each member built their own tradition, there were indeed some shared projects, glossematics being one of them, possibly the most important. The work of building this new theory, glossematics, was carried out mostly, of course, by Louis Hjelmslev and his friend and colleague Hans Jørgen Uldall, who joined his project in 1934. 

Now, we have already spoken about Louis Hjelmslev, but very little is known about Hans Jørgen Uldall, who was a remarkable figure in his own right. He was first and foremost a very talented phonetician and collaborated with Daniel Jones, who was arguably the greatest British phonetician of the twentieth century. Uldall’s phonetic transcriptions were also known to be extremely precise, and yet he was also trained as a field anthropologist, another interesting aspect of his life. He travelled all across America, especially around California, carrying out research for Franz Boas. This gave him an incredible background that complemented Louis Hjelmslev’s own strong comparative and epistemological approach very well. 

Of Hjelmslev and Uldall’s collaboration during the 1930s, it was reported that they couldn’t say where one person’s ideas finished and the other’s started. I really believe this is such a brilliant example of collaboration between two scholars. But of course, there were also other members. If you take the proceedings of the Sixth International Congress of Linguists, for example, which was held in Paris in 1949, you can find a nice summary of the activity of the Copenhagen Circle since its very foundation. It’s a very informative summary, because it gives a clear idea about how the circle understood itself, or rather, how it wanted to present itself to the audience. Its motto sounded like: “We deal with general grammar and morphology over everything else”.

Hjelmslev worked on the internal structure of morphological categories: case, pronouns, articles, and so forth. Brøndal, too, in a way: he was trying to describe the structural nature of such systems and their variability as two complementary aspects connected to logical levels of semantic nature. But then there were also Paul Diderichsen, Knud Togeby, Jens Holt, and Hans Christian Sørensen, four fascinating figures.

If we look at Knud Togeby, he is probably the best known of these four, at least beyond the borders of Denmark. He wrote \textit{La structure immanente de la langue française} in 1951, a kind of compendium in which he described French in all its layers, from grammar to phonology, and was harshly criticized by Martinet. If you pay attention to the way Togeby used the very term ``immanent'', \textit{structure immanente de la langue française}, you’ll recognize the imprint of Hjelmslev: after all, it was Hjelmslev that stressed the need of an immanent description in the first place.

Paul Diderichsen was originally a pupil of Brøndal, and became a follower of Hjelmslev only later. He is mostly known for having developed the so-called fields theory, which is basically a valency model for syntax that works particularly well for Germanic languages and that played a big role in how Danish was analysed – and how it is still analysed today. He also developed what he called graphematics, which means a description of written language in conformity with the framework of glossematics, since it was based on graphemes conceived as formal units. However, Diderichsen became frustrated with this system and cast it aside.

Then we have Jens Holt and Hans Christian Sørensen, two figures that I personally feel very close to. They were both comparative linguists; they both struggled with Hjelmslev’s theory while trying to apply it to the morphological category of aspect, and they both ended up reworking some points of Hjelmslev’s theory in their own ways. For instance, Jens Holt tried to develop his own “rational semantics” – and here again we find this strange urge to qualify a theory as rational, something that tells us a lot about the general epistemological posture taken back then. He called his model “pleremics” – that is, an investigation of content entities in plain reference to glossematics, which was indeed its natural framework. 

Finally, we should mention Eli-Fischer Jørgensen, who cannot be left out of the picture. We can think of her as the Danish version of Lady Welby, the glue of the Circle. She corresponded with the most important figures of linguistics and phonetics at that time, and had a life-long correspondence with Roman Jakobson. She began her studies in syntax but found it too philosophical, so she decided to change, landing in phonology and phonetics instead. 

Now, despite the consonances between the members, and despite their ties to Hjelmslev, no school as such was established, no consistent tradition. They were all tapping into Louis Hjelmslev’s ideas, but they did that according to their own needs, as glossematics was the most consistent theory discussed back then. Yet because of (or perhaps thanks to) their different backgrounds, they could keep their own stances and views about linguistics and glossematics too. That must have been a source of some discomfort for Hjelmslev himself later on, to have his theory modified in various ways to suit different ends.

\paragraph*{JMc:} How did the Copenhagen Circle relate to other linguistic schools active at this time, in particular the Prague Linguistic Circle?

\paragraph*{LC:} In order to answer your second question, we will use the strategy that was developed by Homer in the Iliad. You know, portraying in poetic terms the clash between two whole armies is a hell of a job. Homer’s trick to describe the war between the two armies, the Greeks and the Trojans, was to collapse the armies into champions. So instead of having complicated, confused war scenes, we have battle scenes between two champions. This is what I would like to do here, because it was actually … well, I would not call it a war, but a conflict to be sure, in a way. That was really what happened back then.

The Prague Circle and the Copenhagen Circle had a relationship that could be called a friendly competition, or perhaps a competitive friendship. This doesn’t characterize the attitude of every single member of the two circles, but if we boil it down to the relationship between our main actors or “champions”, as I suggested – namely Roman Jakobson, Nikolai Trubetzkoy, Viggo Brøndal and Louis Hjelmslev – the label is pretty accurate. 

Let us take, for instance, what happened at the Second Congress of Phonetic Sciences in London in 1935. The backstory for it is that they had all met at the First International Congress of Linguists in the Hague, in 1928, so they knew each other. Then around 1932, Jakobson – or rather, the Prague Circle – would write to Hjelmslev asking, “Wouldn’t you be interested in providing a phonological description of modern Danish?” To which Hjelmslev answered, “Yeah, I can do that.” Then two years passed, Hjelmslev met Uldall, they discovered that both the content and the expression side of language – roughly, the signified and the signifier – could be described in parallel, so their approach changed somewhat, and in 1934, Hjelmslev wrote to Uldall saying, “You know what? We are not going to give what Jakobson asked us for. We will give our own talks, and put forward our own theory.” That theory was “phonematics”, the first sprout of glossematics, still centered on the expression plane. “Let us show them that we are a battalion, that \textit{wir marschieren}.” And I am quoting here. 

All this – mind you – happened at the very heart of the Second International Congress of Phonetic Sciences (London 1935), and at the very same session in which Trubetzkoy was speaking. It must have been quite disruptive, it must have looked like a sort of a declaration of war. And it was certainly understood as such, given that Trubetzkoy himself wrote to Jakobson wondering whether Hjelmslev was a friend to the phonological cause or rather an enemy.

As you know, in the past we have probably been a little bit too keen in considering this kind of competition on a personal level, as if between Trubetzkoy and Hjelmslev there was a personal animosity or rancour. I personally do not think so, or rather, if it was so, it was because scientific contrasts were felt in a very serious way, as back then it was of paramount importance to gauge one’s commitment to a common cause, namely the building of a new discipline -- structural linguistics. 

Indeed, starting from 1935, Hjelmslev and Uldall put a lot of effort into disseminating their view, stressing the fact that it was complementary and not identical to the one that the Prague Circle was developing. Hence, for instance, the stress that Hjelmslev put on the fact that research into phonology should focus on the possible pronunciations of linguistic elements and not be limited to the concrete or the factual (“realized”) pronunciation. 

Their view on language was becoming ever broader, and correspondingly, their frustration grew, too. In the same years, around 1935--1936, Hjelmslev was invited by Alan Ross to give lectures on his new science in Leeds in Great Britain, and after sensing the rather sceptical attitude in the audience, Hjelmslev wrote back to Uldall saying, “No one seems to understand what we are trying to do. They all want old traditional Neogrammarian phonetics. Oh, Uldall, I really want to go back to the continent.” 

A rich ground for confrontation between the different approaches was the theory of distinctive features, or \textit{mérismes}, as Benveniste would have called them. Prague was keen on analysing a phoneme into smaller features of a phonetic nature, while for Hjelmslev, this procedure was too hasty. If phonemes are of an abstract, formal nature, they should be analysed further into formal elements rather than straight into phonetic features. Such basic formal elements were called glossemes and represent the very goal of glossematics, which was accordingly called the science of glossemes, the basic invariants of a language. And then there was, of course, the aspect of markedness and binarism. This is the idea which Jakobson stubbornly maintained throughout his life, that distinctive features always occur in pairs defined by logical opposition, an idea that Hjelmslev never endorsed and actually actively fought. 

\begin{sloppypar}So overall, I think one could say that the relationship between these approaches – Prague versus Copenhagen and Paris – was twofold. Viewed from the outside, they gave the idea of being a uniform approach, a single front opposed to the one of traditional grammar or traditional linguistics of the past. They were indeed trying to build what Hjelmslev hoped for: a new classicism. However, viewed from the inside – if we increase, so to speak, the focus of our lens – we begin to notice massive differences that might appear a matter of detail, but that are quite significant in themselves. We have, at the same time, both unity and diversity, a key aspect that needs to be taken into account if you want to give an accurate picture of what happened in structural linguistics back then.\end{sloppypar}

\paragraph*{JMc:} What became of the Copenhagen Circle? Did it continue over several generations, or does it have a contained, closed history with a clear endpoint? What lasting effects did the scholarship of the Copenhagen Linguistic Circle have on linguistics?

\paragraph*{LC:} Well, the Copenhagen Linguistic Circle is still alive and kicking, actually. It has changed direction somewhat, and we may say that the structural approach or the structural generation has flowed into the new generation, which has a functional orientation. But this would be to oversimplify the state of affairs. I do believe the flow from one generation to the other wasn’t a matter of simple acknowledgement or rejection of approaches, methods, and ideas.

The modern approach, the functional one, understands itself as having some continuity with the broad framework of structuralism, even of glossematics. Yet in many respects, the modern framework is also a reaction against the purely formal stance that glossematic structuralism represented in Copenhagen, as well as with Hjelmslev’s somewhat oversized figure not just in scholarship and intellectual activity, but also in academic bureaucracy. It is, after all, a game of positions, of theoretical postures. Some of them can be seen as interpretation or explanations of previous positions. Others are original claims that are not necessarily linked with the previous theories. 

I should mention first of all that functionalism in linguistics can be seen as sort of a combination of insights coming from the structural framework, with the addition of some ideas that were developed later on, especially by Simon Dik in Amsterdam, and also with some ideas coming from cognitivism. 

At the very core of Danish functionalism, even if it may be trivial to mention it, is the attention given to how linguistic elements are used in given contexts. So in functionalism, it’s how they function, or what is functional in a given context, that matters. In this perspective, thus, “function” has little to do with the notion of “relation”, which was the key notion used by Hjelmslev. So here we have the first difference between “old structuralism” and functionalism: “function” in terms of relation was what linguistic structuralism and Hjelmslev’s approach wanted to use. In the new context of functionalism, “function” is rather interpreted as a role, and it is strongly tied to the idea of language as a communicative tool. 

This is very interesting, because such a definition may appear so obvious and trivial, right? Language as communication. But actually, this is not. After all, this was not how language was conceived in other structural contexts. For Hjelmslev, but to some extent to Uldall, and possibly to many other structuralists, too, the point was not communication, but formation or articulation. According to this hypothesis, language is a way to communicate only because it is in the first place a tool to articulate meanings in relation to expressions and vice versa. It’s also a way to represent subjectivity as such, a position that was explained so well by Oswald Ducrot, for instance, and which is echoed in Cassirer.

So, claiming that language serves to communicate can be seen as a position that was held in reaction to what a certain structural tradition was trying to do, and this entails some other theoretical consequences, like how expression and content – so signifier and signified – were interpreted and are interpreted nowadays, a cascade of differences and of conceptual claims that may seem a matter of details, once again, but which we need to be aware of. 

I cannot elaborate further on this point without entering into details, but let me just say that these differences are not just terminological. How function and form are defined in linguistics nowadays is not how they were defined back then, so we cannot assume these concepts are universal, or trivial, or commonsensical. Not at all. The task of a language scientist is also to draw attention to these epistemological stances, since they have a deep influence on his or her work, and this is, I think, the best way to understand our job, too, and a nice way to update what Saussure felt himself about the urge to show what linguists do. This is why I don’t particularly like the label “historiography of linguistics”. I prefer something like “comparative epistemology” because this is actually what we do. So I hope to have answered your question, James. Thank you once again.

\paragraph*{JMc:} Yeah, that was great. Thanks very much.

\nocite{travail}
\nocite{bulletins}
\nocite{bulletinsb}
\nocite{rapport}
\nocite{diderichsen1960a}
\nocite{hjelmslev1931a}
\nocite{holt1946a}
\nocite{jespersen1913a}
\nocite{jespersen1918a}
\nocite{jespersen1924a}
\nocite{saussure1922a}
\nocite{ducrot1968a}
\nocite{cigana2022a}
\nocite{thomas2019a}
\nocite{rasmussen1987a}


\sloppy
\PrintPrimarySources{}
\PrintSecondarySources{}
\end{document}
