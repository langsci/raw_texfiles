\documentclass[output=paper]{langscibook}
\ChapterDOI{10.5281/zenodo.7096290}
\author{Clara Stockigt\orcid{}\affiliation{University of Adelaide} and James McElvenny\affiliation{University of Siegen}}
\title{Missionary grammars in Australia}
\abstract{\noabstract}

\begin{document}
\maketitle 
%\shorttitlerunninghead{}%%use this for an abridged title in the page headers


\paragraph*{JMc:}  As we saw in the previous interview with Jürgen Trabant, a key issue animating much European language scholarship in the nineteenth century was capturing and accounting for the diverse forms found in the world’s languages. In this interview, we take a peek at some of the sources from which European scholars derived their knowledge of non-Indo-European languages. To introduce us to this topic, we’re joined by Clara Stockigt from the University of Adelaide, who’s a specialist in the history of language documentation in Australia.

Before we get started, we have to note that our discussion today focuses rather narrowly on the technical details of the grammatical description of Australian languages and the intellectual networks within which the authors of early grammars operated. We therefore miss the broader – and in many ways much more important – story of settler colonialism in Australia and the world more generally and how this was intertwined with scientific research. This is a topic that we address elsewhere in the podcast series. 

So Clara, to put us in the picture, could you tell us which languages were the first to be described in detail in Australia?


\paragraph*{CS:}  The languages that were described initially were those spoken around the colonial capitals. You had, for example, missionary Lancelot Threlkeld writing a grammar of the language spoken near Newcastle, which is reasonably close to Sydney. The languages spoken close to Adelaide on the coast were described by Lutheran missionaries in the 1840s. Charles Symmons, who was the Protector of Aborigines in Western Australia, described the language spoken close to Perth, so in the very early era you have a pattern where the languages spoken close to the colonial capitals were described.

But those first missions didn’t last very long, and the languages, the people, dispersed quite quickly. Subsequently the Lutherans established missions in South Australia among the Diyari and the Arrernte, and at those missions, there was this intergenerational tradition of linguistic description where Aboriginal people and the missionaries worked alongside each other in what was an economic unit.


\paragraph*{JMc:}  So did the languages that were described in these centres all belong to a single family? How many language families are there in Australia?


\paragraph*{CS:}  We have the Pama-Nyungan family, which covers most of the Australian continent, and so all of the languages that we’re talking about, having been grammatically described in the nineteenth century, belonged to this Pama-Nyungan family of languages, which is a higher-order overarching umbrella under which different languages belong. About 250 Pama-Nyungan languages are said to have been spoken in Australia at the time of colonization.


\paragraph*{JMc:}  And so what was the motivation of the missionaries to study these languages in Australia?


\paragraph*{CS:}  As everywhere around the world Australian missionaries described languages in order to preach in the local language and convert people to Christianity. They believed that if people could hear the Gospel in their mother tongue, they would necessarily be converted to Christianity. The Lutheran missionaries in Central Australia prepared and had printed vernacular literacy materials in Diyari and in Arrernte. As Aboriginal people became literate in their own language, they were able to use hymn books and books of prayer in the schools and in church services. It’s also clear that many missionaries wanted to describe the complexity of the languages in order to show that the people speaking the language were intelligent. From their point of view, you couldn’t covert a people to Christianity unless they were intelligent, and by laying out the complexity of the language, they were, in a way, demonstrating the promise of their mission work. Missionary grammarians in Australia also realized that their work was going to preserve the languages that they were describing. You know, there was a perception that Australian languages and Aboriginal people were disappearing very quickly in the aftermath of European settlement. Lancelot Threlkeld, who was Australia’s earliest grammarian, who wrote a first complete grammar in \citeyear{threlkeld1834a}, he perceived that he had actually outlived the last speakers of the language he described in the 1820s and 1830s.


\paragraph*{JMc:}  “Disappearing” sounds a bit passive and euphemistic. How did the missionaries, people like Threlkeld, describe the situation themselves?


\paragraph*{CS:}  They used the word ”disappearing”.


\paragraph*{JMc:}  OK.


\paragraph*{CS:}  Yes. “Vanishing”.


\paragraph*{JMc:}  It seems a bit euphemistic, doesn’t it? Do you think that that is how someone like Threlkeld genuinely felt about it, or do you think he was more interested in not offending his European readership?


\paragraph*{CS:}  I think he perceived Australian populations were being decimated and dying out. And, of course, the nineteenth-century records collected by the missionaries are increasingly important today in reconstructing Australia’s pre-invasion linguistic ecology because of the high rate of extinction of Australian Indigenous languages since colonization and also – or because – a large proportion of Australian Indigenous populations today now speak English, or Aboriginal English, or creoles as their first language.


\paragraph*{JMc:}  And were missionaries just writing for other missionaries? Did they intend their grammars to be read only by other members of their missionary society?


\paragraph*{CS:}  Some missionaries did, especially the ones who just wrote their grammars as German manuscripts, but those who knew that the work was going to be published often had a little section in the introduction saying that they hoped the work would be interesting, would be of value, to the interested philologist, so there was a definite sense that the missionaries were aware that their linguistic knowledge was valuable to readers outside the field. They were courting a relationship with European philologists.


\paragraph*{JMc:}  What kind of experience did these missionaries have in grammar or in learning foreign languages which might have given them exposure to grammatical description of other languages?


\paragraph*{CS:}  The missionaries who wrote grammars of Australian languages had received different degrees of linguistic training in preparation for mission work. Those trained at the Jänicke-Rückert schools, or at Neuendettelsau in Germany, or at the Basel Mission institute in Switzerland are said to have received a rigorous linguistic training with exposure to nineteenth-century grammars of Latin, Greek, and Hebrew.


\paragraph*{JMc:}  Hebrew as well as Latin and Greek. Hebrew’s a non-Indo-European language, of course, so structurally quite different from Latin and Greek.


\paragraph*{CS:}  Yes.


\paragraph*{JMc:}  So they would have been familiar with languages that have a structure not the same as their own native language?


\paragraph*{CS:}  That’s right, but only some of the grammarians had looked at Hebrew.


\paragraph*{JMc:}  So it was a minority.


\paragraph*{CS:}  Yes, I think so. On the other hand, other missionary grammarians, such as the Congregationalist George Taplin and Missionary Threlkeld of the London Mission Society, had received little formal training, and grammars written by the Protectors of Aborigines were founded in a well-rounded education and a knowledge of schoolboy Latin. So an assumption that a rigorously trained grammarian who had studied a greater number of classical languages would make better analyses of Australian linguistic structures than grammarians with lesser training is actually not upheld when we compare the quality of the description with what is known about a grammarian’s training.


\paragraph*{JMc:}  And why do you think that might be?


\paragraph*{CS:}  Well, it’s a bit odd. It might be because the sample size in Australia is reasonably small. But the fact that it appears to have little bearing on the quality of a grammatical description is probably because the strength of an individual description has more to do with the length of time and the type of exposure that a grammarian had with the language and probably also just to do with his inherent intelligence and aptitude.


\paragraph*{JMc:}   OK. Although you’d think that you’d need to have some sort of grammatical framework that you could use as a scaffolding to even begin making your description.


\paragraph*{CS:}  I think just a basic knowledge of Latin, a very basic knowledge of Latin, was enough to get you there.


\paragraph*{JMc:}  Sort of bootstrapping.


\paragraph*{CS:}  Yeah, and some missionary grammarians in Australia also had previous exposure to the structure of other exotic languages or non-European languages like Hebrew. Early Lutherans trained at the Jänicke-Rückert school were probably also aware of descriptions of Tamil because of missionary work in India.


\paragraph*{JMc:}  OK, and Tamil is of course a Dravidian language, so another very different language, different kind of structure.


\paragraph*{CS:}  Yes.


\paragraph*{JMc:}  And I guess we should probably point out that we’re using this term “exotic” a bit, but that’s a category that the missionaries would have used themselves to describe these unfamiliar languages.


\paragraph*{CS:}  Yeah. The missionaries in Australia tended to use the word “peculiar” rather than “exotic”.

Missionary Threlkeld had worked in Polynesia, so he had some knowledge of the description of Polynesian languages from the mission field, and the Basel-trained missionary Handt had worked in Sierra Leone, so the way in which these experiences may have influenced the early description of Australian languages requires a lot more research, I think. Nobody’s really looked into that too much.


\paragraph*{JMc:}  OK, so this is an unexplored area of missionary linguistics.


\paragraph*{CS:}  I think so, and especially the connection between the early description of languages in Polynesia and in Australia, because there were strong connections with missionaries from the London Missionary Society.


\paragraph*{JMc:}  So how did these people writing grammars and word lists of Australian languages approach them, would you say?


\paragraph*{CS:}  So as was the case with the description of other exotic languages–


\paragraph*{JMc:}  Or peculiar languages, as the case may be.


\paragraph*{CS:}  Or peculiar languages, yeah. Eurocentric linguistic understanding skewed the nineteenth-century representations of Australian linguistic structures. When we look at the attempts to represent the sound systems of Australian languages, we see that nineteenth-century linguists were presented with really significant challenges. Consonants in Australian languages typically show few articulatory manners and an absence of fricatives and affricates, but extensive places, extensive sets of place of articulation contrasts, some having two series of palatal and two series of apical phonemes for stops, nasals, and laterals. And it was difficult for European ears to distinguish these sounds, let alone to decide on a standardized way to represent them. So before the middle decades of the twentieth century, the orthographic treatments of Australian phonologies grossly underrepresented phonemic articulation contrasts, and all sources just fell well short of the mark. And I think it’s this type of failure that has contributed to the outright dismissal of the early descriptions of Australian languages by some later twentieth-century researchers.


\paragraph*{JMc:}  Even though the orthographies that early grammarians designed for these languages might have been insufficient, do you think that they still understood the principles of how the phonology of those languages worked, or do you think it just completely went past them?


\paragraph*{CS:}  I think they understood that there was a greater level of complexity or there were things going on that they weren’t grappling with, and they were frustrated with the inconsistencies in the system. So in the 1930s, when people started to look back at the earlier nineteenth-century sources, they could see that there was a great inconsistency, and even though early grammarians often aimed towards a uniform orthography and stated that they were following the conventions established by the Royal Geographical Society, they really just were not getting anywhere near an adequate method of representing the languages, and I don’t think they understood what was going on, necessarily.


\paragraph*{JMc:}  So those are the phonological features of the languages. What about in terms of the grammar?


\paragraph*{CS:}  So missionary grammarians, by and large, opted to scaffold their developing understanding of Australian languages within the traditional European descriptive framework that they were familiar with from their study of classical languages. And as a consequence, missionary grammarians in Australia tended to attempt to describe features that were just not present in Australian languages, including indefinite and definite articles, the comparative marking of adjectives, passive constructions, and relative clauses signalled by relative pronouns.


\paragraph*{JMc:}  So do you think that the missionaries were actually implying that those categories were universals and were projecting them into the languages they were describing, or do you think it was intended more as a heuristic, as a learner’s guide, like they were writing for an audience that might want to express the equivalent of a passive construction in their own language in this Australian language, and so the grammar is saying, “If you had this kind of structure in a European language, you would then use this”?


\paragraph*{CS:}  That’s exactly what they were doing. So on the other hand, grammarians who became reasonably familiar with an Australian language encountered an array of foreign – or, as they called them, “peculiar” – morphosyntactic features that were not originally accommodated within the descriptive model, and they invented new terminology and descriptive solutions in order to describe these peculiarities. And so they were able to account for Australian features like the marking and function of ergative case, the large morphological case systems of Australian languages, sensitivity of case marking to animacy, systems of bound pronouns, inalienably possessed noun phrases, inclusive and exclusive pronominal distinction and the morphological marking of clause subordination. All of these features were described in the earliest era in Australia. And some early Australian grammarians were certainly aware that the traditional grammatical framework was inadequate to properly describe Australian structures.  In \citeyear{schuermann1844a}, for instance, Lutheran missionary Schürmann advised that grammarians of Australian languages should “divest their mind as much as possible of preconceived ideas, particularly of those grammatical forms which they may have acquired by the study of ancient or modern languages.”


\paragraph*{JMc:}  Wow, so that’s a direct quote from Schürmann..


\paragraph*{CS:}  Yeah, and that’s 1844, so a reasonably early perception, I think. But nevertheless, these missionary grammarians appear unwilling to wean themselves off the framework designed to accommodate classical European languages, even when they knew that the framework was less than adequate. And this is probably because the traditional framework conveyed peculiar structures in a way that was most accessible and easy for the reader to understand, as you were suggesting earlier.

So these grammarians who perceived that the framework was inadequate still managed to describe foreign linguistic structures by subverting the traditional framework. Section or chapter headings that are built into the traditional framework that accommodated European structures that are not found in Australian languages sometimes provided a useful, vacant slot into which these newly encountered peculiarities could be inserted into the description. So an example here, just to get a bit technical, is the description of the case suffix marking allative function, which tended to be underrepresented in the early grammars because allative function is not marked by the morphological case systems of European languages.


\paragraph*{JMc:}  OK, so allative is like going to a place.


\paragraph*{CS:} \begin{sloppypar}Yeah, that’s right. But there was a group of grammarians in Australia who exemplified allative case marking under the heading “correlative pronouns”, which is an unnecessary descriptive category when it’s applied to Australian languages. So under this heading, “correlative pronouns”, we see noun phrases translated as “from X in ablative case” and “to X in allative case”, but there’s no suggestion that the morphology that was described under this heading, “correlative pronouns”, was in any way pronominal. And similarly, while grammarians happily accommodated the large morphological case systems of Australian languages within an early chapter of the grammar headed “Nouns” by presenting case paradigms of up to 11 cases, these same grammarians presented the same morphology again in a later section of the grammar under a final chapter headed “Prepositions”. A contradiction in describing suffixing affixes under the word class heading “preposition” doesn’t appear to have perturbed the grammarian. Newly encountered Australian features tended to be accounted for in sections of the grammar that conventionally conveyed a Europeanism that was perceived as functionally equivalent to the Australian feature – in this instance, nouns marked for cases that needed to be translated by an English prepositional phrase being described as a preposition. And other instances of this type of substitution process in which the traditional framework was colonized by foreign structures include the construal of ergative morphology as marking passive constructions, the depiction of bound or enclitic pronouns as verbal inflections for number and person, and the description of deictic forms as third-person neuter pronouns.\end{sloppypar}


\paragraph*{JMc:}  And how widespread is this representation of ergative morphology as a kind of passive construction? How many different scholars do that?


\paragraph*{CS:}  Quite a few. Even though they made a good account of ergative morphology when they’re talking about case, either conceiving of the ergative case as a second nominative or a type of ablative case, but often when it comes to the description of the passive or the part of the grammar where you’re expected to describe passive functions, there will be ergative morphology given there as well.


\paragraph*{JMc:}  What connections were there between the people in the field writing descriptions of Australian languages and linguistic scholars in Europe and other parts of the world? Were there active networks of communication between the field and the metropolitan centres, and did these language descriptions feed back into the development of linguistic theory?


\paragraph*{CS:}  Generally not. I think connections between missionary grammarians in Australia and Europe were quite limited. Australian linguistic material tends to be absent from nineteenth-century comparative philological literature, and European philologists commonly mention a scarcity, or they’re frustrated about a scarcity, of Australian linguistic data. There’s no reference to Australian languages in \citet{pott1884a}, nor in Friedrich Max \citet{mueller1864a}, although there is a reasonably comprehensive discussion of Australian material in the final volume of Prichard’s \textit{Physical History of Mankind}, Volume 5, \citeyear{prichard1847a}.


\paragraph*{JMc:}  OK, and that’s quite early, 1847.


\paragraph*{CS:}  Yeah.


\paragraph*{JMc:}  So what material did he have to work with?


\paragraph*{CS:} \begin{sloppypar}He had the grammars that had been published at that stage, which were from South Australia and New South Wales, so there was a relatively small amount of material, but he had looked at what was available at that time, which makes it odd that these later compilations of linguistic material from around the world don’t reference the Australian material.\end{sloppypar}


\paragraph*{JMc:}  So were these Australian grammars published, or were they manuscripts?


\paragraph*{CS:}  The ones that he referred to were published grammars. There was a wave of publications of materials in the 1830s and 1840s, and then not a lot of published material until towards the end of that century.


\paragraph*{JMc:}  And were they published in Australia or in Europe?


\paragraph*{CS:}  They were published in Australia, generally by colonial authorities.


\paragraph*{JMc:} \begin{sloppypar}The missionary grammarians themselves, was there contact between them, out in the field, or did they work alone mostly?\end{sloppypar}


\paragraph*{CS:}  They pretty much worked alone, not only from developments in Europe, but also in intellectual isolation from each other. Many early grammarians appear to have written their grammars without any knowledge of previous descriptions of Australian languages. Where schools of Australian linguistic thought did develop or where ideas about the best way to describe Australian languages were handed down to future grammarians, you see a regional pattern of ideas about the best way to describe Australian languages developing. And this occurred within different Christian denominations which were ethnically and linguistically distinct and which had their headquarters in different pre-Federation Australian colonial capitals.


\paragraph*{JMc:}  And what were the main regions?


\paragraph*{CS:}  So we had a school of description developing in New South Wales – the earliest grammars of Australian languages were written there – and then the school of description developing in South Australia mostly with the Lutheran missionaries, and then a later descriptive school developing in Queensland. So this decentralized nature of the development of linguistics in Australia hampered improvements to the understandings and descriptive practices in the country, but also to the movement of ideas in and out of the country. But just as some of the early grammarians had flirted with the interested philologist in the introductory passages, the linguistic knowledge of some grammarians was actively sought by some scholars outside the country. The pathways through which ideas about Australian languages were exchanged remain largely untraced, although there has been focused interest on the enduring communication between the Lutheran missionary Carl Strehlow, who worked with the Arrernte populations in Central Australia, and his German editor, Moritz von Leonhardi. And this relationship kept Strehlow abreast of early twentieth-century European ethnological thinking, although linguistics played a relatively small part in their intellectual exchange.


\paragraph*{JMc:}  When was Carl Strehlow working?


\paragraph*{CS:}  He was working with the Arrernte from 1894 until his death in 1922.


\paragraph*{JMc:}  OK, so this is right at the end of the nineteenth century.


\paragraph*{CS:}  Yeah, in the beginning of the twentieth century. But other interactions deserve more scholarly attention, including the interaction between Wilhelm Bleek, who was the German philologist based in South Africa and who, in \citeyear{bleek1858a}, catalogued Sir George Grey’s philological library, and missionary George Taplin, who was in South Australia, and himself collated comparative lexical material of South Australian languages. There’s an interesting exchange between these two people that I think would be worthy of further investigation.


\paragraph*{JMc:}  And of course, George Grey was a sort of wandering colonial official, wasn’t he, so he had previously been in South Australia before he went to South Africa.


\paragraph*{CS:}  Yeah. And in New Zealand as well. It was George Grey who supported the work of the Lutheran missionaries in South Australia in those very early years.


\paragraph*{CS:}  Other lesser-known exchanges between Australia and Europe are Hans Conan von der Gabelentz’s and Friedrich Müller’s reframing of Australian ergative structures as passive, which were both based on a grammar written by the Lutheran missionary Meyer in \citeyear{meyer1843a}. These were given in Gabelentz’s \textit{Über das Passivum} in \citeyear{gabelentz1861a} and Müller’s \textit{Grundriß der Sprachwissenschaft} in \citeyear{mueller1882a}.


\paragraph*{JMc:}  Do you think that that is a fair interpretation of Hans Conan von der Gabelentz? Because I guess his \textit{Über das Passivum} is really an early typological work, and he’s talking essentially about a functional category and looking at how it is realized in what we would now call the different voice systems of languages around the world. So he doesn’t just have Australian languages in there, for example. He also has Tagalog and numerous other diverse languages of the world. So do you think it’s fair to say that he was reframing the ergative as a passive, or rather, he just used “passive” as a sort of typological term to describe this kind of voice structure in the languages of the world?


\paragraph*{CS:}  No, I actually do think he reframed the structure and he reinterpreted the material that Meyer had presented in a way that Meyer had not intended, and I don’t think it’s a fair representation of the structure in an Australian language in order to support his theory.


\paragraph*{JMc:}  OK. And how representative was the situation in Australia in comparison with other places that were subject to European colonialism in this period? So especially settler colonialism. The comparison, I guess, would be with South and especially North America and South Africa, and parts of the Pacific, like New Zealand.


\paragraph*{CS:}  I think there’s a lot more work to be done in comparing what occurred in these different areas, but I think the situation in Australia does differ quite a lot. No nineteenth-century descriptive linguist in Australia managed to truly bridge the divide between being a missionary or field-based linguist and academia, so Australia has no scholars equivalent to Franz Boas in North America or Wilhelm Bleek in South Africa. Channels of communication between Europe and Australia were much less developed than between Europe and other colonies.


\paragraph*{JMc:}  Why is that? Just because it’s so far away?


\paragraph*{CS:}  Possibly because it’s so far away, and I think because linguistics as a discipline wasn’t centralized, and we just didn’t happen to have a Wilhelm Bleek here or a Franz Boas. There wasn’t a centralized development of ideas in the country and we have this haphazard regional, ad hoc development of ideas in different mission fields that weren’t really feeding into a central body that was communicating with Europe. And I think also the exchange of ideas was largely unidirectional flowing out of the country rather than into the country, so for instance, the presentation of sound systems of Australian languages in systematic diagrams that set out consonant inventories in tables, mapping place of articulation against manner of articulation, occur reasonably regularly and early in European publications commencing with Lepsius in \citeyear{lepsius1855a}, who presented the phonology of Kaurna in such a sort of gridded system. Also, Friedrich Müller in \citeyear{mueller1867a} did a similar thing, and later European works right up until the 1930s were representing Australian phonologies in this way, but such presentations appear not to have been read by any grammarian in Australia, or if they were read, they weren’t understood and they weren’t assimilated into Australian practice. The earliest reasonable graphic representation of consonants made by an Australian researcher didn’t occur until Arthur Capell's \citeyear{Capell1956} work entitled \textit{A New Approach to Australian Languages}. I think the slow speed with which phonological science entered Australia is illustrative of what could almost be seen as a linguistic vacuum in the country before about 1930.


\paragraph*{JMc:}  Capell had a university position, didn’t he? So I guess it’s this academic influence that you’re pointing to.


\paragraph*{CS:}  He did, yes. The university connection commenced very early in the 1930s: you had the first dissertations on Australian Aboriginal languages being written within the Department of Classics at the University of Adelaide and within the Department of Anthropology at the University of Sydney, but it wasn’t until a few decades later that you had linguistic researchers within academic institutions working on Australian languages.


\paragraph*{JMc:}  OK. Up until now, I thought that Australian linguistics burst forth fully formed from the brow of Bob Dixon.


\paragraph*{CS:}  Some would have us believe that.


\paragraph*{JMc:}  OK, so thank you very much for telling us all about the situation in Australia with missionary linguistics.


\paragraph*{CS:}  Absolute pleasure, James. Thanks for inviting me.


\nocite{bleek1858a}
\nocite{flierl1880a}
\nocite{gabelentz1861a}
\nocite{grey1839a}
\nocite{grey1841a}
\nocite{grey1845a}
\nocite{kempe1891a}
\nocite{leonhardi1901a}
\nocite{lepsius1855a}
\nocite{lepsius1863a}
\nocite{meyer1843a}
\nocite{mueller1867a}
\nocite{mueller1882a}
\nocite{mueller1854a}
\nocite{pott1884a}
\nocite{prichard1847a}
\nocite{society1885a}
\nocite{schuermann1844a}
\nocite{strehlow1907a}
\nocite{symmons1841a}
\nocite{threlkeld1834a}
\nocite{taplin1879a}
\nocite{dixon2010a}
\nocite{simpson2019a}
\nocite{stockigt2015a}
\nocite{stockigt2017a}
\nocite{stockigt2022a}


\sloppy
\PrintPrimarySources{}
\PrintSecondarySources{}

\end{document}
