\documentclass[output=paper]{langscibook}
\ChapterDOI{10.5281/zenodo.7096300}
\author{H. Walter Schmitz\orcid{}\affiliation{University College London} and James McElvenny\affiliation{University of Siegen}}
\title{Victoria Lady Welby}
\abstract{\noabstract}


\begin{document}
\maketitle 


\paragraph*{JMc:} Today we’re joined by Walter Schmitz, Emeritus Professor of Communication Science at the University of Duisburg-Essen. He’s going to talk to us about Victoria Lady Welby, an important and yet perhaps still somewhat underappreciated figure in the history of semiotics. To get us started, could you please tell us about Victoria Welby’s work and the background to it? What were her major contributions to semiotic thought?

\paragraph*{WS:} Lady Welby was born in 1837 and died in 1912. She didn’t have any formal education. Instead, she had private lessons, and travelled a lot, especially with her mother – to the United States, northern Africa, and Syria. Later she became Maid of Honour to Queen Victoria, in the 1860s, for two years.

After her marriage, she turned to the study of hermeneutics and problems of interpretation. Her starting point was trying to find arguments against the fundamentalist interpretation of the Athanasian creed and other theological and biblical texts. Afterwards, she studied philosophy and natural sciences. Everywhere she found puzzling terminology, but she found nobody cared for meaning, for meaning of terminology, for meaning of ordinary words, and so she embarked on a critique of terminology and ordinary language, and she found that the language that many scholars were using was not in agreement with the results of the sciences. Through her work, she introduced the study of meaning as a topic into British philosophy, psychology, and even linguistics. In 1896, she published her first article in the philosophical journal \textit{Mind} on sense, meaning and interpretation.

Lady Welby’s contribution to semiotics was quite different from others at the time. Unlike C. S. Peirce, she did not proceed from definitions of signs and their features in order to investigate the relations into which signs with certain features can enter. Rather, she started from the other side, so to speak, and concentrated on the problem of meaning; that is, on questions of interpretation and the communicative use of signs. This is the essential merit of her contribution.

\paragraph*{JMc:} You say that Welby started with Bible interpretation, or hermeneutics. This task of Bible interpretation also played a major role in German intellectual life in the nineteenth century, and in the study of meaning in the nineteenth century in Germany. Why do you think Lady Welby turned to hermeneutics? Was it because she was particularly religious, or did it have more to do with the fact that this was one of the few intellectual outlets that was available to her because she couldn’t get a formal education?

\paragraph*{WS:} I think the roots were in practical problems. She was a mother and had to educate her children, and as a very independent person – mentally, financially, and in every respect independent – she asked herself: How can I educate my children in religious questions? She couldn’t find a suitable answer in the ecclesiastical books available to her, so she started to study biblical texts and ask herself: How should I understand these texts? Do I need a new interpretation, a contemporary interpretation?

Lady Welby couldn’t read German or French, but only English, so she concentrated on what she could find written in English, especially in ecclesiastical books. But at that time it was not permitted for women to have such interests. That was a serious problem for her. Indeed, when she published her first book in 1881 – with a second edition in 1883 – many of her relatives claimed to find it offensive. She had to defend herself against the aggression of church people and even those in her own circle of acquaintances.

\paragraph*{JMc:} Let’s briefly expand on this problem of her not being able to get a formal education. As you mentioned, she was born in 1837, and as far as I’m aware, the first English university that allowed women to attend classes was the University of London in 1868, so she would have already been an adult by that stage. Even then, the female students at the University of London weren’t allowed to take degrees, so they were still second-class citizens in the university world.

Lady Welby is in fact the first woman to have appeared in our podcast series so far, right at the end of the nineteenth century. It’s nice that we’ve finally been able to find a woman scholar who was able to fight against all of the restrictions that were put on her gender in this period, but on the other hand, it’s still a story of privilege, isn’t it? She was a member of the high aristocracy, she was financially independent, she was the Maid of Honour to Queen Victoria – in fact, she was named after Queen Victoria, who was her godmother.

\paragraph*{WS:} Yes, she knew how to use her privilege in order to get on with her studies. She invited other scholars to come to her manor in Lincolnshire and held discussions with them. Her guests included such figures as the psychologist G. F. Stout, or philosophers like Ferdinand Canning Scott Schiller, or mathematicians and philosophers like Bertrand Russell. She sent them her essays and discussed the essays with them. Through her correspondence, she got feedback on her writings and eventually arrived at publishable versions. So she collaborated with others and used her privilege in order to overcome lack of knowledge, lack of experience. Even in writing scientific texts, she had assistants who helped her to write the books.

\paragraph*{JMc:} On a purely political level, I believe she could not be considered a feminist. She was an opponent of the suffragettes, for example – she didn’t support women’s suffrage.

\paragraph*{WS:} In political questions, I think that’s not the only argument to call her very conservative. Especially in the discussions with Frederik van Eeden – a Dutch poet and psychiatrist she corresponded with and knew very well – she was a vocal supporter of the British Empire in the Boer War against the Dutch colonists in South Africa. In those questions, she was a conservative, a member of her class.

But in her scholarship, she was very progressive. She brought the topic of meaning into British philosophy. Ogden and Richards’ later work on meaning, and even Bertrand Russell’s interest in the topic all started with Lady Welby’s work. On the folder in which he kept his correspondence with Lady Welby, Russell wrote: ``From Lady Welby, who turned my attention to linguistic questions.'' I think at that time, in \citeyear{russell1905a}, for example, when Russell wrote about ``On denoting'', he didn’t understand her very well. She was far in advance of him, and she argued against Russell in the same way as P. F. Strawson did many years later. So in this respect she was progressive, but in political respects, she was a conservative, yes.

\paragraph*{JMc:} But didn’t Russell write in some of his correspondence that he refused the invitation to go to Welby’s house because he would have had to be honest with her, and he thinks that it’s a shame that everyone is encouraging her?

\paragraph*{WS:} I think at that time Russell didn’t think very highly of Lady Welby. But later on, he recognized that she showed the right way. Even in \citeyear{schiller1920a}, when there was a symposium organized on the meaning of meaning, Russell participated in that symposium, but he wrote a paper that took a very behaviouristic approach to meaning, while Ferdinand Canning Scott Schiller, the philosopher, was a defender of Lady Welby’s approach. Russell needed more time to learn than others.

\paragraph*{JMc:} Do you think Russell ever did learn? He was still fighting ordinary language philosophers in the 1950s and ’60s.

\paragraph*{WS:} I’d say even when he wrote, ``She called attention to linguistic questions'', it was a kind of misunderstanding. Lady Welby wasn’t interested in linguistic questions; she was interested in ways of interpreting signs, and that’s a more general question than a linguistic one. For her, the word outside of use has a verbal meaning, but it doesn’t have sense; it has no meaning. Her interest was in the meaning of signs and not in words or in a systematic description of language.

\paragraph*{JMc:} OK, so this is a keyword that brings us to the heart of her doctrines, namely this trichotomy that she set up between sense, meaning and significance. Could you explain what that means?

\paragraph*{WS:} Let’s begin with sense. Lady Welby sought a very broad concept of sense, and it was a kind of organismic concept. “Sense”, in its broadest sense, is for Lady Welby the suitable term for that which constitutes the value of experience in this life on this planet. The value of the experience which is had consists of the sort of organic reaction (touch, smell, taste, hearing or sight) to a stimulus which is at the same time an interpretation or translation of the stimulus influenced by the physiology of human senses (so Fritz Mauthner speaks quite correctly of our \textit{Zufallssinne} ``chance senses'').

\largerpage
But words or utterances have sense or acquire sense through the interpretation of the hearer or reader. The first reaction is the sense of the utterance, while meaning is the intention which is combined with the utterance, so the interpreter has to find out the difference between sense and meaning. 

The sense is what we get almost immediately, but in order to get to the meaning of an utterance, we have to draw conclusions. For example, somebody might ask me, ``Where is Peter?'' and I answer, ``Yesterday, I saw a yellow Porsche in front of house number seven.'' The sense of my utterance might be, ``Yesterday, there was such and such an event which I experienced'', but the meaning of my utterance is quite different: ``Peter was in that house.'' 

Now we come to significance. The significance is a consequence of or an implication of the utterance or even an event, even experience. So it might be that there is a woman who lives in house number seven, and the person who asked me where Peter is was perhaps Peter’s wife. She may be afraid that Peter went to another woman. So the consequence or the implication of my utterance might be of very great importance to her. Significance is the third meaning events or utterances or words may have.

\paragraph*{JMc:} In 1909, C. S. Peirce wrote to Welby in a letter that his own tripartition of immediate interpretant, dynamical interpretant, and final interpretant, ``nearly coincides with your sense, meaning and significance''. You mentioned at the beginning of the interview the two different directions that Peirce and Welby approach the problems of semiotics from, but Peirce seems to have thought himself that his own views and Welby’s were very close.

\largerpage
\paragraph*{WS:} Peirce did indeed write that to Lady Welby, and I think there are some similarities between their views, but they aren’t identical. Immediate interpretant, for instance, has some similarities with sense but it’s not quite identical, and the dynamical interpretant is even less similar to Lady Welby’s meaning. Perhaps final interpretant and significance might be more similar than even Peirce thought, but Peirce’s and Welby’s respective approaches were so different that we couldn’t expect that their terms should be used to name the same concepts.

The differences shouldn’t be overlooked. I think for Peirce, it was important to have somebody to discuss semiotic questions with, somebody to explain his ideas on semiotics to, so that at times he overlooked the differences. She did much the same thing. Peirce wasn’t interested in communication and interpretation. He was interested in the development of a general semiotic system, and he left it as an empirical question to find out where and how these classes of signs were realized in real events. That’s a very different approach, and it has to get to very different aims.

\paragraph*{JMc:} So could you tell us then a bit about what happened to Lady Welby’s legacy, to her work in later generations? There was the Dutch Significs movement, as it’s known, a group of scholars in the Netherlands who took Lady Welby’s work as an inspiration and continued in that line, but I think it’s probably fair to say that since that time there hasn’t been much interest in her work, except a resurgence, say, since the 1980s onwards with semioticians looking at the history of semiotics. But these semioticians weren’t deploying her theories actively to make new analyses, but rather just trying to uncover the past.

\paragraph*{WS:} That’s right. Even the Signific movement in the Netherlands didn’t go the same way as Lady Welby. She was just a source of inspiration, and the Dutch scholars, especially Gerrit Mannoury, developed a kind of psychological communication theory. What happened to Lady Welby’s ideas was a kind of unacknowledged, clandestine continuation. If you look at the book by Ogden and Richards, \textit{The meaning of meaning}, there you find a lot of elements of Lady Welby’s work. Even if Ogden and Richards tried to hide it, they were standing on Lady Welby’s shoulders.

\paragraph*{JMc:} And of course, Ogden was one of Lady Welby’s assistants.

\paragraph*{WS:} Yes, yes. And Ogden copied, for example, the important letters from Peirce to Welby, and printed them in the appendix of \textit{The meaning of meaning}. The personal idealism of Ferdinand Canning Scott Schiller was also in some respects influenced by Lady Welby’s theory. Even in the novels of H. G. Wells, you can find traces of Lady Welby. Especially the late novel \textit{The shape of things to come} gets to Ogden and Richards and to Lady Welby, and he knows very well the connection between Welby and Ogden, for example. Another more or less clandestine trace is in General Semantics. Korzybski and Hayakawa were very familiar with the writings of Lady Welby.

Lady Welby was important in her day, but in a certain respect it was good to go beyond the initial inspiration she provided. But semiotics has never again found a way to study the use of signs in communication. Semioticians since Lady Welby’s time have all focused exclusively on creating a taxonomy of signs, as Peirce and Saussure did, but the question of how signs are used in communication is largely neglected in semiotics today. Perhaps it has wandered into conversational analysis, but it has left semiotics.

\paragraph*{JMc:} Thank you very much for this interview.

\paragraph*{WS:} Thank you for your interest in the topic.


\nocite{hayakawa1939a}
\nocite{ogden1949a}
\nocite{russell1905a}
\nocite{schiller1920a}
\nocite{strawson1950a}
\nocite{welby1883a}
\nocite{welby1897a}
\nocite{welby1983a}
\nocite{welby1985a}
\nocite{welby1985b}
\nocite{welby1985c}
\nocite{welby1902a}
\nocite{wells1933a}
\nocite{heijerman1991a}
\nocite{mcelvenny2014a}
\nocite{mcelvenny2018a}
\nocite{petrilli2009a}
\nocite{petrilli2015a}
\nocite{schmitz1985a}
\nocite{schmitz1990a}
\nocite{schmitz1990b}
\nocite{schmitz1993a}
\nocite{schmitz1995a}
\nocite{schmitz1998a}
\nocite{schmitz2009a}
\nocite{schmitz2011a}
\nocite{schmitz2014a}
\nocite{unknown2013a}


\sloppy
\PrintPrimarySources{}
\PrintSecondarySources{}

\end{document}
