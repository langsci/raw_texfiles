% This file was converted to LaTeX by Writer2LaTeX ver. 1.4
% see http://writer2latex.sourceforge.net for more info
\documentclass[12pt]{article}
\usepackage[utf8]{inputenc}
\usepackage[T1]{fontenc}
\usepackage[ngerman]{babel}
\usepackage{amsmath}
\usepackage{amssymb,amsfonts,textcomp}
\usepackage{array}
\usepackage{supertabular}
\usepackage{hhline}
\usepackage{hyperref}
\hypersetup{colorlinks=true, linkcolor=blue, citecolor=blue, filecolor=blue, urlcolor=blue}
% Text styles
\newcommand\textstyleInternetlink[1]{#1}
\makeatletter
\newcommand\arraybslash{\let\\\@arraycr}
\makeatother
\raggedbottom
% Paragraph styles
\renewcommand\familydefault{\rmdefault}
\newenvironment{styleStandard}{\setlength\leftskip{0cm}\setlength\rightskip{0cm plus 1fil}\setlength\parindent{0cm}\setlength\parfillskip{0pt plus 1fil}\setlength\parskip{0in plus 1pt}\writerlistparindent\writerlistleftskip\leavevmode\normalfont\normalsize\writerlistlabel\ignorespaces}{\unskip\vspace{0in plus 1pt}\par}
\newenvironment{styleListParagraph}{\renewcommand\baselinestretch{1.25}\setlength\leftskip{0.4925in}\setlength\rightskip{0in plus 1fil}\setlength\parindent{-0.4925in}\setlength\parfillskip{0pt plus 1fil}\setlength\parskip{0in plus 1pt}\writerlistparindent\writerlistleftskip\leavevmode\normalfont\normalsize\writerlistlabel\ignorespaces}{\unskip\vspace{0in plus 1pt}\par}
% List styles
\newcommand\writerlistleftskip{}
\newcommand\writerlistparindent{}
\newcommand\writerlistlabel{}
\newcommand\writerlistremovelabel{\aftergroup\let\aftergroup\writerlistparindent\aftergroup\relax\aftergroup\let\aftergroup\writerlistlabel\aftergroup\relax}
\setlength\tabcolsep{1mm}
\renewcommand\arraystretch{1.3}
\title{}
\author{James McElvenny}
\date{2022-06-14}
\begin{document}
\begin{styleStandard}
\textbf{Interview with Jürgen Trabant on Wilhelm von Humboldt}
\end{styleStandard}

\begin{styleStandard}
\textbf{Jürgen Trabant}
\end{styleStandard}

\begin{styleStandard}
\textbf{\textit{Free University of Berlin}}
\end{styleStandard}

\begin{styleStandard}
\textbf{James McElvenny}
\end{styleStandard}

\begin{styleStandard}
\textbf{\textit{University of Siegen}}
\end{styleStandard}

\begin{flushleft}
\tablefirsthead{}
\tablehead{}
\tabletail{}
\tablelasttail{}
\begin{supertabular}{m{0.24415985in}m{5.89766in}}
\textbf{JMc}\newline
 &
Today we’re joined by Jürgen Trabant, Emeritus Professor of Romance Languages at the Free University of Berlin, who’ll be talking to us about Wilhelm von Humboldt. Jürgen is the author of numerous works on Humboldt in several languages. You can find a selection of his most significant works in the references list.

So, Jürgen, what would you say is the foundation of Humboldt’s philosophy of language? You have written about what you call Humboldt’s “anti-semiotics”. Could you tell us about what this is and how it fits into the philosophical landscape of Humboldt’s time?\\
\textbf{JT}\newline
 &
Yes, the anti-semiotics of Humboldt is very interesting, and it goes to the very philosophical heart of Humboldt’s language philosophy, because he was, in that point, anti-Aristotelean. The semiotic conception of language was for centuries linked to the European reception of the \textit{De Interpretatione} of Aristotle. Aristotle had the idea that languages are pure means of communication, hence signs. Aristotle, introduced the term “sign”, \textit{semeion}, into the history of language philosophy. The idea was that: Here are the humans. They are everywhere the same, and they think the same everywhere, and they create ideas, their thoughts, universally in the same way. And when they want to communicate those thoughts, they use signs. They use sounds which are signs and which are completely arbitrary, or as Aristotle says, \textit{kata syntheken}. 

Hence we have this idea that words and languages are arbitrary signs, which is then taken up by Saussure – but in a different way, by the way. What Humboldt and other European thinkers realize, mainly in the 17\textsuperscript{th} and 18th centuries, is that languages, words are not signs in that way, but that languages in a sense shape thought in different ways. This was a catastrophic insight for the British philosophers – for Bacon, for Locke. They realized that the common languages – or the languages of extra-European people more so – shaped thought in different ways. So the Europeans realized that it was difficult to say what the Christians wanted to communicate in Nahuatl or Otomi, in American languages, and hence they realized that the languages create different thought. And this is the idea Humboldt takes up through Leibniz, mainly, and which he then transforms into his language philosophy and which he transforms also into his linguistic project. The aim of his linguistic project is exactly inquiry into the diversity of human thought. And this is why his title is \textit{Über die Verschiedenheit des menschlichen Sprachbaues}, \textit{On the Diversity of Human Language Construction}. So I think the anti-semiotics leads us to the very centre of Humboldt’s linguistic philosophy.\\
\textbf{JMc}\newline
 &
OK, and in terms of the immediate philosophical context in which he was working, do you think that Humboldt’s thought came out of a particularly German tradition or was it pan-European?\\
\textbf{JT}\newline
 &
I would say the discovery that different languages create different thought, that was pan-European. But it was mainly in the British world that it was seen as a catastrophic insight because then communication becomes even more difficult than after the Tower of Babel. Now we have really different thought systems, and the German side of it is that Leibniz transformed this idea, this insight, into a celebration of diversity. Leibniz said it’s \textit{la merveilleuse variété des opérations de notre esprit}, the marvellous variety of the operations of our spirit, of our mind, and this celebration of diversity is what Humboldt takes up. He was educated by Leibnizian philosophers. His teacher was a Leibnizian, and his first education was very much formed by this Leibnizian joy of individualism, of diversity, of the wealth of being diverse. And then, of course, Humboldt became a Kantian, which is another story, but Kant then, in a certain way, is the general background for his construction of a philosophy of language. But I would say the very idea of creating a new linguistics is Leibniz, and it’s Herder, and hence it is very German because it’s this celebration, this joy of diversity which is the German contribution to linguistics, because only if you see that the languages of the world are different worldviews, that they create different semantics, different insights, then the research into those languages becomes a worthy thing. Otherwise, why would you research languages if they are only means of communication?\\
\textbf{JMc}\newline
 &
Hans Aarsleff has made the case that Humboldt’s time studying in Paris played an important role at least in turning his attention to language, if not in shaping his outlook, but do you think that plays a significant role at all in Humboldt’s thinking?\\
\textbf{JT}\newline
 &
No, I mean, we, the German scholars, researched this for some time. Aarsleff invented this legend, and I think we really found that this was not the case, I mean that Humboldt was not a German ideologist, \textit{un idéologue allemand}. He was 30 years old when he came to Paris, and he was a complete Kantian, and he tried to convince the French philosophers of his Kantian insights. And the idea that Humboldt is a French philosopher is completely absurd, and I think this was proven by years of research into that idea. But what is certainly right is that Humboldt discovered in Paris his linguistic interest, not via \textit{les idéologues}, but via his encounter with the Basque language, so he encountered this very strange language – before that he was, he had already written about language. But then he finds this very strange language, and his question is how can you think in such a strange language, which is completely different from what he knew from the Indo-European languages, and from Hebrew – these were the languages he knew – and then he goes into that strange language. He travels to the Basque country. He travels to his New World, in a certain way, and then he is fascinated by it, by languages, and he becomes a real linguist trying to get into the structure of languages. Then, as you know, his brother brings American languages, American grammars and dictionaries to Rome.\\
\textbf{JMc}\newline
 &
So Alexander von Humboldt.\\
\textbf{JT}\newline
 &
Alexander von Humboldt, yes. This is also very important: Alexander brings these twelve books, which I consider to be the very first moment in real comparative descriptive linguistics, so he brings these books to Europe, and Friedrich Schlegel reads them first, and then after Schlegel, because Wilhelm doesn’t have the time to read them. But when he has got the time after 1811 and in the 1820s, he studies these books, and he tries to describe those American languages and their really different structural personality. 

I think this is very important, because Humboldt is really not a philosopher from the very beginning. He is a real linguist, and from his linguistics, he goes into philosophy. We have to consider his initial education. When he was young, he was looking for something to do, some contribution he could make. He was not a poet, and he discovered that he was not a philosopher, philosophy was done by Kant, and he believed in Kant. Kant is his master and the master of Germany. But what he discovered and what he was really good at was anthropology. What is anthropology? Anthropology is the description and the study of the concrete manifestations of humanity – not philosophy, not the universal, but the concrete, historical, particular, individual manifestations of humans. And this is what he starts with first. He goes to Paris in order to write a book on, an anthropological study of France. This is what his project is, and then he discovers languages, and he finds that in the very centre of the \textit{anthropos}, of the human, we have language, language as the creation of thought. And now, when he studies languages, at the same time, he writes or he tries to develop his philosophy. 

If you look at what Humboldt really published – he published very few things during his lifetime, practically only some of his speeches at the Berlin Academy – we often forget the book on the Basque because it’s not very Humboldtian. He publishes only eight discourses from the Academy, but he presents I think something like 18 or 17 topics there. So he is 50 years old, when he starts publishing. And what does he publish? He publishes linguistics, linguistic descriptions, grammatical problems on Sanskrit and so on and so forth, on the American languages, and then, of course, at the end of his life, on the Pacific Austronesian languages. So what he presents, really, to the public is linguistic things, but what he does not publish, but what he is working on, is the philosophical part of it, because he has to justify to himself why he is doing this, why he is studying languages. And hence he has to develop a philosophy of language, which is published only after his death, in the first volume of his main work on the Kawi-Sprache.\\
\textbf{JMc}\newline
 &
OK, so that’s a good connection to our next question, which is, how would you say does Humboldt’s concrete study of language, of human language and particular languages, relate to his overall philosophy, in particular the distinction that Humboldt makes between the “construction” or the “organism” of a language and its “character”?\\
\textbf{JT}\newline
 &
Yes, that is a very important question. We first have to say what this opposition is. Studying the construction or the structure, he calls it \textit{den} \textit{Bau}, and in French he calls it \textit{structure}, \textit{charpente}, so it’s the term “structure” which comes up here. And he says we have to study the structures of languages. He also calls these structures the “organisms”. We have to do a systematic study of languages as structures. This is the first step, and then he says this is only the dead skeleton, \textit{das tote Gerippe}, of languages. But languages are not a dead skeleton, languages are spoken. They are action. They are \textit{energeia}. They are activity, and hence to really see what languages are, we have to look at them in action, in speech, in literature. He adds to the description of the construction another chapter on the character. He says if we really want to grasp the very individuality of languages, we have to look into literature, and hence he joined linguistics – and he says \textit{Linguistik –} to philology, \textit{Philologie}. So for him, linguistics, structural linguistics, and the history of that language in its texts are two parts of language study. 

What is so interesting in the 19th century is that, because this dichotomy in the 19th century is very strong, the philologists – so those are the classicists – \ are immediately against linguistics, because linguistics becomes a natural science, it becomes structural, it becomes very technical, and the philologists, they want to stay with their texts. Humboldt sees both together, structure and texts, and he wants them not to be separate, but two chapters, in a certain way, of language studies. But then, of course, in the 19th century, these things get separated. Steinthal is perhaps the last one who tries, again, to put these two together. He has what he called \textit{Stilistik}, stylistics. \textit{Stilistik} is actually the study of the character of languages. But the 19th century is not a century of character, but it comes up in the 20th century and afterwards, so there are linguists who think that language is something living, is an activity, and that we have to study the active usage of language, but I would say this comes in the 20th century with people like Karl Vossler, with so-called idealism, which is then considered by the linguists of the 19th century as non-linguistic.\\
\textbf{JMc}\newline
 &
So you were saying that Humboldt has these two compartments, the structure and the character. But is it not the case that Humboldt felt that the character was more important than the structure? He calls character the \textit{Schlussstein}, the keystone.\\
\textbf{JT}\newline
 &
Yes, it’s the \textit{Schlussstein}. The final aim would be the description of the character of a language. But he never succeeds in describing the character in his Nahuatl grammar, which is the only grammar he really finished and he really nearly published also, which Manfred Ringmacher only published in the nineties. There, he has a chapter on the character, but the chapter is very weak because he does not have texts. Humboldt does not have Nahuatl texts, or very few, only translations, and hence he can’t grasp the character. Hence this chapter on the character is rather deceptive, and when you look for what Humboldt is thinking of when he talks of character, he says we have to study the literature and how the people talk, and then he has one footnote where he refers to a history of Greek literature. He says the history of Greek prose \ might be a description of the character of the Greek language. It’s very hidden, but at the same time, it’s also very true, because what is the description of an individual? The scientific description of an individual is his or her story, her history or his history. So there is no definition of an individual, but in order to say scientifically something about an individual, you have to write his or her history. And this, I think, is the wisdom of that footnote in Humboldt, but he himself, he never succeeds in writing such a description of character. He himself writes grammars, descriptions of the dead skeleton, and writes sketches of other American Indian languages. 

What is also important to remember is that we only know the linguistic work of Humboldt, because Mueller-Vollmer realized – when he saw the material that was not published – that we have to join Humboldt’s linguistic descriptions to his philosophy. Humboldt is known as a philosopher of language, but he was also a real linguist, and he tried to deal with linguistic structure, and the American languages of which he had some knowledge came in grammars which were framed in terms of Latin or Spanish grammar. So you had paradigms like \textit{rosa}, \textit{rosae}, \textit{rosae}, \textit{rosam}, etc., and of course, the Spanish priests who wrote those descriptions followed the Latin, European, Indo-European Spanish grammar, and hence we have descriptions which do not at all render even the individual structure of those languages. So in a certain way, those descriptions even destroy the individuality of the American Indian languages, and Humboldt was very much aware of that problem. What he tried to do in the Nahuatl grammar is to get through those Indo-European descriptions of Nahuatl, for instance, and to show what categories, what grammatical categories are working in Nahuatl, what the structure of that language is.\\
 &
So I think this is really important, but we did not know this of Humboldt. The Nahuatl grammar was not published until 1994, and nobody knew Humboldt as a descriptive linguist.\\
\textbf{JMc}\newline
 &
So linguists in the 19th century were much more interested in the in this dead skeleton of the languages and took absolutely no interest in the character, and as you were saying yourself, Humboldt never really succeeded in developing his linguistics of character himself.\\
\textbf{JT}\newline
 &
Yes.\\
\textbf{JMc}\newline
 &
Why do you think that might be?\\
\textbf{JT}\newline
 &
There are also political reasons. German linguists, like Grimm and Bopp, were also reconstructing the past of the nation, and of Europe. The Grimms dealt with German, Germanic languages. I mean, they called their grammar \textit{Deutsche Grammatik}, but it’s a Germanic grammar. It’s a comparative grammar of the Germanic languages, not at all a German grammar. And here comes Bopp, and what does he do? He compares the Indo-European languages. He does not go beyond, and he even tries to integrate non-Indo-European languages into the Indo-European family, like Polynesian, for instance. He writes against Humboldt. He actively wants to integrate the Austronesian languages into the Indo-European family, and Humboldt was trying to show just the contrary. So I think Germany, Europe were the aim, the final aim of historical linguistics. And the other guys who dealt with non-Indo-European languages, they were the minority. They mostly they were Orientalists, Sinologists, and so on dealing with oriental languages, Chinese, Egyptian, but they were not at the very centre of linguistics.\\
\textbf{JMc}\newline
 &
But a figure like Schleicher, for example, was at the very centre mid-19th-century, and of course Schleicher developed his theory of morphology, which is essentially a kind of typology from a present-day perspective and does have pretensions to accounting for the structure of all languages.\\
\textbf{JT}\newline
 &
Yes, of course, but here, I would say, we do not have the European or German theme anymore; here we have the scientific theme, so we have Darwinism, and of course the influence of natural sciences is very strong here. Hence we have to create, like Darwin did for the species, a tree for the development of all languages of mankind. Yes, that is true, but morphology was always at the very centre. I mean, morphology, this is what what Schlegel, Friedrich Schlegel, discovered when he said we have to look at the \textit{Struktur}. He uses the term \textit{Struktur}, \textit{innere Struktur}, for the first time, and we have to look at the \textit{Struktur} and not at the vocabulary for the comparison of languages. And this is what Bopp does immediately when he writes a \textit{Conjugationssystem}. It’s on \textit{Konjugation}. It’s not on semantics. It does not compare, as Peter Simon Pallas for instance did, words, lexicon, as the basis of his comparative approach, but he then already goes into \textit{Konjugation}, and then, of course, the Grimms go into \textit{Deutsche Grammatik}. First, they write the \textit{Deutsche Grammatik} before they go on to the \textit{Wörterbuch}. And then, of course, after the Grimms, everybody in Europe writes comparative grammars – grammar of the Romance languages, grammar of the Slavic languages, and so on and so forth – so this becomes a huge success. After the Grimms, Bopp and then all the others do comparative grammars, and hence the focus is on morphology, and morphology means also they’re not dealing so very much with the meaning of those morphemes, but they’re more with the form, with the material form of morphemes.\\
\textbf{JMc}\newline
 &
Yes, that’s very true. I mean, Schleicher says himself that he can’t penetrate into the inner form of languages. He just sticks to the surface. So this brings us to the last question, which is about Humboldt’s term “inner form”. This is probably one of the most iconic Humboldtian terms, “inner form”, but Humboldt used the term only in passing himself, and later scholars, right up to the 20th century, have used it in myriad different senses. So why do you think this term has captured people’s imaginations in the way that it has, and what do you think the significance of the term was for Humboldt himself?\\
\textbf{JT}\newline
 &
Let’s start with the first part. “Inner form” comes up in the \textit{Kawi-Einleitung.} After writing some chapters on external form, \textit{äußere Form}, or the \textit{Lautform}, Humboldt writes a chapter on inner form, \textit{innere Sprachform.} What is \textit{innere Sprachform}? What does Humboldt talk about in this chapter? He talks about the semantics of words, and he talks about the semantics of grammatical categories, so this is \textit{innere Form}. \textit{Innere Form} just means the meaning, and then he goes on and talks about the conjunction of meaning and sound. So the next chapter after the chapter on \textit{innere Sprachform} is about both meaning and sound going together. So, and I think the term \textit{innere Sprachform} has been exaggerated by the readers of Humboldt, certainly, but I think they saw something really correct in the end, because this is the very centre of his thought. If we go back to my answer to your first question, I think that going into semantics and into the meaning of categories of morphemes, this is the inner form. 

And this is really what is the very centre of Humboldt’s dealing with languages, because he wants to show \textit{la merveilleuse variété des opérations de notre esprit}, the marvellous variety of variety of the operations of our mind. And mind is the inner form, so even if the chapter on inner form is very short, the readers of Humboldt were correct in focusing on this term, because this is the very novelty of his approach, to look not only at the variety of the sounds. That languages have different sounds was clear from Aristotle onwards, and this material diversity was clear from antiquity on. But \ Bacon, Locke, Leibniz, and Herder, Humboldt, they see: no, it’s not only sound that differs in languages. It’s the meaning. It’s the mind. It’s the inner form, and I think therefore the focus on inner form is really justified.\\
\textbf{JMc}\newline
 &
OK, although I guess meaning and semantics, those are potentially anachronistic terms, because if you think of how semantics is done today – like truth-functional semantics, for example – there’s an idea that meaning is something objective, but for Humboldt inner form is perhaps something much more mystical, talking about the operations of the mind.\\
\textbf{JT}\newline
 &
No, not so much. No, because for instance, in his first discourse at the Academy, where he tries to find an answer, why we have to do linguistics, he proposes that we now have to describe all the languages of the world. We have to do \textit{vergleichendes Sprachstudium}, descriptive-comparative, descriptive \textit{Linguistik}.\\
\textbf{JMc}\newline
 &
OK.\\
\textbf{JT}\newline
 &
And then Humboldt asks, why do we do comparative linguistics, and then at the end, he talks about the semantics of words, quite clearly. He says that words that refer to feelings, to interior operations of the mind, differ more from language to language. Words for exterior objects, they differ less. However, they still differ. A sheep might be something different in, let’s say, in Nahuatl and in French and so on. So I think there is this focus on the meaning, which he calls \textit{Begriff}, by the way. He does not talk about \textit{Bedeutung}. His term is \textit{Begriff}, and the \textit{Begriff} here can be different in different languages.\\
\textbf{JMc}\newline
 &
So you might call \textit{Begriff} “concept” in English, do you think?\\
\textbf{JT}\newline
 &
Yes, I would say concept. But \textit{Begriff }or concept, after Hegel and rationalism, was too closely aligned with the mind. A better word is perhaps \textit{Vorstellung}, because it’s less rationalistic, because this is exactly what the mind does. The mind does create \textit{Vorstellungen} – this is how Humboldt describes it: the world goes through the senses into the mind, and the mind then creates \textit{Vorstellungen}. And they are immediately connected to sound, so they’re immediately words.\\
\textbf{JMc}\newline
 &
So in English we might say “representation” or “image” for \textit{Vorstellung}, do you think?\\
\textbf{JT}\newline
 &
Why not?\\
\textbf{JMc}\newline
 &
Yeah. Why not?\\
\textbf{JT}\newline
 &
{\textquotedbl}Image{\textquotedbl} is also not bad because the word, as Humboldt says, is something between image and sign. Sign is the completely arbitrary thing with the universal concept. Image is something concrete, which depicts the world, and the word is something in between. It has a special structure, a special position between sign and image. Sometimes the word can be an \textit{Abbild}, an image, and sometimes it can also be used as a sign, but this is possible because it is in between the sign and the image. And perhaps one word on this problem: right in the chapter on the \textit{innere Form}, he adds that we might compare the word, or the work of the mind creating a language, with the work of an artist. So that is exactly what he is thinking. He says that languages work like artists, you see: they create images.\\
\textbf{JMc}\newline
 &
OK. Excellent. Well, thank you very much for this conversation.\\
\textbf{JT}\newline
 &
Thank you very much for the interesting questions.\\
\end{supertabular}
\end{flushleft}
\begin{styleStandard}
\textbf{References}
\end{styleStandard}

\begin{styleStandard}
\textbf{Primary sources}
\end{styleStandard}

\begin{styleListParagraph}
Bopp, Franz (1816), \textit{Über das Conjugationssystem der Sanskritsprache in Vergleichung mit jenem der griechischen, lateinischen, persischen und germanischen Sprache}, Frankfurt am Main: Andräische Buchhandlung. 
\end{styleListParagraph}

\begin{styleListParagraph}
Bopp, Franz (1820), \textit{Analytical Comparison of the Sanskrit, Greek, Latin, and Teutonic Languages, shewing the origingal identity of their grammatical structure}, \textit{Annals of Oriental Literature}, 1, pp. 1–64. 
\end{styleListParagraph}

\begin{styleListParagraph}
Bopp, Franz (1833–1852), \textit{Vergleichende Grammatik des Sanskrit, Zend, Griechischen, Lateinischen, Lithauischen, Gothischen und Deutschen}, 6 vols, Berlin: Dümmler.\newline
(2nd ed. 1857–1861, \textit{Vergleichende Grammatik des Sanskrit, Send, Armenischen, Griechischen, Lateinischen, Litauischen, Altslavischen, Gothischen und Deutschen}, 3 vols.)\newline
(English trans.: Edward B. Eastwick, 1845–1853, \textit{A Comparative Grammar of the Sanskrit, Zend, Greek, Latin, Lithuanian, Gothic, German, and Sclavonic Languages}, London: Madden and Malcolm, 3 vols.)
\end{styleListParagraph}

\begin{styleListParagraph}
Grimm, Jacob (1819), \textit{Deutsche Grammatik}, vol. 1, Göttingen: Dieterich’sche Buchhandlung.\newline
(2nd ed. 1822–1837, \textit{Deutsche Grammatik}, 4 vols., Göttingen: Dieterich’sche Buchhandlung.)
\end{styleListParagraph}

\begin{styleListParagraph}
Grimm, Jacob and Wilhelm Grimm et al., eds. (1854–1960), \textit{Deutsches Wörterbuch}, 16 vols., Leipzig: Hirzel.
\end{styleListParagraph}

\begin{styleListParagraph}
Humboldt, Wilhelm von (1836), ‘Über die Verschiedenheit des menschlichen Sprachbaues’, \textit{Über die Kawi-Sprache auf der Insel Java}, vol. 1, ed. Alexander von Humboldt, Berlin: Dümmler. \href{https://archive.org/details/berdiekawisprac00unkngoog/page/n6}{\textstyleInternetlink{archive.org}}\newline
(English trans. \textit{On Language. The diversity of human language structure and ist influence on the mental development of mankind} [1988], trans. Peter Heath, Cambridge: Cambridge University Press.)
\end{styleListParagraph}

\begin{styleListParagraph}
Humboldt, Wilhelm von (1994), \textit{Mexikanische Grammatik}, ed. Manfred Ringmacher, Paderborn: Schöningh.
\end{styleListParagraph}

\begin{styleListParagraph}
Humboldt, Wilhelm von (1997), \textit{Essays on Language}, trans. Theo Harden and D. Farrelly, Frankfurt am Main: Lang.
\end{styleListParagraph}

\begin{styleListParagraph}
Humboldt, Wilhelm von (2012), \textit{Baskische Wortstudien und Grammatik}, ed. Bernhard Hurch, Paderborn: Schöningh.
\end{styleListParagraph}

\begin{styleListParagraph}
Schlegel, Friedrich (1808), \textit{Ueber die Sprache und Weisheit der Indier}, Heidelberg: Mohr und Zimmer. (English trans. ‘On the Indian Language, Literature and Philosophy’ [1900], \textit{The Æsthetic and Miscellaneous Works of Friedrich von Schlegel}, ed. and trans. E. J. Millington, pp. 425–536, London: George Bell and Sons.)
\end{styleListParagraph}

\begin{styleListParagraph}
Schleicher, August (1850), \textit{Die Sprachen Europas in systematischer Übersicht}, Bonn: König.
\end{styleListParagraph}

\begin{styleListParagraph}
Schleicher, August (1863), \textit{Die Darwinsche Theorie und die Sprachwissenschaft, offenes Sendschreiben an Herrn Dr. Ernst Haeckel, o. Professor der Zoologie und Direktor des zoologischen Museums an der Universität Jena}, Weimar: Hermann Böhlau. (English trans.: (1869), \textit{Darwinism Tested by the Science of Language}, trans. Alex V. W. Bikkers, London: John Camden Hotton.)
\end{styleListParagraph}

\begin{styleListParagraph}
Vossler, Karl (1904), \textit{Positivismus und Idealismus in der Sprachwissenschaft}, Heidelberg: Winter.
\end{styleListParagraph}

\begin{styleStandard}
\textbf{Secondary sources}
\end{styleStandard}

\begin{styleListParagraph}
Aarsleff, Hans, and John L. Logan (2016), ‘An essay on the context and formation of Wilhelm von Humboldt’s linguistic thought’, \textit{History of European Ideas }42.6: 729–807.
\end{styleListParagraph}

\begin{styleListParagraph}
McElvenny, James (2016), ‘The fate of form in the Humboldtian tradition: the Formungstrieb of Georg von der Gabelentz’, \textit{Language and Communication} 47: 30–42. 
\end{styleListParagraph}

\begin{styleListParagraph}
McElvenny, James (2018), ‘August Schleicher and materialism in nineteenth-century linguistics’, \textit{Historiographia Linguistica} 45.1, 133-152.
\end{styleListParagraph}

\begin{styleListParagraph}
Mueller-Vollmer, Kurt and Markus Messling (2017), ‘Wilhelm von Humboldt’, \textit{The Stanford Encyclopedia of Philosophy} (Spring 2017 Edition), ed. Edward N. Zalta. \url{https://plato.stanford.edu/archives/spr2017/entries/wilhelm-humboldt/}
\end{styleListParagraph}

\begin{styleListParagraph}
Ringmacher, Manfred (1996), \textit{Organismus der Sprachidee: H. Steinthals Weg von Humboldt zu Humboldt}, Paderborn: Schöningh.
\end{styleListParagraph}

\begin{styleListParagraph}
Trabant, Jürgen (1986), \textit{Apeliotes oder der Sinn der Sprache, Wilhelm von Humboldts Sprachbild}, München: Wilhelm Fink. \href{https://digi20.digitale-sammlungen.de/de/fs1/object/display/bsb00042733_00001.html}{\textstyleInternetlink{Bayerische Staatsbibliothek}}\newline
(French trans. \textit{Humboldt ou le sens du langage} [1992], avec François Mortier et Jean-Luc Evard, Liège: Mardaga.)
\end{styleListParagraph}

\begin{styleListParagraph}
Trabant, Jürgen (2012), Weltansichten: Wilhelm von Humboldts Sprachprojekt, München: C.H. Beck.
\end{styleListParagraph}

\begin{styleListParagraph}
Trabant, Jürgen (2015), Wilhelm von Humboldt Lectures, Université de Rouen. \url{https://webtv.univ-rouen.fr/channels/#2015-wilhelm-von-humboldt-lectures}
\end{styleListParagraph}

\begin{styleListParagraph}
Trabant, Jürgen (2020), ‘Science of Language: India vs America: the Science of Language in 19th-Century Germany’, \textit{Doing Humanities in Nineteenth-Century Germany}, ed. Efraim Podoksik, 189–213, Leiden: Brill.
\end{styleListParagraph}

\end{document}
