% This file was converted to LaTeX by Writer2LaTeX ver. 1.4
% see http://writer2latex.sourceforge.net for more info
\documentclass[12pt]{article}
\usepackage[utf8]{inputenc}
\usepackage[T1]{fontenc}
\usepackage[ngerman]{babel}
\usepackage{amsmath}
\usepackage{amssymb,amsfonts,textcomp}
\usepackage{array}
\usepackage{supertabular}
\usepackage{hhline}
\usepackage{hyperref}
\hypersetup{colorlinks=true, linkcolor=blue, citecolor=blue, filecolor=blue, urlcolor=blue}
% Text styles
\newcommand\textstyleInternetlink[1]{#1}
\makeatletter
\newcommand\arraybslash{\let\\\@arraycr}
\makeatother
\raggedbottom
% Paragraph styles
\renewcommand\familydefault{\rmdefault}
\newenvironment{styleStandard}{\setlength\leftskip{0cm}\setlength\rightskip{0cm plus 1fil}\setlength\parindent{0cm}\setlength\parfillskip{0pt plus 1fil}\setlength\parskip{0in plus 1pt}\writerlistparindent\writerlistleftskip\leavevmode\normalfont\normalsize\writerlistlabel\ignorespaces}{\unskip\vspace{0in plus 1pt}\par}
% List styles
\newcommand\writerlistleftskip{}
\newcommand\writerlistparindent{}
\newcommand\writerlistlabel{}
\newcommand\writerlistremovelabel{\aftergroup\let\aftergroup\writerlistparindent\aftergroup\relax\aftergroup\let\aftergroup\writerlistlabel\aftergroup\relax}
\setlength\tabcolsep{1mm}
\renewcommand\arraystretch{1.3}
\title{}
\author{James McElvenny}
\date{2022-06-14}
\begin{document}
\begin{styleStandard}
\textbf{Interview with John Joseph on Saussure}
\end{styleStandard}

\begin{styleStandard}
John Joseph
\end{styleStandard}

\begin{styleStandard}
\textit{University of Edinburgh}
\end{styleStandard}

\begin{styleStandard}
James McElvenny
\end{styleStandard}

\begin{styleStandard}
\textit{University of Siegen}
\end{styleStandard}

\begin{flushleft}
\tablefirsthead{}
\tablehead{}
\tabletail{}
\tablelasttail{}
\begin{supertabular}{m{0.24415985in}m{5.89766in}}
\textbf{JMc}\newline
 &
In this interview, we’re joined by John Joseph, Professor of Applied Linguistics at the University of Edinburgh. He’ll be talking to us about the great Genevan linguist Ferdinand de Saussure. John is the author of many works relevant to our topic today, the most significant of which would have to be his 2012 biography of Saussure, published with Oxford University Press.

So, John, please tell us about Saussure. Saussure is perhaps best known for his \textit{Course in General Linguistics}, which is widely considered a foundational text of linguistic structuralism. What’s your view on this matter? Would you say that Saussure’s course was a truly groundbreaking work that single-handedly brought structuralism into being?\\
\textbf{JEJ}\newline
 &
For my part, James, I’m still struggling to understand what ‘structuralism’ meant and means. The linguists who called their approach structural weren’t all doing the same thing; they agreed on some principles and vigorously disputed others. One thing they shared was an impulse to analyse and write about languages in a way that was modern – modernist even – and in the \textit{Course in General Linguistics} they found a model for doing that. Nothing about language and intelligence, or language and the national soul, or culture, and an out-and-out rejection of any connection of language with race. No deep philosophical ruminations. Some later structuralists would make links with philosophy, and vice-versa. But for linguists, whatever philosophical implications may have been latent in the \textit{Course} could be left aside, and they could focus on its very sleek, minimalist model of a system of linguistic signs, each made up of a value – a value that was purely its difference from the other elements in the system. That’s modernist, and especially in the wake of World War I, when there was a desire to move forward in a new scientific direction, it had great appeal.\\
\textbf{JMc}\newline
 &
What influence did Saussure’s \textit{Course} have on linguistic scholarship of the time? So the Prague School certainly appealed to Saussure quite often, but did they really follow him? And what about their contemporaries in the English-speaking world, such as Leonard Bloomfield and Edward Sapir in the US or even John Rupert Firth in England?\\
\textbf{JEJ}\newline
 &
I’ll start with the Prague School, and Roman Jakobson, who introduced the term structuralism as a literary and linguistic method or approach. No one did more to disseminate Saussure’s \textit{Course} and proclaim its fundamental importance than Jakobson did – and there was hardly any position taken by Saussure that Jakobson didn’t contest, or even reject out of hand. That includes the fundamental precept that linguistic signs are purely differential. Saussurean phonology is what’s nowadays called a ‘substance-free’ phonology, where it’s all about patterns in the mind, and the sounds don’t matter. Jakobson and his collaborator Nikolai Trubetzkoy said no, some sounds in a language are very distinctive to the ear, whilst others are harder to distinguish, and those maximally distinctive sounds are in various respects more fundamental. 

Jakobson wrote an article called ‘Why \textit{mama} and \textit{papa}?’, why across the world’s languages is it disproportionately the case that /m/ and /p/ or /b/, and the vowel /a/, figure in the words by which children call the two most important people in their lives? The answer lies for Jakobson in the maximal distinctiveness of these sounds to the ear, making them the easiest and first sounds for children to master, to produce systematically. A sound such as /$\theta $/ is hard to distinguish from /s/ or /f/ or /t[2B0?]/, and it’s no coincidence that /$\theta $/ is relatively rare amongst the world’s languages, is learned late by children and is unstable over time. The number which follows two is \textit{three} for me, but \textit{tree} in many Irish dialects, and \textit{free} in a growing number of English dialects. Saussurean phonology can’t account for this; all it can say is that /$\theta $/ is a phoneme by virtue of its difference from /s/, /f/ and /t/ – degrees of difference don’t enter into the equation. So here Jakobson directly contradicts Saussure on a fundamental matter – yet Jakobson was always the first to say that only because of Saussure’s \textit{Course} was he able to make this step at all. 

Prague wasn’t the only place where structural linguistics was moving forward in the 1920s and 30s. Louis Hjelmslev had left Copenhagen to study with Saussure’s former pupil Antoine Meillet in Paris, and Hjelmslev’s 1928 book \textit{Principles of General Grammar} is deeply Saussurean in orientation. So is the first volume of his next book, \textit{Categories of Case} from 1935 – but by the second volume, two years later, he’s come into the orbit of Jakobson, and from then on the Copenhagen School’s relationship to Saussure is comparable to Jakobson’s own, where Saussure is revered as the founding figure who has made it possible for them to move beyond what he himself taught. In Paris, too, Émile Benveniste’s efforts at the end of the 1960s to extend linguistics beyond the semiotic are characterised as simultaneously surpassing and accomplishing Saussure’s project. 

With Sapir and Bloomfield, Saussure’s \textit{Course} figures in their writings starting already in the 1920s. Frustrated at criticism of his book \textit{Language} for not citing Saussure more, Bloomfield wrote to one of his students that Saussure’s influence is on every page. Sapir, as an anthropologist, had been well prepared for Saussurean linguistics through his work with Franz Boas, whose 1911 \textit{Handbook of American Indian Languages} shares the modernist spirit of Saussure’s \textit{Course}. On the other hand, Sapir wanted to extend his linguistic enquiry into the psychological dimension, whereas Saussure resolutely left psychology to the psychologists. Not that he dismissed it, by any means; but he’d been brought up with constant admonitions to choose a particular discipline and not stray beyond it. Saussure’s expertise was as a ‘grammarian’, as he usually called himself; any view he might venture on the psychology of language would be nothing more than opinion, not expertise, and could only damage his scholarly reputation. 

Finally, you asked about J R Firth. My emeritus colleague Ron Asher, Firth’s student, tells me that he can’t recall a single lecture by Firth in which Saussure wasn’t discussed. In 1950 Firth wrote that all linguists were now defined as Saussureans, anti-Saussureans, post-Saussureans, or non-Saussureans. Firth himself somehow managed to be all four. The \textit{system} – that was the crucial thing Firth took from Saussure, but Saussure, in his modernist impulse, had pared the system down to something oversimplified. Firth set out to rectify this, with systems within systems, tiered systems: and a concern with including linguistic \textit{meaning} within the system, not just in the sense of the ‘signified’, that part of the linguistic sign which is conceptual but internal to the language. Meaning \textit{beyond} language – what connects language to the people who speak it, them to one another and to the world they inhabit. Again, what Saussure cut off as lying beyond what he as a grammarian was qualified to talk about. It was the business of philosophers, psychologists and other specialists. For Firth, as for Ogden and Richards in their book \textit{The Meaning of Meaning}, that would always be Saussure’s great limitation.\\
\textbf{JMc}\newline
 &
What then are the innovative features of Saussure’s \textit{Course} and why do you think it has \ been elevated to this status akin to that of holy scripture?\\
\textbf{JEJ}\newline
 &
‘Holy scripture’ is an exaggeration, to put it mildly! Much of the innovation lies, as I’ve said, in what it doesn’t talk about, or pushes out of the centre and into the hinterland of the later chapters. At the centre it puts the linguistic sign, and that’s always been received as the most innovative aspect. Saussure defines a language as a system of linguistic signs – not sounds, or words, or sentences, not as something that, because it’s always evolving, has no stable existence that would allow it to be the subject of scientific enquiry in terms of what it is and how it works at a given time. 

None of these issues is ignored – rather, they’re laid out as alternative ways of analysing a language. And crucially, Saussure points out that the way you study it actually determines what the nature is of the thing you’re studying. He said: ‘the point of view determines the object’. You can study the system, \textit{la langue}, the socially-shared language, or you can study utterances and texts, \textit{la parole}, the speech of an individual. Both are valid, and each is necessary for an understanding of the other. You can study them across time, diachronically, or at a moment in time, synchronically. 

Other linguists hadn’t been mapping out the field of study in this widescreen way, with all these options. They proclaimed \textit{the} way – and so entrenched was this mindset that the \textit{Course} was widely read as if it too fit that pattern. As if Saussure was saying that linguistics had to be about \textit{langue}, not \textit{parole}, had to be synchronic, not diachronic. That he denied any link between linguistic signifieds and things in the world, referents in Frege’s term – when he simply left that to philosophers and psychologists to deal with as their specialised domain. 

In terms of style, too, the \textit{Course} is innovative in deriving from lectures, and only in part from the author’s own lecture notes. As is well known, students’ notes from the three academic years over which he gave the lectures were collated, and a plan was made based mainly on how things were arranged in the last version of the course. Saussure had been trying and failing to write books about big methodological questions in the study of languages since his early 20s. The problem was that he was a perfectionist, determined that every word from his pen had to be precisely the right word – hence the thousands of draft manuscript pages in his archives that lay unpublished until recent years, in which the same thought is often recomposed ten, twenty times, then scratched through and abandoned. 

If he had written the \textit{Course in General Linguistics} – if he could have written it – it might have been as turgid a book as the one on the primitive Indo-European vowel system which made his reputation at the age of 21, but which only a relatively small number of specialists have ever managed to work their way through. The posthumous \textit{Course} is quite the opposite – not the easiest book to read, but neither is every claim nailed down with a fixity that would protect it from any quibble. It’s a very open text – it invites readers into a world of ideas and questions in which they can make their own interpretations and give their own answers. Hence its eventual popularity, though that didn’t come until 50 years after it was published. The price of its textual openness and popularity is of course that it gets read very differently by different people, hence the large amount of scholarly work aimed at trying to understand what Saussure actually thought, which in many cases remains a mystery.\\
\textbf{JMc}\newline
 &
Do you think it would be fair to say that Saussure was simply perpetuating – and perhaps refining, but essentially perpetuating – ideas and methods that were already current among the generation of his teachers, the Neogrammarians?\\
\textbf{JEJ}\newline
 &
No, it would unsustainable to assert that Saussure was just teaching what everyone else was saying at the time. The academic economy demands continuity; anyone who tries to teach or write something without starting from the status quo of academic authority wouldn’t be hailed as a revolutionary, but banished as a crackpot. It’s a common enough game to point to the continuities and say, look, Freud said nothing that Charcot wasn’t already teaching, just sexed-up. So you get Eugenio Coseriu, for instance, claiming in 1967 that all of Saussure is already there in Georg von der Gabelentz – nothing against Gabelentz, a great linguist, but it’s as easy to build a case based just on the continuities as it is a counter-case based on the differences. 

If we want to make a realistic historical assessment of how Saussure’s linguistics relates to the ideas and models of the Neogrammarians, we should look first at how Saussure’s \textit{Course} was received by the linguists of the time, who after all were mostly practising the methods laid down by the Neogrammarians. In their eyes, what Saussure taught embodied a sea change from accepted ideas. That starts with his two colleagues who edited the \textit{Course}, Albert Sechehaye and Charles Bally – in fact, it started before them, with the students whom Saussure taught in his first job, in Paris from 1881 to 1891. They included Antoine Meillet, who always credited Saussure as creator of the radically new linguistic analysis which, led in Paris by Meillet, would develop into structuralism. 

Book reviewers of the \textit{Course} hailed its novelty, whilst also seizing upon links to their own ideas when they could be used to strengthen their position – thus you see Leonard Bloomfield in 1924 claiming that Saussure’s signifier and signified are in effect the stimulus and response of the behaviourism that Bloomfield himself had begun to follow. Again, I’ve stressed how the modernism of the \textit{Course} contributed to it sweeping away existing doctrines, including those of the Neogrammarians, which had acquired that musty smell that %
%You’ve been doing Arabic numerals, so this looks inconsistent.
%Anonymous
%22. Mai 2022, 08:45
forty{}-year-old ideas get. But it wasn’t the case that Saussure had recycled them in a new rhetorical dress and with some refinements. Just look at the core Saussurean concept of the language system as a system of values as pure difference, divorced from their phonetic realisation – when phonetic physicality is at the heart of Neogrammarian ‘sound laws’, with the psychological phenomenon of analogy admitted as a necessary explanatory escape hatch. For Saussure, the reverse: analogy, as mental processing, is placed at the centre, and phonetics becomes an adjunct to linguistics. So no wonder the \textit{Course} had the impact it did.\\
\textbf{JMc}\newline
 &
So in these cases where Saussure broke with his contemporaries and immediate predecessors, would you say that the alternative ideas he put forward were novel or that he was just drawing on even older ideas that had been forgotten or were considered superseded in the academic linguistics of the late 19th century?\\
\textbf{JEJ}\newline
 &
Again, we mustn’t forget the forces of academic economy, which demand that novel ideas be grounded in established authority: the classic example is Noam Chomsky’s \textit{Cartesian Linguistics}, in which he claims that his transformational-generative linguistics is restoring the great 17th-century tradition of understanding language and mind, after its illegitimate usurpation by linguists after Wilhelm von Humboldt. The \textit{Course in General Linguistics} accomplished something similar, though without any overt claim to be doing so. Chomsky’s ‘Cartesians’ weren’t really connected to Descartes, but never mind – his principal heroes were Lancelot and Arnauld, authors of the Port-Royal Grammar and Logic, which laid out the idea of a \textit{grammaire générale}, a universal grammar. This became established in French education, and over the course of the 18th century it came to include as one of its key components the idea of the linguistic sign, the conjunction of a signifying sound or set of sounds, and a signified concept, joined arbitrarily, which is to say with no necessary ‘natural’ link of sound to concept. 

In France, the \textit{grammaire générale} tradition in education, by which I mean secondary education, didn’t survive the Napoleonic period, when virtually everything was reformed. However, Geneva, whilst French-speaking, isn’t France, and the \textit{grammaire générale} tradition didn’t get reformed out of education in Geneva until much later. The young Saussure was in the last cohort of students taught by venerable men in their 70s who had been trained in \textit{grammaire générale} in the first third of the century, and included the theory of linguistic signs in their courses. It was something he and his age-mates had all been taught, and perhaps took to be common sense. In any case, he certainly didn’t imagine that when he included it in his courses in general linguistics almost forty years later that anyone would think it was his original idea. If so he would have pointed out its historical legacy, going back to antiquity. As fate would have it, that legacy was sufficiently forgotten that all but a few readers of the \textit{Course} experienced its theory of the linguistic sign as something radically new and modern.

This part of the \textit{Course} is one that had a very strong impact, perhaps the strongest, across a vast range of fields. But the theory of signs in the \textit{Course} becomes radically different from any that went before when he adds in the dimension that signifiers aren’t sounds, and signifieds aren’t things; he formulates them as mental patterns, sound patterns and concepts; but even this isn’t the definitive formulation, just something his students can get their head around more easily than they could with what is his ultimate view – namely, that each signifier is a value generated by difference from every other signifier within the same system, just as each signified is a value generated by difference from every other signified. That’s a core example of what makes the \textit{Course in General Linguistics} unique. To every question you ask me about whether it draws on earlier ideas or is novel, the answer is: 100\% both, somehow. Which is impossible. And ok, perhaps that’s what makes your sacred scripture analogy tempting: this book defies explanation. Its own author couldn’t write it. It was assembled from notes from three courses over which ideas were evolving and shifting, and were jotted down by various students in often incompatible ways. The editors did their best, but got some important things wrong, and the book isn’t devoid of internal contradictions. Yet somehow the result was extraordinary. You might even say miraculous.\\
\textbf{JMc}\newline
 &
Ah. Well, thanks very much for talking to us about Saussure. I’m sure you’ve inspired many of our listeners to go out there and read more about him.\\
\textbf{JEJ}\newline
 &
Thanks very much, James. \\
\end{supertabular}
\end{flushleft}
\begin{styleStandard}
\textbf{References}
\end{styleStandard}

\begin{styleStandard}
\textbf{Primary Sources}
\end{styleStandard}

\begin{styleStandard}
Arnauld, Antoine and Claude Lancelot (1660), \textit{Grammaire générale et raisonnée contenant les fondemens de l’art de parler, expliqués d’une manière claire et naturelle}, Paris: Pierre le Petit. (English version: \textit{General and Rational Grammar: The Port-Royal Grammar}, trans. by Jacques Rieux \& Bernard E. Rollin, The Hague: Mouton, 1975.)
\end{styleStandard}

\begin{styleStandard}
Benveniste, Émile (2012), \textit{Dernières leçons : Collège de France, 1968 et 1969, }ed. by Jean-Claude Coquet and Irène Fenoglio, Paris: École des Hautes Études en Sciences Sociales, Gallimard, Seuil. (English version: \textit{Last Lectures: Collège de France, 1968 and 1969}, trans. by John E. Joseph, Edinburgh: Edinburgh University Press, 2019.
\end{styleStandard}

\begin{styleStandard}
Bloomfield, Leonard (1924), Review of Saussure (1922), \textit{Modern Language Journal} 8, 317–319. DOI: \href{https://doi.org/10.2307/313991}{\textstyleInternetlink{10.2307/313991}}
\end{styleStandard}

\begin{styleStandard}
Boas, Franz, ed. (1911), \textit{Handbook of American Indian Languages}, Part 1, Washington, D.C.: Government Printing Office. \href{https://archive.org/details/handbookamerica03unkngoog}{\textstyleInternetlink{archive.org}}
\end{styleStandard}

\begin{styleStandard}
Firth, J. R. (1950), ‘Personality and language in society’, \textit{Sociological Review} 42, 37–52.\newline
(Repr. in Firth (1957), \textit{Papers in Linguistics, 1934-1951}, London \& New York: Oxford University Press, pp. 177–189.)
\end{styleStandard}

\begin{styleStandard}
Hjelmslev, Louis (1928), \textit{Principes de grammaire générale}, Copenhagen: Munksgaard.
\end{styleStandard}

\begin{styleStandard}
Hjelmslev, Louis (1935–1937), \textit{La catégorie des cas. Étude de grammaire générale}, Aarhus: Universitetsforlaget.
\end{styleStandard}

\begin{styleStandard}
Jakobson, Roman (1962 [1959]), ‘Why “mama” and “papa”?’, in \textit{Selected Writings, vol. I: Phonological studies}, The Hague: Mouton de Gruyter, pp. 538–545. \href{https://archive.org/details/selectedwritings01jako/page/538/mode/2up}{\textstyleInternetlink{archive.org}}
\end{styleStandard}

\begin{styleStandard}
Jakobson, Roman (1971 [1929]), ‘Retrospect’, in \textit{Selected Writings, vol. II: Word and language}, The Hague: Mouton de Gruyter, pp. 711–722. \href{https://archive.org/details/selectedwritings02jako/page/712/mode/2up}{\textstyleInternetlink{archive.org}}
\end{styleStandard}

\begin{styleStandard}
Meillet, Antoine (1921, 1936), \textit{Linguistique historique et linguistique générale}, 2 vols., Paris: Champion. Vol. I: \href{https://archive.org/details/linguistiquehist00meil}{\textstyleInternetlink{archive.org}}, Vol. II: \href{https://gallica.bnf.fr/ark:/12148/bpt6k936597h/f7.texteImage}{\textstyleInternetlink{BNF Gallica}}
\end{styleStandard}

\begin{styleStandard}
Ogden, Charles K. and Ivor A. Richards (1949 [1923]), \textit{The Meaning of Meaning: A study of the influence of language upon thought and the science of symbolism}, London: Routledge. \href{https://archive.org/details/TheMeaningOfMeaningC.K.OgdenAndI.A.Richards/page/n5/mode/2up}{\textstyleInternetlink{archive.org}}
\end{styleStandard}

\begin{styleStandard}
Saussure, Ferdinand de (1879), \textit{Mémoire sur le système primitif des voyelles dans langues indo-européennes}, Leipzig: B.G. Teubner. \href{https://archive.org/details/memoiresurlesyst00saus/page/n7/mode/2up}{\textstyleInternetlink{archive.org}}
\end{styleStandard}

\begin{styleStandard}
Saussure, Ferdinand de (1922 [1916]), \textit{Cours de linguistique générale}, ed. by Charles Bally and Albert Sechehaye, Paris: Payot. 3rd edition, 1931: \href{https://gallica.bnf.fr/ark:/12148/bpt6k314842j/f11.item}{\textstyleInternetlink{BNF Gallica}}\newline
(English translation: Ferdinand de Saussure, 1959 [1916], \textit{Course in General Linguistics}, trans. by Wade Baskin, New York: Philosophical Library. \href{https://archive.org/details/ferdinanddesaussurewadebaskintrans.courzlib.org/page/n1/mode/2up}{\textstyleInternetlink{2011 edition available from archive.org}})
\end{styleStandard}

\begin{styleStandard}
\textbf{Secondary Sources}
\end{styleStandard}

\begin{styleStandard}
Chomsky, Noam (2009 [1966]), \textit{Cartesian linguistics: A chapter in the history of rationalist thought}, ed. by James McGilvray, Cambridge: Cambridge University Press.
\end{styleStandard}

\begin{styleStandard}
Coseriu, Eugenio (1967), ‘Georg von der Gabelentz et la linguistique synchronique’, \textit{Word} 23, 74–110.
\end{styleStandard}

\begin{styleStandard}
Coseriu, Eugenio (1967), ‘Georg von der Gabelentz et la linguistique synchronique’, \textit{Word} 23: 74–110.
\end{styleStandard}

\begin{styleStandard}
Joseph, John E. (2012), \textit{Saussure}, Oxford: Oxford University Press.
\end{styleStandard}

\begin{styleStandard}
Joseph, John E. (2017), ‘Ferdinand de Saussure’, \textit{Oxford Research Encyclopedia of Linguistics}. DOI: \href{https://doi.org/10.1093/acrefore/9780199384655.013.385}{\textstyleInternetlink{10.1093/acrefore/9780199384655.013.385}}
\end{styleStandard}

\begin{styleStandard}
Joseph, John E. (2020), ‘Structure, mentalité, société, civilisation : les quatre linguistiques d’Antoine Meillet’, \textit{SHS Web of Conferences} 78, \url{https://www.shs-conferences.org/articles/shsconf/abs/2020/06/shsconf_cmlf2020_15002/shsconf_cmlf2020_15002.html}
\end{styleStandard}

\begin{styleStandard}
McElvenny, James (2017), ‘Georg von der Gabelentz’, \textit{Oxford Research Encyclopedia of Linguistics}. \textbf{DOI:}~\href{https://doi.org/10.1093/acrefore/9780199384655.013.379}{\textstyleInternetlink{10.1093/acrefore/9780199384655.013.379}}
\end{styleStandard}

\begin{styleStandard}
McElvenny, James (2018), \textit{Language and Meaning in the Age of Modernism: C. K. Ogden and his contemporaries}, Edinburgh: Edinburgh University Press.
\end{styleStandard}

\end{document}
