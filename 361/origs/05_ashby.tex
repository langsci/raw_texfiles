% This file was converted to LaTeX by Writer2LaTeX ver. 1.4
% see http://writer2latex.sourceforge.net for more info
\documentclass[12pt]{article}
\usepackage[utf8]{inputenc}
\usepackage[T1]{fontenc}
\usepackage[ngerman]{babel}
\usepackage{amsmath}
\usepackage{amssymb,amsfonts,textcomp}
\usepackage{array}
\usepackage{hhline}
\usepackage{hyperref}
\hypersetup{colorlinks=true, linkcolor=blue, citecolor=blue, filecolor=blue, urlcolor=blue}
% Text styles
\newcommand\textstyleInternetlink[1]{#1}
\raggedbottom
% Paragraph styles
\renewcommand\familydefault{\rmdefault}
\newenvironment{styleStandard}{\setlength\leftskip{0cm}\setlength\rightskip{0cm plus 1fil}\setlength\parindent{0cm}\setlength\parfillskip{0pt plus 1fil}\setlength\parskip{0in plus 1pt}\writerlistparindent\writerlistleftskip\leavevmode\normalfont\normalsize\writerlistlabel\ignorespaces}{\unskip\vspace{0in plus 1pt}\par}
% List styles
\newcommand\writerlistleftskip{}
\newcommand\writerlistparindent{}
\newcommand\writerlistlabel{}
\newcommand\writerlistremovelabel{\aftergroup\let\aftergroup\writerlistparindent\aftergroup\relax\aftergroup\let\aftergroup\writerlistlabel\aftergroup\relax}
\title{}
\author{James McElvenny}
\date{2022-06-14}
\begin{document}
\begin{styleStandard}
\textbf{Interview with Michael Ashby on the emergence of phonetics as a field}
\end{styleStandard}

\begin{styleStandard}
Michael Ashby
\end{styleStandard}

\begin{styleStandard}
\textit{University College London}
\end{styleStandard}

\begin{styleStandard}
James McElvenny
\end{styleStandard}

\begin{styleStandard}
\textit{University of Siegen}
\end{styleStandard}

\begin{styleStandard}
\textbf{JMc:} In this interview, we’re joined by phonetician and historian of linguistics Michael Ashby. Michael is a former Senior Lecturer in Phonetics at University College London, the current President of the International Phonetic Association, and the Treasurer of the Henry Sweet Society for the History of Linguistic Ideas. He’s going to talk to us about the history of phonetics from the 19\textsuperscript{th} century to the early 20\textsuperscript{th} century. 
\end{styleStandard}

\begin{styleStandard}
So, Michael, can you tell us about the beginnings of modern phonetic scholarship? When did the modern field of phonetics begin to emerge, and how did it fit in with the intellectual and academic landscape of the time? Was it primarily a pure field interested in the accumulation of knowledge for its own sake, or was it more applied, connected to language teaching, orthography reform and so on?
\end{styleStandard}

\begin{styleStandard}
\textbf{MA:} The 19\textsuperscript{th} century was when phonetics became clearly defined and got a name. It grew up at the intersection of linguistic science with two other fields. One of them is mathematics and physical science, chiefly acoustics, and the other, medical science, especially physiology. If we start with physiology, there had been over centuries an accumulating body of knowledge about the articulation of speech, but there were also many bizarre misconceptions. The 19\textsuperscript{th} century was when scientific medicine really got going, and it was only to be expected that physiologists would turn their attention to the speech organs, especially the larynx, and there were big steps in the early 19\textsuperscript{th} century. 
\end{styleStandard}

\begin{styleStandard}
A very significant event for linguists was the publication of von Brücke’s \textit{Grundzüge der Physiologie} in 1856, because von Brücke is the person who gets articulatory phonetics more or less right for the first time. For instance, he drew separate vocal tract diagrams illustrating the production of various sounds, just like those in a modern phonetics text. You could use them today. Well, his book was soon joined by others, and von Brücke himself went to a second edition later in the century. So long story cut short, but that’s the physiological background. 
\end{styleStandard}

\begin{styleStandard}
Turning to mathematics and acoustics, it’s a parallel story, really. Again, ancient antecedents, but rapid ground-breaking advances in the early 19\textsuperscript{th} century, new light thrown on vowel production, the nature of resonance, and in 1862, Hermann Helmholtz published his great work \textit{Die Lehre von den Tonempfindungen}. That’s to say, the science of sensations of tone. It’s a comprehensive work on sound, covering analysis, synthesis, hearing, and taking into account the sounds of speech. 
\end{styleStandard}

\begin{styleStandard}
Helmholtz was translated into English by Alexander Ellis, a pioneer phonetician who in his day was President of the Philological Society. So he brings us to the third component: linguistic science itself. It was linguists, really, who defined the scope of the subject and gave it a name. The noun ‘phonetics’ as the name for a field of study started to be used in the 1840s, and in the 1870s, two particularly significant and closely contemporary linguistic phoneticians came to the fore: in Germany, Eduard Sievers, and in Britain, Henry Sweet, and their major phonetics handbooks appeared in successive years: 1876, 1877. 
\end{styleStandard}

\begin{styleStandard}
You ask about pure or applied research. Well, as often I think it was both. Certainly, practical applications were never far away. The teaching of the deaf had been a goal for centuries. Von Brücke’s \textit{Physiologie} explicitly says in the title that it’s for linguists and teachers of the deaf. As for orthography reform, yes, many phoneticians were also advocates of spelling reform. Sweet’s 1877 \textit{Handbook of Phonetics} has a sizeable appendix devoted to the topic, and some phoneticians kept up this interest well into the 20\textsuperscript{th} century. As for the connection of phonetics with language teaching, that became particularly important in the last quarter of the 19\textsuperscript{th} century because of the Reform Movement. 
\end{styleStandard}

\begin{styleStandard}
An excellent contemporary view of the development of phonetics and its place in the intellectual and scientific climate of the time can be got from one of Max Müller’s \textit{Lectures on the Science of Language} delivered in 1863. It’s called ‘The Physiological Alphabet’. Müller identifies the same three contributing fields exactly as I did just now, so he’s read von Brücke and Helmholtz, and he knows the writings of Ellis, but it’s all new and exciting and unfolding around him at the point when he’s writing, and he’s interpreting it for the Royal Institution audience. It’s a brilliant piece and must have done a great deal to popularize the idea of phonetics in the mid-19\textsuperscript{th} century.
\end{styleStandard}

\begin{styleStandard}
\textbf{JMc:} So what role do you think advances in recording and other sound technology played in the development of phonetics as a science in the 19\textsuperscript{th} century?
\end{styleStandard}

\begin{styleStandard}
\textbf{MA:} Developments in technology did play a very significant role, though maybe not in the way your question might suggest, at least not at first, because the actual accumulation of archives of recorded language samples on any scale doesn’t begin until the early 20\textsuperscript{th} century. 
\end{styleStandard}

\begin{styleStandard}
The earliest device which picked up sound and did something with it was the phonautograph. It draws waveforms. It’s a primitive oscillograph. It was announced in 1859, and it was almost immediately put to use in speech research. People had wondered whether vowels were characterized by what we now call formants – that is, resonances determined by the vocal tract position – or by specific harmonics – that is, fixed characteristics of the voice at a given pitch. The Dutch physiologist Donders analysed some vowel waveforms and reached the correct conclusion that the quality of vowels is determined by what he called overtones with a characteristic frequency, and that’s what we’d now call formants. 
\end{styleStandard}

\begin{styleStandard}
The phonautograph draws pictures, but it can’t play the sounds back; that came in 1877, when Edison announced the phonograph. Now people were quick to see that if the microscopic phonograph groove could somehow be enlarged for examination, a great deal could be learnt about the speech signal. By July of the following year, two British engineers, Jenkin and Ewing, published a substantial report in which they described their method of enlarging the groove 400 times, and they subject the resulting waveforms to quantitative harmonic analysis. What they’re describing in 1878 just a few months after the invention of the phonograph is now the very basis of all work in acoustic analysis of speech, though now, of course, a computer performs all the calculations they had to do laboriously by hand.
\end{styleStandard}

\begin{styleStandard}
It’s not only sound recording devices, but other instruments and techniques began to be applied to speech. In 1872, a London dentist, James Oakley Coles, described the technique we now call palatography. He painted the upper surface of the mouth with a mixture of flour and gum, made a single articulation, and then used a mirror to look at the wipe-off pattern showing tongue contact. Others refined the technique; later it became more usual to use an artificial palate which could be removed for easier examination. Around the same time, 1876, the kymograph, which was a physiological recording device, was first applied to the study of dynamic speech movements. 
\end{styleStandard}

\begin{styleStandard}
Instruments became altogether more numerous, and in 1891, Rousselot submitted a ground-breaking dissertation using a whole battery of instruments together to investigate his own variety of French. It was widely regarded as epoch-making, and those who enthusiastically followed his lead explicitly say that they were participating in a paradigm shift. 
\end{styleStandard}

\begin{styleStandard}
From the 1890s onwards, therefore, there has been something of a division – Sweet’s word was ‘antagonism’ – between traditional linguistic ear phonetics on one side and laboratory-based experimental phonetics on the other. In my view, it is to a large degree a manufactured division, a manufactured antagonism, but that’s another story.
\end{styleStandard}

\begin{styleStandard}
\textbf{JMc:} What connections were there in the 19\textsuperscript{th} century and the early 20\textsuperscript{th} century between phonetic scholarship and linguistic theory in such areas as historical-comparative linguistics, the documentation of non-European languages, and general linguistics? Did phoneticians pay attention to work in these areas, and did linguists take note of advances in phonetic science in formulating their theories?
\end{styleStandard}

\begin{styleStandard}
\textbf{MA:} Just how and why phonetics matters is set out brilliantly in the first few lines of Sweet’s \textit{Handbook} of 1877. \ That’s where he famously describes phonetics, and this is a quote, as ‘the indispensable foundation of all study of language, whether theoretical or practical.’ The fact is that phonetics was absolutely central to the comparative-historical enterprise, which is after all founded on regular sound correspondences. As Sweet says, ‘Without phonetics,’ and this is another quote from him, ‘philology, whether comparative or historical, is mere mechanical enumeration of letter changes.’ 
\end{styleStandard}

\begin{styleStandard}
As the century went on, I think the importance of phonetics as the explanatory basis of language variation and change just grew and grew. If we go back to von Brücke’s \textit{Grundzüge der Physiologie}, yes, he was a physiologist, but it wasn’t that he wrote a physiology text which then just turned out to be useful to linguists. He knew several languages himself, he had an interest in linguistic theory, he had friends who were active in Indo-European linguistics. He deliberately set out to produce a physiology text to provide the basis for linguistic science.
\end{styleStandard}

\begin{styleStandard}
Similarly with Sievers later in the century. Sievers himself was a Neogrammarian. He even has an Indo-European sound law named after him, Sievers’ Law, and his phonetics manual is number one in a series devoted to Indo-European grammars. It was planned as the foundation of the whole thing. I think at the end of the century, the Neogrammarians’ phonetics reading list is just those two, von Brücke and Sievers. 
\end{styleStandard}

\begin{styleStandard}
Now, the role of phonetics in documenting unwritten languages is, again, something stressed in the opening lines of Sweet’s 1877 \textit{Handbook}. There were two interesting major efforts in the 19\textsuperscript{th} century in the direction of producing a universal notation system that would be suitable for dealing with unwritten languages. One is the Prix Volney, a prize essay series given in accordance with the terms of a \ bequest, where – to begin with, at least – the question posed by the committee of judges was precisely that of creating a universal alphabet. This produced a series of analyses and proposals from 1822 onwards. Now, the motivation for the Prix Volney is general linguistic inquiry into whether such an alphabet was feasible, and many of the answers are rather philosophical in character. 
\end{styleStandard}

\begin{styleStandard}
Another important impetus came from the Protestant missionary effort. Here, the focus is not on language documentation as an end in itself, but as a means to the spreading of Christianity and translation of the Gospel. In 1854, the so-called ‘Alphabetical Conferences’ were held in London. Actually, in modern terms, it was one conference. What was plural was sessions on three days within a week. They were organised by Christian Karl Bunsen, who was a Prussian diplomat and scholar living in London, and he invited a galaxy of leading scientists, scholars, and churchmen to a high-powered brainstorming session, really, on the question of developing the universal alphabet for missionary use. 
\end{styleStandard}

\begin{styleStandard}
Max Müller was one of those attending, and he presented his own candidate missionary alphabet, although it wasn’t adopted. Another participant was the Prussian linguist and Egyptologist Karl Richard Lepsius, who presented the first form of his Standard Alphabet. Eventually, a revised version of that alphabet was published in English with funding from the Church Missionary Society and did see fairly widespread use, especially in Africa, and it was adopted indeed by some general linguists – Whitney, for example. 
\end{styleStandard}

\begin{styleStandard}
The truth is, though, that a great deal of language documentation throughout the 19\textsuperscript{th} century and into the 20\textsuperscript{th} was done without a good phonetic foundation. It’s not so much the lack of a uniform notation that matters. It’s lack of practical phonetic training and awareness, so that observers just fail to notice important features of the languages they’re dealing with. 
\end{styleStandard}

\begin{styleStandard}
That’s coupled with prejudice, too, about what could and could not be likely features of languages. The most graphic example of that I can give you is Max Müller on clicks at the Alphabetical Conferences. Clicks are a problem if you’re making an alphabet. You don’t have enough letters left over to deal with them. What shall you do? Well, Müller’s solution was not to symbolize them, but to abolish them. After all, there are African languages nearby that haven’t got them, so they can’t be necessary. And they are barbarous noises. ‘Barbarous’ is the word he uses. So he seriously suggests that under the civilizing influence of the missionaries, speakers of the languages in question may be induced to give up the clicks. 
\end{styleStandard}

\begin{styleStandard}
\textbf{JMc:} Can you tell us about the founding of the International Phonetic Association? What was the impetus behind it, and what was the mission of the Association in its early years? How has this changed up to the present? I guess one of the most surprising things about the society is the nature of its journal. Since 1970, it’s had the very academic and matter-of-fact title \textit{Journal of the International Phonetic Association}, but prior to that it was called \textit{The Phonetic Teacher} and then \textit{Le Maître Phonétique}. Perhaps even more remarkable is the fact that up until 1970, everything in the journal was printed in the International Phonetic Alphabet. What happened in 1970, and what do these changes say about the evolving character of phonetics as a field?
\end{styleStandard}

\begin{styleStandard}
\textbf{MA:} Yes, well, while the question of a universal alphabet remained unresolved, there were by the 1870s very viable phonetic notations – at least for English and other major European languages – using Latin letters and in many ways very similar to the phonetic alphabet we use today. The IPA came into existence not from the desire to create a new notation so much, but from a movement to use this already existing type of phonetic notation in the teaching of modern languages. 
\end{styleStandard}

\begin{styleStandard}
IPA means two things: the International Phonetic Alphabet, yes, but also the International Phonetic Association. It was an association that came first in 1886, but it wasn’t actually called the International Phonetic Association until 1897. Before that, it was the Phonetic Teachers’ Association, and the original membership was just a handful of teachers of English in Paris. The driving force behind this group was a young teacher called Paul Passy. 
\end{styleStandard}

\begin{styleStandard}
They’d all been inspired by a new trend in language teaching, the one we call the Reform Movement, and that had been launched on its way in 1882, just previously, with a rousing manifesto by Wilhelm Viëtor. He called for a complete change of direction in language teaching, and he was quickly supported by leading figures such as Henry Sweet who had himself not long previously called for reform of what he termed the ‘wretched’ system of studying modern languages then in wide use. 
\end{styleStandard}

\begin{styleStandard}
Now, the use of phonetic transcription in teaching was an important plank of this new approach. The membership of the group snowballed, and members joined from around the world, most of them being schoolteachers. At the same time, leading linguists were members. Jespersen and Sweet had been members right from the beginning, and others joined. Interestingly, de Saussure joined in 1891, and he remained a member until his death. Now, while they were certainly interested in language teaching, figures like Sweet and Jespersen also had bigger concerns. 
\end{styleStandard}

\begin{styleStandard}
Right from the start, Jespersen tried to steer the Association in the direction of an international phonetic association, and he had Sweet’s support, but it took more than 10 years before the ordinary membership agreed to the change. The Association’s journal, \textit{Le Maître Phonétique}, which had begun as a sort of homely newsletter, started to include articles and reviews that were more theoretical and unlikely to be of any direct use to a language teacher in a school. 
\end{styleStandard}

\begin{styleStandard}
Over time, the Association’s aims and practices have evolved, and the constituency from which the membership is drawn has changed correspondingly. The teaching of modern languages went on being identified as one of the Association’s leading priorities well into the 20\textsuperscript{th} century, but it began to fade as the century went on, and if you look through today’s membership, you probably wouldn’t find any modern language schoolteachers at all. 
\end{styleStandard}

\begin{styleStandard}
And yes, as you say, from the beginning right up until 1970, everything in the journal was printed in phonetic script – not just the language samples meant for reading practice, but the editorial matter, book reviews, obituaries, even the Association’s financial reports. This is partly because many of the early supporters were also advocates of spelling reform, though the Association never did throw its weight behind any specific proposals for spelling reform in the way that it did ultimately formulate and promote its own phonetic alphabet. 
\end{styleStandard}

\begin{styleStandard}
By the mid-20\textsuperscript{th} century, the use of phonetic script in the journal had become as much a habit as anything else. It was an eccentricity in some people’s minds, and spelling reform, by this stage, I think, was a lost cause. My own view would be that it was a lost cause all along, but mid-20\textsuperscript{th} century, it was an eccentric affection to use phonetic script for everything, and in the late 1960s, the IPA’s governing council voted to drop the use of phonetic script in the journal and at the same time to change the title of the journal to \textit{Journal of the International Phonetic Association}. Those changes came into force in 1971, and that’s where we are today.
\end{styleStandard}

\begin{styleStandard}
\textbf{JMc:} With the use of phonetic script for writing articles in the journal, was it a phonemic transcription of the language that the article was written in, or was it a much narrower phonetic transcription representing the accent of the author of the article?
\end{styleStandard}

\begin{styleStandard}
\textbf{MA:} Well, I recommend you to have a look. It’s all kinds of things and many different languages. The most extraordinary thing ever published, I think, is an article reviewing a book on Spanish, but the article is written in Welsh, transcribed Welsh – and if you think you know French or German, reading it in an experimental transcription from the late 19\textsuperscript{th} century is great fun. So trying to make out what Viëtor is saying in transcription is a real test.
\end{styleStandard}

\begin{styleStandard}
It’s not quite true to say that it’s in transcription. I used the word ‘phonetic script’. I’m following what Mike MacMahon did. Most people who contributed were using phonetics as a kind of writing system. It’s not that they’re transcribing speech. They’re doing written language, but they’re using phonetic symbols rather than conventional orthography, so it’s mixed in with ordinary punctuation. Numbers are written just with numbers. If a student were to put the date as ‘2021’ in a transcription, it would be a mistake today, but they wrote numbers just using Arabic numerals. And they used quotes and italics and all kinds of devices of written language. They just didn’t use ordinary spelling. But different people tried out different transcriptions, and indeed some transcription systems were first trialled in the journal. People tried them out to see how they worked, see what kind of a reaction they got.
\end{styleStandard}

\begin{styleStandard}
\textbf{JMc:} So were the authors given free rein?
\end{styleStandard}

\begin{styleStandard}
\textbf{MA:} I think so, yes.
\end{styleStandard}

\begin{styleStandard}
\textbf{JMc:} So the editors of the journal didn’t try to standardise the use of the phonetic alphabet. 
\end{styleStandard}

\begin{styleStandard}
\textbf{MA:} They did not try to standardise. I’ve looked for evidence of that. Rousselot, you know, who I’ve mentioned as the originator of the idea of a phonetics laboratory, was allowed by Passy, who was the editor of the journal, to print an article in the journal that was not in phonetic script. Rousselot thought the IPA was wrong in this, and Passy allowed him this rare privilege of writing in ordinary orthography. There’s a bit by Scripture, who was an American, and yet the transcription looks suspiciously British, so there’s a case where I think maybe a British phonetician at UCL had possibly transcribed a bit of ordinary text that Scripture had submitted, but apart from that, no, people were given free rein. 
\end{styleStandard}

\begin{styleStandard}
And sometimes it’s Italian. Sometimes it’s Spanish. Sometimes it’s German. French was the official language of the Association until 1970 again. There were a few articles in French published even after that date, but suddenly the other languages disappeared.
\end{styleStandard}

\begin{styleStandard}
\textbf{JMc:} Thank you very much for your answers to those questions. That’s given us an excellent picture of phonetic study in the 19\textsuperscript{th} century and up into the beginning of the 20\textsuperscript{th} century.
\end{styleStandard}

\begin{styleStandard}
\textbf{MA:} Well, thank you, James. It’s been a pleasure.
\end{styleStandard}

\begin{styleStandard}
\textbf{References}
\end{styleStandard}

\begin{styleStandard}
\textbf{Primary Sources}
\end{styleStandard}

\begin{styleStandard}
Brücke, Ernst Wilhelm von (1856), \textit{Grundzüge der Physiologie und Systematik der Sprachlaute für Linguisten und Taubstummenlehrer}, Wien: Carl Gerold’s Sohn. 
\end{styleStandard}

\begin{styleStandard}
Bunsen, Christian Karl Josias (1854), \textit{Christianity and mankind: their beginnings and prospects}, London: Longman, Brown, Green, and Longmans. [Alphabetical conferences: pages 377–488] 
\end{styleStandard}

\begin{styleStandard}
Helmholtz, Hermann Ludwig Ferdinand von (1885), \textit{On the sensations of tone, as a physiological basis for the theory of music}, trans. by Alexander John Ellis, 2nd edn., London: Longmans, Green \& Co. 
\end{styleStandard}

\begin{styleStandard}
Lepsius, Richard (1863), \textit{Standard alphabet for reducing unwritten languages and foreign graphic systems to uniform orthography in European letters}, London: Williams \& Norgate. 
\end{styleStandard}

\begin{styleStandard}
Müller, Friedrich Max (1864), \textit{Lectures on the science of language delivered at the Royal Institution of Great Britain in February, March, April \& May 1863: Second series}, London: Longman. [“The physiological alphabet”: pages 103–175] 
\end{styleStandard}

\begin{styleStandard}
Rousselot, Pierre Jean (1891), \textit{Les modifications phonétiques du langage, étudiées dans le patois d’une famille de Cellefrouin (Charente)}, Paris: H. Welter. 
\end{styleStandard}

\begin{styleStandard}
Sievers, Eduard (1881), \textit{Grundzüge der Phonetik[202F?]: zur Einführung in das Studium der Lautlehre der indogermanischen Sprachen}, 2nd edn., Leipzig: Breitkopf und Härtel. 
\end{styleStandard}

\begin{styleStandard}
Sweet, Henry (1877), \textit{A handbook of phonetics}, Oxford: Clarendon Press. 
\end{styleStandard}

\begin{styleStandard}
\textbf{Secondary Sources}
\end{styleStandard}

\begin{styleStandard}
Ashby, Michael George (2016), \textit{Experimental phonetics in Britain, 1890–1940}, Oxford: Oxford DPhil. \href{https://ora.ox.ac.uk/objects/uuid:d8bbffae-8a4e-478e-ba65-0f5a5bbd66e1}{\textstyleInternetlink{Oxford University Research Archive}}
\end{styleStandard}

\begin{styleStandard}
Ashby, Michael \& Marija Tabain (2020), ‘Fifty years of JIPA’, \textit{Journal of the International Phonetic Association} 50.3, 445–448. DOI: \href{https://doi.org/10.1017/S0025100320000298}{\textstyleInternetlink{10.1017/S0025100320000298}}
\end{styleStandard}

\begin{styleStandard}
Kemp, J. A. (2006), ‘Phonetic transcription: history’, in Keith Brown \& Anne H. Anderson (eds.), \textit{Encyclopedia of language and linguistics}, 396–410, 2nd edn., Amsterdam: Elsevier.
\end{styleStandard}

\begin{styleStandard}
Kohler, Klaus (1981), ‘Three trends in phonetics: The development of phonetics as a discipline in Germany since the nineteenth-century’, in Ronald Eaton Asher \& Eugénie J. A Henderson (eds.), \textit{Towards a History of Phonetics: In Honour of David Abercrombie}, 161–178, Edinburgh: Edinburgh University Press.
\end{styleStandard}

\begin{styleStandard}
Leopold, Joan (1999), \textit{The Prix Volney. 1a, 1b}, Dordrecht, Boston (Ma): Kluwer Academic Publishers.
\end{styleStandard}

\begin{styleStandard}
Lepsius, Richard (1981), \textit{Standard alphabet for reducing unwritten languages and foreign graphic systems to uniform orthography in European letters} (Amsterdam Studies in [the] Theory and History of Linguistic Science v. 5), J. A. Kemp (ed.), [New ed. of the] 2nd, revised ed. (London, 1863), Amsterdam: Benjamins.
\end{styleStandard}

\begin{styleStandard}
MacMahon, M. K. C. (1986), ‘The International Phonetic Association: The first 100 years’, \textit{Journal of the International Phonetic Association}, 16.1, 30–38. DOI: \href{https://doi.org/10.1017/S002510030000308X}{\textstyleInternetlink{10.1017/S002510030000308X}}.
\end{styleStandard}

\begin{styleStandard}
McElvenny, James (2019), ‘Alternating sounds and the formal franchise in phonology’, \textit{Form and Formalism in Linguistics}, ed. James McElvenny, 35–58. Berlin: Language Science Press.
\end{styleStandard}

\end{document}
