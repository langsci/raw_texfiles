% This file was converted to LaTeX by Writer2LaTeX ver. 1.4
% see http://writer2latex.sourceforge.net for more info
\documentclass[12pt]{article}
\usepackage[utf8]{inputenc}
\usepackage[T1]{fontenc}
\usepackage[ngerman]{babel}
\usepackage{amsmath}
\usepackage{amssymb,amsfonts,textcomp}
\usepackage{array}
\usepackage{supertabular}
\usepackage{hhline}
\usepackage{hyperref}
\hypersetup{colorlinks=true, linkcolor=blue, citecolor=blue, filecolor=blue, urlcolor=blue}
\makeatletter
\newcommand\arraybslash{\let\\\@arraycr}
\makeatother
\raggedbottom
% Paragraph styles
\renewcommand\familydefault{\rmdefault}
\newenvironment{styleStandard}{\setlength\leftskip{0cm}\setlength\rightskip{0cm plus 1fil}\setlength\parindent{0cm}\setlength\parfillskip{0pt plus 1fil}\setlength\parskip{0in plus 1pt}\writerlistparindent\writerlistleftskip\leavevmode\normalfont\normalsize\writerlistlabel\ignorespaces}{\unskip\vspace{0in plus 1pt}\par}
% List styles
\newcommand\writerlistleftskip{}
\newcommand\writerlistparindent{}
\newcommand\writerlistlabel{}
\newcommand\writerlistremovelabel{\aftergroup\let\aftergroup\writerlistparindent\aftergroup\relax\aftergroup\let\aftergroup\writerlistlabel\aftergroup\relax}
\setlength\tabcolsep{1mm}
\renewcommand\arraystretch{1.3}
\title{}
\author{James McElvenny}
\date{2022-06-14}
\begin{document}
\begin{styleStandard}
\textbf{Interview with Floris Solleveld on disciplinary linguistics in the 19th century}
\end{styleStandard}

\begin{styleStandard}
Floris Solleveld
\end{styleStandard}

\begin{styleStandard}
\textit{University of Leuven}
\end{styleStandard}

\begin{styleStandard}
James McElvenny
\end{styleStandard}

\begin{styleStandard}
\textit{University of Siegen}
\end{styleStandard}

\begin{flushleft}
\tablefirsthead{}
\tablehead{}
\tabletail{}
\tablelasttail{}
\begin{supertabular}{m{0.24415985in}m{5.89766in}}
\textbf{JMc}\newline
 &
In this interview, we’re joined by Floris Solleveld from the University of Leuven, who’s going to give us an overview of how linguistics emerged as a discipline in the 19th century. \ 

So Floris, what was the character of language scholarship and the humanities more generally in the 19th century? We have already talked a little bit in this podcast about how 19th-century language scholars emphasized the novelty of what they were doing, that there were frequent proclamations of a revolution in the language sciences. You’ve examined this question yourself in quite a bit of detail. Do you think that there was a decisive break in the study of language and the human world in the 19th century, and could it be described as a scientific revolution?\\
\textbf{FS}\newline
 &
Hi, James. Thanks for having me here. Well, the question to what extent you can speak of a scientific revolution in the humanities is a question that I have pondered for some six years, and my general, unspectacular answer is: Kind of. A lot of things happened, a lot of things changed, around 1800. There is a lot of revolutionary rhetoric surrounding these changes, and whether you call it a scientific revolution depends on your theoretical perspective and on your personal preferences. 

But what happens in linguistics actually is quite dramatic. What you see is a breaking of paper trails, which is a good indication that somethingdrastic is happening, if people stop using work from a previous period, stop quoting from it. And that is what happens in 19th-century linguistics. They’re not using 18th-century work much anymore, and there is a staple of revolutionary rhetoric surrounding it. 

Friedrich Schlegel is the outstanding example. The man is a serial proclaimer of revolutions. Even as a student, he proclaims a revolution in the study of antiquity. Then he invents the Romantic movement, and then he proclaims an Oriental renaissance in \textit{Über die Sprache und Weisheit der Indier}. And most of his proclamations get picked up, although not exactly in the way that he intended them. That is, he is not the guy who founds modern classical philology. His Oriental renaissance turns out to become the basis of comparative linguistics rather than the basis of a spiritual rejuvenation of the West, but to get that instead is not a crass failure either. 

And if you look at that rhetoric in retrospect, which is what happens in the 19th century as the discipline develops, you see that people actually look back on it in those terms, as a revolution. 

But there is a bit of a grey area. For instance, the first person to actually speak of a scientific revolution in the study of language is Peter Stephen Du Ponceau. And what does he cite as an example? He doesn’t cite Schlegel. He cites Adelung, \textit{Mithridates}, which is the text that people now typically use to contrast the previous paradigm and new historical-comparative linguistics. But then Adelung was still used as a source of data, and that is remarkable: Adelung is basically the only or one of the few that are still used as a source of information after the beginning of the century.\\
\textbf{JMc}\newline
 &
Do you think even though there are all of these proclamations of revolutions and people are not citing their predecessors that this really represents a break in continuity between the way people were doing the study of language in the 19th century and their predecessors and also a break in the way that they thought about language, the philosophy of language and the philosophy of science that lies behind the discipline of linguistics?\\
\textbf{FS}\newline
 &
Yes, I do think so, and not just in having this historical-comparative perspective, which of course is very pre-eminent in 19th-century linguistics. There is also a break, for instance, in the realization that there are these different language families, each with their own character, or with the idea that you can actually analyse language structures in different ways, because these different language families have different organizational principles. And that is reflected in the way linguistic material is used, in the mapping of sound systems or the analysis of different ways of ordering particles. You already see Humboldt splitting up Polynesian languages morphologically in \textit{Über die Kawi-Sprache}. You already see Richard Lepsius drawing up diagrams of sound systems in the presentation of his phonetic alphabet, and that is the sort of analysis of language which really doesn’t happen in the 18th century. So yes, I do think that there is this drastic discontinuity. 

You also see that the term “linguistics” comes up in this period. Actually, the remarkable thing again is that the first people to use the term “linguistics” are late 18th-century German compilers who very much work within an early modern compilatory style, so in that regard you never really have a clean break. But then scientific revolutions aren’t like political revolutions where you storm the Bastille or the Winter Palace, you chop off the king’s head and you say it’s a revolution and nobody doubts it. 

With scientific revolutions, you always have this sort of unclarity about what the measure of a complete conceptual break should be. And this is one reason why there has been a lot of skepticism about the notion of scientific revolutions in the history of science, and why some people want to get rid of the phrase. Lorraine Daston and Katherine Park talked about getting rid of that “ringing three-word phrase.” Steven Shapin said that “There was no such thing as the Scientific Revolution, and this is a book about it.” 

And that sort of sums up the \textit{communis opinio} among historians of science. But in the history of scholarship, the question has been addressed far less. Within the humanities, I think the history of linguistics stands out for this sort of really radical conceptual break and break in ways in which material is organized and knowledge is being produced. For the humanities at large, my answer is more like “kind of”, maybe a qualified yes, but linguistics really is one of the strongest arguments in favour of that.\\
\textbf{JMc}\newline
 &
So would you say that accompanying the scientific revolution in linguistics there was a fundamental change in the sociological constitution of the field, and in scholarship more generally, in the 19th century? To describe the scholarly community up until the end of the 18th century, it’s usual to talk about the Republic of Letters. Do you think that this was superseded in the 19th century by clear-cut university-based disciplines, or do you think that there was continuity from this earlier idea of the Republic of Letters?\\
\textbf{FS}\newline
 &
The Republic of Letters is a container notion for the learned world, which perceives itself as an independent commonwealth, hence republic, \textit{res publica}, of letters. And “letters” here is an early modern term for learning at large; “letters” really means what it means in the name-shield of the Faculty of Letters. And three things actually hold that community together, which is (a) a correspondence network reinforced by learned journalism, (b) a symbolic economy, and (c) the sense of an academic community. Now, these things, these three aspects, they actually persist. We still perceive ourselves as part of an imagined community. We still correspond with each other. We still trade in information and prestige, and we don’t get rich, generally. So to that extent, that sort of infrastructure persists. 

Still the notion of Republic of Letters pretty much fades out from use in the early 19th century. I’ve traced that, and it is pretty much a sad story of how the term goes out of use. Some people try to reinvent it – doesn’t work. And there are very clear explanations for that. First of all, the notion of “republic” is appropriated by the French Revolution, and gets different connotations. The notion of “letters” changes, or “literature” becomes a term for literature as an art form instead for learning at large. We still speak of the literature in our field, and that is sort of a remnant of that early modern use. And also, people now address their peers, or they address a wider public, or in some cases they address the nation, and they don’t address the learned community in that sense anymore. 

So it didn’t make that much sense for 19th-century scholars to appeal to the Republic of Letters any more. But it did make sense, for instance, for late 17th-century Huguenot journalists who reinvented the notion, and it did make sense for the \textit{parti philosophique}, who appropriated – or rather, violently took over – the Republic of Letters in the mid-18th century. It made sense also for German academics who were trying to position themselves in the 18th century. 

But this idea of an amateur community being superseded by professionalism, that story has to be seriously qualified, because scholarship already is concentrated at universities in the German lands in the late 17th and 18th century. That is actually what gives the German-speaking countries an edge in the 19th century, because then it turns out that universities are a much more effective model for concentrating learning than they seem to be in the late early modern period, whereas what happens in the French- and English-speaking world is that this concentration of scholarship at universities goes a lot slower. 

It’s actually only in the second half of the 19th century, and especially after 1870, that this model really becomes so predominant that amateur or independent scholarship becomes the great exception. 1870, of course, in France, means the end of the Second Empire because they lose the Franco-Prussian War, and then the Second Empire becomes the Third Republic. In Britain, 1870 is not such a big break, but from the 1860s onward there is a huge wave of new university foundations, so-called red brick universities, and that really leads to a change in the academic landscape. There had been new university foundations before, King’s College, University College London, Durham University, but those were more like additions to the Oxbridge duopoly and the Scottish big four or big five. 

What happens with red brick universities is an intensification of academic research. If you look at the number of university staff and students from 1700 to 1850, it’s pretty constant. There are some serious interruptions when the Jesuit Order is banished or when the French Revolution closes all the universities or when half the German universities die in the period between 1795 and 1818, but on the whole, still, it’s pretty constant. From the second half of the 19th century onward, it expands exponentially. So yes, the notion of Republic of Letters goes out of use in the early 19th century, but no, it’s not as if there is this clean break from an amateur learned community to institutional professional scholarship within well-delineated disciplines. 

But I do want to add a footnote to that, because Ian McNeely recently wrote an article about Humboldt’s \textit{Über die Kawi-Sprache} as the last project of the Republic of Letters. He says that Humboldt then pieced his information together from all kind of previous language gathering exercises like Adelung, like Hervás y Panduro, like the British colonial administrators in Southeast Asia, particularly Marsden, who then fed all that information into Humboldt’s coffers - and then Humboldt, as a retired statesman and independent scholar, writes this big compendium which really still radiates the ghost of this imagined learned community. That is not untrue, but again, this is McNeely’s schematism: he thinks of the Republic of Letters as a sort of reified scholarly community rather than as a notion that you use strategically to present your own situation. 

If you look at how the languages of the world are mapped throughout the long nineteenth century, then quite a lot of these people actually are not university-based scholars. There is a process of institutionalization around historical-comparative linguistics. A small part of that is about linguistics proper and about Sanskrit, but a much larger part is about German studies, French studies, G\textit{ermanistik }and\textit{ Romanistik, }Slavonic studies a bit later, English studies, which are then informed by Indo-European comparative linguistics. But if you look at people who mapped the languages of India, the languages of Australia, the languages of Oceania, or the languages of the Americas, those are to a large part colonial administrators or people co-ordinating missionary networks. And those people do not operate any more within what they would describe as a Republic of Letters. George Grey in Cape Town and Auckland did not think of himself as a citizen of the Republic of Letters. George Grierson mapping the languages of India did not think of himself as a citizen of the Republic of Letters. Well, maybe Peter Stephen Du Ponceau in Philadelphia – who, after all, was born in the 18th century and who still basically thrives on this correspondence network – maybe he thought of himself as a citizen of the Republic of Letters.\\
\textbf{JMc}\newline
 &
But how did they think of themselves, and how were they seen by the newly emerging caste of professional linguists in universities? Was their work received in the centre of disciplinary linguistics, in Indo-European comparative linguistics? Did it feed into that, or were they just doing something separate that was still considered to be an amateur project?\\
\textbf{FS}\newline
 &
Well, no, what you see is that they do take on board professional expertise. George Grey, again, is the outstanding example, for what does he do when he becomes Governor of South Africa and sets forth his language-gathering project which he already had been doing in Adelaide and Auckland? He hires Wilhelm Bleek, a German philologist with a PhD – actually the first student to get his PhD on African languages – to organize his library and to put a stamp of scientific approval on what George Grey had been doing. 

You also see it with George Grierson, who writes – or co-ordinates – \textit{The Linguistic Survey of India} and who tries to avoid acquiring a strong institutional foothold – although he has affiliations – so as to retain some sort of independence. He hires an assistant, Sten Konow, who is university-based. He gets honorary doctorates, he goes to Orientalist congresses. 

Several of these people mapping the languages of the world get the Prix Volney. Peter Stephen Du Ponceau wins the Prix Volney. Sigismund Koelle wins the Prix Volney. Richard Lepsius, who later becomes a professor of Egyptology, gets the Prix Volney. So there is this sort of interaction between this broader ethnolinguistic project and the narrower discipline formation within linguistics, and you also see that some tools, especially phonetic alphabets, get developed within this broader network rather than within this narrow academic sphere. 

Indo-European historical-comparative linguistics is predominant because they have institutional firepower. If you look at who holds the chairs in Germany – where indeed there are chairs in these fields much earlier onward – it’s largely Sanskritists and Germanists. And if you look at the number of people who are actually engaged in this mapping of the languages of the world, the number of people involved in a secondary sense that they supply information for it runs in thousands, but the number of people who actually put together these collections and make comparative grammars and language atlases — that’s a dozen, two dozen. It’s really not such a big community.\\
\textbf{JMc}\newline
 &
Did this community of language scholars work largely in isolation from other fields that were developing at the time, or are there interactions between linguistics and other sciences such as ethnography, psychology, history?\\
\textbf{FS}\newline
 &
Well, one of the greatest interactions that you haven’t mentioned yet actually is with geography. One way of literally mapping the languages of the world is through language atlases, and the people who invent the language atlas are geographers. It’s Adriano Balbi working in Paris who also makes an \textit{Atlas ethnographique du monde} ( ethnographic atlas of the world), which is actually an overview of the languages of the world, and it’s Julius Klaproth, who is a self-taught Sinologist, who then turns to studying the languages of Asia and who also is a geographer, literally a map maker. In the Bibliothèque Nationale in Paris there are hundreds of his map designs.For Julius Klaproth, there really is this strong intersection between linguistics and geography. 

But ethnology is indeed the most direct sister of linguistics within this project of what I call the mapping of the world, because language is one of the clearest denominators of ethnic boundaries on a non-political level. Everyone who studied languages in the 19th century was aware that the overlap was not complete, that you can also learn a language if you are not part of that people, but generally, a people and the language community are overlapping unities. 

Of course, this notion of “people” was involved with all kinds of projections of their own, especially in German, \textit{Volk}, but for the sake of making distinctions between different peoples, it makes sense. If your aim is to know what the main differences are between peoples in a particular region and how we should relate to them, then language really is the most common denominator. What you also see is that — and this of course is one of the dark heritages of the 19th-century colonial project — this classification is then reinforced or formulated in terms of physical anthropology, in terms of theories of race. 

But one of the remarkable things here is that these scholars are aware that there are such things as miscegenation, both on a linguistic and on a racial level, and there also is actually far less consensus about racial classification than there is about linguistic classification. This is surprising, but people nowadays tend to talk about racial theory in the 19th century as if it is this one big dark thing, and it is pretty dark — I wouldn’t want to deny that — but it’s not one thing. There are something like half a dozen conflicting racial theories, and it is common knowledge that they are leaking on all sides. There are theories that simply divide people into different colours. Black, white, red, yellow, and maybe also brown. Or that divide them into different facial forms. Or that divide them by types of hair growth. That’s actually the most comical one. It’s Ernst Haeckel who comes up with it. He says that colour is an arbitrary standard because it changes depending on the climate. Physical proportions are a continuum. But the different hair types are discrete sets, so he divides people into those with sleek hair, and those with curly hair, and those with woolly hair.\\
\textbf{JMc}\newline
 &
And I believe that’s the basis of Friedrich Müller’s linguistic classification.\\
\textbf{FS}\newline
 &
Yes, so then you have these %
%Not sure if the Akkusativ plural {}-n also needs to be preserved in English quotation
%Floris Solleveld
%14. Juni 2022, 09:36
\textit{wollhaarigen}\textit{ Sprachen}, a classification which really doesn’t pass the giggle test.\\
\textbf{JMc} &
I guess also that, by the end of the 19th century, scholars who were trying to come up with rigorous scientific definitions for racial theory found that it didn’t stack up and abandoned eventually abandoned it.\\
\textbf{FS}\newline
 &
What you see indeed is that there is a growing awareness, at least within the scientific community, that these distinctions are somewhat arbitrary, but the practice still continues. Physical anthropology continues indeed until after World War II. What happens is that racial theory, because it is “natural science”, has this sort of appeal as a more rigid quantitative approach. The practice continues even after Franz Boas starts not only noticing that the categories leak, but actively gathering lots of anthropometric data with the express aim of showing that anthropometry is not the right way to quantify people. 

Another interesting example is Pater Wilhelm Schmidt, the man who basically represents Catholic ethnolinguistics, who writes an atlas of the world’s languages, devises the classification of Australian Aboriginal languages that still more or less holds today, and reorganizes the collections of the Propaganda Fide into the Vatican Missionary-Ethnological Museum. Schmidt is firmly convinced you should look at culture, not race, but he says you should do that because ethnology is a separate scientific discipline. Meanwhile he also keeps treating racial theory as a fully bona fide scientific approach. So there is this oddly funny – well, it depends on your sense of humour – there is this very paradoxical outcome that he writes a tract \textit{Rasse und Volk} in the 1920s, and then after the Nazis take over, he reformulates it into a longer tract: \textit{Rasse und Volk. Ihre allgemeine Bedeutung, ihre Geltung im deutschen Raum} (“Race and People: their General Meaning and their Significance in the German Area”). In spite of its title, this book gets banned by the Nazis because what Schmidt says about the meaning of racial theories is that they are irrelevant for understanding what a people is and what a language is. Obviously, Pater Wilhelm Schmidt is not my hero – let’s be clear about that – but he does represent a parting of the ways in this program.\\
\textbf{JMc}\newline
 &
Thanks very much, Floris, for talking to us about linguistic scholarship in the long nineteenth century.\\
\textbf{FS}\newline
 &
Thank you very much, James, for this service to the Republic of Letters.\\
\end{supertabular}
\end{flushleft}
\begin{styleStandard}
\textbf{References}
\end{styleStandard}

\begin{styleStandard}
\textbf{Primary Sources}
\end{styleStandard}

\begin{styleStandard}
Adelung, Johann Christoph and Johann Severin Vater (1806–1817), \textit{Mithridates, oder allgemeine Sprachenkunde}, Berlin: Vossische Buchhandlung. 
\end{styleStandard}

\begin{styleStandard}
Balbi, Adriano (1826), \textit{Atlas ethnographique du Globe}, Paris: Rey. 
\end{styleStandard}

\begin{styleStandard}
Bleek, Wilhelm (1858–1859), \textit{The library of His Excellency Sir George Grey K.C.B.: Philology}, London: Trübner \& co. 
\end{styleStandard}

\begin{styleStandard}
Boas, Franz (1940), \textit{Race, Language, and Culture}, New York: Macmillan.
\end{styleStandard}

\begin{styleStandard}
Du Ponceau, Peter Stephen (1838), \textit{Mémoire sur le système grammatical des langues de quelques nations Indiennes de l’Amerique du Nord}, Paris: Pihan de la Forest. 
\end{styleStandard}

\begin{styleStandard}
Grierson, George (1903–1926), \textit{The Linguistic Survey of India}, 11 in 20 vols, Calcutta: Govt. Printing House. 
\end{styleStandard}

\begin{styleStandard}
Haeckel, Ernst (1868), \textit{Natürliche Schöpfungsgeschichte: Gemeinverständliche wissenschaftliche Vorträge über die Entwickelungslehre im Allgemeinen und diejenige von Darwin, Goethe, und Lamarck im Besonderen} […], Berlin: Reimer. 
\end{styleStandard}

\begin{styleStandard}
Hervás y Panduro, Lorenzo (1787), \textit{Vocabolario poliglotto, con prolegomeni sopra più de CL lingue}, Cesena: Biasini. 
\end{styleStandard}

\begin{styleStandard}
Hervás y Panduro, Lorenzo (1787), \textit{Saggio practicco delle Lingue con prolegomeni e una raccolta di orazioni dominicali in più di trecento lingue e dialetti}, Cesena: Biasini. 
\end{styleStandard}

\begin{styleStandard}
Humboldt, Wilhelm von (1836), ‘Über die Verschiedenheit des menschlichen Sprachbaues’, \textit{Über die Kawi-Sprache auf der Insel Java}, vol. 1, ed. Alexander von Humboldt, Berlin: Dümmler. \newline
(English trans. \textit{On Language. The diversity of human language structure and ist influence on the mental development of mankind} [1988], trans. Peter Heath, Cambridge: Cambridge University Press.)
\end{styleStandard}

\begin{styleStandard}
Klaproth, Julius (1823), \textit{Asia polyglotta}, Paris: Schubart.
\end{styleStandard}

\begin{styleStandard}
Koelle, Sigismund (1854), \textit{Polyglotta Africana}, London: Church Missionary House.
\end{styleStandard}

\begin{styleStandard}
Lepsius, C. R. (1854), \textit{Das allgemeine linguistische Alphabet: Grundsätze der Übertragung fremder Schriftsysteme und bisher noch ungeschriebener Sprachen in europäische Buchstaben}, Berlin: Hertz. \newline
(Lepsius, C. R. (1863), \textit{Standard Alphabet for Reducing Unwritten Languages and Foreign Graphic Systems to a Uniform Orthography in European Letters}, 2nd rev. edn. London: Williams \& Norgate.)
\end{styleStandard}

\begin{styleStandard}
Marsden, William (1782), “Remarks on the Sumatran Languages. In a Letter to Sir Joseph Banks, Bart. President of the Royal Society”, \textit{Archaeologia} VI: 154-158.
\end{styleStandard}

\begin{styleStandard}
Marsden, William (1827), \textit{Bibliotheca Marsdeniana philologica et orientalis: A Catalogue of Books and Manuscripts, Collected with a View to the General Comparison of Languages, and to the Study of Oriental Literature}, London: Cox. 
\end{styleStandard}

\begin{styleStandard}
Müller, Friedrich (1876–1888), \textit{Grundriß der Sprachwissenschaft}, 4 vols. Vienna: Hölder.
\end{styleStandard}

\begin{styleStandard}
Raffles, Thomas Stamford (1830 [1817]), \textit{The History of Java}, 2 vols. London: Murray et al.
\end{styleStandard}

\begin{styleStandard}
Schlegel, Friedrich (1808), \textit{Ueber die Sprache und Weisheit der Indier}, Heidelberg: Mohr und Zimmer. (English trans. ‘On the Indian Language, Literature and Philosophy’ [1900], \textit{The Æsthetic and Miscellaneous Works of Friedrich von Schlegel}, ed. and trans. E. J. Millington, pp. 425–536, London: George Bell and Sons.)
\end{styleStandard}

\begin{styleStandard}
Schmidt, Wilhelm (1919), \textit{Die Gliederung der australischen Sprachen}, Vienna: Mechitaristen-Verlag.
\end{styleStandard}

\begin{styleStandard}
Schmidt, Wilhelm (1926), \textit{Die Sprachfamilien und Sprachenkreise der Erde}, 1+1 vols, Heidelberg: Winter.
\end{styleStandard}

\begin{styleStandard}
Schmidt, Wilhelm (1927), \textit{Rasse und Volk. Eine Untersuchung zur Bestimmung ihrer Grenzen und zur Erfassung ihrer Beziehungen}, München: Kösel \& Pustet. [2nd ed., 1935, \textit{Rasse und Volk. Ihre allgemeine Bedeutung, ihre Geltung im deutschen Raum}, Salzburg/Leipzig: Pustet.]
\end{styleStandard}

\begin{styleStandard}
\textbf{Secondary Sources}
\end{styleStandard}

\begin{styleStandard}
Alter, Stephen (1999), \textit{Darwinism and the Linguistic Image: Language, Race, and Natural Theology in the Nineteenth Century}, Baltimore: Johns Hopkins.
\end{styleStandard}

\begin{styleStandard}
Daston, Lorraine and Catherine Park (2006), “Introduction: The Age of the New”, in L, Daston \& C. Park (eds.), \textit{The Cambridge History of Science}, Vol. III: Early Modern Science, Cambridge: Cambridge University Press.
\end{styleStandard}

\begin{styleStandard}
Marchand, Suzanne (2003), “Priests among the Pygmies: Wilhelm Schmidt and the Counter-Reformation in Austrian Ethnology”, in H. G. Penny \& M. Bunzl (eds.), \textit{Worldly Provincialism: German Anthropology in the Age of Empire}, Ann Arbor: University of Michigan Press, pp. 283–316.
\end{styleStandard}

\begin{styleStandard}
McNeely, Ian (2020), “The Last Project of the Republic of Letters: Wilhelm von Humboldt’s Global Linguistics”, \textit{Journal of Modern History} 92: 241–273.
\end{styleStandard}

\begin{styleStandard}
Majeed, Javed (2018), \textit{Colonialism and Knowledge in Grierson’s Linguistic Survey of India}, London: Routledge.
\end{styleStandard}

\begin{styleStandard}
Messling, Markus (2016), \textit{Gebeugter Geist – Rassismus und Erkenntnis in der modernen europäischen Philologie}, Göttingen: Wallstein.
\end{styleStandard}

\begin{styleStandard}
Shapin, Steven (1996), \textit{The Scientific Revolution}, Chicago: University of Chicago Press.
\end{styleStandard}

\begin{styleStandard}
Solleveld, Floris (2016), “How to make a Revolution. Revolutionary Rhetoric in the Humanities around 1800”, \textit{History of Humanities} 1.2: 277–301.
\end{styleStandard}

\begin{styleStandard}
Solleveld, Floris (2019), “Language, People, and Maps: The Ethnolinguistics of George Grierson and Franz Boas” [review essay], \textit{History of Humanities} 4.2: 461-471.
\end{styleStandard}

\begin{styleStandard}
Solleveld, Floris (2020), “Lepsius as a Linguist: Fieldwork, Philology, Phonetics, and the ‘Hamitic Hypothesis'”, \textit{Language \& History} 63.3.
\end{styleStandard}

\begin{styleStandard}
Solleveld, Floris (2020), “Expanding the Comparative View: Humboldt’s Über die Kawi-Sprache and its Language Materials”, \textit{Historiographia Linguistica} 47.1: 52–82.
\end{styleStandard}

\begin{styleStandard}
Solleveld, Floris (2020), “Afterlives of the Republic of Letters. Learned journals and scholarly community in the early 19th century”, \textit{Erudition and the Republic of Letters} 5.1: 82-116.
\end{styleStandard}

\begin{styleStandard}
Solleveld, Floris (2020), “Klaproth, Balbi, and the Language Atlas”, in E. Aussant \& J.-M. Fortis (eds.), \textit{History of Linguistics 2017: Selected papers from the 14th International Conference on the History of the Language Sciences} (ICHoLS XIV), Amsterdam: John Benjamins, pp. 81–99.
\end{styleStandard}

\begin{styleStandard}
Solleveld, Floris (2020), “Language Gathering and Philological Expertise: Sigismund Koelle, Wilhelm Bleek, and the Languages of Africa”, in J. François (ed.), \textit{Les Linguistes allemands du XIXème siècle et leurs Interlocuteurs étrangers}, Paris: Éditions de la Société de Linguistique de Paris, pp. 169–200.
\end{styleStandard}

\end{document}
