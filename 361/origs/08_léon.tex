% This file was converted to LaTeX by Writer2LaTeX ver. 1.4
% see http://writer2latex.sourceforge.net for more info
\documentclass[12pt]{article}
\usepackage[utf8]{inputenc}
\usepackage[T1]{fontenc}
\usepackage[ngerman]{babel}
\usepackage{amsmath}
\usepackage{amssymb,amsfonts,textcomp}
\usepackage{array}
\usepackage{hhline}
\usepackage{hyperref}
\hypersetup{colorlinks=true, linkcolor=blue, citecolor=blue, filecolor=blue, urlcolor=blue}
\raggedbottom
% Paragraph styles
\renewcommand\familydefault{\rmdefault}
\newenvironment{styleStandard}{\setlength\leftskip{0cm}\setlength\rightskip{0cm plus 1fil}\setlength\parindent{0cm}\setlength\parfillskip{0pt plus 1fil}\setlength\parskip{0in plus 1pt}\writerlistparindent\writerlistleftskip\leavevmode\normalfont\normalsize\writerlistlabel\ignorespaces}{\unskip\vspace{0in plus 1pt}\par}
% List styles
\newcommand\writerlistleftskip{}
\newcommand\writerlistparindent{}
\newcommand\writerlistlabel{}
\newcommand\writerlistremovelabel{\aftergroup\let\aftergroup\writerlistparindent\aftergroup\relax\aftergroup\let\aftergroup\writerlistlabel\aftergroup\relax}
\title{}
\author{James McElvenny}
\date{2022-06-14}
\begin{document}
\begin{styleStandard}
\textbf{Interview with Jacqueline Léon on Firth, Malinowski, and the London School}
\end{styleStandard}

\begin{styleStandard}
Jacqueline Léon
\end{styleStandard}

\begin{styleStandard}
\textit{University of Paris, CNRS Histoire des theories linguistiques}
\end{styleStandard}

\begin{styleStandard}
James McElvenny
\end{styleStandard}

\begin{styleStandard}
\textit{University of Siegen}
\end{styleStandard}

\begin{styleStandard}
\textbf{JMc:} Today we explore the work of the London School of linguistics, whose institutional figurehead was John Rupert Firth, and which had many links outside disciplinary linguistics, perhaps most notably to the ethnographic work of Bronisław Malinowski. To take us through this topic, we’re joined by Jacqueline Léon, from the CNRS Laboratory for the History of Linguistic Theories in Paris. 
\end{styleStandard}

\begin{styleStandard}
A key concept for both Firth and Malinowski was the “context of situation”. You’ve argued, Jacqueline, that this concept represents a kind of anticipation of ideas that were later reinvented or rediscovered under the rubrics of ethnography of communication and conversation analysis. What exactly are the common points between Firthian linguistics and these later approaches? And are there direct historical connections between them or were the later ideas developed independently?
\end{styleStandard}

\begin{styleStandard}
\textbf{JL:} One can say that there is a direct connection between Firth and Malinowski’s ideas and ethnography of communication, since its pioneers, Dell Hymes and John Gumperz, consider Malinowski and Firth among the notable sources of the field. In his introductory book to ethnographic communication, \textit{Language in Culture and Society}, published in 1964, Hymes reproduces the second part of Firth’s text “The technique of semantics” of 1935 under the title of “Sociological linguistics”. Remember that, in that text, Firth starts to elaborate the notion of context of situation in the wake of Malinowski. In the same book, Hymes also reproduces a text by Malinowski of 1937 called “The Dilemma of Contemporary Linguistics”. 
\end{styleStandard}

\begin{styleStandard}
Later, in their introductory book \textit{Directions in Sociolinguistics, The Ethnography of Communication}, published in 1972, Hymes and Gumperz underline what dialectology and variation studies owe to Firth, in particular with the notions of context of situation, speech community, and verbal repertories, and how their notion of frame comes from the functional categories of the context of situation. They also claim their affiliation to Firth’s article “Personality and language in society”, published in 1950.
\end{styleStandard}

\begin{styleStandard}
As for conversation analysis, the connection is less direct: Sacks and Schegloff, the pioneers of conversation analysis, never quote Firth or Malinowski. However, they both refer to Hymes, and Sacks is one of the authors of \textit{Directions in Sociolinguistics}, edited by Hymes and Gumperz in 1972, so that one can claim that they were acquainted with Firth’s and Malinowski’s writings. 
\end{styleStandard}

\begin{styleStandard}
Now, let’s look into this in more detail, specifically Malinowski’s and Firth’s context of situation and their conception of language as a mode of action. In \textit{Coral Gardens and their Magic}, Malinowski’s context of situation includes not only linguistic context but also gestures, looks, facial expressions and perceptual context. More broadly, context of situation is identified with the cultural context comprising all the people participating in the activity, as well as the physical and social environment. In other words, context of situation is the nonverbal matrix of the speech event. Malinowski gives words the power to act, that is to say, long before Austin’s \textit{How to Do Things with Words} (published in 1955 then 1962). Malinowski says, “Words in their first and essential sense do, act, produce and realize.”
\end{styleStandard}

\begin{styleStandard}
As for Firth, as early as 1935, in “The technique of semantics”, he emphasizes the importance of conversation for the study of language. I quote: “Conversation is much more of a roughly prescribed ritual than most people think. Once someone speaks to you, you are in a relatively determined context and you are not free just to say what you please. […] Neither linguists nor psychologists have begun the study of conversation; but it is here that we shall find the key to a better understanding of what language really is and how it works.” 
\end{styleStandard}

\begin{styleStandard}
In this text, Firth presents a linguistic treatment of the context of situation. He groups the contexts by type of use, genres, and what was later called “register”. First: common, colloquial, slang, literary, technical, scientific, conversational, dialectal. Two: speaking, hearing, writing, reading. Three: familiar, colloquial, and more formal speech. Four: the languages of the schools, the law, the Church, and the specialized forms of speech. These categories become the basis of his notion of ‘restricted languages’, which he developed from 1945. 
\end{styleStandard}

\begin{styleStandard}
To these types of monological uses, Firth adds those created by the interactions between several people where the function of phatic communion identified by Malinowski is at work. The examples he gives are acts of ordinary conversation, such as addresses, greetings, mutual recognition, etc., or belong to institutions like the Church, the tribunal, administration, where words are deeds. I quote Firth again: “In more detail we may notice such common situations as:
\end{styleStandard}

\begin{styleStandard}
(a) Address: ‘Simpson!’ ‘Look here, Jones’, ‘My dear boy’, ‘Now, my man’, ‘Excuse me, madam’.
\end{styleStandard}

\begin{styleStandard}
(b) Greetings, farewells, or mutual recognition of status and relationship on contact, adjustment of relations after contact, breaking off relations, renewal of relations, change of relations.
\end{styleStandard}

\begin{styleStandard}
(c) Situations in which words, often conventionally fixed by law or custom, serve to bind people to a line of action or to free them from certain customary duties in order to impose others. In Churches, Law Courts, Offices, such situations are commonplace.” 
\end{styleStandard}

\begin{styleStandard}
However, the notion of situation, and the classification of these situations, seemed to him insufficient to account for language as action. Instead, he proposes linguistic functions reduced to linguistic expressions: he speaks of the languages of agreement, of disagreement, encouragement, approval, condemnation; the action of wishing, blessing, cursing, boasting; the language of challenge, flattery, seduction, compliments, blame, propaganda and persuasion. 
\end{styleStandard}

\begin{styleStandard}
Here we can recognize the first objects studied by the first conversation analysts in their research on talk-in-interaction, that is, greetings, compliments, agreement and disagreement, etc. In \textit{The Tongues of Men}, published in 1937, two years after “The technique of semantics”, appeared what was later formalized as turn-taking organization and action sequences by the conversation analysts. Firth evokes the mutual expectations aroused in the interlocutors as well as the limited range of possibilities of responses to a given turn. 
\end{styleStandard}

\begin{styleStandard}
As for the notions relating to language variation, which will prove to be very important for ethnographers of communication, they were developed by Firth from 1950. Firth already developed the notion of ‘specialized languages’ in his efforts to teach Japanese to British air force officers during the Second World War. These were subsets of the full language confined to certain domains; that is, the vocabulary, grammar and other constructions one would need to communicate in a specific situation. A few years later, this concept of specialized languages became restricted languages. For Firth, even restricted languages are affected by variation and context. Even in the restricted languages of weather or mathematics, which can nevertheless be regarded as extremely constrained, there are dramatic variations according to the cultures in which they are embedded and to the climates where they are used. 
\end{styleStandard}

\begin{styleStandard}
In Firth’s last paper, of 1959, we come across the idea of repertory, according to which each person is in command of a varied repertory of language roles, of a constellation of restricted languages. The notion of repertory was developed by ethnographers of communication as crucial for the study of variation.
\end{styleStandard}

\begin{styleStandard}
With this final paper, where restricted languages refer to speakers’ individual repertories, we could say that Firth gave the outline of the notion of register later developed by his followers, especially Michael Halliday, Angus McIntosh and Paul Strevens in their book \textit{The Linguistic Sciences and Language Teaching}, published in 1964. At first, they worked out the notion of register to address the issue of language variety in connection with foreign language teaching. Linguistic variety should be studied through two distinct notions, dialect and register, to account for linguistic events (Firth’s term to designate the linguistic activity of people in situations).
\end{styleStandard}

\begin{styleStandard}
They oppose dialect (that is, variety according to user: varieties in the sense that each speaker uses one variety and uses it all the time) to register (that is, variety according to use: in the sense that each speaker has a range of varieties and chooses between them at different times). The category of “register” refers to the type of language selected by a speaker as appropriate to different types of situations. Within this framework, restricted languages are referred as specific, constrained types of registers which, I quote, “employ only a limited number of formal items and patterns.”
\end{styleStandard}

\begin{styleStandard}
It should be added that the authors – that is, Halliday et al. – refer to Ferguson and Gumperz’s work on \textit{Linguistic diversity in South Asia}, Weinreich’s \textit{Languages in Contact} and Quirk’s \textit{Use of English}, in addition to Firth’s work, so that it should be said that registers had not been the direct successors of restricted languages. They have been established on Firthian views already revisited by Hymes and Gumperz, and then by Halliday and his colleagues.
\end{styleStandard}

\begin{styleStandard}
In conclusion, one can claim that Firth’s context of situation, linguistic events, restricted languages, and repertories raised crucial issues for early sociolinguistics.
\end{styleStandard}

\begin{styleStandard}
\textbf{JMc:} So Firthian linguistics would seem to have a very pragmatic and applied character. What’s the relationship of Firthian theory to what the British call “applied linguistics”? And how does this relate to the Firthian notion of “restricted languages”, which you just mentioned in your answer to the previous question?
\end{styleStandard}

\begin{styleStandard}
\textbf{JL:} To answer this question, I must recall that there is a specific tradition of applied linguistics coming from British empiricism, which, since the nineteenth century, has rested on the articulation between theory, practice and applications based on technological innovations. Firth played an important role in the development of practical and applied linguistics, which became institutionalized only after his death, in the 1950–60s, with two pioneering trends, in the US and in Britain. Michael Halliday, one of his most famous pupils, was one of the founders of the AILA, Association Internationale de Linguistique Appliquée, in 1964, and of BAAL, the British Association for Applied Linguistics, in 1967. 
\end{styleStandard}

\begin{styleStandard}
Henry Sweet was probably the nineteenth-century linguist who best exemplified the establishment of close links between linguistic theory and its application. Firth was a big admirer of Sweet (in particular, he mentions having learned his shorthand method at 14) and is in line with Sweet’s “living philology” in several ways: the priority given to phonetics in the description of languages, the attention paid to text and phonetics, the absence of distinction between practical grammar and theoretical grammar, the important place of descriptive grammar, and finally the involvement in language teaching. 
\end{styleStandard}

\begin{styleStandard}
In this last area, Sweet advocated the use of texts written in a simple and direct style, containing only frequent words, instead of learning lists of isolated words or sentences off by heart, which was the usual way of teaching languages in his time. These texts – which he called “connected coherent texts” – recall the restricted languages that Firth would recommend later for language teaching and also for all kinds of applications, such as translation and the study of collocations. 
\end{styleStandard}

\begin{styleStandard}
Firth developed restricted language in 1956 – in his article entitled “Descriptive linguistics and the study of English” – even if the idea of specialized language appeared as early as 1950. Firth’s major concern at the time was to set up the crucial status of descriptive linguistics, against Saussurian and Neo-Bloomfieldian structural linguistics. Restricted languages were a way to question the monosystemic view of language shared by European structuralists (especially Meillet’s view of language as a one-system whole \textit{où tout se tient}), and to criticize pointless discussions on metalanguage. Restricted languages are at the core of his conception of descriptive linguistics, where practical applications are guided by theory. Firth developed restricted languages according to three levels, “language under description”, “language of description”, “language of translation”, each of them determining a step in the description process.
\end{styleStandard}

\begin{styleStandard}
The language under description is the raw material observed, transcribed in the form of “text” located contextually. From a methodological point of view, restricted languages under description should be authentic texts – that is, written texts or the transcription of the raw empirical material. They may be materialized in a single text, such as Magna Carta in Medieval Latin, or the American Declaration of Independence. The language of description corresponds to linguistic terminology and transcription systems – we must know that Firth rejected the concept of metalanguage.
\end{styleStandard}

\begin{styleStandard}
Finally, the translation language includes the source and target languages, and the definition languages of dictionaries and grammars. Firth insists that restricted languages are more suited than general language to carrying out practical purposes, such as teaching languages, translating, or building dictionaries, and to study collocations, a major topic in his later work. Likewise, defined as limited types of a major language, for example subsets of English, contextually situated, they are the privileged object of descriptive linguistics. The task of descriptive linguistics, he said, is not to study the language as a whole, but to study restricted, more manageable languages, which should have their own grammar and dictionary, which he called micro-grammar and micro-glossary.
\end{styleStandard}

\begin{styleStandard}
Firth uses the phrase “the restricted language of X” in order to address the different types of restricted languages: the restricted language of science, technology, sport, defence, industry, aviation, military services, commerce, law and civil administration, politics, literature, etc.
\end{styleStandard}

\begin{styleStandard}
Firth died in 1960, the year of decolonization in Africa, also called “the year of Africa”. His last two texts are posthumous speeches at two congresses, organized respectively by the British Council and the Commonwealth on the teaching of English as a foreign language and as a second language in the former colonies. The research on restricted languages initiated by Firth is a central theme addressed in these lectures, under the title “English for special purposes”, and it is the Neo-Firthians, as his followers are sometimes called, including Michael Halliday, who took up these questions.
\end{styleStandard}

\begin{styleStandard}
\textbf{JMc:} Thank you very much for your very detailed answers to these questions.
\end{styleStandard}

\begin{styleStandard}
\textbf{JL:} Thank you.
\end{styleStandard}

\begin{styleStandard}
\textbf{References}
\end{styleStandard}

\begin{styleStandard}
\textbf{Primary Sources}
\end{styleStandard}

\begin{styleStandard}
Archives Firth : John Rupert Firth collection, PP MS75, School of Oriental and African Studies, London.
\end{styleStandard}

\begin{styleStandard}
Biber, Douglas 1988. \textit{Variation across Speech and Writing}. Cambridge : Cambridge University Press.
\end{styleStandard}

\begin{styleStandard}
Brown, Keith \& Law, Vivien (eds.) 2002. \textit{Linguistics in Britain: Personal Histories}, Oxford: Publications of the Philological Society.
\end{styleStandard}

\begin{styleStandard}
Firth, John Rupert 1930. \textit{Speech.} London: Benn’s Sixpenny Library.
\end{styleStandard}

\begin{styleStandard}
Firth, John Rupert 1957 [1935]. “The technique of semantics”, in \textit{Papers in Linguistics (1934–1951).} Oxford: Oxford University Press, pp. 7–33.
\end{styleStandard}

\begin{styleStandard}
Firth, John Rupert 1970 [1937]. \textit{The Tongues of Men}. London: Oxford University Press,
\end{styleStandard}

\begin{styleStandard}
Firth, John Rupert 1957 [1950]. “Personality and language in society”, in \textit{Papers in Linguistics (1934–1951)}. Oxford: Oxford University Press, pp. 177–189.
\end{styleStandard}

\begin{styleStandard}
Firth, John Rupert 1957 [1951]. “General Linguistics and Descriptive Grammar”, in\textit{ Papers in Linguistics (1934–1951), }pp. 216–228.
\end{styleStandard}

\begin{styleStandard}
Firth, John Rupert 1968 [1956]. “Descriptive linguistics and the study of English”, in Palmer F. R. (ed.), pp. 96–113.
\end{styleStandard}

\begin{styleStandard}
Firth, John Rupert 1968 [1957a].“Ethnographic analysis and language with reference to Malinowski’s views”, in Palmer F. R. (ed.), \textit{Selected papers of J.R. Firth (1952–59). }London and Bloomington: Longman and Indiana University Press, pp. 137–67.
\end{styleStandard}

\begin{styleStandard}
Firth, John Rupert [posthumous manuscript 1]. \textit{Conference~on University training and research in the use of English as a second / foreign language}, British Council 15–17 December 1960 [J. R. Firth collection, SOAS, London, PP MS 75, box 2]
\end{styleStandard}

\begin{styleStandard}
Firth, John Rupert [posthumus manuscript 2]. October 1960 \textit{Commonwealth Conference of the teaching of English as a second language}, Makerere, Uganda, January 1961 [J. R. Firth collection, SOAS, London, Personal File].
\end{styleStandard}

\begin{styleStandard}
Gumperz, John Joseph \& Hymes, Dell 1972. \textit{Directions in sociolinguistics, The Ethnography of Communication}. New York: Holt Rinehart \& Winston.
\end{styleStandard}

\begin{styleStandard}
Halliday, Michael Alexander Kirkwood 1966. “General linguistics and its application to language teaching”, in M. A. K. Halliday and A. McIntosh (eds.), \textit{Patterns of Language: Papers in General, Descriptive and Applied Linguistics}. London: Longman, pp. 1–41.
\end{styleStandard}

\begin{styleStandard}
Halliday, Michael Alexander Kirkwood, McIntosh, Angus \& Strevens, Paul 1964.\textit{ The Linguistic Sciences and Language Teaching}. London, Longmans.
\end{styleStandard}

\begin{styleStandard}
Hymes, Dell 1964. \textit{Language in Culture and Society}. \textit{A reader in Linguistics and Anthropology}. New York: Harper \& Row. 
\end{styleStandard}

\begin{styleStandard}
Malinowski, Bronislaw 1923. “The problem of meaning in primitive languages”, Supplement to Ogden C. K. \& Richards I. A., \textit{The Meaning of Meaning. A study of the influence of Language upon Thought and of the science of symbolism}. London: Routledge \& Kegan Paul, pp. 296–337.
\end{styleStandard}

\begin{styleStandard}
Malinowski, Bronislaw 1935. \textit{Coral Gardens and their Magic, The Language of Magic and Gardening, }vol. II. London: Allen \& Unwin.
\end{styleStandard}

\begin{styleStandard}
Malinowski, Bronislaw 1937. “The Dilemma of Contemporary Linguistics,” review of: \textit{Infant Speech: a Study of the Beginnings of Language}, by M. M. Lewis, \textit{Nature} 140: 172–173.
\end{styleStandard}

\begin{styleStandard}
Mitchell, Terence Frederick~1975. \textit{Principles of Firthian Linguistics}. London: Longman.
\end{styleStandard}

\begin{styleStandard}
Sacks, H., Schegloff, E.~\& Jefferson, G. 1974. “A simplest systematics from the organization of turn taking for conversation”, \textit{Language} 50: 696-735.
\end{styleStandard}

\begin{styleStandard}
Sweet, Henry 1891–98. \textit{A New English Grammar: Logical and Historical,} 2 vols. Oxford: Clarendon Press.
\end{styleStandard}

\begin{styleStandard}
Sweet, Henry 1891. \textit{The Practical Study of languages: A guide for teachers and learners}. Oxford: Oxford University Press. (Reprinted in 1964 by R. Mackin.)
\end{styleStandard}

\begin{styleStandard}
\textbf{Secondary Sources}
\end{styleStandard}

\begin{styleStandard}
Howatt, Anthony Philip Reid 1984. \textit{History of English Language Teaching} [2nd edition, 2004]. Oxford: Oxford University Press.
\end{styleStandard}

\begin{styleStandard}
Léon, Jacqueline 2007. “From Linguistic Events and Restricted Languages to Registers. Firthian legacy and Corpus Linguistics”, \textit{The Bulletin of the Henry Sweet Society for the History of Linguistic Ideas }49: 5–26.
\end{styleStandard}

\begin{styleStandard}
Léon, Jacqueline 2008. “Empirical traditions of computer-based methods. Firth’s restricted languages and Harris’ sublanguages”, \textit{Beiträge zur Geschichte der Sprachwissenschaft~}18.2: 259–274.
\end{styleStandard}

\begin{styleStandard}
Léon, Jacqueline 2011. “De la linguistique descriptive à la linguistique appliquée dans la tradition britannique. Sweet, Firth et Halliday”, \textit{Histoire Epistémologie Langage} 33\textit{.}1: 69–81.
\end{styleStandard}

\begin{styleStandard}
Léon, Jacqueline 2019. “Les sources britanniques de l’ethnographie de la communication et de l’analyse de conversation. Bronislaw Malinowski et John Rupert Firth”, \textit{Linha d’Agua} 32.1: 23–38.
\end{styleStandard}

\begin{styleStandard}
Palmer, Frank Robert 1994. “Firth and the London School”\textit{, The Encyclopedia of Language and Linguistics}, Asher (ed.). Oxford: Pergamon Press, pp. 1257–1260.
\end{styleStandard}

\begin{styleStandard}
Rebori, Victoria 2002. “The legacy of J.R. Firth. A report on recent research”, \textit{Historiographia} \textit{Linguistica} 29.1–2:165–190.
\end{styleStandard}

\begin{styleStandard}
Stubbs, Michael 1992. “InstitutionaI Linguistics: Language and Institutions, Linguistics and Sociology”, in Pütz, Martin (ed.), \textit{Thirty Years of Linguistic Evolution. }Amsterdam: Benjamins, pp. 189–21.
\end{styleStandard}

\end{document}
