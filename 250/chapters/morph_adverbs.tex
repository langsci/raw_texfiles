\chapter{Adverbs}
\label{cpt:morph-adverbs}

In this chapter, spatial \refsec{sec:spatialadverb}, temporal \refsec{sec:TemporalAdverbs}, manner \refsec{sec:MannerAdverbs}, and \is{degree adverb}degree adverbs \refsec{sec:Degree adverbs} are described as well as the productive formation of mostly manner adverbials by means of the suffix \textit{-le} \refsec{sec:FormationOfAdverbialsWithTheSuffixLe}. Adverbs form a rather heterogeneous group in Sanzhi and only certain subclasses of \is{spatial adverb}spatial adverbs and \isi{manner adverbs} have been derived by specialized adverbializing suffixes.


%%%%%%%%%%%%%%%%%%%%%%%%%%%%%%%%%%%%%%%%%%%%%%%%%%%%%%%%%%%%%%%%%%%%%%%%%%%%%%%%

\section{Spatial adverbs}
\label{sec:spatialadverb}


% --------------------------------------------------------------------------------------------------------------------------------------------------------------------------------------------------------------------- %

\subsection{Spatial adverbs derived from demonstrative pronouns}
\label{ssec:SpatialAdverbsDerivedFromDemonstrativePronouns}

Several series of \is{spatial adverb}spatial adverbs can productively be derived from \is{demonstrative pronoun}demonstrative pronouns. The major \isi{derivation} pattern is the suffixation of \textit{-tːu} to the stem of the pronouns (\reftab{tab:Spatial adverbs derived from demonstrative pronounsB}). The full list of the respective base pronouns is given in \refsec{sec:Demonstrative pronouns}. The meanings of the \is{spatial adverb}spatial adverbs are plainly based on the meaning of the base pronouns, showing that their semantics is organized along the meaning components of \is{demonstrative pronoun}demonstrative pronouns (\refsec{ssec:The vertical dimension: iC vs. heC vs. hiC and i(C)tːi vs. he(C)tːi vs. hi(C)tːi}, \refsec{ssec:Proximity, distance, cardinal directions, and height The six horizontal series}):   

\begin{itemize}
	\item	proximity to deictic center (i.e. speech act participants)
	\item	elevation in relation to deictic center
	\item	visibility, aformentionedness, familiarity, etc.
\end{itemize}
As for proximity, there is a three way distinction (see the adverbs in the first three lines of \reftab{tab:Spatial adverbs derived from demonstrative pronounsB}). Elevation distinguishes three meanings, of which `above' and `below' are expressed by dedicated stems (see the last three lines in \reftab{tab:Spatial adverbs derived from demonstrative pronounsB}) whereas all remaining adverbs are used when the meaning `level' is intended. The third meaning component is expressed via the distinction of the word-initial syllable (the three columns \tit{i(C)-} vs. \tit{he(C)-} vs. \tit{hi(C)-} in \reftab{tab:Spatial adverbs derived from demonstrative pronounsB}). The adverbs of the type \textit{he(C)tːu} given in the second column of the table are predominantly used when referring to the immediate geographical surroundings of the speaker (and addressee), when the conversation is about spatial reference points that have been mentioned before, are assumed to be known by the participants or are part of the personal sphere of the speaker \refex{ex:I went away from there carrying a bucket of fuel}, \refex{ex:The dog was there}. In contrast, the \textit{i(C)tːu} adverbs in the first column are commonly used when new spatial reference points are introduced or when talking about reference points whose location is unknown or irrelevant \refex{ex:There, here (her husband) sends her everywhere}, \refex{ex:There is nobody here without a drink in the hands}. The adverbs of the \tit{hi(C)tːu} type given in the third column occur only seldom in my corpus so that I am not able to make any generalizations about their meaning. Note also that there are two series of adverbs with the identical meaning, being formally differentiated only by the stem consonant (\textit{x vs. k'}). The adverbs containing \textit{x} are far more frequently used than the adverbs with \textit{k'}, which might even represent code switching to another Dargwa dialect.
%


\begin{table}
	\caption{Spatial adverbs derived from demonstrative pronouns}
	\label{tab:Spatial adverbs derived from demonstrative pronounsB}
	\small
	\begin{tabular}{llll}
		\lsptoprule
		\multicolumn{1}{l}{\tit{i(C)tːu}}	&	\multicolumn{1}{l}{\tit{he(C)tːu}}	&	\multicolumn{1}{l}{\tit{hi(C)tːu}}	&	translation\\
		\midrule
		iš-tːu 		&	heš-tːu	&	hiš-tːu 	&	\sqt{here, close to the speaker}\\
		il-tːu		&	hel-tːu	&	hil-tːu		&	\sqt{there, away from the speaker and\slash or close to the hearer}\\
		i-tːu 		&	he-tːu 	&	hi-tːu		&	\sqt{there, further away, unspecific distance}\\
		ik'-tːu		&	hek'-tːu	&	hik'-tːu	&	\sqt{here/there above the deictic center}\\
		ix-tːu		&	hex-tːu	&	hix-tːu	&	\sqt{here/there above the deictic center}\\
		iχ-tːu		&	heχ-tːu	&	hiχ-tːu	&	\sqt{here/there below the deictic center}\\
		\lspbottomrule
	\end{tabular}
\end{table}

\begin{exe}

	\ex	\tnm{[talking about a gasoline station]}	\label{ex:I went away from there carrying a bucket of fuel}\\
	\gll	ag-ur-re	heltːu-rka,	badra	salaˁrk'a-la	k-aqː-ib=da  \\
		go.\textsc{pfv}-\textsc{pret}-\textsc{cvb}	there-\textsc{abl}	bucket	fuel-\textsc{gen}	\tsc{down}-carry-\textsc{pret}=1\\
	\glt	\sqt{I went away from there carrying a bucket of fuel.}

	\ex	\tnm{[referring to a dog that is visible to the participants of the conversation]}	\label{ex:The dog was there}\\
	\gll	heχ-tːu-b	χe-b=de	χʷe\\
		\textsc{dem.down}-\textsc{loc}-\textsc{n}	exist.\textsc{down-n}=\textsc{pst}	dog\\
	\glt	\sqt{The dog was there.}

\ex	\label{ex:There, here (her husband) sends her everywhere}
	\gll	itːu=ra		ištːu=ra	lubuj	musːa	r-at-iʁ-ul ca-r  \\
		there=\textsc{add}	here=\textsc{add}	any	place	\textsc{f}-send-come.\textsc{pfv}-\textsc{icvb} \textsc{cop-f}\\
	\glt	\sqt{There, here (her husband) sends her everywhere.}
	
	\ex	\tnm{[referring to people depicted on cards]}	\label{ex:There is nobody here without a drink in the hands}\\
	\gll	deč	kʷi-d	akːʷ-ar	ča-k'al	χe-w-akːu	iχ-tːu-w\\ 
	 	drinking	  in.the.hands-\textsc{npl}   	\textsc{cop.neg}-\textsc{prs.3}	 who-\textsc{indef}	exist.\textsc{down-m}-\tsc{cop.neg}	 \tsc{dem.down}\tsc{-loc-m}\\
	\glt	\sqt{There is nobody there without a drink in the hands.}
\end{exe}

A second series of \is{spatial adverb}spatial adverbs denoting the source is derived by means of the suffix \textit{-ka} \refex{ex:The speaker complains that staying in the hospital} (\reftab{tab:Spatial adverbs denoting the source}). This suffix is probably a cognate of the second part of the complex \isi{ablative} suffix \textit{-r-ka} (\refsec{sec:nouncase}). These adverbs can also have a temporal interpretation \sqt{from time X on, after time X} in addition to the spatial meaning \refex{ex:After this Akhad also met my son in front of the house of Kachar}. As can be seen in the table, the adverbs in the first two lines have the same meaning because the base pronouns are synonyms.

\begin{table}
	\caption{Spatial adverbs denoting the source}
	\label{tab:Spatial adverbs denoting the source}
	\small
	\begin{tabular}{%
		>{\raggedright\arraybackslash}p{150pt}
		>{\itshape}l
		>{\itshape}l
		>{\itshape}l
		l}
		
		\lsptoprule
		base meaning		&	\multicolumn{1}{l}{\tit{heC-ka}}	&	\multicolumn{1}{l}{\tit{iC-ka}}	&	\multicolumn{1}{l}{\tit{hiC-ka}} &	translation\\
		\midrule
		close to the speaker	&	hež-ka		&	iž-ka		&	hiž-ka		&	\sqt{from here}\\
		close to the speaker	&	hej-ka			&	ij-ka		&	hij-ka		&	\sqt{from here}\\
		there, away from the speaker and\slash  	&	hel-ka			&	il-ka		&	hil-ka		&	\sqt{from there}\\
		~~or close to the hearer\\
		further away,	 unspecific distance &	het-ka			&	it-ka		&	hit-ka		&	\sqt{from there}\\
%		~~distance\\
		above the deictic center 	&	hek-ka		&	ik-ka		&	hik-ka		&	\sqt{from above}\\
		below the deictic center	&	heχ-ka		&	iχ-ka		&	hiχ-ka		&	\sqt{from below}\\
		\lspbottomrule
	\end{tabular}
\end{table}

\begin{exe}
	\ex	\label{ex:The speaker complains that staying in the hospital}
	\gll	kat=q'ar	ka-r-ilsː-a-di		ij-ka=ra	kːancːupːe	hej-ka=ra	kːancːupːe	iχ-ka=ra\\
		down=\textsc{mod}	\textsc{down-f}-lay.\textsc{ipfv}-\textsc{hab.pst}-1	this-\textsc{abl}=\textsc{add}	ladder	this-\textsc{abl}=\textsc{add} ladder	\textsc{dem.down}-\textsc{abl}=\textsc{add}\\
	\glt	\sqt{As for lying, I lay, but (there are) stairs from here, stairs from here, and also from there.} (The speaker complains that staying in the hospital is difficult for her because in order to go to the toilet she has to take the stairs)

	\ex	\label{ex:After this Akhad also met my son in front of the house of Kachar}
	\gll	hej-ka	ʡaˁħaˁd=ra	suk ∅-ič-ib ca-w	qːačaʁ-la	qal-sa-w di-la	durħuˁ=ra\\
		this-\textsc{abl}	Ahad=\textsc{add}	meet \textsc{m}-occur.\textsc{pfv}-\textsc{pret} \textsc{cop-m}	bandit-\textsc{gen}	house-\textsc{ante}-\textsc{m}	\textsc{1sg-gen}	boy=\textsc{add}\\
	\glt	\sqt{After this Ahad also met my son in front of the house of Kachar (lit. `bandit')).} (Kachar is the nickname of a man)
\end{exe}

Both series of adverbs can be inflected for the directional cases in the same way as nominals are inflected, but since the adverbs denoting source already express movement, they cannot take the essive case (\reftab{tab:Inflectional paradigms of two spatial adverbs}). The \isi{ablative} of the pronouns in this table can also express temporal meaning, for instance \textit{heltːu-rka} (there-\textsc{abl}) \sqt{then}.

\begin{table}
	\caption{Inflectional paradigms of two spatial adverbs}
	\label{tab:Inflectional paradigms of two spatial adverbs}
	\small
	\begin{tabularx}{0.52\textwidth}[]{%
		>{\raggedright\arraybackslash}p{46pt}
		>{\raggedright\arraybackslash\itshape}X
		>{\raggedright\arraybackslash\itshape}X}
		
		\lsptoprule
		{}		&	\upshape\sqt{here}	&	\upshape\sqt{from here}\\
		\midrule
		essive		&	heštːu-b			&	\tmd\\
		lative		&	heštːu				&	helka\\
		\isit{ablative}	&	heštːu-r(ka)			&	helka-r(ka)\\
		directional	&	heštːu-b-a			&	helka-b-a\\
		\lspbottomrule
	\end{tabularx}
\end{table}

A third series of \is{spatial adverb}spatial adverbs has the meaning \sqt{from X to X}. It is formed by means of the complex suffix \textit{-k-itːu-b-a} \refex{ex:heC-kitːu-gm-a directional}. The suffix is a combination of the \isi{ablative} \textit{-ka} (shortened to \textit{-k}), the locational suffix \textit{-tːu} and the directional marker \textit{-\tsc{gm}-a} (\refsec{sssec:Directional -gm-a}). The last suffix is, in principle, optional, although there are no examples without it in my corpus. According to Sanzhi speakers, the resulting complex adverbs are actually a short variant of combining the adverbs in \reftab{tab:Spatial adverbs denoting the source} with the adverbs in \reftab{tab:Spatial adverbs derived from demonstrative pronounsB}, for example \textit{hetka} + \textit{hetːuba} > \textit{hetkitːuba}. However, the suffix as a whole can also be added to other nominal bases such as personal pronouns, common \isi{nouns} or personal names if they are inflected for the \textsc{loc}-\isi{ablative} case first, such as \textit{nušːa-le-r-kitːu-b-a} (1\textsc{pl}-\textsc{loc}-\textsc{abl}-\textsc{advz}-\textsc{n}-\textsc{dir}) \sqt{from us further away}, \tit{uškul-le-r-kitːu-b-a} (school\tsc{-loc-abl-advz-n-dir}) \sqt{from the school further away}. The series is also available from the other two pronominal stems \textit{iC} and \textit{hiC}, but in my corpus there are only examples of the adverbs from the \textit{heC}-pronouns given in \refexrange{ex:heC-kitːu-gm-a directional}{ex:heC-kitːu-rka ablative}, \refex{ex:if our sons go from here to there to Nizhnekamensk}, \refex{ex:Through my garden up down they brought him to Izberbash}.

\begin{exe}
	\TabPositions{6.5em}
	\ex	\tit{heC-kitːu-\tsc{gm}-a} (directional) \label{ex:heC-kitːu-gm-a directional}
	\begin{xlist}
		\ex	\tit{hež-kitːu-b-a}		\tab	\sqt{from here (= place of speaker) to there}
		\ex	\tit{hej-kitːu-b-a}		\tab	\sqt{from here (= place of speaker) to there}
		\ex	\tit{hel-kitːu-b-a}		\tab	\sqt{from there (= place of the addressee) to there}
		\ex	\tit{het-kitːu-b-a}		\tab	\sqt{from there (= unspecific place) to there}
		\ex	\tit{hek-kitːu-b-a}		\tab	\sqt{from above to there}
		\ex	\tit{heχ-kitːu-b-a}		\tab	\sqt{from down to there}
	\end{xlist}

	\ex	\tit{heC-kitːu-rka} (\isi{ablative}) \label{ex:heC-kitːu-rka ablative}
	\begin{xlist}
		\ex	\tit{hež-kitːu-rka}		\tab	\sqt{from here to there, past, by}
		\ex	\tit{hej-kitːu-rka}		\tab	\sqt{from here to there, past, by}
		\ex	\tit{hel-kitːu-rka}		\tab	\sqt{from there to there, past, by}
		\ex	\tit{het-kitːu-rka}		\tab	\sqt{from there to there, past, by}
		\ex	\tit{hek-kitːu-rka}		\tab	\sqt{from above to there, past, by}
		\ex	\tit{heχ-kitːu-rka}		\tab	\sqt{from down to there, past, by}
	\end{xlist}

	\ex	\label{ex:if our sons go from here to there to Nizhnekamensk}
	\gll	nišːa-la	durħ-ne	hejkitːu-b-a	Nižnekamsk-le	b-uq'-aˁn-ne, \ldots\\
		1\textsc{pl}-\textsc{gen}	boy-\textsc{pl}	from.here.to.there-\textsc{hpl}-\textsc{dir}	Nizhnekamensk-\textsc{loc}	\textsc{hpl}-go-\textsc{ptcp}-\textsc{prs}.3\\
	\glt	\sqt{if our sons go from here to there to Nizhnekamensk, \ldots}
\end{exe}

Finally, there is a \isi{spatial adverb} \textit{itille} \sqt{further, to the side, sideways} that seems to be the pronoun \textit{it} inflected for the locational suffix \textit{-le} \refex{ex:At that side of the village there is a place}.

\begin{exe}
	\ex	\label{ex:At that side of the village there is a place}
	\gll	nišːa-la	šːi-la	itille-b	musːa	te-b \\
		1\textsc{pl}-\textsc{gen}	village-\textsc{gen}	further-\textsc{n}	place	exist-\textsc{n}\\
	\glt	\sqt{At that side of the village there is a place.}
\end{exe}


% --------------------------------------------------------------------------------------------------------------------------------------------------------------------------------------------------------------------- %

\subsection{Spatial adverbs related to postpositions}
\label{ssec:SpatialAdverbsDerivedFromPostpositions}

All \is{spatial postposition}spatial postpositions discussed in \refsec{sec:Spatialpostpositions} can also be used adverbially without a dependent \isi{noun phrase} \refex{ex:adjectivesWithCHIB}. Some of them have not only spatial, but also temporal semantics. They can inflect for all \is{spatial case}spatial cases expressing direction\slash movement (essive, lative, \isi{ablative}, directional). A few examples are provided in \refex{ex:Now the fox says to the wolf you sit down in front}, \refex{ex:The village of Anklukh is pretty high up}.

\begin{exe}
	\ex	\label{ex:adjectivesWithCHIB}
		\TabPositions{12em}
		\textit{hitːi} \sqt{after, behind}			\tab	\textit{sar} \sqt{in front, before, in earlier times} \\
		\textit{hila }\sqt{behind, after}	\tab	\textit{sala} \sqt{in front, before, forward} \\
		\textit{gu} \sqt{down, low, before}		\tab	\textit{xːar(i)} `to the bottom, down(wards)'\\
		\textit{či} \sqt{up, above}		\tab		\textit{qari} \sqt{at/on the top} \\
		\textit{b-i} \sqt{inside}			\tab		\textit{urkːa} \sqt{within, in the middle} \\
		\textit{tːura} \sqt{outside}		\tab		\textit{šːule} `at side, to the side, next to, sidelong'

	\ex	\label{ex:Now the fox says to the wolf you sit down in front}
	\gll	``u	sala	ka-b-iž-e,''		bec'-li-cːe	``du	hila	ka-b-irg-an=da!''	b-ik'-ul ca-b	kːurtːa  \\ 
		2\textsc{sg}	front	\textsc{down-n}-be.\textsc{pfv}-\textsc{imp}	wolf-\textsc{obl}-\textsc{in} 1\textsc{sg} behind \textsc{down-n}-be.\textsc{ipfv}-\textsc{ptcp}=1	\textsc{n}-say.\textsc{ipfv}-\textsc{icvb} \textsc{cop-n}	fox\\
	\glt	\sqt{\dqt{Now,} the fox says to the wolf, \dqt{you sit down in front, and I behind!}}

	\ex	\label{ex:The village of Anklukh is pretty high up}
	\gll	ank'luʁi-la	šːi	ʡaˁħ-le	qari-b=q'al\\
		Anklukh-\textsc{gen}	village	good-\textsc{advz}	up-\textsc{n}=\textsc{mod}\\
	\glt	\sqt{The village of Anklukh is pretty high up.}
\end{exe}

There are four \is{spatial adverb}spatial adverbs that have been derived from \is{spatial postposition}spatial postpositions by means of suffixing \textit{-tːi} to the root: \textit{gu-tːi} \sqt{along downside, at the lower side} (< \textit{gu} \sqt{down, under}), \textit{či-tːi} \sqt{along upside, at the upper side} (<\textit{či} \sqt{on}), \textit{sa-tːi} \sqt{at\slash along the front, as soon as} (< \textit{sa} \sqt{in front, ago}), and \textit{b-i-tːi} \sqt{inside, through} (< \textit{b-i} \sqt{in, inside}) \refex{ex:Through my garden up down they brought him to Izberbash}.

\begin{exe}
	\ex	\label{ex:Through my garden up down they brought him to Izberbash}
	\gll	di-la	qu-la	hetkitːu-w-a,	čitːi	gutːi	∅-iχ-ub-le	w-erč-ib ca-w	Izbir-re\\
		1\textsc{sg-gen}	garden-\textsc{gen}	from.there.to.there-\textsc{m-dir}	along.up along.downside	\textsc{m}-be.\textsc{pfv-pret-cvb}	\textsc{m}-lead.\textsc{pfv}-\textsc{pret} \textsc{cop-m}	Izberbash-\textsc{loc}\\
	\glt	\sqt{Through my garden, up, down, they brought him to Izberbash.}
\end{exe}

There are few more adverbs based on the adverbs/postpositions, namely \textit{hitːille} \sqt{on the back, later} (< \textit{hitːi}), \textit{b-atːura} \sqt{from inside} (< \textit{tːura}), and \textit{qaršːa} \sqt{upper side (of the village)} (< \textit{qar} `at/on the top' plus the \textsc{loc}-form of the noun \textit{šːi} \sqt{village}, which is \textit{šːa}).


% --------------------------------------------------------------------------------------------------------------------------------------------------------------------------------------------------------------------- %

\subsection{Other spatial adverbs}
\label{ssec:OtherSpatialAdverbs}

Sanzhi has some more \is{spatial adverb}spatial adverbs of which the most important ones are given in \refex{ex:adjectivesWithKAt}. A few of them are formed by means of the adverbializing suffix \textit{-le} (\refsec{ssec:The adverbializer -le}). For \is{spatial adverb}spatial adverbs that have the meaning of indefinite pro-forms see \refsec{sec:Indefinite pronouns}.

\begin{exe}
	\ex	\label{ex:adjectivesWithKAt}
		\TabPositions{14em}
		\textit{kat'} \sqt{down}					\tab	\textit{bet}	\sqt{there} \\
		\textit{šːulum} \sqt{by, past} (< \textit{šːal} `side')			\tab	\textit{sat} \sqt{here} \\
		\textit{alaw} \sqt{around, in a circle}			\tab	\textit{bet-sat} \sqt{here and there} \\
		\textit{qili} \sqt{at home} (< \textit{qal} \sqt{house})  \tab \textit{guq-le} \sqt{low} \\
		\textit{haraq-le} \sqt{far}					\tab	\textit{hek-le}  \sqt{close, near} \\
		\textit{b-arx-le}  \sqt{directly, straight}			\tab	\textit{kʷi} \sqt{in the hands} \\
\end{exe}


%%%%%%%%%%%%%%%%%%%%%%%%%%%%%%%%%%%%%%%%%%%%%%%%%%%%%%%%%%%%%%%%%%%%%%%%%%%%%%%%

\section{Temporal adverbs}
\label{sec:TemporalAdverbs}

Many of the \is{spatial adverb}spatial adverbs/postpositions listed in \refex{ex:adjectivesWithCHIB} also express temporal meaning. The adverb/postposition \textit{gu} has the somewhat unexpected meaning \sqt{before, in earlier times} when suffixed with the frozen neuter plural agreement suffix \textit{-d} \refex{ex:with these glasses of yours with which you were seeing before}.

\begin{exe}
	\ex	\label{ex:with these glasses of yours with which you were seeing before}
	\gll	heχ-tːi	ala	gu-d	či-d-ig-an	ʡaˁčkːa-b-a-l\\
		\textsc{dem.down}-\textsc{pl}	2\textsc{sg}.\textsc{gen}	down-\textsc{npl}	\textsc{spr}-\textsc{npl}-see.\textsc{ipfv}-\textsc{ptcp}	glasses-\textsc{pl}-\textsc{obl}-\textsc{erg}\\
	\glt	\sqt{with these glasses of yours with which you were seeing before}
\end{exe}

Adverbs for times of the day are given in \refex{ex:temporal adverbs1}. Deictic \is{temporal adverb}temporal adverbs expressing relative time in days and years can be found in \refex{ex:temporal adverbs2}, \refex{ex:temporal adverbs3}, and seasonal adverbs in \refex{ex:adjectivesWithEbla}. Some of the adverbs in \refex{ex:temporal adverbs1} and \refex{ex:adjectivesWithEbla} are formed by adding the \isi{genitive} case suffix to a base noun.

\begin{exe}
	\ex	\label{ex:temporal adverbs1}
	\begin{xlist}
		\ex	\textit{čːaˁʡaˁlla} \sqt{in the morning} (< \textit{čːaˁʡaˁl} \sqt{morning, tomorrow})
		\ex	\textit{arilla} \sqt{at midday, at lunch time} (< \textit{ari} `daytime')
		\ex	\textit{nisnalla} \sqt{after lunch, afternoon}
		\ex	\textit{ʁerilla} \sqt{early evening, at sunset} (< \tit{ʁeri} \sqt{sunlight})
		\ex	\textit{daˁrχːaˁlla} \sqt{in the evening} (< \tit{daˁrχːaˁ} \sqt{evening})
		\ex	\textit{dučːilla} \sqt{at night} (< \tit{dučːi} \sqt{night})
	\end{xlist}

	\ex	\label{ex:temporal adverbs2}
	\begin{xlist}
		\ex	\textit{xujal bar sar} \sqt{five days ago} (< \textit{xujal} \sqt{five} + \textit{bar} \sqt{day} + \textit{sa-r} ago-\textsc{abl})
		\ex	\textit{hati sar bar} \sqt{three days ago} (< \textit{hati} \sqt{more} + \textit{sa-r} + \textit{bar})
		\ex	\textit{sar bar} \sqt{two days ago} (< \textit{sa-r} + \textit{bar})
		\ex	\textit{sːa} \sqt{yesterday}
		\ex	\textit{ižal} \sqt{today} (< \textit{iž} `this')
		\ex	\textit{čːaˁʡaˁl} \sqt{tomorrow, morning}
		\ex \textit{carabal} \sqt{day after tomorrow} (< \textit{ca-ra} `one=\textsc{add}, other + ?)
		\ex	\textit{xujal bar hitːille} \sqt{in five days} (< \textit{xujal} + \textit{bar} + \textit{hitːi-lle} after-\textsc{advz})
	\end{xlist}

	\ex	\label{ex:temporal adverbs3}
	\begin{xlist}
		\ex	\textit{ʡaˁbc'al sar dus} \sqt{thirty years ago} (< \textit{ʡaˁbc'al} `30' + \textit{sa-r} + \textit{dus} `year')
		\ex	\textit{hati sar dus} \sqt{two years ago} (? < \textit{hati} + \textit{sa-r} + \textit{dus})
		\ex	\textit{sar dus, irig, gur dus, hit dus} \sqt{last year} (< \textit{gur} `away', \textit{hit} `that')
		\ex \textit{hež dus} \sqt{this year} (< hež `this')
		\ex \textit{c'il dus, saˁq'an dus, hilabil dus} \sqt{the following year, next year} (< \textit{c'il} `then', \textit{saˁ-q'-an} \textsc{in.front}-go-\textsc{ptcp}, \textit{hila-b-il} back.side-\textsc{n-ref})
	\end{xlist}
	
	\ex	\label{ex:adjectivesWithEbla}
	\begin{xlist}
		\ex	\textit{ebla} \sqt{in spring} (< \tit{eb} \sqt{spring})
		\ex	\textit{hanišalla} \sqt{in summer} (< \textit{haniša} \sqt{summer})
		\ex	\textit{ibxnella} \sqt{in autumn} (< \textit{ebx} \sqt{autumn})
		\ex	\textit{ganilla} \sqt{in winter} (< \tit{ga} \sqt{winter})
	\end{xlist}
\end{exe}

Some more \is{temporal adverb}temporal adverbs are provided in \refex{ex:adjectivesWithSalarka}. For \is{temporal adverb}temporal adverbs that have the meaning of indefinite pro-forms see \refsec{sec:Indefinite pronouns}. For the expression of dates and the time see the descriptions of the various \is{spatial case}spatial cases that fulfill these functions in \refsec{sec:nouncase}.

\begin{exe}
	\ex	\label{ex:adjectivesWithSalarka}
	\begin{xlist}
		\TabPositions{14em}
		\ex \textit{salar(ka)} \sqt{formerly}	(< \textit{sala-r-ka} before-\textsc{abl-abl}) 
		\ex	\textit{barežij} \sqt{the whole day} (< \textit{bar} `day' + ?)
		\ex \textit{ixʷle} \sqt{early, fast}		
		\ex \textit{ixʷbel} \sqt{long ago} (< \textit{ixʷ} `early' + ?\textit{b-el} \textsc{n}-remain)
		\ex	\textit{q'anne} \sqt{late} (< \textit{q'an-ne} late-\textsc{advz})
		\ex	\textit{ha} \sqt{now, already} 
		\ex	\textit{na} \sqt{now, already} 
		\ex \textit{hana} \sqt{now, then}	(< \textit{ha} + \textit{na})
		\ex \textit{c'il(i)} \sqt{then}		
		\ex \textit{heba} \sqt{then, later} (?< \textit{he-b-a} that-\textsc{n-dir})
		\ex \textit{cacajnaqːel} \sqt{sometimes} (< \textit{ca-ca-jna=qːel} one-one-\textsc{time}=when)
		\ex \textit{cacaqːella} \sqt{sometimes} (< \textit{ca-ca=qːella} one-one=when)
		\ex \textit{urkːa-urkːab} \sqt{sometimes} (< urkːa-urkːa-b middle-middle-\textsc{n})
		\ex \textit{mah-mahle, raχ-raχle} \sqt{sometimes, rarely} (< \textit{raχle} `if')
		\end{xlist}
\end{exe}


%%%%%%%%%%%%%%%%%%%%%%%%%%%%%%%%%%%%%%%%%%%%%%%%%%%%%%%%%%%%%%%%%%%%%%%%%%%%%%%%

\section{Manner adverbs}\label{sec:MannerAdverbs}\largerpage

Sanzhi has a productive way of deriving adverbs of manner from \is{demonstrative pronoun}demonstrative pronouns. These adverbs have the meaning \sqt{like this\slash like that}. They are formed by adding the suffix \textit{-itːe} to the \is{demonstrative pronoun}demonstrative pronouns in the singular (\reftab{tab:Manner adverbs derived from demonstrative pronouns2}). Their meaning is again transparently built on the semantics of the \is{demonstrative pronoun}demonstrative pronouns as described in detail in \refsec{sec:Demonstrative pronouns} and summarized in \refsec{ssec:SpatialAdverbsDerivedFromDemonstrativePronouns} above.


\begin{table}
	\caption{Manner adverbs derived from demonstrative pronouns}
	\label{tab:Manner adverbs derived from demonstrative pronouns2}
	\small
	\begin{tabularx}{0.98\textwidth}[]{%
		>{\raggedright\arraybackslash\itshape}p{33pt}
		>{\raggedright\arraybackslash\itshape}p{33pt}
		>{\raggedright\arraybackslash\itshape}p{33pt}
		>{\raggedright\arraybackslash}X}
		
		\lsptoprule
		\multicolumn{1}{l}{\tit{iC}}	&	\multicolumn{1}{l}{\tit{heC}}	&	\multicolumn{1}{l}{\tit{hiC}}	&	translation\\
		\midrule
		iž-itːe		&	hež-itːe	&	hiž-itːe 	&	\sqt{like this, like something close to the speaker}\\
		il-itːe		&	hel-itːe	&	hil-itːe 	&	\sqt{like that, away from the speaker and\slash or close to the hearer}\\
		it-itːe		&	het-itːe	&	hit-itːe	&	\sqt{like that, like something further away, unspecific distance}\\
		ik'-itːe		&	hek'-itːe	&	hik'-itːe	&	\sqt{like this/that above, higher}\\
		iχ-itːe		&	heχ-itːe	&	hiχ-itːe	&	\sqt{like this/that below, lower}\\
		\lspbottomrule
	\end{tabularx}
\end{table}

Since these adverbs are used when an action or event is compared to another event, the adverbs based on the \textit{heC}-series in the second column are far more common than those from the other two series given in the first and in the third column. The most frequent forms in my corpus are \textit{hel-itːe} \sqt{like that, away from the speaker and\slash or close to the hearer} and to a lesser extent \textit{hež-itːe} \sqt{like this, like something close to the speaker} \refex{ex:Like this alongside we put the oil cloth}, but other forms such as \textit{het-itːe} \sqt{like that, like something further away, unspecific distance}, \textit{hek'-itːe} \sqt{like this/that above} and \textit{heχ-itːe} \sqt{like this/that below} as well as very few occurrence of \textit{ižitːe} \refex{ex:‎‎I do not know whether it was like this or not}, \textit{hilitːe} \refex{ex:‎‎I remember like I was standing there like that}, and \textit{ilitːe} are also attested.

\begin{exe}
	\ex	\label{ex:Like this alongside we put the oil cloth}
	\gll	hež-itːe	b-uqen-ne	ka-b-išː-ib	kilijumk'a	hež-itːe  \\
		this-\textsc{advz}	\textsc{n}-long-\textsc{advz}	\textsc{down-n}-put.\textsc{pfv}-\textsc{pret}	oil.cloth	this-\textsc{advz}\\
	\glt	\sqt{Like this alongside (he) put the oil cloth.}

	\ex	\label{ex:‎‎I do not know whether it was like this or not}
	\gll	na	ca-b=el	iž-itːe	akːu=jal,	a-b-alχ-ul=da  \\
		now	\textsc{cop}-\textsc{n}=\textsc{indq}	this-\textsc{advz}	\textsc{cop.neg}=\textsc{indq}	\textsc{neg}-\textsc{n}-know.\textsc{ipfv}-\textsc{icvb}=1\\
	\glt	\sqt{‎‎I do not know whether it was like this or not.}

	\ex	\label{ex:‎‎I remember like I was standing there like that}
	\gll	hil-tːu	ka-r-icː-ur-re	hil-itːe	han	le-b  ...\\
		that-\textsc{loc}	\textsc{down-f}-stand.\textsc{pfv}-\textsc{pret}-\textsc{cvb}	that-\textsc{advz}	remember	exist-\textsc{n}\\
	\glt	\sqt{‎‎I (fem.) remember like I was standing there like that ...}
\end{exe}

There is another rather small group of four \isi{manner adverbs} with a similar meaning that are also derived from \is{demonstrative pronoun}demonstrative pronouns: \textit{itwaj}, \textit{hetwaj}, \textit{hitwaj}, and \textit{ižwaj} \sqt{like that, and so}. Their usage is illustrated in \refex{ex:They are also like this like Russians even now}, \refex{ex:he is my real grandfather}.

\begin{exe}
		\ex	\label{ex:They are also like this like Russians even now}
		\gll	itːi	itwaj=ra	ʡuˁrusː-e	ʁunab-te	ca-b	hana=ra  \\
			those	like.that=\textsc{add}	Russian-\textsc{pl}	\textsc{eq-dd.pl} 	\textsc{cop-hpl}	now=\textsc{add}\\
		\glt	\sqt{They are also like this, like Russians, even now.}

		\ex	\label{ex:he is my real grandfather}
		\gll	di-la	χatːaj	ca-w,	ala	itwaj	χatːaj	ca-w   \\
			1\textsc{sg}-\textsc{gen}	grandfather	\textsc{cop-m}	2\textsc{sg}.\textsc{gen}	like.that grandfather	\textsc{cop-m}\\
		\glt	\sqt{(He) is my (real) grandfather. For you he is only an old man.} (lit. \sqt{He is like a grandfather of yours.})
\end{exe}

Other \isi{manner adverbs} are usually formed by suffixing \textit{-le} to a root, for example \textit{bahla-l} \sqt{slowly}, \textit{halak-le} \sqt{fast}, \textit{χʷal-le} \sqt{greatly, much, a lot}, \textit{imanne} \sqt{patiently}, \textit{ʡaˁħ-le} \sqt{well}, and so on. This is described in the next section (see also \refsec{ssec:The adverbializer -le}).


%%%%%%%%%%%%%%%%%%%%%%%%%%%%%%%%%%%%%%%%%%%%%%%%%%%%%%%%%%%%%%%%%%%%%%%%%%%%%%%%

\section{Degree adverbs}
\label{sec:Degree adverbs}

Adverbs of degree express the degree of a quality and modify \isi{adjectives} or other adverbs. They precede the modified item \refex{ex:‎‎We liked the meat of the bear very much}. From the formal perspective, \is{degree adverb}degree adverbs are a heterogeneous group of items. Some are simple stems (\textit{arindan}, \textit{bara}), but most of them contain the adverbializing suffix -\textit{le} also used to derive \isi{manner adverbs} \refex{ex:listDegreeAdverbs} (\refsec{sec:FormationOfAdverbialsWithTheSuffixLe}); for \is{comparative construction}comparative constructions involving \is{degree adverb}degree adverbs see \refsec{sec:Comparative constructions}. 

\begin{exe}
	\ex	\label{ex:listDegreeAdverbs}
	\TabPositions{14em}
		\tit{c'aq'-le} \sqt{very, strongly} \tab \tit{ħaˁq'-le} \sqt{very} \\
		\tit{arindan} \sqt{too, too much} \tab \tit{b-aq} \sqt{much} \\
        \tit{χːʷal-le} \sqt{largely} \tab \tit{q'ʷila, bara, kam-le} \sqt{little, few, a bit}

		\ex	\label{ex:‎‎We liked the meat of the bear very much}
	\gll	nišːi-j	b-aq	ʡaˁħ	ka-b-icː-ur ca-b	dig	sːika-la	\\
		\tsc{1pl-dat} 	\tsc{n}-much	good	\tsc{down-n}-stand.\tsc{pfv-pret} \tsc{cop-n}	meat	bear-\tsc{gen} \\
	\glt	\sqt{‎‎We liked the meat of the bear very much.}
	\end{exe}
	
	%%%%%%%%%%%%%%%%%%%%%%%%%%%%%%%%%%%%%%%%%%%%%%%%%%%%%%%%%%%%%%%%%%%%%%%%%%%%%%%%

\section{Formation of adverbials with the suffix \textit{-le}}
\label{sec:FormationOfAdverbialsWithTheSuffixLe}

Manner adverbs and some other adverbs are easily derived by means of the suffix \textit{-le} (allomorphs \textit{-l} after vowels, \textit{-re} after \textit{r}, \textit{-ne} after \textit{n}, and occasionally -\textit{lle}). This suffix is attached to the underived short \isi{adjectives} \refex{ex:The village of Anklukh is pretty high up}, \refex{ex:Like this alongside we put the oil cloth}, \refex{ex:deadjectival adverbs} (\refsec{sec:adjmorphclasses}) and to \isi{nouns} in the \isi{absolutive} or in \is{spatial case}spatial cases \refex{ex:Denominaladverbs} as well as to \is{spatial adverb}spatial adverbs bearing the essive case (\refsec{ssec:The adverbializer -le}). The same suffix is added to verbs (usually bearing the preterite or the \isi{imperfective converb} suffix) in order to form simple converbs (\refsec{ssec:Simple converbs}). More examples of its use can be found in \refsec{ssec:The adverbializer -le}. 

\begin{exe}

	\ex	deadjectival adverbs \label{ex:deadjectival adverbs}
	\begin{xlist}
		\TabPositions{10em,12em}
		\ex	\textit{sark} \sqt{open}		\tab	>	\tab	\textit{sark-le} \sqt{openly}
		\ex	\textit{b-arx} \sqt{direct,right}	\tab	>	\tab	\textit{b-arx-le }\sqt{correctly, directly, straight}
		\ex	\textit{ʡaˁħ} \sqt{good}		\tab	>	\tab	\textit{ʡaˁħ-le} \sqt{well}
		\ex	\textit{kːuš} \sqt{hungry}		\tab	>	\tab	\textit{kːuš-le} \sqt{hungrily}
		\ex	\textit{ħaˁdur }\sqt{ready}	\tab	>	\tab	\textit{ħaˁdur-re} \sqt{readily}
		\ex	\textit{c'aq'} \sqt{strong, mighty}\tab	>	\tab	\textit{c'aq'-le} \sqt{strongly, very} 
		\ex	\textit{pašman} \sqt{sad}		\tab	>	\tab	\textit{pašman-ne }\sqt{sadly}
	\end{xlist}
	
	
	\ex	denominal adverbs \label{ex:Denominaladverbs}
	\begin{xlist}
		\TabPositions{10em,12em}
		\ex	\textit{uruχ} \sqt{fear}		\tab	>	\tab	\textit{uruχ-le} \sqt{fearfully, anxiously}
		\ex	\textit{ʡaˁžat} \sqt{need}		\tab	>	\tab	\textit{ʡaˁžat-le} \sqt{needed, necessarily}
		\ex	\textit{ʡaˁjb} \sqt{guilt, blame}	\tab	>	\tab	\textit{ʡaˁjb-le} \sqt{guilty}
		\ex	\textit{ħisab} \sqt{account}	\tab	>	\tab	\textit{ħisab-le} \sqt{accordingly}
		\ex	\textit{jatin} \sqt{orphan}		\tab	>	\tab	\textit{jatin-ne} `as an orphan,\\
			\mbox{~}				\tab	\mbox{~}	\tab	while being an orphan'
		\ex	\textit{mar} \sqt{truth}		\tab	>	\tab	\textit{mar-le} \sqt{truly}
	\end{xlist}


	\ex	other adverbs \label{ex:other adverbs}
	\begin{xlist}
		\TabPositions{10em,12em}
		\sn	\textit{xurc-le} \sqt{barefoot} (*\textit{xurc}; \textit{xurχ} \textit{b-iχʷ-ij} \sqt{become barefoot})
	\end{xlist}
\end{exe}

