\chapter{Non-indicative verb forms}\label{cpt:verbs-nondeclarative}
\largerpage

Non-indicative (or non-declarative) verb forms occurring in Sanzhi are \isi{imperative} (\refsec{sec:imperative}), \isi{prohibitive} (\refsec{sec:prohibitive}), \isi{optative} (\refsec{sec:optative}) and the \isi{modal interrogative} (\refsec{sec:modalinterrogative}). The \isi{imperative}, the \isi{prohibitive}, and the \isi{modal interrogative} are restricted in their use to the second person for the first two forms and the first person for the last form. The \isi{imperative} and the \isi{prohibitive} share the (partial) distinction between intransitive and \is{transitive verb}transitive verbs expressed through the use of dedicated stem-augment vowels in the suffixes. The same distinction and the same formal means of expressing it are found with synthetic verb forms and conditionals (\refsec{sec:Stem augment vowels and person agreement}).


%%%%%%%%%%%%%%%%%%%%%%%%%%%%%%%%%%%%%%%%%%%%%%%%%%%%%%%%%%%%%%%%%%%%%%%%%%%%%%%%
%%%%%%%%%%%%%%%%%%%%%%%%%%%%%%%%%%%%%%%%%%%%%%%%%%%%%%%%%%%%%%%%%%%%%%%%%%%%%%%% 

\section{Imperative}
\label{sec:imperative}

The form of the \isi{imperative} depends on the inflectional class and on the transitivity of the verb. The suffixes are given in \reftab{tab:imperativesuffixes}. Verbs that have the preterite suffix \tit{-un} have the suffixes \tit{-en} and \tit{-ene(ja)} for singular and plural imperatives respectively, independently of their transitivity. The other three verb classes distinguish (almost) always between intransitive and \is{transitive verb}transitive verbs in the formation of the singular \isi{imperative}: \is{intransitive verb}intransitive verbs employ the suffix \tit{-e}; \is{transitive verb}transitive verbs make use of \tit{-a}. The distinction is absent in the plural \isi{imperative}, which has the suffixes\tit{-ene(ja)} and \tit{-aj(a)} (and \tit{-ere} as an alternative that is not frequently used).\footnote{There is one verb with the meaning `go, leave' that is exceptional because it also allows for the suffixes -\textit{aˁn} and -\textit{aˁne}, cf. the last two lines of \reftab{tab:imperativesuffixes}.} The suffix \tit{-(j)a}, which is optionally found with all plural imperatives as well as the plural forms of the \isi{prohibitive} and the second person plural \isi{optative}, can be analyzed as a plural addressee marker following the suggestion by \citet[163\tnd165]{Sumbatova.Lander2014}. See \refsec{sec:Pause fillers, address particles, exclamatives, and interjections} for some other contexts of its use.

\begin{table}
	\caption{The imperative suffixes}
	\label{tab:imperativesuffixes}
	\small
	\begin{tabularx}{\textwidth}[]{%
		>{\itshape}l
		>{\itshape}l
		>{\itshape}l
		>{\itshape}Q}		
		\lsptoprule
			\normalfont\tnm{singular}
		&	\normalfont\tnm{plural}
		&	\normalfont\tnm{preterite}
		&	\tnm{examples (singular, plural)}\\
		\midrule
			\tit{-e}		&	\tit{-aj(a)}\slash\tit{-ere}	&	\tit{-ib}	&	ka-r-iž-e, ka-d-ig-aj(a) \tnm{\sqt{sit down}}\newline w-aš-(e), d-ax-aj(a) \tnm{\sqt{go}}\newline k'ʷah r-ič-e, d-ik-aj(a) \tnm{\sqt{be silent}}\\
			{}			&	{}				&	\tit{-ub}	&	čar r-iχʷ-e, čar d-iχʷ-aj(a) \tnm{\sqt{come back}}\\
			{}			&	{}				&	\tit{-ur}	&	ha-r-icː-e, ha-d-icː-aj(a) \tnm{\sqt{get/stand up}}\newline b-aχ-e, b-aχ-ere\slash b-aχ-aj(a) \tnm{\sqt{know}}\newline b-at-e, b-at-ere\slash b-at-eja \tnm{\sqt{leave, let}}\\[1mm]

			\tit{-a (-aˁ)}		&	\tit{-aj(a)}			&	\tit{-ib}	&	b-uc-a, b-uc-aj(a) \tnm{\sqt{catch}}\newline b-aˁq-aˁ \tnm{\sqt{wound, hit, strike}}\newline b-arq'-a, b-arq'-aj(a) \tnm{\sqt{do}}\\
			{}			&	{}				&	\tit{-ub}	&	b-ikː-aq-a \tnm{\sqt{burn}}\newline ixʷ-a \tnm{\sqt{throw}}\newline kaxʷ-a, kaxʷ-aj(a) \tnm{\sqt{kill}}\\
			{}			&	{}				&	\tit{-ur}	&	ergʷ-a \tnm{\sqt{sieve}}\newline b-aχ-aq-a \tnm{\sqt{tell, make know}}\newline ka-b-arkː-a \tnm{\sqt{wrap (in)}}\\[1mm]

			\tit{-en}		&	\tit{-ene(ja)}			&	\tit{-un}	&	r-usː-en, d-usː-ene(ja) \tnm{\sqt{sleep}}\newline b-elč'-en, b-elč'-ene(ja) \tnm{\sqt{read}}\newline r-uˁq'-aˁn\slash r-uˁq'-en, d-uˁq'-aˁne\slash d-uˁq'-ene \tnm{\sqt{go, leave}}\\
		\lspbottomrule
	\end{tabularx}
\end{table}

For those verbs that have an imperfective and a perfective stem, the \isi{imperative} is mostly formed from the perfective stem. Regular exceptions to this rule are the imperfective stems of the verbs \sqt{eat} and \sqt{drink} that often behave differently from other imperfective verbs. They have the following imperatives: for the imperfective stem \tit{r-učː-e} (\tsc{sg}, \tsc{f})\slash\tit{d-učː-aja} (\tsc{pl}) \sqt{drink (several times)} and perfective \tit{b-erčː-a} (\tsc{sg})\slash\tit{b-erčː-aja} (\tsc{pl}) \sqt{drink (once)}; imperfective stem \tit{r-uk-en} (\tsc{sg}, \tsc{f})\slash\tit{d-uk-ene(ja)} (\tsc{pl}) \sqt{eat (several times)} and perfective stem \tit{b-erkʷ-en} (\tsc{sg})\slash\tit{b-erkʷ-en(ja)} (\tsc{pl}) \sqt{eat (once)}. Other verbs that allow for the imperfective and the perfective stem to serve as the basis for the \isi{imperative} are, for example, \tit{k.alž-} (\tsc{ipfv})\slash\tit{k.elg-} (\tsc{pfv}) \sqt{remain, stay}. Not all morphosyntactically \is{affective verb}affective verbs allow for an \isi{imperative}, but some do, such as \tit{b-aχ-e} (\tsc{n-}know\tsc{.pfv-imp}) \refex{ex:I am yours, (you) know}. Similarly, the verb in \refex{ex:You pl believe me} has experiental\slash affective semantics (though its subject appears in the \isi{dative}) and allows for the \isi{imperative}. With those types of verbs the \isi{imperative} has rather the meaning of a wish of the speaker or a deontic flavor similar to \sqt{you should}.

Sentences \refexrange{ex:‎Tell it like this}{ex:I am yours, (you) know} illustrate the use of the \isi{imperative}. Though it is not particularly common, \isi{imperative} clauses can contain the second person pronoun referring to the addressee, which can be an argument in the \isi{absolutive} \refex{ex:You pl believe me}, in the \isi{ergative} \refex{ex:‎Do not thrash me}, or in the \isi{dative}.

\begin{exe}
	\ex	\label{ex:‎Tell it like this}
	\gll	hel	ceʁuna	b-urs-a!	χabar		b-urs-aχː-atːe	hitːi	ka-jž-e!\\
		that	which	\tsc{n}-tell.\tsc{pfv-imp}	story	\tsc{n-}tell.\tsc{pfv-cond-cond.2sg}	behind	\tsc{down}-remain\tsc{.pfv-imp}\\
	\glt	\sqt{‎Tell it like this! Sit down if you (\tsc{sg}) narrate the story!}

	\ex	\label{ex:You pl believe me}
	\gll	d-iχ-d-it-ag-aj	ušːa!\\
		\tsc{1/2pl-}believe\tsc{-1/2pl-thither}-go\tsc{.pfv-imp.pl}	\tsc{2pl}\\
		\\
	\glt	\sqt{You (\tsc{pl}) believe (me)!}

	\ex	\label{ex:I am yours, (you) know}
	\gll	ala	ca-w=da	du,	b-aχ-e!\\
		\tsc{2sg.gen}	\tsc{cop-m=1}	\tsc{1sg}	\tsc{n-}know\tsc{.pfv-imp}\\
	\glt	\sqt{I am yours, (you) know!}
\end{exe}\largerpage

As in some other Dargwa varieties (\teg\ Icari, \citealp[98]{Sumbatova.Mutalov2003}, Shiri, \citealp{BelyaevInPreparation}), the \isi{imperative} cannot be used when the P argument of a \isi{transitive verb} is first person. In this case, the \isi{optative} is used instead \refex{ex:Leave me}. With second and third person P arguments the \isi{imperative} is allowed, cf. \refex{ex:Move the chair} below.

\begin{exe}
	\ex	\label{ex:Leave me}
	\gll	w-at-ab-aja	du!\\
		\tsc{m-}let\tsc{.pfv-opt-2pl}	\tsc{1sg}\\
	\glt	\sqt{Leave me!}
\end{exe}

\hspace*{-0.12148pt}The \isi{imperative} can be used in combination with the \isi{antipassive} construction, in which case the verb is intransitive and takes the suffix \tit{-e} if the addressee is singular: 

\begin{exe}
	\ex	\label{ex:Tell stories}
	\gll	χabur-t-a-l	∅-ux-e	/	r-ux-e!\\
		story\tsc{-pl-obl-erg}	\tsc{m-}tell.\tsc{ipfv-imp}	/	\tsc{f-}tell\tsc{.ipfv-imp}\\
	\glt	\sqt{Tell stories!} (E)
\end{exe}

With \isi{labile verbs} the use of \tit{-e} and \tit{-a} in the \isi{imperative} singular is possible, where \tit{-e} indicates that the verb is used intransitively and \tit{-a} that it is used transitively \refexrange{ex:Move the chair}{ex:Move (yourself)}.

\begin{exe}

		\ex	\label{ex:Move the chair}
		\gll	ust'ul	b-it'kak'-a!\\
			chair	\tsc{n-}move\tsc{.pfv-imp}\\
		\glt	\sqt{‎Move the chair!} (E)
	
		\ex	\label{ex:Move (yourself)}
		\gll	w-it'kač'-e!\\
			\tsc{m-}move\tsc{.pfv-imp}\\
		\glt	\sqt{Move (yourself)! (said to a man)} (E)

\end{exe}

The \isi{imperative} of the verb \tit{b-ax-} (\tsc{n}-go-) is exceptional because it can be formally zero (i.e. \textit{waš} instead of \textit{waše}), and functionally it can be used as a cohortative, \tie\ to encourage or discourage to perform an action together with the speaker. Thus, in \refex{ex:‎Let's go (together) and look what is there in the shop} \tit{w-aš-e} is not meant as a command to the addressee to perform the action alone, but is intended as an invitation to go together and have a look at the shop. Morphosyntactically, however, the utterance is addressed to a man because the verb agrees in \isi{gender} with the addressee (masculine singular) but not with whole group. Similarly, in \refex{ex:The women shouted ``Let's go eating!''} the verb `go' is used in the function of a cohortative. However, syntactically the periphrastic cohortative constructions are either asyndetic \is{conjunction}conjunctions of two independent main clauses as in \refex{ex:‎Let's go (together) and look what is there in the shop} (`Go!' and `We will look.') or complex clause constructions consisting of a main clause in the \isi{imperative} followed by a purpose clause in the \isi{infinitive} or \isi{subjunctive} as in \refex{ex:The women shouted ``Let's go eating!''} (`Go in order to eat!').

\begin{exe}
	\ex	\label{ex:‎Let's go (together) and look what is there in the shop}
	\gll	``w-aš-e=kːʷa!''	∅-ik'ʷ-ar,	``ce	k'e-b-il	er	d-urk'-an=da'',	∅-ik'ʷ-ar, ``tuken-ne-b''\\
		\tsc{m-}go\tsc{-imp=prt}		\tsc{m-}say.\tsc{ipfv-prs}	what	exist.\tsc{up-n-ref}	look	\tsc{1/2pl-}look\tsc{.ipfv-ptcp=1}	\tsc{m-}say\tsc{.ipfv-prs}	shop-\tsc{loc-n}\\
	\glt	\sqt{\dqt{‎Let's go (together) and look what is there in the shop!}, he says.}
	
			\ex	\label{ex:The women shouted ``Let's go eating!''}
		\gll	xːunr-a-l waˁw	haʔ-ib,	``d-ax-aj''	b-ik'ʷ-ar, ``d-uk-ij,	d-uk-utːaj!''\\
			woman.\tsc{pl-obl-erg} shout	say.\tsc{pfv-pret}	\tsc{1/2pl}-go-\tsc{imp.pl}	\tsc{hpl}-say.\tsc{ipfv-prs}	\tsc{1/2pl}-eat.\tsc{ipfv-inf}	\tsc{1/2pl}-eat.\tsc{ipfv-subj.2}\\
		\glt	\sqt{The women shouted, ``Let's go eating!''} 
\end{exe}




% --------------------------------------------------------------------------------------------------------------------------------------------------------------------------------------------------------------------- %

\section{Prohibitive}
\label{sec:prohibitive}

The \isi{prohibitive} is formally independent of the \isi{imperative}. It consists of the prefix \tit{ma-} and a suffix. The prefix is positioned between the orientation and the deixis/gravitation \is{preverb}preverbs if there are any. The suffixes are similar to the \isi{habitual present} (\refsec{sec:vis-habitualpresent}) because they make use of the same stem augment, which depends on the transitivity of the verb. Intransitive verbs take \textit{u}; \is{transitive verb}transitive verbs take \textit{i}, and one verb behaves exceptionally (\refsec{ssec:IntroductionPersonAgreement}). In the singular the suffixes are \tit{-ut}\slash\tit{-utːa}, \tit{-it}\slash\tit{-itːa}, and \tit{-aˁt}\slash\tit{-aˁtːa}; in the plural they are \tit{-utːaj(a)}, \tit{-itːaj(a)}, and \tit{-aˁtːaj(a)}. The short and long variants in the singular and plural seem to be in free variation. The \isi{prohibitive} is only formed from the imperfective stem in case a verb has both stems. Exemplary verbs in the \isi{prohibitive} singular are shown in the last column of \reftab{tab:prohibitivesuffixes}.

\begin{table}
	\caption{The prohibitive suffixes}
	\label{tab:prohibitivesuffixes}
	\small
	\begin{tabularx}{0.68\textwidth}[]{%
		>{\itshape\raggedright\arraybackslash}p{40pt}
		>{\itshape\raggedright\arraybackslash}p{40pt}
		>{\itshape\raggedright\arraybackslash}X}
		
		\lsptoprule
			\normalfont\tnm{singular}
		&	\normalfont\tnm{plural}
		&	\normalfont\tnm{examples}\\

		\midrule

			-ut\slash -utːa
		&	-utːaj(a)
		&	ma-k-erg-ut \tnm{\sqt{sit down}}\newline ma-r-ik'-ut \tnm{\sqt{say}}\newline tːura ma-ka-lq-ut \tnm{\sqt{go outside}}\newline ma-r-uk-utːa \tnm{\sqt{eat} (intr.)}\newline er-či-ma-ha-rk'-utːa \tnm{\sqt{look up}}\\
		
		-it\slash -itːa
		&	-itːaj(a)
		&	ma-d-učː-it \tnm{\sqt{drink}}\newline ma-b-irq'-it \tnm{\sqt{do}}\newline ma-b-urs-it \tnm{\sqt{tell}}\newline ma-lukː-it \tnm{\sqt{give}}\newline ma-b-urh-itːa \tnm{\sqt{strike}}\\
		
		
					-at\slash -atːa
		&	-atːaj(a)
		& maˁ-q'-aˁtːa \tnm{\sqt{go}}\\
		\lspbottomrule
	\end{tabularx}
\end{table}

As with the \isi{imperative}, the \isi{prohibitive} is only used with second persons. The second person pronoun functioning as the addressee is mostly omitted, but it can be overtly expressed. Examples \refex{ex:No way you look at them} and \refex{ex:Do not cry, do not wrestle} show \is{intransitive verb}intransitive verbs. Sentences \refex{ex:Do not raise your hand against your wife}, \refex{ex:Do not hit with a stick} illustrate \is{transitive verb}transitive verbs. In addition, \is{affective verb}affective verbs with \isi{dative} addressees (experiencers) are allowed \refex{ex:Do not know this}.

\begin{exe}
	\ex	\label{ex:No way you look at them}
	\gll	warilla.wari	u	iχ-tː-a-j	er	či-ma-ha-rk'-utːa!\\
		no.way	\tsc{2sg}	\tsc{dem.down-pl-obl-dat}	look	\tsc{spr-proh-up}-look\tsc{.ipfv-proh.sg}\\
	\glt	\sqt{No way you look at them (\tie\ the trees)!}

	\ex	\label{ex:Do not cry, do not wrestle}
	\gll	ma-d-isː-utːaj,		ma-d-irħ-utːaj!\\
		\tsc{proh-1/2.pl-}cry\tsc{-proh.pl}	\tsc{proh-1/2.pl-}wrestle\tsc{.ipfv-proh.pl}\\
	\glt	\sqt{Do not cry, do not wrestle!}\pagebreak

	\ex	\label{ex:Do not raise your hand against your wife}
	\gll	naˁq	aq	ma-b-irq'-it	xːunul-li-j!\\
		hand	high	\tsc{proh-n-}do\tsc{.ipfv-proh.sg}	woman\tsc{-obl-dat}\\
	\glt	\sqt{Do not raise your hand against your wife! (\tie\ do not beat your wife)}

	\ex	\label{ex:Do not hit with a stick}
	\gll	dirxːa	ma-b-urh-itːa=n!\\
		stick	\tsc{proh-n-}strike\tsc{.ipfv-proh.sg=prt}\\
	\glt	\sqt{Do not hit with a stick!}

	\ex	\label{ex:Do not know this}
	\gll	at	ma-b-alχ-itːa!\\
		\tsc{2sg.dat}	\tsc{proh}-\tsc{n}-know\tsc{.ipfv}-\tsc{proh.sg}\\
	\glt	\sqt{Do not know this!} (E)
\end{exe}

As mentioned above for the \isi{imperative}, with first person P arguments the \isi{prohibitive} cannot be employed. Instead, the negative \isi{optative} must be used \refex{ex:‎Do not thrash me}.

\begin{exe}
	\ex	\label{ex:‎Do not thrash me}
	\gll	ma-jt-aba	du!	u-l	u	w-it-a!\\
		\tsc{proh-}beat.up\tsc{.m-opt.1}	\tsc{1sg}	\tsc{2sg-erg}	\tsc{2sg}	\tsc{m-}beat.up\tsc{-imp}\\
	\glt	\sqt{‎Do not thrash me (masc.)! Thrash yourself (masc.)!} (E)
\end{exe}

When the \isi{ergative} construction or the \isi{antipassive} construction occur together with the \isi{prohibitive}, the difference in transitivity is reflected in the different stem augment vowels, that is, the \isi{antipassive} construction requires \tit{u} \refex{ex:Do not drink water (regularly)}, whereas the \isi{ergative} construction requires \tit{i} \refex{Do not drink the water}.

\begin{exe}
	\ex
	\begin{xlist}
		\ex	\label{ex:Do not drink water (regularly)}
		Antipassive construction\\
		\gll	hin-ni	ma-d-učː-utːaja!\\
			water-\tsc{erg}	\tsc{proh-1/2pl-}drink\tsc{.ipfv-proh.pl}\\
		\glt	\sqt{Do not drink water (regularly)!}
	
		\ex	\label{Do not drink the water}
		Ergative construction\\
		\gll	hin	ma-d-učː-itːaja!\\
			water	\tsc{proh-npl-}drink\tsc{.ipfv-proh.pl}\\
		\glt	\sqt{Do not drink the water!}
	\end{xlist}
\end{exe}


% --------------------------------------------------------------------------------------------------------------------------------------------------------------------------------------------------------------------- %

\section{Optative}\label{sec:optative}
\largerpage[-2]

The \isi{optative} is formed from perfective verbal stems by means of suffixes (\reftab{tab:optativesuffixes}). The suffixes are complex, and \tit{-ab} can be identified as the \isi{optative} marker to which markers that express \isi{person agreement} are added. The \isi{optative} seems to obey the same rules of \isi{person agreement} that obtain in indicative clauses (\refsec{ssec:Person agreement rules}). The paradigm has a structure that is similar to other person paradigms, that is, syncretism of first singular and plural with the second plural and zero marking in the third person (\refsec{sec:Person agreement}). There is an optional variant \tit{-arte} when the \isi{agreement controller} is plural. The \isi{optative} is negated by means of the prefix \tit{ma-}, which is also used for the \isi{prohibitive} (\refsec{sec:prohibitive}).\pagebreak

\begin{table}
	\caption{The optative}
	\label{tab:optativesuffixes}
	\small
	\begin{tabularx}{0.4\textwidth}[]{%
		>{\centering\arraybackslash}p{10pt}
		>{\itshape\centering\arraybackslash}X
		>{\itshape\centering\arraybackslash}X}
		
		\lsptoprule
			{}	&	\multicolumn{1}{c}{\tnm{singular}}	&	\multicolumn{1}{c}{\tnm{plural}}\\
		\midrule
			1	&	\multicolumn{2}{c}{\tit{-ab-a}}\\
			2	&	-ab-e						&	-ab-a /\\
			{}	&	{}						&	-ab-aj /\\
			{}	&	{}						&	-ab-aja /\\
			{}	&	{}						&	-arte\\
			3	&	-ab						&	-ab\slash -arte\\
		\lspbottomrule
	\end{tabularx}
\end{table}

The functions of the \isi{optative} cover:\largerpage

\begin{enumerate}
	\item	Wishes, blessings, curses, e.g. in greetings and other idiomatic phrases. For instance, \refex{ex:A good day to you} is a typical greeting, and \refex{ex:May God bless him} is a phrase used when pronouncing the name of a deceased. Note that the \isi{gender} agreement in \refex{ex:A good day to you} is frozen. For reasons unclear to me it is impossible to use the neuter singular prefix here, although this would be expected from the structure of the clause (see \refsec{General remarks on gender/number agreement} for more information on default \isi{gender} agreement and frozen agreement affixes).
	\begin{exe}
		\ex	\label{ex:A good day to you}
		\gll	ašːi-j	/	at	bari	ʡaˁħ	d-iχʷ-ab!\\
			\tsc{2pl-dat}	/	\tsc{2sg.dat}	day	good	\tsc{npl-}be\tsc{.pfv-opt.3}\\
		\glt	\sqt{A good day to you!} (lit. \sqt{May the day be good for you\slash to you.})

		\ex	\label{ex:May there be greetings to}
		\gll	Maˁħaˁmmad-la	šːal-li-cːe-r	d-iχʷ-ab	ašːi-j	salam-te!\\
			Mahammad\tsc{-gen}	side\tsc{-obl-in-abl}		\tsc{npl-}be\tsc{.pfv-opt.3}	\tsc{2pl-dat}	greeting\tsc{-pl}\\
		\glt	\sqt{May there be greetings to you from the side of Mahammad!}
	
		\ex	\label{ex:May God bless him}
		\gll	ʡaˁpa	b-arq'-ab	cin-na\\
			commemoration	\tsc{n-}do\tsc{.pfv-opt.3}	\tsc{refl.sg-gen}\\
		\glt	\sqt{May God bless him\slash her!}
	
		\ex	\label{ex:May (nobody) lay down (sleep) like me}
		\gll	du	daˁʡle	ma-ka-jsː-ab,	ja	Allah!\\
			\tsc{1sg}	as	\tsc{proh-down}-sleep\tsc{.pfv.m-opt.3}	oh	Allah\\
		\glt	\sqt{May (nobody) lay down (sleep) like me, oh Allah.} (\tie\ with so many sorrows)
	\end{exe}

	\item	Indifference, when the speaker does not care about a situation or event \xxref{ex:‎May I not remain until the sunset}{ex:‎May I be killed}. Note that in \refex{ex:Let it be lemonade} the agreement on the verb is neuter plural because \isi{nouns} referring to liquids normally control neuter plural agreement (\refsec{sec:Gendernumbermismatchesandexceptions}).
	\begin{exe}
		\ex	\label{ex:‎May I not remain until the sunset}
		\gll	hik'	bari	ruˁħ	b-uq-ij=sat	ma-kelg-ab-a	du\\
			\tsc{dem.up}	sun	disappear	\tsc{n-}go\tsc{.pfv-inf=}as.much	\tsc{proh-}remain\tsc{.pfv-opt-1sg}	\tsc{1sg}\\
		\glt	\sqt{‎May I not remain until the sunset.} (\tie\ May I die before the sunset, I don't mind.)
	
		\ex	\label{ex:Let it be lemonade}
		\gll	limonad	d-iχʷ-ab\\
			lemonade	\tsc{npl-}be\tsc{.pfv-opt.3}\\
		\glt	\sqt{Let it be lemonade.} (\tie\ The bottle on the picture could be lemonade or something else, I don't care.)
	
		\ex	\label{ex:‎May I be killed}
		\gll	``r-ebč'-aq-ab-a!''	r-ik'-ul		``r-isː-an=xːar,	ʡaˁħ-dex	b-akːu''\\
			\tsc{f-}die\tsc{.pfv-caus-opt-1sg}	\tsc{f-}say\tsc{.ipfv-icvb}	\tsc{f-}cry\tsc{-ptcp=conc}	good\tsc{-nmlz}	\tsc{n-}\tsc{cop.neg}\\
		\glt	\sqt{``‎May I be killed\slash may they kill me!'', I say, ``even if I cry it will not be better.''}
	\end{exe}

\item	Indirect commands
	\begin{exe}
		\ex	\label{ex:‎‎May Mahammad give you the sheep!}
		\gll	Maˁħaˁmmad-li	at	macːa	b-ikː-ab!\\
			Mahammad-\tsc{erg}	\tsc{2sg.dat}	sheep	\tsc{n}-give.\tsc{pfv-opt.3}\\
		\glt	\sqt{‎‎May Mahammad give you the sheep!} (E)
	\end{exe}


	\item	Commands (\tie\ \isi{imperative} and \isi{prohibitive} function) with first person P arguments:
	\begin{exe}
		\ex	\label{ex:Help me, save me}
		\gll	``dam	kumek	b-arq'-aja!''	∅-ik'-ul,	``w-erc-aq-ab-aja!''\\
			\tsc{1sg.dat}	help	\tsc{n-}do\tsc{.pfv-imp.pl}	\tsc{m-}say\tsc{.ipfv-icvb}	\tsc{m-}save\tsc{.pfv-caus-opt-2pl}\\
		\glt	\sqt{``Help me, save me!'' he says.}
	\end{exe}
\end{enumerate}

Especially the first and second functions are used in situations where the speaker does not have control over what is going to happen.

The suffix \tit{-arte} can only be used when the \isi{agreement controller} is plural \refex{ex:‎‎‎May Allah leave you (plural) well}.  In \refex{ex:‎May their beloved ones die} the addressee is a not further specified group of people of whom the speaker wishes that one beloved (masculine) person may die, that is, one man\footnote{This explains the masculine singular agreement on the two verbs.} per addressee. Thus, there is a group of people for whom the speaker wishes that they would die, which explains the use of \tit{-arte} and the plural \isi{demonstrative pronoun}. The addressee is also plural (reflected in the plural possessive pronoun). If the speaker had wished that more than one beloved one should die, the verbs would change to \tit{b-ikː-an-te b-ebk'-arte}.

\begin{exe}
	\ex	\label{ex:‎‎‎May Allah leave you (plural) well}
	\gll	Allah-li	ʡaˁħ-le	d-at-arte!\\
		Allah\tsc{-erg}	good\tsc{-advz}	\tsc{1/2pl-}let\tsc{.pfv-opt.pl}\\
	\glt	\sqt{‎‎‎May Allah leave you (plural) well!}

	\ex	\label{ex:‎May their beloved ones die}
	\gll	hiš-tːi	ču-la	w-ikː-an	w-ebk'-arte!\\
		this\tsc{-pl}	\tsc{refl.pl-gen}	\tsc{m-}want\tsc{.ipfv-ptcp}	\tsc{m-}die\tsc{.pfv-opt.pl}\\
	\glt	\sqt{‎May their beloved ones die!}
\end{exe}

There is also the possibility of using the bare verbal stem in the \isi{optative} function \xxref{ex:May your beloved (son) be left (in peace, alive)}{ex:May your bodies and souls remain} with singular and plural addressee. There is no observable semantic difference between the use of the bare stem and the use of the \isi{optative} when expressed by the suffixes given in \reftab{tab:optativesuffixes}.

\begin{exe}
	\ex	\label{ex:May your beloved (son) be left (in peace, alive)}
	\gll	ala	w-ikː-an	w-at!\\
		\tsc{2sg.gen}	\tsc{m-}want\tsc{.ipfv-ptcp}	\tsc{m-}let\tsc{.pfv}\\
	\glt	\sqt{May your beloved (son) be left (in peace, alive)!}

	\ex	\label{ex:May your heart char}
	\gll	ala	urk'i	b-erc'!\\
		\tsc{2sg.gen}	heart	\tsc{n-}fry\tsc{.pfv}\\
	\glt	\sqt{May your heart fry!}

	\ex	\label{ex:May your bodies and souls remain}
	\gll	ašːa-la	žan	d-at!\\
		\tsc{2pl-gen}	organism	\tsc{npl-}let\tsc{.pfv}\\
	\glt	\sqt{May your bodies and souls remain!} (\tie\ \sqt{May you be healthy!}) [modified corpus example]
\end{exe}

The bare \isi{optative} can even be used like a noun and inflected without the need of adding any derivational morphology. Thus, in \refex{ex:When they called the brother} the complete \isi{optative} phrase \tit{urk'i b-ac'} (heart \tsc{n-}thaw\tsc{.pfv}) \sqt{May your/his/her/their heart thaw} has been nominalized and then the \isi{dative} suffix has been added because the nominal functions as the addressee of the verb \sqt{telephone}. The phrase is used with the idiomatic meaning \sqt{idiot}.

\begin{exe}
	\ex	\label{ex:When they called the brother}
	\gll	ucːi-li-j	tilipun	d-arq'-ib-le	urk'i	b-ac'-li-j	k-ač'-e	d-arq'-ib-le,	\ldots\\
		brother\tsc{-obl-dat}	telephone	\tsc{npl-}do\tsc{.pfv-pret-cvb}	heart	\tsc{n-}thaw\tsc{.pfv-obl-dat}		\tsc{down}-come\tsc{.pfv-imp}	\tsc{npl-}do\tsc{.pfv-pret-cvb}\\
	\glt	\sqt{When they called the brother, this idiot, if he had told me to come, \ldots}
\end{exe}


% --------------------------------------------------------------------------------------------------------------------------------------------------------------------------------------------------------------------- %

\section{Modal interrogative}
\label{sec:modalinterrogative}

Sanzhi has a suffix \tit{-ide} (with the allomorph \tit{-ida}), which is only used in content \isi{questions} with first person subject-like arguments of verbs of all \is{valency class}valency classes. These \isi{questions} have a modal meaning covering possibility, deontic modality and future (similar to English \tit{can}, \tit{should}, \tit{will}). The \isi{questions} are sometimes more like rhetorical \isi{questions} to which an answer is not expected \refex{ex:‎How can I know about the history of Sanzhi}, but they can also have real interrogative illocutionary force as \isi{questions} that are uttered to solicit answers \refex{ex:‎Where should I bring the car, sister}. 

\begin{exe}
	\ex	\label{ex:‎How can I know about the history of Sanzhi}
	\gll	sːanži-la	šːi-la	isturija	cet'le	b-aχ-ide	dam?\\
		Sanzhi\tsc{-gen}	village\tsc{-gen}	history	how	\tsc{n-}know\tsc{.pfv-modq} 	\tsc{1sg.dat}\\
	\glt	\sqt{‎How can I know about the history of Sanzhi?}

	\ex	\label{ex:‎‎Now what can we say}
	\gll	na	ce	d-ik'ʷ-ide?\\
		now	what	\tsc{1/2pl-}say\tsc{.ipfv-modq}\\
	\glt	\sqt{‎‎Now what can we say?}

	\ex	\label{ex:‎Where should I bring the car, sister}
	\gll	``mašin	čina	b-ič-ide,	rucːi?''		∅-ik'-ul	ca-w\\
		car	where		\tsc{n-}lead\tsc{.ipfv-modq}		sister		\tsc{m-}say\tsc{.ipfv-icvb}	\tsc{cop-m}\\
	\glt	\sqt{``‎Where should I bring the car, sister?'' he is asking}
\end{exe}

The suffix can be added to perfective as well as to imperfective stems with the usual difference in meaning: habitual/iterative/generic if the verb is imperfective \refex{ex:What shouldcan we do} vs. specific singular event if the verb is perfective \refex{ex:‎Then how should I make (the plough)}.

\begin{exe}
	\ex	\tnm{[talking about the present times and how they have changed]} \label{ex:What shouldcan we do} \\
	\gll	ce	b-irq'-ide?\\
		what	\tsc{n-}do\tsc{.ipfv-modq}\\
	\glt	\sqt{What should/can we do?' or `What should/can be done (in general)?}

	\ex	\label{ex:‎Then how should I make (the plough)}
	\gll	c'il	cet'le	b-arq'-ide?\\
		then	how	\tsc{n-}do\tsc{.pfv-modq}\\
	\glt	\sqt{‎Then how should I make (the plough)?}
\end{exe}

The suffix is also obligatorily used when a second person \isi{absolutive} argument of a \isi{transitive verb} occurs \refex{ex:Where should I bring you, sister}. This deviates from the general rule about \isi{person agreement} because normally in clauses with two speech-act participants both arguments can control \isi{person agreement} (\refsec{ssec:Person agreement rules}). Thus, in an indicative clause we could and often would have a second person controlling agreement, as the answer in \refex{ex:Where should I bring you, sister} shows. This is impossible for the \isi{modal interrogative}. From this we can conclude that the \isi{modal interrogative} marker is not a \isi{person agreement} marker, although its use is restricted by person.

\begin{exe}
	\ex	\label{ex:Where should I bring you, sister}
	\gll	u	čina	r-uč-ide,	rucːi?	du-l u r-uk-ul=de qːala qːurejš-le\\
		\tsc{2sg}	where	\tsc{f-}lead\tsc{-modq}	sister	\tsc{1sg-erg} \tsc{2sg} \tsc{f-}lead\tsc{-icvb=2sg}	fortress	Kurejsh-\tsc{loc}\\
	\glt	\sqt{Where should I bring you, sister? I will bring you to Kala-Kurejsh (place name).} (E)
\end{exe}

Occasionally, \tit{-idel} instead of \tit{-ide} is used (the \tit{l} at the end is the \isi{embedded question marker}\slash complementizer \tit{=l}, \refsec{sec:Subordinate questions}). It seems that there is a slight difference in meaning between \tit{-ide} and \tit{-idel}, which reflects the fact that \tit{-idel} is a kind of \isi{insubordination}, \tie\ a use of an originally subordinate form in a main clause. In the Russian translations this is reflected by the use of an additional adverb \tit{interesno} \sqt{interesting}, which seems to stand for an omitted matrix clause \sqt{it would be interesting to know}.

\begin{exe}
	\ex	\label{ex:How can I forget your dear figure}
	\gll	cet'-le	du-l	qum.ert-idel 	ala	čarχ		bek'	durqa-te?\\
		how\tsc{-advz}	\tsc{1sg-erg}	forget\tsc{.pfv-modq}	\tsc{2sg.gen} figure		head	dear\tsc{-dd.pl}\\
	\glt	\sqt{How can I forget your dear figure?} (lit. figure-head)

	\ex	\label{ex:‎‎Now what can we sayModified}
	\gll	na	ce	d-ik'ʷ-idel?\\
		now	what	\tsc{1/2pl-}say\tsc{.ipfv-modq}\\
	\glt	\sqt{‎‎Now what can we say?} [modified corpus example]
\end{exe}

However, the form \tit{-idel} is far more common in real embedded \isi{questions} \refex{ex:‎This is when he is probably thinking} (see \refsec{sec:Subordinate questions} for more examples).

\begin{exe}
	\ex	\label{ex:‎This is when he is probably thinking}
	\gll	iž	ceqːel=el	iž-itːe	pikri	∅-ik'-ul=el		[d-iʡ-ij	∅-uˁq'-idel	a-w-uˁq'-idel]	∅-ik'-ul	le-w\\
		this	when\tsc{=indq}	this\tsc{-advz}	thought	\tsc{m-}say\tsc{.ipfv-icvb=indq}		\tsc{npl-}steal\tsc{.pfv-inf}	\tsc{m-}go\tsc{.pfv-modq}	\tsc{neg-m-}go\tsc{.pfv-modq}	\tsc{m-}say\tsc{.ipfv-icvb}	exist\tsc{-m}\\
	\glt	\sqt{‎This is when he is probably thinking, should I steal or not.}
\end{exe}

The \isi{modal interrogative} of the verb \tit{b-iχʷ-} (\tsc{pfv}) \sqt{be, become, be able} is also used in epistemic modal constructions (\refsec{ssec:Epistemic modal constructions}).
