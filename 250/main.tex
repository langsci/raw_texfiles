%%%%%%%%%%%%%%%%%%%%%%%%%%%%%%%%%%%%%%%%%%%%%%%%%%%%
%%%                                             			%%%
%%%        Language Science Press Master File       	%%%
%%%             follow the instructions below        	%%%
%%%                                              			%%%
%%%%%%%%%%%%%%%%%%%%%%%%%%%%%%%%%%%%%%%%%%%%%%%%%%%%
 
% Everything following a % is ignored
% Some lines start with %. Remove the % to include them

\documentclass[output=book,
  nonflat,
  modfonts,
%   draftmode,
  smallfont,
%   colorlinks,
%   linkcolor=teal,
%   urlcolor=purple,
%   citecolor=blue,
%   showindex,
    arseneau
		  ]{langsci/langscibook}    


\author{Hossep Dolatian}
\title{Adjarian's {\itshape Armenian dialectology} (1911)}
\subtitle{Translation and commentary}

\renewcommand{\lsSeriesNumber}{4}
\renewcommand{\lsSeries}{loc}

\BackBody{Armenian is an Indo-European language that boasts a rich linguistic landscape comprising Classical Armenian (CA), Standard Western Armenian (SWA or WA), Standard Eastern Armenian (SEA or EA), and numerous non-standard dialects, many of which were tragically lost due to the Armenian Genocide. This book is an English translation and commentary of Hratchia Adjarian's seminal work \armenian{Հայ բարբառագիտութիւն} \textit{Armenian dialectology}, originally written in Armenian in 1911. Adjarian describes 31 non-standard Armenian varieties, offering insights into their linguistic structures and historical roots. To enhance accessibility and understanding, this translation unpacks implicit knowledge embedded in Adjarian's text, providing morpheme segmentation, glossing, and translations. This translation is tailored for three distinct audiences: linguists of non-Armenian, traditional Armenian dialectologists, and linguists of Armenian who were trained outside Armenia. This translation aims to bridge linguistic methodologies and facilitate deeper comprehension of Armenian dialectology. The translator supplements Adjarian's prose with commentary, ensuring clarity and accessibility across diverse readerships. This translation provides access to a linguistic landscape of Armenian before the genocide, with the hope of fostering broader scholarly engagement on Armenian dialects.}

\renewcommand{\lsID}{385}
\renewcommand{\lsISBNdigital}{978-3-96110-489-5}
\renewcommand{\lsISBNhardcover}{978-3-98554-118-8}
\BookDOI{10.5281/zenodo.14008766}
\typesetter{Sebastian Nordhoff}
\proofreader{Yasna}

% add all extra packages you need to load to this file

\usepackage{tabularx,multicol}
\usepackage{url}
\urlstyle{same}

\usepackage{listings}
\lstset{basicstyle=\ttfamily,tabsize=2,breaklines=true}

\usepackage{langsci-basic}
\usepackage{langsci-optional}
\usepackage{langsci-lgr}
\usepackage{langsci-osl}
% \usepackage{./langsci/styles/langsci-lgr}
% \usepackage{./langsci/styles/langsci-osl}
% \usepackage{langsci-gb4e}

\usepackage{tikz}
\usetikzlibrary{patterns,calc}
\pgfdeclarepatternformonly{south east lines}{\pgfqpoint{-0pt}{-0pt}}{\pgfqpoint{3pt}{3pt}}{\pgfqpoint{3pt}{3pt}}{
    \pgfsetlinewidth{0.6pt}
    \pgfpathmoveto{\pgfqpoint{0pt}{3pt}}
    \pgfpathlineto{\pgfqpoint{3pt}{0pt}}
    \pgfpathmoveto{\pgfqpoint{.2pt}{-.2pt}}
    \pgfpathlineto{\pgfqpoint{-.2pt}{.2pt}}
    \pgfpathmoveto{\pgfqpoint{3.2pt}{2.8pt}}
    \pgfpathlineto{\pgfqpoint{2.8pt}{3.2pt}}
    \pgfusepath{stroke}}
    
\usepackage{stmaryrd}
\usepackage{wasysym}
\usepackage{multirow}
\usepackage{caption}
\usepackage{subcaption}
\usepackage{mathrsfs}
\usepackage{qtree}

\usepackage{linguex}


%% hyphenation points for line breaks
%% Normally, automatic hyphenation in LaTeX is very good
%% If a word is mis-hyphenated, add it to this file
%%
%% add information to TeX file before \begin{document} with:
%% %% hyphenation points for line breaks
%% Normally, automatic hyphenation in LaTeX is very good
%% If a word is mis-hyphenated, add it to this file
%%
%% add information to TeX file before \begin{document} with:
%% %% hyphenation points for line breaks
%% Normally, automatic hyphenation in LaTeX is very good
%% If a word is mis-hyphenated, add it to this file
%%
%% add information to TeX file before \begin{document} with:
%% \include{localhyphenation}
\hyphenation{
    Beck-man
    Ngu-yen
    back-chan-nel
    back-chan-nels
    mo-not-o-nous
    ste-reo-typ-i-cal
}

\hyphenation{
    Beck-man
    Ngu-yen
    back-chan-nel
    back-chan-nels
    mo-not-o-nous
    ste-reo-typ-i-cal
}

\hyphenation{
    Beck-man
    Ngu-yen
    back-chan-nel
    back-chan-nels
    mo-not-o-nous
    ste-reo-typ-i-cal
}

\bibliography{SanzhiBibtex} 

\begin{document}     
%pminos do not split footnotes
% \interfootnotelinepenalty=10000 %Footnote in Laporte chapters has to be split SN


%\DeclareIndexNameFormat{default}{%
%\nameparts{#1}%
%\usebibmacro{index:name}%
%{\index[names]}%
%{\namepartfamily}%
%{\namepartgiveni}%
% {}% L1
% {}% L2
%{\namepartprefix}% generates spurious space L3
%{\namepartsuffix}% generates spurious space L4
%}

%  {\DeclareIndexNameFormat{default}{%
%     \usebibmacro{index:name}{\index[names]}{#1}{#3}{#5}{#7}}}

%\DeclareIndexNameFormat{default}{%
%  \usebibmacro{index:name}{\sindex[nom]}{#1}{#3}{#5}{#7}}

%\DeclareIndexNameFormat{default}{%
%  \usebibmacro{index:name}{\sindex[person]}{#1}{#3}{#5}{#7}}
%\DeclareIndexNameFormat{default}{%
%\nameparts{#1} \usebibmacro{index:name}{\sindex[person]]}{\namepartfamily}{‌​\namepartgiven}{\nam‌​epartprefix}{\namepa‌​rtsuffix}}

%\newcommand{\smiley}{:)}

%\renewbibmacro*{index:name}[5]{%
%\usebibmacro{index:entry}{#1}%
%{\iffieldundef{usera}{}{\thefield{usera}\actualoperator}\mkbibindexname{#2}{#3}{#4}{#5}}}

% \newcommand{\noop}[1]{}

%remove for final
%\overfullrule=1mm

\newcommand{\tobi}[2]}}
\renewcommand{\S}[1]{\tobi{#1}{\textsc{*}}}

% this volume references
% puts: [this volume]
% already defined: \citetv
%\newcommand{\citepv}[1]{(\citeauthor{#1} \citeyear*{#1} [this volume])}
\newcommand{\citealtv}[1]{\citeauthor{#1} \citeyear*{#1} [this volume]}

%parentheses around example number
\newcommand{\pref}[1]{(\ref{#1})}

% in-text examples

\newcommand{\lnex}[1]{\textit{#1}} %target lang word
\newcommand{\lnlit}[1]{(lit.: `#1')} %literal reading
\newcommand{\lnlat}[1]{(#1)} % latinization
\newcommand{\lntrans}[1]{`#1'} %translation
\newcommand{\lnexl}[2]%
{\lnex{#1}{} \lnlat{#2}} % ex with latinization
\newcommand{\lnexlat}[3]{\lnex{#1}{} \lnlat{#2}{} \lntrans{#3}} % ex with latinization and tranl.

%ch01
\newcommand{\co}[1]{\mbox{\textbf{#1}}}

%ch09

\newcommand{\cyrbulg}[1]{\begin{otherlanguage*}{bulgarian}#1\end{otherlanguage*}}


%ch10
\newcommand{\nlp}{{\small NLP}}
\newcommand{\mwe}{{\small MWE}}
\newcommand{\rae}{{\small RAE}}
\newcommand{\lvc}{{\small LVC}}
\newcommand{\pos}{{\small P}o{\small S}}
%\newcommand{\todo}[1]{ \textcolor{red}{#1} }

%\renewcommand{\labelenumi}{\theenumi}
%\ainamefmt{{vv}{ll}{, ff}{, jj}} % fullname

\newcommand{\biberror}[1]{{\color{red}#1}}

\newcommand{\osenovaitem}{--~}
 

\maketitle
\frontmatter
% %% uncomment if you have preface and/or acknowledgements

\currentpdfbookmark{Contents}{name} % adds a PDF bookmark
\tableofcontents
\addchap{Preface}

This book has had a rather long gestation period. The papers in this collective volume originate in
a selection of the papers that were presented at a workshop at Freie Universität Berlin in May
2017, which was jointly organised by Ulrike Freywald, then University of Potsdam, and Horst Simon,
FU Berlin. In an elaborate and anonymous peer review process, the papers were selected and revised
under the guidance of UF and HS, and finally refined and brought into their present shape by Stefan
Müller, who had also presented at the workshop, and his team at HU Berlin. 

We thank various institutions and individuals, above all the Deutsche Forschungsgemeinschaft and the
Ernst-Reuter-Gesellschaft der Freunde, Förderer und Ehemaligen der Freien Universität Berlin
e.V. for their financial support of the workshop, then a host of colleagues, who need to remain
anonymous, who have offered their expertise when peer reviewing, and finally a group of student
assistants in the final process of polishing the papers into publishable form, namely Elisabeth
Eberle, Luisa Kalvelage, and Julie Täge.

It is to be hoped that the present volume will testify more to the potential headlessness of the
structures discussed than to the headlessness of the pondering linguist. 

\bigskip
\noindent
Berlin \& Dortmund, \today \hfill \begin{tabular}[t]{@{}r@{}}
                                  Ulrike Freywald\\
                                  Horst J.\ Simon \\
                                  Stefan Müller
                                  \end{tabular}





\addchap{\lsAcknowledgementTitle} 

\textit{Núu'etaa numá'kaaki wáakapusmak wakápusa wakų́'ro'sh}. 

This book would not have been possible without the generosity and foresight of various Mandan speakers who gave their time and energy into recording their language and their knowledge. Some of the key Mandan individuals who have provided to this corpus, which extends back over a century and a half, include: Mrs. James Kipp (a.k.a., Earth Woman), Mrs. Holding Eagle (a.k.a., Scatter Corn), Mr. Ben Benson (a.k.a., Buffalo Bull Head), Mr. Flat Bear (a.k.a., Bear on the Flat), Mr. Walter Face (a.k.a., Wounded Face), Mrs. Little Crow (a.k.a., Otter Woman), Mr. Paul Crows Heart (a.k.a., Crows Heart), Mr. Sitting Rabbit, Mr. Little Crow, Mr. Wolfs Head, Mr. Wolf Ghost, Mr. White Calf, Mr. Foolish Woman, Mr. Little Owl, Mrs. Good Bear, Mrs. Calf Woman, Mrs. Owen Baker, Mrs. Front Woman, Mr. Wolf Chief, Mrs. White Duck, Mrs. Edna Face, Mrs. Nora Baker, Mr. Stephen Bird, Mr. Mark Mato, Mrs. Bessie Medicine Stone, Mrs. Mary Atkins, Mrs. Blanche Benson, Mr. Burr Crows Breast, Mrs. Annie Eagle, Mrs. Mattie Grinnell, Mr. Albert Little Owl, Mr. Ralph Little Owl, Mrs. Otter Sage, Mr. John Stone, Mr. Clyde Baker, Mr. Jacob, Bird, Mr. William Bell, Mrs. Louella Benson Young Bear, Mr. Ernest Medicine Stone, Mr. Carl Whitman, Ms. Ann Solano, Mr. Leon Little Owl, Mr. Corey Spotted Bear, and Mr. Edwin Benson.

I also gratefully acknowledge that I was able to assemble the materials and information found in this book over the years thanks to the generous financial support of the American Philosophical Society's Phillips Fund for Native American Research, Yale University's A. Richard Diebold Jr. Graduate Fellowship and the Frederick W. Hilles Memorial Scholarship Fellowship, Northeastern Illinois University's Dr. Bernard J. Brommel Doctoral Scholarship, and freelance employment by the Language Conservancy through its contract with the Mandan-Hidatsa-Arikara Nation's Department of Education. I also could never have completed this book without the support provided to me by my employer, the University of Oklahoma, where I had been afforded a semester of teaching release after my third year to ensure that this book could be completed.

My long-time advisor at Yale, Prof. Stephen Anderson, saw many early versions of papers and projects that led to what is now this book. His guidance was instrumental to the theoretical conclusions I made in my dissertation, and that dissertation formed the backbone of the present monograph. Prof. Claire Bowern, my other dissertation co-chair and professor was likewise key to my ability to complete my dissertation, especially after I left New Haven and worked on my dissertation \textit{in absentia} for several years. She held me accountable to making progress, and I deeply appreciate the time and energy she spent in getting me across the finish line. Prof. Natalie Weber, a dissertation committee member, looked over many versions of the analyses that became the chapter on Mandan phonetics and phonology in this book. I am grateful for their time in walking me through more effective ways of presenting the data there. Prof. Marianne Mithun, as a member of my dissertation committee, provided valuable insight into how different processes in Mandan worked and how they fit into the typology of North American languages. Her own experience in studying Siouan languages and her breadth of knowledge on language change improved how I presented the morphological data in what has become the verbal morphology chapter of this book.

On a more personal level, my parents Raymond and Kathleen Kasak have contributed to much to the evolution of this book by asking me repeatedly over the years, ``is it done yet?'' Since the spring of 2015, many phone conversations ended with a question as to the status of this work. First, they asked when I was working on my dissertation, which this book uses as a launchpad for a more comprehensive grammar. Later, as I was preparing this manuscript for submission to Language Science Press, they would ask me whether I was done with this chapter or that chapter yet. Any academic surely understands how incredibly welcome and not at all stressful such questions were.

The friends I made at my M.A in Linguistics at Northeastern Illinois University helped get me to Yale and were, therefore, important to getting this book finished. Dr. John Boyle, first as my professor and then as my friend, introduced me to Siouan linguistics, so I could not have even begun to build the foundations of this book without him. I am thankful for him taking me to my first Siouan and Caddoan Languages Conference in Kansas and him introducing me to the late, great Bob Rankin, one of the monumental figures of Siouan linguistics. My former professors Drs. Shahrzad Mahootian and Lewis Gebhart were great moral support in convincing me to apply to grad school and checked in with me often to see how things were going. My former NEIU classmates Dr. Binh Ngo and Galini Gkartzonika have likewise been great friends and supporters throughout my time at Yale and beyond. I also could not have written this book without the editorial assistance of my wonderful friend Hadley Austin and my former student Lizz Evalen, whose proofreading of an earlier version of my dissertation has made the writing of this book much easier.

My final and most important acknowledgment goes to my partner and wife Dr. Colbi Beam, who has been a singular source of inspiration to complete this work. I am forever grateful for her love and support. I have no doubt that this book would have taken far longer to complete without her encouragement and her understanding of what it takes to write academically. Having written a dissertation herself, she understood the nature of writing expansive pieces of work and she was helpful in putting aspects of the writing process in perspective for me during times when I struggled with how to present some data or was unsure how to best resolve some impasse in writing. She has been insightful and inspirational throughout this process. Colbi, you are simply the best.
\addchap{\lsAbbreviationsTitle}
% \addchap{Abbreviations and symbols}
The category labels for abbreviations follow the Leipzig Glossing Rules.\footnote{\url{http://www.eva.mpg.de/lingua/resources/glossing-rules.php}}
\vspace{.5cm}

\begin{tabularx}{.45\textwidth}{lQ}
    ∅ & zero form \\
    \textbackslash.../ & verb stem\\
    . & multi-item gloss (3\textsc{sg.m})\\
    \_ & multi-item lexemes (like\_that)\\
    - & for affixes (\textit{-thé})\\
    = & for clitics (\textit{=en})\\
    | & syncretism (2|3 person)\\
    > & argument structure (1>3 `first person acting on third person')\\
    1 & first person\\
    2 & second person\\
    3 & third person\\
    \textsc{abs} & absolutive\\    
    \textsc{absc} & absconditive (attention getter, `look here')\\
	\textsc{adjz} & adjectivizer\\
    \textsc{all} & allative\\
    \textsc{and} & andative (`away')\\
    \textsc{anim} & animate\\
    \textsc{appr} & apprehensive\\
    \textsc{assoc} & associative case\\
    \textsc{char} & characteristic case\\
    \textsc{dat} & dative case\\
    \textsc{dem} & anaphoric demonstrative\\
\end{tabularx}
\begin{tabularx}{.45\textwidth}{lQ}
    \textsc{dia} & diathetic prefix\\
    \textsc{dim} & diminutive\\
    \textsc{dist} & distal deictic\\
    \textsc{distr} & ditributive\\
    \textsc{du} & dual number\\
    \textsc{dur} & durative\\
    \textsc{emph} & emphatic\\
    \textsc{erg} & ergative case\\
    \textsc{etc} & et cetera (`and all')\\
    \textsc{f} & feminine\\
    \textsc{fut} & future\\
    \textsc{futimp} & future imperative\\
    \textsc{iam} & iamitive (`already')\\
    \textsc{ic} & inclusory case\\
    \textsc{imn} & imminent (`about to')\\
    \textsc{imp} & imperative\\
    \textsc{indf} & indefinite\\
    \textsc{ins} & instrumental case\\
    \textsc{io} & indirect object\\
    \textsc{ipfv} & imperfective\\
    \textsc{ipst} & immediate past\\
    \textsc{irr} & irrealis\\
    \textsc{iter} & iterative\\
    \textsc{loc} & locative case\\
    \textsc{lpl} & large plural\\
    \textsc{m} & masculine\\
    \textsc{med} & medial deictic\\
\end{tabularx}
\newpage
\begin{tabularx}{.45\textwidth}{lQ}
    \textsc{nd} & non-dual\\
    \textsc{neg} & negator\\
    \textsc{nmlz} & nominalizer\\
    \textsc{npl} & non-plural\\
	\textsc{npst} & non-past\\
	\textsc{nsg} & non-singular\\
    \textsc{only} & exclusive marker (`only', `just')\\
    \textsc{pfv} & perfective\\
    \textsc{ph} & placeholder (`thingamajig')\\
    \textsc{pl} & plural\\
    \textsc{pn} & proper name\\
    \textsc{pln} & place name\\
    \textsc{poss} & possessive\\
\end{tabularx}
\begin{tabularx}{.45\textwidth}{lQ}
    \textsc{pot} & potential\\
    \textsc{priv} & privative case\\
	\textsc{prop} & proprietive case\\
    \textsc{prox} & proximal deictic\\
    \textsc{pst} & past\\
    \textsc{purp} & purposive case\\
    \textsc{q} & question\\    
    \textsc{quot} & quotative\\
    \textsc{redup} & reduplication\\
	\textsc{rpst} & recent past\\
    \textsc{sg} & singular\\
    \textsc{simil} & similative case\\
    \textsc{stat} & stative\\
    \textsc{temp} & temporal case\\
    \textsc{vent} & venitive (`towards')\\
\end{tabularx}
\mainmatter
% 
% %%%%%%%%%%%%%%%%%%%%%%%%%%%%%%%%%%%%%%%%%%%%%%%%%%%%
% %%%                                        	%%%
% %%%             Chapters		%%%
% %%%                                          	%%%
% %%%%%%%%%%%%%%%%%%%%%%%%%%%%%%%%%%%%%%%%%%%%%%%%%%%%
% 
\addchap{Introduction}

When conference interpreters interpret a speech simultaneously, they are often faced with the need to quickly and precisely render specialised terminology in the target language. Hence, pre-assignment preparation is fundamental to acquire specialised terminology to aptly express domain-specific knowledge in the target language. In the preparation phase, interpreters usually compile glossaries containing specialised terms likely to be used by the speaker (e.g. \citealt{rutten_informations-_2007,fantinuoli_computer-assisted_2017,will_terminology_2007,gile_basic_2009}). Despite learning the terminological equivalents ahead of the event, interpreters may not always be able to retrieve the target-language equivalent from memory during interpreting. To cope with this difficulty, among other tactics \citep[14]{gile_basic_2009}, they may choose to look up the required term in their glossaries. Traditionally, specialised glossaries have been compiled on paper, or prepared in digital format and printed out for the booth (e.g. \citealt{jiang_interpreters_2013,jiang2015survey}).

With the increasing permeation of the profession by technology over the past couple of decades, the booth is now increasingly paper-less \citep{rutten2017terminology}. On a laptop or a tablet, interpreters can now conduct glossary queries in the digital medium.

As for the software employed to create digital glossaries for interpreting assignments, traditionally this has consisted in text processing programmes or database applications aimed at the general public. For lack of dedicated tools, some interpreters have resorted to computer-assisted translation (CAT) tools. The use of tools for corpus-based terminology work, terminology extraction, and of speech recognition, to name but a few examples, is therefore not the exclusive preserve of translators and terminologists but can also be found in interpreters' terminology work.

Despite the potential usefulness of these technologies for interpreters, scholarship remarked that the nature of interpreting imposes specific demands, both cognitive and related to interpreters' workflow (e.g. \citealt{rutten_why_2004,will_bemerkungen_2000,will_terminology_2007}). Thus, around the same time when CAT tools started to appear, applications geared towards the specific needs of interpreters were created and later increasingly refined which fall under the name of computer-assisted interpreting (CAI) tools.\footnote{For a terminological clarification of the term ``CAI tool'' and its use in the present work, see \sectref{CAIdef}.} Their aim is to support interpreters along several phases of their workflow, especially during preparation, but also for terminology retrieval during the interpreting task. The recent advances in automatic speech recognition (ASR) technology have motivated its integration into CAI tools, which may now offer live support for terminology and other units of information without physical interaction between the interpreter and the machine. The first prototypes of ASR-enhanced CAI tools are already emerging (e.g. \citealt{fantinuoli_computer-assisted_2017}).

In light of these developments, several studies have been conducted on the topic of CAI tools over the past few years. They have mainly explored the tools' potential to improve terminological accuracy during simultaneous interpreting (e.g. \citealt{prandi_uso_2015,prandi_use_2015}), the extent to which ASR improves the rendition of number words and specialized terms in the target language (e.g. \citealt{defrancq_automatic_2020}), or their potential for offering support to interpreters in the consecutive mode (e.g. \citealt{wang2019can}). With a few exceptions \citep{biagini_glossario_2015,frittella_cai-supported_2021}, the focus of such studies has been rather narrow, using specific performance indicators such as the accuracy of interpreted terms and numerals to assess the tools' impact on the overall quality of the interpretation (e.g. \citealt{pisani_measuring_2021}), without taking stock of the interpretation beyond these individual items. Many findings have emerged from small-scale experiments conducted on students in the framework of master's theses (e.g. \citealt{canali_utilizzo_2018,van_cauwenberghe_etude_2020}). Despite the emphasis on the postulated difficulty of integrating CAI tools into the interpreting process and, on the other hand, the widespread enthusiasm for the potential of ASR to alleviate cognitive load during SI, the impact of such solutions on the cognitive subprocesses underlying SI has so far remained largely unexplored. This represents an evident lacuna compared to the large body of research conducted on cognition in the translator-machine interaction. Such interaction has been addressed by numerous empirical studies in the area of Translation Process Research (TPR) from multiple perspectives and with a variety of methods \citep[xvii]{carl_measuring_2021}. Indeed, as ``a research tradition within cognitive translation studies (CTS) [...] exploring factors that determine human translation behavior'' (ibid.), TPR may constitute a valuable reference point and provide useful tools to the analysis of computer-assisted simultaneous interpreting (CASI), especially from a methodological standpoint. However, no empirically-validated methodology for the combined collection of product- and process-oriented data with a markedly cognitive focus has yet been developed to explore the phenomenon of technology-supported SI.

The present doctoral thesis seeks to address this limitation by developing and testing an empirical methodology for a cognitive exploration of CASI. In particular, the present work derives its methods from TPR to analyse the impact on cognitive load of different forms of digital terminological support for interpreters through a within-subject experimental study.

To the best of my knowledge, this is the first study to address com\-put\-er-as\-sist\-ed SI from a cognitive perspective. As such, it presents an exploratory character which aims to provide first findings and, at the same time, to identify open questions and formulate hypotheses for further investigations of the phenomenon.

The following section establishes the paradigm for the present research work and provides a conceptual framework for the cognitive inquiry into the phenomenon of CASI. The rest of the present chapter illustrates how the present work is organised and briefly describes the content of each chapter.


\section*{Choice of paradigm for the present study}\label{paradigm}

Especially in the simultaneous mode, interpreting has often been described as a complex cognitive activity, involving concurrent information processing and temporary storage tasks competing for attentional resources. This view of interpreting\footnote{Here intended in the broader sense of the term and referring not only to the conference setting.} as cognitive information processing represents one of the ``supermemes'' of interpreting, as observed by \citet[51]{pochhacker_introducing_2004}\footnote{Pöchhacker derives the notion of memes and supermemes from \citet{chesterman_memes_2016}: as in translation, these ``socio-biological concepts'' have arisen as metaphors to illustrate particular views of interpreting as an object of study. From this perspective, the interpreter is seen as a ``human processor'' performing several ``cognitive skills […] the combination of which would account for the complex task of interpreting'' \citep[53]{pochhacker_introducing_2004}.}.

Yet, even though this ``internal'' perspective has been particularly prolific in the academic inquiry into the phenomenon, interpreting may also be viewed as a socially embedded human activity, situated in a real communicative context. The supermeme of communicative activity elucidates interpreting as a combination of listening and speaking aimed at facilitating communication beyond the linguistic barrier.

For the scope of the present study, I chose to conduct my inquiry into interpreting from an explicitly cognitive perspective. Selecting one perspective does not, however, mean discarding or denying the other, but rather focusing the spotlight on one aspect of this multi-faceted activity, and is necessary to establish the theoretical framework guiding investigation. The next step lies into the definition of my research paradigm\footnote{See \citet{pochhacker_introducing_2004,pochhacker_introducing_2016} for a detailed account of research paradigms in Interpreting Studies.}. The two supermemes of interpreting as cognitive information processing and as communicative activity are at the core of a number of paradigms that can be divided into social, psycholinguistic, and cognitive approaches following Setton's classification \citep{setton_models_2003}.

The social approach reflects a tradition which looks at interpreting within a broader framework including social and behavioural factors. Prominent issues deriving from this view of interpreting are the interaction between the actors involved in the communicative event, the role of the interpreter and the long-standing issue of neutrality, the view of quality assessment as a question of pragmatics (and not only of identity between the source text and the interpreted message), but also the issue of interpreting strategies. To this category may be ascribe the "target-text oriented, translation-theoretical" \citep[77]{pochhacker_introducing_2004} paradigm exemplified by \citet{salevsky_probleme_1987}, \citet{schjoldager_an_1995/2002}, \citet{pochhacker_simultandolmetschen_1994} and \citet{kalina_strategische_1998}, as well as the ``dialogic discourse-based interaction'' paradigm \citep[79]{pochhacker_introducing_2004} of which \citet{roy_interactional_1996, roy_interpreting_2000} and \citet{wadensjo_double_1993, wadensjo_interpreting_1998} are the most prominent representatives.

Psycholinguistic approaches are grounded in theories of communication and focus on features of discourse rather than on the cognitive processing of interpreting. This perspective is at the core of the interpretive paradigm pioneered by \citet{seleskovitch_interpretation_1976} and \citet{lederer_traduction_1981} with their \textit{théorie du sens}. The models by \citet{chernov_semantic_1979, lambert_message_1994}, \citet{dejean_le_feal_satzsegmentierung_1980, dejean_le_feal_lectures_1981}, \citet{donovan_fidelite_1990}, \citet{laplace_theorie_1994} and \citet{setton_simultaneous_1999} himself may also be included in this category.

Finally, interpreting has been studied from a cognitive perspective and viewed as a matter of information processing. The scholars belonging to this tradition focus on the exploration of the cognitive underpinnings of interpreting. It is not by chance that they derive their methods of investigation from the cognitive sciences. This third approach has generated the cognitive processing paradigm \citep[73]{pochhacker_introducing_2004} initiated by \citet{gerver_empirical_1976} and further exemplified by \citet{lambert_information_1988}, \citet{gerver_information-processing_1978}, \citet{moser_simultaneous_1978} and \citet{hauenschild_process_1997}, \citet{kurz_simultandolmetschen_1996}, \citet{shlesinger_strategic_2000} and \citet{gile_partage_1988, gile_conference_1997, gile_testing_1999} and \citet{seeber_thinking_2007,seeber_cognitive_2011,seeber_multimodal_2017}. To this perspective one can also ascribe the ``neurophysiological/neurolinguistic paradigm'' \citep[75]{pochhacker_introducing_2004} exemplified by \citet{lambert_neurological_1994}, \citet{snell-hornby_look_1994,kurz_simultandolmetschen_1996}, \citet{lambert_non-linguistic_1994,daro_experimental_1997}, \citet{petsche_brain_1993}, \citet{rinne_translating_2000}, and \citet{tommola_mental_1990}.

The focus of the present doctoral thesis lies on the way the use of digital terminological support tools during simultaneous interpreting affects the cognitive processes involved in simultaneous interpreting. Hence, I situate my inquiry in the cognitive processing paradigm, while adopting an interdisciplinary approach to the exploration of simultaneous interpreting with digital terminology support. Methodologically, the study draws heavily on research methods developed and validated in the framework of empirical Translation Process Research (TPR). As translation and interpreting are rather similar activities in terms of their underlying cognitive processes and cognitive control functions, the approaches developed in TPR for the exploration of the translation process are expected to provide a valuable methodological reference point.\label{paradigmEnd}


\section*{Organisation}

The present thesis is organised as follows. \chapref{chapter1} introduces the topic of terminology in interpreting. Specifically, it addresses terminology work in conference interpreting highlighting its commonalities and differences with terminology work in translation. Additionally, it underlines the role of terminology as an important quality factor in translation and interpreting, which motivates the focus on terminological support in the present study. The last section closes \chapref{chapter1} with a discussion of the requirements for CAI tools and of the potential and limitations of non-bespoke terminology tools for interpreters. Against this background, \chapref{chapter2} frames the technology object of inquiry, i.e. CAI tools, within the larger framework of technology applied to interpreting. After an overview of the available technologies for interpreting (\sectref{tech_overview}), CAI tools are discussed in detail (\sectref{CAI_overview}), and InterpretBank, the CAI tool chosen for the experiment, in particular (\sectref{IB}). The chapter offers a review of interpreters' practices in compiling terminological resources ahead of and during the interpreting assignment, specifically in terms of their level of computerisation and choice of tools. This section closes with a review of how CAI tools have been studied in interpreting research thus far (\sectref{CAI_evaluation}) and illustrates current attitudes towards CAI tools (\sectref{CAI_attitudes}), which motivates the present work. Hence, in \chapref{chapter3} I proceed to discuss simultaneous interpreting as a complex cognitive activity, specifically as a question of attention allocation and resource sharing between co-occurring subtasks (\sectref{workingmemory}), a key issue in the inquiry into technology-supported (simultaneous) interpreting. \sectref{CL} illustrates cognitive load as a fundamental construct often encountered in academic discourse around CAI tools, but not yet explored experimentally. These two sections pave the way for the discussion of interpreting as an issue of multi-tasking within the area of Interpreting Studies (\sectref{interpreting_cognitive}). Here, I illustrate in detail two models of SI which address this activity from the perspective of the concurrent performance and coordination of cognitive sub-tasks. On this basis, I motivate the choice of framework to operationalise hypotheses on SI with digital terminological support (\sectref{choiceofmodel}). \chapref{chapter4} describes the methods adopted in TPR and neighbouring disciplines (e.g. cognitive psychology) to measure cognitive load. \chapref{chapter5} presents and discusses the research approach (\sectref{approach}) and the methodology deployed to test the hypotheses formulated for the present study (\sectref{hypotheses}). \sectref{pilot_study} presents the methods and results of the pilot study conducted to test the research methodology and validate the stimuli to be used for data collection. \sectref{main_study} goes into the details of the experimental design adopted in the main study, describing the adaptations conducted in light of the results of the pilot test. In \chapref{chapter6}, the results of the experiment are presented and discussed against the background of the hypotheses formulated in \sectref{hypotheses} and of relevant publications in the area of TPR and CAI research. \sectref{validation} discusses and validates the application of \citet{seeber_thinking_2007,seeber_cognitive_2011,seeber_multimodal_2017} CLM of SI to illustrate task interference and cognitive load in SI with the support of traditional digital glossaries, CAI tools with manual look-up, and ASR-enhanced CAI tools. The limitations of the present study are addressed in \sectref{limitations}. \chapref{conclusions} presents the methodological, didactic and practical implications of the present study and concludes this work with final remarks on potential avenues for future CAI research.

% 
\part{Phonology}
 \chapter{Phonology}\label{ch:2}
\hypertarget{Toc115517752}{}
This chapter begins by introducing the phonemic inventory of Sherbro, first vowels then consonants, discussing some of the variation and analytical problems in their treatment. It next turns to suprasegmental phenomena, tone and syllable structure, and concludes with a presentation of Sherbro's phonological rules.

\section{Phonemic inventory}
\label{sec.2.1}\hypertarget{Toc115517753}{}
The phonemic inventory of Sherbro follows the pattern of other Bolom languages with seven vowels arranged in a symmetrical pattern in the vowel space. Consonants also conform to genetic and areal patterns in the presence of labialvelars and prenasalized stops, though the latter have some distributional peculiarities and are generally voiceless.

\subsection{Vowels}
\label{sec:2.1.1}\hypertarget{Toc115517754}{}
Sherbro vowels are spread out evenly in the accoustic-perceptual vowel space, as represented in the schematic diagram in \tabref{tab:phon:5}, a not uncommon pattern in the area and among Sherbro's closest relatives.

\begin{table}
\caption{\label{tab:phon:5}Sherbro vowels}

\begin{tabular}{lll}
\lsptoprule
 i &  & u\\
 e &  & o\\
 ɛ &  & ɔ\\
& a & \\
\lspbottomrule
\end{tabular}
\end{table}

\subsubsection{/i/}
\label{sec:2.1.1.1}
The high front vowel has no significant allophones in open syllables, being realized invariably as [i]. In closed syllables, as is the case with the other mid to high front vowels, they are lowered and/or centralized; thus there are allophones [ɪ] and [ɨ] in closed syllables. The third example in \tabref{tab:phon:6}, the word for ‘both' is syllabified \textit{li.tiŋ}; only the second /i/ is in a closed syllable and thereby centralized.

\begin{table}
\caption{\label{tab:phon:6}Centralized allophones of /i/}

\begin{tabular}{lll}
\lsptoprule
/bik/ & [bɪk] & ‘a type of mat'\\
/kil/ & [kɪl] & ‘house'\\
/vis/ & [vɨs] & ‘meat, animal'\\
/litiŋ/ & [litɪŋ] & ‘both'\\
\lspbottomrule
\end{tabular}
\end{table}

In borrowings from English, the high vowel [i] is realized as [e] in Sherbro, thus suggesting that the Sherbro vowels /i/ and /e/ are both higher than their English counterparts. Though no acoustic measurements were made, impressionistically the Sherbro [e] does sound higher than its English counterpart.

\ea %14
\label{ex:14}
%(\stepcounter{qwerty}{\theqwerty})  
Kɔfe lɛ hɔ lol.\\
\gll kɔfe    lɛ    hɔ    lol\\
coffee  \textsc{def}  \textsc{ncp}\textsubscript{hɔ}  bitter\\
\glt ‘The coffee is bitter.' (P67 L: 104)\footnote{The material in parentheses indicates the example is from Walter Pichl's 1967\ia{Pichl, Walter} Sherbro-English Dictionary (see \sectref{sec:2.1.1}). The letter indicates the section, and the following number is the specific entry. Thus in \REF{ex:14}, the example sentence is from the entry for the word \textit{lol} which is the 104\textsuperscript{th} entry under the letter L.}
\z

\subsubsection{/e/}
\label{sec:2.1.1.2}
In open syllables, both mid front vowels have higher phonetic values than are indicated by their symbols. In closed syllables, they have centralized allophones [ɪ] and/or [ə], more often the latter, particularly with a coda filled by a liquid or nasal.

\begin{table}
\caption{\label{tab:phon:7}Centralized allophones of /e/}


\begin{tabular}{lll} 
\lsptoprule
/len/ & [lɪn] & ‘thing'\\
/yel/ & [yɪl] & ‘boil' (v.)\\
/kel/ & [kəl] & ‘monkey, bite'\\
\lspbottomrule
\end{tabular}
\end{table}

One example of [ə] in an open syllable and not in a closed one is [peŋkə] ‘first' (also heard as [peŋkɛ]).

Both of the lower and higher mid vowels [e ɛ] and [o ɔ] are higher than their English counterparts, and definitely are not diphthongized. When a word such as \textit{plane} ([pleɪn]) is borrowed into Sherbro, the vowel is reanalyzed as the lower mid vowel /ɛ/ rather than the upper mid vowel /e/.

%(\stepcounter{qwerty}{\theqwerty}) 
\ea %15 
\label{ex:15}
Plɛn dɛ kɔn poto kɛthkɛth hink Kyamp ka.\\
\gll plɛn  dɛ    kɔn  poto    kɛthkɛth    hink  Kyamp    ka\\
plane \textsc{def}  go    Europe  often      from  Freetown  here\\
\glt ‘The plane goes frequently from Freetown to Europe.' (P67 K: 114)
\z

These facts have led to some confusion in the colonial orthography which has persisted to this day. Westerners, including many mapmakers, have confused /u/ and /o/ in particular. For example, the name of the language, here rendered <Bolom>, is often spelled “Bullom,” and [Bom], the name of a river and once the name of a language have both been written “Bum” now recognized as “Bom-Kim\il{Bom-Kim}” (\citealt{Childs2020}). The name of the major girls' initiation society ‘Bondo\il{Bondo}' is spelled <Bundu>, and the port of ‘Tombo' is often rendered as <Tumbu>.

To a lesser extent the front vowels /i/ and /e/ are also confused.

\subsubsection{/ɛ/}
\label{sec:2.1.1.3}
The centralized [ə] occurs in more environments as an allophone of /ɛ/ than it does as an allophone of /e/.

\begin{table}
\caption{\label{tab:phon:8}Centralized allophones of /ɛ/}

\begin{tabular}{lll}
\lsptoprule
/ayɛn/ & [ayən] & ‘truly'\\
/pɛl/ & [pəl] & ‘fishing net, hammock'\\
/pɛmplɛ/ & [pəmp.lɛ] & ‘stalk' (v.)\\
/yɛk/ & [yək] & ‘bedbug' \\
/bɛth/ & [bət̪] & ‘cut' (v.)\\
\lspbottomrule
\end{tabular}
\end{table}

In all cases the schwa allophone is unrounded but, in some cases, can be rhotacized, having almost a retroflex quality: [wantsɚ, wantsɚmi] ‘sister, my sister,' even without a coda consonant.

Diachronic records show that the alternation is possibly a recent phenomenon. \citet{Sumner1921}  gives, for example, ‘earthenware pot' as [bɛl], yet the form is [bəl] in both \citet{Pichl1967} and my own data from 2015-2016. Another explanation would be that Sumner was hearing the sounds phonemically; he was a native speaker.\footnote{Sumner was also a minister in the Methodist Church.}

The vowel [ə] is therefore not considered an independent phoneme, as it is in Mani\il{Mani}. The central vowel allophones range over a considerable part of the acoustic-perceptual vowel space, ranging from a high central allophone [ɨ] to a lower central one [ə], as well as a front centralized [ɪ] and a rhotacized [ɚ]. Despite the many pronunciations, speakers had no problems identifying the sound as one of the phonemic vowels, usually /e/ or /ɛ/. Furthermore, the Sherbro Literacy Committee\is{Sherbro Literacy Committee!} and Methodist missionaries have not used <ə> in the writing system that they have developed and thus do not see it as a separate phoneme.\footnote{Most of the committee's members are native speakers.} \citet{Sumner1921} has included the sound but does not comment on its phonemic status, using a quasi-phonetic system for his transcriptions although \citet{Pichl1967} treats it as an allophone of /ɛ/.

Vowel length is contrastive, and all vowels have long counterparts. However, vowel length is only contrastive for monosyllables or on the first syllable of a disyllabic word. For example, the word for ‘bird' is [vee], but when it appears in a compound, the vowel is shortened: [vebolmin] ‘swallow,' lit. ‘crazy bird.'

On the phonetic side, vowels are compensatorily lengthened before an epenthetic (prenasalized) stop, thus [mbaŋkathoːm ndɛ] from /mbaŋkathom + ɛ /. Below appear some phonemic contrasts in length for which there are a great many minimal pairs.

\ea %16
\label{ex:16} Length contrasts in vowels\\
%(\stepcounter{qwerty}{\theqwerty}) 
\vspace{6pt}
\begin{tabular}[t]{llll}
ha & ‘for' & haa & ‘do' \\
yɛk & ‘bedbug'  & yɛɛk & ‘spoon' \\
ya & ‘\textsc{1sg}'  & yaa & ‘mother'\\
vɛ & ‘that one' & vɛɛ & ‘stone' (v.)\\
ve & ‘be well'   & vee & ‘oyster'\\
ku & ‘call, name'   & kuu & ‘property, estate'\\
gbɛŋ & ‘tomorrow' & gbɛɛŋ & ‘glory'\\
\end{tabular}
\z

Long consonants occur only across morpheme boundaries, typically as a result of syllable restructuring (see \sectref{sec:2.1.2} for some examples of geminates and \sectref{sec:2.3} for syllable restructuring in general).

As with other Bolom languages, there are alternations between front and back vowels: i-u, e-o, and ɛ-ɔ. Speakers accept both front and back alternants for a number of forms, suggesting free variation, despite the phonemic status of the opposition between front and back vowels. For example, the pronunciation of /jo/ ‘eat' was equally acceptable as either [jo] or [je] to some speakers in Shenge\is{Shenge}. As another example, both consultant Adama Mampa and research assistant Abdulai Bendu thought that the pronunciation of ‘tomorrow' as either [gbɛŋ] or [gbɔŋ] was acceptable.

\ea %17
\label{ex:17}
Some front-back alternations in free variation\\
%(\stepcounter{qwerty}{\theqwerty}) 
\vspace{6pt}
\begin{tabular}{ll}
sikɔ / sukɔ & ‘the mast of a ship'\\
wei / woi & ‘fear'\\
thɛm / thɔm & ‘friend'\\
wɛi / wɔi & ‘bad, be ugly'\\
pɛ / pɔ & ‘people'; \textsc{pro}\textsubscript{indef}
\end{tabular}
\z

The variation between the front and back alternants seems to be unconditioned.\footnote{These alternations lead to some variant spellings.} In at least one other closely related language, the alternations have been morphologized; they mark contrasts in the verbal morphology of Kisi\il{Kisi}. In Mani\il{Mani}, there is a trace of vowel harmony\is{vowel harmony!} in the applicative verb extension, perhaps a source for the alternations. In Sherbro, there is a limited case of vowel harmony in the past suffix and in the derivational morphology (see \sectref{sec:4.2}, \sectref{sec:7.1}).

\subsection{Consonants}
\label{sec:2.1.2}\hypertarget{Toc115517755}{}
\tabref{tab:phon:9} presents the consonants of Sherbro. I have used the orthographic symbols of the writing system rather than IPA symbols, as follows. The digraphs represent single phonemes. The symbol “v” is in parentheses because it is an allophone of /w/ but used in the writing system. The symbol “kp” in parentheses, on the other hand, represents a peripheral sound found only in a few words. The voiced prenasalized stops, also in parentheses, similarly have a limited distribution. I make only a few comments on the unusual phonetic and distributional features of Sherbro consonants. 

\begin{table}
\caption{\label{tab:phon:9}Sherbro consonants}

\begin{tabular}{ccccccc}
\lsptoprule
 Bilabial & Dental & Alveolar & Palatal & Velar & Labialvelar & Glottal\\
 \midrule
 m &  & n & ny & ŋ &  & \\
 mp, (mb) &  & nt (nd) &  & ŋk (ŋg) &  & \\
 p, b & th & t, d & ch, j & k & (kp) gb & \\
 f (v) &  & s &  &  &  & h\\
&  & r, l &  &  &  & \\
&  &  & y &  & w & \\
\lspbottomrule
\end{tabular}
\end{table}

\subsubsection{Nasals}
\label{sec:2.1.2.1}
Nasals show little variation except for the nasal assimilation described in \sectref{sec:2.4}. Coda alveolar and velar nasals nasalize preceding vowels. There is also some variation in the lexical form of the final nasal, whether it is individual or dialectal could not be determined. In fact, there is some variation within individuals. The velar nasal alternates with [h] in many forms, as discussed in Orthography and Conventions and below under /h/ (\sectref{sec:1.9} \& \sectref{sec:2.1.2}). The palatal nasal never appears in codas and is a dialectal variant of /h/ before front vowels, e.g., [nyɔl] / [hiɔl] ‘four'. For those interested in “the linguist's delight”, a minimal triplet featuring three of the nasals in word-final position is [pɛŋ] ‘boundary; jump' vs. [pɛn] ‘loud talking' vs. [pɛm] ‘war.'

\subsubsection{Prenasalized stops}
\label{sec:2.1.2.2}
Prenasalized stops in Sherbro form a special category because of their distribution and their phonetics. Although there is some dialectal variation, phonemic prenasalized stops consist of a nasal followed by a homorganic voiceless stop; “voiceless” prenasalized stops are established phonemes. In a few medial contexts, the sequence may be voiced. Voiceless prenasalized stops occur almost exclusively in syllable codas, as exemplified in \REF{ex:18}, but only at four of the six places of articulation for Sherbro stops and three of the four for nasals.


\ea  %18
%\label{bkm:Prenasalizedstopsinsyllablecodas}(\stepcounter{qwerty}{\theqwerty}) 
\label{ex:18}
Prenasalized stops in syllable codas\\
\vspace{6pt}
\TabPositions{1.25cm,2cm,3cm,5cm}

mp \tab \textit{bomp} ‘section,' \textit{kump} ‘helper; plaiting,' \textit{nrɔmp} ‘sickness'\\
nt \tab \textit{tɔnt} ‘creek,' \textit{ntɛnt} ‘near,' \textit{tunt} ‘twist'\\
nth \tab \textit{vunth} ‘push,' \textit{panth} ‘tie' (v.), ‘work' (n.), \textit{santhsanth} ‘grownup'\\
*nch\\
ŋk \tab \textit{tɔŋk} ‘praise' (v.), \textit{thuŋk} ‘deep,' \textit{thɛŋk} ‘put up,' \textit{yeŋk} ‘insect wax'\\
*mŋgb\\
\z

There is some variation between voiceless and voiced prenasalized stops in intervocalic position, e.g., ‘today' recorded as both [nante] and [nande], ‘mango' (a borrowing\is{borrowing}) as both [maŋko] and [maŋgo], with the voiceless variant being the more common one. The voiceless prenasalized stop is sometimes the only variant medially, e.g., [kaŋka] ‘so that,' [peŋka] ‘gun'. Word-initially only a few forms with voiceless prenasalized stops can be found, though they do occur as in \textit{ntent} above in \REF{ex:18} and in the evening greeting \textit{mpikɛ}. Thus, the voiceless prenasalized stops have a stronger claim to phonemic status than voiced ones, particularly in light of the latter's derived status described below.

Only at the beginning of syllables do sequences of voiced prenasalized stops occur, almost all derived. They arise due to the prefixing of a nasal morpheme with a loss of syllabicity and nasal assimilation. Otherwise, voiced prenasalized stops appear only in a few function words, names and borrowings. Voiced prenasalized stops are not considered to be independent phonemes in Sherbro.

\subsubsection{Voiceless stops}
\label{sec:2.1.2.3}
The one unusual feature of voiceless stops is that the language has a contrast between a dental stop and an alveolar one (<th> vs. <t> in the orthography).

\subsubsubsection{/t̪/ (<th>) and /t/}
\label{sec:2.1.2.3.1}
The dental stop /t̪/ has a diagnostic “tinny” sound impressionistically, distinctive from the unaspirated alveolar [t], which is often affricated, e.g., \textit{tu} ‘pound (in a mortar)' is realized as [ʧu] or [ʦu].

\TabPositions{1.25cm,3.75cm,4.75cm,6.75cm,7.75cm,8cm,9cm}
\ea %19
\label{ex:19}
%(\stepcounter{qwerty}{\theqwerty}) 
Dental and alveolar voiceless stops\\
\vspace{6pt}
thetha \tab ‘grandmother' \tab tɛtɛk \tab ‘immature rice' \tab tɛnthe \tab ‘cane stick'\\
thɔk \tab ‘stick' (n.) \tab tɔkɔ \tab ‘about'\\
thu \tab ‘spit' (v.) \tab tu \tab ‘iron pot'\\
thuk \tab ‘be warm' \tab tuk \tab ‘disappear, be lost'\\
\z

\subsubsubsection{/p/ and /k/}
\label{sec:2.1.2.3.2}
There is nothing much to say about the other voiceless stops /p/ and /k/. There are no allophones for the voiceless velar stop. When a voiced counterpart [g] appears in a borrowed word, the borrowing\is{borrowing!} is nativized with a [k], as in \textit{mango} [maŋko] above and [bek]‚ bag,' though some learned borrowings may retain the [g], e.g., \textit{gɔvana} ‘governor'. Similarly there is nothing remarkable about the voiced stops /d/, /b/, /gb/, nor about the affricates /ʧ/ and /ʤ/, although there is some dialectal variation between [ʤi] and [di] for ‘kill' (also reported in \citealt{Pichl1967}).

\subsubsection{Voiceless fricatives}
\label{sec:2.1.2.4}

The voiceless fricatives /f/ and /s/ have no allophones, except for the palatalization of /s/ to [ʃ] before the non-low front vowels.

%(\stepcounter{qwerty}{\theqwerty})   
\ea%20 
\label{ex:20}
\begin{tabular}[t]{lll}
/s/: &  [ʃ] & / {\longrule} V [-lo, -bk]\\
& [s] & elsewhere\\
\end{tabular}

\vspace{6pt}

\begin{tabular}[t]{llll}
/sii/ & [ʃii] & ‘fart' (v.)\\
/setie/ & [ʃeʧie] & ‘Settie' (a chiefdom on Sherbro Island\is{Sherbro Island})\\
/isundɛ/ & [iʃundɛ] & ‘sand' (n.)\\

/biisi/ & [biiʃi] & ‘make tight'\\
/si/ & [ʃi] & ‘know'\\
/silɔ/ & [ʃilɔ] & ‘honey, bee'\\
\end{tabular}
\z

The two examples in \REF{ex:21} come from an early source indicating that [s] / [ʃ] is a long-standing alternation, likely below the level of consciousness of speakers (vs. the dialectal variation of [h] / [w] in \REF{ex:28}) (\citealt{Sumner1921}).

\ea %21
\label{ex:21} [s] / [ʃ] alternation\\
\vspace{6pt}

%\label(\stepcounter{qwerty}{\theqwerty})    
\begin{tabular}[t]{llll}
\relax [simi] \textasciitilde{} [ʃimi]\footnotemark & ‘spoil, become rotten'\\
\text{[yikisi]} \textasciitilde{} [yikiʃi] & \textsc{idph} wiggling gait of a woman (\citealt[21]{Sumner1921})\\
\end{tabular}
\footnotetext{I have changed Sumner's <sh> to [ʃ].}
\z

There are also the alternations [sɔ]{\textasciitilde}[ʃɔ] ‘(in the) morning' and [sɔ]{\textasciitilde}[ʃɔ] ‘till' (a field) (v.), where no conditioning high front vowel appears. A possible explanation comes from the closely related language Kisi\il{Kisi}. The words for ‘morning' and ‘till' in that language are respectively /sìɔ̀/ and /sìɔ̀ɔ́/, where the conditioning [i] vowel is still realized.

Curiously, the alveo-palatal fricative [ʃ] begins the name of the language and the people <Sherbro>. This exonym contrasts with the group's autonym [bolom] (rendered <Bolom> or <Bullom> in various sources), the name now given to the subgroup to which Sherbro belongs (\citealt{Childs2024c}).

In the interests of completeness, I mention the unexpected alternation of [s] with [n] in [si] / [ni], both representing the all-purpose connector ‘with, and,' which also posits temporal and logical relations between clauses. This alternation may represent the diachronic collapse of a former semantic distinction between the two words.

There is also a dialectal alternation in the word for ‘hand' [fi] in the north in Bengeh\is{Bengeh} (also spelled Benge), Bumpeh Chiefdom\is{Bumpeh Chiefdom}  and [sui] in the south in Kagboro Chiefdom. Since [sui] is related to the word for ‘finger' in other related languages, the [s] variant may be older.

\subsubsection{/l/}
\label{sec:2.1.2.5}
The alveolar lateral has no distinct allophones and appears in onsets as well as in codas. Unusually, the lateral can be geminated, as happens also in closely related languages (discussed in \sectref{sec:2.4} as part of a more general process of onset strengthening). As the only long consonant found in the language, the geminate [l] occurs across morpheme boundaries. Epenthesis occurs before the definite marker \textit{ɛ} in \REF{ex:22}. The geminate arises before the question particle \textit{a} in \REF{ex:22} (see also examples in \REF{ex:44}).

\ea %22
%\label{bkm:geminatelallophone}(\stepcounter{qwerty}{\theqwerty})
\label{ex:22}
\ea I amɛn bullɛ ka koŋ wu.\\
\gll hi    a-mɛn    bul  ɛ    ka        koŋ  wu\\
\textsc{1pl}  \textsc{ncm}\textsubscript{ha}{}-five  one  \textsc{def}  \textsc{rem.pst}    \textsc{pfv}  die\\
\glt ‘We are five, one died a while ago.' (007a Agnes J. Simbo: 27)\\
\ex Kɛ mi ŋa mɔ ilella?\\
\gll kɛ    mi      ŋa    mɔ  i-lel        a\\
but  mother  what  \textsc{2sg}  \textsc{ncm}\textsubscript{hɔ}{}-name  \textsc{q}\\
\glt ‘But, Mummy, what is your name?' (007a Agnes J. Simbo: 8)
\z
\z

\subsubsection{/r/}
\label{sec:2.1.2.6}
Phonetically, /r/ is realized as central [ɹ] or even a retroflex approximant [ɻ] in its most common manifestations. \citet{Pichl1967} reports it as a trill [r]. In our work, we heard it as an alveolar tap or trill in pre-vocalic position and as a retroflex central approximant in syllable codas. In Sherbro, just as in Mani, it is a phoneme exhibiting a great deal of variation both phonetically and dialectally\il{Mani}. The phoneme is absent in Bom-Kim\il{Bom-Kim} and in the southern dialect of Kisi\il{Kisi}.

In the coda, /r/ obscures vowel quality in the nucleus as formants are damped and vowels are perceived as more centralized. The same effect occurs with the liquid /l/ or a nasal in the coda, as discussed above. V-/r/ metathesis\is{metathesis} can occur, especially when /r/ is in the coda of a high front vowel, as in the American English alternation, \textit{professor} and the somewhat colloquial or regional \textit{perfessor}. Other variants are possible for such Vr sequences, as illustrated with /tir/ ‘town' and similar words in \tabref{tab:phon:10}. The word for ‘ripe' /dir/ was also pronounced with something like a pharyngeal fricative [ʢ] in place of /r/ in several instances (not shown below), an allophone also found in Mani\il{Mani}.

\begin{table}
\caption{\label{tab:phon:10}Vr/ variation (cf. r/$\emptyset$ alternation in \tabref{tab:phon:11})}

\small

\begin{tabularx}{\textwidth}{lllQQQQl} 
\lsptoprule
& [iɹ] & [ɨɹ] & [əɹ] & [ɚ] / [dɹ̩] & [ɾi] & [ɹə] & [ri]\\
\midrule
/tir/ ‘town' & [tiɹ] & [tɨɹ] & [təɹ] & [tɹ̩] & [tɾi] &  & [tri]\\
/dir/ ‘red, ripe' & [diɹ] &  & [dər] & [dɚ] / [dɹ̩] &  & [dɹə] & \\
/bithir/ ‘bottle' & [bithiɹ] & [bitɨɹ] &  &  &  &  & \\
/kɛntir/ ‘groundnut' & [kɛntiɹ] &  &  &  &  &  & [kɛntri]\\
/kirkir/ ‘round' &  &  &  &  &  &  & [krikri]\\
\lspbottomrule
\end{tabularx}
\end{table}

One dialectal alternation is between [r] and [w], as in the word for ‘push,' pronounced [runt̪] around Shenge\is{Shenge}, shown in \REF{ex:23}. In the Bengeh\is{Bengeh} (also spelled Benge), Bumpeh Chiefdom\is{Bumpeh Chiefdom} dialect to the north, the word is pronounced [wunt̪]. The speaker comes from around Shenge. The sentence is followed by two single-word examples of the alternation.

\TabPositions{3cm,8cm}

\ea%23
%\label{bkm:rwalternation}(\stepcounter{qwerty}{\theqwerty}) 
\label{ex:23}
[r] / [w] alternation\\
\ea \label{ex:23a} 
rɔm / wɔm \tab ‘medicine'\\
rokos / wokos \tab ‘lime'\\
runth / wunth \tab ‘push'

\ex \label{ex:23b}
Lɛ nɔsɛ ha ni gbo kɛkɛ, nrunth gbo, mɔ gbo runth libul, komɔɛ koŋ honi.\\
\gll lɛ  nɔs  ɛ    ha    ni    gbo  kɛkɛ    n    runth    gbo mɔ  gbo  runth    li-bul        komɔ    ɛ    koŋ  honi\\
if  nurse  \textsc{def}  do    \textsc{neg}  just  quickly  \textsc{2sg}  push    just \textsc{2sg}  just  push    \textsc{NCM}\textsubscript{lɔ}{}-one    child    \textsc{def}  \textsc{pfv}  go.out\\
\glt ‘If the nurse does not make it fast, you just push, you just push once, and the baby emerges.' (002a Mabel Lohr, Midwifery: 53)
\z
\z

The “r” in Vr sequences also alternates with “${\emptyset}$,” as shown in \tabref{tab:phon:11}. Sometimes the r-less variant will have a long vowel in the place of the Vr sequence as perhaps a case of compensatory lengthening. Sumner, a native speaker of Sherbro, wrote ‘hoe' as <kar> (\citealt{Sumner1921}), which was recorded in our data with a long vowel [kaa].

\begin{table}
\caption{\label{tab:phon:11}[r] / ${\emptyset}$ alternation (cf. Vr variation in \tabref{tab:phon:10})}
\begin{tabular}[t]{lll}
\lsptoprule
təɹ & tə & ‘waist'\\
her & he & ‘cross'\\
gber & gbe & ‘many, much'\\
pɛr & pɛ & ‘fill'\\
kɛrko & kəko & ‘squirrel'\\
bithir & bithiː & ‘bottle'\\
\lspbottomrule
\end{tabular}
\end{table}

These facts, coupled with the [w] / [r] alternation shown in \REF{ex:23}, underscore the instability of /r/ in Sherbro. Another alternation noticed by a previous writer but not present in our work was between [l] and [r]. \citet{Hanson1979a} noted that both [l] and [r] were heard intervocalically between high vowels.

\ea%24
%(\stepcounter{qwerty}{\theqwerty})
\label{ex:24}
\begin{tabular}[t]{ll}
\relax [čɨrɨŋ] / [čɨlɨŋ] & ‘safe'\\
\relax [pilinni] / [pirinni] & ‘to walk around something' (\citealt[25]{Hanson1979a})\\
\end{tabular}
\z

Neither word appeared in our own data, but an earlier source has both the [l] and the [r] forms for ‘walk around something' (\citealt{Pichl1967}). \citet{Hanson1979a} maintains that [l] “is the actual phoneme.”

The /r/ phoneme exhibits similar instability elsewhere in Bolom-Kisi. For example, the \textit{r/l} contrast has been neutralized in southern dialects of Kisi\il{Kisi} to \textit{l} with significant consequences for the noun class morphology (\citealt{Childs1983}).

Another place where metathesis\is{metathesis} occurs, albeit much less frequently, is with the nasal [n], another resonant. Here the alternation for ‘kneel' is between [bitni] and [bitin] with a perhaps intermediate alternant of [bitəni].

%(\stepcounter{qwerty}{\theqwerty})  

\ea%25
\label{ex:25}
[bitəni] / [bitni] / Pɔ anyaɛ ŋa bitin chɔchai.\\
\gll bitəni    bitni    pɛ      a-nya        ɛ    ŋa    bitni    chɔch-ai\\
kneel    kneel    \textsc{pro}\textsubscript{indef}   \textsc{ncm}\textsubscript{ha}{}-people  \textsc{def}  \textsc{3pl}  kneel    church-in\\
\glt ‘kneel' / ‘kneel' / ‘People kneel in church.' (E14 Albert Yanker: 31)\footnotemark 
\footnotetext{Data citations in the E series are elicitation sessions whose transcriptions can be viewed in the FLEx database for Sherbro lexicon in the Endangered Languages Archive (ELAR).}
\z

\subsubsection{/h/}
\label{sec:2.1.2.7}
Because of heavy nasalization of the vowel after [h], the initial sound is often heard as [ŋ] or as [ɲ] before high front (palatal) vowels (see \sectref{sec:1.9} for ramifications in the orthography).
\clearpage
%(\stepcounter{qwerty}{\theqwerty}) 
\ea%26
\label{ex:26}{[h] / [ŋ] alternation}\\
\vspace{6pt}
\begin{tabular}[t]{ll}
 [ha] / [ŋa]    &  \textsc{subord} \textsc{conj}\\
 {[haa]} / [ŋaa]     & ‘do, make'\\
 {[hɔ]}\footnotemark{} / [ŋɔ] & ‘how'\\
\end{tabular}
\footnotetext{Paramount Chief Madam Doris Lenga-Caulker  Gbabiyor\is{Lenga-Caulker Gbabiyor, Doris} insisted that [hɔ] was the “correct” pronunciation. My suspicion is that she is right that it represents the older form before the advent of the Sherbro Literacy Committee\is{Sherbro Literacy Committee} (see \sectref{sec:1.9} for an account of how the ŋ / h spelling was used to distinguish homonyms.)}
\z

The two orthographic representations <ŋa> and <ha> of the homonymous pair [ha] has been exploited by the Sherbro Literacy Committee to differentiate functionally different forms. Although tone distinguishes some of the pairs, even that, coupled with the distinct spelling, does not differentiate all of the homonymous forms (see \sectref{sec:2.2}).

Another /h/-relevant alternation is between [h] and [w]:

%(\stepcounter{qwerty}{\theqwerty})   
\ea%27
\label{ex:27}{[h] / [w] alternation}\\
\vspace{6pt}

\begin{tabular}[t]{ll}
hɔŋgul / wɔŋɡul & ‘sell'\\
hɔ / wɔ & ‘say'\\
hɔ / wɔ & the class pronoun (see the discussion of /w/)\\
\end{tabular}
\z

The [h] variant seems to be more common in the north in the Bengeh\is{Bengeh} (also spelled Benge), Bumpeh Chiefdom\is{Bumpeh Chiefdom} dialect. The [h] / [w] alternation in ‘say' [hɔ] / [wɔ], sometimes coupled with the front-back alternation [ɛ] / [ɔ] can lead to some confusion for non-native speakers.

A final /h/ alternation is between [h] and ${\emptyset}$, as in [hɔbatokɛ] / [ɔbatokɛ] ‘God,' [hi] / [i] ‘we,' and [hina] / [ina] ‘who.'

\subsubsection{/w/}
\label{sec:2.1.2.8}
Common allophones of /w/ at the beginning of a word are the fully devoiced variant [ʍ] or a partially devoiced one [hw], which might explain the alternation between [h] and [w] discussed above.\footnote{Equally valid transcriptions are [w̥w] and [ʍw].} Some additional examples not presented there can be found in \REF{ex:28}.

\ea%28 
\label{ex:28}{More [h] / [w] alternation}\\ 
\vspace{6pt}
\begin{tabular}[t]{ll}
hwɛ / wɔ & ‘world' (coupled with front-back vowel alternation)\\
hun / wun & ‘come'\\
hu / wu  & ‘die'\\ 
hɔl / wɔl  & ‘eye'\\
\end{tabular}
\z

The labialvelar glide /w/ has an allophone [v], which is treated as a separate letter in the writing system of the Sherbro Literacy Committee\is{Sherbro Literacy Committee}, likely due to the influence of English where <w> and <v> represent distinct phonemes. The Sherbro Literacy Committee\is{Sherbro Literacy Committee} recommendations are followed in this grammar and dictionary, but the alternation is predictable as represented below.\\

\begin{tabular}[t]{lll}
/w/: &  [v] &  / {\longrule} V [-lo, -bk]\\
 & [w] & elsewhere\\
\end{tabular}\\

The distribution of these allophones parallels that found elsewhere in Bolom. Uniquely, however, /w/ has a dialectal variant of [h], as seen in \REF{ex:27} and \REF{ex:28}. Although the precise isogloss of the [w] / [h] alternation cannot be stated, [w] is more often heard with northern and interior speakers, e.g., from Bengeh (also spelled Benge), Bumpeh\is{Bengeh} Chiefdom, rather than with speakers from Shenge\is{Shenge}.

\subsubsection{/y/}
\label{sec:2.1.2.9}
The palatal glide also alternates with [h] in the word for ‘boil' as in \REF{ex:29}.

%(\stepcounter{qwerty}{\theqwerty})
\ea%29
\label{ex:29}
Mɛndɛ ma koŋ yɪl / hɪl.\\
\gll mɛn  ɛ    ma    koŋ   yel / hel\\
water  \textsc{def}  \textsc{ncp}\textsubscript{ma}    \textsc{pfv}  boil / boil\\
\glt ‘The water is boiling (has reached a boiling state).' (E08 Albert Yanker: 14)
\z

Other alternations are in the \textsc{1pl} \textsc{pro} [hi] and [yi], the word for ‘salt' [ihɛl] beside [iyɛl], and the word for ‘four' pronounced both as [yɔl] and [hiɔl] (see the discussion of /ny/ above).

\section{Tone}
\label{sec:2.2}\hypertarget{Toc115517756}{}
On the basis of comparative evidence, historically Sherbro was undoubtedly a tone language; tone was once likely used to mark both lexical contrasts and distinctions in the verbal morphology. Today, because the language has fallen into desuetude, much as is the case with its closest relatives, lexical tone is elusive though grammatical tone is still found in a few environments (\citealt{Childs2002a}).

Motivation for an earlier more tonal state comes from comparative and historical evidence. Both lexical and grammatical tone are found in other Bolom-Kisi languages that are still vital (Mani\is{Mani} and Kisi\il{Kisi}), though tone is less prominent in the most moribund Bolom language Bom-Kim\il{Bom-Kim}. Throughout Mel\il{Mel} in general, the greater group to which the Bolom-Kisi sub-group belongs with Temne\il{Temne}-Baga, tone contrasts have been identified (e.g.,\citealt{Wilson1968}). It is thus likely that tone is reconstructible, as is not the case with Atlantic, the language group to the north with which Mel was once associated but has now been disassociated (\citealt{Childs2004}, \citealt{Childs2024a}). I begin with a characterization of lexical tone in Sherbro.

Sherbro has at least two tones, high and low, as illustrated by the minimal pairs in \REF{ex:30}, representing both grammatical and lexical tone contrasts.

\ea%30
\label{ex:30}Some tonal contrasts\\
\ea \label{ex:30a}
\begin{tabular}[t]{llll}
há / ŋá\footnotemark & ‘you (pl.)' &hà / ŋà & ‘they'\\
háá & ‘do' (optative) & hàà & ‘did' (perfective)\\
kíth & ‘small, short' & kìth & ‘hard to drink'\\
rá & ‘a type of snake' & rà & ‘three'\\
wál & ‘palm leaf' & wàl & ‘resting place'\\
wáŋ & ‘girl' & wàŋ & ‘ten'\\
\end{tabular}
\vspace{6pt}
\ex wáŋ mà àwàŋ\\
\gll wáŋ mà à-wàŋ\\
girl \textsc{ncp}\textsubscript{ma} \textsc{ncm}\textsubscript{ha}-ten\\
\glt ‘ten girls'
\footnotetext{The [h]/[ŋ] alternation, as discussed in \sectref{sec:2.1.2},  represents a spelling rather than a phonemic contrast (see \sectref{sec:1.9}).}
\z
\z

The personal pronoun paradigm contains a minimal pair, first noticed in \citet[13]{Sumner1921} and deemed “important” enough to be marked in his proposed writing system. (Tone was generally not marked in early studies.) The second-person plural pronoun has a high tone, thus \textit{há} (or \textit{ŋá}), and the segmentally identical third-person plural pronoun has a low tone \textit{hà} (or \textit{ŋà}) (see \sectref{sec:1.9} for discussion of the <h/ŋ> variation).

An early study gives the following tonal n-tuplets (\citealt[35]{Sumner1921}; \tabref{tab:phon:12}). Pronouns have been omitted by Sumner for the third-person singular; they often go unexpressed. The main contrast in the verbal morphology is aspectual, which in this book are labelled perfective and imperfective. The particle \textit{ma} is used for both the negative optative and the hypothetical.
\clearpage
\begin{table}
\caption{\label{tab:phon:12}Verbal tone ({\citealt[35]{Sumner1921}})}
\begin{tabular}{ll} 
\lsptoprule
Kɔ́. & ‘He went.'\\
Kɔ̀. & ‘Let him go.' / ‘He should go.'\\
Mà kɔ́. & ‘Let him not go.'\\
Má kɔ̀. & ‘He would have gone.'\\
\tablevspace
À mà kɔ́. & ‘Let me not go.'\\
À má kɔ̀. & ‘I would have gone.'\\
\tablevspace
Yí mà kɔ́. & ‘Let us not go.'\\
Yí má kɔ̀. & ‘We would have gone.'\\
\tablevspace
Hà mà kɔ́. & ‘Let them not go.'\\
Há má kɔ̀. & ‘You (pl.) would have gone.'\\
\lspbottomrule
\end{tabular}
\end{table}

Pronouns can also change their tones depending on context. The 3\textsc{sg} pronoun \textit{wɔ} can be low-toned in subject position and high-toned in object position.\footnote{The high tone on the object may be a consequence of the high tone of the Perfective spreading onto the object. The question was not systematically investigated.}

\ea%31
\label{ex:31}
%(\stepcounter{qwerty}{\theqwerty})  
\ea Wɔ̀ ké mí.\\
‘He saw me.'\\
\ex
Yà ké wɔ́.\\
‘I saw him.'\\
\z
\z

But compare these examples with the tones on \textit{mi} in the examples in \REF{ex:32}. The inflectional marker of past (\textsc{pst}) -ɛ́ has a high tone (see \sectref{sec:4.2}).

\ea%32
%(\stepcounter{qwerty}{\theqwerty})  
\label{ex:32}
\ea Tàmɔ̀ɛ̀ wɔ̀ fɛ̀kiɛ́ mì.\\
\gll tamɔ  ɛ    wɔ    fɛki-ɛ        mi\\
boy  \textsc{def}  \textsc{3sg}  disrespect-\textsc{pst}  \textsc{1sg}\\
\glt ‘The boy has disrespected me.' (E10 Albert Yanker: 9)

\ex Tàmɔ̀ɛ̀   wɔ́ mí fɛ̀kí.\\
\gll tamɔ  ɛ    wɔ    mi    fɛ̀kí\\
boy  \textsc{def}  \textsc{3sg}  \textsc{1sg}  disrespect\\
\glt ‘The boy disrespects me.' (E10 Albert Yanker: 10)
\z
\z

Although nouns generally do not change their tones in different contexts, tone is not stable nor reliably produced, being unpredictably variable for lexical items across speakers and even for individual speakers. Therefore, I have followed the general practice of the Sherbro Literacy Committee\is{Sherbro Literacy Committee} of not marking tone. Full details of the grammatical use of tone are spelled out in Chapter \ref{ch:4} on verbal morphology.

\section{Syllable structure}
\label{sec:2.3}\hypertarget{Toc115517757}{}
Sherbro follows the pattern of other related languages in allowing filled codas (CV(C) syllables are the general pattern), as opposed to the situation in Mende\il{Mende}, the language to which Sherbro speakers are switching. Mande\il{Mande} languages are strictly CV, and Mende is no exception to that generalization (\citealt{Dwyer1989}, \citealt{Vydrin2004}). Long vowels in Sherbro appear in monosyllabic words and in initial syllables of polysyllabic words.

The coda consonant may consist of any of the following (single) segments (prenasalized stops are analyzed as unitary segments). The dental and alveolar stops are privileged in codas, especially when they form part of NC sequences.

\ea%33
%(\stepcounter{qwerty}{\theqwerty}) 
\label{ex:33}
Allowed coda consonants\\

\begin{tabular}[t]{ll}
Liquids & l, r\\
Nasals & m, n, ŋ (never <ny> [ɲ])\\
Prenasalized stops & mp, nth, nt, ŋk (never <nych> [ɲç])\\
Voiceless stops & p, th, t, ch, k\\
\end{tabular}
\z

As mentioned above in the discussion of prenasalized stops, all of which are single segments, there is a skewed distribution of “voiceless” prenasalized stops [mp, nt, ŋk] and voiced prenasalized stops [mb, nd, ŋg]. Namely, the former are found in codas and medially, and occasionally initially, while the latter are found in onsets, usually the result of a (syllabic) nasal prefix being reduced to non-syllabic status.

One exception to the last generalization is when the word following the prefixed nasal does not begin with a voiced stop. In the following example, the nasal prefix of the \textit{ma} class [n-] appears before a number of different consonants: [r t p h], which assimilates to the bilabial in \textit{pakai} ‘papaya'. In addition, the second person subject pronoun [n] assimilates to the velar stop [k] in \textit{kɔ} ‘go'. Thus, virtually any prenasalized sequence is possible initially, though place-of-articulation assimilation seems essential. The voiced prenasalized sequence always involves more than one morpheme.
\clearpage
\ea%34
\label{ex:34}
%(\stepcounter{qwerty}{\theqwerty})
Voiceless prenasalized stops: [ŋk, nt, mp] in initial position\\
\vspace{6pt}
Ŋkɔm lɛnthiɛ nrokos ntiŋ ni mpakai nhiɔl!\\
\gll n    kɔ    mi    lɛnthi  {}-ɛ    n-rokos        n-tiŋ      ni    n-      pakai    n-hiɔl\\
\textsc{2sg}  go    \textsc{1sg}  pluck  {}-\textsc{prt}  \textsc{ncm}\textsubscript{ma}{}-orange    \textsc{ncm}\textsubscript{ma}{}-two and  \textsc{ncm}\textsubscript{ma}{}-  papaya  \textsc{ncm}\textsubscript{ma}{}-four\\
\glt ‘Go pluck me two oranges and four papayas.' (P67 L: 53)
\z

The same generalization holds true for a sequence not shown, [nt̪].

Syllable structure may also vary when /r/ or /n/ is involved (see the discussion in \sectref{sec:2.1.2} for some cases of metathesis\is{metathesis} and epenthesis). A schwa may break up a sequence of disallowed consonants (certainly in compounds but also in stems).

\section{Phonological rules}
\label{sec:2.4}\hypertarget{Toc115517758}{}
Sherbro has both purely phonological rules as well as morphophonological ones. The latter category of rules is treated in the sections on morphology. Here only purely phonological rules are discussed.

\subsection{Nasal assimilation}
\label{sec:2.4.1}
Nasals always agree with the place of articulation of a following obstruent both within words and across morpheme boundaries. Following are examples of the latter phenomenon involving the prefixed second-person singular subject pronoun, which appears only before verbs, as in \REF{ex:35}. Another identical morpheme is the \textit{ma}{}-class prefix (or noun class marker (\textsc{ncm})), as featured in \REF{ex:35b}.

\ea%35
\label{ex:35} Nasal assimilation
%\label{bkm:nasassim}(\stepcounter{qwerty}{\theqwerty})  
\ea \label{ex:35a}
[+nas] $\xrightarrow{}$ [α place] / {\longrule} + C [α place], [m, n, ŋ, m͡ŋ]

\ex\label{ex:35b} \textit{ma}{}-class prefix /n-/\\
\ea Yaaka tallɛ, aka ni ŋaa mpanth ma sobaɛ.\\
\gll ya    a    ka      taa      lɛ    a    ka      ni    ŋaa  n-panth      ma    soba-ɛ\\
\textsc{1sg}  \textsc{1sg}  \textsc{rem.pst}  youth    \textsc{def}  \textsc{1sg}  \textsc{rem.pst}  \textsc{neg}  do    \textsc{ncm}\textsubscript{ma}{}-work  \textsc{ncp}\textsubscript{ma}    sober-\textsc{def}\\
\glt ‘When I was young, I did not do serious work.' (094a Ansu Kagboro:  66)

\ex So lan la ako ha ŋkuath ha ŋɔth.\\
\gll so  lan  la      a    koŋ    ha    n-kuath    ha    ŋɔth\\
so  this  \textsc{pro}\textsubscript{indef}  \textsc{1sg}  \textsc{pfv}    \textsc{opt}  \textsc{ncm}\textsubscript{ma}{}-fear  for    fishing\\
\glt ‘So that is how I became afraid of fishing.' (004a Cyril Manley on Walter Hanson:58)
\z

\ex \label{ex:35c} \textsc{2sg} subject prefix /n-/\\
\ea Nsiɛ tɛm pɛm doki yɛi chaŋ-chaŋdɛ …\\
\gll n    siɛ      tɛm  pɛm  doki  yɛ    yi    chaŋ-chaŋ  yɛ\\
\textsc{2sg}  know    time  war  this  how  \textsc{1pl}  travel      \textsc{prt}\\
\glt ‘You know during the war how we were moving around …' (002a Mabel Lohr, Midwifery: 41)

\ex Mɔm, la ŋka cheni ŋa?\\
\gll mɔm      la    n    ka      che  ni    ŋaa-a\\
\textsc{2sg.emph}  what  \textsc{2sg}  \textsc{rem.pst}  \textsc{prog}  now  do-\textsc{q}\\
\glt ‘You, what have you been doing?' (004a Cyril Manley on Walter Hanson: 45)

\ex Mi mŋgbisiŋɛ?\\
\gll mi      n    gbisiŋɛ\\
Mother  \textsc{2sg}  marry\\
\glt ‘Mummy, are you married?' (007a Agnes J. Simbo: 61)
\z
\z
\z

\subsection{Nasalization}
\label{sec:2.4.2}
An abundance of processes contributes to the ubiquity of nasalization in the language.\footnote{When I played some recordings to a renowned phonetician, he asked, “Don't they ever raise their velums?”} Although there is no contrastive nasalization, the process can be both perseveratory and anticipatory, and can affect consonants as well. Before a nasal consonant, but most dramatically and fulsomely after [h], vowels are nasalized, as has been noticed for related languages, Kisi\il{Kisi}, Mani\il{Mani}, and Bom-Kim\il{Bom-Kim} (\citealt{Childs1995}, \citealt{Childs2011}, \citealt{Childs2020}). When a velar nasal fills a coda, the preceding vowel is (phonetically) nasalized.

\ea%36
%(\stepcounter{qwerty}{\theqwerty})
\label{ex:36}
Anticipatory nasalization\\
\begin{tabular}[t]{lll}
/fuŋfuŋ/ & [f\~uŋf\~uŋ] / [f\~uf\~u] & ‘rice seedlings in a nursery' (P67 F: 38)\\
/wɔŋhul/ & [wɔ̃hũl] & ‘sell'\\
\end{tabular}
\z

But because the velar nasal is prone to disappear in such environments, the only trace of its presence, should it disappear, is the nasalization of the vowel. The vowel can also be compensatorily lengthened with the loss of the velar nasal as in Wong, the name of a sacred island in the Dema\is{Dema} Chiefdom, here with the definite article.

\ea%37
\label{ex:37}
%(\stepcounter{qwerty}{\theqwerty})  
woŋ + \textsc{def}\\
woŋ + ɛ  $\xrightarrow{}$  woŋ + ndɛ  $\xrightarrow{}$  wõːndɛ\\
\z

The colorful term “rhinoglottophilia” is used for the nasalization following glottal [h] and is associated with other glottal sounds (\citealt{Matisoff1975}). The nasalization for [h] is so heavy that the ‘\textsc{1pl} \textsc{pro} we,' /h\~\i/, is sometimes transcribed as [nyĩ]. Note also the two forms for ‘sea.'

\ea%38
%(\stepcounter{qwerty}{\theqwerty})

\label{ex:38}
Rhinoglottophilia\\
\vspace{6pt}
\begin{tabular}[t]{ll}
h\~\i & \textsc{1pl} \textsc{pro} ‘we'\\
hɛ̃l / nyɛ̃l & ‘sea'\\
hãã & ‘make, do'\\
hɔ̃lɛ & ‘whisper'
\end{tabular}
\z

There is also the close association between the velar nasal and [h] represented in such alternations as [ŋa] and [hã] \textsc{2pl} \textsc{pro}, ‘you' discussed above.

As a final nasalization process to mention, there is prenasalization of initial consonants. When a voiced stop begins a word, it can be prenasalized (unpredictably). The name of a chiefdom on Bonthe Island can be pronounced [dema] or [ndema].

For some speakers, nasalization becomes glottalization or creaky voice, affirming the link between nasal and glottal processes as in rhinoglottophilia. Not surprisingly, glottalization again is associated with the “glottal” fricative. The name Kain was sometimes spelled with two syllables and an “h” in the middle <Kahain>.

\ea%39
    \label{ex:39}
     Bia tonkiɛ jali Ka̰ḭn ha kɔnth.\\
    \gll Bia  tonki-ɛ        ja      li-Kain      ha    kɔnth\\
    Bia  summon-\textsc{pst}    matter  \textsc{ncm}\textsubscript{lɔ}{}-Kain    for    seizure\\
    \glt ‘Bia summoned Kain for seizure.' (P67 K: 200)
\z

The same name was used in Bom-Kim country and pronounced the same way (spelled <Kain>). Another word that exhibited glottalization was \textit{kahai} ‘outside' and the name \textit{Mahain}. Thus [kaha̰ḭ] and [maha̰ḭ]. The low back vowel is likely the conditioning factor, even more so when it both precedes and follows [h].\footnote{glottoglottophilia?}

\subsection{Palatalization}
\label{sec:2.4.3}
A number of related processes serve to palatalize alveolar stops. They are summarized here, all described separately in the discussion of the allophones of the respective segments (\sectref{sec:2.1.2}).

\ea%40 
\label{ex:40} Palatalization\\

\begin{tabular}{lllll}
d & $\xrightarrow{}$ & ʤ & / {\longrule} [i, e] & \textit{di/ji}  ‘kill, catch, initiate'\\
s & $\xrightarrow{}$ & ʃ & / {\longrule} [i, e] & \textit{si/shi}  ‘know,' \textit{seko/sheko} ‘fishhook'\\
t, th & $\xrightarrow{}$ & ʧ & / {\longrule} [i, e] & \textit{tii/chii} ‘town,' \textit{the/che} ‘hear, listen' 
\end{tabular}
\z

Related to these phenomena is the labiodental allophone [v] of the labialvelar approximate /w/, occurring in the same environment: w $\xrightarrow{}$ v / {\longrule} [i, e].

I now turn to a phonological rule at the level of the syllable. The syllable structure process described below appears in all languages in Bolom. Generally speaking, the process can be seen as a process of onset strengthening, as described in \citet{Childs1988}. Typically, the onset of a relatively “weak” syllable (featuring a liquid, a glide, or nothing) will borrow phonetic substance from a preceding coda, providing that coda is “stronger” (more prototypically consonantal). How this process plays out in Sherbro is described below. The process affects particles, clitics, suffixes, and other grammatical elements.

It is generally the case that strengthening takes place at morpheme and even word boundaries when the segment on the left is a nasal and the segment on the right is the liquid [l], a glide ([w] or [y]), or an onsetless syllable. Thus:\\


    ${\emptyset}$, [l], [y] and [w] $\xrightarrow{}$ [d] / N + {\longrule}\\


If the preceding coda is empty, no strengthening takes place.

\ea%41
    \label{ex:41}
  Onset strengthening
  \ea\label{ex:41a} Mbolomdɛ\\
\gll n-bolom      ɛ\\
\textsc{ncm}\textsubscript{ma}{}-bolom  \textsc{def}\\
\glt ‘Sherbro language'\\

\ex\label{ex:41b} Nthemdɛ\\
\gll n-them      ɛ\\
\textsc{ncm}\textsubscript{ma}{}-themne  \textsc{def}\\
\glt ‘Themne language'\\

\ex\label{ex:41c} ndoɛ\\
\gll n-loɛ\\
\textsc{ncm}\textsubscript{ma}{}-sleep\\
\glt ‘sleep'
\z
\z

The crucial form in \REF{ex:42} is the locative \textit{lɔ} at the end of the first line of morpheme analysis, which becomes [(n)dɔ] in context after \textit{hun}, as shown in the first line.

\ea%42
\label{ex:42}
Haaŋ mɛŋkɛ ŋɔ apotho aɛ ka hun  dɔ, chal ha pin awok aɛ …\\
\gll haa  mɛŋk    ɛ    ŋɔ      a-potho      a-ɛ        ka      hun  lɔ\\
until  time    \textsc{def}  when    \textsc{ncm}\textsubscript{ha}{}-whites  \textsc{ncm}\textsubscript{ha}{}-\textsc{def}  \textsc{rem.pst}  come  there\\
\gll chal  ha    pin  a-wok      a-ɛ\\
stay  for    buy  \textsc{ncm}\textsubscript{ha}{}-enslaved  \textsc{ncm}\textsubscript{ha}{}-\textsc{def}\\
\glt ‘Until the white man came there and settled to buy enslaved people …' (124aw Yanker, Boy Lost at Sea: 19)
\z
In the following example, there are two instances of Onset Strengthening. The first involves the proximal demonstrative \textit{loki} ‘these' after \textit{tiŋ} ‘two,' and the second is the clause-final binding particle \textit{ɛ} after \textit{thiyeŋ} ‘between.'

\ea%43
\label{ex:43}
    Yɛ   thoŋka ki gbi kɔ haani bɛl siatiŋ doki thiyeŋ dɛ …\\
\gll yɛ    thoŋka  ki    gbi  kɔ    haani    bɛl-si      a-tiŋ      loki  thiyeŋ  ɛ\\
when  arguing  this  all    \textsc{ncp}\textsubscript{kɔ}  happen  rat-\textsc{ncm}\textsubscript{si}  \textsc{ncm}\textsubscript{ha}{}-two  these  between  \textsc{prt}\\
\glt ‘When all this arguing is going on between these two rats …' (123aw Yanker, Rat Wife: 77)
\z

Glide insertion is much less common and even unpredictable. In \REF{ex:44}, [w] is inserted before the clause-final particle -\textit{ɛ} (\textsc{prt}) but not before the definite article \textit{ɛ} (\textsc{def}) in the same sentence after both \textit{de} ‘day' and \textit{lɔkɔ} ‘day.'
\clearpage
\ea%44
%\label{bkm:zerotowy}(44)
\label{ex:44}
\ea Yan deɛ ŋɔ huɛ lɔkɔɛ ŋɔ hu wɛ, aka shilani.\\
\gll ya-n      de    ɛ    wɔ    huɛ  lɔkɔ  ɛ    wɔ    hu  ɛ    a    ka      si    la    ni\\
\textsc{1sg-emph}  day  \textsc{def}  3\textsc{sg}  die  day  \textsc{def}  \textsc{3sg}  die \textsc{prt}  \textsc{1sg}  \textsc{rem.pst}  know  this  \textsc{neg}\\
\glt ‘As for me, the day he died the day he died, I didn't know.' (009--10a Lohr \& Mampa: 317)

\ex  Raiyɛ ŋɔ koŋ tuk.\\
\gll rai      ɛ    hɔ      koŋ  tuk\\
paper    \textsc{def}  \textsc{ncp}\textsubscript{hɔ}    \textsc{pfv}  disappear\\
\glt ‘The document has disappeared.' (002a Mabel Lohr, Midwifery: 41)
\z
\z

In a parallel process, after a left element [l], glides and empty onsets will be strengthened to [l], producing a geminate. Thus:\\

[${\emptyset}$], [y], and [w] $\xrightarrow{}$ [l] / [l] + {\longrule}\\

Again, the process augments the onset of a weak syllable preceded by a stronger one. In \REF{ex:45a}, the \textit{l} at the end of \textit{gbal} ‘line' is geminated before the definite article ɛ. In \REF{ex:45b}, the \textit{y} in the final question particle \textit{ya} is strengthened to \textit{l} and another geminate arises. In \REF{ex:45c}, gemination takes place before the preposition \textit{{}-ai} ‘in.'

\ea%45
    \label{ex:45}
    \ea \label{ex:45a} Inan gballɛ,  ilɔ pɛngipɛngi, ikikkik.\\
    \gll i    nan  gbal  ɛ    i    lɔ    pɛŋgipɛŋgi    i    kikkik\\
    \textsc{1pl}  draw  line  \textsc{def}  \textsc{1pl}  there  jump        \textsc{1pl}  kick\\
\glt ‘We draw the line, we jump there (and) kick.' (005a Jalikatu B. Kumba: 80)

\ex \label{ex:45b}  Kɛ mi ŋa mɔ ilella?\\
\gll kɛ    mi        ŋa    mɔ  i-lel-a\\
but  mother    what  \textsc{2sg}  \textsc{ncm}\textsubscript{hɔ}{}-name-\textsc{q}\\
\glt ‘But, Mummy, what is your name?' (007a Agnes J. Simbo: 8)

\ex \label{ex:45c}  Ŋakɔni fillai ŋa kɔ siŋ.\\
\gll ŋa    kɔn-i    fil    ai    ŋa    kɔ    siŋ\\
\textsc{3pl}  go-then  field  in    \textsc{3pl}  go    play\\
\glt ‘They go to the field and play.' (016a Albert Yanker: 166)
\z
\z

This last set of processes show some unity in that they all serve to strengthen an onset so that it is at least as strong as the coda of a preceding syllable.

\subsection{Vowel harmony}
\label{sec:2.4.4}
A single form in the derivational morphology shows vowel harmony\is{vowel}, the suffix \textit{{}\nobreakdash-il/-ul}, which changes verbs into adjectives. The suffix harmonizes with the [back] specification of the last stem vowel. See \sectref{sec:7.1} for some details and examples.


\part{Nominal categegories}
\include{chapters/morph_nouns}
\include{chapters/morph_pronouns}
\include{chapters/morph_adjectives}
\include{chapters/morph_numerals}
\include{chapters/morph_adverbs}
\include{chapters/morph_postpositions}
\include{chapters/morph_minorpartsofspeech}
\include{chapters/morph_placenames}


\part{Verbal morphology}
\include{chapters/verbs_morphology}
\include{chapters/verbs_verbformation}
\include{chapters/verbs_indicativesynthetic}
\include{chapters/verbs_analytic}
\include{chapters/verbs_periphrastic}
\include{chapters/verbs_copulaeauxiliaries}
\include{chapters/verbs_nondeclarative}
\include{chapters/verbs_nonfinite}

\part{Syntax}
\include{chapters/syntax_valencyclasses}
\include{chapters/syntax_agreement}
\include{chapters/syntax_phrasestructure}
\include{chapters/syntax_simpleclauses}
\include{chapters/syntax_relativeclauses}
\include{chapters/syntax_complementation}
\include{chapters/syntax_adverbialclauses}
\include{chapters/syntax_coordination}
\include{chapters/syntax_constituentorder}
\include{chapters/syntax_interrogativeclauses}
\include{chapters/syntax_reflexivereciprocal}
\include{chapters/syntax_minorconstructions}


 %APPENDICES
\appendix
\include{chapters/appendix_affixesenclitics}



%	[x]*	orthographic conventions
%	[x]*	abbreviations

%	[x]*	introduction

%	[x]*	phonology

%	[x].!	morphology nouns
%	[x].!	morphology pronouns
%	[x].!	morphology adjectives
%	[x]	morphology numerals
%	[x].!	morphology adverbs
%	[x].!	morphology postpositions
%	[x].!	morphology placenames
%	[x]*	minor parts of speech

%	[x]*	morphology verbs
%	[x]	verb formation
%	[x]	synthetic verb forms
%	[x]*	analytic verb forms
%	[x]*	periphrastic verb forms
%	[x]	non-finite verb forms
%	[x]	conditional and concessive clauses
%	[x]	non-declarative verb forms
%	[x]	specialized converbs and subordinating enclitics
%	[x]	copulae and auxiliaries
%	[  ]	causativization [INCOMPLETE!]

%	[x]*	verb valency classes
%	[x]*	agreement
%	[x]*	interrogative clauses
%	[x]*	phrase structure
%	[x]*	reflexive and reciprocal constructions
%	[x]*	relative clauses
%	[x]*	adverbial and conditional clauses
%	[x]*	complementation
%	[x]	coordination
%	[x]*	constituent order
%	[x]*	modification of valency patterns
%	[x]	minor constructions
%	[x]*	simple clauses, copular clauses, and grammatical relations

%	[x]*	appendix: list of affixes

%	[  ]	subject index
%	[  ]	author index

%	[  ]	bibliography



% look for:
%	[NNS:
%	???
%	\parn{


% -lative/-essive prefixes tsc or not? -> nope!	(spr)
% when is gloss <in> in-lative, when adposition in?

% spell check:	copulas 		->	copulae
%			zeros			->	zeroes
%			referent (adj.)	->	referential
%			preterit		->	preterite
%			center 		-> 	centre
%			liter			->	litre
%			color			->	colour
%			behavior		->	behaviour
%			co*			->	co-*
%			bi*			->	bi-*
%			cross*			->	cross-*
%			counter*		->	counter-*

%			-> / <-		->	> / <
%			’			->	'		(only apostrophe!)
%	(in tables)	#			->	\tmd		(in every empty cell!)
%			underline /  [...]	->	\emph{}


%%%%%%%%%%%%%%%%%%%%%%%%%%%%%%%%%%%%%%%%%%%%%%%%%%%%
%%%                                     	%%%
%%%             Backmatter         	%%%
%%%                                        	%%%
%%%%%%%%%%%%%%%%%%%%%%%%%%%%%%%%%%%%%%%%%%%%%%%%%%%%

% There is normally no need to change the backmatter section
%\is{some term| see {some other term}}
%\il{some language| see {some other language}}
%\issa{some term with pages}{some other term also of interest}
%\ilsa{some language with pages}{some other lect also of interest}
 
\iasa{Aikio, Antex}{Ánte, Luobbal Sámmol Sámmol}
\iasa{Ántex, Luobbal Sámmol Sámmol}{Aikio, Ante}
 
 
\backmatter
 
\phantomsection 
\addcontentsline{toc}{chapter}{Index} 

\addcontentsline{toc}{section}{Name index}
\ohead{Name index} 
\printindex 
  
\phantomsection 
\addcontentsline{toc}{section}{Language index}
\ohead{Language index} 
\printindex[lan] 
  
\phantomsection 
\addcontentsline{toc}{section}{Subject index}
\ohead{Subject index} 
\printindex[sbj] 
\end{document} 

% you can create your book by running
% xelatex main.tex
%
% you can also try a simple 
% make
% on the commandline
