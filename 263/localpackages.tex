% add all extra packages you need to load to this file  
\usepackage{tabularx} 

%%%%%%%%%%%%%%%%%%%%%%%%%%%%%%%%%%%%%%%%%%%%%%%%%%%%
%%%                                              %%%
%%%           Examples                           %%%
%%%                                              %%%
%%%%%%%%%%%%%%%%%%%%%%%%%%%%%%%%%%%%%%%%%%%%%%%%%%%% 
%% to add additional information to the right of examples, uncomment the following line
% \usepackage{jambox}
%% if you want the source line of examples to be in italics, uncomment the following line
% \renewcommand{\exfont}{\itshape}
\usepackage{./langsci/styles/langsci-optional}
\usepackage{./langsci/styles/langsci-gb4e}
\usepackage{./langsci/styles/langsci-lgr}
\usepackage{./langsci/styles/langsci-glyphs}

\usepackage[english]{babel}
\usepackage{listings}  
\lstset{ %  
  inputencoding=utf8,
  extendedchars=true,
  basicstyle=\ttfamily,        % the size of the fonts that are used for the code  
  literate=%
  {\{}{{{\color{orange}\{}}}1 
  {[}{{{\color{red}[{}}}}1 
  {]}{{{\color{red}]}}}1 
  {)}{{{\color{green})}}}1  
  {(}{{{\color{green}(}}}1  
  {\}}{{{\color{orange}\}}}}1 
} 
 
 

\usepackage{ifthen} 




% \usepackage[framemethod=tikz]{mdframed}
% % \mdfdefinestyle{firstfoc}{linecolor=gray,innerleftmargin=0.01mm} 
% \mdfdefinestyle{firstfoc}{linecolor=white,backgroundcolor=gray!10!white,innerleftmargin=0.01mm,innertopmargin=0.01mm,innerbottommargin=4mm} 
% % \mdfdefinestyle{secondfoc}{linecolor= white,innerleftmargin=0.01mm,tikzsetting = {draw=black, line width = .4pt,dashed,inner sep=0mm}}
% \mdfdefinestyle{secondfoc}{linecolor=white,backgroundcolor=gray!30!white,innerleftmargin=0.01mm,innertopmargin=0.01mm,innerbottommargin=4mm}
% % \mdfdefinestyle{thirdfoc}{linecolor=white,innerleftmargin=0.01mm,tikzsetting = {draw=black, line width = .4pt,dotted,inner sep=0mm}}
% \mdfdefinestyle{thirdfoc}{linecolor=white,backgroundcolor=gray!50!white,innerleftmargin=0.01mm,innertopmargin=0.01mm}
% 
% % https://tex.stackexchange.com/questions/249812/framed-environment-with-striped-background
% \usetikzlibrary{patterns,backgrounds}
% % defining the new dimensions and parameters
% \newlength{\hatchspread}
% \newlength{\hatchthickness}
% \newlength{\hatchshift}
% \newcommand{\hatchcolor}{}
% % declaring the keys in tikz
% \tikzset{hatchspread/.code={\setlength{\hatchspread}{#1}},
%          hatchthickness/.code={\setlength{\hatchthickness}{#1}},
%          hatchshift/.code={\setlength{\hatchshift}{#1}},% must be >= 0
%          hatchcolor/.code={\renewcommand{\hatchcolor}{#1}}}
% % setting the default values
% \tikzset{hatchspread=3pt,
%          hatchthickness=0.4pt,
%          hatchshift=0pt,% must be >= 0
%          hatchcolor=black}
% % declaring the pattern
% \pgfdeclarepatternformonly[\hatchspread,\hatchthickness,\hatchshift,\hatchcolor]% variables
%    {custom north east lines}% name
%    {\pgfqpoint{\dimexpr-2\hatchthickness}{\dimexpr-2\hatchthickness}}% lower left corner
%    {\pgfqpoint{\dimexpr\hatchspread+2\hatchthickness}{\dimexpr\hatchspread+2\hatchthickness}}% upper right corner
%    {\pgfqpoint{\dimexpr\hatchspread}{\dimexpr\hatchspread}}% tile size
%    {% shape description
%     \pgfsetlinewidth{\hatchthickness}
%     \pgfpathmoveto{\pgfqpoint{0pt}{\dimexpr\hatchspread+\hatchshift}}
%     \pgfpathlineto{\pgfqpoint{\dimexpr\hatchspread+0.15pt+\hatchshift}{-0.15pt}}
%     \ifdim \hatchshift > 0pt
%       \pgfpathmoveto{\pgfqpoint{0pt}{\hatchshift}}
%       \pgfpathlineto{\pgfqpoint{\dimexpr0.15pt+\hatchshift}{-0.15pt}}
%     \fi
%     \pgfsetstrokecolor{\hatchcolor}
% %    \pgfsetdash{{1pt}{1pt}}{0pt}% dashing cannot work correctly in all situation this way
%     \pgfusepath{stroke}
%    }
% 
% \tikzset{
% } 
% 
% \mdfdefinestyle{firstsecondfoc}{linecolor=white,backgroundcolor=gray!10!white,innerleftmargin=0.01mm,innertopmargin=0.01mm,
%   nobreak, 
%   linewidth=0pt,
%   apptotikzsetting={
%     \tikzset{mdfbackground/.append style=
%       { pattern=custom north east lines,
%         hatchspread=12pt,
%         hatchthickness=4pt,
%         hatchcolor=gray!20
%       }
%     }
%   }
% } 
%  
%  
\usepackage{pgfplots}
\usepackage{subfigure}
  
\usepackage{setspace} 

\usepackage{multicol}
