\documentclass[output=paper]{langsci/langscibook}
\ChapterDOI{10.5281/zenodo.4018380}

\author{N. J. Enfield\affiliation{Department of Linguistics, The University of Sydney}}

\title{Recruitments in Lao}

\abstract{This chapter describes the resources that speakers of Lao use when recruiting assistance and collaboration from others in everyday social interaction. The chapter draws on data from video recordings of informal conversation in Lao, and reports language-specific findings generated within a large-scale comparative project involving eight languages from five continents (see other chapters of this volume). The resources for recruitment described in this chapter include linguistic structures from across the levels of grammatical organization, as well as gestural and other visible and contextual resources of relevance to the interpretation of action in interaction. The presentation of categories of recruitment, and elements of recruitment sequences, follows the coding scheme used in the comparative project (see \chapref{sec:coding} of the volume). This chapter extends our knowledge of the structure and usage of Lao with detailed attention to the properties of sequential structure in conversational interaction. The chapter is a contribution to an emerging field of pragmatic typology. }
\maketitle
\label{sec:enfield}
\begin{document}

\section{Introduction}
\subsection{Recruitments}
This chapter describes and analyzes aspects of the system of semiotic practices that speakers of Lao use when getting people to do things in the course of everyday life. As defined in this collaborative project (see \chapref{sec:intro}, \sectref{sec:intro:4}), a recruitment sequence involves one participant A doing or saying something to B, or such that B can see or hear it, and next, as a response, B doing something for or with A. The data are drawn from a corpus of video-recorded interaction collected in home and village settings in Vientiane, Laos. The approach taken here assumes that the relevant unit of analysis is the \textit{pair} of moves that constitutes the recruitment sequence, that is, both an initiating move by Person A -- for example, a request or command, or a visible display of a need or difficulty -- that would precipitate some assisting behavior, as well as the move by Person B that responds to it with a form of assistance or collaboration that benefits Person A or a larger social unit of which A is a part, whether it is an act of compliance, rejection, or something else. In this chapter, we examine properties of both moves in the sequence, and ask as to their forms, functions, and interrelations.

The observations offered here arise from research done in a major crosslinguistic project (see \chapref{sec:intro} and other chapters in this volume).\footnote{With thanks to research collaborators in the \textit{Human Sociality and Systems of Language Use} project (see \chapref{sec:intro}, \sectref{sec:intro:3}). Other publications presenting ideas and findings from this project include \citet{Enfield2011a,Enfield2011b,enfield_human_2014}, \citet{FloydEtAl2014b,FloydEtAl2018}, \citet{Rossi2012,Rossi2014,Rossi2015a}, \citet{DrewCouper-Kuhlen2014a}, \citet{KendrickDrew2016}, \citet{ZinkenRossi2016}, \citet{Floyd2017}.} To maximize comparability, the data collected is tightly defined in scope. The studies rely solely on corpus materials from recordings of everyday home and village life. The interactions take place between relatives, neighbors, or people who otherwise know each other well. This implies that none of the interactions are formal or institutional in kind, which in turn means that the phenomena described in this chapter do not exhaust the resources that Lao speakers use in getting people to do things. For example, we shall see that in Lao village life, people seldom acknowledge the assistance given, for example by saying ‘thank you’, while in the more formal settings that are beyond the scope of this work, an idiom meaning ‘thank you’ is often used. A comprehensive account of the resources that Lao speakers rely on in recruitment sequences would require a broader collection of data.

\subsection{The Lao language}
Lao is an isolating/analytic language of the Southwestern Tai branch of Tai-Kadai. It is spoken by about 20 million people mostly in Laos, Thailand, and Cambodia. It is a tone language, with five lexical tones. The tones are indicated in this chapter by a numeral at the end of each word (see \citealt{Enfield2007} for glossing conventions). Lao has open classes of ideophones, nouns, verbs and adjectives, and closed classes of tense/aspect/modality markers, modifier classifiers and noun class markers, and phrase-final and sentence-final particles. There is no inflectional morphology. Grammatical relations tend to be signaled via constituent order, though there is widespread zero anaphora and movement licensed by information-structure considerations. Several grammars of Lao are available (see \citealt{Enfield2007} and many references therein). For recent work on semantic, pragmatic, and conversational patterns in Lao, see Enfield (\citeyear{Enfield2009,Enfield2010,Enfield2013,Enfield2015a,Enfield2015b}), Zuckerman (\citeyear{zuckerman2017disrupting}).

\section{Data collection and corpus}
The corpus on which this work is based was constructed in accordance with guidelines developed by and for the members of the comparative project being reported on in this volume (see \chapref{sec:intro} for further information). Here are the key properties of the data:

\begin{itemize}
%\small
\item Recordings were made on video
\item Informed consent was obtained by those who participated
\item Target behavior was spontaneous conversation among people who know each other well (family, friends, neighbors, acquaintances), in highly familiar environments (homes, village spaces, work areas)
\item Participants were not responding to any instruction, nor were they given a task -- they were simply aware that the researcher was collecting recordings of language usage in everyday life
\item From multiple interactions that were collected in the larger corpus, the selection for analysis in this study was of a set of 10-minute segments, taken from as many different interactions as possible (allowing that some interactions are sampled more than once), to ensure against bias from over-representation of particular interactions or speakers
\end{itemize}

The corpus from which the cases were drawn included video recordings collected by the author in Vientiane, Laos, between 2001 and 2011. Twelve interactions were sampled, with a combined duration of 2 hours 46 minutes, and a total of 222 cases of recruitments for this study. All interactions involved three or more participants. All recordings were made in family homes and village settings.

\section{Basics of recruitment sequences}

This is a study of recruitment sequences, defined by members of the collaborative subproject on this topic in the following way:\\

“The subproject on recruitment (ways that cooperative action gets mobilized) focuses on sequences in which a move by one participant (“[Move A]”, whether or not the move includes speech) leads immediately to a cooperative uptake behavior by another participant (“[Move B]”; this should be a practical bodily action, such as passing the salt, not simply giving information). We limit our scope to the here and now, thus precluding things like invitations where the uptake behavior would happen at a later place and time. [...] On this definition, “recruitments” straightforwardly subsumes things like requests and proposals, but also includes cases in which it may be unclear or equivocal whether the initiating [Move A] was an overt “request” or similar, so long as it results in the cooperative behavior” \citep{Enfield2011b}.\footnote{The original formulation had “M1” (Move 1) and “M2” (Move 2). Cf. Enfield (\citeyear{Enfield2011a}), Floyd et al. (\citeyear{FloydEtAl2014b}), Rossi (\citeyear{Rossi2015a}), Kendrick \& Drew (\citeyear{KendrickDrew2016}), Floyd (\citeyear{Floyd2017}), and \chapref{sec:intro} of this volume.}

\subsection{Minimal sequence}
A basic or minimal recruitment sequence in the Lao data consists of these two moves, Move A and Move B, by Person A and B respectively. The two moves are indicated by ▶ and ▷ in the transcripts.

Here is a typical example. Person A says ‘grind (it)’, while holding some herbal medicine out for Person B, who is holding the relevant medicine-grinding paraphernalia.

\transheader{ex:enfield:1}{INTCN\_020727a\_326860}\vspace{-1mm}
%
\begin{mdframednoverticalspace}[style=firstfoc]
\begin{transbox}{1}{a}
\begin{verbatim}
fon3  vaj2 ((holding medicine for B to take))
grind IMP.RUSH
\end{verbatim}
grind (it)
\end{transbox}
\end{mdframednoverticalspace}
%
\begin{mdframednoverticalspace}[style=secondfoc]
\begin{transbox}{2}{b}
\begin{verbatim}
((takes the medicine from A, prepares to grind it))
\end{verbatim}
\end{transbox}
\end{mdframednoverticalspace}\\

In another example, Person A is in an outdoor kitchen area, using a hose that delivers water pumped up from a well in the backyard. The pump is an electric one, and the switch that turns it on and off is located inside the house, several meters away from where Person A is standing. Two men are inside the house, close to the switch that turns the pump on and off. Person A calls out to them.

\transheader{ex:enfield:2}{INTCN\_030806e\_191591}\vspace{-1mm}
%
\begin{mdframednoverticalspace}[style=firstfoc]
\begin{transbox}{1}{a}
\begin{verbatim}
mòòt4      nam4  haj5 nèè1
extinguish water give IMP.SOFT
\end{verbatim}
turn off the water please
\end{transbox}
\end{mdframednoverticalspace}
%
\begin{mdframednoverticalspace}[style=secondfoc]
\xtransbox{2}{b}{((one of the men gets up and walks to the switch and turns the power for the water pump off))}
\end{mdframednoverticalspace}\bigskip

A third example involves transfer of an object (as opposed to provision of a service as seen in the last two examples). Person A and Person B are in a household food preparation area. Person A asks Person B to pass a papaya.

\transheader{ex:enfield:3}{CONV\_020723b\_RCR\_978740}\vspace{-1mm}
%
\begin{mdframednoverticalspace}[style=firstfoc]
\begin{transbox}{1}{a}
\begin{verbatim}
qaw3  maak5-hung1     qaw3 maa2 mèè4
take  CM.FRUIT-papaya take come IMP.UNIMP
\end{verbatim}
bring (me) a/the papaya
\end{transbox}
\end{mdframednoverticalspace}
%
\begin{mdframednoverticalspace}[style=secondfoc]
\begin{transbox}{2}{b}
\begin{verbatim}
qaw2 ((passing the papaya to A))
take
\end{verbatim}
(here) take (it)
\end{transbox}
\end{mdframednoverticalspace}

\subsection{Non-minimal sequence}

Recruitment sequences sometimes feature more than one initiating move. Often this is because a first attempt does not get a response, and so the Move A part of the sequence is redone. This happens in the following case, in which Person A is asking her father to pass her a knife. Both attempts are done using an interrogative formulation, with the second attempt being done in more specific fashion than in the first attempt.

\transheader{ex:enfield:4}{CONV\_020723b\_RCR\_970010}\vspace{-1mm}
%
\begin{mdframednoverticalspace}[style=firstfoc]
\begin{transbox}{1}{a}
\begin{verbatim}
miit4 dêê3 phòq1
knife Q    daddy
\end{verbatim}
the knife Daddy?
\end{transbox}
\end{mdframednoverticalspace}
%
\xtransbox{2}{b}{((no response))}
%
\begin{mdframednoverticalspace}[style=firstfoc]
\begin{transbox}{3}{a}
\begin{verbatim}
phòò1  miit4 thaang2 lang3 caw4    mii4  bòò3
father knife way     back  2SG.POL exist QPLR
\end{verbatim}
Dad a knife behind you, is there (one)?
\end{transbox}
\end{mdframednoverticalspace}
%
\begin{mdframednoverticalspace}[style=secondfoc]
\begin{transbox}{4}{b}
\begin{verbatim}
nii4 nii4
here here
\end{verbatim}
here here 
\end{transbox}
\end{mdframednoverticalspace}
%
\emptytransbox{~}{((reaches behind to look for the knife, finds it, passes it towards A))}\\

Another reason a recruitment sequence can be extended beyond the minimal structure is that Person B may immediately delegate to another person, rather than carrying out the action herself (see also Floyd, \chapref{sec:floyd}, \sectref{sec:floyd:3.3.1}; Blythe, \chapref{sec:blythe}, \sectref{sec:blythe:4.2.2}). In the next example, when Person B is asked to go and get some trays in preparation to serve food, she does not carry out the action. Instead, she turns to her younger sibling -- Person C -- and re-issues the initiating move, which Person C then immediately fulfills.

\transheader{ex:enfield:5}{INTCN\_111202n\_RCR\_989020}\vspace{-1mm}
%
\begin{mdframednoverticalspace}[style=firstfoc]
\begin{transbox}{1}{a}
\begin{verbatim}
sòòng3 phaa2      nan4    song1 khaw5 maa2 haa3 kan3 ((to B))
two    tray.table DEM.EXT send  enter come seek COLL
\end{verbatim}
those two tray tables, bring them in here together
\end{transbox}
\end{mdframednoverticalspace}
%
\begin{mdframednoverticalspace}[style=firstfoc]
\begin{transbox}{2}{b}
\begin{verbatim}
paj3 qaw3  maa2 ((to C, eye-pointing to trays))
go   take  come
\end{verbatim}
go get (them)
\end{transbox}
\end{mdframednoverticalspace}
%
\begin{mdframednoverticalspace}[style=secondfoc]
\begin{transbox}{3}{c}
\begin{verbatim}
((goes and gets trays))
\end{verbatim}
\end{transbox}
\end{mdframednoverticalspace}

\subsection{Subtypes of recruitment sequence}\label{sec:enfield:3.3}

The recruitment sequences collected for this study were divided into four categories, distinguished by the kind of behavior they would elicit from Person B (see \chapref{sec:coding}, \sectref{sec:coding:6} for discussion). The four categories are: (i) a service (such as turning off a switch), (ii) transfer of an object (such as a papaya), (iii) altering a current trajectory of action that Person B was on (such as telling someone to stop pouring), and (iv) assistance with some trouble that Person A was perceptibly experiencing (such as holding a door open for someone whose hands are full). Cases of service and object transfer are amply illustrated in above examples, and elsewhere throughout this chapter. We now illustrate the other two categories.

Following is an “alter trajectory” example. In this case, Person B is about to sit down on a rickety railing that appears unlikely to be able to bear his weight without breaking. Person A calls out repeatedly ‘don’t sit down!’. B responds by altering his trajectory of behavior, desisting from his path of sitting down, instead moving to sit elsewhere:

\transheader{ex:enfield:6}{INTCN\_030731b\_441300}\vspace{2mm}
%
\emptytransbox{1}{((B is going to sit on a weak railing))}
%
\begin{mdframednoverticalspace}[style=firstfoc]
\begin{transbox}{2}{a}
\begin{verbatim}
jaa1     paj3 nang1 dêj2     han5     
NEG.IMPV go   sit   FAC.NEWS DEM.DIST
\end{verbatim}
don't sit down
\end{transbox}
\end{mdframednoverticalspace}
%
\begin{mdframednoverticalspace}[style=firstfoc]
\begin{transbox}{3}{~}
\begin{verbatim}
jaa1     paj3 nang1 dêj2     han5     
NEG.IMPV go   sit   FAC.NEWS DEM.DIST
\end{verbatim}
don't sit down
\end{transbox}
\end{mdframednoverticalspace}
%
\begin{mdframednoverticalspace}[style=secondfoc]
\begin{transbox}{4}{b}
\begin{verbatim}
((desists from going to sit down on a rickety railing))
\end{verbatim}
\end{transbox}
\end{mdframednoverticalspace}

Next is a “trouble assist” example. Person A is preparing a salad-type dish, putting ingredients into a large pestle. She is holding the pestle in her hand. This type of dish needs to be tossed prior to serving, and this is normally done using a spoon and a mortar-and-pestle in combination. At the moment in the interaction that we are focusing on in this example, Person B is looking directly at Person A, and can see that Person A does not have a spoon (\figref{fig:enfield:1}\textit{a}). Rather than waiting for person A to ask, or letting them find a spoon themselves, Person B looks for a spoon (\figref{fig:enfield:1}\textit{b}), locates one, picks it up, and places it in reach of Person A (\figref{fig:enfield:1}\textit{c}), where Person A is subsequently able to pick it up and use it (\figref{fig:enfield:1}\textit{d}).

\transheader{ex:enfield:7}{INTCN\_030731b\_385660}\vspace{-1mm}
%
\begin{mdframednoverticalspace}[style=firstfoc]
\xtransbox{1}{a}{((involved in a course of food preparation where next step requires a spoon; does not have a spoon))}
\end{mdframednoverticalspace}
%
\begin{mdframednoverticalspace}[style=secondfoc]
\xtransbox{2}{b}{((looks for spoon, walks to pick one up, places it down within arm's reach for A))}
\end{mdframednoverticalspace}\medskip

\begin{figure}
\captionsetup{width=.45\linewidth}
\subfigure[Person A (seated, toward back of frame), is involved in food preparation in which the next step requires a spoon; Person B (standing) is looking directly at Person A and can see that there is no spoon at hand.]{
% \caption{}
\includegraphics[height=.3\textheight]{figures/lao-img1a}%insert small img here
% \end{subfigure}
}~~
\subfigure[Person B turns and retrieves a spoon.]{
\includegraphics[height=.3\textheight]{figures/lao-img1b}%insert small img here
}

\subfigure[Person B places the spoon within direct reach of Person A.]{
\includegraphics[height=.41\textheight]{figures/lao-img1c}%insert small img here
}~~
\subfigure[Person A picks up the spoon to use.]{
\includegraphics[height=.41\textheight]{figures/lao-img1d}%insert small img here
}

\caption{Trouble assist and transfer of a spoon in \REF{ex:enfield:7}.}
\label{fig:enfield:1}
\end{figure}

Another example of the trouble assist type is discussed in \citet[42]{Enfield2014b}. In the example described there in more detail, Person A is walking up a steep staircase with his arms full, holding a full basket of laundry. He approaches a nearly-closed safety gateway at the top of the stairs, which is blocking his way. He does not have a free hand to open the gate and pass. Seeing this, Person B -- who is sitting at the top of the stairs with the gateway within arm’s reach -- does not wait for Person A to say anything, but reaches out to the gate and holds it open for Person A.

While these trouble assist cases are obviously not requests as such, they are recruitments as defined for the purposes of this study. For Person A to get Person B to do something, it is not necessary that their Move A is an on-record or intended signal for Person B. What is important is that Person B acts upon a sign, in the broadest sense, from Person A, and does so with an action that is, in some relevant sense, \textit{for} Person A. Whether Person A means it or not, in these cases Person A’s behavior results in Person B doing something for them. 

This phenomenon relates to the kinds of action we would call “offers”: often, when one person states a problem, another will offer to help. In that sense, offers are seldom truly initiating moves, but are occasioned by certain types of prior move \citep[see also][]{Curl2006}. In the case just described, Person B does not offer to help. Rather, they simply do the helping action in response to the prior move that revealed the need for assistance.

\tabref{tab:enfield:1} shows the relative frequencies and proportions of the four types of recruitment sequences in the Lao corpus. The relative distribution of the types is heavily skewed. Moves that elicit a service account for over half of all cases, and object transfers for over a third of all cases. By contrast, alter trajectory and trouble assist recruitments are infrequent, together accounting for fewer than one out of ten cases.

\begin{table}
\begin{tabularx}{0.75\textwidth}{Xrr}
\lsptoprule
Type of recruitment sequence & Count & Proportion\\
\midrule
Service & 118 & 56\%\\
Object transfer & 76 & 36\%\\
Alter trajectory & 14 & 6.6\%\\
Trouble assist & 3 & 1.4\%\\
\lspbottomrule
\end{tabularx}
\caption{Relative frequencies and proportions of the four types of recruitment sequence in the Lao corpus.}
\label{tab:enfield:1}
\end{table}

\section{Formats in Move A: The initiating move}

Initiating moves in recruitment sequences may be formulated using verbal material alone (i.e. linguistic forms including words and grammatical constructions), nonverbal material alone (i.e. visible bodily behavior), or a combination of both verbal material and nonverbal material (referred to here as composite, cf. \citealt{Enfield2009}).\footnote{I use \textit{verbal} to roughly denote the linguistic, symbolic, lexico-syntactic, vocal behavior in these Lao data, and \textit{nonverbal} to roughly denote the visual, manual, gestural behavior. I use this distinction in the usual common sense way, despite known problems making the distinction definitive \citep[see][]{Enfield2009}.} As \figref{tab:enfield:2} shows, 97\% of all initiating moves in recruitment sequences have a verbal component, with a third of these featuring a nonverbal component in addition.

\begin{table}
\begin{tabularx}{0.66\textwidth}{Xrr}
\lsptoprule
Modality & Count & Proportion\\
\midrule
Verbal only & 135 & 65\%\\
Composite & 67 & 32.2\%\\
Nonverbal only & 6 & 2.9\%\\
\lspbottomrule
\end{tabularx}
\caption{Modality of initiating move (Move A).}
\label{tab:enfield:2}
\end{table}

\subsection{Purely nonverbal initiating moves}

While fully nonverbal initiating moves are rare, they do occur. In an example, Person A points to a bag that had tamarind in it, which people present had been snacking on. Person B responds by stating that there is none left in the bag, thus orienting to the pointing gesture as something like a request to pass some of the tamarind.

\transheader{ex:enfield:8}{INTCN\_111204x\_RCR\_495541}\vspace{-1mm}
%
\begin{mdframednoverticalspace}[style=firstfoc]
\begin{transbox}{1}{a}
\begin{verbatim}
((points at item))
\end{verbatim}
\end{transbox}
\end{mdframednoverticalspace}\vspace{-2mm}
%
\begin{mdframednoverticalspace}[style=secondfoc]
\begin{transbox}{2}{b}
\begin{verbatim}
bet2     lèèw4
finished PRF
\end{verbatim}
(it’s) finished ((‘there’s none left’))
\end{transbox}
\end{mdframednoverticalspace}\bigskip

For another case, see Enfield (\citeyear[19--21, 46]{Enfield2013}): Person A crawls forward in the direction of a basket that contains betel nut paraphernalia, and B responds by passing the basket to her and saying ‘you’ll chew?’. The behavior of crawling forward and reaching toward the basket was understood to be at least an attempt to obtain the contents of the basket to chew, and was perhaps even designed to elicit the other person’s help.

It is notable that the initiating moves that were fully nonverbal include all of the trouble assist examples.

\subsection{Types of nonverbal behavior in initiating moves}

As noted above, around a third of all initiating moves in recruitments in Lao have a component of visible bodily behavior. These forms of bodily conduct are of course quite varied, but there are some recurring types of visible behavior, as shown in this table.

\begin{table}
\begin{tabularx}{0.66\textwidth}{Xrr}
\lsptoprule
Visible behavior & Count & Proportion\\
\midrule
Pointing & 27 & 38\%\\
Holding out & 18 & 25.4\%\\
Reaching & 9 & 12.7\%\\
Other & 17 & 23.9\%\\
\lspbottomrule
\end{tabularx}
\caption{Visible behavior.}
\label{tab:enfield:3}
\end{table}

A large number of examples involve pointing gestures, either by hand or some other vector-projecting body part (eyes, lips, etc.). These gestures often help to locate something that is being asked for, or they may help to otherwise clarify what is intended. For example, in \REF{ex:enfield:5} above, a speaker eye-points to some tray tables as she asks her younger sibling to go and get them. For other examples in this chapter involving pointing, see \REF{ex:enfield:8} and \REF{ex:enfield:21}.
% (INTCN\_111202n\_RCR\_989020), (INTCN\_111204x\_RCR\_495541), (INTCN\_111203l\_682150) 

Another common visible behavior accompanying initiating moves is for Person A to hold something out towards Person B. The following is a typical example, in which a man asks his son to cut some rattan shoots, while holding out the knife that he should use.

\transheader{ex:enfield:9}{INTCN\_111204x\_RCR\_48410}\vspace{-1mm}
%
\begin{mdframednoverticalspace}[style=firstfoc]
\begin{transbox}{1}{a}
\begin{verbatim}
qaw3 qaw3 – qaw3 tat2
take take   take cut
\end{verbatim}
take ((this knife)) -- cut ((that))
\end{transbox}
\end{mdframednoverticalspace}
%
\begin{mdframednoverticalspace}[style=secondfoc]
\begin{transbox}{2}{b}
\begin{verbatim}
((takes knife to start cutting))
\end{verbatim}
\end{transbox}
\end{mdframednoverticalspace}

\begin{figure}
\caption{Person A (man in foreground, to right of frame) holds out a knife as he says ‘cut ((that))’ to Person B (man seated further back, to left of frame).}
\label{fig:enfield:5}
\includegraphics[width=\textwidth]{figures/lao-img2}


%\textbf{FIGURE} \textbf{5.}

%INTCN\_111204x\_RCR\_48410\_0048.png Person A (man in foreground, to right of frame) holds out a knife as he says ‘Cut (them)’ to Person B (man seated further back, to left of frame).
\end{figure}


The third major category of visible behavior that accompanies initiating moves is reaching for something, usually an object that is being requested. See, for example, \REF{ex:enfield:18} below, in which Person A is asking for a piece of medicinal root as she holds out her hand, as if reaching to receive it. %(INTCN\_111204x\_RCR\_296281)

\subsection{Verbal elements}

\subsubsection{Major sentence types}\label{sec:enfield:4.3.1}

In terms of linguistic form, a majority of the initiating moves in this Lao collection are full clauses marked as one or another of the three main sentence types: declarative, imperative, interrogative. As \tabref{tab:enfield:4} shows, the relative frequency of these types is heavily skewed. Imperative forms account for around four fifths of all cases, with interrogatives and declaratives far less frequent.

\begin{table}
\begin{tabularx}{0.66\textwidth}{Xrr}
\lsptoprule
Sentence type & Count & Proportion\\
\midrule
Imperative & 149 & 82.8\%\\
Interrogative & 20 & 11.1\%\\
Declarative & 11 & 6.1\%\\
\lspbottomrule
\end{tabularx}
\caption{Sentence type.}
\label{tab:enfield:4}
\end{table}

In the following example of declarative formatting, Person A is sitting close to a large pot with live fish at the bottom of it. The water level in the pot is too low. Person B starts pouring water into the pot. As the water level rises, the fish start to thrash about, and water splashes onto Person A. He states ‘(that’s) enough’. This assertion results immediately in Person B desisting and moving back from the pot.

\transheader{ex:enfield:10}{INTCN\_111203l\_618100}\vspace{-1mm}
%
\begin{mdframednoverticalspace}[style=firstfoc]
\begin{transbox}{1}{a}
\begin{verbatim}
qeej4 phòò2  lèèw4 – huaj5
yeah  enough PRF   – INTJ.ANNOYED
\end{verbatim}
hey, (that's) enough, gosh!
\end{transbox}
\end{mdframednoverticalspace}
%
\emptytransbox{~}{((moves body back away from pot that is splashing water from the fish))}\vspace{-1mm}
%
\begin{mdframednoverticalspace}[style=secondfoc]
\begin{transbox}{2}{b}
\begin{verbatim}
((stops pouring water into the pot and moves back))
\end{verbatim}
\end{transbox}
\end{mdframednoverticalspace}

For another example of declarative formatting, see \REF{ex:enfield:20} below, in which Person A’s statement ‘you’re blocking your brother’ is an attempt to get Person B to move away. These examples of declarative formatting illustrate the indirect strategy by which people can get people to do things simply by describing a problem that needs solving. When Person A describes a problem, a cooperative Person B may respond by fixing that problem (see also Kendrick, \chapref{sec:kendrick}, \sectref{sec:kendrick:4.2.3}; Rossi, \chapref{sec:rossi}, \sectref{sec:rossi:3.3.4}; Baranova, \chapref{sec:baranova}, \sectref{sec:baranova:3.3.3}; Dingemanse, \chapref{sec:dingemanse}, \sectref{sec:dingemanse:3.2.2}). %(INTCN\_111204x\_RCR\_153391)

Cases with interrogative formatting in the Lao data are mostly of two types. One type asks as to the existence or whereabouts of an object that Person A wants. For example, here Person A wants some betel nut to chew. She first asks ‘permission’ (a kind of ritual preliminary to issuing a request), and then asks ‘is there anything to chew?’.

\transheader{ex:enfield:11}{INTCN\_020727a\_197007}\vspace{-1mm}
%
\begin{mdframednoverticalspace}[style=firstfoc]
\begin{transbox}{1}{a}
\begin{verbatim}
beng1 dee4      qanuñaat4  dèè1     (  )
look  FAC.ONRCD permission IMP.SOFT
\end{verbatim}
look, if I may (\hspace{0.3cm})
\end{transbox}
\end{mdframednoverticalspace}
%
\begin{mdframednoverticalspace}[style=firstfoc]
\begin{transbox}{2}{~}
\begin{verbatim}
khiaw4 maak5 mii2  ñang3    khiaw4 bòò3
chew   betel exist anything chew   QPLR
\end{verbatim}
(I want to) chew betel nut, is there anything to chew?
\end{transbox}
\end{mdframednoverticalspace}
%
\emptytransbox{~}{((looking around for betel nut, grabbing hold of basket herself))}\vspace{-1mm}
%
\begin{mdframednoverticalspace}[style=secondfoc]
\begin{transbox}{2}{b}
\begin{verbatim}
qoo4 mii2  laø.bòò3 ((allows A to proceed))
INTJ exist of.course
\end{verbatim}
oh, yes of course
\end{transbox}
\end{mdframednoverticalspace}\bigskip

See also \REF{ex:enfield:4} above. In that example, Person A wants a knife for food preparation. She first asks Person B (her father) a very general question, roughly ‘the knife?’, following it up with a more specific question ‘Dad is the knife behind you?’. He then reaches back to retrieve the knife and pass it to her. Questions about where an object is, or whether it is available, are appropriate in precisely those situations in which the question is apposite -- namely, when it is not known that the object can be provided or not (see also Floyd, \chapref{sec:floyd}, \sectref{sec:floyd:3.3.3}; Rossi, \chapref{sec:rossi}, \sectref{sec:rossi:3.3.3}). %(CONV\_020723b\_RCR\_970010)

In a second kind of question that is used for getting others to do things, a question may serve to somewhat indirectly draw attention to a problem that needs solving. In the next example, Person A is sitting at a neighbor’s shop stall, watching as her neighbor threads pieces of meat onto skewers and piles them up, in preparation to grill meat for sale at her stall. Her question -- ‘why don’t you ever grill any of these?’ -- can be interpreted as an oblique way of implying that Person A would like some to eat, and suggesting that Person B start grilling the skewered meat.

\transheader{ex:enfield:12}{INTCN\_111204q\_RCR\_15060}\vspace{-1mm}
%
\begin{mdframednoverticalspace}[style=firstfoc]
\begin{transbox}{1}{a}
\begin{verbatim}
khùù2 bòò1 piing4 cak2 thùa1
why   NEG  grill  any  time
\end{verbatim}
why don't you ever grill any of these
\end{transbox}
\end{mdframednoverticalspace}
%
\begin{mdframednoverticalspace}[style=secondfoc]
\xtransbox{2}{b}{((no response, continues threading meat onto skewers))}
\end{mdframednoverticalspace}

The imperative sentence type is the dominant one used in getting people to do things. In this sentence type, there is usually no subject (or if there is one, it is a form of person reference referring to the addressee, the intended agent of the requested action), and the verb has no marking for aspect or modality. In many cases, there is no further marking, while in others there is explicit marking by means of a sentence-final particle from a dedicated system of such particles.\footnote{Cf. Floyd, \chapref{sec:floyd}, \sectref{sec:floyd:3.3.2} for a similarly extensive imperative system based on morphological marking.} The different particles allow speakers to denote a range of subtle or not-so-subtle distinctions in features such as expectation of compliance, minimization of imposition, and urgency (see \citealt{Enfield2007}:63ff for detailed explication). \tabref{tab:enfield:5} gives the figures for the forms of marking that occur more than five times in the Lao data (accounting for 124 cases).

\begin{table}
\begin{tabularx}{0.5\textwidth}{Xrr}
\lsptoprule
Particle & Count & Proportion\\
\midrule
zero & 46 & 37.1\%\\
\textit{mèè4} & 21 & 16.9\%\\
\textit{dèè1} & 18 & 14.5\%\\
\textit{dee4} & 10 & 8.1\%\\
\textit{duu2} & 10 & 8.1\%\\
\textit{naø} & 7 & 5.6\%\\
\textit{paj3} & 6 & 4.8\%\\
\textit{vaj2} & 6 & 4.8\%\\
\lspbottomrule
\end{tabularx}
\caption{Forms of marking.}
\label{tab:enfield:5}
\end{table}

\largerpage
Finally, there is a dedicated negative imperative marker \textit{jaa1}, meaning ‘desist’. There are three cases in my corpus: see \REF{ex:enfield:6} above, and the following example. In this example, a group of people are seated in a village home, eating and talking. They are seated in a circle, without much space between them. On this occasion, one of the people, a middle-aged man, who is a son-in-law to the household, has been a guest in the house and is preparing to leave the village and not return for an extended period. Extended family members are gathering on this occasion. The man’s niece wants to sit close to him, and she begins to push into the space next to him, requiring people to shift and make space. Her father (the man’s brother-in-law) reacts by telling her not to go too close (this is an example of an “alter trajectory” type of recruitment). She ignores this, which is to say that she simply continues her trajectory of action. %(INTCN\_030731b\_441300)

\transheader{ex:enfield:13}{INTCN\_111204x\_RCR\_196441}\vspace{-1mm}
%
\begin{mdframednoverticalspace}[style=firstfoc]
\begin{transbox}{1}{a}
\begin{verbatim}
nithaa3 jaa1     paj3 kaj4 phen1
N       NEG.IMPV go   near 3.POL
\end{verbatim}
Nithaa don't go close to him
\end{transbox}
\end{mdframednoverticalspace}
%
\begin{mdframednoverticalspace}[style=secondfoc]
\xtransbox{2}{b}{((no verbal response, continues to move into the space next to her uncle))}
\end{mdframednoverticalspace}

\subsubsection{Additional verbal types}

Here I note two further types of linguistic form that were used in initiating moves in the Lao data. First is the “no predicate” type, in which someone simply refers to the object being requested. Here is an example.

\transheader{ex:enfield:14}{INTCN\_111201k\_RCR\_343251}\vspace{-1mm}
%
\begin{mdframednoverticalspace}[style=firstfoc]
\begin{transbox}{1}{a}
\begin{verbatim}
phaa2      khaw5 lêk1  hanø     naø luuk4
tray.table rice  steel DEM.DIST TPC child
\end{verbatim}
the steel tray table, child
\end{transbox}
\end{mdframednoverticalspace}
%
\begin{mdframednoverticalspace}[style=secondfoc]
\xtransbox{2}{b}{((outside the room, eventually returns with the tray table as requested))}
\end{mdframednoverticalspace}

It is worth noting the use here of the kin term \textit{luuk4} ‘child’ as a vocative. This may help contribute to the understanding that the speaker is seeking to mobilize the child’s assistance.

Second is the “bare vocative” type. In this type of utterance, a person is summoned by saying their name. That is, calling out ‘John!’ is functionally equivalent to saying ‘John, come here!’. In \REF{ex:enfield:16}, a foreman wants his tradesmen, who are working in a nearby building, to come and assemble for lunch. He calls out \textit{saang1} \textit{qeej4} ‘hey tradesmen!’. In another case (INTCN\_111202s\_980631), a girl is at her family rice fields, and wants her older brother, who is in a paddy field a hundred or so meters away, to come and help with a task. She calls \textit{qaaj4-dong3} ‘elder brother Dong!’. This would mobilize him to go and help her. %INTCN\_111202n\_RCR\_892800

\section{Formats in Move B: The responding move}

There is a range of things that Person B can do in the response move of a recruitment sequence. \tabref{tab:enfield:6} gives a breakdown, from the 181 cases in the Lao data where it is possible to tell how initiating moves were responded to.

\begin{table}
\begin{tabularx}{0.75\textwidth}{Xrr}
\lsptoprule
Response action & Count & Proportion \\
\midrule
No uptake or “other” & 86 & 47.5\%\\
Quickly fulfills & 53 & 29.3\%\\
Plausibly starts fulfilling & 34 & 18.8\%\\
Rejects & 7 & 3.9\%\\
Initiates repair & 1 & 0.6\%\\
\lspbottomrule
\end{tabularx}
\caption{Response action.}
\label{tab:enfield:6}
\end{table}

It is striking that nearly half of all cases are “no uptake” or “other”. This may seem to imply that requests and similar actions are ignored half of the time (cf. Blythe, \chapref{sec:blythe}, \sectref{sec:blythe:4.2.4}). But this is not what is going on here. Often it is because Person B does not hear or notice that the request is issued (recall that these are noisy village environments). Sometimes it is because people re-issue an initiating move before the other person has heard or had a chance to respond. When we restrict the count to “last of non-minimal sequence” plus “one and only” (see \chapref{sec:coding}, \sectref{sec:coding:6}), the proportions change a bit, specifically the proportion of “quickly fulfills” to “other” (\tabref{tab:enfield:7}).

\begin{table}
\begin{tabularx}{0.75\textwidth}{Xrr}
\lsptoprule
Response action & Count & Proportion\\
\midrule
No uptake or “other” & 40 & 38.5\%\\
Quickly fulfills & 40 & 38.5\%\\
Plausibly starts fulfilling & 17 & 16.3\%\\
Rejects & 6 & 5.8\%\\
Initiates repair & 1 & 1\%\\
\lspbottomrule
\end{tabularx}
\caption{Response action when sequential position is “last of non-minimal” or “one and only”.}
\label{tab:enfield:7}
\end{table}

A different breakdown of responses can be done using simple formal criteria. \tabref{tab:enfield:8} shows the relative frequency and proportions of responses that are (i) nonverbal only, (ii) verbal only, and (iii) composite of both verbal and relevant nonverbal behavior.

\begin{table}
\begin{tabularx}{0.66\textwidth}{Xrr}
\lsptoprule
Modality & Count & Proportion\\
\midrule
Nonverbal only & 137 & 60.7\%\\
Verbal only & 65 & 28.8\%\\
Composite & 24 & 10.6\%\\
\lspbottomrule
\end{tabularx}
\caption{Response modality.}
\label{tab:enfield:8}
\end{table}

The majority of responses in recruitment sequences (nearly two thirds) are fully nonverbal, and nearly three-quarters involve some form of relevant visible behavior.

\subsection{Fully nonverbal responses}

Fully nonverbal responses include behavior like the following:

\begin{itemize}
\item
Person B moves towards the television and reaches and switches it on (INTCN\_111204t\_827370) %13.50
\item
Person B stops what he is doing and walks up the stairs, goes into the kitchen, and tosses the rice (INTCN\_111203l\_427440)
\item
Person B reaches for the thing Person A wants, picks it up, and hands it to Person A \REF{ex:enfield:24} %(INTCN\_111203l\_644660)
\item
Person B slides bowl with juice in direction of Person A \REF{ex:enfield:25} %(INTCN\_030731b\_192570)
\end{itemize}

These are common and straightforward kinds of scenario. Nothing more is done by Person B than simply complying with the desired behavior.

\subsection{Verbal elements of responses}

The functional core of a response in a recruitment sequence is the bodily conduct that constitutes the assisting or collaborating behavior. As noted in the previous section, only a minority of the responses surveyed here have a verbal component. Few generalizations about these verbal aspects of responses are possible, but two points are worth mentioning.

First, there are cases in which the recruited action is itself a piece of verbal behavior, and not a bodily action like turning off a switch or passing something. In the following example, a mother-daughter pair (both adults) are sitting in a village house. The daughter’s baby is asleep in a cradle in a nearby house. The daughter has sent a young girl to go and check on the baby, to see if it has woken up. As the young girl is walking over to the other house, the mother tells her daughter to call out to the young girl and instruct her to bring the baby over to them if it has awoken.

\transheader{ex:enfield:15}{INTCN\_111204t\_769065}\vspace{-1mm}
%
\begin{mdframednoverticalspace}[style=firstfoc]
\begin{transbox}{1}{a}
\begin{verbatim}
khan2 man2     tùùn1 qaw3  maa2 haj5 kuu3     sii4 vaa1
if    3SG.BARE wake  take  come give 1SG.BARE thus say
\end{verbatim}
say “if she’s awake bring her to me” 
\end{transbox}
\end{mdframednoverticalspace}
%
\emptytransbox{~}{((addressed to B, the speaker’s adult daughter))}
%
\begin{mdframednoverticalspace}[style=secondfoc]
\begin{transbox}{2}{b}
\begin{verbatim}
khan2 man2     tùùn1 laø  qaw3 nòòng4    maa2 haj5 dee4
if    3SG.BARE wake  then take y.sibling come give FAC.ONRCRD
\end{verbatim}
if she’s awake, bring her to me, y’hear!
\end{transbox}
\end{mdframednoverticalspace}
%
\emptytransbox{~}{((called out to girl on her way to other house))}\\

In another example, an elderly man is sitting in a village temple building where lunch has been prepared for a group of tradesmen who are working some distance away, in the temple grounds. He is with the tradesmen’s foreman, who has just called out to the tradesmen to come and eat lunch. He wonders if the tradesmen did not hear him, and then asks if the elderly man -- who he says has a suitably loud voice -- could call out to them.

\transheader{ex:enfield:16}{INTCN\_111202n\_RCR\_892800}\vspace{-1mm}
%
\begin{mdframednoverticalspace}[style=firstfoc]
\begin{transbox}{1}{a}
\begin{verbatim}
qoo4 phoø-tuu4   pêê3 nan5    lèq1 (.) siang3 dang3 niø qaw2 (.)
INTJ grandfather P    DEM.EXT PRF  (.) voice  loud  TPC INTJ (.)
\end{verbatim}
oh grandpa Pêê (.) he has a loud voice (.)
\end{transbox}
\end{mdframednoverticalspace}
%
\begin{mdframednoverticalspace}[style=firstfoc]
\begin{transbox}{2}{~}
\begin{verbatim}
khùù2-khùù2  niø (.) hòòng4 beng1 duu2
RDP-suitable TPC (.) call   look  IMP.PLEAD
\end{verbatim}
it is suitable ((for calling out to people far away)) (.) call them please
\end{transbox}
\end{mdframednoverticalspace}
%
\begin{mdframednoverticalspace}[style=secondfoc]
\begin{transbox}{3}{b}
\begin{verbatim}
saang1    qeej4
tradesman VOC
\end{verbatim}
hey tradesmen!
\end{transbox}
\end{mdframednoverticalspace}\bigskip

Second, rejecting a request or declining to comply is often done by verbally stating a reason for the rejection or declination. In 18 cases in the Lao data, there is a clausal statement of a reason. None of these are cases of fulfilling, or plausibly beginning to fulfill, a recruitment. Stated reasons for rejection include, for example, that the thing being asked for is not available, or that the addressee is not free to do the act being recruited. See examples of reasons given with rejections in the following section.

\subsection{Types of rejections}

The Lao examples yield only seven cases of a response that rejects or explicitly signals that the person will not do what is asked. In most of the observed cases, the rejection or declination is done by stating a reason why Person B cannot do what is asked of them. This aligns with the classical analysis of speech acts that refers to the felicity conditions of an action \citep{austin_how_1962}. These conditions need to be presupposed if the intended speech act is to be consummated. For example, if a request is made, there are certain “preparatory conditions”, including that Person B should be able to carry out the requested act \citep[66]{Searle1969}. One way in which a person can reject or decline a request is to state or suggest that a preparatory condition does not hold \citep[87--88]{LabovFanshel1977}.

In an example, Person A asks Person B to turn on the television. Person B’s way of declining is to suggest that the television does not work.

\transheader{ex:enfield:17}{INTCN\_111204t\_818990}\vspace{-1mm}
%
\begin{mdframednoverticalspace}[style=firstfoc]
\begin{transbox}{1}{a}
\begin{verbatim}
peet5 tholathat1 beng1 mèè4
open  television look  IMP.UNIMP
\end{verbatim}
turn on the television for us to watch
\end{transbox}
\end{mdframednoverticalspace}
%
\begin{mdframednoverticalspace}[style=secondfoc]
\begin{transbox}{2}{b}
\begin{verbatim}
peet5 bòø daj4 tii4
open  NEG can  QPLR.PRESM
\end{verbatim}
I’m pretty sure it doesn't work ((‘it can't be turned on’))
\end{transbox}
\end{mdframednoverticalspace}\bigskip

Person B’s line is formally a question, but the use of the question particle \textit{tii4} is a way of conveying that you strongly suspect that the answer to your question is ‘yes’. By suggesting in this way that the television does not work, Person B is directly attending to one of the preparatory conditions of the request or command being issued, namely that it is in fact possible to carry out the service requested (see \extref{ex:enfield:8}, in which the rejection move is done by stating that a requested food is finished up). %INTCN\_111204x\_RCR\_495541

In a second example, Person A directly asks Person B to give them some of the herbal medicinal root that they are holding. Person B declines to do so, by stating that ‘there is only one piece’ of the root.

\transheader{ex:enfield:18}{INTCN\_111204x\_RCR\_296281}\vspace{-1mm}
%
\begin{mdframednoverticalspace}[style=firstfoc]
\begin{transbox}{1}{a}
\begin{verbatim}
qaw3 maa2 qaw3 maa2 ((reaching for requested object))
take come take come
\end{verbatim}
give it here, give it here
\end{transbox}
\end{mdframednoverticalspace}
%
\begin{mdframednoverticalspace}[style=secondfoc]
\begin{transbox}{2}{b}
\begin{verbatim}
mii2  khòò5 diaw3  nùng1
exist joint single one
\end{verbatim}
there's only one piece
\end{transbox}
\end{mdframednoverticalspace}\bigskip

By saying that there is only one piece, Person B directly attends to one of the preparatory conditions of the request, namely that it is possible to fulfill it. Here it is not technically impossible to give the medicine, but the speaker is appealing not so much to what is possible, but to what is reasonable. When it comes to goods such as medicines, Lao speakers tend to be willing to share, but in this case the addressee has only one piece of the medicine, and it is medicine that he is using to treat a current illness. His rejection appeals to the absence of a condition that would define a \textit{reasonable} possibility to comply.

In a third example, Person A directly asks Person B to go and get a mortar and pound some papaya. Person B declines to do so, by conveying that ‘there is no hurry’ to do it, given the time frame of the food preparation that is going on.

\transheader{ex:enfield:19}{INTCN\_030731b\_695170}\vspace{-1mm}
%
\begin{mdframednoverticalspace}[style=firstfoc]
\begin{transbox}{1}{a}
\begin{verbatim}
paj3 qaw3 khok1  maø     tam3  paj3
go   take mortar DIR.ALL pound go
\end{verbatim}
go and get a mortar to do the pounding
\end{transbox}
\end{mdframednoverticalspace}
%
\begin{mdframednoverticalspace}[style=secondfoc]
\begin{transbox}{2}{b}
\begin{verbatim}
qoo4 jaa1     faaw4 thòòq2
INTJ NEG.IMPV rush  INTJ
\end{verbatim}
oh, (let’s) not rush.
\end{transbox}
\end{mdframednoverticalspace}\bigskip

Here, Person B is not disputing that the requested service is appropriate, nor that they are able to carry it out, but rather they are disputing that it needs to be done \textit{now}.

A fourth example is from an “alter trajectory” sequence. A preparatory condition for this type of sequence is that Person B is currently engaged in a behavior that is somehow (potentially) problematic, such that it should be altered or halted. In \REF{ex:enfield:6} above, this condition was satisfied by the evident fact that the railing Person B was about to sit on was rickety. In the following case, Person A states that Person B is ‘blocking her brother’. This kind of statement of a problem is a well-known way of getting someone to do something (see \citealt{Rossi2018} and references therein), or at least, people may respond to such statements by helping, or at least offering to help. But in this case, Person B explicitly disputes the truth of the assertion made, thus denying that there is any problem in need of solving. %(INTCN\_030731b\_441300)

\transheader{ex:enfield:20}{INTCN\_111204x\_RCR\_153391}\vspace{-1mm}
%
\begin{mdframednoverticalspace}[style=firstfoc]
\begin{transbox}{1}{a}
\begin{verbatim}
qaw4       bang3 qaaj4
INTJ.SRPRS block elder.brother
\end{verbatim}
hey you're blocking your brother
\end{transbox}
\end{mdframednoverticalspace}
%
\begin{mdframednoverticalspace}[style=secondfoc]
\begin{transbox}{2}{b}
\begin{verbatim}
bang3 qiñang3 kòq2    qaaj4-nik1         laaw2  hên3 dòòk5
block what    Q.AGAIN elder.brother-Nick 3SG.FA see  FAC.RESIST
\end{verbatim}
what am I blocking? Nick can see fine
\end{transbox}
\end{mdframednoverticalspace}\\

In a final case, a man has been skinning catfish for some time and is now evidently tired of it, but he has not yet finished the job. He directs his wife, who is sitting nearby and also busy with laborious food preparation, to do this for him. She refuses, not by saying ‘no’, but by asking a question ‘why don’t you do it?’.

\transheader{ex:enfield:21}{INTCN\_111203l\_682150}\vspace{-1mm}
%
\begin{mdframednoverticalspace}[style=firstfoc]
\begin{transbox}{1}{a}
\begin{verbatim}
qaw3 nang3 maa2 saj1 phii4
take skin  come put  here
\end{verbatim}
put the skin ((of the fish)) in here 
\end{transbox}
\end{mdframednoverticalspace}
%
\emptytransbox{~}{((pointing in direction of the fish skin, then to the bowl where it is to go))}
%
\begin{mdframednoverticalspace}[style=secondfoc]
\begin{transbox}{2}{b}
\begin{verbatim}
caw4    khùù2 bòò1 hêt1 san4
2SG.POL why   not  do   so
\end{verbatim}
why don't you do it
\end{transbox}
\end{mdframednoverticalspace}\vspace{-1mm}
%
\emptytransbox{~}{((pointing to fish that A already has in front of him, and could skin by himself))}\bigskip

Her question challenges a key presupposition of the initiating move by Person A, namely that her husband cannot (reasonably) do the action himself. This comes across in the context as a blunt refusal, yet it is still done using an indirect strategy.

The various forms of rejection observed in this section have the “indirect” quality that would be predicted by well-known social theories of language use. Brown and Levinson's (\citeyear{BrownLevinson1987}) theory of politeness predicts that “face-threatening acts” such as refusals will be more likely handled by off-record means. Instead of saying ‘no’ in the above cases, people instead give reasons, in the form of a reference to a problem with a preparatory condition for the speech act in question.

\subsection{Acknowledgment in third position}

Lao speakers in informal family and village settings seldom say ‘thank you’ or anything resembling it. There are only two cases in the corpus in which there is arguably an acknowledgment by Person A that Person B has fulfilled a request or otherwise assisted. In both cases, this acknowledgment is a simple interjection of confirmation, meaning ‘yes’ or ‘that’s right’.

\transheader{ex:enfield:22}{INTCN\_111203l\_636171}\vspace{-1mm}
%
\begin{mdframednoverticalspace}[style=firstfoc]
%
\begin{transbox}{1}{a}
\begin{verbatim}
qaw3 nii3 paj3 kaj3-kaj3
take this go   RDP -far
\end{verbatim}
take it far away
\end{transbox}
\end{mdframednoverticalspace}
%
\begin{mdframednoverticalspace}[style=secondfoc]
\xtransbox{2}{b}{((picks up pot to move it))}
\end{mdframednoverticalspace}\vspace{-1mm}
%
\begin{transbox}{3}{a}
\begin{verbatim}
qee5
yeah
\end{verbatim}
yeah ((= yes, that’s right))
\end{transbox}\smallskip

\newpage
\transheader{ex:enfield:23}{INTCN\_020727a\_559100}\vspace{-1mm}
%
\begin{mdframednoverticalspace}[style=firstfoc]
\begin{transbox}{1}{a}
\begin{verbatim}
((crawls forward in direction of basket))
\end{verbatim}
\end{transbox}
\end{mdframednoverticalspace}
%
\begin{mdframednoverticalspace}[style=secondfoc]
\begin{transbox}{2}{b}
\begin{verbatim}
caw4    khiaw4 vaa3 ((passes basket to A))
2SG.POL chew   QPLR.INFER
\end{verbatim}
you'll chew? 
\end{transbox}
\end{mdframednoverticalspace}
%
\begin{transbox}{3}{a}
\begin{verbatim}
mm5
yeah
\end{verbatim}
yeah ((= yes, that’s right))
\end{transbox}\bigskip

The data in this study are from highly informal settings. Acknowledgments of compliance or assistance are almost entirely non-existent in these settings, and when they do happen, as in these cases, they are not of the ‘thank you’ variety. Lao speakers do have a way of saying ‘thank you’ -- the term is \textit{khòòp5 caj3} -- but it is restricted to more formal situations, or when speaking to strangers.

Other kinds of third position uptake practices following compliance moves in recruitment sequences are not frequent in this action context either. The following examples are typical of the Lao data in that they do not feature any acknowledgment following Move B.

\transheader{ex:enfield:24}{INTCN\_111203l\_644660}\vspace{-1mm}
%
\begin{mdframednoverticalspace}[style=firstfoc]
\begin{transbox}{1}{a}
\begin{verbatim}
qaw3 tanaang1 dèèng3 maa2
take netting  red    come
\end{verbatim}
bring the red netting here
\end{transbox}
\end{mdframednoverticalspace}
%
\begin{mdframednoverticalspace}[style=secondfoc]
\xtransbox{2}{b}{((reaches for the thing A wants, picks it up and hands it to A))}
\end{mdframednoverticalspace}\vspace{-1mm}
%
\emptytransbox{3}{((interaction continues))}\smallskip

\transheader{ex:enfield:25}{INTCN\_030731b\_192570}\vspace{-1mm}
%
\begin{mdframednoverticalspace}[style=firstfoc]
\begin{transbox}{1}{a}
\begin{verbatim}
qaw3 maa2
take come
\end{verbatim}
bring it here ((the bowl of leaf juice))
\end{transbox}
\end{mdframednoverticalspace}
%
\begin{mdframednoverticalspace}[style=secondfoc]
\xtransbox{2}{b}{((slides bowl with juice in direction of A))}
\end{mdframednoverticalspace}\vspace{-1mm}
%
\emptytransbox{3}{((interaction continues))}\smallskip

\transheader{ex:enfield:26}{INTCN\_111204q\_RCR\_890111}\vspace{-1mm}
%
\begin{mdframednoverticalspace}[style=firstfoc]
\begin{transbox}{1}{a}
\begin{verbatim}
thêêk5 qan3     nan4     qòòk5 kòòn1  dèè1     luuk4
pour   CLF.INAN DEM.DIST exit  before IMP.SOFT child
\end{verbatim}
pour that stuff out first, child
\end{transbox}
\end{mdframednoverticalspace}
%
\begin{mdframednoverticalspace}[style=secondfoc]
\xtransbox{2}{b}{((pours the water as asked))}
\end{mdframednoverticalspace}\vspace{-1mm}
%
\emptytransbox{3}{((interaction continues))}

\section{Social asymmetries}\label{sec:enfield:6}

Social asymmetries in Lao social interaction can be defined in terms of a metaphor of height (see \citealt{Enfield2015b}). In most dyads, one person is considered to be socially “above” the other person. Naturally it is not always a straightforward judgement as to who is above whom, given the sometimes fluid and contestable nature of social relations. But in the kinds of home and village settings focused on in this study, the social order is clear.\footnote{This is not to say that people follow its associated linguistic norms to the letter; the norms can be flouted, negotiated, and contested in numerous ways.} The core measure of social asymmetry in dyads is the relative birth order of siblings, and associated practices, many of which are linguistic in nature \citep{Enfield2015b}. In the home and village, there is no ambiguity as to how most people relate to each other within this height-based conception of social difference. People are either related by kin or they are classified as such.

Where it was possible to determine the social asymmetries between dyads in the data described in this study -- the three possibilities being that Person A is higher than Person B, the two are equal in status, or Person A is lower than Person B -- here is what I found (\tabref{tab:enfield:9}).

\begin{table}
\begin{tabularx}{0.5\textwidth}{Xrr}
\lsptoprule
Relation & Count & Proportion\\
\midrule
A>B & 123 & 60.9\%\\
A=B & 38 & 18.8\%\\
A<B & 41 & 20.3\%\\
\lspbottomrule
\end{tabularx}
\caption{Social asymmetry.}
\label{tab:enfield:9}
\end{table}

Only one in five recruitment sequences features a lower-ranked person getting help from a higher-ranked one. Three in five are issued in a downward direction. This suggests support for Brown and Levinson's (\citeyear[69--74]{BrownLevinson1987}) flow-chart model by which people select from among various options when planning to carry out potentially face-threatening acts. At the first point of choice in their model, if a person judges that the potential threat to face is particularly high, they can choose not to carry out the act at all. This is arguably what accounts for the lower frequency of requests and similar actions directed toward higher-ranked people (see also Floyd, \chapref{sec:floyd}, \sectref{sec:floyd:6}; Baranova, \chapref{sec:baranova}, \sectref{sec:baranova:6}; Dingemanse, \chapref{sec:dingemanse}, \sectref{sec:dingemanse:5.2}).

When Lao speakers get lower-ranked people than themselves to do things, this is not just a preference, it reflects a strong asymmetry in entitlements (to expect assistance from lower-ranked people) and obligations (to provide assistance to higher-ranked people). This is especially apparent in cases of \textit{delegation}: Person A asks lower-ranked Person B to do something, and Person B immediately delegates the task to Person C, who in turn is ranked lower than B (see also Blythe, \chapref{sec:blythe}, \sectref{sec:blythe:6}).

In a case from a family food preparation scene, when Person B is asked to go and scoop some jugged fish and bring it to use in cooking, she does not carry out the action. Instead, she turns to her younger sibling -- Person C -- and re-issues the command, which Person C then immediately fulfills.

\transheader{ex:enfield:27}{CONV\_020723b\_RCR\_126590}\vspace{-1mm}
%
\begin{mdframednoverticalspace}[style=firstfoc]
\begin{transbox}{1}{a}
\begin{verbatim}
tak2  paø-dèèk5           hêt1 viak4 lèèw4  laø cang1 paj3 ((to B))
scoop CM.FISH-jugged.fish do   work  finish PRF then  go
\end{verbatim}
scoop some jugged fish and do your work, and then go
\end{transbox}
\end{mdframednoverticalspace}
%
\begin{mdframednoverticalspace}[style=firstfoc]
\begin{transbox}{2}{b}
\begin{verbatim}
qee5 khiaw5 paj3 tak2  paø-dèèk5           paj3 ((looking at C))
yeah hurry  go   scoop CM.FISH-jugged.fish go
\end{verbatim}
yeah, go and scoop some jugged fish
\end{transbox}
\end{mdframednoverticalspace}
%
\begin{mdframednoverticalspace}[style=secondfoc]
\xtransbox{3}{c}{((gets up to walk over to jugged fish to scoop some up))}
\end{mdframednoverticalspace}

Another example of delegation by lower-ranked Person B to a yet lower-ranked person is \REF{ex:enfield:5} above. %(INTCN\_111202n\_RCR\_989020)

\section{Conclusion}

This survey of semiotic resources for getting people to do things in Lao has concentrated on home and village interaction. The observations made here are not claimed to hold for the full range of contexts and domains in which Lao speakers operate, such as the formal and institutional settings that people sometimes find themselves in. That said, the informal home and village contexts discussed here are arguably the dominant ones in ordinary people’s lives, and therefore require the core set of practices that any member of the Lao-speaking community should command. The overview presented here is therefore offered as a reference point for further work in this area.

Taken together, the above-described practices that Lao speakers use in getting each other to do things show two striking properties. First, they are varied and textured in kind: Lao speakers draw from a range of semiotic options (linguistic or otherwise) for formulating their moves in recruitment sequences. Second, when observed in operation in a corpus, these sets of options show characteristic properties of a functional system. The numerous statements of relative frequency of options summarized in the many data tables provided above show precisely the skewed frequency distributions that are typical of functional systems across widely varying domains, from national economies to academic citation patterns to TV remote control handsets. Here we see the Pareto Principle -- or the Law of the Vital Few and the Trivial Many -- at work (\citealt{Pareto1971}; see also \citealt{Zipf1949}).\footnote{The phrase “Law of the Vital Few and the Trivial Many” is attributed to Joseph M. Juran in 1941.} While many tools are available, a small number of them will carry the greatest functional load for those who use the system.

\section*{Abbreviations}
% \hspace*{.4\textwidth}
\begin{tabularx}{.52\textwidth}{>{\scshape}lQ}
1/2/3 & first/second/third person\\
bare & bare (non-polite)\\
cm.fish & class marker for fish \\
cm.fruit & class marker for fruit\\
coll & collaborative\\
dem.dist & distal demonstrative\\
dem.ext & exterior demonstrative\\
dir.all & allative directional\\
fac.news & facultative, news-giving\\
fac.onrcd & facultative, putting on record\\
fac.resist & facultative, resisting\\
imp.plead & imperative, pleading\\
imp.rush & imperative, rushing\\
imp.soft & imperative, soft\\
imp.unimp & imperative, no impedance
\end{tabularx}
\begin{tabularx}{.51\textwidth}{>{\scshape}lQ}
intj & interjection\\
intj.annoyed & interjection, annoyed\\
intj.surprised & interjection, surprised\\
neg.impv & negative imperative\\
pol & polite\\
prf & perfect\\
q & question\\
q.again & question, again\\
qplr & polar question\\
qplr.presm & polar question, presuming\\
rdp & reduplication\\
sg & singular\\
top & topic\\
voc & vocative
\end{tabularx}

% \section*{Acknowledgments}

\sloppy
\printbibliography[heading=subbibliography,notkeyword=this]
\end{document}
