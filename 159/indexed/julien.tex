\documentclass[output=paper]{LSP/langsci} 
\author{
Marit Julien	\affiliation{Lund University}
}
\title{Head-initial postpositional phrases in North Sámi} 
% \epigram{Change epigram}
\abstract{Most adpositions in North Sámi are postpositions – they follow their complements in the surface order. Nouns, on the other hand, invariably precede their complements. Strikingly, when the nominal complement of a postposition has its own complement, the complement of the noun follows after the postposition, so that the nominal phrase ends up being discontinuous, split by the postposition. This is an indication that North Sámi postpositions are prepositions underlyingly, and that the surface order is the result of the complement of P moving to the Spec of a higher functional head. The complement of the noun is however spelled out in the lower position. Neither the complement stranding approach of \citet{Sheehan2009} nor the FOFC of \citet{Holmberg2000deriving} and \citet{BiberauerEtAl2008,BiberauerEtAl2014syntactic} can fully explain this pattern. Instead, in North Sámi a more specific requirement appears to be at work, which dictates that  a postposition must follow immediately after the nominal head of its complement. A similar effect is seen with possessors, which precede the possessees but also leave their complements behind in postnominal position.}

\ChapterDOI{10.5281/zenodo.1117706}
\maketitle
\begin{document}
\section{Introduction} %1. /

There are in principle two possible explanations for the ordering contrast between prepositional and postpositional phrases. One is that a (possibly local) head parameter gives prepositions when set to <\textsc{head first}> but postpositions when set to <\textsc{head last}>. The other is that the underlying order is the same in both cases, so that the two surface orders result from one or more movement operations.

In this paper I will present data from \ili{North Sámi} which indicate that postpositional phrases in this language result from movement of the complement of the adposition from a position following the adposition to a position preceding it. In other words, \ili{North Sámi} postpositions are prepositions underlyingly.

A striking feature of \ili{North Sámi} postpositional phrases is that if the nominal complement of the \isi{postposition} has its own complement, then this complement will follow the \isi{postposition}, arguably in the position where it originates. At first glance, the observed pattern appears to be similar to the complement stranding phenomenon described in \citet{Sheehan2009}, while the resulting order appears to be consistent with the Final-Over-Final-Condition\is{Final-over-Final Condition} proposed by \citet{Holmberg2000deriving} and \citet{BiberauerEtAl2008,BiberauerEtAl2014syntactic}. If the constructions in {question} are investigated in more detail, however, it turns out that they are not entirely in accordance with either approach. Instead, the surface order seen in \ili{North Sámi} seems to reflect a specific requirement that a \isi{postposition} must immediately follow the lexical head of its complement.


\section{The ordering of nouns and adpositions in North Sámi} %2. /

In \ili{North Sámi}, most \isi{adpositions}\is{adposition} follow their complement in the surface order, as in \REF{ex:julien:1}. That is, they are postpositions.\footnote{\ili{North Sámi} also has a few prepositions. An example is \textit{miehtá} ‘all over’, which is shown in (i).
\begin{exe}
 \exi{(i)}
 \gll Sii    galget        galledit  joatkkaskuvllaid        \textbf{miehtá}  \ul{riikka}.\\
      they  shall.\textsc{pres.\oldstylenums{3}pl}  visit.\textsc{inf} secondary.school.\textsc{pl.acc}  all.over  country.\textsc{gen}\\
 \glt `They are going to visit secondary schools all over the country.’ \end{exe}} We can also note that complements of \isi{adpositions}\is{adposition} have genitive case.\footnote{The examples in this paper are taken from the \ili{North Sámi} corpus developed by Giellatekno, Centre for Saami language technology at the University of Tromsø. See \url{http://gtweb.uit.no/korp/}.} Here and in following examples I boldface the relevant syntactic head (adposition or noun) and underline its complement.

\ea%1
    \label{ex:julien:1}
    \gll    \ul{mánáidgárddiid}      \textbf{várás}\\
	    kindergarten.\textsc{pl}.\textsc{gen}    for\\
    \glt   ‘for (the) kindergartens’
    \z
  

Nouns, on the other hand, precede their complements, while adnominal adjectives precede nouns. This is shown in \REF{ex:julien:2}, where the adjective \textit{ođđa} ‘new’ and the head noun \textit{láhka} ‘law’ precede the complement PP \textit{mánáidgárddiid várás} ‘for kindergartens’:

\ea%2
    \label{ex:julien:2}
   
    \gll  ođđa  \textbf{láhka}    \ulp{mánáidgárddiid}{~~~~~~~~~~}      \ule{várás}   \\ 
	    new  law.\textsc{nom} kindergarten.\textsc{pl.gen}    for\\
    \glt     ‘a/the new law for kindergartens’
    \z

Now if a structure like \REF{ex:julien:2} is to be embedded under a P, the result is as shown in \REF{ex:julien:3}:

\ea%3
    \label{ex:julien:3}
   
    \gll    \ulp{ođđa}{~~~~}  \ule{lága}      \textbf{birra}    \ulp{mánáidgárddiid}{~~~~~~~~~~}      \ule{várás}  \\ 
	    new  law.\textsc{gen}   about    kindergarten.\textsc{pl}.\textsc{gen}    for\\
    \glt    ‘about a/the new law for kindergartens’

    \z

We see here that the complement of the higher P ends up being discontinuous. While the noun and the adjective precede the higher P \textit{birra} ‘about’, the PP complement of the noun follows the higher P.


The same pattern is seen in all cases where the nominal complement of a \isi{postposition} has a postnominal modifier: the \isi{postposition} invariably appears between the noun and the postnominal modifier of the noun. The postnominal modifier of the noun can be a case-marked noun, as in \REF{ex:julien:4}, where the noun \textit{olbmuid} ‘people’ is modified by the noun \textit{govas}, which carries locative case and means ‘in the picture’. As we see, the \isi{postposition} \textit{birra} ‘about’ intervenes between \textit{olbmuid} and \textit{govas}.


\ea%4
    \label{ex:julien:4}
   
    \gll   \ulp{Eará}{~~~~}  \ule{olbmuid}        \textbf{birra}    \ul{govas}        sus      eai    lean          gal  dieđut.    \\ 
	   other  {person}.\textsc{pl}.\textsc{gen}    about    picture.\textsc{loc}    s/he.\textsc {loc}    \textsc{neg.\oldstylenums{3}pl}      be.\textsc{past.conneg}  \textsc{prt}  information.\textsc{pl.nom}\\
    \glt  ‘S/he had no information about other people in the picture.’
    \z

One can also note here that the modifying noun \textit{govas} ‘in the picture’ is only loosely connected to the head noun \textit{olbmuid} ‘people’ semantically. In some approaches modifiers of this type would be referred to as “adjuncts”. It is clear, though, that all postnominal modifiers of nouns show the same behaviour when their containing noun phrase is the complement of a \isi{postposition}. Thus, in \REF{ex:julien:5} the postpositional phrase \textit{mielli alde} ‘on the river bank’, which modifies the head noun \textit{johtalus(a)} ‘traffic’ and which might be taken to be an adjunct in the nominal phrase, is separated from that head noun by the \isi{postposition} \textit{dáfus} ‘concerning’ in the same way as the modifying PP is separated from the head noun in \REF{ex:julien:3}:


\ea%5
    \label{ex:julien:5}
   
    \gll  \ul{Johtalusa}  \textbf{dáfus}      \ulp{mielli}{~~~~~~~~~~~~~~~~~}          \uline{alde},  ferte          atnit     čielggasin  ahte …  \\ 
	  traffic.\textsc{gen}  concerning  river.bank.\textsc{gen}  on    must.\textsc{pres.\oldstylenums{3}sg}    consider.\textsc{inf} clear.\textsc{ess}  that …\\
    \glt    ‘Concerning traffic on the river bank, one must consider it clear that …’
    \z

Hence, the semantic\is{semantics} relation between the head noun and the modifier does not make any difference. To keep things simple I will refer to all postnominal modifiers as complements.

The complement of the noun can also be an infinitival clause, as in \REF{ex:julien:6}, or a finite\is{finiteness} clause, as in \REF{ex:julien:7}. In either case, the clause follows the \isi{postposition} – \textit{birra} ‘about’ in \REF{ex:julien:6}, \textit{vuostá} ‘against’ in \REF{ex:julien:7} – while the noun and prenominal modifiers precede it.\footnote{Nouns that are complements of numerals in the nominative singular appear in the genitive singular in \ili{North Sámi}, as seen in \REF{ex:julien:7}.}

\ea%6
    \label{ex:julien:6}
   
    \gll      Departemeanta  sáhttá        addit    láhkaásahusaid      \ul{gildosa}   \textbf{birra}    \ulp{guolástit}{~~~~~~~~~~}    \ulp{ja}{~~~~~~~~}    \ulp{bivdit}{~~~~~~~~~~}    \ulp{dihto}{~~~~~~~~~~}    \uline{guovlluin}.\\ 
    department.\textsc{nom}  can.\textsc{pres.\oldstylenums{3}sg}  give.\textsc{inf}  statutory.law.\textsc{pl.acc}    prohibition.\textsc{gen}    about    fish.\textsc{inf}    and  hunt.\textsc{inf}  certain  area.\textsc{pl}.\textsc{loc}\\
    \glt  ‘The department can issue statutory laws about the prohibition of fishing and hunting in certain areas.’
    \z

\ea%7
    \label{ex:julien:7}
   
    \gll    Eanetlohku    jienastii      \ulp{Sáme-dikki}{~~~~~~~~~~~~~~~~~~~~~~~~~~~}          \ule{evttohusa}    \textbf{vuostá}     \uline{ahte}  \ulp{dat}{~~~~~~~~~~~~}      \ulp{njeallje}{~~~~~~}    \ulp{ođđa}{~~~~~~}  \ulp{áirasa}{~~~~~~~~~~~~~~~~~~~~~~~~~~}           \ulp{galget}{~~~~~~~~~}    \ulp{mannat}{~~~~~~~~~}     \ule{dan} \ulp{sohkabeallái}{~~~~~~~~~~}  \ulp{mii}{~~~~~~~~~~~~~~}      \ulp{lea}{~~~~~~~~~~~~}    \uline{unnitlogus}.\\ 
	    majority.\textsc{nom}  vote.\textsc{past.\oldstylenums{3}sg}  Sámi-parliament.\textsc{gen}  proposal.\textsc{gen}  against    that  \textsc{dem.nom}  four.\textsc{nom}  new  representative.\textsc{gen}  shall.\textsc{\oldstylenums{3}pl}  go.\textsc{inf}   \textsc{dem} {gender}.\textsc{ill} which.\textsc{nom}  is    minority.\textsc{loc}\\
    \glt  ‘The majority voted against the proposal from the Sámi Parliament that the four new representatives should go to that {gender} which is in minority.’
    \z

In all likelihood, the noun \textit{lága} in \REF{ex:julien:3} underlyingly forms a constituent with the PP \textit{mánáidgárddiid várás,} just like it does in \REF{ex:julien:2}. Similarly,\textit{} the noun \textit{olbmuid} forms a constituent with the noun \textit{govas} in \REF{ex:julien:4}, and the same holds of the noun and the PP in \REF{ex:julien:5} as well as of the noun and the clause in \REF{ex:julien:6} and \REF{ex:julien:7}. There are two possible ways in which the surface order seen in these examples can be derived. Either the postpositions take their complements to the left, and the final PP in \REF{ex:julien:3}, the locative noun in \REF{ex:julien:4}, the PP in \REF{ex:julien:5} and the clauses in \REF{ex:julien:6} and \REF{ex:julien:7} have been moved to the right of the \isi{postposition}, or else the postpositions take their complements to the right, but parts of the complements move to the left of the \isi{postposition}. In the next section, I will take a closer look at these two possibilities.


\section{Leftward or rightward movement?} %3. /

Let us start by considering the underlying structure that must be postulated for \REF{ex:julien:3} if \ili{North Sámi} postpositions take their complement to the left to begin with. I sketch this structure schematically in \REF{ex:julien:8}. The nominal complement of the \isi{postposition} P\textsubscript{2} precedes P\textsubscript{2}, and it consists of the head noun N, the prenominal adjective A and the postnominal complement PP\textsubscript{1}. In order not to jump to any conclusions concerning the category of the nominal complement, I use the label XP here. 

\ea%8
    \label{ex:julien:8}
   
\ea[]{[\textsubscript{PP2} [\textsubscript{XP} A N PP\textsubscript{1}] P\textsubscript{2}]}

\ex[*]{%  
\gll\relax {[\textsubscript{PP2} [\textsubscript{XP}}  ođđa  lága      {[\textsubscript{PP1}}  mánáidgárddiid      várás]]  birra]\\
           {} new  law.\textsc{gen}   {}    kindergarten.\textsc{pl}.\textsc{gen}    for      about\\
\glt   ‘about a/the new law for kindergartens’}
\z
\z

From the structure in \REF{ex:julien:8}, the order seen in \REF{ex:julien:3} above can be derived by movement of PP\textsubscript{1} to the right of P\textsubscript{2}, as shown in \REF{ex:julien:9}:

\ea%9
    \label{ex:julien:9}
   
\ea{}
[[\textsubscript{PP2} [\textsubscript{XP} A N \sout{PP\textsubscript{1}}] P\textsubscript{2}] PP\textsubscript{1}]

\ex
\gll\relax  {[[\textsubscript{PP2} [\textsubscript{XP}}  ođđa  lága    \st{PP1}]  birra {]  [\textsubscript{PP1}}  mánáidgárddiid      várás]]\\
          {} new  law.\textsc{gen} {}    about   {}     kindergarten.\textsc{pl}.\textsc{gen}    for\\

\glt     ‘about a/the new law for kindergartens’
\z
\z

We see that on the assumption that \ili{North Sámi} postpositional phrases are head-final, the operation that is needed to get the right order is descriptively quite simple: all that is required is movement of the complement of the noun embedded under the higher P. As indicated in \REF{ex:julien:9}, one would probably have to say that the moved constituent right-adjoins to the higher PP. The problem with this proposal is that there is no obvious motivation for the movement operation.


As an alternative, one might want to propose that the landing site of the moved complement is the specifier position of a higher head Y, which has PP\textsubscript{2} as its complement to the left and takes its specifier to the right. The resulting structure would be as shown in \REF{ex:julien:10}.


\ea%10
    \label{ex:julien:10}
 	  [\textsubscript{YP} [\textsubscript{PP2} [\textsubscript{XP} A N \sout{PP\textsubscript{1}}]  P\textsubscript{2}] Y PP\textsubscript{1}]
\z

There are however also certain problems with this proposal. Firstly, few or no cases of specifiers located to the right are attested in natural languages (see e.g. \citealt{Kayne1994}). Secondly, the trigger for the movement remains mysterious – what property of Y could cause attraction of constituents as different as nouns, PPs and finite\is{finiteness} and non-finite\is{finiteness} clauses, across other elements in the nominal phrase?


The assumption that \ili{North Sámi} PPs are underlyingly head-final does not lead to any satisfactory explanation for the orders that arise when the nominal complement of P has its own complement. So let us look instead at the consequences of taking \ili{North Sámi} PPs to be underlyingly head-initial. Phrases like the one in \REF{ex:julien:3} would then have the structure shown in \REF{ex:julien:11} after the higher P has been merged over the nominal phrase:


\ea%11
    \label{ex:julien:11}
   
\ea[]{[\textsubscript{PP2} P\textsubscript{2} [\textsubscript{XP} A N PP\textsubscript{1}]]}
\ex[*]{%
\gll\relax  [\textsubscript{PP2}  \textbf{birra}    [\textsubscript{XP}  ođđa  lága    [\textsubscript{PP1}  mánáidgárddiid      várás]]]\\
          {} about   {}     new  law.\textsc{gen}  {}   kindergarten.\textsc{pl}.\textsc{gen}    for\\
\glt        ‘about a/the new law for kindergartens’}
\z
\z

Then a movement operation applies to \REF{ex:julien:11} which gives as net result the structure sketched in \REF{ex:julien:12}:

\ea%12
    \label{ex:julien:12}
   
  \ea{}     [[\textsubscript{ZP} A N] … P\textsubscript{2} [\textsubscript{XP} \sout{A N} PP\textsubscript{1}]]
\ex \gll\relax  [[\textsubscript{ZP}  ođđa  lága    ]  birra    {[\textsubscript{XP} \sout{ZP}}  mánáidgárddiid      várás]]\\
         {} new  law.\textsc{gen}  {} about       {}   kindergarten.\textsc{pl}.\textsc{gen}   for\\
\glt     ‘about a/the new law for kindergartens’
\z
\z

Here some constituent ZP, which contains the adjective and the noun, has moved to the left of the higher adposition, while the complement of N is left behind. However, there cannot be any constituent that contains the adjective and the noun but excludes the complement of N. This means that there is more involved in the derivation of \REF{ex:julien:3} than what is indicated in \REF{ex:julien:12}.


One possible derivation is shown in \REF{ex:julien:13}. Here PP\textsubscript{1} has moved to the Spec of a head Y which is located above PP\textsubscript{2}. Assuming that \isi{adpositions}\is{adposition} are associated with functional domains, one could take Y to be a head in the functional domain of P\textsubscript{2}. Then the whole nominal phrase, from which PP\textsubscript{1} has been extracted, moves to the Spec of an even higher head Z, while P\textsubscript{2} itself moves to Z, presumably via Y (a step not shown here).


\ea%13
\label{ex:julien:13}
[\textsubscript{ZP}
  [\textsubscript{XP}
    A 
    N 
    \sout{PP\textsubscript{1}}
  ] 
  P2-Z 
  [\textsubscript{YP} 
    PP\textsubscript{1} 
    Y 
    [\textsubscript{PP2} 
      \sout{P\textsubscript{2} 
      [\textsubscript{XP} 
	A 
	N 
	PP\textsubscript{1}
      ]}
    ]
  ]
]
\z 

This derivation might be possible, but it is complicated, and it involves a {number} of movement operations that would have to be motivated. 

I will propose instead a much simpler derivation, which I sketch in \REF{ex:julien:14}. Here the nominal complement of P moves as a whole to the Spec of a functional head p above P. The noun, and the elements that precede it inside the nominal phrase, are spelled out in the higher position, while any phrase YP that is the complement of the noun is spelled out in the lower position. Thus, the surface order is partly a matter of spellout.


\ea%14
    \label{ex:julien:14}
\ea\relax    
[\textsubscript{pP} 
  [\textsubscript{XP} 
    A 
    N 
    \sout{YP}
  ] 
  p 
  [\textsubscript{PP} 
    P 
    [\textsubscript{XP} 
      \sout{A 
      N} 
      YP
    ]
  ]
]
\ex
\gll\relax  [\textsubscript{pP} [\textsubscript{XP} ođđa   {lága \sout{PP}} ] p [\textsubscript{PP} birra    {[\textsubscript{XP} \sout{ođđa lága}  [\textsubscript{PP}}   mánáidgárddiid várás]]]]\\
           {} {} new   law.\textsc{gen}   {} {}  {} about        {}       kindergarten.\textsc{pl}.\textsc{gen}      for\\
\glt      ‘about a/the new law for kindergartens’
\z
\z 

This analysis has a {number} of advantages. The problems related to \isi{extraction} of complements of different types disappear, since there is no \isi{extraction} of complements. The landing site is unproblematic, since it is a higher specifier position to the left, and the trigger is likely to be a feature of the attracting head.


As an alternative to the movement sketched in \REF{ex:julien:14}, one might want to propose that the nominal complement of P raises to the specifier of P. However, it has been proposed for adpositional phrases in many other languages that they contain functional elements – see e.g. \citet{Koopman2000} and the articles in \citet{Cinque2010pps}. It is likely, then, that \ili{North Sámi} postpositional phrases also contain more elements that just P. We also have indications that at least some postpositions in \ili{North Sámi} are structurally complex.

In particular, many postpositions with local meaning come in several variants that are differentiated by suffixes. Some examples are given in \REF{ex:julien:15}:


\ea%15
\label{ex:julien:15}
\begin{tabular}[t]{*{5}{l}}
& \textsc{location}    & \scshape goal  & \scshape path & \\
a.  &  bálddas   &   báldii   &   báldal & ‘beside’                 \\
b.  &  duohken   &   duohkái  &  duogi   & ‘behind’                 \\
c.  &  gaskkas   &   gaskii   & gaskal   & ‘between’                \\
d.  &  geahčen   &   geahčai  &  geaže   & ‘at/to/past the end of’  \\
e.  &  maŋis     &   maŋŋái   & maŋil      & ‘after’                \\
f.  &  vuolde    &   vuollái  &  vuole    & ‘under’                 \\
\end{tabular}
\z

These postpositions consist of an invariant, root-like part plus endings that encode either location, goal of movement or path of movement. In order to show how this works in context, I give in \REF{ex:julien:16} one example with each of the three postpositions that correspond to English \textit{under}:

\ea%16
    \label{ex:julien:16}
\ea
\gll  Gávdnen    iežan      niibbi      \uline{duorggaid}    \textbf{vuolde}.\\
      find.\textsc{past.\oldstylenums{1}sg}  own.\textsc{\oldstylenums{1}sg}    knife.\textsc{acc}  twig.\textsc{pl.gen}  at.under\\
\glt       ‘I found my knife under the (heap of) twigs.’
\ex
\gll  Doppe  čakŋala      \ulp{liegga}{~~~~~} \ule{gokčasa}      \textbf{vuollái}.\\
    there    creep.\textsc{pres.\oldstylenums{3}s}  warm    blanket.\textsc{gen}  to.under\\
\glt       ‘There s/he  creeps under a/the warm blanket.’
\ex
\gll    Geaidnu    manai      \uline{bávtte}    \textbf{vuole}.\\
      road.\textsc{nom}  go.\textsc{past.\oldstylenums{3}sg}  cliff.\textsc{gen}   past.under\\
\glt       ‘The road passed under a cliff.’
\z
\z

\citet{Pantcheva2011} argues that spatial expressions involve a \isi{universal} hierarchy of elements which can be given as follows: Route > Source > Goal > \isi{Place}. The \ili{North Sámi} data are compatible with this claim, although the hierarchical ordering is not directly visible in this language, since there are no containment relations between the relevant markers. We can nevertheless take the postpositions in \REF{ex:julien:15} to reflect the postpositional base, which corresponds to the Axial Part in \citet{Pantcheva2011} and also in \citet{Svenonius2006axialparts}, in combination with an element that spells out\is{Spell-Out} one of the heads Route, Goal or \isi{Place}.\footnote{There is no specialised marking of Source in \ili{North Sámi}. Instead, the forms that encode \isi{Place} can also be interpreted as Source. However, \citet{Svenonius2009} argues that the source reading is always imposed from outside of the phrase that carries the marking – for example by a motion verb or by some other element expressing transition. Hence, there appears to be no reflex of the Source head in this language.} This means that the \ili{North Sámi} postpositions are structurally complex, so that there are landing sites in Spec positions above P for constituents that move out of the complement of P. 

This suffices to motivate the analysis that I propose of the constructions exemplified in (\ref{ex:julien:3}–\ref{ex:julien:7}) above. I will continue to refer to the functional head above P as p, and I will not discuss its identity any further here. The p head has an \isi{EPP} feature which forces the nominal complement of P to move to Spec-pP. This means that the real \isi{postposition} is p. But note that the \isi{morphology} of the complex postpositions shown in \REF{ex:julien:15} suggests that P head-moves to p, since the elements that I take to be realisations of p are suffixed to the postpositional bases, which I take to be realisations of P. In any case, if the nominal complement of P contains a complement of N, then this complement of N will obligatorily be spelled out in the lower position.

\section{\citet{Sheehan2009} on complement stranding} %4. /
\citet{Sheehan2009} discusses data from English which bear a striking resemblance to the \ili{North Sámi} examples shown above. In English, PP complements of nouns and adjectives can in some cases be left behind when the nominal or adjectival phrase moves to a higher position. Two of Sheehan’s examples are given in \REF{ex:julien:17}:\largerpage

\ea%17
    \label{ex:julien:17}
\ea\label{ex:julien:17a}   A new book has come out about String Theory.

\ex\label{ex:julien:17b}  How certain are you that the Mets will win?
\z
\z

As we see, the PP complement of the noun \textit{book} in \REF{ex:julien:17a} and the CP\is{complementizer} complement of the adjective \textit{certain} in \REF{ex:julien:17b} have apparently been left behind after movement of their containing constituent. Sheehan analyses this phenomenon, which she calls \textit{complement stranding}, as a consequence of the linearization procedure. I will summarise her analysis very briefly here.

In the structure in \REF{ex:julien:18}, the phrase β has been moved from the complement position of θ to the specifier position of α. Assuming that asymmetric c-command maps to linear precedence, in accordance with the \isi{Linear Correspondence Axiom} (LCA) proposed by \citet{Kayne1994}, Sheehan notes that β’s complement λ does not asymmetrically c-command anything in either of its positions. Moreover, in its higher position λ\textsubscript{2} it cannot be ordered with respect to α and θ, since it neither asymmetrically c-commands nor is asymmetrically c-commanded by either of them, on her definition of c-command. Consequently, only the base-generated copy of λ is a legitimate target for PF.

\ea%18
\label{ex:julien:18}
\begin{forest}
[α [β\textsubscript{2} [δ\textsubscript{2}] [β\textsubscript{2} [ β\textsubscript{2} ] [  λ\textsubscript{2} ] ] ] [ α [ α ] [ θ [ θ ] [ β\textsubscript{1} [  δ\textsubscript{1} ]  [     β\textsubscript{1}  [ β\textsubscript{1}       ] [λ\textsubscript{1} ] ] ] ] ] ]
\end{forest}
\z

However, as Sheehan also points out, this analysis predicts that in a phrase that moves from a complement position to a specifier position, the complement of the highest head inside that phrase will be stranded. For example, if the highest head in the nominal phrase \textit{many books about morphology} is a Num\is{number} head, with \textit{many} located in the NumP \isi{projection}, we might expect the complement of Num\is{number} to be stranded, as in \REF{ex:julien:19a}, instead of the complement of N, as in \REF{ex:julien:19b}, the grammatical version.

\ea%19
    \label{ex:julien:19}
\ea[*]{\label{ex:julien:19a}Many have been {borrowed} books about {morphology}.}
\ex[]{\label{ex:julien:19b}Many books have been {borrowed} about {morphology}.}
\z
\z

To get around this potential problem, Sheehan suggests that the Num\is{number} head attracts its complement to its specifier position, and that this is the operation where the decision to spell out\is{Spell-Out} the complement of N in the low position is made. After movement of the complement of Num\is{number}, the N will be sitting in a Spec position, so that linearization problems do not arise if the nominal phrase moves further.

Importantly, complement stranding in English is subject to certain restrictions. It is not allowed in specific DPs or in complements of verbs which {force} a concrete reading on their complement. Thus, complement stranding is not possible in \REF{ex:julien:20}, where the moved nominal phrase is definite, nor in \REF{ex:julien:21}, where the moved nominal phrase gets a concrete reading since it is underlyingly the object of \textit{destroy}:

\ea%20
    \label{ex:julien:20}
\ea[]{This book about String Theory has finally come out.}
\ex[*]{This book has finally come out about String Theory.}
\z
\z

\ea%21
    \label{ex:julien:21}
\ea[]{A book about String Theory has been destroyed.}
\ex[*]{ A book has been destroyed about String Theory.}
\z
\z

\citet{Sheehan2009} argues that complement stranding is not possible in nominals that are strong \isi{islands}. Thus, the possibility of stranding goes hand in hand with the possibility of \isi{extraction}. Inspired by \citet{Uriagereka1999}
% \todo{This citation was not included in the original doc, we included it based on other citations in this volume.} 
Sheehan takes islandhood to mean that the phrase in {question} goes to \isi{Spell-Out} as soon as it is formed, and she suggests that a D head triggers \isi{Spell-Out} – presumably because D is a \isi{phase} head. It follows that when a DP is formed, it will go to \isi{Spell-Out} as a unit, and there can be no subsequent \isi{extraction} and no split phonological realisation. But if the nominal phrase is not a DP, it will go to \isi{Spell-Out} together with its containing \isi{phase}, and as a consequence, complement stranding will be triggered.


Following this line of reasoning, Sheehan further suggests that in cases where complement stranding appears to be optional in English, we are actually dealing with two different structures: nominals that strand their complements are NPs, while nominals that do not are DPs.


\section{The Final-Over-Final Condition}

The Final-Over-Final Condition is a constraint on syntactic ordering originally proposed in \citet{Holmberg2000deriving} and formulated as follows in \citet[171]{BiberauerEtAl2014syntactic}:

\ea%22
    \label{ex:julien:22}
	  The Final-Over-Final Condition (FOFC\is{Final-over-Final Condition})\\
	  A head-final phrase αP cannot dominate a head-initial phrase βP, where α and β are heads in the same extended \isi{projection}.
\z

In the paper just mentioned, and also in \citet{BiberauerEtAl2008}, a wealth of data from many languages is presented as evidence that the generalisation holds. 

Concerning the nominal domain, Biberauer, Holmberg \& Roberts take \isi{adpositions}\is{adposition} to belong to the same extended \isi{projection} as their nominal complements. The pattern seen in the \ili{Finnish} examples in \REF{ex:julien:23} (from \citealt[187]{BiberauerEtAl2014syntactic}) can then be explained with reference to the FOFC\is{Final-over-Final Condition}. The \ili{Finnish} adposition \textit{yli} ‘via over, across’ can be a \isi{preposition} or a \isi{postposition}. When it is a \isi{preposition}, it can take a nominal complement which contains a PP following the noun, as in \REF{ex:julien:23a}. But when it is a \isi{postposition}, as in \REF{ex:julien:23b}, it cannot.

\ea%23
    \label{ex:julien:23}
    \ea[]{\label{ex:julien:23a}
\gll    \textbf{yli}      [rajan    [maitten    välillä]]\\
    across   border  countries  between\\
\glt      ‘across the border between the countries’}
\ex[*]{\label{ex:julien:23b}
\gll  [ rajan    [maitten    välillä]]  \textbf{yli}\\
     {} border  countries  between  across\\}
\z
\z

Biberauer, Holmberg \& Robert then observe that \ili{Finnish} has an alternative way to express the contents of the nominal phrase in (\ref{ex:julien:23}ab),
% \todo{Did you mean to cite both examples here?}
with an adjectival expression instead of a complement PP. The adjectival expression is prenominal, and consequently, a nominal phrase containing this expression can be embedded under a \isi{preposition}, as in \REF{ex:julien:24a}, or under a \isi{postposition}, as in \REF{ex:julien:24b}.

\ea%24
    \label{ex:julien:24}
  \ea\label{ex:julien:24a}
\gll    \textbf{yli}      [[ maitten  väli-se-n        ]  rajan]\\
      across   {} countries  between-\textsc{adj-gen} {}  border\\
\glt       ‘across the border between the countries’

\ex\label{ex:julien:24b}
\gll\relax  [[ maitten  väli-se-n        ]  rajan  ]  \textbf{yli}\\
       {} countries  between-\textsc{adj-gen} {}  border {} across\\
\glt        ‘across the border between the countries’
\z
\z

The authors do not comment on the internal structure of the nominal phrases seen in these examples – they just state that the nominal phrase is head-initial in \REF{ex:julien:23} but head-final in \REF{ex:julien:24}. Thus, they take the noun to be the head of the nominal phrase, which means that the adjectival phrase in \REF{ex:julien:24} must be contained in the \isi{projection} of the noun.


The FOFC\is{Final-over-Final Condition} is a descriptive generalisation and does not in itself say anything about the underlying mechanism. \citet{Sheehan2009} points out that if one assumes, with \citet{Kayne1994}, that the LCA holds, and also that head-final orders result from roll-up movement, then only one additional restriction is needed to make sure that all resulting orders comply with the Final-Over-Final Condition – namely, that roll-up movement must begin at the bottom of the tree. It follows that the orders Aux\is{Auxiliary} – Verb – Object and Object – Verb – Auxiliary can be derived, for example, but not the order Verb – Object – Auxiliary. On this point \citet{Sheehan2009} is fully in agreement with \citet{BiberauerEtAl2014syntactic}. The potential advantage of Sheehan’s approach is that it also offers an account of stranded complements. On her analysis, if a head-initial phrase like [Verb Object] is moved to the Spec of an auxiliary\is{Auxiliary}, then the lower copy of the object will be spelled out, so that the result is Verb – Aux\is{Auxiliary} – Object instead of Verb – Object – Auxiliary. In other words, on this approach derivations that violate the FOFC\is{Final-over-Final Condition} are not ungrammatical – they just do not lead to head-initial phrases being spelled out in front of their selecting heads.


\section{North Sámi again} %6. /

The account of complement stranding and of the FOFC\is{Final-over-Final Condition} presented in \citet{Sheehan2009} appear at first glance to be relevant also for the ordering pattern seen in \ili{North Sámi} postpositional phrases, where, if a nominal phrase containing a complement of the noun moves to the left of a selecting adposition, then the complement will be spelled out in the lower position. However, the complement stranding that can be observed in \ili{North Sámi} postpositional phrases differs in several respects from the cases of stranding discussed by Sheehan. Firstly, in the \ili{North Sámi} case there is no optionality. A noun that moves from the complement position of P to a position immediately preceding P in the linear order – to Spec,pP on my analysis – obligatorily leaves its complement behind. This holds also for nominal phrases that get a specific or definite reading, as seen in examples \REF{ex:julien:25} and \REF{ex:julien:26}.

\ea%25
    \label{ex:julien:25}
   
  \gll	  Modealla  lea    hukse-juvvon    \ulp{dan}{~~~~~~~~~}      \uline{ipmárdusa}        \textbf{nala} \ulp{ahte}{~~~~~}  \uline{lea}    \ulp{dásseárvu}{~~~~~~}    \ulp{guovtti}{~~~~~}  \ulp{álbmoga}{~~~~~}    \uline{gaskka}.\\
    model.\textsc{nom}  is    build-\textsc{pass.ptcp}  \textsc{dem}.\textsc{gen}  understanding.\textsc{gen}  upon    that  is    equality.\textsc{nom}  two.\textsc{gen}  people.\textsc{gen}  between\\
\glt  ‘The model is built upon the understanding that there is equality between two peoples.’
\z

\ea%26
    \label{ex:julien:26}
   
    \gll   Filbma    lea    \ulp{sin}{~~~~~~}    \ulp{agálaš}{~~~~~~}  \uline{rahčama}      \textbf{birra}    \ulp{doalahit}{~~~~~~~~~}      \uline{guohtun-eatnamiid}. \\ 
	  film.\textsc{nom}  is    their  eternal  struggle.\textsc{gen}  about    keep.\textsc{inf}    grazing-land.\textsc{pl}.\textsc{acc}\\
    \glt    ‘The film is about their ever-lasting struggle to keep their grazing lands.’
    \z

\ili{North Sámi} does not have obligatory articles,\footnote{Nowadays, due to influence from \ili{Scandinavian}, demonstratives are often used as definite articles while the numeral \textit{okta} ‘one’ or the indefinite pronoun \textit{muhtun} ‘some’ appear as indefinite articles.} and the categorical status of nominal phrases in this language is not entirely clear. It \textit{is} clear, though, that the \isi{demonstrative} \textit{dan} in \REF{ex:julien:25} and the \isi{possessor} \textit{sin} in \REF{ex:julien:26} give their containing nominal phrases a definite reading. In addition, if we assume that adnominal adjectives are located in designated Spec positions above nP/NP, as proposed by \citet{Cinque1994,Cinque2010adjectives}, and that other adnominal modifiers are also related to functional heads, it follows that both nominal phrases have functional structure above the nP/NP level. The presence of a \isi{demonstrative} in \REF{ex:julien:25} and of a \isi{possessor} in \REF{ex:julien:26} might be taken to indicate that both phrases are actually DPs. In any case, both phrases are of the type that would not allow complement stranding in English.


Secondly, while complement stranding is obligatory in postpositional phrases, in other cases a phrase that moves from a complement position to a specifier position can take its complement along. An example is seen in the \isi{passive} construction in \REF{ex:julien:27}, where the nominal phrase headed by \textit{gažaldagat} ‘questions’ has moved from object position to the surface subject position. Notably, the complex postpositional phrase which is the complement of \textit{gažaldagat} is carried along – without this causing any linearization problems. Also note that there are no modifiers in front of the highest nominal here, and the phrase as a whole gets an indefinite reading, so that it appears to be of the type that would strand its complement in English.


\ea%27
    \label{ex:julien:27}
   
    \gll    \textbf{Gažaldagat}      \ulp{doarjagiid}{~~~~~~~~~~~~~~~~~~~~~~~~}            \uline{hárrái}    \ulp{boazodoallo-šiehtadusa}{~~~~~~~~~~~~~~~~~~~~~~~}      \uline{olis}      fertejit    čielggad-uvvot  dábálaš   šiehtadallamiid    bokte. \\ 
  question.\textsc{pl}.\textsc{nom}    support.scheme.\textsc{pl.gen}    concerning  reindeer.husbandry-agreement.\textsc{gen}  based.on    must.\textsc{\oldstylenums{3}pl} clarify-\textsc{pass.inf} regular	 negotiation.\textsc{pl.gen}  by.means.of\\
    \glt ‘Questions concerning the reindeer husbandry agreement funding scheme must be settled by means of regular negotiations.’
    \z

In \REF{ex:julien:28}, a similar nominal phrase, headed by a noun without prenominal modifiers and allowing an indefinite reading, has left its PP complement behind when moving to a higher position. The phrase is the sole argument of an unaccusative verb, and it might be taken to originate in complement position.

\ea%28
    \label{ex:julien:28}
   
    \gll  Dađistaga  leat  \textbf{dutkan-gáldut}        lassán-an    \uline{sápmelaččaid  birra}.   \\ 
	 gradually  are  research-source.\textsc{pl}.\textsc{nom}  expand-\textsc{ptcp}  Sámi.\textsc{pl.gen}  about\\
    \glt   ‘Gradually, research sources about the Sámi have expanded.’
    \z

Now consider \REF{ex:julien:29}, which is another \isi{passive} construction. This time the \isi{passive} subject has a \isi{demonstrative} in initial position, which could be taken to mean that it is a DP, or at least that it has functional structure above nP/NP. Again, it carries its PP complement along to the surface subject position. 

\ea%29
    \label{ex:julien:29}
   
    \gll   Dát  iešguđetge  \textbf{doaimmat}      \ulp{sáme-giela}{~~~~~~~~~~~~~~~~}        \uline{várás}    leat     hábme-juvvon    iešguđetge  sektor-surggiin. \\ 
	these  separate    activity.\textsc{pl.nom}  sámi-language.\textsc{gen}  for      are    form-\textsc{pass.ptcp}  separate    sector-branch.\textsc{pl.loc}\\
    \glt  ‘These different activities for the Sámi language are designed in the different sector branches.’
    \z

It seems clear that complement stranding is not restricted in the same way in \ili{North Sámi} as in English. Moreover, in cases where a nominal phrase moves to the front of a \isi{postposition}, and the complement of the noun is a PP, there is in fact roll-up movement in the lower part of the structure. This was seen in example \REF{ex:julien:3}, which I repeat here as \REF{ex:julien:30a}. The structure is shown schematically in \REF{ex:julien:30b}.


\ea%30
    \label{ex:julien:30}
    \ea\label{ex:julien:30a}
    \gll    \ulp{ođđa}{~~~~~~}  \ule{lága}      \textbf{birra}    \ulp{mánáidgárddiid}{~~~~~~~~~~}      \ule{várás}  \\ 
	     new  law.\textsc{gen}   about    kindergarten.\textsc{pl}.\textsc{gen}    for\\
    \glt      ‘about a/the new law for kindergartens’
    
\ex\relax\label{ex:julien:30b} 
[\textsubscript{pP2} 
  [\textsubscript{XP} 
    A 
    N 
    \sout{pP\textsubscript{1}}
  ] 
  p\textsubscript{2} 
  [\textsubscript{PP2} 
    P\textsubscript{2} 
    [\textsubscript{XP} 
      \sout{A N} 
      [\textsubscript{pP1} 
	XP 
	p\textsubscript{1} 
	[\textsubscript{PP1} 
	    P\textsubscript{1} 
	    \sout{XP}
	]
      ]
    ]
  ]
]
\z
\z 

Since the complement of the lower P has moved to the lower Spec,pP, there should be no linearization problems when the XP containing the lower PP moves to the Spec of the higher p – everything inside XP will precede the lower P, and consequently, all these elements should also precede the higher P after movement. In spite of this, the complement of the noun is left behind when the nominal phrase moves in front of the higher P. Hence, linearization does not appear to be the issue here.


If we go on to consider \ili{North Sámi} postpositional phrases in light of the FOFC\is{Final-over-Final Condition}, the first point to be noted is that \ili{North Sámi} nominal phrases are in fact head-initial. This is seen in the subject nominal phrase in \REF{ex:julien:31}, where the order is \isi{demonstrative} – numeral – adjective – noun. In other words, elements that are located higher up in the syntactic structure consistently precede elements that are located lower down. 


\ea%31
    \label{ex:julien:31}
   
    \gll    \ulp{Dat}{~~~~~~~~~~}      \ulp{guokte}{~~~~~~}    \ulp{maŋemus}{~~~~~~}  \uline{iskosa}  čájehedje     ahte  das    eai      lean          bakteriijat.\\ 
	\textsc{dem.nom}  two.\textsc{nom}  latest      test.\textsc{gen}  show.\textsc{past.\oldstylenums{3}pl}    that  it.\textsc{loc}  \textsc{neg.\oldstylenums{3}pl}  be.\textsc{past.conneg} bacterium.\textsc{pl.nom}\\
    \glt  ‘Those two latest tests showed that there were no bacteria in it.’
    \z


Consistently head-final nominal phrases have the opposite order, noun – adjective – numeral – \isi{demonstrative}, as in the West Greenlandic nominal phrase in \REF{ex:julien:32} (from \citealt[118]{Fortescue1984}):

\ea%32
    \label{ex:julien:32}
   
    \gll    qimmit  qaqurtut    marluk  taakku  \\ 
	  dog.\textsc{pl}  white.\textsc{pl}    two    those\\
    \glt   ‘those two white dogs’
    \z

This means that PPs like those in \REF{ex:julien:25} and \REF{ex:julien:26} clearly violate the FOFC\is{Final-over-Final Condition} – if the p-P complex belongs to the nominal extended \isi{projection}. In \REF{ex:julien:25}, the order is Dem\is{demonstrative} – N – P, while in \REF{ex:julien:26}, it is \isi{Poss}\is{possessive} – A – N – P. In both cases, a head-initial phrase precedes the P.


Confronted with these constructions, the FOFC\is{Final-over-Final Condition} can be saved only if we assume that the \isi{postposition} and the noun define separate extended projections. And in fact, this assumption is not entirely unreasonable, at least not for \ili{North Sámi}. Many \ili{North Sámi} postpositions have developed from nouns, and in many cases the nominal source is still easily recognised. For example, the string \textit{joavkkuid gaskkas} can be parsed either as a \isi{possessor} followed by a possessee in the locative case, as in \REF{ex:julien:33a}, or as a \isi{postposition} preceded by its complement, as in \REF{ex:julien:33b}.


\ea%33
    \label{ex:julien:33}
    \ea \label{ex:julien:33a}
\gll joavkkuid    gaskka-s\\
         group.\textsc{pl.gen}  gap-\textsc{loc}   \\ 
    \glt   ‘in the groups’ gap’

\ex \label{ex:julien:33b}
\gll joavkkuid      gaskkas\\
      group.\textsc{pl.gen}    between\\
\glt       ‘between the groups’
\z
\z\largerpage[2]

In \REF{ex:julien:33a} the nominal \isi{possessor}, while contained in the functional domain of the possessee, necessarily also has its own functional domain. The example in \REF{ex:julien:33b} could be taken to have a similar structure, the main difference being that the head of the larger functional domain here is of the category P. 


If the line of reasoning that I have presented here is correct, neither the complement stranding approach presented in \citet{Sheehan2009} nor the FOFC\is{Final-over-Final Condition} of \citet{Holmberg2000deriving} and \citet{BiberauerEtAl2008,BiberauerEtAl2014syntactic} can explain the ordering seen in \ili{North Sámi} postpositional phrases. In \ili{North Sámi}, there appears to be a restriction that applies specifically to postpositions, dictating that when the constituent immediately preceding the \isi{postposition} is a nominal phrase, it must have the head noun as its final element. The restriction is mainly phonological in nature, since it forces the complement of the noun, if there is one, to be spelled out in the lower position.



An observation can now be added which concerns pronominal phrases. If the complement of a \isi{postposition} is a pronominal phrase, then the pronoun need not be in final position within that phrase. An example is given in \REF{ex:julien:34}:


\ea%34
    \label{ex:julien:34}
   
    \gll   Mun  jurddašan      nu    \ulp{din}{~~~~~~~~~~~~~~~}      \uline{(buohkaid)}    \textbf{birra}. \\ 
	   I    think.\textsc{pres.\oldstylenums{1}sg}  so    you.\textsc{pl.gen}   all.\textsc{pl.gen}    about\\
    \glt     ‘I think so much about you (all).’
    \z

If D is where \isi{person} features are located, as \citet{Longobardi2008} proposes, then \textit{din} ‘you\textsc{’} is in D, while the quantifier \textit{buohkaid} ‘all’ is a position below D. Crucially, there is no requirement that \textit{din} should appear immediately in front of the \isi{postposition}, so that the quantifier can freely be added.\footnote{The pronominal part \textit{din} is also optional, from a formal point of view, but \textit{buohkaid} ‘all’ would get a third \isi{person} interpretation if \textit{din} is left out.}

\section{A note on relative clauses} %7. /

A pattern similar to the complement stranding seen with \ili{North Sámi} postpositions is also obligatory in cases where a noun combines with a \isi{relative} clause. Relative clauses in \ili{North Sámi} follow their correlates, as illustrated in \REF{ex:julien:35}, where the \isi{relative} clause \textit{maid áigguiga rasttildit} \textsc{‘}which the two of them wanted to cross’ follows the noun \textit{joga}\textit{š} ‘small river’:

\ea%35
    \label{ex:julien:35}
   
    \gll     \textbf{joga}\textbf{š}      \ulp{maid}{~~~~~~~~~~~~}      \ulp{áigguiga}{~~~~~~~~~~~~~}        \ule{rasttildit} \\ 
	 river.\textsc{dim.nom}    which.\textsc{acc}  want.\textsc{past.\oldstylenums{3}du} cross.\textsc{inf}\\
    \glt  ‘a small river which the two of them wanted to cross’
    \z

	
If the nominal phrase in \REF{ex:julien:35} is to be the complement of a \isi{postposition}, the result is as shown in \REF{ex:julien:36} – the nominal correlate of the \isi{relative} clause precedes the \isi{postposition} while the \isi{relative} clause follows it:

\ea%36
    \label{ex:julien:36}
   
    \gll    De    olliiga         \uline{jogaža}      \textbf{lusa}  \ulp{maid}{~~~~~~~~~~~}      \ulp{áigguiga}{~~~~~~~~~~~}    \uline{rasttildit}. \\ 
then  arrive.\textsc{past.\oldstylenums{3}du}  river.\textsc{dim.gen}  to    which.\textsc{acc}  want.\textsc {past.\oldstylenums{3}du} cross\textsc{.inf}\\
    \glt ‘Then the two of them reached a small river which they wanted to cross.’
    \z


The construction in \REF{ex:julien:36}, with the \isi{relative} clause separated from its correlate by an intervening \isi{postposition}, is reminiscent of the cases of \isi{relative} clause stranding discussed in \citet{Kayne1994}. Kayne assumes a raising analysis of \isi{relative} clauses, which means that the correlate originates inside the \isi{relative} clause and moves to the highest Spec of that clause. It is then to be expected that the correlate should in principle be able to move even higher, leaving the \isi{relative} clause behind.

The raising analysis of \isi{relative} clauses has attracted much attention. It has however also been challenged, among others by \citet{Platzack2000} and \citet{Schmitt2000}, who both argued that \isi{relative} clauses are CP\is{complementizer} complements of N. A well-known objection against the raising analysis has to do with \isi{morphological} case, which presents a problem that can also be seen in the \ili{North Sámi} example in \REF{ex:julien:37}:


\ea%37
    \label{ex:julien:37}
    \gll    Mun  in        diehtán            maidege   \ulp{dan}{~~~~~~~~~~~}      \ule{dili}       \textbf{birra}    \ulp{mas}{~~~~~~~~~~~~~}      \ulp{son}{~~~~~~~~~~~}      \uline{elii}.\\ 
	  I    \textsc{neg.\oldstylenums{1}sg}    know.\textsc{past.conneg}    anything.\textsc{acc} \textsc{dem.gen}  situation.\textsc{gen}  about    which.\textsc{loc}  s/he.\textsc{nom}  live.\textsc{past.\oldstylenums{3}sg}\\
    \glt   ‘I did not know anything about the situation that s/he lived in.’
    \z


Here the correlate \textit{dili} ‘situation’ has genitive case, as a consequence of being the complement of the \isi{postposition}, while the \isi{relative} pronoun \textit{mas}, which introduces the \isi{relative} clause, has locative case, in accordance with the syntactic function of the relativised element. This situation poses a problem for the raising analysis of \isi{relative} clauses, which takes the \isi{relative} pronoun and the correlate to originate as one constituent. It is possible to get around the problem – see e.g. \citet{Bianchi2000} – but I would like instead to consider here the consequences of assuming that the \isi{relative} clause is the complement of the correlate.

On a complement analysis of \isi{relative} clauses, examples like \REF{ex:julien:38} indicate that in \ili{North Sámi}, complement stranding applies to \isi{relative} clauses in the same way as it applies to other complements of nouns that move to pre-postpositional position, since on this analysis there is no constituent that contains the noun and the prenominal modifiers – a \isi{demonstrative} and an adjective – but excludes the \isi{relative} clause. 


\ea%38
    \label{ex:julien:38}
    \gll  Mun  illudin          \ulp{dan}{~~~~~~~~~~~}    \ulp{čáppa}{~~~~~~~~~~~}      \ule{skeaŋkka}  \textbf{ovddas}   \ule{maid}      \ulp{Vilges}{~~~~~~~~~~~}    \ulp{ledjen}{~~~~~~~~~~~}      \uline{ožžon}.\\ 
    I    be.happy.\textsc{past.\oldstylenums{1}sg}  that.\textsc{gen}  beautiful  gift.\textsc{gen}    for   which.\textsc{acc} Vilge.\textsc{loc}  be.\textsc{past.\oldstylenums{1}sg}  get.\textsc{ptcp}	\\
    \glt   ‘I was happy for the beautiful present that I had got from Vilge.’
    \z

Once \isi{relative} clauses are included in the discussion, it is also possible to draw a parallel between postpositional phrases on the one hand and \isi{possessor} constructions on the other. As we saw already in \REF{ex:julien:33a}, possessors in \ili{North Sámi} precede the possessee, and they are marked with genitive case. But if the \isi{possessor} contains a \isi{relative} clause, the \isi{relative} clause follows the possessee. This is shown in \REF{ex:julien:39}, where the \isi{possessor} phrase is headed by a pronoun, and in \REF{ex:julien:40}, where the \isi{possessor} phrase is headed by a noun.


\ea%39
    \label{ex:julien:39}
   
    \gll    Čájeha        maid  \uline{sin}      \textbf{nummáriid}    \ulp{geat}{~~~~~~~~~~~~~~~}        \ulp{leat}{~~~~~~~~~~~~~~}   \uline{geahččalan}    \ulp{dutnje}{~~~~~~~~~~~}  \uline{riŋget}.\\ 
	 show.\textsc{pres.\oldstylenums{3}sg} also  \textsc{\oldstylenums{3}pl.gen}  number.\textsc{pl.acc}  who.\textsc{pl.nom}  be.\textsc{pres.\oldstylenums{3}pl}   try.\textsc{ptcp}      you.\textsc{ill}  call.\textsc{inf}\\
    \glt   ‘It also shows the numbers of those who have tried to call you.’
    \z

\ea%40
    \label{ex:julien:40}
   
    \gll    Mun  galggan        olbmástalla-goahtit      \ulp{dan}{~~~~~~~~~~~}      \ule{máná}   \textbf{váhnemiin}    \ulp{gii}{~~~~~~~~~~~~~~~}        \ulp{munno}{~~~~~~~~~~~}    \ulp{máná}{~~~~~~~~~~~}      \ulp{lea}{~~~~~~~~~~~}    \uline{givssidan}.\\ 
	 I    shall.\textsc{pres.\oldstylenums{1}sg}    make.friends-begin.\textsc{inf}    \textsc{dem}.\textsc{gen}  child.\textsc{gen} parent.\textsc{com}    who.\textsc{nom}  \textsc{\oldstylenums{1}.du.gen}  child.\textsc{acc}  is    bully.\textsc{ptcp}\\
    \glt  ‘I will begin to make friends with the parent of the child who has bullied our child.’
    \z

In other words, the possessee intervenes between the nominal head of the \isi{possessor} phrase and the complement of that head in the same way as postpositions intervene between the head of its complement and the complement of that head. Two conclusions can be drawn from this fact. First, possessors in \ili{North Sámi} originate in a position which is lower than the surface position of the possessed noun. Second, the same mechanism and the same restriction might be at work in possessed nominal phrases as well as in postpositional phrases. In both cases, the nominal head of a complement nominal phrase must immediately precede the complement-taking head. 


\section{Conclusion} %8. /

We have seen that \ili{North Sámi} postpositions are not strictly postpositional after all. If the nominal complement of a \isi{postposition} has its own complement, then the complement of the noun will follow the \isi{postposition} in the linear order, although the noun itself and all elements that precede it within the nominal phrase precede the \isi{postposition}. In other words, in these cases the complement of the \isi{postposition} gets a discontinuous realisation. 


The observed pattern is not explained by the Final-Over-Final Condition \citep{Holmberg2000deriving}, since the preposed complement of the \isi{postposition} is head initial. If the \isi{postposition} is not a part of the extended \isi{projection} of the noun, then the Final-Over-Final Condition does not apply at all. The Complement Stranding approach of \citet{Sheehan2009} appears to be more relevant, but it turns out that the restrictions that regulate Complement Stranding in English do not carry over to \ili{North Sámi}. 



My conclusion concerning the \ili{North Sámi} pattern is as follows. \ili{North Sámi} postpositions take their complement to the right underlyingly. The complement of P is attracted to the Spec of p, a functional head above P, but if the nominal head of that complement has its own complement, the latter will be spelled out in the lower position. However, this does not happen as a consequence of the linearization problems that \citet{Sheehan2009} discusses. It is instead due to a more specific requirement that the \isi{postposition} must follow immediately after the nominal head of its complement. A similar effect is seen with possessors, which are spelled out in prenominal position but leave their complement behind in postnominal position. As it stands, however, the requirement that I propose here only refers to the surface order. The deeper nature of the requirement will have to be investigated further.


% \section*{Abbreviations}
\section*{Acknowledgements}
The research for this paper was financially supported by Bank of Sweden Tercentenary Foundation, grant no. P12-0188:1. 


{\sloppy\printbibliography[heading=subbibliography,notkeyword=this]}
\end{document}