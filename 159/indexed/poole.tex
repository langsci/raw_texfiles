\documentclass[output=paper]{LSP/langsci} 
\author{Geoffrey Poole  	\affiliation{Newcastle University}
}
% \epigram{Change epigram}
\abstract{On the basis of an extensive overview of verb-second languages and data, \citet[376]{Holmberg2015verbsecond} arrives at the following general characterization of the V2 property: (a) a functional head in the left periphery attracts the finite verb and (b) this functional head requires that a constituent move to its specifier position.  In this paper I argue that this view of the V2 property, together with \citegen{Salvi2012} observations concerning the syntactic positions which precede the finite verb in medieval Romance, suggest that Old Spanish was indeed a verb-second language.  More specifically, I argue that the existence and nature of the features which effect (a) and (b) in Old Spanish find a natural motivation/explanation within \citegen{BiberauerRoberts2010} feature-inheritance approach to typology, in which languages differ with respect to whether EPP- and Tense-features are retained by C, donated to T or shared between the two (cf. \citealt{Ouali2008}).  While I assume, following \citet{BiberauerRoberts2010}, that EPP- and T-features are donated to T in Modern Romance (including Modern Spanish), I suggest that these features were retained by C in Old Spanish.} 

\title{Feature inheritance in Old Spanish: (re)visiting V2}
\ChapterDOI{10.5281/zenodo.1117714}
\maketitle

\begin{document}
 
% [Warning: Draw object ignored]
 

\section{Introduction}\label{sec:poole:1}
As noted by \citet{Salvi2012}, there is a long-standing observation regarding medieval \ili{Romance} to the effect that there are two syntactic positions which precede the finite\is{finiteness} verb:  one which immediately precedes the finite\is{finiteness} verb, which \is{Salvi's Generalization}Salvi calls P2, and one which precedes \textit{that} position, which Salvi calls P1.  With respect to information structure, the P1 position hosts ``thematic material'', while P2 can host either thematic or focal material.  This traditional observation would seem to be expressible naturally within a standard version of an articulated left-periphery (in the sense of \citealt{Rizzi1997} and much subsequent work).

\ea%1
    \label{ex:poole:1}
	  [\textsubscript{ForceP} [$\underbrace{\textrm{\textsubscript{TopicP}}}_{\textrm{P1}}$ [$\underbrace{\textrm{\textsubscript{FocusP} [\textsubscript{TopicP}}}_{\textrm{P2}}$ [\textsubscript{FinP}   [\textsubscript{TP}  ]]]]]
    \z

\noindent The P1 position would seem to straightforwardly map on to the high \isi{Topic} position, while the lower \isi{Focus} and \isi{Topic} projections are a natural locus for \is{Salvi's Generalization}Salvi’s P2 position.\footnote{Throughout the paper, I will use terms such as ``C-related'', ``the low left periphery'', ``the C-domain'', ``V-to-C movement'', etc. as ways of referring to \is{Salvi's Generalization}Salvi’s P2 position in the low left periphery.  I take no position on what more specific \isi{projection} within an articulated CP\is{complementizer} might be relevant as I believe the choice does not materially affect the proposed analysis.  See, for example, \citet{Walkden2015} for some possibilities in Germanic.  Strictly speaking, as discussed in \sectref{sec:poole:4.2}, it is possible that Spanish does not even possess FocusP as a syntactic \isi{projection} during the medieval period because, as I will suggest, the relevant syntactic features have yet to develop.}  

In \sectref{sec:poole:2}, I argue that \is{Salvi's Generalization}Salvi’s generalization does descriptively characterize Spanish during the pre-Golden Age period (i.e., prior to the 16\textsuperscript{th} century), taking P2 to refer to the cluster of low left peripheral positions as in \REF{ex:poole:1}.  These positions could contain only \textit{one} XP, but it could be either topical or focal.  Furthermore, the finite\is{finiteness} verb must be right-adjacent to the left-peripheral element.  However, I then ``extend'' \is{Salvi's Generalization}Salvi’s generalization, primarily by considering wide-focus \isi{fronting} in \ili{Old Spanish} \citep{Mackenzie2010}.  This phenomenon is of particular interest in this context because it indicates that the lone XP occupying P2, in addition to being topical or focal, could be \textit{neither}.  In other words, this construction (and others) appear to show that \is{Salvi's Generalization}Salvi’s P2 position in the low left periphery can, in some cases, be occupied by an element which cannot be interpreted \textit{either} as topical \textit{or} as focal.

In \sectref{sec:poole:3}, I suggest that \is{Salvi's Generalization}Salvi’s extended generalization is naturally captured under the assumption that elements are attracted to the low left periphery during this period by a purely formal \isi{EPP} feature present in the low left periphery (rather than via syntactic features encoding specific discourse interpretations such as [+focus]).  In addition, the requirement that the verb be immediately adjacent to the fronted element suggests that Tense\is{tense} features were also retained in this low area of the C-domain.  Diachronically, it appears that \textit{certain} left-peripheral displacements, including wide focus \isi{fronting} and interpolation (see below), decline to extinction in parallel with verb-raising\is{verb raising} to a high position during the Golden Age period, suggesting a close connection between the EPP- and T-features.  

\sectref{sec:poole:4} observes that \is{Salvi's Generalization}Salvi’s descriptive generalization would be naturally accounted for under the assumption that the \isi{EPP} and Tense\is{tense} features are \textit{retained} by a C-related \isi{projection} in \ili{Old Spanish}, but are \textit{donated} to T in Modern Spanish (thus aligning Spanish with \citegen{BiberauerRoberts2010} feature-inheritance account of modern \ili{Romance}).  This in turn implies, given standard accounts of the left-peripheral displacement of topics and foci, that one of the major changes undergone by Spanish during the Golden Age was a ``syntacticization of discourse'' (in the sense of \citealt{HaegemanHill2013}).   In other words, displacement to the low-left periphery came to be driven, not by a purely formal \isi{EPP} feature as previously, but rather by syntactic features with specific information-structure value (e.g., [+focus]).

\section{P2 and the low left periphery of Old Spanish}\label{sec:poole:2}

As noted by various authors (e.g., \citealt{Sitaridou2011,Poole2013}), \ili{Old Spanish} possessed various constructions in which an element displaced to the low left periphery was interpreted as topical or focal.  The phenomenon of interpolation is of particular relevance, as it appears that the element displaced to the low left periphery could indeed be either topical or focal (\textit{contra} \citegen{Poole2013} analysis).  As such then, \is{Salvi's Generalization}Salvi’s Generalization with respect to the P2 position does seem to correctly describe \ili{Old Spanish}.

\subsection{New information focus}\label{sec:poole:2.2}
In her study of information structure in the \textit{General Estoria} of Alfonso the Wise (13\textsuperscript{th} century), \citet{Sitaridou2011} notes that complements dislocated to a pre-verbal position can bear a number of information structure roles, including new information focus.  \REF{ex:poole:2} for example instantiates an operation she dubs “\isi{Focus} \isi{Fronting}\is{fronting}” (174) (see also \citealt{Cruschina2008}; \citealt{CruschinaSitaridou2009}):

\ea%2
    \label{ex:poole:2}
    (\textit{General Estoria} 4, 13\textsuperscript{th} c., \citealt[(25)]{Sitaridou2011})\\
    \gll  \& los qui se  gozaron con el to derribamiento \textbf{penados} seran por ello.  \\
	  and the who \textsc{refl.\oldstylenums{3}pl} enjoy.\textsc{fut}.\oldstylenums{3}\textsc{pl} with the your fall punished be.\textsc{fut}.\oldstylenums{3}\textsc{pl} for this\\
    \glt  ‘And those who rejoice with your fall they will be punished for that.’
    \z

\noindent Unlike Modern Spanish, left-peripheral focus is not obligatorily contrastive.  According to Sitaridou, the fronted \isi{participle} in \REF{ex:poole:2} simply encodes new information focus.  

\subsection{Topics}\label{sec:poole:2.3}
A clear demonstration that the P2 position could be occupied by topics is necessarily made more difficult by the fact that topics may also occupy \is{Salvi's Generalization}Salvi’s P1 position.  However, cases containing a fronted object such as \REF{ex:poole:3}, which, as \citet[170]{Sitaridou2011} notes, do appear to have a topic interpretation, would seem to be plausible candidates.

\ea\label{ex:poole:3}
(\textit{General Estoria} 1, 13\textsuperscript{th} c., \citealt[(16)]{Sitaridou2011})\\
\gll  e fue / natural duna cibdat q{\textless}ue{\textgreater} dixieron fenis […] \\
      and was.\oldstylenums{3}\textsc{sg} / native of.one city which called.\oldstylenums{3}\textsc{pl} Fenis\\
\gll Y esta cibdat poblo fenis fijo dagenor…\\
and this city inhabited.\oldstylenums{3}\textsc{sg} Fenis son of.Agenor\\
\glt ‘And he was from a city which was called Fenis… and this city was inhabited by Fenis, son of Agenor.’
\z
\textit{Esta cibdat} ‘this city’ in \REF{ex:poole:3} resumes the previously mentioned city Fenis.

Cases described as ``resumptive preposing'' by \citet{Mackenzie2010} are also plausibly instances of topics occupying \is{Salvi's Generalization}Salvi’s P2 position.  Consider \REF{ex:poole:4}, originally discussed by \citet{Fontana1993}:\footnote{Mackenzie’s original examples, which lack morpheme-by-morpheme glosses, are reproduced verbatim in the text.  For \REF{ex:poole:4} the glosses would be:

\begin{exe}
\exi{(4)}
\gll este logar mostro dios a abraam.\\
this place showed.3s God to Abraham\\
\glt ‘God showed this place to Abraham.’
\end{exe}
} 

\ea%4
    \label{ex:poole:4}
(\textit{General Estoria} 1, 13\textsuperscript{th} c., \citealt[(14)]{Mackenzie2010})\\
        este logar mostro dios a abraam  \\\relax
    [this place God showed to Abraham]
    \z

\noindent As Mackenzie points out, the presence of \textit{este} ‘this’ suggests that the preposed object is resuming something in the discourse, and his examination of the context reveals that \textit{este} in fact resumes the phrase \textit{una cabeça mas alta que todo el otro monte} ‘a peak higher than the rest of the mountain’ which is found in the preceding sentence.  In that sense, \REF{ex:poole:4} appears similar to \REF{ex:poole:3} above.

Mackenzie observes that resumptively preposed elements are ``topical within the discourse'' (\citeyear[284]{Mackenzie2010}), though, at the same time, claims that they are not ``a topic in any syntactically relevant sense'' (385).  However, given a more articulated distinction between types of topics, phrases such as \textit{este logar} in \REF{ex:poole:4} would seem to plausibly constitute, for example, G-Topics in the sense of \citet{BianciFrascarelli2010}.  G-Topics are used to retrieve information already present within the conversational common ground content and are associated with topic continuity.  These are also the structurally lowest topics within Bianci \& Frascarelli’s hierarchy.  As such they would seem to naturally align with \is{Salvi's Generalization}Salvi’s P2 position.  Additionally, as \citet[392]{Mackenzie2010} notes, resumptively preposed elements are obligatorily adjacent to the verb, further suggesting that they do not occupy a high topic position along the lines of \is{Salvi's Generalization}Salvi’s P1 position.

\subsection{Interpolation}\label{sec:poole:2.4}
The phenomenon of interpolation (e.g., \citealt{Chenery1905,BatlloriEtAl1995,Poole2013})  is of particular interest in the context of the information structure of left-peripheral \isi{fronting} in \ili{Old Spanish} because it appears as though the interpolated element can be interpreted as either a topic or as a focus (partially \textit{contra} \citealt{Poole2013}).  

In this construction, object and indirect object pronouns can appear separated from the finite\is{finiteness} verb by a short intervening constituent, for example an adverb, a short prepositional phrase, or a subject, as illustrated in \REF{ex:poole:5}:\largerpage[2]

\ea%5
    \label{ex:poole:5}
    (\textit{Castigos e documentos de Sancho IV,} 13\textsuperscript{th} c., \citealt[(4)]{Poole2013})\\
    \gll     … \& vsa mal del buen entendimiento que \textbf{le} \textbf{dios} \textbf{dio}. \\
	     {} and uses badly of.the good understanding that him God gave.\oldstylenums{3}\textsc{sg}\\
    \glt  ‘…and he makes poor use of the good understanding that God gave him.’
\z

\noindent \citet{Poole2013} argues that interpolation targets a low \isi{Topic} position within the left periphery and that the interpolated element acts as a given or familiar topic.  Consider the context in \REF{ex:poole:6} preceding the instance of interpolation of \textit{esto} ‘that’:\footnote{I omit for reasons of space the morpheme-by-morpheme gloss in \REF{ex:poole:6} as the argument hinges on the larger discourse context, the relevant portions of which are explicated in the text.} 

\ea%6
    \label{ex:poole:6}
    (\textit{Crónica de 1344 I,} 14\textsuperscript{th} c., \citealt[(19)]{Poole2013})\\
    Et estonçe les dixo el Rey \ul{que se salliesen de su tierra} Et aquella gente a qujen \ul{esto} dixo fueron se a la villa \& tanto que \textbf{les esto dixo} luego se armaron muy bien \& venjeron se al Rey onde yazia en su alcaçar \& lidiaron conel \& lo mataron. \\
    \glt ‘And then the king said to them that they should leave his land.  And those people to whom he said that went to the town and as soon as \textbf{he said} \textbf{\ul{that}} \textbf{to them} they armed themselves well and went to the king where he rested in his fortress and fought with him and killed him.’

    \z

\noindent In \REF{ex:poole:6}, the interpolated element \textit{esto} ‘that’, resumes the recently mentioned event \textit{que se salliesen de su tierra} ‘that they should leave his land’, with said event also having been resumed by \textit{esto} in the sentence immediately preceding the one in which \textit{esto} is interpolated.  As such, interpolation in \REF{ex:poole:6} seems clearly to be an instance of topic continuity.

However, \citet[90]{Poole2013} notes cases of interpolation such as \REF{ex:poole:7}, which are not straightforwardly associated with topicality.

\ea%7
    \label{ex:poole:7}(\textit{Calila e Dimna}, 13\textsuperscript{th} c., \citealt[(13)]{Poole2007})\\
    \gll   manifiesta cosa es que lo feziste {a tuerto} et sin pecado que \textbf{te} \textbf{él} \textbf{fiziese}. \\
	  manifest thing is that it did.\oldstylenums{2}\textsc{sg} unjustly and without sin that you he did.\oldstylenums{3}\textsc{sg}\\
    \glt ‘It’s clear that you did it unjustly and without him having done you any wrong.’
    \z 

\noindent From the context, the interpolated personal pronoun subject seems contrastive and even mildly emphatic.  Other cases in which a personal pronoun are interpolated would seem to be even clearer.

\ea%8
    \label{ex:poole:8}
(\textit{El Libro de Caballero} \textit{Zifar} 14\textsuperscript{th} c., \citealt[(2b)]{Poole2007})\\  
    \gll     e dixe que \textbf{lo} \textbf{yo} \textbf{auja} muerto.\\
	 and said.\oldstylenums{1}\textsc{sg} that him I had killed\\
    \glt  ‘and I said that I had killed him.’
    \z

\noindent In \REF{ex:poole:8}, the interpolated subject pronoun is identical with the matrix clause subject, and in Modern Spanish would be obligatorily emphatic.  Insofar as \ili{Old Spanish} appears to be identical to Modern Spanish with respect to \textit{pro}-drop, one would expect that the pronoun was interpreted as focal in \ili{Old Spanish} as well.  Thus it appears to be the case that the interpolated element may be interpreted as either a topic or as a focus.

Another interpolation-specific generalization which points to the correctness of\linebreak Salvi's\is{Salvi's Generalization} Medieval \ili{Romance} characterization as it relates to \ili{Old Spanish} concerns the elements which can precede the ``interpolation cluster'' -- that is, the cluster of \isi{clitic} pronoun, interpolated element and finite\is{finiteness} verb.  In general, it is rare for anything to precede the interpolation cluster.  It most commonly follows the Complementizer\is{complementizer} or other subordinating element.  However, as \citet{Poole2013} notes, in those cases in which an element does intervene between the \isi{complementizer} and the interpolation cluster, the element is very plausibly topical. Consider \REF{ex:poole:9}, for example:

\ea%9
    \label{ex:poole:9}(\textit{General Estoria I}, 13\textsuperscript{th} c., \citealt[(35)]{Poole2013})\\
    \gll      \& que \textit{desta} \textit{manera} \textbf{se} \textbf{non} \textbf{contrallan} estas razones de Moysen \& de Josepho.\\
	and that of.that way \textsc{refl} \textsc{neg} contradict.\oldstylenums{3}\textsc{pl} those laws of Moses and of Joseph\\
    \glt‘And in that way the laws of Moses and Joseph were not violated.’
    \z

\noindent Recall that \is{Salvi's Generalization}Salvi claims that the P1 position in Medieval \ili{Romance}, the position which precedes the P2 position, is a position which hosts thematic material.  Given the presence of the \isi{demonstrative} pronoun \textit{esto} ‘that’, and the fact that it refers back to an element of the previous discourse, it seems plausible to assume that it occupies \is{Salvi's Generalization}Salvi’s P1 position.  

\subsection{XP co-occurrence restrictions in the low left periphery}\label{sec:poole:2.5}
The previous sections have illustrated various left-peripheral XP displacements in \ili{Old Spanish}, arguing that \is{Salvi's Generalization}Salvi’s traditional generalization is correct insofar as it states that the P2 position in \ili{Old Spanish} could be occupied by elements which were interpreted as topical or as focal.  In addition however, recall that \is{Salvi's Generalization}Salvi’s generalization claims further that only one XP could occupy this left peripheral position.  This predicts that there should be complementary distribution among the constructions discussed above, and this prediction appears to be correct.  

First, these constructions all require that the pre-verbal element be immediately left-adjacent to the verb, from which complementarity of distribution then follows derivatively.\footnote{See \sectref{sec:poole:3.2} below for more discussion regarding the significance of the verb-adjacency requirement.}   \citet[174]{Sitaridou2011}, for example, notes that the fronted element must be adjacent to the verb in order to be interpreted as new information focus.\footnote{Obviously verb-adjacency is not a requirement for interpretation as a topic, given the availability of the P1 position for topics in addition to the P2 position.}  \citet[392]{Mackenzie2010} observes that verb-adjacency is also required for resumptive preposing, while \citet{Poole2013}, among others, notes that the same is true for interpolation.  

Some further co-occurrence restrictions specific to interpolation also suggest that \is{Salvi's Generalization}Salvi’s generalization indeed applies to \ili{Old Spanish}.  As \citet[94--95]{Poole2013} notes, interpolation is in complementary distribution with wh-operators, but not with \isi{relative} clause operators.  It can be found in all types of \isi{relative} clauses, whether restrictive, non-restrictive or free, as in (\ref{ex:poole:10}--\ref{ex:poole:12}), but there appear to be no examples like the constructed \REF{ex:poole:13}.

\ea%10
    \label{ex:poole:10}(\textit{Siete Partidas}, 13\textsuperscript{th} c.; \citealt[(51)]{Poole2013})\\
    \gll    otra muger con \textit{quien} \textit{lo} \textit{no} \textit{pudiesse} fazer de derecho.  \\
other woman with whom it not could.\oldstylenums{3}\textsc{sg} do.\textsc{inf} of right	\\
    \glt ‘…another woman with whom he had no right to [marry].’
    \z

\ea%11
    \label{ex:poole:11}
    (\textit{Gran Conquista de Ultramar}, 13\textsuperscript{th} c., \citealt[(52)]{Poole2013})\\    
    \gll    ala reyna halabra su madre de \textit{quien} \textit{os} \textit{ya} \textit{diximos} en otros lugares  \\
	    to.the queen H his mother of whom you already said.\oldstylenums{1}\textsc{pl} in other places\\
    \glt ‘…to Queen Halabra his mother, about whom we have already spoken elsewhere,…’
    \z

\ea%12
    \label{ex:poole:12}
(\textit{El emperador Otas de Roma}, 14\textsuperscript{th} c., \citealt[(53)]{Poole2013})\\    
    \gll    \emph{quien} \emph{le} \emph{entonçe} \emph{viese} \emph{griegos} \emph{matar} / \& \emph{espedaçar} \emph{espedaçar} \emph{bien} ternja quele deujan doler los braços  \\
	     who him then saw.\oldstylenums{3}\textsc{sg} Greeks kill {} and butcher butcher well would.have.\oldstylenums{3}\textsc{sg} that.him should.\oldstylenums{3}\textsc{pl} hurt.\textsc{inf} the arms \\
    \glt ‘Whoever saw him killing Greeks and butchering them would have had to have had aching arms….’
   \z


\ea%13
    \label{ex:poole:13}(\citealt[(48)]{Poole2013})\\
    \gll * ca non sabedes quien \emph{lo} \emph{asi} \emph{fiziese}  \\
         {} because not know.\oldstylenums{2}\textsc{pl} who it thus did.\oldstylenums{3}\textsc{sg}	\\
    \glt ‘because you don’t know who did it like that’
 \z

\noindent Under the assumption, following \citet{Poole2013}, that wh-operators occupy FocusP in \ili{Old Spanish} in both main and embedded clauses, this suggests that \is{Salvi's Generalization}Salvi’s generalization is correct that only one element can occupy the P2 position.  Wh-operators, which occupy P2, are incompatible with interpolation, which also occupies P2.  Relative clause\is{relative} operators, which occupy a higher position (the specifier of ForceP), are not.

\subsection{Extending Salvi’s Generalization:  elements which are neither topical nor focal}\label{sec:poole:2.6}
The previous sections motivated \citegen{Salvi2012} generalization\is{Salvi’s Generalization} concerning elements which can precede the finite\is{finiteness} verb.  However, evidence from quantifier \isi{fronting} in \ili{Old Spanish} \citep{Mackenzie2010} shows that the generalization can be extended in an important way:  the single element which immediately precedes the finite\is{finiteness} verb can be not only \textit{either} topical or focal, but also be \textit{neither} topical \textit{nor} focal.  

\citet{Mackenzie2010} notes examples such as \REF{ex:poole:14} and \REF{ex:poole:15}, in which a fronted object quantifier appears in an immediately pre-verbal position.\footnote{Again, for the relevant portions of Mackenzie’s examples in \REF{ex:poole:14} and \REF{ex:poole:15}, the morpheme-by-morpheme glosses are as follows:

\let\eachwordone=\upshape
\begin{exe}
 \exr{ex:poole:14}
 \gll si ell omne algo deue….\\
      If the man something owes\\
 \glt ‘If a man owes something….’
\end{exe}

\let\eachwordone=\upshape
\begin{exe}
\exr{ex:poole:15} 
\gll … quel nada ualiesse de lo que el quierie.\\
     {} that.him.\textsc{dat} nothing values of it that he wanted\\
\glt ‘…[anything] that was of value to him in terms of what he wanted.’
\end{exe} 
}  

\ea%14
    \label{ex:poole:14}(\textit{General Estoria} IV, 13\textsuperscript{th} c., \citealt[(22)]{Mackenzie2010})\\
    Si \textbf{ell} \textbf{omne} \textbf{algo} \textbf{deue}; faze gelo oluidar de guisa ques tiene que mas Rico es que otros omnes.\\
    \glt[If a man owes something, it [wine] makes him forget it so that he holds himself to be richer than other men.]
    \z

\ea%15
    \label{ex:poole:15}(\textit{Estoria de España} II, 13\textsuperscript{th} c., \citealt[(23)]{Mackenzie2010})\\
    Mas pero non fizo y quel \textbf{nada} \textbf{ualiesse} de lo que el querie.   
    \glt{[But he didn’t do there [anything] that was of any value to him in terms of what he wanted.]}

    \z

\noindent As \citet{Poole2016} observes, these fronted quantifiers are in complementary distribution with the other elements discussed above.  Mackenzie himself (\citeyear[392]{Mackenzie2010}) notes that they are in complementary distribution with other focus-fronted elements as well as with wh-elements and \citet{Poole2016} notes that this complementarity extends to include interpolation.  It therefore seems plausible to suggest that these are elements which occupy \is{Salvi's Generalization}Salvi’s P2 position.

The importance of the distributional observation stems from the information-struc\-ture of sentences in which this quantifier \isi{fronting} has taken place.  As \citet[390]{Mackenzie2010} observes, “[h]owever hard one looks at examples like these…, it is impossible to see anything other than neutral assertions”, ultimately concluding (\textit{ibid.}) that constructions such as \REF{ex:poole:14} and \REF{ex:poole:15} instantiate wide or broad focus, and indeed labels the construction Wide \isi{Focus} \isi{Fronting}\is{fronting}.

This intuition is confirmed by \citet{Poole2016}.  As he notes, \isi{fronting} such as that seen in \REF{ex:poole:14} and \REF{ex:poole:15} cannot instantiate any kind of information- or contrastive-focus.  Neither can \REF{ex:poole:14} and \REF{ex:poole:15} instantiate verum/positive \isi{polarity} focus, as the construction can be found in environments such as the complements of \isi{factive} clauses, which, following \citet{LeonettiEscandell-Vidal2009}, strongly disallow it:

\noindent\parbox{\textwidth}{\ea%16
    \label{ex:poole:16}
(\textit{Sermones}, early 16\textsuperscript{th} c., \citealt[(4)]{Poole2013})\\    
    \gll    y     así   atinaron   a  pedir     el bien   y desearlo con grandes ansias   \textbf{viendo} \textbf{que} \textbf{nada} \textbf{podían}.\\
	  and thus aimed.\oldstylenums{3}\textsc{pl} to ask.\textsc{inf} the good and desire.it with grand will   see.\textsc{ger} that nothing {could do.\oldstylenums{3}\textsc{pl}}\\
    \glt ‘And thus they settled for praying for good and for desiring it with all their hearts, seeing that they could do nothing else.’
    \z}

\noindent Neither can quantifiers of this sort serve as topics.  Non-specific quantifiers such as \textit{algo} ‘something’ in \REF{ex:poole:14} simply cannot coherently be ``what the sentence is about''.  Therefore, it appears to be the case that the fronted quantifier itself is not (and indeed cannot be) either a topic or a focus.  If this is the case, then \is{Salvi's Generalization}Salvi’s P2 position can be occupied by not only topical or focal elements, but also elements which are \textit{neither}. 

\section{Explaining Salvi’s Generalization:  EPP and tense features in the low left periphery}\label{sec:poole:3}

The extension of \is{Salvi's Generalization}Salvi’s Generalization to include elements which are neither foci nor topics is significant because it provides a clear direction to pursue with respect to the explanation:  the displacement associated with \is{Salvi's Generalization}Salvi’s P2 position is triggered by a feature, hosted in the low left periphery, which does not itself possess any information-structure value (i.e., a feature such as Chomsky's (\citeyear{Chomsky2000,Chomsky2001}) \isi{EPP-feature}\is{EPP}, a ``formal feature'' in the sense of \citet{Frey2006}; \citet{Light2012}, or \citegen{BiberauerEtAl2014syntactic} ``movement triggering'' feature).  Such an approach would account for the fact that the position can be filled by one element only, and that the information-structure status of the element is irrelevant.

However, if \is{Salvi's Generalization}Salvi’s generalization regarding the P2 position ultimately derives from an extension of \citegen{BiberauerRoberts2010} feature-inheritance typology to \ili{Old Spanish} (and therefore relates ultimately to \citegen{Holmberg2015verbsecond} V2\is{verb second} property), we should see evidence that the low left periphery not only retained an \isi{EPP-feature}\is{EPP}, but also that it retained a Tense\is{tense} feature.  In other words, in addition to the evidence that XPs raise into the low left periphery in \ili{Old Spanish}, we should also find evidence that the verb in \ili{Old Spanish} raises to a position in the C-domain.  

\subsection{Verb-adjacency revisited}\label{sec:poole:3.2}
As mentioned above in \sectref{sec:poole:2.6}, it has been noted by various authors in various contexts that the verb in \ili{Old Spanish} must be linearly adjacent to elements which, by hypothesis, occupy a specifier position in the low left periphery.  \citet[175]{Sitaridou2011}, following \citet{Cruschina2008}, notes for example that strict adjacency is required between the verb and focus fronted elements, as exemplified by \REF{ex:poole:17}:

\noindent\parbox{\textwidth}{\ea%17
    \label{ex:poole:17}
(\textit{General Estoria} 1, 13\textsuperscript{th} c. \citealt[(23b)]{Sitaridou2011})\\    
    \gll     Fuerça \textit{fizieron} los sabios e los altos omnes en el nombre d’ esta cibdad.  \\
{power} made.\oldstylenums{3}\textsc{pl} the savants and the high men in the name of this city	\\
    \glt ‘The savants and the men of high standing imposed power in the name of the city.’
\z}

\noindent Under the assumption that the focus-fronted object occupies a position in the low left periphery (such as FocusP), the strict linear adjacency would be accounted for under the assumption that the verb moves to a head position in the same area of the clause.  As also noted above, verb-adjacency is also a requirement for resumptive preposing and interpolation:  if these elements are correctly analyzed as occupying the low left periphery (see \citealt[Section~3]{Sitaridou2011} and the references cited there), then a natural explanation for the observed linear adjacency with respect to the finite\is{finiteness} verb is that it too has raised to a C-related position.  

\subsection{\citet{Sitaridou2012} on tests for V-raising to the C-domain}\label{sec:poole:3.3}
In addition, contra \citealt{Sitaridou2012}, there do appear to be phenomena which suggest that there is V-raising to the C-domain in \ili{Old Spanish}.  In her survey of a \isi{number} of medieval \ili{Romance} varieties, \citet{Sitaridou2012} enumerates a number of traditional syntactic tests which are claimed to provide evidence that the verb moves to a position higher than T\textsuperscript{o}.\footnote{Her proposal more specifically is that the verb moves to Fin\is{finiteness}\textsuperscript{o}.}  She concludes on the basis of these tests that the verb did obligatorily raise to the C-domain in Old \ili{French}, among other varieties, but that this was not the case in \ili{Old Spanish}.  However, there do appear to be examples in \ili{Old Spanish} which parallel the examples offered for Old \ili{French}, once one moves beyond the one text that Sitaridou examines (the \textit{General Estoria} of Alfonso X).  

One traditional argument/test concerns the position of the verb relative to various adverbs which are very high on \citegen{Cinque1999} adverb hierarchy.  She notes, for example, that in Old \ili{French} the verb can appear higher than adverbs such as \textit{vraiment} ‘really’.

\ea%18
    \label{ex:poole:18}
    \citep[(52b)]{Sitaridou2012}\\
    \gll Et \textit{je} croy vraiement.  \\
         and I think.\oldstylenums{1}\textsc{sg} really\\
    \glt ‘And I really believe.’ 
\z

\noindent Although high, speaker-oriented adverbs are generally not found in \ili{Old Spanish}, one can find examples in which the finite\is{finiteness} verb precedes \isi{polarity} focus \textit{bien} ‘well’ (cf. \citealt{Hernanz2006,BatlloriHernanz2013}).\footnote{All unattributed examples from \ili{Old Spanish} are taken from the \textit{Corpus del Español} \citep{Davies2002}.}

\noindent\parbox{\textwidth}{\ea%19
    \label{ex:poole:19}
    (\textit{Cuento de Tristán de Leonís}, 14\textsuperscript{th} c.)\\
    \gll    \textbf{yo} \textbf{creo} \textbf{bien} que el era tristan ca non es enel  mundo caualler que tanto pudiese fazer.    \\
          I believe.\oldstylenums{1}\textsc{sg} well that he was T because \textsc{neg} is in.the world  man that so.much could    do.\textsc{inf}    	\\
    \glt ‘I really believe that he was Tristan because there is no [other] man in the world who could do so much.’
    \z}

\ea%20
    \label{ex:poole:20}
    (\textit{Estoria de España}, 13\textsuperscript{th} c.)\\
    \gll   Et    todo  omne que viesse   la posada que el çid    tenie   \textbf{dirie} \textbf{bien} que era vna grant hueste.  \\
 And every man  that saw.\oldstylenums{3}\textsc{sg} the ship   that The Cid had.\oldstylenums{3}\textsc{sg} {would say.\oldstylenums{3}\textsc{sg}} indeed that was.\oldstylenums{3}\textsc{sg} a great host	\\
    \glt ‘And everyone who saw the ship that The Cid had would indeed say that it was a great military {force}.’
    \z
%     \todo[inline]{Suggest using \textsc{cond} instead of `would'}


\noindent The use and interpretation of \textit{bien} in examples such as \REF{ex:poole:19} and \REF{ex:poole:20} (particularly \REF{ex:poole:19}) appears entirely parallel to the case of Old \ili{French} \textit{vraiment} above.  More specifically, whether \isi{polarity} focus items occupy a ΣP/PolP phrase between \isi{TP} and the C-domain or some higher \isi{projection} within the C-domain itself, examples such as \REF{ex:poole:19} and \REF{ex:poole:20} would seem to show that the verb in \ili{Old Spanish} did indeed undergo ``V-to-C movement'', on analogy with the \ili{French} cases.

Another class of examples which Sitaridou argues provides evidence that the verb raises to a high position in Old \ili{French} are cases such as \REF{ex:poole:21}:

\ea%21
\label{ex:poole:21}
\citep[(53a)]{Sitaridou2012}\\    
    \gll   pour la grant amour ai \textit{je} pourchacie ...   \\
       	for the great love have.\oldstylenums{1}\textsc{sg} I pursued\\
    \glt ‘For the great love I have pursued’ 

    \z

\noindent Following \citealt{Benincà1994}, Sitaridou argues (\citeyear[589--90]{Sitaridou2012})
% \todo{Replaced `she' with `Sitaridou' here for clarification} 
that this inversion pattern, in which the subject pronoun appears between the auxiliary\is{Auxiliary} and the past \isi{participle}, is evidence of obligatory V-to-C raising.  Once again, moving beyond her very specific corpus, it is not difficult to find examples parallel to the Old \ili{French} example in \REF{ex:poole:21}.

\ea%22
    \label{ex:poole:22}
      (\textit{General Estoria} IV, 13\textsuperscript{th} c.)\\
    \gll    \& de Caripdis. de quien \textbf{auemos} \textbf{nos} \textbf{contado} enla tercera parte desta estoria  \\
	    And of C         of  who   have.\oldstylenums{1}\textsc{pl}  we  related   in.the third  part   of.that history\\
    \glt ‘…and about Caripdis, about whom we have spoken in the   third part of that history…’
    \z


\ea%23
    \label{ex:poole:23}
(\textit{Gran Conquista de Ultramar}, 13\textsuperscript{th} c.)\\    
    \gll    Todo aquesto \textbf{he} \textbf{yo} \textbf{hablado} conel duque Gudufre.  \\
	  All    these     have.1\textsc{sg} I spoken with.the duke G\\
    \glt   ‘I have spoken about all of this with Duke Gudufre.'
    \z


\noindent Therefore, it appears as though there is some parallel evidence based on the traditional tests which \citet{Sitaridou2012} discusses in Old \ili{French} for thinking that the verb in \ili{Old Spanish} does indeed raise to some C-related position, and therefore that the C-domain hosted a Tense\is{tense} feature.

\subsection{The diachrony of P2 fronting and its relation to V-to-C raising}\label{sec:poole:3.4}
A further reason for thinking that both \isi{EPP} and Tense\is{tense} features are located in the low left periphery in \ili{Old Spanish} comes from the diachronic development of some of the constructions discussed above.  An examination of the \textit{Corpus del Español} reveals that those instances of low left-peripheral \isi{fronting} unequivocally triggered by a discourse-neutral \isi{EPP} feature decline in parallel with verb-initial declaratives with a post-verbal object.  If the \isi{EPP-feature}\is{EPP} and the \isi{tense} feature are somehow linked, as suggested by \citegen{BiberauerRoberts2010} typological analysis, this parallel decline would be expected.

Consider first the diachrony of two particular instances of movement to \is{Salvi's Generalization}Salvi’s P2 position:  interpolation and wide focus \isi{fronting}.  Recall from \sectref{sec:poole:2.4} above that the element which intervenes between the \isi{clitic} pronoun and the finite\is{finiteness} verb in interpolation structures can be interpreted in some cases as topics and in other cases as foci.  This suggests that the trigger for the \isi{fronting} of that element is a feature which is independent of any particular information-structure interpretation.  A logical conclusion therefore is that the trigger is a ``pure'' \isi{EPP} or movement-triggering feature.  A similar conclusion can be reached in the case of \citegen{Mackenzie2010} Wide \isi{Focus} \isi{Fronting}\is{fronting} (\sectref{sec:poole:2.6}).  Elements such as non-specific \textit{algo} are attracted to the low left periphery, but insofar as these elements cannot be interpreted either as topics or foci, it must be a pure movement-triggering feature which attracts them.  

An examination of the \textit{Corpus del Español} shows that interpolation of \isi{negation}, while robustly attested during the 13\textsuperscript{th} and 14\textsuperscript{th} centuries, declines significantly in the 15\textsuperscript{th} century and is essentially extinct by the 16\textsuperscript{th}.  Clausal \isi{negation}, as in \REF{ex:poole:24}, is one of the most commonly interpolated elements and \citet{Poole2013} claims that it too instantiates XP movement to the low left periphery.

\ea%24
    \label{ex:poole:24}
    (\textit{El Conde Lucanor}, 14\textsuperscript{th} c.)\\   
    \gll    Et   desque vio        que \textbf{lo} \textbf{non} \textbf{fazia}….  \\
	  and since   saw.\oldstylenums{3}\textsc{sg} that it \textsc{neg} would.do.\oldstylenums{3}\textsc{sg}\\
    \glt  ‘And since he saw that he wouldn’t do it….’
\z

\noindent As \tabref{tab:poole:1} shows, \isi{relative} instances of the subordinating Complementizer\is{complementizer} \textit{que} ‘that’ or \textit{si} ‘if’, followed by an object pronoun, followed by clausal \isi{negation}, followed by a finite\is{finiteness} verb (indicative, conditional or subjunctive) remain unchanged during the 13\textsuperscript{th} and 14\textsuperscript{th} centuries.  However, they decline to less than a quarter of that value in the 15\textsuperscript{th} century, and only a handful of cases are to be found by the beginning of the Golden Age period.\footnote{During the 13\textsuperscript{th} and 14\textsuperscript{th} centuries, non-interpolated clausal \isi{negation} (i.e., the order no/non ObjPn V) is found approximately equally frequently.  However, in the 15\textsuperscript{th} century, non-interpolation appears approximately 5.5 times more frequently, and is over 3100 times more frequent in the 16\textsuperscript{th} century.  See \citet[Section~1]{Poole2013} for further discussion.}  

\begin{table}
\begin{tabular}{ld{2}d{2}d{2}d{2}}
\lsptoprule
 {Period} & \multicolumn{1}{l}{13\textsuperscript{th} c.} & \multicolumn{1}{l}{14\textsuperscript{th} c.} & \multicolumn{1}{l}{15\textsuperscript{th} c.} & \multicolumn{1}{l}{16\textsuperscript{th} c.}\\
 \midrule
 Instances & 1745 & 718 & 479 & 4\\
 Per Million Words & 259.84 & 268.96 & 58.69 & 0.23\\
\lspbottomrule
\end{tabular}
\caption{Corpus del Español \textit{: que/si ObjPn no/non [vi*]/[vc*]/[vs*]}}
\label{tab:poole:1}
\end{table}

  A similar diachronic trajectory is seen with respect to \citegen{Mackenzie2010} Wide \isi{Focus} \isi{Fronting}\is{fronting}.  One relatively common example is the \isi{fronting} of \textit{esto} ‘this’ in examples such as \REF{ex:poole:25}.

\ea%25
    \label{ex:poole:25}
    (\textit{Gran Conquista de Ultramar}, 13\textsuperscript{th} c.)\\
    \gll     Quando \textbf{el}  \textbf{emperador} \textbf{esto} \textbf{oyo} ouo muy gran miedo. \\
	  When  the emperor   this heard had very great fear\\
    \glt  ‘When the emperor heard this, he became very afraid.’
    \z

\noindent Like interpolation, Wide \isi{Focus} \isi{Fronting}\is{fronting} declines significantly in the 15\textsuperscript{th} century relative to the 13\textsuperscript{th} and 14\textsuperscript{th}, and is nearly extinct by the 16\textsuperscript{th}.  

\begin{table}
\begin{tabular}{ld{2}d{2}d{2}d{2}}
\lsptoprule
{Period} & \multicolumn{1}{l}{13\textsuperscript{th} c.} & \multicolumn{1}{l}{14\textsuperscript{th} c.} & \multicolumn{1}{l}{15\textsuperscript{th} c.} & \multicolumn{1}{l}{16\textsuperscript{th} c.}\\
 \midrule
 Instances & 108 & 56 & 102 & 36\\
 Per Million Words & 16.08 & 20.98 & 12.50 & 2.11\\
\lspbottomrule
\end{tabular}
\caption{\textit{Corpus del Español}: Det N \textit{esto} V}
\label{tab:poole:2}
\end{table}

  Interestingly, in parallel with the decline of these XP-\isi{fronting} constructions, there is some evidence to suggest that verb-raising\is{verb raising} to a high position also declines.  As \citet[Section~3.4.2]{Fontana1993} notes, one traditional diagnostic for V-to-C raising in the literature on various Germanic varieties (e.g., Modern \ili{Yiddish} and \ili{Icelandic}) is the grammaticality of verb-initial declarative sentences, and sentences such as \REF{ex:poole:26} are very common in \ili{Old Spanish}, particularly  in main clauses introduced by \textit{and} (or variant).\largerpage[-2]

\noindent\parbox{\textwidth}{\ea%26
    \label{ex:poole:26}
(\textit{Estoria de España}, 13\textsuperscript{th} c.; \citealt[(74a)]{Fontana1993})\\
    \gll   \& \textbf{fizo} el papa penitencia \& \textbf{dixo} Sant Antidio la missa en su lugar  \& consagro la crisma. \\
	\& did the pope penance \& said sant Antidio the mass in his place \& consecrated the host\\
    \glt ‘And the pope did penance \& S. A. said the mass in his place and consecrated the Host.’
    \z}

\noindent Fontana notes (\citeyear[249]{Fontana1993}) that the percentage of verb-initial declaratives followed by a \isi{clitic} pronoun (thereby even more clearly suggesting that the verb has raised to a relatively high position) declines from the 12\textsuperscript{th} to the 16\textsuperscript{th} centuries, and it appears to decline in a way reminiscent of the XP-\isi{fronting} data seen above.
% % \todo[inline]{Check \textbackslash noindent here before publication}

\begin{table}
\begin{tabular}{p{3.5cm}rrrrr}
\lsptoprule
{Period} & \multicolumn{1}{l}{12\textsuperscript{th} c.} & \multicolumn{1}{l}{13\textsuperscript{th} c.} & \multicolumn{1}{l}{14\textsuperscript{th} c.} & \multicolumn{1}{l}{15\textsuperscript{th} c.} & \multicolumn{1}{l}{16\textsuperscript{th} c.}\\\midrule
V – Cl order observed vs. Cl – V order & 84\% & 85\% & 87\% & 68\% & 14\%\\
\lspbottomrule
\end{tabular}
\caption{V-Cl vs Cl-V order}
\label{tab:poole:3}
\end{table}

\noindent Data from a representative search of the \textit{Corpus del Español} paints a similar picture.  Though the \textit{Corpus del Español} does not contain any texts from the 12\textsuperscript{th} century, a search for a coordinating conjunction followed by an indicative verb form with an enclitic plural indirect-object pronoun shows a significant decline from the 14\textsuperscript{th} to the 15\textsuperscript{th} centuries to near extinction in the 16\textsuperscript{th} century.

\begin{table}
\begin{tabular}{ld{2}d{2}d{2}d{2}}
\lsptoprule
{Period} & \multicolumn{1}{l}{13\textsuperscript{th} c.} & \multicolumn{1}{l}{14\textsuperscript{th} c.} & \multicolumn{1}{l}{15\textsuperscript{th} c.} & \multicolumn{1}{l}{16\textsuperscript{th} c.}\\
\midrule
Frequency & 46 & 84 & 82 & 47\\
Per Million Words & 6.85 & 31.47 & 10.04 & 2.76\\
\lspbottomrule
\end{tabular}
\caption{\textit{Corpus del Español:} [cc*] *les.[vi*] minus all 2sg verb forms}
\label{tab:poole:4}
\end{table}

To summarize then, the data in the tables above suggests that XP-\isi{fronting} triggered by a ``pure'' \isi{EPP} or movement-triggering feature undergoes a diachronic decline which bears some resemblance to the decline seen in V-to-C raising.  

\section{Convergence:  EPP/tense feature inheritance and some implications}\label{sec:poole:4}
\subsection{\citegen{BiberauerRoberts2010} feature-inheritance approach to syntactic typology}\label{sec:poole:4.1}
As mentioned at various points above, a central claim of this paper is that the synchronic and diachronic descriptive facts discussed in Sections 2 and 3 can be naturally accounted for using \citegen{BiberauerRoberts2010} feature inheritance typological approach:  EPP- and T-features are \textit{retained} in the C-domain in \ili{Old Spanish}, but \textit{donated} to T in Modern Spanish.

Following \citegen{Ouali2008} classification, uninterpretable features present on the \isi{phase} head C may either be ``kept'', ``shared'' or ``donated''.  They are retained by the phase-head in the first case, but either copied or given over entirely to a phase-internal non-\isi{phase} head in the latter two cases respectively.  Biberauer \& Roberts suggest that Ouali’s feature-inheritance classification system can be usefully extended into language typology.  By way of illustration, Biberauer \& Roberts argue that phi- and T-features are donated to T in \ili{Romance} and English (leading to V-to-T movement in \ili{Romance} because of the presence of rich \isi{tense}) but are kept in Continental Germanic, leading to V-to-C raising, one part of the well-known verb-second effect.  These options also apply in the case of XP movement-triggering features such as the \isi{EPP-feature}\is{EPP}.  In \ili{Mainland Scandinavian}, for example, the \isi{EPP-feature}\is{EPP} present in C is shared with T.  As a result, both a traditional verb-second and English-style \isi{EPP} effect is seen.  (See \citealt[Section~3]{BiberauerRoberts2010} for further discussion and examples.)

For \ili{Old Spanish} then, the claim would be that both the \isi{EPP} and Tense\is{tense} features were retained in the C-domain.  This accounts first for the distribution of elements seen in the low left periphery as part of Salvi’s (Extended) Generalization\is{Salvi's Generalization} regarding the P2 position.  There can be only one element, and because it is attracted by a pure \isi{EPP}/movement-triggering feature, it can be either topical, focal, or neither.  The fact that the T-feature is also kept results in the verb being attracted to this position within the low left periphery, which places it adjacent to the element in Salvi’s P2 position.  

The diachronic data seen in \sectref{sec:poole:3.4} finds a natural account under the assumption that at some point during the Spanish Golden Age period, the \isi{EPP} and Tense\is{tense} features ceased being retained in the C domain, and were instead donated to T.  This explains why certain cases of low-left-peripheral \isi{fronting} appear to decline to extinction in parallel with cases of verb-initial declaratives with an enclitic object pronoun.  

Such an approach to the diachronic data aligns with \citegen{BiberauerRoberts2010} typological account of Modern \ili{Romance}.  As mentioned above, on Biberauer \& Roberts’s analysis of Modern \ili{Romance}, the \isi{EPP} feature is donated to T, but the requirement is met by a deleted\textbf{} pronoun in the case of \isi{null subject} languages such as \ili{Italian} and Spanish.  The T-feature is also donated from C to T, which, because of rich \isi{tense}, results in V-to-T movement.  We therefore have a straightforward characterization of (part of) the diachronic change that took place between \ili{Old Spanish} and Modern Spanish.

\subsection{Some synchronic and diachronic implications of the proposed approach: (re)visiting V2 and a ``syntacticization of discourse''}\label{sec:poole:4.2}
Synchroncially, the proposed account takes a clear position in the debate concerning whether or not \ili{Old Spanish} was a verb-second language.\footnote{The former position is represented by work such as \citet{Fontana1993}, while e.g. \citet{Sitaridou2012} argues for the latter position.}  On \citegen{Holmberg2015verbsecond} characterization, there are two components to the V2\is{verb second} property, which may be independently realized.

\ea%27
    \label{ex:poole:27}
     \ea \label{ex:poole:27a} A functional head in the left periphery attracts the finite\is{finiteness} verb.
     \ex \label{ex:poole:27b} This functional head requires that a constituent move to its specifier position.
\z
\z

\noindent The characteristics in \REF{ex:poole:27} appear to describe exactly the situation in \ili{Old Spanish}, as discussed in the above sections.  Indeed, as Holmberg notes (\citeyear[276]{Holmberg2015verbsecond}) the property in \REF{ex:poole:27b} “may be formalized as a ``generalised \isi{EPP-feature}\is{EPP}'', along the lines of Roberts (2004)”.

Salvi’s Generalization\is{Salvi’s Generalization} regarding the P1 position, the position which immediately precedes P2, then becomes the logical explanation for the well-attested instances of V3 (and other) orders in \ili{Old Spanish}.  The P1 position hosts topics, and I have suggested that it finds a natural correspondent in the high topic position within an articulated CP\is{complementizer}.  Given that topics in this position can be iterated \citep[103]{Salvi2012}, the existence of these orders does not undermine the claim that \ili{Old Spanish} was a verb-second language.\footnote{\citet[Fn~34]{Ott2014} suggests that his ellipsis approach to Contrastive Left-Dislocation could be extended to account for \ili{Romance} \isi{Clitic Left Dislocation} phenomena.  Should such an extension prove to be successful, \ili{Old Spanish} might more closely resemble a ``traditional'' verb-second language such as Modern \ili{German}.}\textsuperscript{,}\footnote{V1 declarative orders do exist in \ili{Old Spanish}, but as \citet{Poole2016} argues, they exhibit a specific information structure interpretation: wide or broad focus.  With respect to the satisfaction of the \isi{EPP-feature}\is{EPP}, there are a number of logical possibilities, including a base-generated ``default'' operator associated with sentence-level focus or declarative \isi{force}, or perhaps even attraction of the entire \isi{TP}, which would plausibly entail a wide focus interpretation.  However, I leave this {question} for future research.}

Diachronically, the proposed change in the behaviour of the features associated with C suggests that Spanish underwent a ``syntacticization of discourse''.\footnote{The term is originally due to \citet{HaegemanHill2013}.  See \citet[160]{Sitaridou2011} for some initial speculation regarding \ili{Old Spanish} and \citet{Poole2016} for much further discussion.}   Consider first the relation between syntax and information structure in \ili{Old Spanish} implied by the analysis outlined above.  Movement to the low left periphery is triggered by a pure movement-triggering feature.  Elements which are attracted by this feature can, however, ultimately receive an information-structure interpretation.  Recall example \REF{ex:poole:2} above:

\begin{exe}
 \exr{ex:poole:2}
    (\textit{General Estoria} 4, 13\textsuperscript{th} c., \citealt[(25)]{Sitaridou2011})\\
    \gll  \& los qui se  gozaron con el to derribamiento \textbf{penados} seran por ello.\\
	  and the who \textsc{refl.\oldstylenums{3}pl} enjoy.\textsc{fut}.\oldstylenums{3}\textsc{pl} with the your fall punished be.\textsc{fut}.\oldstylenums{3}\textsc{pl} for this\\
    \glt  ‘And those who rejoice with your fall they will be punished for that.’
\end{exe}

\noindent Following \citet{Sitaridou2011}, I assume that the fronted \isi{participle} is interpreted as focalized, but this is not because the movement is triggered or driven by a syntactic information-structure-specific feature such as [+focus].  In other words, elements are not attracted to Salvi’s P2 position\is{Salvi’s Generalization} in the low left periphery for discourse or information-structure reasons per se.  It follows therefore that, in \ili{Old Spanish}, information structure interpretation is in some way post-syntactic.\footnote{See, e.g., \citet{Cinque1993}, \citet{Reinhart2006} and \citet{Sheehan2010} for the suggestion that focus might be accounted for in prosodic terms.}  

The analysis of interpolation in particular becomes potentially important in this context.  As discussed above, it appears as though interpolated elements may be interpreted as either topical \REF{ex:poole:6} or focal \REF{ex:poole:8}.

\begin{exe}
 \exr{ex:poole:6} (\textit{Crónica de 1344 I,} 14\textsuperscript{th} c., \citealt[(19)]{Poole2013})\\
    Et estonçe les dixo el Rey \ul{que se salliesen de su tierra} Et aquella gente a qujen \ul{esto} dixo fueron se a la villa \& tanto que \textbf{les esto dixo} luego se armaron muy bien \& venjeron se al Rey onde yazia en su alcaçar \& lidiaron conel \& lo mataron. \\
    \glt ‘And then the king said to them that they should leave his land.  And those people to whom he said that went to the town and as soon as \textbf{he said} \textbf{\ul{that}} \textbf{to them} they armed themselves well and went to the king where he rested in his fortress and fought with him and killed him.’
\end{exe}

\begin{exe}
 \exr{ex:poole:8} (\textit{El Libro de Caballero} \textit{Zifar} 14\textsuperscript{th} c., \citealt[(2b)]{Poole2007})\\
    \gll     e dixe que \textbf{lo} \textbf{yo} \textbf{auja} muerto.\\
	 and said.\oldstylenums{1}\textsc{sg} that him I had killed\\
    \glt  ‘and I said that I had killed him.’
\end{exe}

\noindent Under the assumption that cases such as these are instances of ``the same'' syntactic phenomenon -- that is to say, truly two representations of the same syntactic process -- then information-structure interpretation in \ili{Old Spanish} must have been post-syntactic.  

However, if access to the low left periphery in \ili{Old Spanish} is mediated by an in\-for\-ma\-tion-struc\-tur\-al\-ly neutral \isi{EPP} feature which is retained in the C-domain, and, as discussed above, the diachronic change in Spanish involves the donation of this feature to T, then some method must have been developed by which access to the low left periphery was regained, given that Modern Spanish unquestionably has such access.  

\citet{Poole2016} suggests that one of the major syntactic changes to take place during the Spanish Golden Age period is that information-structure-specific syntactic features, such as [+focus], are developed.\footnote{Note that if in fact the innovated feature which is responsible for focus is [+contrast] rather than [+focus], following \citealt{Lopez2009}, that would account for the fact that left-peripheral focus is obligatorily contrastive in Modern Spanish but not in \ili{Old Spanish}.}  This predicts, for example, the loss of left-peripheral wide focus \isi{fronting}, as seen in \sectref{sec:poole:3} above.  Interestingly, as \citet{Poole2016} notes, two word orders which signal wide/broad focus in \ili{Old Spanish}, \citegen{Mackenzie2010} fronted quantifiers and the verb-initial declarative constructions referred to in the previous section, come in later varieties to signal verum/positive \isi{polarity} focus.  Poole suggests that this is an indication that displacement is driven in these later varieties by an in\-for\-ma\-tion-struc\-tur\-al\-ly specified syntactic feature.  In essence, low left-peripheral wide/focus is precluded because access to low left periphery now requires prior identification as a topicalized or focalized element, and therefore an element in the low left periphery which cannot be a topic must be focalizing \textit{something}.  Poole’s suggestion is that these elements have in fact first been attracted to the specifier of ΣP/PolP, the \isi{projection} which encodes sentence \isi{polarity}, and represent focalization of that category, accounting for the verum focus interpretation.\footnote{See \citealt{Poole2016} for much further discussion and \citealt{Poole2011} for another potentially relevant diachronic development relating to n-words.} 

% \section*{Abbreviations}
\section*{Acknowledgements}
Thanks to Maria Maza, the audience at the 2014 Meeting of the Linguistics Association of Great Britain and two anonymous reviewers for many helpful comments and suggestions.  
 
\printbibliography[heading=subbibliography,notkeyword=this]
\end{document}