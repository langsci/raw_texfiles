\documentclass[output=paper]{LSP/langsci} 
\author{Christer Platzack \affiliation{Lund University}
}
\title{Swedish wh-root-infinitives} 
% \epigram{Change epigram}
\abstract{%
The purpose of this short paper is to present a minimalist account of the syntax of the Swedish Wh-Root-infinitives, trying to characterize the syntax of this generally neglected main clause equivalent while comparing its syntax to the syntax of finite root clauses. See (i) and (ii):

\begin{xlist}[(ii)]
 \exi{\footnotesize(i)} Swedish\\
\gll Varför  sälj-a    huset?\\
      why    sell-\textsc{inf}  house.the\\
\trans   ‘Why sell the house?’

\exi{\footnotesize(ii)} Swedish\\
\gll Varför   sälj-er   ni   huset?\\
      why    sell-\textsc{pres}  you  house.the  \\
\trans  ‘Why   do you sell the house?’ 
\end{xlist}

My account is based on the hypothesis that C minimally probes the sentence type features finite, imperative or infinitive, present in the inflection of the verb. More precisely, I will show that the Swedish facts, like corresponding facts in English, German and Icelandic, follow from a grammar driven by an asymmetry with respect to feature values (see \citealt[6]{Chomsky2007} and subsequent papers), and that all unvalued features must be eliminated before syntax can zip together form and meaning.

}


\ChapterDOI{10.5281/zenodo.1117712}
\maketitle
\begin{document}
% \section{Introduction} 




% \textit{Keywords}:  root infinitives, \textit{varför} ‘why’, \ili{Swedish}, syntax

\section{Root infinitives}

In the Germanic languages, the overwhelming majority of independent sentences are finite\is{finiteness}, with a verb inflected at least for \isi{tense}. Occasionally, however, we find independent sentences with an infinitival verb, as in the English examples in \REF{ex:platzack:1}. These are usually introduced by \textit{why} or \textit{how}:

\noindent\parbox{\textwidth}{\ea%1
    \label{ex:platzack:1}
    English \citep[874]{HuddlestonPullum2003}
    \ea  {Why be so soft with them}?
    \ex  {Why not accept his offer}?
    \ex   {How to persuade her to forgive him}?
    \z
    \z}

	

In this paper I will discuss the syntax of \ili{Swedish} independent \isi{infinitive} sentences or Root-infinitives, introduced by a wh-word (Wh-RIs). This is a minor and mainly neglected sentence type, classified by \textit{The \ili{Swedish} Academy Grammar} \citep[IV: 826--827]{TelemanEtAl1999} as a kind of main clause equivalent. Some \ili{Swedish} examples are presented in \REF{ex:platzack:2}; as in English, \ili{Swedish} mainly prefers root infinitives introduced by a wh-adverb \textit{varför} ‘why’ or \textit{hur} ‘how’:

\ea%2
    \label{ex:platzack:2}
    	    \ili{Swedish}     (\# indicates that the example is not fully productive)\\
   \ea
\gll \#Vad göra?                     \\
      what  do.\textsc{inf}  \\
\glt   ‘What to do?’                       

   \ex
\gll \#Vart    vända    sig?\\
      where  turn.\textsc{inf}    \textsc{refl}\\
\glt   ‘Where to turn?’

   \ex
\gll Varför  inte  gå      på   bio?\\
      why    not  go.\textsc{inf}  to  movie\\
\glt   ‘Why don’t you go to the movie?’

   \ex
\gll Varför  sälja     huset?\\
      why    sell.\textsc{inf}  house.the\\
\glt   ‘Why sell the house?’
   \ex
\gll Hur få      personalen att förstå?   \\
      how get.\textsc{inf}   staff.the  to  understand   \\
\glt   ‘How to get the staff to understand?’ \citep[IV: 826--828]{TelemanEtAl1999}
\z
\z

\ili{German} accepts both wh-arguments and wh-adverbs (including \textit{warum} ‘why’), as shown by the examples in \REF{ex:platzack:3}, mainly from \citet{Reis2003}.

\ea%3
    \label{ex:platzack:3} 
	    \ili{German}\\
   \ea
\gll Was   tun?                     \\
what  do.\textsc{inf}\\
\glt   ‘What to do?’   \citet{Reis2003}, example (37a)

   \ex
\gll Wohin    sich  wenden?           \\
      where.to  \textsc{refl}  turn\\
\glt   ‘Where to turn?’ \citet{Reis2003}, example (37b)

   \ex
\gll Warum  nicht  ins     Kino    gehen? \\
      why    not  to.the    movies   go.\textsc{inf}\\
\glt   ‘Why not go to the movies?’ \citet{Reis2003}, example (38b)

   \ex
\gll Warum  das  Haus    verkaufen?\\
      why    the  house    sell.\textsc{inf}\\
\glt   ‘Why did you sell your house?’
\z
\z

\ili{Icelandic}, finally, according to Halldór Sigurðsson p.c., seems to accept only Wh-RIs with \textit{hvers vegna} ‘why’, preferably together with the infinitival marker \textit{að}, see (\ref{ex:platzack:4}c,d). 

\ea%4
    \label{ex:platzack:4}
	    \ili{Icelandic}\\
   \ea
\gll  *Hvað (að) gera?                   \\
       what    to  do.\textsc{inf}\\
\glt   ‘What to do?’ (Halldór Sigurðsson p.c.)

   \ex
\gll *Hvert (að) snúa     sér?               \\
      where  to  turn.\textsc{inf}  \textsc{refl}\\
\glt   ‘Where to turn?’ (Halldór Sigurðsson p.c.)

   \ex
\gll {Hvers vegna} ekki (að)  fara   í     bíó?      \\
      why        not   to    go.\textsc{inf} to  movie\\
\glt   ‘Why don’t go to the movie?’   (Halldór Sigurðsson p.c.)
\ex
\gll ?{Hvers vegna} (að)   selja     húsið?    \\
       why        to    sell.\textsc{inf}  house.the  \\
       \glt ‘Why sell the house?    (Halldór Sigurðsson p.c.)
\z
\z

None of the other languages investigated seem to accept an \isi{infinitive} Complementizer\is{complementizer} (English \textit{to}, \ili{German} \textit{zu}, \ili{Swedish} \textit{att}) in Wh-RIs. \ili{Swedish} and \ili{German} (see \citealt[156]{Reis2003} for \ili{German}) never use the infinitival marker in Wh-RIs. English, according to \citet[875]{HuddlestonPullum2003}, has two types of Wh-RIs, one without infinitival \textit{to} (\ref{ex:platzack:5}c,d), and one with infinitival \textit{to} that accepts more options, as shown in (\ref{ex:platzack:5}a,b):

\ea%5
    \label{ex:platzack:5}
	    English  \\
   \ea
 *{Where go}?    {Where to go}?\\
\ex
*{What do next}?    {What to do next}?\\
   \ex
 {Why be so soft with them}?\\
   \ex
 {Why not accept his offer}?\\
\z
\z

In short, of the four Germanic languages discussed above, \ili{German} seems to be less restricted with respect to which Wh-RIs that are possible (both argumental and adverbial wh-words, e.g.). \ili{Swedish}, \ili{Icelandic} and English mainly allow Wh-RIs introduced by \textit{why} (English) and its \ili{Swedish} and \ili{Icelandic} counterparts \textit{varför} and \textit{hvers vegna}. English Wh-RIs with \textit{to} also accept other adverbial wh-words. 

  In this paper I will take a closer look at \ili{Swedish} Wh-RIs, claming that Wh-RIs belong to an independent infinitival sentence type and arguing for an analysis that expands Reis’ account of \ili{German} Wh-RIs (\citeyear{Reis2003}). Reis’ analysis is in its turn inspired by \citegen{PlatzackRosengren19971998} account of the imperative sentence type, launched within the \isi{Minimalist}\is{Minimalist/minimalist} program (\citealt{Chomsky1995} and subsequent work).

  The paper is organized in the following way. \sectref{sec:platzack:2} gives a brief introduction to those parts of the \isi{Minimalist}\is{Minimalist/minimalist} program that are vital for understanding my account, and \sectref{sec:platzack:3} outlines the analysis of the ordinary finite\is{finiteness} sentence type. \sectref{sec:platzack:4} presents the non-finite\is{finiteness} sentence type Wh-RIs and comppares its syntax with the syntax of finite\is{finiteness} sentences. A short conclusion and discussion follows in \sectref{sec:platzack:5}. 

\section{A short presentation of the theoretical framework}\label{sec:platzack:2}

In this section I will present some central assumptions of the feature-driven version of the \isi{minimalist} program that I use for my analysis of Wh-RIs in \ili{Swedish}. In general, I will stay close to \citet{Chomsky2007,Chomsky2008} and \citet{PesetskyTorrego2001,PesetskyTorrego2004,PesetskyTorrego2007}.  The assumption that there are three types of independent sentence types, finite\is{finiteness}, infinite and imperative, corresponding to the same three types of basic verbal inflection, is outlined in more detail in \citet{PlatzackRosengren2017}. 

\subsection{Morphology Lexicon and Features}

Following a recent discussion in  \citet{CecchettoDonati2015} I have chosen a version of the \isi{minimalist} program where words are created in an autonomous \isi{morphological} module. Hence a word can be seen as an atomic element from the point of view of syntax. Each word is, according to \citet[6]{Chomsky2007}, “a structured array of  properties (\textit{features}) to which \isi{Merge} and other operations apply to form expressions.”  

Features enter the syntactic computation either as valued or unvalued; the purpose of the computation is to build structure so that all unvalued features become valued.

 \subsection{Merge, EPP and the operation Agree}

The central player of the \isi{Minimalist}\is{Minimalist/minimalist} syntactic derivation is the operation \isi{Merge} that builds structure. \isi{Merge} operates on (bundles of) features (valued or unvalued) that provide the building material for syntactic structure. \isi{Merge} takes a feature bundle and adds it to another feature bundle, creating a minimal structure, see \REF{ex:platzack:6}:

\ea%6
    \label{ex:platzack:6}
      Pick the feature bundle A and merge it to an available feature bundle B:
      \begin{forest}
       [A\slash B [A,edge label={node[midway,left]{A~~B~~~→~~~}}] [B]]
       \end{forest}
\z

The result of merging A and B is labeled either A or B. \isi{Merge} can now take a new feature bundle X from the \isi{lexicon} and merge it to the root of the structure, illustrated in \REF{ex:platzack:7}, or it may take the feature bundle B, already present in the derivation, and remerge it to the root of the structure, yielding \REF{ex:platzack:8}; this operation is also called “Move”:


\ea%7
\label{ex:platzack:7}
\begin{forest}
 [X
  [X] [A\slash B
	[A] [B] ]
 ]
\end{forest}
\z
% \todo{Check placement of example numbers}
\ea%8
\label{ex:platzack:8}
  \begin{forest}
   [A 
    [B] [A\slash B [A] [B]]
   ]
  \end{forest}
\z

The operation Agree\is{agreement}, see \citet[31ff.]{Chomsky2001} and below, establishes a connection between an unvalued and a valued instance of a feature, valuing the unvalued one, see \REF{ex:platzack:9}. The derivation will crash if there is any unvalued feature left at the semantic\is{semantics} interface.

\ea%9 

    \label{ex:platzack:9}
\textit{The operation Agree}

\begin{description}
 \item[Step 1:] Select a \textbf{probe} i.e. a head with at least one unvalued feature [¬F], where [F] is a variable over features.

 \item[Step 2:] Search the c-command domain of the probe for the closest \textbf{goal} with a valued instance of the same feature, [F]. 

  \item[Step 3:] Value the unvalued feature of the probe in accordance with the value of the goal. 
\end{description}
\z

Agree\is{agreement} may be accompanied by movement of the bearer of the valued feature to the bearer of the unvalued feature. This operation presupposes the presence of what I here call “\isi{EPP}” which is associated with an unvalued feature, [¬F\textsuperscript{EPP}], saying that the agree-relation must be visible. 

\subsection{The Morpho-Syntactic interface}

Following \citet[14]{CecchettoDonati2015}, I assume that “[a] word which is delivered by \isi{morphology} to syntax, is intrinsically endowned with a category feature”. For verbs, I assume the verbal feature [v], for nouns the nominal feature [n] and for adjectives the adjectival feature [a]. In addition to the categorial feature, \isi{morphology} also provides inflectional features. With \citet{PlatzackRosengren2017}, I assume three kinds of inflectional features in Germanic verbs, listed in \REF{ex:platzack:10} and exemplified in \REF{ex:platzack:11} with \ili{Swedish} independent main clauses:

\ea%10
    \label{ex:platzack:10}
	 \ea
 \textbf{The finite\is{finiteness} inflection}, introducing the feature [fin] and expressed by the \isi{tense}            suffix in \ili{Swedish}.\\

   \ex
 \textbf{The imperative inflection}, introducing the feature [imp] and expressed by the verbal       stem in \ili{Swedish}.\\

   \ex
 \textbf{The infinitival inflection}, introducing the feature [inf] and expressed by the           \isi{infinitive} suffix -\textit{a} in \ili{Swedish}.\\
\z
\z

 \ea
 \label{ex:platzack:11}  
 \ili{Swedish}\\
   \ea \label{ex:platzack:11a}  
\gll Johan  läs-te    en  roman.\\
      Johan    read-\textsc{tns}  a  novel\\
\glt   ‘Johan read a novel.’

   \ex
\gll Läs  en  roman!\\
      read  a  novel\\
\glt   ‘Read a novel!'        

   \ex
\gll Varför  läs-a    en  roman!?\\
      why    read-\textsc{inf}  a  novel\\
\glt   ‘Why read a novel!?’
\z
\z

Simplifying, the first step in the derivation of a sentence like \ili{Swedish} \REF{ex:platzack:11a} or its \ili{German} counterpart, \textit{Johann las einen Roman},
% \todo{einen Roman} 
is to pick the verb \textit{läste} / \textit{las} ‘read.\textsc{past}’ from the \isi{lexicon} and merge it with a DP \textit{en roman / ein Roman} ‘a novel’ bearing an internal theta-role in relation to the verb. The \ili{Swedish} case is outlined in \REF{ex:platzack:12} and the \ili{German} one in \REF{ex:platzack:13}, the main difference being that the \ili{Swedish} vP structure is VO, the \ili{German} one OV. 

\ea%12
    \label{ex:platzack:12}
 
	  \ili{Swedish} \\{}
    [\textsubscript{vP} Johan  [läste en roman]]
\z

\ea%13
    \label{ex:platzack:13}
	  \ili{German}\\{}
    [\textsubscript{vP} Johan  [ein Roman] las]
\z

Like all DPs, both the external argument \textit{Johan} in Spec,vP and the object \textit{en roman / ein Roman} in the complement of v carry a valued $\varphi $-feature. In addition, v carries a valued [fin] feature. The different order between v and DP in \ili{Swedish} and \ili{German} is mandatory to capture the syntactic differences between VO languages and OV languages, see \citet[5--43]{Haider2010} for a detailed presentation. Among other things, this difference plays a role in accounting for the fact that \ili{Swedish} but not \ili{German} displays subject–object asymmetries, see \citet[79ff]{Haider2010}. Since the VO/OV distinction is not in focus here, I will mainly give structures where the head governs to the right, as in English, \ili{Icelandic} and \ili{Swedish}.

\section{The derivation of the Swedish finite sentence type}\label{sec:platzack:3}

In this section, I will illustrate the functional parts \isi{TP} (called Finite\is{finiteness} Phrase in \citealt{Rizzi1997}) and CP\is{complementizer} (called \isi{Force} Phrase in \citealt{Rizzi1997}).   

  In the absence of sentence adverbs and auxiliaries, which are supposed to be adjoined to vP, the next step after vP is assembled is to pick T from the \isi{lexicon} and merge it to vP. The result is depicted in \REF{ex:platzack:14}, assuming T to carry both unvalued phi-features and an unvalued fin-feature: 

\ea%14
    \label{ex:platzack:14}
  \begin{forest}
   [\isi{TP}, for tree={align=center, base=top, parent anchor=south, child anchor=north}
    [T\\{[}¬ϕ\textsuperscript{EPP}{]}\\{[}¬fin{]}] [vP
      [DP\\Johan\\{[}ϕ{]}] [v'
	[v\\läste\\{[}fin{]}] [DP\\en roman\\{[}ϕ{]}]
	]
      ]
    ]
   ]
  \end{forest}
\z

In \ili{Swedish}, as in the Germanic languages in general, the subject is visible in Spec,\isi{TP}, indicating the presence of an \isi{EPP} feature. The presence of a visible subject is accounted for by postulating that \isi{EPP} is attached to the unvalued [$\varphi $]-features in T, hence the proper formulation of this feature will be [¬$\varphi $\textsuperscript{ EPP}]. This forces the closest c-commanded DP, i.e. \textit{Johan}, to move to Spec,\isi{TP}.   

  The derivation of \isi{TP} as illustrated in \REF{ex:platzack:14} is not complete. T contains an unvalued finiteness feature that must be valued. Acting as a probe, T with feature  [¬fin] will establish an Agree\is{agreement} relation with [fin] in little v and thereby the finite\is{finiteness} feature in T is valued. There is no reason to assume that the verb moves from v to T in \ili{Swedish}; if so, we would, contrary to facts, have expected the finite\is{finiteness} verb to appear in front of the \isi{negation} in an embedded clause, taking for granted that the \isi{negation} is adjoined to vP and thus to the right of a verb that has moved to T.

  The highest \isi{phase} in the derivation of a sentence is the C-\isi{projection} with an unvalued finiteness feature in C, which is valued by merging C to \isi{TP}. In most Germanic languages the \isi{tensed} verb is moved to C, due to \isi{EPP} associated with [¬fin] in C. Among other things, this gives rise to \isi{verb second}. Spec,CP\is{complementizer} may be filled by the subject or object, by an adverb phrase, prepositional phrase and various other elements. Since the main part of this paper discusses the adverb \textit{varför} ‘why’ I will here show how a finite\is{finiteness} clause introduced by \textit{varför} is derived.  Being a wh-adverb, \textit{varför} must move to Spec,CP\is{complementizer}. See  \citet{ShlonskySoare2011} for a partly different account.

  I will assume that a \isi{TP} containing \textit{varför} `why' merges\is{Merge} with C that carries the features [¬fin] and [¬wh], both with \isi{EPP}. This will \isi{force} the closest wh-phrase to move to Spec. CP\is{complementizer}

This is illustrated with the \ili{Swedish} sentence in \REF{ex:platzack:15} where both the unvalued features in C are marked \isi{EPP}. The structure when C has merged is given in \REF{ex:platzack:16}:

\noindent\parbox{\textwidth}{\ea%15
    \label{ex:platzack:15}
	  \ili{Swedish}\\
\gll   Varför  läs-te      Johan   en roman?\\
    why    read-\textsc{past}  Johan    a  novel\\
   \glt ‘Why did Johan read a novel?’
\z}
% \todo{translation is present tense}

\ea%16

    \label{ex:platzack:16}
\begin{forest}
 [CP\is{complementizer}, for tree={align=center, base=top, parent anchor=south, child anchor=north}
    [varför\\{[}wh{]}] [C'
      [C\\{[}¬fin\textsuperscript{EPP}{]}\\{[}¬wh\textsuperscript{EPP}{]}] [\isi{TP}
	[DP\\Johan\\{[}ϕ{]}] [T'
	   [T\\{[}fin{]}] [vP
	      [\st{varför}] [vP
		  [DP\\\st{Johan}\\{[}ϕ{]}] [v'
		     [v\\läste\\{[}fin{]}] [DP\\en roman\\{[}ϕ{]}]
		    ]
		]
	    ]
	  ]
      ]
    ]
 ]
\end{forest}
\z

Let us start with the unvalued finiteness feature in C. When Agree\is{agreement} is applied, C will probe its c-command domain for a goal with valued finiteness feature, which it finds in T. Due to \isi{EPP}, the finiteness feature [fin] will be pronounced in C, a prerequisite for \isi{verb second}.

As seen in \REF{ex:platzack:16}, C hosts a second unvalued feature with \isi{EPP}, viz. [¬wh\textsuperscript{EPP}].\textsuperscript{} This feature probes wh-words, \textit{varför}  ‘why’, in our example. In line with \citet[667]{ShlonskySoare2011}, \textit{why}  (and, I assume, \ili{Swedish} \textit{varför}) is first merged in Spec,ReasonP, here simplified as a high adjunction to vP. In particular, ReasonP is assumed to be to the left of the \isi{negation}, which is adjoined to (a low) vP.

Starting from the structure in \REF{ex:platzack:16}, I will in the next section compare the syntactic properties of the independent \ili{Swedish} Wh-RIs with the properties of the independent \ili{Swedish} finite\is{finiteness} sentence.    

\section{The syntax of the independent  Swedish wh-root-infinitive}\label{sec:platzack:4}

\subsection{ Structural analysis} 

In this section I will present an analysis of the independent \ili{Swedish} root \isi{infinitive} introduced by a wh-phrase, arguing for a structure that differs from the finite\is{finiteness} structure (cf. \REF{ex:platzack:16}) mainly in lacking a T-\isi{projection} between C and v. This idea, which is taken from \citet{Reis2003}, is compatible with the fact that the \isi{infinitive} verb is not inflected for \isi{tense} and lacks a (visible) subject. See also \citet{PlatzackRosengren2017}.
% \todo{do not cite works in prep}

   In short, I will assume that there is no \isi{TP} in root infinitives. The Wh-RI in \REF{ex:platzack:17} will thus get the structure in \REF{ex:platzack:18} when all unvalued features are valued.

\ea%17
    \label{ex:platzack:17}
	  \ili{Swedish}\\
\gll Varför  läs-a    bok-en?\\
why      read-\textsc{inf}  book-the \\
\glt ‘Why read the book?’
\z

\ea%18
    \label{ex:platzack:18}
    \begin{forest}
     [CP\is{complementizer}, for tree={align=center, base=top, parent anchor=south, child anchor=north}
      [varför\\{[}wh{]}] [C'
	[{[wh]}\\{[inf]}] [vP
	  [\st{varför}] [vP
	    [DP\\{[ϕ]}] [v'
	      [v\\läsa\\{[inf]}] [DP\\boken\\{[ϕ]}]
	    ]
	  ]
	]
     ]
    ]
    \end{forest}
\z

As will be shown below, the analysis proposed accounts for the main syntactic differences between Wh-RIs and the finite\is{finiteness} sentence type. \sectref{sec:platzack:4.2} compares the finite\is{finiteness} and infinitival sentences with respect to word order, section 4.3 points out similarities and differences with respect to the external argument in Spec.vP, and section 4.4 briefly discusses \isi{A-bar movement} of \textit{varför} ‘why’ in the two sentence types. I will argue that the syntactic similarities and differences between Wh-RIs and Wh-finite\is{finiteness} clauses follow from the structural differences and similarities of \REF{ex:platzack:16} and \REF{ex:platzack:18}.

\subsection{Word order}\label{sec:platzack:4.2}

As seen above, both the finite\is{finiteness} sentence and the root \isi{infinitive} may be introduced by a wh-phrase with an adverbial function:

\ea%19
    \label{ex:platzack:19}
	  \ili{Swedish}\\
   \ea
\gll   Varför  stäng-er  Lisa  dörr-en?        \\
          why    shut-\textsc{pres} Lisa  door-the\\
    \glt   ‘Why does Lisa shut the door?’

   \ex
\gll   Varför  stäng-a  dörren?\\
          why    shut-\textsc{inf}  door-the\\
    \glt   ‘Why shut the door?'
\z
\z

Only finite\is{finiteness} sentences may be productively introduced by an argumental wh-phrase: 

\ea%20
    \label{ex:platzack:20}
	  \ili{Swedish}   \\
   \ea
\gll Vem  välj-er    vi    denna gång?\\
        who  vote-\textsc{pres}  we    this  time\\
  \glt   ‘Who do we vote for this time?’

   \ex \label{ex:platzack:20b}
\gll \#Vem    välj-a    denna  gång? \\
         who   vote-\textsc{inf}    this    time\\
  \glt   ‘Who vote for this time?’ \citep[IV: 827]{TelemanEtAl1999}
\z
\z

The finite\is{finiteness} verb precedes sentence adverbs and the \isi{negation}, the infinitival verb follows sentence adverbs and the \isi{negation}:

\ea%21
    \label{ex:platzack:21}
	  \ili{Swedish}  \\
   \ea
\gll Varför  stäng-er    Lisa  inte  dörr-en?\\
        why    shut-\textsc{pres}  Lisa   not  door-the\\
  \glt   ‘Why doesn’t Lisa shut the door?’

   \ex
\gll *Varför  stäng-a  inte  dörr-en?\\
          why    shut-\textsc{inf}  not  door-the\\
    \glt   ‘Why not shut the door?’

   \ex \label{ex:platzack:21c}
\gll Varför  inte  stänga  dörr-en?\\
        why    not  shut-\textsc{inf}  door-the\\
  \glt   ‘Why not shut the door?’
\z
\z

The differences and similarities listed in (\ref{ex:platzack:19}--\ref{ex:platzack:21}) are all related to the C-domain. Comparing with the analysis of the finite\is{finiteness} main clause in section 3, the presence of a wh-phrase in first position, more precisely in Spec,CP\is{complementizer}, is accounted for by the presence of an unvalued wh-feature in C, [¬wh\textsuperscript{EPP}], which forces a wh-phrase to take first position. Since also the wh-\isi{infinitive} begins with a wh-phrase, the feature [¬wh\textsuperscript{EPP}] is supposed to be present in the infinitival C as well as in the finite\is{finiteness} C. The specific mechanism that in the absence of a wh-word allows almost any phrase to be fronted in \ili{Swedish} finite\is{finiteness} main clauses (\isi{verb second}) is not present in \ili{Swedish} root infinitives, however. 

In Wh-RIs, as we see in \REF{ex:platzack:20b}, the choice of initial wh-phrase is restricted in wh-infinitives but not in finite\is{finiteness} main clauses. Descriptively, we can account for this restriction if we limit the range of the feature wh in C in wh-infinitives to be only sentence adverbials, roughly adverbials being first merged as a high adjunct to vP. This is more or less the area that \citet{Cinque1999} claims consists of functional projections hierarchically ordered in the same way in all of the world’s languages. I will use the notation wh\textsubscript{vP} to remind the reader of this limitation, claiming that C in wh-infinitives is merged with the feature bundle [¬wh\textsubscript{vP}\textsuperscript{EPP}] and that the agree-relation that is established, see \REF{ex:platzack:9}, will probe the closest goal in vP, supposed to be \textit{varför} ‘why’.

Whereas the wh-feature in C has an \isi{EPP} feature both in finite\is{finiteness} main clauses and in wh-infinitives, there is a difference with respect to the other features in C. In \sectref{sec:platzack:3} we concluded that the C of the \isi{tensed} main clause in \ili{Swedish} hosts an unvalued finiteness feature with \isi{EPP}, [¬fin\textsuperscript{EPP}], forcing V2\is{verb second}. There is no finiteness feature associated with wh-infinitives, but an infinitival feature is associated with the \isi{infinitive} inflection in little v, see \REF{ex:platzack:21c}. 

As (\ref{ex:platzack:21}b,c) indicate, the infinitival verb does not seem to move away from little v, as the finite\is{finiteness} verb does, hence there is no indication of an \isi{EPP} feature associated with the infinitival feature. In other respects the two inflectional features [fin] and [inf] are syntactically used in the same way: they are, for example both merged in v as valued, and in C as unvalued. The unvalued versions of the feature for the two cases are summarized in \REF{ex:platzack:22}:

\ea%22
\label{ex:platzack:22}
  \ea C in the finite\is{finiteness} sentence type contains the unvalued feature [¬fin\textsuperscript{EPP}]\\
  \ex C in the \isi{infinitive} sentence type contains the unvalued feature [¬inf]]\\
  \z
\z

Both finite\is{finiteness} and infinite C will probe its c-command domain, valuing the feature in C. Like the finite\is{finiteness} sentence type, we conclude that the Wh.-RIs contain both CP\is{complementizer} and vP. The main difference is the absence of a T-\isi{projection} between C and v.

\subsection{The external argument}

The external argument, less precisely the subject argument, is visible in \ili{Swedish} finite\is{finiteness} sentences, but not in \isi{infinitive} sentences. 

 \ea\label{ex:platzack:23} 
 \ili{Swedish}  \\
   \ea
\gll Varför  stäng-er    han  fönstr-et?\\
        why    shut-\textsc{pres}  he    window-the\\
  \glt   ‘Why does he shut the window?’

   \ex
\gll Varför  stäng-a   (*han)  fönstr-et?\\
        why    shut-\textsc{inf}    he    window-the\\
  \glt   ‘Why shut the window?’
\z
\z

In the finite\is{finiteness} sentence type introduced by \textit{varför} ‘why’, the subject appears in Spec,\isi{TP}, after the finite\is{finiteness} verb in C. See the analysis in \REF{ex:platzack:16}. The subject is visible in this position, due to \isi{EPP} associated with the Agree\is{agreement} relation between the unvalued [$\varphi $]-feature in T and the valued [$\varphi $]-feature in Spec,vP. The absence of a visible DP in the \isi{infinitive} case indicates that there is no \isi{TP} present to establish an agree relation between T and the external argument in this sentence type, as I argued above.

   An alternative possibility would be to assume that no DP is merged to vP, but there is indirect evidence against such an analysis and in favour of the analysis which assumes the presence of an invisible DP in Spec,vP of the \isi{infinitive} sentence, with the same theta-role as the visible subject in the finite\is{finiteness} sentence. As the examples in \REF{ex:platzack:24} and \REF{ex:platzack:25} show, the invisible DP in \REF{ex:platzack:24}, like the visible counterpart in \REF{ex:platzack:25}, may bind an anaphoric pronoun, or the \isi{possessive} reflexive \textit{sin}, and it agrees with a predicative adjective:

\ea%24
    \label{ex:platzack:24}
	  \ili{Swedish} \isi{infinitive} clauses\\
   \ea
\gll Varför   gömm-a     sig     under     säng-en?\\
        why    hide-\textsc{inf}    \textsc{refl}  under    bed-the\\
  \glt   ‘Why hide  under the bed?’    

\ex  
\gll Varför   gömm-a   sin bok    under sängen?\\
    why    hide-\textsc{inf}  \textsc{refl} book   under   bed-the\\
  \glt   ‘Why hide his book under the bed?'

   \ex
\gll Varför  komm-a     full-ø    /  full-a    till fest-en?\\
        why    come-\textsc{inf}  drunk-\textsc{sg} /  drunk-\textsc{pl}   to  party-the\\
  \glt   ‘Why come drunk to the party?’
\z
\z

\ea%25
    \label{ex:platzack:25}
	  \ili{Swedish} finite\is{finiteness} clauses\\
     \ea
\gll Varför   gömm-er  han sig  under     säng-en?\\
        why    hide-\textsc{pres}  he   \textsc{refl}  under    bed-the\\
  \glt   ‘Why does he hide  under the bed?’    

\ex  
\gll Varför   gömm-er               han   sin bok      unde säng-en?\\
        why    hide-\textsc{pres}  he    {\textsc{poss}.\textsc{refl}}   book under   bed-the\\
  \glt   ‘Why hide his book under the bed?'

   \ex
\gll Varför  komm-er   han   full-ø /    de   full-a      till fest-en?\\
        why    come-\textsc{pres}    he drunk-\textsc{sg}  / they   drunk-\textsc{pl}   to  party-the\\
  \glt   ‘Why does he / they come drunk to the party?’
\z
\z


Hence the binding and predicative \isi{agreement} facts support the analysis that there is an invisible DP in Spec.vP.

From a syntactic point of view there seems to be no reason to expect anything else than a symmetrical distribution of \textit{varför} in both finite\is{finiteness} and infinitival clauses. Thus, if we find a well-formed \isi{infinitive} sentence introduced by \textit{varför} ‘why’, we predict the existence of a corresponding  finite\is{finiteness} sentence introduced by \textit{varför} ‘why’, and vice versa. So far, none of the \ili{Swedish} examples given seems to violate this prediction; here I give another two examples of the same kind.

\ea%26
    \label{ex:platzack:26}
	  \ili{Swedish}\\

\ea
\gll Varför  frukta-r    hon  hundar? \\
      why    fear-\textsc{pres}  she  dogs\\
\glt   ‘Why does she fear dogs?’

   \ex
\gll Varför  frukt-a  hundar?\\
      why    fear-\textsc{inf}  dogs\\
\glt   ‘Why   fear dogs?’
\z

\z

\ea%27
    \label{ex:platzack:27}
	  \ili{Swedish}\\
   \ea
\gll Varför   sprang  Erik  till   affär-en?\\
      why    run.\textsc{past}  Erik  to   shop-the\\
\glt   ‘Why did Erik run to the shop?’

   \ex
\gll Varför  spring-a  till    affär-en?!\\
      why    run-\textsc{inf}  to    shop-the\\
\glt   ‘Why run to the shop?’  
\z
\z

So far, the prediction seems to hold. However, notice that the invisible subject cannot be some arbitrary 3\textsuperscript{rd} \isi{person} feminine in \REF{ex:platzack:26} or invisible \textit{Erik} in \REF{ex:platzack:27}, but must represent the \isi{person} spoken to (an invisible \textit{you}). Thus, there is an interpretative difference between the finite\is{finiteness} and the infinitival sentence types. As can be seen, this interpretative difference also shows up in cases like \REF{ex:platzack:28} and \REF{ex:platzack:29}, where the “spoken-to” interpretation of the invisible DP in Spec.vP is not available, and hence there is no infinitival correspondence: 

\ea \label{ex:platzack:28}
\ili{Swedish} \\
   \ea
\gll Varför  dog    han  efter  operation-en?\\
      why    die.\textsc{past}  he    after  operation-the\\
\glt   ‘Why   did he die after the operation?’

   \ex
\gll *Varför  dö      efter  operation-en!?\\
      why    die.\textsc{inf}  after  operation-the\\
\z
\z
      
\ea%29
  \label{ex:platzack:29}
	 \ea
\gll Varför  sjönk      fartyg-et  snabbt?\\
      why    sink.\textsc{past}  ship-the  fast\\
\glt   ‘Why did the ship sink fast?’

   \ex
\gll *Varför  sjunk-a  snabbt?!\\
      why    sink-\textsc{inf}  fast  \\
\z
\z

To understand why there is no well-formed infinitival correspondent to the well form\-ed finite\is{finiteness} sentences in the a-examples in \REF{ex:platzack:28} and \REF{ex:platzack:29}, and why there is an interpretation restriction in \REF{ex:platzack:26} and \REF{ex:platzack:27}, I will turn to an observation by \citet[186]{Reis2003}. Reis notices with respect to the subject of Wh-RIs, that “[n]o matter how we represent the silent subject argument in RIs in syntax, whether by PRO or nothing (that is, by just suppressing the respective argument variable), one thing is clear: In order for RIs to receive a sensible \isi{utterance} interpretation, the subject \isi{reference} must be specified.” As \citet[186]{Reis2003} notices  “[t]he possible candidates [-{}-{}-] are limited to the participants in the \isi{utterance} situation: speaker(s) and addressees.” A closer look at the ungrammatical b-examples in (\ref{ex:platzack:28}--\ref{ex:platzack:29}), reveals that in these cases, the subject \isi{reference} cannot be a participant in the \isi{utterance} situation, whereas in all the well formed cases it can.

\subsection{A-bar movement of \textit{varför}  ‘why’}

As we saw in the last section, Wh.-RIs do not seem to allow \isi{A-movement}, whereas finite\is{finiteness} sentences do. \isi{A-bar movement}, on the other hand, is found in both sentence types, in the \isi{infinitive} one only in form of Wh-movement. Consider the finite\is{finiteness} sentence in \REF{ex:platzack:30}.

\ea%30
    \label{ex:platzack:30}
	  \ili{Swedish}  \\
\gll Varför  sa        du    att    Johan  skrev      brev-et?\\
 why    say.\textsc{past}  you  that  Johan    write.\textsc{past}  letter-the	\\
\glt ‘Why did you say that Johan wrote the letter?’
\z

This sentence is ambiguous: the speaker is either asking why the subject of the main clause said something (the matrix reading), or why the subject of the embedded clause wrote the letter (the embedded reading). See   \citet{ShlonskySoare2011} and \citet{Simik2006}. Since the wh-word is in the same position in both cases, we must assume that \textit{varför} has moved from its position in the embedded clause to the edge of the matrix clause (Spec,CP\is{complementizer}), supporting the analysis in \sectref{sec:platzack:3} above. Notice that \textit{varför} in this case is first merged in a finite\is{finiteness} domain and has moved to a position within a finite\is{finiteness} domain (Spec,CP\is{complementizer} of the matrix).

The corresponding Wh-RI is \REF{ex:platzack:31}:

\ea%31
    \label{ex:platzack:31}
\gll Varför  säga    att    Johan  skrev    brev-et?!\\
  why    say.\textsc{inf}  that  Johan    wrote    letter-the	\\
\glt    ‘Why say that Johan wrote the letter?!’
\z 

Contrary to the finite\is{finiteness} clause in \REF{ex:platzack:30}, example \REF{ex:platzack:31} only displays the matrix reading; hence in both the finite\is{finiteness} sentence matrix and the infinitival one, one option is that \textit{varför} is merged to matrix vP and moved to matrix Spec,CP\is{complementizer} of the \isi{infinitive} sentence. With regard to the embedded reading, \textit{varför} is first merged to a high Spec,vP inside a finite\is{finiteness} domain, i.e. the embedded \textit{att}{}-clause. Obviously, movement out of this domain to a position in the matrix \isi{infinitive} domain is not allowed. In the sentence with an \isi{infinitive} matrix, \textit{varför}, when extracted, must move out of a finite\is{finiteness} domain and into an \isi{infinitive} domain, which presumably is not allowed. There is no corresponding switch of domains in finite\is{finiteness} sentences like \REF{ex:platzack:30}, hence \isi{extraction} of \textit{varför} from an embedded clause to the matrix one is only possible in finite\is{finiteness} sentences.

\section{Summary and conclusion}\label{sec:platzack:5}

As mentioned in the introduction, the purpose of my paper has been to present a narrow syntactic account of the \ili{Swedish} Wh-Root-infinitives, trying to characterize this often neglected main clause equivalent while comparing it with finite\is{finiteness} root clauses. My account is based on the hypotheses that C minimally hosts the sentence type features finite\is{finiteness}, imperative and \isi{infinitive}, in addition to an edge feature \citep[11 f]{Chomsky2007}. More precisely, I have tried to show that the \ili{Swedish} facts follow nicely from a grammar driven by an asymmetry with respect to features, which come in two guises, valued and unvalued, and that all unvalued features must be eliminated before the semantic\is{semantics} and \isi{pragmatic} interfaces are reached.

Working mainly with one language might be seen as a drawback: after all, our understanding of syntax has improved tremendously over the last 30 years, much as a result of us having a theory that may take variation (especially on a macro-level) into account. However, it may also be a problem for our field of research that we are always ready to take other languages into consideration before we are confident that the machinery we have at our disposal can handle at least one human language. Thirty years of studying macro-variation have taught us that the beautiful generalizations we found initially (the parameters), see e.g. \citet{HolmbergPlatzack1995}, usually fade away under a closer study. And we should not forget that each natural language is a possible outcome of \isi{Universal Grammar}. Therefore, what I have presented in this paper can be seen as the basis for comparative syntactic studies. The outcome of a detailed study of a certain part of a single language will result in lists of properties, which, when used in the computation, predicts particular properties of the language studied. Any change that we are forced to make with respect to the theoretical apparatus that is motivated by a careful description of a single language when we are describing another language from the same perspective must be evaluated both with respect to the basic account of the data and possible accounts we have of other languages. 

\section*{Acknowledgements}

Thanks to two anonymous referees for a number of  helpful and ingenious comments, only a handful of which it has been possible for me to take into consideration. I am also grateful to the editors Laura Bailey and Michelle Sheehan for additional help. A special thank you to professor emerita Inger Rosengren; without her enthusiasm and encouragement, I would never have thought about Wh-RIs in the first place. Thanks also to Halldór Sigurðsson for providing me with some {Icelandic} facts. I take full responsibility for all remaining faults and shortcomings.
{\sloppy
\printbibliography[heading=subbibliography,notkeyword=this]
}



\end{document}