\title{Reflections on language evolution}
\subtitle{From minimalism to pluralism}
\author{Cedric Boeckx}
\renewcommand{\lsSeries}{cfls}
\renewcommand{\lsSeriesNumber}{6}
\BackBody{This essay reflects on the fact that as we learn more about the biological underpinnings of our language faculty, the dominant evolutionary narrative coming out of the linguistic tradition most explicitly oriented towards biology ("biolinguistics") appears increasingly implausible. This text offers ways of opening up linguistic inquiry and fostering interdisciplinarity, taking advantage of new opportunities to provide quantitative, testable hypotheses concerning the complex evolutionary path that led to the modern human language faculty.


The essay is structured around three main themes:
(i) renewed appreciation for the comparative method applied to cognitive questions, leading to the identification of elementary but fundamental abstractions in non-linguistic species relevant to language;
(ii) awareness of the conceptual gaps between disciplines, and the need to carefully link genotype and phenotype without bypassing any ``intermediate'' levels of description (certainly not the brain); and
(iii) adoption of a ``philosophical'' outlook that puts the complexity of biological entities front and center.
}
\BookDOI{10.5281/zenodo.5524633}
\renewcommand{\lsISBNdigital}{978-3-96110-328-7}
\renewcommand{\lsISBNhardcover}{978-3-98554-024-2}
\renewcommand{\lsID}{142}
\proofreader{Jeroen van de Weijer, Lachlan Mackenzie}
