\chapter{Another ``bio-linguistics'' is possible}
One need not be an expert in word formation to figure out that prefixes alter meaning. A post-doctoral position is not the same thing as a doctoral position; a preview is not the same thing as a view; to outplay is not the same thing as to play. The addition of a prefix matters. Accordingly, one should expect some added meaning value when we see the prefix ``bio-'' applied to terms proper to the language sciences, including the very notion of ``bio-linguistics". Yet, I do not think that the theoretical linguistics literature that professes a biological orientation and waves the banner of ``biolinguistics" offers a fair reflection of the difficulty one quickly encounters when one tries to do genuine interdisciplinary work combining  ``bio" and linguistics. If the biological commitment of the field were really taken seriously, many linguistics papers would look quite different. There would be a lot more constructive discussion about the brain, there would be a lot more serious talk about other species, and there would also be a lot less about physical laws in language design, as well as far fewer hand-waving remarks about the ``genetic endowment".

While no one can seriously doubt that there is something about our biology that makes it possible for us to develop and use grammatical systems we call natural languages, there is no denying that the generative tradition explored a specific way of characterizing this ``biological endowment'': a rich set of domain-specific cognitive properties that together form what is widely known as ``Universal Grammar'' (UG). This is where the controversy really arises. Lots of scientists object to this characterization of the biological endowment. They favor domain-general solutions, and differ in the degree to which they see learning and environmental interactions shaping the mature linguistic knowledge in humans. In my view, to the extent one is interested in characterizing this biological endowment, one is doing biolinguistics. One need not be an orthodox UG advocate to be a biolinguist. I mean this in two ways: first, one can be a biolinguist even if one rejects the existence of domain-specific cognitive primitives in the language domain. Second, the mere fact of appealing to UG to account for certain facts about our linguistic knowledge does not make one a biolinguist.

There has never been a better time to focus on this biological endowment, given the amount of relevant data currently waiting to be confronted, tested, and interpreted. Linguists ought to play a much more active role in this enterprise, if only to preserve the import of the insights of the cognitive revolution of the 1950s.

To my mind, Eric Lenneberg, who did so much to get the field of ``biolinguistics" off the ground, put it best, when he wrote: ``Nothing is gained by labeling the propensity for language as biological unless we can use this insight for new research directions — unless more specific correlates can be uncovered" \citep{Lenneberg1967biological}.

In the preface of \textit{Biological foundations of language}, Lenneberg expressed his feeling that biology had been ``badly neglected" in language studies. I think this is still true today, even among those who appeal to biology in the introductory remarks to their works. In the same preface Lenneberg refers to Meader and Muyskens's \textit{Handbook of biolinguistics} \citep{meader1950handbook} but points out that he was aiming at a ``distinct theoretical position" from the one found in that work. As a result, he did not endorse the term ``biolinguistics'' there, to avoid confusion. Perhaps he still would resist the term now, as I have come to do, in light of the way it is used by linguists who make so little contact with data generated by biologists (for further discussion and relevant quotes, see \cite{martins2016we}). I suspect Lenneberg would prefer a term like ``cognitive biology of language'' to describe a discipline where the formal nature of language is recognized (as it was in the appendix to \cite{Lenneberg1967biological}, authored by Chomsky), as it must be if reductive biases are to be avoided, but necessarily translated in ways that admit empirical tests of the sort (other) biologists perform. For, like every interdisciplinary enterprise, ``biolinguistics" is both a goldmine (lots of new opportunities and ``low-hanging fruit'') and a minefield (ideas lost in translation).

Doing biolinguistics means inhabiting an ``interfield''. It means being willing to sit between a rock (biologists' naïve notions of language) and a hard place (linguists' naïve notions of biology). It means building bridges. Bridges for ideas to travel on. As everyone knows, bridge-builders have to work as part of a team; they cannot do it on their own. In addition to figuring out which material to use for the bridge, they have to become deeply acquainted with the nature of the soil of both sides to be united; they have to have an understanding of the landscape, and the flow of traffic around the areas that will be united by the bridge. In this sense, bridge-builders are a bit like translators, who have to know more than one language, but also have to familiarize themselves with the cultures these languages are spoken in. Biolinguists are like that, too. They have to link, and therefore know two fields. They may not need to know everything about both fields. But they must know enough to convey messages back and forth, and make ideas flow in both directions.

Like translators, ``biolinguists'' may always be recognizable by their accents when they move in a culture that is not originally theirs, but the translation exercise is something fundamental, no matter how imperfect it may appear to the natives. Here is what Tim Parks has to say about the added value of translation in his essay ``Gained in Translation"):\footnote{\url{https://www.nybooks.com/daily/2017/12/09/gained-in-translation}}
\begin{quote}
Translators are people who read books for us. Tolstoy wrote in Russian, so someone must read him for us and then write down that reading in our language. Since the book will be fuller and richer the more experience a reader brings to it, we would want our translator, as he or she reads, to be aware of as much as possible, aware of cultural references, aware of lexical patterns, aware of geographical setting and historical moment. Aware, too, of our own language and its many resources. Far from being ``just subjective", these differences will be a function of the different experiences these readers bring to the book, since none of us accumulates the same experience. Even then, of course, two expert translators will very likely produce two quite different versions. But if what we want is a translation of Tolstoy, rather than just something that sounds good enough sentence by sentence, it would seem preferable to have our reading done for us by people who can bring more, rather than less, to the work.
\end{quote}

Lenneberg gave us the seeds of an alternative, richer, bio-linguistic program. Today, such seeds find much more fertile ground than they did fifty years ago. Linguists would be wrong to let others reap the fruits just by keeping their methodological blinders on. If we believe that the target of linguistic theorizing is ``ultimately biology", there is no alternative to going there, and doing some actual biology. What was a logical problem (``Darwin's [Problem]") must find a biological (Darwinian) solution.

