\documentclass[output=paper,
            colorlinks, citecolor=brown
            % ,draft
            ,draftmode
		  ]{langscibook}
\ChapterDOI{10.5281/zenodo.10663779}
		  
\author{Malin Petzell\orcid{0000-0002-0774-5131  }\affiliation{University of Gothenburg} and
        Peter Edelsten\orcid{}\affiliation{SOAS University of London}}

\title[Tense and aspect marking in Bantu languages of Morogoro]
      {Tense and aspect marking in Bantu languages of the Morogoro region, Tanzania}

\abstract{A review of the verbal morphology of several Bantu languages of the Morogoro region, Tanzania, reveals surprising diversity in both their distribution and meaning. Bantu languages are renowned for their rich verbal morphology, including remoteness distinctions in the tenses. However, some Bantu languages of the region have essentially only two tenses (past and non-past), limited aspectual distinctions, and some have no negative tense/aspect markers. This chapter summarises our current knowledge of the tense/aspect systems of five Bantu languages of the Morogoro region: Kagulu (G12), Luguru (G35), Kami (G36), Ndamba (G52) and Pogoro (G51). In particular, the chapter reviews the distribution and meaning of these morphological distinctions, the abundance versus scarcity of specific tense/aspect markers, and the methods of expressing negation.
\keywords{Tense/aspect, verbal morphology, Bantu, Morogoro, negation}
}

\IfFileExists{../localcommands.tex}{
  \addbibresource{../localbibliography.bib}
  \usepackage{langsci-optional}
\usepackage{langsci-gb4e}
\usepackage{langsci-lgr}

\usepackage{listings}
\lstset{basicstyle=\ttfamily,tabsize=2,breaklines=true}

%added by author
% \usepackage{tipa}
\usepackage{multirow}
\graphicspath{{figures/}}
\usepackage{langsci-branding}

  
\newcommand{\sent}{\enumsentence}
\newcommand{\sents}{\eenumsentence}
\let\citeasnoun\citet

\renewcommand{\lsCoverTitleFont}[1]{\sffamily\addfontfeatures{Scale=MatchUppercase}\fontsize{44pt}{16mm}\selectfont #1}
   
  %% hyphenation points for line breaks
%% Normally, automatic hyphenation in LaTeX is very good
%% If a word is mis-hyphenated, add it to this file
%%
%% add information to TeX file before \begin{document} with:
%% %% hyphenation points for line breaks
%% Normally, automatic hyphenation in LaTeX is very good
%% If a word is mis-hyphenated, add it to this file
%%
%% add information to TeX file before \begin{document} with:
%% %% hyphenation points for line breaks
%% Normally, automatic hyphenation in LaTeX is very good
%% If a word is mis-hyphenated, add it to this file
%%
%% add information to TeX file before \begin{document} with:
%% \include{localhyphenation}
\hyphenation{
affri-ca-te
affri-ca-tes
an-no-tated
com-ple-ments
com-po-si-tio-na-li-ty
non-com-po-si-tio-na-li-ty
Gon-zá-lez
out-side
Ri-chárd
se-man-tics
STREU-SLE
Tie-de-mann
}
\hyphenation{
affri-ca-te
affri-ca-tes
an-no-tated
com-ple-ments
com-po-si-tio-na-li-ty
non-com-po-si-tio-na-li-ty
Gon-zá-lez
out-side
Ri-chárd
se-man-tics
STREU-SLE
Tie-de-mann
}
\hyphenation{
affri-ca-te
affri-ca-tes
an-no-tated
com-ple-ments
com-po-si-tio-na-li-ty
non-com-po-si-tio-na-li-ty
Gon-zá-lez
out-side
Ri-chárd
se-man-tics
STREU-SLE
Tie-de-mann
} 
  \togglepaper[1]%%chapternumber
}{}

\begin{document}
\maketitle 
%\shorttitlerunninghead{}%%use this for an abridged title in the page headers



\section{Introduction}\label{sec:petzell:1}
\subsection{Background}\label{sec:petzell:1.1}

This chapter provides an analysis of the tense/aspect systems of five selected Bantu languages of the Morogoro region, Tanzania. The Morogoro region spreads from the area north of Morogoro town to the southern part of the Kilombero valley. Tanzania has about 100 Bantu languages (\citealt{MahoSands2002}), most of them being poorly documented. Of the 100 Bantu languages spoken in Tanzania, 10 are spoken mainly in the Morogoro region:
\newpage

\begin{itemize}
\item 
Kagulu (G12)

\item 
Ngulu (G34)

\item 
Luguru (G35)

\item 
Kami (G36)

\item 
Kutu (G37)

\item 
Vidunda (G38)

\item 
Pogoro (G51)

\item 
Ndamba (G52)

\item 
Sagala (G61)

\item 
Mbunga (P15)

\end{itemize}

In addition, other languages which are spoken primarily in neighbouring regions but which have a significant presence in Morogoro region include:


\begin{itemize}
\item 
Zigula (G31)

\item 
Kwere (G32)

\item 
Zaramo (G33)

\item 
Hehe (G62)

\item 
Bena (G63)

\item 
Ngoni (N12)

\item 
Ngindo (P14)

\end{itemize}

Additionally, Swahili (G42) is spoken all over the country, and all consultants in this study are bilingual in Swahili. 



Bantu languages are known for their rich verbal morphology, including elaborate sets of tense/aspect markers. The five chosen languages, although fairly closely related, show variation not only in the number of markers but also in their function. We will describe and analyse the tense/aspect marking in these languages based on models of Bantu verbal morphology, including tense/aspect, by \citet{Meeussen1967} and \citet{Guthrie1967-1971}. More recently, \citet{Nurse2008} and \citet{NurseDevos2019} present a review of tense/aspect data for 100 Bantu languages from across the entire Bantu language area, providing an analysis of the main patterns found and some proposals for their diachronic evolution, which we will also refer to.



The context and rationale for the present study was that both authors had been studying Bantu languages in different parts of the Morogoro region, and they decided to attempt to synthesise their results in one significant area of Bantu grammar: tense/aspect. The expectation was that, given their close proximity, the tense/aspect systems of the selected languages would show some similarities. In fact, they show a surprising amount of diversity.


\subsection{The languages selected for the study}\label{sec:petzell:1.2}

The five languages analysed in this study were selected on the basis that they are distributed across the region and that they might therefore be expected to expose variation in structures found throughout the region. The approximate locations of the five selected languages, Kagulu (G12), Luguru (G35), Kami (G36), Ndamba (G52) and Pogoro (G51), are shown in \figref{fig:petzell:1}.



\begin{figure}
\includegraphics[width=\textwidth]{figures/petzell-img001.jpg}
\caption{Languages of the study.  Data sourced from ©~openstreetmap.org}
\label{fig:petzell:1}
\end{figure}

In this chapter, all data in the examples are derived from the authors' fieldwork unless otherwise stated.



Kagulu (G12) is a Bantu language spoken in and around the Kagulu or \textit{Itumba} mountains in the north-west of the region. The language is estimated to have between 240,000 \citep{Petzell2008} and 336,000 speakers (\citealt{LanguagesofTanzaniaProject2009}). Some speakers use the autonym \textit{Chimegi} to refer to their language, while others prefer \textit{Chikagulu,} since \textit{Megi} is a derogatory term used by Maasai speakers, meaning `non-Maasai' \citep[251]{Mol1996}. The most prestigious Kagulu dialect stems from the mountains and is referred to as (\textit{Chi})\textit{Tumba}. Data are sourced from \citet{Petzell2008}, supplemented from the authors' more recent fieldwork (2009--2020).



Luguru (G35) is a Bantu language spoken in the Luguru mountains south of Morogoro town. It is reported to have 400,000 speakers \citep{LanguagesofTanzaniaProject2009} and it is a dominant language in the region. Data are sourced from \citet{Mkude1974} and \citet{Petzell2020}, supplemented from the authors' fieldwork.



\citet{Mkude1974} identifies two dialects of Luguru (highland and lowland), which are not well documented. An MA thesis \citep{Moses2018} questions this division and reaches the conclusion that there are indeed different dialects of Luguru, but that the division between highland and lowland is not clear \citep[66]{Moses2018}. The dialects, which are mutually intelligible, are somewhat different in phonology and lexicon \citep{Moses2018}. 



Kami (G36) is a highly endangered Bantu language spoken around Mikese, east of Morogoro town. It is reported to have only 5000 speakers \citep{LanguagesofTanzaniaProject2009}. This figure refers to the number of persons who consider themselves to be Kami speakers, but it does not say anything about the competence of those speakers. There are significantly fewer than 5000 fluent speakers left, which was corroborated during field trips in the area. The youngest consultant we found was in his thirties, and he could understand but not speak Kami, which indicates that the language is not being transmitted to the next generation. Data are sourced from \citet{PetzellAunio2019}, supplemented by the authors' fieldwork.



Pogoro (G51) is a Bantu language spoken in the Pogoro Mountains in the south-east of the region. It is estimated to have 200,000 speakers \citep{LanguagesofTanzaniaProject2009}. Data are sourced from \citet{Hendle1907}, supplemented from the authors' fieldwork. Given the age of Hendle's work, his main conclusions about morphosyntax seem to correlate remarkably well with data collected recently, over 100 years later. Less clear is the current validity of the translation of many of the words in the word list, but this may be as much to do with the evolution of their semantics in their German translations as in the original Pogoro.



Ndamba (G52) is a Bantu language spoken in the Kilombero Valley in the south-west of the region. It is estimated to have between 55,000 \citep{Lewis2009} and 196,000 speakers \citep{LanguagesofTanzaniaProject2009}. Data are sourced from \citet{EdelstenLijongwa2010}, supplemented from the authors' fieldwork.



The variant of Ndamba documented by \citet{Novotna2005} shows some differences from that documented by \citet{EdelstenLijongwa2010}. In particular, Novotna describes phonological features such as verb final -\textit{i} and the loss of inflectional future tenses. These differences may show an influence from Pogoro, which may have contributed to Ndamba and Pogoro being grouped together by \citet{Guthrie1948} as the G50 group of languages, and to the comment by \citet[Appendix 1, p. 177]{Nurse2008} that “G51 and G52 are quite similar".



\citegen{EdelstenLijongwa2010} data, however, point towards Ndamba being somewhat more distinct from Pogoro, with complex verbal tense/aspect morphology, as discussed in \sectref{sec:petzell:3.5}, more reminiscent of neighbouring G60 languages such as Bena \citep{Morrison2011} and Hehe (\citealt[Appendix 1, pp. 178--180]{Nurse2008}).


\subsection{Structure of the chapter}\label{sec:petzell:1.3}

Following this introductory section, \sectref{sec:petzell:2} discusses the verbal template used for the analysis, and how tense/aspect and related morphemes fit into the template in the selected languages. The objective is to provide a consistent basis for comparing the morphological structure of verbs in the languages of the study, while at the same time reviewing whether the generally accepted template proposed by \citet{Meeussen1967}, as amended by \citet{Nurse2008}, is consistent with our template.



This is followed in \sectref{sec:petzell:3} with a discussion of tense/aspect in Bantu languages in general, followed by subsections for each of the five selected languages.



\sectref{sec:petzell:4} discusses related verbal categories in the selected languages, including imperative, subjunctive, conditional, habitual and negative, followed by a final section which draws conclusions from the analysis and provides suggestions for further research.


\section{The verbal template}\label{sec:petzell:2}
\largerpage
Bantu languages are often analysed as using morphological verbal templates, into which various affixes fit (\citealt{Meeussen1967, Nurse2008}). One of the reasons for using a template is to show how the affixes concatenate, since the order of affixes is typically strict. The ordering of syntactic elements, on the other hand, is typically much less restricted.



The exact specification of the template slots varies across Bantu languages, but the five selected languages show some uniformity. To compare the verbal morphology of the five selected languages, the template shown in \tabref{tab:petzell:1}, based on \citet[108--111]{Meeussen1967}, is used in this chapter.


\begin{table}

\begin{tabular}{l>{\scshape}l} 
\lsptoprule
 {Template slot} & {\normalfont Abbreviated to}\\
\midrule
Pre-subject marker         & pre.sm\\
 Subject marker            & sm\\
 Post-subject marker       & post.sm\\
 First tense/aspect marker & ta1\\
 Object marker             & om\\
 Verb root                 & root\\
 Extensions                & ext\\
Second tense/aspect marker & ta2\\
 Passive suffix            & pass\\
 Final vowel               & fv\\
Post-final marker          & post.fm\\
\lspbottomrule
\end{tabular}
\caption{The verbal template}
\label{tab:petzell:1}
\end{table}

Meeussen distinguishes two tense/aspect slots, ``formative" and ``limitive", occurring before the object marker slot, but in \tabref{tab:petzell:1}, we have combined them in slot 4, as does \citet[40]{Nurse2008}.



\citet[40]{Nurse2008} also combines TA2 with FV, such that FV then assumes a complex multi-morpheme role. For clarity, we have kept these slots separate. This is discussed further in \sectref{sec:petzell:2.10}.



\REF{ex:petzell:1} shows the use of the verbal template slots with data from Ndamba.\footnote{ Most Bantu languages are tonal (\citealt{MarloOdden2019}). However, none of the languages selected for the study employs grammatical or lexical tone, and none of the examples in this chapter are therefore marked for tone.}


\ea\label{ex:petzell:1}Ndembo a-ka-mu-somol-el-ile ngwena lwimbo.    \hfill      {(Ndamba)}\\
`The elephant sang the crocodile a song.'
\z





from which \textit{a-ka-mu-somol-el-ile} may be analysed as:



\ea \glll {morphemes:}  a-    ka-    mu-  somol-  el-    ile\\
{slots:} \textsc{sm-}  \textsc{ta1}{{}-} \textsc{om-}  \textsc{root-}    \textsc{ext-}  \textsc{ta2}\\
{gloss:} \textsc{sm}{1-} \textsc{pst}{{}-} \textsc{om}{1-}  \textsc{sing-} \textsc{appl}{{}-} \textsc{pfv}\\
\glt          {}`s/he sang him/her a song'
\z



\tabref{tab:petzell:2} compares how the template slots are used in the five selected languages. Details and examples of how each of the slots is used are described in subsequent sections. The table shows that the way these slots are used is more varied than expected, given the proximity of the selected languages; this is discussed further in \sectref{sec:petzell:3} and \sectref{sec:petzell:4}.


\begin{sidewaystable}
\small
\begin{tabularx}{\textwidth}{lQQQQQQ}

\lsptoprule

&{Slot} & {Kagulu} & {Kami} & {Luguru} & {Ndamba} & {Pogoro}\\
\midrule
1& Pre-subject marker & relative object marker \textit{{}-o-}

temporal \textit{fo-}

negative marker 1

tense/aspect marker & negative marker

temporal \textit{fi-} & temporal \textit{{}-(h)a-}

relative object marker

negative marker & conditional/ temporal \textit{pa-} & tense/aspect marker

negative marker \textit{na-}

relative object marker

temporal \textit{pa-}\\
2& Subject marker & subject marker & subject marker & subject marker & subject marker & subject marker\\
3& Post-subject marker & negative marker 2 & temporal \textit{{}-(h)a-}\footnote{It is impossible to be specific as to whether this morpheme is in slot 3 or slot 4 since it cannot co-occur with any TA marker.} &  &  & \\
4& First tense/aspect marker (TA1) & tense/aspect marker

conditional \textit{\nobreakdash-ngh'a-} & tense/aspect marker & tense/aspect marker & tense/aspect marker & tense/aspect marker\\
5& Object marker & object marker

reflexive/ reciprocal \textit{{}-ki-} & object marker

reflexive/ reciprocal \textit{{}-i-} & object marker

reflexive \textit{{}-i-} & object marker

reflexive \textit{{}-i-} & object marker

reflexive \textit{{}-i-}\\
\midrule
\end{tabularx}
\caption{Use of template slots in the five selected languages}
\label{tab:petzell:2}
\end{sidewaystable}
\begin{sidewaystable}
\small
\begin{tabularx}{\textwidth}{lQQQQQQ}

\midrule

&{Slot} & {Kagulu} & {Kami} & {Luguru} & {Ndamba} & {Pogoro}\\
\midrule
6& Verb root & verb root & verb root & verb root & verb root & verb root\\
7& Extensions & extensions & extensions & extensions & extensions & extensions\\
8& Second tense/ aspect marker (TA2) & habitual/ progressive \textit{{}-ag-}

\textit{{}-ile} & habitual/ progressive \textit{{}-ag-}

\textit{{}-ile} & habitual/ progressive \textit{{}-ag-}

\textit{{}-ile} & {habitual/ progressive}

\textit{{}-agh-}

perfective \textit{{}-ile} & perfective \textit{{}-iti}\\
9& Passive suffix & passive \textit{{}-igw-} & passive \textit{{}-igw-} & passive \textit{{}-igw-} & \textit{passive -w-} & \\
10& Final vowel (FV) & {final vowel} \textit{-a}

subjunctive \textit{{}-e} & {final vowel} \textit{-a}

{subjunctive} \textit{-e} & {final vowel} \textit{-a}

{subjunctive} \textit{-e} & {final vowel} \textit{-a}

{subjunctive} \textit{-e} & {final vowel} \textit{-a}

subjunctive \textit{-e}

future tense marker \textit{{}-i}\\
11& Post-final marker & plural \textit{{}-i} & plural \textit{{}-ni} & plural \textit{{}-ni}

habitual/ progressive \textit{{}-ag} & relative & \\
\lspbottomrule
\end{tabularx}
\repeatcaption{tab:petzell:2}{Use of template slots in the five selected languages}
\end{sidewaystable}

The following sections discuss how the template slots are used in the five selected languages.


\subsection{Slot 1: Pre-subject marker (PRE.SM)}

This slot is used for various pre-verbal affixes, the most common being the conditional/temporal marker (all five languages), the negative affix (four out of five languages) and the relative object marker (three out of five languages). Use of this slot for the two latter affixes is posited by \citet[108]{Meeussen1967} for Proto-Bantu. Furthermore, \citet[32]{Nurse2008} points out that negative and relative object markers are the affixes most commonly marked in this slot.



The negative markers found in this slot are discussed in  \sectref{sec:petzell:4.6}.



Kagulu, Luguru and Pogoro all use a relative morpheme in this slot. Kagulu uses a relative morpheme based on -\textit{o}{}-, which agrees with the noun class of the relativised object, as shown in examples (\ref{ex:petzell:2}), (\ref{ex:petzell:3}) and (\ref{ex:petzell:4}).


\ea\label{ex:petzell:2} Kagulu\\
\gll \textbf{yo}{}-cha-mw-end-ile \\                            
\textsc{{rel}}.\textsc{{om}}{1}{}-\textsc{sm}1\textsc{pl}.\textsc{pst}{}-\textsc{om}1-love-\textsc{pfv}\footnotemark{} \\
\glt `s/he who we loved'
\ex\label{ex:petzell:3} Luguru (\citealt[179]{Mkude1974})\\
\gll mw-alimu,  i-chi-tabu    \textbf{chi}{}-a-mu-ing'-ile          i-mw-ana \\
1-teacher  \textsc{aup}{}-7-book  \textsc{{rel}}.\textsc{\textbf{om7}}{}-\textsc{sm1}{}-\textsc{om1}{}-give-\textsc{pfv}   \textsc{aup}{}-1-child\\
\glt `the book which the teacher gave to the child'
\ex\label{ex:petzell:4}Pogoro\\
\gll chi-gota    chi-gu-kop-iti \\
7-chair    \textsc{rel.om}7-\textsc{sbj}.2\textsc{sg}{}-buy-\textsc{prf}\\
\glt `the chair you bought'
\z
\footnotetext{Note that -\textit{ile} no longer functions as a perfective in Kagulu.}


In Kami, there is no specific marking of object relatives, as shown in examples (\ref{ex:petzell:5}) and (\ref{ex:petzell:6}).


\ea\label{ex:petzell:5}Kami\\
\gll chi-nu  chi-no        wa-chi-sol-a      wa-uz-a\\
7-thing  7-\textsc{dem}.\textsc{prox}    \textsc{sm2}{}-\textsc{sm7}{}-take-\textsc{fv}   \textsc{sm2}{}-sell-\textsc{fv}\\
\glt `This thing (which) they took, they sold.'


\ex\label{ex:petzell:6} 
\gll u-mw-ele     a-kom-ile       nguku     Rahma.\\
\textsc{aup}{}-3-knife  \textsc{sm1}{}-kill-\textsc{pfv}       9. chicken  \textsc{{name}}\\
\glt `The knife with which Rahma killed the chicken.'
\z


Ndamba also does not use slot 1 for a relative marker. Instead, it uses a post-verbal relative morpheme, as shown in example (\ref{ex:petzell:7}). This does not, however, seem to be derived from the pre-verbal relative morphemes used in other neighbouring languages like Bena and Ngoni (\citealt{Morrison2011, Ngonyani2003}).


\ea\label{ex:petzell:7}Ndamba\\
\gll li-piki  tu-ka-li-dumul-ile-\textbf{lyo}        li-ka-pand-il-w-e        na    tati.\\
\textsc{5-}{tree} \textsc{sm1pl-pst-om5-}{cut}\textsc{{}-pfv-{rel.5}}  \textsc{sm5-pst-}{plant}\textsc{{}-pfv-pass-fv} by    1a.father\\
\glt `The tree which we cut down was planted by father.'
\z


\begin{sloppypar}
In Kami, slot 1 may be used for the \textit{fi}{}- conditional/temporal marker, as shown in example (\ref{ex:petzell:8}). However, this is less frequent than the \textit{(h)a-} conditional\slash temporal marker in slot 3/4, as shown further below in example (\ref{ex:petzell:13}). The two markers \textit{fi-} and \textit{(h)a-} are mutually exclusive.
\end{sloppypar}

\ea\label{ex:petzell:8} Kami\\
\gll \textbf{fi}{}-wa-tow-ile      ngoma \ldots\\
\textsc{cond{}-sm}2-play-\textsc{pfv}  9.drum\\
\glt `when they played the drum \ldots'
\z



In Luguru, slot 1 can contain either a conditional/temporal marker or a negation. The two cannot co-occur. Either the conditional/temporal marker is used as shown in example (\ref{ex:petzell:9}), or an adverbial is used together with the negation marker, as shown in example (\ref{ex:petzell:10}). The conditional/temporal is further discussed in \sectref{sec:petzell:4.4}.


\ea\label{ex:petzell:9} Luguru\\
\gll \textbf{ha}{}-ni-gend-ile       ha-tali \ldots \\
\textsc{cond}{}-\textsc{sm}1\textsc{sg}{}-go-\textsc{pfv} 16-distance\\
\glt `if/when I had walked a long distance \ldots'


\ex\label{ex:petzell:10}Luguru\\
\gll \textbf{kama}   si-gend-ile \ldots\\
\textsc{cond}  \textsc{neg.sm}1\textsc{sg} -go-\textsc{pfv} \\
\glt `if/when I did not go \ldots'
\z


In Ndamba, slot 1 is used for the \textit{pa}{}- conditional/temporal marker, as shown in example (\ref{ex:petzell:11}).


\ea\label{ex:petzell:11}Ndamba\\
\gll \textbf{pa}{}-tu-yend-ile        pa-tali \ldots\\
\textsc{cond}{}-\textsc{sm1pl}{}-go-\textsc{pfv} {16-far}\\
\glt `if/when we have walked far \ldots'
\z



In Kagulu, the conditional/temporal marker, or the relative object marker when present, appears before negation, as shown in  (\ref{ex:petzell:12}), although a concatenation of markers such as this is rare, and periphrastic constructions are preferred.


\ea\label{ex:petzell:12}Kagulu\\
\gll \textbf{fo-si}{}-cha-lut-e,                         wa-na      wa-onel-a\\
{\textsc{cond}}{{}-}\textsc{{neg}}{}-\textsc{sm}1\textsc{pl}.\textsc{pst}{}-go-\textsc{fv}     2child     \textsc{sm}2-rejoice-\textsc{fv}\\
\glt `When we could not go, the children were happy.'
\z


In Pogoro, several different morphemes may appear in this slot, including future tense markers, negative \textit{na}{}-, the relative object marker, and conditional\slash temporal \textit{pa}{}-, but it is unclear from the source data in what order they may co-occur.


\subsection{Slot 2: Subject marker (SM)}\label{sec:petzell:2.2}

All five selected languages require a subject marker in this slot, except for imperatives, as is normally the case in Bantu languages \citep[108]{Meeussen1967}. Examples of imperatives with no subject marker are given in  \sectref{sec:petzell:4.1}.


\subsection{Slot 3: Post-subject marker (POST.SM)}\label{sec:petzell:2.3}

In Kagulu, this slot is used for negative markers, which may appear before or after the subject marker (see  \sectref{sec:petzell:4.6}). In Kami, the slot is also used for the conditional/temporal marker \textit{(h)a-} (which can occur in slot 1 as well), as shown in example (\ref{ex:petzell:13}) and discussed further in  \sectref{sec:petzell:4.4}.


\ea\label{ex:petzell:13}Kami\\
\gll wa{{}-}\textbf{ha}{}-to-a       ngoma \ldots\\
\textsc{sm}2-\textsc{cond}{}-play-\textsc{fv}  9.drum\\
\glt `when they play the drum \ldots'
\z


\subsection{Slot 4: First tense/aspect marker (TA1)}\label{sec:petzell:2.4}

This is the principal slot for inflectional tense/aspect markers in all the selected languages.


\subsection{Slot 5: Object marker (OM)}\label{sec:petzell:2.5}

All five selected languages use this slot for an optional object marker which, in most cases, agrees with the noun class of the object. An exception is seen for animate objects in Ndamba, which take class 1/2 agreement. Limited data would suggest that Pogoro also follows a system of animate agreement similar to that of Ndamba.



The slot is also used for the reflexive marker \textit{{}-i-} in Ndamba, -\textit{ki-} in Kagulu and \textit{{}-li-} in Pogoro, as shown in examples (\ref{ex:petzell:14}) to (\ref{ex:petzell:16}). In Kagulu, Kami and Luguru the reflexive marker also acts as the reciprocal marker, as shown in example (\ref{ex:petzell:16}). When the reflexive marker is used with plural subjects, there is ambiguity between the reciprocal and reflexive meaning. The forms can be disambiguated by stress, or of course by a reciprocal independent pronoun (i.e. ‘each other’).


\ea\label{ex:petzell:14}Ndamba, reflexive\\
\gll ka-\textbf{i}{}-gom-ile {→} ke-gom-ile  \\
\textsc{sm}1.\textsc{pst}{}-\textsc{{refl}}{}-hit-\textsc{pfv}\\
\glt `S/he hit him/herself'


\ex\label{ex:petzell:15}Pogoro, reflexive\\
\gll ka-\textbf{li}{}-kom-iti\\
\textsc{sm}1-\textsc{{refl}}{}-hit-\textsc{pfv}\\
\glt `S/he hit him/herself'


\ex\label{ex:petzell:16}Kagulu, reciprocal and reflexive\\
\gll wa-nhu       wa-\textbf{ki}{}-end-a \\
2-person     \textsc{sm}2-\textsc{{recp}}{}-love-\textsc{fv}\\
\glt `People like each other' [The stress is on the morpheme \textit{{}-ki-}] or
`People like themselves’ [The stress is on the verb \textit{{}-enda}]
\z

\subsection{Slot 6: Verb root}\label{sec:petzell:2.6}

The verb root appears in this slot.


\subsection{Slot 7: Extensions (EXT)}\label{sec:petzell:2.7}

In all five selected languages, this slot is used for one or more derivational morphemes. In some cases, the extensions have become unproductive and appear only in specific lexicalised verb stems. The main productive extensions which appear in the selected languages are the applicative (\textit{{}-il-} or \textit{{}-el-}), causative (\nobreakdash-\textit{iz}{}-, -\textit{is}{}-, -\textit{ez}{}- or -\textit{es}{}-), stative (-\textit{ik}{}- or -\textit{ek}{}-) and associative (-\textit{an}{}-) extensions. A more complete description of verbal extensions is outside the scope of this chapter. A useful summary of Bantu verbal extensions may be found in \citet{SchadebergBostoen2019}.



(\ref{ex:petzell:17}) to (\ref{ex:petzell:20}) illustrate the main productive extensions in the selected languages.


\ea\label{ex:petzell:17}Kagulu, applicative\\
\gll ya-ku-chi-golos-\textbf{el}{}-a\\
\textsc{sm}1-\textsc{prs}{}-\textsc{om}1\textsc{pl}{}-do-\textsc{{appl}}{}-\textsc{fv}\\
\glt `s/he is working for us'


\ex\label{ex:petzell:18} Kami, causative\\
\gll Ni-mw-ang'-\textbf{iz}{}-a           ma-zi    m-bwanga.\\
\textsc{sm}1\textsc{sg}{}-\textsc{om1}{}-drink-\textsc{{caus}}{}-\textsc{fv}     6-water  1-boy\\
\glt `I made the boy drink water'


\ex\label{ex:petzell:19}Luguru, stative\\
\gll I-chi-dole    che   mu-gheni   chi-ben-\textbf{ek}{}-a.\\
\textsc{aup-7-}finger  of    1-stranger    \textsc{sm7}{}-break-\textsc{{stat}}{}-\textsc{fv}\\
\glt `The stranger’s finger is broken'


\ex\label{ex:petzell:20}Ndamba, associative\\
\gll va-ku-tov-\textbf{an}{}-a  \\
\textsc{sm2}{}-\textsc{prs}{}-hit-\textsc{{recp}}{}-\textsc{fv}\\
\glt `they are fighting (each other)'
\z

\subsection{Slot 8: Second tense/aspect marker (TA2)}\label{sec:petzell:2.8}

This slot is also used in all the languages for the -\textit{ag(h)-} habitual/progressive marker. \citet[110]{Meeussen1967} states that habitual/progressive marking is the primary use of TA2 in Proto-Bantu.



This slot is also used for the suffix -\textit{ile} (-\textit{iti} in Pogoro), which is perfective in Ndamba and Pogoro but only used in dependant clauses in Kagulu, Kami and Luguru. \citet[111]{Meeussen1967} places this in the FV slot. This implies that the \textit{{}-ag(h)-} suffix in TA2 could co-exist with -\textit{ile} in FV. Nevertheless, this is not the case in any of the selected languages; the \textit{{}-ag(h)-} and \textit{{}-ile} morphemes are mutually exclusive in all of them. We have therefore placed both -\textit{ag(h)-} and \textit{{}-ile} in TA2.


\subsection{Slot 9: Passive suffix (PASS)}\label{sec:petzell:2.9}

This slot is used for the passive derivational suffix (\textit{{}-igw or -w-)} in all the selected languages except Pogoro, as illustrated in examples (\ref{ex:petzell:21}) and (\ref{ex:petzell:22}).


\ea\label{ex:petzell:21}Kagulu\\
\gll cho-kol-\textbf{igw}{}-a   \\
\textsc{sm}7.\textsc{fut}{}-catch-\textsc{{pass}}{}-\textsc{fv}\\
\glt `it will be trapped'


\ex\label{ex:petzell:22}Ndamba\\
\gll u-bagha u-ku-telek-\textbf{w}{}-a  \\
14-food \textsc{sm}14-\textsc{prs}{}-cook-\textsc{{pass}}{}-\textsc{fv}\\
\glt `the food is being cooked'
\z


There is no passive marker in Pogoro. Instead, a periphrastic construction with an impersonal third person plural subject marker is used, as shown in example (\ref{ex:petzell:23}).


\ea\label{ex:petzell:23}Pogoro\\
\gll wa-m-fir-a         nene\\
\textsc{sm2}{}-\textsc{om}1\textsc{sg}{}-love-\textsc{fv}   \textsc{pro}.1\textsc{sg}\\
\glt `I am loved' lit. 'they love me'
\z


\citet{Stappers1967} proposes that the passive suffix was *\textit{{}-u-} in Proto-Bantu. \citet[92]{Meeussen1967} states that *\textit{{}-u-} has the last position, following the pre-final (our slot 8), but does not assign it a specific slot. Similarly, \citet[37]{Nurse2008} states that the passive marker is usually the last “extension" following the pre-final, but again does not assign it a separate slot. For our analysis, however, we assume that the passive marker appears in a separate slot, thus creating a second derivational slot. This is further corroborated by the fact that the passive can co-occur with other extensions (although semantic restrictions apply).



\citet[37]{Nurse2008} and \citet[92]{Meeussen1967} both point out that a tense/aspect morpheme in TA2 may merge with a following passive marker, leaving the final vowel of the morpheme in the FV slot. Examples (\ref{ex:petzell:24}) and (\ref{ex:petzell:25}) illustrate this using data from Ndamba.


\ea\label{ex:petzell:24}Ndamba\\
 {{/}}{}-ile- + -w-{{/}} → {}-il-w-e  \\
\gll lw-imbo    lu-ka-somol-il-w-e\\
11-song    \textsc{sm}11-\textsc{pst}{}-sing-\textsc{pfv}{}-\textsc{pass}{}-\textsc{fv}\\
\glt `the song was sung'


\ex\label{ex:petzell:25}Ndamba\\
{{/}}{}-agha- + -w-{{/}} → {}-egh-w-e   \\
\gll lw-imbo    lu-ka-somol-egh-w-e\\
11-song    \textsc{sm}11-\textsc{pst}{}-sing-\textsc{prog}{}-\textsc{pass}{}-\textsc{fv}\\
\glt `the song was being sung'
\z


{An alternative view of this process is that the passive marker is underlyingly an extension appearing as the last extension in slot 7 and that the merging process is as illustrated in example (\ref{ex:petzell:26}).}


\ea\label{ex:petzell:26}Ndamba\\
{{/}}{}-w- + -ile-{{/} }→ {}-il-w-e   \\
\gll lw-imbo  lu-ka-somol-il-w-e\\
11-song  \textsc{sm}11-\textsc{pst}{}-sing-\textsc{pfv}{}-\textsc{pass}{}-\textsc{fv}\\
\glt `the song was sung'
\z


{This process is an example of a phonological process termed “imbrication" (\citealt{Bastin1983, Kula2001, Chebanne1993}), in which, under certain conditions, a verb-final inflectional morpheme moves to a position prior to the last consonant of the extended base, as shown in (\ref{ex:petzell:27}) for Tswana and (\ref{ex:petzell:28}) for Bemba.}


\ea\label{ex:petzell:27}Tswana (\citealt[4]{Chebanne1993})\\
\gll {/-r}ek-w-ile{/}    → -re-il-w-e  \\
buy-\textsc{pass}{}-\textsc{pfv}\\
\glt `be bought'


\ex\label{ex:petzell:28} Bemba (\citealt[153]{Kula2002phd})\\
\gll {/}βúng-il-ile{/}    →   βúlung-i:l-e  \\
mould-\textsc{appl}{}-\textsc{pfv}\\
\glt `has moulded for'
\z

\subsection{Slot 10: Final vowel (FV)}
\label{sec:petzell:2.10}

The final vowel is normally -\textit{a} in all five languages, as illustrated in  (\ref{ex:petzell:29}) for Kagulu. This is the unmarked default in most Bantu languages \citep[261]{Nurse2008}. However, FV appears as \nobreakdash-\textit{e} in the subjunctive in all five languages, as illustrated in  (\ref{ex:petzell:30}) for Ndamba. In Pogoro, FV also appears as \nobreakdash-\textit{i} as a future tense marker, as illustrated in  (\ref{ex:petzell:31}).


\ea\label{ex:petzell:29}Kagulu, present indicative\\
\gll Di-bwa  di-ku-diy-\textbf{a}  \\
5-dog  \textsc{sm}5-\textsc{prs}{}-eat-\textsc{{fv}}\\
\glt `the dog eats'


\ex\label{ex:petzell:30}Ndamba, subjunctive\\
\gll tu-telek-\textbf{e} \\
\textsc{sm}1\textsc{pl}{}-cook-\textsc{{sbjv}}\\
\glt `let us cook'


\ex\label{ex:petzell:31}Pogoro, future indicative\\
\gll ha-ga-fir-\textbf{i}   \\
\textsc{fut}{}-\textsc{sm1}{}-love-\textsc{{fv}}\\
\glt `s/he will love'
\z

\subsection{Slot 11: Post-final vowel suffix (POST.FM)}\label{sec:petzell:2.11}

Three of the languages (Kagulu, Luguru and Kami) use this slot for a \textit{{}-ni} plural suffix in imperatives (see e.g.  (\ref{ex:petzell:32})), a feature claimed by \citet[111]{Meeussen1967} to be Proto-Bantu. Ndamba and Pogoro do not use this slot for plural imperatives, instead relying on a plural subject marker (see \sectref{sec:petzell:4.1}). Ndamba uses the slot for a relative marker, as illustrated in example (\ref{ex:petzell:33}), and Pogoro has nothing in this slot.


\ea\label{ex:petzell:32}Kagulu, plural imperative marker\\
\gll Ni-ingh'h-e-\textbf{ni} \\
\textsc{om}1\textsc{sg}{}-give-\textsc{sbjv}{}-\textsc{{pl}}\\
\glt `you (\textsc{pl}) give me \ldots'


\ex\label{ex:petzell:33}Ndamba, relative marker\\
\gll va-yis-ile-\textbf{vo}          nalelo \\
\textsc{sm}1\textsc{pl}{}-arrive-\textsc{pfv}{}-\textsc{{rel}}{2}    today\\
\glt `they who have arrived today'
\z

\subsection{Conclusions about the template}\label{sec:petzell:2.12}

A verbal template was established for comparing the verbal morphology of the five languages in the study. This template closely follows the template proposed by \citet{Meeussen1967} and amended by \citet{Nurse2008}, the main differences being


\begin{itemize}
\item 
Meeussen's “formative" and “limitive" slots are combined to form a “first tense/aspect marker" TA1

\item 
A separate derivational slot is included for the passive suffix.

\end{itemize}


\section{Tense/aspect}\label{sec:petzell:3}

This section discusses how tense and aspect are represented in the languages of this study. The section starts with a general introduction to tense and aspect in Bantu, followed by a sub-section for each of the five selected languages. These are followed by further sub-sections dedicated to two specific topics: the suffix \nobreakdash-\textit{ile} and periphrastic constructions, followed by a preliminary summary of the data from the five languages. Periphrastic constructions are very common in Bantu languages and are typically used in languages where the inflectional tense/aspect system is inadequate, as discussed in \sectref{sec:petzell:3.6}.



Negative tenses are subsequently discussed in  \sectref{sec:petzell:4.6}.


\subsection{Models of Bantu tense/aspect}\label{sec:petzell:3.1}

Many Bantu languages have multiple past and future tenses. \citet[103]{Nurse2008} estimates that 80\% of Bantu languages have more than one past tense and nearly 50\% have multiple future tenses. \citet[147]{BotneKershner2008} describe how research comparing the tense/aspect markers of Bantu languages has mostly attempted to fit them into a standard model, based primarily on absolute and relative time-scales, but that this approach has tended to obscure more nuanced semantic details of these systems.



One approach to analysing the Bantu tenses is to distinguish ``tense" and ``aspect" (\citealt{Dahl1985, Nurse2008}). In this model, there are two dimensions: ``tense" encodes the absolute time-scale of an event or action and ``aspect" describes details of how that event or action takes place within a specific time-scale. \citet{BotneKershner2008} makes use of this tense/aspect model to form a system of dimensions in which absolute timescales are represented as one dimension (the P-domain) and other contrasts are represented as multiple D-domain dimensions which operate at different points along the P-domain.



In many Bantu languages, tense and aspect are marked in the two distinct slots of the verbal template: TA1 and TA2 respectively. The sections below describe how these slots are used to express tense/aspect in the five languages of the study.


\subsection{Tense/aspect morphology in Kagulu}\label{sec:petzell:3.2}

Kagulu has three specific tense markers appearing in the TA1 slot: \textit{{}-ku-} non-past (i.e. present or future), -\textit{ka}{}- future and -\textit{o}{}- future. The \textit{{}-o-} marker merges with the preceding SM to produce modified subject markers such as \textit{cho}{}- (class 7 \textit{chi+o}). The three future forms appear to be in free variation and there is no apparent distinction in meaning (for a discussion of this, see \citealt[108--109]{Petzell2008}). In addition to these forms, the past imperfective has \textit{ha}{}- in PRE.SM, while the past perfective carries no overt marker. A summary of Kagulu tense/aspect markers is shown in \tabref{tab:petzell:3}.



\begin{table}


\begin{tabularx}{\textwidth}{>{\raggedright\arraybackslash}p{.25\textwidth}l>{\raggedright\arraybackslash}p{.15\textwidth}Q}

\lsptoprule
{Tense} & {PRE.SM} & {TA1} & {Example}\\
\midrule
Non-past 

(present or future) &  & \textit{ku} & {\gll {\itshape chi-\textbf{ku}{}-lut-a}\\
\textsc{sm}1\textsc{pl}{}-\textsc{{prs}}{}-go-\textsc{fv}\\
\glt `we go/will go'}\\
\tablevspace
Future1 &  & {\textit{o}}

 {([o] merges with SM)} & {\gll {\itshape ch\textbf{o}{}-lut-a}\\
\textsc{sm}1\textsc{pl}.\textsc{{fut}}{}-go-\textsc{fv}\\
\glt `we will go'}\\
\tablevspace
Future2 &  & \textit{ka} & {\gll {\itshape chi-\textbf{ka}{}-lut-a}\\
\textsc{sm}1\textsc{pl}{}-\textsc{{fut}}{}-go-\textsc{fv}\\
\glt `we will go'}\\
\tablevspace
Past perfective &  & \textit{ø} & {\gll {\itshape chi-\textbf{ø}{}-lut-a}\\
 \textsc{sm}1\textsc{pl}{}-\textsc{{pst}}.\textsc{{pfv}}{}-go-\textsc{fv}\\
\glt `we have gone/we went'}\\
\tablevspace
Past imperfective & \textit{ha} & \textit{ø} & {\gll {\itshape \textbf{ha}{}-chi-\textbf{ø}{}-lut-a}\\
\textsc{{ipfv}}{}-\textsc{sm}1\textsc{pl}{}-\textsc{{pst}}{}-go-\textsc{fv}\\
\glt `we were going/we went'}\\
\lspbottomrule
\end{tabularx}
\caption{Kagulu inflectional tense markers}
\label{tab:petzell:3}
\end{table}

\subsection{Tense/aspect morphology in Kami}\label{sec:petzell:3.3}

Kami marks non-past (present or future) with -\textit{o}{}-, which merges with the preceding SM to produce a modified SM such as \textit{to}{}- (\textit{tu+o}). Past tense (perfective and imperfective) has a null marker in the TA1 slot. A summary of Kami tense markers is shown in \tabref{tab:petzell:4}.



\begin{table}
\begin{tabularx}{\textwidth}{QQ>{\raggedright\arraybackslash}p{.4\textwidth}}
\lsptoprule
{Tense} & {TA1} & {Example}\\
\midrule
Non-past 

(present or future) & { \textit{o}}

 {([o] merges with SM)} & {\gll {\itshape t\textbf{o}{}-gend-a}\\
\textsc{sm}1\textsc{pl}.\textsc{{non\_pst}}{}-go-\textsc{fv}\\
\glt `we are going' / ‘we will go’}\\
\tablevspace
Past tense (perfective and imperfective) & \textit{ø} & {\gll \textit{tu-\textbf{ø}{}-himb-a}       \textit{simo}\\
\textsc{sm}1\textsc{pl}{}-\textsc{{pst}}{}-dig-\textsc{fv}  9.hole\\
\glt `We (have) dug a hole.'}\\
\lspbottomrule
\end{tabularx}
\caption{Kami inflectional tense markers}
\label{tab:petzell:4}
\end{table}


\subsection{Tense/aspect morphology in Luguru}\label{sec:petzell:3.4}

In Luguru, the present tense is marked with -\textit{o}{}- (which merges with the preceding SM), the future tense is marked with \textit{{}-tso}\footnote{We believe the future marker has been grammaticalized from the present \textit{{}-o-} combined with a remnant of the verb \textit{{}-za} `come’. The marker has several allomorphs that vary in spelling: -\textit{dzo}{}- and -\textit{nz’o}{}- being the most common (see also  (\ref{ex:petzell:95})--(\ref{ex:petzell:97})).} \textit{{}-}, and the past tense (perfective and imperfective) has a null marker in the TA1 slot.



Apart from these inflectional markers, there is another verbal formative, the temporal/aspectual status of which is not clear. This formative \textit{tsa-} (also realised as \textit{dza-}) encodes some type of shared knowledge or shared reference, and conveys meanings such as ‘at a specific time’, ‘at a place’, ‘as we know’, or even ‘for that reason’ \citep{Petzell2020}. It is used primarily in past-time contexts and refers to something like a ‘definite span’ of time or space, or to more abstract notions, e.g. reasons and expectations. For example, compare (\ref{ex:petzell:34}) with (\ref{ex:petzell:35}).


\ea\label{ex:petzell:34}Luguru\\
\gll ni-gend-a \\
\textsc{sm}1\textsc{sg}{}-go\textsc{{}-fv}\\
\glt `I went'


\ex\label{ex:petzell:35}Luguru\\
\gll \textbf{tsa-}ni-gend-a \\
{at}.{that}.{time/because}-\textsc{sm}1\textsc{sg}{}-go\textsc{{}-fv}\\
\glt `at that time/because I went.'
\z



A summary of Luguru tense markers is shown in \tabref{tab:petzell:5}.



\begin{table}

\begin{tabularx}{\textwidth}{QQ>{\raggedright\arraybackslash}p{.4\textwidth}}

\lsptoprule

{Tense} & {TA1} & {Example}\\
\midrule
Present & { \textit{o}}

 {([o] merges with SM)} & {\gll {\itshape tw\textbf{o}{}-ghend-a}\\
\textsc{sm}1\textsc{pl}.\textsc{{prs}}{}-go-\textsc{fv}\\
\glt `we are going'}\\
\tablevspace
Future & \textit{tso (tsa)} & {\gll {\itshape tu-\textbf{tso}{}-long-a}\\
\textsc{sm}1\textsc{pl}{}-\textsc{{fut}}{}-speak-\textsc{fv}\\
\glt `we will speak'}\\
\tablevspace
Past tense (perfective and imperfective) & \textit{ø} & {\gll {\itshape tu-\textbf{ø}{}-himb-a}       \textit{simo}\\
\textsc{sm}1\textsc{pl}{}-\textsc{{pst}}{}-dig-\textsc{fv}  9.hole\\
\glt `We (have) dug a hole.'}\\
\lspbottomrule
\end{tabularx}
\caption{Luguru inflectional tense markers}
\label{tab:petzell:5}
\end{table}

The future tense marker -\textit{tso-} sometimes surfaces as -\textit{tsa}{}- when followed by the morpheme \textit{ku}{}-. 



Two other markers, -\textit{za-} and -\textit{ya-}, are mentioned by \citet[77, 101]{Mkude1974}, but these appear to have become grammaticalised as future markers in current Luguru. Mkude refers to them as “verb like operators” and states that they represent motion towards and away from the speaker, i.e. `come' and `go' respectively. We assume that \textit{{}-za-} combines with non-past \textit{{}-o-} to form \textit{{}-zo-}, realised as future \textit{{}-tso-,} as is shown in the example in \tabref{tab:petzell:5}. It can either mean `we will speak' or rarely, depending on the context, `lest we speak'. The other morpheme, \textit{-ya-,} does not exist in our data and is rejected by our consultants. 


\subsection{Tense/aspect morphology in Ndamba}
\label{sec:petzell:3.5}
Ndamba has inflectional markers for seven distinct tenses: three past tenses, one present tense and three future tenses. All these tense markers are assembled from combinations of TA1 morphemes and the \textit{{}-ile} suffix in TA2. \tabref{tab:petzell:6} shows a summary of Ndamba inflectional tense markers. 


\begin{table}
\begin{tabularx}{\textwidth}{lllQ}
\lsptoprule
{Tense} & {TA1} & {TA2} & {Example}\\
\midrule
Present & \textit{ku} &  & {\gll {\itshape tu-\textbf{ku}{}-telek-a}\\
\textsc{sm}1\textsc{pl}{}-\textsc{{prs}}{}-cook-\textsc{fv}\\
\glt `we cook'}\\
\tablevspace
Near future & \textit{ta} &  & {\gll {\itshape tu-\textbf{ta}{}-telek-a}\\
\textsc{sm}1\textsc{pl}{}-\textsc{{fut.near}}{}-cook-\textsc{fv}\\
\glt `we will cook (in the near future)'}\\
\tablevspace
Future indefinite & \textit{ala} &  & {\gll {\itshape tw-\textbf{ala}{}-telek-a}\\
\textsc{sm}1\textsc{pl}{}-\textsc{{fut.ind}}{}-cook-\textsc{fv}\\
\glt `we will cook (at some undefined time in the future)'}\\
\tablevspace
Future emphatic & \textit{aa} &  & {\gll {\itshape tw-\textbf{aa}{}-telek-a}\\
\textsc{sm}1\textsc{pl}{}-\textsc{{fut.emphatic}}{}-cook-\textsc{fv}\\
\glt `we will definitely cook'}\\
\tablevspace
Perfect & \textit{ø} &  & {\gll {\itshape tu-\textbf{ø}{}-telek-a}\\
\textsc{sm}1\textsc{pl}{}-\textsc{{prf}}{}-cook-\textsc{fv}\\
\glt `we have cooked'}\\
\tablevspace
Past & \textit{ka} & \textit{ile} & {\gll {\itshape tu-\textbf{ka}{}-telek-\textbf{ile}}\\
\textsc{sm}1\textsc{sg}{}-\textsc{{pst}}{}-go-\textsc{{pfv}}\\
\glt `we cooked'}\\
\tablevspace
Past emphatic & \textit{aa} & \textit{ile} & {\gll {\itshape tw-\textbf{aa}{}-telek-\textbf{ile}}\\
\textsc{sm}1\textsc{pl}{}-\textsc{{pst}}.\textsc{{emphatic}}{}-cook-\textsc{{pfv}}\\
\glt `we definitely cooked'}\\
\lspbottomrule
\end{tabularx}
\caption{Ndamba inflectional tense markers}
\label{tab:petzell:6}
\end{table}

Three of the tenses (future indefinite and future and past emphatic) use a tense/aspect marker in TA1 that is used to express a level of certainty. It is possible that these are related to or derived from degrees of remoteness, but we do not have any data to be conclusive about this.



A way of analysing these tense/aspect markers might be to treat them as evidentiality markers as part of the TAME framework \citep{Dahl2013}. In this framework, evidentiality is added as an additional category to the usual verbal categories of tense, aspect and mood. Evidentiality marking indicates how certain the speaker is about the source of information (the evidence) used to make a statement. Dahl states, based on data from WALS \citep{deHaan2013}, that evidentiality markers are ``almost entirely absent in Africa".



Another approach might be to treat these tense/aspect markers as having a modal meaning, as does \citet{Fleisch2000} for the “definite future" tense of the Angolan language Luchazi (K13), as illustrated in  (\ref{ex:petzell:36}).

\newpage
\ea\label{ex:petzell:36}Luchazi (\citealt[150]{Fleisch2000})\\
\gll nji-ku̬ákù-y-a          ku-Venduka \\
\textsc{sm}1\textsc{sg}{}-\textsc{def}\_\textsc{fut}{}-go-\textsc{fv}    17-Windhoek\\
\glt `I will definitely go to Windhoek / I will have to go to Windhoek'
\z


Another interesting aspect of the Ndamba tense/aspect markers is that they may be grouped into symmetrical pairs of past and future. For example, the two emphatic tenses, marked by \textit{{}-aa-} and \textit{\nobreakdash-aa-} + \textit{{}-ile,} show a symmetry in which the same tense marker is used for both tenses, the contrast being achieved by adding -\textit{ile} for the past tense.



This symmetrical contrast is analogous to that found in Nugunu (A62), which has eight tenses, including three future and three past tenses (\citealt[161]{BotneKershner2008}, based on data from \citealt{Gerhardt1989}). The future and past tenses form three pairs of near, mid and far past/future tenses respectively, in which each past/future tense marker pair uses the same basic tense morpheme, modified with a nasal prefix to convert the future version into the past tense. For example, the mid-future tense marker, high-toned \textit{á,} becomes past tense by prefixing a nasal, as shown in examples (\ref{ex:petzell:37}) and (\ref{ex:petzell:38}). The non-hyphenated orthography is taken from the source.


\ea\label{ex:petzell:37}Nugunu (\citealt[321]{Gerhardt1989})\\
\gll a    \textbf{á}    bolá \\
\textsc{sm1}  \textsc{\textbf{pst}}\textbf{2}  arrive\\
\glt `s/he arrived'


\ex\label{ex:petzell:38}Nugunu (\citealt[326]{Gerhardt1989})\\
\gll a   \textbf{ná}    bola\\
\textsc{sm1}  \textsc{{fut}}\textbf{2}  arrive\\
\glt `s/he will arrive'
\z


Another symmetrical contrast may also be seen with the Ndamba \textit{{}-ka-} + \textit{{}-ile} past tense, in which dropping the final \textit{{}-ile} generates an imperfective meaning of an event that started in the past and continues into the future, as shown by comparing  (\ref{ex:petzell:39}) with  (\ref{ex:petzell:40}).


\ea\label{ex:petzell:39}Ndamba past imperfective\\
\gll tu-\textbf{ka}{}-telek-a  \\
\textsc{sm}1\textsc{sg}{}-\textsc{{pst}}{}-go-\textsc{fv}\\
\glt `we are still cooking'


\ex\label{ex:petzell:40}Ndamba past perfective\\
\gll tu-\textbf{ka}{}-telek-\textbf{ile} \\
\textsc{sm}1\textsc{sg}{}-\textsc{{pst}}{}-go-\textsc{{pfv}}\\
\glt `we cooked'
\z

\subsection{Tense/aspect morphology in Pogoro}
\label{sec:petzell:3.6}
The Pogoro tense markers appear in three separate slots: PRE.SM, TA1 and TA2, as shown in \tabref{tab:petzell:7}. Present tense carries no marking in any of the three slots. Past is marked with -\textit{iti} in TA2. There are two future tenses: near future is marked with \textit{za}{}- in PRE.SM, while far future has \textit{naga-} or \textit{ha}{}- in PRE.SM and \textit{{}-i} as FV. In addition, there are two secondary TA1 morphemes: inceptive -\textit{mku-} and counter-expectational -\textit{na}.


\begin{table}

\begin{tabularx}{\textwidth}{Qlll>{\raggedright\arraybackslash}p{.35\textwidth}}

\lsptoprule

{Tense} & {PRE.SM} & {TA1} & {TA2} & {Example}\\
\midrule
Present &  & \textit{ø} &  & {\gll {\itshape ga-\textbf{ø}{}-fir-a}\\
\textsc{sm}1\textsc{sg-\textbf{prs}}{}-love-\textsc{fv}\\
\glt `s/he loves'}\\
\tablevspace
Near future & \textit{za} & \textit{ø} &  & {\gll {\itshape \textbf{za}{}-gu-ø-gend-a}\\
\textsc{{fut}}{}-\textsc{sm}2\textsc{sg}{}-\textsc{{fut}}{}-go-\textsc{fv}\\
\glt `you (sg) will go'}\\
\tablevspace
Far future & { \textit{naga}}

{ \textit{{or}}}

 \textit{ha} & \textit{ø} &  & {\gll {\itshape \textbf{naga}{}-ga-ø-fir-i}\\
\textsc{{fut}}{}-\textsc{sm}1-\textsc{fut}{}-love-\textsc{fv}\\
\glt `s/he will love'}\\
\tablevspace
Past perfect &  & \textit{ø} & \textit{iti} & {\gll {\itshape ka-ø-gend-\textbf{iti}}\\
\textsc{sm}1-\textsc{pfv}{}-go-\textsc{{pfv}}\\
\glt `s/he has gone'}\\
\tablevspace
Inceptive &  & \textit{mku} &  & {\gll {\itshape na-\textbf{mku}{}-fir-a}\\
\textsc{sm}1\textsc{sg}{}-\textbf{begin}{}-love-\textsc{fv}\\
\glt `I am beginning to love'}\\
\tablevspace
Counter-expectational &  & \textit{na} &  & {\gll {\itshape na-\textbf{na}{}-m-on-i}\\
\textsc{sm}1\textsc{sg}{}-{not\_yet}{}-\textsc{om}1-see-\textsc{fv}\\
\glt `I cannot yet see him/her'}\\
\lspbottomrule
\end{tabularx}
\caption{Pogoro inflectional tense markers}
\label{tab:petzell:7}
\end{table}

\subsection{Loss of the suffix -\textit{ile} in Kagulu, Kami and Luguru}\label{sec:petzell:3.7}

The distribution of the “perfective” suffix \textit{{}-ile} is restricted in Kagulu, Kami and Luguru, and it has lost its primary function of marking perfectivity. In Kagulu, Kami and Luguru, \textit{{}-ile} is used only in conditional/temporal constructions, negative and relative clauses (\cites[126]{Petzell2008}[581--582, 588]{PetzellAunio2019}). That a morpheme is retained in subordinate clauses only is not unusual since subordinate clauses are considered more conservative (cf. \citealt{Bybee2002}, among others). The usage of -\textit{ile} in subordinate clauses is exemplified with Luguru in (\ref{ex:petzell:41}), where the first verb takes conditional/temporal marking plus \textit{{}-ile} and the second verb is an (imperfective) negative. This contrasts with the use of \textit{{}-ile} in the G50 group, where it is used as a productive perfective marker in Ndamba, and (as \textit{{}-iti}) for past tense in Pogoro.


\ea\label{ex:petzell:41}Luguru\\
\gll Ha-fvik-ile             si-lim-\textbf{ile}               bae. \\
\textsc{temp.sm}1-arrive-\textsc{{pfv}}    \textsc{neg}\textsc{.1sg}-cultivate-\textsc{pfv}    \textsc{neg}\\
\glt `When s/he arrived, I was not cultivating.’
\z


Other G30 languages such as present day Zaramo (G33) have also lost the principal use of \textit{{}-ile} as marking perfective (Petzell, field data; Brad Harvey, pers. comm.). This behaviour was also attested in \citeauthor{Nurse2008}'s data from the 1970s (\citeyear[Appendix 1, pp.169--170]{Nurse2008}). \citet[49]{Guthrie1948} also remarks on the unusual behaviour of \textit{{}-ile} in some of the G30 languages, noting that the marker does not occur in “regular” affirmative sentences. Furthermore, in \citegen{Mkude1974} grammatical sketch of Luguru there is only one occurrence of \textit{{}-ile} in an affirmative clause, shown in  (\ref{ex:petzell:42}). This, however, is translated as an applicative by our consultants; see  (\ref{ex:petzell:43}).


\ea\label{ex:petzell:42}Luguru, (\citealt[81]{Mkude1974})\\
\gll a-lim-\textbf{ile} \\
\textsc{sm}1-cultivate-\textsc{{pfv}}\\
\glt `s/he dug'                                   


\ex\label{ex:petzell:43}Luguru\\
\gll a-lim-\textbf{il}{}-e  \\
\textsc{sm}1-cultivate-\textsc{{appl-}fv}\\
\glt `s/he dug (for somebody or at a place)’
\z


What is more, another Luguru consultant explains the \textit{{}-ile} marker in example (\ref{ex:petzell:43}) as having a conditional meaning: 'where/when s/he dug'. What is clear is that \textit{{}-ile} is rejected as a perfective marker in affirmative clauses in today’s Luguru.


\subsection{Periphrastic constructions}\label{sec:petell:3.8}

Comparison of periphrastic tenses (referred to by \citealt[46]{Nurse2008} as “compound constructions”) may be hampered by uneven levels of detail in the descriptions of the languages. Nevertheless, it is interesting to examine the range of periphrastic constructions used in the five languages under study to find patterns of similarity or difference.



In Kagulu, Kami and Luguru, several periphrastic tense/aspect constructions are used. One of the most common verbs used in periphrastic constructions is \textit{kuwa} `to be', as shown in example (\ref{ex:petzell:44}), which can be used for the habitual, among other functions.


\ea\label{ex:petzell:44}Kagulu, imperfective\\
\gll Ya-ku-uw-a           ya-sok-a                   ku-lang-a    filamu. \\
\textsc{sm1-non\_pst}{}-be-\textsc{fv}  \textsc{sm1}{}-(be)come\_tired-\textsc{fv} \textsc{15}{}-watch\textsc{{}-fv}\textsc{} 9.film\\
\glt `S/he gets tired whenever she watches a film.'
\z


Other verbs are used as well, such as modal\footnote{\label{fn:petzell:4} The term ``model verb" is used here in the conventional sense as being a non-affirmative verb expressing mood, often used as an auxiliary \citep[295]{Crystal2003}.} -\textit{daha} `be able' (in Kagulu), -\textit{kala} `remain' for past constructions in Kami and Luguru, and modal -\textit{weza} `can' (in Kami), as shown in  (\ref{ex:petzell:45}) to (\ref{ex:petzell:49}).


\ea\label{ex:petzell:45}Kagulu, modal (\citealt[187]{Petzell2008})\\
\gll Wa-gamb-a    si-chi-ku-dah-a            ku-seng-a.   \\
\textsc{sm}2\nobreakdash-speak\nobreakdash-\textsc{fv}  \textsc{neg\nobreakdash-sm1pl\nobreakdash-pres}\nobreakdash-be\_able\nobreakdash-\textsc{fv}     15\nobreakdash-cut\nobreakdash-\textsc{fv}\\
\glt `They said we cannot cut/cultivate. '


\ex\label{ex:petzell:46}Kami, past (\citealt[583]{PetzellAunio2019})\\
\gll To-kal-a               tu-lim-a.  \\
\textsc{sm}1\textsc{pl}.\textsc{non\_pst}{}-remain-\textsc{fv}    \textsc{sm}1\textsc{pl}{}-cultivate-\textsc{fv}\\
\glt `We (had) cultivated.'


\ex\label{ex:petzell:47}Luguru, past\\
\gll Tu-kal-a           tu-bigh-a. \\
\textsc{sm}1\textsc{pl}{}-remain-\textsc{fv}      \textsc{sm}1\textsc{pl}{}-dance-\textsc{fv}\\
\glt `We had danced.'


\ex\label{ex:petzell:48}Kami, modal (\citealt[584]{PetzellAunio2019})\\
\gll No-wez-a             ku-fik-a?  \\
\textsc{sm}1\textsc{sg}.\textsc{non\_pst}{}-can-\textsc{fv}     \textsc{inf}{}-arrive-\textsc{fv}\\
\glt `Can/may I get (there)?'


\ex\label{ex:petzell:49}Luguru, modal\\
\gll Two-dah-a         ku-himb-a  pondo. \\
\textsc{sm}1\textsc{pl}.\textsc{prs}{}-can-\textsc{fv}    \textsc{inf}{}-dig-\textsc{fv} 5.hole\\
\glt `We can dig a hole.'
\z


Other periphrastic constructions are made up of a defective verb, \textit{ng'(h)ali} {}`be still', as shown in  (\ref{ex:petzell:50}). In Kagulu, and occasionally in Kami, it also conveys the meaning of 'not yet', as shown in  (\ref{ex:petzell:51}). In agreement with Nurse, we assume that \textit{ng'ali} contains a negation, \textit{ng'(h)a,} and the copula \textit{li} `be' \citep[173]{Nurse2008}.


\ea\label{ex:petzell:50}Kami, persistive\\
\gll Di-tunda   di-ng'ali       dyo-d-igw-a. \\
5-fruit     \textsc{sm}{5}{}-be\_still     5.\textsc{non\_pst}{}-eat-\textsc{pass}{}-\textsc{fv}\\
\glt `The fruit is still edible.'


\ex\label{ex:petzell:51}Kagulu, persistive\\
\gll Ni-ng'hali         ku-lim-a. \\
\textsc{sm}1\textsc{sg}{}-be\_still \textsc{inf}{}-cultivate-\textsc{fv} \\
\glt `I have not yet cultivated.'
\z


In Ndamba, a periphrastic future tense may be constructed from -\textit{daya} `want', as shown in  (\ref{ex:petzell:52}).


\ea\label{ex:petzell:52}Ndamba\\
\gll Va-henja  va-ku-day-a      va-yis-e        chilawu\\
\textsc{2-}{guest} \textsc{sm}2-\textsc{prs}{}-like-\textsc{fv}    \textsc{sm}2-come-\textsc{sbjv} {tomorrow}\\
\glt `The guests will arrive tomorrow.'
\z


\textit{va-ku-day-a} may be contracted to a cliticised prefix \textit{da-,} as shown in example (\ref{ex:petzell:53}), showing a process of grammaticalisation. Some speakers defined this as their preferred or only method of constructing the future tense, suggesting that the use of the system of inflectional future tenses described above in  \sectref{sec:petzell:3.5} may be in the process of disappearing.

\largerpage
\ea\label{ex:petzell:53}Ndamba\\
\gll Va-henja  \textbf{da}{}-va-yis-e            chilawu \\
\textsc{2-}{guest} \textsc{fut}{}-\textsc{sm2}{}-come-\textsc{sbjv} {tomorrow}\\
\glt `The guests would like to / will arrive tomorrow.'
\z


Additional tense/aspect constructions may be formed in Pogoro using adverbial or conjunctional particles, as shown in  (\ref{ex:petzell:54}) and (\ref{ex:petzell:55}).


\ea\label{ex:petzell:54}Pogoro, temporal conditional\\
\gll hangu     gu-on-i         wa-ndu \\
when     \textsc{sm}2\textsc{sg}{}-see-\textsc{fv}     2-person\\
\glt `when you see the people \ldots'


\ex\label{ex:petzell:55}Pogoro, far past\\
\gll ka-lewer-a       kala \\
\textsc{sm}1-forbid-\textsc{fv}   long\_ago\\
\glt `s/he forbade it'
\z


Adverbial \textit{kala} in Pogoro, as seen in example (\ref{ex:petzell:55}), may derive from Proto-Bantu *\textit{yikala} ‘be, live, stay’ (\citealt[166]{NursePhilippson2006}). A similar construction is available in Ndamba, as shown in example (\ref{ex:petzell:56}).


\ea\label{ex:petzell:56}Ndamba\\
\gll tu-ka-telek-ile      kala  \\
\textsc{sm}1\textsc{sg}{}-\textsc{pst}{}-go-\textsc{pfv}  {already/long\_ago}\\
\glt `we have already cooked / we cooked long ago'
\z


These examples contrast with the use of \textit{kala} ‘remain’ as an auxiliary in Kami and Luguru, as shown in examples (\ref{ex:petzell:46}) and (\ref{ex:petzell:47}).



In conclusion, comparing the five languages, there seem to be some similarities in periphrastic constructions between the three northern languages, Kagulu, Kami and Luguru, but the two southern languages, Ndamba and Pogoro, are different.


\subsection{Summary of tense/aspect morphology}\label{sec:petzell:3.9}

The tense/aspect morphology of the five selected languages described above show that there are three groups of languages.



The first group, consisting of the two G30 languages, Kami and Luguru, exhibit notably little tense/aspect morphology. They essentially have just one tense marker, based on \textit{{}-o-,} which is used for non-past, apart from Luguru that also has a future marker (-\textit{tso-}). In this group of languages, there is only one past tense, which in turn doubles as a perfective and which carries no overt marking (\citealt{Bar-elPetzell2021}). In addition, the use of the “perfective" marker \textit{{}-ile} has disappeared in these languages, except in certain specific contexts such as dependant clauses. Our data also appear to show that these languages make use of periphrastic constructions to express tense/aspect, enhancing their reduced systems of inflectional markers.



The question is why these languages have such reduced verbal tense morphology compared with most other Bantu languages? \citet[103]{Nurse2008} proposes that this is a result of a two-stage historical process. Proto-Bantu initially had a very rudimentary inventory of tenses, possibly only one past and one future tense. In the first stage of transformation, innovations increased this inventory, resulting in the complex tense systems seen in many Bantu languages today. Some languages, however, went through a second stage of transformation in which multiple tenses reduced back to a minimal set. Nurse's evidence for this is that there is little uniformity across the Bantu languages with reduced tense systems. He goes on to hypothesise that the unusual null marked past tense in Kami and Luguru (and occasionally Kagulu) derives from so called “vowel copy forms” (\citealt[84--85]{Nurse2008}).



The second group consists of two G50 languages, Ndamba and Pogoro, which lie in the southern part of the region and have richer sets of tenses, typical of Bantu languages. Nonetheless, the Ndamba data show that these tense distinctions are based less on temporal remoteness and more on degrees of certainty.



The final group consists of the Kagulu language (G12). This language lies somewhere between the two other groups in terms of the complexity of its system of tenses, while there is no morphological encoding of degrees of certainty.


\section{Other related markers}\label{sec:petzell:4}

This section addresses aspects of verbal morphology in the five languages not covered in \sectref{sec:petzell:3}. The reason for including a discussion of other markers at this stage is that they often interact with the tense/aspect system, such that it becomes difficult to delineate structures which are specific to tense/aspect. For example, \sectref{sec:petzell:4.5} describes the use of conditional affix -\textit{ng'a-}, which typically takes the place of a tense marker. In his cross-linguistic review of tense/aspect systems, \citet{Dahl1985} concludes that Bantu languages have the most complex tense/aspect systems of the languages included in his review. In particular, prefix positions assigned for tense/aspect markers are often also used for other categories which, in other languages, are typically expressed by adverbs \citep[176]{Dahl1985}.



As with the preceding section, the objective is to review similarities and differences in structures used by the five languages in the study. This review is presented in sections covering the verbal categories of imperative, subjunctive, conditional, temporal, habitual/progressive/intensive and negative.


\subsection{Imperative}
\label{sec:petzell:4.1}
The constructions of imperatives are similar across the five languages. In all languages there is a contrast between an emphatic imperative with no SM and FV \textit{{}-a}, and a “polite" imperative formed from the subjunctive -\textit{e} (\cites[28]{Nurse2008}[]{DevosVanOlmen2013}).  (\ref{ex:petzell:57}) shows the emphatic imperative and  (\ref{ex:petzell:58}) shows the polite imperative.


\ea\label{ex:petzell:57}Kagulu, imperative\\
\gll Leuk-\textbf{a}!  \\
go\_away-\textsc{fv}\\
\glt `go away!'


\ex\label{ex:petzell:58}Kagulu, polite imperative\\
\gll Ni-tamil-\textbf{e}! \\
\textsc{om}1\textsc{sg}{}-tell-\textsc{sbjv}\\
\glt `Tell me!'
\z


All five languages require a subject or object marker to precede the verb stem when the polite imperative is used in the singular, as shown in examples (\ref{ex:petzell:59}) to (\ref{ex:petzell:63}).


\ea\label{ex:petzell:59}Luguru\\
\gll \textbf{Mu}{}-himb-e      i-vi-adzi \\
\textsc{{sm2sg}}{}-dig-\textsc{sbjv}  \textsc{aup}{}-7-potato\\
\glt `Dig up (pl.) the potatoes'


\ex\label{ex:petzell:60}Kagulu\\
\gll \textbf{ni}{}-lim-e \\
\textsc{{sm}}\textbf{1}\textsc{{sg}{}-}cultivate-\textsc{sbjv}\\
\glt `I should cultivate'


\ex\label{ex:petzell:61}Kami\\
\gll \textbf{M}{}-kem-e! \\
\textsc{{om1}}{}-call-\textsc{sbjv}\\
\glt `Please call (him/her)!'


\ex\label{ex:petzell:62}Ndamba\\
\gll \textbf{wu}{}-gholok-e  \\
\textsc{{sm2sg}}{}-get-up-\textsc{sbjv}\\
\glt `get up!'


\ex\label{ex:petzell:63}Pogoro\\
\gll \textbf{gu}{}-fir-e! \\
\textsc{{sm2sg}}{}-love-\textsc{sbjv}\\
\glt `love!'
\z


For plural imperatives, three languages (Kagulu, Kami and Luguru) use verb-final \textit{{}-ni,} as illustrated in  (\ref{ex:petzell:64}), whereas Ndamba and Pogoro use a plural SM, as illustrated in  (\ref{ex:petzell:65}). Kagulu and Luguru may also make use of a plural SM as an alternative to the \textit{-ni} suffix, as illustrated in  (\ref{ex:petzell:66}).


\ea\label{ex:petzell:64}Kami, plural polite imperative\\
\gll Himb-e-\textbf{ni}    vi-bogwa!  \\
dig-\textsc{sbjv}{}-\textsc{{pl}}   8-potato\\
\glt `Dig up (pl) the potatoes!'


\ex\label{ex:petzell:65}Ndamba, plural polite imperative\\
\gll \textbf{Mu}{}-telek-e! \\
\textsc{sm}{2}\textsc{pl}{}-cook-\textsc{sbjv}\\
\glt `You (\textsc{pl}) cook!'


\ex\label{ex:petzell:66}Kagulu, plural polite imperative\\
\gll \textbf{Mu}{}-kumul-e!  \\
\textsc{sm}{2}\textsc{pl}{}-open-\textsc{sbjv}\\
\glt `You (\textsc{pl}) cook!'
\z


{\citet[39]{Nurse2008} states that the \textit{{}-ni} suffix is the most common form of plural negative in Bantu languages.}


\subsection{Subjunctive}\label{sec:petzell:4.2}

All five selected languages have a verb final \textit{{}-e} for subjunctive, as illustrated in (\ref{ex:petzell:67}) to (\ref{ex:petzell:71}).  (\ref{ex:petzell:67}) and (\ref{ex:petzell:68}) illustrate the use of subjunctive forms in non-affirmative subordinate clauses.  (\ref{ex:petzell:69}) to (\ref{ex:petzell:71}) illustrate the use of the subjunctive for hortatives. These two uses of the subjunctive are also found in other Bantu languages (cf. \citealt{NurseDevos2019})


\ea\label{ex:petzell:67}Kami\\
\gll no-lond-a             ni-lim-\textbf{e}          m-gunda     w-angu\\
\textsc{sm}1\textsc{sg}.\textsc{non\_pst}{}-want-\textsc{fv}  \textsc{sm}1\textsc{sg}{}-cultivate-\textsc{{sbjv}}  3-farm     3-\textsc{poss.1sg}\\
\glt `I want to cultivate my farm.'


\ex\label{ex:petzell:68}Luguru\\
\gll no-bama-a           ni-lim-\textbf{{e}}             m-gunda     gw-angu \\
\textsc{sm}1\textsc{sg}.\textsc{prs}{}-want-\textsc{fv}    \textsc{sm}1\textsc{sg}{}-cultivate-\textsc{{sbjv}}    3-farm     3-\textsc{poss.1sg}\\
\glt `I want to cultivate my farm.'


\ex\label{ex:petzell:69}Kagulu\\
\gll ni-lim-\textbf{e} \\
\textsc{sm}1\textsc{sg}{}-cultivate-\textsc{{sbjv}}\\
\glt `I should cultivate'


\ex\label{ex:petzell:70} Ndamba\\
\gll tu-telek-\textbf{e} \\
\textsc{sm}1\textsc{pl}{}-cook-\textsc{{sbjv}}\\
\glt `let us cook'


\ex \label{ex:petzell:71}Pogoro\\
\gll ni-fir-\textbf{e} \\
\textsc{sm}1\textsc{sg}{}-love-\textsc{{sbjv}}\\
\glt `I may love'
\z

\subsection{Conditional}\label{sec:petzell:4.3}

The conditional is often marked morphologically in Bantu languages, usually in the TA1 slot \citep[34]{Nurse2008}. Variations of the conditional affix -\textit{ng'a-}, which is reconstructed for Proto-Bantu \citep[113]{Meeussen1967}, are seen in all the languages in this study except Pogoro. In Kagulu, Kami and Luguru, \textit{\nobreakdash-ng'ha-} is used for `if \ldots' conditional clauses, as shown in  (\ref{ex:petzell:72}). In Kami, -\textit{ng'}{}- together with an -\textit{ile} suffix is used in past conditional clauses, as shown in  (\ref{ex:petzell:73}).


\ea\label{ex:petzell:72}Kagulu\\
\gll u-\textbf{ng'ha}{}-ij-a \\
\textsc{sm}2\textsc{sg}{}-\textsc{{cond}}{}-come-\textsc{fv}\\
\glt `if you come \ldots'


\ex\label{ex:petzell:73}Kami\\
\gll kama  u-\textbf{ng'}{}-ez-ile \\
if      \textsc{sm}2\textsc{sg}{}-\textsc{{cond}}{}-come-\textsc{fv}\\
\glt `if you came \ldots'
\z



In Ndamba, -\textit{nga-} is used in both the antecedent and consequent of hypothetical conditional `if \ldots\, then \ldots\, would \ldots' statements, as shown in  (\ref{ex:petzell:74}).


\ea\label{ex:petzell:74}Ndamba\\
\gll ma-huka   gha-\textbf{nga}{}-dumuk-ile      ndi\nobreakdash-\textbf{nga}{}-gha-gol-ile \\
{6}\textsc{{}-}{hoe} \textsc{sm}{6}\textsc{{}-{cond}{}-}{break}\textsc{{}-pfv}    \textsc{sm}{1}\textsc{sg-{cond}{}-om}{6}\textsc{{}-}{mend}\textsc{{}-pfv}\\
\glt `if the hoes were broken, I would mend them'
\z


Instead of -\textit{ng'a-,} Pogoro, uses the affix \textit{{}-ya-} for conditional `if \ldots', as shown in example (\ref{ex:petzell:75}).


\ea\label{ex:petzell:75}Pogoro\\
\gll na-\textbf{ya}{}-m-fir-a             m-dalla    a-yu \\
\textsc{sm}1\textsc{sg}{}-\textsc{{cond}}{}-\textsc{om}1-love-\textsc{fv} 1-woman    \textsc{dem-1}.1\\
\glt `if I loved that woman \ldots'
\z


Apart from these conditional markers, there are also non-hypothetical conditional\slash temporal markers meaning `if/when \ldots', as described in the \sectref{sec:petzell:4.4}.


\subsection{Conditional/temporal `when \ldots'}
\label{sec:petzell:4.4}
Bantu languages often have a marker which may be used both for conditional `if' and temporal `when' \citep[75]{Doke1935}. This is the case for the languages in this study, all of which use a morpheme in the PRE.SM slot for conditional/temporal `if/when \ldots', as shown in examples (\ref{ex:petzell:76}) to (\ref{ex:petzell:80}).


\ea\label{ex:petzell:76}Kagulu\\
\gll \textbf{fo}{}-chi-ku-mal-a \ldots \\
\textsc{cond}-\textsc{sm}1\textsc{pl}{}-\textsc{prs}{}-finish-\textsc{fv}\\
\glt `if/when we finish \ldots'


\ex\label{ex:petzell:77}Kami\\
\gll \textbf{fi}{}-wa-tow-ile        ngoma \ldots  \\
\textsc{cond{}-sm}2-play-\textsc{fv}    9.drum\\
\glt `if/when they played the drum \ldots'


\ex\label{ex:petzell:78} Luguru\\
\gll \textbf{ha}{}-ni-gend-ile         ha-tali \ldots \\
\textsc{cond}{}-\textsc{sm}1\textsc{sg}{}-go-\textsc{pfv} 16-distance\\
\glt `if/when I had walked a long distance \ldots'


\ex\label{ex:petzell:79}Ndamba\\
\gll \textbf{pa}{}-tu-yend-ile          pa-tali \ldots \\
\textsc{cond}{}-\textsc{sm1pl}{}-go-\textsc{pfv} {16-far}\\
\glt `if/when we have walked far \ldots'


\ex\label{ex:petzell:80}Pogoro (\citealt[52]{Hendle1907})\\
\gll \textbf{pa}{}-ga-fik-iti  \\
\textsc{cond}{}-\textsc{sm}1-arrive-\textsc{pfv}\\
\glt `if/when s/he arrived \ldots'
\z


While Luguru, Ndamba and Pogoro make use of what looks like the noun class prefix of class 16, as shown in  (\ref{ex:petzell:78}), (\ref{ex:petzell:79}) and (\ref{ex:petzell:80}) respectively, Kagulu (\ref{ex:petzell:76}) and occasionally Kami (\ref{ex:petzell:77}) use morphemes that can be traced to noun class 8. The origin of the Kagulu \textit{fo-} marker shown in  (\ref{ex:petzell:76}) is the most unclear, since the \textit{fo-} also appears to contain the reference marker \nobreakdash-\textit{o}{}- plus noun class 8 \textit{fi-}. The anaphoric marker \textit{{}-o-} is often used in Bantu languages to refer to something previously mentioned in the discourse \citep[275]{Güldemann2002}.



When Kami uses the less frequent class 8 \textit{fi-}, it appears in slot 1, as shown in  (\ref{ex:petzell:77}), while the more commonly used class 16 \textit{ha-} (often realised only as \textit{a-}) appears in slot 3, as shown in  (\ref{ex:petzell:81}).


\ea\label{ex:petzell:81}Kami\\
\gll wa{{}-}\textbf{(}\textbf{h)a}{}-to-a       ngoma \ldots\\
\textsc{sm}2-{16}{}-play-\textsc{fv}    9.drum\\
\glt `when they play the drum \ldots'
\z


Furthermore, the same Kami speaker may use the \textit{fi-} prefix and \textit{{}-ha-} morphemes interchangeably, as shown in examples (\ref{ex:petzell:77}) and (\ref{ex:petzell:81}) (both examples given during the same elicitation session). This type of variation is not unusual for Kami -- being a small and endangered language, it has borrowed many forms from neighbouring and dominating languages such as Luguru and Swahili (\citealt{PetzellAunio2019}).


\subsection{Habitual/progressive/intensive}
\label{sec:petzell:4.5}
All the languages except Pogoro have an -\textit{ag(h)-} affix which may be used for habitual, progressive, imperfective, continuous or intensive. This affix appears in the TA2 post-extension slot in all four languages, as shown in examples (\ref{ex:petzell:82}) and (\ref{ex:petzell:83}).


\ea\label{ex:petzell:82}Kagulu\\
\gll Ha-ka-ij-\textbf{ag}{}-a.\\
\textsc{pst}{}-\textsc{sm1}{}-come-\textsc{{hab}}{}-\textsc{fv}\\
\glt `s/he came (regularly)'


\ex\label{ex:petzell:83}Luguru\\
\gll Tu-gend-\textbf{ag}{}-a       chila  mara  Dar\_es\_Salaam. \\
\textsc{sm}1\textsc{pl}{}-go-\textsc{{hab}}{}-\textsc{fv}    every  time  place\_name\\
\glt `We go to Dar es Salaam frequently.'
\z



In Pogoro there is a progressive affix \textit{{}-aŋku-} which appears in TA1, as shown in  (\ref{ex:petzell:84}).


\ea\label{ex:petzell:84}Pogoro (\citealt[Appendix 1, p.176]{Nurse2008})\\
\gll tw-\textbf{aŋku}{}-hemer-a \\
\textsc{sm}1\textsc{pl}{}-\textsc{{prog}}{}-buy-\textsc{fv}\\
\glt `we are buying'
\z


The derivation of Pogoro \textit{{}-aŋku-} is unclear, and may not be a variant of \textit{{}-ag(h)-,} given that \textit{{}-ag(h)-} variants usually appear in TA2 \citep[110]{Meeussen1967}.



Some Bantu languages use \textit{{}-ang-} rather than -\textit{ag(h)-} for progressive\slash intensive \citep[263]{Nurse2008}. Ndamba and Pogoro, however, use both variants, showing a distinction between habitual\slash progressive\slash imperfective -\textit{ag(h)}{}- and augmentative\slash intensive \textit{{}-ang-}, as illustrated in  (\ref{ex:petzell:85}) and (\ref{ex:petzell:86}) for Ndamba.


\ea\label{ex:petzell:85}Ndamba\\
\gll a-ku-va-tov-\textbf{agh}{}-a  \\
\textsc{sm}1-\textsc{prs}{}-\textsc{om}2-hit-\textsc{{hab}}{}-\textsc{fv}\\
\glt `s/he usually beats them' or `s/he is beating them'


\ex\label{ex:petzell:86}Ndamba\\
\gll a-ku-va-tov-\textbf{ang}{}-a  \\
\textsc{sm}1-\textsc{prs}{}-\textsc{om}2-beat-\textsc{{aug}}{}-\textsc{fv}\\
\glt `s/he is beating them intensively'
\z


Another distinction between \textit{{}-ang-} and\textit{{}-agh-} in Ndamba is that \textit{{}-ang-} behaves more like a derivational extension than \textit{{}-ag(h)-,} which behaves as expected for an inflectional affix. For example, -\textit{ang-} is affected by reduplication processes, as shown in  (\ref{ex:petzell:87}). 


\ea\label{ex:petzell:87}Ndamba\\
\gll a-ku-va-tov-\textbf{ang}{}-a-tov-\textbf{ang}{}-a  \\
\textsc{sm}1-\textsc{prs}{}-\textsc{om}2-hit-\textsc{{aug}}{}-\textsc{fv}{}-hit-\textsc{{aug}}{}-\textsc{fv}\\
\glt `s/he is continuously and intensively beating them'
\z


However, \textit{{}-ag(h)-} is not affected by reduplication processes, as shown in  (\ref{ex:petzell:88}).


\ea\label{ex:petzell:88}Ndamba\\
\gll a-ku-yend-a-yend-\textbf{agh}{}-a  \\
\textsc{sm}1-\textsc{prs}{}-go-\textsc{fv}{}-go-\textsc{{hab}}{}-\textsc{fv}\\
\glt `s/he usually walks'
\z


Furthermore, the two morphemes -\textit{ang-} and \textit{-ag(h)-} can be used together, as seen in  (\ref{ex:petzell:89}).


\ea\label{ex:petzell:89}Ndamba\\
\gll a-ku-va-tov-\textbf{ang}{}-\textbf{agh}{}-a \\ 
\textsc{sm}1-\textsc{prs}{}-\textsc{om}2-beat-\textsc{{aug}}{}-\textsc{{hab}}{}-\textsc{fv}\\
\glt `s/he usually beats them intensively'
\z


This co-occurrence of -\textit{ang}{}- and -\textit{ag(h)}{}- is also observed in Bena (G63), a neighbouring language to Ndamba, as shown in example (\ref{ex:petzell:90}).


\ea\label{ex:petzell:90}Bena (\citealt[37]{Nurse2008})\\
\gll ndi-laa-gul-\textbf{ang}{}-\textbf{ag}{}-a \\
\textsc{sm}1\textsc{sg}{}-\textsc{fut}{}-buy-\textsc{{aug}}{}-\textsc{{hab}}{}-\textsc{fv}\\
\glt `I'll be buying in quantities'
\z

\subsection{Negatives}
\label{sec:petzell:4.6}
This section discusses how negatives are formed in the selected languages, and how these interact with the tense/aspect system. A summary of how negative strategies interact with Bantu tense/aspect systems is provided by \citet[180--184]{Nurse2008}, who identifies six strategies. The two most common strategies are to use a negative morpheme in the pre-subject marker (PRE.SM) or post-subject marker (POST.SM) slots. This follows a pattern common in Bantu languages \citep{GuéroisEtAlToAppear}. Three of the selected languages (Kagulu, Kami and Luguru) have inflectional negatives using these strategies, while the other two (Pogoro and Ndamba) do not, relying instead on periphrastic forms.



None of the languages uses a strategy of having specific negative tense/aspect morphemes that alternate with their non-negative counterparts, a strategy identified by \citet[34]{Nurse2008} with an example from Nen (A44), which is spoken in Cameroon. A further example of this strategy is the Swahili (G42) past perfect \textit{{}-me} / \textit{{}-ja} alternation, illustrated in  (\ref{ex:petzell:91}) and (\ref{ex:petzell:92}).

\largerpage
\ea\label{ex:petzell:91}Swahili\\
\gll wa-\textbf{me}{}-kul-a \\
\textsc{2.sm-{prf}}{}-eat-\textsc{fv}\\
\glt `they have eaten'


\ex\label{ex:petzell:92}Swahili\\
\gll ha-wa-\textbf{ja}{}-kul-a \\
\textsc{neg-2.sm-{neg.prf}}{}-eat-\textsc{fv}\\
\glt `they have not eaten'
\z


The three languages with inflectional negatives (Kagulu, Kami and Luguru) use the PRE.SM slot for their negative morphemes, which include \textit{si-, hu-, ha-}, and \textit{ng'h}{}-. One of these (Kagulu) also allows the negative markers to occur in the POST.SM position; see \tabref{tab:petzell:8} below for a discussion of this variation. According to \citet[180]{Nurse2008}, use of the POST.SM slot for the negative marker is the most common pattern across the Bantu languages. \tabref{tab:petzell:8} summarises how negatives are formed in the five selected languages, with examples for each language.



\begin{sidewaystable}


\begin{tabularx}{\textwidth}{Q>{\raggedright\arraybackslash}p{.3\textwidth}QQlQ}

\lsptoprule

{\bfseries Tense} & {\bfseries Kagulu} & {\bfseries Kami} & {\bfseries Luguru} & {\bfseries Ndamba} & {\bfseries Pogoro}\\
\midrule
Imperative & \textit{ng'ha} in PRE.SM & periphrastic & periphrastic & periphrastic & \textit{na} (in PRE.SM)\\
Present / non-past & \textit{si} or \textit{ng'ha} in PRE.SM

+ \textit{ku} in TA1

or

\textit{si} in POST.SM

+ \textit{ku} in TA1 & \textit{ha}, \textit{hu}, \textit{si} in PRE.SM & \textit{ha}, \textit{hu}, \textit{si} in PRE.SM &  & periphrastic\\
Future & \textit{si} in PRE.SM + \textit{ka}/\textit{ku} in TA1

or

\textit{si} in POST.SM + \textit{ka} in TA1 &  & \textit{ha, hu, si} in PRE.SM + \textit{tso} in TA1 &  & \\
Past & \textit{si} or \textit{ng'ha} in PRE.SM

+ a \textit{in} TA1 + \textit{ile}

or

\textit{si} in PRE.SM or POST.SM + \textit{ile}

or

\textit{s} in POST.SM

+ \textit{a} in TA1 (+ \textit{ile}) & \textit{ha}, \textit{hu}, \textit{si} in PRE.SM + \textit{ile} & \textit{ha}, \textit{hu}, \textit{si} in PRE.SM + \textit{ile} &  & \\
\lspbottomrule
\end{tabularx}

\caption{Negatives}
\label{tab:petzell:8}
\end{sidewaystable}

Kagulu generally uses the negative marker \textit{si-}, but for second person singular and class 1, \textit{ng'h-} is used. This systematic alternation between two negative morphemes is not unusual and can be traced back to Proto-Bantu. When the former marker is used, its position in the verb phrase is in free variation. The \textit{si-} morpheme appears either in the PRE.SM slot, as shown in  (\ref{ex:petzell:93}), or in the POST.SM slot, as shown in  (\ref{ex:petzell:94}), depending on the speaker's dialect or even idiolect \citep{Petzell2010}. Moreover, the same speaker can switch slots in the middle of an utterance without any apparent change in meaning. This type of variation is highly unusual, not only for this region, but for Bantu languages in general.


\ea\label{ex:petzell:93}Kagulu\\
\gll \textbf{si}{}-chi-ka-lim-a\\
\textsc{{neg}}{}-\textsc{sm}1\textsc{pl}{}-\textsc{fut}{}-cultivate-\textsc{fv}\\
\glt `we will not cultivate'


\ex\label{ex:petzell:94}Kagulu\\
\gll chi-\textbf{si}{}-ka-lim-a \\
\textsc{sm}1\textsc{pl}{}-\textsc{{neg}}{}-\textsc{fut}{}-cultivate-\textsc{fv}\\
\glt `we will not cultivate'
\z


Kami and Luguru use the negation markers \textit{si-, hu-} and \textit{ha-} for first, second, and third person animates (i.e. in class 1) respectively. These negation markers merge with the subject marker, as shown in  (\ref{ex:petzell:95}) to (\ref{ex:petzell:97}).


\ea\label{ex:petzell:95}Luguru\\
\gll \textbf{si}{}-nz'o\footnotemark{}-lim-a                 u-m-gunda  \\
\textsc{{neg.sm1sg}}{}-\textsc{fut}{}-cultivate-\textsc{fv}      \textsc{aup}{}-3-field\\
\glt `I shall not cultivate the field'


\ex\label{ex:petzell:96} Luguru\\
\gll \textbf{hu}{}-nz'o-lim-a                 u-m-gunda \\
\textsc{{neg}}.\textsc{{sm}}{2}\textsc{{sg}}{}-\textsc{fut}{}-cultivate-\textsc{fv}      \textsc{aup}{}-3-field\\
\glt `you (sg) shall not cultivate the field'


\ex\label{ex:petzell:97}Luguru\\
\gll \textbf{ha}{}-nz'o-lim-a                 u-m-gunda \\
\textsc{{neg}}.\textsc{{sm}}\textbf{1}{}-\textsc{fut}{}-cultivate-\textsc{fv}        \textsc{aup}{}-3-field\\
\glt `s/he shall not cultivate the field'
\z
\footnotetext{This is an allomorph of the future marker -\textit{tso}{}-, see footnote \ref{fn:petzell:4}.}

For all other persons and noun classes, \textit{ha-} is used in the PRE.SM slot. \citet[100]{Mkude1974} states that the Luguru negative marker is \textit{ng'(a)- instead of ha-}, which is also occasionally found in our data (see  (\ref{ex:petzell:98}) and (\ref{ex:petzell:99})). Our hypothesis is that \textit{ha-} is a phonological (or possibly dialectal) variant of the same morpheme, conceivably due to the influence of Swahili.


\ea\label{ex:petzell:98}Luguru\\
\gll \textbf{ng'a}{}-wa-mw-on-ile  \\
\textsc{{neg}}{}-\textsc{sm}1\textsc{pl}{}-\textsc{om1}{}-see-\textsc{fv}  \\  
\glt `they did not see him/her{}'


\ex\label{ex:petzell:99}Luguru\\
\gll \textbf{ha}{}-wa-mw-on-ile \\
\textsc{{neg}}{}-\textsc{sm}1\textsc{pl}{}-\textsc{om1}{}-see-\textsc{fv}  \\  
\glt `they did not see him/her{}'
\z


Ndamba and Pogoro do not have inflectional negatives, and instead use periphrastic negatives, as illustrated in  (\ref{ex:petzell:100}) and (\ref{ex:petzell:101}).


\ea\label{ex:petzell:100}Ndamba\\
\gll N-gu-yend-a      \textbf{duhu}\\
\textsc{sm}1\textsc{sg}{}-\textsc{prs}{}-go-\textsc{fv}    {not}\\
\glt `I am not going'


\ex\label{ex:petzell:101}Pogoro\\
\gll gwa-fir-a       \textbf{ndiri} \\
\textsc{sm}2\textsc{sg}{}-love-\textsc{fv}   {not}\\
\glt `you (sg) do not love'
\z


Pogoro does, however, have an inflectional negative imperative morpheme \textit{na-} which appears in the pre-subject marker slot, as shown in example (\ref{ex:petzell:102}).


\ea\label{ex:petzell:102}
\gll \textbf{na}{}-gu-fir-a!\\
\textsc{{neg}}{}-\textsc{sm}\textsc{2}\textsc{sg}{}-love-\textsc{fv}\\
\glt `do not love!'
\z

\subsection{Summary of other related markers}\label{sec:petzell:4.7}

Much of the data in this section shows consistency between the five languages in the study. Imperatives use the same structure across the five languages, with the exception of methods of expressing plurals. All the languages use verb-final \textit{{}-e} for subjunctives. All five languages use a conditional marker in place of the primary tense/aspect marker in TA1, and a conditional/temporal marker in the pre-subject marker slot. All the languages expect Pogoro use variants of the habitual/progressive/intensive marker \textit{{}-ah(h)-} as the secondary tense/aspect marker in TA2. The main differences found in this section relate to how negatives are formed, with several different strategies being used.


\section{Conclusions}\label{sec:petzell:5}

The tense/aspect systems of the five selected languages from the Morogoro region show surprising diversity. One of the languages (Ndamba) has a typical Bantu inflectional system of multiple past and future tenses, while the G30 group of languages (Luguru and especially Kami) have a greatly reduced tense/aspect system, relying heavily on periphrastic forms. The other two languages (Kagulu and Pogoro) are intermediate in terms of tense/aspect system complexity, but they are still fairly reduced compared to most other Bantu languages. A common theme across all five languages is that none has a tense/aspect system showing sharp time distinctions, as documented for many Bantu languages \citep[88--94]{Nurse2008}. These reduced systems, especially the ones with the neutralised past/perfective, are not recognised in the literature. We are still looking into how this affects the aspectual categories and vice versa, and how much temporal (and aspectual) information is conveyed through other constituents such as adverbials.



A particular aim of this study was to look at how negative tenses are handled. This revealed two patterns:



Firstly, the three northern languages (Kagulu, Luguru and Kami) have systems based on pre-verbal markers, while the two southern languages (Ndamba and Pogoro) have no inflectional negatives, relying on periphrastic forms.


\newpage
Secondly, an interesting feature is that in the three languages with inflectional negatives, \textit{{}-ile} surfaces only in non-affirmative contexts, supporting the view that it has lost its primary role of marking past perfective in these languages.



Aspects of the study merit further investigation. As these are under-described languages, the amount of morphological description and analysis available for the languages is limited, although increasing. In particular, the descriptions of their tense/aspect markers lack contextual information. It would be interesting to collect more data on the contexts in which the tense/aspect markers are used. This may help us go further into temporal interpretations for a deeper understanding beyond the standard paradigms.



The available data would suggest that tense/aspect marking is evolving in all the languages of the study due to increased contact with other languages, particularly, but not exclusively, Swahili. Swahili is the dominant language in Tanzania, spoken by nearly everyone, including all of our consultants, and given that it is a related Bantu language, it is unsurprising that other local languages are evolving to incorporate aspects of Swahili. That said, the intense contact does not necessarily imply accommodation to the dominant language, Swahili; it may also be non-accommodation, as described by \citet{PetzellKühl2017}. They analyse the overuse of a nominal marker in Luguru as \textit{stability despite contact} due to \textit{covert prestige}.



It may be interesting to document the evolution of markers more thoroughly by comparing current data with older data in a more systematic manner. Furthermore, for the languages that currently display little overt tense/aspect marking, it may be interesting to see if other strategies are emerging and if periphrastic constructions are becoming more common.



Finally, a specific topic worthy of further investigation would be an exploration of the semantics of the Ndamba tense/aspect markers within the context of the evidentialty component of the TAME framework \citep{Dahl2013}.

\section*{Acknowledgements}

We are immensely grateful to the speakers of the five languages who generously gave of their time and insights, and the anonymous reviewers for valuable comments.  We wish to acknowledge, as well, Riksbankens jubileumsfond for financial support.

\newpage
\section*{Abbreviations}

In addition to the abbreviations listed below, numbers in abbreviations refer to noun classes.

\begin{multicols}{2}
\begin{tabbing}
\textsc{fut.near} \= near future\kill
\textsc{appl} \> applicative\\ 
\textsc{aug} \> augmentative \\ 
\textsc{aup} \> augment prefix \\  
\textsc{caus} \> causative \\ 
\textsc{cond} \> conditional \\  
\textsc{dem} \> demonstrative \\ 
\textsc{ext} \> extension \\ 
\textsc{fut} \> future\\
\textsc{fut.ind} \> future indefinite \\ 
\textsc{fut.near} \> near future \\
\textsc{fv} \> final vowel\\ 
\textsc{hab} \> habitual\\
\textsc{ipfv} \> imperfect(ive)\\
\textsc{inf} \> infinitive\\
\textssc{neg} \> negative\\
\textsc{non\_pst} \> non-past\\
\textsc{om} \> object marker\\
\textsc{pass} \> passive\\
\textsc{pfv} \> perfective\\
\textsc{pl} \> plural\\
\textsc{prf} \> perfect\\
\textsc{pro} \> pronoun\\
\textsc{prog} \> progressive\\
\textsc{prox} \> proximate\\
\textsc{pst} \> past\\
\textsc{recp} \> reciprocal\\
\textsc{refl} \> reflexive\\
\textsc{rel} \> relative\\
\textsc{rel.om} \> relative object marker\\
\textsc{sbjv} \> subjunctive\\
\textsc{sg} \> singular\\
\textsc{sm} \> subject marker\\
\textsc{stat} \> stative
\end{tabbing}
\end{multicols}


\sloppy\printbibliography[heading=subbibliography,notkeyword=this]
\end{document}
