\documentclass[output=paper]{langscibook}
\ChapterDOI{10.5281/zenodo.10663775}

\author{Armindo Ngunga\orcid{}\affiliation{Centro de Estudos Africanos Universidade Eduardo Mondlane} and
        Crisófia Langa da Câmara\orcid{}\affiliation{Centro de Estudos Africanos Universidade Eduardo Mondlane}}

\title{Object marking in four Mozambican Bantu languages}

\abstract{Object marking in Bantu is an area which has received substantial attention (e.g. \citealt{MartenKula2012, Riedel2009, RiedelMarten2012preface, Zeller2014}). In many languages of this group, a morpheme which is co-referential with the object can also be incorporated into the verb structure. The present chapter, which looks at data from four Mozambican languages, aims to document and describe the situation in these languages with respect to object marking. The data show that the languages analyzed in this paper can be divided into three groups: Group 1, composed of Cinyungwe and Ciwutee, in which object marking is not obligatory with particular object NPs; Group 2 comprising of  Citshwa, in which the object marker and the object argument cannot co-occur, object marking is not obligatory with particular object NPs and there is no locative object markers; and  Group 3, Ciyaawo, in which object markers are not obligatory with particular object NPs and only the benefactive object can be expressed by an object marker in double object constructions. Taking into account the data from the four languages, we suggest that the obligatory requirement for an object marker [+OM] associated with some transitive verbs and structures should be added as a seventh parameter to the six parameters of variation in object marking in Bantu put forward by \citet{MartenKula2012}. Due to the existence of transitive verbs subcategorized as [+OM], we further encourage scholars to examine these parameters of variation in other Bantu languages in light of these features of variation.
\keywords{agreement, object, differential object marking, Bantu languages, Mozambique}
}

\IfFileExists{../localcommands.tex}{
  \addbibresource{../localbibliography.bib}
  \usepackage{langsci-optional}
\usepackage{langsci-gb4e}
\usepackage{langsci-lgr}

\usepackage{listings}
\lstset{basicstyle=\ttfamily,tabsize=2,breaklines=true}

%added by author
% \usepackage{tipa}
\usepackage{multirow}
\graphicspath{{figures/}}
\usepackage{langsci-branding}

  
\newcommand{\sent}{\enumsentence}
\newcommand{\sents}{\eenumsentence}
\let\citeasnoun\citet

\renewcommand{\lsCoverTitleFont}[1]{\sffamily\addfontfeatures{Scale=MatchUppercase}\fontsize{44pt}{16mm}\selectfont #1}
   
  %% hyphenation points for line breaks
%% Normally, automatic hyphenation in LaTeX is very good
%% If a word is mis-hyphenated, add it to this file
%%
%% add information to TeX file before \begin{document} with:
%% %% hyphenation points for line breaks
%% Normally, automatic hyphenation in LaTeX is very good
%% If a word is mis-hyphenated, add it to this file
%%
%% add information to TeX file before \begin{document} with:
%% %% hyphenation points for line breaks
%% Normally, automatic hyphenation in LaTeX is very good
%% If a word is mis-hyphenated, add it to this file
%%
%% add information to TeX file before \begin{document} with:
%% \include{localhyphenation}
\hyphenation{
affri-ca-te
affri-ca-tes
an-no-tated
com-ple-ments
com-po-si-tio-na-li-ty
non-com-po-si-tio-na-li-ty
Gon-zá-lez
out-side
Ri-chárd
se-man-tics
STREU-SLE
Tie-de-mann
}
\hyphenation{
affri-ca-te
affri-ca-tes
an-no-tated
com-ple-ments
com-po-si-tio-na-li-ty
non-com-po-si-tio-na-li-ty
Gon-zá-lez
out-side
Ri-chárd
se-man-tics
STREU-SLE
Tie-de-mann
}
\hyphenation{
affri-ca-te
affri-ca-tes
an-no-tated
com-ple-ments
com-po-si-tio-na-li-ty
non-com-po-si-tio-na-li-ty
Gon-zá-lez
out-side
Ri-chárd
se-man-tics
STREU-SLE
Tie-de-mann
} 
  \togglepaper[1]%%chapternumber
}{}

\begin{document}
\maketitle 
%\shorttitlerunninghead{}%%use this for an abridged title in the page headers

\settowidth\jamwidth{[OM-V] [O]}

\section{Introduction}

The Bantu languages are known for the systematic way in which grammatical relations are morphologically marked in the verbal structure (\citet{Ngunga2014}). In the case of object marking, many of these languages exhibit agreement with both the subject and the object. Agreement with the subject is usually grammatically obligatory, while the status of object marking is often more pronominal (\citealt{MartenRamadhani2001}) and may be optional. Object markers are affixes or clitics that identify and cross-reference an object argument. Thus, the phenomenon of object marking on verbs in Bantu languages is a mechanism for referring to discourse-familiar entities, similar to pronominalization (\citealt{SikukuEtAl2018}). In this paper, we use the term object marking to refer to the way a lexical object is reflected within the verb structure through a co-referential marker (\citealt{Baker1988, Corbett2006, Deen2006}, amongst others).

The ambiguity of the status of object markers in Bantu has received considerable attention in the literature (\citealt{MartenKula2012, Riedel2009, RiedelMarten2012preface, Zeller2014} among others). In Bantu languages, a wide range of prefixes (subject, object, tense, aspect, mood, negation, and other markers) and suffixes (derivational and inflectional) can be attached to the lexical verb root. In many languages, the object markers (OMs) are attached directly to the verb stem. See the examples in \REF{ex:ngunga:1} and \REF{ex:ngunga:2} presented below where the objects are cross-referenced by the forms wa- and ci- respectively.

\ea\label{ex:ngunga:1} Kiswahili (G42)  \citep[46]{Riedel2009}\\
\gll A-li-\textbf{wa}{}-won-a\\
    \textsc{sm}1-\textsc{past}-\textsc{om}2-see-\textsc{fv}\\
\glt ‘he saw them’
\ex\label{ex:ngunga:2}Cinyungwe   (N43)\\
\gll Iye  a-da-\textbf{ci}{}-mog-a.\\
     he  \textsc{sm1sg}{}-\textsc{pfv}-\textsc{om}7-jump-\textsc{fv}\\
\glt ‘he jumped it’
\z

The slot immediately before the verb root has largely been identified as the OM slot in the Bantu verb structure (\citet{Ngunga2014}) . However, there is no such consensus about the grammatical status of OMs as pronominal or agreement markers (\citealt{BresnanMchombo1987, Deen2006, Riedel2009}), since in individual languages the OMs behave differently. 

The object argument may or may not co-occur with the object marker depending on a series of syntactic, pragmatic and semantic factors. This means that there is a difference between Bantu languages concerning the possibility of the co-occurrence of the object marker and the corresponding object argument (\citealt{Ngunga2014, Zeller2014}). The example in \REF{ex:ngunga:3} illustrates that the co-occurrence of the object marker and the object argument is possible in IsiZulu, but it is not possible in Kinyarwanda, as example \REF{ex:ngunga:4} illustrates. 

\ea\label{ex:ngunga:3} Isizulu (S42)  \citep[219]{Zeller2012} \\
\gll A-ba-ntwana  ba-ya-si-thand-a  lesi  si-kole\\
\textsc{aug}-2-{child}  \textsc{sm}2-\textsc{dis}-\textsc{om}7-{like}-{\textsc{fv}}   \textsc{dem}7  7-school\\
\glt ‘the children like this school.’
\ex\label{ex:ngunga:4} Kinyarwanda (L12)    \citep[76]{Riedel2009} \\
\gll * A-ba-aana  ba-a-ra-bi-ri-ye    i-bi-ryo  ejo.\\
     {} \textsc{aug}-{2}{}-{child}  \textsc{sm}2-\textsc{rem}-\textsc{dis}-\textsc{om}8-{eat}{}-{\textsc{pfv}}   \textsc{aug}-8-food  {yesterday}\\
\glt \phantom{*} Intended: ‘the children ate the food yesterday’
\z

According to \citet{Iorio2015}, the co-occurrence of the object marker and the co-referring object argument is only possible if the latter is right dislocated. This dislocation results in a definiteness and specificity effect on the object arguments with which they co-occur. See the Bembe examples below: 

\ea\label{ex:ngunga:5}  Bembe (D54)    \citep[196]{Iorio2015} \\
    \ea[]{\label{ex:ngunga:5a}
    \gll mwana    a-a-yak-a     ngyoʔa. \\
  \textsc{sm}1-child   \textsc{sm}1.\textsc{sg}-\textsc{pfv}-kill-{\textsc{fv}}  \textsc{sm}9.snake\\ \jambox{[VO]}
\glt  ‘the child has killed a/*the snake.’
}
   \ex[]{\label{ex:ngunga:5b}
   \gll mwana    a-a-\textbf{ya}{}-yak-a. \\      
  {\textsc{sm}1}{}-child   {\textsc{sm}.\textsc{sg}1-\textsc{pfv}-\textsc{om}9}{}-kill-{\textsc{fv}}\\ \jambox{[OM-V]}
\glt  ‘the child has killed it.’
}
    \ex[*]{\label{ex:ngunga:5c}
    \gll mwana  a-a-\textbf{ya}\textsubscript{i}{}-yak-a    ngyoʔa\textsubscript{i}\\ 
  {\textsc{sm}1}{}-child   {\textsc{sm}1.\textsc{sg}-\textsc{pfv}-\textsc{om}9}{}-kill-{\textsc{fv}}   {\textsc{sm}9}{}.snake\\ \jambox{*[OM-V O]}
\glt  Intd.: ‘the child has killed a/he snake’
}
    \ex[]{\label{ex:ngunga:5d}
    \gll mwana    a-a-\textbf{ya}\textsubscript{i}{}-yak-a,     ngyoʔa\textsubscript{i}\\ 
 {\textsc{sm}1}{}-child   {\textsc{sm}1.\textsc{sg}-\textsc{pfv}-\textsc{om}9}{}-kill-{\textsc{fv}}  {\textsc{sm}9}.snake\\ \jambox{[OM-V] [O]}
 \glt ‘the child has killed it, the/*a snake (that is)’
}
    \z
\z

The co-occurrence of the object marker and the nominal object is not the only variation that is found in object marking in Bantu languages. According to \citet{vanderWal2015}, in languages which allow the occurrence of the object marker and the co-referring, there is a great deal of variation as to which objects are marked by an object marker. In Nyarutu, for example, it is usually the animate, definite and/or given objects that are doubled by an object marker (\citealt{vanderWal2015, vanderWal2016}). Therefore, the example in \REF{ex:ngunga:6b} is ungrammatical because animate objects must be doubled by an object marker. 

\ea\label{ex:ngunga:6}
\ea[]{\label{ex:ngunga:6a}  Nyarutu (F32)    (\citealt[6]{vanderWal2015} [via \citealt[182]{Hualde1989}])  \\
\gll n-a-\textbf{mʊ}{}-onaa      Maria.\\
{\textsc{sm1sg}-\textsc{}\textsc{pst}-}{{1\textsc{om}}}{{}-see\textsc{}\textsc{}}     1.Maria\\
\glt  ‘I saw Maria.’}

\ex[*]{\label{ex:ngunga:6b}  
\gll n-a-onaa      Maria.\\
  \textsc{sm1sg}-\textsc{}\textsc{pst}-{{\textsc{om1}}}{{}-see\textsc\textsc{}}   1.Maria\\
\glt   Intd: ‘I saw Maria.’ }
   \z 
\z

At this point we notice that although OMs occur in the verbal structure of several Bantu languages, their occurrence and/or co-occurrence with the object argument is determined by several factors. According to \citet{MartenRamadhani2001}, in Kiluguru (G30), a language spoken in Tanzania, the distribution of object marking in simple transitive predicates is partly motivated by the syntactic context and semantic aspects. Moreover, in some cases, the object is interpreted as being more definite when the object agreement is present. The other, probably more important, dimension to the analysis of object marking in Kiluguru is pragmatic, since the use of object agreement is more related to the anaphoric structure of the discourse and to evaluations by speakers in relation to what they hear. \citet{MartenRamadhani2001} also claim that in this language, in contrast to simple predicates, verbs such as \textit{{}-ona} ‘see’, \textit{{}-phika} ‘find’ and \textit{{}-ing’a} ‘give’ obligatorily require the use of the OM with the lexical object. This is a point to which we will return also for the languages under examination here in (\sectref{sec:ngunga:3}) also for the languages under examination. 

The high degree of diversity in the morphological marking of object arguments in Bantu languages is reflected in the diversity of proposals made by several authors using data from different languages. \citet{MartenKula2012}, for instance, present six parameters for the investigation of variation in object marking based in 16 Bantu languages. 

A different approach, adopted by \citet{Zeller2014}, divides Bantu languages into three types, namely: Type 1, where object markers are agreement markers; Type 2, where object markers are pronominal clitics; and Type 3, where object marking is a reflex of A-bar movement of the corresponding object.

\citet{Aissen2002} claims that object marking is based on semantic and pragmatic grounds. In some languages, it is the pragmatic character of the object that determines whether it is obligatory or optional, or if it is excluded. In pragmatic and morphological object marking, the objects that most resemble subjects are overtly case-marked, whereas syntactic objects are obligatorily case-marked if they stand in a position that is more marked for an object. This is the case for languages like Hungarian and Malayalam (see \citet{Aissen2002}) for further details. For \citet[437]{Aissen2002}, the prominence scale for direct objects is as follows:

\begin{description}
\item [Animacy:] Human > Animate > Inanimate
\item [Definiteness:] Personal pronoun > Proper name > Definite NP > Indefinite specific NP > Non-specific NP
\end{description}

This variation in object marking reflects the tension between two principles: iconicity, which prefers semantic markedness to be expressed by morphology, and economy, which would rather be devoid of structure whenever possible \citep{Aissen2002}.

The current paper seeks to contribute to the understanding of object marking in Bantu by applying \citegen{MartenKula2012} six parameters to four Mozambican Bantu languages. We also develop a seventh parameter for the four languages analyzed in this paper, which is related to the obligatoriness of object markers with specific transitive verbs and specific structure as we shall see in \sectref{sec:ngunga:3}. 

This paper is organized as follows. Following this introduction, we present \citegen{MartenKula2012} six parameters (\sectref{sec:ngunga:2}). We then apply these parameters to four Mozambican Bantu languages (\sectref{sec:ngunga:3}), before presenting some conclusions (\sectref{sec:ngunga:4}). 

\section{A parametric approach to object marking in Bantu}
\label{sec:ngunga:2}

Studies of object marking have shown differences regarding the realization of object markers in Bantu languages. One of these studies is that of \citet{MartenKula2012}, who identified a number of micro-parameters that determine cross-Bantu variation. \citet{MartenKula2012} present six parameters relating to the investigation of the variation in object marking in 16 Bantu languages \REF{extab:ngunga:1}. The languages of their study are: Bemba (M42), Chaga (Kivunjo) (E62b), Chichewa (N31), Ha (D66), Haya (E22), Kinyarwanda (D60), Lozi (K21), Makhuwa (P31), ciNsenga (N41), Otjiherero (R31), Ruwund (L53), Sambaa (G23), siSwati (S43), Kiswahili (G42), Setswana (S31), and Yeyi (R41).

\ea\label{extab:ngunga:1}
Morphosyntactic parameters of object marking in Bantu (\citealt[5]{MartenKula2012}).\\
\begin{enumerate}
\item[(i)]  Can the object marker and the object argument co-occur? \\
\item[(ii)] Is an object marker obligatory with particular object NPs? \\
\item[(iii)]Are there locative object markers?\\
\item[(iv)] Is object marking restricted to one object marker per verb? \\
\item[(v)]  Can either benefactive or theme objects be expressed by an object marker in double object constructions? \\
\item[(vi)] Is an object marker required/optional/disallowed in object relatives? \\
% \lspbottomrule
\end{enumerate}
\z

\subsection{(i) The co-occurrence of object markers and lexical objects}

In some Bantu languages, such as Kiswahili \REF{ex:ngunga:7}, there are no restrictions of co-occurrence of an object marker and a co-referential overt NP. This means that the object marker can be used together with an overt NP. However, in other languages like Otjiherero, the object marker cannot co-occur with an overt NP \REF{ex:ngunga:8}.\pagebreak[3]

\ea\label{ex:ngunga:7} Kiswahili (G42)  \citep[240]{MartenKula2012}\\
\gll ni-li-mw\textsubscript{i}{}-on-a    Juma\textsubscript{i}.\\
{\textsc{sm1sg}}{}-{\textsc{past}-\textsc{om}1}{}-see-{\textsc{fv}}  Juma\\
\glt ‘I saw Juma’
\ex \label{ex:ngunga:8}  Otjiherero (R31)\citep[240]{MartenKula2012}\\
\gll * mb-é    vé  mún-ù    òvá-nátjè.\\
     {} \textsc{sm1sg}-\textsc{past}  \textsc{om}2 see{{}-\textsc{fv}} 2-children\\
\glt \phantom{*} Intd.: ‘I saw the children’

\z

The examples in \REF{ex:ngunga:7} and \REF{ex:ngunga:8} illustrate that Swahili behaves differently from Otjiherero. In Swahili \REF{ex:ngunga:7}, the co-occurrence of the object marker (-\textit{mw}{}-) and the overt object NP (Juma) yields a grammatical result which is not possible in Otjiherero \REF{ex:ngunga:8}. 

\subsection{(ii) The obligatoriness of object markers with specific classes of objects}

This parameter of variation relates to cases where the co-occurrence of object markers and co-referential NPs is obligatory with specific NPs. This can be found in Swahili for example, where object marking is obligatory with animate objects, particularly nouns which refer to humans, as shown in (\ref{ex:ngunga:9}a, b) below:

\ea\label{ex:ngunga:9}  Kiswahili (G42)  \citep[46]{Riedel2009}\\
 \ea[]{\label{ex:ngunga:9a}   
 \gll ni-li-*(\textbf{mw\textsubscript{i}}){}-on-a  m-toto\textsubscript{i}\\
  {\textsc{sm}-\textsc{pfv}-(\textsc{om1})-see-\textsc{fv}}    {\textsc{sm}1-}child\\
\glt  ‘I saw his child’} 

\ex[*]{\label{ex:ngunga:9b}  
 \gll  ni-li-on-a    m-toto\\
    {\textsc{sm}-\textsc{pfv}}{}-see-{\textsc{fv}}    {\textsc{sm}1}{}-child\\
  \glt  Intd: ‘I saw the child’}
    \z
\z

In contrast, the use of the object marker in Kiswahili is structurally optional with inanimate NPs, as exemplified in \REF{ex:ngunga:9c} from \citet[241]{MartenKula2012}:

\ea\label{ex:ngunga:9c}
\gll ni-li-(ki)-on-a    ki-tabu  \\
{\textsc{sm}-\textsc{pfv}-(\textsc{om7})-see-\textsc{fv}}    7-book\\
\glt ‘I saw a/the book’
\z

There are also languages where the thematic role of the object determines whether it can co-occur with an object marker. For instance, in Ruwund, the benefactive object can co-occur with the object marker \REF{ex:ngunga:10a}, but cannot occur with a theme object \REF{ex:ngunga:9b}.

\ea\label{ex:ngunga:10}  Ruwund (D62)  \citep[241]{MartenKula2012}\\
\ea[]{\label{ex:ngunga:10a}
\gll ka-ma-mu-tum-in  mwâan.\\
\textsc{inf}-\textsc{om}6-\textsc{om}1{}-send-\textsc{appl}  1.child\\
\glt ‘to send the child them’
}

\ex[*]{\label{ex:ngunga:10b}
\gll ka-ma-mu-tum-in  mwâan    ma-long.\\
\textsc{inf}-\textsc{om}6-\textsc{om}1{}-send-\textsc{appl}   1.child    {6}{}-plates\\
\glt ‘to send the child the plates’
}
\z
\z

 \subsection{(iii) The presence of locative object markers}

In languages like Cinsenga and Setswana where locative objects can be expressed by locative object markers, locative nouns and locative object markers can co-occur. This is shown in the examples in \REF{ex:ngunga:11} and \REF{ex:ngunga:12}: 

\ea\label{ex:ngunga:11}  Cinsenga (N41)  \citep[243]{MartenKula2012}\\
\gll ku-Lilongwe  n-a-ku-ziw-a.\\
17-Lilongwe  {\textsc{sm}1}{}-{\textsc{pres}-\textsc{om}17}{}-know-{\textsc{fv}}\\
\glt ‘Lilongwe I know it (there)’
\z

\ea\label{ex:ngunga:12}  Setswana (S31a)\\
\gll ke  a  gó  itsé.\\
{\textsc{sm}1}  {\textsc{pres}}  {\textsc{om}17}  know\\
\glt ‘I know it (there)’
\z

However, some other languages do not have locative object markers.

\subsection{(iv) The multiple object markers}

The fourth parameter of object marking variation proposed by \citet{MartenKula2012} is related to the number of object markers allowed per inflected verb form. There are languages that do not allow multiple object markers in the same verb. This is what happens in Bemba which allows one object marker, as shown in \REF{ex:ngunga:13} below:

\ea\label{ex:ngunga:13}  Kiswahili\\
    \ea[]{\label{ex:ngunga:13a}  \gll ni-li-m-p-a.\\
    {\textsc{sm}1-\textsc{pst}-\textsc{om}1}{}-give-{\textsc{fv}}\\
    \glt ‘I gave him (it)’}

    \ex[*]{\label{ex:ngunga:13b}  \gll ni-li-i-m-p-a.\\
    {\textsc{sm}1-\textsc{pst}-\textsc{om}9-\textsc{om}1}{}-give-{\textsc{fv}}\\
    \glt Intd: ‘I gave him it’}
    \z
\z

In contrast to the Bemba examples present abov, each lexical object in Sambaa may have its corresponding OM in the verb structure. Consider the example in \REF{ex:ngunga:14}:

\ea\label{ex:ngunga:14}  Sambaa (L12)    \citep[72]{Riedel2009}\\
\gll n-za-\textbf{ha}{}-\textbf{ci}{}-\textbf{m}{}-nka    Stella    ki-tabu.\\
{\textsc{sm}-\textsc{pfv}-\textsc{om}16-\textsc{om}7-\textsc{om}1}{}-give  Stella    7-book\\
\glt ‘I gave Stella a book there’
\z

This fact has led scholars like \citet{Henderson2006acal} and \citet{Zeller2014} to argue that in Bantu, OMs can function as agreement markers and pronominal clitics.

It is important to note that in languages with multiple object marking, there is variation as to which objects are marked. In Bemba it is possible to mark more than one object if both object markers are animate \REF{ex:ngunga:15a} or if the object marker closest to the verb is the first person singular \textit{n}{}- \REF{ex:ngunga:15b}. 

\ea\label{ex:ngunga:15}  Bemba (M42)    \citep[245]{MartenKula2012}\\
    \ea\label{ex:ngunga:15a} \gll mù-kà-\textbf{bá}-\textbf{mú}-éb-él-á-kó.\\
  {\textsc{sm}1-\textsc{fut}-\textsc{om}2-\textsc{om}1}{}-tell-{\textsc{appl}-\textsc{fv}-\textsc{pro}17}\\
  \glt ‘you will tell them for him.’     
  
  \ex\label{ex:ngunga:15b} \gll  mú-ká-\textbf{cí}-\textbf{mù}-\textbf{n}-twààl-íj-é-kó.\\
  {\textsc{sm}2-\textsc{fut}-\textsc{om}7-\textsc{om}1-\textsc{om}1}{}-return-{\textsc{appl}}{}-{\textsc{fv}-\textsc{pro}17}\\
  \glt ‘you should return it to him/her for me.’
    \z
\z

However, in other languages object markers can co-occur in an unrestricted manner.

 \subsection{(v) The object marking in double object constructions}

According to \citet{MartenKula2012}, another well-known parameter of variation relates to object marking in double object constructions. There are languages in which only the benefactive object in a double object construction can be object marked and those in which either the benefactive or the theme can be marked. 

The ungrammaticality of \REF{ex:ngunga:16b} illustrates that in Chichewa only the benefactive object can be object-marked on the verb in a double object construction. In contrast, in Otjiherero the theme can also be object-marked \REF{ex:ngunga:17}:

\ea\label{ex:ngunga:16}  Chichewa (N31a)    (\citealt[247]{MartenKula2012})\\
    \ea[]{\label{ex:ngunga:16a} \gll a-lenje    a-ku-wá-phík-ir-á    zí-túmbúwa  (a-nyani).\\
  2-hunters  {\textsc{sm}2-\textsc{pres}-\textsc{om}2}{}-cook-{\textsc{appl}-\textsc{fv}} 8-pancakes  2-baboons\\
  \glt ‘the hunters are cooking (for) them (the baboons) some pancakes’
}
    \ex[*]{\label{ex:ngunga:16b}  \gll a-lenje  a-ku-wá-phík-ir-á    a-nyani  (zí-túmbúwa).\\
  2-hunters  {\textsc{sm}2-\textsc{pres}-\textsc{om}8}{}-cook-{\textsc{appl}-\textsc{fv}} 2-baboons   8-pancakes  \\
}
\z
\z

\ea\label{ex:ngunga:17}  Otjiherero (R31)  
(\citealt[247]{MartenKula2012})\\
\gll  Má-yé  ì  tjángér-é  òvà-nâtjé.\\
    \textsc{pres}.\textsc{sm}  \textsc{om}9 write-{\textsc{appl}-\textsc{fv}} 2-children\\
\glt  ‘they are writing the children it’
\z

\subsection{(vi) The object marking in relative clauses}

The last parameter proposed by \citet{MartenKula2012} pertains to the use of object markers in object relative clauses. In descriptive terms, three groups of language types can be distinguished: (i) those where object markers are required in object relatives (e.g. Setswana); (ii) those where object markers are optional (e.g. Swahili) and (iii) those where object markers are not allowed in object relative clauses (e.g. Lozi). These three types are illustrated by examples \REF{ex:ngunga:18}, \REF{ex:ngunga:19} and \REF{ex:ngunga:20} below.

\ea\label{ex:ngunga:18}  Setswana  (\citealt[248]{MartenKula2012})\\
     \ea[]{\label{ex:ngunga:18a}  \gll di-kwelo  tse  ke    di  bone-ng \ldots\\
    10-books  {\textsc{rel}10}  \textsc{sm}1.\textsc{past}  \textsc{om}10 see{{}-\textsc{rel}}\\
    \glt ‘the books which I saw them…’
}
     \ex[*]{\label{ex:ngunga:18b} \gll di-kwelo  tse  ke    bone-ng \ldots\\
     10-books  {\textsc{rel}10}  {\textsc{sm}1.\textsc{past}} see{{}-\textsc{rel}}\\
    \glt ‘the books which I saw them…’}
    \z
\z

The example in \REF{ex:ngunga:18b} is ungrammatical because the object relative construction does not have an object marker. 

  The example of the second type of language is exemplified by Swahili, where object marking in object relatives is possible, but not required \REF{ex:ngunga:19}:

\ea\label{ex:ngunga:19}  Swahili\\
   \gll   ki-tabu    amba-cho  ni-li-(ki-)som-a \ldots\\
      7-books  \textsc{rel}-7    \textsc{sm}1-\textsc{past}-(\textsc{om}7)-read{{}-\textsc{fv}}\\
    \glt  ‘the book which I read (it)’
\z

The third pattern is shown with the example from Lozi, where object markers in object relatives are not allowed. 

\ea\label{ex:ngunga:20}  Lozi (K21)  (\citealt[248]{MartenKula2012})\\
\gll * buka       ye-ne-ba-(ye)-bon-i     ba-nana     fa-tafule  ki-ye-tuna.\\
  {} 9.book   {9.\textsc{rel}-\textsc{past}-\textsc{sm}2-(\textsc{om}9)}{}-see-{\textsc{fv}} 2-children  16-table  {\textsc{cop}-\textsc{sm}9}{}-big\\
\glt \phantom{*} Intd: ‘the book which the children saw it on the table is big.’
\z

Moreover, example \REF{ex:ngunga:20} is important because it illustrates that objects are not required in relative constructions.

\section{Object marking in four Mozambican Bantu languages}
\label{sec:ngunga:3}

In the present section, we examine properties of object marking in four Mozambican Bantu languages, namely, Cinyungwe, Citshwa, Ciwutee, and Ciyaawo. We analyse these languages using six parameters of variation from \citet{MartenKula2012} The languages analyzed in this paper were chosen on the basis of available information and our own knowledge as native speakers of Ciyaawo (first author) and Cinyungwe (second author). In \sectref{sec:ngunga:3.1} we start our discussion by analyzing data from Cinyungwe.

% \begin{figure}
% \includegraphics[height=.8\textheight]{figures/mozambique.png}
% \caption{Lanuages of Mozambique, based on \citet[55]{ChimbutaneEtAl2022} and \url{https://erccportal.jrc.ec.europa.eu/Maps\#/maps?pageIndex=1&pageSize=200&mt=Base\%20map}}
% \end{figure}

\subsection{Object marking in Cinyungwe} \label{sec:ngunga:3.1}
\begin{sloppypar}
Some Bantu languages show restrictions on the co-occurrence of an object marker and the co-referential object argument. The first parameter presented by \citet{MartenKula2012} identifies the conditions under which an object marker can co-occur with a corresponding object argument after the verb.
\end{sloppypar}

\subsubsection{(i) Can the object marker and the object argument co-occur?}

Cinyungwe is a Mozambican Bantu language spoken in Tete Province by 457,290 speakers (\citealt[108]{NgungaFaquir2011}). In Cinyungwe, the co-occurrence of the object argument and the OM within the same sentence is possible only if the object is a dislocated adjunct (i.e. it is not \textit{in situ}), as illustrated in \REF{ex:ngunga:21} below:

\ea\label{ex:ngunga:21}  Cinyungwe    (N43)\\
     \ea[]{\label{ex:ngunga:21a} \gll  baba    a-da-nyamul-a    m-wana.\\
  1.dad    {\textsc{sm1}-\textsc{pfv}}{}-hold-{\textsc{fv}}  {1}{}-child\\
  \glt ‘dad held a child’   } 

    \ex[]{\label{ex:ngunga:21b} \gll baba    a-da-\textbf{mu}{}-nyamul-a.\\
  1.dad    {\textsc{sm1}-\textsc{pfv}-\textsc{om}1}{}-hold-{\textsc{fv}}  \\
  \glt ‘dad held (her/him) the child’  }

    \ex[*]{\label{ex:ngunga:21c}  \gll baba    a-da-\textbf{mu\textsubscript{i}}{}-nyamul-a  m-wana\textsubscript{i}\\
  1.dad    {\textsc{sm1}-\textsc{pfv}-\textsc{om}1}{}-hold-{\textsc{fv}}  {1}{}-child\\
  \glt Intd: ‘dad held (her/him), the child’ } 

    \ex[]{\label{ex:ngunga:21d} \gll baba    a-da-\textbf{mu\textsubscript{i}}{}-nyamul-a,  (m-wana)\textsubscript{i}\\
  1.dad  {\textsc{sm1}-\textsc{pfv}-\textsc{om}}{}-hold-{\textsc{fv}}  {1}{}-child\\
  \glt ‘dad has held (her/him), the child’  }
    \z
\z

In example \REF{ex:ngunga:21a} \textit{mwana} ‘child’ is non-specific. The presence of the OM -\textit{mu}{}- in \REF{ex:ngunga:21b} means that this is an appropriate response to a question such as “What did dad do to the child?”. The example in \REF{ex:ngunga:21c} is ungrammatical because the object marker and the NP co-occur, which is prohibited in Cinyungwe. The example in \REF{ex:ngunga:21c} shows that in Cinyungwe doubling an object marker with an \textit{in situ} object is unacceptable in neutral discourse contexts. Note however that this sentence is acceptable in a context in which the speaker wants to convince the hearer that the action happened and s/he even saw father holding the child, i.e. for emphatic purposes or for certainty. The pause after the verb in example \REF{ex:ngunga:21d} is obligatory and indicates that the NP is dislocated, and represents the only way such a sentence is acceptable in this context. 

\subsubsection{(ii) Is an object marker obligatory with particular object NPs?}

The other aspect of variation with respect to the co-occurrence of the OM and the object argument found in Bantu languages relates to whether an object marker is obligatory with a specific object argument. In Cinyungwe, object marking is not obligatory with specific object arguments of any type. See the examples presented below:

\ea\label{ex:ngunga:22}  Cinyungwe   
     \ea\label{ex:ngunga:22a} \gll  mw-ana  a-da-won-a    ng’ombe.\\
  {1}-child  {\textsc{sm1sg}-\textsc{pfv}}{}-see-{\textsc{fv}}  9.cow\\
  \glt ‘the child saw the cow’    

     \ex\label{ex:ngunga:22b} \gll mw-ana  a-da-\textbf{yi}{}-won-a,  (ng’ombe).\\
  {1}{}-child  {\textsc{1sm}-\textsc{pfv}}{}-{\textsc{om}9-}see-{\textsc{fv}}  9.cow\\
  \glt ‘the child saw it, (the cow)’
    \z
\ex\label{ex:ngunga:23}
\ea\label{ex:ngunga:23a} \gll  mw-ana  a-da-won-a    mu-ti.\\
  {1}{}-child  {\textsc{1sm}-\textsc{pfv}}{}-see-{\textsc{fv}}  3-tree\\
\glt  ‘the child saw the tree’    

\ex\label{ex:ngunga:23b} \gll mw-ana  a-da-\textbf{wu}{}-won-a,  (mu-ti).\\
  {1}{}-child  {\textsc{1sm}-\textsc{pfv}}{}-{\textsc{om}9-}see-{\textsc{fv}}  3-tree\\
\glt    ‘the child saw it, (the tree)’
    \z
\z

The examples presented above illustrate that in Cinyungwe the OM is not obligatory with a specific object argument because as can be seen \REF{ex:ngunga:22} the object is an animate and in \REF{ex:ngunga:23}, the object is an inanimate. Nonetheless, the co-occurrence of non-animate NPs and object marker is related to definiteness or specificity. In \REF{ex:ngunga:22}, the object argument is animate while in \REF{ex:ngunga:23}, the object argument is non-animate. 

However, in contrast to what we described in \REF{ex:ngunga:22} and \REF{ex:ngunga:23} above, object marking with the verb -\textit{wona} ‘to see’ is obligatory. See the examples in \REF{ex:ngunga:24} and \REF{ex:ngunga:25} below.  

\ea\label{ex:ngunga:24}
\ea\label{ex:ngunga:24a} \gll  a-da-{??}(\textbf{mu})\textsubscript{i}{}-won-a  iye\textsubscript{i}  dzulo.\\
  {\textsc{1sm}-\textsc{pfv}}{}-{\textsc{om}1}{}-see-{\textsc{fv}}  he  yesterday\\
\glt  ‘he saw him yesterday’    

\ex\label{ex:ngunga:24b} \gll  a-da-{??}(\textbf{wa})-won-a  iwo  dzulo.\\
  {\textsc{1sm}-\textsc{pfv}}{}-{\textsc{om}2}{}-see-{\textsc{fv}}  they  yesterday\\
\glt  ‘they saw them yesterday’

\ex\label{ex:ngunga:24c} \gll a-da-{??}(\textbf{wa})-won-a  yavu    dzulo.\\
  {\textsc{1sm}-\textsc{pfv}}{}-{\textsc{om}2}{}-see-{\textsc{fv}}  grandma  yesterday\\
\glt  ‘they saw her (the grandma) yesterday’
    \z
\z

\ea\label{ex:ngunga:25}
\ea\label{ex:ngunga:25a} \gll  a-da-\textbf{mu}\textsubscript{i}{}-pas-a  iye\textsubscript{i}  ci-mbamba.  \\
  {\textsc{1sm}-\textsc{pfv}}{}-{\textsc{om}1}{}-give-{\textsc{fv}}  he  7-beans\\
\glt  ‘he gave him beans’ 

\ex\label{ex:ngunga:25b} \gll a-da-\textbf{wa}\textsubscript{i}{}-pas-a  iwo\textsubscript{i}  ci-mbamba.  \\
  {\textsc{1sm}-\textsc{pfv}}{}-{\textsc{om}2}{}-give-{\textsc{fv}}  they  7-beans\\
\glt  ‘he gave them beans’

\ex\label{ex:ngunga:25c} \gll a-da-\textbf{mu}\textsubscript{i}{}-pas-a  ci-mbamba  mayi.  \\
{\textsc{1sm}-\textsc{pfv}}{}-{\textsc{om}1}{}-give-{\textsc{fv}}  7-beans  1.mother\\
\glt ‘he gave her (the mother) beans’
    \z
\z

The data presented in \REF{ex:ngunga:25} illustrate that in Cinyungwe object marking is obligatory with pronominal objects with the verb -\textit{won}{}- ‘to see’. Moreover, examples \REF{ex:ngunga:24c} and \REF{ex:ngunga:25c} illustrate that the obligatoriness of the object marker in the verb may relate to the verb and not the pronominal object per se. 

\subsubsection{(iii) Are there locative object markers?}

In Cinyungwe, locative objects can be expressed by locative object markers and they can co-occur with their overt locative nouns but not in neutral context. Consider the examples in \REF{ex:ngunga:26}: 

\ea\label{ex:ngunga:26}
\ea[]{\label{ex:ngunga:26a} \gll pa-xikola\textsubscript{i},  nd-a-(\textbf{pa\textsubscript{i}}){}-yend-a.\\
16-school  {\textsc{sm1sg}}{}-{\textsc{pfv}-(\textsc{om}16}){}-go-{\textsc{fv}}\\
\glt ‘to school, I (really) went to (there)’}

\ex[]{\label{ex:ngunga:26b} \gll ku-muyi\textsubscript{i},  u-ndza-(ku\textsubscript{i}){}-pit-a\\
  17-home  {\textsc{sm2sg}-\textsc{fut}-(\textsc{om}17}){}-pass-{\textsc{fv}}\\
\glt ‘home, you will (really) pass by (it)’}

\ex[*]{\label{ex:ngunga:26c} \gll mu-nyumba  u-da-\textbf{mu}{}-pit-a\\
  18-house  {\textsc{sm2sg}-\textsc{pfv}-\textsc{om}18}{}-pass-{\textsc{fv}}\\
\glt Intd: ‘inside the house, you will pass by’}

\ex[]{\label{ex:ngunga:26d} \gll pa-xikola\textsubscript{i}  nd-a-*(\textbf{pa\textsubscript{i}}){}-won-a.\\
16-school  {\textsc{sm1sg}}{}-{\textsc{pfv}-(\textsc{om}16}){}-see-{\textsc{fv}}\\
\glt ‘school I saw (there)’}
\z
\z

In \REF{ex:ngunga:26a} and \REF{ex:ngunga:26b}, we see that only class 16 and 17 locative objects can be expressed by locative object markers on the verb and that locative object markers cannot co-occur with locative objects in the same clause. It is important to note that this co-occurrence happens when the speaker wants to expresses his or her knowledge concerning an issue. In example \REF{ex:ngunga:26c}, class 18 cannot be expressed by a locative object marker in the verb structure while in \REF{ex:ngunga:26d}, omission of the object marker renders the sentence ungrammatical. This means that the verb \textit{{}-wona} ‘to see’ requires an object marker. 

\subsubsection{(iv) Is object marking restricted to one object marker per verb?}

Another parameter discussed in \citet{MartenKula2012} that we focus on here concerns the number of object markers that can occur in an inflected verb structure. In Cinyungwe, only one object marker per inflected verb is permitted. See example \REF{ex:ngunga:27}:

\ea\label{ex:ngunga:27}
\ea\label{ex:ngunga:27a} \gll  mw-ana  a-da-won-es-a    Siriza  mu-ti.\\
  {\textsc{}1}{}-child  {\textsc{sm1}-\textsc{pfv}}{}-see-{\textsc{caus}}{}-{\textsc{fv}}  Siriza  3-tree\\
\glt  ‘the child made Siriza see the tree’    

\ex\label{ex:ngunga:27b} \gll mw-ana  a-da-(*\textbf{mu)}{}-\textbf{wu}{}-won-es-a,  Siriza  mu-ti.\\
  {\textsc{}1}{}-child  {\textsc{sm1}-\textsc{pfv}}{}-{\textsc{(om9)}-}{\textsc{om}3-}see-{\textsc{caus}}{}-{\textsc{fv}}  Siriza  3-tree\\
\glt    ‘the child saw it, (the tree)’

\ex\label{ex:ngunga:27c} \gll  mw-ana  a-da-(*\textbf{wu)}{}-\textbf{mu}{}-won-es-a,  Siriza  mu-ti.\\
  {\textsc{}1}{}-child  {\textsc{sm1}-\textsc{pfv}}{}-{\textsc{(om3)}-}{\textsc{om}1-}see-{\textsc{caus}}{}-{\textsc{fv}}  Siriza  3-tree\\
\glt    ‘the child saw it, (the tree)’
    \z
\z

Example \REF{ex:ngunga:27b} illustrates that only one object marker is permitted. In this sentence the class 3 object marker occurs immediately before the verb root and \REF{ex:ngunga:27c} shows that changing the order of the object markers does not alter the ungrammaticality of the sentence. 

\subsubsection{(v) Can either benefactive or theme objects be expressed by an object marker in double object constructions?}

In Cinyungwe, either benefactive or theme objects can be expressed by an object marker in double object constructions. This is illustrated in the examples in \REF{ex:ngunga:28}.

\ea\label{ex:ngunga:28}
\ea\label{ex:ngunga:28a} \gll Mayi  a-da-\textbf{mu}{}-phik-ir-a    ci-manga,  Siriza.\\
Mayi  {\textsc{sm1}-\textsc{pfv}}{}-{\textsc{om}1}{}-cook-{\textsc{appl}-\textsc{fv}} 7-maize  Siriza\\
\glt ‘the mother cooked her (Siriza) maize’

\ex\label{ex:ngunga:28b} \gll Mayi  a-da-\textbf{ci}{}-phik-ir-a    Siriza,  ci-manga.\\
Mayi  {\textsc{sm1}-\textsc{pfv}}{}-{\textsc{om}7}{}-cook-{\textsc{appl}-\textsc{fv}} Siriza  7-maize  \\
\glt ‘the mother cooked Siriza it (the maize)’
\z
\z

 \subsubsection{(vi) Is an object marker required/optional/disallowed in object relatives?}

In Cinyungwe, object markers are generally optional in object relative clauses \REF{ex:ngunga:29}, although again it is not allowed to mark the object argument with the verb -\textit{won-} ‘to see’ \REF{ex:ngunga:29}.  

\ea\label{ex:ngunga:29} \gll ma-bvembe  y-omwe   mayi    a-ndza-(ma)-bweres-a  yanitapira \ldots\\
      6-watermelon  {\textsc{rel}-6}   1.mother  {\textsc{sm1}-\textsc{fut}}{}-({\textsc{om}6}){}-bring-{\textsc{fv}}    sweet\\
\glt      ‘the watermelons that mum shall bring (them) are sweet’
\z

\ea\label{ex:ngunga:30} \gll ma-bvembe  y-omwe  mayi              a-ndza-*(ma)-won-a yanitapira \ldots\\
      6-watermelon  {\textsc{rel}-6}   1.mother  {\textsc{sm1}-\textsc{fut}}{}-({\textsc{om}6}){}-see-{\textsc{fv}}  sweet\\
\glt      ‘the watermelons that mum shall see (them) are sweet’
\z

The difference between the examples in \REF{ex:ngunga:29} and \REF{ex:ngunga:30} reflects the different object marking properties associated with different verb types in Cinyungwe. We do not explore the impact of the verb types on object marking properties in any further detail here although this would be a good avenue for future research. 

It is important to note that verb types are not part of the \citet{MartenKula2012} parameters and in this paper, we add verb types as seventh parameter. In terms of the parameters under examination here, the answers for Cinyungwe are “yes” for the five parameters (ii), (iii) and (iv), (v) and (vi) and “no” for (i) and (ii) (\tabref{tab:ngunga:2}).

\begin{table}
\caption{\label{tab:ngunga:2} Parametric variation in object marking in Cinyungwe}
\begin{tabularx}{\textwidth}{l@{~}Ql}
\lsptoprule
(i) & Can the object marker and the object argument co-occur? & \ding{55}\\
(ii) & Is an object marker obligatory with particular object NPs? & \ding{55}\\
(iii) & Are there locative object markers? & \ding{51}\\
(iv) & Is object marking restricted to one object marker per verb? & \ding{51}\\
(v) & Can either benefactive or theme objects be expressed by an object marker in double object constructions? & \ding{51}\\
(vi) & Is an object marker required\slash optional\slash disallowed in object relatives? & \ding{51}\\
(vii) & Is an object marker obligatory with particular verb? & \ding{55}\\
\lspbottomrule
\end{tabularx}
\end{table}

After presenting data of object marking in Cinyungwe, in the next section we look at object marking in Citshwa. 

\subsection{Object marking in Citshwa}\label{sec:ngunga:3.2}

Citshwa is a Mozambican Bantu language with 693,386 speakers. Speakers are found in the three southern provinces Inhambane, Gaza and Maputo and in two central provinces Manica and Sofala (\citealt{NgungaFaquir2011}). Citshwa has six dialects: Xikhambani, spoken in Panda District; Xirhonga, spoken in Massinga; Xihlengwe, spoken in Morrumbene, Massinga and Funhalouro Districts; Ximhandla, spoken in Vilankulo District; Xidzhonge (or Xidonge), spoken in Inharrime District; Xidzivi, spoken in Morrumbene and Homoine Districts. The data analyzed in this paper were provided by a speaker of the Ximhandla dialect via elicitation. 

\subsubsection{(i) Can the object marker and the object argument co-occur?}


\citet{Ngunga2014} shows that an object marker and the object argument can co-occur in Citshwa, and provides the examples in \REF{ex:ngunga:31} to support this observation.

\ea\label{ex:ngunga:31}  Tshwa (S51) (\citealt[187]{Ngunga2014})\\
     \ea\label{ex:ngunga:31a} \gll  Polina    a-nyik-ile    pawu    ci-n’wanana\\
  Polina    {\textsc{sm1}}{}-give-{\textsc{pfv}} {5}.bread    {7}{}-child(a.small.one)\\
    \glt ‘Polina gave the child some bread’

    \ex\label{ex:ngunga:31b} \gll Polina    a-\textbf{ci}\textsubscript{i}{}-nyik-ile    ci\textsubscript{i}{}-n’wanana\textsubscript{i}    pawu\\
  Polina    {\textsc{sm1}}{}-{\textsc{om}7}{}-give-\textsc{pfv}     7.child(a.small.one)   {\textsc{}5}.bread\\
    \glt ‘Polina gave the child some bread’

    \ex\label{ex:ngunga:31c} \gll  Polina    a-\textbf{ci}{}-nyik-ile    pawu\\
  Polina    {\textsc{sm1}-\textsc{om}7}{}-give-\textsc{pfv}  5.bread  \\
\glt ‘Polina gave her bread’

    \ex\label{ex:ngunga:31d} \gll  Polina    a-\textbf{gi}{}-nyik-ile    \\
  Polina    {\textsc{sm1}-\textsc{om}7}{}-give-{\textsc{pfv}}      \\
\glt ‘Polina gave her (it)’
    \z
\z

In \REF{ex:ngunga:31a}, there is no OM present in the verb structure. In \REF{ex:ngunga:31b}, the class 7 OM prefix is co-referential with the indirect lexical object NP \textit{cin’wanana} ‘child’. These examples show that when there are two objects, a direct and an indirect object, that it is the indirect object with which the OM in the verb structure agrees. It is also worth noting that the word order changes in such cases. While in \REF{ex:ngunga:31a} the word order is S-V-DO-IO (subject, verb, direct object, indirect object), in \REF{ex:ngunga:31b} the word order is S-V-IO-DO, which seems to suggest a locality (adjacency) principle in the agreement between the OM and the indirect lexical object. In \REF{ex:ngunga:31c}, the indirect object noun is not realized but the construction is acceptable if it is part of a conversation where the referent can be recovered from context. This is also what happens in \REF{ex:ngunga:31d} where the OM cross-references a class 7 noun. 

In \REF{ex:ngunga:32}, we present another example which shows that there are important different pragmatic interpretations to be considered when the object marker and the object argument co-occur in an intransitive verb in Citshwa. According to our consultant, in example \REF{ex:ngunga:32b}, the co-occurrence of the object marker and the object argument does not appear out of the context. For him, any Citshwa speaker hearing this sentence out of the context can ask, “Which meat are you talking about?”, “Why are you telling me that?”. Thus, it seems like in \REF{ex:ngunga:32b}, we are talking about a specific meat. In Citshwa, OM-doubling brings this specificity and giveness reading of the object. That is why we propose that in Citshwa, the object and the co-referring direct object cannot co-occur out of the blue. 

\ea\label{ex:ngunga:32}
\ea\label{ex:ngunga:32a} \gll   mu-fana   w-a-g-a      nyama\\
1-boy    {\textsc{sm1}.\textsc{prs}}{}-eat-\textsc{fv}  9.meat  \\
\glt ‘The boy eats the meat’

\ex\label{ex:ngunga:32b} \gll  mu-fana   w-a-\textbf{yi}{}-g-a    nyama\\
1-boy    {\textsc{sm1}.\textsc{prs}}{}-{\textsc{om}9}{}-eat-\textsc{vf}  9.meat  \\
\glt ‘The boy eats the meat’
    \z
\z

In Citshwa, there are cases where the co-occurrence of the object argument and the OM within the same sentence has a different meaning to the one described in \REF{ex:ngunga:32b} above. Thus, if the speaker avoids the co-occurrence of the object argument and the object marker by dislocating the object argument, this results in emphasis on how the boy loves eating meat. An example of the co-occurrence of the object argument and the object marker and the resulting interpretation is shown in \REF{ex:ngunga:33} below:

\ea\label{ex:ngunga:33}
\gll mu-fana   w-a-\textbf{yi}{}-g-a,    nyama\\
1-boy    {\textsc{sm1}.\textsc{prs}}{}-{\textsc{om}9}{}-eat-\textsc{vf}  9.meat  \\
\glt ‘the boy eats a lot of meat’
\z

The example in \REF{ex:ngunga:33} can also have a totality interpretation when the speaker is telling the hearer not to be afraid thinking that the boy shall not finish the meat the hearer is  giving him because as he knows, the boy loves meat and he can eat it with the bones. 

\subsubsection{(ii) Is an object marker obligatory with particular object NPs in Citshwa?}

  In Citshwa, animate as well as inanimate objects can appear with the object marker, although its presence is optional in both cases. See the examples in (\ref{ex:ngunga:34}a--d), presented below.

\ea\label{ex:ngunga:34}
\ea\label{ex:ngunga:34a} \gll Zabhela  a-won-ile    mbzana\\
Zabhela  {\textsc{sm1}}{}-see-{\textsc{pfv}}    9.dog\\
\glt ‘Zabhela saw a dog’ 

\ex\label{ex:ngunga:34b} \gll Zabhela  a-(\textbf{yi})-won-ile    mbzana\\
Zabhela  \textsc{sm1}-({\textsc{om}9}){}-see-{\textsc{pfv}}  9.dog\\
\glt ‘Zabhela saw it’ 

\ex\label{ex:ngunga:34c} \gll Tereza    a-tsal-ile    papilu\\
Tereza    \textsc{sm1}-write-\textsc{pfv}    5.letter\\
\glt ‘Teresa wrote the letter’

\ex\label{ex:ngunga:34d} \gll Tereza    a-(\textbf{gi})-tsal-ile    papilu\\
Tereza    \textsc{sm1}-({\textsc{om}5}){}-write-{\textsc{pfv}}  5.letter\\
\glt ‘Tereza wrote it’
\z
\z

The examples presented in \REF{ex:ngunga:34} above illustrate that object marking is not obligatory. That is, sentences in (\ref{ex:ngunga:34}a--b) are still grammatical even if the object marker is not present. In (\ref{ex:ngunga:34}c--d), where the object argument is an inanimate argument, we can see that the occurrence of the object marker in the verb structure is still not obligatory. The difference between these examples is that \REF{ex:ngunga:34a} and \REF{ex:ngunga:34c} are statements and \REF{ex:ngunga:34b} and \REF{ex:ngunga:34d} are context-based sentence structures. They are used to clarify what was not previously understood in the first statement. There is also another interpretation statement that can be added in the interpretation of \REF{ex:ngunga:34b} and \REF{ex:ngunga:34d}. For our Citshwa speaker, the example in \REF{ex:ngunga:34b} and \REF{ex:ngunga:34d} can also be used for emphatic purposes where the speaker is trying to make clear how beautiful the words written in the paper were. 

Thus, the OM may not be obligatory but the presence or absence of the OM changes the interpretation of each sentence. 

\subsubsection{(iii) Are there locative object markers?}

In Citshwa, there are no locative prefixes of the form similar to the ones we described in Cinyungwe (cf. \sectref{sec:ngunga:3.1}). Therefore, locativization is expressed by the suffix -\textit{eni} attached to the NP. The examples in \REF{ex:ngunga:35} illustrate that the locative object marker is only recovered from the verb for class 17.

\ea\label{ex:ngunga:35}
\ea\label{ex:ngunga:35a} \gll ci-kolw-eni,   u-ta-famb-a \\
7-school-\textsc{loc}  {\textsc{sm2sg}-\textsc{fut}}{}-go-{\textsc{fv}}   \\
\glt ‘to school, you will go’


\ex\label{ex:ngunga:35b}
\gll ci-kolw-eni,   u-ta-\textbf{ku}{}-famb-a  \\
7-school-\textsc{loc}  {\textsc{sm2sg}-\textsc{fut}}{}-{\textsc{om}17}{}-go-{\textsc{fv}}   \\
\glt ‘to school, you will go there (wanting or not)’


\ex\label{ex:ngunga:35c}
\gll ndlw-ini,   u-ta-nghen-a  \\
10-house-\textsc{loc}  {\textsc{sm2sg}-\textsc{fut}}{}-enter-{\textsc{fv}}   \\
\glt ‘in the house, you will get in’


\ex\label{ex:ngunga:35d}
\gll ndlw-ini,   u-ta-(\textbf{ku)}{}-nghen-a  \\
10-house-\textsc{loc}  {\textsc{sm2sg}-\textsc{fut}}{}-{\textsc{om}17-}enter-{\textsc{fv}}   \\
\glt ‘in the house, you will get in’
\z
\z

The absence of examples with locative object markers for class 16 and 18 in \REF{ex:ngunga:35} indicates that Citshwa does not have OMs for these classes. The answer to this parameter from \citet{MartenKula2012} is therefore ``no'' for Citshwa.

 \subsubsection{(iv) Is object marking restricted to one object marker per verb?}

In Citshwa, only one object can be realised as an object marker for each inflected verb. See the examples in \REF{ex:ngunga:36} and \REF{ex:ngunga:37}: 

\ea\label{ex:ngunga:36}
\ea\label{ex:ngunga:36a} \gll bava    a-bhik-is-a    zva-kuga  nhanyana\\
1.father  {\textsc{sm1}.\textsc{prs}-}cook{{}-\textsc{caus}-\textsc{fv}}  8-food    1.girl      \\
\glt ‘the father made the girl cook the food’


\ex\label{ex:ngunga:36b} \gll  bava    a-\textbf{mu}{}-bhik-is-a    zva-kuga  nhanyana   \\
1.father  {\textsc{sm1}.\textsc{prs}-\textsc{om}1-}cook{{}-\textsc{caus}-\textsc{fv}}  8-food    1.girl\\
\glt ‘the father made her (the girl) cook the food’


\ex\label{ex:ngunga:36c} \gll bava    a-(*\textbf{zva})-\textbf{mu}{}-bhik-is-a  zva-kuga  nhanyana \\
1.father  {\textsc{sm1}.\textsc{prs}-(\textsc{om}8)-\textsc{om}1-}cook{{}-\textsc{caus}-\textsc{fv}}  8-food    1.girl\\
\glt Intd: ‘the father made her (the girl) cook it (the food)’


\ex\label{ex:ngunga:36d} \gll bava    a-(*\textbf{mu})-\textbf{zva}{}-bhik-is-a  zva-kuga  nhanyana \\
1.father  {\textsc{sm1}.\textsc{prs}-(\textsc{om}1})-\textsc{om}8-cook{{}-\textsc{caus}-\textsc{fv}}  8-food    1.girl\\
\glt Intd: ‘the father made her (the girl) cook it (the food)’
\z
\ex\label{ex:ngunga:37}
\ea\label{ex:ngunga:37a} \gll mamani  a-rim-el-a      bava    zvi-pfhaki.\\
1.mother  {\textsc{sm1}.\textsc{prs}}{}-cultivate-{\textsc{appl}}{}-{\textsc{fv}}    1.father  8-maize\\
\glt ‘the mother cultivates maize for the father’

\ex\label{ex:ngunga:37b} \gll  mamani  wa-(*\textbf{zvi)}{}-\textbf{mu}-rim-el-a    bava      zvi-pfhaki.\\
1.mother    {\textsc{sm1}.\textsc{prs}}{}-({\textsc{om}8}){}-{\textsc{om}1}{}-cultivate-{\textsc{appl}}{}-{\textsc{fv}}  1.father  8-maize\\
\glt ‘the mother cultivates it (the maize) for him (the father)’

\ex\label{ex:ngunga:37c} \gll mamani  wa-(*\textbf{mu})-\textbf{zvi}{}-rim-el-a    bava    zvi-pfhaki\\
1.mother    {\textsc{sm1}.\textsc{prs}}{}-({\textsc{om}1}){}-{\textsc{om}8}{}-cultivate-{\textsc{appl}}{}-{\textsc{fv}}  1.father  8-maize\\
\glt ‘the mother cultivates it (the maize) for him (the father)’
    \z
\z

In \REF{ex:ngunga:36} and \REF{ex:ngunga:37}, we have examples that illustrate that there is a space for only one object marker in the Citshwa verb structure. 

\subsubsection{(v) Can either benefactive or theme objects be expressed by an object marker in double object constructions?}

In Citshwa, either the benefactive or theme object can be expressed by an object marker in double object constructions. This is illustrated by examples \REF{ex:ngunga:38b} and \REF{ex:ngunga:38c} below which illustrate that either the benefactive or the theme object can be object marked. 

\ea\label{ex:ngunga:38}
\ea\label{ex:ngunga:38a} \gll bava    a-nyik-a  ti-manga  mu-nghana.\\
1.father  {\textsc{sm1}.\textsc{prs}}{}-give-{\textsc{fv}}  10-peanuts  1-friend\\
\glt ‘the father gave the friend peanuts’

\ex\label{ex:ngunga:38b} \gll bava    wa-\textbf{ti}{}-nyik-el-a    mu-nghana.\\
1.father  {\textsc{sm1}.\textsc{prs}}{}-{\textsc{om}10}{}-give-{\textsc{appl}}{}-{\textsc{fv}}  1-friend  \\
\glt ‘the father is giving them (the peanuts) on behalf of his friend’

\ex\label{ex:ngunga:38c} \gll bava    wa-\textbf{mu}{}-nyik-el-a    ti-manga.\\
1.father  {\textsc{sm1}.\textsc{prs}}{}-{\textsc{om}1}{}-give-{\textsc{appl}}{}-\textsc{fv}  10-peanuts\\
\glt ‘the father is giving the peanuts for him (the friend)’
    \z
\z

This means that Citshwa is a ``symmetrical'' language with respect to object marking in double constructions (cf. \citealt{BresnanMoshi1990}).

\subsubsection{(vi) Is an object marker required/optional/disallowed in object 
 relatives?}

The last parameter presented by \citet{MartenKula2012} has do to with the availability of object markers in relative clauses. In Citshwa, object markers are obligatory only with the verb -\textit{won}{}- ‘to see’. Compare the examples \REF{ex:ngunga:39} and \REF{ex:ngunga:40}.    

\ea\label{ex:ngunga:39} \gll a-ma-din’wa   a-nga-(\textbf{ma})-xav-a     mamani  ma-andziha\\
{\textsc{aug}}{}-6-orange  {\textsc{sm1}-\textsc{perf}.\textsc{rel}-(\textsc{om}6}){}-eat-{\textsc{fv}}    1.mother  6-sweets   \\
\glt ‘the oranges that mother bought (them) are sweet’
\ex\label{ex:ngunga:40} \gll  a-madin’wa  a-nga-*(\textbf{ma)}{}-won-a  mamani  ma-nandziha  \\
{\textsc{aug}}{}-6-orange  {\textsc{sm1}.\textsc{perf}.\textsc{rel}-(}\textsc{om}6){}-see-{\textsc{fv}}  1.mother  6-sweet  \\
\glt ‘the oranges that mother saw (them) are sweet’
\z


The example presented in \REF{ex:ngunga:39} illustrates that object markers are not obligatory in Citshwa. However, just like we saw when we were analyzing object marking in Cinyungwe, example \REF{ex:ngunga:40}, shows that it is obligatory to object mark the object argument in relative constructions in Citishwa. This can be related to what we described in section 3.2.1, in Citshwa the co-occurence of the object marker and the object argument is disallowed, making them optional. \tabref{tab:ngunga:3} summarizes the object marking properties in Citshwa. 

\begin{table}
\caption{\label{tab:ngunga:3} Parametric variation in object marking in Citshwa}

\begin{tabularx}{\textwidth}{lQl}

\lsptoprule

(i) & Can the object marker and the object argument co-occur? & \ding{51}\\
(ii) & Is an object marker obligatory with particular object NPs? & \ding{55}\\
(iii) & Are there locative object markers? & \ding{55}\\
(iv) & Is object marking restricted to one object marker per verb? & \ding{51}\\
(v) & Can either benefactive or theme objects be expressed by an object marker in double object constructions? & \ding{51}\\
(vi) & Is an object marker required/optional/disallowed in object relatives? & \ding{51}\\
(vii) & Is an object marker obligatory with particular verb? & \ding{55}\\
\lspbottomrule
\end{tabularx}
\end{table}

\subsection{Object marking in Ciwutee}\label{sec:ngunga:3.3}

In the present section we look at Ciwutee, spoken by 259,790 people in the central province of Manica. 

\subsubsection{(i) Can the object marker and the object argument co-occur in Ciwutee?}

As we saw for Cinyungwe and Citshwa in \sectref{sec:ngunga:3.1} and \sectref{sec:ngunga:3.2} above, in Ciwutee the object marker and the corresponding object argument cannot co-occur out of the blue. It seems like there is both a specificity/givenness component in OM-doubling. See the examples \REF{ex:ngunga:41b} and \REF{ex:ngunga:41d}. 

\ea\label{ex:ngunga:41}
\ea\label{ex:ngunga:41a} \gll    mhondolo  y-a-rum-a  mbudzi \\
      9.lion    {\textsc{sm9}.\textsc{pfv}}{}-bite-{\textsc{fv}}  9.goat\\
    \glt ‘the lion bit the goat’

\ex\label{ex:ngunga:41b} \gll mhondolo  y-a-\textbf{yi}{}-rum-a    (\#mbudzi) \\
             9.lion    {\textsc{sm9}.\textsc{pfv}-\textsc{om}9}{}-bite-{\textsc{fv}}      9.goat\\
\glt    ‘the lion bit it (the goat)’

\ex\label{ex:ngunga:41c} \gll   mwaramu    a-tem-a    mu-ti\\
            1.brother-in-low  {\textsc{sm1}.\textsc{pfv}}{}-cut-{\textsc{fv}}    3-tree\\
\glt            ‘the brother-in-low cut the tree’

\ex\label{ex:ngunga:41d} \gll   mwaramu    a-\textbf{wu}{}-tem-a    (\#mu-ti)\\
    1.brother-in-low  {\textsc{sm1}.\textsc{pfv}-\textsc{om}3}{}-cut-{\textsc{fv}}  3-tree\\
\glt    ‘the brother-in-low cut it (the tree)’
    \z
\z

The examples in \REF{ex:ngunga:41b} and \REF{ex:ngunga:41d} illustrate that in Ciwutee the object marker and the object argument cannot co-occur in neutral context regardless of the animacy of the object argument. In addition, our consultant also suggested that it seems like all lexical objects behave similarly in that the object marker is prohibited to co-occur with the \textit{in situ} object argument out of the blue. According to our speaker, the examples in \REF{ex:ngunga:41b} and \REF{ex:ngunga:41d} reflect this specificity and giveness reading of the object. This is the reason a Ciwutee speaker hearing this sentence out of context can ask ``Which goat or tree are we talking about''? or ``Why are you telling me that?''. This restriction reminds us of what we described for Citshwa in \sectref{sec:ngunga:3.2}. 

  Moreover, our consultant argued that there are contexts in which the examples in \REF{ex:ngunga:41b} and \REF{ex:ngunga:41d} presented above can be used by the speaker to illustrate that they have evidence, knows the person or the fact described, witnessed it (for more details about evidentiality in Bantu see \citealt{LippardEtAl2021}). In such cases, the examples \REF{ex:ngunga:42a} and \REF{ex:ngunga:42b} repeated again from \REF{ex:ngunga:41c} and \REF{ex:ngunga:41d}, can have the following translation in English. 

\ea\label{ex:ngunga:42}
\ea\label{ex:ngunga:42a} \gll    mhondolo  y-a-\textbf{yi}{}-rum-a    (\#mbudzi) \\
      9.lion    {\textsc{sm9}-\textsc{pfv}-\textsc{om}9}{}-bite-{\textsc{fv}}  9.goat\\
      \glt ‘the lion certainly bit it (the goat)’

\ex\label{ex:ngunga:42b} \gll   mwaramu    a-\textbf{wu}{}-tem-a    (\#muti)\\
     1.brother-in-law  {\textsc{sm1sg}.\textsc{pfv}-\textsc{om}3}{}-cut-{\textsc{fv}}  3-tree\\
\glt       ‘the brother-in-law certainly cut it (the tree)’ 
    \z
\z

As noted above, according to our informant, the Ciwutee speaker can use the OM-doubling structures to tell the hearer that they have evidence of what they are talking about. The speaker is not expressing an opinion, they ar etelling the hearer what they know and so does not want to be challenged about the issue. If the other person insists, arguing about the same issue, this sentence can be used to say ``hear want I am saying and let’s end the conversation''. 

\subsubsection{(ii) Is an object marker obligatory with particular object NPs?}

As noted by \citet{MartenKula2012} amongst others, Bantu languages differ with respect to the obligatoriness of co-occurrence of the OM with specific object argument. As has been described in a few other Mozambican Bantu languages such as Makhuwa (\citealt{vanderWal2015}), Cuwabo \citep{Guérois2015} and Shimakonde \citep{NgungaEtAl2016}, in some languages it is obligatory to object-mark class 1 and class 2 nouns (Makhuwa and Echuwabo) and animate objects (Shimakonde). This is not the case in Ciwutee where it is not obligatory to object mark particular objects. Example \REF{ex:ngunga:43} illustrates that object marking is not obligatory with animate objects, while example \REF{ex:ngunga:44} illustrates that object marking is not obligatory with inanimate objects. 

\ea\label{ex:ngunga:43}
\ea\label{ex:ngunga:43a} \gll nd-a-won-a    Zhambato\\
{\textsc{sm1sg}-\textsc{pfv}}{}-see-\textsc{fv}    Zhambato\\
\glt ‘I saw Zhambato’  

\ex\label{ex:ngunga:43b} \gll  nd-a-(\textbf{mu)-}won-a  Zhambato.\\
{\textsc{sm1sg}-\textsc{pfv}}{}-(\textsc{om5})-see-\textsc{fv}    Zhambato.\\
\glt ‘I saw him (Zhambato)’
    \z
\ex\label{ex:ngunga:44}
\ea\label{ex:ngunga:44a} \gll nd-a-won-a    bhuku.\\
{\textsc{sm1sg}-\textsc{pfv}}{}-see-\textsc{fv}    bhuku\\
\glt ‘I saw the book’  

\ex\label{ex:ngunga:44b} \gll nd-a-(\textbf{ri})\textbf{{}-}won-a  bhuku.\\
{\textsc{sm1sg}-\textsc{pfv}}{}-(\textsc{om5})-see-\textsc{fv}    book\\
\glt ‘I saw it (the book)’
\z
\z

 \subsubsection{(iii) Are there locative object markers?}

The presence or absence of locative markers in Ciwutee is the third parameter of variation examined by \citet{MartenKula2012}. Ciwutee has locative prefixes and they can be expressed by locative objects and similar to Cinyungwe, they can co-occur with their corresponding overt locative nouns. Consider the examples in \REF{ex:ngunga:45}:        

\ea\label{ex:ngunga:45}
\ea\label{ex:ngunga:45a} \gll   ku-munda  ndi-no-*(ku)-ziy-a.\\
           17-field  {\textsc{sm1sg}-\textsc{prs}-(\textsc{om}17}){}-know-{\textsc{fv}}\\
\glt          ‘in the field of cultivation (there), I know’

\ex\label{ex:ngunga:45b} \gll   ku-munda  ndi-no-(ku)-won-a.\\
            17-field  {\textsc{sm1sg}-\textsc{prs}-(\textsc{om}17}){}-see-{\textsc{fv}}\\
\glt ‘the field of cultivation I saw (it)’    

\ex\label{ex:ngunga:45c} \gll mu-mvura  ndi-no-(mu)-pind-a\\
18-marsh  {\textsc{sm1sg}-\textsc{prs}-(\textsc{om}18}){}-enter-{\textsc{fv}}  \\
\glt ‘in the marsh, I enter!’

\ex\label{ex:ngunga:45d} \gll pa-nyumba  ndi-no-(pa)-gum-a\\
  16-home  1\textsc{sm1sg}-\textsc{prs}-(\textsc{om16})-arrive-\textsc{fv}\\
\glt ‘at home, I arrive’
    \z
\z

Ciwutee has the three locative object markers and they can be used to express locative objects. Examples (\ref{ex:ngunga:45}a--d) show different verbs that illustrate that, in Ciwutee, locative objects can be expressed by locative prefixes. The examples also illustrate that it is obligatory to object mark locative objects when they occur in subject position. The verb -\textit{won}{}- ‘see’ \REF{ex:ngunga:45b} reminds us about what was described for Cinyungwe (example \ref{ex:ngunga:26d}) where we said that it was obligatory to object mark locative objects with the verb -\textit{wona} ‘to see’.

\subsubsection{(iv) Is object marking restricted to one object marker per verb?}

Ciwutee allows only one object marker per inflected verb. This is shown in the examples in \REF{ex:ngunga:46b} and \REF{ex:ngunga:46c} which demonstrate that in the Ciwutee’s verb structure there is only one place for the OM.

\ea\label{ex:ngunga:46}
\ea\label{ex:ngunga:46a} \gll mbiya    a-pas-a    huku    ma-gwere\\
1.grandma  {\textsc{sm1}.\textsc{pfv}}{}-give-{\textsc{fv}}  {9}.chicken  6-maize\\
\glt ‘the grandma gave the chicken maize’    

\ex\label{ex:ngunga:46b} \gll mbiya    *a-\textbf{yi}{}-\textbf{ma}{}-pas-a    ma-gwere.\\
1.grandma  {\textsc{sm1}.\textsc{pfv}}{}-{\textsc{om}9}-{\textsc{om}6}{}-give-{\textsc{fv}}    6-maize\\
\glt ‘the grandma gave it (the chicken) maize’

\ex\label{ex:ngunga:46c} \gll mbiya    *a-\textbf{ma}{}-\textbf{yi}{}-pas-a  huku  \\
1{.}grandma  {\textsc{sm1}.\textsc{pfv}-\textsc{om}6}-{\textsc{om}9}{}-give-\textsc{fv}  9.chicken  \\
\glt ‘the grandma gave it (the maize) to the chicken’
    \z
\z

It is important to note that the ungrammaticality of \REF{ex:ngunga:46b} and \REF{ex:ngunga:46c} is not related to the order of the objects, rather it is a strict restriction on the number of object markers possible in a verb form.

\subsubsection{(v) Can either benefactive or theme objects be expressed by an object marker in double object constructions?}

Ciwutee allows either benefactive or theme objects to be expressed by an object marker. 

\ea\label{ex:ngunga:47}
\ea\label{ex:ngunga:47a} \gll Diminga  w-aka-rim-ir-a      mayi    ci-mbamba.\\
Diminga  {\textsc{sm1-pfv}}-cultivate-\textsc{apl}-\textsc{fv}  1.mother  {7}{}-beans\\
\glt ‘Diminga cultivated beans for the mother’

\ex\label{ex:ngunga:47b} \gll Diminga  w-aka-\textbf{mu-}rim-ir-a      ci-mbamba.\\
Diminga  {\textsc{sm1}-\textsc{pfv}-\textsc{om}1}{}-cultivate-{\textsc{appl}-\textsc{fv}}   {7}{}-beans\\
\glt ‘Diminga (really) cultivated for her (the mother) beans’

\ex\label{ex:ngunga:47c} \gll Diminga  w-aka-\textbf{ci}{}-rim-ir-a    mayi.\\
      Diminga  {\textsc{sm1}-\textsc{pfv}-\textsc{om}7}{}-cultivate-{\textsc{appl}-\textsc{fv}}  {1.}mother\\
\glt      ‘Diminga (really) cultivated them (the beans) for the mother’
    \z
\z

In \REF{ex:ngunga:47a}, the class 1 object marker (-\textit{a}{}-) is co-referential with the object argument \textit{mayi} ‘mother’ and in \REF{ex:ngunga:47b}, the class 7 prefix (-\textit{ci}-) is co-referential with \textit{cimbamba} ‘beans’. Therefore, just like in Cinyungwe and Citshwa, in Ciwutee either benefactive or theme objects can be expressed by an object marker. This means that, Ciwutee is also a ``symmetrical'' language.  

\subsubsection{(vi) Is an object marker required/optional/disallowed in object relatives?}

Different from Cinyungwe and Citshwa, in Ciwutee object markers are optional in object relatives, even with the verb -\textit{won}{}- ‘to see’. 

\ea\label{ex:ngunga:48}
\ea\label{ex:ngunga:48a} \gll nyumba  ya nd-a-(\textbf{yi})-won-a  nja  Mazvarira\\
9.house  \textsc{rel} {\textsc{sm1sg}}-{\textsc{pfv}}{}-({\textsc{om}9}){}-see-{\textsc{fv}}  \textsc{cop}  Mazvarira\\
\glt ‘the house that I saw (it) belongs to Mazvarira’  

\ex\label{ex:ngunga:48b} \gll ma-khebe    mayi    a (a)-aka-(ma)-won-a  akatapira\\
6-watermelon    1.mother \textsc{rel} {\textsc{sm1}-\textsc{pfv}-(\textsc{om}6)-\textsc{fut}}{}-see-{\textsc{fv}}  sweet\\
\glt ‘the watermelon that mother saw (it) was sweet’
    \z
\z

The examples in \REF{ex:ngunga:48a} and \REF{ex:ngunga:48b} show that the verb -\textit{won}{}- ‘to see’ does not need an object marker in the verb structure to render the sentence grammatical. \tabref{tab:ngunga:4} below summarizes what we have presented for object marking in the Ciwutee data so far.  

\begin{table}
\caption{\label{tab:ngunga:4} Parametric variation in object marking in Ciwutee}
\begin{tabularx}{\textwidth}{lQl}

\lsptoprule

(i) & Can the object marker and the object argument co-occur? & \ding{51}\\
(ii) & Is an object marker obligatory with particular object NPs? & \ding{55}\\
(iii) & Are there locative object markers? & \ding{51}\\
(iv) & Is object marking restricted to one object marker per verb? & \ding{51}\\
(v) & Can either benefactive or theme objects be expressed by an object marker in double object constructions? & \ding{51}\\
(vi) & Is an object marker required\slash optional\slash disallowed in object relatives? & \ding{51}\\
(vii) & Is an object marker obligatory with particular verb? & \ding{55}\\
\lspbottomrule
\end{tabularx}
\end{table}

\subsection{Object marking in Ciyaawo} \label{sec:ngunga:3.4}

After describing object marking in Cinyungwe, Ciwutee and Citshwa in the previous sections, in the present section we look at Ciyaawo data. Ciyaawo (P21 in \citegen{Guthrie1967-1971} classification) is a Mozambican Bantu language spoken by 454,185 people in the Mozambican northern province of Niassa. 

\subsubsection{(i) Can the object marker and the object argument co-occur?}


In Ciyaawo, there are no restrictions of co-occurrence of the object argument and the OM within the same sentence, as illustrated in \REF{ex:ngunga:49}.

\ea\label{ex:ngunga:49}
\ea\label{ex:ngunga:49a} \gll  baaba    a-dim-il-e    yi-maanga.\\
  dad  {\textsc{sm1}}-cultivate-{\textsc{pfv}-\textsc{fv}}  {8}{}-maize\\
\glt  ‘dad has cultivated maize’    

\ex\label{ex:ngunga:49b} \gll baaba    a-\textbf{yi}{}-dim-il-e.\\
  dad  {\textsc{sm}1\textsc{}-\textsc{om}8}{}-cultivate-{\textsc{pfv}-\textsc{fv}}  \\
\glt  ‘dad has cultivated it (maize)’  

\ex\label{ex:ngunga:49c} \gll baaba    a-\textbf{yi\textsubscript{i}}{}-dim-il-e      yi-maanga\textsubscript{i}\\
  dad  {\textsc{sm1}1\textsc{}-\textsc{om}8}{}-culitvate-{\textsc{pfv}-\textsc{fv}}  {8}{}-maize\\
\glt  ‘dad has cultivated the maize’  
    \z
\z

The examples in \REF{ex:ngunga:49} are all grammatical and acceptable, although \REF{ex:ngunga:49c} would probably be understood as emphatic to mean something like “dad has cultivated the maize very well”. On the other hand, an OM such as -\textit{yi}{}-, as in the verb structure in \REF{ex:ngunga:49b}, is usually included in the verb structure to respond to a question such as “What did dad do to the maize?”. 

\subsubsection{(ii) Is an object marker obligatory with particular object NPs?}

In Ciyaawo, there are no examples where the occurrence of OM is obligatory. That is, all transitive verbs can accommodate an object marker of any object NP regardless of their noun class. However, this is not obligatory under any circumstance.

\ea\label{ex:ngunga:50}
\ea\label{ex:ngunga:50a} \gll  n’nyamaata    ju-dim-il-e    yi-maanga.\\
{\textsc1}.boy    {\textsc{sm}1\textsc{}-}cultivate-{\textsc{pfv}-\textsc{fv}}  {8}{}-maize\\
\glt  ‘the boy has cultivated maize’    

\ex\label{ex:ngunga:50b} \gll n’nyamaata  ju-\textbf{yi}{}-dim-il-e.\\
  {\textsc{}1}.boy  {\textsc{sm}1\textsc{}-\textsc{om}8}{}-cultivate-{\textsc{pfv}-\textsc{fv}}  \\
\glt  ‘boy has cultivated it (maize)’  

\ex\label{ex:ngunga:50c} \gll n’nyamaata  ju-\textbf{yi\textsubscript{i}}{}-dim-il-e      yi-maanga\textsubscript{i}\\
  {\textsc{}1}.boy  {\textsc{sm1}\textsc{}-\textsc{om}8}{}-culitvate-{\textsc{pfv}-\textsc{fv}}  {8}{}-maize\\
\glt  ‘boy has cultivated the maize’  
    \z
\ex\label{ex:ngunga:51}
\ea\label{ex:ngunga:51a} \gll   ngweena  ji-kamw-iil-e    muu-ndu.\\
   {\textsc{}9}.crocodile  {\textsc{sm}9\textsc{}-}grab-{\textsc{pfv}-\textsc{fv}}  {1}{}-person\\
\glt  ‘the crocodile has grabbed a person’    

\ex\label{ex:ngunga:51b} \gll ngweena    ji-\textbf{n’}{}-kamw-iil-e\\
  {\textsc{}9}.crocodile    {\textsc{sm}9\textsc{}-\textsc{om}1-}grab-{\textsc{pfv}-\textsc{fv}}  \\
\glt  ‘the crocodile has grabbed a person’  

\ex\label{ex:ngunga:51c} \gll ngweena    ji-\textbf{n’}\textsubscript{i}{}-kamw-iil-e    muu-ndu\textsubscript{i}.\\
  {\textsc{}9}.crocodile    {\textsc{sm}1\textsc{}-\textsc{om}1}{}-culitvate-{\textsc{pfv}-\textsc{fv}}  {1}{}-person\\
\glt  ‘the crocodile has grabbed a person’    
\z
\z

The examples \REF{ex:ngunga:50a} and \REF{ex:ngunga:51a} illustrate that in Ciyaawo, the OM is not obligatory with a specific object argument. That is to say that the presence of the OM in any transitive verb structures is not obligatory regardless of the class to which the noun belongs. When the OM occurs with the transitive verb, it may or may not co-occur with the lexical object seen in (\ref{ex:ngunga:50}b, c) and (\ref{ex:ngunga:51}b, c). The examples in \REF{ex:ngunga:50b} and \REF{ex:ngunga:51b} all correspond to questions like “What happened to the maize/person?”, while the examples in \REF{ex:ngunga:50c} and \REF{ex:ngunga:51c} respond to open questions like “What has happened?”. 

\subsubsection{(iii) Are there locative object markers?}

Just like in Cinyungwe and Ciwutee, in Ciyaawo, locative objects can be expressed by locative object markers and they can co-occur with the corresponding overt locative nouns. Consider the examples in \REF{ex:ngunga:52}: 

\ea\label{ex:ngunga:52}
\ea[]{\label{ex:ngunga:52a} \gll  pa-cikoola\textsubscript{i},  n-gu-\textbf{pa\textsubscript{i}}{}-won-a.\\
16-school  {\textsc{sm1sg}-\textsc{prs}-\textsc{om}16}{}-see-{\textsc{fv}}\\
\glt Li.t ‘at school, I see at'\\
Intd: ‘I see the place of the school’}

\ex[]{\label{ex:ngunga:52b} \gll ku-musi\textsubscript{i},  n-gu-\textbf{ku}\textsubscript{i}{}-won-a\\
  17-home  {\textsc{sm1sg}-\textsc{prs}-\textsc{om}17}{}-see-{\textsc{fv}}\\
\glt Lit. ‘to home, I see to (it)'\\
Intd: ‘I see there, the home’}

\ex[]{\label{ex:ngunga:52c} \gll mu-nyumba  n-gu-\textbf{mu}{}-won-a\\
  18-house  {\textsc{sm1sg}-\textsc{prs}-\textsc{om}18}{}-see-{\textsc{fv}}\\
\glt   Lit. ‘inside the house I see it'\\
Intd: `I see the interior of the house’}
\z
\ex\label{ex:ngunga:53}
\ea[*]{\label{ex:ngunga:53a} \gll pa-cikoola,  n-gu-won-a.\\
16-school  {\textsc{sm1sg}1}{}-{\textsc{prs}}{}-see-{\textsc{fv}}\\
\glt Lit. 'at school, I see'\\
Intd: ‘I see the place of the school’}

\ex[*]{\label{ex:ngunga:53b} \gll ku-musi\textsubscript{i},  n-gu-won-a\\
  17-home  {\textsc{sm1sg}-\textsc{prs}}{}-see-{\textsc{fv}}\\
\glt Lit. ‘to home, I see to (it)'\\
Intd: ‘I see there, the home’}

\ex[*]{\label{ex:ngunga:53c} \gll mu-nyumba  n-gu-won-a\\
  18-house  {\textsc{sm1sg}-\textsc{prs}}{}-see-{\textsc{fv}}\\
\glt   Lit. ‘inside the house I see it'\\
Intd: `I see the interior of the house’}
\z
\z

In the examples in \REF{ex:ngunga:52}, the three locative prefixes are used as OMs. In \REF{ex:ngunga:53}, the omission of the locative object marker in the verb structure renders the sentence ungrammatical. This means that the verb \textit{{}-wona} ‘to see’, and other verbs with the same lexical properties, require the object marker regardless of the respective noun class to render the sentence grammatical. In this language, locative prefixes therefore behave in the same way as any other noun class prefixes.

\subsubsection{(iv) Is object marking restricted to one object marker per verb?}

Another parameter discussed in \citet{MartenKula2012} concerns the number of object markers that can occur per inflected verb structure. In Ciyaawo, only one object marker is allowed per inflected verb. Consider the examples in \REF{ex:ngunga:54}:

\ea\label{ex:ngunga:54}
\ea[*]{\label{ex:ngunga:54a}  \gll mw-anace  ju-ku-won-esy-a  nguku    yi-maanga.\\
  {\textsc{}1}{}-child  {\textsc{sm1}\textsc{}-\textsc{prs}}{}-see-{\textsc{caus}}{}-{\textsc{fv}}  9.chicken  8-maize\\}

\ex[]{\label{ex:ngunga:54b} \gll mw-anace  ju-ku-\textbf{ji}{}-(*\textbf{yi)}{}-won-esy-a  nguku  yi-maanga.\\
  {\textsc{}1}{}-child  {\textsc{sm1}\textsc{}-\textsc{prs}}{}-{\textsc{om}9-(\textsc{om}8)-}see-{\textsc{caus}}{}-{\textsc{fv}}  9.chicken 8-maize\\
\glt    Lit: the child is making it (the chicken) see the maize.\\
    ‘the child is making the chicken see the maize’}

\ex[]{\label{ex:ngunga:54c} \gll  mw-anace  ju-ku-(*\textbf{yi)}{}-(\textbf{ji)}{}-won-esy-a,    nguku     yi-maanga.\\
  {\textsc{}1}{}-child  {\textsc{sm1}\textsc{}-\textsc{prs}}{}-({\textsc{om}8)-(\textsc{om}9){}-}see-{\textsc{caus}}{}-{\textsc{fv}}  9.chicken 8-maize\\
\glt    Lit: the child is making it (the chicken) see the maize.  \\
‘the child is making the chicken see the maize’}
\z
\z

The example in \REF{ex:ngunga:54a} shows once again that, when inflected, the verb -\textit{won}{}- ‘see’ cannot occur without the obligatory presence of the OM in its structure. The data in (\ref{ex:ngunga:54}a, b) illustrate that only one object marker is permitted in the verb structure. That is, in Ciyaawo, the co-occurrence of two OMs in the verb structure is forbidden. 

\subsubsection{(v) Can either benefactive or theme objects be expressed by an object marker in double object constructions?}

In Ciyaawo, different from Cinyungwe, Citshwa and Ciwutee, theme objects cannot be expressed by an object marker in an applied construction. That is, in this language, the only object marker that is allowed to occur in the verb structure is the benefactive as shown below:

\ea\label{ex:ngunga:55}
\ea[]{\label{ex:ngunga:55a} \gll Maama a-ku-\textbf{n’}{}-telec-el-a    yi-maanga,  mw-aanace.\\
1.mother  {1\textsc{sm}\textsc{}-\textsc{prs}}{}-{\textsc{om}1}{}-cook-\textsc{appl}-\textsc{fv}  8-maize    {1}.child\\
\glt ‘the mother is cooking maize for the child’}

\ex[*]{\label{ex:ngunga:55b} \gll Maama a-ku-\textbf{yi}{}-telec-el-a    Siriza,  yi-maanga.\\
1.mother  {1\textsc{sm}\textsc{}-\textsc{prs}}{}-{\textsc{om}7}{}-cook-{\textsc{appl}-\textsc{fv}} Siriza  {7}{}-maize  \\}
\z
\z

The example \REF{ex:ngunga:55b} illustrates that in Ciyaawo when the benefactive and the theme co-occur it is only the benefactive argument that can have a co-referent OM in the verb structure.  

\subsubsection{(vi) Is an object marker required/optional/disallowed in object relatives?}

Ciyaawo functions as Ciwutee in relation to the use of object markers in object relatives which are generally optional, as illustrated in the following examples: 

\ea\label{ex:ngunga:56}
\ea[]{\label{ex:ngunga:56a} \gll ma-ticiti   ga   c-aa-ci-(\textbf{ga})-dy-a        maama     ga   ku-dyoop-a \\
      {6}{}-watermelon  \textsc{rel}  \textsc{fut}-\textsc{sm}1-\textsc{fut}{}-{(\textsc{om}6)}{}-eat-{\textsc{fv}}   {1}.mother   \textsc{gen}  {15}{}-sweet-\textsc{fv}\\
\glt      ‘the watermelons that mum shall eat (them) are sweet’}

\ex[*]{\label{ex:ngunga:56b} \gll ma-ticiti   ga   c-aa-ci-won-a             maama    ga  ku-dyoop-a \\
      6-watermelon  \textsc{rel}  \textsc{fut}-\textsc{sm}1-\textsc{fut}{}-{\textsc{om}6}{}-see-\textsc{fv} 1.mother \textsc{gen} {15}{}-sweet-\textsc{fv}\\}

\ex[]{\label{ex:ngunga:56c} \gll ma-ticiti   ga   c-aa-ci-\textbf{ga}{}-won-a           maama    ga  ku-dyoop-a \\
      6-watermelon  \textsc{rel}  \textsc{fut}-\textsc{sm}1-\textsc{fut}{}-{\textsc{om}6}{}-see-{\textsc{fv}}   {1}.mother \textsc{gen} {15}{}-sweet-\textsc{fv}\\
\glt      ‘the watermelons that mum shall see (them) are sweet’}
    \z
\z

In Ciyaawo, the occurrence of the OM in the verb structure is optional (cf. \ref{ex:ngunga:56a}). It is important to note that the ungrammaticality of \REF{ex:ngunga:56c} does not have to do with the absence of the OM in relative constructions as such, but with the fact that this verb is one which cannot occur without an OM. 

Finally, we should add that, generally, optionality of the OM in the verb structure of most verbs is related to emphasis and what the speaker wants to express, as illustrated in \REF{ex:ngunga:56}. But this is different from \REF{ex:ngunga:56c} which is marked as ungrammatical because of the specificity of the verb -\textit{wona} whose structure requires the presence of an OM, be it in relative constructions or not.

\subsubsection{(vii) Is an object marker obligatory with particular verbs?}

In Ciyaawo, there are some verbs such as -\textit{won}{}- ‘see’ and -\textit{p}{}- ‘give’ which require an OM even if the lexical object occurs in the sentence. Object doubling can occur with all transitive verbs. But its obligatoriness depends on the lexical properties of the verb. In \REF{ex:ngunga:56} we have examples of verbs that must be categorized as [+OM] to which the lexical object is obligatorily added. This explains the ungrammaticality of \REF{ex:ngunga:57a}, \REF{ex:ngunga:58a} and \REF{ex:ngunga:59a}.\footnote{Here we have imbrication, a phenomenon where, in certain verbs, the past tense marker (\textit{il}-) is not suffixed to the verb root, it is imbricated within the verb root to yield the form \textit{-ween}. }

\ea\label{ex:ngunga:57}  Ciyaawo 
\ea[*]{\label{ex:ngunga:57a}  \gll mw-anace  ju-ween-i  di-goombo.\\
  {\textsc{}1}{}-child  {\textsc{sm1}.\textsc{}}{}-see-{\textsc{fv}}  {5}{}-banana      \\}

\ex[]{\label{ex:ngunga:57b} \gll mw-aanace  ju-\textbf{di}\textsubscript{i}{}-ween-i    di-goombo\textsubscript{i}.\\
  {\textsc{}1}{}-child  {\textsc{sm1}.1\textsc{}-\textsc{om}5-}see-{\textsc{fv}}  {5}{}-banana\\
\glt  ‘the child has seen the banana’}

\ex[]{\label{ex:ngunga:57c} \gll mw-aanace  ju-\textbf{di}{}-ween-i.\\
  {\textsc{}1}{}-child  {\textsc{sm1}.1\textsc{}-\textsc{om}5-}see-{\textsc{fv}}\\
\glt  ‘the child has seen it’}
\z
\ex\label{ex:ngunga:58}
\ea[*]{\label{ex:ngunga:58a}  \gll uwe  tu-p-eel-e     mw-aanace   mi-teela.\\
  We  {\textsc{sm}1\textsc{pl}-}give-{\textsc{pfv}}{}-{\textsc{fv}}  {1}{}-child  {4}{}-tree    \\}
  
\ex[]{\label{ex:ngunga:58b} \gll uwe  tu-\textbf{m}\textsubscript{i}{}-p-eel-e    mw-aanace   mi-teela\textsubscript{i}.\\
  we  {\textsc{sm}1\textsc{pl}-\textsc{om}1-}give-{\textsc{pfv}}{}-{\textsc{fv}}  {1}{}-child  {4}{}-tree\\
\glt    ‘we have given the child the trees’}

\ex[]{\label{ex:ngunga:58c} \gll uwe  tu-\textbf{m}{}-p-eel-e    mi-teela.\\
  we  {\textsc{sm}1\textsc{pl}-\textsc{om}1-}give-{\textsc{pfv}}{}-{\textsc{fv}}  {4}{}-tree\\
\glt    ‘we have given him (the child) the trees’}
\z
\ex\label{ex:ngunga:59}
\ea[*]{\label{ex:ngunga:59a} \gll  uwe  tu-maany-i     mw-aanace.\\
  We  {\textsc{sm}1\textsc{pl}-}know-{(\textsc{pfv})}{}{\textsc{}}  {1}{}-child    \\}
  
\ex[]{\label{ex:ngunga:59b} \gll uwe  tu-\textbf{m}\textsubscript{i}{}-maany-i      mw-aanace\textsubscript{i}.\\
  we  {\textsc{sm}1\textsc{pl}-\textsc{om}1-}know-{(\textsc{pfv})\textsc{}}  {1}{}-child\\
\glt    ‘we have known him the child the trees’}

\ex[]{\label{ex:ngunga:59c} \gll  uwe  tu-\textbf{m}\textsubscript{i}{}-maany-i      (mw-aanace\textsubscript{i})\\
  we  {\textsc{sm}1\textsc{pl}-\textsc{om}1-}know-{(\textsc{pfv})-\textsc{}}  ({1}{}-child)\\
\glt    ‘we have known him (the child)’}
\z
\z

In terms of the parameters under examination here, the answer for Ciyaawo is “yes” for all the parameters proposed by \citet{MartenKula2012} except for (ii) and (v). Apart from that, we have shown that there are verbs like -\textit{wona} ‘see’, -\textit{pa} ‘give’, -\textit{manya} ‘know’ which are characterized by the obligatory presence of OM in their structure. We suggest that these verbs should be subcategorized as [+OM]. \tabref{tab:ngunga:5} summarizes object marking properties in Ciyaawo. 

\begin{table}
\caption{\label{tab:ngunga:5} Parametric variation in object marking in Ciyaawo}


\begin{tabularx}{\textwidth}{l@{~}Ql}

\lsptoprule

(i) & Can the object marker and the object argument co-occur? & \ding{51}\\
(ii) & Is an object marker obligatory with particular object NPs? & \ding{55}\\
(iii) & Are there locative object markers? & \ding{51}\\
(iv) & Is object marking restricted to one object marker per verb? & \ding{51}\\
(v) & Can either benefactive or theme objects be expressed by an object marker in double object constructions? & \ding{55}\\
(vi) & Is an object marker required/optional\slash disallowed in object relatives? & \ding{51}\\
(vii) & Is an object marker obligatory with particular verbs? & \ding{51}\\
\lspbottomrule
\end{tabularx}
\end{table}

\subsection{Summary}\label{sec:ngunga:3.5}

This section has examined the morphosyntactic properties of object marking in four Mozambican languages taking into consideration \citegen{MartenKula2012} parameters, as summarized in \tabref{tab:ngunga:6}. 

\begin{table}
\caption{\label{tab:ngunga:6} Object marking in the four analyzed Mozambican Bantu languages 
following \citet{MartenKula2012}}
\begin{tabularx}{\textwidth}{l@{~}Qcccc}
\lsptoprule
\multicolumn{2}{p{.4\textwidth}}{\multirow{2}{=}{Parameters of variation from \citet{MartenKula2012}}} & \multicolumn{4}{c}{Languages of the present study}\\
\cmidrule(lr){3-6}
&  & Ciyaawo & Ciwutee & Cinyungwe & Tshwa\\
\midrule
(i) & Can the object marker and the object argument co-occur? & \ding{51} & \ding{51} & \ding{55} & \ding{51}\\
(ii) & Is an object marker obligatory with particular object NPs? & \ding{55} & \ding{55} & \ding{55} & \ding{55}\\
(iii) & Are there locative object markers? & \ding{51} & \ding{51} & \ding{51} & \ding{55}\\
(iv) & Is object marking restricted to one object marker per verb? & \ding{51} & \ding{51} & \ding{51} & \ding{51}\\
(v) & Can either benefactive or theme objects be expressed by an object marker in double object constructions? & \ding{55} & \ding{51} & \ding{51} & \ding{51}\\
(vi) & Is an object marker required in object relatives? & \ding{51} & \ding{51} & \ding{51} & \ding{51}\\
(vii) & Are there verbs whose inflection obligatorily require an OM in inflectional structure? & \ding{51} & \ding{55} & \ding{55} & \ding{55}\\
\lspbottomrule
\end{tabularx}
\end{table}

To summarise, only three from the seven parameters (ii), (iv) and (vi) have the same responses across all languages, parameter (ii) for which the value is NO across all  languages of our sample and parameters (iv) and (vi) for which the value is YES across our sample. The four remaining parameters have one language whose response is different from the response of the other languages regardless of whether it is NO for Cinyungwe (i), Cithswa (iii), Ciyaawo (v) or YES for Ciyaawo (vii). Ciwutee is the only language which does not have any feature which is specific to it. This means that considerations like the language contact, multilingualism and language classification alone do not help to explain similarities or differences among the languages according to the different parameter values.  

Considering the data from the four languages, we suggest that the obligatory requirement for an object marker [+OM] associated with some transitive verbs and structures should be added as a seventh parameter to the six parameters of variation in object marking in Bantu put forward by \citet{MartenKula2012}. Due to the existence of transitive verbs subcategorized as [+OM], we further encourage scholars to examine these parameters of variation in other Bantu languages in light of these features of variation.


\section{Conclusions}\label{sec:ngunga:4}

This paper has discussed object marking in four Mozambican Bantu languages, Cinyungwe, Citshwa, Ciwutee and Ciyaawo, based on \citegen{MartenKula2012} parameters. In contrast to Ciyaawo, in Cinyungwe, Citshwa and Ciwutee the co-occurrence of the lexical object and OM in the same sentence is not allowed. Specifically, in Cinyungwe, the co-occurrence of the overt subject NP and the OM within the same sentence can happen only if the object is not \textit{in situ.} In Ciwutee the co-occurrence of the object marker with the overt NP is allowed except in cases of emphasis or communicative strategies. In Citshwa OM-doubling the object marker and object argument results in a definiteness reading. The data illustrate that OM-doubling in Cinyungwe and Ciwutee is associated with an evidential reading, in a sense that the speaker is telling the hearer that s/he is sure of what s/he is talking about and so, s/he does not want to be contradicted (see \citealt{LippardEtAl2021} for more on this issue). On the basis of the data presented here, we also suggest that the feature [+OM] for some transitive verbs like -\textit{manya} ‘know’, -\textit{pa} ‘give’ and -\textit{wona} ‘see’ should be added as the seventh parameter to the six parameters put forward by \citet{MartenKula2012}. 

This research shows that of the four languages, only Ciyaawo has the value YES for the parameter (vii). Linking the Ciyaawo response for this parameter to what is happening in the relative constructions in the other three languages analyzed in this paper, we suggest that the verb -\textit{wona} ‘see’ may have lost its [+OM] feature and remained only in the relative sentences. We need to undertake more research on this issue to check if we can find a trace of this feature in these languages using similar or other verbs because in Kilunguru (G30), for example, OM is obligatory with similar verbs \textit{{}-ona} ‘see’; \textit{{}-ing’a} ‘give’ and with a different verb \textit{{}-phika} ‘find’. 

Overall, this chapter has contributed to our understanding of the morphosyntax of four Bantu languages spoken in Mozambique, the broader properties of object marking in Bantu languages, as well as the use of a parametric approach (following \citet{MartenKula2012}) to better understand variation within Bantu.

It has been noted that our aim was to discuss the \citet{MartenKula2012} six parameters in four Mozambican languages. In the course of this, we have found a number of areas which require further investigation and attention. We leave for future work the discussion about the impact of the verb type on object marking, syntactic status of the OM, and (a)symmetry in double object constructions in the four languages analyzed in this paper. 

\section*{Abbreviations}
The following glosses are used in addition to the Leipzig Glossing Rules: 
\begin{tabbing}
The numbers 1, 5, 7, 9, 10 \ldots\hspace{1ex} \= noun classes\kill
1, 5, 7, 9, 10 \ldots \> noun classes\\
\textsc{fv} \> Final Vowel\\
Intd. \> Intended meaning\\
\textsc{om} \> Object Marker\\
\textsc{sm} \> Subject Marker\\
\textsc{pfv} \> Perfective\\
\textsc{prs} \> Present\\
{\textsubscript{i}} \> co-reference\\

\end{tabbing}

\section*{Acknowledgements}
We would like to thank Carlos Manuel for reading the first version of this paper for his comments and insights. We extend our gratitude to the reviewers for their valorous comments. We are also grateful to the comments of Hannah Gibson and Rozenn Gueróis that improved our discussion.

\printbibliography[heading=subbibliography,notkeyword=this]
\end{document} 
