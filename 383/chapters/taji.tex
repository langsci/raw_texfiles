\documentclass[output=paper,
            colorlinks, citecolor=brown
            % ,draft
            ,draftmode
		  ]{langscibook}
\ChapterDOI{10.5281/zenodo.10663763}
		  
\author{Julius Taji\orcid{}\affiliation{University of Dar es Salaam}}

\title[Demonstratives in Chiyao]{Demonstratives in Chiyao: An analysis of their form, distribution and functions}

\abstract{This chapter aims to provide a description of Chiyao demonstratives with a particular focus on their form, distribution, and functions. It is shown that, morphologically, the Chiyao demonstrative consists of an initial element, which is always the vowel \textit{a-,} followed by an agreement marker, and ends with a final element which changes according to the location of the referent in relation to the speaker or hearer. In cases of demonstrative doubling, the post-nominal demonstrative drops the initial vowel. Thus, it is proposed that the initial vowel in the Chiyao demonstrative is optional. Within the broader classification of demonstratives in Bantu languages, the Chiyao demonstratives appear in four main types. These include pronominal demonstratives, which are used as independent pronouns; adnominal demonstratives, which modify nouns; adverbial demonstratives, which modify verbs and locative nouns; and finally, identificational demonstratives, which occur in copula and non-verbal clauses. With regard to their distribution, it is established that the Chiyao demonstrative can occur either post-nominally, or in both pre-nom\-i\-nal and post-nominal positions simultaneously. Finally, it is shown that demonstratives serve various grammatical and communicative functions, including expressing the location of an entity in relation to interlocutors, showing emphasis, indicating definiteness, and encoding anaphoric reference. These findings contribute to our further understanding of the behavior of demonstratives in Bantu languages and inform future descriptive and typological works in this area.}

\IfFileExists{../localcommands.tex}{
  \addbibresource{../localbibliography.bib}
  \usepackage{langsci-optional}
\usepackage{langsci-gb4e}
\usepackage{langsci-lgr}

\usepackage{listings}
\lstset{basicstyle=\ttfamily,tabsize=2,breaklines=true}

%added by author
% \usepackage{tipa}
\usepackage{multirow}
\graphicspath{{figures/}}
\usepackage{langsci-branding}

  
\newcommand{\sent}{\enumsentence}
\newcommand{\sents}{\eenumsentence}
\let\citeasnoun\citet

\renewcommand{\lsCoverTitleFont}[1]{\sffamily\addfontfeatures{Scale=MatchUppercase}\fontsize{44pt}{16mm}\selectfont #1}
   
  %% hyphenation points for line breaks
%% Normally, automatic hyphenation in LaTeX is very good
%% If a word is mis-hyphenated, add it to this file
%%
%% add information to TeX file before \begin{document} with:
%% %% hyphenation points for line breaks
%% Normally, automatic hyphenation in LaTeX is very good
%% If a word is mis-hyphenated, add it to this file
%%
%% add information to TeX file before \begin{document} with:
%% %% hyphenation points for line breaks
%% Normally, automatic hyphenation in LaTeX is very good
%% If a word is mis-hyphenated, add it to this file
%%
%% add information to TeX file before \begin{document} with:
%% \include{localhyphenation}
\hyphenation{
affri-ca-te
affri-ca-tes
an-no-tated
com-ple-ments
com-po-si-tio-na-li-ty
non-com-po-si-tio-na-li-ty
Gon-zá-lez
out-side
Ri-chárd
se-man-tics
STREU-SLE
Tie-de-mann
}
\hyphenation{
affri-ca-te
affri-ca-tes
an-no-tated
com-ple-ments
com-po-si-tio-na-li-ty
non-com-po-si-tio-na-li-ty
Gon-zá-lez
out-side
Ri-chárd
se-man-tics
STREU-SLE
Tie-de-mann
}
\hyphenation{
affri-ca-te
affri-ca-tes
an-no-tated
com-ple-ments
com-po-si-tio-na-li-ty
non-com-po-si-tio-na-li-ty
Gon-zá-lez
out-side
Ri-chárd
se-man-tics
STREU-SLE
Tie-de-mann
} 
  \togglepaper[1]%%chapternumber
}{}

\begin{document}
\maketitle 
%\shorttitlerunninghead{}%%use this for an abridged title in the page headers


\section{Introduction}\label{sec:taji:1}

Studies across Bantu languages have generally indicated that demonstratives perform spatial-deictic roles referring to entities in three main dimensions, namely proximal (near the speaker), non-proximal or medial (near the addressee), and distal (distant from the speaker or addressee) (\citealt{Nicolle2007a,Nicolle2007b, Asiimwe2014}). Despite this uniformity in the dimensions reflected by demonstratives, there is significant variation among Bantu languages in terms of syntactic ordering of demonstratives within the NP and the functions they serve, in addition to their spatial-deictic uses. For example, with regard to syntactic ordering, Swahili allows both pre-nominal and post-nominal demonstratives as shown in \REF{ex:taji:1}.\footnote{Unless otherwise indicated, all Swahili examples are from the author as a native speaker.}

\ea Swahili
    \label{ex:taji:1}
    \ea\label{ex:taji:1a} \gll   \textbf{m-geni}      \textbf{yule}                      a-li-fika                  mapema \\
1-visitor    1.\textsc{dem.dist}        \textsc{sm1-pst}-arrive    early\\
        \glt ‘That visitor arrived early.’

    \ex\label{ex:taji:1b} \gll    \textbf{yule}                        \textbf{m-geni}              a-li-fika                  mapema\\
1.\textsc{dem.dist}       1-visitor            \textsc{sm1-pst}-arrive    early\\
        \glt ‘That visitor arrived early.’                                                  
        \z
\z

In contrast to Swahili, Sukuma allows only post-nominal demonstratives \REF{ex:taji:2}. Placing the demonstrative before the head noun in Sukuma is ungrammatical, as shown in \REF{ex:taji:3}. 

\ea Sukuma (Nyanda p.c.)
    \label{ex:taji:2}
    \ea\label{ex:taji:2a} \gll  \textbf{u-ng’w-ana}          \textbf{uyu}                      a-tog-ilwe                βu-gali\\
      \textsc{ppx}-1-child          \textsc{dem.prox}      \textsc{sm1}-like-\textsc{pfv}        14-ugali\\
      \glt ‘This child likes ugali.’

    \ex\label{ex:taji:2b} \gll    \textbf{a-ma-shamba}      \textbf{ayo}                                    ga-lim-ilwe\\
        \textsc{ppx}-6-farm          \textsc{dem.non\_prox}        \textsc{sm6}-cultivate-\textsc{pfv}\\
        \glt ‘Those farms have been cultivated.’   
    \z

\ex Sukuma (Nyanda p.c.)
    \label{ex:taji:3}
    \ea[*]{\gll \textbf{uyu}                    \textbf{u-ng’w-ana}                a-tog-ilwe                βu-gali\\
          \textsc{dem.prox}    \textsc{ppx}-1-child                \textsc{sm1}-like-\textsc{pfv}       14-ugali\\
         \glt Int: ‘This child likes ugali.’}\label{ex:taji:3a}

    \ex[*]{\gll \textbf{ayo}                                  \textbf{a-ma-shamba}        ga-lim-ilwe\\
          \textsc{dem.non\_prox}      \textsc{ppx}-6-farm            \textsc{sm6}-cultivate-\textsc{pfv}\\
          \glt Int: ‘Those farms have been cultivated.’}\label{ex:taji:3b}
    \z
\z

A similar pattern is attested in Makhuwa where demonstratives occur after the head nouns they modify in their canonical adnominal function as illustrated in \REF{ex:taji:4} below.

\ea Makhuwa (\citealt[184]{vanderWal2010})
    \label{ex:taji:4}
    \ea\label{ex:taji:4a} \gll  mwálpwá    \textbf{olé}                                          o-hoó-wa\\
      1.dog            1.\textsc{dem.non\_prox}        \textsc{sm1-perf.dj}-come\\
      \glt ‘The/that dog came.’

    \ex\label{ex:taji:4b} \gll  ki-kúm-íh-é-ni                                      nipúró      \textbf{nna}                        vá\\
      1\textsc{sg.om}-exist-\textsc{caus-opt-pla}    5.place    5.\textsc{dem.prox}      16.\textsc{dem.prox}\\
      \glt ‘Get me out of this place.’                              
      \z
\z

Makonde exhibits a more free order as it permits multiple occurrence of demonstratives in both pre-nominal and post-nominal positions simultaneously, as in \REF{ex:taji:5} below. As it will be revealed in the subsequent sections, several other languages of southern Tanzania also display this order.

\ea Makonde (\citealt[171]{Makanjila2019})\\
    \label{ex:taji:5}
    \gll \textbf{a-yu}                  mu-ana              \textbf{a-yu}                  a-ka-pilikan-a                    \textbf{a-yu} \\
\textsc{dem.prox}    1-child              \textsc{dem.prox}   \textsc{sm1-neg}-payheed-\textsc{fv}   \textsc{dem.prox}\\ 
    \glt ‘This child is stubborn.’                                 
\z

In view of the variation displayed by demonstratives across Bantu, this chapter provides a description of demonstratives in Chiyao, a Bantu language of southern Tanzania, with a focus on their forms, syntactic distribution, and functions.

\begin{sloppypar}
The Chiyao data in this paper are based on the Masasi dialect and come from two main sources, namely grammaticality judgments and oral narratives. Through grammaticality judgments, a list of constructions with demonstratives in different orders was provided to ten native speakers of Chiyao who then gave feedback on whether they were acceptable or not. As for oral narratives, constructions with demonstratives were extracted from three traditional stories narrated by three different native speakers of Chiyao. The constructions were then analyzed to determine the form, distribution and function of demonstratives. Unless otherwise indicated, all data in this paper are from Chiyao. 
\end{sloppypar}

The chapter proceeds as follows: \sectref{sec:taji:2} provides a brief account of the Chiyao language;  \sectref{sec:taji:3} discusses the forms of the Chiyao demonstratives;  \sectref{sec:taji:4} outlines the types of demonstratives; \sectref{sec:taji:5} presents the syntactic distribution of demonstratives; \sectref{sec:taji:6} focuses on the functions of demonstratives; and \sectref{sec:taji:7} provides a conclusion.

\section{The Chiyao language}
\label{sec:taji:2}

Chiyao is a cross-border Bantu language which is spoken in Southern Malawi, north-western Mozambique, and southern Tanzania. The language is also referred to as Ciyao \citep{Ngunga1997}, Ciyawo (\citealt{DicksDollar2010}), and Yao \citep{Whiteley1966}. According to \citet{Ngunga2002}, there are about three million Chiyao speakers residing in these three countries. A significant number of Chiyao speakers, mainly emigrants from Malawi, are also found in Zambia and Zimbabwe. The present study is based on the Tanzanian variety of Chiyao, which is mainly spoken in the southern regions of Ruvuma (Tunduru District), and Mtwara (Masasi District). The number of Chiyao speakers in Tanzania is estimated at 400,000 (Languages of Tanzania Project, 2009). \citet{NursePhilippson1980} classified Chiyao under the Ruvuma Bantu branch of the Rufiji-Kilombero Bantu zone. In Guthrie’s updated list by \citet{Maho2009}, the language is coded as P21 and is found in the Yao group, along with Mwera (Shimwela) P22, Makonde (Chimakonde) P23, Ndonde (P24), and Mavia (P25).  

\section{Forms of the Chiyao demonstrative}
\label{sec:taji:3}

It has been argued that demonstratives in many Bantu languages start with the ``initial element'' or ``initial vowel'' (\citealt{Wald1973, DuPlessis1978, DuPlessisVisser1992}). The initial vowel may take various forms depending on the vowel of the agreement prefix, such as \textit{a{}-, e{}-,} or \textit{o{}-} \citep{Visser2008}. However, across languages, not all demonstratives have an initial vowel \citep{Asiimwe2014}. \citet{Asiimwe2014}, observed that in Runyankore-Rukiga, the initial vowel \textit{a-} is realised in the proximal and medial demonstratives, but it is absent in the distal demonstratives (but see also \textcitetv{chapters/asiimwe}).

The situation in Chiyao seems to support the analysis that not all demonstratives take an initial vowel. There are cases where a demonstrative occurs without this initial vowel, notably when a demonstrative occurs as a circumdemonstrative as illustrated in \REF{ex:taji:6} below where the post-nominal demonstrative form does not host the initial vowel \textit{a{}-} (see also \sectref{sec:taji:5.3}).

\ea%6
    \label{ex:taji:6}
    \ea\label{ex:taji:6a} \gll  \textbf{ali}                          lí-koswé      \textbf{li}\\
      5.\textsc{dem.prox}      5-rat            5.\textsc{dem.prox}\\
      \glt ‘This rat’                

    \ex\label{ex:taji:6b} \gll \textbf{achila}                  chi-téla      \textbf{chíla}\\
      7.\textsc{dem.dist}      7-tree        7.\textsc{dem.dist}\\
     \glt ‘That tree (over there)’          
    \z
\z

Given this scenario, we can therefore posit that the demonstrative in Chiyao take different shapes depending on the noun class, as shown in \tabref{tab:taji:1}.

\begin{sidewaystable}

%\small
\begin{tabularx}{\textwidth}{lllllllQ}

\lsptoprule

{Class} & {Prefix} & {Example} & {Gloss} & \multicolumn{3}{c}{ {demonstratives}} &  \\
\cmidrule(lr){5-7}
&  &  &  & {\textsc{prox}} & {\textsc{non\_prox}} & {\textsc{dist}} & {Example}\\
\midrule
1 & mu- & mundu & person & (a)ju & (a)jo & (a)jula & mundu aju/ajo/ajula

‘this/that person’\\
2 & va- & vandu & people & (a)va & (a)vo & (a)vala & vandu ava/avo/avala

‘these/those people’\\
3 & m- & mteela & tree & (a)wu & (a)wo & (a)wula & mteela awu/awo/awula

‘this/that tree’\\
4 & mi- & miteela & trees & (a)ji & (a)jo & (a)jila & miteela aji/ajo/ajila

‘these/those trees’\\
5 & li- & lindanda & egg & (a)li & (a)lyo & (a)lila & lindanda ali/alyo/alila

‘this/that egg’\\
6 & ma- & mandanda & eggs & (a)ga & (a)go & (a)gala & mandada aga/ago/agala

‘these/those eggs’\\
7 & chi- & chipuula & knife & (a)chi & (a)cho & (a)chila & chipula achi/acho/achila

‘this/that knife’\\
8 & i-/y- & ipuula & knives & (a)yi & (a)yo & (a)yila & ipula ayi/ayo/ayila

‘these/those knives’\\
9 & n & ndembo & elephant & (a)ji & (a)jo & (a)jila & ndembo aji/ajo/ajila

‘this/that elephant’\\
\midrule
\end{tabularx}
\caption{\label{tab:taji:1}: Chiyao demonstratives in relation to noun classes and deixis}
\end{sidewaystable}
\begin{sidewaystable}

%\small
\begin{tabularx}{\textwidth}{lllllllQ}

\lsptoprule

{Class} & {Prefix} & {Example} & {Gloss} & \multicolumn{3}{c}{ {demonstratives}} &  \\
\cmidrule(lr){5-7}
&  &  &  & {\textsc{prox}} & {\textsc{non\_prox}} & {\textsc{dist}} & {Example}\\
\midrule
10 & n & ndembo & elephants & (a)si & (a)syo & (a)sila & ndembo asi/asyo/asila

‘these/those elephants’\\
11 & lu- & lusulo & river & (a)lu & (a)lo & (a)lula & lusulo alu/alo/alula

‘this/that river’\\
12 & ka- & kajuni & small bird & (a)ka & (a)ko & (a)kala & kajuni aka/ako/akala

‘this/that small bird’\\
13 & tu- & tujuni & small birds & (a)tu & (a)to & (a)tula & tujuni atu/ato/atula

‘these/those small birds’\\
14 & u- & ukana & liquor & (a)u & (a)wo & (a)wula & ukana au/awo/awula

‘this/that liquor’\\
15 & ku- & kulya & eating & (a)ku & (a)ko & (a)kula & kulya aku/ako/akula

‘this/that eating’\\
16 & pa- & paasi & down & (a)pa & (a)po & (a)pala & paasi apa/apo/apala

‘here/there down’\\
17 & ku- & kumchiji & left & (a)ku & (a)ko & (a)kula & kumchiji aku/ako/akula

‘here/there on the left’\\
18 & mu- & mulisimbo & inside the pit & (a)mu & (a)mo & (a)mula & mulisimbo amu/amo/amula

‘here/there inside the pit\\
\lspbottomrule
\end{tabularx}
\repeatcaption{tab:taji:1}{Chiyao demonstratives in relation to noun classes and deixis}
\end{sidewaystable}

As \tabref{tab:taji:1} below indicates, the proximal demonstrative optionally begins with \textit{a-} and ends with -\textit{u, -a}, or \textit{{}-i}. The non-proximal demonstrative optionally begins with \textit{a-} and ends with \textit{{}-o}. The distal demonstrative optionally begins with \textit{a-} and ends with \textit{{}-la}. Generally, all demonstratives in Chiyao have an agreement marker which, as it is proposed here, can be regarded as a base, and a final element. The agreement marker and the final element change in response to noun class and deixis. Thus, the proximal demonstrative appears in the form \textit{(a)-} + agreement marker + \textit{{}-u/-a/-i} while the non-proximal demonstrative takes the form \textit{(a)}{}- + agreement marker + \textit{{}-o}. As for the distal demonstrative, the form is \textit{(a)-} + agreement marker + \textit{{}-la}. From a phonological point of view, it can be observed that proximal and non-proximal demonstratives take the form of (V)C(C)V while distal demonstratives take the form of (V)CVCV.

Generally, the above scenario seems to suggest that the form of the demonstratives in Chiyao is dependent upon two key factors, namely noun class and deixis. Thus, a demonstrative changes its shape in response to the noun class of the noun it modifies, as well as the spatial relationship between the interlocutors and the referent. It follows that in relation to the spatial relationship, the demonstrative takes one of the three forms for each noun class, namely close to the speaker (proximal demonstrative), near the addressee (non-proximal demonstrative), and far from both the speaker and the addressee (distal demonstrative). The examples in \REF{ex:taji:7} further illustrate this three-way distinction of Chiyao demonstratives and their forms by using a class 7 noun \textit{chiteéngu} ‘chair.’ Example \REF{ex:taji:7a} illustrates a proximal demonstrative, \REF{ex:taji:7b} illustrates a non-proximal demonstrative, and \REF{ex:taji:7c} shows a distal demonstrative.

\ea%7
    \label{ex:taji:7}
    \ea\label{ex:taji:7a} \gll  chi-teengú  \textbf{achí} \\
      7-chair        7.\textsc{dem.prox}\\            
      \glt ‘This is a chair’          

    \ex\label{ex:taji:7b} \gll  chi-teengú    \textbf{achó}\\
      7-chair          7.\textsc{dem.non\_prox}\\
      \glt ‘That is a chair (near you)’

    \ex\label{ex:taji:7c} \gll  chi-teengú    \textbf{achilá}\\
      7-chair      7.\textsc{dem.dist}\\
     \glt ‘That is a chair (over there)’
    \z
\z

In addition to noun class and deixis, the shape of the demonstrative seems to be further influenced by syntactic and pragmatic factors. Pragmatic factors specifically determine the presence or absence of the initial vowel in the demonstratives. Thus, the demonstrative in its full form with an initial vowel commonly occurs in copula constructions to indicate the location of an entity within the interlocutor(s)’ visibility (see \sectref{sec:taji:4.2}) -- this is usually accompanied by a pointing gesture, as in \REF{ex:taji:8}. On the other hand, the reduced form (without the initial vowel) modifies nouns in subject or object position and functions to refer to the entity mentioned earlier in the discourse \REF{ex:taji:8}.

\ea%8
    \label{ex:taji:8}
    \gll m-chanda      \textbf{ájúlá}\\
  1-boy            1.\textsc{dem.dist}    \\
  \glt ‘That is a boy (over there)’

\ex%9
    \label{ex:taji:9}
    \gll m-chanda    \textbf{júla}                a-iíche\\
  1-boy    1.\textsc{dem.dist}      \textsc{sm1}-arrive.\textsc{perf}      \\
  \glt ‘That/the boy has arrived.’
\z

The copula construction illustrated in \REF{ex:taji:8} above cannot take a reduced demonstrative, thus explaining the ungrammaticality of \REF{ex:taji:10}. Similarly, placing a full demonstrative after a noun in a non-copula construction is ungrammatical, as in \REF{ex:taji:11} below:  

\ea[*]{\gll m-chanda      \textbf{júla}\\
   1-boy            1.\textsc{dem.dist}\\    
   \glt Int: ‘A boy is over there’}\label{ex:taji:10}

\ex[*]{\gll m-chanda    \textbf{ájúlá}                    a-iíche\\
   1-boy          1.\textsc{dem.dist}      \textsc{sm1}-arrive.\textsc{perf}\\      
   \glt Int: ‘That/the boy has arrived.’}\label{ex:taji:11}
 \z

The form of the demonstrative in \REF{ex:taji:11} looks similar to \REF{ex:taji:7} in that they both contain initial vowels. However, \REF{ex:taji:11} is ungrammatical because the demonstrative occurs with the NP and thus functions as a modifier while in \REF{ex:taji:7} it occurs within the VP as a subject complement. It can therefore be posited that the initial vowel can be present when Noun + \textsc{dem} are produced in isolation, but it is absent when this sequence is used in a clause as subject or object. 

\section{Classification of demonstratives}\label{sec:taji:4}

There are various approaches to the classification of demonstratives. Included amongst these are those proposed by \citet{Diessel1999morphosyntax}, \citet{Dixon2003}, and \citet{VandeVelde2005}. The first two are general typological works while the latter focuses specifically on Bantu. The analysis in this chapter is based on the approaches of \citet{Diessel1999morphosyntax} and \citet{Dixon2003}. \citegen{VandeVelde2005} approach is referred to in the section on syntactic distribution of demonstratives since it focuses on the positioning of demonstratives in relation to other elements within a Bantu clause. 

\citet{Diessel1999morphosyntax} analyses demonstratives based on the type of elements with which they occur in a clause. His analysis is based on the view that demonstratives can occur independently as arguments, or as modifiers of nouns, or as arguments of copula verbs. On the basis of this view, Diessel establishes four types of demonstratives: pronominal demonstratives, adnominal demonstratives, adverbial demonstratives, and identificational demonstratives. These are discussed below with reference to Chiyao examples.   

\subsection{Pronominal demonstratives}\label{sec:taji:4.1}

Pronominal demonstratives can be used as independent pronouns, as is the case with the word \textit{this} in the English sentence \textit{This is nice}. Since they may take the place of noun phrases, they have also been referred to as demonstrative pronouns \citep[72]{Diessel1999morphosyntax}. In Chiyao, pronominal demonstratives show grammatical distinctions also displayed by nouns. This means that they show agreement in terms of noun class and number (see \tabref{tab:taji:1} above). These demonstrative pronouns can also be cross-referenced on the verb with a subject or object marker. In \REF{ex:taji:12} below, the demonstratives \textit{acho} ‘that’, and \textit{aga} ‘these’ are pronominal as they stand alone and take the place of noun phrases. 

\ea%12
    \label{ex:taji:12}
    \ea\label{ex:taji:12a}\gll  nné    ngú-chi-sáká          \textbf{achó}\\
      I          \textsc{sm1-om7}-want    7.\textsc{dem.nom\_prox}\\
      \glt ‘I want that (one).’

    \ex\label{ex:taji:12b}\gll  \textbf{agá}                        ga-teméche\\
      6.\textsc{dem.prox}     \textsc{sm6}-break.\textsc{pfv}\\  
      \glt ‘These have been broken.’
    \z
\z

The demonstrative \textit{achó} in \REF{ex:taji:12a} refers to a class 7 singular noun, such as \textit{chitengu} ‘chair’ while \textit{agá} in \REF{ex:taji:12b} stands for a class 6 plural noun, such as \textit{majela} ‘hoes’, thus evidencing the fact that demonstratives exhibit noun class agreement. The examples in (\ref{ex:taji:12a}--\ref{ex:taji:12b}) further illustrate that pronominal demonstratives can occur in both subject position \REF{ex:taji:12b} and object position \REF{ex:taji:12a}. In both cases, the nouns which they refer to are also cross-referenced on the verb with either a subject marker or an object marker.

\subsection{Adnominal demonstratives}\label{sec:taji:4.2}

Unlike pronominal demonstratives, which can function like nouns, adnominal demonstratives modify nouns, and across Bantu can appear either before nouns (e.g. in Kirundi) or after nouns (e.g. in Luganda and Ruhaya) (\citealt{VandeVelde2005}). In Chiyao, adnominal demonstratives can occur either post-nominally \REF{ex:taji:13} or in both pre-nominal and post-nominal positions (cf. \sectref{sec:taji:5.3}), as in \REF{ex:taji:14}. The prenominal or postnominal occurrence of the adnominal demonstrative can also be conditioned by other adnominals, as is the case in Ikyaushi (Spier 2022).


\ea%13
    \label{ex:taji:13}
    \gll mbwá        á\textbf{sílá}\\
10.dog      10.\textsc{dem.dist}\\  
\glt ‘Those dogs’



\ex%14
    \label{ex:taji:14}
    \gll \textbf{asila}                      mbwá      \textbf{síla}\\
  10.\textsc{dem.dist}   10.dog      10.\textsc{dem.dist}\\  
  \glt ‘Those dogs’
\z

\subsection{Adverbial demonstratives}\label{sec:taji:4.3}

Adverbial demonstratives indicate location. These are exemplified in English with words such as \textit{here} and \textit{there}. Adverbial demonstratives are used to indicate the location of an event or situation expressed by the corresponding verb. In Chiyao, adverbial demonstratives occur both pronominally and adnominally. When they occur pronominally, they stand as independent pronouns and can function as subjects \REF{ex:taji:15a} or as verbal modifiers \REF{ex:taji:15b}. When the adverbial demonstratives occur adnominally, they co-occur with the locative nouns they modify, as in example \REF{ex:taji:16}.

\ea%15
    \label{ex:taji:15}
    \ea\label{ex:taji:15a}\gll \textbf{akulá}                  ku-talíche\\
      17.\textssc{dem.dist}    \textsc{sm17}-be.far\\
      \glt ‘That place (over there) is far.’

  \ex\label{ex:taji:15b}\gll a-kú-támá              \textbf{apa}\\
      \textssc{1sm-prs}-stay      16.\textsc{dem.prox}\\
      \glt ‘He stays here.’
    \z

\ex%16
    \label{ex:taji:16}
    \ea\label{ex:taji:16a} \gll ku-chi-jíji                  \textbf{kula}                      ku-talíche\\
  17-7-village                17.\textsc{dem.dist}      \textsc{prs}-befar.\textsc{pfv}\\
  \glt ‘There at the village it is far.’

   \ex\label{ex:taji:16b} \gll li-jóká        lí-jinjíle                  mu-lisimbo      \textbf{mo}\\
  5-snake        \textsc{sm5}-enter.\textsc{pfv}      18-hole              18.\textsc{dem.non\_prox}\\
  \glt ‘A snake has entered that hole.’
    \z
\z

The basic forms of the demonstratives in \REF{ex:taji:16a} and \REF{ex:taji:16b} are \textit{akula} and \textit{amo} respectively. However, it is worth noting that in these examples the adverbial demonstratives have lost the initial vowel, which is a common tendency for adverbial demonstratives occurring post-nominally in this language (see also \sectref{sec:taji:3}). 

The dropping of the initial vowel in adverbial demonstratives which occur post-nominally, illustrated in the examples above, further supports the argument put forward in \sectref{sec:taji:3} that the initial vowel does not constitute the core unit of the demonstrative. Moreover, the examples above seem to suggest that the type and syntactic occurrence of the demonstrative can impose some restrictions on the form of the demonstrative -- i.e. when the adverbial demonstrative occurs post-nominally, it drops the initial vowel.

\subsection{Identificational demonstratives}\label{sec:taji:4.4}

Identificational demonstratives, or, as \citet{Diessel1999morphosyntax} calls them, ``demonstrative identifiers'' occur in copula constructions. They function to identify a referent in a speech situation by introducing a new discourse topic or drawing the interlocutors’ attention to some existing discourse entity \citep{Amfo2007}. Identificational demonstratives have alternatively been referred to as predicative demonstratives (\citealt{Denny1982, Heath1984}), deictic predicators \citep{Schuh1977}, predicative pronouns \citep{Marconnes1931}, existential demonstratives \citep{Benton1971}, pointing demonstratives \citep{Rehg1981}, deictic identifier pronouns \citep{Carlson1994}, and presentational pronouns \citep{Moltmann2013}. Identificational demonstratives normally occur pronominally, in subject position, as in the following example from Ewondo, a Bantu language of southern Cameroon \citep[19]{Diessel1999morphosyntax}.

\ea Ewondo (\citealt[19]{Diessel1999morphosyntax}, glosses added)\\
    \label{ex:taji:17}
    \gll káádá      \textbf{ɲɔ}\\
  1.crab     1.\textsc{dem.prox}\\ 
  \glt ‘This is a crab’     
\z

In Chiyao, identificational demonstratives can be found in copula constructions which are formed using the copula \textit{ni}, as in the following examples.

\ea%18
    \label{ex:taji:18}
    \ea\label{ex:taji:18a}\gll \textbf{ajú}                        ní    ambusánga      jwenu\\
    1.\textsc{dem.prox}      is    friend              \textsc{1sg.poss}\\        
  \glt ‘This is your friend.’

    \ex\label{ex:taji:18b} \gll  \textbf{aulá}                   ní    m-gunda    wénu\\
      3.\textsc{dem.dist}      is    3-farm        \textsc{3sg.poss}\\
    \glt  ‘That is your farm.’
    \z
\z

In the context where the interlocutors have prior knowledge of the referent, the identificational demonstrative can occur in both positions, i.e. before and after the copula verb, as shown in (19). 

\ea%19
    \label{ex:taji:19}
    \ea\label{ex:taji:19a}\gll \textbf{ajú}                        ní    ambusánga      jwenu   \textbf{júla}\\
    1.\textsc{dem.prox}      is    friend              \textsc{1sg.poss}    1.\textsc{dem.dist}\\        
  \glt ‘This is that friend of yours (that we talked about).’

    \ex\label{ex:taji:19b} \gll  \textbf{aulá}                   ní    m-gunda    wénu  \textbf{ula}\\
      3.\textsc{dem.dist}      is    3-farm        \textsc{3sg.poss}    3.\textsc{dem.dist}\\
    \glt  ‘That is the farm of yours (that we talked about).’
    \z
\z

The demonstratives \textit{aju} ‘this,’ and \textit{aula} ‘that’ in \REF{ex:taji:18a} and \REF{ex:taji:18b} respectively are identificational as they serve to identify the nouns with which they occur. The occurrence of these demonstratives in the preverbal position without nouns seems to suggest that a subject nominal that would appear before the demonstrative is ``silent'' (or omitted, depending on the theoretical approach adopted) since Chiyao widely allows pro-drop.

\section{Syntactic distributions of demonstratives}\label{sec:taji:5}

\citet{VandeVelde2005} identified three major patterns for the positioning of demonstratives in Bantu languages in relation to the nouns with which they occur. These are i) the pre-nominal position (``preposed demonstratives'' in Van de Velde’s terminology), ii) the post-nominal position (``postposed demonstratives''), and iii) both before and after the head noun (``circumdemonstratives''). I discuss each of these in the context of Chiyao below.

\subsection{The pre-nominal position}\label{sec:taji:5.1}

The pre-nominal occurrence of demonstratives has been observed in several Bantu languages, including Nkore, Rundi, Ha, Bemba, Zulu, Xhosa (\citealt{VandeVelde2005}), and Runyankore-Rukiga (\textcitetv{chapters/asiimwe}). In Xhosa, the prenominal demonstrative performs the emphatic role when it precedes a noun with an augment, as in \REF{ex:taji:20} below. 

 \ea\label{ex:taji:20} \citet[14]{VandeVelde2005}, glosses added\\
    \ea\label{ex:taji:20a} \gll \textbf{loo}                        m-ntu      (non-emphatic)\\
      1.\textsc{dem.dist}    1-person\\
    \glt ‘that person’

    \ex\label{ex:taji:20b} \gll \textbf{lowo}                          u-m-ntu     (emphatic)\\
      1.\textsc{dem.dist}    \textsc{aug}1-person        \\
      \glt ‘that person’
    \z
\z

However, unlike the aforementioned languages, which allow demonstratives to occur before their head nouns, Chiyao does not permit pre-nominal occurrence of demonstratives without an accompanying post-nominal demonstrative form (i.e. the demonstrative without the initial vowel \textit{a-}). Thus, the examples in \REF{ex:taji:20bis} below are ungrammatical while those in \REF{ex:taji:21}, which employ the circumdemonstrative demonstrative are grammatical.

\ea%20
    \begin{multicols}{2}
    \label{ex:taji:20bis}
    \ea[*]{\gll \textbf{aju}                        mú{}-ndú  \\
        1.\textsc{dem.prox}    1-person\\
      \glt ‘This person’}\label{ex:taji:20bisa}

    \ex[*]{\gll \textbf{aú}                        m-gunda\\
        3.\textsc{dem.prox}    3-farm  \\
        \glt ‘This farm’}\label{ex:taji:20bisb}
    \z
    \end{multicols}

\ex%21
    \label{ex:taji:21}
    \ea\label{ex:taji:21a} \gll \textbf{aju}                        mú{}-ndú      \textbf{ju}\\
      1.\textsc{dem.prox}    1-person    1.\textsc{dem.prox}\\
    \glt ‘This person’

    \ex\label{ex:taji:21b} \gll \textbf{aú}                          m-gundá    \textbf{u}\\
      3.\textsc{dem.prox}    3-farm        3.\textsc{dem.prox}\\
      \glt ‘This farm’
    \z
\z

\subsection{The post-nominal position}\label{sec:taji:5.2}

Chiyao also allows demonstratives to occur in the post-nominal position as illustrated in the following examples.

\ea%22
    \label{ex:taji:22}
    \ea\label{ex:taji:22a} \gll li-jóká        \textbf{álílá}\\
      5-snake      5.\textsc{dem.dist}\\  
      \glt ‘A snake is over there.’        

    \ex\label{ex:taji:22b} \gll    chi-sotí    \textbf{achó}\\
        7-hat        7.\textsc{dem.non\_prox}\\
        \glt ‘A hat is over there (near you)’      

    \ex\label{ex:taji:22c} \gll    ku-chi-jíji          \textbf{ákúlá}\\
        17-7-village      17.\textsc{dem.dist}\\  
        \glt ‘There at the village’    
    \z
\z

When other dependents are present in the NP, the post-nominal demonstrative occupies the final slot in the Chiyao NP template. But it is worth noting that only the reduced form of the demonstrative, without the initial vowel is allowed to occur after other dependents, as shown in \REF{ex:taji:23}. These examples further suggest that, canonically, the post-nominal demonstrative occurs last in the Chiyao NP. In \REF{ex:taji:23} below, the demonstratives \textit{síla} \REF{ex:taji:23a} and \textit{lila} \REF{ex:taji:23b} are separate from the head noun, while other modifiers optionally occur between the nominal form and the demonstrative.


\ea%23
    \label{ex:taji:23}
    \ea\label{ex:taji:23a} \gll  mbwá      syetu          syékúlúngwá    \textbf{síla}                        sya-júv-ilé              m-ma-úkútu.\\
10.dog    10.\textsc{poss}    10.big                  10.\textsc{dem.dist}      \textsc{sm10}-hide-\textsc{pfv}    18-6-bush\\
\glt ‘Those big dogs of ours hid in bushes.’



    \ex\label{ex:taji:23b} \gll  li-símú    ly-épilíu    ly-ákúnyákapala    \textbf{líla}                        ly-a-jígeelé                      li-váágó.\\
    5-ogre        5-black      5-ugly                     5.\textsc{dem.dist}      \textsc{sm5-pst}-carry.\textsc{pst}      5-axe\\
    \glt ‘That black ugly ogre carried an axe.’
    \z
\z

When a demonstrative occurs before other modifiers (e.g. the possessive or an adjective), the sentence becomes ungrammatical as in \REF{ex:taji:24} below.


\ea%24
    \label{ex:taji:24}
    \ea[*]{mbwá \textbf{síla}      syetu       syékúlúngwa     sya-júv-ilé m-ma-úkútu.\\
    10.dog      10.\textsc{dem.dist}        10.\textsc{poss}        10.big   \textsc{10sm}-hide-\textsc{pfv}    18\textsc{loc}-6-bush\\
    \glt Int. ‘Those big dogs of ours hid in bushes.’}\label{ex:taji:24a}


    \ex[*]{\gll li-símú      \textbf{lila}                   ly-épilíu    ly-ákúnyákapala    ly-a-jígeelé                    li-váágó.\\
        5-ogre        5.\textsc{dem.dist}   5-black      5-ugly                     \textsc{sm5.pst}-carry.\textsc{pst}  5-axe\\
     \glt  Int. ‘That black ugly ogre carried an axe.’}\label{ex:taji:24b}
     \z
\z

In \REF{ex:taji:24a} the distal demonstrative \textit{sila} ‘those’ precedes the modifiers \textit{syetu    syekulungwa} ‘our big’ while in \REF{ex:taji:24b} the distal demonstrative \textit{vala} ‘that’ precedes the modifiers \textit{vapiliu vakunyakapala} ‘black ugly’. In both cases, the resulting sentences are ungrammatical. 

As many as five modifiers can appear between the head noun and the demonstrative, including possessives, numerals, adjectives, intensifiers and relative clauses, as long as the demonstrative occurs after the modifiers. Example \REF{ex:taji:25} presents a Chiyao sentence with a series of modifiers of these types followed by a demonstrative.

\ea%25
    \label{ex:taji:25}
    \gll chi-pula    chángu    chi-mo      ché-kúlúngwa    nnope    chi-ná-chi-súm-ile          lisó  \textbf{chila}\\  
  7-knife      7.\textsc{poss}    7-one          7-big                    \textsc{int}      \textsc{7rel-sm1}-buy-\textsc{perf}    yesterday  7.\textsc{dem.dist}\\  
  \glt ‘My one very big knife which I bought yesterday’
\z

Considering the position of the demonstrative in relation to other nominal dependents in the Chiyao NP, I propose the following NP template for Chiyao:

\ea%26
    \label{ex:taji:26}
    (\textsc{dem}) + \textsc{n} + \textsc{num} + \textsc{adj/assoc} + \textsc{rel} + \textsc{dem} 
\z

Thus, as the above template shows, the post-nominal demonstrative is the most final element in the Chiyao verb template.

\subsection{The pre-nominal and post-nominal position}\label{sec:taji:5.3}

A number of languages contain demonstratives that occur in sort of pairs; one before the head noun and the other after the head. (see \citealt{Lyons1999}). Due to their nature of occurrence -- before and after the head- they are often referred to as circumdemonstratives (see \citealt[6]{VandeVelde2005}). In Irish, for example, the simultaneous pre-nominal and post-nominal occurrence of demonstratives is evident is such forms as \textbf{\textit{an} }\textit{leabhar} ‘the book’, \textbf{\textit{an} }\textit{leabhar} \textbf{\textit{seo}} ‘this book’, \textbf{\textit{an} }\textit{leabhar} \textbf{\textit{sin}} ‘that book’, and \textbf{\textit{an} }\textit{leabhar} \textbf{\textit{úd}} ‘yonder book’ where the post-nominal particle is obligatory, and the pre-nominal determiner element is the definite article \citep[117]{Lyons1999}. A similar pattern is attested in Chiyao, where a demonstrative occurs first before the head noun in its full form and is then repeated after the noun in a reduced form without the initial vowel. In \REF{ex:taji:27} the distal demonstrative occurs in both post-nominal and pre-nominal positions.

\ea%27
    \label{ex:taji:27}
    \ea\label{ex:taji:27a} \gll n{}-tu{}-saídie            yákutí    pa{}-ku{}-m{}-okóa    \textbf{ajula}                    mu-ndu \textbf{júla}\\
      \textsc{sm1-2om}-help  how        16-\textsc{inf}-save      1.\textsc{dem.dist}    1-person  1.\textsc{dem.dist}\\
    \glt ‘Help us on how to save that person.’      

    \ex\label{ex:taji:27b} \gll nambó    \textbf{alila}                    li{}-kwáta    \textbf{lila}                        vá{}-á{}-tité       va{}-ngáli          ma{}-véngwa        a{}-ka{}-ika\\             
      but            5.\textsc{dem.dist}      5-dance      5.\textsc{dem.dist}    \textsc{sm2-pst}-say.\textsc{pfv}  \textsc{sm2-neg}      6-horn                  \textsc{sm2.neg}-come\\    
    \glt ‘But that dance, they said the one who doesn’t have horns should not attend.’
    \z
\z

The circumdemonstrative is the most frequently used demonstrative in Chiyao and in several neighbouring languages, including Makonde (P23) (see \citealt{Makanjila2019}), Shimwela (P22) (see \citealt{TajiMreta2017}), and Makhuwa (P31) (see \citealt{vanderWal2010}). Below are examples from these languages:

\ea Makonde (\citealt[171]{Makanjila2019})\\
    \label{ex:taji:28}
     ai timu ai inamala kung’ana mpila namene \\
    \gll \textbf{a-i}                         timu          \textbf{a-i}                        i-na-mal-a    ku-ng’an-a          m-pila namene\\         
    9.\textsc{dem.prox}   9.team        9.\textsc{dem.prox}  \textsc{sm9-tam}-know-\textsc{fv} \textsc{inf}-play-\textsc{fv}      3-ball  very\\ 
    \glt ‘This team plays football extremely well.’ 

\ex Shimwela (\citealt{TajiMreta2017})\\
    \label{ex:taji:29}
    \gll \textbf{aji}                         mi-kóngo        \textbf{ji}\\
  4.\textsc{dem.prox}   4-tree              4.\textsc{dem.prox}\\
  \glt ‘These trees’   

\ex Makhuwa (\citealt[201]{vanderWal2010})\\
    \label{ex:taji:30}
    \gll \textbf{ólé}                      nthíyán’      \textbf{uule}                    kh-oóthá aa-páh                    ólumwenku\\     
    1.\textsc{dem.dist}    1.woman    1.\textsc{dem.dist}      \textsc{neg-sm1.impf}-lie \textsc{sm1.impf}-light    14.world\\
    \glt ‘This woman didn’t just lie, she set the world on fire.’ 
\z

In sum, the preceding discussion of the distribution of demonstratives has revealed three main properties relating to Chiyao demonstratives. Firstly, Chiyao does not permit a pre-nominal demonstrative to occur alone without being complemented by a post-nominal demonstrative both of which together form a circumdemonstrative; secondly, the post-nominal demonstrative occupies the most final slot in the NP, after all other modifiers; thirdly, the simultaneous occurrence of demonstratives (also known as the circumdemonstrative) in both pre-nominal and post-nominal positions is the most favoured pattern in Chiyao. In the section that follows, I discuss different grammatical and communicative roles served by demonstratives in Chiyao.

\section{Functions of demonstratives}\label{sec:taji:6}

Demonstratives serve various linguistic roles. In addition to the commonly attested spatial-deictic function whereby they are used to indicate objects in three dimensions (proximal, distal, and non-proximal), demonstratives have also frequently been analyzed as definite markers, as in Bemba, Zulu and Xhosa \citep{Greenberg1978}; relative pronouns, for example in German \citep{Lehmann1984}; third person pronouns, for example in French (\citealt{Harris1978, Lambrecht1981}); sentence connectives, as in Hixkaryana in Brazil \citep{Derbyshire1985}; and possessive markers, as in Supyire in Mali \citep{Carlson1994}. It is important to note that these functions are by no means universal. The following sub-sections discuss the functions of demonstratives in Chiyao. 

\subsection{Spatial-deictic role}\label{sec:taji:6.1}

Demonstratives found in most languages are deictically contrastive. Most languages have a proximal demonstrative denoting closeness to the deictic center and a distal demonstrative denoting some relative distance from the deictic center \citep{Amfo2007}. Some languages have more elaborate demonstrative systems than others. Spanish, for example, has \textit{este}, \textit{ese} and \textit{aquel} which correspond to proximal, medial and distal demonstratives, respectively \citep{Amfo2007}. There are also other languages that make more than a two-way distinction of demonstratives. Tlingit (North West American) and Samal (Philippines), for example, have four deictic dimensions of demonstratives, while Malagasy (Austronesian) has six deictic dimensions \citep{Levinson1983}. Chiyao makes a three-way distinction in deictic demonstratives, namely close to the speaker (proximal), far from the speaker but closer to the addressee (non-proximal), and far from both the speaker and the addressee (distal). The Chiyao examples in \REF{ex:taji:31} are illustrative of this system.

\ea%31
    \label{ex:taji:31}
    \ea\label{ex:taji:31a} proximal\\
    \gll li-jela  ly-ángu      \textbf{ali}\\
      5-hoe    5-mine      5.\textsc{dem.prox}\\
     \glt  ‘My hoe is this one.’

     \newpage
    \ex\label{ex:taji:31b} non\_proximal\\
    \gll li-jela    ly-ángu        \textbf{alyo}\\
        5-hoe    5-mine      5.\textsc{dem.non\_prox}\\
     \glt ‘My hoe is that one.’

    \ex\label{ex:taji:31c} distal\\
    \gll li-jela    ly-ángu      \textbf{alila}\\
        5-hoe    5-mine    5.\textsc{dem.dist}\\
     \glt ‘My hoe is that one (over there).’ 
     \z
\z

The examples in \REF{ex:taji:31} above illustrate the spatial-deictic role of demonstratives in Chiyao by using the class 5 noun \textit{lijela} ‘hoe’. Thus, the demonstrative \textit{ali} \REF{ex:taji:31a} refers to a proximal entity, \textit{alyo} \REF{ex:taji:31b} refers to the non-proximal referent, and \textit{alila} \REF{ex:taji:31c} refers to a distal entity.   

\subsection{Emphatic role}\label{sec:taji:6.2}

Cross-linguistically, demonstratives are known for their property of encoding emphasis. In many languages, the expression of emphasis through demonstratives is done by reinforcing the existing demonstrative morphologically, through for example, the addition of further deictic particles, or through reduplication (\citealt{Lyons1999}; \textcitetv{chapters/asiimwe}). In Swahili, for example, reduplicated forms of demonstratives such as \textit{hikihiki} ‘this’, and \textit{kilekile} ‘that’ are widely attested \citep[116]{Lyons1999}. Like Swahili, Chiyao widely employs demonstrative reduplication to indicate emphasis of some parts of a message conveyed. The emphatic demonstrative is alternatively referred to as the confirmative demonstrative, as it induces the meaning ‘the very (same)’ \citep{Floor1998}. The Chiyao examples in \REF{ex:taji:32} below are illustrative of this phenomenon.

\ea%32
    \label{ex:taji:32}
    \ea\label{ex:taji:32a} \gll  va-lendo      vá-aiché                    u-síku          \textbf{úla}                          \textbf{úla}\\
      2-guest        \textsc{sm3}-arrive.\textsc{past}    14-day        14.\textsc{dem.dist}      14.\textsc{dem.dist}\\
    \glt ‘The guests came on that very day (not any other day).’

    \ex\label{ex:taji:32b} \gll  tw-ápité        li-tálá      \textbf{líla}                        \textbf{líla}\\
      2-\textsc{sm}-pass    5-way      5.\textsc{dem.dist}      5.\textsc{dem.dist}\\
     \glt ‘We used the same way (not any other way)’
    \z
\z

As is evident in \REF{ex:taji:32} above, the demonstratives \textit{ula} \REF{ex:taji:32a} and \textit{lila} \REF{ex:taji:32b} are reduplicated for emphatic purposes. The reduplicated demonstratives drop the initial vowel, a common tendency for post-nominal non-deictic demonstratives in Chiyao. The full forms of these demonstratives would be \textit{aula} \REF{ex:taji:32a} and \textit{alila} \REF{ex:taji:32b}. The emphatic role of demonstratives has also been reported in Makhuwa, as in \REF{ex:taji:33} (\citealt[186]{vanderWal2010}). 

\ea Makhuwa (\citealt[186]{vanderWal2010})\\
    \label{ex:taji:33}
    \gll \textbf{yoólé}                  mpákhá      wa-ámútsy’      aáwe\\
  1.\textssc{dem.dist}    until          16-2.family      2.\textsc{poss}\\
  \glt ‘She/the same went to family’s place.’   
\z

\subsection{Definiteness role}\label{sec:taji:6.3}

It is generally assumed that demonstratives are universally definite, and that definiteness exists in all languages \citep{Lyons1999}. \citet[2]{Lyons1999} associates definiteness with such properties as familiarity and identifiability. Thus, something definite is familiar to and identifiable among interlocutors. In many Bantu languages, demonstratives have been observed to perform a function similar to definite articles in languages which have articles (\citealt{VandeVelde2005}). In this way, the demonstrative is used to refer to a referent which is identifiable to both speaker and hearer. This role of demonstratives has been described in a number of Bantu languages, including Chaga (E62), Nyamwezi (F22), and Dciriku (K62) (\citealt{VandeVelde2005}). Similarly, \citet[53]{Taji2020} reported that demonstratives are important indicators of definiteness in Chiyao. In the examples below, the post-nominal demonstratives \textit{úla} \REF{ex:taji:34b} and \textit{líla} \REF{ex:taji:35b} show that the nouns they modify are familiar among the interlocutors, and thus definite. These contrast with the nouns in \REF{ex:taji:34a} and \REF{ex:taji:35a}, which are not modified by demonstratives, suggesting that they are indefinite.

\ea\citet[53]{Taji2020}\\
    \label{ex:taji:34}
    \ea\label{ex:taji:34a} \gll    m-kologo    u-jitíche \\
        3-alcohol    {\textsc{sm3}-be spilt}  \\
        \glt ‘Alcohol has been spilt.’

     \ex\label{ex:taji:34b} \gll  m-kologo    \textbf{úla}                    u-jitíche \\
        3-alcohol    3.\textsc{dem.dist}   {\textsc{sm3}-be spilt}\\
      \glt ‘That/the alcohol has been spilt.’   
      \z

 \ex \citet[53]{Taji2020}\\\label{ex:taji:35}  
    \ea\label{ex:taji:35a} \gll  m-ka-jigále        li-jela \\
          \textsc{1sm-fut}-take  5-hoe  \\
          \glt ‘Go and bring a hoe.’ 

    \ex\label{ex:taji:35b} \gll  m-ka-jigále     li-jela    \textbf{líla}  \\
          \textsc{1sm-fut}-take  5-hoe    5.\textsc{dem.dist}\\           
          \glt ‘Go and bring that/the hoe.’      
    \z
\z

Demonstratives that are used to express definiteness appear in three deictic distinctions – proximal, i.e. closer to the speaker \REF{ex:taji:36}, non-proximal, i.e closer to the hearer \REF{ex:taji:37}, and distal, i.e. far from both speaker and hearer \REF{ex:taji:38}.

\ea%36
    \label{ex:taji:36}
    \gll \textbf{achi}                        chí-téélá      \textbf{chi}\\
    7.\textsc{dem.prox}      7-tree          7.\textsc{dem.prox}\\
    \glt ‘This tree (near me, speaker)’

\ex%37
    \label{ex:taji:37}
    \gll \textbf{acho}                                  chí-téélá      \textbf{cho}\\
    7.\textsc{dem.non\_prox}      7-tree          7.\textsc{dem.non\_prox}\\         
  \glt ‘That tree (near you, hearer)’

\ex%38
    \label{ex:taji:38}
    \gll \textbf{achila}                    chí-téélá      \textbf{chíla}\\
    7.\textsc{dem.dist}      7-tree          7.\textsc{dem.dist}\\
  \glt ‘That tree (far from both of us)’
\z

In (\ref{ex:taji:36}--\ref{ex:taji:38}), the NPs are definite as they are modified by demonstratives. The definite reading of these sentences is further induced by the spatial deictic nature of the demonstratives used, which show that the referents are within the interlocutors’ visibility. 

As one would expect, there are structural and semantic differences between the sentences in (\ref{ex:taji:34}--\ref{ex:taji:35}) and those in (\ref{ex:taji:36}--\ref{ex:taji:38}) above. The examples in (\ref{ex:taji:34}--\ref{ex:taji:35}) contain only post-nominal demonstratives, and these refer to entities away from the interlocutors’ line of vision, while those in (\ref{ex:taji:36}--\ref{ex:taji:38}) each contain both pre-nominal and post-nominal demonstratives, and these refer to entities within the interlocutors’ vision. As such, they may be accompanied with a pointing gesture. This interpretation is consistent with \citet{Taji2020} who observed that, in Chiyao, demonstrative doubling is related to deictic definite NPs as in (\ref{ex:taji:36}--\ref{ex:taji:38}) above while single occurrence of demonstratives is associated with anaphoric reference as in (\ref{ex:taji:34}--\ref{ex:taji:35}). In anaphoric reference, demonstratives are used to refer to an entity with which the hearer is familiar not from the physical situation but from the broader linguistic context. The hearer is familiar with the entity because it was mentioned earlier in the text or discourse. One important aspect of anaphoric reference that deserves a separate section as an independent role of demonstratives is tail-head linking, and this is discussed in \sectref{sec:taji:6.4} below.

\subsection{Tail-head linking}\label{sec:taji:6.4}

Tail-head linking involves repetition of some part (usually the last -- the tail) of the previous sentence in the immediately following sentence (\citealt[201]{vanderWal2010}). This function of demonstratives is commonly encountered in narratives where an entity or character that has been mentioned in the last part of the previous sentence is co-referenced in the following sentence through a demonstrative. In \REF{ex:taji:39} below, the first sentence introduces the location \textit{mumbugu} ‘inside a cave’ which is referred to in the following sentence through a post-nominal demonstrative.

\ea%39
    \label{ex:taji:39}
    \gll ni          a-jáwile            ku-li-sísa                mú-mbugu. Ambano    mú-mbugu            \textbf{múla}                      mw-álijí                ni       méésí.\\
  and      \textsc{sm1}-go.\textsc{pst}    \textsc{inf-refl}-hide    \textsc{18loc}-9.cave now          \textsc{18loc}-9.cave      18.\textsc{dem.dist}      18-contain.\textsc{pst}    with    water\\
    \glt ‘And he/she went to hide himself/herself inside a cave. Now, inside that cave, there was   water.’
\z

Tail-head linking through demonstratives has also been reported in Makhuwa (\citealt[201]{vanderWal2010}). In \REF{ex:taji:40a} below, a woman is introduced into the story and she is referred to in the next sentence through a doubled demonstrative \REF{ex:taji:40b}:

\ea%40
    \label{ex:taji:40}
    \ea\label{ex:taji:40a}\gll o-m-phwánya                  nthíyáná      m-motsá\\
      \textsc{sm1.perf.dj}-1-meet    1.woman    1-one\\
    \glt ‘He met a woman.’

    \ex\label{ex:taji:40b}\gll  \textbf{ólé}                        nthíyán’      \textbf{uule}                      kh-oóthá      aa-páh                      ólumwenku\\
      1.\textsc{dem.prox}    1.woman    1.\textsc{dem.prox}      \textsc{neg-sm1.impf}-lie \textsc{sm1.impf}-light    14.world\\
    \glt ‘This woman didn’t just lie, she set the world on fire.’
    \z
\z

\section{Conclusion}\label{sec:taji:7}


This chapter has provided a description of Chiyao demonstratives, focusing on their form, distribution, and functions. It has revealed that in terms of form, the Chiyao demonstrative is comprised of an initial element, which is always the vowel \textit{a-,} followed by an agreement marker, and ends with a final element which changes according to the location of the referent in relation to the speaker or hearer. The demonstrative may be doubled, thus occurring in both pre- and post-nominal positions. In such cases, the post-nominal demonstrative drops the initial vowel but the pre-nominal demonstrative remains intact. The dropping of the initial vowel in the post-nominal demonstrative suggests that the initial element in the Chiyao demonstrative is optional. 



The chapter has further provided a syntactic description of demonstratives in Chiyao. The findings in this aspect have shown that, within the broader classification of demonstratives in Bantu languages, the Chiyao demonstratives appear in four main types. The first type includes pronominal demonstratives, which are used as independent pronouns, and the second type encompasses adnominal demonstratives, which modify nouns. The third type of demonstrative that are encountered in Chiyao is that of adverbial demonstratives. These function to modify verbs and locative nouns. The last type includes identificational demonstratives, which occur in copula and non-verbal clauses.



With regard to syntactic distributions of demonstratives, it is established that the Chiyao demonstrative can occur either post-nominally, or in both pre-nom\-i\-nal and post-nom\-i\-nal positions simultaneously. 


Finally, the description of functions of demonstratives in Chiyao has shown that demonstratives serve various linguistic and communication purposes, including expressing the location of an entity in relation to interlocutors, emphasis, definiteness, and tail-head linking in narratives. It is hoped that these findings will contribute to further understanding of the behavior of demonstratives in Bantu languages and will inspire further research in this area.
\section*{Acknowledgements}
I appreciate the inputs I received from the team of researchers of the project Morphosyntactic Variation in Bantu at the manuscript development workshop held in Dar es Salaam in September 2017. I am also grateful to the anonymous reviewers whose comments helped to improve this work to its present form.


\section*{Abbreviations}

\begin{multicols}{2}
\begin{tabbing}
\textsc{non\_prox}\hspace{1ex} \= Adjective\kill
\textsc{adj}       \> Adjective\\
\textsc{assoc}     \> Associative\\
\textsc{aug}       \> Augment\\
\textsc{dem}       \> Demonstrative\\
\textsc{dj}        \> Disjoint \\
\textsc{dist}      \> Distal \\
\textsc{fut}       \> Future\\
\textsc{impf}      \> Imperfective \\
\textsc{inf}       \> Infinitive \\
\textsc{intens}    \> Intensifier \\
\textsc{loc}       \> Locative \\
\textsc{neg}       \> Negative\\
\textsc{non\_prox} \> Non-proximal \\
\textsc{num}       \> Numeral \\
\textsc{om}        \> Object marker \\
\textsc{opt}       \> Optative\\
\textsc{pfv}       \> Perfective\\
\textsc{pla}       \> Plural addressee \\
\textsc{poss}      \> Possessive \\
\textsc{pres}      \> Present \\
\textsc{prox}      \> Proximal \\
\textsc{pst}       \> Past \\
\textsc{rel}       \> Relative clause\\
\textsc{refl}      \> Reflexive
\end{tabbing}
\end{multicols}

\printbibliography[heading=subbibliography,notkeyword=this]
\end{document}
