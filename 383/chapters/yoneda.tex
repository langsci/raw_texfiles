\documentclass[output=paper,
            colorlinks, citecolor=brown
            % ,draft
            ,draftmode
		  ]{langscibook}
\ChapterDOI{10.5281/zenodo.10663773}
    
\author{Nobuko Yoneda\orcid{}\affiliation{} and Judith Nakayiza\orcid{}\affiliation{}}

\title{Multiple-object constructions in Ganda}

\abstract{The possible number of pre-stem object markers (OMs) and the symmetrical\slash asym\-metrical nature of
objects in double-object constructions in Bantu languages have been widely discussed (\citealt{BresnanMoshi1993, Marlo2015, MartenKula2012, Zeller2014}, among others). While both objects
display the same syntactic properties in symmetrical languages, only one object has the syntactic
properties of the primary object in asymmetrical languages. Ganda (JE15), spoken in Uganda, is
considered a language that allows two OMs (\citealt{Ssekiryango2006, Marlo2015}), and also a
symmetrical object language, according to the criteria of \citet{BresnanMoshi1993}. However,
according to our observations, three OMs are possible in certain situations. Although Ganda
predominantly shows the behaviour of a symmetrical object language, some asymmetrical
behaviour can be also observed. This paper demonstrates how object NPs and OMs behave in
multiple-object constructions in Ganda. It also shows the asymmetrical tendency of the language
and where three OMs are possible.}

\IfFileExists{../localcommands.tex}{
  \addbibresource{../localbibliography.bib}
  \usepackage{langsci-optional}
\usepackage{langsci-gb4e}
\usepackage{langsci-lgr}

\usepackage{listings}
\lstset{basicstyle=\ttfamily,tabsize=2,breaklines=true}

%added by author
% \usepackage{tipa}
\usepackage{multirow}
\graphicspath{{figures/}}
\usepackage{langsci-branding}

  
\newcommand{\sent}{\enumsentence}
\newcommand{\sents}{\eenumsentence}
\let\citeasnoun\citet

\renewcommand{\lsCoverTitleFont}[1]{\sffamily\addfontfeatures{Scale=MatchUppercase}\fontsize{44pt}{16mm}\selectfont #1}
   
  %% hyphenation points for line breaks
%% Normally, automatic hyphenation in LaTeX is very good
%% If a word is mis-hyphenated, add it to this file
%%
%% add information to TeX file before \begin{document} with:
%% %% hyphenation points for line breaks
%% Normally, automatic hyphenation in LaTeX is very good
%% If a word is mis-hyphenated, add it to this file
%%
%% add information to TeX file before \begin{document} with:
%% %% hyphenation points for line breaks
%% Normally, automatic hyphenation in LaTeX is very good
%% If a word is mis-hyphenated, add it to this file
%%
%% add information to TeX file before \begin{document} with:
%% \include{localhyphenation}
\hyphenation{
affri-ca-te
affri-ca-tes
an-no-tated
com-ple-ments
com-po-si-tio-na-li-ty
non-com-po-si-tio-na-li-ty
Gon-zá-lez
out-side
Ri-chárd
se-man-tics
STREU-SLE
Tie-de-mann
}
\hyphenation{
affri-ca-te
affri-ca-tes
an-no-tated
com-ple-ments
com-po-si-tio-na-li-ty
non-com-po-si-tio-na-li-ty
Gon-zá-lez
out-side
Ri-chárd
se-man-tics
STREU-SLE
Tie-de-mann
}
\hyphenation{
affri-ca-te
affri-ca-tes
an-no-tated
com-ple-ments
com-po-si-tio-na-li-ty
non-com-po-si-tio-na-li-ty
Gon-zá-lez
out-side
Ri-chárd
se-man-tics
STREU-SLE
Tie-de-mann
} 
  \togglepaper[1]%%chapternumber
}{}

\begin{document}
\maketitle 
\judgewidth{??}
%\shorttitlerunninghead{}%%use this for an abridged title in the page headers



\section{Introduction}\label{sec:yoneda:1}


Multiple pre-stem object markers (OMs) and the symmetrical/asymmetrical nature of objects in Bantu languages have received much attention in Bantu research (\citealt{BresnanMoshi1993, Marlo2015, MartenKula2012, Zeller2014}, among many others). In double-object constructions in symmetrical languages, both objects display the same syntactic properties, whereas in asymmetrical languages, only one object has the syntactic properties of the primary object (see \REF{ex:yoneda:4}), and the other is restricted in its syntactic behaviour (\citealt{BresnanMoshi1993}). 

    Ganda, spoken in Uganda, is considered a language that allows two OMs (\citealt{Ssekiryango2006, Marlo2015, Ranero2019, vanderWal2020}), and is considered a symmetrical language \citep{Ssekiryango2006}. Our observations reveal, however, that three OMs are possible in certain situations. Moreover, although Ganda predominantly shows the behaviour of a symmetrical language, some asymmetrical behaviour can also be observed. 

    This chapter examines how object NPs and OMs behave in multiple-object constructions and in which morpho-syntactic contexts asymmetry may emerge in Ganda. In \sectref{sec:yoneda:2}, we lay out the background of the research on multiple-object constructions. In \sectref{sec:yoneda:3}, we show the behaviour of object NPs and OMs in double-object constructions, while in \sectref{sec:yoneda:4} we focus on triple-object constructions. In \sectref{sec:yoneda:5} we discuss the characteristics of multiple-object constructions in Ganda regarding the behaviour and constraints that are observed in  \sectref{sec:yoneda:3} and \sectref{sec:yoneda:4}. The data are examined against the parameters of object marking proposed by \citet{GuéroisEtAl2017}, in particular parameters 75, 76, 78, 109, and 110.


\section{Background to the research}\label{sec:yoneda:2}



Ganda is spoken in Uganda where it is used both as a first language and as a language of wider communication. In \citegen{Maho2009} classification, Ganda is classified as JE15 in the major group of Nyoro-Ganda. Ganda has been relatively well researched and a reference grammar (\citealt{AshtonEtAl1954}) and numerous textbooks have been written. 



Like many other Bantu languages, the Ganda verb is agglutinative, and constructed with a root and different kinds of affixes: subject and object agreement prefixes (SM, OM), affixes that determine the tense, aspect and mood (PreSM, TAM, and Final), and derivational suffixes (DER), as follows:\footnote{All Ganda data come from the second author.} 



\ea%1
    \label{ex:yoneda:1}
    tebáátúzímbira\\
    \glll PreSM-  \textsc{sm}-      \textsc{tam}-  \textsc{om}-      \textsc{root}    -\textsc{der}    {}-Final\\
         te-      bá-      á-      tú-        zímb    {}-ir      {}-a\\
         \textsc{neg}-    \textsc{sm3pl}-  \textsc{pst}-  \textsc{om1pl}-    build    {}-\textsc{appl}    {}-\textsc{fv}\\
    \glt ‘they did not build for us’
    \z
      


    There are some Bantu languages in which the object NP and the corresponding OM cannot co-occur in a clause, others in which the existence of the OM affects the finiteness, and still others in which the presence/absence of the OM depends on the animacy of the corresponding object. Ganda is a language that allows the co-occurrence of the object NP and the corresponding OM. Moreover, the OM is optional regardless of the animacy of the object NP, and the existence of an OM does not affect the definiteness of the object NP, as shown in \REF{ex:yoneda:2} and \REF{ex:yoneda:3}.  



\ea%2
    \label{ex:yoneda:2}
    \ea\label{ex:yoneda:2a}\gll N-á-lábyé               omu-lénzí.\\
         \textsc{sm1sg-pst}-meet.\textsc{prf}  1-boy\\
    \ex\label{ex:yoneda:2b} \gll N-á-\textbf{mú}{}-lábyé omu-lénzí.\\
    \textsc{sm1sg-pst-om1}-meet.\textsc{prf}    1-boy \\
    \glt ‘I saw a/the boy.’
    \z
\ex%3
    \label{ex:yoneda:3}
    \ea\label{ex:yoneda:3a} \gll N-a-gúl-a               eki-tábo. \\
         \textsc{sm1sg-pst}-buy-\textsc{fv}    7-book\\
    \ex\label{ex:yoneda:3b} \gll N-a-\textbf{kí}{}-gúl-a               eki-tábo.\\
    \textsc{sm1sg-pst-om7}-buy-\textsc{fv}    7-book\\
    \glt ‘I bought a/the book.’
    \z
\z



In the literature, the following are generally assumed to be the main syntactic properties of the primary object in Bantu languages. 



\ea \citet[220]{HymanDuranti1982}
    \label{ex:yoneda:4}
    \ea\label{ex:yoneda:4a} has access to the position immediately following the verb
    \ex\label{ex:yoneda:4b} is capable of assuming the subject role through passivization
    \ex\label{ex:yoneda:4c} can be expressed by an object marker within the verbal complex
    \z
\z


\citet{BresnanMoshi1993} divide Bantu languages into two types according to the syntactic behaviour of these objects, namely symmetrical and asymmetrical languages. In symmetrical languages, both (or all) objects can exhibit the syntactic properties of the primary object shown in \REF{ex:yoneda:4}, whereas in asymmetrical languages, only one object can exhibit the syntactic properties of the primary object (\citealt[47]{BresnanMoshi1993}). Example \REF{ex:yoneda:5} is from Tswana, a symmetrical language, and \REF{ex:yoneda:6} is an example of Herero, an asymmetrical language.


\ea Tswana (symmetrical type)    (\citealt[269]{MartenEtAl2007})
    \label{ex:yoneda:5}
    \ea\label{ex:yoneda:5a} \gll ke         ape-ets-e           ngwana     kuku\\
         \textsc{sm1.prs}   cook{}-\textsc{appl-prf}     1.child     9.chicken\\
    \glt ‘I cooked the child the chicken’
    \ex\label{ex:yoneda:5b} \gll ke         ape-ets-e           kuku       ngwana\\
    \textsc{sm1.prs}  cook{}-\textsc{appl-prf}     9.chicken   1.child\\
    \glt ‘I cooked the chicken for the child’
    \z

\newpage
\ex Herero (asymmetrical type)  (\citealt[269]{MartenEtAl2007})
    \label{ex:yoneda:6}
    \ea[]{\label{ex:yoneda:6a}\gll  mávé       tjàng-ér-é         òvà-nâtjé     ò-mbàpírà \\
         \textsc{prg.sm2}  write{}-\textsc{appl-fv}   2-children     9-letter\\
    \glt ‘They are writing the children a letter’}
    \ex[*]{\label{ex:yoneda:6b}\gll mave       tjang-er-e         o-mbapira   ova-natje\\
    \textsc{prg.sm2}   write{}-\textsc{appl-fv}   9-letter    2-children\\
    \glt (Intended meaning: ‘They are writing a letter to the children’)}
    \z
\z

In Tswana, a symmetrical language, both orders of objects are acceptable in a double-object construction. In Herero, an asymmetrical language, only the benefactive/goal can access the position immediately following the verb. This asymmetry correlates with various factors, such as the semantic role and animacy hierarchy, depending on the language.  


Following this idea, in this chapter we discuss the behaviour of the objects in multiple-object constructions in Ganda in terms of symmetry/asymmetry. The criteria to determine the primary object are as follows: (a) if it can be placed immediately after the verb, (b) if it can be the subject in a passive, and (c) if it can be pronominalized inside the verbal complex. 



Ganda has some ditransitive verbs such as \textit{{}-gamba} ‘tell’ or -\textit{wa} ‘give’ just like English. They are underived ditransitive verbs, as shown in \REF{ex:yoneda:7}. 



\ea%7
    \label{ex:yoneda:7}
    \ea\label{ex:yoneda:7a}\gll N-á{}-gámbyé            emi-kwánó   ama-úlire.\\
        \textsc{sm1sg-pst}-tell.\textsc{prf}    4-friends      6-news  \\
    \glt ‘I told friends the news.’
    \ex\label{ex:yoneda:7b} \gll N-á-wáddé             emi-kwánó   amá-tooke.\\
    \textsc{sm1sg-pst}-give.\textsc{prf}  4-friends      6-bananas\\
    \glt ‘I gave friends bananas.’
    \z
\z

In addition, there are verbs that become ditransitive by derivation, such as in the applicative or causative verb forms of transitive verbs, as shown in \REF{ex:yoneda:8b} and \REF{ex:yoneda:8c}, respectively. 


\ea%8
    \label{ex:yoneda:8}
    \ea\label{ex:yoneda:8a}\gll  N-a-fúmbyé               amá-tooke.\\
         \textsc{sm1sg-pst}-cook.\textsc{prf}    6-bananas\\
    \glt ‘I cooked bananas.’
    \ex\label{ex:yoneda:8b}\gll N-á{}-fúmb-íddé                 aba-ana     amá-tooke.\\
    \textsc{sm1sg-pst}-cook-\textsc{appl.prf}    2-children  6-bananas\\
    \glt ‘I cooked bananas for children.’
    \ex\label{ex:yoneda:8c}\gll N-á{}-fúmb-ísízzá                 aba-ana       amá-tooke.\\
    \textsc{sm1sg-pst}-cook-\textsc{caus.prf}    2-children    6-bananas\\
    \glt ‘I have made children cook bananas.’
    \z
\z


In \sectref{sec:yoneda:3} and \sectref{sec:yoneda:4}, we will examine the behaviour of the objects of underived ditransitive, applicative, and causative verbs.



\section{Double-object constructions}\label{sec:yoneda:3}


This section demonstrates the behaviour of each object in double-object constructions, including applicative and causative constructions besides underived ditransitive verbs. For each, we will discuss the three criteria, word order in \sectref{sec:yoneda:3.1}, passivization in \sectref{sec:yoneda:3.2}, and object marking in \sectref{sec:yoneda:3.3}. Relative animacy of the objects is indicated with A > B (A is higher than B), A < B (A is lower than B), or A = B (equal animacy).


\subsection{Word order of object NPs}\label{sec:yoneda:3.1}



As already shown in \REF{ex:yoneda:4a}, the NP which can be placed immediately after the verb (hereafter IAV) is considered the primary object. Here we examine, using the ditransitive, applicative, and causative verbs with double objects, which of the two objects can be placed IAV, and whether the animacy or semantic role of the objects plays a role. 



\subsubsection{Ditransitive verbs}\label{sec:yoneda:3.1.1}



A typical ditransitive verb is -\textit{wa} ‘give’. The semantic roles associated with the objects of this verb are those of recipient and theme. The recipient is \textit{emikwano} ‘friends’ in \REF{ex:yoneda:9} and \REF{ex:yoneda:10}, and the themes are \textit{amatooke} ‘bananas’ in \REF{ex:yoneda:9} and \textit{embwa} ‘dog’ in \REF{ex:yoneda:10}. In both examples, the recipient is higher in terms of animacy than the theme.      



\ea%9
    \label{ex:yoneda:9} [\textsc{recipient}: friends (human), \textsc{theme}: bananas (entity)]  \textsc{recipient} > \textsc{theme}
    \ea\label{ex:yoneda:9a}\gll Máamá     a-wáddé       emi-kwánó   amá-tooke.\\
         1.mother    \textsc{sm1}-give.\textsc{prf}  4-friends      6-bananas\\
    \glt ‘Mother has given the friends bananas.’
    \ex\label{ex:yoneda:9b}\gll Máamá     a-wádde       amá-tooke   emi-kwánó.\\
    1.mother    SM1-give.PRF  6-bananas    4-friends\\
    \glt ‘Mother has given bananas to the friends.’
    \z
\ex%10
    \label{ex:yoneda:10}
    [\textsc{recipient}: friends (human), \textsc{theme}: dog (animal)]  \textsc{recipient} > \textsc{theme}
    \ea\label{ex:yoneda:10a}\gll Máamá    a-wádde       emi-kwánó   embwá.\\
          1.mother    \textsc{sm1}-give.\textsc{prf}  4-friends      9.dog\\
    \glt ‘Mother has given the friends a dog.’
    \ex\label{ex:yoneda:10b}\gll Máamá     a-wádde         embwá   emi-kwánó.\\
    1.mother    \textsc{sm1}-give.\textsc{prf}  9.dog    4-friends \\
    \glt ‘Mother has given a dog to the friends.’
    \z
\z
       

                 

        

When the recipient is higher than the theme in animacy, either object of the ditransitive verb -\textit{wa} ‘give’ can be placed IAV as shown in \REF{ex:yoneda:9} and \REF{ex:yoneda:10}.

\citet[70]{Ssekiryango2006} shows examples in which only the recipient can be placed IAV when both objects are of equal animacy. \citet[599]{Ranero2019} also states that the order of object NPs in ditransitive sentences is strictly fixed in that the ‘recipient (goal)/benefactive’ is placed before the ‘theme’. In our data, however, both orders are acceptable. Although the order in which the recipient appears in the IAV position is more natural than the other order as shown in \REF{ex:yoneda:11}. However, still the order of these objects seems not to be as strict as \citet{Ranero2019} claims.\footnote{Some speakers say that interpretation 2 in \REF{ex:yoneda:11a}, \REF{ex:yoneda:13a} and interpretation 1 in \REF{ex:yoneda:11b}, \REF{ex:yoneda:13b} are not acceptable. However, these orders in which the theme precedes the recipient or benefactive is often used, especially among older generations. Such differences as the disagreement between our data and data of \citet{Ssekiryango2006} and \citet{Ranero2019} are also seen in other properties. There might be generational or/and areal variation.} The presence of ‘?’ in front of the sentence indicates that the utterance is less natural than the other, but is still grammatical and acceptable. 

\ea%11
    \label{ex:yoneda:11}
    [\textsc{recipient}: daughter (human), \textsc{theme}: hunter (human)]    \textsc{recipient} = \textsc{theme}
    \ea\label{ex:yoneda:11a}\gll  Kabáka   ya-wá             mu-walá     we     omu-yízzi. \\
         1.king    \textsc{sm1.pst}-give.\textsc{fv}    1-daughter    1.his  1-hunter\\
    \glt Interpretation 1:  ‘The king gave his daughter the hunter.’\\
       ?Interpretation 2:  ‘The king gave his daughter to the hunter.’
    \ex\label{ex:yoneda:11b} \gll Kabáka   ya-wá             omu-yízzi  mu-walá    we.\\
    1.king    \textsc{sm1.pst}-give.\textsc{fv}     1-hunter    1-dughter  1.his\\
    \glt ?Interpretation 1:  ‘The king gave the hunter to his daughter.’\\
         Interpretation 2:  ‘The king gave the hunter his daughter.’
    \z
\z  

            

In \REF{ex:yoneda:11}, both objects are human and there is no difference in animacy. Although \REF{ex:yoneda:11a}, in which the recipient NP is placed IAV, is more natural and is preferred for the translation of ‘the king gave his daughter the hunter’, \REF{ex:yoneda:11b} is still possible, and thus, both \REF{ex:yoneda:11a} and \REF{ex:yoneda:11b} are ambiguous with the reading ‘the king gave his daughter to the hunter’.

With ditransitive verbs, the objects are symmetrical with respect to word order, although there is a preference for placing the recipient in the IAV position. 


\subsubsection{Applicative verbs}\label{sec:yoneda:3.1.2}



The applicative verb form in Ganda is formed by adding the derivational suffix -\textit{ir}.\footnote{When this suffix appears with the perfect final \textit{{}-ili}, it appears as \textit{{}-dde} as a result of the application of morpho-phonological rules.}  When a verb appears in the applicative form, a new object is introduced, the applied object (AO), which contrasts with the base object (BO), which is the original object of the base verb without derivation. A typical semantic role associated with the applied object is the benefactive,\footnote{In Ganda, the applicative verb with its applied object X can be interpreted as `for X' or `on behalf of X' depending on the context. We therefore use the standard semantic role `benefactive (\textsc{ben})' for an applied object with either of these interpretations.} and the base object is the theme.

When the applied object is higher than the base object in animacy, either object can be placed IAV as shown in \REF{ex:yoneda:12}.

\ea%12
    \label{ex:yoneda:12}
    [AO (\textsc{ben}): friends (human), BO (\textsc{theme}): bananas (entity)]  AO > \textsc{theme}
    \ea\label{ex:yoneda:12a}\gll Máama     a-fúmb-ír-á          mi-kwánó   gyange   amá-tooke.\\
         1.mother    \textsc{sm1}-cook-\textsc{appl-fv}  4-friends    4.my    6-bananas\\
    \ex\label{ex:yoneda:12b} \gll Máama     a-fúmb-ír-á           amá-tooke  mi-kwánó   gyange.\\
    1.mother    \textsc{sm1}-cook-\textsc{appl-fv}  6-bananas  4-friends    4.my\\
    \glt ‘Mother is cooking bananas for my friends.’
    \z
\z
    


When both objects are of equal animacy, the order in which the applied object precedes the base object \REF{ex:yoneda:13a} is more natural than the other order, and usually the object IAV is interpreted as the applied object. However, the opposite order is still possible, resulting in two possible interpretations, as shown in \REF{ex:yoneda:13}. Only the context allows the hearer to make a choice between the two possible interpretations.

\newpage
\ea%13
    \label{ex:yoneda:13}
    [AO (\textsc{ben}): daughters (human), BO (\textsc{theme}): man (human)]  AO= \textsc{theme}
    \ea\label{ex:yoneda:13a}\gll Máama     a-kúb-íddé           ba-walá       be     omu-sájja.\\
         1.mother    \textsc{sm1}-beat-\textsc{appl.prf}  2-daughters  2.her  1-man \\
    \glt Interpretation 1:  ‘Mother has beaten a man for/because of her daughters.’ \\
     ?Interpretation 2:  ‘Mother has beaten her daughters for/because of the man.’
     \ex\label{ex:yoneda:13b}\gll Máama     a-kúb-íddé           omu-sájja   ba-walá       be.\\
     1.mother    \textsc{sm1}-beat-\textsc{appl.prf}  1-man      2-daughters  2.her\\
     \glt ?Interpretation 1:  ‘Mother has beaten a man for/because of her daughters.’   \\
       Interpretation 2:  ‘Mother has beaten her daughters for/because of the man.’
    \z
\z


    In cases where the semantic role of the applied object is the benefactive, its animacy is rarely lower than that of the base object. However, when the semantic role of the applied object is the reason or motivation, the animacy of the applied object can be lower. The semantic role of the applied object in \REF{ex:yoneda:14} is a reason, and its animacy is indeed lower than that of the base object. When the base object is higher up on the animacy hierarchy than the applied object, the order in which the base object precedes the applied object \REF{ex:yoneda:14a} is more natural than the other order \REF{ex:yoneda:14b},\footnote{This might not be because of the animacy feature, but because of the semantic role. It is obvious that showing an example with benefactive is ideal, but it is not easy to find a good example with benefactive in lower animacy than theme.} although both orders are possible. 

\ea%14
    \label{ex:yoneda:14}
    [AO (\textsc{reason}): party (entity), BO (\textsc{theme}): friends (human)]  AO < \textsc{theme} 
    \ea[]{\label{ex:yoneda:14a}\gll Tú-jjá           ku-yít-ir-a           mi-kwánó   gyaffe   embága.\\
         \textsc{sm1pl-fut}    \textsc{inf}-call-\textsc{appl-fv}  4-friends    4.our    9.party\\}
    \ex[?]{\label{ex:yoneda:14b}\gll Tú-jjá           ku-yít-ir-a             embága   mi-kwánó  gyaffe.\\
    \textsc{sm1pl-fut}    \textsc{ind}-call-\textsc{appl-fv}    9.party   4-friends    4.our\\
    \glt ‘We will call our friends for a party.’
    }
    \z
\z

        
         

     With the applicative verb, as well as with the ditransitive verb, the objects are symmetrical in terms of word order, and neither the semantic role nor animacy determines the ordering of the object NPs, although there seems to be a moderate tendency that either the benefactive or one which is higher in animacy is preferably placed IAV. 


\subsubsection{Causative verbs}\label{sec:yoneda:3.1.3}



The causative verb form in Ganda is formed by adding the suffix -\textit{is}\footnote{When this suffix appears with the perfect final \textit{{}-ili}, it appears as \textit{{}-isizza} as the result of the application of some morpho-phonological rules.}  to the verb root. When the verb appears in the causative form, the causee appears as an object. 



When the causee is a human and the theme an inanimate entity, placing the causee IAV is more natural, but both orders are still grammatical as shown in \REF{ex:yoneda:15}. Therefore, based on these examples, we can conclude that both the causee and theme can be placed IAV.    



\ea%15
    \label{ex:yoneda:15}
    [\textsc{causee}: friends (human), \textsc{theme}: banana (entity)]  \textsc{causee} > \textsc{theme}
    \ea[]{\label{ex:yoneda:15a}\gll Máama     a-fúmb-ísízzá           mi-kwánó   gyange  amá-tooke.\\
         1.mother    \textsc{sm1}-cook-\textsc{caus.prf}    4-friends    4.my    6-bananas\\
        }
    \ex[?]{\label{ex:yoneda:15b}\gll Máama     a-fúmb-ísízzá           amá-tooke  mi-kwánó   gyange.\\
    1.mother    \textsc{sm1}-cook-\textsc{caus.prf}    6-bananas  4-friends    4.my \\
    \glt ‘Mother has made my friends cook bananas.’
    }
    \z 
\z


        

    However, in cases where the causee and theme have equal animacy, only the causee can be placed IAV. In other words, only the NP which is placed IAV is interpreted as the causee as shown in \REF{ex:yoneda:16}. The sign \# indicates that the utterance is grammatical but does not have the intended meaning in which the causee is ‘girls’ and the theme is ‘boy’. Therefore the sentence cannot be used for the interpretation prefixed with *.

\ea%16
    \label{ex:yoneda:16}
    [\textsc{causee}: girls (human), \textsc{theme}: boy (human)]  \textsc{causee} = \textsc{theme}
    \ea[]{\label{ex:yoneda:16a}\gll Máama     a-gób-ésézzá             aba-wála   omu-lénzí.\\
         1.mother    \textsc{sm1}-chase-\textsc{caus.prf}    2-girls      1-boy\\
    \glt ‘Mother has made the girls chase away the boy.’}
    \ex[\#]{\label{ex:yoneda:16b}\gll Máama     a-gób-ésézzá             omu-lénzí  aba-wála.\\
    1.mother    \textsc{sm1}-chase-\textsc{caus.prf}    1-boy      2-girls\\
    \glt Interpretation 1: ‘Mother has made the boy chase away the girls.’\\
 *Interpretation 2: ‘Mother has made the girls chase away the boy.’}
    \z
\z    

              



    In Ganda, an instrument can be expressed as the causee. In \REF{ex:yoneda:17}, the instrument \textit{omúggo} ‘stick’ appears as the causee. The causee is an inanimate entity and the theme is an animal; the animacy of the causee is thus lower than that of the theme. Both the causee \textit{omúggo} ‘stick’ and the theme \textit{embwá} ‘dog’ can be placed IAV as shown in \REF{ex:yoneda:17}. 

\ea%17
    \label{ex:yoneda:17}
    [\textsc{causee}: stick (entity), \textsc{theme}: dog (animal)]  \textsc{causee} < \textsc{theme}\\
    \ea\label{ex:yoneda:17a}\gll Maama    a-kúb-íssá                   omú-ggo    embwa.\\
         1.mother   \textsc{sm1.pst}-beat-\textsc{cause.prf}    3-stick      9.dog \\
    \ex\label{ex:yoneda:17b}\gll Maama    a-kúb-íssá                   embwa   omú-ggo.\\
    1.mother   \textsc{sm1.pst}-beat-\textsc{caus.prf}    9.dog     3-stick\\
    \glt ‘Mother beat a dog with a stick.'
        (literally meaning: ‘mother caused a stick to beat a dog.’)
    \z
\z
        

From examples \REF{ex:yoneda:15} and \REF{ex:yoneda:17}, it seems that with the causative verb as well, objects are symmetrical in terms of word order, except when both objects are of equal animacy. Indeed, \REF{ex:yoneda:16} shows an asymmetrical behaviour in that the causee must be placed in IAV position when both objects are equal in their animacy.  

\subsubsection{Summary of word order of object NPs} \label{sec:yoneda:3.1.4}

When the animacy of the theme is lower than that of the other object, these objects behave symmetrically (although there is a moderate preference for a non-theme to be placed IAV) in all ditransitive, applicative, and causative verbs, with two exceptions. One is with causatives. When both objects of a causative verb are of equal animacy, only the causee can be placed IAV and must precede the other object. It is unlike the case of ditransitive or applicative, in which there may be ambiguity in interpretation. Another exception is seen in applicative. When the animacy of the applied object is lower than that of the theme, placing the theme IAV is preferable. 

\citet[69]{Ssekiryango2006} claims that objects with higher animacy appear IAV as the primary object, and \citet[599]{Ranero2019} claims that the order is fixed according to the semantic roles.  However, our data show that in most cases both objects can appear IAV regardless of their semantic role or animacy.     

These facts are summarized in \tabref{tab:yoneda:1}. We conclude that, in terms of word order, objects are predominantly symmetrical in Ganda. 

\begin{table}
\small
\begin{tabularx}{\textwidth}{QllQllQll}

\lsptoprule
Recipient > Theme & Rec. & Yes & Rec. = Theme & Rec. & Yes & Rec. < Theme & Rec. & {}--\\
 & Theme & Yes &  & Theme & ? &  & Theme & {}--\\
\tablevspace
Benefactive > Theme & Ben. & Yes & Ben. = Theme & Ben. & Yes & Reason < Theme & reason & ?\\
 & Theme & Yes &  & Theme & ? &  & Theme & Yes\\
\tablevspace
Causee > Theme & Causee & Yes & Cau. = Theme & Causee & Yes & Cau. < Theme & Causee & Yes\\
 & Theme & ? &  & Theme & \cellcolor{lsLightGray}* &  & Theme & Yes\\
\lspbottomrule
\end{tabularx}
\caption{IAV positioning of ditransitive objects. *: not acceptable,  ?: less natural but acceptable}
\label{tab:yoneda:1}
\end{table}

\subsection{Passivization}\label{sec:yoneda:3.2}



The passive sentence in Ganda is constructed by adding the derivational suffix \textit{{}-w}\footnote{When this suffix appears with the perfect final \textit{{}-ili}, it appears as \textit{{}-íddwa} or -\textit{éddwá} as a result of the application of some morpho-phonological rules.} after the verb root. No overt marker is used to introduce the agent noun phrase, as is shown in \REF{ex:yoneda:18a} (cf. \REF{ex:yoneda:18b}):


\ea%18
    \label{ex:yoneda:18}
    \ea\label{ex:yoneda:18a}\gll  Eki-tabo    ki-som-eddwa           aba-ntu    bangi.\\
         7-book    \textsc{sm7}-read-\textsc{pass.prf}    2-people    2.many\\
    \glt ‘The book has been read by many people.’
    \ex\label{ex:yoneda:18b} \gll  Aba-ntu   bangi     ba-somye        eki-tabo.\\
    2-people    2.many  \textsc{sm2}-read.\textsc{prf}    7-book\\
    \glt ‘Many people have read the book.’
    \z
\z
        



    The ability to become the subject of the passive sentence is one of the key syntactic properties of the primary object in Bantu languages as shown earlier in \REF{ex:yoneda:4b}. We will see which object can be the subject of a passive sentence in double-object constructions with ditransitive, applicative, and causative verbs.



\subsubsection{Ditransitive}
\label{sec:yoneda:3.2.1}


\REF{ex:yoneda:19a} is a passive example in which the recipient is the subject, and in \REF{ex:yoneda:19b} the theme is the subject of the passive verb.\footnote{No preposition is used to introduce the actor in passive sentences in Ganda. Therefore, the order of recipient\slash theme and actor is also an interesting issue. However, in this paper, we concentrate on their ability to appear as the subject of a passive sentence. In actual use the actor (‘maama’ in \REF{ex:yoneda:19}) is often deleted (see \citeauthor{Ssekiryango2006}'s analysis about Ganda not allowing the presence of the agent in a passive sentence derived from a double-object construction (\citeyear[72]{Ssekiryango2006})).} \REF{ex:yoneda:19c} is the corresponding active sentence.  

\newpage
\ea%19
    \label{ex:yoneda:19}
    [\textsc{recipient}: friends (human), \textsc{theme}: banana (entity)]  \textsc{recipient} > \textsc{theme}
    \ea\label{ex:yoneda:19a}\gll Mi-kwánó  gyange   gí-wéréddwá         amá-tooke   (máama).\\
         4-friends    4.my    \textsc{sm4}-give.\textsc{pass.prf}  6-bananas    1.mother\\
    \glt `My friends have been given bananas (by my mother).'
    \ex\label{ex:yoneda:19b}\gll  Ama-tóóké   gá-wéréddwá           mi-kwánó    gyange   (máama)\\
    6-bananas    \textsc{sm6}-give.\textsc{pass.prf}    4-friends      4.my    1.mother\\
    \glt ‘Bananas have been given to my friends (by my mother).’
    \ex\label{ex:yoneda:19c} \gll Máamá     a-wáddé       mi-kwánó    gyange   amá-tooke.\\
    1.mother    \textsc{sm1}-give.\textsc{prf}  4-friends      4.my    6-bananas\\
    \glt ‘Mother has given my friends bananas.’
    \z
\z

With ditransitive verbs, both objects can be the subject in a passivized sentence when the recipient is higher than the theme in animacy, as shown in \REF{ex:yoneda:19}. 

\REF{ex:yoneda:20} exemplifies cases where both objects are of equal animacy. \REF{ex:yoneda:20a} is an example in which the recipient \textit{muwalá} \textit{wa kabaka} ‘king’s daughter’ is the subject, and in \REF{ex:yoneda:20b} the theme \textit{omuyízzi} ‘a hunter’ is the subject of the passive verb. \REF{ex:yoneda:20c} is the corresponding active sentence.


\ea%20
    \label{ex:yoneda:20}
    [\textsc{recipient}: daughter (human), \textsc{theme}: hunter (human)]  \textsc{recipient} = \textsc{theme}
    \ea\label{ex:yoneda:20a} \gll Mu-walá   wa     kabaka   a-wereddwa           omu-yízzi  \\
         1-daughter \textsc{gen}  1.king    \textsc{sm1}-give.\textsc{pass.prf}  1-hunter\\
    \glt Interpretation 1:   ‘The daughter of king has been given the hunter.’\\
 ?Interpretation 2:  ‘The daughter of king has been given to the hunter.’
    \ex\label{ex:yoneda:20b} \gll Omu-yízzi   a-wereddwa           mu-walá     wa     kabaka.\\
    1-hunter      \textsc{sm1}-give.\textsc{pass.prf}  1-daughter    \textsc{gen}  1.king\\
    \glt ?Interpretation 1:  ‘The hunter has been given to king’s daughter.’  \\
    Interpretation 2:  ‘The hunter has been given the king’s daughter.’ 
    \ex\label{ex:yoneda:20c} \gll Kabáka   a-wadde         mu-walá     we     omu-yízzi.\\
    1.king    \textsc{sm1}-give.\textsc{prf}    1-daughter  1.his  1-hunter\\
    \glt ‘The king has given his daughter the hunter.’
    \z
\z
        
Both objects can be the subject of the passivized sentence, though the preference is given for the recipient to be the subject. As a result, \REF{ex:yoneda:20a} and \REF{ex:yoneda:20b} are both ambiguous in their interpretation.


\subsubsection{Applicative}\label{sec:yoneda:3.2.2}



The examples in \REF{ex:yoneda:21} show passive sentences with an applicative verb. The applied object (benefactive) is the subject in \REF{ex:yoneda:21a}, and the theme is the subject in \REF{ex:yoneda:21b}. \REF{ex:yoneda:21c} is the corresponding active sentence.



\ea%21
    \label{ex:yoneda:21}
    [AO (\textsc{ben}): friends (human), BO (\textsc{theme}): banana (entity)]  AO > \textsc{theme}
    \ea\label{ex:yoneda:21a}\gll Mi-kwánó   gyange   gi-fúmb-ír-íddwá             amá-tooke.\\
         4-friends     4.my    \textsc{sm4}-cook-\textsc{appl-pass.prf}    6-bananas\\
    \glt ‘My friends have been cooked bananas (by mother).’
    \ex\label{ex:yoneda:21b}\gll Amá-tooke   gá-fúmb-ír-íddwá             mi-kwánó   gyange.\\
     6-bananas    \textsc{sm6}-cook-\textsc{appl-pass.prf}    4-friends    4.my \\
     \glt ‘Bananas have been cooked for my friends (by mother).’
     \ex\label{ex:yoneda:21c}\gll Máama     a-fúmb-íddé            mi-kwánó   gyange   amá-tooke.\\
     1.mother    \textsc{sm1}-cook-\textsc{appl.prf}    4-friends    4.my    6-bananas\\
     \glt ‘Mother has cooked bananas for my friends.’
    \z
\z


    With applicative verbs, both objects can be the subject of a passivized sentence as was also shown with the ditransitive verbs above. Therefore, both objects behave symmetrically in terms of their ability to appear as the subject of a passive sentence when the benefactive is higher in animacy than the theme.


    The examples in \REF{ex:yoneda:22} are cases in which both objects are equal in animacy. 

\ea%22
    \label{ex:yoneda:22}
    [AO (\textsc{ben}): daughters (human), BO (\textsc{theme}): man (human)]  AO = \textsc{theme}\\
    \gll aba-walá     ba-kúb-ir-íddwá             omu-sájja.\\
         2-daughters  \textsc{sm}2-beat-\textsc{appl}-\textsc{pass}.\textsc{prf}  1-man\\
    \glt Interpretation 1: ‘Someone beat a man for daughters.’  \\
    Interpretation 2: ‘Someone beat daughters for a man.’
    \z

    

    

    \REF{ex:yoneda:22} can have both interpretations 1 and 2. \textit{Abawala} ‘daughters’ is the benefactive in Interpretation 1, and is the theme in Interpretation 2. That is, both benefactive and theme can be the subject of the passive. Therefore \REF{ex:yoneda:22} is ambiguous.


     \REF{ex:yoneda:23} is an example in which the theme is higher than the applied object in animacy. The semantic role of the applied object is reason. In this case, only the theme can be the passive subject as shown in \REF{ex:yoneda:23b}. \REF{ex:yoneda:23a} in which ‘party’ is the subject is grammatical but has a different meaning. 

\newpage
\ea%23
    \label{ex:yoneda:23}
    [AO (\textsc{reason}): party (entity), BO (\textsc{theme}): friends (human)]  AO < \textsc{theme}
    \ea[\#]{\label{ex:yoneda:23a}\gll Embága    e-jja        ku-yit-ir-w-a              mi-kwánó   gyaffe. \\
         9.party    \textsc{sm9-fut}    \textsc{inf}-call-\textsc{appl-pass-fv}  4-friends    4.our\\
    \glt ‘The party will be held for our friends.’}
    \ex[]{\label{ex:yoneda:23b}\gll Mi-kwano   gyaffe   gi-jja       ku-yit-ir-w-a             embaga. \\
    4-friends      4.our    \textsc{sm4-fut}  \textsc{inf}-call-\textsc{appl-pass-fv}  9.party\\
    \glt ‘Our friends will be called for a party.’
    }
    \ex[]{\label{ex:yoneda:23c}\gll Tú-jjá         ku-yít-ir-a           mi-kwánó   gyaffe   embága.\\
    \textsc{sm1pl-fut}    \textsc{inf}-call-\textsc{appl-fv}  4-friends    4.our    9.party\\
    \glt  ‘We will call our friends for a party.’}
    \z
\z
        

        


    Therefore, in applicatives, when the animacy of the applied object (benefactive) is equal to the theme or higher, objects behave symmetrically, while objects behave asymmetrically when the animacy of the applied object (a reason) is lower than the theme. However, cases like \REF{ex:yoneda:23} in which the semantic role of the applied object is a reason/motivation may need to be treated separately based on a number of other factors. 



\subsubsection{Causative}\label{sec:yoneda:3.2.3}



The examples in \REF{ex:yoneda:24} show passive sentences with a causative verb. The causee is the subject in \REF{ex:yoneda:24a} and the theme is the subject in \REF{ex:yoneda:24b}. 



\ea%24
    \label{ex:yoneda:24}
    {[\textsc{causee}: friends (human), \textsc{theme}: bananas (entity)]}  \textsc{causee} > \textsc{theme}
    \ea\label{ex:yoneda:24a}\gll Mi-kwánó   gyange  gí-fúmb-ísiddwa               amá-tooke.\\
         4-friends      4.my    \textsc{sm4}-cook-\textsc{caus.pass.prf}    6-bananas\\
    \glt ‘My friends have been caused to cook bananas.’ 
    \ex\label{ex:yoneda:24b} \gll Amá-tooke    gá-fúmb-ísíddwá              mi-kwano gyange.\\
    6-bananas    \textsc{sm6}-cook-\textsc{caus.pass.prf}    4-friends 4.my\\
    \glt ‘Bananas have been caused to be cooked by my friends.’
    \z
\z



    As shown, with causative verbs as well, both objects can be the subject of a passivized sentence. However, when the causee and theme are of equal animacy, only the causee and not the theme can be the passive subject, as shown in \REF{ex:yoneda:25}. 


\ea%25
    \label{ex:yoneda:25}
    [\textsc{causee}: girls (human), \textsc{theme}: boy (human)]  \textsc{causee} = \textsc{theme}\\
    \ea[]{\label{ex:yoneda:25a}\gll Aba-wála   ba-gób-és-éddwá               omu-lénzí   máama\\
         2-girls      \textsc{sm2}-chase-\textsc{caus-pass.prf}  1-boy     1.mother\\
    \glt ‘The girls were caused to chase away a boy by mother.’}
    \ex[\#]{\label{ex:yoneda:25b} \gll Omu-lénzí   a-gób-és-éddwá               aba-wála   máama. \\
    1-boy        \textsc{sm1}-chase-\textsc{caus-pass.prf}  2-girls     1.mother\\
    \glt Interpretation 1:   ‘The boy was caused to chase away the girls by mother.’\\
       *Interpretation 2:  ‘The boy was chased away by the girls caused by mother.’}
    \z
\z


      The same holds for cases in which the causee is lower than the theme in animacy, only the causee can be the subject of the passivized sentence, as shown in \REF{ex:yoneda:26}. 


\ea%26
    \label{ex:yoneda:26}
    [\textsc{causee}: stick (entity), \textsc{theme}: dog (animal)]  \textsc{causee} < \textsc{theme}
    \ea[]{\label{ex:yoneda:26a}\gll Omú-ggo    gwa-kúb-ísíddwá                   embwa  máamá.\\
         3-stick     \textsc{sm3.pst}-beat-\textsc{caus.pass.prf}    9.dog    1.mother\\
    \glt ‘A stick was used by mother to beat the dog.’
        (literally meaning: ‘A stick was caused to beat the dog by mother.’) }
    \ex[*]{\label{ex:yoneda:26b}\gll Embwá     ya-kúb-ísíddwá                   omú-ggo   máamá.\\
    9.dog      \textsc{sm9.pst}-beat-\textsc{caus.pass.prf}     3-stick      1.mother\\
    \glt (Intended meaning: ‘The dog was beaten with a stick by mother.’)}
    \z
\z



\subsubsection{Summary of properties of passivization}\label{sec:yoneda:3.2.4}



The data presented thus far can be summarized as in \tabref{tab:yoneda:2}. Concerning the ability of being the subject of a passivized sentence, objects are generally symmetrical with the exception of causative verbs. For causative verbs, the theme can be the subject of the passivized sentence only when its animacy is lower than the causee. The semantic role ``reason'' is again an exception here, and exhibits asymmetry. 


\begin{table}
\small
\begin{tabularx}{\textwidth}{QllQllQll}

\lsptoprule

Recipient > Theme & Rec. & Yes & Rec. = Theme & Rec. & Yes & Rec. < Theme & Rec. & {}--\\
& Theme & Yes &  & Theme & ? &  & Theme & {}--\\
\tablevspace
Benefactive > Theme & Ben. & Yes & Ben. = Theme & Ben. & Yes & Reason < Theme & reason & \cellcolor{lsLightGray}*\\
& Theme & Yes &  & Theme & Yes &  & Theme & Yes\\
\tablevspace
Causee > Theme & Causee & Yes & Cau. = Theme & Causee & Yes & Cau. < Theme & Causee & Yes\\
& Theme & Yes &  & Theme & \cellcolor{lsLightGray}* &  & Theme & \cellcolor{lsLightGray}*\\
\lspbottomrule
\end{tabularx}
\caption{Object ability to be the subject of a passivized sentence}
\label{tab:yoneda:2}
\end{table}

\subsection{Object marking}
\label{sec:yoneda:3.3}


The third syntactic property of the primary object in Bantu languages is pronominalization, namely, which object can be expressed by an OM within the verbal complex.



\subsubsection{Ditransitive}
\label{sec:yoneda:3.3.1}

Example \REF{ex:yoneda:27} shows a ditransitive verb. Both objects appear as noun phrases in \REF{ex:yoneda:27c}. The recipient \textit{emikwánó} ‘friends’ is pronominalized and appears as an OM in \REF{ex:yoneda:27b}, and the theme \textit{amátooke} ‘bananas’ is pronominalized and appears as an OM in \REF{ex:yoneda:27c}. As shown in \REF{ex:yoneda:27}, either object in ditransitive verbs can be pronominalized and expressed as an OM when the recipient is higher than the theme in animacy. 


\ea%27
    \label{ex:yoneda:27}
    [\textsc{recipient}: friends (human), \textsc{theme}: banana (entity)]  \textsc{recipient} > \textsc{theme}
    \ea\label{ex:yoneda:27a}\gll Máama     a-wáddé         emi-kwánó   amá-tooke.  \\
         1.mother    \textsc{sm1}-give.\textsc{prf}    4-friends      6-bananas\\
    \glt ‘Mother has given friends bananas.’
    \ex\label{ex:yoneda:27b} \gll Máama     a-ba-wádde             amá-tooke.\\
    1.mother    \textsc{sm1-om2}-give.\textsc{prf}    6-bananas\\
    \glt ‘Mother has given them (friends) bananas.’
    \ex\label{ex:yoneda:27c}\gll Máama     a-ga-wádde           emi-kwánó.\\
    1.mother    \textsc{sm1-om6}-give.\textsc{prf}  4-friends \\
    \glt ‘Mother has given them (bananas) to friends.’
    \z
\z



Both objects can appear as OMs in an utterance as in \REF{ex:yoneda:28}. In \REF{ex:yoneda:28a}, the recipient OM is placed immediately before the stem (IBS, hereafter). This is the natural order, and the other order in \REF{ex:yoneda:28b} is odd, although not completely ungrammatical (as is marked by “??”).  



\ea%28
    \label{ex:yoneda:28}
    \ea[]{\label{ex:yoneda:28a} \gll Máama     a-ga-bá-wadde.\\
         1.mother    \textsc{sm1-om6-om2}-give.\textsc{prf}\\}
    \ex[??]{\label{ex:yoneda:28b} \gll Máama     a-ba-gá-wadde.\\
    1.mother    \textsc{sm1-om2-om6}-give.\textsc{prf}\\
    \glt ‘Mother has given them (bananas) to them (friends).’}
    \z
\z

    According to \citet[599]{Ranero2019}, the OM that agrees with the theme must precede the one that agrees with the recipient or benefactive. Our data also show that the other order is odd, but it is still not ungrammatical.\footnote{\REF{ex:yoneda:28b} is odd but not ungrammatical. For example, the second author's grandmother used such utterances. However, it is not common and is only marginally acceptable today it seems.}

    This preference for the order of the OMs can be seen more clearly when the OM agrees with a 1\textsuperscript{st} person singular subject as shown in \REF{ex:yoneda:29}. \REF{ex:yoneda:29a}, in which the 1SG OM referring to the recipient appears IBS, is fine, while \REF{ex:yoneda:29b}, in which the theme OM appears IBS, is ungrammatical.


\ea%29
    \label{ex:yoneda:29}
    \ea[]{\label{ex:yoneda:29a}\gll Máama     a-gá-n-wadde.       ( > agampadde)\\
         1.mother    \textsc{sm1-om6-om1sg}-give.\textsc{prf}\\}
    \ex[*]{\label{ex:yoneda:29b}\gll Maama     a-n-ga-wadde\\
    1.mother    \textsc{sm1-om1sg-om6}-give.\textsc{prf}\\
    \glt ‘Mother has given them (bananas) to me.’}
    \z
\z


    In fact, not all Ganda speakers accept \REF{ex:yoneda:28b}. Even then, \REF{ex:yoneda:28b} is still not as bad as \REF{ex:yoneda:29b} for these speakers. This suggests that there is a subtle but clear difference in the acceptability between \REF{ex:yoneda:28b} and \REF{ex:yoneda:29b}. We will return to this issue later in \sectref{sec:yoneda:4.3}.


    The examples in \REF{ex:yoneda:30} show cases in which both objects are equal in animacy. 

\ea%30
    \label{ex:yoneda:30}
    [\textsc{recipient}: daughters (human), \textsc{theme}: hunter (human)]  \textsc{recipient} = \textsc{theme}
    \ea[]{\label{ex:yoneda:30a}\gll Kabáka   ya-ba-wá                  omu-yízzi. \\
         1.king    \textsc{sm1.pst-om2}-give.\textsc{fv}    1-hunter\\
    \glt Interpretation 1:   ‘The king gave the hunter to them (his daughters).’\\
       ?Interpretation 2:  ‘The king gave them (his daughters) to the hunter.’}
    \ex[]{\label{ex:yoneda:30b} \gll Kabáka   ya-mu-wa                 ba-wala       be.\\
    1.king    \textsc{sm1.pst-om1}-give.\textsc{fv}    2-daughters  his\\
    \glt ?Interpretation 1:  ‘The king gave him (the hunter) to his daughters.’\\
         Interpretation 2:   ‘The king gave his daughters to him (the hunter).’}
    \ex[]{\label{ex:yoneda:30c}\gll Kabáka   ya-mu-ba-wá.\\
    1.king    \textsc{sm1.prf-om1-om2}-give.\textsc{fv} \\
    \glt ‘The king gave him (the hunter) to them (his daughters).’}
    \ex[\#]{\label{ex:yoneda:30d}\gll Kabáka   ya-ba-mu-wá.\\
    1.king    \textsc{sm1.prf-om2-om1}-give.\textsc{fv}\\
    \glt Interpretation 1:   ‘The king gave them (his daughters) to him (the hunter).’\\
       *Interpretation 2:  ‘The king gave him (the hunter) to them (his daughters).’
    }
    \z
\z


    Both objects can be pronominalized and appear as OM as shown in \REF{ex:yoneda:30a} and \REF{ex:yoneda:30b}. Data from \citet{Ssekiryango2006} and \citet{Ranero2019} show that only the recipient can be pronominalized when one of the objects is pronominalized. Our data also show that pronominalization of a recipient is more natural. However, the pronominalization of a theme is also possible, and hence there is ambiguity. When both objects appear as OMs simultaneously, the OM that agrees with the recipient must appear closer to the verb stem, as shown in \REF{ex:yoneda:30c} and \REF{ex:yoneda:30d}. This agrees with the data from \citet{Ssekiryango2006} and \citet{Ranero2019}.  


\subsubsection{Applicative}
\label{sec:yoneda:3.3.2}

\REF{ex:yoneda:3a} is an example of applicative verb. Both objects appear as noun phrases in \REF{ex:yoneda:31a}, the benefactive \textit{mikwánó gyange} ‘my friends’ appears as an OM in \REF{ex:yoneda:31b}, and the theme \textit{amátóóké} ‘bananas’ appears as an OM in \REF{ex:yoneda:31c}. Either object can appear as an OM when the benefactive is higher than the theme in animacy, as is also the case with ditransitive verbs.  



\ea%31
    \label{ex:yoneda:31}
    [AO (\textsc{ben}): friends (human), BO (\textsc{theme}): banana (entity)]   AO > \textsc{theme}
    \ea\label{ex:yoneda:31a}\gll Máama     a-fúmb-íddé             mi-kwánó   gyange   amá-tóóké.\\
         1.mother    \textsc{sm1}-cook-\textsc{appl.prf}    4-friends    4.my    6-bananas\\
    \glt ‘Mother has cooked bananas for my friend.’
    \ex\label{ex:yoneda:31b}\gll Máama     ya-ba-fúmb-idde             amá-tóóké.\\
    1mother   \textsc{sm1-om2}-cook-\textsc{appl.prf}  6-bananas\\
    \glt ‘Mother has cooked bananas for them.’
    \ex\label{ex:yoneda:31c}\gll Máama     ya-ga-fúmb-íddé             mi-kwánó   gyange.\\
    1.mother    \textsc{sm1-om6}-cook-\textsc{appl.prf}  4-friends    4.my\\
    \glt ‘Mother has cooked it for my friends.’
    \z
\z
        



It is also possible that both objects appear as OMs at the same time, as shown in \REF{ex:yoneda:32}. Placing the one that agrees with the benefactive (or that of higher animacy) IBS is much more natural than the other order. \REF{ex:yoneda:32b}, in which the theme appears as an OM, is very unnatural, although it is not completely ungrammatical.



\ea%32
    \label{ex:yoneda:32}
    \ea[]{\label{ex:yoneda:32a}\gll Máama     a-ga-bá-fumb-idde.\\
         1.mother    \textsc{sm1-om6-om2}-cook-\textsc{appl.prf}\\}
    \ex[??]{\label{ex:yoneda:32b}\gll Máama     a-ba-gá-fúmb-idde.\\
    1.mother    \textsc{sm1-om2-om6}-cook-\textsc{appl.prf} \\
    \glt ‘Mother has cooked it for them.’
    }
    \z
\z


    The examples in \REF{ex:yoneda:33} show cases in which both objects are equal in animacy. The benefactive (AO) \textit{bawalá be} ‘her daughters’ appears as an OM in \REF{ex:yoneda:33a}, the theme (BO) \textit{omusájja} ‘man’ appears as an OM in \REF{ex:yoneda:33b}, and both objects appear as OMs in \REF{ex:yoneda:33c} and \REF{ex:yoneda:33d}.

\ea%33
    \label{ex:yoneda:33}
    [AO (\textsc{ben}): daughters (human), BO (\textsc{theme}): man (human)]  AO = \textsc{theme} 
    \ea[]{\label{ex:yoneda:33a}\gll Máama     a-ba-kúb-íddé               omu-sájja.\\
         1.mother    \textsc{sm1-om2}-beat-\textsc{appl.prf}  1-man\\
    \glt ‘Mother has beaten a man for/on behalf of them (her daughters).’}
    \ex[\#]{\label{ex:yoneda:33b}\gll Máama     a-mu-kúb-íddé               ba-walá       bwe.\\
    1.mother    \textsc{sm1-om1}-beat-\textsc{appl.prf}  2-daughters  2.her\\
    \glt ‘Mother has beaten her daughters for/on behalf of him (a man).’
    }
    \ex[]{\label{ex:yoneda:33c}\gll Máama     a-mu-ba-kúb-íddé.\\
    1.mother    \textsc{sm1-om1-om2}-beat-\textsc{appl.prf}\\
    \glt ‘Mother has beaten him (a man) for/on behalf of them (her daughters).’ 
    }
    \ex[\#]{\label{ex:yoneda:33d}\gll Máama     a-ba-mu-kúb-íddé.\\
    1.mother    \textsc{sm1-om2-om1}-beat-\textsc{appl.prf}\\
    \glt ‘Mother has beaten them (her daughters) for/on behalf of him (a man).’
    }
    \z
\z



    It is only the benefactive that can be pronominalized and appear as an OM when both objects are equal in animacy, as shown in \REF{ex:yoneda:33a} and \REF{ex:yoneda:33b}. Also when both objects appear as OMs, the one that agrees with the benefactive must appear IBS, as shown in \REF{ex:yoneda:33c}.     


    \REF{ex:yoneda:34} is an example in which the animacy of the theme \textit{mikwánó gyaffe} ‘our friends’ is higher than that of the applied object \textit{embága} ‘party’. Both objects can appear as OMs, just like in \REF{ex:yoneda:31}, where the animacy relation of the two objects is the other way around. 

\newpage
\ea%34
    \label{ex:yoneda:34}
    [AO (\textsc{reason}): party (entity), BO (\textsc{theme}): friends (human)]  AO < \textsc{theme}
    \ea\label{ex:yoneda:34a}\gll Tú-jjá         ku-yít-ir-a           mi-kwánó   gyaffe   embága.\\
         \textsc{sm1pl-fut}    \textsc{inf}-call-\textsc{appl-fv} 4-friends    4.our    9.party\\
    \glt ‘We will call our friends for a party.’
    \ex\label{ex:yoneda:34b}\gll Tú-jjá         ku-gi-yít-ir-a             mi-kwánó   gyaffe.\\
    \textsc{sm1pl-fut}    \textsc{inf-om9}-call-\textsc{appl-fv}    4-friends    4.our\\
    \glt ‘We will call our friends for it (a party).’
    \ex\label{ex:yoneda:34c}\gll Tú-jjá         ku-ba-yít-ir-a             embága.\\
    \textsc{sm1pl-fut}    \textsc{inf-om2}-call-\textsc{appl-fv}    9.party\\
    \glt ‘We will call them (our friends) for a party.’ 
    \z
\z
        

    However, when both objects appear as OMs, placing the OM that agrees with the theme IBS is much more natural than the other order, as shown in \REF{ex:yoneda:35b}.  

\ea%35
    \label{ex:yoneda:35}
    \ea[??]{\label{ex:yoneda:35a}\gll Tú-jjá         ku-ba-gi-yít-ir-a.\\
         \textsc{sm1pl-fut}    \textsc{inf-om2-om9}-call-\textsc{appl-fv}\\}
    \ex[]{\label{ex:yoneda:35b}\gll Tú-jjá         ku-gi-ba-yít-ir-a. \\
    \textsc{sm1pl-fut}    \textsc{inf-om9-om2}-call-\textsc{appl-fv}\\
    \glt ‘We will call them (our friends) for it (a party).’}
    \z
\z


    Therefore, both the symmetrical behaviour (with respect to pronominalization \REF{ex:yoneda:34}) and asymmetrical behaviour (with respect to the order of OMs \REF{ex:yoneda:35}) are observed here.   


\subsubsection{Causative}
\label{sec:yoneda:3.3.3}

\REF{ex:yoneda:36} is an example of a causative verb. Both objects appear as noun phrases in \REF{ex:yoneda:36a}, the causee \textit{mikwánó gyange} ‘my friends’ appears as an OM in \REF{ex:yoneda:36b}, and the theme \textit{amátóóké} ‘bananas’ appears as an OM in \REF{ex:yoneda:36c}. In the case of a causative verb, either of the objects can be pronominalized and appear as an OM, as was also the case with ditransitive and applicative verbs. 



\ea%36
    \label{ex:yoneda:36}
    [\textsc{causee}: friends (human), \textsc{theme}: bananas (entity)]  \textsc{causee} > \textsc{theme}
    \ea\label{ex:yoneda:36a}\gll Máama     a-fúmb-ísízzá           mi-kwánó   gyange   amá-tooke.\\
         1.mother    \textsc{sm1}-cook-\textsc{caus.prf}    4-friends    4.my    6-bananas\\
    \glt ‘Mother has made my friends cook bananas.’
    \ex\label{ex:yoneda:36b}\gll Máama     a-ba-fúmb-ísízzá               amá-tooke.\\
    1.mother    \textsc{sm1-om2}-cook-\textsc{caus.prf}    6-bananas\\
    \glt ‘Mother has made them (my friends) cook bananas.’
    \ex\label{ex:yoneda:36c} \gll Máama     a-ga-fúmbísízzá         mi-kwánó   gyange.\\
    1.mother    \textsc{sm1-om6-caus.prf}  4-friends    4.my\\
    \glt ‘Mother has made my friends cook them (bananas).’
    \z
\z
          


    In the examples in \REF{ex:yoneda:37}, in which both objects appear as OMs, the OM that agrees with the causee \textit{mikwánó gyange} ‘my friends’ is placed IBS in \REF{ex:yoneda:37a}, and the opposite order is shown in \REF{ex:yoneda:37b}. The former order is acceptable, but not the latter. \REF{ex:yoneda:37b} is not ungrammatical, but is very odd.  



\ea%37
    \label{ex:yoneda:37}
    \ea[]{\label{ex:yoneda:37a}\gll  Máama    a-ga-bá-fúmb-ísízzá.\\
         1.mother   \textsc{sm1-om6-om2}-cook-\textsc{caus.prf}\\
    \glt ‘Mother has caused them to cook it.’}
    \ex[??]{\label{ex:yoneda:37b}\gll Máama    a-ba-gá-fúmb-ísízzá.\\
    1.mother   \textsc{sm1-om2-om6}-cook-\textsc{caus.prf}\\
    \glt ‘Mother has caused them to cook it.’}
    \z
\z
 

When the causee is higher than the theme in animacy, either object can be pronominalized as we have seen in \REF{ex:yoneda:37}; however, when both objects are of equal animacy, only the causee can be pronominalized and expressed as the OM, as shown in \REF{ex:yoneda:38}. 

\ea%38
    \label{ex:yoneda:38}
    [\textsc{causee}: girls (human), \textsc{theme}: boy (human)]  \textsc{causee} = \textsc{theme} 
    \ea[]{\label{ex:yoneda:38a}\gll Máama     a-gob-ésézzá           aba-wála   omu-lénzi.\\
         1.mother    \textsc{sm1}-chase-\textsc{caus.prf}  2-girls      1-boy\\
    \glt ‘Mother has made the girls chase away the boy.’}
    \ex[]{\label{ex:yoneda:38b}\gll Máama     a-ba-gób-ésézzá               omu-lénzi.\\
    1.mother    \textsc{sm1-om2}-chase-\textsc{caus.prf}    1-boy\\
    \glt ‘Mother has made them (the girls) chase away the boy.’}
    \ex[\#]{\label{ex:yoneda:38c}\gll Máama     a-mú-gób-ésézzá               aba-wála.\\
    1.mother    \textsc{sm1-om1}-chase-\textsc{caus.prf}    2-girls\\
    \glt ‘Mother has made him (the boy) chase away the girls.’}
    \z
\z
                 

  In \REF{ex:yoneda:38}, the causee \textit{abawála} ‘the girls’ and the theme \textit{omulénzi} ‘boy’ are at the same level of animacy. In this case, only the causee can appear as an OM. Therefore, in \REF{ex:yoneda:38c}, in which \textit{omulénzi} ‘the boy’ appears as an OM, \textit{abawála} ‘the girls’ cannot be interpreted as the causee. Here we can see some clear asymmetrical characteristics determined by the semantic role.


     Both objects can appear as OMs at the same time in the order shown in \REF{ex:yoneda:39}, which is also the case when the two objects are of equal animacy. In this case too, the OM that agrees with the causee must be placed IBS. 



\ea%39
    \label{ex:yoneda:39}
    \ea[]{\label{ex:yoneda:39a}\gll Máama     a-mú-ba-gób-ésézzá.\\
         1.mother    \textsc{sm1-om1-om2}-chase-\textsc{caus.prf}\\
    \glt  ‘Mother has made them (the girls) chase him (the boy) away.’}
    \ex[\#]{\label{ex:yoneda:39b}\gll Máama     a-bá-mu-gób-ésézzá.\\
    1.mother    \textsc{sm1-om2-om1}-chase-\textsc{caus.prf}\\
    \glt  ‘Mother has made him (the boy) chase them (the girls) away.’}
    \z
\z


\REF{ex:yoneda:40} and \REF{ex:yoneda:41} are examples in which the causee is lower than the theme in animacy. The causee is pronominalized in \REF{ex:yoneda:40a} and the theme is pronominalized in \REF{ex:yoneda:40b}. The behaviour is thus symmetrical and both objects can be pronominalized as shown in \REF{ex:yoneda:40a} and \REF{ex:yoneda:40b}. However, when both objects are pronominalized and appear as OMs at the same time, their order is asymmetrical. The OM that is placed IBS is the one that corresponds to the causee, and the other order is odd as shown in \REF{ex:yoneda:41}. 

\ea%40
    \label{ex:yoneda:40}
    [\textsc{causee}: stick (entity), \textsc{theme}: dog (animal)]  \textsc{causee} < \textsc{theme}
    \ea\label{ex:yoneda:40a}\gll Máama    a-gu-kúb-íssá                       embwa. \\
         1.mother    \textsc{sm1.pst-om3}-beat-\textsc{caus.prf}    9.dog \\
    \glt ‘Mother beat a dog with it (stick).’ 
        (literally meaning: ‘Mother caused it (stick) to beat dog.’)  
    \ex\label{ex:yoneda:40b}\gll Máama     a-gi-kúb-íssá                     omú-ggo.\\
    1.mother    \textsc{sm1.pst-om9}-beat-\textsc{caus.prf}     3-stick\\
    \glt ‘Mother beat it (dog) with a stick.’
    \z
\z
        

\ea%41
    \label{ex:yoneda:41}
    \ea[]{\label{ex:yoneda:41a}\gll Máama    a-gi-gu-kúb-íssá.\\
         1.mother    \textsc{sm1.pst-om9-om3}-beat-\textsc{caus.prf}\\
    }
    \ex[??]{\label{ex:yoneda:41b}\gll Máama    a-gu-gi-kúb-íssá.\\
    1.mother    \textsc{sm1.pst-om3-om9}-beat-\textsc{caus.prf}\\
    \glt ‘Mother beat it (a dog) with it (a stick).’}
    \z
\z



    Likewise with causative verbs, both the symmetrical and asymmetrical behaviours are observed. 



\subsubsection{Summary of object marking}
\label{sec:yoneda:3.3.4}


The data presented in this section regarding pronominalization are summarized in \tabref{tab:yoneda:3}. \tabref{tab:yoneda:4} summarizes the facts regarding the possible ordering of OMs when both objects appear at the same time. As \tabref{tab:yoneda:3} shows, the language is symmetrical except when both objects are of equal animacy. On the other hand, the order of OMs is clearly asymmetrical as \tabref{tab:yoneda:4} shows.  


\begin{table}
\small
\begin{tabularx}{\textwidth}{QllQllQll}

\lsptoprule

Recipient > Theme & Rec. & Yes & Rec. = Theme & Rec. & Yes & Rec. < Theme & Rec. & {}--\\
& Theme & Yes &  & Theme & ? &  & Theme & {}--\\
\tablevspace
Benefactive > Theme & Ben. & Yes & Ben. = Theme & Ben. & Yes & Reason < Theme & reason & Yes\\
& Theme & Yes &  & Theme & \cellcolor{lsLightGray}* &  & Theme & Yes\\
\tablevspace
Causee > Theme & Causee & Yes & Cau. = Theme & Causee & Yes & Cau. < Theme & Causee & Yes\\
& Theme & Yes &  & Theme & \cellcolor{lsLightGray}* &  & Theme & Yes\\
\lspbottomrule
\end{tabularx}
\caption{Pronominalization (appearing as an OM)}
\label{tab:yoneda:3}
\end{table}

\begin{table}
\small
\begin{tabularx}{\textwidth}{QllQllQll}

\lsptoprule

Recipient > Theme & Rec. & Yes & Rec. = Theme & Rec. & Yes & Rec. < Theme & Rec. & {}--\\
& Theme & \cellcolor{lsLightGray}?? &  & Theme & \cellcolor{lsLightGray}* &  & Theme & {}--\\
\tablevspace
Benefactive > Theme & Ben. & Yes & Ben. = Theme & Ben. & Yes & Reason < Theme & reason & \cellcolor{lsLightGray}??\\
& Theme & \cellcolor{lsLightGray}?? &  & Theme & \cellcolor{lsLightGray}* &  & Theme & Yes\\
\tablevspace
Causee > Theme & Causee & Yes & Cau. = Theme & Causee & Yes & Cau. < Theme & Causee & Yes\\
& Theme & \cellcolor{lsLightGray}?? &  & Theme & \cellcolor{lsLightGray}* &  & Theme & \cellcolor{lsLightGray}??\\
\lspbottomrule
\end{tabularx}
 \caption{Ability of the OM to be placed immediately before the stem. *: not acceptable,  ?: less unnatural but acceptable,   ??: very odd but not ungrammatical}
\label{tab:yoneda:4}
\end{table}

Compared to the other two properties (object order and passivization), pronominalization seems to most clearly highlight the asymmetry in the language especially with respect to the order of OMs. \citet{Ssekiryango2006} and \citet{Ranero2019} also report that although both objects can be pronominalized symmetrically, the order of OMs is rigidly fixed. Here again, the behaviour of the ``reason'' semantic role is an exception to the rule. 



\subsection{Findings and summary of double-object constructions in Ganda} 
\label{sec:yoneda:3.4}


In double-object constructions in Ganda, both objects can be (i) placed IAV, (ii) the subject of a passive sentence, and (iii) pronominalized. These facts show that Ganda is a symmetrical object language (\citealt{BresnanMoshi1993}). They show that even in cases where the interpretation of the semantic roles of the two objects becomes ambiguous, the grammar allows both objects to equally assume the primary object position. At the same time, however, some asymmetrical characteristics are also observed, such as the preference for the primary object to be a non-theme (recipient, benefactive or causee). Another noticeable asymmetrical feature is the order of OMs. These asymmetrical characteristics seem particularly prominent with causative verbs. The order of OMs is not included in the main syntactic properties of the primary object in Bantu languages shown earlier in \REF{ex:yoneda:4}. However, it is still an important characteristic observed in Ganda. \citet[216--217]{vanderWal2020} points out that although the ordering of OMs does not necessarily follow the thematic roles in other Bantu languages with multiple object markers, the order of OMs is determined by their semantic role in Ganda.    



    When the semantic role of the applied object is not the benefactive but the reason, it behaves differently from other cases. It is not clear at this point if this difference is due to the animacy hierarchy or the semantic role of the ``benefactive'' and ``reason''. This deserves further investigation.     Another important finding is the restriction on the appearance of the OM that agrees with the 1\textsuperscript{st} person singular. We will discuss this in \sectref{sec:yoneda:4.3} below.


\begin{table}
\begin{tabular}{>{\scshape}llcccc}
\lsptoprule
                               & Semantic role & \\
{\normalfont Relative animacy} &  of objects & IAV & Passive & OM & IBS\\\midrule
recipient > theme     & Recipient & Yes & Yes & Yes & Yes\\
                    & Theme & Yes & Yes & Yes & \cellcolor{lsLightGray}??\\
benefactive > theme   & Benefactive & Yes & Yes & Yes & Yes\\
                    & Theme & Yes & Yes & Yes & \cellcolor{lsLightGray}??\\
causee > theme        & Causee & Yes & Yes & Yes & Yes\\
                    & Theme & ? & Yes & Yes & \cellcolor{lsLightGray}??\\
recipient = theme     & Recipient & Yes & Yes & Yes & Yes\\
                    & Theme & ? & ? & ? & \cellcolor{lsLightGray}*\\
benefactive = theme   & Benefactive & Yes & Yes & Yes & Yes\\
                    & Theme & ? & Yes & \cellcolor{lsLightGray}* & \cellcolor{lsLightGray}*\\
causee = theme        & Causee & Yes & Yes & Yes & Yes\\
                    & Theme & \cellcolor{lsLightGray}* & \cellcolor{lsLightGray}* & \cellcolor{lsLightGray}* & \cellcolor{lsLightGray}*\\
reason < theme        & Reason & ? & \cellcolor{lsLightGray}* & Yes & \cellcolor{lsLightGray}??\\
                    & Theme & Yes & Yes & Yes & Yes\\
causee < theme        & Causee & Yes & Yes & Yes & Yes\\
                    & Theme & Yes & \cellcolor{lsLightGray}* & Yes & \cellcolor{lsLightGray}??\\
\lspbottomrule
\end{tabular}
\caption{Symmetrical/asymmetrical nature of double objects}
\label{tab:yoneda:5}
\end{table}

\section{Triple-object constructions}
\label{sec:yoneda:4}

Verbs in Ganda do not allow three object NPs; however, triple-object constructions are possible, albeit restricted. The conditions of triple-objects in Ganda are that (i) they appear with the applicative forms of ditransitive verbs, such as -\textit{wa} ‘give’, -\textit{soba} ‘ask’, and -\textit{gamba} ‘tell’, and (ii) the applied object (benefactive) has to be indicated by an OM. Therefore, the semantic roles of the objects in the triple-object constructions must be the recipient, theme, and benefactive. The different ways in which these can therefore be expressed are as follows:

\ea%40
    \label{ex:yoneda:42}
    \ea\label{ex:yoneda:42a} with an OM (benefactive) + two object NPs (recipient and theme)
    \ex\label{ex:yoneda:42b} with two OMs (benefactive, and recipient or theme) + an object NP (recipient or theme) 
    \ex\label{ex:yoneda:42c} with three OMs (benefactive, recipient, and theme)
    \z
\z


    In this section, we will show the possible orders of the object NPs as schematized in \REF{ex:yoneda:42a}, and the possible orders of OMs as in \REF{ex:yoneda:42b} and \REF{ex:yoneda:42c}.


\subsection{Order of object NPs}
\label{sec:yoneda:4.1}


As mentioned above, placing three NPs following the verb is not allowed \REF{ex:yoneda:43a} and the applied object has to be indicated by an OM in triple-object constructions in Ganda \REF{ex:yoneda:43b}. Alternatively, the benefactive has to appear with a preposition as shown in \REF{ex:yoneda:43c}, in which case it is no longer a triple-object construction.



\ea%41
    \label{ex:yoneda:43}
    \ea[*]{\label{ex:yoneda:43a}\gll N-gámb-idde            máamá     emi-kwano   ama-wulire.\\
         \textsc{sm1sg}-tell-\textsc{appl.prf}  1.mother   4-friends      6-news\\
    \glt (Intended meaning: ‘I have told the news to friends for/on behalf of my mother.’)}
    \ex[]{\label{ex:yoneda:43b}\gll N-mu-gámb-idde              emi-kwáno   ama-wúlire.\\
    \textsc{sm1sg-om1}-tell-\textsc{appl.prf}    4-friends      6-news\\
    \glt ‘I have told the news to friends for/on behalf of her (mother).’ 
    }
    \ex[]{\label{ex:yoneda:43c}\gll N-gámbye          emi-kwáno   ama-wúlire   kulwá   máama.\\
    \textsc{sm1sg}-tell.\textsc{prf}    4-friends      6-news      for      1.mother \\
    \glt ‘I have told friends the news for/on behalf of my mother.’}
    \z
\z


    Since the applied object must be pronominalized and expressed as the OM, the question relates to the order of the other objects, namely, the order of theme object NP and the recipient object NP. The examples in \REF{ex:yoneda:44} show that these two object NPs can appear in either order. 

\ea%42
    \label{ex:yoneda:44}
    \ea\label{ex:yoneda:44a}\gll N-mu-gámb-idde              emi-káno   ama-wúlire.\\
         \textsc{sm1sg-om1}-tell-\textsc{appl.prf}    4-friends    6-news \\
    \ex\label{ex:yoneda:44b}\gll N-mu-gámb-idde                 ama-wúlire   emi-káno.\\
    \textsc{sm1sg-om1}-tell-\textsc{appl.prf}  6-news       4-friends \\
    \glt ‘I have told the news to friends for her (mother).’
    \z
\z


\subsection{The order of OMs} 
\label{sec:yoneda:4.2}


Other possible forms of triple-object constructions are two OMs + one NP \REF{ex:yoneda:42b} and three OMs \REF{ex:yoneda:42c}. Here, we will show the ordering of the OMs.


\subsubsection{Two OMs + one NP}
\label{sec:yoneda:4.2.1}
As mentioned above, the applied object (benefactive) must appear as an OM; therefore, one of the object NPs must be either a recipient or theme. The possible combinations of OMs are \{recipient OM+ benefactive OM\} and \{theme OM+ benefactive OM\}, as exemplified in \REF{ex:yoneda:45} and \REF{ex:yoneda:46} respectively. The benefactive OM is at IBS in \REF{ex:yoneda:45a} and \REF{ex:yoneda:46a}, and the other OM is at IBS in \REF{ex:yoneda:45b} and \REF{ex:yoneda:46b}. As \REF{ex:yoneda:45} and \REF{ex:yoneda:46} show, the order is determined by the semantic roles. The OM that refers to the applied object appears IBS. The other order is very odd, but not ungrammatical. 

\ea%43
    \label{ex:yoneda:45}
    \ea[]{\label{ex:yoneda:45a}\gll A-ba-mú-gámb-idde             ama-úlire.\\
         \textsc{sm1-om2-om1}-tell-\textsc{appl.prf}  6-news\\
    \glt ‘He has told them (friends) the news for her (mother).’}
    \ex[??]{\label{ex:yoneda:45b}\gll A-mu-bá-gámb-idde             ama-úlire.\\
    \textsc{sm1-om1-om2}-tell-\textsc{appl.prf}  6-news \\}
    \z
\z
                 

\ea%44
    \label{ex:yoneda:46}
    \ea[]{\label{ex:yoneda:46a}\gll A-ga-mú-gámb-idde               emi-káno.\\
         \textsc{sm1-om6-om1}-tell-\textsc{appl.prf}    4-friends \\
    \glt ‘He has told it (the news) to friends for her (mother).’}
    \ex[??]{\label{ex:yoneda:46b}\gll A-mu-gá-gámb-idde               emi-káno\\
    \textsc{sm1-om1-om6}-tell-\textsc{appl.prf}    4-friends \\}
    \z
\z

         

    When the OM is 1\textsuperscript{st} person singular, it must be placed IBS, and any other order is ungrammatical. In \REF{ex:yoneda:47}, the applied object is 1\textsuperscript{st} person singular, and its OM must appear IBS. The other order is ungrammatical as shown in \REF{ex:yoneda:47b}. Compare this with \REF{ex:yoneda:45b} or \REF{ex:yoneda:46b}, which are very unnatural but not ungrammatical.  

\ea%45
    \label{ex:yoneda:47}
    \ea[]{\label{ex:yoneda:47a}\gll A-bá-n-gámb-idde                   ama-wúlire.\\
          \textsc{sm1-om2-om1sg}-tell-\textsc{appl.prf}    6-news \\
    \glt Interpretation 1:   ‘He has told them the news for me.’\\
   *Interpretation 2:  ‘He has told me the news for them’}
   \ex[*]{\label{ex:yoneda:47b}\gll A-n-bá-gámb-iddé                   ama-wúlire.\\
   \textsc{sm1-om1sg-om2}-tell-\textsc{appl.prf}    6-news\\}
    \z
\z

      If the OM of 1\textsuperscript{st} person singular is placed IBS, it is only interpreted as the benefactive, never as the recipient, as shown in \REF{ex:yoneda:47a}.  If the benefactive OM of 3\textsuperscript{rd} person plural (class 2) is IBS, it is ungrammatical as shown in \REF{ex:yoneda:47b}.

    The benefactive OM must appear IBS; however, the OM of 1\textsuperscript{st} person singular must appear IBS as well. There is a conflict, then, when the applied object is not 1\textsuperscript{st} person singular. In the case when the 1\textsuperscript{st} person singular OM does not refer to the benefactive, the applicative verb form cannot be used, and the benefactive must be expressed in a prepositional phrase as shown in (\ref{ex:yoneda:48}a,b). In this case, the position IBS is not for the benefactive but for the 1\textsuperscript{st} person singular. Therefore, restriction of 1\textsuperscript{st} person singular OM has priority over the benefactive restriction for IBS. 

\ea%46
    \label{ex:yoneda:48}
    \ea[]{\label{ex:yoneda:48a}\gll A-ga-n-gámbye               kulwábwe.\\
         \textsc{sm1-om6-om1sg}-tell.\textsc{prf}    for.them \\
    \glt ‘He has told it (the news) to me for them’}
    \ex[]{\label{ex:yoneda:48b}\gll A-n-gámbye               ama-wúlire   kulwábwe.\\
    \textsc{sm1-om1sg}-tell.\textsc{prf}    6-news      for.them\\
    \glt ‘He has told me the news for them’}
    \ex[*]{\label{ex:yoneda:48c}\gll A-n-ba-gámbye               ama-wúlire.\\
    \textsc{sm1-om1sg-om2}-tell.\textsc{prf}    6-news\\
    \glt (Intended meaning: ‘He has told me the news for them’)}
    \z
\z

The place restriction of the 1\textsuperscript{st} person singular OM is very strict, and seems to be more than just a ``tendency'' or ``preference'' which was observed for the other hierarchies. We will further discuss this in \sectref{sec:yoneda:4.3}. 

\subsubsection{Three OMs}  
\label{sec:yoneda:4.2.2}

Ganda is considered a language that allows two OMs (\citealt{Ssekiryango2006, Marlo2015}). However, according to our observation, in fact, up to three OMs are possible. Expressions such as the following are therefore widespread.


\ea%47
    \label{ex:yoneda:49}
    \ea[]{\label{ex:yoneda:49a}\gll O-ki-bá-n-gámb-idde.\\
         \textsc{sm2sg-om7-om3pl-om1sg}-tell-\textsc{appl.prf}\\
    \glt ‘You have told it to them for me.’}
    \ex[]{\label{ex:yoneda:49b}\gll O-ki-ba-tú-gámb-idde.\\
    \textsc{sm2sg-om7-om3pl-om1pl}-tell-\textsc{appl.prf}\\
    \glt ‘You have told it to them for us.’}
    
    \z
\z

    All three objects, namely the benefactive, the recipient, and the theme, can be expressed by OMs at the same time. Here, we show the possible orders of these three OMs.


The applied (benefactive) OM must appear IBS, as with the double-object construction (\sectref{sec:yoneda:3.3.2}). The examples in \REF{ex:yoneda:50} show that the 2\textsuperscript{nd} person singular benefactive OM \textit{kú-} is necessarily placed IBS. As long as this condition is met, the order of both theme and recipient OMs is interchangeable, as shown in \REF{ex:yoneda:50a} and \REF{ex:yoneda:50b}. 


\ea%48
    \label{ex:yoneda:50}
    \ea\label{ex:yoneda:50a}\textsc{theme}-\textsc{rec-ben}\\
    \gll N-ki-ba-kú-gámb-idde.\\
         \textsc{sm1sg-om7-om3pl-om2sg}-tell-\textsc{appl.prf}\\
    \ex\label{ex:yoneda:50b} \textsc{rec}-\textsc{theme}-\textsc{ben}\\
    \gll N-ba-kí-kú-gámb-idde.\\
    \textsc{sm1sg-om3pl-om7-om2sg}-tell-\textsc{appl.prf}\\
    \glt ‘I have told it to them for/on behalf of you(sg).’
    \z
\z


\REF{ex:yoneda:51} exemplifies a situation in which the 3\textsuperscript{rd} person plural OM \textit{bá-} placed IBS may not be interpreted as the recipient but rather as the benefactive. The 2\textsuperscript{nd} person singular OM \textit{kú-} thus receives the recipient interpretation, as shown in \REF{ex:yoneda:51a} and \REF{ex:yoneda:51b} respectively.  


\ea%49
    \label{ex:yoneda:51}
    \ea\label{ex:yoneda:51a}\textsc{theme}-\textsc{ben-rec}\\
    \gll \# N-ki-ku-bá-gámb-idde.\\
        {} \textsc{sm1sg-om7-om2sg-om3pl}-tell-\textsc{appl.prf} \\
    \glt *Interpretation 1:  ‘I have told it to them for/on behalf of you(sg).’  \\
Interpretation 2:   ‘I have told it to you(sg) for/on behalf of them.’
    \ex\label{ex:yoneda:51b} \textsc{ben}-\textsc{theme}-\textsc{rec}\\
    \gll \# N-ku-ki-bá-gámb-idde.      \\
      {} \textsc{sm1sg-om2sg-om7-om3pl}-tell-\textsc{appl.prf}\\              
      \glt  *Interpretation 1:  ‘I have told it to them for/on behalf of you(sg).’       \\
    Interpretation 2:   ‘I have told it to you(sg) for/on behalf of them.’
    \z
\z


\REF{ex:yoneda:52} shows examples in which the theme OM \textit{ki-} is placed IBS. This is very unnatural regardless of the ordering of the other OMs, as shown in \REF{ex:yoneda:52a} and \REF{ex:yoneda:52b} (note again the ??).   


\ea%50
    \label{ex:yoneda:52}
    \ea\label{ex:yoneda:52a} \textsc{rec-ben}-\textsc{theme}\\
        \gll   ?? N-ba-kú-kí-gámb-idde. \\
        {} \textsc{sm1sg-om3pl-om2sg-om7}-tell-\textsc{appl.prf}\\
    \glt ??Interpretation 1:  ‘I have told it to them for/on behalf of you (sg).'      \\ 
    *Interpretation 2:  ‘I have told it to you (sg) for/on behalf of them.'
    \ex\label{ex:yoneda:52b} \textsc{ben-rec}-\textsc{theme}\\
    \gll ?? N-ku-ba-kí-gámb-idde.   \\
      {} \textsc{sm1sg-om2sg-om3pl-om7}-tell-\textsc{appl.prf}\\              
       \glt *Interpretation 1:   ‘I have told it to them for/on behalf of you (sg).’     \\  
    ??Interpretation 2:   ‘I have told it to you (sg) for/on behalf of them.’
    \z
\z


In \REF{ex:yoneda:52a}, the 2\textsuperscript{nd} person singular OM \textit{ku-} can still be interpreted as the benefactive. The same holds with the 3\textsuperscript{rd} person plural OM \textit{ba-} in \REF{ex:yoneda:52b}. This suggests that the closer the OM is to IBS, the more likely it is to be interpreted as the benefactive, although these sentences are still very odd. 



    However, here again, the 1\textsuperscript{st} person singular OM \textit{n-} behaves differently. This must be placed IBS as shown in \REF{ex:yoneda:53a} and \REF{ex:yoneda:53b}, and other orders are all ungrammatical as shown in \REF{ex:yoneda:53c}--\REF{ex:yoneda:53f}. 


\ea%51
    \label{ex:yoneda:53}
    \ea[]{\label{ex:yoneda:53a}  \gll  A-ki-bá-\textbf{n-}gámb-idde.\\
      \textsc{sm3sg-om7-om3pl-om1sg}-tell-\textsc{appl.prf}\\              
        }
    \ex[]{\label{ex:yoneda:53b} \gll    A-ba-kí-\textbf{n-}gámb-idde.\\
      \textsc{sm3sg-om3pl-om7-om1sg}-tell-\textsc{appl.prf}\\ }             
    \ex[*]{\label{ex:yoneda:53c} \gll  A-ki-\textbf{n-}bá-gámb-idde.\\
      \textsc{sm3sg-om7-om1sg-om3pl}-tell-\textsc{appl.prf}\\}              
    \ex[*]{\label{ex:yoneda:53d} \gll  A-\textbf{n-}kí-bá-gámb-idde.\\
      \textsc{sm3sg-om1sg-om7-om3pl}-tell-\textsc{appl.prf}\\ }             
    \ex[*]{\label{ex:yoneda:53e} \gll A-\textbf{n-}ba-kí-gámb-idde.\\
      \textsc{sm3sg-om1sg-om3pl-om7}-tell-\textsc{appl.prf}\\}              
    \ex[*]{\label{ex:yoneda:53f} \gll  A-ba-\textbf{n-}kí-gámb-idde.\\
      \textsc{sm3sg-om3pl-om1sg-om7}-tell-\textsc{appl.prf}\\
    \glt ‘He has told it to them for/on behalf of me.’}
    \z
\z

Therefore, when the benefactive is not 1\textsuperscript{st} person singular, it cannot appear as an OM as shown in \REF{ex:yoneda:54}. This is the same restriction we saw earlier in \REF{ex:yoneda:47} in \sectref{sec:yoneda:4.2.1} above. 

\ea%52
    \label{ex:yoneda:54}
    \ea[*]{\label{ex:yoneda:54a} \gll   A-\textbf{n-}kí-ba-gámb-idde.\\
      \textsc{sm3sg-om1sg-om7-om3pl}-tell-\textsc{appl.prf}\\ }             
    \ex[]{\label{ex:yoneda:54b} \gll   A{}-ki-n-gámbye               kulwábwe. \\           
    \textsc{sm3sg-om7-om1sg}-tell.\textsc{prf}    for.them\\
    \glt ‘He has told it to me for them.’
    }
    \z
\z

\subsection{Constraint of the 1\textsuperscript{st} person singular OM \textit{n-}}     
\label{sec:yoneda:4.3}

\citet{Marlo2014} discusses a unique behavior of the 1\textsuperscript{st} person singular OM \textit{n-} when it appears alongside the reflexive as shown in \REF{ex:yoneda:55}.

\ea\label{ex:yoneda:55} Unique properties of 1\textsc{sg} and reflexive OP\footnote{OP = Object Prefix, referred to as the OM in this chapter.} \citep[5]{Marlo2014}

\ea\label{ex:yoneda:55a}
The 1\textsuperscript{st} person singular OP and the reflexive are generally required to surface closest to the verb stem \citep[297]{Polak1983} and may therefore be in different morphological or syntactic positions from other OPs (\citealt{Buell2005, Muriungi2008}). 

\ex \label{ex:yoneda:55b}
The 1\textsuperscript{st} person singular and reflexive are the highest on animacy-topicality and person-number hierarchies, which are known to play a role in object marking (\citealt{Alsina1994, Contini-Morava1983, Duranti1979, Rugemalira1993obj}). 

\ex \label{ex:yoneda:55c}
Most OPs have a CV- shape, but 1SG and reflexive are generally unique in having monophone N- and V-.
\z
\z

As we have seen in \sectref{sec:yoneda:3.3} and \sectref{sec:yoneda:4.2}, this is true for Ganda as well; the 1\textsuperscript{st} person singular OM \textit{n-} must always be placed IBS regardless of its semantic role. This is not a preference or tendency, but rather is obligatory. 



    This constraint only holds for 1\textsuperscript{st} person singular, not for 1\textsuperscript{st} person plural. Unlike the case of 1\textsuperscript{st} person singular OM, the 1\textsuperscript{st} person plural OM can appear even when it is not the benefactive as shown in \REF{ex:yoneda:56}.   


\ea%54
    \label{ex:yoneda:56}
    \gll A-kí-tú-ba-gámb-idde.\\
         \textsc{sm3sg-om7-om1pl-om3pl}-tell-\textsc{appl.prf} \\
    \glt ‘He has told it to us for/on behalf of them.’
    \z

            

                 

      

    The restriction that the 1\textsuperscript{st} person singular OM must be placed at IBS does not seem to be due to the hierarchy, since it only holds for 1\textsuperscript{st} person singular and not for 1\textsuperscript{st} person plural. Of course, it is possible that there is a difference in the hierarchy between singular and plural. As mentioned in \REF{ex:yoneda:55b}, the 1\textsuperscript{st} person singular OM is the highest on the animacy-topicality and person-number hierarchies. However, all the asymmetrical characteristics that can be observed, due to the hierarchy, are not as rigid as the rule regarding the placement of the 1\textsuperscript{st} person singular OM. Therefore, we should think of this constraint on the 1\textsuperscript{st} person singular OM independently of the restrictions on other objects.  

    The most likely explanation is a morpho-phonological one. All OMs in Ganda take the form CV except for the 1\textsuperscript{st} person singular OM. Only the 1\textsuperscript{st} person singular OM \textit{n}{}- does not itself constitute a syllable (see \REF{ex:yoneda:55c}). It must therefore appear either alongside the vowel of the TAM marker, or merge with the initial consonant of the stem in order to form a syllable. This means that it cannot appear in front of other OMs. However, it is still ``the most likely explanation'', but remains a topic that requires further research. 

\subsection{Findings and summary of triple-object constructions in Ganda}
\label{sec:yoneda:4.4}

Based on what we have discussed in \sectref{sec:yoneda:4.1}--\sectref{sec:yoneda:4.3}, the following generalizations can be identified \REF{ex:yoneda:57} regarding the triple-object construction in Ganda: 

\ea%55
    \label{ex:yoneda:57}
    \ea\label{ex:yoneda:57a}    The triple-object construction occurs with the applicative form of ditransitive verbs.

    \ex\label{ex:yoneda:57b}    Placing three object NPs after the verb is not accepted; therefore, at least one of the three objects must appear as an OM.  

    \ex\label{ex:yoneda:57c}    The applied object must always appear as the OM and cannot appear as an NP.

    \ex\label{ex:yoneda:57d}    Up to three OMs can appear at the same time. 

    \ex\label{ex:yoneda:57e}    The OM that appears IBS necessarily has the benefactive interpretation. 

    \ex\label{ex:yoneda:57f}    When the verb has an OM which agrees with the 1\textsuperscript{st} person singular object, it must be placed IBS and must be the benefactive.   
    \z
\z

As summarized in \REF{ex:yoneda:57}, although verbs in Ganda cannot be followed by three object NPs, the triple-object construction itself is possible. 

    In the triple-object construction, the applied object must always be expressed as an OM, and this is a further restriction unique to the triple-object construction. In the double-object construction, the ordering of OMs is asymmetrically determined by semantic role, but pronominalization is symmetric in that any of the two objects can be freely pronominalized. In the triple-object construction, then, the semantic role hierarchy also figures crucially in pronominalization. As a result, the triple-object construction turns out to be more rigidly asymmetrical.


\section{Conclusion}
\label{sec:yoneda:5}


We have shown how object NPs and OMs behave in multiple-object constructions in Ganda. 



Considering the symmetrical versus asymmetrical nature of objects in Ganda, both objects in a double-object construction can be (i) placed IAV,\footnote{As we mentioned in \sectref{sec:yoneda:3.1.1}, \citet{Ranero2019} claims that the order of postverbal objects in the ditransitive is strictly ``goal/ben -- theme''. However, according to our data, both ``goal/ben -- theme'' and ``theme -- goal/ben'' orders are acceptable.} (ii) pronominalized, and (iii) passivized. According to the criteria proposed by \citet{BresnanMoshi1993}, Ganda seems to be a ``perfect'' symmetrical object language. \citet{Ssekiryango2006} claims that Ganda is a symmetrical language although he also shows asymmetrical data. The data we presented in this chapter are perhaps more supportive of \citet{Ssekiryango2006} than his data in support of the idea that Ganda is a symmetrical language. Even if the resulting interpretation of the sentence is ambiguous, objects still behave symmetrically. This can be seen as strong evidence that Ganda is a symmetrical object language. 



However, this language also shows some characteristics of an asymmetrical object language. There is a preference to place the recipient, benefactive, or causee at the IAV position, to pronominalize these elements and to passivize them as opposed to the theme. A preference to treat the recipient, benefactive, and causee as the primary objects can be seen with respect to all three criteria. In addition to these criteria, the order of OMs is fixed asymmetrically. These asymmetrical features are affected by the semantic role, rather than the animacy hierarchy.  


    It is not surprising to find some asymmetrical features in the languages whose objects are considered as symmetrical, and this seems to be very natural. It is unnecessary to reconsider such languages as asymmetrical languages according to such asymmetrical features. Presumably, it must be common that some asymmetrical characteristics exist even in a language which is considered predominantly symmetrical, and languages in which objects behave completely symmetrical must be very rare. However, it is still meaningful, especially for a micro-variation study, to find out where or based on which semantic (or other grammatical) features such asymmetrical characteristics can be observed. Interestingly, in Ganda, the semantic role affects the preference or naturalness of the order of OMs, but the factor which most strongly affects the order of the OMs is a morpho-phonological condition, particularly for the 1\textsuperscript{st} person singular OM \textit{n-}. Therefore, there seems to be different kinds of restrictions in Ganda, one based on semantic roles, and the other, presumably a morpho-phonological condition.


Summarizing the symmetry\slash asymmetry in Ganda, the objects show a symmetrical nature with respect to ordering and passivization, with some exceptions in cases where the animacy of both objects are the same. They are symmetrical with respect to pronominalization, but not in the ordering of OMs. The asymmetry emerges as a result of the hierarchical principles of semantic roles rather than animacy. Regarding the possible number of OMs, although Ganda has previously been considered a language that allows two OMs (\citealt{Ssekiryango2006, Marlo2015}), we have shown that three OMs are possible, and the conditions where this is possible.



The emergence of the asymmetrical characteristics in Ganda can be seen as part of a more general concept of the ``emergence of the unmarked'' (cf. \citealt{Bresnan1997,Bresnan2001}), whereby a grammatical property or restriction that is usually not observed, or ``hidden'', suddenly surfaces or comes in effect in a corner of a language under a certain condition. Many ``exceptions'' in language might be then understood as ``emergent properties'' once we recognize the relevant, crucial conditioning factors. 


\section*{Acknowledgements}
This paper is based on our presentation at the 9\textsuperscript{th} World Congress of African Linguistics held in 2018 in Rabat, Morocco. We would like to thank all those who gave us helpful comments during the conference. We would also like to thank the reviewers, who provided very helpful comments and suggestions for the previous versions of this paper. Of course any mistakes are our own responsibility. This research is supported by JSPS Grants-in-Aid for Scientific Research (Grant Number 19K00550, 19H01254) and the ILCAA Joint Research Project ``Typological Study of Microvariation in Bantu (2)''.

\sloppy\printbibliography[heading=subbibliography,notkeyword=this]
\end{document} 
