\documentclass[output=paper 		  ]{langscibook}
\ChapterDOI{10.5281/zenodo.10663765}

\author{Allen Asiimwe\orcid{0000-0003-1906-519X}\affiliation{Makerere University, Uganda}}

\title[Demonstratives in Run\-yan\-ko\-re-Ru\-ki\-ga]{The structure, distribution and function of demonstratives in Run\-yan\-ko\-re-Ru\-ki\-ga}

\abstract{The aim of this chapter is to give an overview of the morpho-syntactic characteristics and functions of demonstratives in Run\-yan\-ko\-re-Ru\-ki\-ga (Bantu JE.13-JE.14, Uganda). The study shows that a typical Run\-yan\-ko\-re-Ru\-ki\-ga demonstrative is comprised of the core demonstrative morpheme prevalent in many Bantu languages, the noun class concord and a suffix. The suffix indicates the position of the referent from the speaker and/or hearer. There are other forms of the demonstrative discussed, such as the demonstrative \textit{{}-nu} which is used mainly in narratives to mark a continuation topic, and the locative copulative demonstrative realised by the nasal \textit{n}{}-, which expresses a more specific location. A much-neglected form of demonstrative -\textit{ti} with both nominal and verbal properties, whose primary role is to express manner, is also discussed. In terms of position, data indicate that basic demonstrative types can either precede or follow the noun but it appears that there is no clear-cut connection between the position of the demonstrative and the role it plays. The chapter also discusses pragmatic roles of demonstratives categorized into exophoric for the non-anaphoric uses and endophoric for discourse functions of demonstratives. The study draws data from written literary sources, spontaneous speech and elicitation. The study is a pointer to specific and detailed studies that need to be conducted on demonstratives in Run\-yan\-ko\-re-Ru\-ki\-ga and other related Bantu languages.
\keywords{demonstratives, exophoric, endophoric, manner demonstratives, Run\-yan\-ko\-re-Ru\-ki\-ga, Bantu}
}

\IfFileExists{../localcommands.tex}{
  \addbibresource{../localbibliography.bib}
  % add all extra packages you need to load to this file

\usepackage{tabularx,multicol}
\usepackage{url}
\urlstyle{same}

\usepackage{listings}
\lstset{basicstyle=\ttfamily,tabsize=2,breaklines=true}

\usepackage{langsci-basic}
\usepackage{langsci-optional}
\usepackage{langsci-lgr}
\usepackage{langsci-osl}
% \usepackage{./langsci/styles/langsci-lgr}
% \usepackage{./langsci/styles/langsci-osl}
% \usepackage{langsci-gb4e}

\usepackage{tikz}
\usetikzlibrary{patterns,calc}
\pgfdeclarepatternformonly{south east lines}{\pgfqpoint{-0pt}{-0pt}}{\pgfqpoint{3pt}{3pt}}{\pgfqpoint{3pt}{3pt}}{
    \pgfsetlinewidth{0.6pt}
    \pgfpathmoveto{\pgfqpoint{0pt}{3pt}}
    \pgfpathlineto{\pgfqpoint{3pt}{0pt}}
    \pgfpathmoveto{\pgfqpoint{.2pt}{-.2pt}}
    \pgfpathlineto{\pgfqpoint{-.2pt}{.2pt}}
    \pgfpathmoveto{\pgfqpoint{3.2pt}{2.8pt}}
    \pgfpathlineto{\pgfqpoint{2.8pt}{3.2pt}}
    \pgfusepath{stroke}}
    
\usepackage{stmaryrd}
\usepackage{wasysym}
\usepackage{multirow}
\usepackage{caption}
\usepackage{subcaption}
\usepackage{mathrsfs}
\usepackage{qtree}

\usepackage{linguex}


  %pminos do not split footnotes
% \interfootnotelinepenalty=10000 %Footnote in Laporte chapters has to be split SN


%\DeclareIndexNameFormat{default}{%
%\nameparts{#1}%
%\usebibmacro{index:name}%
%{\index[names]}%
%{\namepartfamily}%
%{\namepartgiveni}%
% {}% L1
% {}% L2
%{\namepartprefix}% generates spurious space L3
%{\namepartsuffix}% generates spurious space L4
%}

%  {\DeclareIndexNameFormat{default}{%
%     \usebibmacro{index:name}{\index[names]}{#1}{#3}{#5}{#7}}}

%\DeclareIndexNameFormat{default}{%
%  \usebibmacro{index:name}{\sindex[nom]}{#1}{#3}{#5}{#7}}

%\DeclareIndexNameFormat{default}{%
%  \usebibmacro{index:name}{\sindex[person]}{#1}{#3}{#5}{#7}}
%\DeclareIndexNameFormat{default}{%
%\nameparts{#1} \usebibmacro{index:name}{\sindex[person]]}{\namepartfamily}{‌​\namepartgiven}{\nam‌​epartprefix}{\namepa‌​rtsuffix}}

%\newcommand{\smiley}{:)}

%\renewbibmacro*{index:name}[5]{%
%\usebibmacro{index:entry}{#1}%
%{\iffieldundef{usera}{}{\thefield{usera}\actualoperator}\mkbibindexname{#2}{#3}{#4}{#5}}}

% \newcommand{\noop}[1]{}

%remove for final
%\overfullrule=1mm

\newcommand{\tobi}[2]}}
\renewcommand{\S}[1]{\tobi{#1}{\textsc{*}}}

% this volume references
% puts: [this volume]
% already defined: \citetv
%\newcommand{\citepv}[1]{(\citeauthor{#1} \citeyear*{#1} [this volume])}
\newcommand{\citealtv}[1]{\citeauthor{#1} \citeyear*{#1} [this volume]}

%parentheses around example number
\newcommand{\pref}[1]{(\ref{#1})}

% in-text examples

\newcommand{\lnex}[1]{\textit{#1}} %target lang word
\newcommand{\lnlit}[1]{(lit.: `#1')} %literal reading
\newcommand{\lnlat}[1]{(#1)} % latinization
\newcommand{\lntrans}[1]{`#1'} %translation
\newcommand{\lnexl}[2]%
{\lnex{#1}{} \lnlat{#2}} % ex with latinization
\newcommand{\lnexlat}[3]{\lnex{#1}{} \lnlat{#2}{} \lntrans{#3}} % ex with latinization and tranl.

%ch01
\newcommand{\co}[1]{\mbox{\textbf{#1}}}

%ch09

\newcommand{\cyrbulg}[1]{\begin{otherlanguage*}{bulgarian}#1\end{otherlanguage*}}


%ch10
\newcommand{\nlp}{{\small NLP}}
\newcommand{\mwe}{{\small MWE}}
\newcommand{\rae}{{\small RAE}}
\newcommand{\lvc}{{\small LVC}}
\newcommand{\pos}{{\small P}o{\small S}}
%\newcommand{\todo}[1]{ \textcolor{red}{#1} }

%\renewcommand{\labelenumi}{\theenumi}
%\ainamefmt{{vv}{ll}{, ff}{, jj}} % fullname

\newcommand{\biberror}[1]{{\color{red}#1}}

\newcommand{\osenovaitem}{--~}
  %% hyphenation points for line breaks
%% Normally, automatic hyphenation in LaTeX is very good
%% If a word is mis-hyphenated, add it to this file
%%
%% add information to TeX file before \begin{document} with:
%% %% hyphenation points for line breaks
%% Normally, automatic hyphenation in LaTeX is very good
%% If a word is mis-hyphenated, add it to this file
%%
%% add information to TeX file before \begin{document} with:
%% %% hyphenation points for line breaks
%% Normally, automatic hyphenation in LaTeX is very good
%% If a word is mis-hyphenated, add it to this file
%%
%% add information to TeX file before \begin{document} with:
%% \include{localhyphenation}
\hyphenation{
    Beck-man
    Ngu-yen
    back-chan-nel
    back-chan-nels
    mo-not-o-nous
    ste-reo-typ-i-cal
}

\hyphenation{
    Beck-man
    Ngu-yen
    back-chan-nel
    back-chan-nels
    mo-not-o-nous
    ste-reo-typ-i-cal
}

\hyphenation{
    Beck-man
    Ngu-yen
    back-chan-nel
    back-chan-nels
    mo-not-o-nous
    ste-reo-typ-i-cal
}

  \togglepaper[1]%%chapternumber
}{}

\begin{document}
\maketitle
%\shorttitlerunninghead{}%%use this for an abridged title in the page headers


\section{Introduction}\label{sec:asiimwe:1}

Run\-yan\-ko\-re-Ru\-ki\-ga (JE.13–JE.14) is an interlacustrine Bantu language cluster spoken as one of the main indigenous languages of Uganda. Indigenous languages of Uganda are classified according to their corresponding ethnic groups \citep{SimonsFennig2017}. As such, Rukiga is regarded as a language of the Bakiga and Runyankore as a language of the Banyankore. The population of these two ethnic groups is estimated to be a combined total of about six million, based on the 2014 Uganda National Population and Housing Census report. Due to the high level of mutual intelligibility between Runyankore and Rukiga, which is estimated at approximately 94\% lexical similarity (\citealt{LadefogedEtAl1972, SimonsFennig2017}), and due to their grammatical affinity, they are linguistically regarded as dialects of the same language hence the overall label Run\-yan\-ko\-re-Ru\-ki\-ga.

\begin{sloppypar}
Runyankore and Rukiga belong to the Nyoro-Ganda group (following the classification provided by \citealt{Maho2009}). Other Ugandan Bantu languages in the same group include Runyoro (JE.11), Rutooro (JE.12), Luganda (JE.15) and Lusoga (JE.16), among others. In the same way that Run\-yan\-ko\-re-Ru\-ki\-ga are often referred to under this joint label, Runyoro and Rutooro are also sometimes termed Runyoro-Rutooro due to their mutual intelligibility with approximately 93\% shared lexicon (\citealt{LadefogedEtAl1972, LewisEtAl2013}). Run\-yan\-ko\-re-Ru\-ki\-ga and Runyoro-Rutooro together form ``Runyankitara'' (JE.10) -- a ``language'' which is taught in some universities in Uganda \citep{Bernsten1998}.\footnote{Runyakitara is not a language but a label used to teach the four mutually intelligible languages.}
\end{sloppypar}

Run\-yan\-ko\-re-Ru\-ki\-ga exhibits a concord system which is characteristic of Bantu languages. Elements that modify the noun agree with it in terms of grammatical gender and number. The augment is part of the nominal morphology of Run\-yan\-ko\-re-Ru\-ki\-ga, which is conditioned by syntactic but also dis\-course-prag\-mat\-ic factors (e.g. see Asiimwe et al 2023). The example in \REF{ex:asiimwe:1} illustrates the concord system and the augment in the nominal domain.\footnote{Otherwise indicated, examples used are not inclined towards a particular language variety. When an example is drawn from an individual language variety say Rukiga or Runyankore, this is indicated.}

\ea%1
    \label{ex:asiimwe:1}
    ekyo (é)kikópo (é)kihango (é)kiríkutukura\\
    \gll  e-ki-o  (e)-ki-kopo  (e)-ki-hango  (e)-ki-riku-tukur-a\\
  \textsc{dem}{}-7-\textsc{med}  \textsc{aug-}7-cup  \textsc{aug}{}-7-big  \textsc{aug}{}-7-\textsc{ipfv}{}-red-\textsc{fv} \\
  \glt ‘that big cup which is red’
\z

This chapter offers a general descriptive overview of the demonstrative system in Run\-yan\-ko\-re-Ru\-ki\-ga. Various forms, the distribution, and functions of the demonstrative are discussed. The demonstrative in Run\-yan\-ko\-re-Ru\-ki\-ga is generally comprised of three parts: the initial demonstrative morpheme, the noun class concord and the morpheme that shows distance although, in the proximal demonstrative form, distance is zero-marked. Like most other Bantu languages, Run\-yan\-ko\-re-Ru\-ki\-ga uses a three-way demonstrative system to mark the distance of a referent in relation to the position of the speaker and the hearer (see also \sectref{sec:asiimwe:2}).\footnote{There are however some Bantu languages that present a four-way demonstrative system for example, Nande (JD.42) (\citealt{Valinande1984} via \citealt{VandeVelde2019}), Chidingo (E.73) \citep{Nicolle2007} and Fwe (K402) \citep{Gunnink2018}.} In relation to the position of the referent, the following terms are adopted for the current description: proximal for an entity close to the speaker, medial for a referent that is near to the hearer and distal to refer to an entity that is far from both the speaker and the hearer \REF{ex:asiimwe:2} as indicated in the glosses.

\ea%2
    \label{ex:asiimwe:2}
    \ea\label{ex:asiimwe:2a}  egy’  énju\\
    \gll e-gi  e-n-ju\\
    \textsc{dem-9.prox}  \textsc{aug}{}-9-house\\
    \glt  ‘this house’

    \ex\label{ex:asiimwe:2b} enjw’ égyo\\
    \gll e-n-ju  e-gy-o\\
    \textsc{aug-}9-house  \textsc{dem-9-med}\\
    \glt ‘that house’

    \ex\label{ex:asiimwe:2c} enjw’ éríya\\
    \gll e-n-ju  ∅-e-riya\\
    \textsc{aug}{}-9-house  \textsc{dem}{}-9-\textsc{dist}\\
    \glt ‘that house (far from both speaker and hearer but visible)’
    \z
\z

The initial element in the morphology of the demonstrative is analysed here as the core demonstrative morpheme, which is underlyingly \textit{a-}. The demonstrative core morpheme \textit{a-} has also been described in a number of other Bantu languages (cf. \citealt{Wald1973, DuPlessisEtAl1992, Visser2008}). The analysis of the existence of this initial core morpheme in Run\-yan\-ko\-re-Ru\-ki\-ga is further motivated in \sectref{sec:asiimwe:2} (see also \citealt{Asiimwe2014,Asiimwe2016}).


\largerpage
As an adnominal modifier, the demonstrative occurs either before or after the head noun (\ref{ex:asiimwe:3a}--\ref{ex:asiimwe:3b}). Run\-yan\-ko\-re-Ru\-ki\-ga also allows double demonstratives to modify a single noun \REF{ex:asiimwe:3c} for emphasis or particularisation (\sectref{sec:asiimwe:4.3}). The syntax of demonstratives is discussed in further detail in \sectref{sec:asiimwe:3}.

\ea%3
    \label{ex:asiimwe:3}

    \ea\label{ex:asiimwe:3a} Egi njojo ní nto\\
    \gll \emph{e-gi}  n-jojo  ni  n-to\\
    \textsc{dem}{}-9.\textsc{prox}  9-elephant  \textsc{cop}  9-young\\
  \glt ‘This elephant is young.’

  \ex\label{ex:asiimwe:3b} Enjoj’ égi ni nto\\
    \gll e-n-jojo  \emph{{}-e-gi}  ni  n-to\\
    9-elephant  \textsc{dem}{}-9.\textsc{prox} \textsc{cop}  9-young\\
  \glt ‘This elephant is young.’

  \ex\label{ex:asiimwe:3c} Egy’ (é)njojw’ égi ní nto\\
    \gll \emph{{e}-gi}  (é)-n-jojo  é-gi  ni  n-to\\
    \textsc{dem-9.prox}  \textsc{aug}{}-9-elephant  \textsc{dem-9.prox.cop}  is  9-young\\
    \glt ‘This (particular) elephant is young.’
    \z
\z

Cross-linguistically, demonstratives play two key semantic roles: (i) to mark a referent that is present in the physical environment relative to the deictic centre and (ii) to refer to an entity already established in discourse \citep[7]{Diessel1999}. Hence, demonstratives are generally used to mark familiar and accessible entities \citep{Lyons1999}. Additionally, a demonstrative may serve to activate old knowledge assumed to be familiar to both the speaker and the hearer. This latter type of demonstrative has a ``recognitional'' role according to \citet{Himmelmann1996} and \citet{Diessel1999}, as exemplified in \REF{ex:asiimwe:4}. The analysis of pragmatic functions of demonstratives follows \citegen{Diessel1999} classification of demonstratives into ``exophoric'' and ``endophoric'' uses, discussed in \sectref{sec:asiimwe:4}.

\ea%4
    \label{ex:asiimwe:4}
          Eshááha yaaw’ éríy’ énkúr’ erahe?\\
\gll  e-shaaha  ya-a-we  ∅\emph{{}-e-riya}  e-n-kuru  e-rahe? \\
  \textsc{aug}{}-9.watch  9-\textsc{conn}{}-your  \textsc{dem-}9-\textsc{dist}  \textsc{aug}{}-9-old  9-where\\
\glt ‘Where is that old watch of yours?’
\z

Data for this study come from folktales, newspaper texts, spontaneous speech, and a number of constructions were elicited. The challenge at present is that there is no substantial accessible corpus for Run\-yan\-ko\-re-Ru\-ki\-ga. Through reading various materials written in Run\-yan\-ko\-re-Ru\-ki\-ga, and analysing radio and conversation recordings, I identified sentences or parts of sentences or chunks of discourse which are relevant for the study. As a native speaker of Rukiga, I used introspection but also held consultations with other native speakers of both Runyankore and Rukiga on grammatical judgements of given constructions.\footnote{I wish to thank the following native speakers of Run\-yan\-ko\-re-Ru\-ki\-ga: Fridah Katushemererwe, Justus Turamyomwe, Aron Turyasingura, Celestino Oriikiriza, Misah Natumanya and Emmy Rwomushana for discussing parts of the data analysed in this chapter.}\footnote{Since there was no accessible corpus for Run\-yan\-ko\-re-Ru\-ki\-ga at the time of writing this chapter, it was not possible to present a quantitative analysis of the demonstrative system and the distribution of the forms within a corpus. This will remain an avenue for future research.}

\section{The morphology of demonstratives}
\label{sec:asiimwe:2}

This section discusses the various forms of the demonstrative in Runyakore-Rukiga. The basic form of the demonstrative is composed of three elements: the core demonstrative \textit{a-} (\sectref{sec:asiimwe:2.1}), the agreement morpheme and the suffix that marks the position of the referent from the speaker and/or hearer (\sectref{sec:asiimwe:2.2}). There are other forms of the demonstrative discussed in this section, including the demonstrative –\textit{nu} (\sectref{sec:asiimwe:2.3}) and the ‘identificational’ \textit{n}{}- (\sectref{sec:asiimwe:2.4}). In addition, the locative (\sectref{sec:asiimwe:2.5}), as well as manner demonstrative forms (\sectref{sec:asiimwe:2.6}), are discussed.

\subsection{Argument for the demonstrative root \textit{a}-}
\label{sec:asiimwe:2.1}

Previous studies, such as \citet{MorrisKirwan1972} and \citet[137--138]{Taylor1985}, suggest that the initial morpheme of the demonstrative in Run\-yan\-ko\-re-Ru\-ki\-ga is an initial vowel.\footnote{Other terms such as augment or preprefix are often used. The term augment is chosen for the current discussion.} For instance, \citet[137]{Taylor1985} argues that the morpheme is an augment which is obligatory in the first degree of distance, that is, the proximal demonstrative in the present analysis. \citet{Taylor1985} further points out that this morpheme is deleted when it appears within the scope of a negative operator just like any other augment of any other nominal under the same syntactic conditions. He further states that it is possible for the augment to be retained if the demonstrative is placed before the lexical noun such that, instead, it is the augment of the head noun that is affected by the negative operator. \citet{Taylor1985} again notes that the demonstrative retains its augment when the head noun is implicit. Since \citet{Taylor1985} observes contexts in which the initial morpheme of the demonstrative is retained, it appears that it is indeed a central element in the morphology of the demonstrative.

\citet{Wald1973}, \citet{DuPlessisEtAl1978} and  \citet{DuPlessisEtAl1992} have established that the canonical core demonstrative morpheme which is attested in many Bantu languages is underlyingly the morpheme \textit{a-.} As also seen in a number of other Bantu languages, the initial element of the demonstrative in Run\-yan\-ko\-re-Ru\-ki\-ga appears to be the core of the demonstrative although in some languages such as Chiyao (P.21), the initial element of the demonstrative is not indispensable since it can be dropped depending on the syntactic and pragmatic factors (see \textcitetv{chapters/taji}). \citet[28]{Visser2008} notes that this morpheme may appear allomorphically as \textit{a}{}-, \textit{e}{}-, or \textit{o}{}- depending on the vowel of the agreement prefix of head noun, following the rules of vowel harmony, identical to the form of the respective augment. This is observed for Run\-yan\-ko\-re-Ru\-ki\-ga as well. The quality of the vowel of the class prefix determines the shape of the initial morpheme of the demonstrative. As shown in (5a-c), the initial morpheme appears as: /a/ if the vowel of the class concord is /a/; if the vowel of the noun class prefix is /u/, the core morpheme manifests phonologically as /o/, while /e/ is as a result of having /i/ or /o/ as the vowel of the noun class prefix. The examples in (\ref{ex:asiimwe:5a}--\ref{ex:asiimwe:5c}) show the proximal and medial forms. For the distal form in \REF{ex:asiimwe:5d}, the core demonstrative \textit{a}{}- is morphologically zero-marked.\largerpage

\ea%5
    \label{ex:asiimwe:5}
    \ea\label{ex:asiimwe:5a} amat’áaga\\
     \gll a-ma-te  a-ga\\
     \textsc{aug}{}-6-milk  \textsc{dem-6.prox}\\
     \glt ‘this milk’

    \ex\label{ex:asiimwe:5b}  omut’ óogu/ogwo\\
    \gll o-mu-ti    o-gu/o-gu-o\\
      \textsc{aug}{}-3-tree  \textsc{dem-3.prox/dem-3-med}\\
    \glt  ‘this tree/that tree’

    \ex\label{ex:asiimwe:5c}  e-mit’ éegi/egyo\\
    \gll e-mi-ti  e-gi/e-gi-o\\
    \textsc{aug}{}-4-tree  \textsc{dem-4.prox/dem-4-med}\\
    \glt  ‘these/those trees’

    \ex\label{ex:asiimwe:5d}  omuti gúri(yà)\\
    \gll o-mu-ti  ∅-gu-ri(ya)\\
      \textsc{aug}{}-3-tree  \textsc{dem-3-dist}\\
    \glt    ‘that tree (far from speaker and hearer)’
  \z
\z

In the remaining part of this subsection, I give evidence to support the claim that the initial morpheme of the demonstrative is not an augment as has been claimed in a number of previous studies but, instead  is the core morpheme of the demonstrative. I demonstrate that the initial morpheme of the postnominal adnominal demonstrative following a negative verb is not affected by the negative operator the same way as the augment of other nominals. The examples below show that nouns \REF{ex:asiimwe:6a} and nominal modifiers such as adjectives \REF{ex:asiimwe:6b}, possessives \REF{ex:asiimwe:6c} and some quantifiers \REF{ex:asiimwe:6d} following a negative verb can appear without the augment but the initial element of the demonstrative is not affected by negation as shown in \REF{ex:asiimwe:6e}.

\ea%6
    \label{ex:asiimwe:6}
    \ea\label{ex:asiimwe:6a}  Tíbaareeba (é)ngagi\\
  \gll ti-ba-aa-reeb-a  (e)-n-gagi\\
  \textsc{neg-2.sbj-n.pst}{}-see-\textsc{fv}  \textsc{aug}{}-10-gorilla\\
  \glt ‘They have not seen (the) gorillas.’

    \ex\label{ex:asiimwe:6b}  Tíbaareeba (é)mpango\\
 \gll ti-ba-aa-reeb-a  (e)-n-hango \\
  \textsc{neg-2.sbj-n.pst}{}-see-\textsc{fv}  \textsc{aug}{}-10-big\\
  \glt `They did not see (the) big ones.’

    \ex\label{ex:asiimwe:6c} Tíbaareeba (é)zaawe\\
  \gll ti-ba-aa-reeb-a  (e)-za-awe\\
  \textsc{neg-2.sbj-n.pst}{}-see-\textsc{fv}  \textsc{aug}{}-10-your\\
  \glt ‘They did not see yours.’

    \ex\label{ex:asiimwe:6d}  Tíbaareeba (é)zindi\\
 \gll ti-ba-aa-reeb-a  (e)-zi-ndi\\
  \textsc{neg}{}-\textsc{2.sbj-n.pst}{}-see-\textsc{fv}  \textsc{aug}{}-10-other\\
  \glt ‘They did not see (the) others.’

  \ex\label{ex:asiimwe:6e}  Abarámbuzi tíbareeb’ ézi ngagi\\
    \gll a-ba-rambuz-i  ti-ba-a-reeb-a  \emph{e-zi}  n-gagi\\
    \textsc{aug-}2-tourist-\textsc{nmlz}  \textsc{neg}{}-2.\textsc{sbj-n.pst}{}-see-\textsc{fv}  \textsc{dem}{}-10.\textsc{prox}  10-gorilla\\
  \glt ‘(The) tourists have not seen these gorillas.’
    \z
\z

The example given in \REF{ex:asiimwe:7} below is from \citet[151]{MorrisKirwan1972}, and it shows that the initial morpheme of the demonstrative can be omitted if the demonstrative comes immediately after a negative verb.\footnote{Glosses are mine.} However, in the discourse studied, there are no cases where  the demonstrative following a negative verb loses its initial element. Hence, the effects of the negation rule are likely to be offset by not placing the demonstrative immediately after the negative verb and this is the strategy that speakers mostly use (\ref{ex:asiimwe:8a}--\ref{ex:asiimwe:8b}), as was also observed in \citet[137]{Taylor1985}\footnote{\REF{ex:asiimwe:8b} is a typical Rukiga sentence.}. Besides, in the spoken register, due to phonetic factors, it appears that some speakers may omit the initial morpheme of the demonstrative perhaps to aid production.

\ea%7 
    \label{ex:asiimwe:7} \citet[124]{Asiimwe2014}\\
    Tindíkwenda ki kitabo\\
    \gll ti-n-riku-end-a  \emph{ki}  ki-tabo\\
\textsc{neg}{}-1.\textsc{sbj}{}-\textsc{ipfv}{}-want-\textsc{fv}  7.\textsc{prox}  7-book\\
    \glt ‘I do not want this book.’  (\citealt[151]{MorrisKirwan1972}, glosses added)

\newpage
\ex%8
    \label{ex:asiimwe:8} \citet[181]{Asiimwe2014}\\
    \ea\label{ex:asiimwe:8a} Tindíkwend’ ékitab’ éki\\
    \gll ti-n-riku-end-a    e-ki-tabo  \emph{e-ki}\\
    \textsc{neg-1.sbj-ipfv}{}-want-\textsc{fv}  \textsc{aug-7}{}-book  \textsc{dem-7.prox}\\
    \glt ‘I do not want this book.’

\ex\label{ex:asiimwe:8b}  Ekitab’ eki tíndakyenda \\
    \gll e-ki-tabo  \emph{e-ki}  ti-n-ra-ki-end-a\\
    \textsc{aug-7}{}-book  \textsc{dem-7.prox}  \textsc{neg-1sbj}{}-\textsc{ipfv}-7-want-\textsc{fv}\\
    \glt ‘This book, I do not want it.’
    \z
\z

According to speakers who I worked with for this study, the structures shown in (\ref{ex:asiimwe:8a}--\ref{ex:asiimwe:8b}) are the most natural forms and are the most widely used by speakers.\footnote{I worked with four native speakers of Runyankore and two speakers of Rukiga.} Although some of the speakers I consulted claim that it is acceptable to use a demonstrative without the initial morpheme in spoken discourse, others indicate that without the initial element of the demonstrative (as in \REF{ex:asiimwe:7}), the construction sounds odd.

Note further that, when a noun or its modifier follows the locative element \textit{omu} ‘in’ or \textit{aha} ‘at/on’, the nominal obligatorily loses the augment \citep[88]{Taylor1985}, as shown in \REF{ex:asiimwe:9a}. \footnote{Taylor argues that \textit{omu} and \textit{aha} are prepositions. The idea of a nominal element is adopted from \citet{Asiimwe2014} who argues that these elements have nominal properties (see \citealt[143--145]{Asiimwe2014} for a discussion).} However, when a demonstrative immediately follows either \textit{omu} or \textit{aha} (as in \REF{ex:asiimwe:9b}), the initial element of the demonstrative (which manifests as \textit{e}{}- in \REF{ex:asiimwe:9b}) is retained. In addition, an obligatory suffix -\textit{ri} is added to the locative element.

\ea%9
    \label{ex:asiimwe:9}
    \ea\label{ex:asiimwe:9a}  Engagi nizituur’ ómu kibira\\
    \gll E-n-gagi  ni-zi-tuur-a o-mu  ki-bira\\
    \textsc{aug}{}-10-gorilla  \textsc{ipfv-10.sbj-}live\textsc{{}-fv  aug-}18.in  7-forest\\
  \glt ‘(The) gorillas live in a/the forest.’

  \ex\label{ex:asiimwe:9b}  Engagi nizituur’ ómury’ éki kibira\\
    \gll E-n-gagi  ni-zi-tuur-a  \emph{o-mu-*(ri)}  \emph{e-ki}  ki-bira\\
    \textsc{aug}{}-10-gorilla  \textsc{ipfv-10.sbj-}live\textsc{{}-fv}  \textsc{aug-}18.in-\textsc{suff}  \textsc{dem}{}-7.\textsc{prox}  7-forest\\
  \glt ‘(The) gorillas live in this forest.’
  \z
\z

In addition to the suffix -\textit{ri} being obligatorily added onto \textit{omu/aha} before a demonstrative \REF{ex:asiimwe:9b}, the suffix is required when \textit{omu/aha} precedes nominal elements, which inherently bear no augment such as proper names \REF{ex:asiimwe:10a}, pronouns \REF{ex:asiimwe:10b}, the invariant \textit{buri} ‘every’ \REF{ex:asiimwe:10c} and numerals \REF{ex:asiimwe:10d} (cf. \citealt[88--89]{Taylor1985}). Since the demonstrative requires an obligatory -\textit{ri} suffix on \textit{aha/omu}, it can, therefore, be argued to belong to the same category of nominals which possess no augment.\footnote{The status of the suffix \textit{-ri} remains unclear, and further research is needed to establish its full morphosyntactic properties. The element \textit{-ri} is added when \textit{omu/aha} is immediately followed by a demonstrative, proper name, pronoun, the invariant \textit{buri} ‘every’ and numeral. It may be that it performs a similar function to the augment since nouns belonging to these categories cannot take an augment. See \citet{BeermannAsiimweForthcoming} for further discussion of this.}\textsuperscript{,}\footnote{The locative class 17 -\textit{ku}- is less productive in Run\-yan\-ko\-re-Ru\-ki\-ga (see \citealt{BeermannAsiimweForthcoming}).}

\ea%10
    \label{ex:asiimwe:10}
    \ea\label{ex:asiimwe:10a}  aha*(ri) Kábagaráme\\
    \gll a-ha-ri  Kabagarame\\
    \textsc{aug}{}-16.at-\textsc{suff}  23.Kabagarame\\
    \glt ‘at Kabagarame’

    \ex\label{ex:asiimwe:10b}  omu*(rí)iwe\\
  \gll o-mu-ri  iwe\\
  \textsc{aug-18-suff}  you\\
  \glt ‘in you’

    \ex\label{ex:asiimwe:10c}  omu*(ri) burí kyaro\\
  \gll o-mu-ri  buri  ki-aro\\
\textsc{aug-18-suff}  every  7-village\\
    \glt ‘in every village’

    \ex\label{ex:asiimwe:10d}  aha*(ri) mukáaga\\
  \gll a-ha-ri  mu-kaaga\\
  \textsc{aug-16-suff}  6-six\\
    \glt ‘at six/out of six’
    \z
\z

Further evidence to support the claim that the initial element of the demonstrative \textit{a}{}- is the historical core of the demonstrative, responsible for definiteness meaning, comes from examining the referential (definite) morpheme -\textit{a}, which appears to be the grammaticalised form of this core demonstrative (cf. \citealt{Asiimwe2014,Asiimwe2016}). An appropriate agreement prefix is attached as shown in \REF{ex:asiimwe:11} (but see \citealt[68]{Asiimwe2016} for a list of the forms for noun classes 1--18).

\newpage
\ea%11
    \label{ex:asiimwe:11}
    \ea\label{ex:asiimwe:11a}  gwá muhánda\\
    \gll \emph{gu-a}  mu-hánda\\
    3-\textsc{ref}  3-path\\
    \glt ‘that (other) path’

    \ex\label{ex:asiimwe:11b}  byá bitabo\\
    \gll \emph{bi-a}  bi-tabo \\
    8-\textsc{ref}  8-book\\
    \glt ‘those (other) books’

    \ex\label{ex:asiimwe:11c}  rwá rutindo\\
    \gll     \emph{ru-a}   ru-tindo\\
                11-\textsc{ref}  11-bridge\\
    \glt        ‘that (other) bridge’
    \z
\z

Following \citegen{Diessel1999} characterization of the grammaticalisation process associated with demonstratives, and additional evidence (cf. \citealt{Asiimwe2014,Asiimwe2016}), the demonstrative and the functional element \textit{{}-a} can be seen to share semantic features, meaning that (i) they can be used interchangeably as shown in \REF{ex:asiimwe:11}. Although the medial demonstrative form can be used in the place of -\textit{a} without altering the semantics, it is mostly the distal demonstrative that can replace -\textit{a} to locate a referent that is assumed to be accessible in the mind of the hearer. (ii) Like the demonstrative, the morpheme \textit{{}-a} can be used as a tracking device for a referent already mentioned in the previous discourse. A medial demonstrative (\textit{ogwo}) can replace \textit{wa} as in \REF{ex:asiimwe:12} without altering the semantics of the discourse (this function of the demonstrative is discussed in \sectref{sec:asiimwe:4.1}). (iii) The morpheme \textit{{}-a} shares with the demonstrative the feature of denoting a referent about which both the speaker and the hearer have common knowledge (as in \REF{ex:asiimwe:13}).

The context for both \REF{ex:asiimwe:13a} and \REF{ex:asiimwe:13b} is that speaker (A) asks the hearer (B) whether B got the book which A asked for, a referent that is assumed to have been talked about by A and B previously.\footnote{Note that where Runyankore speakers use the distal demonstrative e.g., \textit{ka-ri(ya)} (see \tabref{tab:asiimwe:1}) Rukiga speakers use \textit{-a} (e.g., \textit{kariya katabo} versus \textit{ka katabo}).}

\ea%12
    \label{ex:asiimwe:12} \citet[215]{Asiimwe2014}

          Enkundi ku erikumara kuragara omuzaire naaruga aha kiriri, \emph{omwana} nibamushohoza   aheeru…Nibaronda omwojo n’omwishiki b’omuka endiijo, reero omwishiki naaheeka \emph{wa}   \emph{mwereere} naagyenda n’ogwo mwojo ou baija hamwe nibaza omu kishaka \ldots \citep[109]{Karwemera1994}\footnote{In the interest of space, long excerpts are not glossed.}

‘\ldots\, After the umbilical cord has fallen off, the days for the mother to remain in the house with the child (after the child has been born) are over, and the \emph{child} is taken outside…they get a boy and   a girl from another family and the girl carries \emph{that} \emph{baby} on her back and goes with that boy, whom she has come with, to the bush \ldots'
\z

\ea%13
    \label{ex:asiimwe:13}
    \ea\label{ex:asiimwe:13a}  Tindíkumanya yáába ká katabo waakantúngîire\\
    \gll ti-n-riku-many-a  yaaba  \emph{k-a}  ka-tabo  w-aa-ka-n-túngi-ire\\
    \textsc{neg-1.sbj-ipfv}{}-know-\textsc{fv}  whether  12-\textsc{ref}  12-book  \textsc{2sg-pst-12-1sg}{}-find-\textsc{pfv}\\
  \glt ‘I do not know whether you got for me that other (small) book.’

\ex\label{ex:asiimwe:13b}  \citet[205]{Asiimwe2014}\\
Tindíkumanya yáába \emph{kárí(ya)} (á)katabo waakantúngîire\\
  \gll ti-n-riku-many-a  yaaba  \emph{ka-ri(ya)}  (a)-ka-tabo  w-aa-ka-n-tungi-ire\\
  \textsc{neg-1.sbj-ipfv}{}-know-\textsc{fv}  whether  12-\textsc{dist}  \textsc{aug}{}-12-book  \textsc{2sg-pst-12-1sg}{}-find-\textsc{pfv}\\
  \glt ‘I do not know whether you got for me that other (small) book.’
    \z
\z

In the phrase \textit{wa mwereere}, ‘that baby’, the distal demonstrative \textit{oriya} can be used in the place of \textit{wa}: \textit{oriya mwereere} ‘that baby’. The two phonological words \textit{wa} and \textit{oriya}, in this case, are in complementary distribution. This shows that they play the same role and, perhaps, the morpheme \textit{{}-a} evolved from the demonstrative core morpheme \textit{a-}.
%Therefore, the initial element of the demonstrative is, in the current study, considered to be the core element in the structure of the demonstrative.

\subsection{Proximal, medial and distal demonstratives}\label{sec:asiimwe:2.2}

Many Bantu languages have a three, or four-way system of demonstratives, expressing the distance of the speaker or hearer (\citealt{Nicolle2012, VandeVelde2019}) from the deictic center. Run\-yan\-ko\-re-Ru\-ki\-ga typically follows a three-way demonstrative system marking the referent close to speaker (proximal), close to the hearer (medial) and far from both the speaker and hearer (distal). Also, most Bantu languages have retained the proximal suffix -\textit{no} or its cognate -\textit{nu} from Proto-Bantu (\citealt{AshtonEtAl1954, Nicolle2012, AhnvanderWal2019}). Run\-yan\-ko\-re-Ru\-ki\-ga maintains the suffix -\textit{nu} commonly used as an anaphoric demonstrative in narratives (\sectref{sec:asiimwe:2.3}). The proximal demonstrative, which typically reflects the position of a referent close to the speaker, has no overt morphological marker for distance as shown in \REF{ex:asiimwe:14a} (also see \tabref{tab:asiimwe:1}). The medial demonstrative, also known as the demonstrative of reference, which refers to an entity near the hearer has the suffix -\textit{o} \REF{ex:asiimwe:14b}. This suffix is also found in many other Bantu languages (cf. \citealt{AshtonEtAl1954, Nicolle2012,Nicolle2014, AhnvanderWal2019}, among others).

\ea%14
    \label{ex:asiimwe:14}
    \ea\label{ex:asiimwe:14a}  eki kibira\\
    \gll \emph{e-ki}  ki-bira\\
    \textsc{dem}{}-7.\textsc{prox}  7-forest\\
  \glt ‘this forest.’

  \ex\label{ex:asiimwe:14b} ekyo ki-bira\\
    \gll \emph{e-ki-o}  ki-bira\\
    \textsc{dem}{}-7-\textsc{med}  7-forest\\
  \glt ‘that forest.’
    \z
\z

According to \citet[135]{Taylor1985}, the distal demonstrative is further divided between reference to visible and invisible entities. For distant objects, which are nevertheless visible to the speaker and hearer, Taylor observes that the suffix \emph{{}-}\textit{riya} is used, while for distant and invisible entities, he argues that \textit{{}-ri} is used. However, in contrast to \citegen{Taylor1985} claim, -\textit{ri} and -\textit{riya} are here considered to be variants of the same form (compare \REF{ex:asiimwe:15} and \REF{ex:asiimwe:16} but see also \tabref{tab:asiimwe:1}). They are used interchangeably for both visible and invisible referents and the use of either of the forms depends mainly on an individual’s choice \citep{Asiimwe2014}.\footnote{Note that with referents which are far from the speaker and the hearer, the core morpheme of the demonstrative is morphologically unmarked.} Example \REF{ex:asiimwe:15} shows that a speaker can choose either the -\textit{ri} or its variant -\textit{riya} for both visible and invisible referents. It should also be noted that referents in noun classes 1 and 9 only occur with the long form (-\textit{riya}) as exemplified in \REF{ex:asiimwe:17a} with a class 9 noun. Note further that the same noun in the plural form (class 10) permits either the short or long form of the distal demonstrative \REF{ex:asiimwe:17b}. Nouns in classes 1 and 9 have only a vowel as their demonstrative concord while the rest use a consonant plus a vowel, which might be the reason why the two noun classes behave differently with the distal demonstrative forms.

\ea%15
    \label{ex:asiimwe:15}
    Bárí/riya báâna nibarónd’ oha?\\
  \gll \emph{∅{}-ba-ri/riya}  ba-ana  ni-ba-rond-a  o-ha?\\
  \textsc{dem-}2-\textsc{dist}  2-child  \textsc{prog-2.sbj}{}-look-\textsc{fv}  1-who\\
  \glt ‘Who are those children looking for?’
\ex%16
    \label{ex:asiimwe:16} \citet[184]{Asiimwe2014}\\
     Omwishiki oú yaashabíre akamugira ati “Shaná wááza kunshwéra obanze óité báríya báána baawe.”  \citep[20]{Karwemera1975}\\
  \gll o-mu-ishiki  o-u  y-aa-shab-ire  a-ka-mu-gir-a  a-ti  “Shana w-aa-za  ku-n-shwer-a  o-banz-e  o-it-e  ∅{}-\emph{ba-riya}  ba-ana  ba-awe \\
\textsc{Aug}{}-1-girl  \textsc{aug-rel.pro}  \textsc{3.sbj-pst}{}-ask-\textsc{pfv}  \textsc{3.sbj-rem}{}-\textsc{1om}{}-say-\textsc{fv}  1-that  May.be 1-\textsc{prs}{}-go  \textsc{inf-1.sbj}{}-marry-\textsc{fv}  \textsc{2sg.sbj}{}-first-\textsc{sbjv}  \textsc{sg}{}-kill-\textsc{sbjv}  \textsc{dem-2-dist}  2-child  2-\textsc{conn}.yours\\
    \glt ‘The girl whom he asked for a hand in marriage told him ‘If you are to marry me, you will first kill those children of yours’.
\ex%17
    \label{ex:asiimwe:17}
    \ea\label{ex:asiimwe:17a}  Embúz’ érí*(ya)\\
    \gll e-n-buzi   ∅{}-\emph{e-ri*(ya)}\\
    \textsc{aug-}9-goat  \textsc{dem-}9-\textsc{dist}\\
    \glt ‘that goat (over there)’

    \ex\label{ex:asiimwe:17b}  Embúzi zíri(ya)\\
    \gll e-n-buzi  ∅{}-\emph{zi-ri(ya)}\\
    \textsc{aug-}10-goat  \textsc{dem-}10-\textsc{dist}\\
    \glt ‘those goats (over there)’
    \z
\z

For referents that are far from the speaker but visible to both the speaker and the hearer, especially in spoken discourse, the final -\textit{a} may be lengthened and pronounced with a high tone for emphasis, as in \REF{ex:asiimwe:18}--\REF{ex:asiimwe:19}.

\ea%18
    \label{ex:asiimwe:18}
    enté ériyááá                  \\
 \gll e-n-te  \emph{∅{}-e-riya}\\
  \textsc{aug}{}-9-cow  \textsc{dem-}9-\textsc{dist}\\ \jambox{\citep[136--137]{Taylor1985}}
\glt ‘that cow right over there’
\ex%19
    \label{ex:asiimwe:19} \citet[184]{Asiimwe2014}\\
    Réébá ékintu kiríkugamba nkírí múríyááá.            \\
\gll ∅-reeb-a  e-ki-ntu  ki-riku-gamb-a  n-ki-ri  \emph{∅{}-mu-riya}\\
  \textsc{2pl-}see-\textsc{fv}  \textsc{aug}{}-7-thing  \textsc{7sbj-prog}{}-talk-\textsc{fv}  \textsc{ld.cop}{}-7-\textsc{dist}  \textsc{dem-16-dist}\\ \jambox{\citep[28]{Mubangizi1966}}
\glt ‘Look, the creature that is making some sound is there, in there (far but visible).’
\z

In contrast, among the speakers of Rukiga, it is the vowel /i/ of the demonstrative suffix that may be lengthened and also pronounced with a high tone to refer to an object that is far but visible: e.g. \textit{múrííya} ‘in there’. The lengthening of the vowel is at times accompanied by pouted lips (cf. \citealt[59]{MorrisKirwan1972}) or a pointing gesture for extra emphasis. \tabref{tab:asiimwe:1} gives a summary of the forms of the demonstratives indicating the distance of the speaker or hearer from the deictic centre.

\begin{table}
\begin{tabular}{lllll}
\lsptoprule
\multicolumn{2}{c}{Noun class} & Proximal (this) & Medial (that) & Distal (that) †\\
\midrule
1 & -mu- & ogu & ogwo & oriya\\
2 & -ba- & aba & abo & bari/bariya\\
3 & -mu- & ogu & ogwo & guri/guriya\\
4 & -mi- & egi & egyo & giri/giriya\\
5 & -ri- & eri & eryo & riri/ririya\\
6 & -ma- & aga & ago & gari/gariya\\
7 & -ki- & eki & ekyo & kiri/kiriya\\
8 & -bi- & ebi & ebyo & biri/biriya\\
9 & -n- & egi & egyo & eriya\\
10 & -n- & ezi & ezo & ziri/ziriya\\
11 & -ru- & oru & orwo & ruri/ruriya\\
12 & -ka- & aka & ako & kari/kariya\\
13 & -tu- & otu & otyo & turi/turiya\\
14 & -bu- & obu & obwo & buri/buriya\\
15 & -ku- & oku & okwo & kuri/kuriya\\
16 & -ha- & aha & aho & hari/hariya\\
17 & -ku- & oku & okwo & kuri/kuriya\\
18 & -mu- & omu & omwo & muri/muriya\\
20 & -gu- & ogu & ogwo & guri/guriya\\
21 & -ga- & aga & ago & gari/gariya\\
\lspbottomrule
\end{tabular}
\caption{Demonstratives in Run\-yan\-ko\-re-Ru\-ki\-ga adapted from \citet[136]{Taylor1985}. †: Visible and invisible referents.}
\label{tab:asiimwe:1}
\end{table}

\subsection{The demonstrative suffix \textit{{}-nu}}\label{sec:asiimwe:2.3}

The demonstrative may also take the suffix form \textit{{}-nu} to which an appropriate noun class prefix is attached. The -\textit{nu} form is commonly used in storytelling to refer to a referent that has already been mentioned (cf. \citealt[59]{MorrisKirwan1972}; also see \citealt{Nicolle2012, Nicolle2014} for Digo). It is usually used in narratives with third-person human referents. Also, -\textit{nu} is used with personified nouns in folktales. The literal translation of -\textit{nu} is ‘this one’, hence a proximal form of the demonstrative. This can be seen in example \REF{ex:asiimwe:20}.

\ea%20
    \label{ex:asiimwe:20} \citet[186]{Asiimwe2014}\\
    Omukáma ayeta ómwigárire Nyakwégira. Kú áija ámubuuza atí “Ógu níwe shó ékiti?” \emph{Onú} ati “Buzima niwé tatá ékiti” \ldots \citep[31]{Mubangizi1966}\\
    \gll o-mu-kama  a-et-a  o-mu-igarire  Nyakwegira.  Ku  a-ij-a a-mu-buuz-a  a-ti  “o-gu  ni-w-e  sho    ekiti?” \emph{o-nu}  a-ti  “Buzima  ni-w-e  tata  ekiti\\
\textsc{aug}{}-1-king  \textsc{1.sbj}{}-call-\textsc{fv}  \textsc{aug}{}-1-princess  1.Nyakwegire  When  \textsc{1sbj-}come-\textsc{fv} \textsc{3.sbj-2om}{}-ask-\textsc{fv}  1-that  \textsc{dem}{}-1.\textsc{prox}  \textsc{cop}{}-1-\textsc{rel.pro}  father.your  real 1-\textsc{prox}  1-that  true  \textsc{cop-1-rel.pro}  1.father  real\\
\glt ‘The king called the princess. And when she came, he asked her “Is this your real father?” Then this one (the princess) answered: “Yes he is my real father”.’
\z

\subsection{Identificational demonstrative \textit{n-}}\label{sec:asiimwe:2.4}

Another demonstrative in Run\-yan\-ko\-re-Ru\-ki\-ga is formed by the nasal \textit{n-}\footnote{This form of the demonstrative should not be confused with the locative demonstrative discussed in \sectref{sec:asiimwe:2.5}.} which appears before the noun class prefix. \citet[138]{Taylor1985} refers to this form as an emphatic demonstrative which means ‘here she/he/it is’. I call this morpheme a locative demonstrative copulative because it marks the location of a referent with a copulative sense. This form is also found in Runyoro-Rutooro (JE.11-JE.12), a language cluster that is closely related Run\-yan\-ko\-re-Ru\-ki\-ga. For Runyoro-Rutooro, \citet{Rubongoya1999} identifies this form as a nasal morpheme that is used for things that are visible and specified. IsiXhosa (S.41), a Bantu language of South Africa, also has this form of demonstrative (\citealt{DuPlessisEtAl1992}), performing the same function as in Run\-yan\-ko\-re-Ru\-ki\-ga. In Run\-yan\-ko\-re-Ru\-ki\-ga, the locative demonstrative copulative morpheme \textit{n-} occurs in all the three demonstrative divisions (\ref{ex:asiimwe:21a}--\ref{ex:asiimwe:21c}).

 \ea\label{ex:asiimwe:21}
    \ea\label{ex:asiimwe:21a}   Amíízi ngága\\
  \gll A-ma-izi  \emph{n-ga-ga}\\
  \textsc{aug}{}-6-water  \textsc{ld.cop}{}-6-6\\
  \glt ‘Here is the water.’

  \ex\label{ex:asiimwe:21b}   Amíízi ngágo\\
  \gll A-ma-izi  \emph{n-ga-ga-o}\\
  \textsc{aug}{}-6-water  \textsc{ld.cop}{}-6-6-\textsc{med}\\
  \glt `The water, there it is.'

  \ex\label{ex:asiimwe:21c}  Amíízi ngárí(ya)\\
 \gll  a-ma-izi  \emph{n-ga-ri(ya)}\\
  \textsc{aug}{}-6-water  \textsc{ld.cop-6-dist}\\
  \glt ‘There is the water (far from speaker and hearer but visible).’

  \ex\label{ex:asiimwe:21d} Amíízi ngagári \\
  \gll a-ma-izi  \emph{n-ga-ga-ri}\\
  \textsc{aug-}6-water  \textsc{ld.cop-6-6-dist}\\
  \glt ‘There is the water (far from speaker and hearer but visible).’
  \z
\z

Observe that with demonstrative locative copulatives, the noun class concord is doubled in the proximal and medial demonstrative types (see \tabref{tab:asiimwe:2} below) and typically not in the distal although in some varieties of Rukiga, the class prefix is duplicated in the distal demonstrative \REF{ex:asiimwe:21d} as well. However, it remains unclear why exactly the noun class concord is duplicated. It might also be argued that one of the two identical morphemes performs a function other than noun class concord. However, this is a question that needs to be investigated further. The prefix \textit{n}{}- has a locative meaning and as such directs the addressee to a more specific location of a referent and an appropriate gesture can accompany this form.

\begin{table}
\begin{tabularx}{\textwidth}{lQQQ}

\lsptoprule

Noun class & Proximal & Medial & Distal\\
\midrule
1 -mu- & ngugu & ngugwo & nguri(ya)\\
2 -ba- & mbaba (n-ba-ba) & nbabo & mbari(ya)\\
3-mu- & ngugu & ngugwo & nguri(ya)\\
4 -mi- & ngigi & ngigyo & ngiri(ya)\\
5 -i/ri- & ndiri (n-ri-ri) & ndiryo (n-ri-ri-o) & ndiri(ya) (n-ri-ri(ya))\\
6 -ma- & ngaga & ngago & ngari(ya)\\
7 -ki- & nkiki & nkikyo (n-ki-ki-o) & nkiri(ya)\\
8 -bi- & mbibi (n-bi-bi) & mbibyo (n-bi-bi-o & mbiri(ya)\\
9 -n- & ngigi & ngigyo (n-gi-gi-o) & ngiri(ya)\\
10 -n- & nzizi & nzizo (n-zi-zi-o) & nziri(ya)\\
11 -ru- & nduru (n-ru-ru) & nduryo (n-ru-ru-o) & nduri(ya)\\
12 -ka- & nkaka & nkako & nkari(ya)\\
13 -tu- & ntutu & ntutyo (n-tu-tu-o) & nturi(ya)\\
14 -bu- & mbubu (n-bu-bu) & mbubwo (n-bu-bu-o & mburi(ya)\\
15 -ku- & nkuku & nkukwo (n-ku-ku-o) & nkuri(ya)\\
16 -ha- & mpaha (n-ha-ha) & mpaho (n-ha-ha-o) & Mpariya (n-ha-ri(ya))\\
17 -ku- & {}- & - & {}- \\
18 -mu- & {}- & - & {}-\\
\lspbottomrule
\end{tabularx}
\caption{The locative demonstrative copulative  (adapted from \citealt[187]{Asiimwe2014})}
\label{tab:asiimwe:2}
\end{table}

The locative demonstrative copulative \textit{n-} is used to locate entities which may be seen or have been referred to previously in a more specific place. Concerning the locative noun classes, the identificational morpheme \textit{n-} can only combine with \textit{ha} (class 16), e.g., \textit{mpaha (n-ha-ha)} ‘Here it (place) is’ but not \textit{ku} (class 17), \textit{*Nkuku} or \textit{mu} (class 18): *\textit{mmumu (n-mu-mu).}

\subsection{Locative demonstratives}\label{sec:asiimwe:2.5}

Locative constructions in Bantu have generally attracted a lot of interest due to their morpho-syntactic versatility (see for example, \citealt{BresnanKanerva1989, Cocchi2000, Marten2012, BloomStröm2015, Zeller2013, Zeller2017}). Run\-yan\-ko\-re-Ru\-ki\-ga has three locative noun classes: \textit{ha} (class 16), \textit{ku} (class 17) and \textit{mu} (class 18). The locative prefix \textit{ha} marks a specific place, \textit{ku} indicates a general location, while \textit{mu} signals an internal location (\citealt{Taylor1985, Asiimwe2014, BeermannAsiimweForthcoming}). Generally, the class 16 locative noun prefix \textit{ha}{}- is more productive than the other two classes, as it is the only prefix that can be attached to verbs and to most nominal modifiers (cf. \citealt{Asiimwe2014}). Nevertheless, locative demonstratives can be formed with all three classes. There are also different realisations of the proximal demonstrative which are used for more specific versus non-specific locations while there are no variants for the medial locative demonstrative. As for the distal demonstrative, a more specific location is marked by lengthening the vowel /i/ of the suffix especially in the Rukiga dialects. The different forms of locative demonstratives are given in \tabref{tab:asiimwe:3}.

\begin{table}
\begin{tabularx}{\textwidth}{QQQQQ>{\raggedright\arraybackslash}p{.2\textwidth}}

\lsptoprule

\multicolumn{2}{l}{Proximal} & \multicolumn{2}{l}{Medial}  & \multicolumn{2}{l}{Distal}\\
\midrule
aha\slash hanu\slash hanuuya & ‘here’ & aho & ‘there’ & hari(ya)\slash hariiya & ‘there (‘at’ place not close to both speaker and hearer’\\
\tablevspace
oku\slash kunu\slash kunuuya & ‘this side\slash place’ & okwo\slash okwe\footnote{There are dialectal differences; the variants \textit{okwe/omwe/muriiya/kuriiya/hariiya} are commonly used in Rukiga.} & ‘that side\slash place’ & kuri(ya)\slash kuriiya & ‘there (wider (unspecified place)’\\
\tablevspace
omu\slash munu\slash munuuya & ‘in here’ & omwo\slash omwe & ‘in there’ & muriya\slash muriiya & ‘in there (an ‘in-location’ far from speaker and hearer)’\\
\lspbottomrule
\end{tabularx}
\caption{Locative demonstratives in Run\-yan\-ko\-re-Ru\-ki\-ga}
\label{tab:asiimwe:3}
\end{table}

\largerpage
Among the forms given in \tabref{tab:asiimwe:3}, there are those which mark a place that is specific or very close to the speaker. These include the following: \textit{hanu}/\textit{hanuuya}, \textit{kunu}/\textit{kunuuya} and \textit{munu}/\textit{munuuya} \REF{ex:asiimwe:22}, and these contrast with the regular forms \textit{aha,} \textit{oku,} and \textit{omu}. Locative demonstratives that denote a more specific location are formed by removing the augment and attaching the suffix -\textit{nu(uya)} to the locative prefix with either a long /u/ or /i/ depending on the dialect. Speakers of Rukiga mostly lengthen the vowel of the suffix \textit{{}-nuuya or -riiya} for the proximal or distal demonstrative respectively (compare \REF{ex:asiimwe:22a} and \REF{ex:asiimwe:22b}).

\ea%22
    \label{ex:asiimwe:22}
    \ea\label{ex:asiimwe:22a}  Amíízi gashuké múnúúya/múríiya        Rukiga\\
    \gll a-ma-izi  ga-shuk-e  \emph{mu-nuuya/mu-riiya}\\
\textsc{aug}{}-6-water  6\textsc{sbj}-pour-\textsc{sbjv}  18-here/18-there\\
\glt ‘(You) Pour the water (exactly) in here/there.’

  \ex\label{ex:asiimwe:22b}   Amíízi gashuké múnu/múríya          Runyankore\\
    \gll a-ma-izi  ga-shuk-e  \emph{mu-nu/mu-riya}\\
\textsc{aug}{}-6-water  \textsc{6sbj-}pour-\textsc{sbjv}  18-here/18-there\\
\glt ‘(You) Pour the water (exactly) in here/there.’
    \z
\z

Locative demonstratives are sometimes categorized as demonstrative adverbs \citep{Dixon2003} (also see \sectref{sec:asiimwe:3.5}). As adverbs, they can appear after the verb indicating a location described by the verb \REF{ex:asiimwe:23}.  They can also appear as locative demonstrative pronouns as shown in \REF{ex:asiimwe:24}.  For future research, a detailed analysis of the categorial status of locative demonstratives is recommended.

\ea%23
    \label{ex:asiimwe:23} Engagi zaarabá áha\\
\gll e-n-gagi  z-aa-rab-a  \emph{a-ha}\\
  \textsc{aug}{}-10-gorilla  \textsc{10.sbj-n.pst-fv}  \textsc{aug}{}-here\\
\glt ‘(The) Gorillas have passed here.’
\z

\ea%24
    \label{ex:asiimwe:24}
    Waagira ngu nookorerá Mbarara? Mbarará ókwo toríkubáása kúmpangirayó ákarimo kúnu kumburíírwe?\\
 \gll  w-aa-gir-a  ngu  ni-o-kor-er-a  Mbarara?  Mbarara o-ku-o  ti-o-riku-baas-a  ku-n-hang-ir-a=yo  a-ka-rimo \emph{ku-nu}  ku  n-buri-ir-w-e?\\
2\textsc{sg-n.pst-}say-\textsc{fv}  that  \textsc{ipfv}{}-\textsc{2sg.sbj-}work-\textsc{appl-fv}  23.Mbarara.  23.Mbarara  \textsc{dem-17-med}  \textsc{neg-2sg.sbj-ipfv}{}-can-\textsc{fv}  \textsc{inf-1sg}{}-fodge-\textsc{apl-fv}=23  \textsc{aug}{}-12-job 17-this.side  but  1\textsc{sg}{}-fail-\textsc{appl-pass-pfv}\\
\glt ‘You have said that you work from Mbarara, right? Is it possible for you to find a small job for me there in Mbarara as I have failed to find one this side?’
\z

\subsection{The manner demonstrative -\textit{ti}}\label{sec:asiimwe:2.6}

Another form of the demonstrative found in Run\-yan\-ko\-re-Ru\-ki\-ga is -\textit{ti} which expresses the manner in which something is done or perceived. This category is close to what \citet{Dixon2003} identifies as verbal demonstrative and it has both nominal and verbal properties. As a nominal demonstrative, it can be used pronominally or adnominally for either proximal or medial location of a referent. It does not refer to entities far from both speaker and hearer. As a verbal demonstrative, it combines with personal pronouns as shown in \REF{ex:asiimwe:25}. This form expresses manner in a similar way to that discussed by \citet{Guérin2015}, also in relation to the meaning ‘do like this’ suggested in \citet[72]{Dixon2003}. In terms of morphology, it appears as a suffix to which an appropriate noun class concord is attached. As mentioned already, the manner demonstrative modifies nominals \REF{ex:asiimwe:26a} and predicates \REF{ex:asiimwe:26b}

\ea%25
    \label{ex:asiimwe:25}
 \glllllll {}  {Proximal (\textit{like this})}  {medial  (\textit{like that})}\\
    {1\textsuperscript{st}  pers.sg.}    n-ti      n-ty-o/n-sy-o\footnotemark{}\\
    {1\textsuperscript{st}  pers. pl.}    tu-ti      tu-ty-o/tu-sy-o\\
    {2\textsuperscript{nd}  pers.sg.}     o-ti      o-ty-o/o-sy-o\\
    {2\textsuperscript{nd} pers. pl.}    mu-ti      mu-ty-o/mu-syo\\
    {3\textsuperscript{rd} pers. sg. human\footnotemark{}}   a-ti       a-ty-o/a-sy-o\\
    {3\textsuperscript{rd} pers. pl. human}   ba-ti      ba-ty-o/ba-sy-o\\
\z
\footnotetext{The form -\textit{syo} is commonly used in Rukiga.}
\footnotetext{For non-human entities, the shape of the prefix takes the concordial form of the modified head noun.}

\ea%26
    \label{ex:asiimwe:26}
    \ea\label{ex:asiimwe:26a}  Context: In a visit to a zoo, one sees a three-horned chameleon for the first time. \\
    Enyarujw’ etí tínkagíreebahóga!\\
    \gll e-nyaruju  \emph{e-ti}  ti-n-ka-gi-reeb-a=ho=ga\\
\textsc{aug-}9.chameleon  9-\textsc{prox}  \textsc{neg-1sg.sbj-asp-9.om-}see-\textsc{fv=part=}never\\
\glt ‘I have never seen a chameleon like this one.’

  \ex\label{ex:asiimwe:26b} Enyarujú neetambúra eti\\
\gll e-nyaruju  ni-e-tambur-a  e-ti\\
\textsc{aug}{}-9.chameleon  \textsc{ipfv-}9-walk-\textsc{fv}  9-\textsc{prox}\\
    \glt ‘A chameleon moves likes this (while demonstrating the way a chameleon moves)’
    \z
\z

Unlike the basic forms of the demonstrative discussed in \sectref{sec:asiimwe:2.2}, which can either be pre- or postposed, the demonstrative -\textit{ti} strictly follows the noun \REF{ex:asiimwe:27} or verb \REF{ex:asiimwe:28} it modifies. In addition, as an adnominal or pronominal demonstrative, it can be replaced by a basic demonstrative as shown \REF{ex:asiimwe:29}.

\ea%27
    \label{ex:asiimwe:27}
    \ea[]{Ondééteré ékitabo kítí\\
    \gll o-n-reet-er-e  e-ki-tabo  ki-tí\\
    \textsc{2sg.om-1sg}{}-bring-\textsc{appl-sbjv}  \textsc{aug}{}-7-book  7-\textsc{prox}\\
    \glt ‘(You) bring for me a book like this (one).’}\label{ex:asiimwe:27a}

  \ex[*]{Ondeetere ki-ti e-ki-tabo\\
    \gll o-n-reet-er-e  ki-ti    e-ki-tabo\\
    \textsc{2sg.sbj-1sg}{}-bring-\textsc{appl}{}-\textsc{sbjv}  7-\textsc{prox}  \textsc{aug}{}-7-book\\
    \glt ‘(You) bring for me a book like this (one).’}\label{ex:asiimwe:27b}

  \ex[]{Ondéétere kítí\\
    \gll o-n-reet-er-e  ki-tí\\
    \textsc{2sg.sbj-1sg}{}-bring-\textsc{appl-sbjv}  7-\textsc{prox} \\
    \glt ‘(You) bring for me like this (one).’}\label{ex:asiimwe:27c}
    \z
\z

\ea%28
    \label{ex:asiimwe:28}
    \ea[]{Abarungí bakora  bátyo\\
    \gll a-ba-rungi  ba-kor-a  ba-ty-o\\
    \textsc{aug}{}-2-good/nice  \textsc{2sbj-}do-\textsc{fv}  2-like.that-\textsc{med}\\
    \glt ‘Lit. ‘Good people do like that.’ ‘Good people behave like that’.’}\label{ex:asiimwe:28a}

  \ex[*]{\gll A-ba-rungi  ba-ty-o  ba-kor-a\\
    \textsc{aug}{}-2-good/nice  2-like.that-\textsc{med}  \textsc{2sbj-}do-\textsc{fv}\\ }\label{ex:asiimwe:28b}
    \z
\z

\newpage
\ea%29
    \label{ex:asiimwe:29}
    Ondééteré ékitabo nk’éki\\
 \gll o-n-reet-er-e    e-ki-tabo  nk’-e-ki\\
  \textsc{2sg.sbj-1sg}{}-bring-\textsc{appl-sbjv}  \textsc{aug}{}-7-book  like-\textsc{dem}{}-7.\textsc{prox}\\
    \glt ‘(You) bring for me a book like this (one).’
\z

Generally, the manner demonstrative has both endophoric and exophoric uses. It can be used for referents in the physical environment and its anchor is an entity, an action, a situation or an activity in a conversation or in a previous text.

So far, we have seen different forms of demonstratives found in Run\-yan\-ko\-re-Ru\-ki\-ga. The next section discusses the syntax of demonstratives, focusing on the position of the demonstrative in relation to the noun it modifies and its distribution in the verbal domain.

\section{Syntax of Run\-yan\-ko\-re-Ru\-ki\-ga demonstratives}
\label{sec:asiimwe:3}

\subsection{Position of the demonstrative in the NP}\label{sec:asiimwe:3.1}

A demonstrative, like most other nominal modifiers in Bantu languages, generally shows concord with the noun it occurs with either as an adnominal or predicative demonstrative. In terms of position in the nominal domain, nominal modifiers in African languages generally follow the noun (c.f., \citealt{Zeller2020}). However, there is considerable flexibility among some nominal modifiers including demonstratives.  \citet{VandeVelde2005} categorises Bantu languages into three on the basis of the position of the adnominal demonstrative: (i) Bantu languages in which the adnominal demonstrative is always preposed, (ii) languages where the demonstrative is postposed and (iii) languages that present a free word order between the noun and the demonstrative. Some of the Bantu languages of East Africa from Van de Velde’s list which allow only postnominal demonstratives include Hema (JE.10), Ganda (JE.15), and Haya (JE.22).\footnote{\textrm{Makhuwa (P.31) also allows only postnominal demonstratives (\citealt{vanderWal2010}).}} However, (Lu)Ganda (JE.15), which is spoken in Uganda and neighbours Run\-yan\-ko\-re-Ru\-ki\-ga, allows a preposed demonstrative as well (Kawalya p.c.) as shown in \REF{ex:asiimwe:30}. \citet{VandeVelde2005} describes Rundi (JE.62) as a language which seems to allow only the prenominal demonstrative. Outside the Van de Velde’s sample, \citet{Gunnink2018} shows that the demonstrative in Fwe (K.402) is preferred in the prenominal position. In Chiyao, the demonstrative is canonically postnominal. In this language,  according to \textcitetv{chapters/taji} a prenominal demonstrative can only be allowed if it co-occurs with a reduced form of the demonstrative (that occurs without the initial element) in the postnominal position  as illustrated in \REF{ex:asiimwe:31}. Run\-yan\-ko\-re-Ru\-ki\-ga falls in the category of languages which allow free word order between the noun and the demonstrative (\citealt{Taylor1985, DeBlois1970, VandeVelde2005, Asiimwe2014}). Other Bantu languages spoken in Uganda which allow the demonstrative in the prenominal and postnominal position include Lugwere \citep{AhnvanderWal2019} and Runyoro-Rutooro \citep{Rubongoya1999}.

\ea%30
    \label{ex:asiimwe:30}Luganda (Kawalya p.c.)\\
    Bano *(á)bávúbuka abámánsámánsá ssénte baziggya wa?\\
 \gll ba-no  *(a)-ba-vubuka  a-ba-mansa-mansa  ssente  ba-zi-ggy-a  wa\\
2-\textsc{prox}  \textsc{aug}{}-2-youth  \textsc{aug}{}-2.\textsc{rel}{}-scatter-scatter
10.money  \textsc{2sbj-}10\textsc{om}{}-get-\textsc{fv}  where\\
    \glt ‘These youths who spend money lavishly, where do they get it from?
\z


\ea%31
    \label{ex:asiimwe:31} Chiyao (\textcitetv{chapters/taji})\\
    \ea[]{\gll \emph{aú}  m-gundá  \emph{u}\\
  3.\textsc{dem.prox.}  3-farm  3.\textsc{dem.prox}\\
    \glt ‘this farm’}\label{ex:asiimwe:31a}

    \ex[*]{\gll \emph{aú}  m-gundá\\
3.\textsc{dem.prox}  3-farm  \\
    \glt ‘this farm’}\label{ex:asiimwe:31b}
    \z
\z



There is no consensus in terms of the canonical position of the adnominal demonstrative in Run\-yan\-ko\-re-Ru\-ki\-ga. According to \citet{Wald1973}, the demonstrative in Run\-yan\-ko\-re-Ru\-ki\-ga was historically prenominal. In contrast, \citet{Taylor1985} considers the demonstrative in Run\-yan\-ko\-re-Ru\-ki\-ga to be canonically a postnominal modifier, which can, however, precede the noun for emphasis \citep[89]{Taylor1985}. Synchronically, nominal modifiers in Run\-yan\-ko\-re-Ru\-ki\-ga apart from the quantifier \textit{buri} ‘every’ typically occur in the postnominal position (confirming \citegen{Zeller2020} observation) although can freely move to the prenominal position. The demonstrative may either precede or follow the head noun as the examples in \REF{ex:asiimwe:32} show. Different factors may influence the position of the demonstrative in the nominal domain, including individuals’ preference, dialectal variation, genre or register, and spoken versus written discourse (cf. \citealt{Asiimwe2014} for further discussion of this). The position of the demonstrative may not influence the role the demonstrative plays in the nominal domain. However, this needs to be investigated, putting into consideration that information structure, for example, might influence word order in the nominal domain.

\ea%32
    \label{ex:asiimwe:32}
    \ea\label{ex:asiimwe:32a}  Ndéétera ébiníka égyo!\\
 \gll N-reet-er-a  e-binika  \emph{e-gi-o}\\
  1\textsc{sg.sbj}{}-bring-\textsc{appl}-\textsc{fv}  \textsc{aug}{}-9.kettle  \textsc{dem}{}-9-\textsc{prox}  \\
    \glt ‘Bring me that kettle!’

  \ex\label{ex:asiimwe:32b}  Ndéétera égyo biníka!\\
 \gll N-reet-er-a  \emph{e-gi-o}  binika\\
  1\textsc{sg.sbj}{}-bring-\textsc{appl}-\textsc{fv}  \textsc{dem}{}-9-\textsc{prox}  9.kettle\\
  \glt ‘Bring me that kettle!’
    \z
\z

The example in \REF{ex:asiimwe:33} is extracted from a weekly local Run\-yan\-ko\-re-Ru\-ki\-ga newspaper called \textit{Orumuri.}\footnote{Publication of the newspaper was suspended in May 2020.} In this newspaper, discourse demonstratives are consistently postnominal. In literary works, the demonstrative occurs either prenominally or postnominally (cf. examples \REF{ex:asiimwe:34}). In spoken discourse, it has been observed that the demonstrative is mostly preposed and the augment on the modified noun retained regardless of its function (see \sectref{sec:asiimwe:3.2}).

\eanoraggedright%33
\label{ex:asiimwe:33} \citet[197]{Asiimwe2014}\\\sloppy
    Bwanyima y’eka y’abantu 4 kwitwa oburwaire butamanyirwe, abashaho bagyerizeho kukyebera nikwo kushanga ngu n’oburwairwe bwa MARBURG […] \emph{Endwara} \emph{egi} ebarukireho omuri Kabale […] \emph{oburwaire} \emph{obu} nibukwata nka Ebola […] \\
    \hfill \hbox{Orumuri newspaper (October 22--28, 2012)}\\
  ‘After four members of one family had died of an unknown disease, doctors carried out tests, and found out the disease to be Marburg […] \emph{This} \emph{disease} broke out in Kabale [district][…] \emph{this} \emph{disease} has signs like those of Ebola [hemorrhagic fever]’
\ex%34
    \label{ex:asiimwe:34} \citet[207]{Asiimwe2014}\\
    Eshaaha y’okwetebeekanisiza okutemba endegye ekarindwa yaahika; ntyo naasiibuurana n’abo abaabaire banshendekyereize. Emigugu yangye naagyehisya haihi, naaza omu runyiriri. Ntyo naahika ahi barikushwijumira tikiti, paasipoota hamwe n’emigugu. Tikiti naagiha \emph{empagare} nungi, nayo ndeeba yaagirabyamu amaisho kandi yaateeraho sitampu. \emph{Emigugu} yangye bagirabya omu kyoma, bangira ngu tiinyine nshonga yoona. \emph{Egyo} \emph{mpangare} engira ngu \emph{egyo} \emph{migugu} niinyija kugishanga Gatiwick. \hfill \hbox{\citep[1]{Mugumya2010}}\\
‘Time to prepare for boarding the plane came. I bid farewell to those who had accompanied me. I got my baggage closer and joined the queue. I then approached the passport, air-ticket and baggage checking desk. I handed my ticket to a \emph{nice} \emph{looking} \emph{lady}. She checked it and then stamped it. My \emph{baggage} was sent on a conveyor belt, and I was told that there was no problem. \emph{That} \emph{lady} told me that I will find \emph{that} \emph{baggage} at Gatwick.’
\z

\subsection{Demonstratives and the head noun augment}\label{sec:asiimwe:3.2}

According to \citet{Taylor1985}, if the noun is preceded by a demonstrative, the augment of the noun is omitted. This is also reported in \citet{Dewees1971} and \citet{DeBlois1970}. However, there is evidence indicating that synchronically, the augment can be retained on the head noun with a prenominal demonstrative \REF{ex:asiimwe:35}. Hence, since the augment and the demonstrative can co-occur, they are not in complementary distribution.

\eanoraggedright\sloppy%35
    \label{ex:asiimwe:35} \citet[199]{Asiimwe2014}\\
    {[}…{]} biriyoni 15 ezi baihire ahari difensi omukama we naagira ngu timurikuziihaho. Kandi nabo abantu bagira ngu Your Excellency kasita eki twabaire nituteeraho esente nyingi aha rutaro tukaba twine orutaro omu north, tukaba twine orutaro nkahi [gap]. Mbwenu hati obu rutakiriho katuzite omu kurwanisa \emph{aba} \emph{abakazi} 16 abarikufa buriizooba, \emph{aba} \emph{abantu} 300 abarikwitwa omushwija buriizooba, \emph{aba} \emph{abaana} 435 abarikufa ahabw’endwara ezi twakubaire nitutamba, tube nikyo twaza kukora kandi nyowe nindeeba tikyakubaire kiri ekizibu.\footnote{Recorded from a radio program, \textit{Katuhurirane}, loosely translated as ‘Let us hear from one another’ on \textit{Radio West} on 21.09.2012: 9.00pm, EAT.}\\
‘[…] 15 billion which was to be deducted from the defense budget, the chief [the president] as for him, he says that they cannot deduct it. And as for them, the people say that ‘Your Excellency’ we were allocating more money to war [defense] because there was war in the north, we had a war… where [gap]? Now that the war is no more, let us use this money in the struggle to minimize the level of death of \emph{these} \emph{16} \emph{women} who die every day [of maternal health related complications], \emph{these} \emph{300} \emph{people} who die of fever [malaria] every day, [and] \emph{these} \emph{435} \emph{children} who die as a result of diseases we could prevent. That is what we should now do and to me, I see that would not be a problem.’
\z

The use of an augment on a noun with a preposed demonstrative appears to be a recent occurrence. While it is commonly found in spoken discourse, it is hardly found in written works. Languages are not static; retaining an augment on the noun with a prenominal demonstrative can be attributed to language change and language contact. It may be the case that Run\-yan\-ko\-re-Ru\-ki\-ga speakers are influenced by Luganda\footnote{Luganda appears to have a great influence on many indigenous languages in Uganda.}, since in Luganda the augment of the noun modified by a prenominal demonstrative is obligatory as in \textit{bano *(a)bavubuka}\footnote{When the augment is removed from the noun on the Luganda phrase, \textit{Bano bavúbúka} becomes a copula clause ‘These are youths.’} ‘these youths’ in \REF{ex:asiimwe:30} above. Bantu languages spoken in Uganda that are in close proximity with Luganda geographically are borrowing heavily from Luganda. Borrowing is witnessed almost at all levels of linguistics, especially at the lexical level.

Regarding the role of the augment on a noun preceded by a demonstrative, some speakers do not attach any meaning to it. However, when it is present on a noun, it appears to play the pragmatic role of adding emphasis to the noun (cf. \citealt{Asiimwe2014}). In the radio recording given in \REF{ex:asiimwe:35} above, the speaker consistently uses the augment on nouns preceded by demonstratives and the augment appears to encode an additional feature of emphasis on these nouns. In the written discourse examined, there is only one case that has been identified where the augment is retained on the head noun with a prenominal adnominal demonstrative, given in \REF{ex:asiimwe:36}. The extract is from a book written by a Rukiga speaker. One can, perhaps, predict that Rukiga speakers may have already introduced the augment on a noun preceded by a demonstrative in the formal discourse, which may be an indication that Rukiga is at a further stage in the process of language change.


\eanoraggedright%36
    \label{ex:asiimwe:36} \citet[199]{Asiimwe2014}\\
    Ku baabaire bahikaho, babashangisa aha irembo enjugano zibaikiriziine. Bakaba babanza kubooreka ente kaasha (tikirikumanyisa ngu \emph{egi} \emph{ente} eine akaasha omu buso, kureka nikimanyisa ente nungi erikuhita ezindi)[…] \hfill  Rukiga \citep[86]{Karwemera1994}\\
‘After arriving there, they would meet at the gate and they [the girl’s relatives] would look at the bride price which had been agreed upon. They would first show them the cow \textit{kasha} (\textit{kasha} does not mean that \emph{that} \emph{(specific)} \emph{cow} has a white spot on its forehead, instead it means a nice looking healthy cow amongst all the cows brought)[…]’
\z


\subsection{Double demonstratives in one NP}\label{sec:asiimwe:3.3}

As reported in \citet{Asiimwe2014}, two demonstratives of the same form can co-occur in Run\-yan\-ko\-re-Ru\-ki\-ga. One demonstrative may appear preposed and the other one postposed, as in \REF{ex:asiimwe:37a}, or both may appear in a sequence in the same position, either prenominally \REF{ex:asiimwe:37b} or postnominally with no difference in interpretation. Two demonstratives of different forms can also co-occur in the same NP. For example, the locative copulative demonstrative can co-occur with a non-locative demonstrative \REF{ex:asiimwe:38} and add a level of emphasis (see also \sectref{sec:asiimwe:4.3}). It is also possible for pronominal forms to co-occur. The use of two adnominal demonstratives adds emphasis but may also be used for confirmatory reasons:  \textit{egi ngigi} ‘this very one’ as in \REF{ex:asiimwe:38}.

\ea%37
    \label{ex:asiimwe:37}
    \ea\label{ex:asiimwe:37a}  Egyo njú égyó éíoríkureeba kúri niharááramú omugabe\\
    \gll \emph{e-gi-o}  n-ju  \emph{e-gi-o}  e-i  o-riku-reeb-a  ku-ri  ni-ha-raar-a=mu  o-mu-gabe\\
    \textsc{dem-9-med}  9-house  \textsc{dem-9-med}  \textsc{aug-9.rel.pro}  \textsc{2sg.sbj.rel-prog-}see-\textsc{fv}
    \textsc{17-dist}  \textsc{ipfv}{}-16\textsc{om}{}-sleep-\textsc{fv=18}  \textsc{aug-}1-king\\
  \glt ‘That house that you see over there, a King lives there.’

  \ex\label{ex:asiimwe:37b}  Egyó égyó énju éí oríkureeba kúri niharááramú omugabe\\
    \gll \emph{e-gi-o}  \emph{e-gi-o}  e-n-ju  e-i  o-riku-reeb-a ku-ri  ni-ha-raar-a=mu  o-mu-gabe\\
    \textsc{dem-9-med}  \textsc{dem-9-med}  \textsc{aug-}9-house  \textsc{aug-9.rel.pro}  \textsc{2sg.sbj.rel-prog-}see-\textsc{fv}
\textsc{17-dist}  \textsc{ipfv}=16\textsc{om}{}-sleep-\textsc{fv=18}  \textsc{aug-}1-king\\
\glt ‘That house that you see over there, a King lives there.’
\z
\ex%38
    \label{ex:asiimwe:38} \citet[210]{Asiimwe2014}\\
    Aha murúndí ógu nkataayaayira ényanja ya Lomond na Ness. Egi ngígí ekabá neegámbwahó  kúkye\\
    \gll a-ha  mu-rundi  o-gu  n-ka-taayaay-ir-a  e-n-yanja  y-a Lomond  na  Ness.  \emph{e-gi}  \emph{n-gi-gi}  e-ka-ba ni-e-gamb-w-a=ho  ku-kye\\
    \textsc{aug}{}-16  3-time  \textsc{dem}{}-3.\textsc{prox}  \textsc{1sg.sbj-rem}{}-visit-appl-\textsc{fv}  \textsc{aug}{}-9-lake  9-conn
23.Lomond  and  23.Ness  \textsc{dem}{}-9.\textsc{prox}  \textsc{ldcop}{}-9.\textsc{prox}{}-9.\textsc{prox}  9-\textsc{rem}{}-be
\textsc{ipfv}{}-9-talk-\textsc{pass-fv}=16.\textsc{part}  15-little \\ \jambox{\citep[59]{Mugumya2010}}
\glt ‘This time around, I visited Lake Lomond and Ness. For this very one, no one talked about it much.’
\z

\subsection{Pronominal demonstratives}\label{sec:asiimwe:3.4}

Demonstratives can play a pronominal role. A pronominal demonstrative replaces a noun phrase in an argument position of a verb. A pronominal demonstrative identifies a referent that has been previously established in the discourse, or a referent that can be identified in the physical environment \REF{ex:asiimwe:39} or generally accessible by the hearer. The example in \REF{ex:asiimwe:39} shows the use of the demonstrative pronoun in an adverbial position (but see next Section). In \REF{ex:asiimwe:40}, \textit{ekyo} is used as an endophoric demonstrative pronoun. Note that the demonstrative -\textit{nu} (see also discussion in \sectref{sec:asiimwe:2.3} and \sectref{sec:asiimwe:4.2}) can only be used pronominally, as shown in \REF{ex:asiimwe:41} and it is mostly used in narratives.

\ea%39
    \label{ex:asiimwe:39}
    Omukázi naakund’ éki\\
  \gll o-mu-kazi  ni-a-kund-a  \emph{e-ki}\\
  \textsc{aug}{}-1-woman  \textsc{ifpv-1.sbj}{}-like-\textsc{fv}  \textsc{dem}{}-7.\textsc{prox}\\
\glt ‘The woman likes this (one).’

\ex \label{ex:asiimwe:40}  Ekyo nookimányahó ki?\\
\gll \emph{e-ki-o}  ni-o-ki-mány-a=ho  ki\\
\textsc{dem-7-med}  \textsc{ipfv-2.sbj-7om-}know-\textsc{fv=16.part}  what\\
\glt ‘What do you know about that?’

\ex\label{ex:asiimwe:41}
   Ónu ebishumuuruzo akabitíína tarábiréesire.\\
\gll \emph{o-nu}  e-bi-shumuurur-o  a-ka-bi-tiin-a  ti-a-ra-bi-reet-ire\\
1-\textsc{prox}  \textsc{aug-}\textsc{8om}-key-\textsc{nmlz}  \textsc{3.sbj-rem-}\textsc{8om}-fear-\textsc{fv}  \textsc{neg-3.sbj-rem-}\textsc{8om}-bring-\textsc{pfv}\\ \jambox{\citep[46]{Mubangizi1997}}
\glt ‘This one feared to bring the keys.’
\z

\subsection{Demonstrative adverbs}\label{sec:asiimwe:3.5}

Demonstratives can be used as verbal modifiers \citep{Diessel1999}. The same forms of adnominal demonstratives occur as demonstrative adverbs (\ref{ex:asiimwe:42a}--\ref{ex:asiimwe:42c}). We noted that the noun class 17 locative prefix is not available as an adnominal demonstrative prefix but can be used when the demonstrative is used pronominally (as shown in \tabref{tab:asiimwe:3} for locative demonstratives) and adverbially \REF{ex:asiimwe:42c}. Generally speaking, demonstrative adverbs are commonly used pronominally. This tendency has also been observed in Makhuwa (\citealt{vanderWal2010}).

\ea%42
    \label{ex:asiimwe:42}
    \ea\label{ex:asiimwe:42a}  Engagi zaaraba aha/hánu\\
    \gll e-n-gagi  z-aa-rab-a    \emph{a-ha/há-nu}\\
    \textsc{Aug}{}-10-gorilla  \textsc{10.sbj-n.pst}{}-pass-\textsc{fv}  \textsc{dem}{}-16/16-\textsc{prox}\\
  \glt ‘Gorillas passed here.’

  \ex\label{ex:asiimwe:42b}  Engagi niziraará ómu\\
    \gll e-n-gagi  ni-zi-raar-a  \emph{o-mu}\\
    \textsc{aug-}10-gorilla  \textsc{ipfv}{}-\textsc{10.sbj-}sleep-\textsc{fv}  \textsc{dem}{}-18.\textsc{prox}\\
    \glt ‘Gorillas sleep in here.’

    \ex\label{ex:asiimwe:42c}  Engagi nizirééba kúri\\
    \gll e-n-gagi  ni-zi-reeb-a  \emph{ku-ri}\\
\textsc{aug}{}-10-gorilla  \textsc{prog}{}-10-see-\textsc{fv}  17-there\\
\glt ‘Gorillas are facing the other side.’
    \z
\z

The same forms of demonstratives can be used for temporal deixis. Any of the three demonstrative types can be used to refer to time (\ref{ex:asiimwe:43a}--\ref{ex:asiimwe:43b}). The proximal demonstrative specifically refers to the current time \REF{ex:asiimwe:43c} while either the proximal or the distal form can be used to refer to time in the past (\ref{ex:asiimwe:43a}--\ref{ex:asiimwe:43b}). The locative demonstrative for the current time may also take the \textit{{}-nu} form as shown in \REF{ex:asiimwe:43d} in a non-verbal clause. The form -\textit{nu}, which is not the locative copulative demonstrative commonly used in narratives (\sectref{sec:asiimwe:2.3}) is presumably borrowed from Runyoro-Rutooro where it is used when reference is made to the present time e.g., \textit{obusumi bunu} ‘these days’.

\ea%43
    \label{ex:asiimwe:43}
    \ea\label{ex:asiimwe:43a}  Obwo tukaba nitusháárura ómugúsha\\
    \gll \emph{o-bw-o}  tu-ka-ba  ni-tu-shaarur-a  o-mu-gusha\\
    \textsc{dem-14-med}  \textsc{1pl.sbj-rem-be}  \textsc{ipfv-1pl-}harvest-\textsc{fv}  \textsc{aug}{}-3-sorghum\\
    \glt ‘At that time, we were harvesting sorghum.’

  \ex\label{ex:asiimwe:43b}  Búríya tukabá nitusháárura ómugúsha\\
    \gll ∅-\emph{bu-riya}  tu-ka-ba  ni-tu-shaarur-a  o-mu-gusha    \\
    \textsc{dem-}14-\textsc{dist}  \textsc{1pl.sbj-rem}{}-be  \textsc{ipfv-1pl-}harvest-\textsc{fv}  \textsc{aug}{}-3-sorghum\\
    \glt ‘At that time, we were harvesting sorghum.’

  \ex\label{ex:asiimwe:43c}  Obu turimú nitusháárura ómugúsha\\
    \gll \emph{o-bu}  tu-ri=mu  ni-tu-shaarur-a  o-mu-gusha\\
    \textsc{dem}{}-14.\textsc{prox}  \textsc{1pl.sbj-}be=18.\textsc{explet}  \textsc{ipfv}{}-1\textsc{pl}{}-harvest-\textsc{fv}  \textsc{aug}{}-3-sorghum\\
    \glt ‘We are now harvesting sorghum.’

  \ex\label{ex:asiimwe:43d}  Búri tibwó búnu\\
\gll ∅-bu-ri  ti-bu-o  bu-nu\\
\textsc{dem}{}-14-\textsc{dist}  \textsc{neg-14-pron}  \textsc{14-prox}\\
\glt     ‘The current time is not the same as that (past) time.’
    \z
\z

Not all forms of the demonstrative discussed in \sectref{sec:asiimwe:2} can modify verbs. The pronominal demonstrative -\textit{nu} (\sectref{sec:asiimwe:2.3}), for instance, does not occur in the verb phrase \REF{ex:asiimwe:44a}. An identificational demonstrative also cannot immediately follow a verb in the main clause. It is only felicitous in a relative clause construction \REF{ex:asiimwe:44b} where it has a pronominal but not adverbial role. In the main clause, an identificational demonstrative may follow the ordinary demonstrative \textit{aha.} When it immediately precedes or follows the verb, it functions as a nominal and requires a (locative) object agreement marker to correspond with (\ref{ex:asiimwe:44c}--\ref{ex:asiimwe:44d}).\footnote{\textit{Mpaho}
    can be used expletively as a confirmatory pragmatic marker. As a pragmatic marker, it also functions to express surprise or noteworthiness:
  \ea
  Mpáho wáákimanya\\
  \gll n-ha-ha-o  w-aa-ki-many-a\\
  \textsc{dem-16-16-med}   \textsc{1sg.sbj-n.pst-7om}-know-\textsc{fv}\\
  \glt ‘(I confirm that) you have understood it.’
  \z}
Moreover, the identificational form \textit{n}{}- only combines with noun class 16 but not 17 and 18 locatives (\sectref{sec:asiimwe:2.4}).

\ea%44
    \label{ex:asiimwe:44}
    \ea\label{ex:asiimwe:44a}  Engagi zaabyama *zinu

  \ex\label{ex:asiimwe:44b}  Mpaha ahú zaaba zíbyami\\
    \gll \emph{n-ha-ha}  a-hu  z-aa-ba  zi-byami\\
    \textsc{ld.cop}{}-16-16  \textsc{aug}{}-16.\textsc{rm}  \textsc{10.sbj-n.pst}{}-be  10.\textsc{sbj-}sleep\\
    \glt ‘It is (exactly) here where they were sleeping.’

    \ex\label{ex:asiimwe:44c}  Wááharénga mpáho\\
    \gll w-aa-ha-reng-a  \emph{n-ha-ha-o}\\
  2\textsc{sg.sbj-n.pst}{}-16\textsc{om}{}-pass-\textsc{fv}  \textsc{ld.cop}{}-16-16-\textsc{med}\\
    \glt It (the place) is right there, you are about to bypass it.’

    \ex\label{ex:asiimwe:44d}  Mpaho wááharenga\\
    \gll \emph{n-ha-ha-o}  w-aa-ha-reng-a\\
  \textsc{ld.cop}{}-16-16-\textsc{med}  2\textsc{sg.sbj-n.pst}{}-16\textsc{om}{}-pass-\textsc{fv}\\
  \glt `It (the place) is right there, you are about to bypass it’
  \z
\z

The goal of this section was to discuss the syntax of demonstratives in Run\-yan\-ko\-re-Ru\-ki\-ga. The section has highlighted the fact that the demonstrative can either occur in the prenominal or postnominal position. We also note that the augment, especially in the spoken discourse, can appear on the noun preceded by a demonstrative. The same forms of demonstratives can be used adnominally, pronominally, as adverbs and as temporal deictic markers although not all forms of demonstratives discussed in \sectref{sec:asiimwe:2} can be used adverbially. Next, I turn to the basic roles demonstratives play in the grammar of Run\-yan\-ko\-re-Ru\-ki\-ga.

\section{Functions of demonstratives}\label{sec:asiimwe:4}

This section discusses the pragmatic roles demonstratives play both in discourse and in the physical context. For purposes of this analysis, I follow \citegen[7]{Diessel1999} categorization of demonstratives into exophoric (\sectref{sec:asiimwe:4.1}) and endophoric demonstratives (\sectref{sec:asiimwe:4.2}). Other functions not covered under these two broad categories, such as emphasis and specificity, are discussed in \sectref{sec:asiimwe:4.3}.

\subsection{Exophoric uses}\label{sec:asiimwe:4.1}

The exophoric category symbolizes the basic use from which all non-anaphoric uses of the demonstrative derive. The exophoric demonstratives accompany referents which are mostly visible and accessible in the spatial environment \REF{ex:asiimwe:45}--\REF{ex:asiimwe:46}. For a referent that is available in the physical environment, the demonstrative can be further accompanied by a pointing gesture or a specific eye gaze or pointing lips (cf. \citealt{Fillmore1997,Diessel1999,Diessel2012, Lyons1999, Levinson2004}) to guide the hearer further to the intended referent.

\ea%45
    \label{ex:asiimwe:45}
    Abarámbuzi tíbaareebá ézo ngagi\\
  \gll A-ba-rambuzi  ti-ba-a-reeb-a  \emph{e-z-o}  n-gagi\\
  \textsc{aug}{}-2-tourist-\textsc{nmlz}  \textsc{neg}{}-2-\textsc{n.pst}{}-see-\textsc{fv}  \textsc{dem}{}-10-\textsc{dist}  10-gorilla\\
  \glt ‘The tourists have not seen those gorillas.'

\ex%46
    \label{ex:asiimwe:46} \citet[201]{Asiimwe2014}\\
    Obú wíízire nsígarira n’ogú mwána nzé kwéreeter’ ótwizi\\
  \gll obu  w-a-iz-ire  n-sigar-ir-a  na  \emph{o-gu}    mu-ana n-z-e  ku-e-reete-er-a  o-tu-izi\\
  since  \textsc{2sg-prs}{}-come-\textsc{pfv}  \textsc{1sg.sbj-}stay-\textsc{appl-fv}  with  \textsc{dem}{}-1.\textsc{prox}  1-child 1\textsc{sg.sbj}{}-go-\textsc{fv}  \textsc{inf}-\textsc{refl}-bring-\textsc{appl-fv}  \textsc{aug}{}-12-water\\ 
  \glt ‘Now that you have come, stay with this child while I go to fetch for myself some water.’
\z


An exophoric demonstrative may also be used to refer to an entity not visible to either the speaker or the hearer but assumed to be present in the physical environment \REF{ex:asiimwe:47}. The medial demonstrative form is used for this purpose.

\ea%47
    \label{ex:asiimwe:47}
    Context: At the sound of a loud bang outside as heard from inside a house.\\
  Eky’ ékyángwa n’énki?\\
\gll  \emph{e-ki-o}  e-ky-a-ngw-a  ni  enki?\\
  \textsc{dem}{}-7-\textsc{prox}  \textsc{aug}{}-7.\textsc{rel-n.pst}{}-fall-\textsc{fv}  \textsc{cop}  what\\
\glt  ‘What is that that has fallen?’
\z

Relatedly, the pronominal proximal demonstrative can be used to refer to an invisible and unknown referent. Imagine a situation where the interlocutors are moving in a car; one looks outside and sees a vacuum flask thrown by the roadside and marvels as in \REF{ex:asiimwe:48}: the person who threw the flask by the roadside cannot be identified nor be seen.

\ea%48
    \label{ex:asiimwe:48}
    Ogu shí furásika yaaginagira ki?\\
  \gll \emph{o-gu}  shí  furásika  y-aa-gi-nag-ir-a  ki\\
\textsc{dem-1.prox}  \textsc{dm}  9.flask  \textsc{3sg.sbj-n.pst-9om}{}-throw-\textsc{appl-fv}  why\\
\glt ‘Why has this one thrown the vacuum flask?’
\z

Exophoric demonstratives further take on a symbolic role (\citealt{Fillmore1997, Levinson2004}). According to \citet[94]{Diessel1999}, the symbolic demonstrative draws on knowledge about a larger situational context, which involves more than what is immediately visible in the surrounding situation. The symbolic use of the demonstrative is exemplified in \REF{ex:asiimwe:49}.

\ea%49
    \label{ex:asiimwe:49} \citet[204]{Asiimwe2014}\\
    Abantu b’ómury’ éky’(e)kyaro n’ábahíngi\\
\gll a-ba-ntu  b-a  o-mu-ri  \emph{e-ki}  (e)-ki-aro  ni  a-ba-hingi\\
  \textsc{aug}{}-2-person  2-\textsc{conn}  \textsc{aug}{}-18-\textsc{suff}  \textsc{dem}{}-7.\textsc{prox}  \textsc{aug}{}-7-village  \textsc{cop}  \textsc{aug}{}-2-farmer\\
\glt ‘People of this village are farmers.’ \\
‘People who live in this village are farmers.’
\z

It is the proximal form of the demonstrative that is used for symbolic reference. Hence, the demonstrative use of \textit{eki} in \REF{ex:asiimwe:49} is based on common knowledge about the larger situational context or the symbolic use of \textit{ekyaro} ‘village’. The symbolic demonstrative is a form of deictic demonstrative which does not take any form of gesture because it involves activating knowledge about the communicative event and the referent \citep[94]{Diessel1999}.

\subsection{Endophoric uses}\label{sec:asiimwe:4.2}

The second major category consists of demonstratives whose role is to identify participants in an ongoing discourse. \citet{Diessel1999} considers such to be typically anaphoric (also see \citealt{Himmelmann1996, Lyons1999, Levinson2004, Guillemin2011} among others). The endophoric category also includes demonstratives with a recognitional role, a term that is attributed to \citet{Himmelmann1996}. The speaker draws the hearer to locate a referent in an on-going discourse. A demonstrative takes an anaphoric role if the referent exists in previous discourse. The antecedent of a demonstrative may be a noun phrase (noun phrase anaphora) or a piece of text -- a clause, a paragraph, or even a full story (this is called ``textual anaphora'' according to \citet[64]{Dixon2003}). Example \REF{ex:asiimwe:50} below illustrates noun phrase anaphora. Proximal demonstratives are not commonly used in Run\-yan\-ko\-re-Ru\-ki\-ga texts for phrase anaphora in contrast to some other Bantu languages such as Digo (\citealt{Nicolle2007,Nicolle2014}). However, they can be used to indicate topic continuation in an ongoing discourse. Medial demonstratives are widespread in discourse especially, to signal a shift in topic \REF{ex:asiimwe:51} back to the major topic introduced previously and mostly used in a full NP. \footnote{For convenience, the passage in \REF{ex:asiimwe:33} is repeated in \REF{ex:asiimwe:50}.} Distal demonstrative forms and the proximal narrative demonstrative are used in turn to indicate shifts and turns between two participants in a narrative. This is possible when there is no intervening NP as exemplified with an excerpt from a folktale by \citet[56--57]{Mubangizi1966} in \REF{ex:asiimwe:52}. After a number of turns taken using only demonstratives (cf. \REF{ex:asiimwe:52}), full NPs are resumed.

\eanoraggedright%50
    \label{ex:asiimwe:50}
    \emph{Ekitóngore} ékí Jáne yaabaire naakorá nakyó kikaba kitári mu mateeka. \emph{Ekitongore} \emph{ékyo} kikakingwa.\\
\gll E-ki-tongore  e-ki  Jane  y-aa-ba-ire  ni-a-kor-a   na=ki-o  ki-ka-ba  ki-ta-ri  mu  ma-teeka.  e-ki-tongore e-ki-o  ki-ka-king-w-a\\
\textsc{aug}{}-7-organisation  a\textsc{ug}{}-7.\textsc{rel.pro}  1.Jane  \textsc{3sm}.\textsc{sbj-pst}{}-be-\textsc{pfv}  \textsc{ipfv}{}-3\textsc{sg}{}-work-\textsc{fv}
with=7-\textsc{pro}  7.\textsc{sbj-rem}{}-be  7-\textsc{neg}{}-be  in  6-law \textsc{aug}{}-7-organisation
\textsc{dem}{}-7-\textsc{med}  7.\textsc{sbj}{}-\textsc{rem}{}-close- \textsc{pass-fv}\\
  \glt ‘The company that Jane worked with was illegally established. That company was closed.’

\ex%51
    \label{ex:asiimwe:51} \citet[207]{Asiimwe2014}\\
   Eshaaha y’okwetebeekanisiza okutemba endegye ekarindwa yaahika; ntyo naasiibuurana n’abo abaabaire banshendekyereize. Emigugu yangye naagyehisya haihi, naaza omu runyiriri. Ntyo naahika ahi barikushwijumira tikiti, paasipoota hamwe n’\emph{emigugu}. Tikiti naagiha \emph{empagare} nungi, nayo ndeeba yaagirabyamu amaisho kandi yaateeraho sitampu. \emph{Emigugu} yangye bagirabya omu kyoma, bangira ngu tiinyine nshonga yoona. \emph{Egyo} \emph{mpangare} engira ngu \emph{egyo} \emph{migugu} niinyija kugishanga Gatiwick. \hfill \citep[1]{Mugumya2010}.\\
‘Time to prepare for boarding the plane came. I bid farewell to those who had accompanied me. I got my baggage closer, and joined the queue. I then approached the passport, air-ticket and baggage checking desk. I handed my ticket to a nice looking lady. She checked it and then stamped it. My baggage was sent on a conveyor belt, and I was told that there was no problem. \emph{That} \emph{lady} told me that I will find \emph{that} \emph{baggage} at Gatwick.’

\ex%52
    \label{ex:asiimwe:52}
    A conversation between Kaaremeera and Rwamunyoro extracted from a folktale by   \citet[56--57]{Mubangizi1966}.\\
  Amugira ati “Kááremeera!”\\
 \gll a-mu-gira  a-ti  Kaaremeera\\
  3\textsc{sg.sbj}{}-1\textsc{om}{}-say  1-that  1.Kaaremeera\\
  \glt “He said “Kareemeera!”\\
 Ónu ati “Éé!”\\
\gll \emph{o-nú}  ati  ee\\
1-\textsc{prox}  \textsc{1-}that  yes\\
  \glt “This one said “Yes!”\\
  Oríya ati “Ente kú ébaagwá ni biihá ébifá busha bitariirwe\\
\gll  \emph{o-ríya}  a-ti  “E-n-te  kú  é-baag-w-á  ni  bi-ihá  é-bi-fá busha  bi-ta-ri-ir-w-e\\
  1-\textsc{dist}  1-that  \textsc{aug}{}-9-cow  when  9.\textsc{sbj-}slaughter-\textsc{pass-fv}  is  8-what  \textsc{aug}-8-waste
  nothing  8-\textsc{neg}{}-eat-\textsc{appl}{}-\textsc{pass-fv}\\
\glt  ‘The other one said, ``When a cow is slaughtered, which parts of it are not eaten?''’\\
  Ónu abanzá áteekáteeka\\
\gll  \emph{ó-nu}  a-banzá  á-teekáteek-a…\\
  1-\textsc{prox}  \textsc{3sg}{}-first-\textsc{fv}  3\textsc{sg}{}-think-\textsc{fv}\\
\glt  ‘This one had to first think…’\\
  Oríya ati f’ókugamba óku óríkubímanya.”\\
\gll  \emph{o-riya}  a-ti  fa  o-ku-gamb-a  o-ku  o-riku-bi-many-a.”\\
  1-\textsc{dist}  1-that  just  \textsc{aug-inf}{}-speak-\textsc{fv}  \textsc{aug}{}-how  \textsc{1.sbj-ipfv}{}-\textsc{8om-}know-\textsc{fv}\\
\glt  ‘The other said, "Just mention those that you know.”\\
Ónu ati “Ndeeb-a nibanaga ámíîsho, nibanaga óbwongo, nibanaga óruhango, nibanagá endurwe\\
\gll  \emph{o-nu}  a-ti  “N-reeb-a  ni-ba-nag-a  a-ma-isho,  ni-ba-nag-a o-bu-ongo,  ni-ba-nag-a  o-ru-hango,  ni-ba-nag-a  e-n-rurwe…\\
  1-\textsc{prox}  1-that  1\textsc{sg}{}-see-\textsc{fv}  \textsc{ipfv}{}-2-throw-\textsc{fv}  \textsc{aug}{}-6-eye  \textsc{ipfv}{}-2-throw-\textsc{fv}
  \textsc{aug}{}-14-brain  \textsc{ipfv}{}-2-throw-\textsc{fv}  \textsc{aug}{}-11-gallbladder  \textsc{ipfv}{}-2-throw-\textsc{fv}  {\textsc{aug}{}-9-bile duct}\\
\glt  “This one said, “I see eyes being thrown, the brain being thrown, I see the bile duct being thrown, I see the gallbladder being thrown…”\\
  Oríya ati ébyo nibyó óríkumanyá byónka?\\
\gll  \emph{o-riya}  a-ti  e-bi-o  ni-bi-o  o-riku-many-a  bi-onka?”\\
  1-\textsc{dist}  1-that  \textsc{dem}{}-8-\textsc{med}  \textsc{cop}{}-8-\textsc{rel.pro}  1-\textsc{ipfv}{}-know-\textsc{fv}  8-only\\
\glt  “The other said, “Is that all you know?”\\
  Rwamunyóro ati “Amahémbe n’ómukíra orábiríire\\
\gll  \emph{Rwamunyoro}  a-ti  “A-ma-hembe  na  o-mu-kira  o-ra-bi-ri-ire?\\
  1.Rwamunyoro  1-that  \textsc{aug}{}-6-horn  and  \textsc{aug}{-3}-tail  2\textsc{sg.sbj}{}-ever-\textsc{8om-}eat-\textsc{pfv}\\
 \glt ‘Rwamunyoro said, “Have you ever eaten the horns and the tail?”’\\
  Karemeera ati “Shana óyenzire kugira ébyokushánzirana nangwá kutabiiré kunjúma\\
\gll  \emph{Karemeera}  a-ti  “Shana  o-yend-ire  ku-gira  e-bi-a  o-ku-shanz-ir-ana nangwá  ku-ta-b-iré  ku-n-júm-a.”\\
  1.Karemeera  1-that  maybe  2\textsc{sg}{}-want-\textsc{pfv}  \textsc{inf}{}-say  \textsc{aug}{}-8-conn  \textsc{aug-}joke-\textsc{appl-recp}
  but  \textsc{inf-neg}{}-be-\textsc{pfv}  \textsc{inf-1sg}{}-abuse-\textsc{fv}\\
\glt  ‘Karemera said “Unless you just want to make fun of me, if not abusing me.”’
\z

A pronominal form in a narrative, as \textit{ogwo} in \REF{ex:asiimwe:53} is used for the referent that forms the most important topic in the previous discourse. The demonstrative \textit{ogwo} is associated with a referent that is already established. A pronominal form is felicitous because the antecedent appears in the recent discourse and \textit{ogwo} can be recognised to refer to a human referent for example, not \textit{ente} ‘cow’ which is non-human and belongs to a different noun class, yet both can be tracked from the immediate preceding discourse.

\eanoraggedright\sloppy%53
    \label{ex:asiimwe:53}\citet[208]{Asiimwe2014}\\
    Ku baahikire omu kyaro ky’owaabo, baateekyerereza Omukama waabo eby’omuhiigo, n’eby’oburungi bwa munyaanya wa Muyanda, n’oburungi bw’ente zi yaabaire atungire. \emph{Omukama} \emph{ku} \emph{yaabihuriire} \emph{yaagira} \emph{ati} \emph{“Ntashwire} \textit{\textbf{ogwo}} \emph{ndyashwera} \emph{oha?”} Omukama ahabwokwenda ngu ashwere ogwo mwishiki kandi anyagye n’ente za Muyanda, akateekateeka eihe ry’okuza kurwanisa. \hfill \citep[21]{Karwemera1975}.\\
  ‘When they returned to their village, they told their King about hunting and the beauty of Muyanda’s sister, and about the beauty of the cows which Muyanda reared. \emph{When} \emph{the} \emph{King} \emph{heard} \emph{all} \emph{that} \emph{he} \emph{said} \emph{‘If} \emph{I} \emph{don’t} \emph{marry} \emph{that} \emph{one,} \emph{whom} \emph{shall} \emph{I} \emph{marry?’} Because the King wanted to marry that girl and to rustle Muyanda's cows, he organised a militia group to go and fight with.'
  \z

The boldfaced anaphoric pronominal demonstrative \textit{ogwo} in \REF{ex:asiimwe:53} also selects the most salient referent from the previous discourse which forms the main topic for the subsequent discourse. This contrasts with the medial demonstrative in a full NP which provides additional information about a major participant in the foregoing discourse. This is especially if there are other NPs introduced in between, such as \textit{egyo mpangare} in \REF{ex:asiimwe:51} which is reactivated after a series of other intervening NPs (see also \citealt{Nicolle2014} for Digo).

Demonstratives not only refer to NPs but also to whole sentences, paragraphs or even a full story. In text anaphora, the antecedent of the demonstrative must refer to the immediately preceding discourse (\citealt[224]{Himmelmann1996}). \textit{Ekyo}, a medial demonstrative in \REF{ex:asiimwe:54}, refers to the whole preceding proposition.

\eanoraggedright\sloppy%54
    \label{ex:asiimwe:54}\citet[209]{Asiimwe2014}\\
    Tukabá tuteera órunyiriri rw’ókuzá omu kinaabiro-kihorónyo kwékoraho. Abándi ab’émicwe etagunjúkire bakabá bamarayó ebyanda, \emph{ékyo} kireeterá abantu kwéshanyá ógwó otaríkwenda kuhéérezá ábandi omugisha gw’ókwéshemeza. \hfill \citep[3]{Mugumya2010}\\
  `We would queue to go to the bathroom to clean ourselves. Other people with no good manners would spend there a long time, and \textit{that} would make people angry at that person who does not want to give others a chance to clean themselves.’
  \z

This section has underscored the diversity of roles that demonstratives perform in Run\-yan\-ko\-re-Ru\-ki\-ga both in the physical world and in discourse. The exophoric demonstratives refer to events and objects in the physical environment but also serve to activate shared knowledge between interlocutors. In discourse, demonstratives activate and reactivate participants but also highlight major discourse participants. Demonstratives also may indicate turns and shifts between discourse participants in a narrative as indicated in \REF{ex:asiimwe:52}, for instance. A detailed study of discourse roles of demonstratives in spontaneous speech and natural contexts with the help of a corpus can reveal more specific roles.
% notwithstanding the challenge of lack of a substantial corpus.

\subsection{Other functions of the demonstrative: emphasis and particularisation}\label{sec:asiimwe:4.3}

Emphasis and particularisation are other roles played by the Run\-yan\-ko\-re-Ru\-ki\-ga demonstrative. These functions are commonly realised when two demonstratives co-occur. Similar to Run\-yan\-ko\-re-Ru\-ki\-ga, \textcitetv{chapters/taji} notes that double demonstratives in Chiyao realise emphasis. By particularisation, this means that an object is singled out from other objects. The demonstratives may be of the same form \REF{ex:asiimwe:55} or of a different form. For example, a locative demonstrative copula can co-occur with a locative demonstrative to particularise or emphasise a more specific location as in \REF{ex:asiimwe:56}. In addition, when an augment on the head noun modified by a prenominal demonstrative is retained, it is said to add emphasis to the noun (see \sectref{sec:asiimwe:3.2})

\ea%55
    \label{ex:asiimwe:55}
    Egyó égyó  ésímu niyo ndíkwenda\\
\gll  \emph{e-gy-o}  \emph{e-gy-o}  e-simu  ni-yo  n-riku-end-a\\
  \textsc{dem}{}-9-\textsc{med}  \textsc{dem-}9-\textsc{med}  \textsc{aug}{}-9.phone  \textsc{cop-9.rel.pro}  \textsc{1sg-ipfv}{}-want-\textsc{fv}\\
\glt ‘It is that phone (particularly that one) that I want [may be accompanied by a gesture].’

\ex%56
    \label{ex:asiimwe:56}
    Aha mpáha bakáhabáha?\\
\gll  \emph{a-ha}  \emph{n-ha-ha}  ba-ka-ha-ba-h-a\\
  \textsc{dem-16.prox}{}-here  \textsc{ldcop}{}-16-16  \textsc{3pl-rem}{}-\textsc{om16-om2}-give-\textsc{fv} \\
\glt ‘Were you given this (exact) place?’
\z

Other demonstrative forms used to encode additional emphasis on the location of an entity include: \textit{okwe nkukwe} ‘exactly there’ for class 17 and \textit{omwe mpaho} ‘exactly in there’ for class 18, depending on the relative distance of location of an entity from both the speaker and the hearer (refer to the various locatives forms in \tabref{tab:asiimwe:3}). The use of these latter forms depends on the dialect of the speaker and age as these forms seem to be common among the young Rukiga speakers. It is observed that Runyankore speakers simply double the demonstrative locative to denote a more specific location of a referent: \textit{aha (a)ha} ‘exactly here’; \textit{okwo okwe} ‘exactly there’ and \textit{omwo omwe} ‘exactly in there’. Moreover, doubling of demonstratives is a common strategy for expressing emphasis in Bantu languages (c.f., discussion in \citealt{Malinga1980} for isiXhosa and \citealt{Gunnink2018} for Fwe).

\section{Conclusion}\label{sec:asiimwe:5}

This chapter has offered an overview of the morphology, the syntax and the functions of demonstratives in Run\-yan\-ko\-re-Ru\-ki\-ga. Various forms of the demonstrative and usage have been discussed. Evidence for the claim that the initial element of the demonstrative is not an augment but a core morpheme, hence an indispensable element has been presented. This morpheme responsible for deictic and anaphoric meanings, as discussed, is prevalent in many Bantu languages. Another key aspect of the current analysis is the manner demonstrative -\textit{ti} which seems to have not received much attention in previous work. A detailed follow up study of this form can establish its connection with the other demonstrative classes, its functions, historical origin and also the study might be extended to other Bantu languages to find out how far it is spread in the Bantu language zones. An additional key issue for further investigation that emerges from this chapter concerns the role of the augment of the noun appearing with a prenominal demonstrative. An augment retained on the noun preceded by an adnominal demonstrative seems to be more pronounced in spoken discourse and is hardly found in written works. This phenomenon shows how flexible the spoken register is, compared to the written form. This occurrence could also be attributed to language contact with the neighboring Luganda where the augment is grammatically required when the noun appears with a pre-modifying demonstrative and many Run\-yan\-ko\-re-Ru\-ki\-ga speakers are also bilingual or trilingual in Luganda.

Another pertinent question to explore further is the extent to which information structure interacts with syntax. Such a study might provide answers to what determines the position of the adnominal demonstrative in the Run\-yan\-ko\-re-Ru\-ki\-ga language cluster. An in-depth study of discourse demonstratives, their distribution and the roles they play, is also still pending. This work would benefit greatly from the presence of an extensive Run\-yan\-ko\-re-Ru\-ki\-ga corpus.

The present chapter is a contribution to descriptive and comparative studies of Bantu languages. Insights can be drawn from the current study to conduct further specific and detailed studies about the morpho-syntax and functions of demonstratives in Run\-yan\-ko\-re-Ru\-ki\-ga and other related Bantu languages.

\section*{Abbreviations}
\begin{multicols}{2}
\begin{tabbing}
\textsc{Rel.pro} \=   locative demonstrative copulative\kill
\textsc{appl} \>   applicative\\
\textsc{asp} \>    aspect\\
\textsc{aug} \> augment\\
\textsc{cop}    \> copula\\
\textsc{dem}    \> demonstrative morpheme\\
\textsc{dist} \>   distal\\
\textsc{dm} \>   discourse marker\\
\textsc{explet} \>   expletive\\
\textsc{fv} \>    final vowel\\
\textsc{conn} \>   connective  \\
\textsc{inf}    \> infinitive \\
\textsc{ipfv} \>   imperfective\\
\textsc{ldcop} \>    locative demonstrative \\ \> copulative\\
\textsc{med}    \> medial\\
\textsc{neg}    \> negation\\
\textsc{nmlz} \>   nominalizer\\
\textsc{n.pst} \>   immediate past\\
\textsc{om} \>    object marker\\
\textsc{part} \>   partitive\\
\textsc{pers.} \>   person\\
\textsc{pl} \>   plural\\
\textsc{prox} \>   proximal\\
\textsc{prog} \>   progressive\\
\textsc{prs}    \> present tense\\
\textsc{ref}    \> referential\\
\textsc{rem}    \> remote past\\
\textsc{refl}   \> reflexive\\
\textsc{rel}    \> relative\\
\textsc{rel.pro} \> relative pronoun\\
\textsc{sbj}    \> subject marker\\
\textsc{sbjv} \>   subjunctive\\
\textsc{sg} \>   singular\\
\end{tabbing}
\end{multicols}

\printbibliography[heading=subbibliography,notkeyword=this]
\end{document}
