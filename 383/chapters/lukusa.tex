\documentclass[output=paper]{langscibook}
\author{Stephen T. M. Lukusa\affiliation{Gothenburg, Sweden}}
\title{A morphosyntactic study of verb object marking in Čilubà}
\abstract{As described in several Bantu languages, object marking in Čilubà goes hand in hand with the capacity of a verb to carry object complements. This discussion of object marking has been inspired by parameters that were used by \citet{MartenEtAl2007}. Contrary to European languages which use independent pronouns to represent secondary and primary objects, Bantu languages use a number of dependent object markers which are prefixed or added to the end of verb stems as post-clitics. The use of object markers in Bantu languages has also been extended to verbal morphemes replacing what is commonly known as locative adjuncts. The present discussion brings a new stone to the building by analysing and commenting on new data from Čilubà with a hope to highlight similarity and difference between this central African language from D.R. Congo and similar languages that have already been investigated in East Africa, Southern Africa and elsewhere.}
\IfFileExists{../localcommands.tex}{
  \addbibresource{../localbibliography.bib}
  \usepackage{langsci-optional}
\usepackage{langsci-gb4e}
\usepackage{langsci-lgr}

\usepackage{listings}
\lstset{basicstyle=\ttfamily,tabsize=2,breaklines=true}

%added by author
% \usepackage{tipa}
\usepackage{multirow}
\graphicspath{{figures/}}
\usepackage{langsci-branding}

  
\newcommand{\sent}{\enumsentence}
\newcommand{\sents}{\eenumsentence}
\let\citeasnoun\citet

\renewcommand{\lsCoverTitleFont}[1]{\sffamily\addfontfeatures{Scale=MatchUppercase}\fontsize{44pt}{16mm}\selectfont #1}
   
  %% hyphenation points for line breaks
%% Normally, automatic hyphenation in LaTeX is very good
%% If a word is mis-hyphenated, add it to this file
%%
%% add information to TeX file before \begin{document} with:
%% %% hyphenation points for line breaks
%% Normally, automatic hyphenation in LaTeX is very good
%% If a word is mis-hyphenated, add it to this file
%%
%% add information to TeX file before \begin{document} with:
%% %% hyphenation points for line breaks
%% Normally, automatic hyphenation in LaTeX is very good
%% If a word is mis-hyphenated, add it to this file
%%
%% add information to TeX file before \begin{document} with:
%% \include{localhyphenation}
\hyphenation{
affri-ca-te
affri-ca-tes
an-no-tated
com-ple-ments
com-po-si-tio-na-li-ty
non-com-po-si-tio-na-li-ty
Gon-zá-lez
out-side
Ri-chárd
se-man-tics
STREU-SLE
Tie-de-mann
}
\hyphenation{
affri-ca-te
affri-ca-tes
an-no-tated
com-ple-ments
com-po-si-tio-na-li-ty
non-com-po-si-tio-na-li-ty
Gon-zá-lez
out-side
Ri-chárd
se-man-tics
STREU-SLE
Tie-de-mann
}
\hyphenation{
affri-ca-te
affri-ca-tes
an-no-tated
com-ple-ments
com-po-si-tio-na-li-ty
non-com-po-si-tio-na-li-ty
Gon-zá-lez
out-side
Ri-chárd
se-man-tics
STREU-SLE
Tie-de-mann
} 
  \togglepaper[1]%%chapternumber
}{}

\begin{document}
\maketitle 
%\shorttitlerunninghead{}%%use this for an abridged title in the page headers


\begin{itemize}
\item \textbf{Introduction}
\end{itemize}

The topic of object marking is closely related to that of verb transitivity which has been extensively discussed by several authors (\textstyleHTMLCite{\textup{Hopper and Thompson, 1980; Naess, 2007; Baker, 1988;}} Chomsky, 1981) and their discussions have revealed diversity in the syntactic argument structure of several languages. With respect to the action or state described by the verb of a sentence, the participants fulfill some functions that are called \textit{semantic roles}, \textit{thematic roles}, \textit{theta roles}, or \textit{participant roles}. Among these thematic roles, \citet{Borsley1991}, \citet{Chomsky1981}, \citet{CookNewson1996}, as well as \citet{Haegeman1991} point to the following:

\begin{itemize}
\item \textbf{Commonly} \textbf{identified} \textbf{thematic} \textbf{roles}
\end{itemize}

\begin{itemize}
\item \textit{The agent} who performs the action designated by the verb
\item \textit{The patient} that suffers/undergoes the action or event designated by the verb
\item \textit{The theme}, i.e., the entity that is moved by the action or event designated by the verb
\item \textit{The experiencer}, i.e., the living entity that experiences/undergoes the action or event denoted by the predicate
\item \textit{The goal/destination} which is the location or entity in the direction targeted by the action of the verb
\item \textit{The benefactive} which is the entity that profits/benefits from the action or event denoted by the predicate
\item \textit{The source} which is the location or entity from which something originates
\item \textit{The instrument}, i.e., the device with which or the medium by which the action or event denoted by the predicate is carried out
\item \textit{The locative} which is the word or word group that situates (or specifies the place) where the action or event denoted by the predicate takes place
\end{itemize}

The remaining part of this chapter is organized along these lines. \sectref{sec:lukusa:2} presents Čilubà, the language which is at the centre of this study and its close relatives, the Luban languages. \sectref{sec:lukusa:3} describes object as a compulsory complement of the verb. In \sectref{sec:lukusa:4}, it was judged helpful to adopt the aforementioned parametric approach as a systematic way in eliciting Čilubà data since it allows to eventually compare the situation in Čilubà to what is found in other languages. Every stage of the discussion starts with a statement of the parameter followed by a statement of its applicability to the language in the light of relevant data and a discussion. \sectref{sec:lukusa:5} closes the chapter with a brief summary and a short conclusion.

\begin{itemize}
\item \textbf{Čilubà} \textbf{and} \textbf{Luban} \textbf{languages}
\end{itemize}

Čilubà is a Bantu language which \citet{Guthrie’s1948} classification labels as L31. It is chiefly spoken in the ‘Kasaï Occidental’ and ‘Kasaï Oriental’ provinces of the Democratic Republic of Congo. \citet{HammarströmEtAl2022} label it as Luba1253 and classify it among Central-Western Bantu languages. It is at the centre of a group of Bantu languages that \textstylest{\citet{Choi1996} established as the} \emph{\textup{Luban languages}}\textstylest{ which constitute half of Guthrie’s Zone L.}

The other familiar languages of group L are Kisongye (L23), Kisanga (L35), Kihemba (L34), Kanyok (L32), Budya (L20), Gipende (L11), Kaonde (L41), Kilubà (L33), and Zela which \citet{Maho2009} labelled \textstyleHTMLCode{L331.} Čilubà (L31) is sometimes called Luba-Kasai in the literature and should not be confused with Kilubà (L33), a sister language in the DRC province of Katanga (known in the literature as Luba-Katanga, Luba-Shaba or sometimes Luba-Shankadi).

Despite the resemblance between Čilubà and Kilubà, and the historical relationships between the ethnic groups speaking these languages, \citet{Guthrie1948} maintains them as two separate languages spoken by two related ethnic groups which are nowadays patrilineal (for the Baluba-Kasai) and matrilineal (for the Baluba-Katanga).

The Čilubà speakers’ territory is bounded on the North by Lele (a subgroup of the Kuba people), Kuba (C30), and Tetela (a language of the Mongo group (C)); on the West by Gipende (L11), and Chokwe (K11); on the South by Mbagani (L22), Kete (L20), Ruund (L50), Chokwe and Kanyok (L32); and on the East by Kilubà (L33) and Kisongye (L23).

Čilubà is not just the language of the Balubà as the name tends to suggest. Its native speakers include (i) the Balubà-Lubìlànjì (in the Eastern Kasai province), (ii) the Luluwà (of Western Kasai), (iii) the Bena Konjì (in Western Kasai) and (iv) the Bàbindji (of Eastern Kasai). It is also used as a second language by ethnic groups speaking different mother tongues such as (i) Bakuba, (ii) Bakete (North), (iii) Bashilele, (iv) Bampende, (v) BaChokwe, (vi) Babìndjì, (vii) Basalampasu, and (viii) Balwalwa (of western Kasai), and (i) Basongye, (ii) Bena Kanyoka, (iii) Bena Tubeya, (iv) Bakete (South), and (v) Bakwa Mputu (of Eastern Kasai). The language has also some pockets of speakers in the major cities of the country.

The number of Čilubà speakers amounts approximately to 11.6 million out of a 2009 UN-estimated population of 66 million people for the whole country (i.e., 17.6 \%). Though probably realistic, these numbers should be seen as mere estimations since they are not based on any actual census. Moreover, this figure applies to the particular year in which it was published. Other sources (such \citet{SimonsFennig2017}) give varying figures depending on the year of publication.

The language has a 5-vowel sound system. Each vowel has a long counterpart whose quality is more or less similar that of its short counterpart. The language also has 21 consonant sounds of which two, namely [ɸ] and [h], are dialectal or positional variants of /p/. The language’s sound system is well known for its consistent harmonization of consonants and vowels in some specific environments, palatalization or raising of some consonants before [i] as well as its widespread tone reversal. The language also displays synchronic mutations which are reminiscent of historical changes (Coupez, 1954: 47) which yielded different consonants in positions where the two high vowels [i] and [u] in a five-vowel system stand for original high (close) vowels of Proto-Bantu seven-vowel system.

Its morphology is agglutinative and is based on an 18-noun class system of which three are very productive locative classes (16, 17 \& 18) and two diminutive classes (12 \& 13) which are now missing in some Southern Bantu languages. The object markers which are at the centre of the present study are some of the concord markers that are found in Čilubà. The language lacks the augments which have survived in a number of Eastern and Southern Bantu languages.

\begin{itemize}
\item \textbf{Discussion} \textbf{of} \textbf{parameters} \textbf{and} \textbf{illustration}
\end{itemize}

As already said earlier, the parametric approach as a systematic way of eliciting Čilubà data will be applied in the subsequent discussion. Every step of the discussion will start with an enunciation of the relevant parameter from \citet{MartenEtAl2007}, followed by a statement of its applicability to the language, illustrated with relevant data that will be subsequently commented on.

\textbf{Parameter} \textbf{1:} \textbf{Can} \textbf{the} \textbf{object} \textbf{marker} \textbf{and} \textbf{the} \textbf{lexical} \textbf{object} \textbf{NP} \textbf{co-occur?}

No, object and OM cannot co-occur in Čilubà.

Under normal circumstances, that is, with SVO order, in Čilubà, it is ungrammatical to see the OM co-occur with the lexical object NP that it is supposed to replace. In \REF{ex:lukusa:2} below, the OM \textit{u-} replaces the object NP \textit{mukanda} ‘letter’.

\ea%2
    \label{ex:lukusa:2}
    \z

            ŋ-â-\textbf{ù}{}-bad-ì

    SM1SG-PRS-\textbf{OM3-}read-FV

\glt ‘I have read \textbf{it}’.

Compare this with the ungrammatical, pleonastic sentence in \REF{ex:lukusa:3} below.

\ea%3
    \label{ex:lukusa:3}
    \z

            \textbf{*}ŋ-â-\textbf{ù}{}-bad-ì               \textbf{mu-kàndà}

      SM1SG-PRS-\textbf{OM3}{}-read-FV    3-letter

    \textbf{*} \textbf{‘}I have read \textbf{it} \textbf{the} \textbf{letter}’.

The form in \REF{ex:lukusa:3} is marked as ungrammatical. It is occasionally heard among non-native speakers or children of native speakers who speak the language as their second language after acquiring Swahili. Hence, their speech form is seen as influenced by Swahili and is considered as non-standard.

\textbf{Parameter} \textbf{2:} \textbf{Is} \textbf{co-occurrence} \textbf{of} \textbf{object} \textbf{marker} \textbf{and} \textbf{object} \textbf{NP} \textbf{required} \textbf{in} \textbf{some} \textbf{contexts?}

Yes, for the sake of emphasis, object NP can be moved to the front position in Čilubà and therefore be allowed to co-occur with the object marker in the same sentence. Moreover, when the OM represents an animate beneficiary, Čilubà requires that an object NP be used in the form of a disjunctive (or better a disjunctive pronoun) in case one needs to emphasize the beneficiary.

Emphasis does sometimes require co-occurrence of object marker and object NP, as explained below.

When an object NP is moved to the front of the verbal word for the sake of emphasis, there is usually a comma after it in writing to represent a brief pause in speaking, as indicated in \REF{ex:lukusa:4}. This is a clear indication that such an object NP and the following predicate belong to different intonation groups.

\ea%4
    \label{ex:lukusa:4}
    \z

            \textbf{mu-kàndà}    \textbf{w-èbè,}      ŋ-â-\textbf{ù}{}-bad-ì              leelù.

\textbf{3-letter}      \textbf{POSS3-2SG}    SM1SG-PRS\textbf{{}-OM3}{}-read-FV     today

\glt ‘Your letter, I have read it today’.

The normal word order would either be SVO without any pre-verbal OM, as in \REF{ex:lukusa:5} below, or SV including a pre-verbal OM without any post-verbal object NP, as in \REF{ex:lukusa:6}.

\ea%5
    \label{ex:lukusa:5}
    \z

            ŋ-â-bad-ì           \textbf{mu-kàndà}    \textbf{w-èbè}       leelù

    SM1SG-PRS-read-FV    \textbf{3-letter}      \textbf{POSS3-2SG}     today

\glt ‘I have read your letter today’.

\ea%6
    \label{ex:lukusa:6}
    \z

            ŋ-â-ù-bad-ì             leelù

    SM1SG-PRS-OM3-read-FV    today

\glt ‘I have read it today’.

Therefore, the following sentence in which object NP still occurs in its usual position while the pre-verbal OM is also used is incorrect.

\ea%7
    \label{ex:lukusa:7}
    \z

            *ŋ-â-ù-bad-ì             \textbf{mu-kàndà}    \textbf{w-èbè}       leelù

      SM1SG-PRS-OM3-read-FV  \textbf{3-letter}      \textbf{POSS3-2SG}    today

* ‘I have read \textbf{it} \textbf{your} \textbf{letter} today’.

As mentioned in the foregoing discussion of Parameter 2, the second context requiring co-occurrence of object marker and object NP in some Čilubà contexts is when, for the sake of emphasis, an object NP in the form of an independent / disjunctive pronoun is used to refer to the same antecedent as the OM.

\ea%8
    \label{ex:lukusa:8}
    \z

            a.  W-a-\textbf{m}{}-fuč-ì               \textbf{même}            ma-bànzà  ànyi    ônso.

      SM1-PRS-\textbf{OM1}{}-pay off-FV(\textbf{as} \textbf{far} \textbf{as} \textbf{I} \textbf{am} \textbf{concerned})    6-debt    6-my    6.all

\glt ‘\textbf{As} \textbf{far} \textbf{as} \textbf{I} \textbf{am} \textbf{concerned}, he  has paid off all my debts.’

b.  Bà-a\textbf{{}-tù}{}-tùm-iny-i              \textbf{twêtu}    mi-kàndà    mi-bì

      SM2-PRS-\textbf{OM1PL}  {}-send-APPL-FV  \textbf{us}        4-letter      4-bad

\glt ‘\textbf{To} \textbf{us} \textbf{particularly}, they sent bad letters.’

c.  W-a-\textbf{kw}{}-amb-id-i             \textbf{wêwe}   cinyi?

      SM1-PRS-\textbf{OM2SG}{}-say-APPL-FV    \textbf{you}    what

\glt ‘What has he told \textbf{you} \textbf{specifically}?’

Notice that underlining in the English translations of (8.a, b and c) represents emphasis added by adverbial to the bold-typed objects in the Čilubà sentence above.

\textbf{Parameter} \textbf{3:} \textbf{Are} \textbf{there} \textbf{locative} \textbf{object} \textbf{markers} \textbf{in} \textbf{Čilubà?}

Yes, there are locative OMs in Čilubà.

Contrary to some Eastern Bantu languages (e.g., Swahili) and South-Eastern Bantu languages (e.g., Setswana and Shekgalagari), Čilubà has preserved all the three locative classes (i.e., 16, 17 and 18) of Proto-Bantu. The locative class prefixes are so productive in the language in that most nouns can still take one (i.e., a locative class prefix) as a secondary prefix, i.e., one that attaches to a noun that already has a primary class prefix or a zero class-prefix and be used as subjects or complements. Normal locative nouns (as well as locative nouns derived in the aforementioned way) can be replaced by a distinct locative low-toned OM (namely \textit{pa-}, \textit{ku-} or \textit{mu-}), depending on the applicable class.

\ea%9
    \label{ex:lukusa:9}
    \z

            Class 18 locative noun

    ŋ-â-bwed-ì            \textbf{mu-n-zùbu}    butùku

    SM1SG-PRS-enter-FV    \textbf{18-9-house}    14.night

\glt ‘I have entered \textbf{(*in)} \textbf{the} \textbf{house} at night’.

With the use of a locative OM, the sentence in \REF{ex:lukusa:9} becomes:

\ea%10
    \label{ex:lukusa:10}
    \z

          Class 18 locative OM\footnote{Notice that the low-toned class 18 OM is a locative one and should not be confused with a high-toned similar class 9 OM which would stand for \textit{the house} in \textit{Nzùbu eu, ng-â-mu-bwed-ì butùku} “this house, I have entered it at night”.}

ŋ-â-\textbf{mù}{}-bwed-  ì             butùku  

SM1SG-PRS-\textbf{OM18}{}-enter-FV    14.night

\glt ‘I have entered \textbf{it} at night’.

\ea%11
    \label{ex:lukusa:11}
    \z

           Class 17 locative noun

ŋ-â-bwed-ì            \textbf{ku-mu-longo}    mpidiewu

    SM1SG-PRS-enter-FV    \textbf{17-3-queue}    now

\glt ‘I have entered/joined \textbf{the} \textbf{queue} now.’

With the use of a locative OM, example \REF{ex:lukusa:11} becomes:

\ea%12
    \label{ex:lukusa:12}
    \z

          Class 17 locative OM\footnote{This L-toned locative OM should not be confused with the 2\textsuperscript{nd} person singular class 1 OM which is H-toned as in \textrm{\textit{[F04E?]}}\textit{{}-â-}\textbf{\textit{ku}}\textit{{}-bwed-ì mpidiewu} ‘I have entered/penetrated \textbf{you} now’ which means ‘I have touched your heart now’.}

ŋ-â-\textbf{kù}{}-bwed-ì             mpidiewu

SM1SG-PRS-\textbf{OM17}{}-enter-FV    now

\glt ‘I have entered / joined \textbf{it} (i.e., the queue) now.’

\ea%13
    \label{ex:lukusa:13}
    \z

           Class 16 locative noun\footnote{In the examples below, the tense/aspect marker is indicated with a zero symbol /-\textrm{${\varnothing}$}{}-/ to show that it has been elided to avoid successive inter-syllabic vowels and therefore facilitate pronunciation. Nevertheless, the analysis indicated that the tense is present.}

ŋ-${\varnothing}${}-imàny-ì           \textbf{pa-lu-kìtà}    lwèndè

SM1SG-PRS-stand-FV    \textbf{16-11-tomb}  POSS.11

\glt ‘I have stood on his tomb’.

With the use of a locative OM, example \REF{ex:lukusa:13} becomes:

\ea%14
    \label{ex:lukusa:14}
    \z

          Class 16 Locative OM

ŋ-${\varnothing}${}-îmàny-ì\textbf{{}-po}

SM1SG-PRS-stand-FV-\textbf{OM16}

\glt ‘I have stood on \textbf{it}{}'

It should be clarified that in contrast with the use of a pre-verbal OM in \REF{ex:lukusa:10} and \REF{ex:lukusa:12}, \REF{ex:lukusa:14} uses a cliticized post-verbal OM to avoid the pre-verbal OM \textit{pa-} whose syllabic configuration requires the application of vowel elision which would lead into resyllabification with the following initial vowel of the verb stem \textit{{}-iman-} ‘stand’, from the Proto-Bantu VC root \textit{{}-im-}. A verb with a CVC root would have taken a pre-verbal OM as indicated in \REF{ex:lukusa:15} and 16) below.

\ea%15
    \label{ex:lukusa:15}
    \z

          ŋ-â-funk-ùny-ì         mu-nu    \textbf{pa-lu-kìtà}    lwèndè

    SM1SG-PRS-point-FV    3-finger    \textbf{16-11-tomb}  POSS.11

\glt ‘I have pointed a finger \textbf{at} \textbf{his} \textbf{tomb.}’

With the use of an OM, \REF{ex:lukusa:15} becomes:

\ea%16
    \label{ex:lukusa:16}
    \z

          ŋ-â-\textbf{pà}{}-funkùny-ì             mu-nu

    SM1SG-PRS-\textbf{OM16-}point-FV    16-finger

\glt ‘I have pointed a finger \textbf{at} \textbf{it.}’

In \REF{ex:lukusa:15} and \REF{ex:lukusa:16}, \textit{kufunkuna munu} is an idiom in which \textit{munu} ‘finger’ is not the object of \textit{kufunkuna}  ‘to point at’, but rather the instrument with which pointing is done. Therefore, what the Čilubà sentence really means is ‘I have pointed at it with the finger.’ This sentence is slightly different from \REF{ex:lukusa:17} and \REF{ex:lukusa:18} below in which the word \textit{tupaya} ‘razor blades’ is used metaphorically for ‘razor blade slashes’ which is the object of the verb ‘to put’.

\ea%17
    \label{ex:lukusa:17}
    \z

          Bědì tupaaya  \textbf{pa} \textbf{čibaŋu.}                    

    B(a)-ěl-ì           tu-paaya          \textbf{pa} \textbf{či-baŋu}

    SM2-PRS.put-FV    12-razor blade slash    \textbf{on} \textbf{7-scar}  

\glt ‘They have put razor blade slashes on the scar.’

With the use of the locative OM, \REF{ex:lukusa:17} becomes:

\ea%18
    \label{ex:lukusa:18}
    \z

           B{ǎ}\textbf{p}ědì tupaaya.

    B(a)-{ǎ}{}-\textbf{p(}{\textbf{a)}}{}-el-ì        tu-paaya

     SM2-PRS-\textbf{OM16-}put  {}-FV  12-razor  blade  slash  

\glt ‘They have put razor blade slashes on it (i.e., on the scar)’

Should the object \textit{tupaaya} ‘razor blade slashes’ need to be replaced with an OM, here is the possible outcome.

\ea%19
    \label{ex:lukusa:19}
    \z

          B{ǎ}\textbf{tw}ědì\textbf{po.}

    B({à})-a\textbf{{}-t}{\textbf{ù}}{}-ěl-ì\textbf{{}-po}

SM2-PRS-\textbf{OM12-}put-FV\textbf{{}-OM16} 

\glt ‘They have put them (i.e., slashes) on it (i.e., the scar)’

A single non-locative OM would not show the same discriminative behaviour as illustrated by the locative OM which occurs before the verb root /-el-/ in \REF{ex:lukusa:18} and post-verbally, i.e., after the FV as a clitic in \REF{ex:lukusa:19}. It will keep the same pre-verbal position regardless of the syllabic configuration of the verb root and regardless of the type of object marker used, as illustrated in \REF{ex:lukusa:20} to \REF{ex:lukusa:23}.

\ea%20
    \label{ex:lukusa:20}
    \z

          ŋâ\textbf{ù}badì  

    ŋ-â-\textbf{ù}{}-bal-ì  

     SM1SG-PRS-\textbf{OM4}{}-read-FV  

\glt ‘I have read it’

\ea%21
    \label{ex:lukusa:21}
    \z

          ŋâ\textbf{mu}lekèdì kashidi    

ŋ-â-\textbf{mu}{}-lek-èl-ì            kashidi  

    SM1SG-PRS-\textbf{OM1}{}-abandon-FV   forever  

\glt ‘I have abandoned her forever.’

\ea%22
    \label{ex:lukusa:22}
    \z

          ŋà\textbf{mw}enzèdì bîmpè  

    ŋ-à-\textbf{mu}{}-enz-èl-ì            bîmpè

    SM1SG-PRS-\textbf{OM1}{}-do-APPL-FV  good

\glt ‘I have done good to him.’

\ea%23
    \label{ex:lukusa:23}
    \z

          ŋà\textbf{mw}enzèjì bîmpè  

    ŋ-à-\textbf{mu}{}-enz-èj-ì            bi-{ì}mpè

    SM1SG-PRS-\textbf{OM1}{}-do-CAUS-FV  good

\glt ‘I have forced him to do better.’

With two object markers, the beneficiary comes before the verb root while the secondary object follows it as a post-clitic.

\ea%24
    \label{ex:lukusa:24}
    \z

          ŋà\textbf{mu}lekèèdì\textbf{u}    

    ŋ{}-à-\textbf{mu}{}-lek-e(l)-èl-ì\textbf{{}-u}

    SM1SG-PRS-\textbf{OM1}{}-leave-APPL-APPL-FV\textbf{{}-OM3}  

\glt ‘I have abandoned it to her.’

\ea%25
    \label{ex:lukusa:25}
    \z

          Twâ\textbf{bà}tùminyi\textbf{čyò}  

    Tu-â-\textbf{bà}{}-tùm-in-i\textbf{{}-čyò}  

    SM1PL-PRS-\textbf{OM2}{}-send-APPL-FV\textbf{{}-OM7}  

\glt ‘We have sent it to them’

\textbf{Parameter} \textbf{4a:} \textbf{Is} \textbf{object} \textbf{marking} \textbf{restricted} \textbf{to} \textbf{one} \textbf{object} \textbf{marker} \textbf{per} \textbf{verb?}

No, object marking is not restricted to one object per verb in Čilubà.

It all depends on the maximum number of objects that each individual verb can take. Some verbs are intransitive and may therefore take no object. Such verbs will, as a consequence, take no OM as there is no object NP for which the OM can stand.

\ea%26
    \label{ex:lukusa:26}
    \z

          Examples of some intransitive verbs in Čilubà

\begin{itemize}
\item kufùùluluka    ‘to resuscitate’
\item kubììka        ‘to wake up’
\item kutàbuluka    ‘to be startled’
\item kuselemuka    ‘to slip/slide (as on a slippery surface)’
\item kwimana      ‘to stand’
\end{itemize}

Čilubà, intransitive verbs can still accept OMs because an intransitive verb such as \textit{{}-iman-} ‘stand’ can take adjuncts of place as shown in \REF{ex:lukusa:13} to \REF{ex:lukusa:14} and such adjuncts can be replaced by OMs in Čilubà.

Other verbs can take one to three objects. Such verbs are transitive and the number of OMs which they can take varies according to the valency of each specific verb. With very few exceptions, mono-transitive verbs are underived.

\ea%27
    \label{ex:lukusa:27}
    \z

          Mono-transitive or one-object verbs

\begin{itemize}
\item ku-bal-a mikànda        ‘to read books’
\item ku-long-a Čilubà          ‘to study Čilubà\footnote{NB. The verb \textit{kulonga} meaning ‘to place’ can take two OMs as \textit{kulonga mik}{\textit{à}}\textit{nd}{\textit{à}} \textit{pa m}{\textit{èè}}\textit{sa} becomes \textit{ku}\textbf{\textit{{}-}}{\textbf{\textit{ì}}}\textit{{}-long-a}\textbf{\textit{{}-p}}{\textbf{\textit{ò}}} including the pre-verbal OM {\textit{ì-} and the post-clitic OM \textit{{}-pò}.}}
\item ku-saam-a munda        ‘to suffer from diarrhoea’
\item ku-sab-uk-a musùlu      ‘to cross a river’
\end{itemize}

On the other hand, ditransitive verbs are mostly derived by suffixing a verb extension to the root. Hence such extensions are described as valency-increasing.

\ea%28
    \label{ex:lukusa:28}
    \z

          Di-transitive or two-object verbs

\begin{itemize}
\item ku-sùmb-il-a \textbf{mwâna} \textbf{bìsàbatà}
\end{itemize}
\glt ‘to buy a pair of shoes for the child’

\begin{itemize}
\item ku-tùb-ul-a \textbf{bakàjì} \textbf{mačì}
\end{itemize}
\glt ‘to pierce women’s ears’

\begin{itemize}
\item ku-jik-ij-à \textbf{muntu} \textbf{mpatà}
\end{itemize}
\glt ‘to end one’s doubts’

\begin{itemize}
\item ku-pààny-in-a \textbf{balongeshi} \textbf{mikàndà}
\end{itemize}
\glt ‘to sell books for the teachers’

\begin{itemize}
\item ku-temb-a \textbf{bakwèbà} \textbf{bule}
\end{itemize}
\glt ‘to exalt/praise your height to your friends’

\begin{itemize}
\item ku-bad-ish-a \textbf{baloŋi} \textbf{mikàndà}
\end{itemize}
\glt ‘to make students read books’

With the exception of \textit{kutemba} ‘to exalt/praise’ in \REF{ex:lukusa:28e}, all the verbs in \REF{ex:lukusa:28} are morphologically complex, i.e., derived by extending the root with a CAUS (\textit{{}-ish-}/\textit{{}-ij-}), APPL (\textit{{}-il-}/\textit{{}-in-}) or REVS (\textit{{}-ul-}) extension. This therefore made them capable of taking more objects and therefore more OMs.

Three-object verbs are rare but can be found in Čilubà where verbs are extended as explained earlier on. In the first of the following examples in \REF{ex:lukusa:29}, the valency of the idiom \textit{kutwà nshìngì} ‘to inject’ which already contains one object \textit{nshìngì} ‘injections’ is increased by inserting the applicative extension \textit{{}-il-} which enables the resulting derived verb form to add two more objects including the recipient \textit{bâna} ‘children’ and the beneficiary \textit{baledi} ‘parents’.

\ea%29
    \label{ex:lukusa:29}
    \z

           Three-object verbs\footnote{In \REF{ex:lukusa:29h} the preposition \textit{ne} ‘with’ has been bracketed to show that it is optional. One can say \textit{kuuja màlòŋò byàkudyà or kuuja màlò}\textrm{\textit{ŋ}}\textit{ò nè byàkudyà} ‘to fill plates with food’.}

\begin{itemize}
\item kutwìla \textbf{baledi} \textbf{bâna} \textbf{nshìŋì}
\end{itemize}
\glt ‘to inject children for the parents’

\begin{itemize}
\item kutàmbushila \textbf{bantu} \textbf{bâna} \textbf{ntàmbù}
\end{itemize}
\glt ‘to baptize children of other people’

\begin{itemize}
\item kusambishila \textbf{baledi} \textbf{bâna} \textbf{mabàkà}
\end{itemize}
\glt ‘to make people’s children move from one marriage to another’

\begin{itemize}
\item kumwèneshela \textbf{baledi} \textbf{bâna} \textbf{makèŋa}
\end{itemize}
\glt ‘to make other parents’ children suffer’

\begin{itemize}
\item kufwànyikijila \textbf{mulùma} \textbf{bâna} \textbf{matàma}
\end{itemize}
\glt ‘to make the husband’s children’s cheeks resemble one’s own cheeks’

\begin{itemize}
\item kwendeshela \textbf{bantu} \textbf{bâna} \textbf{mitù}
\end{itemize}
\glt ‘to trouble the heads of other people’s children’

\begin{itemize}
\item kufwìshila \textbf{bantu} \textbf{bâna} \textbf{bundù}
\end{itemize}
\glt ‘to put shame on other people’s children’

\begin{itemize}
\item kuujila \textbf{baloŋi} \textbf{màlòŋò} (nè) \textbf{byàkudyà}
\end{itemize}
\glt ‘to fill students’ plates with food’

To show that three lexical object complements as those bold-typed in \REF{ex:lukusa:29} are not commonplace occurrence in Čilubà, their replacement with three parallel object markers in the language is not easily workable, unless a homorganic first-person singular object \textit{{}-N-} is used for the beneficiary, as illustrated in the following examples derived by replacing the lexical objects in \REF{ex:lukusa:30} with object markers. Moreover, such complex verbal words including three object markers need to be used with great caution particularly in a context which will not create confusion as in \textit{nshingi y}{ô}\textit{yo sè, ne-à-}\textbf{\textit{bà}}\textit{{}-}\textbf{\textit{n}}\textit{{}-tw-il-a-}\textbf{\textit{yo}} \textit{čyakaaa} ’Injections! He will administer them to them in multitude to my pleasure’, where the first ‘them’ stands for ‘injections’ whereas the second ‘them’ represents the recipient of injections.

\ea%30
    \label{ex:lukusa:30}
    \z

          a.  kubà{ǹtwìlayò}

ku-\textbf{bà}{}-{\textbf{ǹ}{}-tw-ìl-a-\textbf{yò}}

{15-\textbf{OM2-OM1SG}{}-inject-APPL-FV-\textbf{OM10}}

{‘to inject them on my behalf / to my pleasure’}

b.  kubà{ǹtàmbushilayò}

ku-\textbf{bà}  {}-{\textbf{ǹ}{}-tàmb-ush-il-a\textbf{{}-yò}}

15-\textbf{OM2}{}-\textbf{OM1SG}{}-baptize-APPL-FV-\textbf{OM10}

\glt ‘to baptize them on my behalf’

c.  kubà{ǹsambushilayò}

ku-\textbf{bà}  {}-{\textbf{ǹ}{}-samb-ush-il  {}-a\textbf{{}-yò}}

15-\textbf{OM2}{}-\textbf{OM1SG}{}-wander-APPL-FV-\textbf{OM6}

\glt ‘to make them wander from one wedding to another to my bewilderment’

d.  kubà{mfwànyìkijilaò}

ku-\textbf{bà}  {}-{\textbf{m}{}-fwàny-ìk-ij-il-a-\textbf{ò}}

{15-\textbf{OM2-OM1SG}{}-resemble-NEUTR-CAUS-APPL-FV-\textbf{OM6}}

{‘to make them (i.e., their cheeks) resemble mine to my satisfaction’}

e.  kubàŋ{endeshelayò}

ku-\textbf{bà}  {}-\textbf{ŋ}{{}-end-esh-el-a-\textbf{yò}}

{15-\textbf{OM2}  \textbf{{}-OM1SG-}trouble-CAUS-APPL-FV-\textbf{OM6}}

{‘to trouble/confuse them (i.e., their minds) on my behalf/to my pleasure’}

f.  kubà{mfwìshilabù}

ku-\textbf{bà}{}-{\textbf{m}{}-fw-ìsh-il-a-\textbf{bù}}

{15-\textbf{OM2-OMSG}{}-die-CAUS-APPL-FV-\textbf{OM14}}

{‘to embarrass / kill them (with shame) on my behalf/to my pleasure’}

g.  kubàŋ{uujilaò}

ku-\textbf{bà}  {}-\textbf{ŋ}{{}-uuj-il-a-\textbf{ò}}

{15-\textbf{OM2-OM1SG}{}-fill-CAUS-APPL-FV-\textbf{OM6}}

{‘to make them swell to my satisfaction’}

All the verbs listed in the examples in \REF{ex:lukusa:30} are expressions derived respectively from the following idioms or pleonastic expressions:

\ea%31
    \label{ex:lukusa:31}
    \z

          Idiomatic expressions underlying three-object verbs

a.  kutwà nshìŋì            

\glt ‘to inject’

b.  kutàmbula ntàmbù        

\glt ‘to baptize’ (pleonastic expression)

c.  kusamba mabàkà          

\glt ‘to move/wander from one marriage to another’

d.   kumòna makèŋa          

\glt ‘to suffer’ (Lit: to see misery)

e.  kufwànyikija bâna matàma  

\glt ‘to cause children’s cheeks to look like one’s own’

f.  kwendesha bantu mitù      

\glt ‘to trouble people’s heads’ (Lit: to make people’s heads move)

g.  kufwìsha bantu bundù      

\glt ‘to put shame on people’ (Lit: to make people die of shame)

h.  kuuja màlòŋò (nè) byàkudyà  

\glt ‘to fill plates with food’

\textbf{Parameter} \textbf{4b:} \textbf{Are} \textbf{two} \textbf{object} \textbf{markers} \textbf{possible} \textbf{in} \textbf{restricted} \textbf{contexts?}

Yes, two OMs are possible in Čilubà as long as the one of them (which is closest to the verb root) is the homorganic 1\textsuperscript{st} Person singular -N-

Čilubà restricts the sequence of two OMs as follows: (i) if one of the object markers is the 1st person singular object marker \textit{N-} (i.e., the homorganic class 1 nasal) (as illustrated in \REF{ex:lukusa:32} and \REF{ex:lukusa:33} below, it comes closer to and before the verb root because it is phonologically conditioned by homorganic assimilation to the place of articulation of the first sound of the verb root. (ii) If the two OMs include one animate and the other inanimate, the animate one precedes the verb root, and the inanimate OM is relegated to the post verbal clitic position as illustrated in \REF{ex:lukusa:34}. (iii) But if otherwise the two OMs are inanimate, as illustrated in \REF{ex:lukusa:35}, the benefactive comes before the verb root and the instrument is relegated to the post-verbal position as a clitic OM. Since translation and glossing cannot perfectly reflect the order of object NPs and the Object Markers, I have used underlining, italicizing and bold-typing to indicate where the aforementioned grammatical categories keep their positions or change them in Ciluba.

\ea%32
    \label{ex:lukusa:32}
    \z

         OM 1st person singular (\textit{N-}) and a second human OM (from class 1 and/or 2)

a.  nù\textbf{n}tuminà    \textit{bâna}        ‘(You) Send \textit{the children} \textbf{to} \textbf{me}’  >

nù\textit{bà}\textbf{n}tùmìnà            ‘(You) send \textit{them} \textbf{to} \textbf{me}’  

nù-\textit{bà}\textbf{{}-N}{}-tùm-ìn-à  

SM-\textit{OM2}{}-\textbf{OM1SG}{}-send-APPL-FV  

\glt ‘Send \textit{them} \textbf{to} \textbf{me’}

b.  Nù\textbf{n}tùmìnà \textit{mukànd}à      ‘(You) send \textbf{me} \textit{the letter}’       >

nù\textit{ù}\textbf{n}tùmìnà            ‘(You) send \textit{it} \textbf{to} \textbf{me}’  

nù-\textit{ù}{}-\textbf{N}{}-tùm-ìn-à  

SM-\textit{OM3}{}-\textbf{OM1SG}{}-send-APPL-FV  

\glt ‘Send \textit{it} \textbf{to} \textbf{me’}

c.  nù\textbf{m}b{ì}k{ì}d{ì}là    \textit{muloŋeshi}   ‘call \textit{the teacher} \textbf{for} \textbf{me}’    >

nù\textit{mu}\textbf{m}b{ì}k{ì}d{ì}là            ’call \textit{him} \textbf{for} \textbf{me}

nù-\textit{mu}{}-\textbf{N}{}-b{ì}k{ì}l-  {ì}l-à

SM-\textit{OM2}{}-\textbf{OM1SG}{}-call-APPL-FV  

\glt ‘call \textit{him} \textbf{for} \textbf{me’}

\ea%33
    \label{ex:lukusa:33}
    \z

         OM 1. person singular (\textit{N-}) and a second non-animate OM 

a.  Nù\textbf{n}deejà    \textit{cibuta}        ‘show \textit{the bag} \textbf{to} \textbf{me}’    >  

Nù\textit{c}{\textit{ì}}\textbf{n}dee\textbf{j}à                ’show \textit{it} \textbf{to} \textbf{me’}

Nù  {}-\textit{c}{\textit{ì}}{}-\textbf{N}{}-leej-à

SM-\textit{OM7}{}-\textbf{OM1SG}{}-show-FV    

\glt ‘show \textit{it} \textbf{to} \textbf{me’}

b.  (Nùmulekèlè) à\textbf{n}dondèlà    \textit{bwalu}  ‘(you, Let him) tell \textbf{me} \textit{the story’}’  >  

(Nùmulekèlè) à\textit{bù}\textbf{n}dondèlà        ‘(Let him) tell \textit{it} \textbf{to} \textbf{me’}      

(nù-mu-lek-èl-è)        à-bu-\textbf{n}{}-lond-èl-à              

(SM-OM-let-APPL-FV)    SM-OM2-\textbf{OM1SG}{}-tell-APPL-FV    

\glt ‘(Let him) tell \textit{it} \textbf{to} \textbf{me’}

c.  \textbf{N}kwàc{ì}lè   \textit{mw}{â}\textit{na}            ‘Hold \textit{the child} \textbf{for} \textbf{me}’  >

\textit{Mu}\textbf{n}kwàc{ì}lè                ’Hold \textit{him} \textbf{for} \textbf{me’}

\textit{Mu}{}-\textbf{N}  {}-kwàt-il-è  

\textit{OM1}{}-\textbf{OM1SG}{}-hold{}-APPL{}-FV    

\glt ‘Hold \textit{him} \textbf{for} \textbf{me’}

\ea%34
    \label{ex:lukusa:34}
    \z

          One OM is animate and the other inanimate: The inanimate OM is relegated to the post-verbal clitic position.

a.  Wàkaasa mbùji mangùlùbà    ‘He hit the goat with clods (of earth)’   >


Wàkamwasao              ‘He hit it with them.’  



U-aka-mu-as-a-o



SM1-PST-OM9-hit-FV-OM6



\glt ‘He hit it with them.’


b.   Mungàngà wàkatwà mbwa bwanga  ‘The doctor injected the dog with a drug’ >


Mungàngà wàkamutwàb{o}.        ‘The doctor injected her with it.’



Mu-ngàngà    u-àka-mu-tw-à{}-bo



1-doctor      SM1-PST-OM9-inject-FV-OM14



\glt ‘The doctor injected her (i.e., the dog) with it (i.e., the drug).’


\ea%35
    \label{ex:lukusa:35}
    \z

           Both OMs are inanimate: The benefactive comes before the verb root and the instrument is suffixed as a clitic.

a.  {ǔ}d{ì}d{ì} \textbf{bil}à\textbf{mb}à \textit{cyamù}    ‘He has bought \textit{an iron} (INSTR) \textbf{for} \textbf{them} (BEN)’ >

W{ǎ}\textbf{by}{ǔ}d{ì}d{ì}\textit{cyo}          ‘He has bought \textit{it} (INSTR) \textbf{for} \textbf{them} (BEN)’

{u}{}-{ǎ}{}-\textbf{b}{\textbf{ì}}{}-ul-{ì}l-{ì}{}-\textit{cyo}

SM1-PFT-\textbf{OM(BEN)}{}-buy-APPL-FV-\textit{OM(INSTR)}

\glt ‘He has bought \textit{it} \textbf{for} \textbf{them}’

b.  w{ǎ}mbw{ì}d{ì} mvùla makùmb{ì}  ‘He has brought \textit{the umbrellas} \textbf{for} \textbf{the} \textbf{rain}’ >

wà{mwa}mbw{ì}d{ì}o          ‘He has brought \textit{them} (INSTR) \textbf{for} \textbf{it} (BEN)

ù-à-{mu-a}mb-u(l)-{ì}l-{ì-}o

SM1-PFT-{\textbf{OM(BEN)}{}-bring}{}-{APPL}{}-{FV-}\textit{OM(INSTR)}

\glt ‘He has brought \textit{them} \textbf{for} \textbf{it.’}

The following starred combinations in \REF{ex:lukusa:36} to \REF{ex:lukusa:42} are ungrammatical because they do not observe the aforementioned restrictions in Ciluba.

\ea%36
    \label{ex:lukusa:36}
    \z

          a.   Nù\textbf{bà}tùmìnà \textbf{čyesu} kùnu.        ‘Send \textbf{them} \textbf{the} \textbf{pot} over here’   >

nù\textbf{bà}tùmìnà\textbf{čyo}  kùnu

nù-\textbf{bà}{}-tùm-ìn-à-\textbf{čyo}              kùnu

SM2PL-\textbf{OM2}{}-send-APPL-FV-\textbf{OM7}     over here

\glt ‘Send \textbf{it} (i.e., the pot) over here \textbf{to} \textbf{them}’

b.  \textbf{*}nù\textbf{čìbà}tùmìnà  kùnu

nù-\textbf{čì-bà}{}-tùm-ìn  {}-à              kùnu

  SM2PL-\textbf{OM7-OM2}{}-send-APPL-FV     here

\ea%37
    \label{ex:lukusa:37}
    \z

           a.  Ù\textbf{mu}seemèjìlà \textbf{bwâtu}.  ‘That you may push \textbf{the} \textbf{boat} \textbf{for} \textbf{him}’ >

  ù\textbf{mu}seemejìlà\textbf{bu}

  ù-\textbf{mu}{}-seemej-ìl-à-\textbf{bu}

SM2SG-\textbf{OM1}{}-push-APPL-FV\textbf{{}-OM14}

\glt ‘That you may push \textbf{it} (i.e., the boat) \textbf{for} \textbf{him}’

b.  \textbf{*}ù\textbf{bùmu}seemèjìlà

  ù-\textbf{bù}  {}-\textbf{mu}{}-seemèj-ìl-à

SM2SG-\textbf{OM14}{}-\textbf{OM1}{}-push-APPL-FV

       \textbf{*}‘That you \textbf{it}  \textbf{for} \textbf{him} may push’

\ea%38
    \label{ex:lukusa:38}
    \z

           a.  Neà\textbf{tù}kèbèlà \textbf{mubelu}.  ‘He will seek the piece of advice for us’  >

  neà\textbf{tù}kèbèlà\textbf{u}

  ne-à-\textbf{tù}{}-kèb-èl-à\textbf{{}-u}

FUT-SM1-\textbf{OM1PL}{}-seek-APPL-FV-\textbf{OM3}

\glt ‘He will seek \textbf{it} (i.e., the piece of advice) \textbf{for} \textbf{us}’

b.  *neà\textbf{tùù}kèbèlà

  ne-à-\textbf{tù}  \textbf{{}-ù}{}-kèb-èl-à

  FUT-SM1-\textbf{OM1PL-OM3}{}-seek-APPL-FV

\ea%39
    \label{ex:lukusa:39}
    \z

           a.  Ùmusombèshà {čikàsu   ‘that you may lend \textbf{him} the spade’    >}

{ù\textbf{mu}sombèshà\textbf{čyo}}

{  ù-\textbf{mu}{}-somb  {}-èsh-à\textbf{{}-}\textbf{čyo}}

{SM2SG-\textbf{OM1}{}-lend-APPL-FV\textbf{{}-OM7}}

\glt {‘that you may lend \textbf{it} (i.e., the spade) \textbf{to} \textbf{him}’}

{    b.  *ù\textbf{čìmu}sombèshà}

{        ù-\textbf{čì-mu}{}-somb-èsh-à}

{  SM1  {}-\textbf{OM7-OM1}  {}-lend-CAUS-FV}

{(40)   a.} Nùbàpelèshèlà matalà    ‘Get \textbf{the} \textbf{maize} milled \textbf{for} \textbf{them}.’  >

{  nù-\textbf{bà}{}-pel-èsh-èl-à\textbf{{}-}\textbf{o}}

{SM2PL-\textbf{OM2}{}-mill-CAUS-APPL-FV\textbf{{}-OM6}}

{{\textasciigrave}Get \textbf{it} milled (i.e., the maize) \textbf{for} \textbf{them}’}

b.  {*nù\textbf{àbà}pelèshèlà}

{     nù-\textbf{à-bà}{}-pel-èsh-èl-à}

{  SM2PL-\textbf{OM6-OM2}{}-mill-CAUS-APPL-FV}

\ea%41
    \label{ex:lukusa:41}
    \z

           a.  Ù\textbf{mu}sombèshà {\textbf{čikàsu}    ‘That you may lend \textbf{him} \textbf{the} \textbf{spade}’  >}

{  ù-\textbf{mu}{}-somb  {}-èsh-à\textbf{{}-}\textbf{čyo}}

    SM1-\textbf{OM1}{}-borrow-CAUS-FV-\textbf{OM7}

\glt ‘That you may lend \textbf{it} (i.e., the spade) \textbf{to} \textbf{him}’

{b.  *ù\textbf{čìmu}sombèshà}

{   ù\textbf{{}-}\textbf{čì-mu}{}-somb-èsh-à}

       SM1-\textbf{OM7-OM1-}borrow-CAUS-FV

\ea%42
    \label{ex:lukusa:42}
    \z

           a.  À\textbf{tù}noonèlà \textbf{keelè}       ‘Let him sharpen \textbf{the} \textbf{knife} \textbf{for} \textbf{us}’  >

  à{}-\textbf{tù}{}-noon-èl-à-\textbf{ko}

SM1-\textbf{OM1PL}{}-noon-APPL-FV-\textbf{OM13}

\glt ‘Let him sharpen \textbf{it} (i.e., the knife) \textbf{for} \textbf{us}’

b.  *à-\textbf{kà-tù}{}-noon-èl-à

  à{}-\textbf{kà-tù}{}-noon-èl-à

  SM1-\textbf{OM13-OM1PL}{}-sharpen-APPL-FV

The first-person singular OM /-N-/ is homorganic because it gets assimilated to the place of articulation of the first sound of the verb root which follows it and when the following sound is a vowel, it is realized as the velar [ŋ], as \textit{in u-bà-}\textbf{\textit{ŋ}}\textit{{}-el-èl-à mooyo} ‘Say hello to them on my behalf’ or in {\textit{ù}}\textit{{}-}\textbf{\textit{ŋ}}\textit{{}-}{\textit{û}}\textit{j-}{\textit{ì}}\textit{l-è lwesu} ‘Fill the pot for me’.

\textbf{Parameter} \textbf{4c:} \textbf{Are} \textbf{two} \textbf{or} \textbf{more} \textbf{object} \textbf{markers} \textbf{freely} \textbf{available?}

No, two or more OMs are either possible, or not possible in certain contexts in Čilubà.

It is wiser to avoid a sweeping assertion or negation in reply to parameter 4c. Multiple object markers are not completely absent in Čilubà. However, it should be clarified, as already observed earlier in connection with parameters \REF{ex:lukusa:4a} and \REF{ex:lukusa:4b}, that one pre-verbal OM is commonplace in the language. Two pre-verbal OMs do, as well, occur. But they do so under the conditions which have been specified under parameter \REF{ex:lukusa:4b}. On the other hand, three pre-verbal OMs are scarcer. As observed in connection with the examples in \REF{ex:lukusa:35} and \REF{ex:lukusa:37}, they do occur with some verbs that take a pleonastic object (such as \textit{ku}\textbf{\textit{tàmbw}}\textit{isha n}\textbf{\textit{tàmbù}}, ‘Lit: to baptize baptism’) or when they are derived from some idioms by using some valency-increasing extensions such as the applicative /-il-/, /-el-/, /-in-/ or /-iɲ-/, /-en-/, or the causative-applicative /-ish-il-/, /-ij-il-/, /-esh-el-/, /-ej-el-/.

\textbf{Parameter} \textbf{4d:} \textbf{Is} \textbf{the} \textbf{order} \textbf{of} \textbf{multiple} \textbf{object} \textbf{markers} \textbf{structurally} \textbf{free?}

No, the order of multiple objects is not structurally free. It is fixed in Čilubà. 

When a verb has two OMs, they are subjected to a strict order which will be phonologically motivated below. Because there is a requirement in Čilubà that one of the two OMs has to be the first-person singular object marker \textit{N-} (i.e., a homorganic nasal), it is indeed this OM that has to be prefixed first to the verb root. The other OM is in its turn prefixed to this output yielding the following structure of the underived conjugated verbal word. 

\ea%43
    \label{ex:lukusa:43}
    \z

          Structure of the underived conjugated verbal word including two OMs


\begin{tabularx}{\textwidth}{XXXXXX}

\lsptoprule

 SM & TAM & OM\textbf{\textsubscript{2}} & \textbf{OM\textsubscript{1}}\textbf{1SG} & Verb Root & FV\\
\lspbottomrule
\end{tabularx}
Following the order of affixes described above, it has been found useful in this analysis to modify the numbering of OMs suggested in \citet{MartenEtAl2007} by saying that an underived conjugated Čilubà verb including two OMs normally follows the structure suggested in Diagram \REF{ex:lukusa:43} above. It consists of \textbf{Subject} \textbf{Marker} \textbf{+} \textbf{Tense} \textbf{or} \textbf{Aspect} \textbf{Marker} \textbf{+} \textbf{Object} \textbf{Marker} \textbf{2} \textbf{+} \textbf{Object} \textbf{Marker} \textbf{1} \textbf{(i.e.,} \textbf{the} \textbf{First} \textbf{Person} \textbf{Singular} \textbf{Homorganic} \textbf{/-N-/)} \textbf{+} \textbf{Verb} \textbf{Root} \textbf{+} \textbf{FV} \textbf{suffix}. In this verb structure, OM\textsubscript{2} is normally the secondary object while OM\textsubscript{1} is the primary object.

Phonologically, the attachment of the homorganic OM\textsubscript{1} /-N-/ to the verb root is justified by the fact that it blends so well to the pronunciation of the following sound by harmonizing with it through place-of-articulation assimilation (i.e., homorganic assimilation) and it does not seem to change the syllabic configuration or the syllable count of the verb because it becomes part of the onset of the first syllable of the verb root.

\textbf{Parameter} \textbf{5:} \textbf{Can} \textbf{either} \textbf{object} \textbf{be} \textbf{adjacent} \textbf{to} \textbf{the} \textbf{verb?}

No, only one object can be adjacent to the verb in Čilubà.

Double object verbs do not allow the order of the two objects to be interchanged. This is related to the theta role of each argument of the verb. In agreement with \citet{Riedel2009b}, when a sentence has a ditransitive verb, the primary object, which may have the role of the goal (= G), recipient (= RCPT), patient (= PAT) or benefactive (= BEN) argument of the verb, tends to be closer to the verb root, while the secondary object is the theme (= TH) argument and tends to be far from the verb root. 

\ea%44
    \label{ex:lukusa:44}
    \z

           a.  Mwâna ùdi ùfùndila baledi mukàndà

mw-âna    ù-di           ù-fùnd-il-a           ba-ledi    mu-kàndà

1-child     SM1-PRS.be    SM1-write-APPL-FV    2-parent   3-letter

\glt ‘The child is writing a letter to the parents’

b.  *Mw-âna   ù-di     ù-fùnd-il-a     mu-kàndà   ba-ledi  

\ea%45
    \label{ex:lukusa:45}
    \z

           a.  Vwàdikà       mwâna      mutèèlu

      vwàl-ik-à       mw-âna      mu-tèèlu

wear-CAUS-FV    1-child      3-shirt

\glt ‘Dress the child with a shirt’

    b.  *Vwad-ik-a       mu-teelu     mw-âna

  wear-CAUS-FV    3-shirt       1-child

\ea%46
    \label{ex:lukusa:46}
    \z

         a.  Kòsa        mutèèlu      mabòko

      kòs-a        mu-tèèlu    ma-bòko

      cut\_off-FV    3-shirt      6-sleeve

\glt ‘Cut off the sleeves of the shirt’

    b. *Kòs-a       ma-bòko   mu-tèèlu  

        cut\_off-FV    6-sleeve    3-shirt

\ea%47
    \label{ex:lukusa:47}
    \z

            a.  Tùùla          nzòòlo    masàlà

      tùùl-a        n-zòòlo    ma-sàlà

pluck\_off-FV    9-chicken  6-feather

\glt ‘Pluck off the chicken’s feathers’

    b. *Tùùl-a        ma-sàlà    nzòòlo

  pluck\_off-FV    6-feather  9.chicken

\ea%48
    \label{ex:lukusa:48}
    \z

           a.  Twà    mukàndà    {č}ìtampì

      tw-à    mu-kàndà    {č}ì-tampì

put-FV  3-letter      7-stamp

\glt ‘Put the stamp on the letter’

    b.  \textbf{*}Tw-à    cì-tampì    mu-kàndà

  put-FV    7-stamp   3-letter

\ea%49
    \label{ex:lukusa:49}
    \z

          a.  Bàdi        bàloŋolola      nzùbu    bi-manu

      bà-di        bà-loŋolol-a    n-zùbu    bi-manu

SM2-PRS.be  SM2-repair-FV    9-house    8-wall

\glt ‘They are repairing the walls of the house’

    b.  *Bà-di      bà-loŋolol-a    bi-manu    n-zùbu

  SM2-PRS.be  SM2-repair-FV    8-wall    9-house

\textbf{Parameter} \textbf{6:} \textbf{Can} \textbf{either} \textbf{object} \textbf{become} \textbf{subject} \textbf{under} \textbf{passivization?}

No, only one object can become subject under passivization in Čilubà.

\ea%50
    \label{ex:lukusa:50}
    \z

          Mwâna    ùdi    ùtùmina          baledi    mukenji

    mw-âna    ù-di    ù-tùm-in-a        ba-ledi    mu-kenji

1-child    SM1-be  SM1-send-APPL-FV  \textbf{2-parent}  \textbf{3-message}

\glt ‘The child is sending the message to the parents’

\ea%51
    \label{ex:lukusa:51}
    \z

          Mukenji      ùdi        ùtùmiibwa        kùdì  mwâna

    mu-kenji    ù-di        ù-tùm-iibw-a      kùdì  mw-âna

\textbf{3-message}    SM3-PRS.be  SM3-send-PASS-FV  by    \textbf{1-child}

\glt ‘The message is being sent by the child’

Neither of the two following sentences is acceptable in Čilubà for the reasons given below.

\ea%52
    \label{ex:lukusa:52}
    \z

          \textbf{*Mukenji}    ùdi        ùtùmiibw-a          baledi  kùdì    \textbf{mwâna}

      \textbf{mu-kenji}    ù-di        ù-tùm-iibw-a        ba-ledi  kùdì    \textbf{mw-âna}

  \textbf{3-message}  SM3-PRS.be  SM3-send-PASS-FV    2-parent by      \textbf{1-child}

\glt ‘The message is being sent to parents by the child’

\ea%53
    \label{ex:lukusa:53}
    \z

          \textbf{*Baledi}       bàdi    bàtùm-iibw-a         mukenji       kùdi mwâna

      \textbf{ba-ledi}       bà-di     bà-tùm-iibw-a       mu-kenji     kùdi mw-âna

  \textbf{2-parent}    SM2-be  SM2-send-PASS-FV    3-message    by    1-child

\glt ‘The parents are being sent the message by the child’\footnote{Notice that there is nothing wrong in this English translation which is an acceptable English sentence.}

The form in \REF{ex:lukusa:52} is far-fetched and unusual. Using \textit{baledi} ‘parents’ as the object of -\textit{tum-} ‘send’ in Čilubà amounts to assigning to it the role of beneficiary which requires having recourse to the applicative extension which does not normally co-occur with the passive extension. The normal way of conveying the meaning glossed in \REF{ex:lukusa:52} is to use the participle \textit{mutùmina} ‘sent’ as in \REF{ex:lukusa:54} below.

\ea%54
    \label{ex:lukusa:54}
    \z

          \textbf{Mukenji}      ùdi      mutùmina\footnote{The fact that the participle \textit{mutumina} starts with the CM (i.e., the ‘Concord Marker’) /mu-/ which is similar to the adjectival prefix enables it to be used as an adjective in agreement with the attributive use explained in the paragraph following \REF{ex:lukusa:54}.}           baledi    kùdì  \textbf{mwâna}

    \textbf{mu-kenji}    ù-di      mu-tùm-in-a        ba-ledi    kùdì  \textbf{mw-âna}

\textbf{3-message}  SM3-PRS.be  CM3-send-APPL-FV    2-parent    by   \textbf{1-child}

\glt ‘The message is being sent to parents by the child’

The verb \textit{ùdi} ‘to be’ can even be omitted from this attributive sentence and the outcome will be a noun phrase meaning ‘the message sent to parents by the child’. Sentence \REF{ex:lukusa:53}, too, is unnatural and wrong because using the primary object now as the grammatical subject of the sentence is misleading. The resulting sentence is clumsy because Čilubà does not allow the real performer of the action \textit{mwana} (= the child) to co-occur with a fake subject \textit{baledi} (= parents) in the same passive sentence. That is why \REF{ex:lukusa:51} in which \textit{baledi} ‘parents’ is omitted is acceptable while \REF{ex:lukusa:52} and \REF{ex:lukusa:53} are wrong.

\textbf{Parameter} \textbf{7:} \textbf{Can} \textbf{either} \textbf{object} \textbf{be} \textbf{expressed} \textbf{by} \textbf{an} \textbf{object} \textbf{marker?}

Yes, either object can be expressed by an object marker.

However, it is worth remarking that the thematic role difference between the two objects is clearly shown in Čilubà both through their sequential order within the sentence and the position of their respective marker in the verbal word. Just as the primary object which plays the role of the goal, recipient, patient or benefactive argument of the verb tends to be closer to the verb root, and the secondary object (or the theme argument) tends to be far from the verb root, the former (i.e. the primary object) is generally expressed by a pre-verbal OM, while the secondary object comes as a cliticized post-verbal OM, as illustrated in \REF{ex:lukusa:55} to \REF{ex:lukusa:59} below. The class numbers beside the SM and the OM are a clear indication of how they respectively relate to the subject noun and the object nouns.

\ea%55
    \label{ex:lukusa:55}
    \z

          a.  Mw-âna    ú-di      ú-fùnd-il-a          \textbf{ba-ledi}    \textbf{mu-kànda}

      1-child    SM1-be    SM1-write-APPL-FV    \textbf{2-parent}  \textbf{3-letter}

\glt ‘The child is writing a letter to the parents’

b.  Mw-âna    ù-di      ù-\textbf{bà}{}-fùnd-il-a\textbf{{}-ù}

1-child    SM1-be    SM1-\textbf{OM2}{}-write-APPL-FV\textbf{{}-OM3}

\glt ‘The child is writing it to them’

\ea%56
    \label{ex:lukusa:56}
    \z

          a.  Vwàd  {}-ik-à      \textbf{mw-âna}    \textbf{mu-tèèlu}

      wear-CAUS-FV    \textbf{1-child}    \textbf{3-shirt}

\glt ‘Dress the child with a shirt’

b.  \textbf{Mu}{}-vwàd-ik-à\textbf{{}-u}

\textbf{OM1}{}-dress-CAUS-FV\textbf{{}-OM3}

\glt ‘Dress him with it’

\ea%57
    \label{ex:lukusa:57}
    \z

           a.   Bà-di      bà-tùm-in-a      \textbf{ba-ledi}      \textbf{mi-kànda}

      SM2-be    SM2-send-APPL-FV  \textbf{2}{}-\textbf{parent}    \textbf{4-letter}

\glt ‘They are sending the letters to the parents’

b.  Bà-di      bà-\textbf{bà}  {}-tùm-in-a\textbf{{}-yò}

      SM2-be    SM2-\textbf{OM2}{}-send-APPL-FV\textbf{{}-OM4}

\glt ‘They are sending them (i.e., the letters) to them (i.e., the parents)’

\ea%58
    \label{ex:lukusa:58}
    \z

          a.   Tùùl-a      \textbf{n-zòòlo}    \textbf{ma-sàlà}!

      pluck\_off-FV  \textbf{9-chicken}  \textbf{6-feather}

\glt ‘Pluck off the chickens’  feathers!’

b.   \textbf{Mu}{}-tùùl-à\textbf{{}-o}

\textbf{OM9}{}-pluck\_off-FV\textbf{{}-OM6}

\glt ‘Pluck them off it / Pluck them from it’

\ea%59
    \label{ex:lukusa:59}
    \z

          a.   Bà-di        bà-loŋ-olol-a      \textbf{mw-âna}    \textbf{n-sukì}

      SM2-PRS.be  SM2-set-INTS-FV    \textbf{1-child}    \textbf{10-hair}

\glt ‘They are setting the child’s hair’

b.   Bà-di        bà-\textbf{mu}{}-loŋ-olol-a\textbf{{}-yò}

      SM2-PRS.be  SM2-\textbf{OM1}{}-set-INTS-FV\textbf{{}-OM10}

\glt ‘They are setting it (i.e., the hair) on her’

\textbf{Parameter} \textbf{8:} \textbf{Does} \textbf{the} \textbf{relative} \textbf{marker} \textbf{agree} \textbf{with} \textbf{the} \textbf{head} \textbf{noun?}

Yes, the relative marker agrees with the head noun.

Agreement between pronominal markers (e.g., SM, OM, REL (i.e., relative markers)) and the head noun (or antecedent) controlling each is an important characteristic in the grammars of Bantu languages. In Čilubà, the relative marker generally resembles the SM in shape, but tone usually intervenes to differentiate the two.

\ea%60
    \label{ex:lukusa:60}
    \z

          Relative marker

Mu-ntu    \textbf{u}{}-di        nè      m-êsu … 

    1-person  \textbf{REL1}{}-PRS.be  with    6-eye ...

\glt ‘A human being who has eyes ...’

Not to be confused with

\ea%61
    \label{ex:lukusa:61}
    \z

          Subject marker

Mu-ntu    \textbf{ù}{}-di        nè    m-êsu … 

1-human  \textbf{SM1}{}-PRS.be  with  6-eye ...

\glt ‘A human being has eyes.’

As one can see in \REF{ex:lukusa:60} and \REF{ex:lukusa:61}, the relative marker and the subject marker tend to compete for the verb-initial position. However, when both are used in the same sentence as illustrated in \REF{ex:lukusa:62} and \REF{ex:lukusa:63}, the relative marker has the upper hand and comes before the subject marker. The reason for this prevalence of the relative marker on the SM can be justified by the fact that it serves as the connector between its antecedent and the following verb or auxiliary and the rest of the verbal word. Sometimes, the SM is even relegated to the post-verbal\footnote{Or post-auxiliary position if the verb is accompanied by an auxiliary.} position to avoid a clash when the SM is not the 1\textsuperscript{st} or 2\textsuperscript{nd} person (SG or PL).

\ea%62
    \label{ex:lukusa:62}
    \z

          Relative \& pre-auxiliary / pre-verbal subject marker

a.  Mu-ntu    \textbf{u-ù}{}-dì            w-amb-a  

1-person  \textbf{REL1-}  \textbf{SM2SG-}PRS.be  SM2SG-say-FV

\glt ‘The person that \textbf{you} \textbf{are} \textbf{saying} \textbf{...’}

b.  Mu-ntu    \textbf{u}{}-{\textbf{ǹ}}{}-dì            ŋ-àmb-a

1-person  \textbf{REL1-SM1SG}{}-PRS.be  SM1SG-say-FV  

\glt ‘The person that \textbf{I} \textbf{am} \textbf{saying} \textbf{...’}

c.   Mu-kàjì      u-ŋ-{ǎ}{}-mòny-i

1-woman    \textbf{REL1-SM1SG}{}-PRS-see-FV

\glt ‘The woman that \textbf{I} \textbf{have} \textbf{seen} \textbf{...’}

\ea%63
    \label{ex:lukusa:63}
    \z

          Relative marker \& post-auxiliary/post-verbal SM

a.   Mu-ntu    \textbf{u{}-}di\textbf{{}-bò}          b-àmb-a  

1-person  \textbf{REL1-}PRS.be\textbf{{}-SM2} SM2-say-FV

\glt ‘The person that \textbf{they} \textbf{are} \textbf{saying} \textbf{...’}

b.   Mu-ntu    \textbf{u}{}-di\textbf{{}-yè}            w-àmb-a

1-person  \textbf{REL1-}PRS.be\textbf{{}-SM1}    SM1-say-FV  

\glt ‘The person that \textbf{s/he} \textbf{is} \textbf{saying} \textbf{...’}

c.  Mu-kàjì    \textbf{u}{}-di-\textbf{bò}          nendè ...  

1-woman  \textbf{REL1-PRS.be-SM2}  with\_her

\glt ‘The woman whom \textbf{they} have ...’

d.   Mu-kàjì    \textbf{u}{}-di-\textbf{yè}            w-àmb-a...

1-woman  \textbf{REL1-}  \textbf{PRS.be-SM1}    SM1-say-FV

\glt ‘The woman that \textbf{he} is saying ...’

e.   Mu-kàjì      \textbf{w}{}-â-mòny-i-\textbf{yè}  

1-woman    \textbf{REL1-}TAM-see  {}-FV\textbf{{}-SM1}

\glt ‘The woman that \textbf{he} has seen ...’

The examples combining both REL and SM should not be confused with the following ones in which one of the two is used while the other is not.

\ea%64
    \label{ex:lukusa:64}
    \z

          Mu-ntu    \textbf{u{}-}di        \textbf{w-}amb-a  

    1-person  \textbf{REL1-}PRS.be  \textbf{REL1-}say-a

\glt ‘The person who is saying ...’

\ea%65
    \label{ex:lukusa:65}
    \z

          Mu-ntu    \textbf{ù{}-}di        \textbf{w-}àmb-a  

    1-person  \textbf{SM1}{}-PRS.be  \textbf{SM1-}  say-FV

\glt ‘The person is saying ...’

\textbf{Parameter} \textbf{9a:} \textbf{Is} \textbf{an} \textbf{object} \textbf{marker} \textbf{required} \textbf{in} \textbf{object} \textbf{relatives?}

No, an object marker is not required in object relative clauses.

What this parameter is questioning is whether Čilubà allows the use of a resumptive OM which repeats the antecedent of the relative clause as in the Setswana or Swahili sentences given respectively in \REF{ex:lukusa:66} and \REF{ex:lukusa:67} below.

\ea%66
    \label{ex:lukusa:66}
    \z

          Setswana

\textbf{di-buka}    \textbf{tse}    ke      \textbf{di-}rat-ng  

\textbf{10-book}    \textbf{REL10}  SM1SG  \textbf{OM10-}like-REL

\glt ‘The books which I like …’ (Lit.: ‘the books which I like \textbf{them}...’)

\ea%67
    \label{ex:lukusa:67}
    \z

          Swahili

\textbf{Ki-tu}      \textbf{amba-cho}    ni-na-\textbf{chi-}pend-a 

\textbf{7-thing}    \textbf{REL-7}        SM1SG-PRS-OM7\textbf{{}-}like-FV

\glt ‘The thing that I like …’ (Lit.: ‘the thing that I like \textbf{it} ...’)

Indeed, such a resumptive use of the OM is not allowed in Čilubà because the relative marker plays both the role of a pronoun replacing the antecedent noun and a connector linking the relative clause to its antecedent, as explained earlier. The most natural way of saying what example \REF{ex:lukusa:67} means in Čilubà would be:

\ea%68
    \label{ex:lukusa:68}
    \z

          Allowed relative clause (in Čilubà)

{\textbf{či-ntu}    \textbf{čî-n}{}-dì            mu}\footnote{The concord marker /mu-/ refers back to SM1, the subject marker which stands for the omitted Subject ‘m{\textrm{ê}}me’ (= me).}{{}-sw-è}

{\textbf{7-thing}    \textbf{REL7-SM1SG}{}-PRS.be  CM1-like-FV}

{\textbf{‘The} \textbf{thing} \textbf{that} I like...’}

Rather than

\ea%69
    \label{ex:lukusa:69}
    \z

          Disallowed relative clause (in Čilubà)

{\textbf{*či-ntu}    \textbf{čî-n}{}-dì          mu-\textbf{čî}  {}-sw-è}

{        \textbf{7-thing}    \textbf{REL7-SM1}{}-PRS.be  SM1-\textbf{OM7-}like-FV}

{Intd.: \textbf{‘The} \textbf{thing} \textbf{that} I like \textbf{it}...’}

The disallowed Čilubà relative structure in \REF{ex:lukusa:69} resembles the allowed Setswana and Swahili relative clauses in \REF{ex:lukusa:66} and \REF{ex:lukusa:67} above in that the pre-verbal OM repeats the antecedent of REL (i.e., the relative marker).

\textbf{Parameter} \textbf{9b:} \textbf{Is} \textbf{an} \textbf{object} \textbf{marker} \textbf{disallowed} \textbf{in} \textbf{object} \textbf{relatives?}

No, an object marker is not disallowed in object relative clauses.

\ea%70
    \label{ex:lukusa:70}
    \z

          W-à-pe{č-i          bi-ntu  \textbf{bi}{}-ŋ-à-\textbf{mu}{}-tùm-iny-i}

SM1-PRS-receive{{}-FV    8-thing  \textbf{REL8}{}-SM1SG-PRS-\textbf{OM1}{}-send-APPL-FV}

\glt {‘He has received the things \textbf{that} I have sent \textbf{him}’}

Though OMs are not disallowed in Čilubà object relative clauses, one clarification is worth making. Only the secondary OM is not allowed in Čilubà relative clauses for the simple reason that, the relative marker (REL) serves both as (i) the link between the relative clause and (ii) a pronominal element that is both co-referential to the head of the relative clause and a substitute of the secondary object of the verb of the relative clause. So, only primary objects (such as those that fulfill the role of the goal/destination/target/recipient/patient/benefactive argument of the verb) are the ones that occur as OMs in Čilubà object relative clauses.

\ea%71
    \label{ex:lukusa:71}
    \z

          Bal-à      \textbf{mu-kànda}     u-ŋ-à-\textbf{mu}{}-fùnd-id-i

read-FV    \textbf{3-letter}       REL3-SM1SG-PRS-\textbf{OM1-}write-APPL-FV

\glt ‘Read \textbf{the} \textbf{letter} that I have written \textbf{her}’

Sentence \REF{ex:lukusa:71} results from merging \textit{bal-à} \textbf{\textit{mu-kàndà}} (= ‘read \textbf{the} \textbf{letter}’) and \textit{ŋ-à-mu-fùnd-id-i} \textbf{\textit{mu-kàndà}} (= ‘I have written her \textbf{a} \textbf{letter}’).

\ea%72
    \label{ex:lukusa:72}
    \z

           Mòn-a      \textbf{mi-tèèlu}    i-ŋ-à\textbf{{}-ku}{}-sùmb-id-i  

look.at-FV    \textbf{4-shirt}    REL4-SM1SG-PRS\textbf{{}-OM2SG}{}-buy-APPL-FV  

\glt ‘Look at \textbf{the} \textbf{shirts} that I have bought for \textbf{you}’

Sentence \REF{ex:lukusa:72} results from merging \textit{mòn-a} \textbf{\textit{mi-tèèlu}} (= ‘look at the shirts’) and \textit{ŋ{}-à-}\textbf{\textit{ku}}\textit{{}-sùmb-id-i} \textbf{\textit{mi-tèèlu}} (= ‘I have bought you the shirts’).

\ea%73
    \label{ex:lukusa:73}
    \z

          Amb-à    \textbf{ma-alu}      a-tw-à-\textbf{mw}{}-amb-ìd-ì

report-FV  \textbf{6-problem}    REL6-SM1PL-PRS-\textbf{OM1}{}-tell-APPL-FV

\glt ‘Report the problems that we have told \textbf{him’}

Sentence \REF{ex:lukusa:73} results from merging \textit{amb-à} \textbf{\textit{ma-alu}} (= ‘report the problems’) and \textit{tw-â-}\textbf{\textit{mw}}\textit{{}-amb-ìd-ì} \textbf{\textit{ma-alu}} (= ‘we have told him the problems’).

\ea%74
    \label{ex:lukusa:74}
    \z

           Kùn-a      bi-lòŋò    bi-ŋ-à-\textbf{ku}{}-leej-ì

plant-FV    8-flower    REL8-SM1SG-PRS  {}-\textbf{OM2SG}{}-show-FV

\glt ‘Plant the flowers that I have shown \textbf{you}’

Sentence \REF{ex:lukusa:74} results from merging \textit{kùn-a} \textbf{\textit{bi-lòŋò}} (= ‘plant the flowers’) and \textit{ŋ{}-à-}\textbf{\textit{ku}}\textit{{}-leej-ì} \textbf{\textit{bi-lòŋò}} (= ‘I have shown you the flowers’).

\textbf{Parameter} \textbf{10:} \textbf{Is} \textbf{locative} \textbf{inversion} \textbf{thematically} \textbf{restricted} \textbf{to} \textbf{intransitives?}

Yes, locative inversion is thematically restricted to intransitives?

As already said under parameter 3, Čilubà has preserved all the three locative classes (i.e., 16, 17 and 18) of Proto-Bantu. In this language, the locative class prefixes are so productive that almost any noun can take a locative class prefix as a secondary prefix and be used as a subject or a complement/adjunct. 

What parameter 10 implies is that a locative complementing an intransitive verb can be used as a subject of that intransitive verb in an inverted structure while the original subject becomes a pseudo-object. With intransitive verbs, one can say for instance,

\ea%75
    \label{ex:lukusa:75}
    \z

           \textbf{b-enyi}    \textbf{bà-bûŋi}    b-à-bwed-ì        \textbf{mu-n-zùbu} 

\textbf{2-visitors}  \textbf{2-many}    SM2-PRS-enter-FV  \textbf{18-9-house} 

\glt ‘Many visitors have entered the house’

This can be inverted as

\ea%76
    \label{ex:lukusa:76}
    \z

          \textbf{Mu-n-zùbu}      mw-a-bwed-i        \textbf{b-enyi}    \textbf{bà-bûŋi} 

18-9-house      SM18-PRS-enter-FV    2-visitor    2-many

\glt ‘Many visitors have entered the house’

In a like manner, the sentence in \REF{ex:lukusa:77} can be inverted as illustrated in \REF{ex:lukusa:78}

\ea%77
    \label{ex:lukusa:77}
    \z

           \textbf{Bi-ntu}    \textbf{bi-leŋèle}  m-bi-men-à      \textbf{mu-n-kèlèndè} 

8-things    8-good    COP-SM8-grow-FV  18-10-thorn

\glt ‘Good things grow amongst thorns’

\ea%78
    \label{ex:lukusa:78}
    \z

           \textbf{Mu-n-kèlèndè}    m-mu-men-à        \textbf{bi-ntu}    \textbf{bi-leŋèle} 

18-10-thorn    COP-SM18-grow-FV    8-thing    8-good

\glt ‘Amongst thorns grow good things’

Two major observations need to be made about the inverted structures in \REF{ex:lukusa:77} and \REF{ex:lukusa:78}. First, when the locative moves from the sentence-final position to the initial position, it becomes the subject of the sentence. The second observation is that a locative that is used as a subject commands concord with the verb which follows it in that sentence like any other subject in a normal sentence.

Any attempt to invert locatives in sentences including transitive verbs results in ungrammatical sentences, as illustrated below.

\ea%79
    \label{ex:lukusa:79}
    \z

          Ba-ntu    bà-di        bà-nw-a      ma-alà  mu-n-zùbu

2-person  SM2-PRS.be  SM2-drink-FV  6-beer  18-9-house

\glt ‘People are drinking beer in the house’

Sentence \REF{ex:lukusa:79} may neither be inverted as

\ea%80
    \label{ex:lukusa:80}
    \z

           \textbf{*}Mu-n-zùbu  mù-di      mù-nw-a      ma-alà  ba-ntu

  18-9-house  SM18  {}-PRS.be  SM18-drink  {}-FV  6-beer  2-people

nor as

\ea%81
    \label{ex:lukusa:81}
    \z

           \textbf{*}Mu-n-zùbu  mù-di        mù-nw-a       ba-ntu    ma-alà

  18-9-house  SM18  {}-PRS.be    SM18  {}-drink-FV  2-people  6-beer

Notice however that, when locatives are found in existential sentences, Čilubà rather uses the equivalent of ‘to be’ and allows the locative to be either the subject or the adjunct of \textit{{}-ikala} (i.e., to be).

\ea%82
    \label{ex:lukusa:82}
    \z

           The locative as the subject

\textbf{Mu-n-zùbu}     \textbf{mù}{}-di        ba-ntu    bà-bûŋi

\textbf{18-9-house}      \textbf{SM18}{}-PRS.be    2-people  2-many

\glt ‘There are many people in the house’

As said above, sentence \REF{ex:lukusa:82} may be inverted as

\ea%83
    \label{ex:lukusa:83}
    \z

          The locative as the adjunct

\textbf{Ba-ntu}    \textbf{bà-bû.ŋi}    \textbf{bà}{}-di        \textbf{mu-n-zùbu}

\textbf{2-people}  \textbf{2-many}    \textbf{SM2}{}-PRS.be  \textbf{18-9-house}  

The locative which was the subject in \REF{ex:lukusa:82} becomes the post-verbal adjunct in \REF{ex:lukusa:83} while \textit{ba-ntu bà-bûŋi} (= many people) becomes the subject.

\textbf{Parameter} \textbf{11:} \textbf{Are} \textbf{there} \textbf{three} \textbf{different} \textbf{locative} \textbf{subject} \textbf{markers?}

Yes, there are.

As already said under parameter 3, all the three locative classes (i.e., 16, 17, and 18) of Proto-Bantu have survived in Čilubà and agreement with the verb uses a specific SM related to the class of the locative in question, as shown in the following examples.

\ea%84
    \label{ex:lukusa:84}
    \z

           \textbf{Pa-}{\textbf{č}}\textbf{y-âta}    \textbf{pà}{}-di      lu-pwishi  lw-a-bûŋi

\textbf{16-7-mat}    \textbf{SM16}{}-be  11-dust    11-GEN-many

\glt ‘There is a lot of dust on the mat’

\ea%85
    \label{ex:lukusa:85}
    \z

           \textbf{Ku-mbèlù}    \textbf{kw-}èndè    \textbf{kù}{}-di      \textbf{kù}{}-teem-a        m-indà

\textbf{17-9.plot}    \textbf{17-}his      \textbf{SM17}{}-be  \textbf{SM17-}glitter-FV    4-light

\glt ‘His plot is glowing with lights / Lights are shining at his plot’

\ea%86
    \label{ex:lukusa:86}
    \z

          \textbf{Mu-nz-ùbù}    \textbf{mw}{}-ètù    \textbf{mù-}di        \textbf{mu-}fììka

\textbf{18-9-plot}    \textbf{18-}our    \textbf{SM18-}PRS.be    \textbf{18-}dark

\glt ‘The interior of our house is dark’

\textbf{Parameter} \textbf{12:} \textbf{Is} \textbf{partial} \textbf{agreement} \textbf{with} \textbf{conjoined} \textbf{NPs} \textbf{possible?}

No, it is not possible.

Parameter 12 questions whether agreement with conjoined nouns is commanded by one of the two conjoined NP’s or both. Examples (87a \& b) below prove that in Čilubà, when the two conjoined nouns are [- human] they compromise by using class 8 SM regardless of their own classes and no matterf which of the two nouns comes first. It should be noted here that class 8 is the plural class of things (i.e. [-human] nouns) in many Bantu languages.

\ea%87
    \label{ex:lukusa:87}
    \z

          Conjoined [- human] nouns

a.   \textbf{Kà-làvwandà}   nè   \textbf{mu-kàbà}  \textbf{bì-}di        \textbf{bi-}fwànàŋàne

\textbf{13-tie}       and   \textbf{3-belt}    \textbf{SM8}{}-PRS.be  \textbf{8-}similar

\glt ‘The tie and the belt are similar’

b.  \textbf{Mu-kàbà}   nè   \textbf{kà-làvwandà}    \textbf{bì-}di        \textbf{bi-}fwànàŋàne

\textbf{3-belt}     and   \textbf{13-tie}        \textbf{SM8}{}-PRS.be  \textbf{8-}similar

\glt ‘The tie and the belt are similar’

On the other hand, when the conjoined nouns are [+ human], they use a class 2 SM regardless of their own classes and regardless of which of the two nouns comes first. It is also worth noting that class 2 is used to form the plural form of class 1 i.e. [+ human] nouns.

\ea%88
    \label{ex:lukusa:88}
    \z

          Conjoined [+ human] nouns

a.   \textbf{Ka-màmà}               nè   \textbf{mu-loŋeshi}    \textbf{bà\--}di        \textbf{bà}{}-tànd-aŋan-a

\textbf{13-speech.impaired.person}   and   \textbf{1-teacher}    \textbf{SM2}{}-\-PRS.be  \textbf{SM2}{}-quarrel-RECI-FV

\glt ‘The speech-impaired person and the teacher are quarrelling’

b.   \textbf{Mu-loŋeshi}   nè   \textbf{ka-màmà}              \textbf{bà\--}di        \textbf{bà}{}-tànd-aŋan-a

\textbf{1-teacher}     and   \textbf{13-speech.impaired.person}  \textbf{SM2-}PRS.be  \textbf{SM2}{}-quarrel-RECIP-FV

\glt ‘The speech-impaired person and the teacher are quarrelling’

When the two nouns are disjoined, the story is different. In this case, since Čilubà is an SVO language, it is the noun that comes before the verb which commands concord with the verb.

\ea%89
    \label{ex:lukusa:89}
    \z

           Disjoined nouns 

a.  \textbf{Kà-làvwandà}  \textbf{kà-}di        \textbf{ka}{}-fwànàŋàne    ne    \textbf{mu-kàbà}

\textbf{13-tie}      \textbf{SM13}{}-PRS.be  \textbf{13-}similar      to    \textbf{3-belt}

\glt ‘The tie is similar to the belt’

b.  \textbf{Mu-kàbà}  \textbf{ù}{}-di        \textbf{mu-}fwànàŋàne  nè    \textbf{kà-làvwandà}

\textbf{3-belt}    \textbf{SM3-}PRS.be  \textbf{3-}similar        to    \textbf{13-tie}

\glt ‘The belt and the tie are similar’

c.  \textbf{Ka-màmà}              \textbf{kà-}di        \textbf{kà}{}-tànd-aŋan-a        nè   \textbf{mu-loŋeshi}

\textbf{13-speech.impaired.person}  \textbf{SM13}{}-PRS.be  \textbf{SM13}{}-quarrel-RECI-FV    with   \textbf{1-teacher}

\glt ‘The speech-impaired person is quarrelling the with the teacher’

d. \textbf{Mu-loŋeshi}    \textbf{ù-}di        \textbf{ù}{}-tànd-aŋan-a        nè   \textbf{ka-màmà}

\textbf{1-teacher}    \textbf{SM1}{}-PRS.be  \textbf{SM1}{}-quarrel-RECI-FV  with \textbf{13-speech.imp.pers.}

\glt ‘The speech-impaired person is quarrelling the with the teacher’

\textbf{Parameter} \textbf{13:} \textbf{Is} \textbf{there} \textbf{a} \textbf{(tonal)} \textbf{distinction} \textbf{between} \textbf{conjoint} \textbf{⁄} \textbf{disjoint} \textbf{forms?}

No, there is no tonal distinction between conjoint / disjoint forms.

No such distinction can be found in Čilubà.

\textbf{Parameter} \textbf{14:} \textbf{Is} \textbf{there} \textbf{a} \textbf{(tonal)} \textbf{distinction} \textbf{of} \textbf{nominal} \textbf{‘cases’?}

No, there are no tonal distinction of nominal cases.

There is no difference in the tonal pattern of nouns according to their position in the clause. The following examples show that the noun phrase \textit{bâna bakesè} ‘few/small children’ keeps the same tonal pattern regardless of whether it is found before the predicate as a subject or after it. The same is true for the locative \textit{mu-kàlasà} ‘in the classroom’ which is toned the same way regardless of its position at the end of the sentence in \REF{ex:lukusa:90} as the locative adjunct or as the subject of the existential sentence in \REF{ex:lukusa:91}.

\ea%90
    \label{ex:lukusa:90}
    \z

          B-âna    ba-kesè    bà-di        mu-kà-laasà

2-child  2-few      SM2-PRS.be  18-13-classroom

\glt ‘Few students are in the classroom’

\ea%91
    \label{ex:lukusa:91}
    \z

           Mu-kà-laasà      mù-di        bâ-na     ba-kesè

18-13-classroom    SM18-PRS.be    2-children  2-few

\glt ‘There are few students in the classroom’

\textbf{4.} \textbf{SUMMARY} \textbf{AND} \textbf{CONCLUDING} \textbf{REMARKS}

\textbf{Summary}

After a scrupulous examination of Čilubà examples given in answer to the parameters that were used, the foregoing discussion has yielded the results that are briefly presented in the following table:

\ea%92
    \label{ex:lukusa:92}
    \z

           Summary of the Results


\begin{tabularx}{\textwidth}{XXXX}

\lsptoprule

\multicolumn{2}{c}{ \textbf{Parameter}} & \textbf{Answer} & \textbf{Further} \textbf{Elaboration} \textbf{of} \textbf{the} \textbf{findings}\\
\textbf{No.} & \textbf{Summary} \textbf{Statement} &  & \\
1 & OM – Obj.NP co-occurrence? & No (and yes to some degree) & Object noun and OM cannot co-occur. It is ungrammatical in the Standard language which is based on L1 Čilubà speakers. 

Variability in acceptance, and that grammaticality depends in part on the specific valency, but also in part on the difference between first and second language speakers (e.g., the latter being influenced by contact with Swahili) \\
2 & OM Obligatory? & Yes & For the sake of emphasis, an object NP can be fronted in Čilubà and therefore be allowed to co-occur with the object marker in the same context\\
3 & Loc OMs? & Yes & There are three locative OMs in Čilubà\\
4a & One OM restriction? & No & Object marking is not restricted to one object per verb in Čilubà\\
4b & Restricted 2 OMs? & Yes & Two OMs are possible in certain Čilubà contexts.\\
4c & Possible Multiple OMs? & No & Two or more OMs are either possible, or not possible in certain Čilubà contexts\\
4d & Free order of multiple OMs? & No & The order of multiple OMs is not structurally free. It is fixed in Čilubà.\\
5 & Symmetrical order of OMs? & No & Only one object can be adjacent to the verb (root) in Čilubà.\\
6 & Symmetrical Passive? & No & Only one Object (the secondary object) can become the subject of a multiple object sentence in Čilubà.\\
7 & Symmetrical OMs? & Yes & Either object (secondary or primary) can be expressed by an OM in Čilubà\\
8 & Agreement of Rel. Marker Obligatory? & Yes & The relative marker agrees with the head noun in Čilubà\\
9a & Rel. OM Obligatory? & No & The object marker is not required in Čilubà object relative clauses\\
9b & Rel.OM barred? & No & An OM is not disallowed in object relative clauses\\
10 & Loc. Inversion Restricted? & Yes & Locative Inversion is thematically restricted to intransitives\\
11 & Full Loc. Subject Marking? & Yes & There are three fully different locative subject markers in Čilubà\\
12 & Partial Agreement & No & In case of conjoined NPs function as the subject, partial agreement is not possible\\
13 & Tonal distinction between Conj./Disjoined? & No & There is no tonal distinction between conjoined / disjoined forms of the same NP in Čilubà\\
14 & Tone case? & No & There are no tonal distinctions of nominal cases\\
\lspbottomrule
\end{tabularx}
\textbf{Concluding} \textbf{Remarks}

To complete this discussion of object marking in Čilubà, it is useful to wrap it up with the following concluding remarks:

\begin{itemize}
\item Čilubà has preserved 18 of the 23 noun classes that have been identified in different Bantu languages.
\item In this language, object marking is strictly controlled by the grammatical rule of agreement which relates each and every OM to the relevant antecedent (regardless of whether it is present or omitted by ellipsis).
\item In case of conjoined NPs, Čilubà uses class 8 or 2 concord marker depending on whether the conjoined nouns are respectively [- Human] or [+ Human].
\item Under normal circumstances, the antecedent lexical object NP and the OM referring to it cannot co-occur in a Čilubà sentence. However, co-occurrence is possible when object NP is fronted for the sake of emphasis. Under such circumstances, the former is fronted and set apart by a pause in speaking which is reflected by a comma in writing.
\item Contrary to the situation in Swahili, there is variability in the acceptance of sentences in which a lexical OM is followed by a pleonastic OM (which literally repeats the former). This variability is regulated by: (i) Grammaticality depending on the specific valency of a verb, but also in part on the difference between first and second language speakers (e.g., the latter, under the influence of Swahili (their first language) transfer the Swahili pleonastic use of the OM to Čilubà which is reluctant to allow it.~
\item Single OM occurrences are common when mono-transitive verbs are used. Double object sentences are also possible when a ditransitive verb is used. Such a verb normally supports a primary object, which may have the role of the goal, recipient, patient or benefactive argument of the verb and which tends to be closer to the verb root; while the secondary object which is the theme argument tends to be far from the verb root.
\item In Čilubà, the thematic role difference between the two objects is reflected both through their sequential order within the sentence and the position of the respective marker in the verbal word.
\item In the competition between the relative marker and the subject marker for the verb-initial position, the relative marker has the upper hand in that it always comes before the subject marker. Sometimes, the SM is even relegated to the post-verbal position to avoid a clash.
\end{itemize}

\sloppy\printbibliography[heading=subbibliography,notkeyword=this]
\end{document}
