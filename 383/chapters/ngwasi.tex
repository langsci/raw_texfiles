\documentclass[output=paper,
            colorlinks, citecolor=brown
            % ,draft
            ,draftmode
		  ]{langscibook}
\ChapterDOI{10.5281/zenodo.10663783}
    
\author{Lengson Ngwasi\orcid{}\affiliation{University of Dar es Salaam} and Abel Mreta\orcid{}\affiliation{University of Dar es Salaam}}

\title{The historical development of the reflexive-reciprocal polysemy in Hehe}

\abstract{This chapter describes the encoding of reflexive and reciprocal events in Hehe, a Bantu language spoken in Tanzania. It is argued that the reflexive prefix has historically developed into a reciprocal marker, thus, replacing the reflex of the Proto-Bantu reciprocal suffix *-\textit{an}-. As such, the reflexive prefix encodes both reflexive and reciprocal meanings. The data presented and analyzed in this chapter show that there are some remnants of the Proto-Bantu reciprocal suffix *-\textit{an}- in a very few list of verbs encoding inherent reciprocal events, suggesting that this suffix was productive at some point in the history of the Hehe language. The analysis of the development from reflexive to reciprocal marker follows the three stages of grammaticalization theory proposed by \citet{Heine1993} and  applied in the analysis of German reflexive and reciprocal constructions by \citet{HeineNarrog2009}. Following \citet{Ngwasi2021}, it is shown in this chapter that, unlike German, Hehe attests a fourth stage in the grammaticalization from reflexive to reciprocal marker. The fourth stage is evidenced by the recruitment of the reflexive prefix encoding events, such as chaining and associative, that are closer to the reciprocal prototype.
}

\IfFileExists{../localcommands.tex}{
  \addbibresource{../localbibliography.bib}
  % add all extra packages you need to load to this file

\usepackage{tabularx,multicol}
\usepackage{url}
\urlstyle{same}

\usepackage{listings}
\lstset{basicstyle=\ttfamily,tabsize=2,breaklines=true}

\usepackage{langsci-basic}
\usepackage{langsci-optional}
\usepackage{langsci-lgr}
\usepackage{langsci-osl}
% \usepackage{./langsci/styles/langsci-lgr}
% \usepackage{./langsci/styles/langsci-osl}
% \usepackage{langsci-gb4e}

\usepackage{tikz}
\usetikzlibrary{patterns,calc}
\pgfdeclarepatternformonly{south east lines}{\pgfqpoint{-0pt}{-0pt}}{\pgfqpoint{3pt}{3pt}}{\pgfqpoint{3pt}{3pt}}{
    \pgfsetlinewidth{0.6pt}
    \pgfpathmoveto{\pgfqpoint{0pt}{3pt}}
    \pgfpathlineto{\pgfqpoint{3pt}{0pt}}
    \pgfpathmoveto{\pgfqpoint{.2pt}{-.2pt}}
    \pgfpathlineto{\pgfqpoint{-.2pt}{.2pt}}
    \pgfpathmoveto{\pgfqpoint{3.2pt}{2.8pt}}
    \pgfpathlineto{\pgfqpoint{2.8pt}{3.2pt}}
    \pgfusepath{stroke}}
    
\usepackage{stmaryrd}
\usepackage{wasysym}
\usepackage{multirow}
\usepackage{caption}
\usepackage{subcaption}
\usepackage{mathrsfs}
\usepackage{qtree}

\usepackage{linguex}


  %pminos do not split footnotes
% \interfootnotelinepenalty=10000 %Footnote in Laporte chapters has to be split SN


%\DeclareIndexNameFormat{default}{%
%\nameparts{#1}%
%\usebibmacro{index:name}%
%{\index[names]}%
%{\namepartfamily}%
%{\namepartgiveni}%
% {}% L1
% {}% L2
%{\namepartprefix}% generates spurious space L3
%{\namepartsuffix}% generates spurious space L4
%}

%  {\DeclareIndexNameFormat{default}{%
%     \usebibmacro{index:name}{\index[names]}{#1}{#3}{#5}{#7}}}

%\DeclareIndexNameFormat{default}{%
%  \usebibmacro{index:name}{\sindex[nom]}{#1}{#3}{#5}{#7}}

%\DeclareIndexNameFormat{default}{%
%  \usebibmacro{index:name}{\sindex[person]}{#1}{#3}{#5}{#7}}
%\DeclareIndexNameFormat{default}{%
%\nameparts{#1} \usebibmacro{index:name}{\sindex[person]]}{\namepartfamily}{‌​\namepartgiven}{\nam‌​epartprefix}{\namepa‌​rtsuffix}}

%\newcommand{\smiley}{:)}

%\renewbibmacro*{index:name}[5]{%
%\usebibmacro{index:entry}{#1}%
%{\iffieldundef{usera}{}{\thefield{usera}\actualoperator}\mkbibindexname{#2}{#3}{#4}{#5}}}

% \newcommand{\noop}[1]{}

%remove for final
%\overfullrule=1mm

\newcommand{\tobi}[2]}}
\renewcommand{\S}[1]{\tobi{#1}{\textsc{*}}}

% this volume references
% puts: [this volume]
% already defined: \citetv
%\newcommand{\citepv}[1]{(\citeauthor{#1} \citeyear*{#1} [this volume])}
\newcommand{\citealtv}[1]{\citeauthor{#1} \citeyear*{#1} [this volume]}

%parentheses around example number
\newcommand{\pref}[1]{(\ref{#1})}

% in-text examples

\newcommand{\lnex}[1]{\textit{#1}} %target lang word
\newcommand{\lnlit}[1]{(lit.: `#1')} %literal reading
\newcommand{\lnlat}[1]{(#1)} % latinization
\newcommand{\lntrans}[1]{`#1'} %translation
\newcommand{\lnexl}[2]%
{\lnex{#1}{} \lnlat{#2}} % ex with latinization
\newcommand{\lnexlat}[3]{\lnex{#1}{} \lnlat{#2}{} \lntrans{#3}} % ex with latinization and tranl.

%ch01
\newcommand{\co}[1]{\mbox{\textbf{#1}}}

%ch09

\newcommand{\cyrbulg}[1]{\begin{otherlanguage*}{bulgarian}#1\end{otherlanguage*}}


%ch10
\newcommand{\nlp}{{\small NLP}}
\newcommand{\mwe}{{\small MWE}}
\newcommand{\rae}{{\small RAE}}
\newcommand{\lvc}{{\small LVC}}
\newcommand{\pos}{{\small P}o{\small S}}
%\newcommand{\todo}[1]{ \textcolor{red}{#1} }

%\renewcommand{\labelenumi}{\theenumi}
%\ainamefmt{{vv}{ll}{, ff}{, jj}} % fullname

\newcommand{\biberror}[1]{{\color{red}#1}}

\newcommand{\osenovaitem}{--~} 
  %% hyphenation points for line breaks
%% Normally, automatic hyphenation in LaTeX is very good
%% If a word is mis-hyphenated, add it to this file
%%
%% add information to TeX file before \begin{document} with:
%% %% hyphenation points for line breaks
%% Normally, automatic hyphenation in LaTeX is very good
%% If a word is mis-hyphenated, add it to this file
%%
%% add information to TeX file before \begin{document} with:
%% %% hyphenation points for line breaks
%% Normally, automatic hyphenation in LaTeX is very good
%% If a word is mis-hyphenated, add it to this file
%%
%% add information to TeX file before \begin{document} with:
%% \include{localhyphenation}
\hyphenation{
    Beck-man
    Ngu-yen
    back-chan-nel
    back-chan-nels
    mo-not-o-nous
    ste-reo-typ-i-cal
}

\hyphenation{
    Beck-man
    Ngu-yen
    back-chan-nel
    back-chan-nels
    mo-not-o-nous
    ste-reo-typ-i-cal
}

\hyphenation{
    Beck-man
    Ngu-yen
    back-chan-nel
    back-chan-nels
    mo-not-o-nous
    ste-reo-typ-i-cal
}
 
  \togglepaper[1]%%chapternumber
}{}

\begin{document}
\maketitle 
%\shorttitlerunninghead{}%%use this for an abridged title in the page headers


\section{Introduction}\label{sec:ngwasi:1}

Many Bantu languages distinguish two morphemes for encoding reflexive and reciprocal events in terms of their forms and their morphological distribution (see \citealt{Meeussen1967, Schadeberg2003, SchadebergBostoen2019}). The reflexive events are most often encoded by a reflexive prefix that occurs in the OM slot, located immediately before the verb root in the morphological structure of the verb. The reflexive prefix’s shape can be a single vowel, such as -\textit{i}{}- in Hehe (G62) (\citealt{Msamba2013, Ngwasi2016,Ngwasi2021}), or CV, such as -\textit{ki}{}- in Kagulu (G12) \citep{Petzell2008} and some languages of the Kikongo Language Cluster (KLC) (\citealt{DomKulikov2019}). In turn, reciprocal events are predominantly encoded by a suffix which has the form -\textit{an}{}-, or a compound form involving -\textit{an}{}-, e.g., \textit{angan}{}- or -\textit{asan}{}-, as in the KLC (\citealt{BostoenEtAl2015}); or in Runyambo (JE21) \citep{Rugemalira1993runyambo}. Interestingly, \citet{Polak1983} (see also \citealt{Marlo2015exceptional}) notes that in some Bantu languages, the reflexive and reciprocal events are encoded by the same verbal morpheme. On the one hand, there are languages where the reflexive prefix has been recruited to encode reciprocal events (e.g., Bolia (C35b), Chokwe (K11), Ganda (JE15), Lunda (L52) etc.), whereas, in other languages, the reciprocal suffix encodes reflexive events (e.g., Ewondo (A72) and Tsogo (B31)) \citep{Marlo2015exceptional}. The first case is more widespread while the second case is extremely rare in Bantu languages. This chapter focuses on the first case by describing how the reflexive prefix has developed to encode reciprocal events, thus being polysemous in Hehe, particularly in Dzungwa\footnote{The data presented in this chapter were collected at Bomalang’ombe village, one of the villages where the Dzungwa dialect is spoken. Data from the other dialect come from \citet{Msamba2013}. In the rest of this chapter, we will use the term Hehe or Dzungwa when referring to the Dzungwa dialect, and where data are cited from the other dialect, we will use the term ``Standard'' Hehe.} dialect. The analysis for the development from reflexive to reciprocal follows the three stages of grammaticalization theory proposed by \citet{Heine1993} and applied by \citet{HeineNarrog2009} in the grammaticalization of the reflexive marker to reciprocal marker in German. I will add the fourth stage that has not been applied by \citet{HeineNarrog2009} in the analysis of the grammaticalization from reflexive to reciprocal marker in German, but this stage has been applied by \citet{Ngwasi2021}. It will be argued that the reflexive\nobreakdash-reciprocal marker in Hehe is a result of the grammaticalization process leading from reflexive to reciprocal marker, taking over the role of the reflex of the Proto-Bantu reciprocal suffix *-\textit{an}{}- which is no longer productive in this language. The remnants of the Proto-Bantu reciprocal suffix *-\textit{an}{}- are found in a very few verbs encoding inherent reciprocal events. 

Before introducing the language under study, we first define the terms used in this chapter, which are: (i) prototypical reflexive event or situation, (ii) prototypical reciprocal event or situation, and (iii) inherent reciprocal event or situation. By a prototypical reflexive event, we refer to a two-participant event type where the agent and the patient/theme refer to the same participant (see \citealt{Faltz1985, Haspelmath2019, Kemmer1993}). In other words, as \citet[107]{Moyse-Faurie2008} points out, prototypical reflexive events express actions or events that one usually performs on other entities being performed on oneself, as exemplified in \REF{ex:ngwasi:1}. 

\ea\label{ex:ngwasi:1}
John hit \textbf{himself} with a hammer.
\z

A prototypical reciprocal event encodes a similar or symmetric relation between two participants A and B, where A acts on B and B acts on A (see \citealt{Haspelmath2007, Kemmer1993, KönigKokutani2006}). It should be noted that the prototypical reciprocal events are neither necessarily nor very frequently semantically reciprocal (see \citealt{DomEtAlForthcoming, Haiman1983}). And they include events such as ‘punching each other’, ‘seeing each other’, ‘hitting each other’, ‘cutting each other’, ‘killing each other’, etc. The example in \REF{ex:ngwasi:2} illustrates the construction encoding a prototypical reciprocal event for English, where John and Bill are mutually involved in the punching action.

\ea\label{ex:ngwasi:2}
John and Bill punched \textbf{each} \textbf{other}.
\z

An inherent or natural reciprocal event is an event type that necessarily or very frequently expresses reciprocal situations (see \citealt{Kemmer1993, KönigKokutani2006, Nedjalkov2007}). \citet[104]{Kemmer1993} lists verbs which encode inherent reciprocal events cross-linguistically, which are: verbs of antagonistic actions (‘fight’, ‘quarrel’, ‘wrestle’), verbs of affectionate actions (‘kiss’, ‘embrace’, ‘make love’), verbs of encountering and associations (‘meet’, ‘greet’, ‘shake hands’), verbs of actions denoting unintentional physical contact (‘bump into’, ‘collide’), verbs of physical convergence or proximity (‘touch’, ‘join’, ‘unite’, ‘be close together’), verbs of exchanging (‘trade’, ‘share’, ‘divide’, ‘split’), verbs of agreement/disagreement  (‘converse’, ‘argue’, ‘gossip’, ‘correspond’), and verbs of similarity/dissimilarity (‘resemble’). The examples in \REF{ex:ngwasi:3} illustrate a construction encoding inherent reciprocal event for English.

\ea\label{ex:ngwasi:3}
John and Bill \textbf{met}.
\z

With this brief introduction of reflexive and reciprocal events, we turn to the introduction of Hehe language. Hehe is spoken mainly in the Iringa region of Tanzania. It is classified as G62 by \citet{Guthrie1948, Guthrie1967-1971} and \citet{Maho2009}, and it is closely related to other G60 languages, such as Sangu (G61), Bena (G63), Pangwa (G64), Kinga (G65), Wanji (G66), and Kisi (G67). Hehe was reported to have approximately 598,839 native speakers by \citet{LOT2009}, but recently, Ethnologue Languages of the World reports the number of native speakers to be approximately 1,210,000, as of 2016 (\citealt{EberhardEtAl2020}). In terms of dialects, there is no agreement among scholars on the number of dialects of Hehe. For instance, \citet{Madumula1995} identifies five dialects, \citet{Mpalanzi2010} identifies three dialects, while \citet{Haonga2013} identifies two dialects called ``Standard'' Hehe and Dzungwa (also called Tsungwa by its native speakers). We follow \citegen{Haonga2013} analysis of the dialectal variation of Hehe since it is the only source that is solely based on linguistic evidence, i.e., phonological, morphological, syntactic, and semantic evidence. As already noted above, this chapter focuses on the Dzungwa dialect with sporadic reference to the other dialect, the so-called ``Standard'' dialect, where the data are accessible. 

Like many other Bantu languages, Hehe is ``verby'' in the sense that the verb root can be attached with several morphemes for various inflectional and derivational functions (see \citealt{Nurse2008}). The structure of Hehe verbs can be demonstrated by examples \REF{ex:ngwasi:4}--\REF{ex:ngwasi:7}, elaborating the templatic structure shown in \tabref{tab:ngwasi:1} below, as extracted from \citet{Ngwasi2016}.

\begin{table}
%\small
\begin{tabularx}{\textwidth}{lllllllllll}

\lsptoprule

1 & 2 & 3 & 4 & 5 & 6 & 7 & 8 & 9 & 10 & 11\\
\textsc{rel} & \textsc{neg1} & \textsc{sm} & \textsc{neg2} & \textsc{tam} & \textsc{om/refl}\nobreakdash-\textsc{rec} & \textsc{vr} & \textsc{ext} & \textsc{pfv} & \textsc{fv} & \textsc{clit}\\
\lspbottomrule
\end{tabularx}
\caption{The structure of Hehe verbs \citep[50]{Ngwasi2016}}
\label{tab:ngwasi:1}
\end{table}

\ea\label{ex:ngwasi:4}
\glll \textit{yesiakutsági}\\
ye-si-a-ku-ts-ág-i\\
 \textsc{rel-neg1-sm1-tam}-come-\textsc{hab-fv}\\
 \glt ‘S/he who does not normally come.’

\ex\label{ex:ngwasi:5}
\glll \textit{witóve}\\
 u-i-tóv-e\\
\textsc{sm1-refl}-beat-\textsc{imp/sbj}\\
\glt ‘Beat yourself’

\ex\label{ex:ngwasi:6}
\glll \textit{alakulimítsa}\\
a-la-ku-lim-íts-a\\
\textsc{sm1-neg2.sbj-tam}-cultivate-\textsc{caus-fv}\\
\glt ‘S/he should not make you cultivate.’


\ex\label{ex:ngwasi:7}
\glll \textit{vaitseengíte}\footnotemark{}\\
va-i-tseeng-íte\\
\textsc{sm2-om9}-build-\textsc{pfv}\\
\glt ‘They have built it.’
\z
\footnotetext{It should be noted that the class 9 object prefix -\textit{i}{}-, unlike the reflexive prefix -\textit{i}{}-, does not trigger the deletion of the vowel \textit{a} of the subject marker \textit{va}{}- (see \citet{Ngwasi2016} on vowel deletion and glide formation triggered by the reflexive prefix in Hehe).}


As can be seen from \tabref{tab:ngwasi:1}, the productive reciprocal marker occupies slot 6, the slot for OM and reflexive markers in many Bantu languages. As will be argued later in this chapter, the reciprocal marker occupies this slot as a result of the historical development (grammaticalization) whereby the reflexive prefix has undergone grammaticalization and has taken over the role of the reflex of the Proto-Bantu reciprocal suffix *\textit{\nobreakdash-an-}. As such, both reflexive and reciprocal meanings are productively expressed by the same morpheme, the reflexive prefix -\textit{i-}, occupying the OM slot, as will be discussed further in \sectref{sec:ngwasi:2}.

The remainder of this chapter is divided into four sections. \sectref{sec:ngwasi:2} provides an overview of the construction types where the Hehe reflexive prefix has various functions, particularly those encoding reflexive and reciprocal events. \sectref{sec:ngwasi:3} introduces grammaticalization theory and discusses the rise of the reflexive-reciprocal polysemy in Hehe, as explained from a grammaticalization perspective. \sectref{sec:ngwasi:4} briefly highlights the loss of the reflex of the Proto-Bantu reciprocal suffix *-\textit{an}{}- in Hehe and the emergence of the reflexive prefix -\textit{i}{}- as a new means of encoding reciprocal events. \sectref{sec:ngwasi:5} concludes the discussion.

\section{Reflexive-reciprocal polysemy: An overview of construction types}\label{sec:ngwasi:2}

This section describes various constructions where the reflexive prefix -\textit{i}{}- encodes exclusive reflexive events, ambiguous reflexive-reciprocal events, and exclusive reciprocal events. The construction types which we focus on in this section are infinitive constructions (\sectref{sec:ngwasi:2.1}), constructions with singular subjects (\sectref{sec:ngwasi:2.2}), and constructions with plural subjects with ambiguous reflexive-reciprocal interpretation and those with exclusive reciprocal interpretation (\sectref{sec:ngwasi:2.3}).

\subsection{Infinitive constructions}\label{sec:ngwasi:2.1}

The reflexive prefix -\textit{i}{}- encodes ambiguous reflexive-reciprocal meanings or exclusively reciprocal meaning in infinitive constructions, as can be exemplified by the examples in \REF{ex:ngwasi:8} and \REF{ex:ngwasi:9}. As shown in \REF{ex:ngwasi:8}, the reflexive prefix -\textit{i}{}- has an ambiguous reflexive-reciprocal interpretation. This is because the verb in this construction is neither necessarily nor frequently semantically reciprocal, while in \REF{ex:ngwasi:9}, the reflexive prefix -\textit{i}{}- has only a reciprocal interpretation because the verb is semantically reciprocal (see \citealt{Nedjalkov2007} for an overview of cross-linguistic encoding of inherent reciprocal events).

\ea\label{ex:ngwasi:8}
\glll \textit{kw}\textbf{\textit{í}}\textit{bumíla}\\
kú-\textbf{i}{}-bumíl-a\\
\textsc{inf-refl-rec}-hit-\textsc{fv}\\
\glt ‘to hit oneself’ or ‘to hit each other’

\ex\label{ex:ngwasi:9} 
\glll \textit{kw}\textbf{\textit{í}}\textit{huungíla}\\
kú-\textbf{i}{}-huungíl-a\\
\textsc{inf-rec}-greet-\textsc{fv}FV\\
\glt ‘to greet each other’
\z

\subsection{Constructions with singular subjects}\label{sec:ngwasi:2.2}

The reflexive prefix \textit{{}-i-} renders only reflexive meaning with constructions having singular subjects in Hehe, as exemplified in \REF{ex:ngwasi:10}. In fact, the constructions with singular subjects have only reflexive meaning because the plurality of the participants, which is a key defining property of constructions encoding reciprocal events, is not available. As already noted above in \sectref{sec:ngwasi:1}, the definition of reciprocal events or situations requires plural participants (see \citealt{Frajzyngier2000, HeineMiyashita2008, Lichtenberk2000}).


\ea\label{ex:ngwasi:10}
\textit{Juma ak}\textbf{\textit{i}}\textit{bumyé}\footnotemark{}\\
\gll Juma   a-ka-\textbf{i}{}-bumíl-íle\\
Juma   \textsc{sm1-pst-refl}-hit-\textsc{pfv}\\
\glt ‘Juma hit himself.’
\z
\footnotetext{It should be noted that the perfective suffix -\textit{íle} triggers imbrication with some verb roots or stems in Hehe, as can be seen in example \REF{ex:ngwasi:10}. See \citet{Bastin1983} and \citet{Hyman1995} for a detailed discussion on imbrication in Bantu.}

It should be noted that constructions with singular subjects can optionally occur with emphatic reflexive pronouns for emphasis in Hehe, as illustrated in \REF{ex:ngwasi:11}. The emphatic reflexive pronouns, just like in English, can also follow the subject NP it emphasises.

\ea\label{ex:ngwasi:11}
\textit{Juma ak}\textbf{\textit{i}}\textit{bumyé} \textbf{\textit{yimwene}}\\
\gll Juma   a-ka-\textbf{i}{}-bumíl-íle    \textbf{yimwene}\\
Juma   \textsc{sm1-pst-refl}-hit-\textsc{pfv}  \textsc{emph}\\
\glt ‘Juma hit himself.’
\z


\subsection{Constructions with plural subjects}\label{sec:ngwasi:2.3}

The reflexive prefix \textit{{}-i-} encodes ambiguous reflexive-reciprocal meaning in constructions with plural subjects and verbs which do not trigger inherent reciprocal interpretation. This is unlike the constructions with singular subjects discussed in \sectref{sec:ngwasi:2.2} which have reflexive interpretation only. This is illustrated by the example in \REF{ex:ngwasi:12}, where the reflexive prefix -\textit{i}{}- has an ambiguous reflexive\nobreakdash-reciprocal interpretation.


\ea\label{ex:ngwasi:12}
\textit{Kiliani na Naftali vak}\textbf{\textit{i}}\textit{bumyé}\\
\gll Kiliani   na   Naftali   va-ka-\textbf{i}{}-bumíl-íle\\
Kiliani   \textsc{com}   Naftali   \textsc{sm2-pst-refl-rec}-hit-\textsc{pfv}\\
\glt ‘Kiliani and Naftali hit each other.’ or ‘Kiliani and Naftali hit themselves.’
\z



Hehe speakers use emphatic pronouns to remove this ambiguity and rule out a reciprocal interpretation in favour of the reflexive interpretation in constructions with plural subjects, as \REF{ex:ngwasi:13} exemplifies. If the intended meaning is the reciprocal interpretation, the speakers can employ discontinuous reciprocal constructions to rule out the reflexive interpretation, as can be seen in \REF{ex:ngwasi:14}. It should be noted that in discontinuous reciprocal constructions, one of the two participants follows a verb and is introduced by a comitative preposition (see \citealt{Dimitriadis2004,Dimitriadis2008, Haspelmath2007}), i.e., the comitative \textit{na} in the case of Hehe in \REF{ex:ngwasi:14}. The example \REF{ex:ngwasi:13} shows that the emphatic reflexive pronoun functions as the disambiguator for the reflexive interpretation just as in other languages cross-linguistically, such as French \textit{eux\nobreakdash-mêmes} and German \textit{sich} (see \citet{Cable2014} for examples), while the discontinuous reciprocal construction in \REF{ex:ngwasi:14} functions as a disambiguation strategy for the reciprocal interpretation (see \citealt{Dimitriadis2004, SeidlDimitriadis2003} for this disambiguation strategy in Swahili (G42) and German).


\ea\label{ex:ngwasi:13}
\textit{Kiliani na Naftali vak}\textbf{\textit{i}}\textit{bumyé} \textbf{\textit{vavene}}\\
\gll Kiliani na Naftali   va-ka-\textbf{i}{}-bumíl-íle     \textbf{vavene}\\
Kiliani \textsc{com} Naftali \textsc{sm2-pst-refl}-hit-\textsc{pfv}  \textsc{emph.refl}\\
\glt ‘Kiliani and Naftali hit themselves.’


\ex\label{ex:ngwasi:14}
\textit{Kiliani ak}\textbf{\textit{i}}\textit{bumyé} \textbf{\textit{na} \textbf{\textit{Naftali}}}\\
\gll Kiliani a-ka-\textbf{i}{}-bumíl-íle     \textbf{na}   \textbf{Naftali}\\
Kiliani \textsc{sm1-pst-rec}-hit-\textsc{pfv}  \textsc{com}  {Naftali}\\
\glt ‘Kiliani and Naftali hit each other.’
\z


It is important to note that, unlike some Bantu languages where both singular and plural subject markers are acceptable in the case of discontinuous reciprocal constructions (cf. Mwera (P22) and Cilubà (L31a), see \cites[763--764]{BostoenEtAl2015}[183]{SchadebergBostoen2019}), in Hehe, the subject marker continues to show singular agreement with the remaining lexical NP in the subject position.

The reflexive prefix -\textit{i}{}- is also used as a productive means of encoding inherent reciprocal events in constructions with plural subjects and verbs that are semantically frequently or necessarily reciprocal. The example \REF{ex:ngwasi:15} illustrates the reflexive prefix \textit{\nobreakdash-i}{}- encoding inherent reciprocal events with plural subject NPs.

\ea\label{ex:ngwasi:15}
\textit{Juma na Ali vak}\textbf{\textit{i}}\textit{húunje}\\
\gll Juma   na   Ali   va-ka-\textbf{i}{}-huungíl-íle\\
Juma \textsc{com}  Ali   \textsc{sm2-pst-rec}-greet-\textsc{pfv}\\
\glt ‘Juma and Ali greeted each other.’
\z

There are some verbs which trigger inherently reciprocal interpretation that have retained the reflex of the Proto\nobreakdash-Bantu reciprocal suffix *-\textit{an}{}-. Such verbs are listed in \REF{ex:ngwasi:16} below. It should be noted that the verbs \textit{{}-leka} ‘leave/abandon’ and -\textit{hwaana} ‘resemble’ also take the reflexive prefix -\textit{i}{}- in the synchronic state of the language. In addition, the verb \nobreakdash-\textit{hwaana} does not occur without the reciprocal suffix *\nobreakdash-\textit{an-}. As such, the reciprocal suffix -\textit{an}{}- is fossilized (has become part of the verb stem) in this verb. The same fossilized reciprocal suffix is observed on the verb -\textit{taang’ána} ‘meet’, which also requires the reflexive prefix to be present in order to express reciprocity. Thus it is the reflexive prefix that encodes reciprocal meaning in such cases.

\ea\label{ex:ngwasi:16}
Verbs with the reflex of the Proto-Bantu reciprocal suffix *-\textit{an}{}-\\
\gllllll \textit{kúgav}\textbf{\textit{án}}\textit{a}  \textit{kú-gav-án-a}    {‘to share’}  < \textit{kúgava} {‘to distribute’}\\
\textit{kúlek}\textbf{\textit{án}}\textit{a}  \textit{kú-lek-án-a}    {‘to divorce’}   < \textit{kúleka} {‘to leave/}\\
{} {} {} {} {} {abandon’}\\
\textit{kúhwa}\textbf{\textit{án}}\textit{a}  \textit{kú-hwaán-a}    {‘to resemble’}\\
\textit{kúloong}\textbf{\textit{án}}\textit{a}  \textit{kú-loong-án-a} {‘to chat’}  < \textit{kúloonga} {‘to talk’}\\
\textit{kw}\textbf{\textit{í}}\textit{taang’}\textbf{\textit{an}}\textit{a}  \textit{kú-i-taang’án-a} {‘to meet’}\\
\z


Besides encoding inherent reciprocal events, the reflexive prefix -\textit{i}{}- also encodes other events such as chaining reciprocal and associativity. By chaining reciprocal events, following \citet{Kemmer1993}, we refer to events that involve an ordered sequence or series of participants who are in a certain relation, while associative events refer to events or actions that are carried out jointly. The following examples in \REF{ex:ngwasi:17} and \REF{ex:ngwasi:18} illustrate constructions encoding chaining and associative events, respectively.  

\ea\label{ex:ngwasi:17}
\textit{Avanyashule vak}\textbf{\textit{i}}\textit{fwaatíte}\\
\gll  a-va-nyashule    va-ka-\textbf{i}{}-fwaat-íte\\
  \textsc{aug}-2-student    \textsc{sm2-pst2-rec}-follow-\textsc{pfv}\\
\glt  ‘The students followed each other.’


\ex\label{ex:ngwasi:18}
\textit{Avanu vakitaanzíle}\\
\gll a-va-nu    va-ka-\textbf{i}{}-taang-íle\\
\textsc{aug}-2-person    \textsc{sm2-pst-rec}-help/do-\textsc{pfv}\\
\glt ‘The people did together.’ (lit. ‘The people worked together’)
\z


In general, we can conclude that the reflexive prefix -\textit{i}{}- is a productive means of encoding reflexive, reciprocal, chaining, and associative events in Hehe. Within the domain of reciprocal events, it is used to encode both prototypical and inherent reciprocal events, as well as other events related to the reciprocal prototype, such as chaining and associative events. Having described the way, the reflexive prefix -\textit{i}{}- encodes these events, we turn to the discussion of the historical development of the reflexive-reciprocal polysemy in Hehe in \sectref{sec:ngwasi:3}.

\section{The historical development of the reflexive-reciprocal polysemy in Hehe: A grammaticalization perspective}\label{sec:ngwasi:3}

This section discusses the historical development of the reflexive-reciprocal polysemy in Hehe by applying \citegen{Heine1993} Overlap Model on the grammaticalization from reflexive to reciprocal markers. This Model has been applied in other languages, particularly German by \citet{HeineMiyashita2008, HeineNarrog2009} who examine the grammaticalization from the reflexive pronoun \textit{sich} to reciprocal marker. The Model presupposes three synchronic stages, reflecting a historical development leading from reflexive to reciprocal marker. In addition to the three stages of the Model, following \citet{Heine2002} (see also \citealt{HeineKuteva2007}), we add the fourth stage called \textit{conventionalization}. Before discussing the stages of grammaticalization from reflexive to reciprocal marker, we briefly define grammaticalization and the mechanisms of change by providing examples from other languages on the grammaticalization from reflexive to reciprocal marker.

\subsection{Grammaticalization and its parameters}\label{sec:ngwasi:3.1}

The term ``grammaticalization'' has been used in linguistics in two ways. First, it is used to refer to a process of language change. Second, it is used to refer to the theoretical framework that is used to account for the processes of language change (see \citealt{CampbellJanda2001, Heine2003, HeineNarrog2009}). According to \citet{Croft2006}, to understand what grammaticalization means, we need to understand first the processes that create the grammar of a particular language.  In general, as \citet{HeineNarrog2009} define it, grammaticalization is a process in which lexical items become grammatical items, or grammatical items become more grammatical. From this definition, there are two types of grammaticalization. First, there is primary grammaticalization which involves a change from lexical to grammatical items. Second, secondary grammaticalization which involves a change from already grammatical(ized) items to more grammatical ones. This chapter is based on secondary grammaticalization because there is no lexical source reconstructable for Proto-Bantu as a source of the reflexive marker in Bantu. 

As for grammaticalization as a theoretical framework, it is meant to explain what causes grammaticalization, and how grammatical or more grammatical categories are developed and structured in languages (\citealt{Heine2003, HeineKuteva2007}). Thus, it is an explanatory tool for the grammaticalization phenomenon. 

As a process of change, grammaticalization involves four parameters, namely: ``extension'', ``desemanticization'', ``decategorialization'', and ``erosion''. Although these parameters are mainly associated with primary grammaticalization, they are worth exploring because they have been used to explain the grammaticalization from reflexive to reciprocal where lexical sources are attested. They can also equally be used with secondary grammaticalization in many respects. Each of these parameters is explained in the following paragraphs as applied in the grammaticalization from reflexive to reciprocal marker in other non-Bantu languages, in particular, German.

The first parameter, \textit{extension}, involves the rise of new grammatical meanings for a particular form, especially in a new context (semantic component). This is to say, the linguistic item with its meaning receives a new meaning in another context (context-induced reinterpretation) (see \citealt{Heine2002, HeineDunham2010, HeineKuteva2007}). It also involves the extension of the use of a linguistic item in its usual or primary context to a new set of context(s) (\citealt{HeineDunham2010}), such that it is no longer limited to a particular defined context (text\nobreakdash-pragmatic component). As \citet{HeineKuteva2007} argue, all these come out due to some sociolinguistic component whereby speakers, usually a group, start employing a new usage or meaning of the existing linguistic item, and later on adopted by the entire speech community. The German reflexive pronoun \textit{sich}, for example, was extended to encode reciprocal events in constructions with plural subjects or antecedents (see \citealt{HeineMiyashita2008, HeineNarrog2009}). As can be seen in \REF{ex:ngwasi:19}, the reflexive pronoun \textit{sich} has a reflexive meaning only, but in \REF{ex:ngwasi:20}, it is reinterpreted as encoding also reciprocal meaning since the context -- the plurality of the participants -- leads to its reinterpretation as a reciprocal marker while maintaining its source meaning, i.e., reflexive. Hence, the construction becomes ambiguous between the source meaning and the new meaning.

\ea\label{ex:ngwasi:19}
\gll \textit{Er}   \textit{wusch} \textbf{\textit{sich}}\\
He   wash.\textsc{pst}  \textsc{refl}\\
\glt ‘He washed (himself).’


\ex\label{ex:ngwasi:20}
\gll \textit{Sie}   \textit{wuschen} \textbf{\textit{sich}}\\
They  wash.\textsc{pst.pl}  \textsc{refl-rec}\\
\glt ‘They washed themselves.’ \\ 
‘They washed each other.’        (\citealt[410]{HeineNarrog2009}) 
\z


The second parameter, \textit{desemanticization} or \textit{semantic bleaching}, refers to the process whereby a linguistic item loses its old or source meaning or use due to the reinterpretation in the new context of use  (\citealt{HeineDunham2010, HeineNarrog2009}). This parameter follows from extension because the extended linguistic item may lose part of its primary meaning in specific contexts. With respect to the reflexive pronoun \textit{sich} in German, it loses the reflexive meaning when used with verbs that trigger inherent reciprocal interpretation, as exemplified in \REF{ex:ngwasi:21}. 

\ea\label{ex:ngwasi:21}
\gll \textit{Sie}  \textit{küssten} \textbf{\textit{sich}}\\
They  kiss.\textsc{pst.pl}  \textsc{refl}\\
\glt ‘They kissed (each other).’         (\citealt[410]{HeineNarrog2009})
\z


The third parameter, \textit{decategorialization}, involves the loss of the morphosyntactic characteristics of the linguistic item after being desemanticized. This means that the morphosyntactic properties which the linguistic item had before its extension and desemanticization are no longer available in the new usage context. This may include, among others: (i) Loss of ability to be inflected;  (ii) Loss of ability to take on the derivational morphology; (iii) Loss of ability to take modifiers; (iv) Loss of independence as an autonomous linguistic item, leading to an increased dependence on some other linguistic item; (v) Loss of syntactic freedom of a linguistic item, such as, the ability to be moved in a sentence; (vi) Loss of ability to be referred to anaphorically; and (vii) Loss of membership to a grammatical paradigm (see \citealt{Heine2003, HeineDunham2010}). In addition to these, \citet{HeineMiyashita2008} attribute decategorialization to a limited set of contexts, both syntactic and pragmatic, where the grammaticalized item can occur. According to \citet[196--197]{HeineMiyashita2008}, the most widespread decategorialization involving reflexive markers that become reciprocal markers is the constraint on the category of number. In other words, the reflexive-reciprocal marker, when used to encode reciprocal meaning, becomes restricted to “a smaller set of syntactic and pragmatic contexts” compared to when it is used as a reflexive marker. As such, the reciprocal interpretation is restricted to constructions with plural subjects or antecedents only. This is to say, for example, the pronoun \textit{sich} in German can only be interpreted as encoding reciprocal meaning with plural subjects. In contrast, with singular subjects, it continues to encode reflexive meaning.

The fourth parameter, \textit{erosion} or \textit{phonetic reduction}, refers to the loss in phonetic substance of the linguistic item undergoing a change in grammaticalization. This may involve the loss of an entire syllable, phonetic simplification, loss of phonetic autonomy as well as the adaptation to adjacent phonetic units, or loss of segmental properties such as stress, tone, or intonation (see \citealt{HeineDunham2010, HeineMiyashita2008, HeineNarrog2009}). For the German reflexive pronoun \textit{sich}, \citet{HeineNarrog2009} argue that it loses the stress that it bears when encoding reflexive events when used to encode reciprocal events.

It is argued that grammaticalization is a continuous process or a ``chain-like'' development in the sense that it usually follows the parameters from extension to phonetic reduction (see \citealt{Heine2000, HeineKuteva2007, HeineNarrog2009}). However, it should be noted that the grammaticalization process can stop at any point of development, and it does not necessarily replace older linguistic forms expressing the same grammatical meaning (see \citealt{Heine2000, HeineKuteva2007, Hopper1991}). With this note on the mechanisms or parameters of grammaticalization, we turn to the grammaticalization of the reflexive prefix -\textit{i}{}- in Hehe in \sectref{sec:ngwasi:3.2}.

\subsection{The stages of grammaticalization of the reflexive prefix -\textit{i}{}- in Hehe}\label{sec:ngwasi:3.2}

The four stages of grammaticalization involving the reflexive markers mentioned at the beginning of \sectref{sec:ngwasi:3} are explained in this subsection with reference to the data presented in \sectref{sec:ngwasi:2}. The data presented in \sectref{sec:ngwasi:2} where the reflexive prefix -\textit{i}{}- has other functions are analyzed from the grammaticalization perspective.

The first stage (stage I) in this Model is called the ``Initial stage''.  In this stage, as \citet{Heine2002} argues, the linguistic item has its original meaning, and it is not restricted in terms of contexts where it can occur. This stage in Hehe, in the synchronic state of the language, is represented by the constructions with reflexive interpretation only (those with singular subjects), but it can be hypothesized that before its grammaticalization to reciprocal marker, it was not restricted to constructions with singular subjects. This stage is illustrated by the constructions with singular subjects and the verbs that do not trigger inherent reciprocal interpretation, as in the example \REF{ex:ngwasi:10} above.

The second stage (stage II) is called ``bridging context''. The linguistic item gets reinterpreted with reference to the source meaning and the new target meaning. Thus, the linguistic item becomes ambiguous. For Hehe, this stage is represented by constructions with an ambiguous reflexive\nobreakdash-reciprocal interpretation, i.e., the constructions with plural subjects and the constructions with infinitive prefixes with verbs of prototypical two\nobreakdash-participant event verbs, as in the examples \REF{ex:ngwasi:8} and \REF{ex:ngwasi:12} above. In fact, the constructions in this stage differ from the constructions in stage I in that the subjects in these constructions are plural, and the constructions with an infinitive prefix. This stage is an intermediate stage for grammaticalization from reflexive to reciprocal marker. As examples \REF{ex:ngwasi:8} and \REF{ex:ngwasi:12} show, the constructions are simultaneously interpreted with reference to the source or original meaning (reflexivity) and the target or new meaning (reciprocity).

The third stage (stage III) is called ``switch context''. In this stage, the linguistic item is interpreted with the new or target meaning only \citep[85]{Heine2002}. In other words, the source meaning is no longer accessible. This stage in Hehe is represented by constructions with plural subjects (and infinitive constructions), just like the ones in stage II, but the difference is based on the type of verbs used at this stage. Unlike the verbs used at stage II, the verbs used at stage III constructions trigger an inherently reciprocal interpretation with the reflexive prefix -\textit{i}{}-. In switch contexts, the target function or meaning, encoding reciprocal in this case, is the only available interpretation. In other words, there is no source function at this stage (the reflexive function of the prefix -\textit{i}{}- is excluded at this stage). So, the reflexive interpretation of the reflexive prefix -\textit{i}{}- is infelicitous in stage III. It is inappropriate for the examples \REF{ex:ngwasi:9} and \REF{ex:ngwasi:15} above to mean ‘to greet oneself’, or ‘Juma and Ali greeted themselves’. The only appropriate interpretation of this construction is reciprocal, i.e., ‘to greet each other’ or ‘Juma and Ali greeted each other’.

The fourth stage (stage IV) is called the ``conventionalization stage''.  In this stage, as \citet[86]{Heine2002} argues, the linguistic item may be used in other new contexts because it is no longer restricted to its source function. In Hehe, the reflexive prefix -\textit{i}{}- is also recruited to encode chaining and collective or associative events, apart from encoding prototypical and inherent reciprocal events, as we have already seen in examples \REF{ex:ngwasi:17} and \REF{ex:ngwasi:18} above. This is because the language speakers have conventionalized it to be their new means of encoding reciprocal events. Thus, it is also extended to encode other less core reciprocal functions of the reflex of the Proto-Bantu reciprocal suffix *-\textit{an}{}-, in particular the constructions with verbs that trigger chaining and associative reciprocal interpretation.

The four stages of grammaticalization of the reflexive prefix -\textit{i}{}- in Hehe are summarized in \tabref{tab:ngwasi:2} below, following \citet{Heine2002}.

\begin{table}
\small
\begin{tabularx}{\textwidth}{lQ>{\raggedright\arraybackslash}p{.25\textwidth}}

\lsptoprule

{Stage}  & {Context}  & {Resulting interpretation} \\
\midrule
I. Initial & Not restricted & Reflexive\\
\tablevspace
II. Bridging context & Plural subjects/Infinitive prefix, prototypical two\nobreakdash-participant event verbs & Reflexive-reciprocal\\
\tablevspace
III. Switch context & Plural subjects/Infinitive prefix, verbs resulting to inherent reciprocal interpretation & Reciprocal \\
\tablevspace
IV. Conventionalization & Plural subjects/Infinitive prefix, verbs resulting to chaining reciprocal interpretation, and associative interpretation & Chaining reciprocal, associative interpretation\\
\lspbottomrule
\end{tabularx}
\caption{The stages of grammaticalization from reflexive to reciprocal of the reflexive prefix \nobreakdash-\textit{i}{}-}
\label{tab:ngwasi:2}
\end{table}

\section{The loss of the reflex of the Proto-Bantu reciprocal suffix *-\textit{an}{}- and the emergence of the reflexive-reciprocal polysemy}\label{sec:ngwasi:4}

A number of facts indicate that in an earlier stage, Hehe conformed to the common Bantu situation, in that it had the reflexive prefix for encoding reflexive events and the reciprocal suffix for encoding reciprocal events. First, the fact that there are some verbs with the reflex of the Proto-Bantu reciprocal suffix *-\textit{an-}, as shown in \REF{ex:ngwasi:16} above, is a piece of evidence that the reciprocal suffix was a productive reciprocal marker in Hehe. Second, \citet{Msamba2013} argues that while most speakers of ``Standard'' Hehe prefer to use the reflexive prefix -\textit{i-} instead of the reflex of the Proto-Bantu reciprocal suffix *\textit{\nobreakdash-an-} (the suffix -\textit{an}{}-) to express reciprocity, a few speakers, especially elders, still use the reciprocal suffix with some verbs, as shown in \tabref{tab:ngwasi:3} below. This indicates that even in this dialect, the reflexive prefix is becoming conventionalized as a productive means of encoding reciprocal events, replacing the reflex of the Proto-Bantu reciprocal suffix *-\textit{an}{}-.



\begin{table}

\begin{tabularx}{\textwidth}{lllQlQ}

\lsptoprule

Verb stem & Gloss & Reflexive & Gloss & Reciprocal & Gloss \\
\midrule
{}-\textit{tova} & ‘beat’ & {}-\textbf{\textit{i}}\textit{tova} & ‘beat oneself’ & {}-\textit{tov}\textbf{\textit{an}}\textit{a} & ‘beat each other’\\
{}-\textit{heka} & ‘laugh’ & {\itshape {}-\textbf{i}heka} & ‘laugh at oneself’ & {}-\textit{hek}\textbf{\textit{an}}\textit{a} & ‘laugh at each other’\\
{}-\textit{kwega} & ‘pull’ & {}-\textbf{\textit{i}}\textit{kwega} & ‘pull oneself’ & {}-\textit{kweg}\textbf{\textit{an}}\textit{a} & ‘pull each other’\\
{}-\textit{homba} & ‘pay’ & {\itshape {}-\textbf{i}homba} & ‘pay oneself’ & {}-\textit{homb}\textbf{\textit{an}}\textit{a} & ‘pay each other’\\
\lspbottomrule
\end{tabularx}
\caption{The coexistence of the reflexive prefix and reciprocal suffix in encoding reciprocal events in ``Standard'' Hehe \citep[59]{Msamba2013}}
\label{tab:ngwasi:3}
\end{table}

It is important to note that the grammaticalization from reflexive to reciprocal described in this chapter and summarized in \tabref{tab:ngwasi:2} for Hehe should be regarded as a means of creating a new grammatical item for encoding reciprocal events, taking over the role of the reflex of the Proto\nobreakdash-Bantu reciprocal suffix *-\textit{an}{}-. It has been hypothesized by \citet{Schladt1998} that the Proto\nobreakdash-Bantu reciprocal suffix *-\textit{an}{}- developed from the comitative marker \textit{na}.  He argues that the development from the comitative marker to reciprocal suffix resulted from a serial construction following the grammaticalization chain: V-a \textit{na} > V-a-\textit{na} > V-\textit{an}{}-a (note: V-a stands for the verb root + the default final vowel). This hypothesis has been adopted in other work on Bantu  languages, i.e.,  \citet{SchadebergBostoen2019} and \citet{BostoenEtAl2015}.\footnote{Another suffix (verb extension) that has been hypothesized to have developed from a lexical source is the extensive suffix \textit{*-al-.} According to  \citet{Schadeberg2003}, it is from the lexical item \textit{*-jal-} ‘to spread’. For other verb extensions, there is no suggested lexical sources (see  \citealt{Schadeberg2003} and \citealt[174]{SchadebergBostoen2019}).}

The fact that there is evidence for the existence of the reflex of the Proto-Bantu reciprocal suffix *\nobreakdash-\textit{an}{}- in Hehe, means that it can be concluded that this suffix went through the grammaticalization chain hypothesized by \citet{Schladt1998} before it fell out of favour by Hehe speakers.

According to \citet[22--23]{Hopper1991}, when a linguistic item is taking over the functional role of another linguistic item, it is expected that the new item and the old item may coexist for a certain period. This means that the new linguistic item does not immediately replace an already existing item. In the case of the grammaticalization of the reflexive prefix -\textit{i}{}- in Hehe, the verbs where the reciprocal suffix -\textit{an}{}- is still existent synchronically (cf. example \REF{ex:ngwasi:16}) offer evidence of the coexistence stage in the history of the language. In addition to the examples in \REF{ex:ngwasi:16}, the data in \tabref{tab:ngwasi:3} from \citet[59]{Msamba2013} showing the coexistence of the reflexive prefix -\textit{i}{}- and the reciprocal suffix -\textit{an}{}- especially in the speech of elders in ``Standard'' Hehe illustrate this phenomenon. Similar coexistence has been reported by \citet[249]{Morrison2011} for Bena (G63) (a language which is spoken in close geographic proximity to Hehe), as can be exemplified in \REF{ex:ngwasi:22}.

\ea\label{ex:ngwasi:22}
Bena (G63) \citep[249]{Morrison2011}
    \ea\label{ex:ngwasi:22a} \textit{Tuhwíwona}\\
    \gll tu-hu-\textbf{i}{}-won-a\\
    \textsc{sm2-e-refl-rec}-see-\textsc{fv}\\
    \glt ‘We see each other/We see ourselves.’

    \ex\label{ex:ngwasi:22b} \textit{Twíwonana}\\
    \gll tu-i-won-\textbf{an}{}-a\\
    \textsc{sm2-prs}-see-\textsc{rec-fv}\\
    \glt ‘We see each other.’ 
    \z
\z

In general, the coexistence of the reflexive prefix -\textit{i}{}- and the reciprocal suffix -\textit{an}{}- in encoding reciprocal events provides evidence that the reciprocal suffix -\textit{an}{}- had been productively used as a reciprocal marker, and the reflexive prefix -\textit{i}{}- is now taking over the role of the reciprocal suffix in Bena. The reflexive prefix might ultimately be the only productive means of encoding reciprocal events as has happened in Hehe.

\section{Conclusion}\label{sec:ngwasi:5}

Based on the Hehe data presented and analyzed in this chapter, it is evident that the reflexive prefix -\textit{i-} has developed from being a dedicated reflexive marker into a polysemous marker encoding both reflexive and reciprocal events. We have argued that the various present\nobreakdash-day uses of the reflexive prefix can be interpreted as distinct stages illustrating the diachronic grammaticalization process leading from a prototypical reflexive marker to a reciprocal marker. The reflex of the Proto-Bantu reciprocal suffix *\textit{{}-an-}, which occurs throughout Bantu languages as a productive reciprocal marker, is still found with some verbs encoding inherent reciprocal events. However, the grammaticalization of the reflexive prefix -\textit{i-} is becoming dominant to such an extent that it is also used and preferred with some of these archaic lexicalized reciprocal verbs (e.g., \textit{kw}\textbf{\textit{í}}\textit{taang’}\textbf{\textit{án}}\textit{a} ‘to meet’).  Finally, we have shown that the reflexive prefix -\textit{i}{}- after grammaticalizing and becoming a new productive means of encoding reciprocal events has been extended to encode chaining and associative events, the events which are closer to the reciprocal prototype. These two events are also encoded by the reflexes of the Proto-Bantu reciprocal suffix *\textit{{}-an-} in the languages where the reciprocal suffix is still productive.

\section*{Acknowledgements}
We thank our language consultants, who dedicated their time to us for data elicitation. We also appreciate the comments and insights from the reviewers that helped us reach this final version. This work could not have been possible without their thoughts and insights. Lastly, we appreciate the work by the editors of this volume from the beginning of this project until its completion.  


\section*{Abbreviations and symbols}
\begin{multicols}{2}
\begin{tabbing}
\textsc{emph.refl}\hspace{1ex}  \=  Emphatic pronoun\kill
\textsc{aug} \>    Augment\\
\textsc{cl} \>    Noun class\\
\textsc{clit} \>    Clitic\\
\textsc{com} \>    Comitative\\
\textsc{e} \>    Epenthetic morpheme\\
\textsc{emph.refl} \>    Emphatic pronoun\\
\textsc{ext} \>    Verb extension\\
\textsc{fv} \>    Final vowel\\
\textsc{inf} \>    Infinitive\\
\textsc{neg} \>    Negative marker\\
\textsc{om} \>    Object marker\\
\textsc{pfv} \>    Perfective\\
\textsc{pl} \>    Plural\\
\textsc{rec} \>    Reciprocal\\
\textsc{refl} \>    Reflexive\\
\textsc{rel} \>    Relative\\
\textsc{sg} \>    Singular\\
\textsc{sm} \>    Subject Marker\\
TAM \>   Tense, Aspect, Mood\\
> \>   to
\end{tabbing}
\end{multicols}


\sloppy\printbibliography[heading=subbibliography,notkeyword=this]
\end{document} 
