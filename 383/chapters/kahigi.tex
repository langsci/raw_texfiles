\documentclass[output=paper		  ]{langscibook}
\ChapterDOI{10.5281/zenodo.10663777}

\author{Kulikoyela Kahigi\orcid{}\affiliation{Saint Augustine University of Tanzania, Mwanza}}

\title[Verb extensions and morphosyntactic variation in  Sumbwa]{Verb extensions and morphosyntactic variation in Bantu: The case of Sumbwa}

\abstract{This study has two aims: (a) to describe verbal extensions in Sumbwa and their valence, and (b) to contribute to Bantu comparative data that may be used in research on Bantu morphosyntactic parametric variation as proposed in \citet{GuéroisEtAl2017}. The description covers all extensions that could be identified in the data at hand (e.g. \citealt{Capus1898, Kahigi2008a, Kahigi2008b}), focusing on the forms and their various meanings, their valence possibilities, their productivity and co-occurrence constraints. The study reveals that most of the Proto-Bantu verb extensions reconstructed by \citet{Meeussen1967} and \citet{Guthrie1971} are still active in the language. Some extensions are found to be highly productive (applicative, passive, causative (also instrumental), associative, stative, and frequentative), others moderately or semi-productive (persistive, reversive, impositive and denominative) and quite a number may be regarded as being unproductive (associative/reciprocal \textit{{}-an-}, reiterative, static, contactive and other minor extensions). The second aim of the study is to consider how Sumbwa compares to other Bantu languages by drawing on the parameters on verbal derivation identified in \citet{GuéroisEtAl2017}. Some of the findings agree with what is found in the majority of Eastern Bantu languages, e.g. the verb derivational strategies follow closely those mapped out by the PB reconstructions, except for a few innovations among the minor extensions (e.g. \textit{{}-agan-}, \textit{{}-agil-}). However, Sumbwa does not have \textit{ba}-passives found in Bemba, nor does it have the CARP order as postulated by \citet{Hyman2002} for Bantu. Some of the notable characteristics of Sumbwa verb extensions include the fact that (a) the causative and instrumental share extensions, (b) the associative markers include the post-verbal \textit{{}-an-} and the pre-verbal \textit{{}-i-}, which is homophonous with the reflexive; (c) the applicative conveys \textit{benefactive, directive, location}, and \textit{reason} meanings; (d) there is no systematic fixed order of extensions, except that in all co-occurrences, the passive comes last.
\keywords{Sumbwa verb extensions, Valence, Parameters of morphosyntactic variation}
}

\IfFileExists{../localcommands.tex}{
  \addbibresource{../localbibliography.bib}
  \usepackage{langsci-optional}
\usepackage{langsci-gb4e}
\usepackage{langsci-lgr}

\usepackage{listings}
\lstset{basicstyle=\ttfamily,tabsize=2,breaklines=true}

%added by author
% \usepackage{tipa}
\usepackage{multirow}
\graphicspath{{figures/}}
\usepackage{langsci-branding}

  
\newcommand{\sent}{\enumsentence}
\newcommand{\sents}{\eenumsentence}
\let\citeasnoun\citet

\renewcommand{\lsCoverTitleFont}[1]{\sffamily\addfontfeatures{Scale=MatchUppercase}\fontsize{44pt}{16mm}\selectfont #1}
   
  %% hyphenation points for line breaks
%% Normally, automatic hyphenation in LaTeX is very good
%% If a word is mis-hyphenated, add it to this file
%%
%% add information to TeX file before \begin{document} with:
%% %% hyphenation points for line breaks
%% Normally, automatic hyphenation in LaTeX is very good
%% If a word is mis-hyphenated, add it to this file
%%
%% add information to TeX file before \begin{document} with:
%% %% hyphenation points for line breaks
%% Normally, automatic hyphenation in LaTeX is very good
%% If a word is mis-hyphenated, add it to this file
%%
%% add information to TeX file before \begin{document} with:
%% \include{localhyphenation}
\hyphenation{
affri-ca-te
affri-ca-tes
an-no-tated
com-ple-ments
com-po-si-tio-na-li-ty
non-com-po-si-tio-na-li-ty
Gon-zá-lez
out-side
Ri-chárd
se-man-tics
STREU-SLE
Tie-de-mann
}
\hyphenation{
affri-ca-te
affri-ca-tes
an-no-tated
com-ple-ments
com-po-si-tio-na-li-ty
non-com-po-si-tio-na-li-ty
Gon-zá-lez
out-side
Ri-chárd
se-man-tics
STREU-SLE
Tie-de-mann
}
\hyphenation{
affri-ca-te
affri-ca-tes
an-no-tated
com-ple-ments
com-po-si-tio-na-li-ty
non-com-po-si-tio-na-li-ty
Gon-zá-lez
out-side
Ri-chárd
se-man-tics
STREU-SLE
Tie-de-mann
} 
  \togglepaper[1]%%chapternumber
}{}

\begin{document}
\maketitle 
%\shorttitlerunninghead{}%%use this for an abridged title in the page headers

\section{Introduction}\label{sec:kahigi:1}

{For the purposes of this study, we shall adopt \citegen{Guthrie1962} use of the term ``verb extension'', which is interchangeable with ``verb derivation'' (cf. \citealt{KatambaStonham2006}). As such, a verb extension is a suffix added to the verb root or base that changes the sense of the root/base.} 

Verb extensions are one of the most important features of Bantu languages and have been investigated and described since the inception of Bantu studies. A survey of descriptions of Bantu verb extensions across the years (e.g. \citealt{Madan1903, Ashton1947, Johnson1939, Guthrie1962, Eastman1967, Scotton1967, Bokamba1975, Khamis1972, Khamis1985, Rugemalira1993runyambo,Rugemalira2005, Schadeberg2003}) shows that verb extensions constitute one of the most important topics in Bantu linguistics. This is borne out by research on the topic, which has continued to produce valuable descriptive and theoretical contributions (cf. \citealt{Baker1985, Alsina1999, Marten2003, Hyman2002, KatambaStonham2006, Waweru2005, Khumalo2007, Chabata2007, KulaMarten2010, DomEtAl2018}, etc.).

{The focus of this study is on Sumbwa, a West Tanzania Bantu language, classified as F23}\footnote{{This is according to the widely used \citet{Guthrie1948} classification for identifying individual Bantu languages. In this classification, the Bantu area is divided into zones and the zones are divided into groups. Sumbwa belongs to Zone F, Group F20 (Sukuma-Nyamwezi Group).}} {by \textcites[]{Guthrie1948}[11]{Guthrie1970}. Known as Sisuumbwa by its native speakers, this largely undescribed language is mainly spoken in Geita, Shinyanga, Tabora and Kagera regions.} {Other Sumbwa speakers, known as the Bayeke, are found in the DRC, in the current Yeke chiefdom, whose capital is Bunkeya, in Lualaba province (cf. \citealt{MunongoGrevisse1967}). The dialect addressed in this study is the Ushilombo/Lunzewe dialect spoken in Bukombe district, Geita region.} 

{This study has two aims. The first aim is to describe verbal extensions in Sumbwa. I follow \citegen{Guthrie1962} approach, which focuses on identifying the morphological shapes of the extensions, their meanings, and their syntactic effects (i.e. valence). In addition, a brief statement of the productivity of the extensions is provided. The concept of valence used here is the traditional one; it refers to the potential of the verb to take an argument (i.e. subject, direct object or indirect object) (cf. \citealt{Humphreys1999, HaspelmathMüller-Bardey2004}). In most Bantu languages, some verb extensions trigger a change in valence by either adding or deducting an argument, while some extensions do not affect the basic valence of the verb. Thus, an extension may be referred to as valence-increasing, valence-decreasing or valence-maintaining (cf. \citealt{Chabata2007, Payne1997, Hyman2007}). The concept of productivity is also used here in its traditional sense; a verb extension is viewed as being productive if it is used to coin new words (cf. \citealt[121]{Plag2006}). Regarding Bantu extensions, productivity is considered to be a scalar concept, that is on a scale “from totally unproductive expansions occurring in just a few verbs to fully productive suffixes” \citep[73]{Schadeberg2003}. Thus, in Sumbwa we have what we may call highly or fully productive extensions (e.g. the passive extension); moderately or semi-productive extensions like the frequentative} {\textit{{}-agul-}} {which is} {mainly restricted to disyllabic roots; and non-productive extensions such as the static} {\textit{{}-am-}} {that can hardly be used in coining new words in the language. Besides valence and productivity, this study also briefly examines the co-occurrence restrictions of the extensions.} 

{The more general aim of the study is to contribute to Bantu comparative data that may be used in research on Bantu parametric morphosyntactic variation (cf. \citealt{MartenEtAl2007, GuéroisEtAl2017}). \citet{GuéroisEtAl2017} is a master list of 141 parameters in 12 morphosyntactic areas. These include nouns and pronouns, noun modifiers, nominal derivation, lexicon, verbal derivation, verbal inflection, relative clauses, clefts and questions, verbless clauses, simple clauses, constituent order, complex sentences, and expression of focus. Each parameter begins with a question, followed by possible answers, e.g.} 

\begin{quote}
{\textit{Parameter} \textbf{36}}{. Canonical Passive: Is the canonical passive productively expressed through a verbal extension?} 

{Possible answers:}

{null = unknown,} 

{no = another strategy is used to express passivisation, e.g. an impersonal construction…,} 

{yes = specify whether there is one or several possible forms.} 
\end{quote}

{The possible answers vary depending on the nature of the parameter.}

{One of the goals of the project is to collect Bantu morphosyntactic data with a view to identifying variation at the micro level. Of the 12 areas in the master list, the focus of the present study will be on area 5: verbal derivation (parameters number 36 to 48).}

{The second aim of this study,  consequently, addresses these parameters, responding to the possible answers given for each parameter. Due to space limitations, the discussion is mostly restricted to Sumbwa data, although, occasionally, other Bantu languages will be referred to for comparison. The focus here will be to observe} {\textit{whether}} {and} {\textit{how}} {the proposed parameters involving verb extensions occur in Sumbwa. The data used in this study is mainly from \citet{Capus1898} and \citet{Kahigi2008a, Kahigi2008b}; the first source is an earlier grammar while the other two include data collected directly from the field between 1976 and 2004.}

{The rest of the paper is organized as follows. \sectref{sec:kahigi:2} deals with the salient aspects of the verb extensions in the language: the extensions and, their meanings,  syntactic effects (i.e. valence) and  productivity. The co-occurrence constraints of the verbal extensions are presented in \sectref{sec:kahigi:3}. \sectref{sec:kahigi:4} focuses on the parameters of morphosyntactic variation \citep{GuéroisEtAl2017}, and, lastly, \sectref{sec:kahigi:5} presents the conclusion.} 

\section{Sumbwa verb extensions}\label{sec:kahigi:2}

{To facilitate comparison, we will use \citeauthor{Meeussen1967}'s (\citeyear[92]{Meeussen1967}, \citeyear{Meeussen1969}) and \citegen[144]{Guthrie1971} verb extension reconstructions. \tabref{tab:kahigi:1} shows the modern Sumbwa reflexes of these} {reconstructions. Some less well-known extensions, i.e.} \mbox{\textit{{}-agil-}}, \mbox{\textit{{}-agan-}}, {\textit{{}-al-}}, and {\textit{{}-l-}}{, are not shown in \tabref{tab:kahigi:1} but will be covered in the discussion below.} 

\begin{table}
\begin{tabularx}{\textwidth}{llQ}
\lsptoprule
{Verb extension}  & {Proto-Bantu form} & {Sumbwa form}\\
\midrule
 {Applicative}\footnote{The terms used here to refer to the extensions are by no means universal. Alternative terms for applicative include e.g. applied, directive, prepositional, dative; for persistive e.g. intensive, double prepositional; and for frequentative e.g. augementative (cf. \citealt{Madan1903, Johnson1939, Ashton1947, Guthrie1962, Lodhi2002, Schadeberg2003}).} & *-id- & {{{}-il-}\footnote{The extensions are represented in their basic form. The surface form is determined by vowel harmony, i.e. (i) the extension vowel /i/ is lowered to [e] in the environment of root vowels /ε/ or /ɔ/, e.g. /sεk-il-a/ becomes [sεkεla] ‘laugh for’ and /βɔl-il-a/ becomes [βɔlεla] ‘rot for’; (ii) the extension vowel /u/ is lowered to [o] in the environment of root vowel /ɔ/, e.g. /dɔd-uulul-a/ becomes [dɔdɔɔlɔla] ‘unsew’.}}\\
Passive & *-u-, *-ibu- & {}-w-, iβw-\\
{{Causative}} & *-i, *ici- & {{{}-i-, -iisi- (also instrumental)}}\\
Impositive & {{*-ik-}} & {}-ik-\\
Stative & *-ik-/*-uk-/*-ad- & {}-ik-, -uk-, inkan-\\
Associative & *-an- & {}-an-, -i-\\
Reversive (active) & *-ud-, *-udud- & {}-ul-, -ulul\\
Reversive (stative) & *-uk- & {}-uk-\\
Persistive & {{*-idid-}} & {{{}-ilil-, -ilizi-}}\\
Frequentative (tr.) & *-agud- & {}-agul-\\
{{Frequentative (intr.)}} & *-aguk- & {}-aguk-\\
Denominative & *-ap-, *-p-, *-apad- & {}-h-, -ahal-\\
Reiterative & *-udud- & {}-uul-, -ul-\\
Static & *-am- & {}-am-\\
Contactive & *-at- & {}-at-\\
\lspbottomrule
\end{tabularx}
\caption{Bantu verb extension reconstructions and reflexes in Sumbwa. These PB reconstructions and their reflexes in Sumbwa do not include “imbricated” forms that characterize perfective constructions. These imbricated forms, which are found in many Bantu languages, are characterized by mutations which do not occur in simple verbal constructions and non-perfective contexts (cf. \citealt{Berger1937-8, Givón1970, Mould1972, Bastin1983, Kahigi1989, Hyman1995, Kula2001}).}
\label{tab:kahigi:1}
\end{table}

{What follows is a presentation and discussion of the verb extensions in terms of their morphological shapes, their meanings,  syntactic effects (i.e. change in valence) and  productivity.} 

\subsection{The applicative: \textit{{}-il{}-}}\label{sec:kahigi:2.1}

{The applicative extension denotes an action performed on behalf of, on, at, towards an entity, etc. (cf. \cites[]{Madan1903}[218--220]{Ashton1947}[46]{Rugemalira2005}). The surface forms of the} {applicative extension are} {\textit{{}-il-}} {and} {\textit{{}-el-,}} which are distributed in accordance with vowel harmony. Examples are given in \tabref{tabex:kahigi:1}.

\begin{table}
\begin{tabularx}{\textwidth}{lll>{\raggedright\arraybackslash}p{.15\textwidth}Q}
\lsptoprule
     &     Verb root &  Gloss &  Applicative stem &  Examples\\
          \midrule
 {a.} & {\textit{{}-nó-}} & drink & {\textit{{}-nó-il-a [nwééla]}} & {\gll \textit{a-la-mu-nó-il-a} \textit{βusele}\\
 \textsc{sm1-pst-om1}-drink-\textsc{appl-fv} beer\\
\glt ‘he drank beer for him’}\\
 {b.} & {\textit{{}-tem-}} & cut & {\textit{{}-tem-il-a [temela]}} & {\gll {\textit{a-la-mu-tem-il-a}}  {\textit{muti}}\\
\textsc{sm1-pst-om1}-cut-\textsc{appl-fv} {{tree}}\\
\glt {{‘he cut a tree for him’}}}\\
 {c.} & {\textit{{}-dod-}} & {sew} & {\textit{{}-dod-il-a [dodela]}} & {\gll \textit{a-ø-dod-el-a} \textit{kaaya}\\
\textsc{sm1-hab}-sew-\textsc{appl-fv} home\\
\glt {{‘he sews wood at home’}}}\\
 {d.} & {\textit{{}-huul-}} & whip & {\itshape {}-huul-il-a} & {\gll \textit{a-ø-mu-huul-il-a}   \textit{βuzoβe}\\
\textsc{sm1-hab-om1}-whip-\textsc{appl-fv} laziness\\
\glt {{‘he whips her for laziness’}}}\\
 {e.} & {\textit{{}-iluk-}} & run & {\itshape {}-iluk-il-a} & {\gll \textit{a-li-iluk-il-a} \textit{mu-numba}\\
\textsc{sm1-pst-}run-\textsc{appl-fv} 17-house\\
\glt ‘he ran into the house’}\\
\lspbottomrule
\end{tabularx}
\caption{Examples of the applicative extension}
\label{tabex:kahigi:1}
\end{table}

{The examples in \tabref{tabex:kahigi:1}  illustrate the various functions of the applicative in Sumbwa. Examples a and b represent the sense of ‘do sth on behalf of’ or ‘for the benefit of'; example c represents the sense of ‘location’; example d  the sense of ‘reason’; and example e the ‘directional’ sense. The applicative extension is a valence-increasing extension, i.e. it adds an extra argument to the verb, cf. example b which without the applicative extension would be} {\textit{a-la-tem-a}} {(\textsc{sm1-pst-}cut-\textsc{fv}), ‘he cut’. Verbs of all types (transitive, intransitive) take the applicative extension; consequently, it is one of the most productive extensions in the language.} 

\subsection{The passive: \textit{{}-u-, -iβu-}}\label{sec:kahigi:2.2}

{The passive construction is found in many languages around the world. In Bantu, the Proto-Bantu passive extension *}{\textit{{}-u-}}{/*}{\textit{{}-ibu-}} {is still directly reflected in many existing languages, although some of them (e.g. the A70 group) use other extensions for the passive due to innovations that occurred in those languages (\citealt{BostoenNzang-Bie2010}).}

{Sumbwa has retained the reconstructed extensions:} {\textit{-u-/-iβu-}}{, with the vowel /u/ becoming the glide /w/, a diachronic process common in Bantu; thus} {\textit{{}-u-/-iβu-}} {becomes} {\textit{-w-/-iβw-.} }{The short extension} {\textit{{}-w-}} {occurs in the environment after consonant-final roots, while} {\textit{{}-iβw}}{}- occurs after vowel-final roots or extended stems. Examples are given in \tabref{tabex:kahigi:2}.


\begin{table}

\begin{tabularx}{\textwidth}{llllQ}
\lsptoprule
         & Verb & Gloss & Passive & Gloss\\
         \midrule
  {a.} & {\textit{{}-lim-}} & cultivate; farm & {{\textit{{}-lim-u-a  [limwa]}}} & be cultivated\\
  {b.} & {\textit{{}-tem-}} & cut & {{\textit{{}-tem-u-a  [temwa]}}} & be cut\\
  {c.} & {\textit{{}-dul-}} & bore & {{\textit{{}-dul-u-a  [dulwa]}}} & be bored\\
  {d.} & {\textit{{}-tah-}} & draw (water) & {{\textit{{}-tah-u-a  [tahwa]}}} & be drawn\\
  {e.} & {\textit{{}-li-}} & eat & {{\textit{{}-li-iβu-a  [lííβwa]}}} & be eaten\\
  {f.} & {\textit{{}-fil-}} & take sb or sth somewhere &  {{\textit{{}-fil-u-a   [filwa]}}} & {{be taken somewhere}}\\
  \lspbottomrule
\end{tabularx}
\caption{Examples of the passive extension}
\label{tabex:kahigi:2}
\end{table}

{The passive construction licenses an optional prepositional phrase, which indicates the logical agent.  Examples in \REF{ex:kahigi:3} illustrate this point:}

\ea\label{ex:kahigi:3}
    \ea\label{ex:kahigi:3a} \gll \textit{mu-kiima}   \textit{a-la-léét-a}           \textit{si-taβo}\\
 1-woman     \textsc{sm1-pst-}bring-\textsc{fv}    7-book\\
\glt {‘A woman brought a book.’}

    \ex\label{ex:kahigi:3b} \gll \textit{si-taβo}    \textit{si-la-léét-w-a}             \textit{ne}    \textit{mu-kiima}\\
7-book       \textsc{sm7-pst-}bring-\textsc{pass-fv}  by    1-woman\\
\glt {‘A book was brought by a woman.’}
    \z
\z

Apart from this standard construction, Sumbwa also has some other passive constructions that need to be noted. One of these is the ‘passive form of infinitival nouns’, usually placed in class 15 in the noun class system. Note the following examples in \REF{ex:kahigi:4}:

\ea\label{ex:kahigi:4}
    \ea\label{ex:kahigi:4a} \textit{[kufwíílwa kwiinki kulalééta naku niinki]}\\
\gll ku-fu-il-u-a           ku-inki     ku-la-léét-a             naku       n-inki\\
15-die-\textsc{appl-pass-fv}   15-much  \textsc{sm15-pst-}bring-\textsc{fv}    10.misery   10-much\\
\glt {‘Being bereaved many times brought a lot of misery.’}

    \ex\label{ex:kahigi:4b} \textit{[kulekaniisiβwa mukazi waamwe kulamusaayiisja]}\\
\gll ku-lek-an-isi-iβu-a              mu-kazi   wa-amwe ku-la-mu-saay-isi-a\\
15-leave-\textsc{ass-caus-pass-fv}   1-wife   1-his \textsc{sm15-pst-om1-}be.angry-\textsc{caus-fv}\\
\glt {‘Separating from his wife made him angry.’}
    \z
\z

{In the data we have, constructions of this type do not occur with the ``by phrase'' noted above. Another construction to note is the ``passive with the locative noun''.  This is illustrated in \REF{ex:kahigi:5}:}

\ea\label{ex:kahigi:5}    
    \ea\label{ex:kahigi:5a} \textit{[malaβo galapaambwamo munúúmba]}\\
\gll ma-laβo     ga-la-pamb-w-a-mo              mu-numba\\
 6-flower   \textsc{sm6-pst-}decorate-\textsc{pass-fv-cl18}   18-house\\
\glt {‘Flowers were decorated in the house.’}

    \ex\label{ex:kahigi:5b} \textit{[munúúmba halapaambwamo malaβo]}\\
\gll mu-numba   ha-la-pamb-w-a-mo               ma-laβo\\
18-house     \textsc{sm16-pst-}decorate-\textsc{pass-fv-cl18}   6-flower\\
\glt {Lit.: ‘In the house there was decorated flowers’}
    \z
\z

{These two examples illustrate what is known as locative inversion. In the first example, the locative noun} {\textit{munúúmba} }{occurs after the passivized verbal construction} {\textit{galapaambwamo}}{; in the second example, the locative noun is in subject position. The final} {\textit{{}-mo}} {is a locative enclitic. What needs to be noted is that in both sentences the passive construction does not need to, but can have a ``by phrase'', as illustrated in \REF{ex:kahigi:6} below:}

\ea\label{ex:kahigi:6}
    \ea\label{ex:kahigi:6a} \textit{[malaβo galapaambwamo munúúmba ne mukiima]}\\
\gll ma-laβo   ga-la-pamb-w-a-mo              mu-numba   ne     mu-kiima\\
6-flower   \textsc{sm6-pst-}decorate-\textsc{pass-fv-cl18}   18-house   by   1-woman\\
\glt {‘Flowers were decorated in the house by a woman’}

 \ex\label{ex:kahigi:6b} \textit{[munúúmba halapaambwamo malaβo ne mu-kiima]}\\
\gll mu-numba   ha-la-pamb-w-a-mo               ma-laβo   ne   mu-kiima\\
18-house     \textsc{sm16-pst-}decorate-\textsc{pass-fv-cl18}    6-flower   by   1-woman\\
\glt {Lit.: ‘In the house there was decorated flowers by a woman’}
    \z
\z

{A final construction to note is the ``passive with the past participle''. This construction is illustrated in example \REF{ex:kahigi:7} below:}

\ea\label{ex:kahigi:7} \gll \textit{mu-gunda}   \textit{gu-li}      \textit{βu-lim-w-e}\\
 3-farm    \textsc{sm3}-be     \textsc{pp}-cultivate-\textsc{pass-fv}\\
\glt {‘The farm is cultivated.’}
\z

{As can be noticed here, the passive past participle in the language is characterized by the verb -}{\textit{li} }{‘be’ followed by main verb with the structure:} {\textit{βu-}}\textsc{vrt-pass}{\textit{{}-e}}{. In the data we have, this type of construction is also not followed by the ``by phrase''. It is used when one intends to imply a ``state'' of an entity.}

{The passive extension is a valence-decreasing extension, i.e. it deducts an argument from the verb as the examples in \REF{ex:kahigi:3a} and \REF{ex:kahigi:3b} show. In addition, all transitive verbs take the passive extension; it is thus a productive extension in the language.} 

\subsection{The causative: \textit{{}-i-/-iisi-}}\label{sec:kahigi:2.3}

{The extension generally denotes ‘causing someone to perform some action’. Historically, the extension} {\textit{{}-i-} }{has caused what is known as spirantization in Bantu (cf. \citealt{Schadeberg1995, Bostoen2008}). This has resulted in morphophonemic alternations in root-final position in all roots or bases ending in /p, b, l, t, d, k, g/. These change to [f, v, z, s, z, s, z] respectively. These changes are illustrated in \tabref{tabex:kahigi:8} below. Root-final consonants which change are shown as well as consonants which do not change. Notice that all roots with final vowels take the -}{\textit{iisi-}} {extension, while some roots (e.g.} {\textit{{}-βamb-}} {‘peg out’) take both extensions. All instances of} {\textit{i}} {and} {\textit{u}} {occurring at morpheme boundaries change to the corresponding glides [j] and [w], respectively.} 

\begin{table}
\small\tabcolsep=.66\tabcolsep
\begin{tabularx}{\textwidth}{llQQQQ}
\lsptoprule
 & Verb root & \multicolumn{2}{c}{Causative \textit{-i-}} & Causative \textit{-iisi-} & Causative gloss\\\cmidrule(lr){3-4}
 & & {{Change of root-final C}} & {{No change of root-final C}} &  & \\
 \midrule
 {a.} & {\textit{{}-βamb-}} & {{{}-}{\textit{βamv-i-a}}} &  &  {{{}-}{\textit{βamb-iisi-a}}} & cause to peg out\\
 {b.} & {\textit{{}-puup-}} & {\itshape {}-puuf-i-a} &  &  & cause to be light \\
 {c.} & {\textit{{}-tem-}} &  & {\itshape {}-tem-i-a} & {\itshape {}-tem-iisi-a}

{\itshape [teméésja]} & cause to cut\\
 {d.} & {\textit{{}-lil-}} &  {{\-}{\textit{{}-liz-i-a}}} &  &  & cause to cry\\
 {e.} & {\textit{{}-sees-}} &  &  & {\itshape {}-sees-iisi-a}

{\itshape [seeseesja]} & {{cause to spill}}\\
 {f.} & {\textit{{}-kan-}} &  & {\itshape {}-kan-i-a} & {\itshape {}-kan-iisi-a} & cause to groan or creak\\
 {g.} & {\textit{{}-swiiz-}} &  &  & {\itshape {}-swiiz-iisi-a} & cause to filter or strain\\
 {h.} & {\textit{{}-dod-}} & {\itshape {}-doz-i-a} &  & {\itshape {}-dod-iisi-a}

{{\textit{[dodéésja]}}} & cause to sew\\
 {i.} & {\textit{{}-hit-}} & {\itshape {}-his-i-a} &  &  & cause to pass\\
 {j.} & {\textit{{}-ak-}} & {\itshape {}-as-i-a} &  &  & cause to burn\\
 {k.} & {\textit{{}-og-}} & {\itshape {}-oz-i-a} &  & {\itshape {}-og-iisi-a}

{\itshape [ogéésja]} & cause to bathe\\
 {l.} & {\textit{{}-saay-}} &  &  & {\itshape {}-saay-iisi-a} & cause to be angry\\
 {m.} & {\textit{{}-oβah-}} &  & {\itshape {}-oβah-i-a} &  & cause to fear\\
 {n.} & {\textit{{}-no-}} &  &  & {\itshape {}-no-iisi-a}

{\itshape [nwéésja]} & cause to drink\\
 {o.} & {\textit{{}-li-}} &  &  & {\itshape {}-li-iisi-a} & cause to eat\\
 {p.} & {\textit{{}-gu-}} &  &  & {\itshape {}-gu-iisi-a} & cause to fall\\
\lspbottomrule
\end{tabularx}
\caption{Examples of the causative extension}
\label{tabex:kahigi:8}
\end{table}

{The only root-final consonants that do not change are} {\textit{m, s, z, n,}} {y, and} {\textit{h}}{. As can be seen in examples (\tabref{tabex:kahigi:8}a--p), an extra argument has been added to the basic valence of the respective verbs, e.g.} {\textit{a-la-gu-a} }\textsc{(sm1-pst-}fall-\textsc{fv}) ‘she fell’ with the causative extension becomes: {\textit{a-la-mu-gu-iisi-a} }\textsc{(sm1-pst-om1}-fall-\textsc{caus-fv}) ‘he caused her to fall’. As to productivity, the causative extension is highly productive; it applies to intransitive and transitive verbs.

\subsection{The instrumental: \textit{-i-,} \textit{-iisi-}}\label{sec:kahigi:2.4}

{In Sumbwa, the causative extensions} {\textit{{}-i-/-iisi-}} {are also instrumental, as the following examples show:}

\ea\label{ex:kahigi:9}
    \ea\label{ex:kahigi:9a} \textit{[alamulíísja mwaana]}\\
 \gll a-la-mu-li-iisi-a                mu-ana\\
 \textsc{sm1-pst-om1-}eat-\textsc{caus-fv}   1-child\\
\glt {‘She caused the child to eat.’ (i.e. she fed the child)}

    \ex\label{ex:kahigi:9b} \textit{[alalíísja siliko]}\\
 \gll a-la-li-iisi-a                si-liko\\
 \textsc{sm1-pst-}eat\textsc{-caus-fv}   7-spoon\\
\glt {‘She ate with a spoon.’}
    \z
\z

{As can be seen here, there is a crucial difference between the first and the second sentence, despite the fact that both use the same extension} {\textit{{}-iisi-}}{. In \REF{ex:kahigi:9a},} {\textit{{}-iisi-}} {is used in its causative sense, while in \REF{ex:kahigi:9b} it is used in its instrumental sense. This is clearly a case of ``causative-instrumental syncretism''. This case is also found in other Bantu languages. For example, \citet{Jerro2017} discusses a similar case in Kinyarwanda, and \citet{Wald1998} argues for a split between Bantu languages that use the applicative and those that use the causative extension to mark instrumentals; he hypothesizes that the latter is an innovation.}

{In Sumbwa, the instrumental is a valence-increasing extension, as shown in \tabref{tabex:kahigi:8}b above. It is also quite productive.} 

\subsection{The impositive: \textit{-ik-}}\label{sec:kahigi:2.5}

{In Bantu,} {\textit{{}-ik-}} is also the extension for the stative/neuter, which is described below. The impositive differs from the stative in that it has a ‘causative’ meaning. The action of the verb results in "causing something or somebody to be in some position or state". In \tabref{tabex:kahigi:10} are some examples.

\begin{table}
\begin{tabularx}{\textwidth}{llllQ}
\lsptoprule
& Verb root & Gloss & Impositive & Gloss\\
\midrule
 {a.}& {\textit{{}-tumb-}} & {{increase (\textsc{intr})}} & {{\textit{{}-tumb-ik-a}}} & gather together in a heap\\
 {b.}& {\textit{{}-loβ-}} & get wet & {\itshape {}-loβ-ik-a} & soak\\
 {c.}& {\textit{{}-oβah-}} & be afraid & {\itshape {}-oβah-ik-a} & frighten\\
 {d.}& {\textit{{}-om-}} & {{dry (\textsc{intr})}} & {\itshape {}-om-ik-a} & dry sth over fire\\
 {e.}& {\textit{{}-hengam-}} & {{tilt (\textsc{intr})}} & {\itshape {}-hengam-ik-a} & cause to tilt sideways\\
 \lspbottomrule
\end{tabularx}
\caption{Examples of the impositive extension}
\label{tabex:kahigi:10}
\end{table}

{Since it is causative in meaning, the impositive extension is valence-increasing. However, it is not as pervasive as the normal causative} {\textit{{}-i-/-isi-}}{; it may be characterized as slightly productive.} 

\subsection{The stative: \textit{-ik-}}\label{sec:kahigi:2.6}

{The stative is also referred to as ``neuter'' \citep{Schadeberg2003}. This extension denotes a state already completed or still going on. It also denotes ``potentiality'', depending on the context. Most verbs use} {\textit{{}-ik-}}{, but a few use {}-}{\textit{inkan-}}{. It is the} {\textit{{}-ik-}} {extension that is still productive.}

\begin{table}
\begin{tabularx}{\textwidth}{lllll}
\lsptoprule
 & Verb root & Gloss & Stative & Gloss\\
 \midrule
 {a.} & {\textit{{}-lim-}} & cultivate; farm & {\itshape {}-lim-ik-a} & {{be cultivated}}\\
 {b.} & {\textit{{}-tem-}} & cut & {\itshape {}-tem-ek-a} & be cut\\
 {c.} & {\textit{{}-nó-}} & {{drink}} & {\itshape {}-nó-ik-a} & {{be drinkable}}\\
 {d.} & {\textit{{}-dul-}} & {{bore}} & {\itshape {}-dul-ik-a} & be bored\\
 {e.} & {\textit{{}-bhon-}} & {{see}} & {\itshape {}-bhon-inkan-a} & be seen\\
 {f.} & {\textit{{}-mani-}} & {{know}} & {\itshape {}-mani-inkan-a} & be known\\
 {g.} & {\textit{{}-suh-}} & forget & {\itshape {}-suh-inkan-a} & be forgotten\\
\lspbottomrule
\end{tabularx}
\caption{Examples of the stative extension}
\label{tabex:kahigi:11}
\end{table}

{Note that the Sumbwa} {\textit{{}-ik-/-inkan-}} {extension is also found in other Bantu languages such as Swahili (e.g.} {\textit{{}-pik-}} {‘cook’;} {\textit{{}-pik-ik-a}} {[pikika] ‘be cooked’;} {\textit{{}-on-}} {‘see’;} {\textit{{}-on-ikan-a}} {[onekana] ‘be visible’). The stative is valence-decreasing, e.g.} {\textit{a-la-dul-a lubaβo} }\textsc{(sm1-pst-}bore-\textsc{fv} a plank) ‘he made holes into a plank’ becomes {\textit{lubaβo lu-la-dul-ik-a} }{‘a/the plank was bored’. It is a characteristic of the Bantu stative that the agent is never expressed.}

\subsection{The associative: \textit{{}-i-/-an-}}\label{sec:kahigi:2.7}

{We use the term ``associative'' following \citet{Ashton1947} and \citet[164]{MagangaSchadeberg1992}. Ashton states:}

\begin{quote}
{The term “Associative” is used instead of the more generally accepted term “Reciprocal” as found in the} {\textit{Standard Swahili-English Dictionary}}{, for in addition to reciprocity -NA expresses other aspects of association such as concerted action, interaction and interdependence (and in some cases disassociation) (\citeyear[240]{Ashton1947}).}
\end{quote}

{In Sumbwa, there are two affixes that denote action performed mutually or associatively,} {\textit{{}-i-}} {and} {\textit{{}-an-}}{. The first, \--}{\textit{i-}}{, is a pre-verb root affix that is also used as a reflexive marker. Thus, it is polysemous and highly productive. Examples for the associative} {\textit{{}-i-}} are in \tabref{tabex:kahigi:12}.

\begin{table}
\begin{tabularx}{\textwidth}{llllQ}
\lsptoprule
 & Verb & Gloss & Associative \textit{-i-} & Gloss\\
 \midrule
 {a.} & {\textit{ku-huul-a}} & {{to hit}} & {{\textit{ku-i-huul-a}}} & {{to hit each other}}\\
 {b.} & {\textit{ku-taahi-a}} & {{say farewell}} & {\itshape ku-i-taahi-a} & to say farewell to each other\\
 {c.} & {\textit{ku-gú-a}} & to fall & {\itshape ku-i-gú-il-a} & to fall on each other\\
 {d.} & {\textit{ku-li-a}} & to eat & {\itshape ku-i-li-a} & to eat each other\\
 \lspbottomrule
\end{tabularx}
\caption{Examples of the associative -\textit{i}-}
\label{tabex:kahigi:12}
\end{table}

{Due to the ambiguity of the} {\textit{{}-i-} }{affix, the above associative constructions could also be glossed as ‘to hit oneself’, ‘to bid farewell to oneself’, ‘to fall on oneself’, and ‘to eat oneself’, respectively. It is important to note that the meaning of the construction with the} {\textit{{}-i-} }{affix will depend on the linguistic  context. A reflexive meaning will always imply that the subject NP and the object NP are identical, as illustrated in \REF{ex:kahigi:13} below:}

\ea\label{ex:kahigi:13}
    \ea\label{ex:kahigi:13a} \gll {\textit{mu-ana}}{\textit{\textsubscript{i}}} \textit{a-la-huul-a}    \textit{mu-ana}{\textit{\textsubscript{i}}} {(Subject NP = Object NP)}\\
 {1-child}{\textsubscript{i}} \textsc{sm-pst-}hit\textsc{-fv}   1-child{\textsubscript{i}}\\
 
    \ex\label{ex:kahigi:13b} \gll \textit{mu-ana}\textsubscript{i}  \textit{a-la-i}{\textit{\textsubscript{i}}}-\textit{huul-a}\\
     {1-child}{\textsubscript{i}} \textsc{sm-pst-ref}{\textsubscript{i}}-hit-\textsc{fv}\\ 
    \glt ‘the child hit him/herself’
    \z
\z

{As can be seen here, while the reflexive in \REF{ex:kahigi:13b} allows a singular subject (i.e. an agent which is at the same time a patient), the reciprocal in \REF{ex:kahigi:14} below can only allow a plural subject, which is consistent with reciprocity or associativeness (i.e. two or more individuals doing the same thing to each other).}

\ea\label{ex:kahigi:14}  
 %^   \ea\label{ex:kahigi:14a} 
 %   \gll \textit{mu-ana}\textit{\textsubscript{i}} \textit{a-la-huul-a} 
 % \textit{mu-ana}\textit{\textsubscript{j}}\\
% {1-child}{\textsubscript{i}} \textsc{sm-pst-}hit-\textsc{fv} 1- %child\textsubscript{j}\\   
%    \ex\label{ex:kahigi:14b} 
\gll \textit{βa-ana}\textit{\textsubscript{i}} 
 \textit{βa-la-i-huul-a}\\ 
 {2-child}{\textsubscript{i}} \textsc{sm-pst-ref}\textsubscript{i}-hit-\textsc{fv}\\ 
    \glt ‘the children hit each other’
    \z
% \z

{A further point to note here is that Sisumbwa is not the only language in zone F to express associativeness using the pre-verbal {\textit{{}-i-}}. Expression of associativeness using {\textit{{}-i-}} has also been found in Rimi \citep[172--174]{Olson1964}, Rangi \citep[144]{Stegen2002} and Sukuma \citep[172--173]{Batibo1985}. The extent to which this phenomenon is widespread in Bantu awaits investigation.} 

{The second affix, -}{\textit{an-}}{, competes with} {\textit{{}-i-}} {as an associative marker. Our analysis shows that it is unproductive; reciprocity/associativeness in Sumbwa is more frequently expressed by the pre-verbal} {\textit{{}-i-}}{. Some examples illustrating} {\textit{{}-an\--}} are given in \tabref{tabex:kahigi:15}.


\begin{table}
\begin{tabularx}{\textwidth}{llQQ>{\raggedright\arraybackslash}p{.35\textwidth}}
\lsptoprule
 & Verb root & {Gloss} & Associative -\textit{an}- & {Gloss}\\
  \midrule
 {a.}& {\textit{{}-taag-}} & throw away & {\itshape {}-taag-an-a} & leave or abandon each other\\
 {b.}& {\textit{-las-}} & shoot using a bow & {\itshape {}-las-an-a} &  {{shoot at each other; fight using bows and arrows}}\\
 {c.}& {\textit{-som-}} & stab & {\itshape {}-som-an-a} & stab each other\\
 {d.}& {\textit{-bhut-}} & give birth & {\itshape {}-bhut-an-a} &  {{give birth in great numbers}}\\
 {e.}& {\textit{-tol-}} & backbite & {\itshape {}-tol-an-a} & backbite each other\\
\lspbottomrule
\end{tabularx}
\caption{Examples of the associative -\textit{an}-}
\label{tabex:kahigi:15}
\end{table}

{An inspection of the ``-}{\textit{an-}} {associative'' in actual speech and in \citet{Kahigi2008a} shows that the verb roots targeted are either -CV(V)C- (many) or -CV(V)CVC- (fewer). There are, in addition, a few verbs that form their associative forms with} {\textit{{}-aan-}} {instead of} {\textit{{}-an-}}. Examples of these are given in \tabref{tabex:kahigi:16}.

\begin{table}
\begin{tabularx}{\textwidth}{lllll}
\lsptoprule
 & {{{{Verb} {root}}}} & {Gloss} & {{{{Associative} {-\textit{aan}-}}}} & {Gloss}\\
 \midrule
 {a.} & {\textit{{}-lek-}} & leave & {\itshape {}-lekaana} & leave each other\\
 {b.} & -{\textit{sang-}} & find & {\itshape {}-sangaana} & {{meet}}\\
 {c.} & {\textit{{}-gabh-}} & divide & {\itshape {}-gabhaana} & {{share}}\\
\lspbottomrule
\end{tabularx}
\caption{Examples of the associative -\textit{aan}-}
\label{tabex:kahigi:16}
\end{table}

There are also a few examples which appear to have an associative meaning but are not related to any corresponding verb roots. These are shown in \tabref{tabex:kahigi:17}.

\begin{table}
\begin{tabularx}{\textwidth}{llQlQ}
\lsptoprule
& {Verb root} & {Gloss} & {Associative}  & {Gloss}\\\midrule
 {a.}& {\textit{-fu-}} & {{die}} & {{\textit{{}-fw-aan-a}}} & {{quarrel}}\\
 {b.}& {\textit{{}-nó-}} & {{drink}} & {\itshape nw-aan-a} &  {{become friends or well-mixed}}\\
 {c.}& {\textit{{}-lag-}} & {{say goodbye to king}} & {\itshape lag-an-a} & {{promise}}\\
 {d.}& {\textit{{}-sas-}} & make bed & {\itshape sas-an-a} & take from each other by force\\
 {e.}& {\textit{{}-tong-}} & claim for payment of debt & {\itshape tong-an-a}  & {{quarrel}}\\
\lspbottomrule
\end{tabularx}
\caption{Examples of the associative -\textit{aan}-/-\textit{an}-}
\label{tabex:kahigi:17}
\end{table}

{Although the forms} {\textit{{}-fu-,}} {\textit{{}-nó-, -lag-, -sas-,} }{and} {\textit{-tong-}} {by themselves are found in the language, their current meanings have nothing to do with the associative forms.}

As examples in \REF{ex:kahigi:18} below show, the associative {\textit{{}-i-}}/{\textit{{}-an-}} is valence-decreasing; the affixes can only license an external argument.

\ea\label{ex:kahigi:18}
%    \ea\label{ex:kahigi:18a} \gll \textit{a-la-mu-las-a}                   %\textit{mudugu} \textit{wamwe}\\
 %   \textsc{sm1-past-}shoot.with.arrow-\textsc{fv}     relative   his\\
 %   \glt {‘He shot his relative with arrows.’}

    \ea\label{ex:kahigi:18a} \gll \textit{βa-la-i-las-a}\\
    \textsc{sm2-past-rec-}shoot.with.arrows-\textsc{fv}\\
    \glt {‘They shot each other with arrows.’}

    \ex\label{ex:kahigi:18b} \gll \textit{βa-la-las-an-a}\\
    \textsc{sm2-past-rec-}shoot.with.arrows-\textsc{fv}\\ 
    \glt {‘They shot each other with arrows.’} 
    \z
\z

% {It is interesting to note that while ‘shoot’ with} {\textit{{}-i-}} {in %\REF{ex:kahigi:18a} only means ‘shoot each other’, ‘shoot’ with} {\textit{{}-%an-}} {may mean ‘shoot each other’ or ‘fight’ (with arrows or spears).}

\subsection{The reversive: \textit{-ul-/uk-{\textasciitilde}-uul-/-uuk-{\textasciitilde}-ulul-/-uluk-}}
\label{sec:kahigi:2.8}

Although there is variation in the form of the reversive extension, the overall meaning is the same: it denotes the opposite of the meaning of the verb root. The reversive involves both the active and stative/neuter forms.\footnote{The \textit{{}-ulul-} extension may also be considered to be a “double reversive” \textit{{}-ul-ul-} (an intensive reversive), while \textit{{}-uluk-} may be regarded as a combination of the reversive extension \textit{{}-ul-} followed by the reversive stative \textit{{}-uk-}.} Examples are given in \tabref{tabex:kahigi:19} and \tabref{tabex:kahigi:20}.


\begin{table}
\begin{tabularx}{\textwidth}{llQlQ}
\lsptoprule
& {{{{Verb} {root}}}} & {Gloss} & {{{{Reversive} {active}}}} & {{{{Gloss}}}}\\
\midrule
 {a.}& { \textit{-anz-}} & spread e.g. bedclothes & {\itshape {}-anz-ulul-a} & take off sth spread\\
 {b.}& { \textit{-βamb-}} & peg out & {\itshape {}-βamb-ulul-a} & unpeg\\
 {c.}& { \textit{-tung-}} & thread & {\itshape {}-tung-ulul-a} & unthread\\
 {d.}& { \textit{-fung-}} & close & {\itshape {}-fung-ulul-a} & open\\
 {e.}& { \textit{-gongom-}} & bend; stoop & {\itshape {}-gongom-ul-a} & raise\\
 {f.}& { \textit{-semb-}} & tie with rope/ bandage & {\itshape {}-semb-ul-a} & unwrap; untie\\
\lspbottomrule
\end{tabularx}
\caption{Examples of the reversive active -\textit{ulul}-}
\label{tabex:kahigi:19}
\end{table}

\begin{table}
\begin{tabularx}{\textwidth}{llQlQ}
\lsptoprule
 & {{{{Verb} {root}}}} & {Gloss} & {{{{Reversive} {stative}}}} & {{{{Gloss}}}}\\
 \midrule
 {a.}& {\textit{-anz-}} & spread e.g. bedclothes & {\itshape {}-anz-uluk-a} & (of bed) have bedclothes taken off\\
 {b.}& {\textit{-βamb-}} & peg out & { {\textit{{}-βamb-uluk-a}}} & become unpegged\\
 {c.}& {\textit{-tung-}} & { {thread}} & {\itshape {}-tung-uluk-a} & be become unthreaded\\
 {d.}& {\textit{-fung-}} & { {close}} & {\itshape {}-fung-uluk-a} & be opened\\
 {e.}& {\textit{-gongom-}} & bend; stoop & {\itshape {}-gongom-uk-a} & be raised\\
 {f.}& {\textit{-semb-}} & tie with rope/ bandage & {\itshape {}-semb-uk-a} & become unwrapped or untied\\
\lspbottomrule
\end{tabularx}
\caption{Examples of the reversive stative -\textit{uluk}-}
\label{tabex:kahigi:20}
\end{table}

{All the above cases show that the reversive extension is suffixed directly to the verb root. There are some cases, however, where the basic verb root does not exist in its simple form. Here we have what we may call ``complex verb roots''; but the reversive suffix is never attached to them directly. The  examples in \tabref{tabex:kahigi:21} illustrate this point.} 


\begin{table}
\begin{tabularx}{\textwidth}{llQlQ}
\lsptoprule
 & {{{{Verb} {root}}}} & {Gloss} & {{{Reversive}}} & {{{Gloss}}}\\
 \midrule
 {a.}& {\textit{{}-anikil-}} & spread out to dry & {\itshape {}-an-ul-a} & take in sth spread out to dry\\
 {b.}& {\textit{{}-hanik-}} & hang up & {\itshape {}-han-uul-a} & take down sth suspended \\
 {c.}& {\textit{{}-higik-}} & arrange cooking stones & {\itshape {}-hig-ul-a} & move (e.g. cooking stones)\\
 {d.}& {\textit{{}-siβik-}} & { {tether}} & {\itshape {}-siβ-ul-a} & untether\\
\lspbottomrule
\end{tabularx}
\caption{Examples of the reversive -\textit{ul}-}
\label{tabex:kahigi:21}
\end{table}

{In \tabref{tabex:kahigi:21}, comparison of the reversive forms with the complex verb roots shows that the ``complex verb roots'' have been reanalyzed; they have been shortened to} {\textit{{}-an-, -han-, -hig-,} }{and}{ \textit{-siβ-,} }{respectively, before the attaching of the reversive extension. The ``shortened'' roots in the reversive are probably older forms which are no longer used. Swahili has a similar pattern, e.g.} {\textit{{}-anik-}} {‘spread out to dry’, -}{\textit{an-u-a}} {‘take in sth spread out to dry’;} {\textit{{}-angik-}} {‘hang up; suspend’,} {\textit{{}-ang-u-a}} {‘pick; knock down’.}

{Regarding valence, the active reversive} {\textit{{}-ul-}} {is valence-maintaining, while the stative} {\textit{{}-uk-}} {is valance-decreasing. Overall, the extension is slightly productive.}

\subsection{The persistive \textit{-ilil-, -ilizi-}}\label{sec:kahigi:2.9}

{\citet[144]{Guthrie1971} uses the term ``persistive'' for} {\textit{*-idid-}}{, which is reflected in Sumbwa as} {\textit{{}-ilil-.}} {The form is literally the doubling of the applicative} {\textit{{}-il-}}{ which is why \citet{Johnson1939} called it ``double-prepositional''. The extension, however, does not have any applicative meaning; rather, it denotes action performed persistently or continuously but intensively. This is probably why \citet[214, 243--245]{Ashton1947} uses the term ``augmentative''.} 

{The second extension,} {\textit{{}-ilizi-}}{, is assumed to be a direct outcome of the spirantization that occurred from a combination of *}{\textit{{}-idid- + -i-} }{(causative).}

{Both of these extensions occur in many Eastern Bantu languages, although only slightly productively. For instance, in Swahili, where the} {\textit{*-idid-}} {extension is no longer very productive, we have:} {\textit{{}-end-} }{‘go’}{ \textit{> -end-ele-a} }{‘progress’}{\textit{, -shik-} }{‘hold’}{\textit{> -shik-ili-a} }{‘hold on tightly or insist’}{\textit{, -pend-} }{‘like’} {\textit{> -pend-ele-a} }{‘favour’; and for *}{\textit{{}-idid- + -i-} } {we  have:} {\textit{{}-end-} }{‘go’ > -}{\textit{end-elez-a} }{‘cause to progress’, etc. (\citealt[214, 243--245]{Ashton1947}).} 

{In Sumbwa,} {\textit{{}-ilil-}} {occurs mostly with the meaning of ‘persistence, continuity or intensity’, while} {\textit{{}-ilizi-}} {occurs with two meanings: 1) doing sth persistently, continuously or intensively, and 2) doing sth persistently, continuously or intensively for payment. Given these meanings, the extension does not trigger any change in the basic valence of the verb. We exemplify each of these below.}

\subsubsection{The persistive \textit{-ilil-}}\label{sec:kahigi:2.9.1}

This extension is illustrated in the  examples in \tabref{tabex:kahigi:22}.

\begin{table}
\begin{tabularx}{\textwidth}{llllQ}
\lsptoprule
& {{{{Verb} {root}}}} & {Gloss} & {\itshape -ilil-}  & {Gloss}\\
 \midrule
{a.}& {\textit{-ling-}} & {{look}} & {\itshape {}-ling-ilil-a} & look at for a long time\\
 {b.}& {\textit{{}-lind-}} & {{await}} & {\itshape {}-lind-ilil-a} & await for a long time\\
 {c.}& {\textit{-lond-}} & {{follow}} & {\itshape {}-lond-elel-a} & follow for a long time\\
 {d.}& {\textit{-sek-}} & {{laugh}} & {\itshape {}-sek-elel-a} & laugh for a long time\\
 {e.}& {\textit{-kwáát-}} & hold, seize & {\itshape {}-kwáát-ilil-a} & hold firmly for a long time\\
 \lspbottomrule
\end{tabularx}
\caption{Examples of the persistive -\textit{ilil}-}
\label{tabex:kahigi:22}
\end{table}

We also note probable instances of lexicalization in the  two examples in \tabref{tabex:kahigi:23}.

\begin{table}
\begin{tabularx}{\textwidth}{llllQ}
\lsptoprule
 & {{{{Verb} {root}}}} & {{{{Gloss}}}} & {persistive -\textit{ilil}-}  & {{{{Gloss}}}}\\
 \midrule
 {a.} & \textit{-mani-} & { {know}} & {\itshape {}-mani-ilil-a} & get accustomed to\\
 b.& \textit{-fuk-} & pour (water) & {\itshape {}-fuk-ilil-a} & irrigate\\
\lspbottomrule
\end{tabularx}
\caption{Examples of lexicalized persistive -\textit{ilil}-}
\label{tabex:kahigi:23}
\end{table}

{It should be noted that the meanings of the extended verb bases} {\textit{{}-mani-ilil-}} {and} {\textit{{}-fuk-ilil-}}{, though relatable to the meaning ‘doing sth persistently, continuously or intensively’, may be argued to be qualitatively different from the meanings of the verb roots} {\textit{{}-mani-}} {and} {\textit{{}-fuk-}}.

\subsubsection{The persistive: \textit{-ilizi-}\textit{\textsuperscript{1}}}
\label{sec:kahigi:2.9.2}

{The extension} {\textit{{}-ilizi-}}{\textit{\textsuperscript{1}}} {retains much of the meaning ‘doing sth persistently, continuously or intensively’, as the  examples in \tabref{tabex:kahigi:24} show.}


\begin{table}
\begin{tabularx}{\textwidth}{llQlQ}
\lsptoprule
 &{{{{Verb} {root}}}} & {{{{Gloss}}}} & {{\textit{{}-ilizi-}}}{{\textit{\textsuperscript{1}}}} & {{{{Gloss}}}}\\
 \midrule
 {a.} &{\textit{{}-an-}} & groan & {\itshape {}-an-ilizi-a} & groan for a long time; yell\\
 {b.} &{\textit{{}-anguh-}} & hasten & {{\textit{{}-anguh-ilizi-a}}} & hasten overmuch\\
 {c.} &{\textit{{}-gum-}} & harden & {\itshape {}-gum-ilizi-a} & persevere\\
 {d.}&{\textit{-gelek-}} & put sth on top of another & {\itshape {}-gelek-elezi-a} & pile up to the top\\
 {e.} &{\textit{{}-kooβ-}} & look for & {{\textit{{}- kooβ-elezi-a}}} & search for a long time\\
\lspbottomrule
\end{tabularx}
\caption{Examples of persistive -\textit{iliz-\textsubscript{1}}}
\label{tabex:kahigi:24}
\end{table}

\subsubsection{The persistive: \textit{-ilizi-}\textit{\textsuperscript{2}}}
\label{sec:kahigi:2.9.3}

{The extension} {\textit{{}-ilizi-}}{\textit{\textsuperscript{2}}}{, besides the meaning ‘doing sth persistently, continuously or intensively’, has the meaning ‘doing some work continuously for payment’.  Examples are given below:}

\begin{table}
\begin{tabularx}{\textwidth}{llQlQ}
\lsptoprule
& {{{{Verb} {root}}}} & {{{{Gloss}}}} & {{{\textit{{}-ilizi-}}}{{\textit{\textsuperscript{2}}}}} & {Gloss}\\
\midrule
 {a.}& {\textit{-diim-}} & herd & {\itshape {}-diim-ilizi-a} & herd for payment\\
 {b.}& {\textit{{}-hakul-}} & harvest honey from beehive & {\itshape {}-hakul-ilizi-a} & harvest honey for payment\\
 {c.}& {\textit{{}-tumam-}} & work & {\itshape {}-tumam-ilizi-a} & work for payment\\
 {d.}& {\textit{-fufúúl-}} & { {clear farm ready for planting}} & {\itshape {}-fufúúl-ilizi-a} & clear farm for payment\\
 {e.}& {\textit{{}-lim-}} & { {cultivate}} & {\itshape {}-lim-ilizi-a} & { {cultivate for payment}}\\
\lspbottomrule
\end{tabularx}
\caption{Examples of persistive -\textit{iliz-\textsubscript{2}}}
\label{tabex:kahigi:25}
\end{table}

{The use of this extension for the meaning ‘doing some work continuously for payment’ is still moderately productive, i.e. it can be used with any verb describing work that one does for payment.}

\largerpage
\subsection{Frequentative \textit{{}-agul, -aguk-}}\label{sec:kahigi:2.10}

{These are moderately productive extensions, not only in Sumbwa but also in related languages, e.g. Sukuma (\citealt{RichardsonMann1966}) and Nyamwezi (\citealt{MagangaSchadeberg1992}). I follow \citet[144]{Guthrie1971} in using the term ``frequentative'' for these extensions. Other terms have also been used: for example, augmentative \citep{Lodhi2002} and iterative-separative (\citealt[167]{MagangaSchadeberg1992}).} 

{I take} {\textit{*-agud-}} {to be present in Proto-Bantu as reconstructed by \citet[144]{Guthrie1971}. I also assume that the stative form} {\textit{*-aguk-}} {was also present.}\footnote{ {Another possibility would be to assume \textit{*-agul-} and \textit{*-aguk-} to have evolved as a combination of \textit{*-ag-} (frequentative) and \textit{*-ul-} (intensive, active) and \textit{*-uk-} (intensive, stative). The extension \textit{*-ag-/-ang-} is glossed as “repetitive” and noted to behave “tonally as an extension” but functions also as an inflectional suffix with the meaning “durative/habitual” \citep[72]{Schadeberg2003}.}} {In Sumbwa, these extensions are reflected as} {\textit{{}-agul-}} {and} {\textit{{}-aguk-.}} 

{The usual meaning for these extensions is ‘doing something quickly or hurriedly, excessively or clumsily, and repeatedly’;} {\textit{{}-agul-}} {has an active meaning, and does not add any argument to the basic valence of a verb, while} {\textit{{}-aguk-}} {has a stative meaning, and deducts an argument from the basic valence of a verb. The various contexts in which these extensions occur are spelt out below.}

\subsubsection{Examples involving both extensions}\label{sec:kahigi:2.10.1}

{These examples include verbs which are transitive when used with} {\textit{{}-agul-}} {but become intransitive when used with} {\textit{{}-aguk-.}} {Consider \tabref{tabex:kahigi:26}:}

\begin{table}
\setlength{\tabcolsep}{3pt}
\small
\begin{tabularx}{\textwidth}{lllQQQQ}
\lsptoprule
 &\makecell[tl]{{{Verb}}\\{{root}}} & {Gloss} & {{{{Frequentative} {{active}}}}} & {{{{Gloss}}}} & {{{{Frequentative} {{stative}}}}} & {{{{Gloss}}}}\\
 \midrule
 {a.}& {\textit{-simb-}} & {{dig}} & {\itshape {}-simb-agul-a} & dig intensively & { {\textit{{}-simb-aguk-a}}} & become dug up intensively\\
 {b.} &{\textit{{}-bel-}} & {{break}} & {\itshape {}-bel-agul-a} & break into small pieces & {\itshape {}-bel-aguk-a} & be broken into small pieces\\
 {c.} &{\textit{{}-kat-}} & {{cut}} & { {\textit{{}-kat-agul-a}}} & cut into small pieces & {\itshape {}-kat-aguk-a} & be cut into small pieces\\
{d.}& {\textit{{}-kuul-}} & \makecell[tl]{extract;\\uproot} & {\itshape {}-kuul-agul-a} & extract/

uproot repeatedly & {\itshape {}-kuul-aguk-a} & become  extracted/ uprooted intensively\\
 {e.} &{\textit{{}-dul-}} & {{bore}} & {\itshape {}-dul-agul-a} & pierce with many holes & {\itshape {}-dul-aguk-a} & { {be riddled with many holes}}\\
\lspbottomrule
\end{tabularx}
\caption{Examples involving frequentative -\textit{agul}-, -\textit{aguk}-}
\label{tabex:kahigi:26}
\end{table}

\subsubsection{Examples involving \textit{{}-agul-} only}\label{sec:kahigi:2.10.2}

{There are many verbs which take the} {\textit{{}-agul-}} {extension, but not} {\textit{{}-aguk-}} {. In \tabref{tabex:kahigi:27} are a few examples.} {As can be noted here, the verbs involved are all transitive verbs which do not allow the} {\textit{{}-aguk-} }{extension.}

\begin{table}
\begin{tabularx}{\textwidth}{ll>{\raggedright\arraybackslash}p{.15\textwidth}>{\raggedright\arraybackslash}p{.2\textwidth}Q}
\lsptoprule
 &{{{{Verb} {root}}}} & {{{{Gloss}}}} & {{{{Frequentative} {{active}}}}} & {Gloss}\\
 \midrule
 {a.}&{\textit{-βoh-}} & { {tie}} & {\itshape {}-βoh-agul-a} & tie clumsily and quickly\\
 {b.} &{\textit{{}-βeez-}} & carve & {\itshape {}-βeez-agul-a} & carve clumsily\\
 {c.}&{\textit{-tuk-}} & insult & {\itshape {}-tuk-agul-a} & { {insult excessively}}\\
 {d.}&{\textit{-tah-}} & draw (water, etc.) & {\itshape {}-tah-agul-a} & draw (water, etc.) excessively or quickly\\
 {e.} &{\textit{{}-moog-}} & shave & {\itshape {}-moog-agul-a} & shave clumsily\\
\lspbottomrule
\end{tabularx}
\caption{Examples involving -\textit{agul}- only}
\label{tabex:kahigi:27}
\end{table}





\subsubsection{Examples involving -\textit{aguk-} only}
\label{sec:kahigi:2.10.3}

{Notice that all the verbs involved here are intransitive, and the suffixing of} {\textit{{}-aguk}}{{}- to the verb root results in some sort of ‘state’ or ‘condition’.}

\begin{table}
\begin{tabularx}{\textwidth}{lll>{\raggedright\arraybackslash}p{.2\textwidth}Q}

\lsptoprule
 & { {{{Verb} {root}}}} & {Gloss} & {{{{Frequentative} {{stative}}}}} & {{{{Gloss}}}}\\
 \midrule
{a.}& {\textit{{}-lul-}} & {{be bitter}} & {\itshape {}-lul-aguk-a} & become excessively bitter\\
 {b.}& {\textit{{}-gin-}} & be fat & {\itshape {}-gin-aguk-a} & become excessively fat\\
 {c.}& {\textit{{}-duuh-}} & be blunt & {\itshape {}-duuh-aguk-a} & become blunt quickly\\
 {d.}& {\textit{{}-nunk-}} & {{smell}} & {\itshape {}-nunk-aguk-a} & stink\\
\lspbottomrule
\end{tabularx}
\caption{Examples involving -\textit{aguk}- only}
\label{tabex:kahigi:28}
\end{table}

\subsubsection{Cases involving -CVCVC- verb roots}\label{sec:kahigi:2.10.4}


All the above examples involve -CVC- verb roots, which represent the overwhelming majority of verbs targeted by these extensions. Occurrence of these extensions with                {}-CVCVC- verb roots is not commonly observed, although the examples in \tabref{tabex:kahigi:29} have been attested.


\begin{table}
\begin{tabularx}{\textwidth}{lll>{\raggedright\arraybackslash}p{.2\textwidth}Q}

\lsptoprule
 & {{{Verb} {root}}} & {{{{Gloss}}}} & {{{{Frequentative} {active}}}} & {{{{Gloss}}}}\\
 \midrule
 {a.}& {\textit{-tafun-}} & chew & {\itshape {}-tafun-agul-a} & chew intensively or clumsily\\
 {b.} &{\textit{{}-heken-}} & chew & { {\textit{{}-heken-agul-a}}} & chew excessively\\
\lspbottomrule
\end{tabularx}
\caption{Cases involving -CVCVC- verb roots}
\label{tabex:kahigi:29}
\end{table}

It should be noted that the majority of -CVCVC- verb roots may take any of the extensions after the elision of the final VC. In Tables~\ref{tabex:kahigi:30} and \ref{tabex:kahigi:31} are some examples.


\begin{table}
\begin{tabularx}{\textwidth}{ll>{\raggedright\arraybackslash}p{.2\textwidth}>{\raggedright\arraybackslash}p{.2\textwidth}Q}

\lsptoprule
 & {Verb} {root} & {Gloss} & { {{{Frequentative} {active}}}} & {Gloss}\\
 \midrule
 {a.}& {\textit{{}-hepul-}} & { {cause to be hungry}} & {\itshape {}-hep-agul-a} & cause to be excessively hungry\\
 {b.}& {\textit{{}-hogol-}} & break sth off & {\itshape {}-hog-agul-a} & { {break sth off quickly} }\\
 {c.}& {\textit{{}-gangul-}} & crack sth & {\itshape {}-gang-agul-a} & crack sth quickly\\
 {d.}& {\textit{{}-tandul-}} & tear & {\itshape {}-tand-agul-a} & tear quickly or excessively\\
\lspbottomrule
\end{tabularx}
\caption{Examples of frequentative -\textit{agul}- in -CVCVC- verb roots}
\label{tabex:kahigi:30}
\end{table}

\begin{table}[t]
\begin{tabularx}{\textwidth}{ll>{\raggedright\arraybackslash}p{.2\textwidth}>{\raggedright\arraybackslash}p{.2\textwidth}Q}

\lsptoprule
 & {Verb} {root} & {{{{Gloss}}}} & {Frequentative} {stative} & {{{{Gloss}}}}\\
 \midrule
 {a.}& {\textit{{}-hepul-}} & cause to be hungry & {\itshape {}-hep-aguk-a} & become excessively hungry\\
 {b.}& {\textit{{}-hogol-}} & { {break sth off}} & {\itshape {}-hog-aguk-a} & become broken off quickly\\
 {c.}& {\textit{{}-gangul-}} & crack sth & {\itshape {}-gang-aguk-a} & become cracked quickly or excessively\\
 {d.}& {\textit{{}-tandul-}} & { {tear}} & {\itshape {}-tand-aguk-a} & become torn quickly or excessively \\
\lspbottomrule
\end{tabularx}
\caption{Examples of frequentative -\textit{aguk}- in -CVCVC- verb roots}
\label{tabex:kahigi:31}
\end{table}

\begin{sloppypar}
As can be noted here, the verb roots \textit{-hepul-, -hogol-, -gangul-, -tandul-,} \mbox{\textit{-sambul-},} and {\textit{{}-konyol-}} {are not used with the extensions;}\footnote{ {It is possible that the} {\textit{{}-ul-} }{in} {\textit{{}-hep-ul-, -hog-ol-, -tand-ul-,}} {etc. was once used as an extension in prehistory.}} {instead, the truncated forms, i.e.} {\textit{{}-hep-, hog-,}} {etc. are used.} 
\end{sloppypar}

\subsubsection{Cases of lexicalization of \textit{{}-agul-, -aguk-}}\label{sec:kahigi:2.10.5}

Cases of lexicalization include words whose verb roots do not have any meaning connection with the extended form. These words include the examples in \tabref{tabex:kahigi:32}. {The asterisk (*) in the verb-root column indicates that these are not attested verb roots but ``reconstructed'' ones.}

\begin{table}
\small
\begin{tabularx}{\textwidth}{llQQQQ}

\lsptoprule
 & {Verb} {root} & {{{{Frequentative} {active}}}} & {{{{Gloss}}}} & {{{{Frequentative} {stative}}}} & {{{\textit{Gloss}}}}\\
 \midrule
 {a.}& {\textit{*-ken-}} & {\itshape {}-ken-agul-a} & destroy; spoil & {\itshape {}-ken-aguk-a} & become destroyed/ spoiled\\
 {b.}& {\textit{*-pos-}} & {\itshape {}-pos-agul-a} & { {break into} {many pieces}} & {\itshape {}-pos-aguk-a} & become broken into many pieces\\
 {c.}& {\textit{*-tamp-}} & {\itshape {}-tamp-agul-a} & pierce repeatedly with pointed weapon & { {\textit{{}-tamp-aguk-a}}} & become riddled with piercings\\
 {d.}& {\textit{*-hunz-}} & {\itshape {}-hunz-agul-a} & exhaust & {\itshape {}-hunz-aguk-a} & be exhausted\\
\lspbottomrule
\end{tabularx}
\caption{Cases of lexicalization of -\textit{agul}- and -\textit{aguk}-}
\label{tabex:kahigi:32}
\end{table}

{The asterisk (*) in the verb-root column indicates that these are not attested verb roots but ``reconstructed'' ones.}

\subsection{The denominative \textit{{}-h-}, \textit{{}-ahal-}}\label{sec:kahigi:2.11}

These extensions are used in the derivation of verbs from nouns or adjectives. In \tabref{tabex:kahigi:33} are examples.

\begin{table}
\begin{tabularx}{\textwidth}{lQQQQ}

\lsptoprule
 & {{{{Adjective} {or} {nominal} {base}}}} & {Gloss} & {Derived verb} & {Gloss}\\
 \midrule
 {a.}& {\textit{{}-angu}} & quick & {\itshape angu-h-a} & hurry\\
 {b.}& {\textit{{}-gazi}} & wide & {\itshape gazi-h-a} & become wide\\
 {c.}& {\textit{{}-ganzi}} & favourite & { {\textit{ganzi-h-a}}} & become favourite\\
 {d.}& {\textit{{}-bhanz}} & brave & {\itshape bhanzi-h-a} & become brave\\
 {e.}& {\textit{{}-ingi}} & many & {\itshape ingi-h-a} & become many\\
 {f.}& {\textit{{}-daasa}} & { {sterile (of animals)}} & {\itshape daas-ahal-a} & become sterile\\
 {g.}& {\textit{{}-guzu}} & strength & { {\textit{{}-guzu-ahal-a [guzuhala]}}} & become strong\\
\lspbottomrule
\end{tabularx}
\caption{Examples of the denominative -\textit{h}- and -\textit{ahal}-}
\label{tabex:kahigi:33}
\end{table}

{There are many more examples. The above derived examples may in turn accept other extensions, such as causative, applicative and passive.} 

\subsection{The reiterative: \textit{{}-ul-, uul-}}\label{sec:kahigi:2.12}

{This term is \citegen[144]{Guthrie1971}, and the implication is one of ``added quantity or quality or intense effort''. The actual meaning will depend on the meaning of the root. In Sumbwa, there appears to be two forms of this extension:} {\textit{{}-uul-}} {and} {\textit{{}-ul-.} }Examples for the first one are in \tabref{tabex:kahigi:34}.

\begin{table}
\begin{tabularx}{\textwidth}{llQ>{\raggedright\arraybackslash}p{.17\textwidth}Q}

\lsptoprule
 & {{{{Verb} {root}}}} & {Gloss} & {{{Reiterative} {-\textit{uul}-}}} & {{{{Gloss}}}}\\
\midrule
 {a.}& {\textit{{}-kam-}} & squeeze (as when milking cow) & { {\textit{{}-kam-uul-a}}} & squeeze tightly (as when preparing juice with hands)\\
 {b.}& {\textit{{}-kemb-}} & trim; pare & {\itshape {}-kemb-uul-a} & { {trim evenly}}\\
 {c.}& {\textit{{}-tah-}} & { {draw (e.g. water)}} & {\itshape {}-tah-uul-a} & scoop or draw in large quantities\\
 {d.}& {\textit{{}-han-}} & { {admonish}} & {\itshape {}-han-uul-a} & advise strongly\\
 {e.}& {\textit{{}-sem-}} & { {bevel}} & {\itshape {}-sem-uul-a} & { {cut evenly a large portion of}}\\
\lspbottomrule
\end{tabularx}
\caption{Examples of the reiterative -\textit{uul}-}
\label{tabex:kahigi:34}
\end{table}

{As can be seen, the meanings include ‘squeeze tightly’, ‘trim evenly’, ‘draw in large quantities’, ‘advise strongly’, and ‘cut a large portion of’. There are also a few items where the} {\textit{{}-ul-}} {instead of} {\textit{{}-uul-}} is used, as in \tabref{tabex:kahigi:35}.

\begin{table}
\begin{tabularx}{\textwidth}{llllQ}
\lsptoprule
 & {{{{Verb} {root}}}} & Gloss & { {{{Reiterative} {-\textit{ul}-}}}} & {Gloss}\\
 \midrule
 {a.}& {\textit{-simb-}} & {{dig}} & {\itshape {}-simb-ul-a} & uproot\\
 {b.}& {\textit{{}-hel-}} & grind coarsely & {\itshape {}-hel-ul-a} & { {grind excessively coarsely}}\\
 {c.}& {\textit{{}-seng-}} & {{cut}} & {\itshape {}-seng-ul-a} & cut trees for building\\
\lspbottomrule
\end{tabularx}
\caption{Examples of the reiterative -\textit{ul}-}
\label{tabex:kahigi:35}
\end{table}

{The reiterative} {\textit{{}-ul-}} {and} {\textit{{}-uul-}} {extension is only slightly productive and does not affect the basic valence of the verb.}

\subsection{The static \textit{{}-am-}}\label{sec:kahigi:2.13}

The general meaning for this extension is ‘assume or be in a position or state’.  Consider the  examples in \tabref{tabex:kahigi:36}.

\begin{table}
\begin{tabularx}{\textwidth}{l>{\raggedright\arraybackslash}p{.2\textwidth}llQ}

\lsptoprule
 & {{{{Verb/nominal} {root}}}} & {{{{Gloss}}}} & {{{{Static} {-\textit{am}-}}}} & {{{{Gloss}}}}\\
 \midrule
 {a.}& {\textit{-fuk-}} & kneel & { {\textit{{}-fuk-am-a}}} & kneel obediently; menstruate\\
 {b.}& {\textit{i-papa}} & wing & {\itshape {}-pap-am-a} & beat (of bird’s wing); palpitate\\
 {c.}& {\textit{{}-gazi}} & wide & {\itshape {}-gaz-am-a} & { {widen}}\\
 {d.}& {\textit{{}-gond-}} & bend & {\itshape {}-gond-am-a} & bend\\
 {e.}& {\textit{{}-hanga}} & alive & {\itshape {}-hang-am-a} & live for a long time\\
\lspbottomrule
\end{tabularx}
\caption{Examples of the static -\textit{am}-}
\label{tabex:kahigi:36}
\end{table}

In \tabref{tabex:kahigi:36}, examples a and d involve verb roots; example b involves a noun, and c and e involve adjectival roots. The static meaning of the derived verb is quite clear.

There are also some examples involving verb roots whose meanings have been lost but may be recoverable through connection with related static and impositive derived forms. In \tabref{tabex:kahigi:37} are examples.


\begin{table}
\begin{tabularx}{\textwidth}{lllllQ}

\lsptoprule
 & {{{Verb} {root}}} & Gloss & {{{{Static} {-\textit{am}-}}}} & {{{{Gloss}}}} & {{{{Impositive} {forms}}}}\\
 \midrule
 {a.}& {\textit{-gol-}} & * & {\itshape {}-gol-am-a} & {{\textsc{intr} bend}} & {{\textit{{}-go-lek-a}} {(\textsc{tr} bend)}}\\
 {b.}& {\textit{{}-heng-}} & * & {\itshape {}-heng-am-a} & {{\textsc{intr} tilt}} & {{\textit{{}-heng-ek-a}} {(\textsc{tr} tilt)}}\\
 {c.}& {\textit{{}-send-}} & * & {\itshape {}-send-am-a} & be leaning & {{\textit{{}-send-ek-a}} {(\textsc{tr} lean sth against sth else)}}\\
 {d.}& {\textit{{}-in-}} & * & {\itshape {}-in-am-a} & bend, stoop & {{\textit{{}-in-ik-a}} {(\textsc{tr} lay over on one side)}}\\
\lspbottomrule
\end{tabularx}
\caption{Examples of the static -\textit{am}- involving verb roots whose meanings have been lost}
\label{tabex:kahigi:37}
\end{table}

{All the static forms in \tabref{tabex:kahigi:36} and \tabref{tabex:kahigi:37} may take the applicative and causative extensions. As can be noted, the extension is only slightly productive.}

\subsection{The contactive: \textit{{}-at-}}\label{sec:kahigi:2.14}

This extension is not productive. The original meaning of the extension implies some contact by an agent on a beneficiary or patient. In \tabref{tabex:kahigi:38} are the few examples available in our data.

\begin{table}
\begin{tabularx}{\textwidth}{lllll}

\lsptoprule
 & {{{{Verb} {root}}}} & {{{{Gloss}}}} & {{{{Contactive} {-\textit{at}-}}}} & Gloss\\
 \midrule
 {a.}& {\textit{{}-kwa-}} & {}-{}-{}- & {\itshape {}-kwa-at-a} & {{hold}}\\
 {b.}& {\textit{{}-kumb-}} & cover & {\itshape {}-kumb-at-a} & embrace\\
 {c.}& {\textit{{}-fumb-}} & close & {\itshape {}-fumb-at-a} & embrace\\
 {d.}& {\textit{{}-lam-}} & {}-{}-{}- & {\itshape {}-lam-at-a} & stick firmly; adhere\\
\lspbottomrule
\end{tabularx}
\caption{Examples of contactive -\textit{at}-}
\label{tabex:kahigi:38}
\end{table}

{In these examples, only} {\textit{{}-fumb-}} {and} {\textit{{}-kumb-}} {have meanings that may be grasped by a native speaker. The other verb roots have no meaning that may be related to the meaning of the extended bases. The extension does not change the basic valence of the verb.}

\subsection{Other less-known verb extensions}\label{sec:kahigi:2.15}

{In the data we have, there are four non-productive extensions,} {\textit{{}-agil-, -agan}}{{}-, -}{\textit{al-} }{and} {\textit{{}-l-.}}

\subsubsection{\textit{-agil-}}\label{sec:kahigi:2.15.1}

{This is probably a combination of *}{\textit{{}-ag-}} {(a repetitive}{ }{extension, cf. \citealt[96]{Schadeberg2003}) and the applicative *}{\textit{{}-id-}}. In Sumbwa, it appears to have a ‘persistive meaning’, as the  examples in \tabref{tabex:kahigi:39} show.

\begin{table}
\begin{tabularx}{\textwidth}{lllll}
\lsptoprule
{a.} &{\textit{{}-sunt-}}& {limp}& {\textit{sunt-agil-a} }& {limp along}\\
{b.} &{\textit{{}-yomb-}}& {speak}& {\textit{yomb-agil-a}}& {talk too much}\\
{c.}& {\textit{-kand-}}& {step on}& {\textit{kand-agil-a}}& {walk fast in hot sun}\\
{d.}& {\textit{-kump-}}& {stumble}&{\textit{kump-agil-a}}& {stumble along}\\
{e.}& {\textit{-don-  ?}}& ? &{\textit{don-agil-a}}& {start to walk (infant)}\\
{f.}& {\textit{-met-  ?}}& {shine}& {\textit{met-agil-a}} &{strut about}\\
{g.}& {\textit{-kum-  ?}}& {gather}& {\textit{kum-agil-a}} &{move}\\
{h.}& {\textit{-zwi-}} &{ideophone} &{\textit{zwi-agil-a [zwiigila]}} &{squack (like a baby)} \\
\lspbottomrule
\end{tabularx}
\caption{Examples involving -\textit{agil}-}
\label{tabex:kahigi:39}
\end{table}

{Notice that the only examples where there is a close relation between the basic root and the extended base are the first four. The last example -}{\textit{zwi}}{{}- is ideophonic: it imitates the cry of an infant. In the data we have, there are only about ten words with the} {\textit{{}-agil-}} {extension.}

\subsubsection{\textit{{}-agan-}}\label{sec:kahigi:2.15.2}

This also appears to be a persistive extension. There are not many examples (\tabref{tabex:kahigi:40}).

\begin{table}
\begin{tabularx}{\textwidth}{llllQ}

\lsptoprule
{a.}&{\textit{βumb-}} &{mould (with clay)}& {\textit{βumb-agan-a} }&{stick together (soil)}\\
{b.}& {\textit{{}-om- (intr)}}& {harden; dry} &{\textit{om-agan-a}}& {harden from drying; solidify}\\
 {c.}&{\textit{-vimb-}}&{swell} &{\textit{vimb-agan-a}}&{swell very much}\\
{d.}&{\textit{-tab-}} &{confuse} &{\textit{tab-agan-a}} &{get quite confused}\\
 {e.}&{\textit{-tonto(lok)-}}& {weaken}&{\textit{tonto-gan-a}} &{weaken further}\\
\lspbottomrule
\end{tabularx}
\caption{Examples involving -\textit{agan}-}
\label{tabex:kahigi:40}
\end{table}

{As can be observed here, the examples show a clear meaning relationship between the root and the extended base. There is, however, one example where the} {\textit{{}-agan-}} {extension functions as a denominative extension:}

\ea\label{ex:kahigi:41} {\textit{ma-paβa}}\hspace{2em} {mischief}\hspace{2em} {\textit{{}-paβ-agan-a}}\hspace{2em} {be mischievous}
\z

{In this example,}{ }{\textbf{\textit{{}-}}}{\textit{paβa} }{‘mischief’}{ }{is a nominal stem; when suffixed with} {\textit{{}-agan-}} {it changes into a verb.}

\subsubsection{\textit{{}-al-}}\label{sec:kahigi:2.15.3}

{The} {\textit{*-ad- > -al-}} {extension was recognized by \citet[90]{Meeussen1967}, but he notes that it ‘appears partly as an expansion, partly as a suffix with ill-defined meaning’. Examples given include:} {\textit{*-dúad-} }{‘be ill’,} {\textit{*-démad}}{{}- ‘be crippled’,} {\textit{*-ikal-}} {‘sit’ whose Sumbwa reflexes are}{\textit{: -lúal-, -lémal-, -ikal-,}} {respectively. \citet[72]{Schadeberg2003} calls it ``extensive'', by which he means ‘to be in a spread-out position’. This meaning is probably borne out by such Sumbwa words as} {\textit{{}-samb-aal-a}} {‘spread’, but there are not many. However, inspection of the various examples in Sumbwa shows that there is no single meaning that may be attributed to this extension. For some of the examples, the meaning of this extension is ‘change into a state’, as illustrated in \tabref{tabex:kahigi:42} below:}

\begin{table}
\begin{tabularx}{\textwidth}{lQlQl}

\lsptoprule
 & {{{{Verb/nominal} {root/stem}}}} & {{{{Gloss}}}} & {\itshape {}-al-} & {{{Gloss}}}\\
 \midrule
 {a.}& {\textit{syaha}} {(n)} & anger & { {\textit{syah-al-a  [syaahala]}}} & be angry\\
 {b.}& {\textit{{}-humb-}} & be stupid & { {\textit{{}-humba-al-a  [humbaala]}}} & lose one’s mind\\
 {c.}& {\textit{{}-lema}} {(n)} & lame & { {\textit{{}-lem-al-a  [lemala]}}} & become lame\\
\lspbottomrule
\end{tabularx}
\caption{Examples involving -\textit{al}- ‘change into a state’}
\label{tabex:kahigi:42}
\end{table}

{For the remaining few examples, there are different senses attached to the extension. For instance, the extension}{ }{has an intensive}{ }meaning in the  examples in \tabref{tabex:kahigi:43}.

\begin{table}
\begin{tabularx}{\textwidth}{llQlQ}
\lsptoprule
a. & {\textit{{}-sees-}}&{pour out, e.g. water}& {\textit{seesek-al-a  [seesekala]}}& {pour out completely}\\
b. & {\textit{{}-siis-}}& {spoil (\textsc{tr})}& {\textit{siisik-al-a  [siisikala]}}& {spoil completely} \\
\lspbottomrule
\end{tabularx}
\caption{Examples involving -\textit{al}- ‘intensive’}
\label{tabex:kahigi:43}
\end{table}


{In our last example,} {\textit{{}-al-}} {acts as a denominative suffix, creating a verb which indicates action:}

\ea\label{ex:kahigi:44} {\textit{i-suβa}}\hspace{2em} {urine container}\hspace{2em} {\textit{{}-suβa-al-a} }{[suβaala]\hspace{2em}    urinate}
\z

{In this example,}{ }{\textbf{\textit{{}-}}}{\textit{suβa} }{‘urine container’}{ }{is a nominal stem; if suffixed with} {\textit{{}-al-}} {it changes into the verb -}{\textit{suβa-al-a}}.

\subsubsection{\textit{-l-}}
\label{sec:kahigi:2.15.4}

This extension is also noted by \citet[91]{Meeussen1967}, and he gives examples such as
\textit{*-ganud-} ‘narrate’ \textit{< *-ganú} {‘tale’},
\textit{-púmúd-} ‘breathe’, \textit{*-púmu} ‘breath, rest’,
\textit{*-pokud-} ‘make blind’ \textit{< *-poku} ‘blind’, etc.
In Sumbwa, examples showing the \textit{-l-} extension are presented in \tabref{tabex:kahigi:45}.

\begin{table}
\begin{tabularx}{\textwidth}{lllQQ}

\lsptoprule
{a.} &{\textit{{}-hofu} } &{‘blind’}& {\textit{{}-hofu-l-a}} {\textit{→  [hofula]}}& {‘be blind’}\\
{b.} &{\textit{{}-panti}} &{‘deaf’}& {\textit{{}-panti-l-a}} {\textit{→  [pantila}]}&    ‘become deaf’\\
{c.} &{\textit{{}-sefu}} &{‘nausea’}& {\textit{{}-sefu-l-a}} {\textit{→ [sefula]}}& {‘nauseate’}\\
\lspbottomrule
\end{tabularx}
\caption{Further examples involving -\textit{l}- (with various meanings)}
\label{tabex:kahigi:45}
\end{table}


{As can be seen in \tabref{tabex:kahigi:45}, the meaning of the examples a and b is ‘change into a state’, but the meaning of example c} {\textit{{}-sefu-l-a}} {‘to nauseate’ is causative.}

\section{Co-occurrence constraints}\label{sec:kahigi:3}

{Co-occurrence constraints, otherwise referred to as “suffix ordering” constraints in the literature, have been the subject of intense discussion in Bantu linguistics studies for quite some time (cf. e.g. \citealt{Baker1985}, \citealt{Alsina1999}, \citealt{Hyman2002}). The main debate is whether there are Pan-Bantu constraints that govern multiple affixation. Three main approaches may be identified: a semantic or compositional approach (whereby affix order is based on ‘relevance’ -- the most relevant is closer to the verb root, and the least farthest from the verb root -- cf. \citet{Bybee1985}); a syntactic approach (whereby affix ordering reflects syntactic derivation, cf. \citegen{Baker1985} ``mirror principle'') and the strictly morphological approach (whereby suffix ordering is strictly governed by morphological criteria in the majority of cases, while exceptions are handled in reference to semantic or syntactic criteria cf. \citealt{Hyman2002}). For our purposes, we consider \citegen{Hyman2002} morphocentric approach to be germane as a point of departure. We summarize the main ideas, and consider whether the proposed orderings are supported by the Sumbwa data.}

{\citet{Hyman2002}, using the Optimality Theory framework, assumes that Bantu suffix ordering is determined by the ranking of two licensors:}

\begin{enumerate}
\item \textsc{carp (caus -- appl -- recp -- pass)} Template -- which licenses suffix ordering in most Bantu languages; quite general and highly ranked;

\item {Non-templatic constraints (i.e. semantic compositionality or MIRROR constraints which deal with all cases which do not follow the CARP template).}
\end{enumerate}


To formulate these postulates, Hyman used data from Chichewa, Kinande, Chibemba, Chimwiini, Luganda, Ciyao, Emakua, Nyakyusa, Tonga and other Bantu languages. \tabref{tab:kahigi:2} shows some examples of suffix orders licensed by the two proposed licensors: the CARP template and the non-templatic constraints. The suffix order examples are from Chichewa: {\textit{{}-mang-}} {= ‘tie’,} {\textit{{}-its-}} {= CAUS,} {\textit{{}-ir-}} {APPL,} {\textit{{}-an-}} {= RECP.}

\begin{table}
\begin{tabularx}{\textwidth}{lQlQ}

\lsptoprule

\multicolumn{2}{p{.45\textwidth}}{{{Suffix orders governed by the CARP template}}} & \multicolumn{2}{p{.45\textwidth}}{{{Suffix orders governed by non-templatic constraints}}}\\
\midrule
{CAR} &{\textit{mang-its-ir-an}}

{ { }{‘cause to tie for each other’}} & {CRA}& {\textit{mang-its-an-ir-an-}}

{  {‘cause to tie for each other’}}\\
{CA}& {\textit{mang-its-ir}}

{ { }{‘cause to tie for’}} & \\

AR & \textit{mang-ir-an}

{  {‘tie for each other’}} & \\
\lspbottomrule
\end{tabularx}
 \caption{Examples of suffix ordering in Bantu}
 \label{tab:kahigi:2}
 \end{table}

{Now, what is the situation like in Sumbwa? One important difference which sets Sumbwa (and other similar languages) apart from languages like Chichewa is that the former does not have a  productive reciprocal/associative extension} {\textit{{}-an-}}{; instead, reciprocity is expressed by the pre-verb root} {\textit{{}-i-}}{, which is also a reflexive marker (cf. \sectref{sec:kahigi:2.7}). Due to its pre-verb root position, the reciprocal} {\textit{{}-i-}} {cannot participate in suffix ordering. We have to keep this in mind as we present the suffix ordering facts as they pertain to Sumbwa.}

{The first attempt to state Sumbwa suffix ordering constraints was in \citet[71]{Kahigi2008b}. Below I present a modified statement of these constraints:}

{1. An affix type cannot be repeated in the same verb stem (as \tabref{tab:kahigi:2} shows, there is repetition of} {\textit{{}-an-} }{in Chichewa} {\textit{{}-mang-its-an-ir-an-}}{; this does not occur in Sumbwa). The only exception observed has to do with one verb,} {\textit{{}-zi}}{{}- ‘go’, and the applicative extension, as shown below:}

\ea\label{ex:kahigi:46}
    \ea\label{ex:kahigi:46a} \gll {\textit{a-la-zi-a}} \\
     \textsc{sm1-pst-}go\textsc{-fv}\\
        \glt ‘he went’

    \ex\label{ex:kahigi:46b} \gll \textit{a-la-zi-il-a}               \textit{si-ntu}\\
 \textsc{sm1-pst-}go\textsc{-appl-fv}  7-thing\\
 \glt ‘he went for sth’

    \ex\label{ex:kahigi:46c} \gll \textit{a-la-mu-zi-il-il-a}                   \textit{si-ntu}\\
 \textsc{sm1-pst-om1-}go-\textsc{appl-appl-fv}  7-thing\\
 \glt {‘he went for sth for him’}
    \z
\z

{2. The maximum number of affixes that can co-occur in a verb stem is four. Example:}

\ea\label{ex:kahigi:47}   \textsc{rev + fre + pers + pass} \\
\gll dod-ool-agul-iliz-ibhw-a > [dodoólagulizibwa]\\
 sew-\textsc{rev-fre-pers-pass-fv} \\
\glt {‘be caused to sew quickly for pay’}
\z

{3. The Passive may follow the Bare Verb Root, applicative, instrumental, persistive, frequentative and causative.}

\ea\label{ex:kahigi:48}
    \ea {\textit{{}-kat-u-a      → [katwa]}} {‘be cut’              \hfill\textsc{vr+pass}}

    \ex { \textit{-kat-il-u-a    → [katilwa]}} {‘be cut for’          \hfill\textsc{appl+pass}}

    \ex {\textit{{}-kat-iisi-iβu-a  → [katiisiβwa]}} {‘be cut with’        \hfill\textsc{inst+pass}}

    \ex {\textit{{}-vig-ilizi-iβu-a  → [vigiliziβwa]}} {‘be squeezed tightly’  \hfill\textsc{pers+pass}}

    \ex {\textit{{}-kat-agul-u-a  → [katagulwa]}} {‘be cut repeatedly’    \hfill\textsc{fre+pass}}

    \ex -{\textit{kat-iisi-iβu-a  → [katiisiβwa]}} {‘be caused to cut’    \hfill\textsc{caus+pass}}
    \z
\z

{4. The associative/reciprocal} {\textit{{}-i-}} {may occur with the following: Bare Base, \textsc{appl, fre, pers}. It never occurs with \textsc{pass} and \textsc{st/neu}. As noted earlier (cf. \sectref{sec:kahigi:2.7}), this associative marker is also the reflexive marker; hence, all such constructions are ambiguous.}

\newpage
\ea\label{ex:kahigi:49}
    \ea {\textit{{}-i-kat-a      →  [ikata]}} {‘cut each other’/’cut oneself’}

    \ex {\textit{{}-i-kat-il-a    →  [ikatila]}} {‘cut for each other’/’cut for oneself’}

    \ex {\textit{{}-i-kat-agul-a  →  [ikatagula]}} {‘cut each other repeatedly’/’cut oneself...’}

    \ex {\textit{{}-i-kwat-ilil-a  →  [ikwaatilila]}} {‘hold each other tightly’/’hold oneself ...’}
    \z
\z

{The associative} {\textit{{}-an-}} {is predominantly restricted to -CVC- verb roots. In the data we have, there are only a few examples that show co-occurrence with other extensions:}

\ea\label{ex:kahigi:50}
    \ea\label{ex:kahigi:50a} \gll \textit{{}-lek-an-iisi-iβu-a}  → [lekaniisiβwa]\\
 leave-\textsc{recp-caus-pass-fv}\\
 \glt ‘be separated from each other’

 \ex\label{ex:kahigi:50b} \gll \textit{{}-βi-h-il-an-a}    → [βiihilana]\\
  bad-\textsc{dec-appl-recp-fv}\\
 \glt ‘be bad for each other’ {(i.e. ‘be angry with each other’)}

    \ex\label{ex:kahigi:50c} \gll \textit{{}-li-iisi-an-a}    → [liisjana]\\
        eat-\textsc{caus-recp-fv}\\
 \glt {‘feed each other’}
    \z
\z

{The meaning in \REF{ex:kahigi:50b} suggests that the example} {\textit{{}-βiihilana}} {(‘be angry with each other’) appears to be lexicalized.}

{5. In all co-occurrence cases, the Passive occurs last before the final element, FV.}

{The statements in 1 -- 5 may be summarized as follows:}

\begin{enumerate}
\item \textsc{appl + appl}


\item \textsc{rev + fre + pers + pass}

\item  \begin{enumerate}
    \item  BB + \textsc{pass}


  \item \textsc{appl + pass}


  \item \textsc{inst + pass}


  \item \textsc{fre + pass}


  \item \textsc{caus + pass}
    \end{enumerate}
\item Constructions with {\textit{{}-i-}} {(\textsc{recp}) may allow \textsc{appl, fre, pers}, but not \textsc{pass}, ST/NEU. On the other hand, constructions with} {\textit{{}-an-}} may allow the following orderings:

    \begin{enumerate}
    \item \textsc{recp + caus + pass}


    \item \textsc{den + appl + recp}


    \item \textsc{caus + recp}
    \end{enumerate}
\end{enumerate}

Considerations in this section lead to the following conclusions:

\begin{enumerate}
\item {Sumbwa does not have a single example illustrating a complete CARP ordering.}

\item {The only examples that could be taken to partially follow the CARP template are:} {\textit{{}-βi-h-il-an-a}} {(APPL + RECP),} {\textit{{}-li-iisi-an-a}} {(CAUS + RECP) and PASS, which occurs last in the ordering.}

\item The remaining examples do not fit in the CARP template.
\end{enumerate}


\section{Sumbwa verb extensions and parameters of Bantu morphosyntactic variation}\label{sec:kahigi:4}

{Having presented the verb extensions in Sumbwa in \sectref{sec:kahigi:2} and the co-occurrence constraints in \sectref{sec:kahigi:3}, we are now in a position to deal with the parameters of Bantu morphosyntactic variation as presented in \citet{GuéroisEtAl2017}. As pointed out in the introduction (\sectref{sec:kahigi:1}), the relevant parameters are in \sectref{sec:kahigi:5} of the master list, i.e. parameters 36-48 which deal with verbal derivation. The parameters have to do with the canonical passive, the ‘impersonal’ passive, agent noun phrase, bare agent, reciprocal, other functions of} {\textit{{}-an-}}{, causative, instrumental causative, applicative, applicative functions, multiple applicative extensions, neuter/stative, and the order of suffixes. The objective of the exercise is to provide data that may be used in identifying micro-variation among Bantu languages with respect to the proposed parameters. Some of the questions have already been answered in \sectref{sec:kahigi:2}. In this section, we summarize the relevant points and provide further discussion of any points not covered in previous sections.}

\subsection{Canonical passive (Parameter 36)}\label{sec:kahigi:4.1}

{A canonical passive is taken to be a normal passive which is a “construction by which the subject of an active clause is demoted to an oblique or remains unexpressed, while the object is promoted to subject status” (cf. \citealt[2]{KulaMarten2010}). It is thus a result of classical passivization, which involves a transitive verb, and which can be expressed in a  rule format as NP}{\textsubscript{1} }{+ V + NP}{\textsubscript{2}} {→ NP}{\textsubscript{2} }{+ V}{\textit{{}-w-a}} ({\textit{na} }{+ NP}{\textsubscript{1}}{), describing the canonical Swahili passive,  where} {\textit{{}-w-}} {represents the passive extension (with its allomorphs),} {\textit{{}-a}} {the FV (with its allomorphs), and} {\textit{na}} {the preposition that is the head of the optional oblique NP.}

{In Sumbwa, as in most Bantu languages, the canonical passive is expressed through a verbal extension. As shown in \sectref{sec:kahigi:2}, passives in Sumbwa are marked by} {\textit{{}-u-}} {(occurring after consonantal-final verb roots) and} {\textit{{}-iβu-}} {(occurring after vowel-final verb-roots). Examples shown in \sectref{sec:kahigi:2.2} summarize the facts on Sumbwa passivization.}

\subsection{``Impersonal'' passives (Parameter 37)}\label{sec:kahigi:4.2}

{The so-called impersonal passives are non-canonical. A case in point is the} {\textit{ba}}{{}-passive construction in Bemba, a language of Zambia, where ``\ldots\, the active clause subject, as in typical passives, is demoted to an oblique position introduced by a preposition or remains unexpressed. The preferred preposition to introduce agents is} {\textit{ku-}}{/}{\textit{kuli-} }{`by', while} {\textit{na} }{`by/with' is more frequent with instruments'' (\citealt[118]{KulaMarten2010}). An example of the}{ \textit{ba}}{{}-passive is given in \REF{ex:kahigi:51} below, where  \REF{ex:kahigi:51a}  is active while \REF{ex:kahigi:51b} is passive:}

\ea\label{ex:kahigi:51} \citet[119]{KulaMarten2010}\\
    \ea\label{ex:kahigi:51a} \gll \textit{umw-áàna}      \textit{bá-alí-mu-ít-a}              \textit{ku}  \textit{mu-mbúlu}\\
    1-child        \textsc{sm2-past-om1-}call-\textsc{fv}   by  3-wild.dog\\
    \glt ‘The child was called by the wild dog.’

    \ex\label{ex:kahigi:51b} \gll \textit{bá-alí-it-a}             \textit{umw-áàna}      \textit{ku}    \textit{mu-mbúlu}\\
    \textsc{sm2-past-}call-\textsc{fv}  1-child        by    3-wild.dog\\
    \glt ‘The child was called by the wild dog.’
    \z
\z



{A characteristic of the passive in \REF{ex:kahigi:51b} is that “the theme argument is not clearly promoted to subject position: It remains in situ in post-verbal position” (\citealt[119]{KulaMarten2010}).} This construction does not occur in Sumbwa.

\subsection{Agent noun phrase (Parameter 38)}\label{sec:kahigi:4.3}

{The agent noun phrase in Sumbwa is introduced by the preposition} {\textit{ne}}.

\ea\label{ex:kahig:52}
    \ea\label{ex:kahigi:52a} \gll \textit{mu-ana}   \textit{a-la-tem-a}         \textit{mu-ti}\\
    1-child   \textsc{sm1-pst-}cut\textsc{-fv}  3-tree\\
    \glt ‘the child cut a tree’

    \ex\label{ex:kahigi:52b} \gll \textit{mu-ti}     \textit{gu-la-tem-w-a}         \textit{ne}   \textit{mu-ana}\\
    3-tree   \textsc{sm1-pst-}cut\textsc{-pass-fv}  by   1-child \\
    \glt ‘a tree was cut by the child’
    \z
\z

{It is important to note that the agent noun phrase may be dropped if the focus is on the patient that is the new subject:}

\ea\label{ex:kahigi:53} \gll \textit{mu-ti}     \textsc{gu-la-tem-w-a}\\
 3-tree   \textsc{sm1-pst-}cut\textsc{-pass-fv} \\
 \glt {‘a tree was cut’}
\z

{There are also other constructions where the agent noun phrase is not needed, as noted in \sectref{sec:kahigi:2.2}.}

\subsection{Bare agent (Parameter 39)}\label{sec:kahigi:4.4}

{Can the preposition} {\textit{ne} }{be omitted and the passive construction remain grammatical? In Sumbwa, such omission will always result in ungrammatical sentences, and is not allowed, as the following examples shows:}

\ea\label{ex:kahigi:54}
\ea[]{ \gll \textit{mu-ti}   \textit{gu-la-tem-w-a}         \textit{ne}   \textit{mu-ana}\\
 3-tree   \textsc{sm1-pst-}cut-\textsc{pass-fv}   by   1-child\\
\glt ‘a tree was cut by the child’}\label{ex:kahigi:54a}

 \ex[*]{\textit{mu-ti}     \textit{gu-la-tem-w-a}         \textit{mu-ana}\\
 3-tree     \textsc{sm1-pst-}cut-\textsc{pass-fv}   1-child\\
    \glt ‘a tree was cut by the child’}\label{ex:kahigi:54b}
    \z
\z

{As can be noted here, \REF{ex:kahigi:54a} with} {\textit{ne}} {is grammatical, while \REF{ex:kahigi:54b} without is not.}

\subsection{Reciprocal (Parameter 40)}\label{sec:kahigi:4.5}

{As shown in \sectref{sec:kahigi:2.7}, there are two reciprocal/associative markers in Sumbwa,} {\textit{{}-i-}} {and} {\textit{{}-an}}{{}-, the former occurring in pre-verbal position and the latter in post-verbal position. It should be noted that the marker} {\textit{{}-i-}} {is the more frequent one. There are a few examples which use the extension} {\textit{{}-aan-} }{instead of} {\textit{{}-an-}}{. As already noted,} {\textit{{}-i-}} {is also a reflexive marker, making it ambiguous. Examples are:}

\ea\label{ex:kahigi:55}
    \ea[]{ \gll \textit{βa-la-li-il-an-a} \\
 \textsc{sm2-pst-}eat-\textsc{appl-rec-fv}\\
 \glt {‘they ate at each other’s home’}}\label{ex:kahigi:55a}

 \ex[*]{\gll \textit{βa-la-i-li-a}   \textit{[βalíílja]}\\
 \textsc{sm2-pst-past-rec}-eat-\textsc{fv}\\
 \glt {‘they ate each other’}}\label{ex:kahigi:55b}
    \z
\z

As can be noted here, which of the two reciprocal/associative marker is used is not a free choice. The choice depends on several factors, some of which are:

\begin{enumerate}
\item {The meaning of the verb involved; if the meaning is incongruous as in \REF{ex:kahigi:55b} above the polysemous \-}{\textit{{}-i-} }{is avoided}

\item {\textit{{}-an-} }{is restricted to shorter verb roots, usually -CVC-, and occasionally -CVCVC-}
\end{enumerate}


\subsection{Other functions of the associative (Parameter 41)}\label{sec:kahigi:4.6}

{In addition to the reciprocal function of the associative} {\textit{{}-i-}} {and} {\textit{{}-an-}}{, there are some examples which indicate the comitative function, as follows:}

\ea\label{ex:kahigi:56}
    \ea\label{ex:kahigi:56a} \gll \textit{βa-la-gaβ-aan-a}\\
 \textsc{sm2-pst-}divide-\textsc{ass-fv}\\
 \glt {‘they shared’}

 \ex\label{ex:kahigi:56b} \gll \textit{βa-la-lek-aan-a}\\
 \textsc{sm2-pst-}leave-\textsc{ass-fv}\\
 \glt {‘they separated’}
    \z
\z

{There is also one example which does not indicate either reciprocal or comitative function:}

\ea\label{ex:kahigi:57} \gll \textit{a-la-zí-an-a}              \textit{i-kóóti}\\
 \textsc{sm2-pst-}go-\textsc{ass-fv}        {5-coat}\\
 \glt {‘s/he took a coat with her/him’ (literally: ‘s/he went with a coat’)}
\z

\subsection{Causative (Parameter 42)}\label{sec:kahigi:4.7}

{The causative extensions are} {\textit{{}-i-} }{and}{ \textit{-iisi-}} {(cf. \sectref{sec:kahigi:2.3}). Briefly, these forms are distributed as follows:} {\textit{{}-i-}} {occurs in verb roots with final consonants. It is accompanied by spirantization of /p, b, t, d, l, k, g/ into [f, v, s, z, z, s, z], respectively.} {\textit{{}-iisi-}} {occurs in verb roots with final vowels or consonants. In the examples in \tabref{tabex:kahigi:58},} {\textit{a}}{{}- is the class 1 subject marker, and -}{\textit{la}}{{}- is the past tense marker.}

\begin{table}
\begin{tabularx}{\textwidth}{l>{\raggedright\arraybackslash}p{.16\textwidth}QQ}

\lsptoprule
 & {{{{Verb} {root}}}} & {{{{Causative} {-\textit{i}-}}}} & {{{{Causative} {-\textit{iisi}-}}}}\\
 \midrule
  {a.} & {\textit{{}-βáámb}}{{}-}

{  {‘peg out’}} & { {\textit{a-la-βáámb-i-a → [álaβáámvja]}} }

‘he caused (sth) to be pegged out’ & { {\textit{a-la-βáámb-iisi-a]}} {\textit{→[alaβáámbíisja]}}}

‘he caused (sth) to be pegged out’\\
  b.&   -{\textit{dod-} }

{  {‘sew’}} & { {\textit{a-la-dod-i-a}} {→} {\textit{[aladozja]}}}

‘he caused (sth) to be sewn’ & { {\textit{a-la-dod-iisi-a → [aladodeesja]}}}

{ {‘he caused (sth) to be sewn’}}\\
 c.&   -{\textit{og-} }

{  {‘take a bath’}} & { {\textit{a-la-og-i-a  →  [aloozja]}}}

‘he bathed (sb)’ & { {\textit{a-la-og-iisi-a  → [aloogeesja]}}}

‘he bathed (sb)’\\
\lspbottomrule
\end{tabularx}
\caption{Examples of the causative -\textit{i}- and -\textit{iisi}-}
\label{tabex:kahigi:58}
\end{table}

As can be seen in \tabref{tabex:kahigi:58}, the causative \textit{{}-i-} in a--c involves two rules: spirantization (/b/ → [v], /d, g/ → [z] and gliding /i/→[j].

\subsection{Instrumental causative (Parameter 43)}\label{sec:kahigi:4.8}

{As already noted in \sectref{sec:kahigi:2.4}, the extensions} {\textit{{}-i-} }{and}{ \textit{-iisi-}} {are used for the causative and instrumental. Consider the following example, which was given in \sectref{sec:kahigi:2.4} and is repeated here as \REF{ex:kahigi:59}.}

\ea\label{ex:kahigi:59}
    \ea\label{ex:kahigi:59a} \textit{[alamulíísja mwaana]}\\
 \gll \textit{a-la-mu-li-iisi-a}                \textit{mu-ana}\\
 \textsc{sm1-pst-om1-}eat\textsc{caus-fv}  1-child\\
 \glt {‘She caused the child to eat.’ (i.e. she fed the child)}

    \ex\label{ex:kahigi:59b} \textit{[alalíísja siliko]}\\
 \gll \textit{a-la-li-iisi-a}                \textit{si-liko}\\
 \textsc{sm1-pst-}eat-\textsc{caus-fv}   7-spoon\\
\glt {‘She ate with a spoon.’}
    \z
\z

{The first sentence is a causative construction while the second is an instrumental.}

\subsection{Applicative (Parameter 44)}\label{sec:kahigi:4.9}

{Applicative constructions are formed by using the extension} {\textit{{}-il-}}{, as was noted in \sectref{sec:kahigi:2.1}.}

\subsection{Applicative functions (Parameter 45)}\label{sec:kahigi:4.10}

{Of the five functions noted by \citet[218--221]{Ashton1947} for the Swahili applicative, at least four may be recognized for Sumbwa, as exemplified in the examples given in \sectref{sec:kahigi:2.1}, some of which are repeated in \tabref{tabex:kahigi:60} for convenience.}

\begin{table}
\begin{tabularx}{\textwidth}{lllQ}
\lsptoprule
 & Verb root & Applicative & Example\\
 \midrule
  {a.}& {\textit{{}-tem-a} }{‘cut’} & {\itshape {}-tem-il-a [temela]} & {\gll \textit{a-la-mu-tem-il-a} \textit{muti} \\
    \textsc{sm1-pst-om1-}cut\textsc{-appl-fv} tree\\
    \glt ‘he cut a tree for her’\\
    (Benefactive)}\\
  {b.}& {\textit{{}-iluk-a} }{‘run’} & {\itshape {}-iluk-il-a} & {\gll \textit{a-la-iluk-il-a} \textit{mu-numba}\\
    \textsc{sm1-pst-}run-\textsc{appl-fv}  17-house\\
    \glt {‘he ran into the house’}\\
    (Directional)}\\
  {c.} &{\textit{{}-húúl-a} }{‘whip’} & {\itshape {}-húúl-il-a} & {\gll  \textit{a-ø-mu-húúl-il-a}   \textit{βuzoβe}\\
    \textsc{sm1-hab-om1-}whip-for-\textsc{fv} laziness\\
    \glt ‘he whips her for laziness’\\
    (Reason)}\\
  {d.} &{\textit{{}-dod-a} }{‘sew’} & { {\textit{{}-dod-il-a [dodela]}}} & {\gll \textit{a-Ø–dod-il-a} \textit{kaaya}\\
    \textsc{sm1-hab-}sew-\textsc{appl-fv} home\\
    \glt {‘he sews at home’}\\
    (Location)}\\
\lspbottomrule
\end{tabularx}
\caption{Examples of functions of the applicative}
\label{tabex:kahigi:60}
\end{table}

{The ‘reason’ meaning expressed in c in \tabref{tabex:kahigi:60} is also found in found in `why' questions such as} {\textit{a-Ø-mu-húúl-il-a si?} }{‘Why does he whip her?’. The location meaning in d agrees with the corresponding question} {\textit{a-Ø-dod-il-a hi?} }{‘Where does he sew?’.}

{This multiplicity of functions of the applicative extension, recognized quite early by Bantuists (cf. \cites[xii]{Madan1903}[218--221]{Ashton1947}), is true of many Eastern Bantu languages.}

\subsection{Multiple applicative extensions (Parameter 46)}\label{sec:kahigi:4.11}

{In Sumbwa, as in other Bantu languages, there is what appears to be a ``multiple applicative'' extension due to the fact that it is a reduplication of the usual applicative extension,} {\textit{{}-il-.}} {In this study, following \citet[144]{Guthrie1971}, we have called it the persistive}{ \textit{-ilil-}} {(cf. \sectref{sec:kahigi:2.9.1}). In the Swahili-English Dictionary of 1939, Johnson called it ``double prepositional''.}

{Its function is to express intensity, repetition or completeness. It does not allow addition of an argument other than the one licensed by the verb root. This is shown in the following examples:}

\ea\label{ex:kahigi:61}
    \ea\label{ex:kahigi:61a} \gll \textit{a-la-kwáát-a}      \textit{si-ntu}\\
    \textsc{sm1-pst-}hold-\textsc{fv}  7-thing\\
    \glt {‘he held a thing’}

    \ex\label{ex:kahigi:61b} \gll \textit{a-la-mu-kwáát-il-a}            \textit{si-ntu}\\
 \textsc{sm1-pst-om1-}hold-\textsc{appl-fv}   7-thing\\
    \glt {‘he held a thing for him’}

    \ex\label{ex:kahigi:61c} \gll \textit{a-la-kwáát-ilil-a}           \textit{si-ntu}\\
 \textsc{sm1-pst-}hold-\textsc{pers-fv}    7-thing\\
 \glt {‘he held the thing tightly’}
    \z
\z

{In \REF{ex:kahigi:61a}, the verb root -}{\textit{kwáát-}} {allows an argument} {\textit{sintu} }{(thing). In \REF{ex:kahigi:61b}, the applicative extension} {\textit{{}-il-}} {allows an extra argument, marked as} {\textit{{}-mu-} }{i.e. the object marker, while in \REF{ex:kahigi:61c} the persistive} {\textit{{}-ilil-}} {does not allow any extra argument other than the one allowed by the verb root -}{\textit{kwáát-}}{. So, in general, the persistive does not allow addition of an argument.}

{In Sumbwa, however, we find one exception, which was given in \REF{ex:kahigi:46} and is repeated as \REF{ex:kahigi:62} below:}

\ea\label{ex:kahigi:62}
    \ea\label{ex:kahigi:62a} \gll \textit{a-la-zi-a}\\
           \textsc{sm1-pst-}go-\textsc{fv}\\
        \glt ‘he went’

    \ex\label{ex:kahigi:62b} \gll \textit{a-la-zi-il-a}               \textit{si-ntu}\\
            \textsc{sm1-pst-}go-\textsc{appl-fv}   7-thing\\
    \glt {‘he went for sth’}

    \ex\label{ex:kahigi:62c} \gll \textit{a-la-mu-zi-il-il-a}                   \textit{si-ntu}\\
            \textsc{sm1-pst-om1-}go\textsc{-appl-appl-fv}   7-thing\\
    \glt {‘he went for sth for him’}
    \z
\z

{In \REF{ex:kahigi:62a}, the verb root} {\textit{{}-zi-}} {‘go’ does not allow an extra argument because it is intransitive. But the example in \REF{ex:kahigi:62b}, which is applicative, allows addition of one argument. In \REF{ex:kahigi:62c},} {there is an addition of the applicative extension which is accompanied by the addition of an argument,} {\textit{{}-mu-}}{, as beneficiary.}

\subsection{Neuter/stative (Parameter 47)}\label{sec:kahigi:4.12}

{Bantuists have attributed two functions to the stative extension: to express state without implying agency, and to express ``potentiality'' (cf. \citealt[227--228]{Ashton1947}).}


In this study, we have covered the following extensions which express stative meanings in Sumbwa:

\begin{enumerate}
\item {The usual stative marker} {\textit{{}-ik-}} {(cf. \sectref{sec:kahigi:2.6})}

\item {The reversive stative} {\textit{{}-uk-/-uuk-/-uuluk-}} {(cf. \sectref{sec:kahigi:2.8})}

\item {The frequentative stative} {\textit{{}-aguk-} }{(cf. \sectref{sec:kahigi:2.10}).}
\end{enumerate}


{Since these extensions have been dealt with at length in the foregoing, reference should be made to the respective sections.}

\subsection{Order of suffixes (Parameter 48)}\label{sec:kahigi:4.13}

{Co-occurrence constraints have been stated in \sectref{sec:kahigi:3}. Here we shall be brief. Is there a specific order for Sumbwa productive verbal extensions? Does Sumbwa have the causative-applicative-reciprocal-passive (CARP) order postulated by \citet{Hyman2002} for Bantu?}

{It is important to note, first, that, of the four extensions involved, i.e. causative, applicative, reciprocal, and passive, only three (causative, applicative and passive) enjoy very high productivity. The fourth, the associative/reciprocal, is expressed by two separate forms,} {\textit{{}-i-}} {and} {\textit{{}-an-}}{, of which the former is a pre-verb-root affix (and highly productive) and the latter is an unproductive extension. Given the fact that} {\textit{{}-an-}} {is restricted mostly to -CVC- verb roots, and is currently unproductive, it becomes evident that} {\textit{{}-an-}} {cannot have the freedom to combine freely with the other extensions.}

{Now, what are the orders that are allowed? These orders were spelt out in \sectref{sec:kahigi:3} above, but for convenience we present a few grammatical and ungrammatical examples here to show the orders allowed and not allowed:}

\ea\label{ex:kahigi:63}
    \ea[]{   \gll -\textit{li-il-an-a}  \\
    eat-\textsc{appl-rec-fv}\\
    \glt ‘eat at each other’s home’ \hfill \textsc{appl-rec}}\label{ex:kahigi:63a}

    \ex[]{   \gll -\textit{li-iisi-an-a} \\
    eat-\textsc{caus-rec-fv}\\
    \glt {‘feed each other’} \hfill \textsc{caus-rec}}\label{ex:kahigi:63b}

    \ex[*]{ \gll \textit{-li-iisi-il-an-iβu-a}  \\
    eat-\textsc{caus-appl-rec-pass-fv}\\
    \glt {‘be caused to feed each other’} \hfill \textsc{caus-appl-rec-pass (CARP)}}\label{ex:kahigi:63c}

    \ex[]{\gll \textit{{}-lek-an-iisi-iβu-a} \\
    leave-\textsc{rec-caus-pass-fv}\\
    \glt {‘be caused to leave each other’} \hfill \textsc{rec-caus-pass}}\label{ex:kahigi:63d}

    \ex[]{\gll \textit{{}-dod-ool-agul-iliz-ibhu-a} \\
    sew-\textsc{rev-fre-appl-caus-pass-fv}\\
    \glt {‘be caused to sew clumsily and quickly’} \hfill \textsc{rev-fre-pers-pass}}\label{ex:kahigi:63e}

    \ex[*]{\gll \textit{-dod-iisi-il-an-ibhu-a} \\
    sew-\textsc{caus-appl-rec-pass-fv}\\
    \glt {‘be caused to sew each other’} \hfill \textsc{caus-appl-rec-pass (CARP)}}\label{ex:kahigi:63f}
    \z
\z

{The orders that are allowed are those in (\ref{ex:kahigi:63a}, \ref{ex:kahigi:63b}, \ref{ex:kahigi:63d}, \ref{ex:kahigi:63e}). The orders in \REF{ex:kahigi:63c} and \REF{ex:kahigi:63f}, based on the CARP hypothesis, are ungrammatical. Whether the orders above reflect non-templatic constraints as spelt out in \citet{Hyman2002} is an issue for further study.}

\section{Conclusion}\label{sec:kahigi:5}

{This study has revealed the following important facts about verb extensions in Sumbwa: their productivity, co-occurrence constraints, their valence possibilities, and their behavior in relation to the parameters of morphosyntactic variation proposed by \citet{GuéroisEtAl2017}.}

{The study reveals that most of the Proto-Bantu verb extensions reconstructed by Guthrie and Meeussen are still active in the language. The extensions may} {\textit{roughly}} be categorized into three groups: highly productive, moderately productive and least productive. The highly productive extensions are the applicative, passive, causative (also instrumental), stative, and frequentative. The pre-verbal associative {\textit{{}-i-}} is also highly productive, while the status of the associative {\textit{{}-an-} needs further investigation. The moderately productive extensions are the persistive, reversive, impositive and denominative. The least productive are reiterative, static, contactive and other minor extensions.} {Overall, the productivity ranking is similar to that in other Bantu languages (cf. \citealt{MagangaSchadeberg1992, Rugemalira1993runyambo,Rugemalira2005, Schadeberg2003, Stegen2002, Waweru2005, Chabata2007}).}

The study also shows that Sumbwa verb extensions may be categorized as either valence-increasing, valence-decreasing or valence-maintaining, as in \tabref{tab:kahigi:3}.

\begin{table}
\begin{tabularx}{\textwidth}{QQQ}

\lsptoprule

Valence-increasing & {Valence-decreasing} & {Valence-maintaining}\\
\midrule
Applicative & Passive & Reversive active\\
{{Causative}} & Stative/Neuter & Persistive \\
Instrumental & Associative & Reiterative\\
Impositive & {{Frequentative Stative}} & Frequentative active\\
& Reversive Stative & \\
\lspbottomrule
\end{tabularx}
 \caption{Extensions categorized in terms of valence}
\label{tab:kahigi:3}
\end{table}

{The categorization in \tabref{tab:kahigi:3} generally reflects the valence possibilities for most Bantu languages.}

{Furthermore, in the answer to the questions in the Master List of the Parameters of Morphosyntactic Variation, the study has revealed some interesting facts that may be useful in Bantu comparative morphosyntax. These are summarized in \tabref{tab:kahigi:4}.}

\begin{table}
\begin{tabularx}{\textwidth}{l>{\raggedright\arraybackslash}p{.2\textwidth}Q}

\lsptoprule

{No.} & {Topic} & Important characteristics\\\midrule
{ {36}} & Canonical Passive & { {1. It is expressed through a verbal extension -}{\textit{u-/-iβu-}}}

2. There is no other strategy to express passivization.\\
37 & ``Impersonal'' passive & { {There are no} {\textit{ba-}}{passives} }\\
38 & Agent Noun Phrase & { {The Agent NP in a passive construction is introduced by the preposition} {\textit{ne}}}\\
39 & Bare agent & { {The preposition which introduces the agent cannot be omitted.}}\\
40 & Reciprocal & { {Through the use of the pre-verbal} {\textit{{}-i-}} {(which is also the reflexive marker) and the suffix} {\textit{{}-an-}}}\\
41 & { {Other functions for} {\textit{{}-an-}}} & { {Yes, it has the comitative function}}\\
42 & Causative & { {It is expressed through suffixes} {\textit{{}-i-}} {and} {\textit{{}-iisi-}}}\\
43 & Instrumental Causative & Yes, the causative extension is also used to introduce prototypical instruments\\
44 & Applicative & { {Applicative constructions are formed through the use of the suffix} {\textit{{}-il-}}}\\
45 & Applicative functions & { {In addition to benefactive meaning, applicative constructions convey the following meanings: directional, location, reason.}}\\
46 & Multiple applicative extensions & { {1. What appears to be a case of multiple applicative extension (i.e.} {\textit{{}-ilil-}}{), is in fact a persistive extension.}}

{ {2. The only possible case of multiple applicative extension is}{ \textit{-zi-il-il-a} }{‘go for (sth) for (sb)}}\\
47 & Neuter/Stative & { {In addition to the stative} {\textit{{}-ik-,} }{the language has the reversive stative and the frequentative stative.}}\\
48 & Order of suffixes & { {1. CARP is not possible in the language}}

{ {2. There is no systematic fixed order}}

{ {3. The Passive always occurs last.}}\\
\lspbottomrule
\end{tabularx}
\caption{Characteristics of the Parameters manifested in Sumbwa}
\label{tab:kahigi:4}
\end{table}

{Most of the characteristics listed in \tabref{tab:kahigi:4} are found in the majority of Bantu languages. There are, however, characteristics that are peculiar to Sumbwa (and other languages similar to it). These include:}

\begin{enumerate}
\item {The reflexive-reciprocal syncretism marked by the pre-verb root affix} {\textit{{}-i-.}} {Reciprocity is expressed in a productive way by} {\textit{{}-i-}}{, while the Proto-Bantu reciprocal extension} {\textit{{}-an-}} {occurs only in restricted contexts. This characteristic is not restricted to Sumbwa; it has been reported in Rimi (F.32; \citealt{Olson1964}) and Rangi (F.33; \citealt{Stegen2002}). Other zone F languages should be investigated in connection with this feature to find out whether it is a characteristic for the zone.}

\item  {The causative-instrument syncretism marked by the causative extensions} {\textit{{}-i-}} {and} {\textit{{}-iisi-}}{. This syncretism has been discussed in the literature (cf. \citealt{Wald1998}). In this case, Sumbwa belongs to Bantu languages which no longer uses} {\textit{{}-il-}} {to mark the instrumental role.}

\item  {Sumbwa data do not support the CARP template as formulated in \citet{Hyman2002}. This is probably because the productive affix for reciprocity\slash associativeness is no longer} {\textit{{}-an-}} {but} {\textit{{}-i-}} {which occurs in pre-verb-root position and is not a suffix.}
\end{enumerate}


{As a final remark, we need to stress the limited nature of this study and that all the above issues (and others listed in the above table) are of interest to Bantu comparative linguistics and require further in-depth investigation.}

\section*{Acknowledgements}

I am grateful to the editors and anonymous reviewers for comments which have improved parts of this chapter.

\section*{Abbreviations}

\begin{tabularx}{.45\textwidth}{lQ}
\textsc{act} & active\\
AM & agreement marker\\
\textsc{appl} & applicative\\
\textsc{ass} & associative\\
BB & bare base\\
\textsc{caus} & causative\\
\textsc{cli} & clitic\\
\textsc{den} & denominative\\
\textsc{fre} & frequentative\\
\textsc{fre-act} & frequentative-active\\
\textsc{fre-st} & frequentative stative\\
FV &final vowel\\
\textsc{hab} & habitual\\
\textsc{imp} & impositive\\
\textsc{inst} & instrumental\\
\end{tabularx}
\begin{tabularx}{.45\textwidth}{lQ}
\textsc{intr} & intransitive\\
OM & object marker\\
\textsc{pass} & passive\\
\textsc{pers} & persistive\\
\textsc{pst} & past\\
\textsc{recp} & reciprocal\\
\textsc{reit} & reiterative\\
\textsc{rev} & reversive\\
SM & subject marker\\
\textsc{st} & stative\\
TAM & tense-aspect-modal marker\\
\textsc{tr} & transitive\\
TRV &transitive verb\\
VR & verb root
\end{tabularx}·

\printbibliography[heading=subbibliography,notkeyword=this]
\end{document}
