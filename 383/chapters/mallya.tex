\documentclass[output=paper]{langscibook}
\ChapterDOI{10.5281/zenodo.10663769}
    
\author{Aurelia Mallya\orcid{}\affiliation{University of Dar es Salaam}}

\title{The morphosyntax of locative expressions in Kiwoso}

\abstract{This paper discusses the morphological and syntactic properties of locative expressions in Kiwoso. It provides an account of the locative forms and their properties in relation to nominal and verbal morphology. The findings show that locative nouns in Kiwoso are formed by means of a locative suffix -\textit{(i)n}. It also shows that the traditional Bantu locative class prefixes (\textit{ku}{}-, \textit{pa}{}-, \textit{mu}{}-) are unproductive in Kiwoso. However, the locative class 17 prefix \textit{ku}{}- triggers agreement on all nominal and verbal modifiers, indicating that locative meanings are still part of the noun class system in the language. The data show that Kiwoso exhibits two post-final locative enclitics -- =\textit{ho} and =\textit{u}. Both particles are used to indicate locative objects, albeit with different interpretations. The post-final =\textit{ho} relates directly to the semantics of the locative noun \textit{kundo}, while =\textit{u} corresponds to the interpretation of the locative noun \textit{ando}. This paper contributes to the understanding of locatives within the Bantu language family in general, and offers new insights about locatives in Kiwoso, an area which has not received extensive treatment in the previous literature. 
\keywords{Kiwoso, locative expressions, locative suffixes, nominal domain}
}

\IfFileExists{../localcommands.tex}{
  \addbibresource{../localbibliography.bib}
  \usepackage{langsci-optional}
\usepackage{langsci-gb4e}
\usepackage{langsci-lgr}

\usepackage{listings}
\lstset{basicstyle=\ttfamily,tabsize=2,breaklines=true}

%added by author
% \usepackage{tipa}
\usepackage{multirow}
\graphicspath{{figures/}}
\usepackage{langsci-branding}

  
\newcommand{\sent}{\enumsentence}
\newcommand{\sents}{\eenumsentence}
\let\citeasnoun\citet

\renewcommand{\lsCoverTitleFont}[1]{\sffamily\addfontfeatures{Scale=MatchUppercase}\fontsize{44pt}{16mm}\selectfont #1}
   
  %% hyphenation points for line breaks
%% Normally, automatic hyphenation in LaTeX is very good
%% If a word is mis-hyphenated, add it to this file
%%
%% add information to TeX file before \begin{document} with:
%% %% hyphenation points for line breaks
%% Normally, automatic hyphenation in LaTeX is very good
%% If a word is mis-hyphenated, add it to this file
%%
%% add information to TeX file before \begin{document} with:
%% %% hyphenation points for line breaks
%% Normally, automatic hyphenation in LaTeX is very good
%% If a word is mis-hyphenated, add it to this file
%%
%% add information to TeX file before \begin{document} with:
%% \include{localhyphenation}
\hyphenation{
affri-ca-te
affri-ca-tes
an-no-tated
com-ple-ments
com-po-si-tio-na-li-ty
non-com-po-si-tio-na-li-ty
Gon-zá-lez
out-side
Ri-chárd
se-man-tics
STREU-SLE
Tie-de-mann
}
\hyphenation{
affri-ca-te
affri-ca-tes
an-no-tated
com-ple-ments
com-po-si-tio-na-li-ty
non-com-po-si-tio-na-li-ty
Gon-zá-lez
out-side
Ri-chárd
se-man-tics
STREU-SLE
Tie-de-mann
}
\hyphenation{
affri-ca-te
affri-ca-tes
an-no-tated
com-ple-ments
com-po-si-tio-na-li-ty
non-com-po-si-tio-na-li-ty
Gon-zá-lez
out-side
Ri-chárd
se-man-tics
STREU-SLE
Tie-de-mann
} 
  \togglepaper[1]%%chapternumber
}{}

\begin{document}
\judgewidth{**}
\maketitle 

\section{Introduction}
\label{sec:mallya:1}
Locative constructions have received extensive attention in the previous literature on Bantu languages. Descriptive accounts suggest that locative expressions are marked differently both within and between Bantu languages (\citealt{MartenEtAl2007, PersohnDevos2017}). In a number of Bantu languages, locative expressions are derived by attaching the class 16, 17 and 18 prefixes to a noun (see \citealt{Rugemalira2004, Petzell2008, RiedelMarten2012, Guérois2016, VandeVelde2019}, among others). However, while some languages such as Kagulu \citep{Petzell2008}, Bemba \citep{Marten2012} and Chichewa (\citealt{BresnanKanerva1989}) have maintained all of the three locative prefixes, others like Kivunjo-Chaga \citep{Moshi1995} and Sesotho (\citealt{DemuthMmusi1997}) exhibit only two productive prefixes. Moreover, languages such as Haya and Zulu exhibit only one productive locative prefix (\citealt{RiedelMarten2012}).

In addition to the prefixation strategy, locative nouns in Bantu languages may also be derived by means of suffixation (cf. \citealt{Grégoire1975, Guérois2016}). This strategy is predominantly attested in Eastern and Southern Bantu languages and most of the languages that employ a locative suffix lack locative prefixes. Scholars have also noted that there are Bantu languages that employ both prefixes and suffixes in marking locatives (\citealt{Marten2010, Marten2012}). This is the case in Nguni languages of Southern Africa, for example, in which locative noun class 25 (\textit{e}{}-) and the suffix \textit{{}-(i)ni} are used jointly to derive locative nouns (\citealt{vanderSpuy2014}). It has also been noted that in languages in zone P30 (spoken in Mozambique), the traditionally recognized locative prefixes (those of classes 16-18) can be used in combination with the locative suffix -\textit{ni} to derive locative expressions \citep{Guérois2016}.

The variation in locative constructions has attracted the attention of a wide range of scholars who are interested in investigating the nature of locative expressions in individual Bantu languages, particularly in relation to the domain of morphosyntax. The present paper contributes to the on-going description and discussion of the morphosyntax of locative nouns in Bantu, using data from the Tanzanian Bantu language Kiwoso. The chapter aims to address issues regarding the morphosyntax of Kiwoso locative expressions, with reference to \citegen{GuéroisEtAl2017} parameters. \citet{GuéroisEtAl2017} propose 142 descriptive parameters aimed at examining morphosyntactic variation in Bantu languages. For the purposes of the present study, I have selected four parameters to address key issues pertaining to Bantu locative constructions: i) the forms of locative expressions in Kiwoso, ii) agreement patterns, iii) locative subject and object marking, and iv) the presence or absence of locative postverbal enclitics. 

The rest of this paper is organized as follows: \sectref{sec:mallya:2} provides a brief linguistic profile of Kiwoso, while an overview of the noun class system of the language is presented in \sectref{sec:mallya:3}. Locative nouns, their forms and the associated agreement system are discussed in \sectref{sec:mallya:4}. \sectref{sec:mallya:5} summarizes and concludes the discussion offered in this chapter.

The Kiwoso data presented in this work are based on the intuition of the author as a native speaker, complemented by acceptability judgements provided by two other native speakers of Kiwoso. The primary data are supplemented by secondary data obtained from existing written documents, particularly the dissertations by \citet{Mallya2011, Mallya2016} and \citet{Mushi2005}. Examples from other languages used in this chapter are taken from various sources and are acknowledged accordingly.


\section{Linguistic profile of Kiwoso}
\label{sec:mallya:2}

Kiwoso is an eastern Bantu language spoken predominantly in the Kilimanjaro region of Tanzania. In the survey carried out by the Languages of Tanzania Project, it was reported that Kiwoso is spoken by approximately 81,000 people who are scattered across different districts of the Kilimanjaro region \citep{LanguagesofTanzaniaProject2009}. Native speakers of Kiwoso are mainly found in the administrative areas of Moshi Rural, Hai, Siha, and Moshi Town Districts. \citet{Maho2009} classifies Kiwoso as one of the Zone E languages belonging to the Chagga group (E60) and Kiwoso specifically is coded as E621D \citep{Maho2009}.

Kiwoso is one of a large number of under-studied and under-described languages of Tanzania. The only available literature on Kiwoso is a dictionary (\citealt{KagayaOlomi2009}), two unpublished MA dissertations (\citealt{Mallya2011, Mushi2005}) and a PhD thesis \citep{Mallya2016}. Although the present paper is not intended to provide a full linguistic description of Kiwoso, some background information on the noun class system is presented before embarking on the more specific discussion of the morphology and the syntax of locatives, the primary focus of this paper.

\section{The Kiwoso noun class system}
\label{sec:mallya:3}

Kiwoso displays the typical Bantu noun class system and exhibits 14 noun classes, as illustrated in \tabref{tab:mallya:1}. For each noun class presented in the table, the nominal prefix, an example word, the subject and object agreement morphemes, adjective and possessive prefixes, and the three forms of demonstrative are also shown.

\begin{sidewaystable}
%\footnotesize
\begin{tabular}{lllllllllll}
\lsptoprule
{Class} & {Nominal} {Prefix} & {Example} & {Gloss} & {SM} & {OM} & {Adj} {Prefix} & {Poss} {Prefix} & {Dem1} & {Dem2} & {Dem3}\\
\midrule
1 & {\itshape mu-} & {\itshape mu-na} & child & \textit{a-} & \textit{n-} & \textit{n-} & \textit{o-} & {\itshape e-tu} & {\itshape e-to} & {\itshape u-lya}\\
2 & {\itshape wa-} & \textit{wa-na} & children & \textit{wa-} & \textit{wa-} & \textit{wa-} & {\itshape wa-} & {\itshape e-wa} & {\itshape e-wo} & {\itshape wa-lya}\\
3 & {\itshape n-} & {\itshape n-ji} & tree & \textit{u-} & \textit{i-} & \textit{n-} & \textit{o-} & {\itshape e-tu} & {\itshape e-to} & {\itshape u-lya}\\
4 & {\itshape mi-} & {\itshape mi-ji} & trees & \textit{i-} & \textit{i-} & \textit{mi-} & {\itshape ta-} & {\itshape e-ti} & {\itshape e-to} & {\itshape tya}\\
5 & {\itshape i-} & {\itshape i-du} & ear & \textit{lyi-} & \textit{lyi-} & \textit{lyi-} & {\itshape lya-} & {\itshape e-lyi} & {\itshape e-lyo} & {\itshape lya}\\
6 & {\itshape ma-} & {\itshape ma-du} & ears & \textit{a-} & \textit{wa-} & \textit{ma-} & {\itshape a-} & {\itshape e-wa} & {\itshape e-wo} & {\itshape alya}\\
7 & {\itshape ki-} & {\itshape ki-andu} & knife & \textit{ki-} & \textit{ki-} & \textit{ki-} & {\itshape ki-} & {\itshape e-kyi} & {\itshape e-kyo} & {\itshape kya}\\
8 & {\itshape shi-} & \textit{shi-andu} & knives & \textit{shi-} & \textit{shi-} & \textit{shi-} & {\itshape shi-} & {\itshape e-shi} & {\itshape e-sho} & {\itshape shya}\\
9 & \textit{N-} & {\itshape mburu} & goat & \textit{i-} & \textit{i-} & \textit{ngi-} & {\itshape a-} & {\itshape e-yi} & {\itshape e-yo} & {\itshape iya}\\
10 & \textit{N-} & {\itshape mburu} & goats & \textit{ti-} & \textit{ti-} & \textit{ngi-} & {\itshape ta-} & {\itshape e-ti} & {\itshape e-to} & {\itshape tya}\\
11 & {\itshape u-} & {\itshape u-dende} & leg & \textit{lu-} & \textit{lu-} & \textit{lu-} & {\itshape lo-} & {\itshape e-lu} & {\itshape e-lo} & {\itshape lou}\\
14 & {\itshape u-} & {\itshape u-doko} & laziness & \textit{lu-} & \textit{lu-} & \textit{lu-} & {\itshape lo-} & {\itshape e-lu} & {\itshape e-lo} & {\itshape lou}\\
16 & {\itshape a-} & {\itshape a-ndo} & place & \textit{ku-} & \textit{ku-} & \textit{ku-} & {\itshape ko-} & {\itshape yaa} & \textit{yoo} & {\itshape alya}\\
17 & \textit{ku} & {\itshape ku-ndo} & place & \textit{ku-} & \textit{ku-} & \textit{ku-} & {\itshape ko-} & {\itshape kunu} & \textit{kulya} & {\itshape kulya}\\
\lspbottomrule
\end{tabular}
\caption{\label{tab:mallya:1} The Kiwoso noun class system}
\end{sidewaystable}

Most of the noun classes in classes 1–10 appear in a singular-plural pairing. More specifically, classes 1, 3, 5, 7, 9, and 11 contain singular nouns, while classes 2, 4, 6, 8, and 10 contain their plural counterparts. However, not all classes conform to this pairing system. For example, class 11 nouns form their plural counterparts in class 6, and class 14 nouns lack plural counterparts. The singular-plural pairing system of noun classes found in Kiwoso is illustrated in \figref{fig:mallya:1} below. The class 11/6 plural pairing is exemplified by the examples in \REF{ex:mallya:1}.

\begin{figure}
\begin{tikzpicture}[node distance=5pt and 2cm]
\node(one){1};
\node(two)[right =of one]{2};
\node(three)[below =of one]{3};
\node(four)[right =of three]{4};
\node(five)[below =of three]{5};
\node(six)[right =of five]{6};
\node(seven)[below =of five]{7};
\node(eight)[right =of seven]{8};
\node(nine)[below =of seven]{9};
\node(ten)[right =of nine]{10};
\node(eleven)[below =of nine]{11};
\node(fourteen)[below =of eleven]{14};

\draw (one.east) -- (two.west);
\draw (three.east) -- (four.west);
\draw (five.east) -- (six.west);
\draw (seven.east) -- (eight.west);
\draw (nine.east) -- (ten.west);
\draw (eleven.east) -- (four.west);
\end{tikzpicture}

\caption{\label{fig:mallya:1} Singular/plural noun class pairings in Kiwoso}
\end{figure}

\ea\label{ex:mallya:1}    
  \ea\label{ex:mallya:1a}
  \gll Lelo    ni-a-le-many-a      u-dende  na    kyaara\\
    Lelo    \textsc{init}-2\textsc{sm}-\textsc{pst}-cut-\textsc{fv}      11-leg      by    7.axe\\
  \glt ‘Lelo cut (his) leg by (means of) an axe’

    \ex\label{ex:mallya:1b}
    \gll Lelo    ni-a-le-many-a      ma-dende    na    kyaara\\
         Lelo    \textsc{init}-2\textsc{sm}-\textsc{pst}-cut-\textsc{fv}    6-leg            by    7.axe\\
    \glt ‘Lelo cut (his) legs by (means of) an axe’
    \z
\z

In many Bantu languages, the class 15 prefix {\textit{ku}}{{}- is the prefix for infinitival nouns (\citealt{Katamba2003, VandeVelde2019}). However, Kiwoso differs from the majority of Bantu languages in relation to the infinitive marker. In Kiwoso, infinitives are marked with the class 5 prefix} {\textit{i-}} {which also triggers class 5 subject and object agreement similarly to other class 5 nouns. The} {infinitive morpheme in Kiwoso can be illustrated using infinitives such as} {\textit{ikora}} {‘to cook’,} {\textit{idema}} {‘to cultivate’,} {\textit{isoma}} {‘to read’, and} {\textit{iseka}} {‘to laugh’. Interestingly, the Tanzanian Bantu language Rangi also employs some class 5 infinitives. However, in Rangi, the class 5 infinitive no longer appears to be the productive (nor dominant) noun class for the formation of infinitives. Rather, it is used in addition to the more widespread class 15 infinitive marking \citep{Gibson2012}.}

\begin{sloppypar}
Note also that while many Bantu languages form diminutives by assigning nouns to classes 12 and 13, which are amongst the classes reconstructed for diminutives in Proto-Bantu \citep{Meeussen1967}, diminutives in Kiwoso are expressed by a shift into classes 7/8 and the associated prefixes \textit{ki}{}-/\textit{shi}{}-. For example, \textit{iwee} ‘stone’, \textit{kiwee} ‘small stone’ \textit{shiwee} ‘small stones’ and \textit{uwoko} ‘hand’ \textit{kiwoko} ‘small hand’ \textit{shiwoko} ‘small hands’.
\end{sloppypar}

The following section discusses the locative noun classes 16 (\textit{a}{}-) and 17 (\textit{ku}{}-) which are of particular relevance in this paper.

\section{Locative nouns in Kiwoso}
\label{sec:mallya:4}

\subsection{Unproductive locative prefixes}
\label{sec:mallya:4.1}

In Kiwoso, there are two locative nouns, namely \textit{ando} and \textit{kundo,} both signifying ‘place’. However, the nouns \textit{ando} and \textit{kundo} are pragmatically different in that the former (\textit{ando)} can be interpreted as a place which is definite, specific, known, and near to both the speaker and the hearer, whereas the latter (\textit{kundo)} refers to a place which is indefinite, unspecific, unknown, and far from both the speaker and the hearer.

The two shades of meaning associated with the locative nouns \textit{ando} and \textit{kundo} can be seen in examples \REF{ex:mallya:2} and \REF{ex:mallya:3}, respectively. It is important to note from the outset that the only grammatically active locative classes in the language are class17 marked by \textit{ku}{}- and class 16 marked by \textit{a-}. However, in contrast, locative agreement in Kiwoso is regularly marked with class 17 (see also \sectref{sec:mallya:4.3.1}).

\ea%2
    \label{ex:mallya:2}
    \ea\label{ex:mallya:2a}
    \gll a-lya            a-ndo        ku-cha\\
 16-\textsc{dem}3    16-place      17-nice\\
    \glt {‘There is a nice place’ [definite].} 
    
    \ex\label{ex:mallya:2b} 
    \gll lya   wa-na   wa-le-ch-a      a-le-end-a    a-ndo      ka-woiya-u      sau\\
    \textsc{rel}  2-child 2-\textsc{pst}-come-\textsc{fv}      3\textsc{sg}-\textsc{pst}-go-\textsc{fv}  16-place  \textsc{consc}-keep-there  silent\\
    \glt {‘When the children arrived, s/he went to that place and kept silent’ [definite]}
    \z
\ex\label{ex:mallya:3}
    \ea\label{ex:mallya:3a} 
    \gll ku-lya          ku-ndo    ku-cha\\
      17-\textsc{dem}3  17-place    17-nice\\
    \glt  ‘There is a nice place’ [indefinite].

  \ex\label{ex:mallya:3b}
  \gll wa-ka           wa-le-fik-a   ku-ndo ku-lya       umbe       ti-lekumb-o\\
    2-woman      2-\textsc{pst}-arrive-\textsc{fv} place    17-\textsc{dem}3    10.cow    10-sell-\textsc{pass}\\
    \glt  ‘Women reached the place where cows were sold’. [indefinite]
    \z
\z

Examples \REF{ex:mallya:2} and \REF{ex:mallya:3} suggest that, in common with many other Bantu languages, locative prefixes in Kiwoso can function as noun class markers in the sense that they can be attached to nominal stems yielding locative meanings. Note also that the locative noun class reconstructed as class 18 *\textit{mʊ{}-} in Proto-Bantu and found synchronically as a variant of \textit{mu-} in a number of other Bantu languages, does not exist in Kiwoso.

In addition to specific place names such as \textit{Dar es Salaam}, \textit{Arusha}, \textit{Tanga}, and \textit{Kampala,} there are general place names in Kiwoso such as \textit{kinaange} ‘market’, \textit{mmba} ‘house’, \textit{shuule} ‘school’ \textit{kai} ‘attic’ and \textit{bo} ‘home’. These names are inherently locative in nature and have to be unmarked for locative, as ungrammatical construction in \REF{ex:mallya:4b} illustrates. However, similarly, to derived locative nouns, inherent locative nouns take class 17 agreement, as exemplified in the locative inversion construction in \REF{ex:mallya:4c}, based on the sentence in \REF{ex:mallya:4a}. (See also \sectref{sec:mallya:4.3.1}).

\ea\label{ex:mallya:4}
    \ea[]{\label{ex:mallya:4a}
    \gll wa-ka        wa-le-koon-a      kinaange\\
    2-woman  2\textsc{sm}-\textsc{pst}-meet-\textsc{fv}    market\\
    \glt ‘The women met at the market (place).’}

    \ex[*]{\label{ex:mallya:4b}
    \gll wa-ka        wa-le-koon-a      kinaangen\\
     2-woman  2\textsc{sm}-\textsc{pst}-meet-\textsc{fv}    market\\
    \glt  ‘The women met at the market (place).’}

     \ex[]{\label{ex:mallya:4c}
     \gll kinaange   ku-le-koon-a        wa-ka\\
      market        17-\textsc{pst}-meet-\textsc{fv}      2-woman\\
      \glt ‘The Market is a place where women used to meet.’}
      \z
\z

Locative constructions such as in \REF{ex:mallya:4} are interpreted differently in terms of discourse-pragmatics. In \REF{ex:mallya:4a}, the locative noun \textit{kinaange} ‘market’ serves as a focus while in \REF{ex:mallya:4c} the noun encodes a topic. (See \citealt{MartenGibson2016}, \citealt{MartenvanderWal2014} and \citealt{Mallya2020} for further details on locative inversion constructions). 

\subsection{Locative suffixation}
\label{sec:mallya:4.2}

The present section provides a brief introduction to the locative suffix \textit{{}-(i)ni} in Bantu languages, before discussing the morphology of locative nouns in Kiwoso. As shown in the introduction, apart from the commonly established pattern of locative marking which involves the three locative prefixes from classes 16, 17 and 18, some Bantu languages derive locative nouns by means of the locative suffix \textit{{}-(i)ni} (or variants thereof).

Although the suffix \textit{{}-(i)ni} is widely attested in eastern and southern Bantu languages, there is currently no consensus on its origins. Different scholars have put forth different proposals on the source of this suffix. For example, \citet{Meinhof1942} as cited in \citet[128]{SamsonSchadeberg1994} proposes that the locative suffix is derived from the locative class prefix 18 (\textit{mu}{}-). Meinhof’s proposal is further supported by \citet{Güldemann1999} who argues that the suffix \textit{{}-(i)ni} was originally a marker of inessive relations which later developed into a general locative. However, \citet{SamsonSchadeberg1994} have convincingly demonstrated that the locative suffix is the result of grammaticalization of the word *-\textit{ini} ‘liver’.

Some Bantu languages use double locative marking, combining both prefixation and suffixation. For example, this pattern is found in the P30 languages spoken in Mozambique \citep{Guérois2016} and southern Bantu Nguni languages (\citealt{Fleisch2005cognitive, vanderSpuy2014}). The P30 languages use the prefixes of classes 16, 17 and 18 in addition to the locative suffix -\textit{ni}, whereas the Southern Bantu languages use a combination of the class 25 prefix \textit{e}{}- and the locative suffix (-\textit{i})\textit{ni}. In contrast, locative marking in Kiwoso is solely achieved through suffixation, as will be further shown in the following section.

Locative nouns in Kiwoso are derived by attaching a locative suffix -(\textit{i})\textit{n} to the noun. This contrasts with Bantu languages in which locative expressions are achieved by means of locative prefixes such as Bemba (\citealt{Marten2010,Marten2012}), Kagulu \citep{Petzell2008}, and Chichewa (\citealt{BresnanKanerva1989}). Examples of the use of the locative suffix in Kiwoso are shown in \tabref{tab:mallya:2} below. The data presented in this paper indicate that there is an instance of vowel coalescence in Kiwoso, when the locative suffix (-\textit{i})\textit{n} is attached to nouns that end with the vowel \textit{{}-a.} In such instances, the vowel changes into -\textit{e}, as the examples in \tabref{tab:mallya:2} illustrate. 

\begin{table}
\begin{tabular}{llll}
\lsptoprule
{ordinary nouns} & {gloss} & nouns with -(\textit{i})\textit{n} & {gloss}\\
\midrule
{\itshape ndubhi} & { {‘calabash’}} & {\itshape ndubhin} & { {‘in the calabash’}}\\
{\itshape nlango} & { {‘door’}} & {\itshape nlangon} & { {‘at the door’}}\\
{\itshape nlima} & { {‘mountain’}} & {\itshape nlimen} & { {‘on/at/ the mountain’}}\\
{\itshape nungu} & { {‘pot’}} & {\itshape nungun} & { {‘in the pot’}}\\
{\itshape muda} & { {‘water’}} & {\itshape muden} & { {‘in the water’}}\\
{\itshape kitara} & { {‘bed’}} & {\itshape kitaren} & { {‘on bed’}}\\
{\itshape umbe} & { {‘cow’}} & {\itshape umben} & { {'at/among cows'}}\\
{\itshape irike} & { {‘warmth’}} & {\itshape iriken} & { {‘in the warmth’}}\\
\lspbottomrule
\end{tabular}
\caption{\label{tab:mallya:2} Locativised nouns in Kiwoso}
\end{table}


{Depending on the context, the locative suffix -(}{\textit{i}}){\textit{n}} {in Kiwoso demonstrates all shades of meanings expressed by the traditionally recognized locative prefixes} {\textit{pa}}{{}-,} {\textit{ku}}{{}-, and} {\textit{mu}}{{}-.} {Suffixation as a means of deriving locative nouns has been attested in other East African Bantu languages such as Kikuyu \citep{Mugane1997}, Kiswahili \citep{Grégoire1975}, Kamba \citep{Kioko2005}, and in southern Bantu languages such as Tswana \citep{Creissels2011} and Swati \citep{Marten2010}.} 


Prototypically, in many Bantu languages, class 16 expresses nearness, specific and definite location. Class 17 denotes remoteness, unspecific and indefinite location, while class 18 indicates interiority, inside or location within (see \citealt{Grégoire1975, Maho1999, Fleisch2005cognitive, MartenEtAl2007, Guérois2016}). Although the specific meaning expressed by these prefixes differs across languages, the aforesaid are the general meanings associated with the locative classes. Illustrative examples are provided in \REF{ex:mallya:5}.

\ea\label{ex:mallya:5}      
    \ea\label{ex:mallya:5a}
    \gll wa-ndu      wa-le-id-a      ruko-n\\
        2-people    2\textsc{sm}-\textsc{pst}-enter-\textsc{fv}    9.kitchen-\textsc{loc}\\
    \glt ‘People entered in (i.e., inside) the kitchen.’

    \ex\label{ex:mallya:5b}
    \gll wa-ka        wa-le-lal-a        ki-tare-n\\
       2-woman  2\textsc{sm}-\textsc{pst}-sleep-\textsc{fv}    7-bed-\textsc{loc}\\
     \glt ‘Women slept on the bed.’

    \ex\label{ex:mallya:5c}
    \gll wa-na    wa-le-shaam-a        n-lime-n\\
     2-child  2\textsc{sm}-\textsc{pst}-climb-\textsc{fv}        3-mountain-\textsc{loc}\\
     \glt ‘Children went to the mountain.’

    \ex\label{ex:mallya:5d}
    \gll duke-n         ku-le-ch-a      wa-ndu\\
      9.shop-\textsc{loc}    17-\textsc{pst}-come-\textsc{fv}    2-people\\
   \glt ‘At the shop there came people.’
\z
\z

The locative expressions \textit{ruko-n} ‘in the kitchen’ in \REF{ex:mallya:5a} denotes an inside or interior location, \textit{kitare-n} ‘on the bed’ in \REF{ex:mallya:5b} and \textit{nlime-n} ‘to the mountain’ in \REF{ex:mallya:5c} indicate general and non-specific locations, whereas \textit{duken} ‘at the shop’ in \REF{ex:mallya:5d} expresses a specific, definite location. These examples show that the locative suffix -(\textit{i})\textit{n} in Kiwoso can be used to express a range of nuances of meaning which are associated the locative classes 16, 17 and 18 cross-Bantu. As mentioned in \sectref{sec:mallya:4.1}, the locative suffix *-\textit{ini} and related forms has been considered to be the grammaticalized form of the lexeme meaning ‘liver’, and it is thought to have originally been used to denote interior location before it expanded further to denote other locative relations in languages such as Kiwoso. (For further details about the suffix across Bantu languages see \citet[185--204]{Grégoire1975} and \citet[51--52]{Güldemann1999}).

The available evidence suggests that semantically, locative suffixes cannot occur with animate nouns in some languages. For example, in Kiswahili, nouns such as \textit{mtu} ‘person’ and \textit{nguruwe} ‘pig’ and \textit{paka} ‘cat’ cannot be locativised by means of the suffix -\textit{ni} \citep{Rugemalira2004}. This means that the constructions such as \textit{mtu-ni}, \textit{nguruwe-ni}, and \textit{paka-ni} are unacceptable. In contrast, in Kiwoso, the locative suffix -(\textit{i})\textit{n} can be affixed to animate nouns to form locative nouns. The examples in \REF{ex:mallya:6} demonstrate this process.

\ea%6
    \label{ex:mallya:6}
  \gllllll        {Ordinary nouns}        {gloss}            {locativised nouns}          {gloss}\\
   \textit{wandu}                          \normalfont‘people’        \textit{wandun}                            {\normalfont‘at/in/with/by the people’}\\
  \textit{mburu}                         ‘goat’            \textit{mburun}                            {‘by the goat’}\\
  \textit{umbe}                          ‘cow’            \textit{umben}                              {‘by the cow'}\\
  \textit{baka}                            ‘cat’              \textit{baken}                                {‘by the cat'}\\
  \textit{kite}                             ‘dog’            \textit{kiten}                                  {‘by the dog’}\\
\z

Apart from the nouns exemplified in \REF{ex:mallya:6}, the locative suffix -(\textit{i})\textit{n} in Kiwoso is also used to mark abstract locations. The suffix can be attached to abstract nouns, such as \textit{reema} ‘darkness’, \textit{mmbari} ‘sun’, and \textit{ngoo} ‘heart’ to form locative nouns, as the forms in \REF{ex:mallya:7} illustrate.

\ea%7
    \label{ex:mallya:7}
   \gllll {Ordinary nouns}      {gloss}                    {locativised nouns}        {gloss}\\
        \textit{reema}                      \normalfont‘darkness’            \textit{reemen}                        {\normalfont‘in the darkness’}\\
    \textit{mmbari}                   ‘sun’                      \textit{mmbarin}                   {‘in the sun’}\\
   \textit{ngoo}                        ‘heart’                    \textit{ngoon}                          {‘in/from/the heart’}\\
   \z

The data presented in \REF{ex:mallya:6} and \REF{ex:mallya:7} suggest that in Kiwoso, animate and inanimate nouns, as well as abstract entities can express places or locations by simply adding the locative suffix -(\textit{i})\textit{n}. Note also that agreement on the dependents of the nouns is marked by the invariant locative class 17 prefix \textit{ku-}. Agreement patterns are explained further in \sectref{sec:mallya:4.3.1} below.

\subsection{Locative agreement patterns}
\label{sec:mallya:4.3}

\subsubsection{Locative marking within NPs}
\label{sec:mallya:4.3.1}

Locative expressions are also realized differently in terms of morphology. In Bantu languages, locative nouns are often associated with different types of agreement markers. Usually, in languages where locative classes 16, 17, and 18 are productive, a series of concordial class prefixes are associated with the derived nouns. In Cuwabo and Makhuwa \citep{Guérois2016}, Bemba \citep{Marten2012} and Chichewa (\citealt{BresnanKanerva1989, Carstens1997}), for example, all three locative prefixes exhibit full agreement with other elements in a construction. This is demonstrated in the examples in \REF{ex:mallya:8} from Chichewa \citep[362]{Carstens1997}.

\ea%8
    \label{ex:mallya:8}
    \ea\label{ex:mallya:8a} 
    \gll pa-nyumba      pa-ku-on-ek-a        ngati     pa-ku-psy-a\\
        16-9house      16-\textsc{asp}-see-\textsc{stat}-\textsc{fv}    like         16-\textsc{asp}-burn-\textsc{fv}\\
    \glt    ‘The house and surrounding yard look like they are burning.’

  \ex\label{ex:mallya:8b} 
  \gll ku-nyumba      ku-ndi      ku-tali\\
    17-9house    17-\textsc{dem}    17-far\\
   \glt ‘That house and its environs are far away.’

  \ex\label{ex:mallya:8c}
  \gll mu-nyumba  mu-ku-nunkh-a\\
      18-9house  18-\textsc{asp}-stink-\textsc{fv}\\
   \glt `Inside the house stinks.’
    \z
\z

In contrast to languages such as Chichewa, some Bantu languages such as Herero and Lozi \citep{MartenEtAl2007} distinguish three locative noun classes, but only one or two of the classes are reflected in the agreement pattern of these languages. Lozi and Herero for example exhibit a three-way distinction in the class prefix of locative nouns, but subject agreement is exclusively marked by the class 17 prefix. Similarly, in Kinyarwanda, locative agreement on predicates is invariably marked by the class 16 prefix (\citealt{ZellerNgoboka2018}).

This observation suggests that the absence of a three-way locative class prefix distinction on derived nouns does not preclude a three-way locative noun class prefix system on nominal modifiers and verb agreement. \citet{Grégoire1975} has pointed out that locative nouns in languages such as Kiswahili, Shambala, and Bondei are consistently achieved by means of the locative suffix -\textit{(i)ni}, but agreement markers on dependents reflect the three-way class distinction, as the examples in \REF{ex:mallya:9} from Kiswahili show \citep[402]{Carstens1997}.

\ea%9
    \label{ex:mallya:9}
    \ea\label{ex:mallya:9a}
    \gll nyumba-ni          pa-ngu   pa-zuri\\
            9.house-\textsc{loc}    16-my        16-good\\
            
    \ex\label{ex:mallya:9b}
    \gll nyumba-ni        kw-angu    ku-zuri\\
              9.house-\textsc{loc}    17-my        17-good\\
     \ex\label{ex:mallya:9c}
     \gll nyumba-ni       mw-angu    m-zuri\\
              9.house-\textsc{loc}    18-my        18-good\\
     \glt  ‘at/in my good house’
    \z
\z

Indeed, this is the system that is seen in Kiwoso. In common with other languages that express location by means of the suffix -\textit{(i)ni}, and in which locative prefixes are unproductive, Kiwoso exhibits agreement markers on different locative nominal modifiers. However, in Kiwoso, agreement on dependents is marked by the invariant locative class 17 prefix \textit{ku}{}-. Examples in \REF{ex:mallya:10} are illustrative of this pattern.

\ea\label{ex:mallya:10}      
    \ea[]{\label{ex:mallya:10a}
    \gll ruko-n             ko-ke            ku-cha\\
    9.kitchen-\textsc{loc}    17-\textsc{poss}    17-nice  \\
    \glt ‘at/in his/her nice kitchen’}

    \ex[**]{\label{ex:mallya:10b}
    \gll ruko-n              lya-ke        lyi-cha\\
     9.kitchen-\textsc{loc}        9-\textsc{poss}    9-nice  \\}

    \ex[]{\label{ex:mallya:10c}
    \gll ki-tare-n        ko-ke          ku-cha\\
    7-bed-\textsc{loc}    17-\textsc{poss}    17-nice \\
    \glt ‘on his/her nice bed’}

    \ex[**]{\label{ex:mallya:10d}
    \gll ki-tare-n        ki-ake    ki-cha\\
    7-bed-\textsc{loc}    7-\textsc{poss}  7-nice  \\}
    \z
\z

The examples in \REF{ex:mallya:10} show that it is the locative suffix \textit{n}{}- which controls agreement on other modifiers, such as possessives and adjectives and not the prefix of the inherent noun. \citet{Marten2012} describes such an agreement system as ‘outer’ agreement. Like many other Bantu languages, the noun \textit{kundo} ‘place’ in Kiwoso reflects a remnant of a locative class 17 prefix. The data presented above further indicate that unlike Zigua and Kamba \citep{Marten2012} which show agreement with the original noun class of the locative noun, Kiwoso does not license inner agreement, as the unacceptability of the examples in \REF{ex:mallya:10b} and \REF{ex:mallya:10d} show. In Kiwoso, when the locative class prefix \textit{ku}{}- triggers agreement on the modifiers, as in \REF{ex:mallya:10a} and \REF{ex:mallya:10c}, the emphasis is on the location (i.e., the modifier provides information about the location). 

\subsubsection{Locative verbal marking}
\label{sec:mallya:4.3.2}

Locative nouns in Bantu languages such as Kagulu \citep{Petzell2008}, Chichewa (\citealt{BresnanKanerva1989}) and Haya \citep{Riedel2010} exhibit subject agreement on the verb, as examples in \REF{ex:mallya:11} and \REF{ex:mallya:12} from Chichewa and Kagulu, respectively, demonstrate.

\ea%11
    \label{ex:mallya:11}(\citealt[3]{BresnanKanerva1989})\\
    \gll m-mi-tengo    mw-a-khal-a    a-nyani\\
  18-4-tree      18-\textsc{perf}-sit-\textsc{fv}    2-baboon\\
  \glt ‘In the trees are sitting baboons’.
\ex%12
    \label{ex:mallya:12}\citep[75]{Petzell2008}\\
    \gll ku-m-lomo    ku-fimb-a      ku-gati\\               
  17-3-mouth  17-swell-\textsc{fv}  17-inside\\ 
  \glt ‘The mouth has swollen inside’.
\z

Chichewa and Kagulu are examples of Bantu languages which distinguish three locative noun classes, viz. 16-18 \citep{MartenEtAl2007}. In Chichewa, locative noun class agreement on the verb is reflected by the presence of subject markers for classes 16, 17, and 18, as shown in \REF{ex:mallya:13}.

\ea%13
    \label{ex:mallya:13}
    \ea\label{ex:mallya:13a}
    \gll pa-{msika}-{pa}          {pa}-{badw}-{a}     {nkhonya}\\
    16-3market-6\textsc{dem} 16\textsc{sm}-be\_born-\textsc{fv}  10fist\\
    \glt ‘At this village the fight is going to break out.’

    \ex\label{ex:mallya:13b}
    \gll ku-mu-dzi      ku-na-bwer-a    a-lendo\\
    17-3-village    17\textsc{sm}-\textsc{pst}-come-\textsc{fv}    2-visitor\\
    \glt ‘To the village came visitors.’

    \ex\label{ex:mallya:13c}
    \gll m-nkhalango    mw-a-khal-a        mi-kango\\
    18-9forest          18\textsc{sm}-PRF-remain-\textsc{fv}      4-lion\\
    \glt ‘In the forest have remained lions.’
    \z
\z

However, not all Bantu languages reflect the full three-way locative noun class distinctions. In Kinyarwanda, Subwa and Sukuma, for example, locative agreement on the verb is restricted to class 16 regardless of the class of the locative noun (cf. \citealt{Maho1999}). Similarly, in Lozi locative subjects are invariably marked by class 17 \textit{ku}{}- \citep{MartenEtAl2007}.

Subject agreement with locative nouns is not only attested in languages with a locative prefix. The agreement is also exhibited in the languages which mark locative nouns with the locative suffix -\textit{(i)ni}. Example \REF{ex:mallya:14} from Kiswahili shows that agreement on the verb is marked by locative classes 16-18, regardless of the fact that the language does not mark locative nouns through locative class prefixes \citep[402]{Carstens1997}.

 \ea\label{ex:mallya:14}
 \gll nyumba-ni        pa-/ku-/m-na      wa-tu          wengi\\
    9.house-\textsc{loc}    16\textsc{sm}-/17\textsc{sm}-/18\textsc{sm}-has    2-people    2.many\\
 \glt ‘In/at the house has many people.’
\z

As has been shown in \sectref{sec:mallya:4.2}, Kiwoso derives locative expressions by means of the suffix -(\textit{i})\textit{n}. However, subject agreement is consistently with locative class 17 prefix for all locative nouns. Examples in \REF{ex:mallya:15} are illustrative of this.

\ea\label{ex:mallya:15}
    \ea\label{ex:mallya:15a}
    \gll duke-n          ku-le-ch-a      wa-ndu      wa-fye\\
       9.shop-\textsc{loc}    17\textsc{sm}-\textsc{pst}-come-\textsc{fv}  2-people    2-many\\
    \glt  ‘In/at the shop came many people’.

    \ex\label{ex:mallya:15b}
    \gll ku-le-ch-a        wa-ndu        wa-fye\\
       17-\textsc{pst}-come-\textsc{fv}      2-person      2-many\\
    \glt  ‘There came many people’.

    \ex\label{ex:mallya:15c}
    \gll mmba      ku-le-id-a          mbefu\\
     9.house    17-\textsc{pst}-enter-\textsc{fv}    10.ants\\
    \glt ‘In the house entered ants’.
    \z
\z

Note that the locative prefix \textit{ku}{}- in \REF{ex:mallya:15} indicates a location or a place. The prefix is a grammatical locative subject marker; thus, constructions with the locative prefix \textit{ku}{}- cannot be interpreted as impersonal constructions in Kiwoso. This interpretation (locative) holds even when the lexical locative subject is not mentioned, as \REF{ex:mallya:15b} exemplifies. Example \REF{ex:mallya:15c} shows that inherent locative nouns, i.e., locative nouns without locative morphology, also trigger class 17 subject prefix on the verb (see also \citealt{Mallya2016}).

In addition to locative subject, locative expressions in a number of Bantu languages trigger locative object agreement. Examples from Kivunjo-Chaga \citep[138]{Moshi1995} and Haya (\citealt[282]{RiedelMarten2012}) in \REF{ex:mallya:16} and \REF{ex:mallya:17}, respectively, demonstrate this.

  \ea\label{ex:mallya:16}      
  \gll wa-fee      wa-ku-ichi      (kayi)\\
    2-parent    2-\textsc{om}17-know    (9.attic)\\
   \glt ‘The parents know there (the attic place)’.
\z

 \ea\label{ex:mallya:17}
 \gll n-ka-ha-gul-a\\
  1\textsc{sm}-\textsc{pst}-\textsc{om}16-buy-\textsc{fv}\\
 \glt ‘I bought it (the place)’.
\z

Examples in \REF{ex:mallya:16} and \REF{ex:mallya:17} illustrate locative object marking in two Bantu languages. However, not all Bantu languages can license locative object markers. Studies indicate that languages such as Lozi, Chasu, Yeyi and the languages of the Nguni group do not realize locative object markers \citep{MartenEtAl2007}. Languages of zone P30 such as Cuwabo and Makhuwa also lack locative object markers \citep{Guérois2016}, making locative object marking another area of variation amongst Bantu languages.

However, Kiwoso can realise locative objects on the verb. The locative object is marked by the locative class 17 prefix \textit{ku}{}- only, as illustrated by the examples in \REF{ex:mallya:18}.

    \ea\label{ex:mallya:18}
    \ea\label{ex:mallya:18a}
    \gll mmba      wa-le-me-\textbf{{ku}}-loly-a\\
      9.house    2\textsc{sm}-\textsc{pst}-\textsc{perf}-\textsc{om}17-see-\textsc{fv}\\
     \glt ‘In the house they have seen (it) there.’

    \ex\label{ex:mallya:18b}
    \gll wa-ndu      wa-le-\textbf{{ku}}-many-a      (Muchi)\\
     2-people    2\textsc{sm}-\textsc{pst}-\textsc{om}17-know-\textsc{fv}    Moshi\\
     \glt ‘People knew (recognized) (it) there (Moshi).’
    \z
\z

\subsection{Locative verbal enclitics}
\label{sec:mallya:4.5}

A locative enclitic, as commonly found in many Bantu languages, is a morpheme that can be attached to the verb to license locative expressions. A large number of Bantu languages exhibit a locative enclitic which establishes the location in which a particular event takes place (see \citealt{PersohnDevos2017} for further discussion and examples of this). Kiwoso exhibits two postverbal locative enclitics, namely =\textit{ho} and =\textit{u}. These markers are considered to be enclitics since they occur after all other suffixes, including the final vowel (see example \ref{ex:mallya:19b}). These locative enclitics can only be attached to the verb to contribute the locative semantics, as exemplified in \REF{ex:mallya:19}.

  \ea\label{ex:mallya:19}    
  \ea\label{ex:mallya:19a}
  \gll wa-na   wa-le-bhik-a     ki-tabu     i-kari-n\\
    2-child  2-\textsc{pst}-put-\textsc{fv}    7-banana    5-car-\textsc{loc}\\
   \glt  ‘The children put a book in the car’

    \ex\label{ex:mallya:19b}
    \gll wa-na   wa-le-bhik-a=\textbf{{ho/u}} ki-tabu\\
     2-child  2-\textsc{pst}-put-\textsc{fv}=\textsc{loc}    7-book  \\
    \glt ‘The children put there a book’

    \ex\label{ex:mallya:19c}
    \gll wa-na   wa-le-bhik-a     ki-tabu\\
     2-child  2-\textsc{pst}-put-\textsc{fv}    7-book    \\
    \glt ‘The children put the book’
    \z
\z

The examples in \REF{ex:mallya:19} show that a locative enclitic =\textit{ho/u} is an obligatory part of the verb \textit{bhika} ‘put’ when a full locative noun is omitted, as the unacceptability of the sentence in \REF{ex:mallya:19c} demonstrates. The obligatory locative enclitic =\textit{ho/u} in example \REF{ex:mallya:19b} refers to an object argument.

The data from Kiwoso show that although the two elements (=\textit{ho} and =\textit{u}) function as true locative objects, their interpretation is slightly different from each other. On the one hand, =\textit{ho} is used to indicate a place or a location which is indefinite, non-specific and which is far from both the speaker and the hearer. On the other hand, =\textit{u} is used when both the speaker and the hearer are certain about the place or the location, and such a location or a place is specific and closer to both the speaker and the hearer. For example, in \REF{ex:mallya:20a} the locative noun \textit{nnda} ‘land/field’ is assumed to be far from both the speaker and the hearer. This contrasts with example \REF{ex:mallya:20b}. The use of the demonstratives \textit{kulya} ‘there’ (afar) and \textit{alya} ‘there’ (near) serve to confirm the difference between the enclitics =\textit{ho} and =\textit{u}. In other words, \textit{kulya} cannot co-occur with =\textit{u} and \textit{alya} cannot co-occur with =\textit{ho}. 

\ea%20
    \label{ex:mallya:20}
    \ea\label{ex:mallya:20a}
    \gll wa-ka       wa-le-ur-a     nnda   (kulya)   wa-ka-dema=ho             soko\\
    2-woman 2\textsc{sm}-\textsc{pst}-buy-\textsc{fv} land        (\textsc{dem}3)  2\textsc{sm}-\textsc{perf}-cultivate=\textsc{loc}   beans\\
    \glt  ‘Women bought land (there) and planted there beans’.

    \ex\label{ex:mallya:20b}
    \gll wa-ka       wa-le-ur-a     nnda   (alya)     wa-ka-dema=u           soko\\
        2-woman 2\textsc{sm}-\textsc{pst}-buy-\textsc{fv} land      (\textsc{dem}2)  2\textsc{sm}-\textsc{perf}-cultivate=\textsc{loc}   beans\\
    \glt  ‘Women bought land (there) and planted there beans’.
    \z
\z

In terms of interpretation, the clitic =\textit{ho} mirrors the meaning assigned to the locative noun \textit{kundo} ‘place’, while the semantics of the clitic =\textit{u} matches the one associated with the locative noun \textit{ando} ‘place’, as also shown in \REF{ex:mallya:20} (cf. \sectref{sec:mallya:4.1} for details on the semantic differences of the nouns \textit{ando} and \textit{kundo}).

More examples of the use of the post-final locative enclitics as objects in Kiwoso is exemplified in \REF{ex:mallya:21}. 

\ea\label{ex:mallya:21}
    \ea[]{\label{ex:mallya:21a}
    \gll duke-n            ku-le-ch-a=\textbf{{ho}} wa-ndu\\
    9.shop-\textsc{loc}      17-\textsc{pst}-come-\textsc{fv}=\textsc{loc}    2-people\\
     \glt ‘At the shop came (there) people.’}

    \ex[]{\label{ex:mallya:21b}
    \gll ku-le-ch-a-=\textbf{{ho}} wa-ndu    (duke-n)\\
     17-\textsc{pst}-come-\textsc{fv}=\textsc{loc}  2-people    (9.shop-\textsc{loc})\\
    \glt ‘There came (there) people (at the shop).’}

    \ex[]{\label{ex:mallya:21c}
    \gll wa-ndu            wa-le-many-a=\textbf{{ho}}\\
      2\textsc{sm}-people    2\textsc{sm}-\textsc{pst}-know-textsc{fv}=\textsc{loc}\\
     \glt  ‘People knew (recognized) (it) there (the place).’}

    \ex[**]{\label{ex:mallya:21d}
    \gll wa-ndu      wa-le-ku-many-a=\textbf{{ho}}\\
     2\textsc{sm}-people      2-\textsc{pst}-17-know-\textsc{fv}=\textsc{loc}\\
     \glt ‘People knew (recognized) there (the place).’}
    \z
\z

The examples in \REF{ex:mallya:21} indicate that locative enclitics in Kiwoso can optionally co-occur with the corresponding lexical object noun \textit{duken} ‘at the shop’ \REF{ex:mallya:21b}, but not with the locative object agreement prefix \textit{ku}{}- \REF{ex:mallya:21d}. This implies that the prefix \textit{ku}{}- is an object agreement marker and the post-verbal locative enclitics =\textit{ho/=u} in Kiwoso are in complementary distribution. However, both enclitics, =\textit{ho} and =\textit{u} can co-occur with the lexical locative subject as well as locative subject agreement marker, as evidenced in \REF{ex:mallya:21a} and \REF{ex:mallya:21c}. In \REF{ex:mallya:21a}, the enclitic =\textit{ho} is an anaphoric locative agreement marker, whereas in \REF{ex:mallya:21c}, it functions as a true locative object.

In the majority of Bantu languages, locative enclitics, when present, correspond to the three locative noun classes, 16, 17 and 18. For example, the three locative enclitics =\textit{vo}, =\textit{wo}, and =\textit{mo} in Cuwabo originate from the three locative noun prefixes \textit{va}{}-, \textit{o}{}-, and \textit{mu}{}-, respectively (see \citealt[5]{Guérois2017}). Additionally, \citet[3]{Gunnink2017} reports that in Fwe, the verbal locative enclitics =\textit{ho}, =\textit{ko}, and =\textit{mo} correspond to the locative noun classes 16, 17 and 18, respectively. However, unlike Cuwabo and Fwe where enclitics are derived from demonstrative forms of different locative noun classes, it is not easy to ascertain the origin of the two locative enclitics (=\textit{ho} and =\textit{u}) in Kiwoso because the language lacks the locative prefixes and the demonstrative forms of the two available locative noun classes (16 and 17) do not correspond the locative enclitics identified in Kiwoso. As mentioned earlier, locative nouns in Kiwoso are derived through a suffix -\textit{(i)n} and class 17 prefix \textit{ku}{}- is only productive in agreement (see \sectref{sec:mallya:4.2}).

\section{Summary and conclusion}
\label{sec:mallya:5}

This chapter has examined the morphosyntax of locative expressions in Kiwoso. It has shown that locative expressions in Kiwoso are achieved by means of a locative suffix -(\textit{i})\textit{n}. The data presented show that although the three traditionally recognized locative noun class prefixes are not productive in Kiwoso, the locative agreement prefix \textit{ku}{}- is consistently used with all nominal and verbal modifiers. Although the locative noun class prefixes in Kiwoso are unproductive, the language has maintained some features of the locative system common to Bantu languages, as evidenced in both the nominal and verbal morphology. The analysis offered in this chapter indicates that Kiwoso further has two post-verbal locative enclitics which function as locative arguments.

The use of the locative suffix -(\textit{i})\textit{n} can be viewed as an innovation to compensate for the lost locative prefixes in the language. The chapter has also looked at the forms of locative expressions in Kiwoso and some of their syntactic properties. It would be interesting to conduct further research on the post-final locative enclitics so as to establish their different forms, origin, and their broader functions other than as locative arguments.

\newpage
\section*{Abbreviations}
\setlength{\columnsep}{2cm}
\begin{multicols}{2}
\begin{tabbing}
1, 2, 3\ldots\, \=   unacceptable sentence\kill
\textsc{adj} \>               adjective marker\\
\textsc{asp} \>             aspect\\
\textsc{consc} \>       consecutive\\
\textsc{dem1} \>           demonstrative of the first series\\
\textsc{dem2} \>           demonstrative of the second series\\
\textsc{dem3} \>            demonstrative to the second series\\
\textsc{fv} \>               final vowel\\
\textsc{init} \>            Initial element\\
\textsc{loc} \>            locative\\
\textsc{om} \>               object marker\\
\textsc{perf} \>           perfective\\
\textsc{poss} \>           possessive marker\\
\textsc{prn} \>             pronoun\\
\textsc{pro} \>             pronominal\\
\textsc{pst} \>             past tense\\
\textsc{sm} \>              subject marker\\
\textsc{stat} \>           stative\\
1, 2, 3\ldots \>         noun classes 1, 2, 3\ldots\\
*  \>        Proto-Bantu\\
(**…) \>                 unacceptable sentence
\end{tabbing}
\end{multicols}

\sloppy\printbibliography[heading=subbibliography,notkeyword=this]
\end{document} 
