\documentclass[output=paper]{langscibook}
\ChapterDOI{10.5281/zenodo.10663767}

\author{Amani Lusekelo\orcid{0000-0001-6275-237X}\affiliation{University of Dar es Salaam}}

\title[Concord and agreement in Eastern Bantu]{Concord and agreement in Eastern Bantu: The augment and noun classes in Nyakyusa}

\abstract{The recent discussion of parameters of morphosyntactic variation motivated further scrutiny of the properties of augment and object markers available on bare nouns and complex noun phrases in Nyakyusa (M31). The focus of this chapter is on the (non-)occurrence of the V-aug\-ment and CV-particle, the role of demonstratives, and the word-order within the noun phrases. The CV-particle appears to derive from the proximal demonstrative. This is confirmed by its complementary distribution with both the proximal demonstrative and the V-aug\-ment. The main role of the CV-particle is to indicate contrastive focus of the referent. In addition, the anaphoric demonstrative \textit{{}-la} ‘that/those’ occurs in complementary distribution with the augment, as both are related to the indication of definiteness. With regard to the role of object prefixes, Nyakyusa reveals that object marking may provide definite readings with verbs which take optional object markers (e.g. \textit{piija} ‘cook’ and \textit{bɪɪka} ‘put’). However, definite interpretations are mandatory with verbs which require obligatory object prefixes, e.g. \textit{bona} ‘see’ and \textit{bʊʊla} ‘inform’. Therefore, object marking is not associated with the realisation of the augment. }

\IfFileExists{../localcommands.tex}{
  \addbibresource{../localbibliography.bib}
  % add all extra packages you need to load to this file

\usepackage{tabularx,multicol}
\usepackage{url}
\urlstyle{same}

\usepackage{listings}
\lstset{basicstyle=\ttfamily,tabsize=2,breaklines=true}

\usepackage{langsci-basic}
\usepackage{langsci-optional}
\usepackage{langsci-lgr}
\usepackage{langsci-osl}
% \usepackage{./langsci/styles/langsci-lgr}
% \usepackage{./langsci/styles/langsci-osl}
% \usepackage{langsci-gb4e}

\usepackage{tikz}
\usetikzlibrary{patterns,calc}
\pgfdeclarepatternformonly{south east lines}{\pgfqpoint{-0pt}{-0pt}}{\pgfqpoint{3pt}{3pt}}{\pgfqpoint{3pt}{3pt}}{
    \pgfsetlinewidth{0.6pt}
    \pgfpathmoveto{\pgfqpoint{0pt}{3pt}}
    \pgfpathlineto{\pgfqpoint{3pt}{0pt}}
    \pgfpathmoveto{\pgfqpoint{.2pt}{-.2pt}}
    \pgfpathlineto{\pgfqpoint{-.2pt}{.2pt}}
    \pgfpathmoveto{\pgfqpoint{3.2pt}{2.8pt}}
    \pgfpathlineto{\pgfqpoint{2.8pt}{3.2pt}}
    \pgfusepath{stroke}}
    
\usepackage{stmaryrd}
\usepackage{wasysym}
\usepackage{multirow}
\usepackage{caption}
\usepackage{subcaption}
\usepackage{mathrsfs}
\usepackage{qtree}

\usepackage{linguex}


  %pminos do not split footnotes
% \interfootnotelinepenalty=10000 %Footnote in Laporte chapters has to be split SN


%\DeclareIndexNameFormat{default}{%
%\nameparts{#1}%
%\usebibmacro{index:name}%
%{\index[names]}%
%{\namepartfamily}%
%{\namepartgiveni}%
% {}% L1
% {}% L2
%{\namepartprefix}% generates spurious space L3
%{\namepartsuffix}% generates spurious space L4
%}

%  {\DeclareIndexNameFormat{default}{%
%     \usebibmacro{index:name}{\index[names]}{#1}{#3}{#5}{#7}}}

%\DeclareIndexNameFormat{default}{%
%  \usebibmacro{index:name}{\sindex[nom]}{#1}{#3}{#5}{#7}}

%\DeclareIndexNameFormat{default}{%
%  \usebibmacro{index:name}{\sindex[person]}{#1}{#3}{#5}{#7}}
%\DeclareIndexNameFormat{default}{%
%\nameparts{#1} \usebibmacro{index:name}{\sindex[person]]}{\namepartfamily}{‌​\namepartgiven}{\nam‌​epartprefix}{\namepa‌​rtsuffix}}

%\newcommand{\smiley}{:)}

%\renewbibmacro*{index:name}[5]{%
%\usebibmacro{index:entry}{#1}%
%{\iffieldundef{usera}{}{\thefield{usera}\actualoperator}\mkbibindexname{#2}{#3}{#4}{#5}}}

% \newcommand{\noop}[1]{}

%remove for final
%\overfullrule=1mm

\newcommand{\tobi}[2]}}
\renewcommand{\S}[1]{\tobi{#1}{\textsc{*}}}

% this volume references
% puts: [this volume]
% already defined: \citetv
%\newcommand{\citepv}[1]{(\citeauthor{#1} \citeyear*{#1} [this volume])}
\newcommand{\citealtv}[1]{\citeauthor{#1} \citeyear*{#1} [this volume]}

%parentheses around example number
\newcommand{\pref}[1]{(\ref{#1})}

% in-text examples

\newcommand{\lnex}[1]{\textit{#1}} %target lang word
\newcommand{\lnlit}[1]{(lit.: `#1')} %literal reading
\newcommand{\lnlat}[1]{(#1)} % latinization
\newcommand{\lntrans}[1]{`#1'} %translation
\newcommand{\lnexl}[2]%
{\lnex{#1}{} \lnlat{#2}} % ex with latinization
\newcommand{\lnexlat}[3]{\lnex{#1}{} \lnlat{#2}{} \lntrans{#3}} % ex with latinization and tranl.

%ch01
\newcommand{\co}[1]{\mbox{\textbf{#1}}}

%ch09

\newcommand{\cyrbulg}[1]{\begin{otherlanguage*}{bulgarian}#1\end{otherlanguage*}}


%ch10
\newcommand{\nlp}{{\small NLP}}
\newcommand{\mwe}{{\small MWE}}
\newcommand{\rae}{{\small RAE}}
\newcommand{\lvc}{{\small LVC}}
\newcommand{\pos}{{\small P}o{\small S}}
%\newcommand{\todo}[1]{ \textcolor{red}{#1} }

%\renewcommand{\labelenumi}{\theenumi}
%\ainamefmt{{vv}{ll}{, ff}{, jj}} % fullname

\newcommand{\biberror}[1]{{\color{red}#1}}

\newcommand{\osenovaitem}{--~} 
  %% hyphenation points for line breaks
%% Normally, automatic hyphenation in LaTeX is very good
%% If a word is mis-hyphenated, add it to this file
%%
%% add information to TeX file before \begin{document} with:
%% %% hyphenation points for line breaks
%% Normally, automatic hyphenation in LaTeX is very good
%% If a word is mis-hyphenated, add it to this file
%%
%% add information to TeX file before \begin{document} with:
%% %% hyphenation points for line breaks
%% Normally, automatic hyphenation in LaTeX is very good
%% If a word is mis-hyphenated, add it to this file
%%
%% add information to TeX file before \begin{document} with:
%% \include{localhyphenation}
\hyphenation{
    Beck-man
    Ngu-yen
    back-chan-nel
    back-chan-nels
    mo-not-o-nous
    ste-reo-typ-i-cal
}

\hyphenation{
    Beck-man
    Ngu-yen
    back-chan-nel
    back-chan-nels
    mo-not-o-nous
    ste-reo-typ-i-cal
}

\hyphenation{
    Beck-man
    Ngu-yen
    back-chan-nel
    back-chan-nels
    mo-not-o-nous
    ste-reo-typ-i-cal
}
 
  \togglepaper[1]%%chapternumber
}{}

\begin{document}
\maketitle 
%\shorttitlerunninghead{}%%use this for an abridged title in the page headers

\settowidth\jamwidth{[Nyankore-Kiga]}

\section{Introduction}\label{sec:lusekelo:1}  %1. /
\largerpage
The contribution of this chapter is two-fold. Firstly, the parameters for morphosyntactic variation in Bantu languages, as articulated by \citet{GuéroisEtAl2017}, opened another avenue to re-examine the morphosyntactic properties of nouns and noun phrases provided in previous studies for Nyakyusa (\citealt{DeBlois1970, Lusekelo2009, Mbope2016, Persohn2017}). For instance, \citet{DeBlois1970} suggested that both V-aug\-ments and CV-aug\-ments occur in Nyakyusa. Data offered in sections 2 and 3 substantiates that both the V-aug\-ment, CV-aug\-ment (termed CV-particle herein) and prenominal demonstrative realise definiteness in Nyakyusa. Based on the theory of definiteness \citep{Lyons1999} and contrast focus \citep{Repp2010}, the proper functions of the augment are provided in this chapter. I establish the role of the augment as a marker of (in)definiteness. 

Secondly, the interpretation of (in)definite sentence(s) involves both the presence and/or absence of the augment and object markers in Bantu languages with augments. For instance, \citet{Visser2010} claims that definite readings in Xhosa are obtained once an object marker is cliticized on the verb and the object noun is marked with an augment. This claim is confirmed in Bantu languages without augments. For instance, with regard to object marking, \citet{MartenKula2012} argue that the use of the object marker with non-animate NPs is associated with definiteness in Swahili. However, \citet{Riedel2009} argues that some sentences provide definite readings without prefixing the object marker in Swahili. In \sectref{sec:lusekelo:4}, I argue that the distinction between definiteness and indefiniteness in Nyakyusa is partly associated with object marking with verbs which take an object prefix optionally. In verbs which require a mandatory object prefix, the object prefix does not indicate definiteness. 

\section{The shape and concord of bare nouns}\label{sec:lusekelo:2}  %2. /
\subsection{The shape and distribution of the augment}\label{sec:lusekelo:2.1}  %2.1 /

The first parameter of \citet{GuéroisEtAl2017} requires an investigation of the morphology of the augment. Data shows that bare nouns\footnote{The notion “bare noun” is used to refer to nouns composed of an augment, noun class prefix and a stem without any modification or quantification. A bare noun is a noun in isolation. This is opposed to complex noun phrases which are comprised of a head noun and at least one modifier or quantifier.}  in Nyakyusa\footnote{Most examples of bare nouns for Nyakyusa come from \citet{Felberg1996}. Some sentences for Nyakyusa come from \citet{Lusekelo2012} and \citet{Persohn2017}. Other examples were constructed by the author who is a native speaker of the language.}  consist of an augment (always a V-aug\-ment), a noun class prefix and a stem, as exemplified in \REF{ex:lusekelo:1}. Notice that more examples are provided for noun classes 5/6 due to variations in the shape of the noun class 5.  

\ea%1
    \label{ex:lusekelo:1}
\begin{tabbing}
    cl. 18 \=  \textit{a-ma-isi} \textit{(amiisi)}  \= ‘person’   \kill
    cl. 1 \> \textit{ʊ{}-mʊ{}-ndʊ} \> ‘person’     \\
  cl. 2 \> \textit{a-ba-ndʊ} \> ‘persons’\\
  cl. 3 \> \textit{ʊ{}-m-piki} \> ‘tree’  \\
    \> \textit{ʊ}{}-\textit{n-kota} \>  ‘medicine’\\
  cl. 4 \> \textit{ɪ{}-mɪ{}-piki} \> ‘trees’   \\
    \> \textit{ɪ{}-mɪ{}-kota} \>  ‘medicines’\\
  cl. 5  \> \textit{ɪ{}-lɪ{}-lasi} \> ‘bamboo tree’\\
    \> \textit{ɪ{}-lasi} \> ‘bamboo tree’  \\
    \> \textit{ɪ{}-lɪ{}-bwe} \> ‘stone’   \\
   \> \textit{ɪ{}-bwe} \> ‘stone’   \\
   \> \textit{ɪ{}-lɪ{}-fumbi} \> ‘egg’   \\
   \> \textit{ɪ{}-fumbi} \> ‘egg’\\
    \> \textit{ɪ{}-ly-abi} \> ‘underpant’ \\
   \> \textit{ɪ{}-ly-osi} \> ‘smoke’ \\
  cl. 6  \> \textit{a-ma-boko} \> ‘arms, hands’  \\
   \> \textit{a-ma-lasi} \> ‘bamboo trees’\\
   \> \textit{a-ma-isi} \textit{(amiisi)} \> ‘water’ \\
   \> \textit{a-ma-bwe} \> ‘stones’   \\
   \> \textit{a-ma-abi} \> ‘underpants, underwear’  \\
  cl. 7 \> \textit{ɪ{}-kɪ{}-lundi} \> ‘leg’ \\
   \> \textit{ɪ{}-kɪ{}-boko} \> ‘arm, a hand’  \\
   \> \textit{ɪ{}-kɪ{}-paale} \> ‘calabash’     \\
  cl. 8 \> \textit{ɪ{}-fɪ{}-lundi} \> ‘leg’ \\
   \> \textit{ɪ{}-fɪ{}-boko} \> ‘arms, hands’ \\
   \> \textit{ɪ{}-fɪ{}-paale} \> ‘calabashes’\\
  cl. 9 \> \textit{ɪ{}-nguku} \> ‘fowl’ \\
   \> \textit{ɪ{}-mbwa} \> ‘dog’ \\
  cl. 10 \> \textit{ɪ{}-nguku} \> ‘fowls, chickens’ \\
   \> \textit{ɪ{}-mbwa} \> ‘dogs’ \\
   \> \textit{ɪ{}-mbabu} \> ‘firewood (\textsc{pl})’\\
  cl. 11  \> \textit{ʊ{}-lʊ{}-kama} \> ‘milk’  \\
   \> \textit{ʊ{}-lʊ{}-babu} \> ‘firewood (\textsc{sg})’  \\
  cl. 12 \> \textit{a-ka-kuku} \> ‘chick’ \\
   \> \textit{a-ka-lasi} \> ‘small bamboo tree’      \\
  cl. 13 \> \textit{ʊ{}-tʊ{}-kuku} \> ‘chicks’ \\
   \> \textit{ʊ{}-tʊ{}-lasi} \> ‘small bamboo trees’  \\
  cl. 14 \> \textit{ʊ{}-bʊ-ndʊ} \> ‘humanity’ \\
  cl. 15 \> \textit{ʊ{}-kʊ{}-lima} \> ‘to farm, to hoe’ \\
   \> \textit{ʊ{}-kʊ{}-seka} \> ‘to laugh’\\
  cl. 16 \> \textit{pa-kaja} \> ‘at home/homestead’ \\
  cl. 17 \> \textit{ku-kaja} \> ‘to the home/homestead’ \\
  cl. 18 \> \textit{mu{}-ngaja} \> ‘in the grove’
\end{tabbing}
\z

The data in \REF{ex:lusekelo:1} substantiate five points regarding Nyakyusa. The first point concerns the distribution of the augment across noun classes. The underived nouns are primarily characterized by the word-structure \textsc{aug-ncp}-root because an augment is found before noun class prefix. The shape of the augment is regularly a vowel-augment (V-aug\-ment) in bare nouns. The V-aug\-ment is realised as \textit{ɪ}{}-, \textit{a}{}- and \textit{ʊ}{}-, as discussed  in previous works (\citealt{DeBlois1970, Felberg1996, Lusekelo2009, Lusekelo2013, Mbope2016, Persohn2017}). In fact, the vowel quality of the augment is a copy of the noun class prefix vowel. Locative classes 16--18 do not contain the augment.  

The \textit{ɪ{}-}augment prolifically occurs in six noun classes: 4, 5, 7, 8, 9 and 10. In addition, the loanwords which are integrated into noun classes 9/10 take the \textit{ɪ}{}-augment irrespective of the presence or absence of the nasal consonant noun prefix, as exemplified in \REF{ex:lusekelo:2}. 

\ea%2
    \label{ex:lusekelo:2}
    \begin{tabbing}
cl. 9/10 \= \textit{ɪ{}-ndalama} \= [<\textit{cabbage}: English] \= ‘blouse’\kill
 cl. 9/10 \> \textit{ɪ{}-bulausɪ} \> [<\textit{blouse}: English] \> ‘blouse’ \\
   \> \textit{ɪ{}-hela} \> [<\textit{Heller}: German] \> ‘money’ \\
   \> \textit{ɪ{}-kaabatɪ} \> [<\textit{kabati}: Swahili] \> ‘cupboard’ \\
   \> \textit{ɪ{}-kaabɪkɪ} \> [<\textit{cabbage}: English] \> ‘cabbage’ \\
   \> \textit{ɪ{}-katani} \> [<\textit{katani}: Swahili] \> ‘sisal’ \\
   \> \textit{ɪ{}-naulɪ} \> [<\textit{nauli}: Swahili] \> ‘bus fare’ \\
   \> \textit{ɪ{}-ndalama} \> [<\textit{dirham}: Arabic] \> ‘money’ \\
   \> \textit{ɪ{}-ndobo} \> [<\textit{ndoo}: Swahili] \> ‘bucket’   \\
   \> \textit{ɪ{}-sisala} \> [<\textit{scissor}: English] \> ‘a pair of scissor’
   \end{tabbing}
\z

The \textit{ɪ}{}-augment occurs in other nouns in classes 9/10 which do not bear the nasal. The \textit{ɪ}{}-augment is for both singularity and plurality, as shown in \REF{ex:lusekelo:3}.  

\ea%3
    \label{ex:lusekelo:3}
\glll {\normalfont cl. 9/10}  \textit{ɪ{}-sekema}  \normalfont‘fever’ \\
  {}  \textit{ɪ{}-fula}  ‘rainfall’ \\
  {}  \textit{ɪ{}-sanu}  ‘weed(s)’\\
\z

The \textit{ʊ}{}-augment is also productive, as it occurs in classes 1, 3, 11 and 13. The \textit{a}{}-augment occurs in noun classes 2, 6 and 12. Both augments occur in loans, e.g.  \textit{alumasɪ} [<Swahili: \textit{almasi}] ‘diamond’, noun class 6, \textit{amafʊta} [<Swahili: \textit{mafuta}] ‘oil, fat’,  noun class 1, \textit{ʊnnesi} [<Swahili: \textit{nesi}] ‘nurse’ and \textit{ʊnsinjala} [<English: \textit{messenger}] ‘office clerk’. 

The borrowed nouns in Nyakyusa obtain the shape \textsc{aug-ncp}-root when they are incorporated into the lexicon, as illustrated in \REF{ex:lusekelo:4}.

\newpage
\ea%4
    \label{ex:lusekelo:4}
    \begin{tabbing}
    cl. 8 \= \textit{ɪ{}-mɪ{}-papaju} \= [<\textit{kikombe}: Swahili] \= ‘a police officer, a soldier’\kill
cl. 1 \> \textit{ʊ{}-n-sikali} \> [<\textit{askari}: Swahili] \> ‘a police officer, a soldier’ \\
  cl. 2 \> \textit{a-ba-sikali} \> [<\textit{askari}: Swahili] \> ‘police officers, soldiers’ \\
  cl. 3 \> \textit{ʊ{}-m-papaju} \> [<\textit{mpapai}: Swahili] \> ‘a paw paw tree’  \\
  cl. 4 \> \textit{ɪ{}-mɪ{}-papaju} \> [<\textit{mipapai}: Swahili] \> ‘paw paw trees’\\
  cl. 5 \>  \textit{ɪ{}-lɪ{}-}{\textit{g}}\textit{alasi} \> [<\textit{glass}: English] \> ‘a glass, spectacles’ \\
  cl. 7 \> \textit{ɪ{}-kɪ{}-kombe} \> [<\textit{kikombe}: Swahili] \> ‘a cup’  \\
  cl. 8  \> \textit{ɪ{}-fɪ{}-kombe} \> [<\textit{vikombe}: Swahili] \> ‘cups’
  \end{tabbing}
\z

The augment is absent in some noun classes in Nyakyusa. \Citet[119]{DeBlois1970} states correctly that Nyakyusa has no V-aug\-ments in locative classes 16, 17 and 18. The data in \REF{ex:lusekelo:1} confirm that the augment does not surface in any of the locative noun classes 16, 17 and 18.

The second point concerns the status of the \textit{ɪ}{}-augment in noun class 5. Data in \REF{ex:lusekelo:1} above show the optionality of the noun class -\textit{lɪ}{}- (cl. 5). Nouns in class 5 may occur or may not occur with a noun prefix, as exemplified by \textit{ɪlɪbwe/ɪbwe} ‘stone’ and \textit{ɪlɪlasi/ɪlasi} ‘bamboo tree’. More data with the optional singular class prefix, together with their corresponding plural pairings, are provided in \REF{ex:lusekelo:5} below. 

\ea%5
    \label{ex:lusekelo:5}
    \begin{tabbing}
    cl. 5 \= \textit{ɪ{}-(lɪ){}-kumbuulu} \= ‘machine’ \=  cl. 6 \= \textit{a-ma-kumbuulu} \= ‘hoes’\kill
    cl. 5 \> \textit{ɪ{}-(lɪ){}-jabʊ} \> ‘cassava’  \> cl. 6  \> \textit{a-ma-jabʊ} \> ‘cassava (\textsc{pl})’ \\
  cl. 5 \> \textit{ɪ{}-(lɪ){}-kina} \> ‘machine’ \>  cl. 6 \> \textit{a-ma-kina} \> ‘machines’\\ 
  cl. 5 \> \textit{ɪ{}-(lɪ){}-jɪko} \> ‘kitchen’  \> cl. 6 \> \textit{a-ma-jɪko} \> ‘kitchens’ \\
  cl. 5 \> \textit{ɪ{}-(lɪ){}-koonda} \> ‘wasp’  \> cl. 6 \> \textit{a-ma-koonda} \> ‘wasps’\\ 
  cl. 5 \> \textit{ɪ{}-(lɪ){}-kumbuulu} \> ‘a hoe’  \> cl. 6 \> \textit{a-ma-kumbuulu} \> ‘hoes’ \\
  cl. 5 \> \textit{ɪ{}-(lɪ){}-lopa} \> ‘blood’  \>   cl. 6  \> \textit{a-ma-lopa} \> ‘blood’ \\
  cl. 5 \> \textit{ɪ{}-(lɪ){}-sosɪ} \> ‘tear’   \>  cl. 6 \> \textit{a-ma-sosɪ} \> ‘tears’
\end{tabbing}
\z

Some nouns in class 5 occur with the noun prefix, as demonstrated by \textit{ɪlyabi} ‘underpant’ and \textit{ɪlyosi} ‘smoke’ in \REF{ex:lusekelo:1}. Here the removal of the noun class prefix -\textit{lɪ}{}- is unacceptable. Therefore, the forms *\textit{ɪabi} and *\textit{ɪosi} are unacceptable. 

In nouns with optional -\textit{lɪ}{}-, the status of the augment \textit{ɪ{}-} in class 5 is worth mentioning here. First and foremost, do we treat the noun without a class prefix as being preceded by an augment or a noun class prefix? The answer to this question will be that the augment does not function as the class prefix, as found by \citet{Legère2005} for Kwanyama (spoken in Namibia). In Nyakyusa, nouns in class 5 may contain a silent nominal prefix. The augment is always overt.    

The third point substantiated by data in \REF{ex:lusekelo:1} above surrounds claims made in previous studies for Nyakyusa. \Citet[114]{DeBlois1970} characterised Nyakyusa as lacking an augment in kinship terms, proper names, titles, and their plural forms (i.e. nouns in classes 1a and 2a). \citet[41]{Persohn2017} suggested that the \textit{ʊ}{}-augment is optional for kinship terms, proper names, some living beings, and some loans. This use of the augment is related to specific grammatical conditions, which have not been articulated by De Blois and Persohn.

Nyakyusa allows the use of the \textit{ʊ}{}-augment and the \textit{a}{}-augment in kinship terms, some proper names and titles, as illustrated in \REF{ex:lusekelo:6}. 

\ea%6
    \label{ex:lusekelo:6}
    \small
    \begin{tabbing}
    cl. 1a \= \textit{ʊ{}-Mwasekage} \= ‘their mother’  \= cl. 2a \= \textit{a-ba-malalafyale} \= ‘the Mwakaseges’\kill
    cl. 1a  \> \textit{ʊ{}-nyoko} \> ‘your mother’ \>  cl. 2a  \> \textit{a-ba-nyoko} \> ‘your mothers’ \\
  cl. 1a \> \textit{ʊ{}-taata} \> ‘my father’ \>  cl. 2a \> \textit{a-ba-taata} \> ‘my fathers’ \\
  cl. 1a \> \textit{ʊ{}-maama} \> ‘mother’ \>  cl. 2a \> \textit{a-ba-maama} \> ‘mothers’ \\
  cl. 1a \> \textit{ʊ{}-baaba} \> ‘father’  \> cl. 2a \> \textit{a-ba-baaba} \> ‘fathers’ \\
  cl. 1a \> \textit{ʊ{}-juuba} \> ‘mother’  \> cl. 2a \> \textit{a-ba-juuba} \> ‘mothers’ \\
  cl. 1a \> \textit{ʊ{}-Tuntufye} \> `Tuntufye' \>  cl. 2a \> \textit{a-ba-Tuntufye} \> ‘the Tuntufyes’ \\
  cl. 1a \> \textit{ʊ{}-Mwasekage} \> `Mwasekage' \>  cl. 2a \> \textit{a-ba-Mwasekage} \> ‘the Mwakaseges’\\
  cl. 1a \> \textit{ʊ{}-Malija} \> ‘Mary, Maria’ \>  cl. 2a \> \textit{a-ba-Malija} \> ‘the Marys’ \\
  cl. 1a \> \textit{ʊ{}-m-puuti} \> ‘priest’  \> cl. 2a \> \textit{a-ba-puuti}  \> ‘priests’ \\
  cl. 1a \> \textit{ʊ{}-malafyale} \> ‘chief’  \>   cl. 2a \> \textit{a-ba-malalafyale} \> ‘chiefs’ \\
  cl. 1a \> \textit{ʊ{}-n-nabo} \> ‘their mother’  \> cl. 2a \> \textit{a-ba-nabo} \> ‘their mothers’
\end{tabbing}
\z

Some proper names that contain the prefix for class 11 do not contain the \textit{ʊ}{}-augment. Nouns which contain the augment belong to class 11, as illustrated in \REF{ex:lusekelo:7}.   

\ea%7
    \label{ex:lusekelo:7}
    \begin{tabbing}
    cl. 11 \= \textit{ʊ{}-Lupakisyo} \= ‘happiness’   \=  cl. 1a \= \textit{Lupakisyo} \= ‘proper name’\kill
   cl. 11 \> \textit{ʊ{}-lusajo} \> ‘blessing’ \>  cl. 1a \> \textit{lusajo} \> ‘proper name’\\
  cl. 11 \> \textit{ʊ{}-lusekelo} \> ‘happiness’ \>  cl. 1a \> \textit{lusekelo} \> ‘proper name’ \\
  cl. 11 \> \textit{ʊ{}-lupakisyo} \> ‘fear’  \>   cl. 1a \> \textit{lupakisyo} \> ‘proper name’ \\
  cl. 11 \> \textit{ʊ{}-lu}{\textit{g}}\textit{ano} \> ‘love’  \>   cl. 1a  \> \textit{lu}{\textit{g}}\textit{ano} \> ‘proper name’ \\
  cl. 11 \> \textit{ʊ{}-lusubilo} \> ‘hope’   \>  cl. 1a \> \textit{lusubilo} \> ‘proper name’\\
  cl. 11 \> \textit{ʊ{}-lutufyo} \> ‘praise’  \> cl. 1a \> \textit{lutufyo} \> ‘proper name’
  \end{tabbing}
\z

The fourth point concerns the V-aug\-ments \textit{ɪ}{}-, \textit{a}{}- and \textit{ʊ}{}- which remain present in derived nouns in Nyakyusa. Therefore, derived nouns are characterised by the configuration \textsc{aug-ncp}-stem-\textsc{noml}, as exemplified in \REF{ex:lusekelo:8}. In Nyakyusa, nomilising suffixes (marked as \textsc{noml}) are V-shaped, realised mainly as -\textit{e}, -\textit{i} and -\textit{o}. The applicative is often used to derive deverbatives in combination with these nominalising suffixes, such as \textit{ɪ{}-simb-il-o} ‘a pen’ (< \textit{samba} ‘write’). 

\ea%8
    \label{ex:lusekelo:8}
    \begin{tabbing}
    \textit{sambula} \= ‘make noise’ \= cl. 18  \= \textit{ɪ{}-fɪ{}-sambul-il-o} \= ‘ladles, scoopers’\kill
  \textit{bina} \> ‘become ill’ \> cl. 1 \> \textit{ʊ{}-m-bin-e} \> ‘patient’\\
  \textit{paapa} \> ‘give birth’ \> cl. 2 \> \textit{a-ba-paap-i} \> ‘parents’\\
  \textit{jwee}{\textit{g}}\textit{a} \> ‘make noise’ \> cl. 3 \> \textit{ʊ{}-n-jwee}{\textit{g}}\textit{{}-o} \> ‘noise’\\
  \textit{simba} \> ‘write’ \> cl. 9 \> \textit{ɪ{}-simb-il-o} \> ‘pen’ \\
  \textit{koma} \> ‘hit’ \> cl. 5 \> \textit{ɪ{}-lɪ{}-kom-el-o} \> ‘playing/dancing tail’ \\
  \textit{sambula} \> ‘scoop’ \> cl. 8 \> \textit{ɪ{}-fɪ{}-sambul-il-o} \> ‘ladles, scoopers’\\
  \textit{seka} \> ‘laugh’ \> cl. 11 \> \textit{ʊ{}-lʊ{}-sekel-o} \> ‘happiness’
\end{tabbing}
\z

\begin{sloppypar}
The fifth and last point concerns the diminutive class. \citet[3--5]{Felberg1996} lists inherent nouns in classes 12/13 (\textit{aka-/ʊtʊ{}-}), which are few under primary classification in Nyakyusa. Typical nouns in this class include \textit{akabalilo}{}-\textit{ʊtʊbalilo} ‘time’, \textit{akaaja-ʊtwaja} ‘homestead(s)’, \textit{akalʊʊlʊ{}-ʊtʊlʊʊlʊ} ‘ululation(s)’ and \textit{akasumo-ʊtʊsumo} ‘tale(s)’. 
\end{sloppypar}

Most class 12/13 nouns are derived diminutive nouns. \citet[43]{Persohn2017} uses the examples \textit{akapango} ‘story’ and \textit{ʊtʊpango} ‘stories’, which are examples of derived nouns which come from the verb \textit{panga} ‘narrate, tell a story’. Diminutive nouns (and sometimes pejorative nouns) are derived from classes 1 to 11, as illustrated in \REF{ex:lusekelo:9}. 

\ea%9
    \label{ex:lusekelo:9}
\begin{tabbing}
    cl. 11  \= \textit{ʊ{}-lʊ{}-kama} \= ‘bamboos’  \=   cl. 12 \= \textit{ʊ{}-tʊ{}-kama} \= ‘small bamboo’\kill
  cl. 1 \> \textit{ʊ{}-mw-ana} \> ‘child’  \>   cl. 12 \> \textit{a-ka-ana} \> ‘small child’ \\
  cl. 2 \> \textit{a-ba-ana} \> ‘children’ \>  cl. 13 \> \textit{ʊ{}-tw-aana} \> ‘small children’  \\
  cl. 3 \> \textit{ʊ{}-m-piki} \> ‘tree’  \>   cl. 12 \> \textit{a-ka-piki} \>  ‘small tree’ \\
  cl. 4 \> \textit{ɪ{}-mɪ{}-piki} \> ‘trees’  \>   cl. 12 \> \textit{ʊ{}-tʊ{}-piki} \> ‘small trees’  \\
  cl. 5 \> \textit{ɪ-(lɪ){}-lasi} \> ‘bamboo’  \> cl. 12 \> \textit{a-ka-lasi} \> ‘small bamboo’ \\
  cl. 6 \> \textit{a-ma-lasi} \> ‘bamboos’  \> cl. 13 \> \textit{ʊ{}-tʊ{}-lasi} \> ‘small bamboo’ \\
  cl. 7 \> \textit{ɪ-kɪ{}-kota} \> ‘chair’   \>  cl. 12 \> \textit{a-ka-kota} \> ‘small chair’ \\
  cl. 8 \> \textit{ɪ-fɪ{}-kota} \> ‘chairs’ \>  cl. 13 \> \textit{ʊ{}-tʊ{}-kota} \> ‘small chairs’ \\
  cl. 9 \> \textit{ɪ{}-mbene} \> ‘goat’  \>   cl. 12  \> \textit{a{}-ka-pene} \> ‘small goat’ \\
  cl. 10 \> \textit{ɪ{}-mbene} \> ‘goats’  \> cl. 13 \> \textit{ʊ{}-tʊ{}-pene} \> ‘small goats’ \\
  cl. 11 \> \textit{ʊ{}-lʊ{}-}{\textit{g}}\textit{oje} \> ‘rope’   \>  cl. 12  \> \textit{a-ka-}{\textit{g}}\textit{oje} \> ‘small rope’ \\
  cl. 11 \> \textit{ʊ{}-lʊ{}-kama} \> ‘milk’   \>  cl. 12 \> \textit{ʊ{}-tʊ{}-kama} \> ‘little milk’
\end{tabbing}
\z

\subsection{The function of the augment in Nyakyusa in relation to its role in Eastern Bantu}\label{sec:lusekelo:2.2}  %2.2 /
\largerpage

Previous studies suggested that the augment functions to mark definiteness in Bantu languages (cf. \citealt{DeBlois1970, Bokamba1971, HymanKatamba1993, Legère2005, Visser2010, Goodness2013, Asiimwe2014, PetzellKühl2017}). They suggest that the augment functions like articles in English, Hungarian or French. Three exemplary cases are provided below. 

\citet{Bokamba1971} found in Dzamba (C322, spoken in the DRC) that the augment functions like articles in Western European languages (mainly the English definite article {\textit{the}}). It is argued that Dzamba permits augments on definite nouns only, as shown in (\ref{ex:lusekelo:10}--\ref{ex:lusekelo:11}). 

\ea%10
    Dzamba \citep[220]{Bokamba1971}\label{ex:lusekelo:10}\\
    \gll mɔ{}-ibi  (mɔɔ)  anyɔlɔki  o-ndaku\\ 
  1-thief  (one)  entered  \textsc{au}{}-9.house\\
   \glt  ‘A thief entered the house.’
\ex%11
    \label{ex:lusekelo:11}
    \gll  o-mɔ{}-ibi    (*mɔɔ)  anyɔlɔki  o-ndaku\\
  \textsc{aug}{}-1-thief  (*one)  entered  \textsc{au}{}-9.house          \\
 \glt ‘The thief entered the house.’
\z

In both examples the object noun phrases carry the augment and have definite interpretation. However, the subject noun phrases differ. In example \REF{ex:lusekelo:10}, \textit{mɔibi} ‘a thief’ is not preceded by an augment, hence we obtain an indefinite reading of the subject noun phrase. Example \REF{ex:lusekelo:11} \textit{omɔibi} ‘the thief’ is preceded by an augment, hence we get a definite interpretation. These examples also reveal that the augment and the numeral \textit{mɔɔ} ‘one’ cannot co-occur \citep{Bokamba1971}. Based on such data, it is argued that an NP without an augment indicates an indefinite reading. The presence of the augment warrants a definite interpretation in Dzamba.  

In support of the claim that the augment functions as an (in)definite marker, \citet[71]{Asiimwe2014} shows the following examples from Luganda (JE15, spoken in Uganda) in which the presence and absence of the augments signal (in)def\-i\-nite\-ness. The presence of the augment in \REF{ex:lusekelo:12} marks definiteness while its absence in \REF{ex:lusekelo:13} signals indefiniteness. 

\ea%12
  Luganda \citep[71]{Asiimwe2014}  \label{ex:lusekelo:12}\\
   \gll {U-mw-ana}  {a-a-fwaaya}  {i-ci-tabu}  \\
  \textsc{aug}-1-child  3\textsc{pl}-\textsc{pst}-want  \textsc{aug}-7-book \\
  \glt ‘The child wanted the book.’
\ex%13
    \label{ex:lusekelo:13}
    \gll {U-mw-ana}  {a-a-fwaaya}  {ci-tabu}\\
  \textsc{aug}-1-child  3\textsc{pl}-\textsc{pst}-want  7-book\\
  \glt ‘The child wanted a book.’
\z

Irrespective of the suggestion above, the use of an augment is often difficult to describe in Luganda and Nyankore-Kiga. Most speakers of Nyankore-Rukiga are not certain of the proper choices for the use of deletion of the augment \citep[183--184]{Asiimwe2014}. This kind of uncertainty is also mentioned by \citet{HymanKatamba1993} and \citet{Goodness2013}. In fact, the use of the augment to mark definiteness is not a straightforward mechanism in other Bantu languages, as confirmed by \citet{HymanKatamba1993} for Luganda, \citet{Goodness2013} for Shinyiha and \citet{PetzellKühl2017} for Luguru (both spoken in Tanzania). \citet{PetzellKühl2017} show that the use of demonstratives plays an important role for providing definite readings in Luguru. This may point to the suggestion that the occurrence or non-occurrence of the augment is not related in a straightforward way with the use of the definite article \textit{the} in English. 

The augment is used in a variety of functions in Nyakyusa. Its distribution is robust, but it has limited connection with the expression of (in)definiteness. For instance, \citet[343]{Persohn2017} provides some texts whose nouns bear no augment. Text \REF{ex:lusekelo:14} shows that only the noun \textit{iisikʊ} ‘day’ bears a V-aug\-ment.   

\ea%14
    \label{ex:lusekelo:14}
    \ea \gll {po}  {tʊ-kʊ-tɪ}  \textbf{kalʊlʊ}  {na}  \textbf{nsyɪsyɪ}  {ba-a-lɪ}  {ba-manyaani}  {fiijo}\\
  then  1\textsc{pl}-\textsc{prs}-say  hare  \textsc{com}  skunk  2-\textsc{pst}-\textsc{cop}  2-friend  intense\\
  \glt ‘We say, Hare and Skunk were good friends.’

  \ex \gll {b-end-aga}  {b-oosa}  {kʊkʊtɪ}  {kʊ-no}  {bi-kʊ-bʊʊk-a}\\
  2-travel-\textsc{ipfv}  2-all  every  17-prox  2-\textsc{prs}-go-\textsc{fv}\\
  \glt ‘They went together wherever they went.’

  \ex \gll {po}  \textbf{ii-sikʊ}  {lɪ-mo}  \textbf{kalʊlʊ}   {na}  \textbf{nsyɪsyɪ}    {ba-a-bʊʊkile}   {n-kʊ-fwɪm-a} \\
    then  5-day  5-one  hare   \textsc{com}  skunk    2-\textsc{pst}-{go}{}-\textsc{pfv}  18-15-hunt-\textsc{fv}\\
    \glt ‘So one day Hare and Skunk went to hunt.’
    \glt \citep[343]{Persohn2017}
\z
\z

\citet[352]{Persohn2017} provides other texts whose nouns bear a V-aug\-ment. He suggests that the definiteness and indefiniteness distinction is realised using the augment. Text \REF{ex:lusekelo:15} shows that all nouns bear the V-aug\-ment, e.g. \textit{abatasi} ‘ancestors’ and \textit{ɪndingala} ‘drum’. 

\ea%15
    \label{ex:lusekelo:15}
\ea \gll   {ba-a-li=po}  \textbf{a-ba-ndʊ}  {b-a}  {ijolo}  \textbf{a-ba-tasi}\\
  2-\textsc{pst}-\textsc{cop}  \textsc{aug}-2-person  2-assoc  old times  \textsc{aug}-2-ancestor\\
\glt  ‘There were the people of old times, the ancestors.’

\ex \gll {ba-a-li}  n=\textbf{ii-penenga}  \textbf{ɪ-ndingala}  {ɪ-jɪ}  {j-aa-job-igw-aga}\\
  2-\textsc{pst}-\textsc{cop}  \textsc{com}=5-type.of.drum  \textsc{aug}-9-drum  \textsc{aug}-prox  9-\textsc{pst}-speak-\textsc{pass}-\textsc{ipfv}\\
\glt  ‘They had the drum.’ \citep[352]{Persohn2017}
\z
\z

Irrespective of the presence or absence of the V-aug\-ment, the texts in \citet{Persohn2017} are provided with definite and indefinite readings. This entails that the definite and indefinite interpretations of the sentences require some context to be used. For example, the sentence in \REF{ex:lusekelo:16} can be interpreted as indefinite or definite. 

\ea%16
    \label{ex:lusekelo:16}
\gll   ʊ{}-malafyale  a-fw-ele    ɪ{}-ngiga \\
  \textsc{aug}-1.chief  \textsc{sm}1-wear-\textsc{pfv}  \textsc{aug}-9.crown    \\
\glt  ‘a/the chief wears a/the crown.’
\z

The sentence in \REF{ex:lusekelo:16} will be interpreted as indefinite if interlocutors are unaware of the chief and crown. However, if the chief is known by interlocutors, a definite reading obtains. Likewise, once the crown is known by interlocutors, a definite reading obtains. 

The use of demonstratives is paramount in drawing the distinction between definite and indefinite readings. \citet{Lyons1999} pointed out that the anaphoric use of demonstratives provides a definite reading, while the referential use of demonstratives gives a deictic interpretation of nouns. In Bantu languages, scholars have shown that the anaphoric use of demonstratives is linked to definiteness and focus (\citealt{Visser2010, Asiimwe2014, Ndomba2017, Kimambo2018a}). The Nyakyusa text in \REF{ex:lusekelo:18} shows the use of demonstratives to indicate definiteness. 

\ea%18
    \label{ex:lusekelo:18}
    \gll {ʊ-n-nyambala}  {jʊ-mo}  {a-fik-ile}  {ku-buhesya.}  {a-ba-ag-ile} {a-ba-fwimi.}  {a-ba-fwimi}  {ba-la}  {ba-li-mw-amb-ile} {ʊ-n-nyambala}  {jʊ{}-la.}\\
  \textsc{aug-1}{}-man  1-one  \textsc{sm1}{}-arrive-\textsc{pfv}  17-new-land  \textsc{sm1-om2}{}-find-\textsc{pfv} \textsc{aug-2-}hunter  \textsc{aug-2-}hunter  2-those  \textsc{sm2-\textsc{pst}-om1}{}-invite-\textsc{pfv}    \textsc{aug-1}{}-man  1-that  \\
  \glt ‘A certain man arrived at a new country. He found hunters. The hunters   invited the man.’  
\z

The data in \REF{ex:lusekelo:18} above shows that the augment occurs in the indefinite noun \textit{ʊnnyambala} \textit{jʊmo} ‘certain man’. Also, it occurs in definite nouns \textit{ʊnnyambala} \textit{jʊla} ‘that man’. However, it is the use of anaphoric demonstratives \textit{bala} ‘those’ and \textit{jʊla} ‘that’ which provides definite readings. \citet{Lyons1999} states that anaphoric demonstratives tend to focus the referent. In this case, we obtain definite and focused nouns in Nyakyusa. 

The prenominal demonstratives are also anaphoric in Bantu languages (\citealt{Visser2008, Kimambo2018a}). Nyakyusa allows prenominal demonstratives, as shown in \REF{ex:lusekelo:19}. Usually, the prenominal demonstrative occurs in complementary distribution with the augment \citep{Lusekelo2009}. Thus, either the V-aug\-ment or prenominal demonstratives indicate definiteness.  

\ea%19
    \label{ex:lusekelo:19}
\gll {jʊ-la}   {n-nyambala}  {a-ba-buul-ile}  {ba-la}  {ba-fwimi}\\
  \textsc{1}{}-that  1-man  \textsc{sm1-om2}{}-tell-\textsc{pfv}  2-those  2-hunter\\
\glt  ‘The man informed the hunters.’
\z

The role of the augment in Dzamba, Luganda, Nyankore-Kiga and Xhosa is directly related to the function of articles in English, French and Hungarian. Its presence entails a definite reading. This is not the case for Nyakyusa. Rather, the prenominal demonstrative and V-aug\-ment occur in complementary distribution, which is an indication that focused nouns would not require the V-aug\-ment. 

In the context that provides an indefinite reading, the V-aug\-ment is not required, as exemplified in \REF{ex:lusekelo:20a}. However, in a context that provides a definite reading, the V-aug\-ment is required, as shown in \REF{ex:lusekelo:20b}. 

\ea%20
    \label{ex:lusekelo:20}
   \ea\label{ex:lusekelo:20a} \gll {u-pimb-ile}      {ma-toki}   {ma-ki?}\\
    \textsc{sm2sg-}carry-\textsc{pfv}  6-banana  6-what\\
    \glt ‘What kind of bananas are you carrying?’

  \ex\label{ex:lusekelo:20b} \gll  {u-pimb-ile}      {a-ma-toki}     {ga-liku?}\\
    \textsc{sm2sg-}carry-\textsc{pfv}  \textsc{aug}-6-banana  6-what\\
  \glt  ‘Which bananas are you carrying?’

  \ex\label{ex:lusekelo:20c} \gll {m-bimb-ile}       {ma-toki}     {ma-bifwe}\\
    \textsc{sm1sg-}carry-\textsc{pfv}     6-banana    6-ripe\\
  \glt  ‘I am carrying ripe bananas.’

  \ex\label{ex:lusekelo:20d} \gll {m-bimb-ile}       {a-ma-toki}     {a-ma-bifwe}\\
    \textsc{sm1sg-}carry-\textsc{pfv}     \textsc{aug-}6-banana    \textsc{aug-}6-ripe \\
  \glt  ‘I am carrying the ripe bananas.’ 
    \z
\z

The V-aug\-ment is not required in interrogatives \REF{ex:lusekelo:20a}, although it is attested in this context as well \REF{ex:lusekelo:20b}. Both are indefinite contexts. In the case of an indefinite answer, no V-aug\-ment appears as in \REF{ex:lusekelo:20c}. In contrast, the V-aug\-ment is typically (although not obligatorily) required in definite contexts \REF{ex:lusekelo:20d}. The context of speech also provides clues to definiteness.  

\subsection{CV-particle in the nominal domain in Nyakyusa}\label{sec:lusekelo:2.3}

The discussion about the presence of the CV-aug\-ment in Nyakyusa begins with \citet[93]{DeBlois1970}. He reported that the reconstruction of the augment in Proto-Bantu included a CV-shape, mainly *\textit{ga}{}-, *\textit{ba}{}-, *\textit{j\k{i}}{}- and *\textit{j\k{u}}{}- as reported in \citet{Meeussen1967}. The interpretation of nouns preceded by a CV-aug\-ment is “with definite meaning” \citep[99]{Meeussen1967}. The reconstructed proto-words include *\textit{j\k{u} muntu} ‘the person’, *\textit{ba} \textit{bantu} ‘the persons’, *\textit{ji} \textit{mbúa} ‘the dog’ and *\textit{s\k{i} mbúa} ‘the dogs’ (\citeyear[99]{Meeussen1967}). In many Bantu languages, the CV-aug\-ment was reduced to a V-aug\-ment, except a few languages (including Nyakyusa) whose CV-aug\-ment is restricted to emphasized nouns (\citealt[92--93]{DeBlois1970}).    

\Citet{DeBlois1970} shows that Nyakyusa possesses the CV-aug\-ment, and that it is still in use alongside the normal V-aug\-ment. He notes that “the CV-type is used instead of the V-type to give more prominence or emphasis to the noun” (\citealt[98]{DeBlois1970}). This means that when the CV-aug\-ment is used, it shows a special meaning of emphasis, as illustrated by \textit{jʊ{}-mʊ{}-ndʊ} ‘the very man’ and {\textit{g}}\textit{a-ma-heelu} ‘only terms of abuse’.

\citet[44]{Persohn2017} separated the CV-aug\-ment from the V-aug\-ment. With regard to the former, he suggested that “nouns carrying a pronominal prefix instead of an augment, express an emphatic notion translatable along the lines of ‘the very X’ as in \textit{lʊ{}-lw-ala} ‘the very grindstone’ and \textit{gʊ{}-n-tʊ} ‘the very head’” (\citeyear[44]{Persohn2017}). 

The data in \REF{ex:lusekelo:21} show that the CV-aug\-ment is present in all 18-noun classes in Nyakyusa. Even the locative classes 16, 17 and 18 bear the CV-aug\-ment. Note that the morphology of the CV-aug\-ment is represented in \tabref{tab:lusekelo:1}.    

\ea%21
    \label{ex:lusekelo:21}
    \begin{tabbing}
    cl. 16 \= \textit{pa{}-pa-sʊkʊʊlʊ} \= ‘at the very school’ \kill
  cl. 1 \> \textit{jʊ{}-m-puuti} \> ‘only the priest’\\
  cl. 2 \> \textit{ba-ba-ndʊ} \> ‘the very persons’         \\
  cl. 3 \> \textit{gʊ-m-papaju} \> ‘the very paw paw tree’  \\
  cl. 4 \> \textit{gɪ{}-mɪ-papaju} \> ‘the very paw paw trees’ \\
  cl. 5 \> \textit{lɪ{}-lɪ-lasi} \> ‘only the bamboo tree’ \\
  cl. 6 \> \textit{ga{}-ma-lasi} \> ‘only the bamboo trees’\\
  cl. 7 \> \textit{kɪ{}-kɪ{}-kota} \> ‘the very chair’ \\
  cl. 8 \> \textit{fɪ{}-fɪ{}-kota} \> ‘the very chairs’ \\
  cl. 9 \> \textit{jɪ{}-mbwa} \> ‘only the dog’ \\
  cl. 10 \> \textit{sɪ{}-mbwa} \> ‘only the dogs’ \\
  cl. 11 \> \textit{lʊ-lʊ-kama} \> ‘only the milk’\\
  cl. 12 \> \textit{ka{}-ka-kʊkʊ} \> ‘only the chick’ \\
  cl. 13 \> \textit{tʊ-tʊ-kʊkʊ} \> ‘only the chicks’\\
  cl. 14 \> \textit{bʊ-bʊ-ndʊ} \> ‘only the humanity’\\
  cl. 15 \> \textit{kʊ-kʊ-kuuta} \> ‘only to cry’ \\
  cl. 16 \> \textit{pa{}-pa-sʊkʊʊlʊ} \> ‘at the very school’ \\
  cl. 17 \> \textit{kʊ-kʊ-kaja} \> ‘to the very home’\\
  cl. 18 \> \textit{mʊ-mʊ-supa} \> ‘in the very bottle’
\end{tabbing}
\z


However, what is called the CV-aug\-ment is not attached to the lexical noun; rather it occurs as a separate element, as correctly suggested by \citet{Persohn2017}. This CV-particle derives from proximal demonstratives, as shown in \tabref{tab:lusekelo:1}.  

\begin{table}
\begin{tabular}{lllllll}
\lsptoprule
Cl. & \multicolumn{3}{l}{Proximal demonstrative} & \multicolumn{3}{l}{CV-aug\-ment}\\
\midrule
{1} & {\itshape ʊmʊndʊ} & {\itshape ʊjʊ} & {this person} & {\textit{jʊ}} & {\itshape mʊndʊ} & {the person}\\
{2} & {\textit{abandʊ}} & {\itshape aba} & {these persons} & {\itshape ba} & {\itshape bahesya} & {the persons}\\
{3} & {\itshape ʊmwenda} & {\itshape ʊgʊ} & {this dress} & {\textit{gʊ}} & {\itshape mpiki} & {the tree}\\
{4} & {\textit{ɪmɪpiki}} & {\textit{ɪgɪ}} & {these trees} & {\textit{gɪ}} & {\itshape mipiki} & {the trees}\\
{5} & {\itshape ɪlyabi} & {\itshape ɪlɪ} & {this pant} & {\textit{lɪ}} & {\itshape lyabi} & {the pant}\\
{6} & {\itshape amaabi} & {\itshape aga} &  these pants & {\itshape ga} & {\itshape maabi} & {the pants}\\
{7} & {\itshape ɪkɪtili} & {\textit{ɪkɪ}} & {this cap} & {\textit{kɪ}} & {\itshape kitili} & {the cap}\\
{8} & {\itshape ɪfɪtili} & {\itshape ɪfɪ} & {these caps} & {\textit{fɪ}} & {\itshape fitili} & {the caps}\\
{10} & {\textit{ɪmbwa}} & {\itshape ɪjɪ} & {this dog} & {\textit{jɪ}} & {\itshape mbwa} & {the dog}\\
{10} & {\textit{ɪmbwa}} & {\textit{ɪsɪ}} & {these dogs} & {\textit{sɪ}} & {\itshape mbwa} & {the dogs}\\
{11} & {\textit{ʊlʊkili}} & {\itshape ʊlʊ} & {this stick} & {\itshape lʊ} & {\textit{lʊkili}} & {the stick}\\
{12} & {\itshape akakuku}  & {\itshape aka} & {this chick} & {\itshape ka} & {\itshape kakuku}  & {the chick}\\
{13} & {\textit{ʊtʊkuku}} & {\textit{ʊtʊ}} & {these chicks} & {\itshape tu} & {\itshape tukuku} & {the chicks}\\
{14} & {\textit{ʊbʊtatu}} & {\itshape ubu} & {this trinity} & {\textit{bʊ}} & { \textit{bʊtatu}} &  the trinity\\
{15} & {\itshape ukulima} & {\textit{uku}} &  this farming & {\textit{kɪ}} & {\itshape kulima} & {the farming}\\
{16} & {\itshape pakaja} & {\itshape apa} & {this homestead} & {\itshape pa} & {\itshape pakaja} & {at the homestead}\\
{17} & {\itshape kukaja} & {\itshape aka} & {that homestead} & {\itshape ka} & {\itshape kukaja} & {to the homestead}\\
{18} & {\itshape ntwaja} & {\textit{ʊmo}} &  in homestead & {\textit{mʊ}} & {\itshape ntwaja} & {in the homestead}\\
\lspbottomrule
\end{tabular}
\caption{Proximal demonstrative and CV-particle}
\label{tab:lusekelo:1}
\end{table}

The prenominal demonstrative and the CV-particle occur in complementary distribution. Example \REF{ex:lusekelo:22} shows the use of the V-aug\-ment, while the prenominal demonstrative restricts its occurrence, as shown in \REF{ex:lusekelo:23}. The CV-aug\-ment restricts the occurrence of the V-aug\-ment \REF{ex:lusekelo:24}, as well as the prenominal demonstrative \REF{ex:lusekelo:25}.   

\ea[]{%22
    \label{ex:lusekelo:22}
  \gll     {ɪ-mbeba}  sɪ{}-tafwine  ɪ-ndefu\\
  \textsc{aug}-10.rat  \textsc{sm}10-tear.\textsc{pfv}  \textsc{aug}-10.mat \\
\glt  ‘Rats damaged mats.’} 
\ex[]{%23
    \label{ex:lusekelo:23}
\gll  {ɪsɪ}  mbeba  sɪ{}-ta-fwine  ɪ-ndefu\\
  \textsc{dem}10  10.rat  \textsc{sm}10-tear.\textsc{pfv}  \textsc{aug}-10.mat \\
\glt  ‘(Specifically) the rats damaged mats.’}
\ex[]{%24
    \label{ex:lusekelo:24}
\gll  {sɪ}  {mbeba}  sɪ{}-ta-fwine  ɪ-ndefu\\
  \textsc{cv-10}  10.rat  \textsc{sm}10-tear.\textsc{pfv}  \textsc{aug}-10.mat \\
\glt  ‘(Specifically) the rats damaged mats.’}
\ex[*]{%25
    \label{ex:lusekelo:25}
\gll {sɪ}   {ɪsɪ}   {mbeba}  sɪ{}-ta-fwine  ɪ-ndefu\\
    \textsc{cv-10}  {\textsc{dem}10}  10.rat  \textsc{sm}10-tear.\textsc{pfv}  \textsc{aug}-10.mat\\
}
\z

Two theories can account for the definiteness marking. On the one hand, based on the theory of definiteness and specificity \citep{Lyons1999}, the examples above demonstrate that the CV-aug\-ment indicates typically a specific and focused referent in Nyakyusa. 

The theory of information structure, mainly contrastive focus (\citealt{Aboh2004, Repp2010}), can account for the role of the CV-aug\-ment and prenominal demonstrative. In example \REF{ex:lusekelo:26}, the CV-particle is used to focus the contrast of \textit{bwalwa} ‘alcohol’ from other foodstuff. It means that the alcoholics did not vomit anything else but alcohol. Likewise, the contrastive focus in example \REF{ex:lusekelo:27} indicates that {Tuntufye} wears nothing but \textit{kɪtili} ‘cap’. It means that {Tuntufye}, who is perhaps a child, had not put on any other articles of clothing.   

\ea%26
    \label{ex:lusekelo:26}
\gll {A-ba-leefi}  {ba-teek-ile}  {bʊ}        {bwalwa} \\
  \textsc{aug}-2-alcoholic  \textsc{sm}2-vomit-\textsc{pfv}  \textsc{aug}.14  14.alcohol \\
\glt  ‘The alcoholics vomited nothing but alcohol.’
\ex \label{ex:lusekelo:27}  
\gll {Tuntufye}  {a-fw-ele}  {kɪ}      {kɪ{}-tili} \\
  1.Tuntufye  \textsc{sm}1-wear-\textsc{pfv}  \textsc{aug}.7  7-cap  \\
\glt  ‘Tuntufye put on only a cap.’ 
\z

The tools of information structure help to analyse properly the emphatic purpose of the CV-aug\-ment highlighted by \citet{DeBlois1970} and \citet{Persohn2017}. It means that the CV-aug\-ment is used to provide contrastive focus in which of the many alternative referents, it foregrounds one. 

\section{Configuration and concord in complex noun phrases} \label{sec:lusekelo:3}

\citet{Rijkhoff2002} highlights the fact that complex noun phrases contain the lexical noun with its dependents. The dependents of the lexical noun include articles, demonstratives, adjectives, numerals etc. \citep{AlexiadouEtAl2007}. Both articles and demonstratives are used to indicate definiteness \citep{Lyons1999}.  

Given this backdrop, two points are discussed in this section. Firstly, I investigate whether the (non-)occurrence of the augment on adjectives has implications for the (in)definite interpretation in Nyakyusa. In Nyankore-Kiga and Rutooro, the occurrence of the augment on the adjectives indicates a definite reading, and its absence indicates indefinite interpretation (\citealt{Kaji2009, Asiimwe2014}).   

In Nyankore-Kiga, example \REF{ex:lusekelo:29} shows an adjective \textit{murungi} ‘good’ without an augment, indicating indefiniteness, while example \REF{ex:lusekelo:30} shows that an adjective \textit{umurungi} ‘good’ has an augment which provides definiteness \citep{Asiimwe2014}. Similarly, Rutooro provides an indefinite interpretation for the adjective without the augment \REF{ex:lusekelo:31}, and a definite reading for the adjective with the augment \REF{ex:lusekelo:32} \citep{Kaji2009}.   

\ea%29
    Nyankore-Kiga \citep[120]{Asiimwe2014}\label{ex:lusekelo:29}\\
\gll {o-mu-shaija}  {mu-rungi} \\
  \textsc{aug}-1-man  1-good  \\ 
  \glt ‘a good man’  
\ex%30
    \label{ex:lusekelo:30}
\gll {o-mu-shaija}  {u-mu-rungi} \\
  \textsc{aug}-1-man  \textsc{aug}-1-good    \\
\glt  ‘the good man’
\ex%31
   Rutooro \citep[246]{Kaji2009}\label{ex:lusekelo:31}\\
\gll {e-ki-tábu}  {ki-ríngi}\\ 
  \textsc{aug}-7-book  7-good \\
\glt  ‘a good book’ 
\ex%32
    \label{ex:lusekelo:32}
\gll   {e-ki-tábu}  {e-ki-ríngi} \\
  \textsc{aug}-7-book  \textsc{aug}-7-good    \\
\glt  ‘the good book’
\z

Secondly, I investigate further whether the co-occurrence of the augment and demonstratives overlap in the marking of definiteness in Nyakyusa. In some Bantu languages, the prenominal demonstrative, which provides anaphoric reference to the noun (\citealt{Rugemalira2007, PetzellKühl2017, Kimambo2018a}), restricts the augment occurring on the noun (\citealt{Visser2002, VandeVelde2005}). 

In Xhosa, the canonical order is demonstrative > noun, which provides a deictic reading. The augment does not occur in the nouns in \REF{ex:lusekelo:33}, while example \REF{ex:lusekelo:34} shows the occurrence of the augment and provides a definite reading \citep{Visser2002}. 

\ea%33
   Xhosa \citep[287]{Visser2002} \label{ex:lusekelo:33}\\
\gll {lo}  {m-fana}  {m-de} \\
  this  1-young man  1-tall \\ 
\glt  ‘this tall young man’ 
\ex%34
    \label{ex:lusekelo:34}
\gll {o-m-fana}  {o-m-de}\\
  \textsc{aug}-1-young man  \textsc{aug}-1-tall\\
\glt  ‘the tall young man’
\z

\citet{Lusekelo2009} showed that the prenominal demonstrative restricts the augment in Nyakyusa. In addition, as highlighted in section 2.2, both the augment and prenominal demonstrative have implications for the marking of definiteness in Nyakyusa. In this section, I discuss the role of the demonstrative and adjective in marking definiteness. 

Nyakyusa reveals this pattern of noun phrase order: (\textsc{dem}) > \textit{noun} > \textsc{dem} > \textsc{adj} \citep{Lusekelo2009}. The example in \REF{ex:lusekelo:35} shows the word-order N > \textsc{dem} > \textsc{num}, while example \REF{ex:lusekelo:36} shows the word-order N > \textsc{dem} > \textsc{adj}. These examples confirm the canonical pattern of deictic referent nouns in Nyakyusa. 

\ea%35
    \label{ex:lusekelo:35}
\gll  {Tu-ku-songol-a}    \textbf{ɪ{}-fɪ{}-kota}    \textbf{ɪfɪ}  \textbf{fɪ{}-hano}\\
  \textsc{sm}1\textsc{pl}-\textsc{pres}-curve-\textsc{fv}  \textsc{au}-8-chair    \textsc{dem}8  8-five    \\
\glt  ‘We curve these five chairs.’
\ex \label{ex:lusekelo:36}
\gll \textbf{a-ba-ana-ngu}  \textbf{aba}  \textbf{a-ba-tali} {ba-fik-ile} {ʊlʊ}\\
  \textsc{aug}-2-child-\textsc{poss}1\textsc{sg}  2.\textsc{dem}  \textsc{aug}-2-tall  \textsc{sm}2-arrive-\textsc{pfv}   now\\
\glt  ‘These tall children of mine have just come now.’
\z

The word order \textsc{dem} > \textit{noun} > \textsc{adj} provides a definite interpretation. In fact, the prenominal demonstrative, which indicates contrastive focus, restricts the augment from occurring on the noun, as shown in (\ref{ex:lusekelo:37}--\ref{ex:lusekelo:38}). Given this pattern, both the augment and demonstrative function to indicate the definite determiner in the nominal domain of Nyakyusa. 

\ea\label{ex:lusekelo:37}
\gll {aba}  {ba-ana-ngu}  {a-ba-tali} {ba-fik-ile} {ʊlʊ}\\
  2.\textsc{dem}  2-child-\textsc{poss}1\textsc{pl}  \textsc{aug}-2-tall  \textsc{sm}2-arrive-\textsc{pfv}  now\\
\glt  ‘The two tall children of mine have just come now.’
\ex\label{ex:lusekelo:38}
\gll  aga {ma-boko}    ma-bili {ga-nya-lile}\\
  6.\textsc{dem}  6-hand    6-two    \textsc{sm}6-be.dirty-\textsc{pfv}\\
\glt  ‘The two arms have become unclean.’
\z

In the complex noun phrase, the occurrence of the augment on adjectives is not restricted, as shown in \REF{ex:lusekelo:37} above. However, the numeral does not host the augment \REF{ex:lusekelo:38}. 

The anaphoric use of the demonstrative is also determined by the shape in Bantu languages. \citet[60]{Mwamzandi2014} highlights that the demonstrative -\textit{le} ‘that\slash those’ provides deictic and anaphoric references when introducing the topic referent in Swahili. The word order \textit{noun} > \textsc{dem} introduces an activated and recent referent \REF{ex:lusekelo:39}, while the word order \textsc{dem} > \textit{noun} reintroduces the inactive referent into current discourse \REF{ex:lusekelo:40}. 

\ea%39
    \label{ex:lusekelo:39}
\gll {M-sichana}  {yule}  {a-li-ingi-a} \\
  1-girl     1-\textsc{dem}  \textsc{sm}1-\textsc{pst}-inter-\textsc{fv}  \\
\glt  ‘That girl entered.’
\ex%40
    \label{ex:lusekelo:40}
\gll  {Yu-le}   {m-sichana}  {a-li-ingi-a}   \\
  1-\textsc{dem}  1-girl    \textsc{sm}1-\textsc{pst}-enter-\textsc{fv}  \\
\glt  ‘That girl entered.’
\z

The anaphoric demonstraitve in Nyakyusa is -\textit{la} ‘that/those’. It occurs in all noun classes, as exemplified below. In example \REF{ex:lusekelo:41}, the anaphoric demonstrative is \textit{gʊla} ‘that’ for class 3, while example \REF{ex:lusekelo:42} contains the anaphoric \textit{gala}~‘those’ for class 6. In example \REF{ex:lusekelo:43}, the anaphoric demonstrative is for noun class 2. 

\ea\label{ex:lusekelo:41}
\gll {pa-kyinja}  {ba-a-gʊ{}-bwene}    {ʊ-mw-esi} {gʊ-la}\\
  16-year  \textsc{sm}2-\textsc{pst}-\textsc{om}6-see.\textsc{pfv}  \textsc{aug}-6-moon  3-that\\
\glt  ‘Last year, they saw that moon.’
\ex%42
    \label{ex:lusekelo:42}
\gll  {a-ma-gali}    {ga-la}   {go{}-onangike} lɪlɪno\\ 
  \textsc{aug}-6-car  6-those  \textsc{sm}6-wreck  today    \\
\glt  ‘Those cars have wrecked today.’
\ex%43
    \label{ex:lusekelo:43}
\gll  {a-ba-ndu}    {ba-la}  {ba-pimb-ile}   {i-ly-afuli}\\
  \textsc{aug}-2-person   2-{\textsc{dem}}   \textsc{sm2-}carry-\textsc{{pfv}}   \textsc{aug-5-}umbrella\\
\glt  ‘Those people carry umbrellas.’ 
\z

The canonical word order of noun > \textsc{dem} is also attested for the anaphoric demonstratives, as illustrated above. However, it also occurs in prenominal position, as shown in (\ref{ex:lusekelo:44}--\ref{ex:lusekelo:45}). The NPs \textit{bala} \textit{bandʊ} ‘the very people’ and \textit{lila} \textit{lɪlasi} ‘the very bamboo tree’ have the word order \textsc{dem} > noun. 

\ea%44
    \label{ex:lusekelo:44}
\gll  \textbf{bala}   \textbf{ba-ndʊ}  {ba-ø-fik-ile}              \\
  2.those  2-person  \textsc{sm}2-\textsc{pst}-arrive-\textsc{pfv}  \\
\glt  ‘The (expected/known) people have arrived.’
\ex%45
    \label{ex:lusekelo:45}
\gll  {Atu}   {a-tem-ile}    \textbf{lila}   \textbf{lɪ{}-lasi}  \\
  1.Atu  \textsc{sm}1-cut-\textsc{pfv}  5.that   5-bamboo     \\
\glt  ‘Atu has cut the (exact) bamboo tree.’
\z

At this juncture, two points are underscored here. On one hand, the presence of the augment may indicate both the indefinite and definite referents in Nyakyusa. And the absence of the augment indicates clefting, which is associated with focus. Any referent in focus is foregrounded; therefore, the augment cannot occur on the noun. This is confirmed by its complementary distribution with the prenominal demonstrative, which indicates definite and focused nouns \citep{Lyons1999}. Therefore, the augment in Nyakyusa behaves like a typical determiner. 

Word order is not the primary mechanism to indicate the definite interpretation of referents in Nyakyusa. The anaphoric demonstrative --\textit{la} ‘that/those’ provides two kinds of definiteness: (i) the word order noun > \textsc{dem} is associated with a deictic referent which is active in the discourse; (ii) the pattern \textsc{dem} > noun is related to the introduction of an inactive referent which is referred to in the current situation. 

Based on the observations above, the occurrence or non-occurrence of the augment cannot yield definiteness independent of the context. The proper identification of indefiniteness has to be determined by the context of speech, whether signalling active or inactive referents, as also discussed by \citet{Mwamzandi2014}.  

{Since the position of the determiner is now identified, I suggest the structure of the Nyakyusa nominal domain in \REF{ex:lusekelo:46}. The augment, the prenominal demonstrative, and the postnominal demonstrative are all indicators of definiteness.} 

\ea \label{ex:lusekelo:46} 
\gll (determiner)  \textit{noun}  (determiner)  (modifiers)\\
  {(\textsc{dem}) (\textsc{aug})}  {}  (\textsc{dem})  {(\textsc{adj}) (\textsc{num})}\\
\z

{However, the postnominal demonstrative occurs in complementary distribution with the possessive, which is another category that indicates definiteness \citep{Lyons1999}. In Nyakyusa, the possessives occur postnominally and do not host the augment, as illustrated in (\ref{ex:lusekelo:47}--\ref{ex:lusekelo:48}). In these sentences, the possessives provide a specificity interpretation of the referents.} 

\ea%47
    \label{ex:lusekelo:47}
\gll   ʊ-m-piki  gʊ-ake   gʊ-mel-ile\\
  \textsc{aug}-3-tree  3-his  \textsc{sm}3-grow-\textsc{pfv}  \\
\glt  ‘His tree has grown.’
\ex%48
    \label{ex:lusekelo:48}
\gll  ɪ{}-mɪ{}-piki    gɪ{}-ake   gɪ{}-mel-ile\\
  \textsc{aug}-4-tree  4-his  \textsc{sm}4-grow-\textsc{pfv}  \\
\glt  ‘His trees have grown.’
\z

The co-occurrence of possessives and demonstratives is permitted in Nyakyusa \citep{Lusekelo2009}. In example \REF{ex:lusekelo:49}, the demonstrative precedes the possessive, while in example \REF{ex:lusekelo:50}, the possessive precedes the demonstrative. These examples substantiate the fact that both possessives and demonstratives occur in the determiner slot in the nominal domain.  

\ea\label{ex:lusekelo:49}
\gll {a-ba-ana}  {aba}  {ba-angu}  {ba-a-fik-ile}\\
  \textsc{aug}-2-child  2.these  2-mine  \textsc{sm}2-\textsc{pst}-arrive-\textsc{pfv}\\
\glt  ‘These children of mine arrive.’
\ex\label{ex:lusekelo:50}
\gll {a-ba-ana}  {ba-angu}  {aba}  {ba-a-fik-ile}\\
  \textsc{aug}-2-child  2-mine  2.these  \textsc{sm}2-\textsc{pst}-arrive-\textsc{pfv}\\
\glt  ‘These children of mine arrive.’
\z

Given the fact that both demonstratives and possessives occur in postnominal position, I propose the structure of the Nyakyusa nominal domain in \REF{ex:lusekelo:51}. Since the augment is introduced, this structure differs from the one suggested in \citet{Lusekelo2009}.

\ea\label{ex:lusekelo:51}
\gll (determiner)  \textit{noun}  (determiner)  (modifiers)\\
  {(\textsc{dem}) (\textsc{aug})}  {}  {(\textsc{dem}) (\textsc{poss})}  {(\textsc{adj})(\textsc{num})}\\
\z

\section{Object marking and the status of the augment in bare nouns} 
\label{sec:lusekelo:4}
Object marking relates to the marking of (in)definiteness and (non-)specificity in Bantu languages (\citealt{MartenEtAl2007, Riedel2009, MartenKula2012, Visser2010, Kimambo2018b}). For instance, \citet[242]{MartenKula2012} argue that the use of the object marker with non-animate NPs is associated with definiteness or specificity in Swahili. The example in \REF{ex:lusekelo:52} demonstrates an indefinite/non-specific reading with no object marker in the verb, while the example in \REF{ex:lusekelo:53} demonstrates a definite/specific reading because of the object prefix. 

\ea%52
    \label{ex:lusekelo:52}
    Swahlii, \citet[242]{MartenKula2012}\\
\gll {Ni-li-on-a}  {ki-tabu} \\
  \textsc{sm1-pst}{}-see-\textsc{fv}  7-book \\
\glt  ‘I saw a/the book.’ 
\ex%53
    \label{ex:lusekelo:53}
\gll {Ni-li-\textbf{ki}-on-a}       {ki-tabu} \\
  \textsc{sm1-pst-}{\textsc{om7}}{}-see-\textsc{fv}    7-book \\
\glt  ‘I saw the book.’ 
\z

\citet{Visser2010} argues that object marking and occurrence of the V-aug\-ment yield definite and specific readings. The example \REF{ex:lusekelo:54} shows a non-specific reading of \textit{ngubo} ‘blanket’ because there is no object prefix and no V-aug\-ment on the noun. But the example in \REF{ex:lusekelo:55} shows that the object prefix and V-aug\-ment indicate a specific reading of the object noun \textit{ingubo} ‘blanket’.  

\newpage
\ea%54
    Xhosa \citep[302]{Visser2010}\label{ex:lusekelo:54}\\
\gll  {\textbf{I}-intombi}    {a-zi-hlambi}    {ngubo}\\ 
  \textsc{au}{}-10.girl  \textsc{neg}-\textsc{sm}10-\textsc{pres}-wash  9.blanket\\ 
\glt  ‘(The) girls do not wash any (not specific) blanket.’ 
\ex%55
    \label{ex:lusekelo:55}
 \gll   {\textbf{I}-intombi}    {a-zi-yi-hlambi}    {\textbf{i}-ngubo}   \\
  \textsc{au}{}-10.girl  \textsc{neg}-\textsc{sm}10-\textsc{om}9-wash  \textsc{aug}{}-9.blanket    \\
\glt  ‘The girls do not wash the (specific) blanket.’
\z

However, the (in)definite interpretation is not always associated with object marking for some Bantu languages without augments. \citet{Riedel2009} offers cases from Shambala and Swahili. The example \REF{ex:lusekelo:56} confirms that in Shambala “object marking with an inherently definite noun phrase is optional” and example \REF{ex:lusekelo:57} shows that in Swahili “definite readings are available without object marking” \citep[49]{Riedel2009}.  

\ea%56
   Shambala \citep[49]{Riedel2009} \label{ex:lusekelo:56}\\
\gll  {N-za-(mw)-ona}    {ng’wanae}\\   
  \textsc{sm}1-\textsc{pfv}.\textsc{dj}-\textsc{om}1-see    1child.\textsc{poss}.3\textsc{s}\\ 
\glt ‘I saw his child.’
\ex%57
   Swahili \citep[49]{Riedel2009}  \label{ex:lusekelo:57}\\
\gll   {Ni-li-penda}  {sana}  {ki-tabu}    {chake}  {cha}   {kwanza}\\
\textsc{sm}1-\textsc{pst}-like  much   7-book    7.her  7\textsc{ass}. first\\ 
\glt ‘I liked her first book a lot.’
\z

Object marking in Nyakyusa is determined by the lexical semantics of the verb \citep{Lusekelo2012}, similar to Chiyao and Luguru (\citealt{MartenRamadhani2001, Taji2020verb}). Some verbs take compulsory object prefixes, while other kinds of verbs take optional object markers. 

Some of the verbs taking compulsory object prefixes include \textit{bona} ‘see’ (\ref{ex:lusekelo:58}--\ref{ex:lusekelo:59}) and \textit{bʊʊla} ‘inform’ \REF{ex:lusekelo:60}. Notice also that the augment occurs on the lexical object nouns \textit{abahesya} ‘guests’, \textit{ɪkɪkota} ‘chair’ and \textit{abakipanga} ‘congregation'. 

\ea%58
    \label{ex:lusekelo:58}
\gll  ʊ{}-lʊgano   a-a-*(ba)-bwene    a-ba-hesya  mmajolo\\
  \textsc{aug}-1.Lugano   \textsc{sm}3-\textsc{pst}-\textsc{om}2-see.\textsc{pfv}  \textsc{aug}-2-guest  yesterday \\
\glt  ‘Lugano saw (the) guests yesterday.’ 
\ex%59
    \label{ex:lusekelo:59}
\gll  ʊ{}-lʊgano  \textbf{a-*(kɪ)}-bwene  (ɪ{}-kɪ{}-kota)\\
  \textsc{aug}-1.Lugano  {\textsc{sm}1}{}-{\textsc{om}7}{}-see.\textsc{pfv}   \textsc{au}-7-chair\\
\glt ‘Lugano saw a/the chair.’ 
\ex%60
    \label{ex:lusekelo:60}
\gll  {ʊ-m-pʊʊtɪ}  {a-*(\textbf{ba})}{-bʊʊl-ile}  {a-ba-kipanga}.\\
  \textsc{ppx}{}-1-priest  \textsc{sm}{}-{\textsc{om2}}{}-tell-\textsc{pfv}  \textsc{ppx}{}-2-congregation\\
\glt  ‘A/the priest told (the) congregation.’
\z

Other verbs do not require an object prefix, as illustrated by the verb \textit{piija} ‘cook’ \REF{ex:lusekelo:61} and \textit{bɪɪka} ‘put’ \REF{ex:lusekelo:62}. Notice also that the lexical nouns \textit{ɪ{}fɪ{}ndu} ‘food’ and \textit{ɪndeko} ‘pot’ host the augment. Object marking in this type of verbs would mark definite referents. 

\ea%61
    \label{ex:lusekelo:61}
\gll  ʊ{}-lʊgano    a-piij-ile  ɪ{}-fɪ{}-ndu\\
  \textsc{aug}-1.Lugano  \textsc{sm}1-cook-\textsc{pfv}   \textsc{aug}-8-food\\
\glt ‘Lugano has cooked some food.’ 
\ex \label{ex:lusekelo:62}
\gll {ba-a-bɪɪk-ile}  ɪ{}-ndeko  kʊ-sofu\\
  \textsc{sm}2-\textsc{pst}-put-\textsc{pfv}  \textsc{aug}-10.pot  17-inner-room \\
\glt  ‘They put pots in the inner room.’ 
\z

In examples (\ref{ex:lusekelo:63}--\ref{ex:lusekelo:64}), object marking is associated with a definite interpretation. The object prefix is therefore used to signal definite referents, and the augment still occurs in both lexical object nouns \textit{abaana} ‘children’ and \textit{ɪndeko} ‘pots’. 

\ea%63
    \label{ex:lusekelo:63}
\gll   ʊ{}-lʊgano    a-ba-piij-il-e    a-ba-ana    ɪ{}-fɪ{}-ndu\\
  \textsc{aug}-1.Lugano  \textsc{sm}1-\textsc{om}2-cook-\textsc{appl}-\textsc{pfv}   \textsc{aug}-2-child   \textsc{aug}-8-food\\
 \glt ‘Lugano has cooked for the children the food.’ 
\ex \label{ex:lusekelo:64} 
\gll {ba-a-si-bɪɪk-ile}  ɪ{}-ndeko  kʊ-sofu\\
  \textsc{sm}2-\textsc{pst}-\textsc{om}10-put.\textsc{pfv}  \textsc{aug}-10.pot  17-inner-room \\
\glt  ‘They put the pots in the inner room.’ \citep{Lusekelo2012}
\z

\largerpage
\section{Conclusions} \label{sec:lusekelo:5}

This chapter described the morphosyntactic properties of the augment in Nyakyusa. The bare nouns in Nyakyusa contain this morphology: \textsc{aug}+\textsc{ncp}+stem. The V-aug\-ment is prolific in the language. I showed that the augment cannot co-occur with the prenominal demonstrative because both have the same function of indication of definiteness. 

The nominal domain in Nyakyusa reveals the use of the CV-particle, previously called CV-aug\-ment (\citealt{DeBlois1970, Mbope2016, Persohn2017}). The CV-particle functions to indicate contrastive focus of the referent. As a result, it is in complementary distribution with both the proximal demonstrative and the V-aug\-ment. 

\begin{sloppypar}
The occurrence of the anaphoric demonstrative, realised as -\textit{la} ‘that/those’, yields the word orders noun > \textsc{dem} and \textsc{dem} > noun. The former word order introduces the deictic and active referent, while the latter introduces the inactive referent in discourse. The prenominal anaphoric demonstrative occurs in complementary distribution with the V-aug\-ment and CV-aug\-ment. This substantiates the fact that they realise one single function of indication of definiteness. 
\end{sloppypar}

The adnominal demonstrative which occurs in the postnominal position has similar functions as the possessives. \citet{Lyons1999} shows that possessives indicate definiteness. In the foregoing discussion, I showed that the nominal domain in Nyakyusa hosts the elements captured in \REF{ex:lusekelo:65}. 

\ea\label{ex:lusekelo:65}
\gll (determiner)  \textit{noun}  (determiner)  (modifiers)\\
  {(\textsc{dem}) (\textsc{aug})}   {} {(\textsc{dem}) (\textsc{poss})}  {(\textsc{adj}) (\textsc{num})}\\
\z

Morever, the object prefix is required in some verbs such as \textit{bona} ‘see’ and optional in other verbs such as \textit{piija} ‘cook’. As a result, it has no influence on the occurrence or non-occurrence of the augment in Nyakyusa. However, when it occurs optionally (in verbs like \textit{piija} ‘cook’), it yields a definite reading. 

\section*{Abbreviations}

%Glossing conventions follow the Leipzig Glossing Rules with the %following additions: 
\begin{multicols}{2}
\begin{tabbing}
1, 2, 3 etc. \= noun class number\kill
1, 2, 3 etc. \> noun class number\\
\textsc{adj} \> adjective\\
\textsc{aug} \> augment\\
\textsc{dem} \> demonstrative\\
\textsc{fv} \> (default) final vowel\\
\textsc{om} \> object marker\\
\textsc{ncp} \> noun class prefix\\
\textsc{num} \> numeral\\
\textsc{pfv} \> perfective aspect marker\\
\textsc{pl} \> plural\\
\textsc{poss} \> possessive\\
\textsc{prox} \> proximal\\
\textsc{pst} \> past tense marker\\
\textsc{sg} \> singular\\
\textsc{sm} \> subject marker\\  
\end{tabbing}
\end{multicols}



\printbibliography[heading=subbibliography,notkeyword=this]
\end{document} 
