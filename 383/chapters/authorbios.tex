\addchap{Author biographies}


\newcommand{\ORCID}[1]{{\itshape (ORCID \href{https://orcid.org/#1}{#1})}}
\subsubsection*{Allen Asiimwe \ORCID{0000-0003-1906-519X}}
Allen Asiimwe is a Senior Lecturer in the Department of African Languages, School of Languages, Literature and Communication at Makerere University in Uganda. Her research interests lie in morphology and syntax, and the interface between information structure and syntax in the Bantu language cluster of Ru-nyankore-Rukiga. She was one of the collaborators on the project on ‘Bantu Syntax and Information Structure’ (BaSIS) (\url{https://bantusyntaxinformationstructure.com/}) investigating the influence of information structure on the syntax of Rukiga (Bantu, JE 14). She also has a passion for documentation and preservation of cultural heritage of the Bakiga. She recently documented Rukiga personal names and conducted a linguistic analysis of these names. The ultimate goal of this project was to come up with an encyclopaedic collection of Rukiga personal names preserved for future reference, useful to scholars in linguistics, gender studies, and anthropology, but also to contribute to the debate on African onomastics.

\subsubsection*{Peter Edelsten}
Peter Edelsten is a Research Associate in Linguistics at SOAS University of London. He has a background in mathematics, agriculture, finance and linguistics. His principal linguistic publication is as co-author, with Chiku Lijongwa, of a \textit{Grammatical Sketch of Chindamba}, an endangered language of southern Tanzania (Cologne: Köppe, 2010). {Since 2016} he has provided mathematical and computing assistance to the project ‘Morphosyntactic variation in Bantu: Typology, contact and change’ housed in the Linguistics Department at SOAS under the direction of Lutz Marten, and contributed to publications generated by the project.

\subsubsection*{Hannah Gibson  \ORCID{0000-0003-2324-3147}}

Hannah Gibson is a Professor of Linguistics at the University of Essex. She holds a PhD in Linguistics from SOAS University of London. Before joining Essex, she was a Japan Society for the Promotion of Science postdoctoral researcher at Osaka University in Japan and a British Academy Postdoctoral Research Fellow at SOAS University of London. Her research examines linguistic variation with a focus on the morphosyntax of the Bantu languages of Eastern and Southern Africa. She also works on language contact, language change and multilingualism. Hannah is involved in a number of externally-funded research projects, including as Principal Investigator on the project ‘Grammatical variation in Swahili: contact, change and identity’ (2021-2025) and ‘Microvariation in youth language practices in Africa’ (2021-2024). She has carried out research in Tanzania, Kenya and South Africa.

\subsubsection*{Rozenn Guérois \ORCID{0000-0003-3602-8622}}
Rozenn Guérois is a Researcher at the laboratory LLACAN (Langage, Langues et Cultures d’Afrique) of the French National Centre for Scientific Research (CNRS), and an affiliated researcher at the Department of Linguistics of the University of KwaZulu-Natal (UKZN). She received her PhD from the University of Lyon 2 in 2015. Before joining the CNRS in 2020, she was a postdoctoral researcher at SOAS University of London (2015-2018) and Ghent University (2018-2020). Her research interests include the description of Bantu languages spoken in Mozambique, morphosyntactic variation, typology, and comparative and historical linguistics. She has recently obtained an ANR Grant (French National Research Agency) for the ‘OriKunda’ project (2023-2027). Since 2020, she has been a member of the committee of the journal \textit{Linguistique et Langues Africaines} (LLA).

\subsubsection*{Fabiola Hassan \ORCID{0000-0003-3407-7025}}
Fabiola F. Hassan is a Lecturer of Kiswahili and Bantu linguistics in the Department of Kiswahili at the University of Dodoma, Tanzania.  She holds a PhD from the University of Dar es Salaam. Her areas of specialization include phonology, morphology, syntax, semantics, and lexicography. Her doctoral thesis is titled \textit{Upinduzi wa Kimahali katika Lugha ya Kiswahili} (‘Locative Inversion in Swahili’). She has published a number of articles and books on historical and comparative linguistics, psycholinguistics, applied linguistics and descriptive linguistics. Currently her duties include teaching, research, and consultancy in Tanzania. She is interested in conducting research on Kiswahili and other Bantu languages.

\subsubsection*{Kyle Jerro  \ORCID{0000-0002-1467-2524}}
Kyle Jerro~is a Senior Lecturer at the University of Essex. They are interested in semantics, syntax, and the interfaces between syntax, semantics, and pragmatics. In particular, their focus has been on the question of possible verb meanings and how the meaning of a verb derives argument realization, especially on the semantics of valency changing morphology like applicatives and causatives in Bantu languages. They have also conducted research in the areas of aspect and temporal reference, information structure, the semantics of location and directed motion, reference and fictional worlds, and the psycho-semantic reality of noun classes in Bantu languages.~

\subsubsection*{Kulikoyela Kahigi}
Kulikoyela Kahigi is Associate Professor of Linguistics in the Department of Languages and Linguistics, Saint Augustine University of Tanzania (SAUT), Mwanza. Before moving to SAUT, he taught Swahili language and linguistics at the University of Dar es Salaam (UDSM) for many years. While at UDSM, he was, at various times, Chief Editor of the journals \textit{UTAFITI}, \textit{KIOO CHA LUGHA} and \textit{KISWAHILI}. He was also involved in the Microsoft Kiswahili localisation project as Chief Moderator and Language Expert (2004-8), and was part of the Management Committee, Research Team and Editorial Panel of the ‘Languages of Tanzania’ (LOT) project which published 20 dictionaries and the \textit{Atlasi ya Lugha za Tanzania} (‘Atlas of the languages of Tanzania’, Dar es Salaam: LOT, 2009). His current interests are in Bantu linguistics (synchronic and diachronic), discourse analysis, stylistics, and poetry. His publications include \textit{Sema Kiswahili: Furahia Tanzania} (Dar es Salaam: US Peace Corps Tanzania, 1997) as consultant and co-author, ‘Structural and cohesion dimensions of style: A considerations of some texts in  {Maw 1974}’ (\textit{Journal of Linguistics and Language Education} 3: 1-79, 1997), \textit{Misingi ya Kusoma} (‘Foundations of Reading’) (Dar es Salaam: Children’s Book Project Tanzania, 2003), \textit{Sumbwa-English-Swahili/English-Sumbwa-Swahili dictionary} (Dar es Salaam: LOT, 2008), \textit{Kahe-English-Swahili/English-Kahe-Swahili lexicon,} Dar es Salaam: LOT, 2008); \textit{Misingi ya Semantiki na Pragmatiki ya Kiswahili} (Dar es Salaam: KAUTTU, 2019). His most recent collections of poetry include \textit{Mageuzi} (Dar es Salaam: KAUTTU, 2019) and \textit{New Horizons} (Dar es Salaam: E \& D Vision Publishers, 2018).

\subsubsection*{Crisófia Langa da Câmara \ORCID{0000-0001-5786-991X}}
Crisófia Langa da Câmara is a researcher at the Department of Language and Communication of the Centro de Estudos Africanos at the Universidade Eduardo Mondlane, Mozambique. Her research focuses on the description of Mozambican Bantu languages, minority and endangered language documentation, morphology and syntax of Bantu languages, language policy, multilingualism and gender. Her research work also involves the development of teaching manuals and bilingual education teacher training materials. She has published one book, several journal articles and book chapters, and edited one book.

\subsubsection*{Amani Lusekelo \ORCID{0000-0001-6275-237X}}
Amani Lusekelo is an Associate Professor of Linguistics at the University of Dar es Salaam. He holds a PhD (African Languages and Literature) from the University of Botswana and MA (Linguistics) from University of Dar es Salaam in Tanzania. He teaches graduate courses covering the structure of African languages, contact linguistics, field methods, and advanced morphology. His research areas include the noun phrase, lexical borrowing, ethnobotany, linguistic landscape, and tense and aspect, focusing mainly on Akie, Datooga, Hadzabe, Nyakyusa, Nyamwezi, Sandawe, Sukuma and Swahili. {Since 2013}, he has supervised to completion ten doctoral candidates and currently has six ongoing PhD students at the University of Dar es Salaam and University of Dodoma. His selected authored and co-authored publications include \textit{Plant Nomenclature and Ethnobotany of the Hadzabe Society of Tanzania} (Cape Town: University of Western Cape, 2023), ‘Nominal morphology and syntax in Rwa-Meru’ (in Blasius Achiri-Taboh (ed.), \textit{The Bantu Noun Phrase: Issues and Perspectives}, 78-95. London: Routledge, 2023), ‘Portrayal of the COVID-19 pandemic in political cartoons in Tanzania’ (\textit{Cogent Arts and Humanities} 10(1), 2023), ‘The V and CV augment and exhaustivity in Kinyakyusa’ (\textit{Studies in African Linguistics} 51(2): 323-345, 2022), ‘Locating the Hadzabe in the wilderness’ (\textit{Utafiti} 17(1): 130–154, 2022), ‘Linguistic and social outcomes of interaction of Hadzabe and Sukuma in north-western Tanzania’ (\textit{Utafiti} 15(1): 348–373, 2020), and ‘African linguistics in eastern Africa’ (in H. Ekkehard Wolff (ed.), \textit{A History of African Linguistics}, 133–152. Cambridge: CUP, 2019).~ ~

\subsubsection*{Aurelia Mallya \ORCID{0000-0001-7842-1022}}
Aurelia Mallya is a Senior Lecturer in the Department of Foreign Languages and Linguistics, College of Humanities, University of Dar es Salaam, Tanzania. Her research focus is in syntax, morphology, lexical semantics, and their interfaces, particularly in Bantu languages. She is also interested in language documentation and descriptive linguistics. Dr Mallya completed her Bachelor's and Master's degrees at the University of Dar es Salaam, and her PhD at Stellenbosch University in South Africa. Some of her recent publications include ‘Passive and anticausative constructions: A new perspective to morphosyntax and lexical semantic interfaces’ (\textit{Southern African Linguistics and Applied Language Studies} 37(4): 315-324, 2019) and ‘Locative-subject alternation constructions in Kiwoso’ (\textit{Ghana Journal of Linguistics} 9(2): 1-21, 2020). She is currently the Head of the Department of Foreign Languages and Linguistics, University of Dar es Salaam.

\subsubsection*{Gastor Mapunda \ORCID{0000-0001-5683-0175}}
Gastor Mapunda is Associate Professor of English and Linguistics in the Department of Foreign Languages and Linguistics, University of Dar es Salaam. He completed his Bachelor’s and Master’s degrees at the University of Dar es Salaam, and his PhD at the Universities of Dar es Salaam and Bristol on a split-site arrangement. His areas of research are applied linguistics, sociolinguistics, and discourse analysis. His doctoral research was on classroom interaction in multilingual communities. He has also researched extensively on language contact. He has served in various positions, including deputy chair of the Governing Council of the Tanzania Institute of Education (TIE) (2017-2021), member of the National Technical Committee for Curricula Review (2022-2023), and convenor of the African Humanities  {Conference 2023} held at the University of Dar es Salaam. His current research ties are with colleagues from the Universities of Dar es Salaam, Essex, and St. John’s University of Tanzania, among others. His most recent publications include ‘“You must be crazy!” Teacher corrective feedback and student uptake in two Tanzanian secondary schools’ (\textit{Journal of Modern Research in English Language Studies} 10(4): 1-20, 2023) and ‘A morphological analysis of Kemunasukuma personal names’ (\textit{Linguistik Online} 123(5): 95–113, 2023).

\subsubsection*{Lutz Marten \ORCID{0000-0002-2289-5701}}
Lutz Marten is Professor of General and African Linguistics at SOAS University of London, where he has served as Dean of the Faculty of Languages and Cultures and as Head of the SOAS Doctoral School. He is interested in linguistic theory, comparative and historical linguistics, and questions of language and identity. Most of his work focuses on African languages and he has conducted research in Eastern and Southern Africa. His publications include \textit{At the Syntax-Pragmatics Interface}  (Oxford: OUP, 2002), \textit{A Grammatical Sketch of Herero} (with Jekura Kavari and Wilhelm Möhlig, {Cologne: Köppe, 2002}), \textit{The Dynamics of Language} (with Ronnie Cann and Ruth Kempson,  {Elsevier, 2005}), and \textit{Colloquial Swahili} (with Donovan McGrath,  {London: Routledge, 2003}/2012). He is the Founding Chair of the International Conference on Bantu Languages and he is the current editor of the \textit{Transactions of the Philological Society}, the oldest scholarly journal devoted to the general study of language and languages with an unbroken tradition.

\subsubsection*{Judith Nakayiza \ORCID{0000-0002-7302-1932}}
Judith Nakayiza is a researcher with a specific interest in sociolinguistics, language planning and policy (ideology, beliefs and identity), language practices in diverse or multilingual spaces, the sociolinguistics of Bantu languages and Ugandan languages. She previously worked as a lecturer in the School of Languages, Literature and Communication, Makerere University, Uganda for over 15 years. She has also conducted post-doctoral research at SOAS University of London, funded by the Commonwealth Scholarship Commission. She is currently an Independent Researcher with affiliations to SOAS University of London, UK and Makerere University, Uganda. Some of her publications present research on the sociolinguistic situation of English, implementing language rights in multilingual Uganda, language attitudes and Ideologies in Uganda, the sociolinguistics of multilingualism in Uganda, Luganda in Uganda and literacy acquisition in selected Ugandan rural primary schools.

\subsubsection*{Armindo Ngunga}
Armindo Saúl Atelela Ngunga, PhD in Linguistics (from UC Berkeley), Chairperson of the Northern Integrated Development Agency, is Professor of Linguistics at the Eduardo Mondlane University in Mozambique, where he taught general linguistics, phonetics, phonology and morphology of Bantu languages since 1984. He has supervised several graduate and Master's dissertations and PhD theses. He served as Secretary of State in Cabo Delgado Province (2020-2021), Vice-Minister of Education and Human Development (2015-2020), Director of the Centre of African Studies (2007-2015) and Dean of the Faculty of Arts and Social Sciences at the Eduardo Mondlane University (1999-2007). He has published tens of journal articles and book chapters, and has written and edited books on grammar and lexicography of Mozambican languages. He has edited a book series called \textit{As nossas línguas} (‘Our languages’), with twelve titles since 2009.

\subsubsection*{Lengson Ngwasi \ORCID{0000-0001-9114-6491}}
Lengson Ngwasi is a Lecturer in Linguistics in the Department of Foreign Languages and Linguistics at the University of Dar es Salaam, Tanzania. He received his PhD in 2021 from the University of Gothenburg, Sweden. He also has a Master degree in Linguistics from the University of Dar es Salaam (received in 2016), as well as a Bachelor's degree in Arts with Education from the same university that he received in 2013. {In 2016}, during his MA studies, he secured a Linneaus-Palme Exchange position for six months at the University of Gothenburg. His research interest includes historical and comparative linguistics, grammaticalization, language variation, and language typology. His research works are his MA dissertation on \textit{Reflexive marking in Kihehe} and his PhD dissertation \textit{The multiple functions of the reflexive prefix in Hehe, Sukuma, Nilamba, and Nyaturu}. He is currently working on a paper entitled ‘The typology of reciprocal marking in Hehe, Sukuma, Nilamba, and Nyaturu’.

\subsubsection*{Malin Petzell  \ORCID{0000-0002-0774-5131}}
Malin Petzell is a Reader (Docent) in African Languages at the Department of Languages and Literatures at the University of Gothenburg, Sweden. Her speciality{~}lies in the documentation, description and analysis of both under-described and unwritten languages, and particularly Tanzanian Bantu languages. Her work focuses primarily on verbal morphosyntax including tense and aspect, and aspectual classification of verbs. Two other areas of interest are comparative studies of closely related languages, and field methods and data management.

\subsubsection*{Daisuke Shinagawa \ORCID{0000-0002-9320-6520}}
Daisuke Shinagawa is Associate Professor at the Research Institute for Languages and Cultures of Asia and Africa (ILCAA), Tokyo University of Foreign Studies. He has been carrying out his descriptive field linguistic research on underdescribed Kilimanjaro Bantu languages including Rwa (E621A), Siha (E621C), Uru (E622D), and Rombo-Mkuu (E623C). His research interests are based on structural diversity of Bantu languages and typological principles lying behind. He has also been working on the structural variations of Swahili, particularly focusing on urban varieties generally known as Sheng. His current publications include ‘A micro-parametric survey on typological covariation related to focus marking strategies, based on the Bantu Morphosyntactic Variation database’ (with Lutz Marten, \textit{Linguistique et langues africaines} 9(1): 1-25, 2023), \textit{Descriptive Materials of Morphosyntactic Microvariation in Bantu} \textit{Vol.2: A microparametric survey of morphosyntactic microvariation in Southern Bantu languages} (co-edited with Seunghun J. Lee and Yuko Abe, Tokyo: ILCAA, 2021), \textit{Variation in Swahili, special issue of Swahili Forum 26} (co-edited with Nico Nassenstein, 2019), and \textit{Descriptive Materials of Morphosyntactic Microvariation in Bantu} (co-edited with Yuko Abe, Tokyo: ILCAA,  2018).

\subsubsection*{Julius Taji \ORCID{0000-0003-0516-0583}}
Julius Taji is a Senior Lecturer in the Department of Foreign Languages and Linguistics, University of Dar es Salaam, Tanzania and a Research Associate in the Department of Linguistics and Language Practice at University of the Free State, South Africa. He holds a PhD (Linguistics), an M.A. (Linguistics), and a B.A. (Education) from the University of Dar es Salaam. His areas of specialization include morphosyntax of Bantu languages, lexicography and sign language linguistics. Since the completion of his PhD in 2017, Julius has participated in collaborative research projects with academics from several universities, including SOAS and the University of Essex in the UK, the Research Institute for Languages and Cultures of Asia and Africa (ILCAA) in Japan, and the University of the Free State, South Africa. He has also won several research awards, including the PANGeA Early Career Fellowship Award hosted by the Graduate School of the Faculty of Arts and Social Sciences at Stellenbosch University, South Africa in 2019, a Full-Time Postdoctoral Research Fellowship in the Faculty of Humanities, University of the Free State, South Africa in 2020 and the African Humanities Programme (AHP) Fellowship in 2021. Currently, Julius is involved in a collaborative research project on grammatical variation in Swahili with colleagues from SOAS, the University of Essex, and Kenyatta University.

\subsubsection*{Nobuko Yoneda}
Nobuko Yoneda started her doctoral course at the Graduate School of Tokyo University of Foreign Studies in 1995, majoring in African Linguistics, and received her PhD in 2000. She worked as a Lecturer at Osaka Jogakuin  {University from 2000} to 2002. She has worked at Osaka University since 2002, first as Associate  {Professor (2008}-2010) and then since 2010 as Professor, teaching Swahili grammar, Bantu linguistics, and African sociolinguistics. Her research interests are descriptive linguistics in Bantu languages and sociolinguistics. She has been conducting fieldwork in Africa, mainly in Tanzania and Namibia, since 1993. Her main publications include ‘Noncausal/causal verb alternations in Swahili’ (\textit{Linguistique et Langues Africaines} 8(2), 2022), ‘Noun-modifying constructions in Swahili and Japanese’ (in Prashant Pardeshi and Taro Kageyama (eds.), \textit{Handbook of Japanese Contrastive Linguistics}, 433-452. Berlin: de Gruyter Mouton, 2018), ‘Conjoint/disjoint distinction and focus in Matengo (N13)’ (in Larry Hyman and Jenneke van der Wal (eds.), \textit{The Conjoint/Disjoint Alternation in Bantu}, 14-60. Berlin: de Gruyter Mouton, 2017) and ‘Word order in Matengo (N13): Topicality and informational roles’ (\textit{Lingua} 121(5): 754-771, 2011).
