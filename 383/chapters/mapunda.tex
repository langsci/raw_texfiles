\documentclass[output=paper ]{langscibook}
\ChapterDOI{10.5281/zenodo.10663771}

\author{Gastor Mapunda\orcid{}\affiliation{University of Dar es Salaam} and
        Fabiola Hassan\orcid{}\affiliation{University of Dodoma}}

\title[The locative system in South-Tanzanian Bantu languages]{A comparative study of the locative system in South-Tanzanian Bantu languages}

\abstract{The paper presents a comparative analysis of locative expressions in four South-Tanzanian Bantu languages, namely Bena (G60), Ngoni (N12), Yao (P21), and Makhuwa (P31). We more particularly explore the locative marking strategies within noun phrases, the issue of locative agreement, and locative inversion constructions. The article pursues two objectives: (1) To describe the form of locative affixes in each of the four languages, and (2) to establish resemblances and dissimilarities between four neighbouring languages spoken in the south of Tanzania. The findings show that, although the locative systems of the four sampled languages are overall very similar, Makhuwa still exhibits a few divergent features. 
\keywords{Locative expressions, locative morphology, agreement, inversion constructions, Eastern Bantu}}

\IfFileExists{../localcommands.tex}{
  \addbibresource{../localbibliography.bib}
  % add all extra packages you need to load to this file

\usepackage{tabularx,multicol}
\usepackage{url}
\urlstyle{same}

\usepackage{listings}
\lstset{basicstyle=\ttfamily,tabsize=2,breaklines=true}

\usepackage{langsci-basic}
\usepackage{langsci-optional}
\usepackage{langsci-lgr}
\usepackage{langsci-osl}
% \usepackage{./langsci/styles/langsci-lgr}
% \usepackage{./langsci/styles/langsci-osl}
% \usepackage{langsci-gb4e}

\usepackage{tikz}
\usetikzlibrary{patterns,calc}
\pgfdeclarepatternformonly{south east lines}{\pgfqpoint{-0pt}{-0pt}}{\pgfqpoint{3pt}{3pt}}{\pgfqpoint{3pt}{3pt}}{
    \pgfsetlinewidth{0.6pt}
    \pgfpathmoveto{\pgfqpoint{0pt}{3pt}}
    \pgfpathlineto{\pgfqpoint{3pt}{0pt}}
    \pgfpathmoveto{\pgfqpoint{.2pt}{-.2pt}}
    \pgfpathlineto{\pgfqpoint{-.2pt}{.2pt}}
    \pgfpathmoveto{\pgfqpoint{3.2pt}{2.8pt}}
    \pgfpathlineto{\pgfqpoint{2.8pt}{3.2pt}}
    \pgfusepath{stroke}}
    
\usepackage{stmaryrd}
\usepackage{wasysym}
\usepackage{multirow}
\usepackage{caption}
\usepackage{subcaption}
\usepackage{mathrsfs}
\usepackage{qtree}

\usepackage{linguex}


  %pminos do not split footnotes
% \interfootnotelinepenalty=10000 %Footnote in Laporte chapters has to be split SN


%\DeclareIndexNameFormat{default}{%
%\nameparts{#1}%
%\usebibmacro{index:name}%
%{\index[names]}%
%{\namepartfamily}%
%{\namepartgiveni}%
% {}% L1
% {}% L2
%{\namepartprefix}% generates spurious space L3
%{\namepartsuffix}% generates spurious space L4
%}

%  {\DeclareIndexNameFormat{default}{%
%     \usebibmacro{index:name}{\index[names]}{#1}{#3}{#5}{#7}}}

%\DeclareIndexNameFormat{default}{%
%  \usebibmacro{index:name}{\sindex[nom]}{#1}{#3}{#5}{#7}}

%\DeclareIndexNameFormat{default}{%
%  \usebibmacro{index:name}{\sindex[person]}{#1}{#3}{#5}{#7}}
%\DeclareIndexNameFormat{default}{%
%\nameparts{#1} \usebibmacro{index:name}{\sindex[person]]}{\namepartfamily}{‌​\namepartgiven}{\nam‌​epartprefix}{\namepa‌​rtsuffix}}

%\newcommand{\smiley}{:)}

%\renewbibmacro*{index:name}[5]{%
%\usebibmacro{index:entry}{#1}%
%{\iffieldundef{usera}{}{\thefield{usera}\actualoperator}\mkbibindexname{#2}{#3}{#4}{#5}}}

% \newcommand{\noop}[1]{}

%remove for final
%\overfullrule=1mm

\newcommand{\tobi}[2]}}
\renewcommand{\S}[1]{\tobi{#1}{\textsc{*}}}

% this volume references
% puts: [this volume]
% already defined: \citetv
%\newcommand{\citepv}[1]{(\citeauthor{#1} \citeyear*{#1} [this volume])}
\newcommand{\citealtv}[1]{\citeauthor{#1} \citeyear*{#1} [this volume]}

%parentheses around example number
\newcommand{\pref}[1]{(\ref{#1})}

% in-text examples

\newcommand{\lnex}[1]{\textit{#1}} %target lang word
\newcommand{\lnlit}[1]{(lit.: `#1')} %literal reading
\newcommand{\lnlat}[1]{(#1)} % latinization
\newcommand{\lntrans}[1]{`#1'} %translation
\newcommand{\lnexl}[2]%
{\lnex{#1}{} \lnlat{#2}} % ex with latinization
\newcommand{\lnexlat}[3]{\lnex{#1}{} \lnlat{#2}{} \lntrans{#3}} % ex with latinization and tranl.

%ch01
\newcommand{\co}[1]{\mbox{\textbf{#1}}}

%ch09

\newcommand{\cyrbulg}[1]{\begin{otherlanguage*}{bulgarian}#1\end{otherlanguage*}}


%ch10
\newcommand{\nlp}{{\small NLP}}
\newcommand{\mwe}{{\small MWE}}
\newcommand{\rae}{{\small RAE}}
\newcommand{\lvc}{{\small LVC}}
\newcommand{\pos}{{\small P}o{\small S}}
%\newcommand{\todo}[1]{ \textcolor{red}{#1} }

%\renewcommand{\labelenumi}{\theenumi}
%\ainamefmt{{vv}{ll}{, ff}{, jj}} % fullname

\newcommand{\biberror}[1]{{\color{red}#1}}

\newcommand{\osenovaitem}{--~} 
  %% hyphenation points for line breaks
%% Normally, automatic hyphenation in LaTeX is very good
%% If a word is mis-hyphenated, add it to this file
%%
%% add information to TeX file before \begin{document} with:
%% %% hyphenation points for line breaks
%% Normally, automatic hyphenation in LaTeX is very good
%% If a word is mis-hyphenated, add it to this file
%%
%% add information to TeX file before \begin{document} with:
%% %% hyphenation points for line breaks
%% Normally, automatic hyphenation in LaTeX is very good
%% If a word is mis-hyphenated, add it to this file
%%
%% add information to TeX file before \begin{document} with:
%% \include{localhyphenation}
\hyphenation{
    Beck-man
    Ngu-yen
    back-chan-nel
    back-chan-nels
    mo-not-o-nous
    ste-reo-typ-i-cal
}

\hyphenation{
    Beck-man
    Ngu-yen
    back-chan-nel
    back-chan-nels
    mo-not-o-nous
    ste-reo-typ-i-cal
}

\hyphenation{
    Beck-man
    Ngu-yen
    back-chan-nel
    back-chan-nels
    mo-not-o-nous
    ste-reo-typ-i-cal
}
 
  \togglepaper[1]%%chapternumber
}{}

\begin{document}
\maketitle 
%\shorttitlerunninghead{}%%use this for an abridged title in the page headers

\settowidth\jamwidth{[Makhuwa]}

\section{Introduction}\label{sec:mapunda:1} %1. /

In Eastern Bantu Languages, locative expressions have received enormous attention from various scholars (cf., among others, \citealt{Harries1965, Rugemalira2004, Buell2007, Marten2012, Barlew2013, MartenvanderWal2014, Guérois2016, ZellerForthcoming}). These different studies give insight into the high degree of variation of Bantu locatives. 

  The present article aims to show how Bena, Ngoni, Yao, and Makhuwa, four Eastern Bantu languages, vary in the expression of their locative noun phrases and locative clauses. The languages mentioned above have been selected because they represent different language groups that are found in the Eastern Bantu area, and are geographically close. Additionally, these languages are familiar to the authors of this chapter. Bena is an Eastern Bantu language spoken in the Southern Highlands of Tanzania, mostly in Njombe District. It is also spoken in the north-western part of Songea District, the north-eastern part of Mbeya District, the southern part of Mufindi District, and the south-western part of Ulanga District. \citet{Guthrie1971} classifies Bena under zone (G60) together with Ki-Hehe, Shi-Sango, Ki-Kinga, Ki-Kisi, and Ki-Wanji. \citet[115]{Chaula1989} identifies seven main dialects of Bena, namely Lupembe, Masakati, Sovi, Maswamu, Mavemba, Ilembula, and Ulanga. Makhuwa (P31) is spoken in the north of Mozambique (Cabo Deldago, Nampula, Niassa, and Zambézia provinces), in Malawi (Mulanje and Tyholo), and in the southern part of Tanzania \citep{Kröger2005}. In Tanzania, the principal regions where Makhuwa speakers live are Mtwara, Lindi, Morogoro, and the Coast. \citet{Ismail2000} lists no less than twelve dialects,\footnote{{As we have not engaged with a comparative study of these twelve dialects, we cannot comment on their similarities and differences.}} most of which are located in Mozambique. This article focuses on the Imithupi dialect spoken in Tanzania, next to the three other languages analyzed in this chapter. Yao (P21) is spoken in Malawi, Mozambique, Tanzania, Zambia, and Zimbabwe. The current article uses data from Yao spoken in Tanzania, specifically in Masasi and Tunduru Districts. Finally, the Tanzanian Ngoni (N12) has four dialects, namely Maposeni-Peramiho, Likonde-Kigonserat, Matimira, and Rwanda. The data for the current study are based on the Maposeni-Peramiho dialect, which is also the best known (\citealt{Mapunda2015, Ngonyani2003}). These four variants analyzed in this chapter are spoken in Tanzania, as shown in Map 1 below. The four variants are in close geographical proximity, and the speakers of these dialects understand each other well.

  
\begin{figure}
\includegraphics[width=\textwidth]{figures/MapundaHassanfinal-img001.pdf}
 

\caption{Approximate locations of the Bena, Ngoni, Yao and Makhuwa speaking areas in Tanzania (map produced by Sebastian Nordhoff based on work by G. Mapunda and F. Hassan)}
\label{fig:mapunda:1}
\end{figure}

Specifically, the article describes and is structured along the following lines, which resume some of the morphosyntactic parameters proposed by \citet{GuéroisEtAl2017}: i) what are the formal strategies of locative marking on nouns? (\sectref{sec:mapunda:2}); ii) how does locative agreement operate within NPs and VPs? (\sectref{sec:mapunda:3}); iii) are locative inversion constructions attested? (\sectref{sec:mapunda:4}). As a general result, the paper shows how Makhuwa tends to behave differently from the other three languages. 

The primary data used in this study were obtained through interviews with two adult consultants from each language. More specifically, consultants were prompted to translate Swahili sentences into their language. Then, the translations were cross-checked for consistence. Follow-up questions were also asked when additional information was required.  

\section{Locative marking strategies}\label{sec:mapunda:2} %2. /

In many Bantu languages location is marked by nominal prefixation \citep{Rugemalira2004}. Four locative prefixes were reconstructed to Proto-Bantu: class 16 \mbox{*pa-,} class 17 \mbox{*ku-,} class 18 \mbox{*mu-,} and class 23/25 *i/e-\footnote{{This locative affix is attested in very few Bantu languages. Ganda (JE15 Uganda) is an example, e.g.} {\textit{e-Kampala}} {‘in Kampala’.} } (\citealt{Bleek1862-69, Guthrie1948, Guthrie1967-1971, Meeussen1967}). In semantics terms, \textit{*pa-} means nearness, adjacency, definiteness, specificity, limitedness or known location; \textit{*ku-} implies  remoteness, farness, unspecificity, generalness, unlimitedness, not necessarily known or direction\slash to\-wardness; and \textit{*mu-} denotes withinness, interiority or enclosed location. Whilst many Bantu languages have retained the first three historical locative prefixes, e.g. Shona [S10] illustrated in (\ref{ex:mapunda:1}), other languages have retained two affixes, e.g. Vunjo [E62] \citep{Mcha1979} and yet others have retained only one class, e.g. Kiwoso [E621d] \citep{Mallya2011}. 


\ea 
\label{ex:mapunda:1}
\glll {class 16}    \textbf{\textit{pa}}\textit{{}-imba}   {‘at the house’}\\
  {class 17}        \textbf{\textit{ku}}\textit{{}-imba} {‘to the house’}\\      
  {class 18}  \textbf{\textit{mu}}\textit{{}-imba}   {‘in the house’} \hfill (Shona, \citealt{NgungaMpofu2013}: 45)\\
  \z

A second strategy to expression locative consists in suffixing -\textit{(i)ni} {\textasciitilde} -\textit{n} {\textasciitilde} -\textit{eng} to the end of the noun stem. Examples in (\ref{ex:mapunda:2}) illustrate this strategy in several Eastern languages.


\ea 
\label{ex:mapunda:2}
    \ea\label{ex:mapunda:2a} Swahili [G42] \textit{nyumba-}\textbf{\textit{ni}}  ‘at/to/in the house’
    \ex\label{ex:mapunda:2b} Sesotho [S33]  \textit{thab-}\textbf{\textit{eng}   }‘to/on the mountain’ \hfill \citep[120]{Machobane1995}
    \ex\label{ex:mapunda:2c} Chaga [E30]  \textit{ruko-}\textbf{\textit{nyi}   }‘at/to/in the kitchen’ \hfill \citep[131]{Moshi1995}
    \z
\z

Note that the loss of locative morphology in these languages is restricted to noun class prefixes. As will be made clear in \sectref{sec:mapunda:2}, noun modifiers and verb forms controlled by a locative head noun necessarily host locative agreement markers \citep{Mpiranya2015}.

Double affixation, which involves both a locative noun class prefix and a locative suffix, is a third strategy very rarely attested across Bantu. As far as we know, only P30 languages productively exhibit double affixation,\footnote{{In Swati [S43], some locative nouns necessarily combine the class 25 locative prefix} {\textit{e}}{{}- and the locative suffix} {\textit{{}-ini}}{, e.g.} {\textit{e-ndl-ini}} {‘at/to/in the house’ \citep[254]{Marten2010}, but this double affixation is restricted to a few nouns only.}} as illustrated in (\ref{ex:mapunda:3}) with Cuwabo [P34] and Makhuwa-Enahara [P31] which both combine class 17 prefix \textit{o-} and locative suffix \textit{{}-ni}.


\ea
\label{ex:mapunda:3}
    \ea\label{ex:mapunda:3a} Cuwabo  \textbf{\textit{o}}\textit{{}-ma-básá-}\textbf{\textit{ni}}  (cl.17)  ‘at work’
    \ex\label{ex:mapunda:3b} Makhuwa-Enahara  \textbf{\textit{o}}\textit{{}-n-tékô-}\textbf{\textit{ni} }(cl.17)  ‘at work’ \hfill \citep[51]{Guérois2016}
    \z
\z

\citet{Guérois2016} suggests that whilst locative prefixes were inherited, the suffixation of \textit{{}-ni} is a later innovation resulting from a contact situation with Swahili. 

Finally, it should be noted that names of places or cities do not commonly host locative marking. In Swahili, for instance, cities like Tokyo, London, or Paris are not modified when used locatively \citep[153]{Mkude2005}. Cities from Tanzania also do not host locative markers (e.g. Arusha, Dar es Salaam, Dodoma, Mbeya).   

Two of the strategies above are attested in our sample of languages, namely prefixation (for Bena, Ngoni, and Yao) and double affixation (for Makhuwa), as shown in (\ref{ex:mapunda:4}).\footnote{{N is a homorganic nasal, i.e. its surface realization depends on its phonetic environment, such as N > [m] / \_ bilabial C and N > [ŋ] / \_ velar C.} }  


\ea 
    \label{ex:mapunda:4}
    \gllll {} Bena    Ngoni    Yao    Makhuwa{}-I.  \\
cl.16  \textbf{\textit{pa}}\textit{{}-kaye}   \textbf{\textit{pa}}\textit{{}-nyumba} \textbf{\textit{pa}}\textit{{}-nyumba} \textbf{\textit{va}}\textit{{}-nupa-}\textbf{\textit{ni}}  {‘at the house’}\\  
cl.17        \textbf{\textit{ku}}\textit{{}-kaye} \textbf{\textit{ku}}\textit{{}-nyumba}   \textbf{\textit{ku}}\textit{{}-nyumba}   \textbf{\textit{u}}\textit{{}-nupa-}\textbf{\textit{ni}}   {‘to the house’}\\  
cl.18  \textbf{\textit{mu}}\textit{{}-kaye}  \textbf{\textit{mu}}\textit{{}-nyumba}   \textbf{\textit{n}}\textit{{}-nyumba}   \textbf{\textit{n}}\textit{{}-nupa-}\textbf{\textit{ni}}   {‘in the house’}\\
\z

In each sampled language, locative prefixes are additive, i.e. they are added to the stem which has an inherent noun class prefix. For instance, in Makhuwa, the locative prefixes \textit{va}{}-, \textit{u}{}- and \textit{N}{}- can be added to the noun stem \textit{mwiri} ‘tree’ which has an inherent noun class 3 prefix \textit{mw}{}- as seen in example (\ref{ex:mapunda:5}).


\ea 
\label{ex:mapunda:5}
\glll cl.16  \textbf{\textit{va}}\textit{{}-mw-iri-}\textbf{\textit{ni}}  {‘at the tree’} \hspace{4cm}[Makhuwa-I.]\\
cl.17        \textbf{\textit{u}}\textit{{}-mw-iri-}\textbf{\textit{ni}}   {‘to the tree’}\\  
cl.18  \textbf{\textit{m}}\textit{{}-mw-iri-}\textbf{\textit{ni}}   {‘in the tree’}\\
\z

Looking back at \tabref{tab:mapunda:1}, we see that Makhuwa-Imithupi behaves as Makhuwa-Enahara illustrated in (\ref{ex:mapunda:3}), i.e. locative nouns are by default built upon the combination of a locative prefix (class 16 \textit{va-}, class 17 \textit{u-}, and class 18 \textit{{N-}}) and the locative suffix -\textit{ni}. Only a few nouns deviate from this pattern by not admiting the locative suffix. These are lexicalized locatives such as \textit{vachula} ‘at the top’ and \textit{uchulu} ‘to the top’, and proper locative nouns such as names of towns, countries and continents, as in (\ref{ex:mapunda:6}). 


\ea 
\label{ex:mapunda:6}Makhuwa
    \ea\label{ex:mapunda:6a}
        \ea[]{\label{ex:mapunda:6ai} \gll {\textbf{\textit{u}}\textit{{}-Dar es salaam}}\\   
                                    {17-{Dar es Salaam}}\\
                                    \glt ‘to Dar es Salaam’}
        \ex[*]{\label{ex:mapunda:6aii} \gll {\textbf{\textit{u}}\textit{{}-Dar es salaam-}\textbf{\textit{ni}}} \\
                                    {17-{Dar es Salaam}-\textsc{loc}}\\}
        \z
    \ex\label{ex:mapunda:6b}  
        \ea[]{\label{ex:mapunda:6bi} \gll \textbf{\textit{u}}\textit{{}-Tanzania} \\     
                                    17-Tanzania\\
                                    \glt ‘to Tanzania’}
        \ex[*]{\label{ex:mapunda_6bii} \gll \textbf{\textit{u}}\textit{{}-Tanzania-}\textbf{\textit{ni}}\\
                                    17-Tanzania-\textsc{loc}\\}
        \z
    \z
\z  

On the other hand, mere locative prefixation with common nouns seems to be strictly prohibited in Makhuwa-Imithupi, as shown in (\ref{ex:mapunda:7}). Both nouns \textit{patsári} ‘market’ and \textit{matta} ‘field’ are made locative by double affixation, i.e. by one of the three locative prefixes and by the locative suffix -\textit{ni} (\textit{u-patsári-ni} ‘to the market’ and \textit{m-matta-ni} ‘in the field’). In this respect, it is worth noting a dialectal difference with Makhuwa-Enahara, whereby certain common nouns may be marked for locative uniquely through prefixation (\ref{ex:mapunda:8}). Some others freely add the locative suffix -\textit{ni} (\ref{ex:mapunda:8}).  


\ea 
\label{ex:mapunda:7}Makhuwa-Imithupi
    \ea[*]{\label{ex:mapunda:7a} \gll \textbf{\textit{u}}\textit{{}-patsári}\\ 
                                17-9.market \\          
                            \glt ‘to the market’ }        
    \ex[*]{\label{ex:mapunda:7b} \gll \textbf{\textit{m}}\textit{{}-matta}\\
                                    18-field\\
                                \glt ‘in the field’}
    \z

\ex
\label{ex:mapunda:8}Makhuwa-Enahara \citep[53--54]{Guérois2016}
    \ea\label{ex:mapunda:8a} \gll  \textbf{\textit{o}}\textit{{}-patsári} \\         
                                    17-market \\        
                            \glt ‘at/to the market’        
    \ex\label{ex:mapunda:8b} \gll \textbf{\textit{m}}\textit{{}-mátta(-}\textbf{\textit{ni}})\\
                                    18-field-\textsc{loc}\\
                            \glt ‘in the field’
    \z
\z

If the locative suffix -\textit{ni} is present in certain locative expressions in Bena, Ngoni, and Yao, its use is not productive at all. In these three languages, locative NPs are expressed through prefixation only. In Ngoni, for instance, \textit{-ni} is present in two specific contexts, i.e. in lexicalized locative NPs (\ref{ex;mapunda:9})\footnote{However,
 it cannot be excluded that these lexicalized locative NPs are originally loans from Swahili where locative is marked by suffixation.}
and in borrowed locative NPs (\ref{ex:mapunda:10}). Since -\textit{ni} is not segmentable in these words, it does not convey any locative meaning, hence locative prefixation is still needed.



\ea\label{ex;mapunda:9} 
\gll \textit{mfuleni}   ‘well’     \textit{pa-mfuleni}   {‘at the well’}     \hspace{3.5cm}[Ngoni]\\ 
{\textit{bomani}} {‘town’} {\textit{pa-bomani}} {‘to/in (?) town’}\\



\ex\label{ex:mapunda:10}
\gllll {\textit{m-ji}\textbf{\textit{ni}} ‘town’   < Swahili \textit{mjini}} {}         \hspace{3.3cm}[Ngoni]\\
    {{\textbf{\textit{pa}}}{\textit{{}-m-ji}}{\textbf{\textit{ni}}} (cl.16)}     {‘at the town’}\\ 
    {{\textbf{\textit{ku}}}{\textit{{}-m-ji}}{\textbf{\textit{ni}}} (cl.17)}     {‘to the town’}\\
    {{\textbf{\textit{mu}}}{\textit{{}-m-ji}}{\textbf{\textit{ni}}} (cl.18)}     {‘in the town’}\\
\z


{In Yao,} {the locative suffix} {\textit{{}-ni}} {may optionally be added on loan words (\ref{ex:mapunda:11}), reminiscent of equivalent Makhuwa double affixed locative NPs} {\textit{umsikitini}} {‘to the mosque’,} {\textit{ukanisani}} {‘to the church’,} {\textit{ushuleni}} {‘to school’ and} {\textit{umahakamani}} {‘to the court’. Different from Yao, in Bena the locative marker -}{\textit{ni}} {cannot be added \REF{ex:mapunda:12}.} 



\ea 
\label{ex:mapunda:11}Yao \\
\gllll \textit{msikiti} ‘mosque’ (cl.3)  → \textit{\textbf{pa}{}-m-sikiti(-\textbf{ni})}   {‘at the mosque’} (cl.16)\\ 
\textit{kanisa}  ‘church’ (cl.5)    →  \textit{\textbf{ku}{}-kanisa(-\textbf{ni})}    {‘to the church’}   (cl.17)\\  
\textit{shule}    ‘school’ (cl.9)   →  \textit{\textbf{ku}{}-shule(-\textbf{ni})}    {‘to the school’} (cl.18)\\
\textit{mahakama} ‘court’ (cl.9)   →  \textit{\textbf{mu}{}-mahakama(-\textbf{ni})}  {‘in the court’} (cl.18)\\  


\ex\label{ex:mapunda:12}Bena\\
\gllll \textit{msikiti} ‘mosque’ (cl.3)  → \textit{\textbf{pa}{}-m-sikiti(*-ni)}  {‘at the mosque’} (cl.16)\\ 
\textit{kanisa}  ‘church’ (cl.5)    →  \textit{\textbf{ku}{}-kanisa(*-ni)}  {‘to the church’}   (cl.17)\\  
\textit{sule}    ‘school’ (cl.9)   →  \textit{\textbf{ku}{}-sule(*-ni)}    {‘to the school’} (cl.18)\\
\textit{mahakama} ‘court’ (cl.9)   →  \textit{\textbf{mu}{}-mahakama(*-ni)}  {‘in the court’} (cl.18)\\
\z


\tabref{tab:mapunda:0} shows a summary of locative marking in Bena, Ngoni, Yao, and Makhuwa.

\begin{table}
\begin{tabularx}{\textwidth}{>{\raggedright\arraybackslash}p{.22\textwidth}Ql>{\raggedright\arraybackslash}p{.16\textwidth}}

\lsptoprule

 locative marking strategies & prefixation only & \makecell[tl]{suffixation\\only} & prefixation + suffixation\\
 \midrule
{Bena}

{Ngoni}

 Yao & yes & no & {yes}

{in (Swahili) loans}\\
\tablevspace
 Makhuwa-I. & {restricted to}

 lexicalized locative nouns + proper geographical names & no & {yes}

 by default\\
\lspbottomrule
\end{tabularx}

\caption{Locative marking in Bena, Ngoni, Yao, and Makhuwa}
\label{tab:mapunda:0}
\end{table}

As seen in \tabref{tab:mapunda:0}, the locative marking strategies attested in the selected languages are prefixation and double affixation. The sampled languages differ from other Eastern Bantu languages spoken in Tanzania such as Swahili and Chagga, whose nouns become locative via suffixation only. 

\section{Locative agreement} \label{sec:mapunda:3}

Several linguists have discussed locative agreement systems in Bantu languages (e.g., among others, \citealt{Stucky1976, Harford1983, Kahigi2005, Marten2012, NgungaMpofu2013}). Agreement occurs: i) within locative NPs, between the locative head noun and its modifiers; ii) within clauses, between the locative head noun and its dependent verb. The two types of agreement are discussed in the two following subsections.

\subsection{Agreement within NPs}\label{sec:mapunda:3.1} %3.1 /

Locative agreement is a morphosyntactic process whereby the dependent elements in the locative NP agree with the locative. Noun dependents here involve possessives, associatives, adjectives, and demonstratives. They are commonly referred to as modifiers. Agreement-wise, languages show a three-way distinction (e.g. \citealt{Marten2012, Machobane1995}). Firstly, there are languages with an inner agreement system, whereby the inherent noun class prefix of a noun controls the agreement between the locative head and its dependents. This is shown in (\ref{ex:mapunda:13}) with Runyambo [JE21], where the first person singular possessive stem \textit{nje} agrees in noun class with \textit{citabo} ‘book’, i.e. class 7. Secondly, there are languages with an outer agreement system, whereby noun modifiers receive locative agreement prefixes. In (\ref{ex:mapunda:14}), the Swahili first person singular possessive stem \textit{angu} takes class 18 agreement to express withinness. Thirdly, there are languages which exhibit both outer and inner agreement systems. In these languages, the inherent noun class prefix of a noun or the locative prefix controls the agreement between the locative head and its dependents. In Tshiluba [L31], demonstratives modifying locative nouns may agree either with the leftward locative prefix (\ref{ex:mapunda:15a}) or with the inherent noun prefix (\ref{ex:mapunda:15b}). 


\ea 
\label{ex:mapunda:13}
\gll o-mu-\textbf{ci}{}-tabo     \textbf{ca}{}-nje \\
    \textsc{aug}{}-18-7-book     7-my \\
\glt      ‘In my book’\hfill (Runyambo, \citealt[6]{Rugemalira2004})


\ex 
\label{ex:mapunda:14}
\gll chumba-ni  \textbf{mw}{}-angu \\
 7.room-\textsc{loc}  18-\textsc{poss.1sg}\\
\glt      ‘In my room’\hfill (Swahili, \citealt[154]{Mkude2005})


\ex 
\label{ex:mapunda:15}
    \ea\label{ex:mapunda:15a} \gll   \textbf{mu}{}-di-kopu  e-\textbf{mu}     mu-di     mu-tooke \\
              18-5-cup   \textsc{dem}{}-18  \textsc{sm18}{}-be  18-clean\\
     \glt ‘This cup is clean inside’

    \ex\label{ex:mapunda:15b} \gll mu-\textbf{di}{}-kopu  e-\textbf{di}    mu-di     mu-tooke  \\
          18-5-cup   \textsc{dem}{}-5     \textsc{sm18}{}-be  18-clean\\
     \glt ‘The space inside this cup is clean’\hfill (Tshiluba, \citealt[180]{Stucky1976})
    \z
\z

Based on this typology, our data show that Bena and Yao both have outer and inner types of agreement on all types of modifiers, namely adjectives, connectives, demonstratives, and possessives, just like Tshiluba in (\ref{ex:mapunda:15}). Bena data are provided in \tabref{tab:mapunda:1} and Yao data in \tabref{tab:mapunda:2}. The difference in meaning is not entirely clear-cut, but it seems that outer agreement gives more emphasis on the locative aspect of the event, i.e. it relates to a place and not somewhere else. On the other hand, inner agreement gives more importance to the modifier as such. For instance, \textit{pakaye inofu} ‘to a good house’ in Bena, provided in \tabref{tab:mapunda:1}, underlies the fact that the house is good (and not bad).  


\begin{table}
\caption{Outer and inner agreement in Bena}
\label{tab:mapunda:1}

\begin{tabularx}{\textwidth}{p{.16\textwidth}QQ} 
\lsptoprule
& Outer AGR & Inner AGR \\
\midrule
\textit{ahele} ... 

‘he has gone’ & {\gll \textbf{pa}{}-kaye \textbf{pa}{}-nofu \\
16-9.house 16-good\\
\glt ‘to a good house’} & {\gll pa-\textbf{kaye} \textbf{i}{}-nofu \\
16-9.house 9-good\\
\glt ‘to the place where the house is good’}\\
\tablevspace
& {\gll \textbf{pa}{}-kaye \textbf{pa} vaanu\\
16-9.house 16.\textsc{con} 2.people\\
\glt ‘to the people’s house’} & {\gll pa-\textbf{kaye} \textbf{ja} vaanu\\
16-9.house 9.\textsc{con} 2.people\\
\glt ‘to the place where the house is of the people’}\\
\tablevspace
& {\gll \textbf{pa}{}-kaye \textbf{pa}{}-la\\
16-9.house 16.\textsc{dem.ii}\\
\glt ‘to that house’} & {\gll pa-\textbf{kaye} \textbf{i}{}-la\\
16-9.house 9.\textsc{dem.ii}\\
‘to the place of that house’} \\
\tablevspace
& {\gll \textbf{pa}{}-kaye \textbf{pa}{}-angu\\
16-9.house 16-\textsc{poss.1sg}\\
\glt ‘to my house’} & {\gll pa-\textbf{kaye} \textbf{ya}{}-angu\\
16-9.house 9-\textsc{poss.1sg}\\
\glt ‘to my house’} \\
\lspbottomrule
\end{tabularx}
\end{table}

\begin{table}
\caption{Outer and inner agreement in Yao}
\label{tab:mapunda:2}
\begin{tabularx}{\textwidth}{p{.16\textwidth}QQ} 
\lsptoprule
& Outer AGR & Inner AGR \\
\midrule
\textit{ajawile}... 

‘he has gone’ & {\gll \textbf{pa}{}-nyumba \textbf{pa}{}-ambone \\
16-9.house 16-good\\
\glt ‘to the good house’} & {\gll pa-\textbf{nyumba} \textbf{ja}{}-ambone \\
16-9.house 9-good \\
\glt ‘to the place where the house is good’}\\
\tablevspace
& {\gll \textbf{pa}{}-nyumba \textbf{pa} vandu\\
16-9.house 16.\textsc{con} 2.people\\
\glt ‘to the people’s house’} & {\gll pa-\textbf{nyumba} \textbf{ja} vandu\\
16-9.house 9.\textsc{con} 2.people \\
\glt ‘to the place where the house is of the people’}\\
\tablevspace
& {\gll \textbf{pa}{}-nyumba a-\textbf{pa}{}-la\\
16-9.house \textsc{aug-}16-\textsc{dem.iii}\\
\glt ‘to that house’} & {\gll pa-\textbf{nyumba} a-\textbf{ja}{}-la\\
16-9.house \textsc{aug}{}-9-\textsc{dem.iii}\\
\glt ‘to the place of that house’}\\
\tablevspace
& {\gll \textbf{pa}{}-nyumba \textbf{pa}{}-angu\\
16-9.house 16-\textsc{poss.1sg}\\
\glt ‘to my house’} & {\gll pa-\textbf{nyumba} \textbf{ja}{}-angu\\
16-9.house 9-\textsc{poss.1sg}\\
\glt ‘to the place of my house’}\\
\lspbottomrule
\end{tabularx}
\end{table}


Ngoni resembles Runyambo (illustrated in (\ref{ex:mapunda:13}) above): outer and inner types of agreement are only attested between the locative head noun and demonstratives. The other modifiers (adjectives, connectives, possessives) may only receive inner agreement, whereas outer agreement is ungrammatical, as seen in \tabref{tab:mapunda:3}. 

\begin{table}
\caption{Inner agreement in Ngoni}
\label{tab:mapunda:3}

\begin{tabularx}{\textwidth}{>{\raggedright\arraybackslash}p{.17\textwidth}QQ}

\lsptoprule

\textit{ahambi}... 

‘he has gone’ & Outer AGR & Inner AGR \\
\midrule
`to a good house' & {\gll \textbf{*pa}{}-nyumba \textbf{pa}{}-bwina \\
16-9.house 16-good\\} & {\gll pa-\textbf{nyumba} ya{}-bwina \\
 16-9.house 9-good\\}\\
 \tablevspace
`to the house of the people' & {\gll \textbf{*pa}{}-nyumba \textbf{pa{}-}vanu\\
16-9.house 16.\textsc{con}-2.people\\} & {\gll pa-\textbf{nyumba} \textbf{ya} vanu\\
16-9.house 9.\textsc{con} 2.people\\}\\
\tablevspace
`to that house' & {\gll \textbf{pa}{}-nyumba \textbf{pa}{}-la\\
16-9.house \textsc{9}{}-\textsc{dem.iii}\\} & {\gll pa-\textbf{nyumba} y\textbf{i}{}-la\\
16-9.house 9-\textsc{dem.iii}\\}\\
\tablevspace
`to my house' & {\gll \textbf{*pa}{}-nyumba *\textbf{pa}{}-angu\\
16-9.house 16-\textsc{poss.1sg}\\} & {\gll pa-\textbf{nyumba} \textbf{ya}{}-angu\\
16-9.house 9-\textsc{poss.1sg}\\}\\
\lspbottomrule
\end{tabularx}
\end{table}

From our sample, Makhuwa differs the most, as it only displays outer agreement (as in Swahili in (\ref{ex:mapunda:14}) above). This is illustrated in \tabref{tab:mapunda:4}. 

\begin{table}
\caption{Inner agreement in Makhuwa-Imithupi}
\label{tab:mapunda:4}

\begin{tabularx}{\textwidth}{>{\raggedright\arraybackslash}p{.17\textwidth}QQ}

\lsptoprule

\textit{ahorwa}... 

‘he has gone’ & Outer AGR & Inner AGR \\
\midrule
to a good house & {\gll \textbf{va}{}-i-nupa-ni \textbf{va}{}-orera \\
16-9-house-loc 16-good\\} & {\gll *va-i-nupa-ni *\textbf{y-}orera  \\
16-9-house-loc 9-good\\}\\
\tablevspace
to the house of the people & {\gll \textbf{va}{}-i-nupa-ni \textbf{va}{}-atu\\
16-9-house-loc 16.\textsc{con}{}-people\\} & {\gll *va-i-nupa-ni *\textbf{y-}atu\\
16-9.house-loc 9.\textsc{con}{}-people\\}\\
\tablevspace
to that house & {\gll \textbf{va}{}-i-nupa-ni \textbf{va}{}-le\\
16-9-house \textsc{9}{}-\textsc{dem.iii}\\} & {\gll *va-\textbf{i}{}-nupa-ni *\textbf{i}{}-le\\
16-9-house 9-\textsc{dem.iii}\\}\\
\tablevspace
to my house & {\gll \textbf{va}{}-i-nupa-ni \textbf{va}{}-aka\\
16-9-house 16-\textsc{poss.1sg}\\} & {\gll *va-\textbf{i}{}-nupa-ni *\textbf{y}{}-aka\\
16-9-house 9-\textsc{poss.1sg}\\}\\
\lspbottomrule
\end{tabularx}
\end{table}

\newpage
\subsection{Agremeent within VPs}\label{sec:mapunda:3.2} %3.2 /

Within VPs, locative indexation on the verb usually involves subject, object and relative prefixation as well as locative cliticization. Locative verbal enclitics are not attested in the selected languages. Therefore, in this chapter, we only discuss locative subject and object prefixation. In the four sampled languages, locative subject prefixes exist for the three historical locative classes. Class 16 \textit{pa-} (or variant \textit{va-} in Makhuwa-I.) is illustrated in \REF{extab:mapunda:5}. More examples of locative verbal agreement are described in \sectref{sec:mapunda:4} on locative inversion construction.

\ea
Class 16 locative subject prefixes\\
\label{extab:mapunda:5}
\gllllllll \textbf{Bena}  \textbf{pa}{}-i-nung-a    a-ma-futa         \textbf{pa-kaye}\\
~ \textsc{sm16-prs}{}-smell-\textsc{fv}  \textsc{aug}{}-6-oil         16-9.house  \\
\textbf{Ngoni}  \textbf{pa}{}-gi-nung’-a    ma-huta           \textbf{pa-nyumba}\\
~ \textsc{sm16}{}-smell-\textsc{fv}  6-oil                16-9.house\\
 \textbf{Yao}  \textbf{pa}{}-ku-nung-a    ma-huta           \textbf{pa-musi}\\
~ \textsc{sm16-prs}{}-smell-\textsc{fv}  6-oil                 16-9.house\\
 \textbf{Makhuwa} \textbf{va}{}-no-nukh-a    ma-khura        \textbf{va-nupa-ni}\\
~ \textsc{sm16-prs}{}-smell-\textsc{fv}  6-oil                16-9.house- \textsc{loc}  \\
\glt ‘It~smells oil at~the~house.’\\
\z

Locative object marking is also attested in our sample, except in Makhuwa where object marking is restricted to classes 1 and 2 \REF{extab:mapunda:xx}.

\ea
Class 16 locative object prefixes\\
\label{extab:mapunda:xx}

% \begin{tabularx}{\textwidth}{lQ}
% \lsptoprule
\gllllllll \textbf{Bena} ~~u-mw-ana  a-ku-\textbf{pa-}nogw-a    \textbf{pa-sule} \\
{} ~~\textsc{aug}{}-1-child   \textsc{sm1-prs}{}-\textsc{om16}{}-like-\textsc{fv}  16-school\\
\textbf{Ngoni} ~~mw-ana  a{}-\textbf{pa-}gan-i    \textbf{pa-shuli}\\
{} ~~1-child   \textsc{sm1-om16}{}-like-\textsc{fv}  16-school\\
\textbf{Yao}  ~~mw-anache  a-ku-\textbf{pa}{}-sak-a      \textbf{pa-shule} \\
{} ~~1-child   \textsc{sm1-prs-om16}{}-like-\textsc{fv}  16-school\\
\textbf{Makhuwa}  *mw-ana  a-no-\textbf{va}{}-tun-a      \textbf{va-shule-ni} \\
{} ~~1-child   \textsc{sm1-prs-om16}{}-like-\textsc{fv}  16-9.school-\textsc{loc}\\

\glt ‘The child likes school.’ (lit. ‘The child likes there at the school.’)\\
\z

\subsection{Summary}\label{sec:mapunda:3.3} %3.3 /

\tabref{tab:mapunda:x} summarizes the locative agreement system as found in each sampled language. As can be seen, Bena and Yao behave alike: both languages allow locative inner and outer agreement within NPs and both have locative subject and object verbal markers. Ngoni is very similar, except for outer agreement which is restricted to the demonstratives, whereas it is observed with all modifiers in Bena and Yao. Makhuwa, in turn, differs from the other three languages in two respects: first it prohibits inner agreement, second it does not have locative object markers.

\begin{table}
\caption{Overview of locative agreement systems\label{tab:mapunda:x}}
\begin{tabular}{llcccc}
\lsptoprule
\multicolumn{2}{l}{Agreement system} & {Bena}  & {Ngoni} & {Yao} & {Makhuwa}\\
\midrule
\multicolumn{2}{l}{\itshape within NPs} &  &  &  & \\
\multicolumn{2}{l}{inner agreement} &  &  &  & \\
& with adjectives & \ding{51}& \ding{51}& \ding{51}& \ding{55}\\
& with connectives  & \ding{51}& \ding{51}& \ding{51}& \ding{55}\\
& with demonstratives & \ding{51}& \ding{51}& \ding{51}& \ding{55}\\
& with possessives  & \ding{51}& \ding{51}& \ding{51}& \ding{55}\\
\tablevspace
\multicolumn{2}{l}{outer agreement} &  &  &  & \\
& with adjectives  & \ding{51}& \ding{55}& \ding{51}& \ding{51}\\
& with connectives  & \ding{51}& \ding{55}& \ding{51}& \ding{51}\\
& with demonstratives  & \ding{51}& \ding{51}& \ding{51}& \ding{51}\\
& with possessives  & \ding{51}& \ding{55}& \ding{51}& \ding{51}\\
\midrule
\multicolumn{2}{l}{\itshape within VPs} &  &  &  & \\
& locative subject marker & \ding{51}& \ding{51}& \ding{51}& \ding{51}\\
& locative object marker & \ding{51}& \ding{51}& \ding{51}& \ding{55}\\
\lspbottomrule
\end{tabular}
\end{table}

\section{Locative inversion constructions}\label{sec:mapunda:4}

Locative inversion (LI) is part of those inversion constructions whereby a logical subject, i.e. the highest thematic role selected by the verb, occupies a postverbal position and the locative phrase is raised to the preverbal position where it grammatically behaves like a regular subject, i.e. it controls agreement on the verb. This change in word order is often motivated by information-structural considerations (\citealt{MartenvanderWal2014, HamlaouiForthcoming}). Two types of LI are traditionally distinguished \citep{Buell2007}: formal agreeing LI and semantic agreeing LI. The former relies on locative morphology and implies that languages have maintained a productive locative system. This is the case in Chewa as shown in (\ref{ex:mapunda:16}) where the verb \textit{li} ‘be’ agrees with the preverbal locative phrase \textit{kumudzi} ‘to the village’. Semantic agreeing LI, in turn, involves nouns which are inherently locative without any additional locative marking. This is illustrated in Zulu (\ref{ex:mapunda:17}) with \textit{lezi zindlu} ‘(in) these houses’, which triggers subject agreement on \textit{hlala} ‘live’. 


\ea 
\label{ex:mapunda:16}
    \ea\label{ex:mapunda:16a} \gll  Chi-tsime  chi-li    \textbf{ku-mu-dzi}. \\
                                     7-well  \textsc{sm7}{}-be    17-3-vilage\\   \jambox*{[Chewa]}
                                \glt ‘The well is to the village.’

    \ex\label{ex:mapunda:16b} \gll  \textbf{Ku-mu-dzi}    \textbf{ku}{}-li    chi-tsime.\\
                          17-3-village    \textsc{sm17}{}-be  7-well\\
                            \glt  ‘To the village there is a well.’ \citep[5]{Salzman2005}
    \z

\ex 
\label{ex:mapunda:17}
    \ea\label{ex:mapunda:17a} \gll  \textbf{Aba}{}-\textbf{ntu}    aba-dala  \textbf{ba-}hlala        \textbf{lezi}        \textbf{zi-ndlu}. \\ 
                                     2-people      2-old    \textsc{sm2}{}-live          10.\textsc{dem.i}  10-house\\ \jambox*{[Zulu]}
                         \glt ‘Old people live in these houses.’ 

    \ex\label{ex:mapunda:17b} \gll \textbf{Lezi}    \textbf{zi-ndlu}  \textbf{zi-}hlala  aba-ntu  aba-dala.\\
                             10-\textsc{dem.i}  10-house   \textsc{sm10}{}-live  2-people  2-old\\
                         \glt ‘(In) these houses live old people.’ \hfill\citep[107--108]{Buell2007}
    \z
\z

More recently, \citet{Guérois2014} shows that both locative LI and semantic LI exist in Cuwabo [P34]. Other languages such as Olutsootso [JE32b] and Swahili (\citealt{MartenvanderWal2014}), and Kinyarwanda [JD61] \citep{Ngoboka2016} show the same feature. Our collected data show no evidence of semantic LI constructions; only formal LI is attested in the 4 sampled languages, in accordance with the most common Bantu pattern (\citealt{MartenvanderWal2014}). An example of each language is provided below. 


\ea 
\label{ex:mapunda:18}
    \ea\label{ex:mapunda:18a} \gll  a-ma-futa    ma-gi-nung’{}-a    mu{}-shumba \\    
                                \textsc{aug}{}-6-oil    \textsc{sm6-prs}{}-smell-\textsc{fv}  18{}-7.room\\ \jambox*{[Bena]}
                            \glt  ‘Oil is smelling in the room’

    \ex\label{ex:mapunda:18b} \gll  \textbf{mu-shumba}    \textbf{mu}{}-gi-nung-a    a-ma-futa \\
                 18-7.room    \textsc{sm18}{}-\textsc{prs}{}-smell-\textsc{fv}  \textsc{aug}{}-6-oil\\
                \glt  ‘In the room is smelling oil’
    \z

\ex \label{ex:mapunda:19}
    \ea\label{ex:mapunda:19a} \gll  ma-huta  ma-gi-nung’-a    mu-chumba \\       
                                  6-oil    \textsc{sm6}{}-\textsc{prs}{}-smell-\textsc{fv}  18-7.room\\ \jambox*{[Ngoni]}
                            \glt  ‘Oil is smelling in the room’

    \ex\label{ex:mapunda:19b} \gll  \textbf{mu-chumba}    \textbf{mu}{}-gi-nung’-a  ma-huta \\
                          18-7.room    \textsc{sm17}{}-\textsc{prs}{}-smell-\textsc{fv}  6-oil\\
                        \glt  ‘In the room is smelling oil’
    \z

\ex\label{ex:mapunda:20} 
    \ea\label{ex:mapunda:20a} \gll ma-huta  ma-ku-nung-a    mu-ch-umba \\ 
                          6-oil    6-\textsc{prs}{}-smell-\textsc{fv}  18-7-room\\ \jambox*{[Yao]}
                        \glt  ‘Oil is smelling in the room’

    \ex\label{ex:mapunda:20b} \gll  \textbf{mu-chumba}    \textbf{mu}{}-ku-nung-a    ma-huta  \\
                      18-7.room-\textsc{loc}  \textsc{sm18}{}-\textsc{prs}{}-smell-\textsc{fv}  6-oil\\
                \glt  ‘In the room is smelling oil’
    \z

\ex 
\label{ex:mapunda:21}
    \ea\label{ex:mapunda:21a} \gll ma-khura  a-no-nukh-a    n-ch-umba-ni \\
                          6-oil    \textsc{6sm-prs}{}-smell-\textsc{fv}  18-7-room-loc\\ \jambox*{[Makhuwa]}
                            \glt  ‘Oil is smelling in the room’

    \ex\label{ex:mapunda:21b} \gll  \textbf{n-chumba-ni}   \textbf{n}{}-no-nukh-a    ma-khura \\
                      18-7.room-\textsc{loc}  \textsc{sm18}{}-\textsc{prs}{}-smell-\textsc{fv}  6-oil\\
                        \glt  ‘In the room is smelling oil’
    \z
\z

Other examples with the copula verb \textit{li} {\textasciitilde} \textit{ri} ‘be’ are provided in (\ref{ex:mapunda:22})--(\ref{ex:mapunda:25}). 


\ea 
\label{ex:mapunda:22} \gll \textbf{mu-shumba}  \textbf{mu}{}-li    mw-ana\\ 
                            18-7.room  \textsc{sm18}{}-be  1-child\\ \jambox*{[Bena]}
                    \glt ‘In the room there is a child.’


\ex\label{ex:mapunda:23} 
\gll \textbf{mu-chumba}  \textbf{mu}{}-wi    (na)  mw-ana\\ 
18-7.room  \textsc{sm18}{}-be   (with)   1-child\\ \jambox*{[Ngoni]}
\glt ‘In the room there is (with) a child.’


\ex\label{ex:mapunda:24} 
\gll \textbf{mu-nyumba}  \textbf{mu-}li    mw-anache \\ 
18-9.house  \textsc{sm18}{}-be  1-child\\\jambox*{[Yao]}
\glt ‘In the house there is a child.’


\ex
\label{ex:mapunda:25}
\gll \textbf{m-nupa-ni}    \textbf{m}{}-ri    mw-ana \\ 
18-9.house-\textsc{loc}    \textsc{sm18}{}-be  1-child \\ \jambox*{[Makhuwa]}
\glt ‘In the house there is a child.’
\z

The preverbal locative phrase behaves, in many ways, just like a regular subject. Like in most Bantu languages, finite verbs in the four sampled languages have an obligatory subject prefix that agrees with the subject NP in noun class. In LI constructions, the subject prefix of the verb obligatorily agrees with the preverbal locative phrase, in one of the three locative noun classes. Such agreement is a clear indicator of the subject status of the fronted locative phrase. 

As a grammatical subject and discourse topic, the fronted locative NP may be dropped or may be postponed clause-finally. In both cases, it keeps licensing subject agreement on the verb. This is shown below.


\ea \label{ex:mapunda:26}
\gll \textbf{mu}{}-li    mw-ana   (\textbf{mu-shumba})\\   
\textsc{sm18}{}-be    1-child    (18-7.room)\\ \jambox*{[Bena]}
\glt ‘There is a child (in the room).’


\ex\label{ex:mapunda:27}
\gll \textbf{mu}{}-wi       (na)  mw-ana (\textbf{mu-chumba})\\   
\textsc{sm18}{}-be    (with)  1-child    (18-7.room)  \\\jambox*{[Ngoni]}
\glt ‘There is (with)  a child (in the room).’


\ex \label{ex:mapunda:28}
\gll \textbf{mu-li}    mw-anache (\textbf{mu-nyumba})\\   
\textsc{sm18}{}-be    1-child    (18-9.house)  \\\jambox*{[Yao]}
\glt ‘There is a child (in the room).’


\ex \label{ex:mapunda:29}
\gll \textbf{m}{}-\textbf{ri}    mw-ana (\textbf{m-nupa-ni})    \\ 
\textsc{sm18}{}-be    1-child    (18-9.house-\textsc{loc)}    \\\jambox*{[Makhuwa]}
\glt ‘There is a child (in the room).’
\z

On the other hand, the inverted subject appears immediately after the verb, i.e. the object position, but maintains a thematic role of subject. Its presence is mandatory. Omitting the inverted subject would make the sentence ungrammatical, as seen in the examples below. 


\ea[*]{\label{ex:mapunda:30}
\gll mu-shumba  mu-li \\    
18-7.room  \textsc{sm18}{}-be\\  \jambox*{[Bena]}
\glt lit. ‘In the room there is.’}


\ex[*]{\label{ex:mapunda:31}
\gll mu-chumba  mu-wi(na) \\  
18-7.room  \textsc{sm18}{}-be \\\jambox*{[Ngoni]} 
\glt lit. ‘In the room there is.’}

\ex[*]{\label{ex:mapunda:32} 
\gll mu-nyumba  mu-li \\    
18-9.house  \textsc{sm18}{}-be  \\\jambox*{[Yao]}
\glt lit. ‘In the house there is.’}


\ex[*]{\label{ex:mapunda:33}
\gll m-nupa-ni    m-ri \\    
18-9.house-\textsc{loc}    \textsc{sm18}{}-be  \\\jambox*{[Makhuwa]}
\glt lit. ‘In the house there is.’}
\z

Despite its postverbal object position, the inverted subject does not really behave as an object. First, it cannot be object-marked on the verb as seen in examples (\ref{ex:mapunda:34})--(\ref{ex:mapunda:37}). 


\ea[*]{\label{ex:mapunda:34}
\gll mu-sh{}-umba    mu-i-\textbf{ma}{}-nung-a    \textbf{a-ma-futa} \\ 
18-7-room    \textsc{sm18-prs-om6}{}-smell-\textsc{fv}  \textsc{aug}{}-6-oil\\ \jambox*{[Bena]}
\glt ‘In the room is smelling it, oil’}

\ex[*]{\label{ex:mapunda:35}
\gll mu-ch-umba    mw-i-\textbf{mu}{}-nung’{}-a    \textbf{ma-huta} \\ 
18-7-room    \textsc{sm18-prs-om6}{}-smell-\textsc{fv}  6-oil\\ \jambox*{[Ngoni]}
\glt ‘In the room is smelling it, oil’}


\ex[*]{\label{ex:mapunda:36}
\gll mu-chumba   mu-\textbf{ma}{}-kunung-a    \textbf{ma-huta} \\ 
18-7.room  \textsc{sm18-om6}{}-smell-\textsc{fv}    6-oil\\ \jambox*{[Yao]}
\glt ‘In the room is smelling it, oil’}


\ex[*]{\label{ex:mapunda:37}
\gll m-chumba-ni   m-no-\textbf{mw}{}-unl-a    \textbf{mw-ana} \\ 
18-7.room-\textsc{loc}  \textsc{sm18-prs-om6}{}-smell-\textsc{fv}  1-child\\ \jambox*{[Makhuwa]}
\glt ‘In the room is crying him, the child’}
\z

Second, the logical subject cannot be passivized, as seen in (\ref{ex:mapunda:38})--(\ref{ex:mapunda:41}).


\ea[*]{\label{ex:mapunda:38}
\gll \textbf{ma-futa}  ma-i-nung-\textbf{w}{}-a    (ni  mu-ki-yumba)\\       
  6-oil     \textsc{sm6-prs}{}-smell-\textsc{pass-fv}   (by  18-7-room)\\ \jambox*{[Bena]}
\glt  ‘Oil is smelled (in the room)’}


\ex[*]{\label{ex:mapunda:39} 
\gll \textbf{ma-huta}  ma-inung’-iw-a    (ni  mu-chumba) \\    
  6-oil     \textsc{sm6}{}-smell-\textsc{pass-fv}  (by  18-7.room)\\ \jambox*{[Ngoni]}
\glt  ‘Oil is smelled (in the room)’}


\ex[*]{\label{ex:mapunda:40} 
\gll \textbf{ma-huta}  ma-kungung-\textbf{w}{}-a  (ni  mu-chumba) \\    
  6-oil     \textsc{sm6}{}-smell-\textsc{pass-fv}   (by  18-7.room)\\ \jambox*{[Yao]}
\glt  ‘Oil is smelled (in the room)’}


\ex[*]{\label{ex:mapunda:41}
\gll \textbf{ma-khura}  ma-no-nukh-\textbf{iy}{}-a    (ni  m-chumba-ni)\\   
  6-oil     \textsc{sm6-prs}{}-smell-\textsc{pass-fv}   (by  8-7.room-\textsc{loc})\\ \jambox*{[Makhuwa]}
\glt  ‘Oil is smelling (in the room)’}
\z

Third, the logical subject cannot be extracted by relativization, as seen in \REF{ex:mapunda:42}--\REF{ex:mapunda:45}.


\ea[*]{\label{ex:mapunda:42}
\gll ani   ye   mu-kaye   i-vemb-a? \\ 
  who   1.\textsc{dem}  18-9.house   \textsc{sm1.prs}{}-cry-\textsc{fv.rel}\\ \jambox*{[Bena]}
\glt  ‘Who is it that is crying in the house?’}


\ex[*]{\label{ex:mapunda:43}
\gll yani   mwe   mu-nyumba   i-vemb-a? \\  
who   1.\textsc{dem}  18-9house   \textsc{sm1.prs}{}-cry-\textsc{fv.rel} \\ \jambox*{[Ngoni]}
\glt  ‘Who is it that is crying in the house?’}


\ex[*]{\label{ex:mapunda:44}
\gll nduni   jwelejo   m-nyumba   a-ku-lil-a? \\   
    who   1.\textsc{dem}     18-9.house   \textsc{sm1.prs}{}-cry-\textsc{fv.rel} \\ \jambox*{[Yao]}
\glt  ‘Who is it that is crying in the house?’}


\ex[*]{\label{ex:mapunda:45}
\gll mpani  yo   m-nupa-ni     a-no-unl-a? \\ 
      who   1.\textsc{dem} 18-9.house-\textsc{loc}   \textsc{sm1-prs}{}-cry-\textsc{fv.rel} \\ \jambox*{[Makhuwa]}
\glt  ‘Who is it that is crying in the house?’}
\z

As noted by \citet{BresnanKanerva1989}, the impossibility to object-mark, to passivize and to relativize the postverbal logical subject of a LI construction, suggests that it is not a typical object complement of the verb. Yet, its inflexible immediate-after-the-verb position and its obligatory presence still liken it to a core argument rather than an adjunct.  

Argument structures involved in LI may differ. For example, \citet{DemuthMmusi1997} argue that in Tswana, LI is possible with active transitive verbs. In contrast, in Chewa, \citet{BresnanKanerva1989} observe that those verbs do not allow LI. In the four sampled languages, LI is possible with unaccusative verbs, i.e. intransitive verbs which take one argument with the semantic role of theme. The verb may in most cases also take a locative argument. Examples of these verbs are ‘smell’, ‘be full’, ‘spread’, and ‘germinate’. Examples in (\ref{ex:mapunda:18})--(\ref{ex:mapunda:21}) above illustrate the point with the verb ‘smell’. However, LI is no longer possible when unaccusative verbs are used in the passive voice. Examples in (\ref{ex:mapunda:46})--(\ref{ex:mapunda:49}) illustrate this point. 


\ea[*]{\label{ex:mapunda:46}
\gll ku-sh{}-umba  ku-i-nung’{}-w-a    a-ma-futa  (na va-ana)\\   
17-7-room   \textsc{sm17-prs}{}-smell-\textsc{pass-fv}  \textsc{aug}{}-6-oil  (by 2-child)\\ \jambox*{[Bena]}
\glt  ‘To the room is being smelled the oil (by the children)’}


\ex[*]{\label{ex:mapunda:47}
\gll ku-ch-umba  ku-i-nung’{}-iw-a            ma-huta       (na va-ana)\\ 
17-7-room  \textsc{sm17}{}-\textsc{prs}{}-smell-\textsc{pass-fv}     6-oil      (by 2-child)\\ \jambox*{[Ngoni]}
\glt ‘To the room is being smelled the oil (by the children)’}


\ex[*]{\label{ex:mapunda:48}
\gll mu-ch-umba    mu-ku-nung-w-a    ma-huta   (ni va-ana)\\ 
  17-7-room-\textsc{loc}  \textsc{sm17-prs}{}-smell-\textsc{pass-fv}   6-oil       (by 2-child)\\ \jambox*{[Yao]}
\glt  ‘In the room is being smelled the oil (by the children)’}

\ex[*]{\label{ex:mapunda:49}
\gll n-ch-umba-ni     n-no-nukh-w-a         ma-khura    (na ashana) \\ 
  18-7-room-\textsc{loc}  \textsc{sm18-prs}{}-smell-\textsc{pass-fv}   6-oil           (by 2.child)\\ \jambox*{[Makhuwa]}
\glt  ‘In the room is being smelled the oil (by the children)’}
\z

On the other hand, unergative verbs do not allow LI. Unergative verbs are intransitive verbs that are semantically distinguished by having an agent argument. Examples of these verbs are ‘vomit’, ‘defecate’, ‘run’, and ‘cry’. Bena examples in (\ref{ex:mapunda:50}) illustrate the point with the verb \textit{vemba} ‘cry’.


\ea\label{ex:mapunda:50}
    \ea[*]{\label{ex:mapunda:50a}
        \gll mu-shumba    mu-vemb-a  mw-ana \\ 
      18-7.room    \textsc{sm18}{}-cry-\textsc{fv}  1-child\\ \jambox*{[Bena]}
     \glt ‘The child is crying in the room.’}

    \ex[*]{\label{ex:mapunda:50b}
        \gll mu-chumba  mu-vemb-a  mw-ana  \\ 
      18-7.room    \textsc{sm18}{}-cry-\textsc{fv}  1-child\\ \jambox*{[Ngoni]}
     \glt ‘The child is crying in the room.’}

    \ex[*]{\label{ex:mapunda:50c}
    \gll mu-ki-yumba  mu-vemb-a  mw-ana \\ 
      18-7-room    \textsc{sm18}{}-cry-\textsc{fv}  1-child\\ \jambox*{[Yao]}
     \glt ‘The child is crying in the room.’}

    \ex[*]{\label{ex:mapunda:50d}
    \gll n-chumba-ni  n-no-unl-a    mw-ana \\ 
      18-7.room-loc  \textsc{sm18}{}-prs-cry-\textsc{fv}  1-child\\ \jambox*{[Makhuwa]}
     \glt ‘The child is crying in the room.’}
     \z
\z

In the same way, passivised unergative verbs cannot appear in LI. Example in (\ref{ex:mapunda:51}) illustrates the point with the verb \textit{vembwa} ‘cried by’.


\ea\label{ex:mapunda:51}
    \ea[*]{\label{ex:mapunda:51a}  
    \gll mu-sh{}-umba  mu-vemb-w-a    (ni mw-ana) \\ 
  18-7-room    \textsc{sm18}{}-cry-\textsc{pass-fv}  (by 1-child)\\ \jambox*{[Bena]}
  \glt ‘It is being cried in the room (by the child).’}

    \ex[*]{\label{ex:mapunda:51b}
    \gll mu-ch{}-umba  mu-vemb-w-a    (ni mw-ana) \\ 
  18-7-room    \textsc{sm18}{}-cry-\textsc{pass-fv}  (by 1-child)  \\ \jambox*{[Ngoni]}
  \glt ‘It is being cried in the room (by the child).’}

    \ex[*]{\label{ex:mapunda:51c}
    \gll mu-ki-yumba  mu-vemb-w-a    (ni mw-ana) \\ 
  18-7-room  \textsc{sm18}{}-cry-\textsc{pass-fv}  (by 1-child)\\ \jambox*{[Yao]}
  \glt ‘It is being cried in the room (by the child).’}

    \ex[*]{\label{ex:mapunda:51d}
    \gll n-ch-umba-ni  n-no-unl-w-a      (ni mw-ana) \\ 
   18-7-room  \textsc{sm18}{}-prs-cry-\textsc{pass-fv}  (by 1-child)\\ \jambox*{[Makhuwa]}
   \glt ‘It is being cried in the room (by the child).’}
    \z
\z

Transitive verbs, which add a thematic object to the argument structure, fail to undergo LI. This is expected when the thematic object precedes the inverted subject, as the latter necessarily follows the verb. The order inverted subject-theme is nevertheless just as ungrammatical. Infelicitous examples are provided in (\ref{ex:mapunda:52})--(\ref{ex:mapunda:55}) with the verbs ‘cultivate’ and ‘put’. 


\ea\label{ex:mapunda:52}
    \ea[*]{\label{ex:mapunda:52a}   \gll a-pa-ono               pa-limil-e    i-ki-tu    kuku  \\ 
   \textsc{aug}{}-16-place  \textsc{sm16}{}-cultivate-\textsc{prf}  \textsc{aug}{}-7-thing  1.grandfather\\ \jambox*{[Bena]}
   \glt ‘Grandfather has cultivated something on the place.’}

    \ex[*]{\label{ex:mapunda:52b}   \gll a-pa-ono         pa-limil-e    kuku               i-ki-tu     \\
   \textsc{aug}{}-16-place    \textsc{sm16}{}-cultivate-\textsc{prf}  1.grandfather    \textsc{aug}{}-7-thing  \\
   \glt ‘Grandfather has cultivated something on the place.’}
   \z


\ex\label{ex:mapunda:53}
    \ea[*]{\label{ex:mapunda:53a}  \gll ap-a          naha    pa-limil-e    chi-tu    gogu \\ 
   16-\textsc{dem}    int       \textsc{sm16}{}-cultivate-\textsc{prf}  7-thing    1.grandfather\\ \jambox*{[Ngoni]}
   \glt ‘Grandfather has cultivated something on the place.’}

    \ex[*]{\label{ex:mapunda:53b}  \gll ap-a          naha    pa-limil-e    gogu          chi-tu    \\
   16-\textsc{dem}    int     \textsc{sm16}{}-cultivate-\textsc{prf}  1.grandfather  7-thing    \\
  \glt ‘Grandfather has cultivated something on the place.’}
  \z


\ex\label{ex:mapunda:54}
    \ea[*]{\label{ex:mapunda:54a}   \gll pa-m-keka   pa-vichil-e  chi-ndu  baba \\ 
    16-3-mat  \textsc{sm16}{}-put-\textsc{prf}  7-thing     1.father\\ \jambox*{[Yao]}
    \glt  ‘Father has put something on the mat.’}

    \ex[*]{\label{ex:mapunda:54b}  \gll pa-m-keka    pa-vichil-e    baba     chi-ndu \\
  16-3-mat      \textsc{sm16}{}-put-\textsc{prf}  1.father  7-thing     \\
   \glt ‘Father has put something on the mat.’}
   \z


\ex\label{ex:mapunda:55}
    \ea[*]{\label{ex:mapunda:55a}   \gll va-m-pasa-ni   va-ho-wesh-a    i-tu  athatha \\ 
       16-3-mat-\textsc{loc}  \textsc{sm16}{}-\textsc{prf}{}-put-\textsc{fv}  7-thing   1.father\\ \jambox*{[Makhuwa]}
       \glt ‘Father has put something on the mat.’}

    \ex[*]{\label{ex:mapunda:55b}   \gll va-m-pasa-ni   va-ho-wesh-a    athatha    i-tu \\
       16-3-mat-\textsc{loc}  \textsc{sm16}{}-\textsc{prf}{}-put-\textsc{fv}  1.father  7-thing\\
       \glt ‘Father has put something on the mat.’}
       \z
\z

On the other hand, passivized transitive verbs do allow LI. Examples in (\ref{ex:mapunda:56})--(\ref{ex:mapunda:59}) illustrate the point with the verb ‘being put’.


\ea\label{ex:mapunda:56}
\gll a-pa-ono   pa-limil-w-e    i-ki-tu    (ni  kuku) \\ 
  \textsc{aug}{}-16-place  \textsc{sm16}{}-put-\textsc{pass-prf}  \textsc{aug}{}-7-thing  (by   1.grandfather)\\ \jambox*{[Bena]}
\glt  ‘Something has been cultivated on the place (by grandfather).’


\ex\label{ex:mapunda:57}
\gll ap-a       naha    pa-lim-iw-e        i-ki-tu       (na  gogu) \\ 
16-place  int \textsc{sm16}{}-put-\textsc{pass-prf}    \textsc{aug}{}-7-thing    (by  1.grandfather)\\ \jambox*{[Ngoni]}
\glt  ‘Something has been cultivated on the place (by grandfather).’


\ex\label{ex:mapunda:58} 
\gll pa-m-keka   pa-vichil-w-e    chi-ndu  (ni  baba) \\ 
  16-3-mat  \textsc{sm16}{}-put-\textsc{pass-prf}  7-thing     (by  1.father)\\ \jambox*{[Yao]}
\glt   ‘Something has been put on the mat (by father).’


\ex\label{ex:mapunda:59}
\gll va-m-pasa-ni   va-ho-wesh-iy-a  i-tu     (ni  athatha) \\ 
  16-3-mat-\textsc{loc}  \textsc{sm16}{}-\textsc{prf}{}-put-\textsc{fv}  7-thing   (by  1.father)\\ \jambox*{[Makhuwa]}
\glt   ‘Something has been put on the mat (by father).’
\z

\tabref{tab:mapunda:6} summarizes the findings for LI. As argued above, the four sampled languages behave alike, both in terms of types of LI allowed (formal versus semantic) and the interaction between LI and the argument structure. 

\begin{table}
\caption{\label{tab:mapunda:6}Locative inversion in Bena, Ngoni, Yao, and Makhuwa}

\begin{tabular}{lcccc}

\lsptoprule

{Parameters} & {Bena} & {Ngoni} & {Yao} & {Makhuwa}\\
\midrule
\multicolumn{5}{l}{\textit{Types of LI}}\\


formal agreeing LI
 & \ding{51} & \ding{51} & \ding{51} & \ding{51}\\


semantic agreeing LI
 & \ding{55} & \ding{55} & \ding{55} & \ding{55}\\
 \midrule
\multicolumn{5}{l}{\textit{LI and argument structure}}\\

active unaccusative verb 
 & \ding{51} & \ding{51} & \ding{51} & \ding{51}\\


passive unaccusative verb 
 & \ding{55} & \ding{55} & \ding{55} & \ding{55}\\


active unergative verb 
 & \ding{55} & \ding{55} & \ding{55} & \ding{55}\\


passive unergative verb 
 & \ding{55} & \ding{55} & \ding{55} & \ding{55}\\


active transitive verb 
 & \ding{55} & \ding{55} & \ding{55} & \ding{55}\\


passive transitive verb 
 & \ding{51} & \ding{51} & \ding{51} & \ding{51}\\
\lspbottomrule
\end{tabular}
\end{table}

\section{Conclusion}\label{sec:mapunda:5} %5. /

This paper has provided a comparative description of the locative system of four South-Tanzanian Bantu languages, namely Bena, Ngoni, Yao and Makhuwa. The study shows that these languages overall exhibit similar locative constructions with similar properties. This is particularly clear with LI constructions, which show identical properties. Furthermore, the four languages make a productive use of the three historical locative prefixes of class 16, 17 and 18 in both nominal and verbal domains. While Bena and Yao are strictly identical for all properties discussed in this paper, Ngoni differs from the three others in that it does not allow outer agreement within NPs (except with demonstrative modifiers). The most notable differences come from Makhuwa. In this language, in addition to locative prefixation, locative nouns are further marked with a locative suffix -\textit{ni}. The only cases of exception are lexicalized locatives and nouns refering to administrative-geographical entities, such as names of towns or countries. What looks like double affixation in Bena, Ngoni or Yao is attested in loanwords only, especially from Swahili. Furthermore, Makhuwa is the only sampled language which does not allow inner agreement within NPs. Only outer agreement is attested. One last major difference observed in Makhuwa is the absence of a full paradigm of object prefixes. The system eroded to such a point that only classes 1/2 have object agreement markers in the language. In contrast, Bena, Ngoni and Yao have full object markers paradigms, which includes locative object markers.

Bena and Yao, in spite of sharing identical locative features, are  geographically not proximal. In fact, and as already shown in Map 1, Bena and Yao areas are separated by the Ngoni linguistic group. Influence from Swahili, as a lingua franca across north-eastern Bantu, is perceptible in all four languages, with lexical borrowing of words such as \textit{mafuta/mahuta} ‘oil’, or in Bena and Ngoni, \textit{chumba} ‘room’, and \textit{lima} ‘cultivate’. As far as the locative system is concerned, however, only Makhuwa seems to have been more directly affected by Swahili through the suffixation of \textit{{}-ni} on locativized nouns. Beyond Swahili influence, the few examples retrieved in this paper may not warrant any conclusion on mutual influence within the sampled languages.

Avenues for future research would at least involve extending the study to include locative verbal enclitics which have been excluded from this paper because of a lack of clear data in the selected languages and the difficulty to further investigate on them ex situ. As explained in the introduction, the languages surveyed here represent a convenience sample. Further light could be shed on the micro-variation of locative systems in Eastern Bantu through a broader comparative work covering a certain number of Eastern Bantu languages to see how our four sampled languages fit in a wider geographical area. 


\section*{Abbreviations}
\begin{multicols}{2}
\begin{tabbing}
TAM \= Tense Aspect Mood\kill
\textsc{fv} \> Final Vowel\\
\textsc{loc} \> Locative\\
\textsc{om} \> Object Marker\\
\textsc{prf} \> Perfective\\
\textsc{prs} \> Present\\
\textsc{pass} \> Passive\\
\textsc{rel} \> Relative\\
\textsc{sm} \> Subject Marker\\
\textsc{tam} \> Tense Aspect Mood\\
\textsc{dem} \> demonstrative\\
\textsc{int} \> intensifier\\
\end{tabbing}
\end{multicols}

\sloppy\printbibliography[heading=subbibliography,notkeyword=this]
\end{document} 
