\documentclass[output=paper]{langscibook}
\ChapterDOI{10.5281/zenodo.10663761}

\author{Lutz Marten\orcid{}\affiliation{SOAS University of London} and Hannah Gibson\orcid{}\affiliation{University of Essex} and Rozenn Guérois\orcid{}\affiliation{LLACAN CNRS; University of KwaZulu-Natal} and  Gastor Mapunda\orcid{}\affiliation{University of Dar es Salaam}}

\title{Introduction}
\abstract{}
\IfFileExists{../localcommands.tex}{
  \addbibresource{../localbibliography.bib}
  % add all extra packages you need to load to this file

\usepackage{tabularx,multicol}
\usepackage{url}
\urlstyle{same}

\usepackage{listings}
\lstset{basicstyle=\ttfamily,tabsize=2,breaklines=true}

\usepackage{langsci-basic}
\usepackage{langsci-optional}
\usepackage{langsci-lgr}
\usepackage{langsci-osl}
% \usepackage{./langsci/styles/langsci-lgr}
% \usepackage{./langsci/styles/langsci-osl}
% \usepackage{langsci-gb4e}

\usepackage{tikz}
\usetikzlibrary{patterns,calc}
\pgfdeclarepatternformonly{south east lines}{\pgfqpoint{-0pt}{-0pt}}{\pgfqpoint{3pt}{3pt}}{\pgfqpoint{3pt}{3pt}}{
    \pgfsetlinewidth{0.6pt}
    \pgfpathmoveto{\pgfqpoint{0pt}{3pt}}
    \pgfpathlineto{\pgfqpoint{3pt}{0pt}}
    \pgfpathmoveto{\pgfqpoint{.2pt}{-.2pt}}
    \pgfpathlineto{\pgfqpoint{-.2pt}{.2pt}}
    \pgfpathmoveto{\pgfqpoint{3.2pt}{2.8pt}}
    \pgfpathlineto{\pgfqpoint{2.8pt}{3.2pt}}
    \pgfusepath{stroke}}
    
\usepackage{stmaryrd}
\usepackage{wasysym}
\usepackage{multirow}
\usepackage{caption}
\usepackage{subcaption}
\usepackage{mathrsfs}
\usepackage{qtree}

\usepackage{linguex}


  %pminos do not split footnotes
% \interfootnotelinepenalty=10000 %Footnote in Laporte chapters has to be split SN


%\DeclareIndexNameFormat{default}{%
%\nameparts{#1}%
%\usebibmacro{index:name}%
%{\index[names]}%
%{\namepartfamily}%
%{\namepartgiveni}%
% {}% L1
% {}% L2
%{\namepartprefix}% generates spurious space L3
%{\namepartsuffix}% generates spurious space L4
%}

%  {\DeclareIndexNameFormat{default}{%
%     \usebibmacro{index:name}{\index[names]}{#1}{#3}{#5}{#7}}}

%\DeclareIndexNameFormat{default}{%
%  \usebibmacro{index:name}{\sindex[nom]}{#1}{#3}{#5}{#7}}

%\DeclareIndexNameFormat{default}{%
%  \usebibmacro{index:name}{\sindex[person]}{#1}{#3}{#5}{#7}}
%\DeclareIndexNameFormat{default}{%
%\nameparts{#1} \usebibmacro{index:name}{\sindex[person]]}{\namepartfamily}{‌​\namepartgiven}{\nam‌​epartprefix}{\namepa‌​rtsuffix}}

%\newcommand{\smiley}{:)}

%\renewbibmacro*{index:name}[5]{%
%\usebibmacro{index:entry}{#1}%
%{\iffieldundef{usera}{}{\thefield{usera}\actualoperator}\mkbibindexname{#2}{#3}{#4}{#5}}}

% \newcommand{\noop}[1]{}

%remove for final
%\overfullrule=1mm

\newcommand{\tobi}[2]}}
\renewcommand{\S}[1]{\tobi{#1}{\textsc{*}}}

% this volume references
% puts: [this volume]
% already defined: \citetv
%\newcommand{\citepv}[1]{(\citeauthor{#1} \citeyear*{#1} [this volume])}
\newcommand{\citealtv}[1]{\citeauthor{#1} \citeyear*{#1} [this volume]}

%parentheses around example number
\newcommand{\pref}[1]{(\ref{#1})}

% in-text examples

\newcommand{\lnex}[1]{\textit{#1}} %target lang word
\newcommand{\lnlit}[1]{(lit.: `#1')} %literal reading
\newcommand{\lnlat}[1]{(#1)} % latinization
\newcommand{\lntrans}[1]{`#1'} %translation
\newcommand{\lnexl}[2]%
{\lnex{#1}{} \lnlat{#2}} % ex with latinization
\newcommand{\lnexlat}[3]{\lnex{#1}{} \lnlat{#2}{} \lntrans{#3}} % ex with latinization and tranl.

%ch01
\newcommand{\co}[1]{\mbox{\textbf{#1}}}

%ch09

\newcommand{\cyrbulg}[1]{\begin{otherlanguage*}{bulgarian}#1\end{otherlanguage*}}


%ch10
\newcommand{\nlp}{{\small NLP}}
\newcommand{\mwe}{{\small MWE}}
\newcommand{\rae}{{\small RAE}}
\newcommand{\lvc}{{\small LVC}}
\newcommand{\pos}{{\small P}o{\small S}}
%\newcommand{\todo}[1]{ \textcolor{red}{#1} }

%\renewcommand{\labelenumi}{\theenumi}
%\ainamefmt{{vv}{ll}{, ff}{, jj}} % fullname

\newcommand{\biberror}[1]{{\color{red}#1}}

\newcommand{\osenovaitem}{--~} 
  %% hyphenation points for line breaks
%% Normally, automatic hyphenation in LaTeX is very good
%% If a word is mis-hyphenated, add it to this file
%%
%% add information to TeX file before \begin{document} with:
%% %% hyphenation points for line breaks
%% Normally, automatic hyphenation in LaTeX is very good
%% If a word is mis-hyphenated, add it to this file
%%
%% add information to TeX file before \begin{document} with:
%% %% hyphenation points for line breaks
%% Normally, automatic hyphenation in LaTeX is very good
%% If a word is mis-hyphenated, add it to this file
%%
%% add information to TeX file before \begin{document} with:
%% \include{localhyphenation}
\hyphenation{
    Beck-man
    Ngu-yen
    back-chan-nel
    back-chan-nels
    mo-not-o-nous
    ste-reo-typ-i-cal
}

\hyphenation{
    Beck-man
    Ngu-yen
    back-chan-nel
    back-chan-nels
    mo-not-o-nous
    ste-reo-typ-i-cal
}

\hyphenation{
    Beck-man
    Ngu-yen
    back-chan-nel
    back-chan-nels
    mo-not-o-nous
    ste-reo-typ-i-cal
}
 
  \togglepaper[1]%%chapternumber
}{}

\begin{document}
\maketitle 
%\shorttitlerunninghead{}%%use this for an abridged title in the page headers

\section{Background}
\noindent The present volume \textit{Morphosyntactic Variation in East African Bantu Languages} has, as indicated in the title, three interacting foci of interest: Variation in morphosyntax, the study of Bantu languages, and a regional focus on East Africa. Each of these foci deserves a little bit of discussion to illuminate the motivation for this book. 

Morphosyntactic variation is a comparatively recent field of study in the wider domain of comparative linguistics. Both phonological and lexical comparative studies have been an established part of the field for a long time, especially in historical linguistics. With respect to Bantu languages, a proto language had been reconstructed by the end of the nineteenth century, based on phonological and lexical reconstruction, although even early studies of Bantu languages reflected an interest in morphology and, to a lesser extent, syntax (e.g. \citealt{Bleek1862}, \citealt{Meinhof1906}). However, recent years have seen an impressive growth in research examining morphosyntactic variation in Bantu languages, highlighted, for example, in the volume on comparative Bantu grammar edited by \citealt{Mchombo1993}). Work in this tradition includes studies which look at specific construction types from a cross-Bantu perspective or in a given language, such as the examinations of double object constructions (\citealt{BresnanMoshi1990,Rugemalira1993obj}), locative inversion (\citealt{DemuthMmusi1997,Morimoto2000,Khumalo2010}) and object marking (\citealt{Beaudoin-LietzEtAl2004,Riedel2009}).

A particular approach to the study of morphosyntactic variation in Bantu, following \citet{MartenEtAl2007}, employs a set of surface parameters or variables to conduct systematic comparative studies often involving a larger group of languages (e.g. \citealt{BaxDiercks2012,PetzellHammarström2013,MartenvanderWal2014,ZellerNgoboka2015,Mtenje2016,ShinagawaAbe2019,ShinagawaMarten2021}). A comprehensive list of such parameters is developed in \citet{GuéroisEtAl2017} and the approach is also adopted by some of the papers in the current volume. Two further recent studies in the field of morphosyntactic variation and comparative grammar are the edited volumes by \citet{BostoenEtAl2022}, which presents historical-comparative reconstructions for a number of Bantu grammatical structures, and Bloom Ström et al. (forthcoming), which contains chapters on variation from across the Bantu-speaking area.

The approximately 500 Bantu languages spoken across vast areas of Central, Eastern and Southern Africa provide the wider empirical backdrop to the present volume. Bantu languages are united by the presence of a number of broad typological similarities, including, for example, complex noun class systems, agglutinative verbal morphology with a rich array of verbal affixes, and basic SVO word-order which is subject to pragmatically motivated word-order variation. However, within this similarity, the languages also exhibit a high degree of fine-grained micro-variation across all linguistic domains. This micro-variation results in part from independent diachronic developments such as processes of grammaticalisation and reanalysis, and in part from language contact both between Bantu languages and between Bantu languages and neighbouring non-Bantu languages. The high number of different Bantu languages and lects, the geographic density of the Bantu-speaking area, and the specific and often multilingual ecologies in which Bantu languages are spoken make the language group an important area of research for our understanding of developments and processes of morphosyntactic variation.

The regional focus of the volume is East Africa. Our conception of East Africa is inclusive, and the papers in the volume report on research on languages spoken in Kenya, Tanzania, Uganda, Malawi, Mozambique and the Democratic Republic of the Congo. Linguistically speaking, East Africa is a diverse area, in which languages of four African linguistic families are spoken and which is also home to the Rift Valley linguistic area (\citealt{KiesslingEtAl2007,HarveyEtAlForthcoming}).

There is a long tradition of comparative work on Eastern Bantu languages. Eastern Bantu was the subject of the lexicostatistical survey of \citet{NursePhilippson1980}, while \citet{HinnebuschEtAl1981} examined several phonological and morphological features of languages in this area. \citet{Nurse1985} then addresses the question of phonological areal features in North-Eastern Bantu, and (\citealt{NurseMuzale1999}) examined changes in tense and aspect in the same linguistic zone. A comprehensive historical-comparative reconstruction of Proto-Sabaki was developed in \citet{NurseHinnebusch1993}. More recently, Nicolle has developed three studies concentrated on Eastern Bantu languages, two of which deal with discourse strategies in narrative texts (\citealt{Nicolle2014,Nicolle2015}), and a third which compares the expression of information structure \citep{Nicolle2016}. In terms of Bantu linguistic classification, however, the area has not been identified with a specific sub-group of the family. Rather, in most lexically-based classifications, East African Bantu languages are grouped together with Southern and Central Bantu languages as part of the larger ‘Osthochland Gruppe’ \citep{HeineEtAl1977} or the ‘Eastern’ group \citep{GrollemundEtAl2015}. On the other hand, in terms of non-lexical data, the languages of the East African region in particular have been noted to share high degrees of structural similarity (e.g. \citealt{HinnebuschEtAl1981}). More recently, \citet{EdelstenEtAl2022} identify several morphosyntactic aspects which may serve to distinguish East Bantu languages from non-East Bantu languages. These include a symmetric pattern in ditransitive constructions, negation marking in dependent clauses by a post-verbal negative marker, widespread subject inversion constructions, and the co-occurrence of formal and semantic locative inversion constructions.

\section{Origins of the book}

The book, in part, has its origins in two closely related projects. The first was the Leverhulme Trust funded project ‘Morphosyntactic Variation in Bantu: Typology, contact and change’ (RPG-2014-208), which was led by Professor Lutz Marten and based at SOAS University of London (2014-2018). The project aimed to investigate linguistic similarities in a sample of Bantu languages, with a view to better understanding how the structures of different Bantu languages have been shaped by the interaction of processes of historical innovation, language contact, and universal functions of human language. 

One of the key outputs of the project was the development of a list of 142 descriptive level parameters of morphosyntactic variation \citep{GuéroisEtAl2017}, and the creation of a database which enabled the storage and representation of data related to languages of the sample with respect to these parameters – the Bantu Morphosyntactic Variation (BMV) database \citep{MartenEtAl2018}. The project set up partnerships and collaborations with researchers in Africa, Europe and Asia. Additionally, as part of the project, a workshop on approaches to morphosyntactic variation in Bantu was held at the University of Dar es Salaam on 20/21 {July 2016}, in which some of the work reported in this volume was first discussed.

The second project was a collaboration between researchers at SOAS University of London and the University of Dar es Salaam, and was led by Professor Lutz Marten and Dr Gastor Mapunda. This project was funded by the British Academy and was entitled ‘Parametric approaches to morphosyntactic variation in Eastern Bantu languages’ (2017-18). The goal of the project was to build on aspects of morphosyntactic variation which had previously been described for Bantu languages and on the emerging parametric approach, and to extend this to grammatical variation in Eastern Bantu. The project sought to build on original empirical evidence from some twenty Bantu languages of Eastern Africa, with a view to contributing to a better understanding of the historical, comparative and typological patterns that have shaped the linguistic landscape of East Africa. In specific terms, the project enabled a second workshop which was held at the University of Dar es Salaam 13-15 {September 2017}, bringing together researchers working on East African Bantu in a practical, interactive session to explore the parametric approach. The exchange also enabled collaborators at the University of Dar es Salaam to visit SOAS.

Many of the chapters in the current volume have their origins in the workshops that took place as part of this project. However, we were also fortunate enough to have received a number of contributions from those who did not attend the original workshops, which have further strengthened and broadened the empirical coverage and the theoretical and methodological breadth of the volume. 

\section{Chapters in the volume}

Against this background, the present volume includes chapters which offer both comparative and descriptive accounts of Bantu languages spoken in East Africa. Several of the languages discussed (e.g. Shinyiha, Runyankore-Rukiga, Kiwoso, Kihehe and Sumbwa) have not been the subject of previous extensive descriptions. The volume thus presents new empirical data, improving the descriptive status of the languages discussed in the volume. In addition, and including other more well-known Bantu languages (e.g. Swahili, Nyakyusa, Ciyao and Sena), the contributions present new or little-treated aspects of their morphosyntax, as well as providing novel data. The results presented in the volume enable close morphosyntactic comparison between languages spoken in the specific geographic area, some of which are in contact with each other. 

The volume consists of twelve chapters, in addition to this introduction (Chapter 1). These chapters are grouped into three sections, devoted to the examination of the nominal domain, the verbal domain, and analyses adopting comparative and historical approaches. A number of the chapters address the question of morphosyntactic variation through an in-depth examination of a single morphosyntactic phenomenon in a small sample of languages. Others embrace a wider perspective with more parameters of variation throughout the northeastern region.

Chapter 2, by Julius Taji, presents a description of the form, function and distribution of demonstratives in Chiyao. The form of the Chiyao demonstrative is shown to be determined by the location of the referent in relation to the speaker or hearer. In terms of distribution, the demonstrative can appear either in pre- or post-nominal position. Ciyao also employs circumdemonstratives in which a demonstrative appears both before and after the nominal. The demonstrative is shown to exhibit a range of grammatical and communicative functions: in addition to expressing the location of an entity in relation to the interlocutors, it can also express emphasis, definiteness and encode anaphoric reference.  

Chapter 3, by Allen Asiimwe, examines demonstratives in Runyankore-Ru\-ki\-ga, a language of Uganda. The chapter explores the more common functions of demonstratives observed across Bantu, such as encoding proximity of the referent to the speaker/hearer. However, the study also reports on less frequently examined features such as the nominal and verbal properties of demonstratives and their use to express manner. Pragmatic properties of Runyankore-Rukiga demonstratives, which are divided into exophoric and endophoric demonstratives (the former used for non-anaphoric functions), are also explored. The study draws on data from written literary sources, spontaneous speech and data gathered via elicitation.

Chapter 4, by Amani Lusekelo, discusses the distribution and function of the augment and object markers in Nyakyusa, spoken in Tanzania. Adopting a parametric approach \citep{GuéroisEtAl2017}, and including both bare nouns and complex noun phrases, the chapter focuses on the (non-)occurrence of the V-augment and CV-particle, the role of demonstratives, and the word-order within the noun phrase. It shows that the main role of the CV-particle is to indicate contrastive focus of the referent, while for the anaphoric demonstrative \textit{-la} ‘that/those’, the augment and object marking are related to the indication of definiteness. In addition, it is shown that object marking can be optional or obligatory, depending on the verb.

Chapter 5, by Aurelia Mallya, discusses morphological and syntactic properties of locative expressions in the Tanzanian language Kiwoso. The study provides an account of the locative forms and their properties in relation to nominal and verbal morphology. While the locative class 17 prefix \textit{ku-} is used to show agreement on all nominal and verbal modifiers in Kiwoso, the nominal locative prefixes \textit{ku}-, \textit{pa}-, \textit{mu}- are unproductive in the language. Kiwoso also exhibits two post-final locative enclitics (=\textit{ho} and =\textit{u}) which are used to cross-reference locative objects. The chapter contributes to the understanding of locatives in Kiwoso as well as locatives in the Bantu language family in general.

Chapter 6, by Gastor Mapunda and Fabiola Hassan, presents a comparative analysis of locative expressions in Bena, Ngoni, Makhuwa, and Yao, i.e. four languages spoken around the Ruvuma River which separates Tanzania and Mozambique. The chapter shows that while these languages share similar features in terms of locative morphology, Makhuwa slightly departs from the three other languages in several respects. This study constitutes an exploration of micro-variation among languages which are typologically very similar.   

Chapter 7, by Nobuko Yoneda and Judith Nakayiza, focusses on the Ugandan language Ganda and examines how object noun phrases and object markers behave in multiple-object constructions, focussing in particular on the contexts in which asymmetry between these objects may emerge. The chapter makes two crucial contributions. First, Ganda allows three object markers, not only two as previously reported by other studies on this topic. Second, the symmetricity of the language is sometimes mitigated by semantic and phonological factors, suggesting that there can be variation between constructions within a language, and that “(a)symmetry” is probably not a parameter determined by language, but is more fine-grained.

Chapter 8, by Armindo Ngunga and Crisófia Langa da Câmara, presents an exploration of object marking in four languages of Mozambique. The study adopts six of the parameters of variation detailed in \citet{MartenKula2012} and examines the properties of these four languages with regard to these features. The study shows that the four languages exhibit three different patterns with regard to the obligatoriness of object marking and the co-occurrence of object markers. The authors also propose an additional parameter of variation which examines the obligatoriness of the presence of an object marker with certain transitive verbs. The study is micro-comparative in nature and furthers the descriptive status of these four languages as well as our understanding of variation in object marking in Bantu.

Chapter 9, by Kulikoyela Kahigi, describes the verbal extensions in the Tanzanian language Sumbwa and their valence in the context of Bantu comparative data, adopting the parametric approach of \citet{GuéroisEtAl2017}. The study reveals that the verb derivational strategies in Sumbwa closely follow those mapped out by the Proto-Bantu reconstructions, excluding a few innovations among the minor extensions (e.g. \textit{-agan-}, \textit{-agil-}). It also shows that the causative and instrumental share extensions, that the associative markers include the post-verbal \textit{-an-} and the pre-verbal \textit{-i-}, that the applicative conveys benefactive, directive, location, and reason meanings, and that there is no systematic fixed order of extensions, except that in all co-occurrences, the passive comes last.

Chapter 10, by Malin Petzell and Peter Edelsten, presents a review of the tense-aspect systems of five Bantu languages of Morogoro region in Tanzania: Kagulu, Luguru, Kami, Ndamba and Pogoro. The study shows that the languages investigated show significant diversity in TAM marking ranging from only two tenses (past and non-past) and limited aspectual distinctions to a system with multiple pasts and futures. The chapter discusses the distribution and meaning of these morphological distinctions, the abundance versus scarcity of specific tense-aspect markers, and the methods of expressing negation, thereby highlighting an unusual diversity in both the distribution and meaning of tense-aspect marking as well as negation across Bantu languages. 


Chapter 11, by Daisuke Shinagawa, presents a comparative overview of the tense and aspect systems in Kilimanjaro Bantu languages, including those from which comprehensive information about the tense-aspect system has not been previously made available in the literature. The chapter presents data from eight varieties of Kilimanjaro Bantu, namely Rwa, Siha, Mashami, Kibosho, Uru, Vunjo, Rombo-Mkuu, and Gweno. The data show a general picture of geographical distribution and formal correspondences of shared tense-aspect markers. The chapter also examines systematic correspondences – grammaticalisation chains – and explores the historic processes of change which has given rise to shared tense-aspect markers. The chapter is micro-parametric and comparative in nature and provides possible typological generalisations which might lie behind the variation found in the Kilimanjaro Bantu tense-aspect system.

Chapter 12, by Lengson Ngwasi and Abel Mreta\textsuperscript{†}, describes the historical development of reflexive-reciprocal polysemy in Kihehe by employing the three stages of the Overlap Model of Grammaticalisation Theory proposed by \citet{Heine1993}. The paper discusses an interesting feature of reflexive-reciprocal polysemy encoded by the originally reflexive prefix, a polysemy type reported in the grammar of several Bantu languages, but not in great detail (except for \citealt{BostoenForthcoming} on South-Western Bantu languages). The present paper is, therefore, an important contribution to the field, in that it provides an account of this polysemy type in a little-described language. It also contributes to our understanding of variation in (East) Bantu languages because this morphosyntactic phenomenon deviates from the common Bantu situation in which reflexivity and reciprocity are encoded by two different verbal morphemes.

Chapter 13, by Lutz Marten, Hannah Gibson, Rozenn Guérois and Kyle Jerro, compares written poetic texts of Old Swahili with present-day Standard Swahili in terms of morphosyntactic features developed in \citet{GuéroisEtAl2017}. Results of this study show significant differences between the two varieties. In particular, it shows that the relation between Old Swahili and Standard Swahili is characterised by a loss of morphosyntactic forms and a loss of variability. The authors argue that these results are likely to reflect the processes of standardisation and regularisation involved in the development of Swahili as a language of wider communication. The findings of the study shed new light on morphosyntactic variation since they show the effect of standardisation and a particular trajectory of morphosyntactic development.

Overall, the chapters brought together in this volume provide a snapshot of the state of the art in the study of morphosyntactic variation in the region, drawing on a range of languages and providing novel empirical data for many of them. The papers give a good impression of the variation encountered, and how different aspects of this play out in different languages. We hope that this contribution will lead to further work on the morphosyntax of East African Bantu languages, where much work remains to be done. 

\section{Editorial considerations}\largerpage

We make two further notes and observations here which are relevant for the volume as a whole. Firstly, we did not adopt a prescriptive approach to language naming and have instead left chapter authors to employ the naming conventions they consider fit for their chapters and which are used in the local context. This means that in some instances languages appear with their prefix – e.g. \textit{Kiswahili} – while in other contexts they may appear without the prefix – e.g. \textit{Swahili}. We acknowledge that there are different arguments for one convention or the other, and we are also aware that the question of names and naming of languages often has context-specific historical and political significance.

Secondly, we took a similar approach to the representation of data, glossing and abbreviations. We asked authors to be internally consistent in terms of how they represent data and the terms and abbreviations they use, in all cases encouraging the adaptation of the Leipzig Glossing Rules. However, in some chapters, authors have needed to use additional abbreviations, and have, thus, followed different styles that are widespread in their local contexts.

\section{Next steps}

It is our hope that the present volume serves as a reference point for those interested in Bantu languages in particular, as well as those interested in variation in morphosyntax and the East African region more broadly. It combines a number of different methodological approaches and insights, but also furthers the descriptive status of a number of the languages examined and mentioned herein, thereby serving as a reference point \-for future work. 

We hope that other scholars might be inspired by the work contained herein, as well as being exposed to data and findings which makes them reassess or revisit current work and emerging ideas, and so that the volume contributes to the further academic investigation and public awareness of African languages.

\section{Acknowledgements}\largerpage

As editors, we would like to extend our thanks to all the authors whose work contributes to the volume, as well as the attendees of the two workshops where this project started. We are also grateful to the British Academy and the Leverhulme Trust who generously funded the two workshops, as well both the project ‘Morphosyntactic variation in Bantu: typology, contact and change’ and ‘Parametric approaches to morphosyntactic variation in Eastern Bantu languages’.

We are grateful to all of the reviewers for their detailed comments and input they made on earlier versions of the chapters. Their comments served to strengthen and improve the clarity of the work. We are also grateful to the editors of the \textit{Contemporary African Linguistics} series for their initial and ongoing support of this project. We are also immensely grateful to the team at Language Science Press and here Sebastian Nordhoff must receive a special mention for being an unwavering source of support, enthusiasm and advice during the process of putting the book together.

\section{Dedication}

We would like to dedicate this volume to the late Dr Abel Mreta. Dr Mreta was a central member of the Department of Foreign Language and Linguistics at the University of Dar es Salaam, at one time the Head of the Department, and a staunch supporter of linguistic research on East African languages. He generously shared his knowledge and experience and he was involved in the training of a large proportion of the linguists working in Tanzania today.

Dr Mreta was born in Kilimanjaro Region, Tanzania. He completed his BA Education and MA Linguistics at the University of Dar es Salaam, and then his PhD at the University of Bayreuth, Germany. Dr Mreta was employed as a tutorial assistant at the University of Dar es Salaam in {May 1987}. His areas of interest were morphology, historical and comparative linguistics, sign language, and language documentation. In his career, he rose to the rank of Senior Lecturer in {May 2008}. He worked as a visiting lecturer at Leiden University (the Netherlands), Gothenburg University (Sweden) and Hankuk University of Foreign Studies (South Korea). Dr Mreta taught and mentored most of the linguists currently working in Tanzania and beyond, including many of the contributors to this volume.


In  {September 2017}, Dr Mreta participated in the workshop that discussed the parametric approach to the study of morphosyntactic variation in Bantu languages jointly organised by the Department of Foreign Languages and Linguistics (University of Dar es Salaam) and SOAS University of London, and it was at this workshop where Dr Mreta, together with other scholars, started working on their chapter contributions. Dr Mreta was a central contributor and supporter of the overall collaborative project and this book project. This contribution is in part reflected in the co-authored chapter by Dr Mreta and Lengson Ngwasi (Chapter~11). Dr Mreta passed on while the manuscripts were still in the review process. He will be remembered fondly by all who knew him and his legacy will live on through his work.

\printbibliography[heading=subbibliography,notkeyword=this]
\end{document}
