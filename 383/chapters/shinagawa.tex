\documentclass[output=paper]{langscibook}
\ChapterDOI{10.5281/zenodo.10663781}
		  
\author{Daisuke Shinagawa\orcid{}\affiliation{ILCAA, Tokyo University of Foreign Studies}}

\title[A comparative sketch of TA markers in Kilimanjaro Bantu]{A comparative sketch of TA markers in Kilimanjaro Bantu: In search of the directionality of semantic shift and micro-parametric correlation}

\abstract{This chapter presents a comparative overview of the tense and aspect (TA) systems in Kilimanjaro Bantu languages (KB), including those from which comprehensive information about the TA system has not been available in the literature. Fundamental description about the TA system of the eight varieties of KB, namely Rwa, Siha, Mashami, Kibosho, Uru, Vunjo, Rombo-Mkuu, and Gweno, reveals a general picture of geographical distribution and formal correspondences of shared TA markers. Based on the systematic correspondences, which can be described as grammaticalisation chains, we further discuss historical processes of semantic development of shared TA markers, as well as possible typological generalizations lying behind the observed variation of the TA systems in KB.}


\IfFileExists{../localcommands.tex}{
  \addbibresource{../localbibliography.bib}
  % add all extra packages you need to load to this file

\usepackage{tabularx,multicol}
\usepackage{url}
\urlstyle{same}

\usepackage{listings}
\lstset{basicstyle=\ttfamily,tabsize=2,breaklines=true}

\usepackage{langsci-basic}
\usepackage{langsci-optional}
\usepackage{langsci-lgr}
\usepackage{langsci-osl}
% \usepackage{./langsci/styles/langsci-lgr}
% \usepackage{./langsci/styles/langsci-osl}
% \usepackage{langsci-gb4e}

\usepackage{tikz}
\usetikzlibrary{patterns,calc}
\pgfdeclarepatternformonly{south east lines}{\pgfqpoint{-0pt}{-0pt}}{\pgfqpoint{3pt}{3pt}}{\pgfqpoint{3pt}{3pt}}{
    \pgfsetlinewidth{0.6pt}
    \pgfpathmoveto{\pgfqpoint{0pt}{3pt}}
    \pgfpathlineto{\pgfqpoint{3pt}{0pt}}
    \pgfpathmoveto{\pgfqpoint{.2pt}{-.2pt}}
    \pgfpathlineto{\pgfqpoint{-.2pt}{.2pt}}
    \pgfpathmoveto{\pgfqpoint{3.2pt}{2.8pt}}
    \pgfpathlineto{\pgfqpoint{2.8pt}{3.2pt}}
    \pgfusepath{stroke}}
    
\usepackage{stmaryrd}
\usepackage{wasysym}
\usepackage{multirow}
\usepackage{caption}
\usepackage{subcaption}
\usepackage{mathrsfs}
\usepackage{qtree}

\usepackage{linguex}


  %pminos do not split footnotes
% \interfootnotelinepenalty=10000 %Footnote in Laporte chapters has to be split SN


%\DeclareIndexNameFormat{default}{%
%\nameparts{#1}%
%\usebibmacro{index:name}%
%{\index[names]}%
%{\namepartfamily}%
%{\namepartgiveni}%
% {}% L1
% {}% L2
%{\namepartprefix}% generates spurious space L3
%{\namepartsuffix}% generates spurious space L4
%}

%  {\DeclareIndexNameFormat{default}{%
%     \usebibmacro{index:name}{\index[names]}{#1}{#3}{#5}{#7}}}

%\DeclareIndexNameFormat{default}{%
%  \usebibmacro{index:name}{\sindex[nom]}{#1}{#3}{#5}{#7}}

%\DeclareIndexNameFormat{default}{%
%  \usebibmacro{index:name}{\sindex[person]}{#1}{#3}{#5}{#7}}
%\DeclareIndexNameFormat{default}{%
%\nameparts{#1} \usebibmacro{index:name}{\sindex[person]]}{\namepartfamily}{‌​\namepartgiven}{\nam‌​epartprefix}{\namepa‌​rtsuffix}}

%\newcommand{\smiley}{:)}

%\renewbibmacro*{index:name}[5]{%
%\usebibmacro{index:entry}{#1}%
%{\iffieldundef{usera}{}{\thefield{usera}\actualoperator}\mkbibindexname{#2}{#3}{#4}{#5}}}

% \newcommand{\noop}[1]{}

%remove for final
%\overfullrule=1mm

\newcommand{\tobi}[2]}}
\renewcommand{\S}[1]{\tobi{#1}{\textsc{*}}}

% this volume references
% puts: [this volume]
% already defined: \citetv
%\newcommand{\citepv}[1]{(\citeauthor{#1} \citeyear*{#1} [this volume])}
\newcommand{\citealtv}[1]{\citeauthor{#1} \citeyear*{#1} [this volume]}

%parentheses around example number
\newcommand{\pref}[1]{(\ref{#1})}

% in-text examples

\newcommand{\lnex}[1]{\textit{#1}} %target lang word
\newcommand{\lnlit}[1]{(lit.: `#1')} %literal reading
\newcommand{\lnlat}[1]{(#1)} % latinization
\newcommand{\lntrans}[1]{`#1'} %translation
\newcommand{\lnexl}[2]%
{\lnex{#1}{} \lnlat{#2}} % ex with latinization
\newcommand{\lnexlat}[3]{\lnex{#1}{} \lnlat{#2}{} \lntrans{#3}} % ex with latinization and tranl.

%ch01
\newcommand{\co}[1]{\mbox{\textbf{#1}}}

%ch09

\newcommand{\cyrbulg}[1]{\begin{otherlanguage*}{bulgarian}#1\end{otherlanguage*}}


%ch10
\newcommand{\nlp}{{\small NLP}}
\newcommand{\mwe}{{\small MWE}}
\newcommand{\rae}{{\small RAE}}
\newcommand{\lvc}{{\small LVC}}
\newcommand{\pos}{{\small P}o{\small S}}
%\newcommand{\todo}[1]{ \textcolor{red}{#1} }

%\renewcommand{\labelenumi}{\theenumi}
%\ainamefmt{{vv}{ll}{, ff}{, jj}} % fullname

\newcommand{\biberror}[1]{{\color{red}#1}}

\newcommand{\osenovaitem}{--~} 
  %% hyphenation points for line breaks
%% Normally, automatic hyphenation in LaTeX is very good
%% If a word is mis-hyphenated, add it to this file
%%
%% add information to TeX file before \begin{document} with:
%% %% hyphenation points for line breaks
%% Normally, automatic hyphenation in LaTeX is very good
%% If a word is mis-hyphenated, add it to this file
%%
%% add information to TeX file before \begin{document} with:
%% %% hyphenation points for line breaks
%% Normally, automatic hyphenation in LaTeX is very good
%% If a word is mis-hyphenated, add it to this file
%%
%% add information to TeX file before \begin{document} with:
%% \include{localhyphenation}
\hyphenation{
    Beck-man
    Ngu-yen
    back-chan-nel
    back-chan-nels
    mo-not-o-nous
    ste-reo-typ-i-cal
}

\hyphenation{
    Beck-man
    Ngu-yen
    back-chan-nel
    back-chan-nels
    mo-not-o-nous
    ste-reo-typ-i-cal
}

\hyphenation{
    Beck-man
    Ngu-yen
    back-chan-nel
    back-chan-nels
    mo-not-o-nous
    ste-reo-typ-i-cal
}
 
  \togglepaper[1]%%chapternumber
}{}

\begin{document}
\maketitle 
%\shorttitlerunninghead{}%%use this for an abridged title in the page headers

\section{Introduction} %1. /
\label{sec:shinagawa:1}

This chapter provides comparative lists of tense and aspect markers (TAM) from eight Kilimanjaro Bantu languages (KB) covering all of the three major subgroups, namely Western Kilimanjaro (WK), Central Kilimanjaro (CK), and Rombo, with the goal of presenting an overall picture of the distribution and semantic variation of common TA markers in KB\footnote{For a linguistic overview, see \citet{PhilippsonMontlahuc2003} and \citet{ShinagawaForthcoming-a}.}. Based on the data, I will discuss the diachronic processes and typological implications of TA systems, namely i) the semantic development and grammaticalisation processes of TA markers across boundaries between languages and between subgroups, and ii) highlight micro-typological correlations between morphosyntactic parameters including those related to TA systems.

The languages presented in this chapter constitute two groups. The first group includes the languages described by the author with a full list of core TA markers, which includes Rwa (E621A\footnote{The five-digit codes shown in parentheses after language names are from the updated list of Guthrie codes by \citet{Maho2009}.}, WK), Siha (E621C, WK), Uru (E622D, CK), and the Mkuu variety of Rombo (E623C). The second group consists of languages with reliable descriptions of TA forms in the existing literature; this group includes Mashami (E621B, WK; \citealt{RugemaliraPhanuel2012}), Kibosho (Kiw’oso, E621D, WK; \citealt{Kagaya1989}), Vunjo (E622C, CK; \citealt{Nurse2003a}, \citealt{Moshi1994}), and Gweno (E65; \citealt{PhilippsonNurse2000}).

The study of TA systems in KB was pioneered by Derek Nurse. His \citetitle{Nurse2003a} \citep{Nurse2003a} provides a comprehensive overview of TA concepts and forms found in KB with a special focus on Vunjo. Following in this vein, this study intends to provide descriptive data for the languages which are only referred to in a limited way in his study (especially WK languages), as well as to present a more in-depth account of both historical and typological aspects of TA systems in KB.

 
\begin{figure}
\includegraphics[width=\textwidth]{figures/shinagawa-img001.pdf}
\\ {\tiny Maps © \url{www.thunderforest.com}, Data © \url{www.osm.org/copyright}}
\caption{Geographical distribution of the KB languages examined.}
\label{map:shinagawa:1}
\end{figure}

\largerpage
  It should be noted here that since the primary focus of this chapter is on the form-meaning correspondences of common TA markers in KB, the following topics are not included in the scope of this study: i) compound tenses, ii) forms with so-called limitatives (cf. \citealt{Meeussen1967}), iii) TA in negative constructions, and iv) modality markers.\footnote{Though these topics require further investigation in future research, the following points can be briefly mentioned as somewhat common features in KB: i) As for compound tenses, most of KB utilises the construction for aspectual forms with future tense, where the lexical stem meaning ‘find’ or ‘get’ (e.g. \textit{koóya} or related forms in WK) is used as an auxiliary conveying future tense. ii) The limitative \textit{ka-} is widely attested in KB with the meaning of consecutive; however, in Uru, it also expresses “ironical negation”, the use of which may have developed from “T inceptive” `already, not yet' in \citet[109]{Meeussen1967}. iii) Negation is also an essential part of the TA system and there seems to be crosslinguistic variation in terms of strategies of negation marking in relation to tense and aspect categories. For example, in Siha, NEG\textsubscript{2} \textit{ta-}, which is generally used as a non-main clause negation marker in other KB languages, has even spread to main clause verb forms; [Sih.] \textit{tikaváa} `We will hit' vs. \textit{ti}\textbf{\textit{ta}}\textit{kaváa pfo} `We will not hit'. iv) Most languages can take elements grammaticalized from ‘come’ and ‘go’ verbs in the TA slot, which essentially denote the modal concept that can be labelled as “certainty”. I will only briefly mention the forms when they are interpretable as denoting TA notions.} Although these points are of importance for a comprehensive understanding of TA expressions, it should be noted that the core system of TA forms can be structurally described independently from these elements.\footnote{As \citet[73]{Nurse2003a} mentions, generally in KB, negation in independent clauses is morphologically marked by a negative particle which has little influence on the morphological structure of the verb, i.e., it is relatively independent from broader TA-marking strategies.}

  This chapter is organised as follows. First, comparative lists of common TA markers from the above-mentioned languages are presented in \sectref{sec:shinagawa:2}, followed by a comparative analysis of semantic variation and its interrelation with each TA marker in \sectref{sec:shinagawa:3}. Based on these observations, the grammaticalisation processes and micro-typological correlations found in the TA systems of KB will be further discussed in \sectref{sec:shinagawa:4}. Conclusions are presented in \sectref{sec:shinagawa:5}.

\section{Comparative lists of TAM}\label{sec:shinagawa:2}

This section presents comparative lists of TAM of selected sample KB languages. Where available, sample sentences are tone-marked based on surface realizations, while the location of underlying lexical high tones is shown by an underscore. The lists are presented in the form of tables in which the morphological structure of the verb is indicated in rows and the core TA categories are shown in columns. 

The morphological template of the verb in KB basically follows the typical Bantu structure, i.e., \{Preinitial=SM-NEG\textsubscript{2}{}-TAM-OM${\neq}$Stem-Final=Postfinal\} (cf. \citealt{RoseEtAl2002}), out of which the structurally essential parts are SM, TAM, Stem, and Final. In addition to these elements, some of the lists include Preinitial\footnote{The Mashami table includes the Preinitial slot, simply because all the examples suitable for examining TA expressions are presented as forms with the element in the original source, i.e., it does not (necessarily) mean that the Preinitial (generally understood as a Focus marker) directly affects the TA notion of the verb.} and Postfinal slots, if they are relevant to the expressions of the core TA categories in the language in question.

Another point to be noted regarding the structure relates to the multiplicity of TAM slots. As the literature shows, a string of TAMs in a single verb structure is quite typical in KB. Reflecting this morphological feature, the list has three slots for TAMs to capture the gradual nature of the TAMs in a simplified way, i.e., TAM\textsubscript{0} and TAM\textsubscript{2} are positively defined on the scale and TAM\textsubscript{1} is defined as “in-between”. TAM\textsubscript{0} includes forms which are phonologically fused with the preceding SM (resulting in a monosyllabic cluster) and structurally self-standing, i.e., they can be realised without the co-occurrence of other TAMs, while TAM\textsubscript{2} forms tend to be realised in combination with preceding TAM(s), especially when denoting a past reference, and their lexical sources are relatively clear, i.e., they can be regarded as recently grammaticalised.

As for the core TA concepts, eight categories have been chosen to ensure sufficient semantic coverage and a clear formal distinction of the TAMs, namely two tense categories (past and future), four aspectual categories (progressive, anterior, completive, and habitual), and two combined categories (present and past statives). The so-called general present tense is not included in the list because, as \citet[115--117]{Nurse2008} explains, the present time reference is normally expressed with an aspectual focus, with a typical situation involving the progressive aspect with a dynamic verb like “She is walking” or a stative aspect with an inchoative verb like “He sleeps”. Thus, a pure exponent of the perfective (i.e., aspectually unmarked) present is not always clearly identifiable, or in other words the category may be described as “empty” in TA matrices in many Bantu languages (see also \citealt[77]{Nurse2003a}).

As shown in the following sections, all of the languages examined in this study have more than two distinct pasts and the non-WK languages have multiple futures. The four aspectual categories can be distinguished by TAMs throughout KB (although anterior and completive are not necessarily distinguished in some languages). Statives are adopted here as it may help identify the existence of a TA category that can be (tentatively) labelled as past imperfective, which may not be a common TA (combined\footnote{As a TA concept, this may not necessarily be a “combined” category. Rather, the form in question may be regarded as a special form of past marker that appears in various contexts of imperfective aspects. For further discussion, see \sectref{sec:shinagawa:3.1.5}.}) concept in Bantu languages in general, but is widely attested in KB at least as a category frequently morphologised by a TA marker (cf. \citealt[80]{Nurse2003a}). Examining these categories will thus help us identify how a TA marker diachronically expands or shifts its semantic coverage across KB languages. 

\subsection{Rwa}\label{sec:shinagawa:2.1}

Rwa, the geographically western-most WK language, has three morphologically distinct pasts and a single future which is marked by Final \textit{{}-áa}, apparently originating from a Prefinal-Final cluster *\textit{{}-ag-a}. Its tonally modified form \textit{{}-aá} is used as a habitual marker. The formal distinction between PST\textsubscript{1} and PST\textsubscript{3} is only tonal, i.e., PST\textsubscript{3} has a grammatically assigned high tone\footnote{This high tone may cause high tone plateauing, i.e., tone-less TBUs between this and an immediately preceding high tone may be realised as high flat tones, e.g., \textit{avatobírílíá} ‘S/he made (something) for them’.} on the final vowel of the stem, e.g., [PST\textsubscript{1}] \textit{va-a-ʃí${\neq}$kab-ís-a m-biíri} vs. [PST\textsubscript{3}] \textit{va{}-a{}-ʃí${\neq}$kab-ís-á m-biíri} \{SM2-PST\textsubscript{1/3}{}-OM7${\neq}$hit-CAUS\#CPx9-stick\}\footnote{Both SM2 and OM7 have an underlying high tone and the former is realised on the following syllable, while the latter is deleted by the so-called anti-Meeussen’s Rule (i.e., HH > LH).} ‘they hit (something in cl.7) by a stick’. However, if the verb stem is monosyllabic, the tonal difference is neutralised and the distinction is made by replacing the Final \textit{{}-á} with \textit{{}-é}\footnote{From a historical point of view, it may be suggested that this element might be regarded as an irregular manifestation of *\textit{{}-i̜le}, whose regular realisation is -\textit{íe}.}: [Rwa] \textit{t-ā${\neq}$l-}\textbf{\textit{ā}}\{SM1sg-PST\textsubscript{1}${\neq}$eat-F\} “we ate (PST\textsubscript{1})” vs. \textit{t-ā${\neq}$l-}\textbf{\textit{ē}}\{SM1sg-PST\textsubscript{3}${\neq}$eat-F\} “we ate (PST\textsubscript{3})”. Though the historical background of PST\textsubscript{2} \textit{nde-}\footnote{Note that \textit{nde-} of PST2 is structurally different from \textit{nde-} as a modality marker, relatively recently grammaticalised from the verb ‘go’, which can only appear with other tense markers.} is uncertain, one may relate it to \textit{le-} as a common past marker or the segmentally identical \textit{ndé-} in Gweno, which is a “verb-focusing” near past (cf. Philippson and Nurse [2000: 254]; see also \sectref{sec:shinagawa:2.8}).

The common markers \textit{keé-} (from *\textit{kad} ‘sit’), \textit{m̩-}, and \textit{maa-} (both from{*\textit{mad} ‘finish’) are attested as denoting progressive, anterior, and completive, respectively. Stative is marked by \textit{{}-ié} which originates from *\textit{{}-ile} and its past tense is expressed by TAM \textit{i-} with a lengthened final vowel (shown as \textit{=V} in \tabref{tab:shinagawa:1}). This morphological template of past marking occurs repeatedly not only in various imperfective aspects of regular verbs including progressive, anterior/completive, and habitual, but also in non-verbal predicates such as existentials: [Rwa]} \textit{ti-∅${\neq}$ifó} \{SM1pl-PRS${\neq}$EXT\} `we are (in a specific place)' vs. \textit{tí-}\textbf{\textit{i}}\textit{${\neq}$ifo=}\textbf{\textit{ó}} `we were (in a specific place)'. {Thus, it seems possible to posit a TA category morphologised by this template, which will be tentatively referred to as past imperfective. As shown in the following sections, this TA-combined category is often grammaticalized in other KBs as well}.\footnote{According to \citet[77]{Nurse2003a}, the category marked by \textit{we-} in Vunjo seems to correspond to this combined category. On the other hand, the recurrent morphologisation of this rather uncommon category can support Nurse’s (ibid.: 85) claim from a cross-KB perspective that “TA categories are more stable than the morphemes which carry them.”}\footnote{Examples in the lists presented in Tables~\ref{tab:shinagawa:1}--\ref{tab:shinagawa:4} are provided in morphophonemic description and those in the body of the text are in phonemic description. Others are quoted without modification from the source. For detailed information of the phonemic inventories of Rwa, Uru, and Rombo, see \citet{ShinagawaForthcoming-a}. The phonemic inventories of Mashami, Kibosho, and Gweno are presented in their original sources. A Cross-KB comparative list of phonemes is provided in \citet{PhilippsonMontlahuc2003}.}


\begin{table}
\small\tabcolsep=.66\tabcolsep
\begin{tabularx}{\textwidth}{>{\scshape}llllllllQ}
\lsptoprule
{\normalfont TA category} & SM & {TAM\textsubscript{0}} & {TAM\textsubscript{1}} & {TAM\textsubscript{2}} & stem & F & PoF & translation\\
\midrule
pst\textsubscript{1} & {\itshape t-} & {\itshape a-} &  &  & {\textit{k\ul{a}b}} & {\itshape {}-á} &  & `We hit'\\
pst\textsubscript{2} & {\itshape ti-} &  & {\itshape nde-} &  & {\textit{k\ul{a}b}} & {\itshape {}-á} &  & `We hit'\\
pst\textsubscript{3} & {\itshape t-} & {\itshape á-} &  &  & {\textit{k\ul{á}b}} & {\itshape {}-á} &  & `We hit'\\
fut                  & {\itshape ti-} &  &  &  & {\textit{k\ul{a}b}} & {\itshape {}-áa} &  & `We will hit'\\
prs.stat             & {\itshape ti-} &  &  &  & { \textit{lol}} & {\itshape {}-ié} &  & {`We see/are watching'}\\
pst.stat             & {\itshape tí-} &  & {\itshape í-} &  & {\textit{lol}} & {\itshape {}-ié} & {\itshape =V} & `We saw/were watching'\\
prog                 & {\itshape ti-} &  &  & {\textit{k\ul{e}é-}} & {\textit{k\ul{a}b}} & {\itshape {}-á} &  & {`We are hitting'}\\
ant                  & {\itshape t-} & {\itshape a-} & {\itshape m̩-} &  & {\textit{k\ul{a}b}} & {\itshape {}-á} &  & `We have hit'\\
comp                 & {\itshape t-} & {\itshape a-} & {\itshape m̩-} & {\itshape maa-} & {\textit{k\ul{a}b}} & {\itshape {}-á} &  & {`We have finished hitting'}\\
hab                  & {\itshape ti-} &  &  &  & {\textit{k\ul{a}b}} & {\itshape {}-āā} &  & {`We hit (regularly)'}\\
\lspbottomrule
\end{tabularx}
\caption{List of the core TA markers in Rwa}
\label{tab:shinagawa:1}
\end{table}


In addition to these TAMs, Rwa has two markers grammaticalised from ‘come’ and ‘go’ (referred to here as COM and GOM), which are \textit{ʃe-} and \textit{nde-}, respectively (N.B. this is a distinct form from the \textit{nde-} of PST\textsubscript{2}). Although the former, in particular, is often used for future marking in some languages of CK and Rombo, these two markers clearly denote “certainty” as a modal concept in this language\footnote{Interestingly, this semantic interpretation of GOM and COM is apparently the opposite to those found in Vunjo. For example, where COM marks a “more definite” intention while GOM marks a “less definite” intention (cf. \citealt[87]{Nurse2003a}). For further discussion, see \citet{ShinagawaForthcoming-a}.}: [Rwa] \textit{fua y-a-}\textbf{\textit{ndé}}\textit{${\neq}$nis-á} \{SM9-PST\textsubscript{1}{}-GOM${\neq}$rain-F\} `It (certainly) rained/It rained (as expected)' vs. \textit{fua y-a-}\textbf{\textit{ʃé}}\textit{${\neq}$nis-á} \{SM9-PST\textsubscript{1}{}-COM${\neq}$rain-F\} `It rained (unexpectedly)', where \textit{nde-} expresses a past event (“raining”) as more certain (i.e., it is recognised by the speaker that the event certainly or expectedly happened), while \textit{ʃe-} indicates less certainty and/or expectedness. Note also that they are used in a past tense (marked by \textit{a-}), suggesting that both COM and GOM are highly grammaticalised as modality markers.


\subsection{Siha}\label{sec:shinagawa:2.2}

As in Rwa, Siha also has a tripartite past and a single future. Since this distinction is also attested in Mashami (see \sectref{sec:shinagawa:2.5}), it may be regarded as a typical tense division in WK. Note, however, that some exponents are different from those in Rwa (see \tabref{tab:shinagawa:1}). PST\textsubscript{1} is marked by \textit{le}{}-, a common past marker throughout KB, while PST\textsubscript{2} uses the same form with an extra lengthened vowel in the Final, which can be regarded as semantically parallel to and morphologically homogenous with \textit{=V} of the past stative in Rwa. Stativeness can be expressed by \textit{{}-ile} or \textit{{}-i}, a possible shortened allomorph of *\textit{{}-ile}, and its past is also marked by an extra lengthened vowel.


\begin{table}
\small\tabcolsep=.66\tabcolsep
\begin{tabularx}{\textwidth}{>{\scshape}llllllllQ}
\lsptoprule
{\normalfont TA category} & SM & {TAM\textsubscript{0}} & {TAM\textsubscript{1}} & {TAM\textsubscript{2}} & stem & F & PoF & translation\\\midrule
pst\textsubscript{1} & {\itshape ti-} &  & {\itshape (l)e-} &  & {\textit{k\ul{á}v}} & {\itshape {}-á} &  & `We hit'\\
pst\textsubscript{2} & {\itshape ti-} &  & {\textit{(l)e-}} &  & {\textit{k\ul{a}v}} & {\itshape {}-á} & {\itshape =V} & `We hit'\\
pst\textsubscript{3} & {\itshape t-} & {\itshape á-} &  &  & {\textit{k\ul{á}v}} & {\itshape {}-á} &  & `We hit'\\
fut                  & {\itshape ti-} &  &  &  & {\itshape lóli} & {\itshape {}-áa} &  & `We will see'\\
prs.stat             & {\itshape to-} &  &  &  & {\textit{\ul{ó}n}} & {\itshape {}-i} &  & `We see/are watching'\\
pst.stat             & {\itshape tó-} &  &  &  & {\textit{\ul{o}n}} & {\itshape {}-í} & {\itshape =V} & `We saw/were watching'\\
prog                 & {\itshape ti-} &  & {\itshape li-} &  & {\textit{k\ul{á}v}} & {\itshape {}-á} &  & {`We are hitting'}\\
ant                  & {\itshape ti-} &  & {\itshape (l)e-} &  & {\textit{k\ul{á}v}} & {\itshape {}-á} &  & `We have hit'\\
comp                 & {\itshape ti-} &  & {\itshape (l)e-} & {\itshape me-} & {\textit{k\ul{á}v}} & {\itshape {}-á} &  & `We have finished hitting'\\
hab                  & {\itshape ti-} &  &  &  & {\itshape loli} & {\itshape {}-aa} &  & `We see (regularly)'\\
\lspbottomrule
\end{tabularx}
\caption{List of the core TA markers in Siha}
\label{tab:shinagawa:2}
\end{table}

As for aspectual forms, progressive is marked by \textit{li}{}-, which is also a common marker regarded as being grammaticalised from a copula *\textit{li} ‘be’. Otherwise, the aspectual exponents are similar to Rwa, i.e., completive is marked by \textit{me-} (<*\textit{mad}) and \textit{{}-aa} (<*\textit{{}-ag-a}) is used as habitual, though its tonal behaviour differs.

\subsection{Uru}\label{sec:shinagawa:2.3}\largerpage

The tense system of Uru, a CK language, shows configurational differences from that of WK, since both past and future\footnote{The number of tense categories of future is sometimes unclear since futurity can also be expressed by present progressive (cf. “present-used-as future” in \citealt{Nurse2003a}) and this can be applied to other CK languages and Rombo. However, it is relatively clear that future tense is exclusively indicated by a single form (mostly with \textit{{}-aa}) in WK.} are divided into two sub-categories with an extra future tense limitedly realised in main clause verbs. However, their exponents are interrelated with those in WK. The past markers \textit{e-}(PST\textsubscript{2}) and \textit{le-}(PST\textsubscript{1}) are parallel to PST\textsubscript{3} and PST\textsubscript{2} of Mashami, while \textit{i-} of \textsc{fut}\textsubscript{1} is regarded as a cognate of progressive \textit{li-} in Siha, which is justified in terms of historical sound change (loss of intervocalic /l/, cf. \citealt[79]{Nurse2003a}), as well as the fact that the semantic development process from progressive to future is a universally attested grammaticalisation pattern (cf. \citealt{BybeeEtAl1994}). \textsc{fut}\textsubscript{2}\textit{tʃi-} is a common future marker in CK which is regarded as having been grammaticalised from *\textit{ci} ‘know’ (cf. \citealt[76]{Nurse2003a}). Stativeness is marked by \textit{{}-ie} or \textit{{}-i} as in Siha, while its past is marked by \textit{e-} of PST\textsubscript{2}. Progressive is marked by \textit{ke}{}- (<*\textit{kad}) as in Rwa, contrary to its habitual usage in another CK language, Vunjo. The anterior marker \textit{a-} (realised as \textit{o-} in \tabref{tab:shinagawa:3} as a result of vowel coalescence) is consistently fused with SM, as is widely attested in other CK languages and Rombo. Habitual has no segmental exponent in the TAM slots and is marked only by a high tone assigned to Final.\footnote{The structural interpretation of this form can be rather controversial. It is reasonable to regard this as a “zero form,” which “refers to timeless action, an activity which does or can occur over a vast present” \citep[81]{Nurse2003a}. However, it is also possible to regard this as a descendant form of *\textit{∅${\neq}$}(Stem)-\textit{ag-a}, which will be discussed further in \sectref{sec:shinagawa:3.3}.}


\begin{table}
\small\tabcolsep=.66\tabcolsep
\begin{tabularx}{\textwidth}{>{\scshape}lllllllQ}
\lsptoprule
{\normalfont TA category} & SM & {TAM\textsubscript{0}} & {TAM\textsubscript{1}} & {TAM\textsubscript{2}} & stem & F & translation\\
\midrule
pst\textsubscript{1}   & {\itshape lú-} &  & {\itshape le-} &  & {\textit{\ul{ó}lok}} & {\itshape {}-a} & `We fell down'\\
pst\textsubscript{2}   & {\itshape lw-} & {\itshape é-} &  &  & {\textit{\ul{o}lók}} & {\itshape {}-a} & `We fell down'\\
fut\textsubscript{1}   & {\itshape lu-} &  & {\itshape í-} &  & {\textit{k\ul{á}p}} & {\itshape {}-a} & `We will hit'\\
fut\textsubscript{2}   & {\itshape lú-} &  & {\itshape tʃi-} &  & {\textit{k\ul{á}p}} & {\itshape {}-a} & `We will hit'\\
fut/cond               & {\textit{a\'{} }} &  & {\itshape e-} &  & {\textit{w\ul{ó}n}} & {\itshape {}-a} & `(If...) s/he would see'\\
prs.stat               & {\itshape lú-} &  &  &  & {\textit{w\ul{o}n}} & {\itshape {}-í} & `We see'\\
pst.stat               & {\itshape lw-} & {\itshape é-} &  &  & {\textit{w\ul{o}n}} & {\itshape {}-íé} & `We saw'\\
prog                   & {\itshape lú-} &  &  & {\itshape ke-} & {\textit{k\ul{á}p}} & {\itshape {}-a} & `We are hitting'\\
ant                    & {\itshape l-} & {\textit{ó-} (\{\textit{a-}\})} &  &  & {\textit{w\ul{ó}n}} & {\itshape {}-a} & `We have seen' \\
comp                   & {\itshape l-} & {\textit{o-} (\{\textit{a-}\})} &  & {\itshape m̩-} & {\textit{w\ul{ó}n}} & {\itshape {}-a} & `We have finished seeing' \\
hab                    & {\itshape lú-} &  &  &  & { \textit{ɾ\ul{e}í}} & {\itshape {}-á} & `We write (regularly)' \\
\lspbottomrule
\end{tabularx}
\caption{List of the core TA markers in Uru}
\label{tab:shinagawa:3}
\end{table}

It should be mentioned here that apart from the forms listed in \tabref{tab:shinagawa:3}, there is another morpheme that can be slotted in the TAM position. The morpheme \textit{we-} is mentioned in \citet{Nurse2003a} as ``anomalous'' in that in Vunjo it seems to indicate aspectual concepts (or more precisely the pastness of imperfective aspects), while it appears in the leftmost position of the TA string, where tense markers are usually slotted (ibid.: 77). However, in Uru \textit{we-} appears between the TAM\textsubscript{1} and TAM\textsubscript{2} positions and seems to denote a kind of predicate focus function as in; \textit{ɲálě:tʃa} ‘S/he came’ vs \textit{ɲálewê:tʃa} ‘S/he also came’ (subject additive focus) or ‘S/he came again’ (event recurrence). This will be further discussed in \sectref{sec:shinagawa:3.2}.

\subsection{Rombo-Mkuu}\label{sec:shinagawa:2.4}

In the current classification, e.g., \citet{Maho2009} based on the classifications of \citet{Nurse1981} and \citet{PhilippsonMontlahuc2003}, Rombo as a subgroup is further classified into (at least) four varieties, namely Useri, Mashati, Mkuu, and Keni (from North to South). However, its dialectal variation seems more diverse and complicated than this division suggests.

The tense system is basically comparable with that of Uru, i.e., both past and future are bipartite, though there may be more exponents denoting futurity. \textsc{fut}\textsubscript{1}\textit{ʃe-} is regarded as having grammaticalised from \textit{ʃa} ‘come’, denoting a near future reference (possibly with epistemic modal connotations). The inherited form of *\textit{{}-ile} is used as part of past tense marking, while stativeness is marked by its shortened form. Its past tense is indicated by a possibly relevant form of \textit{we-} in Vunjo. The aspectual patterns are similar to Siha, except that (present) habitual is marked by \textit{e-}, which is segmentally identical to \textsc{fut}\textsubscript{2}\textit{e-}. However, it should be noted that the two markers occupy different slots, as shown in \tabref{tab:shinagawa:4}.


\begin{table}
\small\tabcolsep=.8\tabcolsep
\begin{tabular}{>{\scshape}llllllll}
\lsptoprule
{\normalfont TA category} & SM & {TAM\textsubscript{0}} & {TAM\textsubscript{1}} & {TAM\textsubscript{2}} & stem & F & translation\\\midrule
pst\textsubscript{1}   & {\itshape dú-} &  & {\itshape le-} &  & {\itshape lolj} & {\itshape {}-a} & `We saw'\\
pst\textsubscript{2}   & {\itshape dú-} &  &  &  & {\itshape lol} & {\itshape {}-íé} & `We saw'\\
fut\textsubscript{1}   & {\itshape dú-} &  &  & {\itshape ʃe-} & {\textit{ɾ\ul{u}nd}} & {\itshape {}-a} & `We will work'\\
fut\textsubscript{2}   & {\itshape du-} &  & {\itshape é-} &  & {\textit{ɾ\ul{u}nd}} & {\itshape {}-a} & `We will work'\\
prs.stat               & {\itshape dú-} &  &  &  & {\textit{k\ul{u}nd}} & {\itshape {}-i} & `We want'\\
pst.stat               & {\itshape dú-} &  & {\itshape ve-} &  & {\textit{k\ul{u}nd}} & {\itshape {}-i} & `We wanted'\\
prog                   & {\itshape du-} &  & {\itshape í-} &  & {\itshape eleke} & {\itshape {}-a} & {`We are heading for'}\\
ant                    & {\itshape dw-} & {\itshape á-} &  &  & {\itshape lolj} & {\itshape {}-a} & `We have seen' \\
comp                   & {\itshape dw-} & {\itshape á-} &  & {\itshape me-} & {\itshape lolj} & {\itshape {}-a} & `We have finished seeing' \\
hab                    & {\itshape dw-} & {\itshape é-} &  &  & {\textit{k\ul{a}b}} & {\itshape {}-a} & `We hit (regularly)' \\
\lspbottomrule
\end{tabular}
\caption{List of the core TA markers in Rombo-Mkuu}
\label{tab:shinagawa:4}
\end{table}

\subsection{Mashami (\citealt{RugemaliraPhanuel2012})}\label{sec:shinagawa:2.5}

As examined in previous sections, it is clear from the data presented in \tabref{tab:shinagawa:5} that the tense system of Mashami, as expected, clearly follows the WK type. Moreover, the TA system as a whole seems quite similar to that of Rwa.

\begin{table}
\small\tabcolsep=.66\tabcolsep
\begin{tabularx}{\textwidth}{>{\scshape}llllllllQ}
\lsptoprule
{\normalfont TA category} & PreI & SM & {TAM\textsubscript{0}} & {TAM\textsubscript{1}} & { TAM\textsubscript{2}} & stem & F & { translation}\\
\midrule
pst\textsubscript{1} & {\itshape n=} & {\itshape lw-} & {\itshape á-} &  &  & {\itshape many} & {\itshape {}-a} & `We knew'\\
pst\textsubscript{2} & {\itshape n=} & {\itshape lú-} &  & {\itshape le-} &  & {\itshape mány} & {\itshape {}-a} & `We knew'\\
pst\textsubscript{3} & {\itshape n=} & {\itshape lw-} & {\itshape é-} &  &  & {\itshape mány} & {\itshape {}-a} & `We knew'\\
fut                  & {\itshape n=} & {\itshape ʃí-} &  &  &  & {\itshape kór} & {\itshape {}-aa} & `I will cook'\\
prs.stat             & {\itshape n=} & {\itshape lu-} &  &  &  & {\itshape salal} & {\itshape {}-ye} & `We are standing'\\
pst.stat             & {\itshape n=} & {\itshape lu-} &  & {\itshape é-} & {\itshape ké-} & {\itshape many} & {\itshape {}-a} & `We were understanding'\\
prog                 & {\itshape n=} & {\itshape lú-} &  &  & {\itshape ké-} & {\itshape many} & {\itshape {}-a} & `We are knowing' [sic]\\
ant\footnote{According to the source, anterior can be marked without \textit{a-}, as in \textit{ku-n${\neq}$shani-shi-a${\neq}$kya} \{SM2sg-ANT-come\#\textsc{foc}-SM1sg-PST1${\neq}$be cured\} `Since you have come, I am safe (Sw: Kwa kuwa umekuja, nimepona)'.}/comp & {\itshape n=} & {\itshape lw-} & {\itshape á} &  & {\itshape m-} & {\itshape many} & {\itshape {}-a} & `We have already known'\\
hab                 & {\itshape n=} & {\itshape ʃí-} &  &  &  & {\itshape kor} & {\itshape {}-aa} & `I cook'\\
\lspbottomrule
\end{tabularx}
\caption{List of the core TA markers in Mashami}
\label{tab:shinagawa:5}
\end{table}


However, there are two points to be noted which are not explicitly shown in \tabref{tab:shinagawa:5}. First, TAM\textsubscript{1}\textit{e-} as a past marker of stative verbs also marks the pastness of imperfective forms such as habitual: [Mas.] \textit{n=lu-}\textbf{\textit{é}}\textit{${\neq}$many-aa} \{\textsc{foc}=SM1pl-PST.IMPF${\neq}$know-\textsc{hab}\} “We used to know”. Second, final \textit{{}-aa} (<*\textit{{}-ag-a}) can be used not only as a future marker but also to indicate progressive meaning: [Mas.] \textit{ni-a-koy}${\neq}$\textit{aa \# i}${\neq}$\textit{kor-}\textbf{\textit{aa} }\{\textsc{foc}\textit{{}-}SM3sg${\neq}$find\textit{{}-}\textsc{fut} \# INF${\neq}$cook\textit{{}-}\textsc{prog}\} “He will be cooking”.

\subsection{Kibosho \citep{Kagaya1989}}\label{sec:shinagawa:2.6}

Though available data are rather limited, it can be said that Kibosho,\footnote{Though Kibosho is normally classified as WK in the linguistic literature, it is regarded as a variety of Vunjo in local narratives, according to \citet{Kagaya2006}.} a WK language, shows a somewhat unique pattern.\footnote{Note however, that there seem to be dialectal patterns which are more complicated than that which is presented here, e.g. \textit{∅}{\textit{${\neq}$}}(Stem)\textit{{}-ie} is also attested as a remote past form or as a variant of hodiernal past, while future may be denoted by a null-marked form or by \textit{ende-} as GOM. I acknowledge Gérard Philippson for this information.} First, the tense distinction seems to be a bipartite past and a single future, which is slightly different from the 3:1 pattern in the other WK languages examined in this study. Second, the descendant of *\textit{{}-ag} does not denote futurity as in other WK, but indicates progressive aspect, which is, however, observed in limited contexts in Mashami. While \citet[829]{Kagaya1989} describes future as marked by the vowel lengthening of SM (shown as \textit{V-} in \tabref{tab:shinagawa:6}), this element may well be identified as \textit{i-} in other languages and the same element is attested in progressive forms as well, according to my own data.\footnote{The examples of \textsc{fut}, \textsc{prog}, and \textsc{hab} in \tabref{tab:shinagawa:6} were confirmed through elicitation with a native speaker in his 30s in my field research carried out in August 2018.}


\begin{table}
\small\tabcolsep=.66\tabcolsep
\begin{tabularx}{\textwidth}{>{\scshape}lllllllQ}
\lsptoprule
{\normalfont TA category} & SM & {TAM\textsubscript{0}} & {TAM\textsubscript{1}} & {TAM\textsubscript{2}} & stem & F & translation\\\midrule
pst\textsubscript{1} & {\itshape l-} & {\textit{o-} (\{a-\})} &  &  & {\itshape ch} & {\itshape {}-a} & `We arrived' (Hodienal)\\
pst\textsubscript{2} & {\itshape lu-} &  & {\itshape le-} &  & {\itshape ch} & {\itshape {}-a} & {`We arrived' (Remote past)}\\
fut                  & {\itshape lú-} &  & {\textit{V-} (\{i-\})} &  & {\itshape som} & {\itshape {}-a} & `We will arrive'\\
prs.stat             & {\itshape lu-} &  &  &  & {\itshape ke} & {\itshape {}-i} & `We are (at a place)'\\
pst.stat             & {\itshape lw-} & {\itshape e-} &  &  & {\itshape ke} & {\itshape {}-i} & `We were (at a place)'\\
prog                 & {\itshape lú-} &  & {\textit{V-} (\{i-\})} &  & {\itshape som} & {\itshape {}-áa} & `We read/are reading'\\
hab                  & {\textit{lú-}} &  &  &  & {\textit{som}} & {\textit{{}-aá}} & `We read (regularly)'\\
\lspbottomrule
\end{tabularx}
\caption{List of the core TA markers in Kibosho}
\label{tab:shinagawa:6}
\end{table}

It is also to be noted that this language, too, seems to morphologise past imperfective, which is marked by \textit{e-}: [Kib.] \textit{n̩=lw-}\textbf{\textit{e}}\textit{${\neq}$som-aa} `We were reading', where \textit{e-}encodes the past tense of the progressive aspect.

\subsection{Vunjo \citep{Nurse2003a}}\label{sec:shinagawa:2.7}

Vunjo exhibits a typical CK-Rombo pattern of a tense system with a bipartite distinction both in past and future. The exponent of \textsc{fut}\textsubscript{1} is \textit{ci-}, which is shared with Uru. Though it is not shown in \tabref{tab:shinagawa:7}, there is a form which is realised in the past tense of some imperfective forms, which is \textit{we-} as mentioned in \sectref{sec:shinagawa:2.3}: [Vun.] \textit{lu-we-i${\neq}$kap-a} \{SM1pl-P.I.-\textsc{prog}${\neq}$hit-F\} `We were hitting'. As observed above, \textit{we-} in this example may well be regarded as a form functionally equivalent to \textit{ve-} in Rombo and \textit{i-} in Rwa, which denotes a certain range of past imperfective. As for aspectual categories, it should be noted that progressive \textit{i-} is shared with Rombo, while habitual is marked by \textit{ke-}, which is identical to the progressive marker in Uru.


\begin{table}
\small\tabcolsep=.66\tabcolsep
\begin{tabularx}{\textwidth}{>{\scshape}lllllllQ}
\lsptoprule
{TA category} & SM & {TAM\textsubscript{0}} & {TAM\textsubscript{1}} & {TAM\textsubscript{2}} & stem & F & translation\\
\midrule
pst\textsubscript{1} & {\itshape l-} & {\itshape o-} &  &  & {\itshape kap} & {\itshape {}-a} & `We hit'\\
pst\textsubscript{2} & {\itshape lu-} &  & {\itshape le-} &  & {\itshape kap} & {\itshape {}-a} & `We hit'\\
fut\textsubscript{1} & {\itshape lw-} & {\itshape (e-)} &  & {\itshape ci-} & {\itshape kap} & {\itshape {}-a} & `We will hit'\\
fut\textsubscript{2} & {\itshape lw-} & {\itshape e-} &  &  & {\textit{kap}} & {\itshape {}-a} & `We will hit'\\
prog                 & {\itshape lw-} &  & {\itshape i-} &  & {\itshape kap} & {\itshape {}-a} & `We are hitting'\\
ant                  & {\itshape lu-} &  &  &  & {\itshape kap} & {\itshape {}-ie} & `We have hit'\\
comp?                & {\itshape l-} & {\itshape o-} &  & {\itshape m-} & {\itshape kap} & {\itshape {}-a} & `We have/had already hit'\\
hab                  & {\itshape lu-} &  &  & {\itshape ke-} & {\itshape kap} & {\itshape {}-a} & `We hit regularly'\\
\lspbottomrule
\end{tabularx}
\caption{List of the core TA markers in Vunjo}
\label{tab:shinagawa:7}
\end{table}

\subsection{Gweno (\citealt{PhilippsonNurse2000})}\label{sec:shinagawa:2.8}

According to \citet{PhilippsonNurse2000}, the community of Gweno speakers has long lived in Northern Pare, remote from Kilimanjaro, which may have caused the language to retain some archaic features and to develop differently from other languages of the three main subgroups.

The configuration of the Gweno tense system (\tabref{tab:shinagawa:8}) seems identical to the CK-Rombo type, where both pasts are morphologised by the *\textit{{}-ile} form. The TA forms which are more or less relatable to common forms in other KB languages include \textit{le-} and less reliably \textit{nde-} and \textit{tʃe-}. Progressive is not marked by a specific morpheme but is expressed as a “focused form of general present”. Though habitual \textit{tʃi-} appears to be identical to \textsc{fut}\textsubscript{1} in Vunjo, it is pointed out in Philippson and Nurse (ibid.: 255) that it is a shortened form of the verb \textit{tʃiβia} ‘be accustomed to’.


\begin{table}
\small
\begin{tabularx}{\textwidth}{>{\scshape}llllllllQ}
\lsptoprule
{\normalfont TA category} & {Pre-Ini} & SM & {TAM\textsubscript{0}} & {TAM\textsubscript{1}} & {TAM\textsubscript{2}} & stem & F & {translation}\\
\midrule
{pst\textsubscript{1}/ant} &  & {\itshape fu-} &  &  &  & {\itshape βúk} & {\itshape {}-íe} & `We left'\\
{pst\textsubscript{2}}      &  & {\itshape ni-} &  & {\itshape lé-} &  & {\itshape ɣend} & {\itshape {}-ie} & `I went'\\
{fut\textsubscript{1}}      &  & {\itshape fw-} & {\itshape a-} & {\itshape ɣe-} &  & {\itshape ʃiɣ} & {\itshape {}-a} & `We will look for'\\
{fut\textsubscript{2}}      &  & {\itshape fw-} & {\itshape a-} & {\itshape tʃe-} &  & {\itshape ɣu} & {\itshape {}-a} & `We will buy'\\
{prog\footnote{Progressive aspect seems to be expressed in several different ways in Gweno. According to \citet[253]{PhilippsonNurse2000}, in addition to the form cited in \tabref{tab:shinagawa:8}, there is another progressive marker, \textit{ky-a-}, whose lexical origin is not explicitly stated in the source.}} 
                            & {\itshape ḿ=} & {\itshape fw-} & {\itshape â} &  &  & {\itshape ɣu} & {\itshape {}-a} & `We are buying'\\
ant/comp                    &  & {\itshape í} &  & {\itshape nde-} & {\itshape (mi-)} & {\itshape pfw} & {\itshape {}-á} & `It has died'\\
{hab}                       &  & {\itshape ni} &  &  & {\itshape tʃi} & {\itshape ɣend} & {\itshape {}-a} & `I go'\\
\lspbottomrule
\end{tabularx}
\caption{List of the core TA markers in Gweno}
\label{tab:shinagawa:8}
\end{table}


\section{Semantic correspondence of each TAM}\label{sec:shinagawa:3}

  Based on the above observations, the distribution of all the common TA markers in the eight sample languages examined in this study are summarised in \tabref{tab:shinagawa:9}.


\begin{table}
\small
\begin{tabularx}{\textwidth}{QQQQQQQQQ} 
\lsptoprule
& \multicolumn{4}{c}{WK} & \multicolumn{2}{c}{CK} & Rombo & Gwe.\\
\cmidrule(lr){2-5}\cmidrule(lr){6-7}\cmidrule(lr){8-8}
& Rwa & Sih. & Mas. & Kib. & Vun. & Uru & Mkuu & \\
\midrule
{\textit{a\textsubscript{1}}\textit{{}-}} & {\scshape pst.n} & \multirow{2}{=}{\scshape pst.r} & \multirow{2}{=}{\textsc{pst.n}} & \multirow{2}{=}{\scshape pst.n} & \multirow{2}{=}{\scshape pst.n} & \multirow{2}{=}{\scshape ant} & \multirow{2}{=}{\scshape ant} & \multirow{2}{=}{{\scshape prs}

{\scshape /fut}

{\scshape /cont}}\\
{\textit{a\textsubscript{2}}\textit{{}-}} & {\scshape pst.r} &  &  &  &  &  &  & \\
\tablevspace
{\textit{e\textsubscript{1}}\textit{{}-}} &  & {\scshape p.i.} & {\scshape pst.r/}

{\scshape p.i.} & {\scshape p.i.} &  & {\scshape pst.r/}

{\scshape p.i.} &  & \\
\tablevspace
{\textit{e\textsubscript{2}}\textit{{}-}} &  &  &  &  & {\scshape fut.r} & {\scshape fut} & {\scshape fut/}

{\scshape hab} & \\
\tablevspace
{\itshape V-} &  &  &  & {\scshape fut} &  &  &  & \\
\tablevspace
{\itshape le-} & {\scshape pst.m} & {\scshape pst.n/ m} & {\scshape pst.m} & {\scshape pst.r} & {\scshape pst.r} & {\scshape pst.n} & {\scshape pst.n} & {\scshape pst.r}\\
{\itshape li-} & {\scshape p.i.} & {\scshape cont} &  &  & {\scshape cont} & {\scshape fut.n} & {\scshape cont} & \\
\tablevspace
{\itshape we-, ve-} &  &  &  &  & {\scshape p.i.?} & { \textsc{foc}} & {\scshape p.i} & {\scshape pst.m?}\\
\tablevspace
{\itshape ci-} &  &  &  &  & {\scshape fut.n} & {\scshape fut.r} &  & \\
\tablevspace
{\itshape ke-} & {\scshape cont} &  & {\scshape cont} &  & {\scshape hab} & {\scshape cont} &  & {\scshape cont}\footnote{Used as an auxiliary verb.} \\
\tablevspace
{\itshape (ker-

i-)} &  &  &  &  & {\scshape cont} & {\scshape cont} &  & \\
\tablevspace
{\itshape ʃe-} & {\scshape cert↓} &  &  &  & {\scshape cert↑} &  & {\scshape fut} & {\scshape fut.r}\\
\tablevspace
{\itshape nde-} & {\scshape cert↑} &  &  &  & {\scshape cert↓} &  &  & \\
\tablevspace
{\textit{m̩{}-, mi-}} & {\scshape ant} & {\scshape ant} & {\scshape ant/}

{\scshape comp} &  & {\scshape comp} & {\scshape comp} & {\scshape comp} & {\scshape ant}\\
\tablevspace
{\itshape maa-} & {\scshape comp} & {\scshape comp}\footnote{Used as an auxiliary verb.} & {\scshape comp}  &  &  &  &  & \\
\tablevspace
{\itshape {}-ile} & {\scshape stat} & {\scshape stat} & {\scshape stat} & {\scshape stat} & {\textsc{ant/} \textsc{stat}} & {\scshape pst.st} & {\scshape pst.r} & {\scshape pst}\\
\tablevspace
{\itshape {}-i} &  & {\scshape stat} &  & {\scshape stat} &  & {\scshape stat} & {\scshape stat} & \\
\tablevspace
{\itshape {}-ag} & {\scshape fut/ hab} & {\scshape fut/ hab} & {\scshape fut/ hab} & {\scshape cont/ hab} &  & {\scshape hab} &  & \\
\tablevspace
{\itshape =V} & {\scshape p.i.} & {\scshape pst.r?} &  &  &  &  &  & \\
\lspbottomrule
\end{tabularx}
\caption{Comparative list of TAMs in eight sample languages of KB}
\label{tab:shinagawa:9}
\end{table}


The following sections provide brief notes on the variation of TA categories that each TAM covers in the different languages.

\subsection{Inherent markers: TAM\textsubscript{0} and TAM\textsubscript{1}}\label{sec:shinagawa:3.1}

\subsubsection{\textit{a-}}\label{sec:shinagawa:3.1.1}

This prefix is attested in all the languages examined. While its conceptual coverage in each language varies from past through anterior and even to future, the geographical distribution of each type appears to overlap with the boundaries of the subgroups, i.e., it is used as past in WK (N.B. two tonally distinctive markers in Rwa), anterior in CK and Rombo, and “general present-future” (including progressive when the verb is in Focus) in Gweno.\footnote{For this apparently uncommon grammaticalisation process, see \citet[74--75]{Nurse2003a}.}

\subsubsection{\textit{e-}}\label{sec:shinagawa:3.1.2}

The data suggest that two distinct TA categories can be encoded by the prefix \textit{e-}. One is past imperfective, as a frequently morphologised category across KB, as attested in Uru (as past tense of stative, progressive, and anterior) and in Kibosho (as past tense of progressive): [Uru] \textit{lw-é-m̩${\neq}$ɾeí-a} \{SM1pl-P.I.-ANT${\neq}$write-F\} “We had written”, [Kib.] \textit{n̩=lw-e{}-som-aa} \{\textsc{foc}=SM1pl-P.I.${\neq}$read-\textsc{prog}\} “We were reading”. The other is a series of markers denoting the future-habitual references as in Vunjo (remote future), Uru (future [with conditional]), and Rombo (remote future): [Uru] \textit{kaʃiká ʃimbó, n=\H{a}-}\textbf{\textit{e}}\textit{${\neq}$wón-a m̩meéku} \{\textsc{foc}=SM3sg-\textsc{fut}.COND${\neq}$see-F\} {“If s/he arrives in Shimbwe (a name of a village), s/he will meet an old man”.}

  What is significant here is that in Uru and Rombo-Mkuu, the segmentally identical exponent appears in different TAM slots for marking different TA concepts, e.g., [Mkuu] \textit{dw-é${\neq}$kab-a} “We hit (regularly)” vs. \textit{du-é${\neq}$ɾund-a} “We will work”. This may suggest that while the two TA categories are regarded as close to each other, they are still systematically differentiated, and that the multiple TAM slots are used in order to morphologise the semantic difference of the TA categories.

\subsubsection{\textit{le-}}\label{sec:shinagawa:3.1.3}

  This marker is attested in all the languages examined except Rwa and consistently denotes past tense throughout KB. It should be noted that in Siha, \textit{le-} is also realised as \textit{e-} especially when preceded by a SM with the syllable structure /Ci/ or /Cu/: [Sih.] \textit{ti-le${\neq}$káv-á} {\textasciitilde} \textit{te-e${\neq}$káv-á} “We hit (near past)”.

\subsubsection{\textit{li-}}\label{sec:shinagawa:3.1.4}

  \textit{li-,} or its weakened form \textit{i-}, is also a common TAM attested across KB and its typical function is progressive, as seen in Siha, Vunjo (cf. labelled as “present-used-as-future” in \citealt{Nurse2003a}), and Rombo-Mkuu. Uru also uses this prefix to encode a future reference more clearly. However, as mentioned in \sectref{sec:shinagawa:2.1}, \textit{i-} in Rwa covers the past reference of imperfective aspects including progressive, anterior, habitual, stative, and even copula constructions.

\subsubsection{\textit{we-}}\label{sec:shinagawa:3.1.5}

  The forms originating from *\textit{wa} ‘be, become’ are attested in CK and Rombo, and they are realised in past tense forms of various imperfective aspects such as anterior and progressive in Vunjo (\textit{we-}), and the past of stative verbs in Rombo (\textit{ve-}). In contrast, \textit{we-} in Uru clearly shows a different direction of its grammaticalisation process, i.e., it appears to have developed as a marker of predicate focus. The interpretation that \textit{we-} in Uru marks predicate focus can be justified by the fact that it is not allowed to appear in the "out-of-focus" positions such as before a clause-final negation particle (as mentioned in \sectref{sec:shinagawa:2.3}) or in relative clauses, where the preverbal clitic \textit{ni=}, which denotes “main clause-ness” in the language, is also avoided, as in [Uru] \textit{a-le-tʃ-a=se ‘}The one who came again’ vs. *\textit{a-le-we-tʃ-a} (for further discussion of \textit{ɲi=}, see \citealt{ShinagawaForthcoming-b}).\footnote{See \citet{HymanWatters1984} for various examples of structural co-occurrence restrictions on (inherent) focused forms in Bantu. For a cross-Bantu discussion of the developmental process of progressive into focus, see \citet{Güldemann2003} and \citet{Gibson2019}.}

\subsection{Recently grammaticalised markers: TAM\textsubscript{2}}\label{sec:shinagawa:3.2}
\subsubsection{*\textit{ci}}\label{sec:shinagawa:3.2.1}

  This marker is observed only in CK and scarcely found elsewhere. Its function is consistently future marking and it can be regarded as having grammaticalised from the verb *\textit{ci} ‘know’ \citep[76]{Nurse2003a}, suggesting that it can be viewed as a rather recent innovation in the CK area. An apparent homophonic form (\textit{tʃi-}) is also attested in Uru and Gweno, denoting habitual in both languages. However, as  \citet[255]{PhilippsonNurse2000} clarify, \textit{tʃi}{}- should be regarded as a shortened form of a verb stem \textit{tʃiβia} ‘be accustomed to’, since in Uru, too, speakers replace habitual \textit{tʃi-} with a periphrastic structure involving \textit{tʃiβia}.

\subsubsection{*\textit{kad}}\label{sec:shinagawa:3.2.2}

  TAMs grammaticalised from *\textit{kad} ‘sit, stay’ are broadly attested in WK (Rwa, Mashami), in CK (Vunjo, Uru) and in Rombo. Though the semantic category it denotes can be basically recognised as progressive aspect, it is also used as a habitual marker, as attested in Vunjo. If we follow the general tendency of the direction of grammaticalisation paths involving both concepts, it can be presumed that habitual is further grammaticalised from progressive (cf. \cites[93]{HeineKuteva2002}[48]{Haspelmath1998}). This grammaticalisation path from progressive to habitual (then to future) will be discussed again in \sectref{sec:shinagawa:4.1.2}.

\subsubsection{*\textit{mad}}\label{sec:shinagawa:3.2.3}

  TAMs originating from *\textit{mad} ‘finish’ stably denote anterior meanings (then to completive in some WK languages) and distributed throughout KB.\footnote{\textit{me-} in Rombo can be regarded as a relic of an anterior form of *\textit{mad} (\textit{*mad-i̜de} > \textit{me}), which is also attested in Old Moshi (Gérard Philippson, p.c.).} Two distinctive forms \textit{maa-} vs. \textit{m̩-} have developed in some WK languages, where the former is obviously a later innovation than the latter and seems to be in complementary distribution with anterior (to past) use of *\textit{{}-ile}.

\subsection{Suffixes}\label{sec:shinagawa:3.3}
\subsubsection{*\textit{{}-ag}}\label{sec:shinagawa:3.3.1}

  As previously pointed out by \citet{PhilippsonMontlahuc2003}, the reflexes of *\textit{{}-ag} are distributed only in WK. In the data, it is attested (as \textit{{}-aa} with various tonal realization) in Kibosho, Siha, Mashami, and Rwa, i.e., exactly covering the WK area. As \citet{Shinagawa2015} suggests, the absence of *-\textit{ag} in non-WK languages may at least partly be explained by the lack (or weakness) of vowel length contrast in those languages. In this sense, it may be possible to regard the present habitual form of Uru, which has no segmentally overt TAM and is only marked by a high tone on the final vowel,\footnote{Of course, the form can be simply seen as a so-called “null form” which is unmarked for tense and aspect, thus denoting the general present or a generic situation (cf. \citealt[117--118]{Nurse2008}). More investigation on this issue is needed.} as a shortened form of a relic of *\textit{{}-ag-a}. Though its semantic coverage is essentially habitual and future as described in the literature, it is worth mentioning that \textit{{}-aa} in Kibosho denotes progressive aspect. The historical process of the semantic split of this marker will be further discussed in \sectref{sec:shinagawa:4.1.2}.

\subsubsection{*\textit{{}-ile}}\label{sec:shinagawa:3.3.2}

  Reflexes of *\textit{{}-ile}, mostly realised as \textit{{}-ie}, as well as its historical allomorphic form \textit{{}-i}, are widespread throughout KB. While \textit{{}-ie} denotes anteriority or past tense in CK,\footnote{\textit{{}-ie} in Vunjo seems to exhibit an intermediate situation, i.e., \citet{Nurse2003a} describes its function as anterior (presupposing that inchoative verb stems such as \textit{won} ‘see,’ \textit{lal} ‘lie, sleep’ etc. express a stative meaning when attached to the anterior marker), while \citet{Moshi1994} rather consistently insists that, though it is greatly influenced by the lexical meanings of the verb, the core semantics of \textit{{}-ie} forms can be regarded as stativity.} Rombo, and Gweno, it encodes stativity in WK. For example, in Rwa, this element derives stative verb stems that are paradigmatically differentiated from default (\textit{{}-a} ending) verb stems in that they follow different past-marking paradigms. In the following examples, the past tense is marked by the combination of TAM \textit{i-} and a lengthened final vowel just as in existential predicates (see \sectref{sec:shinagawa:2.1}): [Rwa] \textit{ti-∅${\neq}$tisiɾ-ié} \{SM1pl-PRS${\neq}$write-STAT\} “We have written (resultant state)” vs. \textit{tí-í${\neq}$tisiɾ-ié=e} “We had written (resultant state)”. Parallel morphosyntactic behaviour is observed in stative \textit{{}-i} in Rombo-Mkuu, where the past imperfective \textit{ve-} denotes the past tense of stative verbs: [Mkuu] \textit{du-∅${\neq}$kund-i} “We love” vs. \textit{dú-ve${\neq}$kund-i} “We loved”.

\subsubsection{Vowel copy suffix}\label{sec:shinagawa:3.3.3}

  In some western WKs, there is a Postfinal clitic which is a copy of the Final (realised as a lengthened vowel), denoting the past tense of specific aspect forms. As stated in \sectref{sec:shinagawa:2.1} and \sectref{sec:shinagawa:3.3.2} the past tense of stative predicates is expressed by a combination of past imperfective \textit{i-} and this vowel copy (VC) suffix in Rwa. Moreover, Siha apparently expands its use into default verb forms to encode remoteness of past tense (cf. the formal distinction between PST\textsubscript{1} vs. PST\textsubscript{2} in \tabref{tab:shinagawa:2}).

\section{Grammaticalisation chains and a microparametric approach to regionally shared features}\label{sec:shinagawa:4}
\subsection{Grammaticalisation chains}\label{sec:shinagawa:4.1}

  As presented in the previous sections, there are several TAMs which are shared by different languages in the same subgroup or even across sub-group boundaries. TA categories encoded by such TAMs can be either shared throughout languages (e.g., \textit{maa-} in WK is consistently used as a completive marker) or gradually shift from language to language. In the latter case, a gradual and systematic shift of the semantic coverage of common TAMs may clarify the grammaticalisation process of each TAM and shed light on the historical development of TA systems in KB. \figref{fig:shinagawa:1} is a simplified illustration of this type of correspondence, found between Vunjo and Rwa.

\begin{figure}

\begin{tikzpicture}[align=left, anchor=west]
\node at (0,6) {Aspectual Concept/Category};
\node at (5,6) {Vunjo};
\node at (7,6) {Rwa};

\node at (0,5) {Habitual};
\node(ke) at (5,5) {\bfseries\itshape ke-};
\node at (7,5) {{}-aá};

\node at (0,4) {Progressive};
\node(il) at (5,4) {\bfseries\itshape i-};
\node(kee) at (7,4) {\bfseries\itshape kee-};

\node at (0,3) {Past Imperfective};
\node at (5,3) {\itshape we-};
\node(ir) at (7,3) {\bfseries\itshape i-};

\node at (0,2) {Completive};
\node(m) at (5,2) {\bfseries\itshape m-};
\node at (7,2) {\itshape maa-};

\node at (0,1) {Anterior};
\node(iel) at (5,1) {\bfseries\itshape {}-ie};
\node(mi) at (7,1) {\bfseries\itshape m̩-};

\node at (0,0) {Stative}; 
\node(ier) at (7,0) {\bfseries\itshape {}-ie};

\draw (ke.east) -- (kee.west);
\draw (il.east) -- (ir.west);
\draw (m.east) -- (mi.west);
\draw (iel.east) -- (ier.west);

\end{tikzpicture}
\caption{Schematised interrelation of TAMs between Vunjo and Rwa}
\label{fig:shinagawa:1}
\end{figure}

The corresponding relation shown in \figref{fig:shinagawa:1} clearly demonstrates that habitual and progressive are interconnected by a shared TAM grammaticalised from *\textit{kad} ‘sit, stay’ and, in parallel with the connection, the progressive \textit{i-} of Vunjo in turn corresponds to the past imperfective in Rwa. Similarly, the common TAM \textit{m̩-} denotes completive in Vunjo and anterior in Rwa, while anterior \textit{{}-ie} in Vunjo encodes stative in Rwa, and so on. The following sections will focus on two clusters of TA categories, in which such corresponding connections across languages are clearly observed. Note, however, that the following discussion will focus only on the interrelation between the three main subgroups, due to the limited materials on Gweno itself and its historical linguistic relations with other KB languages.

\subsubsection{PST-ANT-STAT continuum}\label{sec:shinagawa:4.1.1}

  As shown in \tabref{tab:shinagawa:10}, a semantic area ranging over past, anterior, and stative forms a cluster of TA categories interconnected by common TAMs such as \mbox{\textit{a-},} \textit{le-}, \textit{e-}, -\textit{ie}, and various markers originating from *\textit{mad} ‘finish’. If we regard the prototypical semantic area of *\textit{{}-ie} as anterior, the following process can be suggested.

  In WK, probably motivated by the connotation of resultant state that likely emerged from the concept of anterior (cf. \citealt[96]{Nurse2003b}), the semantic area of \textit{{}-ie} shifted to stative; and successively \textit{m̩-}, most probably grammaticalised as completive, was pulled to anterior. The apparently redundant innovation of \textit{maa-} may be explained by the empty gap made after the movement of \textit{m̩-}. In contrast, \textit{{}-ie} in Rombo-Mkuu moved in the other direction to past, which seems to be a more usual semantic shift of *\textit{{}-ile}, while \textit{a-}occupies the semantic area of anterior.

  The TAM \textit{a-} in Vunjo and WK, on the other hand, plays the role of past-marking as in many other Bantu languages (cf. \citealt[82]{Nurse2008}). With the exception of Siha, where \textit{a-} (with a high tone) denotes the remotest past, the relative order of temporal distance denoted by these TAMs seems stable throughout different TA systems, i.e., \textit{a-} occupies the temporal space closest to the present (or time of utterance), followed by \textit{le-}, and \textit{e-}occupies the furthest.\footnote{The order of temporal distance between \textit{a-} and \textit{le-} has already been proposed by \citet[74]{Nurse2003a}.}


\begin{table}

\begin{tabularx}{\textwidth}{XXXXXXXX} 
\lsptoprule
& \multicolumn{4}{c}{WK} & \multicolumn{2}{c}{CK} & Rombo\\
\cmidrule(lr){2-5}\cmidrule(lr){6-7}\cmidrule(lr){8-8}
& Rwa & Mas & Sih. & Kib. & Vun. & Uru & Mkuu\\
\midrule
{\scshape pst.r} & {\itshape a-} & {\itshape e-} & {\itshape a-} & {\itshape le-} & {\itshape le-} & {\itshape e-} & {\itshape {}-ie}\\
{\scshape pst.m} & \textit{nde-} & {\itshape le-} & \mbox{\textit{le-} (\textit{=V})} &  &  &  & \\
{\scshape pst.n} & {\itshape a-} & {\itshape a-} & {\itshape le-} & {\itshape a-} & {\itshape a-} & {\itshape le-} & {\itshape le-}\\
{\scshape stat} & {\itshape {}-ie} & {\itshape {}-ie} & {\itshape {}-ie} & \textit{{}-ie} & {\itshape {}-ie} & {\itshape {}-ie} & {\itshape {}-i}\\
{\scshape ant} & {\itshape m̩-} & {\itshape m-} & {\itshape m(e)-} &  &  & {\itshape a-} & {\itshape a-}\\
{\scshape comp} & \mbox{\itshape m̩-maa-} & {\itshape m̩-maa} & {\itshape m̩-maa}  &  & {\itshape m-} & {\itshape a-m̩-} & {\itshape a-me-}\\
\lspbottomrule
\end{tabularx}
\caption{Formal correspondence of TAMs denoting PST-ANT-STAT continuum}
\label{tab:shinagawa:10}
\end{table}

\subsubsection{\textsc{prog}-\textsc{fut}-\textsc{hab} continuum}\label{sec:shinagawa:4.1.2}

The other TA cluster connected by common TAMs ranges over the progressive, future, and habitual areas. To clarify this interrelation, it would be necessary to identify the original semantic area of *\textit{{}-ag-a} in the TA system at the Proto-KB stage. Though it is widely assumed that its prototypical meaning throughout Bantu may be a broad range of imperfective, with habituality or iterativity as core concepts (cf. \cites[110]{Meeussen1967}[262--263]{Nurse2008}), the data examined here suggest that while \textit{{}-aa} denotes habitual and future in Rwa, Siha, and Mashami, it is also used as a progressive marker in Kibosho. Moreover, its progressive use is also, though limitedly, attested in Mashami, where habitual and future usages are also attested, i.e., the Mashami system can be regarded as an intermediate stage of the semantic shift of \textit{{}-ag}. If this is the case, then it is reasonable to posit that the direction of semantic change may have started from progressive to habitual or future and not vice versa, as suggested by a widely attested tendency of the grammaticalisation path developed from present progressive to habitual or future (cf. \citealt[158]{BybeeEtAl1994}). Along the same lines, \citet{Haspelmath1998} also explains how an old present form (*\textit{{}-ag} in this case) changes into a new future as a side effect of the emergence of a new present (\textit{ke(e)-} as an apparently recent innovation of grammaticalisation) in a wide variety of languages. If we follow these assumptions, the distribution summarised in \tabref{tab:shinagawa:11} would suggest the following process.

In Rwa and Mashami, *\textit{{}-ag-a} shifted and split into future and habitual due to the innovation of a novel progressive marker originating from *\textit{kad}.\footnote{For more detail on the semantic change of *\textit{{}-ag} in Rwa, see \citet{Shinagawa2009}.} In Siha, the same process was initiated by the progressive \textit{li-}, which plays the same role in Vunjo, where \textit{ke-} shifted to habitual aspect. As suggested in  \sectref{sec:shinagawa:2.3} and \sectref{sec:shinagawa:3.3.1}, \textit{{}-á} of habitual in Uru might be regarded as a relic of *\textit{{}-ag-a}, which has completely vanished in non-WK areas probably due to the lack or weakness of vowel length contrast. Though the process is still uncertain, it is worth mentioning that in Rwa a reflex of *\textit{li}, which is used as a progressive marker in most languages, denotes past imperfective, which is encoded by \textit{we-} in Vunjo and Rombo-Mkuu and by \textit{e-} in other languages.


\begin{table}
\small
\begin{tabularx}{\textwidth}{XXXXXXXX} 
\lsptoprule
& \multicolumn{4}{c}{WK} & \multicolumn{2}{c}{CK} & Rombo\\
\cmidrule(lr){2-5}\cmidrule(lr){6-7}\cmidrule(lr){8-8}
& Rw. & Ms. & Sh. & Kb. & Vn. & Ur. & Mkuu\\
\midrule
{\scshape fut2} & \multirow{2}{=}{\itshape {}-áa} & \multirow{2}{=}{\itshape {}-áa} & \multirow{2}{=}{\itshape {}-áa} & \multirow{2}{=}{(SM-)\textit{V-}} & {\itshape e-} & {\itshape ci-} & \textit{e\textsubscript{1}}\textit{{}-}\\
{\scshape fut1} &  &  &  &  & {\itshape ci-} & {\itshape i-} & {\itshape ʃe-}\\
{\scshape hab} & {\itshape {}-aá} & {\itshape {}-aa} & {\itshape {}-aa} & \textit{{}-áa} & {\itshape ke-} & {\itshape {}-á} & \textit{e\textsubscript{0}}\textit{{}-}\\
{ \textsc{prog}} & {\itshape kée-} & {\itshape ke-} & {\itshape li-} & {\itshape {}-aá} & {\itshape i-} & {\itshape ke-} & {\itshape i-}\\
{\scshape p.i.} & {\itshape i-} & {\itshape e-} & {\itshape e-} & {\itshape e-} & {\itshape we-} & \textit{e\textsubscript{0}}\textit{{}-} & {\itshape ve-}\\
\lspbottomrule
\end{tabularx}
\caption{Formal correspondence of TAMs denoting \textsc{prog}-\textsc{fut}-\textsc{hab} continuum}
\label{tab:shinagawa:11}
\end{table}

\subsection{Microparametric approach to regional features}\label{sec:shinagawa:4.2}

 This section briefly presents a provisional sketch of micro-typological correlations between morphosyntactic parameters related to TA systems. As observed in \sectref{sec:shinagawa:2}, tense-marking systems in KB can be divided into two types, i.e., one is the WK type with a tripartite past and a single future, and the other is the non-WK type with a bipartite system for both past and future. If we focus on the future marking, the former can be classified as the mono-future type, while the latter as the pluri-future type. This distinction largely overlaps with the distribution of *\textit{{}-ag-a}, i.e., the languages classified as the mono-future type are those with *\textit{{}-ag}, while the others lack this element.

The first point of note is that this structural difference itself may suggest a typological correlation, i.e., if a language is the pluri-future type, then the language is likely to denote future time reference with pre-stem TAMs, while if a language is the mono-future type, then future tense is likely to be marked in the Final slot. This provisional correlation may be justified by the difference of “openness” of the morphological slots, i.e., pre-stem slots are relatively open for newly grammaticalised elements, while the Final slot is a rather closed slot for a limited number of inflectional elements.

The second point involves the relationship with the degree of grammaticalisation of the markers COM and GOM. As presented in \sectref{sec:shinagawa:2.1}, these markers are essentially used to denote modality in Rwa and the situation seems similar in Vunjo \citep{Nurse2003a}.\footnote{However, their meaning seems discrepant between WK and Vunjo, i.e., GOM denotes strong certainty in WK, while it denotes relatively weak certainty in Vunjo compared to COM (cf. \citealt{ShinagawaForthcoming-a}).} However, in pluri-future type languages, COM is more like a future tense marker (e.g., Rombo-Mkuu is a typical case, see \sectref{sec:shinagawa:2.4}), while GOM is asymmetrically less grammaticalised, e.g., as shown in \sectref{sec:shinagawa:2.1}. On the other hand, GOM is used even with past tense forms in Rwa, while its cooccurrence with a past marker is not grammatically acceptable in Rombo-Mkuu. While this may suggest a correlation between types of future tense categorisation and grammaticalisation types of ‘come’ and ‘go’ verbs, it should be noted that languages without a fully developed GOM also tend to have a fully developed focus-marking system with \textit{ní-} (cf. \citealt{Shinagawa2015, ShinagawaForthcoming-b}).


\section{Concluding remarks}\label{sec:shinagawa:5}

This study has presented an overview of the distribution of common TAMs in KB and their semantic correspondences across the languages of this group, some of which have scarcely been examined in previous studies. Based on the systematic correspondences revealed by the data, it was shown that there are two clusters of TA categories, namely past-anterior-stative and progressive-future-habitual. These categories are interconnected across the boundaries of languages or subgroups through grammaticalisation chains of common TAMs. The findings have also suggested that some regionally shared features may also be regarded as reflecting micro-typological correlations, i.e., a typological distinction between a mono-future (WK) type and a pluri-future (non-WK) type may be correlated with different components of the TA-marking system such as grammaticalization types of COM and GOM and/or the existence of \textit{ni-} as a focus-marking strategy in a broad sense.

However, there are various issues to be investigated further. First, comprehensive descriptions of the TA systems of under-described languages are needed to fill the gaps with reliable data. Second, the scope of description should also be expanded to cover the entire range of TA expressions including compound tenses, TA in dependent clauses, negative clauses. Finally, it would be valuable to explore the whole range of morphosyntactic microvariation of each language in order to investigate the possible micro-parametric correlations between TA systems and other logically independent properties of grammar, which may shed light on shared principles of KB grammar underlying the group’s internal linguistic diversity.

\section*{Acknowledgements}
This work is made possible by financial support from JSPS’s Grants-in-Aid for Scientific research (C) \#16K02630 and \#19K00568, and is part of the academic results of ILCAA’s joint research project “Typological Study of Microvariation in Bantu (Phase 2).” My sincere thanks go to anonymous reviewers for their helpful comments and especially Gérard Philippson, whose detailed comments and suggestions helped me to avoid errors in the analysis and interpretation of the data. However, of course, remaining errors or misunderstandings are my own.

\section*{Abbreviations}

\begin{tabbing}
(↑/↓) CERT \ldots\hspace{1ex} \= Class numbers\kill
1, 2, 3 \ldots \> Class numbers\\
1sg/pl \ldots \> Person + Singular/Plural\\
\textsc{ant} \ldots \> Anterior\\
(↑/↓) \textsc{cert} \ldots \> (high/low) Certainty\\
\textsc{com} \> grammaticalised marker from ‘come’\\
\textsc{comp} \> Completive\\
\textsc{prog} \> Progressive\\ 
\textsc{ext} \> Existential\\
\textsc{foc} \> Focus marker\\
\textsc{fut}\textsubscript{(1{\textasciitilde}2)} \> Future (near {\textasciitilde} remote)\\
G \> Glide\\
\textsc{gom} \> grammaticalised marker from ‘go’\\
H \> High tone\\
\textsc{hab} \> Habitual\\
\textsc{neg}\textsubscript{2} \> Secondary Negative\\
\textsc{p.i} \> Past Imperfective (Past tense for Imperfective aspect forms)\\
\textsc{prs} \> Present\\
\textsc{pst}\textsubscript{(1{\textasciitilde}3)} \> Past (near {\textasciitilde} remote)\\
\textsc{sm} \> Subject Marker\\
\textsc{stat} \> Stative\\
%\textsc{subj} \> Subjunctive\\
\textsc{tam} \> Tense and Aspect marker\\
\textsc{v} \> Vowel (including a copied vowel of a preceding element)\\
\textsc{vc} \> Vowel Copy (clitic)\\
\textsc{vlc} \> Vowel Length Contrast\\
- \> Affix boundary\\
= \> Clitic boundary\\
${\neq}$ \> Verb stem boundary\\
\# \> Word boundary\\
Tonal annotation (broad phonetic): \> \\
{[á]} \> high\\
{[ꜛá]} \> upstepped high (descriptive expression of [\H{a}]: super high)\\
{[ꜜá]} \> downstepped high\\
{[ā]} \> middle\\
{[â]} \> falling\\
{[ǎ]} \> rising\\
\end{tabbing}

\sloppy\printbibliography[heading=subbibliography,notkeyword=this]
\end{document}
