\chapter{Neutral vowels}\label{sec:6}
\is{velar fronting islands|(}\il{Highest Alemannic|(}
\section{Introduction}\label{sec:6.1}

In the SwG dialects discussed below velar fronting is an active synchronic process creating a palatal (e.g. [ç]) from the corresponding velar (e.g. /x/), but there are also many regular instances of underlying velars (e.g. /x/) surfacing unexpectedly without change (e.g. [x]) in the front vowel context. For example, in one dialect, the /i/ component of the diphthong /ei/ triggers the fronting of a following /x/ to [ç], but /x/ underapplies after the /i/ component of the diphthong /øi/, i.e. [eiç] vs. [øix]. The aberrant \isi{vocoid} -- that is, the [i] in [øi] -- is a \isi{neutral vowel} (\sectref{sec:2.4.2}), defined as a phonetically front vowel lacking the place feature [coronal], as in \REF{ex:6:1b}. By contrast, \isi{nonneutral vowels} like the /i/ in /ei/ are phonetically front and phonologically [coronal], as in \REF{ex:6:1a}.\footnote{In \REF{ex:6:1}, [αF] is an abbreviation for all other distinctive properties (e.g. the major class feature [+sonorant], height features [±high] and [±low], the \isi{tenseness} feature [±tense], or manner features like [±nasal] if there is a contrast between oral and nasalized vowels). The presence of [αF] in representation \REF{ex:6:1b} makes neutral vowels distinct from \isi{schwa}, which bears only [+sonorant] and [--consonantal], but no additional features.}

\ea%1
\label{ex:6:1}Representations for front vowels:
\begin{multicols}{2}\raggedcolumns
\ea Front \isi{nonneutral vowels}:\\\label{ex:6:1a}
    \begin{forest}
      [\avm{[−cons\\αF]} [\avm{[coronal]}]]
    \end{forest}\columnbreak
\ex Neutral vowels:\\\label{ex:6:1b}
    \begin{forest}[\avm{[−cons\\αF]}]\end{forest}
\z 
\end{multicols}
\z 

The representation in \REF{ex:6:1b} is intended to indicate the absence of [coronal]. Front rounded neutral vowels (/y/ or /ʏ/), present in both dialects discussed below, must bear a feature capturing roundedness to make them distinct from their unrounded counterparts.

The \isi{neutral vowel} in \REF{ex:6:1b} derived historically from a back vowel, e.g. /øi/ < /ou/; Recall \figref{fig:2.10}. The vocalic change that created \REF{ex:6:1b} therefore exemplifies \isi{Vowel Fronting}, which in this case involved the deletion of the backness feature for back vowels without the addition of the [coronal]. In that type of example, the historical process of velar fronting underapplied after historically back sounds like /øi/; hence, velar sounds like [x] in the context of neutral vowels like [øi] are opaque.

In the remainder of this chapter I present two case studies from \il{Highest Alemannic}HstAlmc illustrating neutral vowels (\sectref{sec:6.2}, \sectref{sec:6.3}). In \sectref{sec:6.4} I consider how neutral vowels emerged historically, and in \sectref{sec:6.5} I provide some discussion. The chapter concludes in \sectref{sec:6.6}.

\section{Highest Alemannic (part 1)}\label{sec:6.2}

The present section investigates the patterning of dorsal fricatives and affricates in the \il{Highest Alemannic}HstAlmc variety described in detail by \citet{Wipf1910}, spoken in the town of \ipi{Visperterminen} in the Swiss canton of Valais (Wallis; \mapref{map:2}). \ipi{Visperterminen} is part of a large \isi{velar fronting island} comprising \ipi{Upper Valais} because it is surrounded by non-velar fronting regions; see \sectref{sec:15.8} for discussion.

The patterning of dorsal fricatives and affricates in \ipi{Visperterminen} can only be understood by considering first the phonetics and especially the phonology of vowels (\sectref{sec:6.2.1}). The intricate distribution of velar and palatal stops and affricates is discussed in \sectref{sec:6.2.2}.

\subsection{Phonetics and phonology of vowels}\label{sec:6.2.1}

\ipi{Visperterminen} has phonemic oral and nasalized vowels. The monophthongs consist of front vowels (oral /iː i yː eː e ɛː ɛ  æː æ/ and nasalized /{\~\i}ː \~i  ỹː ẽː  \~ɛː \~ɛ/) and back vowels (oral /(uː) u oː o ɑː ɑ/ and nasalized /\~u  õː õ ɑ͂ː ɑ͂/). /uː/ is parenthesized because it occurs in only a very small number of words \citep[11]{Wipf1910}. The only front rounded monophthongs are /yː/ and /ỹː/.\footnote{\label{fn:6:2}[yː] derives historically from \ili{OHG} [uː], which underwent a context-free fronting, e.g. OHG [fuːl] > [fyːl] ‘lazy’. Nasalized vowels arose historically through the assimilation of nasality from a nasal consonant to a preceding vowel followed by the deletion of that nasal consonant before a fricative. The details of that change \citep[44--45]{Wipf1910} exceed the goals of the present analysis.} There are six phonemic diphthongs, namely oral /øi ei yo iæ/ and nasalized /\~ɑi ẽi/. Note that two front rounded vowels occur in diphthongs which are absent in the system of monophthongs, i.e. /ø/ in /øi/ and /y/ in /yo/.

\ipi{Visperterminen} is extremely conservative in the sense that it preserves a number of features from \ili{OHG}. One such feature is the retention of \isi{full vowels} in unstressed syllables, which were ultimately reduced to \isi{schwa} in \ili{MHG}, e.g. [hilffu] ‘help-\textsc{1sg}’ (cf. \il{Standard German}StG [hɛlfə]); \citet[146]{Wipf1910}. Since \isi{Vowel Reduction} (\chapref{sec:4})  never occurred in \ipi{Visperterminen}, [ə] is not a phonemic vowel. Wipf transcribes the second element of the diphthong [iæ] as \isi{schwa} (⟦ə⟧). However, she notes that the pronunciation with [ə] as the second component only holds for fast speech (“das rasche zusammenhängende Sprechenˮ; \citealt{Wipf1910}: 12). In the same passage she observes that the pronunciation of the second part of the diphthong in question with the low front vowel [æ] is typical for slower speech. The example she gives is the word \textit{Fieber} ‘fever’, which can be pronounced [fiəber] or [fiæber]. In the related dialect discussed below in \sectref{sec:6.3} \citep{Brun1918}, that author makes a similar observation, but he consistently transcribes the second component of the diphthong in question as ⟦æ⟧; (\citealt{Brun1918}: 18--19). I transcribe the diphthong [iæ] henceforth with [æ]; more significantly, there is evidence discussed throughout this chapter that [i] and [æ] in [iæ] are phonologically front ([coronal]) vowels. I see the pronunciation [iə] as a consequence of a rule of \isi{phonetic implementation} that is not relevant for the phonology.

The two components of the six diphthongs can be made distinct by referring to features referring to height, roundedness, backness, and nasality. Consider first /øi/ and /ei/. What those two diphthongs have in common is that the first part is mid and front and the second part high (i.e. /i/). The difference between /ø/ and /e/ in the first component of /øi/ and /ei/ involves only \isi{rounding}. For the two oral diphthongs /yo/ and /iæ/ the first component is high (/y/ or /i/) and the second component nonhigh (/o/ or /æ/). Note that the two vowels /o/ and /æ/ differ in terms of backness. What the two nasalized diphthongs /\~ɑi/ and /ẽi/ have in common is that the first component is nonhigh (/\~ɑ/ or /ẽ/), and the second component is high (/i/). The difference between /\~ɑ/ or /ẽ/ is one of backness. Phonological representations for the diphthongs are provided below.

The four oral diphthongs /øi ei yo iæ/ are phonemic because they contrast with one another and with monophthongs. Although actual minimal pairs were not found in the original source, it is not difficult to find examples of words in which those diphthongs appear in very similar environments. In \REF{ex:6:2} I present monosyllabic words in which the four oral diphthongs surface between two consonants.\footnote{{All of the oral monophthongs can also surface as the V in CVC words, although I present no examples here. I conclude that there is no evidence that the diphthongs in \REF{ex:6:2} derive from monophthongs as is sometimes proposed for other languages (\sectref{sec:2.2.3}).}}

\TabPositions{.1\textwidth, .25\textwidth, .4\textwidth, .66\textwidth}
\ea%2
\label{ex:6:2}
\ea\label{ex:6:2a} teiff  \tab [teiff] \tab tief  \tab ‘deep’             \tab 37
\ex\label{ex:6:2b} briəf  \tab [briæf] \tab Brief \tab ‘letter’           \tab 38
\ex\label{ex:6:2c} böim   \tab [bøim]  \tab Baum  \tab ‘tree’             \tab 38
\ex\label{ex:6:2d} büob   \tab [byob]  \tab Bube  \tab ‘single young man’ \tab 40
\z 
\z 

From the point of view of phonology, the four diphthongs in \REF{ex:6:2} are not derived. Thus, they are present in the underlying representations as /ei/, /iæ/, /øi/, and /yo/ and surface without change as [ei], [iæ], [øi], and [yo]. However, I show below that there are also regular Umlaut-based alternations involving /ei/{\textasciitilde}/øi/ and /iæ/{\textasciitilde}/yo/.

The status of the two nasalized diphthongs /\~ɑi/ and /ẽi/ in the synchronic phonology (in particular [\~ɑi]) is not as clear-cut as the status of the four oral diphthongs. Since /\~ɑi/ and /ẽi/ derived historically from an oral vowel plus nasal consonant sequence before a fricative (recall \fnref{fn:6:2}), they occur primarily in the context before the fricative, e.g. [x\~ɑiʃt] ‘can-\textsc{2sg}’ (cf. \il{Standard German}StG \textit{kannst}), [gʃpẽiʃt] ‘ghost’ (cf. \il{Standard German}StG \textit{Gespenst}). Word-finally, /ẽi/ surfaces in words like [klẽi] ‘small’ (cf. \il{Standard German}StG \textit{klein}). In that context it contrasts with the oral diphthongs, e.g. [hiə] ‘here’, [fryo] ‘early’. \citet[45]{Wipf1904} notes that native speakers often pronounce [\~ɑi] as [ɑŋ], which suggests that the former is synchronically derived from the latter. I analyze [ẽi] and [\~ɑi] in words like the ones given above as phonemic (i.e. /ẽi/ and /\~ɑi/), although it will be clear below that an analysis in which [ɑi] is synchronically derived from /ɑŋ/ is compatible with my treatment.\footnote{{The mirror-image change (\isi{nasalized vowel} is realized as the corresponding oral vowel plus [ŋ]) has a parallel in the realization of \ili{French} loanwords in \il{Standard German}StG, e.g. [pɑʀf\~{œ}ː]>[pɑʀfœŋ] ‘perfume’; \citet[65]{Mangold2005}.}} It will be seen below that there are \isi{Umlaut} alternations involving [ẽi]{\textasciitilde}[\~ɑi].

The correct features for vocalic segments can be established by considering the way in which they behave phonologically. It is shown on the basis of vocalic alternations that certain front vowels in diphthongs require neutral representations as in \REF{ex:6:1b} and others the nonneutral representation in \REF{ex:6:1a}.

As indicated in \REF{ex:6:3}, vocalic alternations (\isi{Umlaut}) fall into one of three categories. First, back monophthongs alternate with the corresponding front unrounded monophthongs in (\ref{ex:6:3a}).\footnote{{[uː]{\textasciitilde}[iː] alternations are apparently unattested because [uː] is a rare sound. No examples could be found in the original source in which the nasalized vowels [õ] or [õː] occur in the context for \isi{Umlaut}. I omit from the present discussion the short low back vowel [ɑ], which surfaces in the \isi{Umlaut} context in some morphemes as [e] and in others as [æ]. That type variation exemplifies a complication that exceeds the goals of the present work.} } Second, front rounded monophthongs are the umlauted counterparts of the corresponding front unrounded monophthongs, as in (\ref{ex:6:3b}). Third, diphthongs show the pattern of alternation illustrated in \REF{ex:6:3c}. Note that the first component of the diphthongs in the two alternating pairs [øi]{\textasciitilde}[ei] and [yo]{\textasciitilde}[iæ] exhibits the same pattern as in \REF{ex:6:3b}. By contrast, the second part of [øi] remains unchanged in [ei], while the second component of [yo] corresponds to [æ] in the \isi{Umlaut} context.

\ea%3
\label{ex:6:3}
\ea\label{ex:6:3a}
  \relax [u]{\textasciitilde}[i]\\
  \relax [ũ]{\textasciitilde}[ĩ]\\         
  \relax [o]{\textasciitilde}[e]\\       
  \relax [oː]{\textasciitilde}[eː]\\
  \relax [ɑː]{\textasciitilde}[æː]\\

\ex\label{ex:6:3b} \relax [yː]{\textasciitilde}[iː]\\
    \relax [ỹː]{\textasciitilde}[ĩː]\\
    

\ex\label{ex:6:3c} \relax [øi]{\textasciitilde}[ei]\\
    \relax [yo]{\textasciitilde}[iæ]\\
    \relax [\~ɑi]{\textasciitilde}[ẽi]\\
\z 
\z 

The three patterns in \REF{ex:6:3} are displayed in (\ref{ex:6:4}--\ref{ex:6:6}). The morphological contexts for \isi{Umlaut} in these examples are the plural of nouns and diminutives.

\TabPositions{.125\textwidth, .33\textwidth, .5\textwidth, .75\textwidth}
\ea%4
\label{ex:6:4}
\ea\label{ex:6:4a}  hund      \tab  [hund]  \tab Hund    \tab ‘dog’       \tab 122 \\
     hind      \tab  [hind]  \tab Hunde   \tab ‘dog-\textsc{pl}’      \tab 136 \\
\ex\label{ex:6:4b}  su͔       \tab   [sũ]   \tab  Sohn   \tab  ‘son’      \tab  122\\
     si͔       \tab   [sĩ]   \tab  Söhne  \tab  ‘son-\textsc{pl}’     \tab  122\\
\ex\label{ex:6:4c}  xopf      \tab  [xopf]  \tab Kopf    \tab ‘head’      \tab  93 \\
     xepf      \tab  [xepf]  \tab Köpfe   \tab ‘head-\textsc{pl}’     \tab  93 \\
\ex\label{ex:6:4d}  flō       \tab  [floː]  \tab Floh    \tab ‘flea’      \tab  122\\
     flē       \tab  [fleː]  \tab Flöhe   \tab ‘flea-\textsc{pl}’     \tab   35\\
\ex\label{ex:6:4e}  fɑ̄lt      \tab  [fɑːlt] \tab Falte   \tab ‘wrinkle’   \tab  122\\
     f\={æ}lt  \tab  [fæːlt] \tab Falten  \tab  ‘wrinkle-\textsc{pl}’ \tab  122\\
\z 
\ex%5
\label{ex:6:5}
\ea\label{ex:6:5a}  krǖt    \tab [kryːt]   \tab Kraut   \tab ‘herb’   \tab 93  \\
     krītter \tab [kriːter] \tab Kräuter \tab ‘herb-\textsc{pl}’  \tab 93  \\
\ex\label{ex:6:5b}  tsü͔    \tab  [tsỹː]   \tab  Zaun   \tab  ‘fence’ \tab  122\\
     tsī͔    \tab  [tsĩː]   \tab  Zäune  \tab  ‘fence-\textsc{pl}’\tab  122\\
\z 
\ex \label{ex:6:6}
\ea\label{ex:6:6a}  bøim    \tab [bøim]    \tab Baum         \tab ‘tree’                   \tab  38 \\
     beim    \tab [beim]    \tab Bäume        \tab ‘tree-\textsc{pl}’                  \tab  39 \\
\ex\label{ex:6:6b}  brüoder \tab [bryoder] \tab Bruder       \tab  ‘brother’               \tab  40 \\
     briədri \tab [briædri] \tab Bruder, dim  \tab ‘brother\textsc{{}-dim}’ \tab  40 \\
\ex\label{ex:6:6c}  He͔iši  \tab  [hẽiʃi]  \tab  Hans, dim   \tab  ‘Hans\textsc{{}-dim}’   \tab  168\\
     Hɑ͔isi  \tab  [h\~ɑisi]  \tab  Hans, dim   \tab  ‘Hans\textsc{{}-dim}’   \tab  168\\
\z 
\z 

If a front unrounded vowel occurs in the \isi{Umlaut} context, then that vowel does not exhibit an alternation, e.g. [rind] ‘cow’ vs. [rinner] ‘cow-\textsc{pl}’.

I propose that monophthongs have the representations in \REF{ex:6:7}. In those structures, front unrounded segments are nonneutral and hence [coronal]; see \REF{ex:6:7a}. In contrast to all of the dialects considered in previous chapters, the back monophthongs of \ipi{Visperterminen} are [peripheral]; see \REF{ex:6:7c}. That structure follows \citet{Rice2002}, who proposes that [peripheral] expresses backness and/or roundedness in vowels. Neither [dorsal] nor [labial] are necessary in the representation of back monophthongs given the structure in \REF{ex:6:7c}. The representation for front rounded sounds is presented in \REF{ex:6:7b}. In contrast to \REF{ex:6:7a} and \REF{ex:6:7c}, the one in \REF{ex:6:7b} is a complex structure with [coronal] and [peripheral]. The advantages of analyzing the monophthongs in (\ref{ex:6:7b}, \ref{ex:6:7c}) as [peripheral] and not as [dorsal] and/or [labial] are discussed below.

\ea%7
\label{ex:6:7}
\ea Front unrounded:\\\label{ex:6:7a}

\begin{forest}
[ \avm{[−cons\\αF]}
   [\avm{[coronal]}]
]
\end{forest}
\ex Front rounded:\\\label{ex:6:7b}
\begin{forest}
[ \avm{[−cons\\αF]}
   [\avm{[coronal]}]
   [\avm{[peripheral]}]
]
\end{forest}
\ex Back:\\\label{ex:6:7c}
\begin{forest}
[ \avm{[−cons\\αF]}
   [\avm{[peripheral]}]
]
\end{forest}
\z 
\z 

The contrast between the simplex representations in \REF{ex:6:7a}/\REF{ex:6:7c} and the complex structure in \REF{ex:6:7b} derives support from \isi{markedness}. Regardless of how that term is defined, it is uncontroversially the case that front rounded vowels like [yː] are more marked than both their back ([uː]) and front ([iː]) counterparts (e.g. \citealt{DeLacy2006}, \citealt{Rice2007}).

Individual monophthongs are assigned distinctive features, as indicated in Table \ref{ex:6:8} for the oral vowels. Note that [peripheral] is assigned twice, depending on whether or not it corresponds to the backness or the roundedness dimension. The distinction between short and long vowels is ignored here.

\begin{table}%8
\caption{\label{ex:6:8}Distinctive features for vowels (Visperterminen)}
\begin{tabular}{lcccccccc}
\lsptoprule
         & iː i & yː & eː e & ɛː ɛ & æː æ & uː u & oː o & ɑ ɑː\\\midrule
\relax [coronal] & \ding{51} & \ding{51} & \ding{51} & \ding{51} & \ding{51} &  &  & \\
\relax [peripheral] &  &  &  &  &  & \ding{51} & \ding{51} & \ding{51}\\
\relax [high] & + & + & {}-- & {}-- & {}-- & + & {}-- & {}--\\
\relax [low] &  &  & {}-- & {}-- & + &  & {}-- & +\\
\relax [tense] &  &  & +  & {}-- &  &  &  & \\
\relax [peripheral] &  & \ding{51} &  &  &  &  &  & \\
\lspbottomrule
\end{tabular}
\end{table}

I classify the six diphthongs in terms of the values of the feature [±nasal] and a height feature ([±high] or [±low]) for each component, as in \REF{ex:6:9}. Note that four of these diphthongs consist of [--high] followed by [+high] in (\ref{ex:6:9a}--\ref{ex:6:9d}) and two are [+high] followed by [+low] in (\ref{ex:6:9e}, \ref{ex:6:9f}). I consider additional features for diphthongs below.

\ea%9
\label{ex:6:9}
\begin{multicols}{2}
\ea\label{ex:6:9a} /øi/:\smallskip\\ \avm{[−nasal\\−high] \hspace{1ex} [−nasal\\+high]}  
\ex\label{ex:6:9b} /ei/:\smallskip\\ \avm{[−nasal\\−high] \hspace{1ex} [−nasal\\+high]}
\ex\label{ex:6:9c} /\~ɑi/:\smallskip\\ \avm{[+nasal\\−high] \hspace{1ex} [+nasal\\+high]}  
\ex\label{ex:6:9d} /ẽi/:\smallskip\\ \avm{[+nasal\\−high] \hspace{1ex} [+nasal\\+high]}
\ex\label{ex:6:9e} /yo/:\smallskip\\ \avm{[−nasal\\+high] \hspace{1ex} [−nasal\\+low]}    
\ex\label{ex:6:9f} /iæ/:\smallskip\\ \avm{[−nasal\\+high] \hspace{1ex} [−nasal\\+low]}
\z 
\end{multicols}
\z 

The feature [+low] can be justified in \REF{ex:6:9f} because /æ/ is phonetically low, but the same cannot be said about \REF{ex:6:9e} because /o/ is phonetically mid and not low. It needs to be stressed that the features adopted here are intended to capture phonological patterns and not the phonetics of the sounds in question. Recall the discussion of vowels in \sectref{sec:2.2.3} and the analysis of \ipi{Rheintal} /ɛː ɛ œː œ/ as phonologically [+low] in \sectref{sec:3.4}. The /o/ component of the /yo/ diphthong in \REF{ex:6:9e} and the /æ/ component of /iæ/ in \REF{ex:6:9f} do not bear the same features as the respective monophthongs /o/ and /æ/ in Table \ref{ex:6:8}, although my analysis does not crucially depend on this.

The complete featural representations for the four [--high]-[+high] diphthongs are presented in \REF{ex:6:10}. In the following discussion I concentrate on the place features.

\protectedex{\ea%10
    \label{ex:6:10}
    \begin{multicols}{2}\raggedcolumns
\ea /øi/:\smallskip\\\label{ex:6:10a}
    \begin{forest}
    [,phantom
      [\avm{[−cons\\−nas\\−high]} [\avm{[peripheral]}]]
      [\avm{[−cons\\−nas\\+high]}]
    ]
    \end{forest}
\ex /ei/:\smallskip\\\label{ex:6:10b}
    \begin{forest}
    [,phantom
      [\avm{[−cons\\−nas\\−high]} [\avm{[coronal]},name=target]]
      [\avm{[−cons\\−nas\\+high]},name=parent]
    ]
    \draw (parent.south) -- (target.north);
    \end{forest}
\ex /\~ɑi/:\smallskip\\\label{ex:6:10c}
    \begin{forest}
    [,phantom
      [\avm{[−cons\\+nas\\−high]} [\avm{[peripheral]}]]
      [\avm{[−cons\\+nas\\+high]}]
    ]
    \end{forest}    
\ex /ẽi/:\smallskip\\\label{ex:6:10d}
    \begin{forest}
    [,phantom
      [\avm{[−cons\\+nas\\−high]} [\avm{[coronal]},name=target]]
      [\avm{[−cons\\+nas\\+high]},name=parent]
    ]
    \draw (parent.south) -- (target.north);
    \end{forest} 
\z\end{multicols}
\z}

The diphthongs in \REF{ex:6:10b} and \REF{ex:6:10d} consist of a sequence of two front vowels which share the feature [coronal] by the \isi{OCP} (recall \sectref{sec:2.2.3}). Both parts of /ei/ and /ẽi/ are therefore \isi{nonneutral}, as in \REF{ex:6:1a}.

The diphthongs in (\ref{ex:6:10b}, \ref{ex:6:10d}) can now be compared with the ones in (\ref{ex:6:10a}, \ref{ex:6:10c}): The second part of /øi/ and /\~ɑi/ is specified for a height feature but not for [coronal] or [peripheral]; hence, the /i/ in /øi/ and /\~ɑi/ -- but not the /i/ in /ei/ or /ẽi/ -- is a \isi{neutral vowel}, as in \REF{ex:6:1b}. The first part of /øi/ bears the place feature [peripheral]; [coronal] is redundant for /ø/ in /øi/ because there are no diphthongs in \ipi{Visperterminen} consisting of the corresponding back vowel plus /i/, i.e. /oi/. Some evidence that the feature [coronal] is absent in the representation for /ø/ comes from phonetics. \citet[11--12]{Wipf1910} notes that her informants pronounced that vowel as [ø] but that other informants appeared to be pronouncing [o]. The fact that the front rounded vowel in /øi/ vacillates between a front vowel and a back vowel supports a structure like the one in \REF{ex:6:10a} in which the frontness feature ([coronal]) is absent.\footnote{Wipf also notes that the second part of /øi/ can be rounded, i.e. /øi/ can be realized as [øy]. This type of variation is also consistent with the representation in \REF{ex:6:10a} because a feature for \isi{rounding} is absent.}\largerpage

The vowel /\~ɑ/ in /\~ɑi/ in \REF{ex:6:10c} is phonetically back and phonologically [peripheral]. Note that there is no contrast between /\~ɑi/ and a nasalized diphthong whose first member is low and front (/\~{æ}i/); hence, the feature [peripheral] can be interpreted in the phonetics as a back vowel and not as a front vowel.\footnote{It was noted above that [\~ɑi] may be derived synchronically from /ɑŋ/. That type of analysis would require that [+nasal] spreads from /ŋ/ onto the preceding vowel and that /ŋ/ changes into a \isi{nasalized vowel} (i.e. /\~\i/). If that were the correct analysis then the change from [ŋ] to the \isi{nasalized vowel} would require [+consonantal] to change to [--consonantal] and [peripheral] to be deleted.}

In \REF{ex:6:11} I give the representations for the two [+high]-[+low] diphthongs in (\ref{ex:6:9e}, \ref{ex:6:9f}):

\ea%11
    \label{ex:6:11}\begin{multicols}{2}
\ea  /yo/:\\\label{ex:6:11a}
     \begin{forest}
      [,phantom
      [\avm{[−cons\\−nas\\+high]} [\avm{[peripheral]},name=source]]
      [\avm{[−cons\\−nas\\+low]}, name=target]
      ]
      \draw (source.north) -- (target.south);
     \end{forest}
\ex  /iæ/:\\\label{ex:6:11b}
     \begin{forest}
      [,phantom
      [\avm{[−cons\\−nas\\+high]} [\avm{[coronal]},name=source]]
      [\avm{[−cons\\−nas\\+low]}, name=target]
      ]
      \draw (source.north) -- (target.south);
     \end{forest}
   \z\end{multicols}
\z

Both parts of the diphthong /iæ/ in \REF{ex:6:11b} are front and therefore marked [coronal] in the phonological representation. The representation in \REF{ex:6:11b} is therefore akin to the structures in (\ref{ex:6:10b}, \ref{ex:6:10d}) for the other two front diphthongs.

The diphthong /yo/ in \REF{ex:6:11a} consists of a single [peripheral] component. As was the case with /øi/, it is not necessary to include the feature [coronal] for the /y/ component of /yo/ because there is no diphthong in \ipi{Visperterminen} consisting of the corresponding back vowel plus /o/, i.e. /uo/. The front rounded vowel in \REF{ex:6:11a} does not bear the feature [coronal] and is therefore \isi{neutral vowel}, as in \REF{ex:6:1b}.

The representations for monophthongs and diphthongs presented above hold regardless of whether or not the sound in question participates in \isi{Umlaut} alternations. For example, the diphthongs /ei/ and /iæ/ from \REF{ex:6:2} do not alternate with other vowels. However, the /ei/ and /iæ/ alternate with /øi/ and /yo/ in (\ref{ex:6:6a}, \ref{ex:6:6b}).

Given the structures for vowels posited above, \isi{Umlaut} alternations in \ipi{Visperterminen} are expressed as in (\ref{ex:6:12a}) for diphthongs and as in (\ref{ex:6:12b}) for monophthongs. The abbreviation ‘mcat’ is the set of morphological categories (e.g. singular{\textasciitilde}plural in nouns).\largerpage

\ea%12
\label{ex:6:12}
\ea\label{ex:6:12a}   
\begin{tikzpicture}[baseline=(matrix-1-1.base)]
\matrix (matrix) [matrix of nodes, ampersand replacement=\&, 
                  nodes in empty cells, row sep=3ex]
  {(/ ...\& \avm{[−cons]} \& ... \& \avm{[−cons]} ... /)\textsubscript{mcat} \& $\sim$\\
         \& \avm{[peripheral]} \&     \&                                     \&       \\
  };
\matrix (matrix2) [below=\baselineskip of matrix.base west,matrix of nodes, 
                   ampersand replacement=\&, nodes in empty cells, row sep=3ex,
                   anchor=north west]
  {
     (/ ... \avm{[−cons]} ... \& \avm{[−cons]} \& ... /)\textsubscript{mcat}\\
      \avm{[coronal]}  \& \\                              
  };
\draw (matrix-1-2.south) -- (matrix-2-2.north);
\draw (matrix2-1-1.south) -- (matrix2-2-1.north);
\draw (matrix2-1-2.south) -- (matrix2-2-1.north);
\end{tikzpicture}
\pagebreak
\ex\label{ex:6:12b}
\begin{tikzpicture}[baseline=(matrix-1-1.base)]
\matrix (matrix) [matrix of nodes, ampersand replacement=\&, nodes in empty cells, row sep=3ex]
  {(/ ... \& \avm{[−cons]} \& .../)\textsubscript{mcat} \& $\sim$ \& (/... \& \avm{[−cons]} \& .../)\textsubscript{mcat}\\
          \& \avm{[peripheral]}                      \&            \&  \& \& \avm{[coronal]} \& \\    
  };
\draw (matrix-1-2.south) -- (matrix-2-2.north);
\draw (matrix-1-6.south) -- (matrix-2-6.north);
\end{tikzpicture}
\z 
\z 

The advantage of analyzing back monophthongs as [peripheral] and not as [dorsal] (and/or [labial]) is that the umlauted vowels also include front rounded monophthongs. Recall from \REF{ex:6:3b} that /yː/ and /ỹː/ alternate with /iː/ and /ĩː/. In an alternative featural system in which front rounded monophthongs are [coronal] and [labial], it is not clear how \REF{ex:6:3a} and \REF{ex:6:3b} can be unified.

In \REF{ex:6:13} I illustrate the alternations involving the diphthongs in \REF{ex:6:3c}:

\ea%13
\label{ex:6:13}
\begin{tabular}[t]{@{}lccc@{}}
a. & /øi/ & \textasciitilde & /ei/\\
   &  \begin{forest}
      [,phantom
      [\avm{[−cons\\−nas\\−high]} [\avm{[peripheral]},name=source]]
      [\avm{[−cons\\−nas\\+high]}, name=target]
      ]
     \end{forest} & & \begin{forest}
      [,phantom
      [\avm{[−cons\\−nas\\−high]} [\avm{[coronal]},name=source]]
      [\avm{[−cons\\−nas\\+high]}, name=target]
      ] 
      \draw (target.south) -- (source.north);
      \end{forest}\\
b. & /ɑ̃i/ & \textasciitilde & /ẽi/\\
   & \begin{forest}
      [,phantom
      [\avm{[−cons\\+nas\\−high]} [\avm{[peripheral]},name=source]]
      [\avm{[−cons\\+nas\\+high]}, name=target]
      ]
     \end{forest} & & \begin{forest}
      [,phantom
      [\avm{[−cons\\+nas\\−high]} [\avm{[coronal]},name=source]]
      [\avm{[−cons\\+nas\\+high]}, name=target]
      ] 
      \draw (target.south) -- (source.north);
      \end{forest}\\
c. & /yo/ & \textasciitilde & /iæ/\\
   & \begin{forest}
      [,phantom
      [\avm{[−cons\\−nas\\+high]} [\avm{[peripheral]},name=source]]
      [\avm{[−cons\\−nas\\+low]}, name=target]
      ]
      \draw (target.south) -- (source.north);
     \end{forest} & & \begin{forest}
      [,phantom
      [\avm{[−cons\\−nas\\+high]} [\avm{[coronal]},name=source]]
      [\avm{[−cons\\−nas\\+low]}, name=target]
      ] 
      \draw (target.south) -- (source.north);
      \end{forest}\\
\end{tabular}
\z

The representations for vowels were posited on the basis of \isi{Umlaut} alternations. The structures defended above include \isi{nonneutral vowels} as well as neutral vowels. The following predictions can be made regarding the vowels of \ipi{Visperterminen}:\largerpage[-1]\pagebreak

\ea%14
\label{ex:6:14}
\ea\label{ex:6:14a}/i/ in /øi/ and /\~ɑi/ does not behave phonologically like a coronal;
\ex\label{ex:6:14b}/y/ in /yo/ does not behave phonologically like a coronal;
\ex\label{ex:6:14c}/iæ/ behaves phonologically like a coronal;
\ex\label{ex:6:14d}/ø/ in /øi/ does not behave phonologically like a coronal;
\ex\label{ex:6:14e}/o/ in /yo/ does not behave phonologically like a dorsal
\z 
\z 

In \sectref{sec:6.2.2} I demonstrate that the predictions in (\ref{ex:6:14a}--\ref{ex:6:14c}) are correct on the basis of the patterning of dorsal fricative and dorsal \isi{affricate} allophones. By contrast, predictions (\ref{ex:6:14d}, \ref{ex:6:14e}) are shown to be untestable.

\subsection{Dorsal fricatives and affricates}\label{sec:6.2.2}

\ipi{Visperterminen} possesses two singleton dorsal fricatives, namely velar [x] (=⟦x⟧) and palatal [ç] (=⟦χ⟧); \citet[14]{Wipf1910}. [x] and [ç] also have geminate counterparts, namely [xx] (=⟦xx⟧) and [çç] (=⟦χχ⟧). It is clear from the original source \citep[16]{Wipf1910} that the geminate articulation is the surface realization of a dorsal fricative after a short vowel. (“Der Spirant \textit{x} resp. \textit{χ} kommt nur nach kurzem Vokal als Geminata vorˮ). By contrast, the singleton counterparts [x] and [ç] occur in the elsewhere case, i.e. after a long vowel or consonant or word-initially. I assume that singletons and geminates are allophones, although I do not provide a formal treatment.

As in \ipi{Rheintal} (\sectref{sec:3.4}), \ipi{Visperterminen} also possesses the two dorsal affricates, namely velar [kx] (=⟦kx⟧) and palatal [kç] (=⟦kχ⟧). Affricates are phonemic because they contrast with stops and fricatives at the same place of articulation, e.g. after [u] in [luk] ‘loose’ vs. [bruxx] ‘fracture’ vs. [ʃtukx] ‘piece’. The distribution of [kx] and [kç] is shown below to mirror the distribution of the corresponding fricatives. The relationship between velar and palatal fricatives and affricates (ignoring the geminate realizations) is depicted in \REF{ex:6:15} for word-initial and postsonorant position.

\ea\label{ex:6:15}
\begin{forest} 
     [,phantom
         [/x/,calign=first  [{[x]}]    [{[ç]}]]         
         [/kx/,calign=first [{[kx]}]   [{[kç]}]]
     ]
      \end{forest} 
\z 


The intricate facts involving the distribution of the sounds in \REF{ex:6:15} are summarized in \REF{ex:6:16} and \REF{ex:6:17}. These statements mirror very closely the historical observations in the original source (\citealt{Wipf1910}: 92, 93, 96).\largerpage[-1]\pagebreak

\ea%16
{[x]/[kx] and [ç]/[kç] in word-initial onsets:}\label{ex:6:16}
\ea\label{ex:6:16a}\relax [ç] occurs word-initially only before high front vowels but not before [yo];
\ex\label{ex:6:16b}\relax [x] occurs word-initially before nonhigh front vowels, back vowels, or coronal sonorant consonants;
\ex\label{ex:6:16c}\relax [x] occurs word-initially before [yo];
\ex\label{ex:6:16d}\relax [kç] and [kx] have the same distribution as word-initial [x]/[ç].
\z
\ex%17
{[x]/[kx] and [ç]/[kç] after a sonorant:}\label{ex:6:17}
\ea\label{ex:6:17a}\relax [ç] occurs after high front vowels with the exception of [øi] and [\~ɑi];
\ex\label{ex:6:17b}\relax [ç] occurs after [iæ];
\ex\label{ex:6:17c}\relax [x] occurs after nonhigh front vowels or back vowels (not including [iæ]);
\ex\label{ex:6:17d}\relax [x] occurs after [øi] and [\~ɑi];
\ex\label{ex:6:17e}\relax [ç] occurs after high front vowels followed by a liquid;
\ex\label{ex:6:17f}\relax [x] occurs after any other vowel followed by a liquid;
\ex\label{ex:6:17g}\relax [kç] and [kx] have the same distribution as [x]/[ç].
\z
\z 

The generalizations in \REF{ex:6:16} and \REF{ex:6:17} together indicate that palatal and velar fricatives and palatal and velar affricates do not contrast.

Distributional statement \REF{ex:6:16a} is revealed in \REF{ex:6:18}: Word-initial [ç] occurs before a high front vowel, namely [iː] in (\ref{ex:6:18a}), [i] in (\ref{ex:6:18b}), [iæ] in (\ref{ex:6:18c}), or [ỹː] in (\ref{ex:6:18d}). The historical source for velar and palatal fricatives and affricates in \REF{ex:6:18} and in all subsequent datasets is \ili{WGmc} \textsuperscript{+}[k] or \textsuperscript{+}[x], although a few assimilated loanwords are included as well. In a number of examples presented below there are front stem vowels that were originally back; thus, \isi{Vowel Fronting} fed velar fronting. I comment on those examples below.

\ea%18
\label{ex:6:18}Word-initial [ç] (from /x/):
\ea\label{ex:6:18a}χībe     \tab [çiːbe]   \tab zürnen  \tab ‘be angry-\textsc{inf}’ \tab  35
\ex\label{ex:6:18b}χind     \tab [çind]    \tab Kind    \tab ‘child’                 \tab 124
\ex\label{ex:6:18c}χiəl     \tab [çiæl]    \tab kühl    \tab ‘cool’                  \tab  92
\ex\label{ex:6:18d}χü͔χ  lɑ \tab  [çỹːçlɑ] \tab  Kunkel \tab  ‘explosive pellet’     \tab   94
\z
\z 


The absence of words beginning with a dorsal fricative followed by the oral vowel [yː] is accidental. Evidence that [yː] behaves as a front vowel -- like its nasal counterpart [ỹː] {}-- comes from the occurrence of the word-initial palatal \isi{affricate} before that vowel (see below). It is also shown that /x/ is realized as palatal in the context after [yː].

The data presented below reveal that [x] surfaces in word-initial position in the elsewhere case (=\ref{ex:6:16b}, \ref{ex:6:16c}). In \REF{ex:6:19}, word-initial [x] is followed by a back vowel in (\ref{ex:6:19a}), a non-high front vowel in (\ref{ex:6:19b}), or [yo] in (\ref{ex:6:19c}). Note that sequences like [xyo] reveal \isi{underapplication} \isi{opacity}. No examples were found in the original source in which a word-initial dorsal fricative is followed by the back vowel [ɑː] or before the nonhigh front vowel [æː]. I hold these gaps to be accidental.

\ea%19
\label{ex:6:19}Word-initial [x] (from /x/):
\ea\label{ex:6:19a}  xuxxi  \tab [xuxxi]    \tab   Küche \tab ‘kitchen’ \tab 93\\
     xopf   \tab [xopf]     \tab  Kopf   \tab ‘head’    \tab 93\\
     xōru   \tab [xoːru]    \tab   Korn  \tab ‘grain’   \tab 93\\
     xɑtsɑ  \tab [xɑtsɑ]    \tab   Katze \tab ‘cat’     \tab 92\\
\ex\label{ex:6:19b}  xebjɑ  \tab  [xebjɑ]   \tab  Käfig  \tab ‘cage’    \tab 93\\
     xertsɑ \tab  [xertsɑ]  \tab  Kerze  \tab ‘candle’  \tab 93\\ 
     xɛnnu  \tab  [xɛnnu]   \tab  können \tab ‘be able\textsc{{}-inf}’ \tab 93\\
     xællɑ  \tab  [xællɑ]   \tab  Kelle  \tab ‘trowel’                 \tab 93\\
     xeišto \tab  [xeiʃto]  \tab  Keim   \tab ‘germ’                   \tab 93\\
     xøiffu \tab   [xøiffu] \tab  kaufen \tab ‘buy\textsc{{}-inf}’ \tab 71\\
\ex\label{ex:6:19c}  xüo    \tab [xyo]      \tab   Kuh   \tab ‘cow’ \tab 127\\
     xüoffɑ \tab [xyoffɑ]   \tab  Kufe   \tab ‘vat’ \tab 40 \\
  \z                        
\z 

The examples in \REF{ex:6:20} show that velar [x] -- but not palatal [ç] -- occurs in word-initial position before a coronal sonorant consonant, namely [n] in (\ref{ex:6:20a}), [l] in (\ref{ex:6:20b}), or [r] in (\ref{ex:6:20c}); recall \REF{ex:6:16b}. There are no restrictions governing the type of vowel that can follow the sonorant consonant in question. In particular, that vowel can be high and front, but that high front vowel exerts no influence on the initial dorsal fricative, which consistently surfaces as [x].

\ea%20
\label{ex:6:20}Word-initial [x] (from /x/):
\ea\label{ex:6:20a} xnɑll    \tab [xnɑll]  \tab Knall  \tab ‘bang’                   \tab 93 \\
      xnæxt  \tab [xnæxt]  \tab Knecht \tab ‘vassal’                 \tab 121\\
\ex\label{ex:6:20b} xlɑɡu    \tab [xlɑgu]  \tab klagen \tab ‘complain-\textsc{inf}’  \tab 93 \\
      xliwwe \tab [xliwwe] \tab Kleie  \tab ‘bran-\textsc{pl}’       \tab 93 \\
\ex\label{ex:6:20c} xrīts    \tab [xriːts] \tab Kreuz  \tab ‘cross’                  \tab 93 \\
      xrɑnts \tab [xrɑnts] \tab Kranz  \tab ‘wreath’                 \tab 93 \\
\z 
\z

Wipf includes in her grammar [x]{\textasciitilde}[ç] alternations like the ones in \REF{ex:6:21}, which suggest that the complementary distribution between word-initial [x] and [ç] described above is a rule-governed process. In the first example in \REF{ex:6:21a} the stem vowel is [u], which alternates with [i], as in the second example. The pair of words in \REF{ex:6:21b} is similar to the word pair in \REF{ex:6:21a}, although the stem vowel in [çiːrli] shows the effects of an apparently idiosyncratic process of raising (together with \isi{Umlaut}). Significantly, the [x] in [xoːru] ‘grain’ is replaced by [ç] in [çiːrli] ‘grain-\textsc{dim}’ because the vowel [iː] follows [x]. The examples in \REF{ex:6:21c} demonstrate that the stem vowel [o] alternates with [e] but that [x] does not change to [ç] after the latter vowel because [e] is not high and front. The most significant examples are the ones in \REF{ex:6:21d} because they indicate that opaque [x] is only present before the one diphthong [yo]. When that diphthong is replaced with [iæ] in the plural, opaque [x] surfaces as [ç] as expected.

\ea%21
\label{ex:6:21}
Dorsal fricatives (from /x/) before alternating vowels:
\ea \label{ex:6:21a}
     xurts   \tab [xurts]   \tab kurz   \tab ‘short’   \tab  93 \\
     χirtzer \tab [çirtser] \tab kürzer \tab ‘shorter’ \tab  93 \\
\ex\label{ex:6:21b} xōru  \tab [xoːru]  \tab Korn       \tab ‘grain’               \tab 93 \\
    χīrli \tab [çiːrli] \tab Korn, dim. \tab ‘grain-\textsc{dim}’  \tab 93 \\
\ex\label{ex:6:21c} xopf  \tab [xopf]   \tab Kopf       \tab ‘grain’               \tab 122\\
    xepf  \tab [xepf]   \tab Köpfe      \tab ‘head-\textsc{pl}’               \tab 122\\
\ex\label{ex:6:21d} xüo   \tab [xyo]    \tab Kuh        \tab ‘cow’                 \tab 127\\
    χiə   \tab [çiæ]    \tab Kühe       \tab ‘cow-\textsc{pl}’                \tab 127\\
\z
\z 

The data presented up to this point show that [x] and [ç] stand in complementary distribution in word-initial position, although the [yo] context is characterized by \isi{opacity}.

The examples in \REF{ex:6:22} demonstrate that the distribution of the velar \isi{affricate} [kx] and its palatal counterpart [kç] parallels the distribution of the corresponding fricatives (=\ref{ex:6:16d}). Thus, [kç] occurs in word-initial position before a high front vowel in (\ref{ex:6:22a}) and [kx] in the elsewhere case in (\ref{ex:6:22b}). The second example in \REF{ex:6:22a} is important because it illustrates the occurrence of the palatal \isi{affricate} before [yː]; recall the discussion above on the absence of word-initial dorsal fricatives before that vowel.\footnote{{The affricates in [kçitsjot] and [kxeːrt] are synchronically derived from the past participle prefix /k/, which coalesces with the stem-initial fricative (/x/), i.e. /k-xitsjot/ and /k-xeːrt/.  The remaining examples in \REF{ex:6:22} show that there is also a phonemic \isi{affricate} /kx/.}}\largerpage[-1]\pagebreak

\ea%22
Word-initial dorsal affricates (from /kx/):\label{ex:6:22}
\TabPositions{.15\textwidth, .33\textwidth, .5\textwidth, .75\textwidth}
\ea\label{ex:6:22a}    kχitsjot  \tab [kçitsjot]   \tab gekitzelt \tab ‘tickle-\textsc{part}’ \tab 69\\
       kχǖr      \tab [kçyːr]      \tab Kur       \tab ‘health resort’         \tab 95\\
\ex\label{ex:6:22b}    kxɑffḗ    \tab  [kxɑffeː]   \tab Kaffee    \tab ‘coffee’                \tab 95\\
       kxumpíəru \tab  [kxumpiæru] \tab kopieren  \tab ‘copy-\textsc{inf}’     \tab 95\\
       kxērt     \tab  [kxeːrt]    \tab gekehrt   \tab ‘sweep-\textsc{part}’   \tab 69\\
    \z
\z 

No examples in the original source were found in which a word-initial dorsal \isi{affricate} surfaces before any of the diphthongs. I interpret this gap as accidental.

The examples in \REF{ex:6:23} show that [k] does not have a palatal realization. In word-initial position, [k] surfaces before any vowel. Example \REF{ex:6:23a} has [k] before a high front vowel and the ones in \REF{ex:6:23b} have [k] before other vowels.

\ea%23
\label{ex:6:23}Word-initial [k] (from /k/):
\ea\label{ex:6:23a} kinte  \tab [kinte]  \tab Launen    \tab ‘mood-\textsc{pl}’                  \tab 96\\
\ex\label{ex:6:23b} keittu \tab [keittu] \tab schwanken \tab ‘fluctuate-\textsc{inf}’ \tab 96\\
    kætter \tab [kætter] \tab Gitter    \tab ‘grate’                  \tab 97\\
    koffrɑ \tab [koffrɑ] \tab Koffer    \tab ‘suitcase’               \tab 95\\
    kunto  \tab [kunto]  \tab Konto     \tab ‘account’                \tab 95\\
\z                       
\z 

The conclusion is that the velars [x]/[kx] and the corresponding palatals [ç]\slash\relax[kç] do not contrast in word-initial position. The distribution of those sounds follows if the underlying velars (/x/ and /kx/) and the surface palatals ([ç] and [kç]) have the representations in \REF{ex:6:24}. Given those structures, the rule fronting word-initial /x/ and /kx/ is given in \REF{ex:6:25}. Recall from \sectref{sec:2.2.2} that stops are [--sonorant, --continuant], affricates are [--sonorant, --continuant, +continuant], and fricatives are [--sonorant, +continuant]. The target of Wd-Initial Velar Fronting{}-4 is expressed as the natural class of [{}--sonorant, +continuant, peripheral] sounds, i.e. /x/ and /kx/ in \REF{ex:6:24a}. The stop /k/ is not a target since that sound is [--continuant]. No spreading occurs from /r l n/ because none of those sounds is [+high].

\ea%24
Representations for dorsal fricatives/affricates:\label{ex:6:24}
\begin{multicols}{2}\raggedcolumns
\ea /x/, /kx/:\\\label{ex:6:24a}
    \begin{forest}
      [\avm{[−son\\+cont]} [\avm{[peripheral]}]]
    \end{forest}
\ex\relax [ç], [kç]:\\\label{ex:6:24b}
    \begin{forest}
      [\avm{[−son\\+cont]} [\avm{[coronal]}] [\avm{[peripheral]}]]
    \end{forest}
\z
\end{multicols}
\ex%25
\isi{Wd-Initial Velar Fronting-4}:\label{ex:6:25}\\
\begin{forest}
[,phantom
  [\avm{[−son\\+cont]},name=source [\avm{[peripheral]},tier=word]]
  [\avm{[+high]}  [\avm{[coronal]},name=target,tier=word]]
]
\node [left=1ex of source] {\textsubscript{wd} [};
\draw [dashed] (source.south) -- (target.north);
\end{forest}
\z

The structures in \REF{ex:6:24} differ only minimally from the ones presupposed for velar and palatal fricatives in earlier chapters: Velars in \ipi{Visperterminen} are [peripheral] (and not [dorsal]), while palatals are [coronal] and [peripheral] (and not [coronal] and [dorsal]). \isi{Wd-Initial Velar Fronting-4} in \REF{ex:6:25} differs from the corresponding rule posited in the dialects discussed in earlier case studies because the trigger for \REF{ex:6:25} is restricted to [+high] sounds.

\isi{Wd-Initial Velar Fronting-4} spreads [coronal] from a high front segment (e.g. /i/). Recall from (\ref{ex:6:7a}, \ref{ex:6:7b}) that all front \isi{nonneutral vowels} are [coronal]. The natural class of high [coronal] vowels also includes the /i/ in the diphthong /iæ/, as in \REF{ex:6:11b}. By contrast, word-initial /x/ surfaces as opaque [x] before /yo/ because the /y/ in that diphthong -- as a \isi{neutral vowel} -- lacks [coronal], as in \REF{ex:6:11a}; recall \REF{ex:6:16c}.

Distributional generalization \REF{ex:6:17a} is exhibited in (\ref{ex:6:26a}--\ref{ex:6:26g}): Palatal fricatives ([ç] or [çç]) surface only after a high front vowel. The categories within \REF{ex:6:26} illustrate the individual high front vowels, i.e. the oral vowels [i iː yː ei] and the nasalized vowels [ỹː {\~\i}ː ẽi]. Note that the palatal fricatives can surface either in word-final position after a vowel or between vowels.  Generalization \REF{ex:6:17b} is exemplified with example \REF{ex:6:26h}.

\ea%26
Postvocalic palatal fricatives (from /x/):\label{ex:6:26}
\ea\label{ex:6:26a}  līχt   \tab [liːçt]         \tab  leicht  \tab ‘easy’    \tab 35
\ex\label{ex:6:26b}  štiχχ   \tab [ʃtiçç]         \tab  Stich   \tab ‘sting’   \tab 93
\ex\label{ex:6:26c}  bǖχ    \tab [byːç]          \tab  Bauch   \tab ‘stomach’ \tab 35
\ex\label{ex:6:26d}  weiχ   \tab [weiç]          \tab  weich   \tab ‘soft’    \tab 94
\ex\label{ex:6:26e}  χü͔χlɑ \tab  [çỹːçlɑ]       \tab   Kunkel \tab  ‘explosive pellet’   \tab  94
\ex\label{ex:6:26f}  wī͔χill\tab  [w{\~\i}ːçill] \tab   Winkel \tab  ‘angle’              \tab  94
\ex\label{ex:6:26g}  de͔iχu \tab  [dẽiçu]        \tab   denken \tab  ‘think-\textsc{inf}’ \tab  94
\ex\label{ex:6:26h}  liǝχt  \tab [liæçt]         \tab  Licht   \tab ‘light’               \tab 38
\z 
\z

The examples in (\ref{ex:6:27a}--\ref{ex:6:27h}) illustrate the occurrence of velar fricatives ([x] or [xx]) after back vowels or nonhigh front vowels (=\ref{ex:6:17c}). Those eight categories represent the individual vowels, i.e. [u o ɑ ɑː yo e ɛː æ]. The nonoccurrence of words with a velar fricative after the other vowels (e.g. [ɛ eː æː oː] and the nasalized monophthongs) is accidental. The items listed in (\ref{ex:6:27i}, \ref{ex:6:27j}) exemplify \REF{ex:6:17d}: The opaque velar fricative underapplies after the two diphthongs [øi \~ɑi].

\ea%27
\label{ex:6:27}Postvocalic velar fricatives (from /x/):
\ea\label{ex:6:27a} bruxx  \tab [bruxx]  \tab Bruch   \tab ‘fracture’ \tab  93
\ex\label{ex:6:27b} loxx   \tab [loxx]   \tab Loch    \tab ‘hole’     \tab  93
\ex\label{ex:6:27c} bɑxx   \tab [bɑxx]   \tab Bach    \tab ‘stream’   \tab  94
\ex\label{ex:6:27d} nɑ̄x    \tab [nɑːx]   \tab nahe    \tab ‘near’     \tab  34
\ex\label{ex:6:27e} süoxu  \tab [syoxu]  \tab suchen  \tab ‘search-\textsc{inf}’  \tab   156
\ex\label{ex:6:27f} dexxi  \tab [dexxi]  \tab Decke   \tab ‘blanket’  \tab 93
\ex\label{ex:6:27g} nɛ̄xšt  \tab [nɛːxʃt] \tab nächst  \tab ‘next’     \tab 34
\ex\label{ex:6:27h} blæx   \tab [blæx]   \tab Blech   \tab ‘tin’      \tab 94
\ex\label{ex:6:27i} øix    \tab [øix]    \tab auch    \tab ‘also’     \tab 95\\
    røix   \tab [røix]   \tab Rauch   \tab ‘smoke’    \tab 94
\ex\label{ex:6:27j} ɑ͔ixo  \tab  [\~ɑixo]  \tab  Butter \tab  ‘butter’  \tab  94\\
    dɑ͔ixu \tab  [d\~ɑixu] \tab  danken \tab  ‘thank-\textsc{inf}’ \tab  94
    \z
\z 

[x]{\textasciitilde}[ç] alternations in postsonorant position are presented in (\ref{ex:6:28a}--\ref{ex:6:28e}). The two stems in \REF{ex:6:28a} are lexically listed because the vowels are not related by a regular synchronic process, i.e. /ræxt/, /rixt-ig/. \isi{Umlaut} alternations in (\ref{ex:6:28b}--\ref{ex:6:28e}) reflect the final two patterns in \REF{ex:6:3c}. The pair in \REF{ex:6:28f} exhibits [o]{\textasciitilde}[e] \isi{Umlaut} alternations (=\ref{ex:6:3a}), but velar [xx] stays velar [xx] after [e] because that vowel is not [+high].


\TabPositions{.1\textwidth, .25\textwidth, .55\textwidth, .8\textwidth}
\ea 
 \label{ex:6:28}Dorsal fricatives (from /x/) after front vowels:
\ea\label{ex:6:28a} ræxt   \tab [ræxt]   \tab recht      \tab ‘right’   \tab 29 \\
    riχtiɡ \tab [riçtig] \tab richtig    \tab ‘correct’ \tab 29 \\
\ex\label{ex:6:28b} büox   \tab [byox]   \tab Buch       \tab ‘book’    \tab 40 \\
    biǝχer \tab [biæçer] \tab Bücher     \tab ‘book-\textsc{pl}’   \tab 40 \\
\ex\label{ex:6:28c} tüox   \tab [tyox]   \tab Tuch       \tab ‘towel’   \tab 171\\
    tiǝχji \tab [tiæçji] \tab Tüchlein   \tab ‘towel-\textsc{dim}’\tab  171 \\
\ex\label{ex:6:28d} brüox  \tab [bryox]  \tab Pferdegeschirr        \tab ‘horse harness’   \tab  94 \\
    briǝχ  \tab [briæç]  \tab Pferdgeschirr, pl.    \tab ‘horse harness-\textsc{pl}’ \tab  94 \\
\ex\label{ex:6:28e} bɑ͔ix  \tab  [b\~ɑix]  \tab  Bank                 \tab  ‘bench’          \tab   94\\
    be͔iχ  \tab  [blẽiç]  \tab  Bänke                \tab  ‘bench-\textsc{pl}’      \tab   94\\
    xlɑ͔ix \tab  [xl\~ɑix] \tab  (Glocken-)klang      \tab  ‘sound of bell’  \tab   94\\
    xle͔iχ \tab  [xlẽiç] \tab  (Glocken-)klang, pl. \tab  ‘sound-\textsc{pl} of bell’ \tab   94\\
\ex\label{ex:6:28f} loxx   \tab [loxx]   \tab Loch                  \tab ‘hole’            \tab 124 \\
    lexxer \tab [lexxer] \tab Löcher                \tab ‘hole-\textsc{pl}’           \tab  33 \\
    \z
\z 

The most significant examples above involve the occurrence of the opaque velar fricative after the [i] component of [\~ɑi] and the transparent palatal after the [i] component of [ẽi] in \REF{ex:6:28e}; recall (\ref{ex:6:10c}, \ref{ex:6:10d}).

The examples in \REF{ex:6:29} demonstrate that palatal [kç] surfaces after a high front vowel, while the data in \REF{ex:6:30} show that the velar [kx] occurs after all other sounds (=\ref{ex:6:17g}). The separate categories in (\ref{ex:6:29}--\ref{ex:6:30}) indicate the individual vowels. No examples were found in the original source with dorsal affricates after neutral vowels.


\TabPositions{.2\textwidth, .35\textwidth, .55\textwidth, .8\textwidth}
\ea%29
\TabPositions{.125\textwidth, .275\textwidth, .4\textwidth, .6\textwidth}
\label{ex:6:29}Postvocalic palatal [kç] (from /kx/):
\ea\label{ex:6:29a} dikχ    \tab [dikç]    \tab dick     \tab ‘fat’                 \tab 96
\ex\label{ex:6:29b} bleikχu \tab [bleikçu] \tab bleichen \tab ‘bleach-\textsc{inf}’ \tab 96
\z
\ex%30
\label{ex:6:30}Postvocalic velar [kx] (from /kx/):
\ea\label{ex:6:30a}  štukx \tab [ʃtukx] \tab Stück   \tab ‘piece’              \tab 96\\
\ex\label{ex:6:30b}  bokx  \tab [bokx]  \tab Bock    \tab ‘buck’               \tab 96\\
\ex\label{ex:6:30c}  sɑkx  \tab [sɑkx]  \tab Sack    \tab ‘sack’               \tab 96\\
\ex\label{ex:6:30d}  dekxu \tab [dekxu] \tab decken  \tab ‘cover-\textsc{inf}’ \tab 96\\
\ex\label{ex:6:30e}  rɛ̄kx  \tab [rɛːkx] \tab bitter  \tab ‘bitter’             \tab 96\\
\ex\label{ex:6:30f}  bækxu \tab [bækxu] \tab picken  \tab ‘peck-\textsc{inf}’  \tab 96\\
  \z
\z 

\ipi{Visperterminen} also has words containing [k] after a high front vowel, which show that [k] has no palatal realization, e.g. [rik] ‘back’ \citep[98]{Wipf1910}.

The examples in (\ref{ex:6:26}--\ref{ex:6:30}) reveal that velars and the corresponding palatals do not contrast after a vowel. The palatals are derived from velars by \REF{ex:6:31}:

\ea%31
\label{ex:6:31}{Velar Fronting-6}:\\

\begin{forest}
    [,phantom
       [\avm{[+high]} [\avm{[coronal]},tier=word,name=target] ]
       [\avm{[−son\\+cont]},name=source  [\avm{[peripheral]},tier=word] ]
    ]
    \draw [dashed] (source.south) -- (target.north);
\end{forest}

% \begin{forest}
%   [,phantom
%     [\avm{[+high]} [\avm{[coronal]},name=target]]
%     [\avm{[−son]\\+cont},name=source [\avm{[peripheral]}]]
%   ]
%   \draw [dashed] (source.south) -- (target.north);
% \end{forest}
\z 

\isi{Velar Fronting-6} does not apply after /øi/ and /\~ɑi/ (=\ref{ex:6:27i}, \ref{ex:6:27j}) because the /i/ component of both diphthongs is a neutral sound and therefore lacks [coronal]; recall (\ref{ex:6:10a}, \ref{ex:6:10c}). By contrast, spreading occurs after /ei/ (=\ref{ex:6:26d}) and /ẽi/ (=\ref{ex:6:26g}) because the /i/ in those diphthongs are [coronal]; recall (\ref{ex:6:10b}, \ref{ex:6:10d}).

Example \REF{ex:6:26h} illustrates that \isi{Velar Fronting-6} creates palatals after /iæ/. This is possible because that diphthong is [coronal]; recall \REF{ex:6:11b}. The spreading of [coronal] in /iæ/ occurs as expected: /liæxt/→[liæçt].

The data in this section support predictions (\ref{ex:6:14a}--\ref{ex:6:14c}). \REF{ex:6:14a} is correct because velars and not palatals occur after /\~ɑi/ and /øi/, and \REF{ex:6:14c} is substantiated because palatals and not velars surface after /iæ/. The data from word-initial position support \REF{ex:6:14b} because velars and not palatals occur in that position before [yo]. Since \isi{Velar Fronting-6} and \isi{Wd-Initial Velar Fronting-4} are both triggered by high front vowels, neither \REF{ex:6:14d} nor \REF{ex:6:14e} can be (dis)confirmed.

I conclude this section by considering the distribution of the dorsal fricatives and affricates after a consonant. Unlike all of the dialects discussed in the preceding chapters, velars ([x]/[kx]) and palatals ([ç]/[kç]) both occur after a (liquid) consonant; there are no dorsal fricatives or affricates before [n] because nasals deleted in that context by a historical process \citep[44--45]{Wipf1910}. The relevant generalization is that the place of articulation of the dorsal sound is determined by the vowel immediately preceding the liquid (=\ref{ex:6:17e}, \ref{ex:6:17f}). In \REF{ex:6:32} I show that [ç] occurs after a liquid if the immediately preceding vowel is high and front. The palatal fricative can be either word-final or word-internal before a vowel. In \REF{ex:6:32a} the liquid in question is [l] and in \REF{ex:6:32b} it is [r]. In all of the examples presented in \REF{ex:6:32} the high front vowel preceding the liquid is [i]. The absence of examples with [yː] in that context can be attributed to the lack of \ili{OHG} words with the cognate vowel [uː] followed by a liquid plus dorsal fricative (\fnref{fn:6:2}). I speculate that there are similar historical reasons accounting for the lack of words with [iː] or any of the high front nasalized vowels followed by a sequence of liquid plus dorsal fricative.

\ea%32
\label{ex:6:32}Postconsonantal [ç] (from /x/):
\ea\label{ex:6:32a} χilχɑ \tab       [çilçɑ]  \tab Kirche   \tab ‘church’            \tab 94\\
    milχ        \tab [milç]   \tab Milch    \tab ‘milk’              \tab 94\\
\ex\label{ex:6:32b} firχtu      \tab [firçtu] \tab fürchten \tab ‘fear-\textsc{inf}’ \tab 42\\
    birχɑ \tab       [birçɑ]  \tab Birke    \tab ‘birch’             \tab 42\\
   \z
\z 

The examples in \REF{ex:6:33} indicate that velar [x] surfaces after a liquid if the preceding vowel is either back or nonhigh and front. The liquid is [l] in \REF{ex:6:33a} and [r] in \REF{ex:6:33b}.

\ea%33
\label{ex:6:33}Postconsonantal [x] (from /x/):
\ea\label{ex:6:33a} wulxɑ \tab [wulxɑ] \tab Wolke  \tab ‘cloud’ \tab 94\\
    xɑlx  \tab [xɑlx]  \tab Kalk   \tab ‘lime’  \tab 94\\
    mælxu \tab [mælxu] \tab melken \tab  ‘milk-\textsc{inf}’ \tab 94\\
\ex\label{ex:6:33b} sɑ̄rx      \tab [sɑːrx] \tab Sarg    \tab ‘coffin’ \tab 94\\
    lerx      \tab [lerx]  \tab Lärche  \tab ‘larch’  \tab 94\\
    w\={æ}rx  \tab [wæːrx] \tab Werk    \tab ‘work’   \tab 94\\
    \z
\z 

Dorsal affricates have an identical distribution to the corresponding fricatives. Two representative examples given in \REF{ex:6:34}.

\ea%34
\label{ex:6:34}Postconsonantal dorsal affricates (from /kx/):
\ea\label{ex:6:34a} wirkχu \tab [wirkçu] \tab wirken \tab ‘seem\textsc{{}-inf}’ \tab 96
\ex\label{ex:6:34b} merkxu \tab [merkxu] \tab merken \tab ‘notice-\textsc{inf}’ \tab 96
\z
\z 

I argue that front vowel plus liquid sequences undergo the \isi{OCP}-motivated change in \REF{ex:6:35}, which merges the two [coronal] features into one. I assume that the first vowel in \REF{ex:6:35} is not restricted to [+high] sounds; there are no examples suggesting that the change does or does not occur after a nonhigh vowel. Since there are no nasal consonants that can potentially undergo \isi{Coalescence-1}, I omit [--nasal] from the second segment in \REF{ex:6:35}. \isi{Coalescence-1} has a function similar to the two \isi{schwa} fronting changes posited in \sectref{sec:3.4} (for \ipi{Rheintal}) and in \sectref{sec:5.4} (for a number of HG varieties). See also \sectref{sec:12.8.1} for further discussion.

\ea%35
\label{ex:6:35}\isi{Coalescence-1}:\\
\begin{forest} for tree={grow'=north}
  [,phantom
    [\avm{[coronal]} [\avm{[−cons]},tier=word]]
    [\avm{[coronal]} [\avm{[+cons\\+son]},tier=word]]
    [→,tier=word]
    [\avm{[coronal]} [\avm{[−cons]}] [\avm{[+cons\\+son]}]]
  ]
\end{forest}
\z 

In examples like [milç] ‘milk’ in \REF{ex:6:32a}, \isi{Coalescence-1} \isi{feeds} \isi{Velar Fronting-6}: /milx/→[milç]. By contrast, \isi{Coalescence-1} does not affect the /ul/ sequence in examples like [wulxɑ] ‘cloud’ in \REF{ex:6:33a}; hence, \isi{Velar Fronting-6} does not apply: /wulxɑ/→[wulxɑ].

\section{Highest Alemannic (part 2)}\label{sec:6.3}

\citet{Brun1918} describes a \il{Highest Alemannic}HstAlmc dialect spoken in the community (Gemeinde) of \ipi{Obersaxen} (now known as \ipi{Obersaxen} Mundaun) in the Swiss canton of Grisons (Graubünden); see \mapref{map:2}.

\ipi{Obersaxen} is an area in Switzerland settled by people originally from the canton of Valais during the \is{Walser migration}\textsc{Walser migrations} (Walserwanderungen); see \citet{Bohnenberger1913} and \citet[904]{Wiesinger1983b}. Hence, the dialect described by \citet{Brun1918} is one variety of \is{Walser German}\textsc{Walser German} (Walderdeutsch). \ipi{Obersaxen} is a unique dialect because it is a German-language island \citep{Wiesinger1983b} completely surrounded by areas in which a \ili{Romance} language is the dominant tongue (\ili{Romansh}). See \sectref{sec:15.6} for further discussion.

In his discussion of the sounds of \isi{Walser German}, \citet[173]{Bohnenberger1913} observes that /kx/ and /x/ are realized as palatal depending on the nature of the preceding vowel. It is tempting to interpret Bohnenberger’s observation as evidence that \isi{Walser German} as a whole is characterized by velar fronting. The problem with this interpretation is twofold. First, not all varieties of \isi{Walser German} have velar fronting (e.g. \ipi{Schanfigg}; \citealt{Kessler1931}; \mapref{map:2}). Second, varieties of \isi{Walser German} with velar fronting do not have the same rule (see \chapref{sec:15} for discussion).

Although \ipi{Obersaxen} is shown below to possess a \isi{neutral vowel} and is hence structurally similar to \ipi{Visperterminen} (\citealt{Wipf1910}; \sectref{sec:6.2}), it needs to be stressed that the two SwG varieties are spoken in different cantons and that they are therefore separated by conservative non-velar fronting varieties. Neutral vowels in \ipi{Visperterminen} and \ipi{Obersaxen} therefore developed independently.

I consider first the phonetics/phonology of the vowels (\sectref{sec:6.3.1}) and then the patterning of dorsal fricatives and affricates (\sectref{sec:6.3.2}).

\subsection{Phonetics and phonology of vowels}\label{sec:6.3.1}

\ipi{Obersaxen} possesses front vowels (/i y yː e eː æ æː/), back vowels (/u o oː ɑ ɑː ə/), and six diphthongs (/æʊ ʊæ ʏu æɪ ɪæ ɪi/).\footnote{{Three surface monophthongs are ignored, namely [ɪ ø ɛ].  [ø ɛ] only occur rarely (\citealt{Brun1918}: 45, 67) and apparently never in the context of a dorsal fricative or \isi{affricate}. [ɪ] is a stressless allophone of /i/. Two diphthongs are not considered below ([ɪə ʊə]) because they do not occur in the neighborhood of dorsal fricatives or affricates.} } The diphthongs are placed into two categories based on how they behave with respect to \isi{Umlaut}: /æʊ ʊæ ʏu/ bear [peripheral] and /æɪ ɪæ ɪi/ [coronal]; see below for representations. The most important diphthong for present purposes /ʏu/, whose phonetically front component /ʏ/ is shown below to be a \isi{neutral vowel}, cf. the equivalent diphthong in \ipi{Visperterminen} /yo/.

Vocalic alternations involving \isi{Umlaut} are essentially the same as in \ipi{Visperterminen}: Back monophthongs alternate with the corresponding front unrounded monophthongs in (\ref{ex:6:36a}); front rounded monophthongs surface in the context of \isi{Umlaut} as the corresponding front unrounded monophthongs in (\ref{ex:6:36b}). Diphthongs exhibit the pattern of alternation in \REF{ex:6:36c}.

\ea%36
    \label{ex:6:36}
\ea\label{ex:6:36a} \relax [u]{\textasciitilde}[i]  \\  
    \relax [o]{\textasciitilde}[e]  \\  
    \relax [oː]{\textasciitilde}[eː]\\
    \relax [ɑ]{\textasciitilde}[æ]  \\
    \relax [ɑː]{\textasciitilde}[æː]\\
\ex\label{ex:6:36b} \relax [yː]{\textasciitilde}[iː]\\
    \relax [y]{\textasciitilde}[i]  \\
\ex\label{ex:6:36c} \relax [ʊæ]{\textasciitilde}[ɪæ]\\
    \relax [æʊ]{\textasciitilde}[æɪ]\\
    \relax [ʏu]{\textasciitilde}[ɪi]\\
\z 
\z 

The three patterns in \REF{ex:6:36} are illustrated in (\ref{ex:6:37a}--\ref{ex:6:37c}). The morphological contexts for \isi{Umlaut} in these examples are the comparative or superlative of adjectives, the plural of nouns and the derivational suffixes [-ər] and [-lɪçç]. Synchronic alternations involving the pair [ʏu]{\textasciitilde}[ɪi] in \REF{ex:6:36c} are difficult to come by; a crucial example involving that pair of diphthongs as it interacts with the distribution of dorsal fricatives is discussed in \sectref{sec:6.3.2}.


\TabPositions{.2\textwidth, .4\textwidth, .6\textwidth, .8\textwidth}
\ea%37
    \label{ex:6:37}
\ea\label{ex:6:37a}  ksunt      \tab [ksunt]       \tab gesund      \tab ‘healthy’              \tab  61\\
     ksintər    \tab [ksindər]     \tab gesünder    \tab ‘healthier’            \tab  61\\
     ɡrop       \tab [grop]        \tab grob        \tab ‘rough’                \tab  61\\
     ɡrebər     \tab [grebər]      \tab gröber      \tab ‘rougher’              \tab  61\\
     ɡrōss      \tab [groːss]      \tab groß        \tab ‘large’                \tab 160\\
     ɡrēšt      \tab [greːʃt]      \tab größt-      \tab ‘largest’              \tab 160\\
     štɑrxx     \tab [ʃtɑrxx]      \tab stark       \tab ‘strong’               \tab 160\\
     štærxšt    \tab [ʃtærxʃt]     \tab stärkst-    \tab ‘strongest’            \tab 160\\
     nɑ̄t        \tab [nɑːt]        \tab Naht        \tab ‘seam’                 \tab  57\\
     n\={æ}tlɩ  \tab [næːtlɪ]      \tab Naht, dim   \tab ‘seam-\textsc{dim}’    \tab  57\\
\ex\label{ex:6:37b}  fǖšt       \tab [fyːʃt]       \tab Faust       \tab ‘fist’                 \tab 155\\
    fīšt        \tab [fiːʃt]       \tab Fäuste      \tab ‘fist-\textsc{pl}’                \tab 155\\
    hüt         \tab [hyt]         \tab Haut        \tab ‘skin’                 \tab  75\\
    hittæ       \tab [hittæ]       \tab häuten      \tab ‘skin-\textsc{inf}’    \tab  75\\
\ex\label{ex:6:37c}  šuæl       \tab [ʃʊæl]        \tab Schule      \tab ‘school’               \tab  55\\
    šiælər      \tab [ʃɪælər]      \tab Schüler     \tab ‘student’              \tab  66\\
    ɡlæubæ      \tab [glæʊbæ]      \tab glauben     \tab ‘believe-\textsc{inf}’ \tab  81\\
    uŋklæɩplɩχχ \tab [uŋklæɪplɪçç] \tab unglaublich \tab ‘unbelievable’         \tab  63\\
\z 
\z 

Front unrounded monophthongs are \isi{nonneutral} and hence [coronal]; see \REF{ex:6:7a}, while back monophthongs are [peripheral]; see \REF{ex:6:7c}. The correct representation for front rounded monophthongs is \REF{ex:6:7b}. Individual monophthongs are assigned distinctive features, as indicated in \tabref{tab:fromex:38}. In contrast to \ipi{Visperterminen} (=~\tabref{ex:6:8}), [low] must be assigned before [high] so that high and mid front vowels all bear the feature [--low]. [peripheral] is assigned twice, depending on whether or not it corresponds to the backness or the roundedness dimension.

\begin{table}%38
\caption{\label{tab:fromex:38}Distinctive features for vowels (Obersaxen)}
\begin{tabular}{lccccccc}
\lsptoprule
       & i & yː y & eː e & æː æ & u & oː o & ɑː ɑ\\\midrule
\relax [coronal] & \ding{51} & \ding{51} & \ding{51} & \ding{51} &  &  & \\
\relax [peripheral] &  &  &  &  & \ding{51} & \ding{51} & \ding{51}\\
\relax [low] & − & − & − & + & − & − & +\\
\relax [high] & + & + & − & − & + & − & −\\
\relax [peripheral] &  & \ding{51} &  &  &  &  & \\
\lspbottomrule
\end{tabular}
\end{table}


The featural representations for the six diphthongs are presented in \REF{ex:6:39}. Note that both components of those diphthongs bear either a positive or negative specification of the feature [high]. The feature [high], together with the place features [coronal] and [peripheral], suffices to make all six diphthongs distinct. For that reason, the feature [±low] is redundant, as is [±tense]. The fact that certain components of the diphthongs are phonetically lax and others are phonetically tense is captured in the phonetics and not in the phonology.

\ea%39
\label{ex:6:39}
\parbox[t]{.45\textwidth}{
a. /æʊ/\\
\begin{forest}
[,phantom
    [\avm{[--cons\\--high]} [\avm{[peripheral]},name=target]]
    [\avm{[--cons\\+high]}, name=source]
]
\draw (source.south) -- (target.north);
\end{forest}
}
\parbox[t]{.45\textwidth}{
b. /æɪ/\\
\begin{forest}
[,phantom
    [\avm{[--cons\\--high]} [\avm{[coronal]},name=target]]
    [\avm{[--cons\\+high]}, name=source]
]
\draw (source.south) -- (target.north);
\end{forest}
}

\parbox[t]{.45\textwidth}{
c. /ʊæ/\\
\begin{forest}
[,phantom
    [\avm{[--cons\\+high]} [\avm{[peripheral]},name=target]]
    [\avm{[--cons\\--high]}, name=source]
]
\draw (source.south) -- (target.north);
\end{forest}
}
\parbox[t]{.45\textwidth}{
d. /ɪæ/\\
\begin{forest}
[,phantom
    [\avm{[--cons\\+high]} [\avm{[coronal]},name=target]]
    [\avm{[--cons\\--high]}, name=source]
]
\draw (source.south) -- (target.north);
\end{forest}
}


\parbox[t]{.45\textwidth}{
e. /ʏu/\\
\begin{forest}
[,phantom
    [\avm{[--cons\\+high]}  [\avm{[peripheral]},name=target]]
    [\avm{[--cons\\+high]} , name=source]
]
\draw (source.south) -- (target.north);
\end{forest}
}
\parbox[t]{.45\textwidth}{
f. /ɪi/\\
\begin{forest}
[,phantom
    [\avm{[--cons\\+high]} [\avm{[coronal]},name=target]]
    [\avm{[--cons\\+high]} , name=source]
]
\draw (source.south) -- (target.north);
\end{forest}
}
\z

Given the structures for monophthongs in \REF{ex:6:7} and diphthongs in \REF{ex:6:39}, \isi{Umlaut} is expressed as in \REF{ex:6:12}.

\subsection{Dorsal fricatives and affricates}\label{sec:6.3.2}

\ipi{Obersaxen} has two singleton dorsal fricatives, namely [x] (=⟦x⟧) and [ç] (=⟦χ⟧), which also have geminate counterparts [xx] (=⟦xx⟧) and [çç] (=⟦χχ⟧). In contrast to \ipi{Visperterminen}, geminates can occur in \ipi{Obersaxen} after a long vowel. The basic facts involving the distribution of dorsal fricatives and affricates in \ipi{Obersaxen} are very similar -- but not identical -- to the facts for \ipi{Visperterminen}. The reader is referred to the detailed discussion in the original source (\citealt{Brun1918}: 113--118; 121--122). The relationship between the velars and corresponding palatals is depicted in (\ref{ex:6:15}) for word-initial and postsonorant position.

The distribution of the velar and palatal sounds in question is summarized in \REF{ex:6:40} and \REF{ex:6:41}:

\ea%40
\label{ex:6:40}[x]/[kx] and [ç]/[kç] in word-initial onsets:
\ea\label{ex:6:40a}\relax [ç] occurs word-initially only before nonlow front vowels but not before [ʏu];
\ex\label{ex:6:40b}\relax [x] occurs word-initially in the elsewhere case (also before [ʏu]);
\ex\label{ex:6:40c}\relax [kç] and [kx] have the same distribution as word-initial [x]/[ç].
\z
\ex%41
\label{ex:6:41}[x]/[kx] and [ç]/[kç] after a sonorant:
\ea\label{ex:6:41a}\relax [ç] occurs after a nonlow front vowel;
\ex\label{ex:6:41b}\relax [x] occurs after other vowels (including [ɪæ]);
\ex\label{ex:6:41c}\relax [ç] occurs after a nonlow front vowel followed by a liquid;
\ex\label{ex:6:41d}\relax [x] occurs after any other vowel followed by a liquid;
\ex\label{ex:6:41e}\relax [kç] and [kx] have the same distribution as [x]/[ç].
\z
\z 

There are two crucial differences between \ipi{Obersaxen} and \ipi{Visperterminen}: First, in \ipi{Visperterminen} palatals occur in the neighborhood of a [+high] coronal, but in \ipi{Obersaxen} palatals surface when adjacent to a [--low] sound. Second, in \ipi{Obersaxen}, [x] surfaces after the diphthong [ɪæ], but in \ipi{Visperterminen}, palatal [ç] surfaces after the equivalent diphthong ([iæ]).

 In word-initial position, [ç] occurs before a nonlow front vowel (=\ref{ex:6:40a}). The vowel referred to here can be [i] in (\ref{ex:6:42a}), [yː] in (\ref{ex:6:42b}), [e] in (\ref{ex:6:42c}), [eː] in (\ref{ex:6:42d}), or [ɪæ] in \REF{ex:6:42e}. No examples with a word-initial dorsal fricative were found in \citet{Brun1918} in which the vowel following that fricative is [y] or [ɪi] -- gaps I interpret as accidental. The dorsal fricatives and affricates in \REF{ex:6:42} and subsequent datasets derive historically from \ili{WGmc} \textsuperscript{+}[k] or \textsuperscript{+}[x].


\TabPositions{.15\textwidth, .3\textwidth, .5\textwidth, .8\textwidth}
\ea%42
Word-initial [ç] (from /x/):\label{ex:6:42}
   \ea\label{ex:6:42a} χint     \tab [çint]     \tab Kind     \tab ‘child’            \tab 113
   \ex\label{ex:6:42b} χǖχχlæ   \tab [çyːççlæ]  \tab Kunkel   \tab ‘explosive pellet’ \tab 113
   \ex\label{ex:6:42c} χeɡəl    \tab [çegəl]    \tab Kegel    \tab ‘pin’              \tab 113
   \ex\label{ex:6:42d} χēl      \tab [çeːl]     \tab Kohl     \tab ‘cabbage’          \tab  47
   \ex\label{ex:6:42e} χɩæholts \tab [çɪæholts] \tab Kienholz \tab ‘resinous wood’    \tab  54
  \z
\z 


As shown in \REF{ex:6:43}, before any other segment, the word-initial dorsal fricative surfaces as [x] (=\ref{ex:6:40b}). Thus, word-initial [x] occurs before a back vowel in (\ref{ex:6:43a}), a nonhigh front vowel in the diphthongs [æu] and [æi] in (\ref{ex:6:43b}), or the diphthong [ʏu] in (\ref{ex:6:43c}). The latter example is crucial because [ʏ] is a high front vowel and, as such, would be expected to pattern like the examples in \REF{ex:6:42}. Thus, a surface sequence of velar followed by [ʏu] exemplifies the \isi{underapplication} of velar fronting.

\ea%43
\label{ex:6:43}Word-initial [x] (from /x/):
\ea\label{ex:6:43a}  xunšt  \tab [xunʃt]  \tab Kunst   \tab ‘art’  \tab 113\\
     xopf   \tab [xopf]   \tab Kopf    \tab ‘head’ \tab 113\\
     xɑ̄lt   \tab [xɑːlt]  \tab kalt    \tab ‘cold’ \tab 61
\ex\label{ex:6:43b}  xæuwæ  \tab [xæʊwæ]  \tab kauen   \tab ‘chew-\textsc{inf}’ \tab 113\\
     xæisər \tab [xæɪsər] \tab  Kaiser \tab ‘emperor’           \tab 113
\ex\label{ex:6:43c}  xüuwæ  \tab [xʏuwæ]  \tab Kuh     \tab ‘cow’               \tab 113
\z 
\z 


In \citegen[113]{Brun1918} description of the distribution of word-initial [x] and [ç], he writes that the former sound occurs before the vowels [ɑ o u æ æɪ æʊ ʏu] and the palatal before [i y e ɪæ]. (“Velare Spirans x......vor den Vokalen \textit{ɑ o u æ æi æu} und \textit{üu}; Palatale χ......vor den Palatalvokalen \textit{i u ɩæ e ǖ}ˮ). Note in particular that Brun classifies the front part of the diphthong [ʏu] with the back vowels and the nonlow front vowels.

The examples in \REF{ex:6:44} indicate that [x] -- but not [ç] -- occurs in word-initial position before a coronal sonorant consonant, which can be [n] in (\ref{ex:6:44a}), [l] in (\ref{ex:6:44b}), or [r] in (\ref{ex:6:44c}). The second example in \REF{ex:6:44a} illustrates that the realization of the word-initial dorsal is not determined by the vowel following /r/.


\TabPositions{.2\textwidth, .4\textwidth, .55\textwidth, .8\textwidth}
\ea%44
\label{ex:6:44}Word-initial [x] (from /x/):
\ea\label{ex:6:44a}  xnæu     \tab [xnæu]     \tab  Knie      \tab ‘knee’                \tab 113\\
     xnæxt    \tab [xnæxt]    \tab  Knecht    \tab ‘vassal’              \tab 34 \\
\ex\label{ex:6:44b}  xrɑnts   \tab [xrɑnts]   \tab  Kranz     \tab ‘wreath’              \tab 113\\
     xrits    \tab [xrits]    \tab  Kreuz     \tab ‘cross’               \tab 113\\
\ex\label{ex:6:44c}  xlɑ̄r     \tab [xlɑːr]    \tab  klar      \tab ‘clear’               \tab 113\\
     xlæppæræ \tab [xlæppæræ] \tab  klappern  \tab ‘rattle-\textsc{inf}’ \tab 113\\
\z 
\z 


The Umlaut alternations in \REF{ex:6:45a} trigger a change from velar [x] to palatal [ç] before a nonlow front vowel. The same vocalic change occurs in the pair in \REF{ex:6:45b}. Note that the diphthong in the singular noun is the \isi{neutral vowel} [ʏu], which is preceded by a surface velar [x]. The fronted counterpart of that \isi{neutral vowel} is [ɪi] in the plural noun, which is preceded by a surface palatal [ç] because [ɪi] is a \isi{nonneutral vowel}.

\ea%45
Dorsal fricatives (from /x/) before Umlaut alternations:\label{ex:6:45}
\ea\label{ex:6:45a}  xɑ̄lt   \tab [xɑːlt]  \tab kalt   \tab ‘cold’   \tab  61\\
     χeltər \tab [çeltər] \tab kälter \tab ‘colder’ \tab  61\\
\ex\label{ex:6:45b}  xüuwæ  \tab [xʏuwæ]  \tab Kuh    \tab ‘cow’    \tab 155\\
     χɩijæ  \tab [çɪijæ]  \tab Kühe   \tab ‘cow-\textsc{pl}’   \tab 155\\
\z 
\z 

Word-initial velar and palatal affricates showing the same distribution as the corresponding fricatives are presented in \REF{ex:6:46}; recall \REF{ex:6:40c}. \citet[113]{Brun1918} is clear in that the distribution of word-initial dorsal affricates is the same as the distribution of the corresponding fricatives.

\ea%46
\label{ex:6:46}Word-initial affricates (from /kx/):
\ea\label{ex:6:46a}  kχits   \tab [kçits]   \tab Werg     \tab ‘oakum’  \tab  114
\ex\label{ex:6:46b}  kxuntæ  \tab [kxuntæ]  \tab Rechnung \tab ‘bill’   \tab   38\\
     kxæuffæ \tab [kxæʊffæ] \tab kaufen   \tab ‘buy-\textsc{inf}’ \tab 38
\z 
\z 

In word-initial position velars and palatals do not contrast. As indicated above, I analyze the underlying sound as a velar (/x/ or /kx/), which shifts to the corresponding palatal before a [--low] vowel by \REF{ex:6:47}. No native words begin with [k], although a small number of apparently integrated loanwords have [k] in that context, e.g. [kiŋklæ] ‘rabbit’. Since word-initial [k] is not realized as palatal before nonlow front vowels, the set of targets for \REF{ex:6:47} consists of fricatives and affricates only (=\ref{ex:6:24a}).

\ea%47
\label{ex:6:47}\isi{Wd-Initial Velar Fronting-5}:

\begin{forest}
[,phantom
    [\avm{[−son\\+cont]},name=source [\avm{[peripheral]},tier=word]]
    [\avm{[−low]}  [\avm{[coronal]}, tier=word, name=target]]
]
\node [left=1ex of source] {\textsubscript{wd} [}; 
\draw [dashed] (source.south) -- (target.north);      
\end{forest}
\z 

\isi{Wd-Initial Velar Fronting-5} fails to spread [coronal] from a consonant (/r l n/) to a preceding /x/ because [±low] is not distinctive for consonants. Hence, word-initial /x/ in \REF{ex:6:44} surfaces without change as [x].

The distribution of velar and palatal fricatives after a vowel (=\ref{ex:6:41a}--d) is shown in \REF{ex:6:48}: Palatals surface after a nonlow front vowel in (\ref{ex:6:48a}), while velars occur after a low front vowel in (\ref{ex:6:48b}) or a back vowel in (\ref{ex:6:48c}). The examples in \REF{ex:6:48d} exhibit the occurrence of velar fricatives after the diphthong [ɪæ].\footnote{\label{fn:6:10}The discussion in \citet[114]{Brun1918} is clear that palatals only surface after the vowels I analyze as nonlow. In the context of that discussion the author notes a complication: If a dorsal fricative occurs between a low front vowel and /ɪ/, then the fricative in question is fronted, e.g. the /xx/ in the word /ʃtæxxɪk/ ‘malicious’. I do not take that type of example into consideration below because I see the fronted articulation of /xx/ as the product of a coarticulatory fronting and not of a discreet phonological process. Brun himself notes that the fronted dorsal fricative in words like /ʃtæxxɪk/ is articulatorily between velar and palatal.}

\TabPositions{.2\textwidth, .4\textwidth, .55\textwidth, .8\textwidth}

\ea%48
\label{ex:6:48}Postvocalic dorsal fricatives (from /x/):
\ea\label{ex:6:48a} rīχχ       \tab [riːçç]      \tab  reich                     \tab ‘rich’                         \tab 46 \\
    ksiχt      \tab [ksiçt]      \tab  Gesicht                   \tab ‘face’                         \tab 121\\
    r\={æ}tɩχ  \tab [ræːtɪç]     \tab  Rettig                    \tab (unclear gloss)                \tab 44 \\
    χeχχ       \tab [çeçç]       \tab  Köche                     \tab ‘cook-\textsc{pl}’                        \tab 116\\
    sēχtæ      \tab [seːçtæ]     \tab  Wäsche in die Lauge legen \\ \tab \tab  ‘put-\textsc{inf} wash in lye’ \tab 45 \\
\ex\label{ex:6:48b} fæxtæ      \tab [fæxtæ]      \tab  fechten                   \tab ‘fence-\textsc{inf}’           \tab 121\\
\ex\label{ex:6:48c} bruxx      \tab [bruxx]      \tab  Bruch                     \tab ‘fracture’                     \tab  39\\
    loxx       \tab [loxx]       \tab  Loch                      \tab ‘hole’                         \tab  37\\
    bɑxx       \tab [bɑxx]       \tab  Bach                      \tab ‘stream’                       \tab 116\\
    dɑ̄x        \tab [dɑːx]       \tab  Docht                     \tab ‘wick’                         \tab  43\\
    ræuxx      \tab [ræuxx]      \tab  Rauch                     \tab ‘smoke’                        \tab 116\\
\ex\label{ex:6:48d} sɩæxx      \tab [sɪæxx]      \tab  krank                     \tab ‘sick’                         \tab  54\\
    ərnɩæxtæræ \tab [ərnɪæxtæræ] \tab  Schnapps                  \tab ‘kind of Schnapps’             \tab  55\\
\z 
\z


Note in particular the data in \REF{ex:6:48d}: [ɪæ] is followed by a velar fricative in contrast to the data from \REF{ex:6:26h} indicating that a palatal fricative follows [iæ] in \ipi{Visperterminen}.

The \isi{Umlaut} alternations in \REF{ex:6:49} indicate that [x] surfaces after a back vowel in the singular but that [ç] occurs after the fronted (nonlow) vowel in the plural.

\ea%49
\label{ex:6:49}Dorsal fricatives (from /x/) after fronted vowels:
\ea\label{ex:6:49a}  fruxt \tab [fruxt] \tab Frucht   \tab ‘fruit’  \tab 155
\ex\label{ex:6:49b}  friχt \tab [friçt] \tab Früchte  \tab ‘fruit-\textsc{pl}’ \tab 155
\z 
\z

The examples in \REF{ex:6:50} show that velar and palatal affricates have a parallel distribution to the corresponding fricatives (=\ref{ex:6:41e}). Thus, palatal [kç] surfaces after a [--low] front vowel in (\ref{ex:6:50a}) and velar [kx] after any other vowel in (\ref{ex:6:50b}).

\ea%50
\label{ex:6:50}Postvocalic dorsal affricates (from /kx/):
\ea\label{ex:6:50a}  glikχ \tab [glikç] \tab Glück \tab ‘fortune’ \tab 116\\
\ex\label{ex:6:50b}  štukx \tab [ʃtukx] \tab Stück \tab ‘piece’   \tab 42 \\
     špækx \tab [ʃtækx] \tab Speck \tab ‘bacon’   \tab 116\\
\z 
\z 


Postvocalic velars and palatals are derived from /x/ or /kx/ after a [--low] front vowel by \REF{ex:6:51}. As in word-initial position, the target for postsonorant fronting in \REF{ex:6:51} does not include /k/, e.g. [ek] ‘corner’.

\ea%51
\label{ex:6:51}\isi{Velar Fronting-7}:\\
\begin{forest}
    [,phantom
       [\avm{[−low]} [\avm{[coronal]},tier=word,name=target] ]
       [\avm{[−son\\+cont]},name=source  [\avm{[peripheral]},tier=word] ]
    ]
    \draw [dashed] (source.south) -- (target.north);
\end{forest}
\z 


The examples in \REF{ex:6:48d} indicate that the dorsal fricative surfaces as velar after the diphthong /ɪæ/, e.g. /sɪæxx/→[sɪæxx] ‘sick’. The reason [coronal] cannot spread from the diphthong /ɪæ/ to /x/ is that the trigger for fronting (\isi{Velar Fronting-7}) is [--low]. Recall from \REF{ex:6:39} that the two components of the six diphthongs are distinguished from one another with the positive or negative value of the feature [high] alone (together with [coronal] and or [peripheral]), but that [low] is not a distinctive feature for diphthongs.

I consider now the distribution of the dorsal fricatives after a consonant (=\ref{ex:6:41c}, \ref{ex:6:41d}). In \REF{ex:6:52} I show that the palatal [ç] occurs after a liquid if the immediately preceding vowel is nonlow and front. In \REF{ex:6:52a} the liquid in question is [l], and in \REF{ex:6:52b} it is [r].


\TabPositions{.1\textwidth, .25\textwidth, .5\textwidth, .75\textwidth}
\ea%52
\label{ex:6:52}Postliquid dorsal fricatives (from /x/):
\ea\label{ex:6:52a}  milχχ \tab [milçç] \tab Milch            \tab ‘milk’          \tab  37\\
     χelχχ \tab [çelçç] \tab Kelch            \tab ‘chalice’       \tab  32
\ex\label{ex:6:52b}  mælxx \tab [mælxx] \tab leicht zu melken \tab ‘easy to milk-\textsc{inf}’  \tab 116
\ex\label{ex:6:52c}  xɑlxx \tab [xɑlxx] \tab Kalk             \tab ‘lime’          \tab  37
\z 
\z 


Front vowel plus liquid sequences undergo \isi{Coalescence-1} (=\ref{ex:6:35}). In \REF{ex:6:52a} \isi{Velar Fronting-7} applies because the front vowel is [--low], e.g. /milxx/→[milçç] ‘milk’. Since the vowel preceding the liquid is not front in \REF{ex:6:52c} \isi{Coalescence-1} does not apply, and the dorsal fricative surfaces as velar, e.g. /xɑlxx/→[xɑlxx] ‘lime’. In \REF{ex:6:52b} the front vowel plus liquid sequence undergoes \isi{Coalescence-1}, but the dorsal fricative after the liquid fails to undergo \isi{Velar Fronting-7} because the front vowel does not bear the feature [--low], e.g. /mælxx/→[mælxx] ‘easy to milk-\textsc{pl}’.

\section{{Emergence} {of} {neutral} {vowels}}\label{sec:6.4}

As noted above, neutral vowels were historically back. The change from an original back sound to the neutral structure in \REF{ex:6:1b} exemplifies \isi{Vowel Fronting}, which requires the deletion of the feature characterizing back sounds ([peripheral]) but crucially not the addition of the front vowel feature ([coronal]). That type of change is depicted in \REF{ex:6:53a} in the context before a velar and in \REF{ex:6:53b} in the context after a word-initial velar. \isi{Vowel Fronting} – depicted here as /ɑ/ > /i/ – deleted the [peripheral] feature from the back sound. The significant point is that the frontness feature ([coronal]) was not added to the new front vowel /i/, which is the \isi{neutral vowel} represented in \REF{ex:6:1b}. \REF{ex:6:53} depicts both underlying and surface representations, which are the same.

\ea \label{ex:6:53}
  \ea \label{ex:6:53a}
  \begin{tikzpicture}[baseline={(matrix-1-1.base)}]
    \matrix (matrix) [matrix of nodes, nodes in empty cells]
      { /ɑ & x/ &[10mm] &[10mm] /i & x/\\[5mm]
        {[\textsc{peripheral}]} & {[\textsc{peripheral}]} & &  & {[\textsc{peripheral}]}\\
                             &                     & > &                   & \\
        \relax {[ɑ} & x] & & {[i} & x]\\[5mm]
        \relax {[\textsc{peripheral}]} & {[\textsc{peripheral}]} & &  & {[\textsc{peripheral}]}\\
      };
      \foreach \i in {1,2,5} \draw (matrix-1-\i) -- (matrix-2-\i);
      \foreach \i in {1,2,5} \draw (matrix-4-\i) -- (matrix-5-\i);
  \end{tikzpicture}
  \ex\label{ex:6:53b}
  \begin{tikzpicture}[baseline={(matrix-1-1.base)}]
    \matrix (matrix) [matrix of nodes, nodes in empty cells]
      { /x & ɑ/ &[10mm] &[10mm] /x & i/\\[5mm]
        \relax{[\textsc{peripheral}]} & {[\textsc{peripheral}]} & & {[\textsc{peripheral}]} & \\
                             &                     & > &                   & \\
        \relax {[x} & ɑ] & & {[x} & i]\\[5mm]
        \relax {[\textsc{peripheral}]} & {[\textsc{peripheral}]} & & {[\textsc{peripheral}]} & \\
      };
      \foreach \i in {1,2,4} \draw (matrix-1-\i) -- (matrix-2-\i);
      \foreach \i in {1,2,4} \draw (matrix-4-\i) -- (matrix-5-\i);
      \node [left=1mm of matrix-1-1.center, overlay] {\textsubscript{wd}[};
      \node [left=1mm of matrix-1-4.center, overlay] {\textsubscript{wd}[};
  \end{tikzpicture} 
  \z
\z


In some of the examples from \ipi{Visperterminen} and \ipi{Obersaxen} presented earlier, \isi{Vowel Fronting} involves not simply the deletion of [peripheral] from the original back vowel, but also the addition of [coronal] to those new front vowels, thereby creating the \isi{nonneutral} representation in \REF{ex:6:1a}. That type of vocalic change is depicted in \REF{ex:6:54a} for the context before a velar (/x/) and in \REF{ex:6:54b} for the context after a word-initial velar (/x/). Note that \isi{Vowel Fronting} -- represented here as /ɑ/ > /i/ -- \isi{feeds} velar fronting because the new front vowel created by the former (/i/) serves as a trigger for the latter.

\ea \label{ex:6:54}
  \small
  \ea \label{ex:6:54a}
  \hspace*{-5mm}\begin{tikzpicture}[baseline={(matrix-1-1.base)}]
    \matrix (matrix) [matrix of nodes, nodes in empty cells]
      { /ɑ & x/ &[5mm] &[5mm] /i & x/\\[5mm]
        {[\textsc{peripheral}]} & {[\textsc{peripheral}]} & & {[\textsc{coronal}]} & {[\textsc{peripheral}]}\\
                             &                     & > &                   & \\
        \relax {[ɑ} & x] & & {[i} & ç]\\[5mm]
        \relax {[\textsc{peripheral}]} & {[\textsc{peripheral}]} & & {[\textsc{coronal}]} & {[\textsc{peripheral}]}\\
      };
      \draw (matrix-4-5.south) -- (matrix-5-4.north);
      \foreach \i in {1,2,4,5} \draw (matrix-1-\i) -- (matrix-2-\i);
      \foreach \i in {1,2,4,5} \draw (matrix-4-\i) -- (matrix-5-\i);
  \end{tikzpicture}
  \ex\label{ex:6:54b}
  \hspace*{-5mm}\begin{tikzpicture}[baseline={(matrix-1-1.base)}]
    \matrix (matrix) [matrix of nodes, nodes in empty cells]
      { /x & ɑ/ &[5mm] &[5mm] /x & i/\\[5mm]
        \relax{[\textsc{peripheral}]} & {[\textsc{peripheral}]} & & {[\textsc{peripheral}]} & {[\textsc{coronal}]} \\
                             &                     & > &                   & \\
        \relax {[x} & ɑ] & & {[ç} & i]\\[5mm]
        \relax {[\textsc{peripheral}]} & {[\textsc{peripheral}]} & & {[\textsc{peripheral}]} & {[\textsc{coronal}]} \\
      };
      \foreach \i in {1,2,4,5} \draw (matrix-1-\i) -- (matrix-2-\i);
      \foreach \i in {1,2,4,5} \draw (matrix-4-\i) -- (matrix-5-\i);
      \draw (matrix-4-4.south) -- (matrix-5-5.north);
      \node [left=1mm of matrix-1-1.center, overlay] {\textsubscript{wd}[};
      \node [left=1mm of matrix-1-4.center, overlay] {\textsubscript{wd}[};
  \end{tikzpicture}
  \z
\z


I consider now three representative words in \REF{ex:6:55} from \ipi{Visperterminen} for the two types of \isi{Vowel Fronting}. Examples (\ref{ex:6:55a}, \ref{ex:6:55b}) exhibit the emergence of neutral vowels (=\ref{ex:6:53}) and the one in \REF{ex:6:55c} of \isi{nonneutral vowels} (=\ref{ex:6:54}). The reconstructed forms in the second column are my own.


\TabPositions{.1\textwidth, .15\textwidth, .2\textwidth, .3\textwidth, .5\textwidth, .7\textwidth}
\ea%55
    \label{ex:6:55}
\ea\label{ex:6:55a}\relax [xyo]   \tab < \tab \textsuperscript{+}[xuo]  \tab ‘cow’     \tab cf. OHG \textit{kuo}  \tab (from \ref{ex:6:19c})
\ex\label{ex:6:55b}\relax [røix]  \tab < \tab \textsuperscript{+}[rouh] \tab ‘smoke’   \tab cf. OHG \textit{rouh} \tab (from \ref{ex:6:27i})
\ex\label{ex:6:55c}\relax [byːç]  \tab < \tab \textsuperscript{+}[buːx] \tab ‘stomach’ \tab cf. OHG \textit{būh}  \tab (from \ref{ex:6:26c})
\z 
\z 


Since \isi{Vowel Fronting} is simply a cover term for any change from any etymological back vowel to any type of vowel that loses the backness feature, there is no reason to assume that the vocalic changes in \REF{ex:6:55} were necessarily coterminous. In fact, I show that the changes creating neutral vowels in (\ref{ex:6:55a}, \ref{ex:6:55b}) probably came about later than the ones creating \isi{nonneutral} structures, as in \REF{ex:6:55c}.

The vocalic changes in (\ref{ex:6:55a}, \ref{ex:6:55b}) are expressed formally in \REF{ex:6:56}. The featural structure to the left of the wedge in \REF{ex:6:56} captures the two original diphthongs, which consisted of back vowels (/ou, uo/). According to \is{neutral vowel}Neutral Vowel Formation the feature [dorsal] is replaced with [peripheral] and additional features are added to the two components, namely [--nasal] and the height features [±high] and [±low], which are represented in \REF{ex:6:56} with the two variables [αF] and [αG]. The crucial aspect of the change is that the second component of the diphthongs to the right of the wedge does not acquire the feature [coronal]. \is{neutral vowel}Neutral Vowel Formation in \REF{ex:6:56} can be compared with \isi{Coronalization} in \REF{ex:6:57}, which is required for example \REF{ex:6:55c}; recall \fnref{fn:6:2}. \isi{Coronalization} also involves a replacement of [dorsal] with [peripheral], but crucially the structure to the right of the wedge acquires the feature [coronal].

\ea%56
\label{ex:6:56}\is{neutral vowel}Neutral Vowel Formation:\\
\begin{minipage}[t]{.4\textwidth}
\centering  /ou/, /uo/  \\
            \begin{forest}
            [\avm{[dorsal]},calign=first,grow'=90
              [\avm{[−cons]}]
              [\avm{[−cons]}]
            ]
            \end{forest}
\end{minipage}\begin{minipage}[t]{.1\textwidth} \centering > \end{minipage}\begin{minipage}[t]{.4\textwidth}
\centering  /øi/, /yo/\\
            \begin{forest}
            [,phantom
                [\avm{[−cons\\−nas\\αF]} [\avm{[peripheral]}]]
                [\avm{[−cons\\−nas\\αG]}]
            ]
            \end{forest}
\end{minipage}
\ex%57
\label{ex:6:57}Coronalization:\\
\begin{minipage}[t]{.4\linewidth}
\centering /uː/\\
\begin{forest}
  [\avm{[−cons\\+high]} [\avm{[dorsal]}]]
\end{forest}
\end{minipage}\begin{minipage}[t]{.1\textwidth} \centering > \end{minipage}\begin{minipage}[t]{.4\textwidth}
\centering  /yː/\\
\begin{forest}
  [\avm{[−cons\\+high]}
     [\avm{[coronal]}] [\avm{[peripheral]}]
  ]
\end{forest}
\end{minipage}
\z 


In \REF{ex:6:58} I provide the three examples from \REF{ex:6:55} as well as the word [weiç] ‘soft’ from (\ref{ex:6:26d}). The first and third items represent the neutral vowels [yo] and [øi], the second example shows a high front \isi{nonneutral vowel} deriving from a historical back vowel, and the fourth example illustrates an inherited high front \isi{nonneutral vowel} ([i] in [ei]). At Stage 1 velars surfaced without change as velars. Stage 2 reflects the point where velar fronting was phonologized as an allophonic (transparent) process, and Stage 3 represents the dialect as it was described by Elisa Wipf in 1910. The subscripts indicate whether or not the segment in question is peripheral (“pˮ), dorsal (“dˮ) or coronal (“cˮ). I assume that all instantiations of [dorsal] at Stage 2 changed to [peripheral] at Stage 3.

\ea%58
    \label{ex:6:58}
\begin{tabular}[t]{@{} lllll @{}}
\relax  /x\textsubscript{d}u\textsubscript{d}o\textsubscript{d}/ & /buː\textsubscript{d}x\textsubscript{d}/    & /ro\textsubscript{d}u\textsubscript{d}x\textsubscript{d}/  &  /we\textsubscript{c}i\textsubscript{c}x\textsubscript{d}/    &         \\
\relax  [x\textsubscript{d}u\textsubscript{d}o\textsubscript{d}] & [buː\textsubscript{d}x\textsubscript{d}]    & [ro\textsubscript{d}u\textsubscript{d}x\textsubscript{d}]  &  [we\textsubscript{c}i\textsubscript{c}x\textsubscript{d}]    &  Stage 1\\\tablevspace
\relax  /x\textsubscript{d}u\textsubscript{d}o\textsubscript{d}/ & /byː\textsubscript{c}x\textsubscript{d}/    & /ro\textsubscript{d}u\textsubscript{d}x\textsubscript{d}/  &  /we\textsubscript{c}i\textsubscript{c}x\textsubscript{d}/    &         \\
\relax  [x\textsubscript{d}u\textsubscript{d}o\textsubscript{d}] & [byː\textsubscript{c}ç\textsubscript{cd}]   & [ro\textsubscript{d}u\textsubscript{d}x\textsubscript{d}]  &  [we\textsubscript{c}i\textsubscript{c}ç\textsubscript{cd}]   &  Stage 2\\\tablevspace
\relax  /x\textsubscript{p}y\textsubscript{p}o\textsubscript{p}/ & /byː\textsubscript{cp}x\textsubscript{p}/   & /rø\textsubscript{p}ix\textsubscript{p}/                   &   /we\textsubscript{c}i\textsubscript{c}x\textsubscript{p}/   &         \\
\relax  [x\textsubscript{p}y\textsubscript{p}o\textsubscript{p}] &  [byː\textsubscript{cp}ç\textsubscript{cp}] & [rø\textsubscript{p}ix\textsubscript{p}]                   &   [we\textsubscript{c}i\textsubscript{c}ç\textsubscript{cp}]  &  Stage 3\\\tablevspace
\relax   \textit{Kuh}                                            &   \textit{Bauch}                            & \textit{Rauch}                                             &   \textit{weich}                                              &   \il{Standard German}StG\\
\relax   ‘cow’                                                   &   ‘stomach’                                 & ‘smoke’                                                    &   ‘soft’                                                      &         \\
\end{tabular}
\z 


\isi{Coronalization} created a front (\isi{nonneutral}) vowel from a historical back vowel in [byːç]. When that restructuring occurred (=Stage 2), the new front vowel fed velar fronting, which created a palatal that was fully transparent. By contrast, the examples [xyo] and [røix] exemplify the historical \isi{underapplication} of velar fronting. In particular, at Stage 3 \is{neutral vowel}Neutral Vowel Formation converted the historical back vowels in those examples to diphthongs containing neutral vowels.

In \sectref{sec:2.5} I posited a historical model which involves the interaction between speakers and listeners in \isi{acquisition}. Consider how that approach accounts for the emergence of neutral vowels in \REF{ex:6:58}. At Stage 2 the speaker (P\textsubscript{1}) utters words like [weiç] (from /weix/) and [roux] (from /roux/). At Stage 3 the listener (P\textsubscript{2}) correctly hears [weiç] and -- on the basis of similar examples with [ç] and [x] -- deduces that the underlying representation is /weix/ with a rule of velar fronting. By contrast, the diphthong in [roux] is misperceived as a diphthong consisting of a front component ([ø]) followed by a high vowel that is no longer back but also not as front as the second component of [ei]. The second part of the new diphthong is therefore misperceived as something other than [i]. I speculate that when the change from /ou/ to /øi/ was phonologized the new diphthong was probably pronounced as [øi̠], where [i̠] represents a slightly retracted [i]. But the change from Stage 2 to Stage 3 did not simply involve P\textsubscript{2}’s misperception and pronunciation of that new vowel. It also crucially entailed the interpretation of that vowel in phonological units as one which is neither front nor back, but instead neutral, as in \REF{ex:6:56}. In \citeyear{Wipf1910} when Elisa Wipf published her book on the sounds of \ipi{Visperterminen} the second component of [ei] and [øi] had fallen together; hence, at that point there was no longer a phonetic difference between the [i] in [ei] and the [i̠] in [øi̠], but the unique phonological representation in \REF{ex:6:56} was retained.\footnote{{If the second component of [ei] and [øi] is now truly the same then it needs to be clarified how generations of \ipi{Visperterminen} listeners since 1910 have correctly acquired phonological representations with neutral vowels. I hypothesize that there remains a very subtle difference between the [i] in [ei] and the [i] in [øi] to the present day which serves as a cue to language learners that only the first but not the second serves as a trigger for velar fronting. Future work on \ipi{Visperterminen} can (dis)confirm my hypothesis.}}

\section{Discussion}\label{sec:6.5}\largerpage

\subsection{Alternative analyses}\label{sec:6.5.1}

Recall from \sectref{sec:2.4.2} that there is precedence in the cross-linguistic literature for neutral vowels. The example discussed in that section \citep{Dresher2009} involved Barrow Inupiaq, which has both a \isi{nonneutral}, Palatalization-triggering /i/, as well as a neutral, Palatalization-inhibiting /i/. In present terms, the former /i/ is marked phonologically for the feature that spreads in Palatalization ([coronal]), while the neutral /i/ does not have that feature. Significantly, neutral /i/ derived historically from a back vowel.

The material from Barrow Inupiaq lends strong support to the analysis of the two SwG varieties discussed in this chapter because it establishes a precedence for the two representations in \REF{ex:6:1}. In spite of that independent evidence one might claim that coronalless structures like the one in \REF{ex:6:1b} can be eschewed by adopting an alternative analysis. I discuss and reject three such alternatives below.

The weakest alternative to \REF{ex:6:1b} (Analysis A) is to assert that velars like [x] and palatals like [ç] are phonemes and to deny that there are any processes fronting the former to the latter. If /x/ and /ç/ -- as well as the corresponding affricates -- are phonemic, then one might assume that representations like \REF{ex:6:1b} are superfluous. Analysis A is untenable because velars and palatals never contrast in either of the \il{Highest Alemannic}HstAlmc varieties discussed above. For example, in \ipi{Visperterminen} postvocalic [ç] occurs only after any high front vowel with the exception of the [i] in [øi], but [x] surfaces only after back vowels and the [i] in [øi]. [x] and [øi] are therefore allophones according to any definition. That point aside, the reader should recall that \REF{ex:6:1b} derives independent support from \isi{Umlaut} alternations.

A second alternative to \REF{ex:6:1b} (Analysis B) is to derive palatals from the corresponding velars with versions of velar fronting which simply list the segmental triggers. For example, Analysis B would state \isi{Wd-Initial Velar Fronting-4} and \isi{Velar Fronting-6} as in \REF{ex:6:59}. An analysis along these lines is endorsed by \citet[509--511]{Anderson1981}, who assumes a synchronic rule of \isi{Velar Palatalization} in \ili{Icelandic} that is triggered by a list of segments and not a set of features.

\ea%59
\label{ex:6:59}Alternative rules (rejected):
\ea\label{ex:6:59a}/x  kx/ →  [ç  kç] /     \textsubscript{wd} [ {\longrule}\     /i iː iæ ỹː/
\ex\label{ex:6:59b}/x  kx/ →  [ç  kç] /     /i iː yː ei iæ {\~\i}ː ỹː ẽi/  {\longrule}
\z
\z 

The crucial difference between \REF{ex:6:59} and the rules of fronting posited above is that the rules in \REF{ex:6:59} are not expressed in terms of features. For example, \REF{ex:6:59a} is triggered by the four vowels /i iː iæ ỹː/ but not by the high front vowel /y/ in the diphthong /yu/ because /yu/ is not included in the list of triggers. Likewise \REF{ex:6:59b} applies after the vowels /i iː yː ei iæ {\~\i}ː ỹː ẽi/ but not after the /i/ in the diphthongs /øi/. Given that palatals are derived when adjacent to an arbitrary list of vowels -- and not to a natural class expressed in terms of features -- there is no need to analyze neutral vowels as placeless. Thus, the /y/ in /yo/ and then /i/ in /øi/ and /ɑi/ can be analyzed as [coronal].

A number of criticisms can be directed towards Analysis B. Observe that the treatment’s rejection of neutral vowels comes at the expense of relying on rules that do not apply to a natural class. That contrasts with velar fronting in all of the other German dialects investigated in this book. A more serious drawback is that it is not clear how Analysis B accounts for the vocalic alternations described in \sectref{sec:6.2.2} and \sectref{sec:6.3.2}.

A third alternative to \REF{ex:6:1b} (Analysis C) is to treat the aberrant words as \isi{lexical exceptions}. On that analysis, the reason [x] surfaces in a word in \ipi{Visperterminen} like [øix] ‘also’ is not because the /i/ has a coronalless representation, but instead because of the specific morpheme in which the sounds in question occur.

Analysis C can therefore be thought of as a morpheme-based analysis, which contrasts with the present treatment (a vowel-based analysis). There are two arguments against the former approach.

First, Analysis C cannot explain why the exceptional velars only surface in the neighborhood of the same vowels. For example, word-initial [x] surfaces not only in the morpheme [xyo] ‘cow’, but also in all other morphemes containing [yo]. But [x] fails to surface in word-initial position before other high front vowels. The same points hold for the [x] in \ipi{Visperterminen} examples like [røix] ‘smoke’. The fact that opaque velars occur only in the context of certain high front vowels but not in the context of others is captured directly by the vowel-based approach, but the facts are coincidental in the morpheme-based treatment.

Second,  if morphemes were marked as exceptional then there would be no explanation for \isi{Umlaut} alternations. For example, the morpheme ‘cow’ surfaces in \ipi{Visperterminen} as [xyo] in the singular, but the plural is [çiæ]. The morpheme [b\~ɑix] ‘bank’ likewise surfaces with the palatal [ç] in the plural (i.e. [bẽiç]). The change from [x] to [ç] in these examples makes sense given my treatment (which is vowel-based) because the [y] in [yo] and the [i] in [\~ɑi] but not the [i] in [iæ] or [ẽi] are neutral vowels. But if morphemes and not vowels were marked as exceptions as per Analysis C, there would be no explanation for the fact that the same morpheme sometimes obeys the rule and other times does not.

\subsection{Directionality}\label{sec:6.5.2}

Reference was made to a \isi{directionality} parameter in the typological literature on \isi{Velar Palatalization} (\sectref{sec:2.3.5}). Thus, the works cited in that section demonstrate that \isi{Velar Palatalization} can apply either regressively (right-to-left) or progressively (left-to-right). A hypothetical example illustrating regressive Palatalization is /ɑki/→[ɑci] and progressive Palatalization is /ikɑ/→[icɑ]. Both choices are attested in the languages of the world, although there is a clear preference for regressive spreading.

The \isi{directionality} parameter has not been discussed in the context of velar fronting in German dialects because postsonorant velar fronting always applies from left-to-right, cf. \il{Standard German}StG [kuːxən] ‘cake’ vs. [kʏçə] ‘kitchen’. In these items it can be seen that the trigger for velar fronting (e.g. /ʏ/) is to the immediate left of the target (/x/). The reason the trigger cannot be the vowel to the right of the target is that that vowel is always \isi{schwa} (/ə/) in native words. Schwa cannot trigger the spreading of the frontness feature because it is not a front vowel. Recall that \isi{schwa} in examples like [kuːxən] ‘cake’ vs. [kʏçə] ‘kitchen’ was etymologically a \isi{full vowel} (cf. \ili{OHG} \textit{kuohho} ‘cake’, OHG \textit{kuhhina} ‘kitchen’) which underwent \isi{Vowel Reduction}. \il{Standard German}StG also has many nonnative words (including names), in which the velar fronting target (/x/) is between two \isi{full vowels} (Appendix~\ref{appendix:g}), e.g. [ɛço] ‘echo’, \textit{Achim} [ɑxɪm] ‘(name)’. The reason \il{Standard German}StG tolerates words like these with \isi{full vowels} in unstressed syllables is that \isi{Vowel Reduction} is no longer active synchronically. More to the point, examples like [ɛço] and [ɑxɪm] confirm that velar fronting spreads the frontness feature progressively and not regressively. Nonnative words like these are not considered in this book because they are usually not discussed in the original sources.

The topic of \isi{directionality} is relevant in this chapter because \isi{Vowel Reduction} never occurred in \ipi{Visperterminen} (recall \sectref{sec:6.2.1}) and only applied to a limited extent in \ipi{Obersaxen}. Hence -- in contrast to all other dialects of German -- potential triggers for velar fronting can be present in both of those SwG varieties after the targets even in native words. Four representative examples from \ipi{Visperterminen} with velar fronting targets (/x/ and /kx/) situated between two \isi{full vowels} are repeated in \REF{ex:6:60}. Words like these confirm that spreading is progressive. Thus, in \REF{ex:6:60a} the (high front) vowel to the left of the target is a trigger, while the (back) vowel to the right of that target is not a trigger. However, the vowel to the right of the target (/xx/) in \REF{ex:6:60b} is high and front, while the vowel to the left of the trigger in those words is not high and front. Since the target /xx/ surfaces without change as velar in \REF{ex:6:60b} it can be concluded that velar fronting cannot spread the frontness feature from right-to-left. (Recall from \fnref{fn:6:10} that the regressive spreading attested in \ipi{Obersaxen} is the result of coarticulatory fronting and not discreet phonological fronting).
\TabPositions{.2\textwidth, .4\textwidth, .6\textwidth, .8\textwidth}

\ea \label{ex:6:60}
\ea\label{ex:6:60a}  de͔iχu   \tab  [dẽiçu]  \tab  ‘think-\textsc{inf}’  \tab  (from \ref{ex:6:26g})\\
     bleikχu  \tab [bleikçu] \tab ‘bleach-\textsc{inf}’  \tab (from \ref{ex:6:29b})\\
\ex\label{ex:6:60b}  xuxxi    \tab [xuxxi]   \tab ‘kitchen’              \tab (from \ref{ex:6:19a})\\
     dexxi    \tab [dexxi]   \tab ‘blanket’              \tab (from \ref{ex:6:27f})\\
\z 
\z 

The reason data like the ones in \REF{ex:6:60b} are significant is that they show velar fronting could potentially apply regressively in native words. Since outputs like *[xuççi] and *[deççi] are incorrect, velar fronting was phonologized in pre-\ipi{Visperterminen} as a rule applying progressively even though the opposite direction was available to native speakers. Interestingly, speakers of pre-Vis\-per\-ter\-mi\-nen did not opt for the preferred regressive direction. I return to the topic of \isi{directionality} in the context of when velar fronting was phonologized in \sectref{sec:16.5}.

\section{{Conclusion}}\label{sec:6.6}

What the two case studies discussed above have in common is that they possess neutral vowels, which by definition are phonetically front but which lack the phonological feature [coronal].  From the historical perspective, neutral vowels were once back ([dorsal]) sounds that were restructured to neutral vowels when historical processes eliminated the backness feature (\isi{Vowel Fronting}) failed to add the frontness feature [coronal]. The occurrence of velars like [x] in the neighborhood of those historical back vowels therefore exemplifies the historical \isi{underapplication} of velar fronting.

This chapter and the preceding one both consider cases involving the synchronic and/or diachronic \isi{underapplication} of velar fronting. The reason \isi{underapplication} occurs is that there were changes eliminating the original backness feature ([dorsal]), but those changes (e.g. \isi{Vowel Fronting} in the present chapter) failed to \isi{feed} velar fronting. In the following three chapters I consider the consequences of changes eliminating the feature for historically front sounds ([coronal]) in the context of velars undergoing fronting. It is demonstrated in those chapters that the type of change referred to here (e.g. \isi{Vowel Retraction}) led to a historical \isi{overapplication} of velar fronting and opaque palatals in the neighborhood of front vowels.\is{velar fronting islands|)}\il{Highest Alemannic|)}
