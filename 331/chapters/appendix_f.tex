\chapter{Modern reflexes of historical dorsal sounds}\label{appendix:f}

A central goal of the present book is to determine the realization of original (WGmc) velars in modern HG and LG dialects. As a point of reference this appendix shows how historical velars developed into those modern HG dialects on which \il{Standard German}StG is based (henceforth HG). The sounds discussed below also include the \isi{etymological palatal} glide; hence the appendix considers the modern reflexes of dorsal sounds.

The changes discussed below have been discussed at length in the earlier literature, e.g. \citet{Wright1907}, \citet{Prokosch1938}, \citet{vonKienle1969}, \citet{Russ1978, Russ1982}, \citet{Szulc2002}, and \citet{Fulk2018}. Two works discussing the development of original velars into modern German dialects include \citet{Behaghel1911} and especially \citet{Schirmunski1962}.

I consider first the development of \ili{WGmc} dorsal sounds in terms of their probable phonetic realizations based on the conclusions drawn from scholars of Gmc like the ones cited above. At the end of this appendix I show how the phonetic dorsals of \ili{WGmc} fit into a system of contrastive sounds (phonemes).

\ili{WGmc} velars surfacing in word-initial position were \textsuperscript{+}[k ɣ], as well as the \textsuperscript{+}[k] in \textsuperscript{+}[sk] clusters. \ili{PGmc} \textsuperscript{+}[x] did not occur in word-initial position in \ili{WGmc} because it either debuccalized to [h] before a vowel in (\ref{ex:appendix:f:1}a) or deleted before a consonant in (\ref{ex:appendix:f:1}b). Phonetic representations for the words listed in \REF{ex:appendix:f:1} and below can be inferred from the \il{Standard German}StG orthography.

\ea%1
    \label{ex:appendix:f:1}
\ea  PGmc \textsuperscript{+}[x] > HG [h]  Heer ‘army’, Herz ‘heart’
\ex   PGmc \textsuperscript{+}[x] > HG ∅  lachen ‘laugh\textsc{{}-inf}’ (cf. Go \textit{hlahjan}),

          rein ‘pure’ (cf Go \textit{hrains})

\z
\z

All instances of word-initial [x]/[ç] in HG are loanwords (Appendix~\ref{appendix:g}). The reason why no native word begins with [x]/[ç] is that the earlier reflex of those sounds (\ili{PGmc} \textsuperscript{+}[x]) either underwent h-Deletion or \isi{Debuccalization}. Since there were no independent (German-specific) changes that introduced new instances of word-initial [x]/[ç] in HG, there are no native words beginning with those sounds.

The modern reflex of \ili{WGmc} \textsuperscript{+}[k] in word-initial position is [k] in (\ref{ex:appendix:f:2a}), while \ili{WGmc} \textsuperscript{+}[sk] is now realized as [ʃ] in (\ref{ex:appendix:f:2b}). \ili{WGmc} \textsuperscript{+}[ɣ] in word-initial position is [g] in (\ref{ex:appendix:f:2c}).

\ea%2
    \label{ex:appendix:f:2}
\ea\label{ex:appendix:f:2a}
WGmc \textsuperscript{+}[k] >  HG [k]  Kuh ‘cow’, Kind ‘child’
\ex\label{ex:appendix:f:2b}
WGmc \textsuperscript{+}[sk] > HG [ʃ]  Schaf ‘sheep’, schöpfen ‘ladle\textsc{{}-inf}’,

          schlafen ‘sleep\textsc{{}-inf}’
\ex\label{ex:appendix:f:2c}
WGmc \textsuperscript{+}[ɣ] > HG [g]  Gast ‘guest’, gelb ‘yellow’, Glas ‘glass’

\z
\z

The traditional phonetic symbol for \ili{WGmc} \textsuperscript{+}[ɣ] is "g", although most scholars confusingly consider that word-initial sound to be a lenis fricative ([ɣ]) and not the corresponding stop ([g]). The reason the velar in question was realized as a fricative word-initially is that this is how it was realized in most of the earliest attested \ili{WGmc} languages, i.e. \ili{OE}, \ili{OLF}; see \citet[173]{Moulton1972} and \citet{Ringe2006} for a similar conclusion concerning \ili{PGmc}. The same generalization must also be true for the earliest stages of LG because an initial dorsal fricative (from \ili{WGmc} \textsuperscript{+}[ɣ]) is the norm in LG (\il{Westphalian}Wph) dialects described at the end of the nineteenth and early twentieth centuries (\chapref{sec:4}). It is therefore assumed throughout the present book that the initial sound in words like the ones in \REF{ex:appendix:f:2c} was a phonetic fricative (\ili{WGmc} \textsuperscript{+}[ɣ]), which shifted to [g] in an early stage (OHG).

The developments in \REF{ex:appendix:f:1} and \REF{ex:appendix:f:2} are depicted in \REF{ex:appendix:f:3}:

\ea%3
    \label{ex:appendix:f:3}
          Modern reflexes of historical velar obstruents in word-initial position: \\

\begin{forest} for tree={align=center}
[~, phantom
  [\textsuperscript{+}{[x]}
    [\textsuperscript{+}{[h]}
        [{[h]}]
    ]
  ]
  [\textsuperscript{+}{[k]}
    [\textsuperscript{+}{[k]}
        [{[k]}]
    ]
  ]
  [\textsuperscript{+}{[sk]}
    [\textsuperscript{+}{[sk]}
        [{[ʃ]}]
    ]
  ]
  [\textsuperscript{+}{[ɣ]}
    [\textsuperscript{+}{[ɣ]}
        [{[ɡ]}]
    ]
  ]
  [PGmc, no edge
    [WGmc, no edge
        [HG, no edge]
    ]
  ]
]
\end{forest}
\z


WGmc velars surfacing after a sonorant were \textsuperscript{+}[k x ɣ], as in \REF{ex:appendix:f:4}:

\ea%4
    \label{ex:appendix:f:4}
\ea  \label{ex:appendix:f:4a}
WGmc \textsuperscript{+}[x] > HG [x]/[ç]  Furche ‘furrow’, Nacht ‘night’,

          fechten ‘fence\textsc{{}-inf}’
\ex \label{ex:appendix:f:4b}
WGmc \textsuperscript{+}[k] > HG [x]/[ç]  Dach ‘roof’, Reich ‘empire’
\ex \label{ex:appendix:f:4c}
WGmc \textsuperscript{+}[ɣ] > HG [g]  Wagen ‘car’, liegen ‘lie\textsc{{}-inf}’,

          folgen ‘follow\textsc{{}-inf}’
\ex \label{ex:appendix:f:4d}   WGmc \textsuperscript{+}[ɣ] > HG [ç]  König ‘king’

\z
\z

The original fortis fricative is retained as a fricative, which undergoes velar fronting in the context of front sounds in (\ref{ex:appendix:f:4a}). \ili{WGmc} \textsuperscript{+}[k] is realized as a velar fricative in postsonorant position in (\ref{ex:appendix:f:4b}) by the \isi{High German Consonant Shift} \citep{Braune2004}. The new velar fricative created by the latter change undergoes velar fronting in the context after front segments. Since the \isi{High German Consonant Shift} did not affect LG, the LG reflex of \ili{WGmc} \textsuperscript{+}[k] is [k]. As a consequence, there are significantly more words containing [x]/[ç] in HG than in LG. In the default case, \ili{WGmc} \textsuperscript{+}[ɣ] is realized in HG as [g] in (\ref{ex:appendix:f:4c}), but in the context after [ɪ] in coda position, it is realized as [ç] in (\ref{ex:appendix:f:4d}).

  Comparative evidence from the earliest attested \ili{WGmc} languages supports treating the original velar in (\ref{ex:appendix:f:4c}, \ref{ex:appendix:f:4d}) as a fricative (\textsuperscript{+}[ɣ]) and not as a stop, but the same conclusion can be drawn from HG and LG dialect data. As attested in a number of varieties discussed in this book, the original \ili{WGmc} sound in (\ref{ex:appendix:f:4c}, \ref{ex:appendix:f:4d}) is retained as a velar/palatal fricative after any vowel; hence, the [g] in the HG words in \REF{ex:appendix:f:4c} is realized as [ɣ]/[ʝ]. The same generalization holds in final position, e.g. words like \textit{Tag} ‘day’ and \textit{Sieg} ‘victory’ where the final sound is [k] in HG is [x]/[ç] in many HG and LG varieties.

Historical geminate velar stops underwent \isi{Degemination} in (\ref{ex:appendix:f:5}a, \ref{ex:appendix:f:5}b). In (\ref{ex:appendix:f:5}c) it can be seen that \ili{WGmc} \textsuperscript{+}[xx] degeminated and now surfaces as velar or palatal depending on the nature of the preceding sound.

\ea\label{ex:appendix:f:5}
\ea WGmc \textsuperscript{+}[kk] > HG [k]  Rock ‘skirt’, recken ‘stretch\textsc{{}-inf}’
\ex WGmc \textsuperscript{+}[gg] > HG [k]  Brücke ‘bridge’, \ipi{Mücke} ‘mosquito’
\ex WGmc \textsuperscript{+}[xx] > HG [x]/[ç]  lachen ‘laugh\textsc{{}-inf}’, Küche ‘kitchen’
\z
\z

The \ili{WGmc} geminates in \REF{ex:appendix:f:5} were typically derived from the corresponding singletons before [j] by \isi{WGmc Gemination} (\citealt{Simmler1974}, \citealt{MurrayVennemann1983}, \citealt{Murray1986}, \citealt{Ham1998}, \citealt{Denton1998}, \citealt{Fulk2018}). Others emerged after a short vowel from the \isi{High German Consonant Shift}.

  The developments in \REF{ex:appendix:f:4}--\REF{ex:appendix:f:5} are illustrated in \REF{ex:appendix:f:6}. Not depicted here is the velar nasal (HG [ŋ]), which only surfaced in early Gmc in nasal-stop clusters, e.g. \textsuperscript{+}[ŋk] and \textsuperscript{+}[ŋk].

\ea%6
    \label{ex:appendix:f:6}
          Modern reflexes of historical velar obstruents in postsonorant position:\\
          \begin{forest} for tree={align=center}
[~, phantom
  [\textsuperscript{+}{[x]}
    [\textsuperscript{+}{[x]},name=plusx
        [{[x]},name=x]
    ]
  ]
  [\textsuperscript{+}{[k]}
    [\textsuperscript{+}{[k]},name=plusk
        [{[ç]},name=ç1]
    ]
  ]
  [\textsuperscript{+}{[ɣ]}
    [\textsuperscript{+}{[ɣ]},name=plusɣ
        [{[ɡ]}]
    ]
  ]
  [\textsuperscript{+}{[sk]}
    [\textsuperscript{+}{[sk]}
        [{[ʃ]}]
    ]
  ]
  [~
    [\textsuperscript{+}{[kk]}, name=kk, no edge
        [~, no edge]
    ]
  ]
  [~
    [~, no edge
        [{[k]}, no edge, name=k]
    ]
  ]
  [~
    [\textsuperscript{+}{[gg]}, no edge, name=gg
        [~, no edge]
    ]
  ]
  [~
    [\textsuperscript{+}{[xx]}, no edge, name=xx
        [{[x]}]
    ]
  ]
  [~, no edge
    [~, no edge
        [{[ç]},name=ç2, no edge]
    ]
  ]
  [PGmc, no edge
    [WGmc, no edge
        [HG, no edge]
    ]
  ]
]
\draw(plusx.south)--(ç1.north);
\draw(plusk.south)--(x.north);
\draw(plusɣ.south)--(ç1.north);
\draw(kk.south)--(k.north);
\draw(gg.south)--(k.north);
\draw(xx.south)--(ç2.north);
\end{forest}

\z

The \ili{WGmc} palatal glide -- referred to throughout this work as the \isi{etymological palatal} -- is retained as a palatal in word-initial position in HG; see (\ref{ex:appendix:f:7}a). In some modern varieties the original word-initial palatal glide is retained as a glide (e.g. in Almc; see \chapref{sec:3}); however, in many other varieties the original glide is now realized as a palatal fricative, e.g. \chapref{sec:4} for LG, \chapref{sec:9} for HG (CG). These two realizations of the original glide are depicted in \REF{ex:appendix:f:8}. In some LG varieties (\chapref{sec:10}) \ili{WGmc} \textsuperscript{+}[j] in examples like the ones in (\ref{ex:appendix:f:7}a) is now realized as a \isi{sibilant} fricative ([ʒ]). In contexts other than word-initial position, the original palatal glide deletes, as in (\ref{ex:appendix:f:7}b).

\ea%7
    \label{ex:appendix:f:7}
\ea   WGmc \textsuperscript{+}[j] > HG [j]/[ʝ]  ja ‘yes’, Jugend ‘youth’
\ex  WGmc \textsuperscript{+}[j] > HG ∅  recken ‘stretch\textsc{{}-inf}’, bitten ‘ask\textsc{{}-inf}’
\z
\z

\ea%8
    \label{ex:appendix:f:8}
          Modern reflexes of the palatal glide:\\
\begin{forest}
[~,phantom
  [{\textsuperscript{+}[j]},name=plusj, no edge
    [{[j]\strut}]
  ]
  [~,phantom
    [{[ʝ]\strut}, name=ʝ]
  ]
  [WGmc\strut, no edge
    [HG\strut, no edge]
  ]
]
\draw(plusj.south)--(ʝ.north);
\end{forest}
\z

Among the \ili{WGmc} velar sounds discussed above there is agreement among scholars that \textsuperscript{+}[k] was phonemic (/k/) because it contrasted with other consonants (e.g. /p/, /b/, /t/, /d/). \textsuperscript{+}[h] and \textsuperscript{+}[x] stood in complementary distribution, where the former surfaced only word-initially and the latter elsewhere. I capture that distribution with the \ili{WGmc} phoneme /x/, which was realized as \textsuperscript{+}[h] in word-initial position by the synchronic reflex of the historical change referred to above (\isi{Debuccalization}). Note that the allophonic distribution of \textsuperscript{+}[h] and \textsuperscript{+}[x] is inherited into many modern varieties of HG, e.g. \ipi{Maienfeld} (\sectref{sec:3.3}). The velar nasal was an allophone of /n/, since \textsuperscript{+}[ŋ] only occurred before a homorganic stop (\textsuperscript{+}[ŋk] and \textsuperscript{+}[ŋg]) and \textsuperscript{+}[n] elsewhere (see \citealt{Moulton1972}: 171 for \ili{PGmc}). Thus, the \ili{WGmc} phoneme was /n/, which was realized as \textsuperscript{+}[ŋ] before a velar sound by \isi{Regressive Nasal Place Assimilation}. (The \ili{WGmc} phoneme /m/ contrasted with /n/ initially, medially, and finally). The two lenis velars \textsuperscript{+}[ɣ] and \textsuperscript{+}[g] are considered by most scholars to be allophones of a single phoneme. In early Gmc (e.g. \ili{OE}, \ili{OLF}) the fricative had a much wider distribution than the stop: [g] surfaced only after \textsuperscript{+}[ŋ] and in gemination (\textsuperscript{+}[gg]) and [ɣ] in the elsewhere case (initially, medially between a vowel or liquid and a vowel, and finally after a vowel or liquid); see \citet[173]{Moulton1972} and \citet[113--114]{Szulc2002} on \ili{PGmc}. It is not always clear from the scholarly literature how the synchronic relationship between \textsuperscript{+}[ɣ] and \textsuperscript{+}[g] should be expressed. Here are two options: (a) There was a \ili{WGmc} phoneme /g/ that was realized as \textsuperscript{+}[ɣ] in the contexts listed above, or (b) there was a \ili{WGmc} phoneme /ɣ/ that was pronounced \textsuperscript{+}[g] after a homorganic nasal and in gemination. For purposes of this book I adopt (b) and not (a) because of the wider distribution of \ili{WGmc} \textsuperscript{+}[ɣ]. As a consequence I posit that there was a change I call g-Formation (e.g. \chapref{sec:3} and elsewhere), which shifted that original fricative /ɣ/ to the stop [g]. Finally, the \isi{etymological palatal} (\ili{WGmc} \textsuperscript{+}[j]) is a phonemic (underlying) glide (/j/). No scholarly works to my knowledge have actually argued that /j/ is phonemic (as opposed to being synchronically derived from another sound, presumably /i/), but the basic line of argumentation discussed in \citet{Hall2017} for the glides of \ili{MHG} can be extended to \ili{WGmc} as well.

