\chapter{Phonemicization of palatals (part 3)}\label{sec:10}\largerpage[2]
\is{alveolopalatalization|(}
\section{{Introduction}}\label{sec:10.1}

A common pattern characterized by many CG varieties involves the historical merger of the original palatal (nonsibilant) allophone [ç] (/x/), together with the inherited postalveolar (\isi{sibilant}) fricative [ʃ] (/ʃ/) to a new \isi{sibilant} fricative, namely alveolopalatal [ɕ] (/ɕ/). That change (\textsc{alveolopalatalization}) is de\-pict\-ed provisionally in \REF{ex:10:1}.\footnote{\label{fn:10:1}\isi{Alveolopalatalization} is referred to in much of the recent literature cited below as “\isi{Koronalisierung}ˮ (“coronalizationˮ). I eschew the latter term because [ʃ], [ɕ] and [ç] are all [coronal]. Much has been written on the phonetics of the sibilants referred to here, both from the cross-linguistic perspective (e.g. \citealt{LadefogedMaddieson1996}) and from the perspective of German dialects (e.g. \citealt{Herrgen1986}, \citealt{Gilles1999}). To simplify, alveolopalatal [ɕ] is usually described in the dialect literature as being articulated with unrounded (spread) lips, while postalveolar [ʃ] is pronounced with \isi{lip rounding} and protrusion. I discuss the way in which phonological representations mirror those articulations below.} Not shown here is the retention of the original velar allophone [x] (/x/), which surfaces in the context after back vowels.

\ea%1
\label{ex:10:1}\isi{Alveolopalatalization} (first version):\\
\begin{forest}
 [{[ɕ]},grow'=90 [{[ç]}] [{[ʃ]}] ]
\end{forest}
\z 


As a consequence of alveolopalatalization (\il{Standard German}StG) words like [ɪç] ‘I’, [fɪʃ] ‘fish’, and [frɔʃ] ‘frog’ are realized as [ɪɕ], [fɪɕ], and [frɔɕ] respectively, but words with historical [x] after a back vowel retain that velar, e.g. [lɔx] ‘hole’. Since [ɕ] and [x] contrast in the context after a back vowel (cf. [frɔɕ] vs. [lɔx]), alveolopalatalization involves the \isi{phonemicization} of /ɕ/. Significantly, the development in \REF{ex:10:1} did not result in the loss of velar fronting, which remains active as a rule of \isi{neutralization} relating words with alternations triggered by back vs. front vowels, e.g. [lɔx] (/lɔx/) ‘hole’ vs. [lœɕɐ] (/lœx-ɐ/) ‘hole-\textsc{pl}’.

\isi{Alveolopalatalization} has been studied extensively in the German dialect literature where it has been demonstrated that the change has been ongoing in Central Germany from the late nineteenth century to the present day. Some of the works on this topic include \citet{Mitzka1972}, \citet{Robinson2001}, \citet{Hall2014b}, and \citet{Féry2017}, although the most comprehensive treatment is undoubtedly \citet{Herrgen1986}.

In the system described above there is a contrast between velar ([x]) and alveolopalatal ([ɕ]) after certain back vowels (e.g. [frɔɕ] ‘frog’ vs. [lɔx] ‘hole’), but in the context after front vowels only [ɕ] occurs (e.g. [ɪɕ] ‘I’, [lœɕɐ] ‘hole-\textsc{pl}’). That system is therefore akin to the one for the postsonorant velar vs. palatal contrasts classified in \chapref{sec:9} as Contrast Type B, which can be extended to alveolopalatalizing dialects  as in \REF{ex:10:2}, where [i] and [ɑ] are cover symbols for front vowels and back vowels respectively.

\ea%2
\label{ex:10:2}
Contrast Type B:\\
\fbox{
\begin{tabular}{@{}ll@{}}
	\relax [...iɕ...] & [...ɑɕ...] \\
    & [...ɑx…] \\
\end{tabular}
}
\z 

The occurrence of the front sound [ɕ] in the context after a back vowel in \REF{ex:10:2} does not involve \isi{overapplication} \isi{opacity} because it was not the product of velar fronting. For example, the [ɕ] in [frɔɕ] ‘frog’ derived historically from the coronal \isi{sibilant} [ʃ] and not from a velar (cf. \ili{MHG} \textit{vrosch}).

I argue that the changes in \REF{ex:10:1} involved an intermediate stage not depicted above:

\begin{exe}%3
\ex\label{ex:10:3}
\begin{forest} for tree = {fit=band}
  [,phantom
   [/x/ [{[x]}]]   
   [/ʃ/ [{[ʃ]}]]  
   [>]  
   [/x/, calign=first [{[x]}] [{[ç]}]]  
   [/ʃ/ [{[ʃ]}]]  
   [>]   
   [/x/,calign=first [{[x]}] [{[ɕ]}]]  
   [/ʃ/ [{[ʃ]}]]  
   [>]  
   [/x/ [{[x]}]] 
   [/ɕ/ [{[ɕ]}]]
  ]
\end{forest}
\end{exe} 

In \REF{ex:10:3} the original velar is realized as velar after any kind of sound, and later the palatal (nonsibilant) allophone ([ç]) develops. The intermediate stage absent in \REF{ex:10:1} is one in which the earlier palatal allophone ([ç]) is realized as alveolopalatal [ɕ], which exists side by side with the inherited \isi{sibilant} [ʃ]. In the final stage of \REF{ex:10:3} the earlier allophone [ɕ] and the inherited fricative [ʃ] merge to [ɕ]. At that point [x] and [ɕ] contrast only in the context after a back vowel, as in \REF{ex:10:2}. The final stage of \REF{ex:10:3} illustrates another instantiation of a \isi{phonemic split} triggered by merger (\chapref{sec:8}).

Evidence for the final two stages in \REF{ex:10:3} comes from German dialects. Some dialects reveal the Contrast Type B system in the final stage in \REF{ex:10:3}, while others represent the pre-Contrast Type B system where [ɕ] is still an allophone of [x] (/x/) which exists side by side with [ʃ].

In \sectref{sec:10.2} I provide a more in-depth discussion of the historical stages depicted in \REF{ex:10:3}. In the remainder of the chapter I discuss Contrast Type B dialects (\sectref{sec:10.3}) as well as dialects in which [ɕ] is still an allophone of [x] (\sectref{sec:10.4}). In \sectref{sec:10.5} I discuss the areal distribution of alveolopalatalizing varieties. \sectref{sec:10.6} considers three topics in greater detail, namely the origin and spread of alveolopalatalization, the realization of the lenis fricative [ʝ] as an alveolopalatal \isi{sibilant}, and the way in which certain underlying representations are restructured in the course of alveolopalatalization.  In \sectref{sec:10.7} I conclude.

\section{{Alveolopalatalization} {deconstructed}}\label{sec:10.2}

It is argued below that alveolopalatalization consists of the stages depicted in \REF{ex:10:3}, which are made more explicit in \REF{ex:10:4}. Stage A corresponds to what has been referred to in earlier chapters as Stage 2 (\figref{fig:2.3}). Reference is made at Stage A to the distinctive features for [ʃ] (/ʃ/) described in \sectref{sec:2.2.2}. Recall from that section that the category “\isi{sibilant}” is not relevant for the phonology of German dialects under investigation in the present book and that phonological representations for sounds like /s/ and /ʃ/ consequently lack the nondistinctive feature [±strident]. The realization of sounds like /s/ and /ʃ/ as sibilants at the level of Speech is accomplished with rules of \isi{phonetic implementation}, which are discussed in greater detail below.

\begin{exe}
\ex Historical stages for alveolopalatalization:\label{ex:10:4}
\begin{description}
\item[Stage A:] Velar ([x]) and palatal ([ç]) are allophones related by velar fronting (from /x/). Phonemic [ʃ] (/ʃ/) is also present; that segment is phonologically a simplex coronal distinct from [s] (/s/) by either [±anterior] or [±high]. Phonetic implementation ensures that all simplex coronal fricatives (but crucially not the complex segment [ç]) are realized at the level of Speech as sibilants.

\item[Stage B:] Velar fronting continues to be active. It has the same form as the eponymous process at Stage A and therefore creates a [coronal, dorsal] fricative. [ʃ] (/ʃ/) is also present, which is reanalyzed phonologically as a complex [coronal, labial] segment. [coronal, labial] and [coronal, dorsal] fricatives are interpreted by \isi{phonetic implementation} at the level of Speech as the sibilants [ʃ] and [ɕ] respectively.

\item[Stage C:] Phonemic [ʃ] (/ʃ/) undergoes a change to a complex segment ([coronal, dorsal]), thereby merging with the Stage B [coronal, dorsal] which itself was the product of velar fronting. That complex fricative is interpreted in \isi{phonetic implementation} as the alveolopalatal \isi{sibilant} [ɕ]. Velar fronting remains active as a \isi{neutralization} rule capturing alternations between velar [x] and its fronted variant.
\end{description}
\end{exe}

A simplification of the progression from Stage A to Stage C is given in \REF{ex:10:5}. In the final column I list the dialects discussed below representing Stages B and C. In the remainder of this chapter I discuss first Stage C dialects because the descriptions for those varieties are more detailed (and impressionistically more common) than the ones for Stage B.

\ea%5
\label{ex:10:5}\isi{Alveolopalatalization} (final version):\\
\begin{forest} for tree = {grow'=90}
  [ {[ɕ]},name=C 
    [{[ɕ]} [{[ç]}]]
    [{[ʃ]},name=B [{[ʃ]},name=A]]  
  ]
  \node [right=of C] (C-description) {Stage C (\ipi{Schlebusch}, Lxm, \ipi{Leipzig})};
  \path let \p1=(B), \p2=(C-description.base west) in node [anchor=west] at (\x2,\y1) 
     {Stage B  (\ipi{Cologne}, \ipi{Frankfurt am Main}/\ipi{Montabaur})};
  \path let \p1=(A), \p2=(C-description.base west) in node [anchor=west] at (\x2,\y1) 
     {Stage A  (many dialects)};
\end{forest}
\z 

The palatal fricative [ç] undergoing alveolopalatalization can have more than one synchronic (and diachronic) source. As described above, Stage A [ç] can derive synchronically from /x/. What is not depicted in \REF{ex:10:5} is that the original [ç] that undergoes alveolopalatalization can also be the coda realization of lenis \.{/}ɣ/ after a front segment, e.g. Stage B/C /iɣ/→{\textbar}ix{\textbar}→[iɕ]. In dialects like those, /ɣ/ undergoes velar fronting after a front segment in a word-internal onset, surfacing as nonsibilant [ʝ] but not as the lenis alveolopalatal fricative [ʑ].

Velar fronting at Stage B/C necessitates the same change as in other dialects, namely one in which the feature [coronal] from a front segment spreads to [dorsal] sound, thereby creating a complex corono-dorsal segment. However, velar fronting does not create alveolopalatal [ɕ] directly; instead, the [coronal, dorsal] sound produced by velar fronting is interpreted as alveolopalatal by the \isi{phonetic implementation} made explicit below.

An \isi{analogy} can be made involving the rhotic consonant /r/. That segment is defined phonologically with a particular feature complex (e.g. [+consonantal, +sonorant, +continuant, coronal]), but those features tell us nothing about whether or not /r/ is realized in Speech as a trill or approximant (or something else). The feature complex [+consonantal, +sonorant, +continuant, coronal] is the phonological representation for /r/, but rules of \isi{phonetic implementation} specify whether or not that segment is articulated as a trill (in one language, dialect, idiolect) or approximant (in another language, dialect, idiolect). Likewise the features [coronal, dorsal] for a fortis fricative say nothing about whether or not that segment is a \isi{sibilant} ([ɕ]) or a nonsibilant ([ç]). That kind of fine-grained distinction is captured in the phonetics and not in the phonology.

There are a number of different ways to express the place contrast involving /s/, /ʃ/, and /ɕ/ According to one -- alluded to briefly in \sectref{sec:2.2.2} -- /s/ and /ʃ/ (e.g. in \il{Standard German}StG) are distinguished with either [±anterior] or [±high]. My analysis adopts the proposal made in some of the recent work on alveolopalatalization referred to above that \isi{lip rounding} (recall \fnref{fn:10:1}) is phonologically distinctive for /ʃ/ in alveolopalatalizing dialects; hence, /ʃ/ is analyzed as [coronal, labial], /s/ is [coronal] without a [labial] component, and /ɕ/ is [coronal, dorsal]. One advantage of that treatment (not discussed below) is that it expresses the connection between alveolopalatalization and the unrounding of front rounded vowels \citep{Hall2014b}. A second advantage is that the complex segment analysis of /ʃ/ simplifies the rules of \isi{phonetic implementation} referred to above. As indicated below, \isi{phonetic implementation} specifying sibilancy for one or more coronal fricative differs slightly according to the historical stage. I only list phonetic symbols for fortis fricatives here.

\eanoraggedright%6
\label{ex:10:6}Phonetic Implementation (Stage A):
\ea\label{ex:10:6a} Fortis complex fricatives ([coronal, dorsal]) are interpreted as nonsibilants ([ç]).
\ex\label{ex:10:6b} Simplex [coronal] fricatives are interpreted as sibilants ([s ʃ]).
\z 
\ex%7 
\label{ex:10:7}Phonetic Implementation (Stage B):
\ea\label{ex:10:7a} Fortis complex fricatives are interpreted as sibilants ([coronal, labial] as [ʃ], [coronal, dorsal] as [ɕ]).
\ex\label{ex:10:7b} Simplex [coronal] fricatives are interpreted as sibilants ([s]).
\z 
\ex%8
\label{ex:10:8}Phonetic Implementation (Stage C):
\ea\label{ex:10:8a} Fortis complex fricatives ([coronal, dorsal]) are interpreted as sibilants ([ɕ]).
\ex\label{ex:10:8b} Simplex [coronal] fricatives are interpreted as sibilants ([s]).
\z 
\z 

Significantly, (\ref{ex:10:6}--\ref{ex:10:8}) refer to whether or not the phonological structures are simplex coronals or if they have a complex place structure. The important point to observe is that there were two changes involving (\ref{ex:10:6}--\ref{ex:10:8}). First, the requirement that complex fricatives surface as nonsibilants (=\ref{ex:10:6a}) changed to one specifying those sounds as sibilants (=\ref{ex:10:7a}). Second, the phonological representation of /ʃ/ changes from a simplex coronal at Stage A and a complex sound at Stage B.

\section{{Stage} {C} {dialects}}\label{sec:10.3}

\subsection{Ripuarian (part 1)}\label{sec:10.3.1}\il{Ripuarian|(}

\citet{Bubner1935} describes the \il{Ripuarian}Rpn dialect spoken in \ipi{Schlebusch} (Leverkusen) in the German state of North Rhine-Westphalia (Nordrhein-Westfalen; \mapref{map:8}), which has the phonemic front vowels /iː y yː e eː ɛ ɛː ø øː œ œː/ and the phonemic back vowels /u uː o oː ɔ ɔː ɑ ɑː ə/. Three diphthongs end in a front vowel (/ɑi ei øy/) and one ends in a back vowel (/ou/).

As in many other varieties of \il{Ripuarian}Rpn, \ipi{Schlebusch} has the three dorsal fricatives [x ɣ ʝ], (=⟦x γ j⟧) but no fortis palatal [ç]. In addition to [s z] (=⟦s z⟧), the dialect has a third \isi{sibilant} fricative (=⟦š⟧), which Bubner describes (p. 6) as an alveolar-cerebral fricative (“alveolarer-zerebraler Reibelautˮ). Since no further details are given, it is not possible to determine with certainty whether or not that sound corresponds to [ʃ] or [ɕ]. On the basis of phonological patterning I argue that ⟦š⟧ is alveolopalatal [ɕ] and not postalveolar [ʃ].  Further support that ⟦š⟧ represents [ɕ] is that other Stage C dialects are attested in which the output of that change is [ɕ], but no Stage C dialect to my knowledge is attested in which the output is [ʃ].

\ipi{Schlebusch} has the palatal nonsibilant lenis fricative [ʝ] -- but no fortis counterpart ([ç]) -- and the coronal (alveolopalatal) \isi{sibilant} fricative [ɕ], but no lenis counterpart ([ʑ]). The phonemic fricatives in question and their allophones are illustrated for word-initial and postsonorant position in \REF{ex:10:9}. Not depicted here is /g/, which contrasts with the fricatives listed in \REF{ex:10:9}.

\ea%9
\label{ex:10:9}\begin{multicols}{2}
\ea\label{ex:10:9a}  \begin{forest}
[,phantom [/ɕ/ [{[ɕ]}]]    [/ʝ/  [{[ʝ]}]] ]
\end{forest}
\ex\label{ex:10:9b} \begin{forest}
 [,phantom [/x/  [{[x]}]]   [/ɕ/ [{[ɕ]}]]     [/ɣ/,calign=first [{[ɣ]}] [{[ʝ]}]]]
 \end{forest}
\z 
\end{multicols}
\z 

The two velars [x] (/x/) and [ɣ] (/ɣ/) differ in terms of a laryngeal dimension. As implied by the phonetic symbols, [ʝ] and [ɕ] represent two distinct places of articulation from the point of view of phonetics. \citet[6]{Bubner1935} therefore describes the place of articulation for [ʝ] (=⟦j⟧) as “palatalˮ, which is different from the “alveolar-cerebralˮ category for [ɕ] (=⟦š⟧). In \REF{ex:10:10} I list the four fricatives in \REF{ex:10:9} from the point of view of phonetics; hence, each of the four columns reflect a separate place of articulation. For comparison, I also include [s] and [z], which are uncontroversially underlying segments.\largerpage

\ea
\label{ex:10:10}Coronal and dorsal fricatives (arranged according to phonetics):\\
\begin{tabular}{lllll}
fortis & [s] & [ɕ] &  & [x]\\
lenis & [z] &  & [ʝ] & [ɣ]\\
\end{tabular}
\z 

From the point of view of phonology, fortis [ɕ] (/ɕ/) and lenis [ʝ] (/ʝ/) are paired together just as other fortis vs. lenis pairs, namely [s] (/s/) and [z] (/z/); [x] (/x/) and [ɣ] (/ɣ/). Alternations between [x] and [ɕ] described below are likewise juxtaposed the same way alternations involving [ɣ] and [ʝ] are. The \isi{sibilant} [ɕ] therefore occupies the slot other dialects fill with the nonsibilant [ç]. The analysis described here is depicted in \REF{ex:10:11}.

\ea%11
\label{ex:10:11}Coronal and dorsal fricatives represented phonologically:\\
\begin{tabular}{llll}
{[+fortis] }& [s] (/s/) & [ɕ] (/ɕ/) & [x] (/x/)\\
{[--fortis]} & [z] (/z/) & [ʝ] (/ʝ/) & [ɣ] (/ɣ/)\\
\end{tabular}
\z 

I demonstrate below that the three pairs in \REF{ex:10:11} are alike featurally: /s z/ are simplex [coronal], /x ɣ/ simplex [dorsal], and /ɕ ʝ/ complex ([coronal, dorsal]).

In word-initial position [ɕ] surfaces before any type of vowel or before a coronal consonant in (\ref{ex:10:12a}). The alveolopalatal in such examples derives from historical coronal sounds (cf. MHG \textit{schūm}, MHG \textit{slōz}). Palatal [ʝ] occurs word-initially before any vowel in (\ref{ex:10:12b}). [ʝ] in examples like those derives from a historical palatal (\ili{WGmc} \textsuperscript{+}[j]) or velar (\ili{WGmc} \textsuperscript{+}[ɣ]). The fronting of an original velar before any kind of segment (as in \ipi{Schlebusch}) is investigated in \chapref{sec:14}.

\TabPositions{.15\textwidth, .33\textwidth, .5\textwidth, .75\textwidth}
\ea%12
\label{ex:10:12}\ipi{Schlebusch} [ɕ] (from /ɕ/) and [ʝ] (from /ʝ/):
\ea\label{ex:10:12a}  šum  \tab [ɕum]  \tab Schaum  \tab ‘foam’   \tab 78\\
     šlǫs \tab [ɕlɔs] \tab Schloss \tab ‘lock’   \tab 78
\ex\label{ex:10:12b}  jęl  \tab [ʝɛl]  \tab gelb    \tab ‘yellow’ \tab 72\\
     jǭ   \tab [ʝɔː]  \tab ja      \tab ‘yes’    \tab 88
\z 
\z 

The following data illustrate that [x] surfaces after back vowels in (\ref{ex:10:13}) and [ɕ] after front vowels in (\ref{ex:10:14a}) or back vowels in (\ref{ex:10:14b}). The [x] in \REF{ex:10:13} derived historically from a velar (\ili{WGmc} \textsuperscript{+}[k] or \textsuperscript{+}[x]) and the [ɕ] in \REF{ex:10:14} from coronal [ʃ] (cf. MHG \textit{zwischen, visch, droschen, vrosch}).


\ea\label{ex:10:13}\ipi{Schlebusch} [x] (from /x/):\\
bux \tab [bux] \tab Bauch \tab ‘stomach’ \tab 65\\
lǫx \tab [lɔx] \tab Loch \tab ‘hole’ \tab 65\\
bōːx \tab [boːx] \tab Buch \tab ‘book’ \tab 65\\
hǫːx \tab [hɔːx] \tab Haken \tab ‘hook’ \tab 65\\
wɑ̄x \tab [βɑːx] \tab wach \tab ‘awake’ \tab 65\\
bɑ̄x \tab [bɑːx] \tab Bach \tab ‘stream’ \tab 65\\
\pagebreak
\ex%14
\label{ex:10:14}\ipi{Schlebusch} [ɕ] (from /ɕ/):
\ea\label{ex:10:14a} krīšǝ \tab [kriːɕǝ] \tab weinen \tab ‘cry-\textsc{inf}’ \tab 79\\
    tøšǝ \tab [tøɕǝ] \tab zwischen \tab ‘between’ \tab 78\\
    veš \tab [veɕ] \tab Fisch \tab ‘fish’ \tab 78\\
    flęš \tab [flɛɕ] \tab Flasche \tab ‘bottle’ \tab 78\\
    vlēš \tab [vleːɕ] \tab Fleisch \tab ‘meat’ \tab 78
\ex\label{ex:10:14b} rūšǝ \tab [ruːɕǝ] \tab rauschen \tab ‘rustle-\textsc{inf}’ \tab 79\\
    drošǝ \tab [droɕǝ] \tab droschen \tab ‘thresh-\textsc{pret}’ \tab 110\\
    vrǫš \tab [vrɔɕ] \tab Frosch \tab ‘frog’ \tab  21
   \z
\z 

Significantly, [x] and [ɕ] contrast after a back vowel, e.g. [vrɔɕ]  ‘frog’ vs. [lɔx] ‘hole’, but after a front vowel only [ɕ] occurs (=\ref{ex:10:2}).

The absence of [x] after a front vowel is also reflected in the regular replacement of [x] with [ɕ] after a front vowel in morphophonemic alternations like the ones in (\ref{ex:10:15}). As indicated here, [x] after a back vowel corresponds to [ɕ] after a stem vowel mutation. The material presented in the original source suggests that there are no exceptions to the alternating pattern in \REF{ex:10:15}, e.g. a stem with a back vowel plus [x] in which the [x] surfaces without change as [x] and not as [ɕ] after the alternant with a front vowel.  The dorsal fricatives in words like the ones in \REF{ex:10:15} derived historically from a fortis velar (\ili{WGmc} \textsuperscript{+}[k] or \textsuperscript{+}[x]).

\ea%15
\label{ex:10:15}\ipi{Schlebusch} [x]-[ɕ] Alternations (from /x/):

\ea\label{ex:10:15a}   ruxǝ \tab [ruxǝ] \tab riechen \tab ‘smell-\textsc{inf}’ \tab 35\\
      ryš \tab [ryɕ] \tab riecht \tab ‘smell-\textsc{3sg}’ \tab 35
\ex\label{ex:10:15b}   lǫx \tab [lɔx] \tab Loch \tab ‘hole’ \tab 96\\
      l\={ø}šǝ \tab [løːɕǝ] \tab Löcher \tab ‘hole-\textsc{pl}’ \tab 96
\ex\label{ex:10:15c}   šprōːxǝ \tab [ʃproːxǝ] \tab sprachen \tab ‘speak-\textsc{pret}’ \tab 112\\
      špręšǝ \tab [ʃprɛɕǝ] \tab sprechen \tab ‘speak-\textsc{inf}’ \tab 112
\z 
\z 
    
The alternating examples in \REF{ex:10:15} are captured synchronically with an underlying /x/ which undergoes the version of velar fronting posited below. The reason /ɕ/ cannot be taken as basic with a rule retracting that sound to [x] after a back vowel is that there are a number of morphemes containing a nonalternating [ɕ], as in \REF{ex:10:14b}. Additional examples are provided in \REF{ex:10:16}. Note that the stem vowels in \REF{ex:10:16} set display the same kind of vowel alternations as in \REF{ex:10:15}. Were /ɕ/ the sound underlying the alternations in \REF{ex:10:15}, then the rule retracting that sound to [x] after a back vowel would incorrectly apply to some of the examples in \REF{ex:10:16}.\pagebreak

\ea%16
\label{ex:10:16}\ipi{Schlebusch} nonalternating [ɕ] (from /ɕ/):
\ea\label{ex:10:16a} dręšǝ \tab [drɛɕǝ] \tab dreschen \tab ‘thresh-\textsc{inf}’ \tab 110\\
    drošǝ \tab [droɕǝ] \tab droschen \tab ‘thresh-\textsc{pret}’ \tab 110
\ex\label{ex:10:16b} vrǫš \tab [vrɔɕ] \tab Frosch \tab ‘frog’ \tab 21\\
    vrøš \tab [vrøɕ] \tab Frösche \tab ‘frog-\textsc{pl}’ \tab  22
\ex\label{ex:10:16c} wūǝš \tab [βuːǝɕ] \tab Wurst \tab ‘sausage’ \tab 26\\
    w\={y}ǝš \tab [βyːǝɕ] \tab Würste \tab ‘sausage-\textsc{pl}’ \tab 26
   \z
\z 

The [ɕ] in \REF{ex:10:16} derives synchronically from /ɕ/, but the diachronic source for that sound was [ʃ].

The items listed in \REF{ex:10:17} contain a surface [ɕ] deriving etymologically from a fortis velar sound (\ili{WGmc} \textsuperscript{+}[k]) in the context after a front vowel. No examples were found in the original source in which [ɕ] occurs after a coronal sonorant consonant.

\ea%17
\label{ex:10:17}\ipi{Schlebusch}  nonalternating [ɕ] (from /x/):\\
  eš \tab [eɕ] \tab ich \tab ‘I’ \tab 65\\
  zēš \tab [zeːɕ] \tab kurze Sense \tab ‘short scythe’ \tab 65\\
  bręšǝ \tab [brɛɕǝ] \tab brechen \tab ‘break-\textsc{inf}’ \tab 65\\
  jøšǝ \tab [ʝøɕǝ] \tab jucken \tab ‘itch-\textsc{inf}’ \tab 65
\z 

As indicated above, the [ɕ] in \REF{ex:10:17} -- in contrast to the examples in \REF{ex:10:15} -- does not alternate with [x]. As a Contrast Type B dialect, \ipi{Schlebusch} does not contrast [x] and [ɕ] after front vowels; hence, I analyze the underlying representation in words like the ones in \REF{ex:10:17} with a velar (/x/), which is simply inherited from pre-\ipi{Schlebusch} (Stage B in \ref{ex:10:4}). Note that there are two types of words with [ɕ] after a front vowel: Those in which [ɕ] is underlyingly /ɕ/ (=\ref{ex:10:16}) and those in which [ɕ] derives from /x/ (=\ref{ex:10:17}). A question I discuss in \sectref{sec:10.6.3} is how speakers acquiring the \ipi{Schlebusch} system who are not knowledgeable about etymology are able to determine the correct underlying representation.

Velar [ɣ] surfaces in a word-internal onset only after a full back vowel in (\ref{ex:10:18a}), while palatal [ʝ] is found in a word-internal onset only after a front vowel in (\ref{ex:10:18b}). The items in \REF{ex:10:18c} indicate that [ʝ] can also occur in a word-internal onset after a \isi{schwa} if that \isi{schwa} is preceded by a coronal consonant. I analyze the [ʝ] in (\ref{ex:10:18b}, \ref{ex:10:18c}) as a realization of /ɣ/. The [ɣ ʝ] in these examples derives historically from the lenis velar fricative (\ili{WGmc} \textsuperscript{+}[ɣ]).

\ea%18
\label{ex:10:18}\ipi{Schlebusch} [ɣ] and [ʝ] (from /ɣ/):
\ea\label{ex:10:18a} ōːɣǝ \tab [oːɣǝ] \tab Augen \tab ‘eye-\textsc{pl}’ \tab 73\\
    zɑ̄ːɣǝ \tab [zɑːɣǝ] \tab sagen \tab ‘say-\textsc{inf}’ \tab 73
\ex\label{ex:10:18b} bēːjǝ \tab [beːʝǝ] \tab biegen \tab ‘bend-\textsc{inf}’ \tab 73\\
    z\={ę}jǝ \tab [zɛːʝǝ] \tab sägen \tab ‘saw-\textsc{inf}’ \tab 73
\ex\label{ex:10:18c} o·rǝjǝl \tab [o·rǝʝǝl] \tab Orgel \tab ‘organ’ \tab 73\\
    he·lǝjǝ \tab [he·lǝʝǝ] \tab Heiligen \tab ‘saint-\textsc{pl}’ \tab 73\\
    zɑ̄ːnǝjǝ \tab [zɑːnǝʝǝ] \tab sandiger \tab ‘sandy-\textsc{infl}’ \tab 73
    \z
\z 

I argue that the pre-[ʝ] \isi{schwa} in \REF{ex:10:18c} is epenthetic and that it acquires [coronal] from the segment to its left (/r l n/). That fronted \isi{schwa} then spreads [coronal] to /ɣ/, thereby creating a palatal  (recall \sectref{sec:5.4}), e.g. /o·rɣǝl/ → {\textbar}o·rǝɣǝl{\textbar} → {\textbar}o·rə̟ɣǝl{\textbar} → [o·rə̟ʝǝl]. In contrast to the varieties discussed in \sectref{sec:5.4}, the rule fronting \isi{schwa} is triggered by all coronal sonorants, while \isi{schwa} epenthesis applies between a coronal sonorant and a noncoronal consonant (/ɣ/) that can be in a word-internal onset. I do not indicate the fronted \isi{schwa} in the phonetic representations in \REF{ex:10:18c} and in similar examples presented below because my transcriptions are broad and not narrow.

Postsonorant /ɣ/ (<\ili{WGmc} \textsuperscript{+}[ɣ] /ɣ/) participates in alternations involving laryngeal and place features like the ones in \REF{ex:10:19}. In coda position, /ɣ/ undergoes \isi{Final Fortition} to [x], as in the second example in (\ref{ex:10:19a}, \ref{ex:10:19b}). If /ɣ/ is preceded by a front vowel, it is realized as [ʝ] in a word-internal onset, as in the first example in (\ref{ex:10:19b}--\ref{ex:10:19e}) and as [ɕ] in the coda, as in the final example in (\ref{ex:10:19c}--\ref{ex:10:19e}). That palatals in \REF{ex:10:19e} result when [ə̟] spreads its [coronal] feature to /ɣ/.

\ea%19
\label{ex:10:19}\ipi{Schlebusch} laryngeal and place alternations (from /ɣ/):
\ea\label{ex:10:19a} zɑ̄ːɣǝ \tab [zɑːɣǝ] \tab sagen \tab ‘say-\textsc{inf}’ \tab 73 \\
    zɑ̄ːx \tab [zɑːx] \tab sage \tab ‘say-\textsc{1sg}’ \tab 74
\ex\label{ex:10:19b} bǝdrēːjǝ \tab [bǝdreːʝǝ] \tab betrügen \tab ‘cheat-\textsc{inf}’ \tab 108\\
    bǝdrox \tab [bǝdrox] \tab betrog \tab ‘cheat-\textsc{pret}’ \tab 108
\ex\label{ex:10:19c} z\={ę}jǝ \tab [zɛːʝǝ] \tab sägen \tab ‘saw-\textsc{inf}’ \tab 73\\
    z\={ę}š \tab [zɛːɕ] \tab sägt \tab ‘saw-\textsc{3sg}’ \tab 74
\ex\label{ex:10:19d} fleːjǝ \tab [fleːʝǝ] \tab fliegen \tab ‘fly-\textsc{inf}’ \tab 73\\
    flyšt \tab [flyɕt] \tab fliegt \tab ‘fly-\textsc{3sg}’ \tab 74
\ex\label{ex:10:19e} zɑ̄ːnǝjǝ \tab [zɑːnǝʝǝ] \tab sandiger \tab ‘sandy-\textsc{infl}’ \tab 73\\
    zɑ̄ːnǝš \tab [zɑːnǝɕ] \tab sandig \tab ‘sandy’ \tab 73
    \z
\z 

In sum, [ɣ] and [ʝ] do not contrast; hence, those two sounds derive synchronically from /ɣ/, as indicated in the heading for \REF{ex:10:19}. Velar fronting thus not only must capture the relationship between [x] and [ɕ] in \REF{ex:10:15} but also the one between [ɣ] and [ʝ] in \REF{ex:10:18} and \REF{ex:10:19}.

In \REF{ex:10:20} I provide representations for the four fricatives discussed above (/ɕ x ɣ ʝ/). I also include /s/ for comparison. The structures given here are the ones present at the underlying level and in the phonetic representation. Recall from \REF{ex:10:9} that [ʝ] is an allophone of /ɣ/ in postvocalic position, but that it is an underlying palatal in word-initial position.

\todo{[ɕ] /ɕ/ is a heading and starts at the very left, like all headings. The geometry of the tree afterwards has no bearing on its placement}

\todo{please check ordering of a b c d e}
\ea\label{ex:10:20}

\begin{multicols}{2}\raggedcolumns
\ea\label{ex:10:20a}\relax [s] /s/\\
\begin{forest}
[\avm{[−son\\+cont\\+fortis]} [\avm{[coronal]}]]
\end{forest}
\ex\label{ex:10:20b} \relax [ɕ] /ɕ/\\
\begin{forest}
[\avm{[−son\\+cont\\+fortis]} [\avm{[coronal]}] [\avm{[dorsal]}]]
\end{forest}
\columnbreak

\ex\label{ex:10:20c} \relax [x] /x/\\      
\begin{forest}
[\avm{[−son\\+cont\\+fortis]} [\avm{[dorsal]}]]
\end{forest}
\ex\label{ex:10:20d} \relax [ɣ]  /ɣ/\\   
\begin{forest}
[\avm{[−son\\+cont\\−fortis]} [\avm{[dorsal]}]]
\end{forest}
\z
\end{multicols}
\begin{xlist}
\setcounter{xnumii}{4}
\ex\label{ex:10:20e} \relax [ʝ]  /ʝ/\\
\begin{forest}
[\avm{[−son\\+cont\\−fortis]} [\avm{[coronal]}] [\avm{[dorsal]}]]
\end{forest}
\z
\z


The most important structure here is the one for /ɕ/ in \REF{ex:10:20b}, which I analyze as a complex corono-dorsal segment.

My claim that the place structure for [ɕ] and [ʝ] is the same derives support from the patterning of those two sounds after coronal sonorants. The examples in (\ref{ex:10:19c}--\ref{ex:10:19e}) show that [ʝ] in a word-internal onset surfaces as [ɕ] in coda position. Since \isi{Final Fortition} uncontroversially only alters a laryngeal feature, the implication is that [ʝ] and [ɕ] have the same place structure.

The fronting of velar (/x ɣ/) to a complex segment is accomplished with \REF{ex:10:21}. The set of triggers for the /x/ target consists of front vowels. As noted above, no examples are attested for /x/ after a coronal sonorant consonant, although the data in \REF{ex:10:18c} illustrate that liquids indirectly trigger the fronting of /ɣ/ by spreading [coronal] to \isi{schwa}, which \isi{feeds} \REF{ex:10:21}. \isi{Velar Fronting-1} -- together with \isi{Final Fortition} -- also accounts for derived corono-dorsal sounds ([ɕ ʝ]) in \REF{ex:10:19}.

\ea%21
\label{ex:10:21}\isi{Velar Fronting-1}:\\
\begin{forest}
[,phantom
 [\avm{[+son]} [\avm{[coronal]},tier=word,name=target]]
 [\avm{[−son\\+cont]},name=parent [\avm{[dorsal]},tier=word]]
]
\draw [dashed] (parent.south) -- (target.north);
\end{forest}
\z

\isi{Velar Fronting-1} creates a derived corono-dorsal fricative which is realized at the level of Speech as the \isi{sibilant} [ɕ] given the [+fortis] target segment /x/. The rule of \isi{phonetic implementation} is stated in \REF{ex:10:8a} above. By contrast, the palatal ([ʝ]) created from the lenis velar fricative /ɣ/ is not interpreted as a \isi{sibilant} (e.g. alveolopalatal [ʑ]) because \REF{ex:10:8a} only affects a [+fortis] sound. [ʝ] is also not affected by \REF{ex:10:8b}, which targets simplex coronal fricatives.

Recall from \REF{ex:10:5} that there were two historical progenitors of alveolopalatal [ɕ], namely palatal [ç] and postalveolar [ʃ]. Since [ç] are [ɕ] are the same segment phonologically, the change from the former to the latter simply involves only the change in \isi{phonetic implementation} rule \REF{ex:10:6a} to \REF{ex:10:7a} or \REF{ex:10:8a}; see below for discussion. The shift from [ʃ] to [ɕ] necessitates a restructuring of the former to the latter. It was proposed earlier that Stage B (pre-\ipi{Schlebusch}) /ʃ/ was complex [coronal, labial]. The alveolopalatalization of that /ʃ/ therefore required \REF{ex:10:22}, which entailed both the loss of [labial] as well as the addition of [dorsal]. The segments to the left and to the right of the wedge in \REF{ex:10:22} are interpreted as a sibilants by \isi{phonetic implementation} (=\ref{ex:10:7a} or \ref{ex:10:8a}).

\ea%22
 \label{ex:10:22}\isi{Delabialization}:\\
 \begin{forest} for tree = {fit=band}
 [,phantom
     [\avm{[−son\\+cont\\+fortis]} [\avm{[coronal]}] [\avm{[labial]}]]
     [>]
     [\avm{[−son\\+cont\\+fortis]} [\avm{[coronal]}] [\avm{[dorsal]}]]
 ]
 \end{forest}
\z

\isi{Delabialization} was a historical merger because the representation to the right of the wedge is identical to the representation created by \isi{Velar Fronting-1} with the target segment /x/ or with the target segment /ɣ/ in coda position. As stated in \REF{ex:10:22} the change was context-free; hence any [ʃ] (/ʃ/) was restructured to alveolopalatal. However, some evidence discussed below in \sectref{sec:10.4.2} indicates that there are alveolopalatalizing dialects in which \isi{Delabialization} occurs only in a specific context.

 In \REF{ex:10:24} I present six examples at the three historical stages in \REF{ex:10:4}. The first two words represent a [x]{\textasciitilde}[ɕ] alternating pair in which the two fricatives derived historically from an earlier velar. The third and fourth items exemplify words with [ɣ] after a back vowel and [ʝ] after a front vowel. The fifth and sixth items are words with [ɕ] (<[ʃ]) which do not have an alternant with [x]. Stage C represents \ipi{Schlebusch} as it was described in 1935. Stage B (Pre-\ipi{Schlebusch}) is the point before \isi{Delabialization} entered the language when \isi{Velar Fronting-1} was an allophonic rule for both /x/ and /ɣ/. I indicate in \REF{ex:10:23} with subscripts whether or not a fricative is simplex ([coronal] only) or complex (corono-labial) for [ʃ] or corono-dorsal (for [ɕ], [ʝ], or [ç]).

\ea%23
\label{ex:10:23}Representations for simplex/complex segments:\\
\begin{forest}
[,phantom
    [\avm{[−son\\+cont]} [\avm{[coronal]} [P, no edge,name=1]]]
    [\avm{[−son\\+cont]} [\avm{[coronal]} [Q, no edge, xshift=1.1cm,name=2]] [\avm{[labial]}]]
    [\avm{[−son\\+cont]} [\avm{[coronal]} [R, no edge, xshift=1.1cm,name=3]] [\avm{[dorsal]}]]
]
\foreach \i in {1,2,3} \draw [-{Triangle[]}] (\i) -- ++(0,1cm);
\end{forest}
\ex%24
\label{ex:10:24}
\begin{tabular}[t]{@{} *{7}{l} @{}}
\relax   /lɔx/          &    /løːx-ǝ/                           & /oːɣǝ/           & /beːɣ-ǝ/                           & /vrɔʃ\textsc{\textsubscript{p}}/  & /veʃ\textsc{\textsubscript{p}}/ &         \\
\relax  [lɔx]           &   [løːç\textsc{\textsubscript{r}}ǝ]   & [oːɣǝ]           & [beːʝ\textsc{\textsubscript{r}}ǝ]  & [vrɔʃ\textsc{\textsubscript{p}}]  & [veʃ\textsc{\textsubscript{p}}] &  Stage A\\\tablevspace
\relax  /lɔx/           &      /løːx-ǝ/                         & /oːɣǝ/           & /beːɣ-ǝ/                           & /vrɔʃ\textsc{\textsubscript{q}}/  & /veʃ\textsc{\textsubscript{q}}/ &         \\
\relax  [lɔx]           &   [løːɕ\textsc{\textsubscript{r}}ǝ]   & [oːɣǝ]           & [beːʝ\textsc{\textsubscript{r}}ǝ]  & [vrɔʃ\textsc{\textsubscript{q}}]  & [veʃ\textsc{\textsubscript{q}}] &  Stage B\\\tablevspace
\relax  /lɔx/           &      /løːx-ǝ/                         & /oːɣǝ/           & /beːɣ-ǝ/                           & /vrɔɕ\textsc{\textsubscript{r}}/  & /veɕ\textsc{\textsubscript{r}}/ &         \\
\relax  [lɔx]           &   [løːɕ\textsc{\textsubscript{r}}ǝ]   & [oːɣǝ]           & [beːʝ\textsc{\textsubscript{r}}ǝ]  & [vrɔɕ\textsc{\textsubscript{r}}]  & [veɕ\textsc{\textsubscript{r}}] &  Stage C\\\tablevspace
\relax   \textit{Loch}  &   \textit{Löcher}                     & \textit{Augen}   & \textit{biegen}                    & \textit{Frosch}                   & \textit{Fisch}                  &  \il{Standard German}StG \\
\relax   ‘hole’         &    ‘hole-\textsc{pl}’                            & ‘eye-\textsc{pl}’           & ‘bend-\textsc{inf}’                & ‘frog’                            &  ‘fish’                         &         \\
\end{tabular}
\z 

At Stage A, [ʃ\textsc{\textsubscript{p}}] and [ç\textsc{\textsubscript{r}}] were distinct in the phonological component: The former was a simplex coronal, while [ç] was a complex corono-dorsal segment, as in all of the dialects discussed in the present work with that sound. At Stage A, \isi{Velar Fronting-1} affected both /x/ and /ɣ/. Since the output ([ç\textsc{\textsubscript{r}}]/[ʝ\textsc{\textsubscript{r}}]) was not an underlying segment, \isi{Velar Fronting-1} was an allophonic rule. [ʃ\textsc{\textsubscript{p}}] is interpreted as a \isi{sibilant} by \REF{ex:10:6b}. At Stage B \isi{Velar Fronting-1} remains active as an allophonic rule in examples [beːʝ\textsc{\textsubscript{r}}ǝ] ‘bend-\textsc{inf}’ and [løːɕ\textsc{\textsubscript{r}}ǝ] ‘hole-\textsc{pl}’. [ɕ] in the latter word and [ʃ\textsc{\textsubscript{q}}] in words like [vrɔʃ\textsc{\textsubscript{q}}] ‘frog’ and [veʃ\textsc{\textsubscript{q}}] ‘fish’ are interpreted as sibilants by \REF{ex:10:7a}. At Stage C, \isi{Delabialization} altered the underlying representation of /ʃ\textsc{\textsubscript{q}}/ to /ɕ\textsc{\textsubscript{r}}/ in [vrɔɕ\textsc{\textsubscript{r}}] ‘frog’ and [veɕ\textsc{\textsubscript{r}}] ‘fish’; hence that new \isi{sibilant} merged with the [ɕ\textsc{\textsubscript{r}}] allophone from Stage C in [løːɕ\textsc{\textsubscript{r}}ǝ] ‘hole-\textsc{pl}’. At this point (\ipi{Schlebusch} in 1935), \isi{Velar Fronting-1} remained in the grammar by applying as a \isi{neutralization} to /x/ in alternating morphemes after a front vowel, e.g. in the second example in \REF{ex:10:24}. Stage C [ɕ\textsc{\textsubscript{r}}] is interpreted as a \isi{sibilant} by \REF{ex:10:8a}.\il{Ripuarian|)}

\subsection{Moselle Franconian (Luxembourgish)}\label{sec:10.3.2}\il{Moselle Franconian|(}\il{Luxembourgish|(}

\citet{Gilles1999} provides a detailed account of the phonetics and phonology of several varieties of Lxm (\mapref{map:10}). In terms of German dialectology, Lxm is classified as a variety of \il{Moselle Franconian}MFr (Appendix~\ref{appendix:a}). Lxm has the phonemic front vowels /i iː e eː ɛ/ and the phonemic back vowels /u uː o oː ɑ ɑː ǝ/ as well as diphthongs ending in a back vowel, i.e. /iǝ uǝ ɑu ɑːu ou/ and diphthongs ending in a front vowel, i.e. /ei ɑi ɛːi/. In the following discussion I concentrate on the realization of fortis dorsal fricatives, i.e. the change from historical [ç ʃ] to alveolopalatal [ɕ]. I do not discuss lenis fricatives ([ʝ ɣ]) which can have different realizations depending on the dialect. See \citet{Gilles1999} for discussion and \citet{Hall2014b} for a phonological treatment.\largerpage

\citet{Gilles1999} notes that traditional sources for Lxm invariably transcribe the historical dorsal fricatives as [x] after a back vowel and elsewhere (e.g. after front vowels) as [ç] -- as in \il{Standard German}StG -- and that these same sources likewise render etymological [ʃ] in modern Lxm as [ʃ]. An example of that type of source is LSA, which has [heiç] ‘high’ (Map 122) and [ʃɔn] ‘already’ (Map 121) throughout the central and southern parts of \ipi{Luxembourg}. According to the acoustic-phonetic investigation conducted by \citet{Gilles1999}, there is no evidence that the sound usually transcribed as [ç] is a palatal fricative. Instead, the author concludes that the historical palatal allophone of the phoneme /x/ has undergone an exceptionless, context-free shift to the sound he transcribes as [ɕ], which he calls “alveolar-palatalˮ. The change referred to here appears to be ongoing and subject to some speaker-specific variation. Gilles writes (p. 238): “In keiner der Aufnahmen wurde ein Beleg mit \textit{ç} transkribiert. Es findet sich ausschliesslich koronalisiertes \textit{ɕ}. Auch in der [sic.] Aufnahmen der älteren Generation konnte kein \textit{ç} gefunden werdenˮ. (“In none of the recordings was a token transcribed with \textit{ç}. Only a coronalized \textit{ɕ} was found. In the recordings of the older generation no instance of \textit{ç} could be found eitherˮ).\footnote{Gilles’s observation is corroborated by \citet[636]{Newton1993}, who writes that many speakers of Lxm have extreme difficulty in acquiring the ich-Laut, substituting instead an allophone approximating to the [ʃ], though realized without the labialization associated with \il{Standard German}StG.}

Examples illustrating the shift from palatal to alveolopalatal are presented in \REF{ex:10:25}. Gilles does not give the phonetic representations for these words, but in his phonetic investigation of these examples he determined that the fricative corresponding to \textit{ch} is alveolopalatal [ɕ] and not palatal [ç] or postalveolar [ʃ].\footnote{Gilles's evidence (\citeyear[237, 239--241]{Gilles1999}) is based on sonograms of the relevant fricatives. He concludes that [ɕ] is characterized by a higher tonality than [ʃ]. What is more, [ɕ] differs from [ç] in that the latter lacks a local maximum at the lowest frequency (see the spectrograms for [ʃ ɕ ç] in \citealt{Gilles1999}: 237).} The data in \REF{ex:10:25} and below are drawn from speakers from Central, South, and East Lxm. By contrast, North Lxm (\ipi{Nordösling}) displays a very different pattern (\sectref{sec:14.5}).

\ea%25
\label{ex:10:25}\isi{Alveolopalatalization} of Lxm [ç] (/x/) to [ɕ] (/x/):\\
  héich  \tab hoch    \tab ‘high’     \tab 239\\
  Kichen \tab  Küche  \tab ‘kitchen’  \tab 239\\
  Dicher \tab  Tücher \tab  ‘towel-\textsc{pl}’  \tab 239\\
  fiicht \tab  feucht \tab  ‘damp’    \tab 239\\
\z 

The examples in \REF{ex:10:25} are intended to show that all instances of the historical palatal fricative  participated in the shift to alveolopalatal.

It is clear from the discussion in \citet{Gilles1999} that the velar fricative [x] (/x/) is also present in the context after a back vowel, e.g. [nɑx] ‘still’. As in all other velar fronting dialects, alternations between [x] and [ɕ] are presumably present. For example, the third item in \REF{ex:10:25} has [ɕ] after the front vowel [i], but the corresponding fricative in the singular form (cf. \il{Standard German}StG \textit{Tuch}) has [x] because the preceding vowel is back.\largerpage

The acoustic measurements made by Gilles were intended not only to determine whether or not the dorsal fricatives in \REF{ex:10:25} are alveolopalatal (which they are), but also to consider the nature of the historical postalveolar \isi{sibilant} [ʃ]: Does this sound surface for the speakers of Lxm who have the alveolopalatal in \REF{ex:10:25} as [ʃ] or did the historical [ʃ] also undergo a change to [ɕ]? The results of Gilles’s investigations showed the latter result. Examples of words with the etymological postalveolar are presented in \REF{ex:10:26}. Gilles is clear that the fricative corresponding to \textit{s} or \textit{sch} in the examples in \REF{ex:10:26} is alveolopalatal [ɕ] and not postalveolar [ʃ]; hence, the historical postalveolar [ʃ] shifted to [ɕ] by \isi{Delabialization} in \REF{ex:10:22}.

\TabPositions{.15\textwidth, .33\textwidth, .55\textwidth, .75\textwidth}
\ea%26
\label{ex:10:26}\isi{Alveolopalatalization} of Lxm [ʃ] (/ʃ/) to [ɕ] (/ɕ/):
\ea\label{ex:10:26a} Spigel \tab  Spiegel \tab ‘mirror’ \tab 239\\
    stoen \tab  stehen \tab ‘stand\textsc{{}-inf}’ \tab  239\\
    schléit \tab  schlägt \tab ‘beat\textsc{{}-3sg}’ \tab 239\\
    schéin \tab  schön \tab ‘beautiful’ \tab 239
\ex\label{ex:10:26b} Dësch \tab  Tisch \tab ‘table’ \tab 239\\
    Biischt \tab  Bürste \tab ‘brush’ \tab 239\\
    éischten \tab  erster \tab ‘first\textsc{{}-masc.sg}’ \tab 239\\
    Fräsch \tab  Frosch \tab ‘frog’ \tab 239
    \z 
\z

From the synchronic perspective, Lxm exemplifies the Contrast Type B system depicted in \REF{ex:10:2} and possesses the representations for the fortis corono-dorsal fricatives presented in (\ref{ex:10:20}a-c). The [ɕ] in \REF{ex:10:25} derives synchronically from /x/ by \isi{Velar Fronting-1} and is interpreted as an alveolopalatal \isi{sibilant} by \REF{ex:10:8a}. The /ɕ/ in \REF{ex:10:26} is a [coronal, dorsal] fricative also targeted by \REF{ex:10:8a}.\il{Moselle Franconian|)}\il{Luxembourgish|)}

\subsection{Upper Saxon}\label{sec:10.3.3}\il{Upper Saxon|(}

\citet{Große1957} documents a series of sound changes which were occurring in the 1950s in colloquial speech in primarily urban varieties of \il{Upper Saxon}USax (\mapref{map:12}). He concentrates on the dialect as it is spoken in \ipi{Leipzig}, although Große observes that the facts are similar in other urban areas in the same region, e.g. \ipi{Dresden} and \ipi{Chemnitz} (formerly Karl-Marx-Stadt). Große does not give a list of the phonemic vowels, although it can be concluded from the data presented in that source that the dialect possesses the phonemic front (unrounded) vowels /i iː ɛ e eː æː/, the phonemic back vowels /u uː ɔ oː ɑ ɑː ə/, and the diphthongs /ɑe/ and /ɑo/.

\ipi{Leipzig} possesses one dorsal fricative ([x]) as well as the \isi{sibilant} fricative Große transcribes as ⟦χ´⟧), which he describes as a (fortis) fricative acoustically between (‘akustisch und schallphysiologisch zwischen’; p. 182) [ç] (=⟦χ⟧) and [ʃ] (=⟦š⟧). I transcribe ⟦χ´⟧ below as [ɕ], which Große (1957: 182) observes is articulated without \isi{lip rounding} or lip protrusion, as opposed to [š] (Große 1957: 182). The dialect has no [ɣ] or [ʝ] because those historical fricatives (from \ili{WGmc} \textsuperscript{+}[ɣ]) merged with the corresponding fortis sounds. Contrasts between [x] and [ɕ] in postsonorant position require that those sounds be phonemic, as depicted in \REF{ex:10:27}.  The only dorsal fricative surfacing in word-initial position is [ɕ].

\ea%27
\label{ex:10:27}
\begin{forest}
[,phantom
   [/x/ [{[x]}]]  [/ɕ/ [{[ɕ]}]]
]
\end{forest}
\z 

The postsonorant system in \REF{ex:10:27} exemplifies \REF{ex:10:2}.

Examples illustrating the occurrence of [ɕ] are presented below in the context after a front vowel in (\ref{ex:10:28a}) and after a coronal sonorant consonant in (\ref{ex:10:28b}). Example \REF{ex:10:28c} reveals the occurrence of velar [x] after a back vowel. [ɕ] and [x] in these examples derived historically from a velar (\ili{WGmc} \textsuperscript{+}[x] or \textsuperscript{+}[ɣ]).

\ea%28
\label{ex:10:28}\ipi{Leipzig} [ɕ] (/ɕ/) and [x] (/x/):
\ea\label{ex:10:28a} līχ´n     \tab [liːɕn̩] \tab  liegen \tab  ‘lie\textsc{{}-inf}’\tab  183\\
    niχ´      \tab [niɕ]    \tab nicht   \tab ‘not’                \tab 183\\
    wɑ̄χ´ \tab [væːɕ]   \tab Weg     \tab ‘path’               \tab 183
\ex\label{ex:10:28b} mōrχ´n    \tab [moːrɕn̩]\tab  morgen \tab  ‘tomorrow’           \tab 183\\
    fęlχ´n    \tab [fɛlɕn̩] \tab  Felgen \tab  ‘wheel rim-\textsc{pl}’         \tab 183
\ex\label{ex:10:28c} mɑxə      \tab [mɑxə]   \tab mache   \tab ‘do\textsc{{}-1sg}’  \tab 189
    \z
\z 

Since Große’s concern is the change from [ç ʃ] to [ɕ], he does not discuss the distribution of [x], although it can be inferred that there are many morphemes displaying an alternation between [ɕ] and [x] depending on the quality of the preceding vowel. For example, the fricative [ɕ] in [lɛɕɔr] (=⟦lęχ´ɔ\textsuperscript{r} ⟧) ‘hole-\textsc{pl}’ is [ɕ] after the front vowel, but the fricative in the singular noun \textit{Loch} (cf. \il{Standard German}StG [lɔx]) is presumably [x] because the preceding vowel is back. The [x] and [ɕ] in alternating pairs like that one derived historically from a velar (\ili{WGmc} \textsuperscript{+}[k]).

The examples in \REF{ex:10:29} illustrate the occurrence of [ɕ] after a front vowel in (\ref{ex:10:29a}), back vowel in (\ref{ex:10:29b}), noncoronal consonant in (\ref{ex:10:29c}), or word-initially in (\ref{ex:10:29d}). The [ɕ] in these examples derived from historical [ʃ] (/ʃ/) by \isi{Delabialization}.

\ea%29
\label{ex:10:29}\ipi{Leipzig} [ɕ] (/ɕ/):
\ea\label{ex:10:29a}ęŋliχ´  \tab [ɛŋliɕ] \tab  englisch \tab ‘English’ \tab  183
\ex\label{ex:10:29b}mɑχ´ə   \tab [mɑɕə]  \tab  Masche   \tab ‘mesh’    \tab 189
\ex\label{ex:10:29c}lębχ´   \tab [lɛpɕ]  \tab  läppisch \tab  ‘petty’  \tab 183
\ex\label{ex:10:29d}χ´leχ´d \tab [ɕleɕt] \tab  schlecht \tab ‘bad’     \tab 183
\z 
\z

The examples in \REF{ex:10:28} and \REF{ex:10:29} together exemplify \REF{ex:10:2}: [ɕ] surfaces after front vowels and back vowels, but [x] only occurs after back vowels. Note the minimal pair in \REF{ex:10:28c} vs. \REF{ex:10:29b}.

  In word-initial position [ɕ] surfaces before back vowels in (\ref{ex:10:30a}) or front vowels in (\ref{ex:10:30b}).

\ea%30
\label{ex:10:30}\ipi{Leipzig} [ɕ] (/ɕ/) in word-initial position:
\ea\label{ex:10:30a}  χ´ɑ̊̄\textsuperscript{r} \tab  [ɕɑːɐ]\tab  Jahr \tab  ‘year’\tab  182
\ex\label{ex:10:30b}   χ´ęds                  \tab [ɕɛts] \tab jetzt \tab ‘now’  \tab 182
\z 
\z 

The [ɕ] in \REF{ex:10:30} is different from all of the other examples discussed above (including the ones in \ipi{Schlebusch}). The reason is that the historical source for [ɕ] was neither [ʃ] nor the palatal ([ç]) created by velar fronting. Instead, the initial sound in \REF{ex:10:30} is the \isi{etymological palatal} (<\ili{WGmc}  \textsuperscript{+}[j] /j/), which underwent \isi{Glide Hardening} to [ʝ] (/ʝ/). When the fortis vs. lenis contrast among fricatives was neutralized that distinctive laryngeal feature was subsequently lost.

\ipi{Leipzig} displays Stage C: The original palatal allophone of /x/ ([ç]) merged with the historical postalveolar \isi{sibilant} [ʃ] (/ʃ/) to [ɕ] (/ʃ/). \citet[183]{Große1957} emphasizes that this is a true merger on the basis of word pairs like the ones in \REF{ex:10:31}, which are completely homophonous.

\ea
\label{ex:10:31}\ipi{Leipzig} merger of /x/ ([ç]) and [ʃ] (/ʃ/) to [ɕ] (/ʃ/):
\ea\label{ex:10:31a} diχ´ \tab [diɕ] \tab Tisch \tab ‘table’ \tab 183\\
    diχ´ \tab [diɕ] \tab dich \tab ‘you\textsc{{}-acc.sg}’ \tab 183
\ex\label{ex:10:31b} lęʳχ´ɔ\textsuperscript{r} \tab [lɛɕɔr] \tab  Löscher \tab ‘extinguisher’ \tab 183\\
    lęʳχ´ɔ\textsuperscript{r} \tab [lɛɕɔr] \tab Löcher \tab ‘hole-\textsc{pl}’ \tab 183
\ex\label{ex:10:31c} bręχ´n \tab  [brɛɕn̩] \tab brechen \tab ‘break\textsc{{}-inf}’ \tab 183\\
    bręχ´n \tab  [brɛɕn̩] \tab breschen \tab ‘breach\textsc{{}-inf}’ \tab 183
\ex\label{ex:10:31d} lɑon\textsuperscript{i}χ´ \tab [lɑoniɕ] \tab launig \tab ‘witty’ \tab 183\\
    lɑon\textsuperscript{i}χ´ \tab [lɑoniɕ] \tab launisch \tab ‘moody’ \tab 183
    \z
\z 

From the formal perspective, \ipi{Leipzig} has two phonemic dorsal fricatives (/x~ɕ/), which are represented as in \REF{ex:10:32}.

\ea%32
\label{ex:10:32}
\begin{multicols}{2}
\ea\label{ex:10:32a}
    \begin{forest}
      [/ɕ/\\\avm{[−son\\+cont]}, align=center [\avm{[coronal]}] [\avm{[dorsal]}]]
    \end{forest}
\ex\label{ex:10:32b} 
    \begin{forest}
      [/x/\\\avm{[−son\\+cont]},align=center [\avm{[dorsal]}]]
    \end{forest}
\z
\end{multicols} 
\z 

The alternations involving [x] and [ɕ] alluded to above are captured with underlying /x/, which shifts to a [coronal, dorsal] fricative {\textbar}ç{\textbar} by \isi{Velar Fronting-1}. That feature complex is interpreted as [ɕ] by \REF{ex:10:8a}.\footnote{{The difference between \REF{ex:10:32} and the ones in (\ref{ex:10:20b}, \ref{ex:10:20c}) for \ipi{Schlebusch} /ɕ x/ is the presence/absence of a distinctive laryngeal feature. As noted above, \ipi{Schlebusch} /x/ must be marked [+fortis] because it contrasts with [--fortis] /ɣ/. Since \ipi{Leipzig} fricatives do not display a laryngeal contrast, the two fricatives in that dialect lack specification for [±fortis]. The dialect possesses the rules of \isi{phonetic implementation} in \REF{ex:10:8}, although \REF{ex:10:8a} makes no reference to fortis fricatives.}}\il{Upper Saxon|)}

\section{{Stage} {B} {dialects}}\label{sec:10.4}

\subsection{Ripuarian (part 2)}\label{sec:10.4.1}\il{Ripuarian|(}

\citet{Heike1964} offers a phonetic study grounded in traditional phonemic theory of the Stage B variety spoken in \ipi{Cologne} (Köln; \mapref{map:8}). Note that \ipi{Cologne} is in the direct vicinity of Stage C \ipi{Schlebusch} (\sectref{sec:10.3.1}). Stage B is also implicit in the phonetic transcriptions provided in one of the dictionaries for \ipi{Cologne} German (KWb).\footnote{{Two other sources for Stage B can be mentioned here: (i) the phonetic study of \ipi{Gleuel} (\il{Ripuarian}Rpn; \mapref{map:8}) conducted by \citet{Heike1970} and (ii) the treatment of \ipi{Gabsheim} (\il{Rhenish Franconian}RFr; \mapref{map:10}) offered by \citet[40]{Post1987}.}}

The \ipi{Cologne} variety has large number of vocalic contrasts, which Heike analyzes as phonemic. Those segments consist of the front vowels /ɪ iː ʏ yː ɛ ɛː e eː œ œː ø øː/, the back vowels /ʊ uː ɔ ɔː o oː ɑ ɑː ə/, and the diphthongs /ei/, /øy/ and /ou/.\footnote{{Heike’s choice of symbols for the phonemic vowels and diphthongs is not exactly the same as my own. The differences between the two transcriptional systems are immaterial.} } Heike observes that his dialect possesses a fricative reflex of historical [ç], which I interpret as [ɕ] (=⟦£ʽ⟧), as well as [ʃ] (=⟦ʃ⟧). The author describes ⟦£ʽ⟧ a “… a more or less strongly palatalized [ʃ] ... and is articulated with unrounded lipsˮ. (“... ist ein mehr oder weniger stark palatalisiertes [ʃ] ... und wird mit entrundeten Lippen artikuliertˮ, p. 45).

In postsonorant position the \ipi{Cologne} dialect has two phonemic velar fricatives: /x/ and /ɣ/. As indicated in the postsonorant system depicted in \REF{ex:10:33} the former is realized as [x] (=⟦x⟧) or [ɕ] (=⟦£ʽ⟧) and the latter as [ɣ] (=⟦ɣ⟧) or [ʝ] (=⟦j⟧). [ɕ] only occurs after a front vowel and [x] after a back vowel. The distribution of [ɣ] and [ʝ] is essentially the same as their fortis counterparts, although palatal [ʝ] (/ʝ/) also occurs word-initially (as in \ipi{Schlebusch}). It is demonstrated below that [ʃ] never contrasts with [ɕ].

\ea%33
\label{ex:10:33}\begin{forest} for tree = {fit=band}
[,phantom
[/x/,calign=first [{[x]}] [{[ɕ]}]]   [/ʃ/ [{[ʃ]}]]    [/ɣ/,calign=first [{[ɣ]}] [{[ʝ]}]]
]           
\end{forest}
\z 

The system in \REF{ex:10:33} does not illustrate Contrast Type B in \REF{ex:10:2} because [ɕ] only occurs in the context after a front segment but not after a back vowel. Instead, \ipi{Cologne} exemplifies Stage B in (\ref{ex:10:5}): [ɕ] (<[ç]) and [ʃ] (<[ʃ]) have not yet merged together and are still distinct.\largerpage[-1]\pagebreak

Examples illustrating the occurrence of [ɕ] in the context after a front vowel are presented in \REF{ex:10:34a} and [x] after a back vowel in \REF{ex:10:34b}. [ɕ] derives historically from a velar (\ili{WGmc} \textsuperscript{+}[k]). The [ɣ]{\textasciitilde}[ʝ] alternation in \REF{ex:10:34c} exemplifies the allophonic relationship involving /ɣ/, which is realized as [ɣ] after a back vowel and as [ʝ] after a front vowel. I analyze [ɕ] in \REF{ex:10:34a} as an allophone of /x/ and [ʝ] in \REF{ex:10:34c} as an allophone of /ɣ/. It is not possible to provide a complete set of data with [ɕ ʝ] after every phonemic front vowel because Heike does not give them. In contexts other than after a front vowel, [ʃ] occurs. A representative example for word-initial position is presented in \REF{ex:10:34d}.\footnote{{\citet[46]{Heike1964} analyzes [g] (not depicted in \ref{ex:10:33}) and [ɣ] as allophones. I do not discuss the patterning of [g] because that topic is peripheral. Most of Heike’s examples are given in broad transcriptions in diagonal slashes representing phonemes (//). It is possible to reach conclusions on the distribution of the sounds in \REF{ex:10:33} on the basis of the author’s remarks on allophones and on the basis of his narrow transcriptions enclosed in square brackets. In contrast to my treatment, Heike analyzes [ʃ] and [ɕ] as allophones of the same phoneme because they never contrast.} }

\ea%34
\label{ex:10:34}\ipi{Cologne} [ɕ] (/x/):
\ea\label{ex:10:34a} ɪ£ʽ \tab  [ɪɕ] \tab ich \tab ‘I’ \tab 45\\
    mɪ£ʽ \tab [mɪɕ] \tab mich \tab ‘me-\textsc{acc}.\textsc{sg}’ \tab 45\\
    œːntlɪ£ʽ \tab [œːntlɪɕ] \tab ordentlich \tab ‘orderly’ \tab 46\\
    jǝzeː£ʽ \tab [ʝǝzeːɕ] \tab Gesicht \tab ‘face’ \tab 46\\
    bøː£ʽɒ \tab [bøːɕɒ] \tab Bücher \tab ‘book-\textsc{pl}’ \tab 46\\
    kʀeː£ʽpɔts \tab [kʀeːɕpɔts] \tab Griechenpforte \tab ‘(street name)’ \tab 112
\ex\label{ex:10:34b} ˙ba:x \tab [bɑːx] \tab Bach \tab ‘stream’ \tab 90
\ex\label{ex:10:34c} fUɣǝl \tab [fʊɣǝl] \tab Vogel \tab ‘bird’ \tab 50\\
    fγjǝl \tab [fʏʝǝl] \tab Vögel \tab ‘bird-\textsc{pl}’ \tab 50
\ex\label{ex:10:34d} ʃlai£ʽǝ \tab [ʃlaiɕǝ] \tab schleichen \tab ‘creep\textsc{{}-inf}’ \tab 84
    \z
\z 

It is clear from the discussion in the original source that [ʃ] and [ɕ] never contrast. In Heike’s own words: “Oppositionen zwischen ʃ and £ʽ … existieren nicht … ˮ. (“Oppositions between ⟦ʃ⟧ and ⟦£ʽ⟧ ... do not exist ... ˮ). For example, in the context after a front vowel, dialect speakers are unable to distinguish historical [ʃ] from [ɕ]. Recall from \REF{ex:10:31} that \ipi{Leipzig} has completely neutralized that contrast to [ɕ] in words like \textit{Löscher} ‘extinguisher’ (cf. \il{Standard German}StG [lœʃɐ]) vs. \textit{Löcher} (cf. \il{Standard German}StG [lœçɐ]) ‘hole-\textsc{pl}’. \citet[46]{Heike1964} observes that a similar generalization holds for \ipi{Cologne}, suggesting that \isi{Delabialization} occurred -- or is in the process of occurring -- although only in the context after a front vowel. The complementary distribution of [ʃ] and [ɕ] is also clear in the narrow transcription of two texts read by native dialect speakers (pp. 131--132): [ɕ] (=⟦£ʽ⟧) surfaces after front vowels and [ʃ] elsewhere. The [ɕ] in those examples derives historically from a velar.\footnote{In contrast to \citet{Große1957, Heike1964} does not say explicitly that words like \textit{Löscher} ‘extinguisher’ and \textit{Löcher} ‘hole-\textsc{pl}’ are homophonous, only that dialect speakers cannot distinguish the fricatives in question.}

From the formal perspective, the velars /x ɣ/ are represented as in (\ref{ex:10:20}c,d) and /ʃ/ as [coronal, labial]. Both /x ɣ/ serve as targets for \isi{Velar Fronting-1}. In the case of target /ɣ/ \isi{Velar Fronting-1} creates {\textbar}ʝ{\textbar}, which surfaces as the nonsibilant [ʝ] in a word-internal onset, e.g. [fʏ.ʝǝl] ‘bird-\textsc{pl}’ from \REF{ex:10:34c}. In the case of target /x/ the same process produces a complex corono-dorsal segment which is interpreted as the \isi{sibilant} [ɕ] by \REF{ex:10:7a}.\il{Ripuarian|)}

\subsection{West Central German}\label{sec:10.4.2}

\citet{Féry2017} provides the results of a phonetic investigation involving alveolopalatalization in the speech of four speakers of WCG dialects. Three of the four speakers are from \ipi{Frankfurt am Main} (\il{Central Hessian}CHes; \mapref{map:11}), and the fourth is from \ipi{Montabaur}  (\il{Moselle Franconian}MFr; \mapref{map:10}). I do not provide a list of the phonemic vowels because they are not made explicit in the original source.

In Féry’s experiment the four speakers were asked to read sentences which included a selection of words containing \il{Standard German}StG [ç] and [ʃ]. The result showed that there is a strong tendency to replace [ç] and [ʃ] with [ɕ], although there was not a complete \isi{neutralization} indicative of Stage C dialects. Instead, the alveolopalatalization of [ç ʃ] to [ɕ] (also Féry’s symbols) led to a system in which both [ʃ] and [ɕ] are present. In contrast to all of the dialects discussed above the experiment also indicates the change from historical [ç] to [ʃ]. The results of the experiment are summarized in \REF{ex:10:35} and \REF{ex:10:36}. Some of the items listed there are loanwords not discussed in the case studies in Chapters~\ref{sec:3}--\ref{sec:9}. Féry does not provide full phonetic transcriptions for the German examples; the type of vowels and consonants referred to in the categories listed below can be inferred from the orthography. The realization of [ʃ] as [ɕ] in \REF{ex:10:35d} and of [ç] as [ʃ] in (\ref{ex:10:36a}, \ref{ex:10:36b}) indicate a tendency and not Neogrammarian sound change. By contrast, the change from [ç] to [ɕ] in \REF{ex:10:36c} is a regular development. I have simplified the categories presented in \citet{Féry2017} in \REF{ex:10:35} and \REF{ex:10:36}, although those changes are immaterial. \citet{Féry2017} notes that all of her speakers retain [x] in the context after a back vowel, e.g. \textit{noch} ‘still’, \textit{Kuchen} ‘cake’.\largerpage[-1]

\ea%35
\label{ex:10:35}Reflexes of historical [ʃ] in \ipi{Frankfurt am Main}/\ipi{Montabaur}:
\ea\label{ex:10:35a}\relax [ʃ] in syllable-initial position before a back vowel or consonant:\\
    schon ‘already’, Schuhe ‘shoe-\textsc{pl}’, Schnee ‘snow’
\ex\label{ex:10:35b}\relax [ʃ] after a back vowel:\\
rasch ‘quick’, Sushi ‘sushi’
\ex\label{ex:10:35c}\relax [ʃ] after a (front or back) rounded vowel optionally separated by a consonant:\\
Kusch ‘shoo!’, Bosch ‘(name)’, Lösch ‘delete\textsc{{}-imp.sg}’, hübsch ‘pretty’
\ex\label{ex:10:35d}\relax [ɕ] after a front unrounded vowel:\\
Fisch ‘fish’, Tisch ‘table’, Fleisch ‘meat’
    \z
\ex%36
\label{ex:10:36}Reflexes of historical [ç]:
\ea\label{ex:10:36a}\relax [ʃ] in syllable-initial position before a front vowel:\\
China ‘China’, Chemie ‘chemistry’
\ex\label{ex:10:36b}\relax [ʃ] in coda position after a consonant:\\
Dolch ‘dagger’, Mönch ‘monk’, durch ‘through’
\ex\label{ex:10:36c}\relax [ɕ] after a front unrounded vowel optionally separated by a consonant:\\
ich ‘I’, Blech ‘tin’, echt ‘genuine’, Milch ‘milk’
    \z
\z 

The results of the experiment illustrate the merger of historical [ç ʃ] to [ɕ], but only in the context after a front unrounded vowel. In contrast to all of the other dialects discussed in this chapter, the data presented above also reveal the change from [ç] to [ʃ] in word-initial position before a front vowel or after a consonant.

From the synchronic perspective, the \il{Central Hessian}CHes/\il{Rhenish Franconian}RFr speakers described above have the system of fortis fricatives as in \REF{ex:10:37}. That system captures both postsonorant and word-initial position.

\ea%37
\label{ex:10:37}
\begin{forest}
[,phantom
   [/x/,calign=first [{[x]}] [{[ɕ]},name=mid]]  [/ʃ/,name=parent2 [{[ʃ]}]]
]
\draw (parent2.south) -- (mid.north);
\end{forest}
\z 

[x] and [ɕ] never contrast because the former only surfaces after a back vowel and the latter only after a front unrounded vowel; hence, \REF{ex:10:37} does not reflect \REF{ex:10:2}. The [ɕ] in examples like the ones in \REF{ex:10:36c} is an allophone of /x/ which is realized as [coronal, dorsal] by \isi{Velar Fronting-1} and is interpreted as the \isi{sibilant} [ɕ] by \REF{ex:10:7a}.\footnote{{Based on \REF{ex:10:35d} and \REF{ex:10:36c} it appears that [ɕ] is restricted in its occurrence to the context after a front unrounded vowel. No example was found in the original source in which [ɕ] surfaces after a rounded vowel (e.g. \il{Standard German}StG} \textrm{\textit{Löcher}} \textrm{‘hole-\textsc{pl}’); hence, one cannot know for sure whether or not the set of triggers for velar fronting consists solely of front unrounded vowels.} } Postalveolar /ʃ/ is clearly phonemic; note that [ʃ] and [x] contrast after a back vowel, e.g. \textit{noch} ‘still’ with [x] vs. \textit{Bosch} ‘(name)’ with [ʃ]. In contrast to all of the other dialects discussed in this chapter, \isi{Delabialization} as stated in \REF{ex:10:22} does not apply. Instead, that change is restricted to the context after front unrounded vowels. Thus, /ʃ/ is realized as [ɕ] in words like \textit{Fisch} ‘fish’, but otherwise surfaces without change as [ʃ]. See \citet{Féry2017} for an analysis of that change. From the historical perspective [ç] changed to [ʃ] in the contexts specified in (\ref{ex:10:36a}, \ref{ex:10:36b}).

\section{Areal distribution of alveolopalatalization}\label{sec:10.5}\largerpage
\tabref{tab:10:1} provides a list of the alveolopalatalizing varieties of German discussed earlier, but I also include a number of others. All of these places are indicated on the maps for the dialect listed in the second column. \ipi{Sępóno Krajeńskie} (in the final box in that table) can be found on \mapref{map:18}. Many of the sources listed here have been cited in the earlier literature on alveolopalatalization (in particular \citealt{Herrgen1986}). I do not indicate which of the stages from \sectref{sec:10.2} are attested in which variety because that information is not always clear from the source.\footnote{A few of the works listed in \tabref{tab:10:1} make only passing reference to alveolopalatalization. For example, \citet[8]{Freiling1929} observes that the articulation in question (constriction between the alveolar ridge and the hard palate) is typical for \ipi{Bad König}, which is about 4km from Zell am Mūmlingtal. (Freiling’s data discussed in \sectref{sec:9.3} from the latter place do not contain alveolopalatal segments).} The sources given here are placed into four separate boxes corresponding to dialect area.

\begin{longtable}{>{\raggedright}p{.4\textwidth}l>{\raggedright\arraybackslash}p{.3\textwidth}}
\caption{Alveolopalatalizing varieties of HG and LG\label{tab:10:1}}\\
\lsptoprule Place & Dialect & Source\\\midrule\endfirsthead
\midrule Place & Dialect & Source\\\midrule\endhead
\endfoot\lspbottomrule\endlastfoot
\ipi{Mainz} & \il{Rhenish Franconian}RFr & \citet{Reis1892}\\
\ipi{Ludwigshafen am Rhein} & \il{Rhenish Franconian}RFr & \citet{Krell1927}\\
\ipi{Saarbrücken} & \il{Rhenish Franconian}RFr & \citet{Kuntze1932}, \citet{Steitz1981}\\
\ipi{Bad König} & \il{Rhenish Franconian}RFr & \citet{Freiling1929}\\
\ipi{Plankstadt} & \il{Rhenish Franconian}RFr & \citet{Treiber1931}\\
\ipi{Speyer} & \il{Rhenish Franconian}RFr & \citet{Waibel1932}\\
\ipi{Pfungstadt} & \il{Rhenish Franconian}RFr & \citet{Grund1935}\\
\ipi{Nußdorf} & \il{Rhenish Franconian}RFr & \citet{Bertram1937}\\
\ipi{Eberbach} & \il{Rhenish Franconian}RFr & \citet{Kilian1951}\\
\ipi{South Odenwald}/\ipi{Ried} & \il{Rhenish Franconian}RFr & \citet{Bauer1957}\\
\ipi{Darmstadt} & \il{Rhenish Franconian}RFr & \citet{Keller1961}\\
\ipi{Oftersheim} & \il{Rhenish Franconian}RFr & \citet{Liébray1969}\\
\ipi{Zweibrücken} & \il{Rhenish Franconian}RFr & \citet{Castleman1975}\\
\ipi{Wackernheim}, \ipi{Nackenheim}, \ipi{Alzey}, \ipi{Wallertheim}, Bechtheim & \il{Rhenish Franconian}RFr & \citet{Karch1981}\\
\ipi{Gabsheim} & \il{Rhenish Franconian}RFr & \citet{Post1987}\\
\ipi{Michelstadt} & \il{Rhenish Franconian}RFr & \citet{DurrellDavies1989}\\

\ipi{Birkenfeld} & \il{Moselle Franconian}MFr & \citet{Baldes1896} \\
\ipi{Kenn} & \il{Moselle Franconian}MFr & \citet{Thomé1908}\\
Kreis \ipi{Ottweiler} & \il{Moselle Franconian}MFr & \citet{Scholl1912}\\
\ipi{Arzbach} & \il{Moselle Franconian}MFr & \citet{Bach1921}\\
\ipi{Burg-Reuland}& \il{Moselle Franconian}MFr & \citet{Hecker1972}\\
\ipi{Bell} & \il{Moselle Franconian}MFr & \citet{Mattheier1987}\\
\ipi{Horath} (Hunsrück) & \il{Moselle Franconian}MFr & \citet{Reuter1989}\\
Beuren\ip{Beuren (Trier)} & \il{Moselle Franconian}MFr & \citet{Peetz1989}\\
\ipi{Luxembourg} & \il{Moselle Franconian}MFr & \citet{Gilles1999}\\
\ipi{Montabaur} & \il{Moselle Franconian}MFr & \citet{Féry2017} \\

\ipi{Cologne} & \il{Ripuarian}Rpn & \citet{Wahlenberg1877}\\
Area north of \ipi{Aachen} & \il{Ripuarian}Rpn & \citet{Schmitz1893}\\
\ipi{Schlebusch} & \il{Ripuarian}Rpn & \citet{Bubner1935}\\
\ipi{Aachen} & \il{Ripuarian}Rpn & \citet{Welter1938}\\
\ipi{Cologne} & \il{Ripuarian}Rpn & \citet{Heike1964}\\
\ipi{Gleuel} & \il{Ripuarian}Rpn & \citet{Heike1970}\\
Elsenborn & \il{Ripuarian}Rpn & \citet{Hecker1972}\\
\ipi{Burscheid} & \il{Ripuarian}Rpn & \citet{Heinrichs1978}\\
\ipi{Krefeld} & \il{Ripuarian}Rpn & \citet{Bister-Broosen1989}\\
\ipi{Erp} (Erftstadt) & \il{Ripuarian}Rpn & \citet{Kreymann1994}\\
\ipi{Niederbachem}, \ipi{Oberbachem} & \il{Ripuarian}Rpn & \citet{Fuss2001}\\

\ipi{Frankfurt am Main} & \il{Central Hessian}CHes & \citet{Rauh1921}, \citet{BethgeBonnin1969}, \citet{Féry2017}\\
\ipi{Petersberg} (Fuda) & \il{East Hessian}EHes & \citet{Schwarz1992}\\

Kreis \ipi{Rosenberg} & \il{High Prussian}HPr & \citet{Kuck1933}\\
In and around \ipi{Chemnitz} & \il{Upper Saxon}USax & \citet{Große1955}\\
\ipi{Leipzig} & \il{Upper Saxon}USax & \citet{Große1957} \\
\ipi{Vorerzgebirge} & \il{Upper Saxon}USax & \citet{Bergmann1965}\\
Kreis \ipi{Oschatz} & \il{Upper Saxon}USax & \citet{BethgeBonnin1969}\\
\ipi{Chemnitz} & \il{Upper Saxon}USax & \citet{KahnWeise2013}\\
\ipi{Gera} & \il{Thuringian}Thrn & \citet{Dietrich1957}\\
\ipi{East Thuringia} & \il{Thuringian}Thrn & \citet{Spangenberg1974,Spangenberg1989}\\
\ipi{Berlin} & \il{North Upper Saxon-South Markish}NUSax-SMk & \citet{Schönfeld2001}\\

Aschafftal & \il{East Franconian}EFr & \citet{Hirsch1971}\\
Barr & \il{Low Alemannic}LAlmc & \citet{Keller1961}\\
\ipi{Benfeld} & \il{Low Alemannic}LAlmc & \citet{Rünneburger1985}\\
\ipi{Colmar} & \il{Low Alemannic}LAlmc & \citet{PhilippBothorel-Witz1989}\\


\ipi{Sępóno Krajeńskie} & \il{East Pomeranian}EPo & \citet{Darski1973}\\
\end{longtable}

Data from linguistic atlases complement my own findings on the areal distribution of alveolopalatalization in \tabref{tab:10:1}. In particular, the following three atlases reveal that alveolopalatalization is the norm throughout the \il{Ripuarian}Rpn/\il{Moselle Franconian}MFr/\il{Rhenish Franconian}RFr dialect areas: (a) MRhSA for \il{Moselle Franconian}MFr/\il{Rhenish Franconian}RFr; (b) SNBW for the northwest corner of the German state of \ipi{Baden} Württemberg between Mannheim and Heidelberg (\il{Rhenish Franconian}RFr); and (c) SUF for Northwest Bavaria in the general vicinity of Aschaffenburg (\il{Rhenish Franconian}RFr).{\interfootnotelinepenalty=10000\footnote{Maps 8 and 11 in WSAH document alveolopalatalization in parts of the German state of Hesse between Gießen and \ipi{Darmstadt} (\il{Central Hessian}CHes and \il{Rhenish Franconian}RFr). Other linguistic atlases reveal that there are parts of WCG with very little alveolopalatalization, e.g. ALLG. The maps in that source (e.g. Map 269 for \textit{Milch} ‘milk’) show that alveolopalatalization (=⟦š⟧) is the exception rather than the rule in German Lorraine.}\footnote{In several dialect dictionaries alveolopalatalization is either commented on in the pronunciation guide and/or expressed directly in the spelling \textit{sch} (for etymological [ç]). Examples for \il{Ripuarian}Rpn include AaWb, DrWb, KWb, TrWb, WbKM. \il{Rhenish Franconian}RFr is represented by SaWb. \isi{Alveolopalatalization} is also evident from the phonetically transcribed texts in towns and villages throughout the \il{Ripuarian}Rpn/\il{Moselle Franconian}MFr dialect areas in \citet{CornelissenEtAl1989}. Several places from that source in the \il{Ripuarian}Rpn dialect region are indicated on \mapref{map:8}.}}\largerpage

The most important conclusion to be drawn from \tabref{tab:10:1} is precisely what \citet{Herrgen1986} determined over thirty years ago: \isi{Alveolopalatalization} is feature of CG. That assessment is illustrated visually in \mapref{map:16}, which indicates that alveolopalatalizing varieties (black squares) predominate in CG areas. Also indicated on \mapref{map:16} are the CG varieties listed in Appendix~\ref{appendix:c} which make no reference to alveolopalatalization (white squares).

\begin{map}
% \includegraphics[width=\textwidth]{figures/VelarFrontingHall2021-img022.png}
\includegraphics[width=\textwidth]{figures/Map16_10.1.pdf}
\caption[Areal distribution of alveolopalatalization]{Areal distribution of alveolopalatalization. High German (Central German and Low Alemannic) varieties (and one variety of Low German) with alveolopalatalization are indicated with black squares. Varieties of Central German without alveolopalatalization are indicated with white squares.}\label{map:16}
\end{map}

On the basis of \tabref{tab:10:1} and \mapref{map:16} five generalizations can be made: (A) \isi{Alveolopalatalization} is much more robustly attested in WCG than in ECG; (B) within WCG, alveolopalatalization is considerably more common in CFr (\il{Ripuarian}Rpn\slash\il{Moselle Franconian}MFr) and \il{Rhenish Franconian}RFr than in \il{North Hessian}NHes\slash\il{Central Hessian}CHes/\il{East Hessian}EHes; (C) within ECG, alveolopalatalization is typical of \il{Upper Saxon}USax and (East) \il{Thuringian}Thrn but not at all for \il{Silesian}Sln; (D) even in the CG regions where alveolopalatalization is most prevalent, there are still conservative places which retain the original palatal fricative [ç]; and (E) alveolopalatalization is also attested in a few places outside of the CG dialect region (i.e. one attestation for \il{High Prussian}HPr and three for \il{Low Alemannic}LAlmc). I consider below (D) in more detail. (E) is discussed in \sectref{sec:10.6}.\pagebreak

\begin{sloppypar}
Generalization (D) is shown on \mapref{map:16} by the presence of many white squares. (D) can also be illustrated by focusing on specific alveolopalatalizing areas. Consider \il{Rhenish Franconian}RFr (\mapref{map:10}). As indicated on that map, many of the sixteen alveolopalatalizing \il{Rhenish Franconian}RFr places from \tabref{tab:10:1} are situated within close proximity. However, the other sources for \il{Rhenish Franconian}RFr indicated on \mapref{map:10} do not document alveolopalatalization, e.g. \citet[4]{Heeger1896}, \citet[67]{Wanner1908}, \citet[44]{Wenz1911}, \citet[9, 74]{Reichert1914}, and \citet[57--58]{Seibt1930} to name a few. A similar finding for \il{Ripuarian}Rpn is discussed in \citet[398--399]{Cornelissen2000}, who provides a map of alveolopalatalizing and non-alveolopalatalizing towns in the area between \il{Ripuarian}Rpn and LFr (recall \mapref{map:8}). Map 349 for \textit{Kirche} ‘church’ in volume 4 of MRhSA similarly depicts a number of places with [ç] surrounded by places with the alveolopalatal.
\end{sloppypar}

Although a number of conclusions concerning alveolopalatalization can be drawn from \tabref{tab:10:1} and \mapref{map:16}, there is an additional factor that has unfortunately not been taken into consideration, namely the time dimension. The point is that it is possible for a dialect in a particular place to be be non-al\-ve\-o\-lo\-pal\-a\-tal\-i\-zing at one point in time but as alveolopalatalizing at a later point. Consider the following two examples:

\citet{Jardon1891} discussed the \il{Ripuarian}Rpn dialect spoken in and around \ipi{Aachen} at the end of the nineteenth century and gave no indication in his book for alveolopalatalization. Forty-seven years later \citet{Welter1938} also described the \ipi{Aachen} dialect, but he consistently transcribed the fortis palatal fricative [ç] with ⟦š⟧, suggesting that the various stages of alveolopalatalization posited above had been complete at the time he conducted his fieldwork. Welter’s observations concerning the realization of [ç] has also been documented in the 1970 dictionary for the \ipi{Aachen} dialect (AaWb), p. XL, XLI. A similar conclusion can be drawn for descriptions of the \ipi{Saarbrücken} dialect: \citet{Kuntze1932} transcribed the historical palatal fricative with the traditional symbol ⟦χ⟧ and only mentioned in passing (p. 94) that ⟦χ⟧ is often replaced with ⟦š⟧. Forty-nine years later, \citet{Steitz1981} transcribed historical [ç] -- and historical [ʃ] -- consistently as ⟦ʃ⟧ in his description of the \ipi{Saarbrücken} dialect, but he made no mention at all of the earlier pronunciation with [ç]. In the pronunciation guide of the 1984 dictionary for the \ipi{Saarbrücken} dialect (SbWb) the distinction between \textit{ch} ([ç]) and \textit{sch} ([ʃ]) is likewise completely neutralized to \textit{sch} ([ʃ]), e.g. \textit{Fisch} ‘fish’ and \textit{Biescher} ‘book-\textsc{pl}’ (pp. 11--22). No mention is made in SbWb of the ich-Laut.\footnote{On the other hand, alveolopalatalization is attested early in other places, e.g. \citet[11]{Rauh1921} is explicit that \ipi{Frankfurt am Main} already had it in 1921, long before \citegen{Féry2017} study confirmed that finding for a later generation of speakers. See \sectref{sec:10.6.1} for even earlier attestations of alveolopalatalization in other cities.}\pagebreak

\begin{sloppypar}
These examples confirm the conclusion already made by \citet{Große1957} for \il{Upper Saxon}USax: \isi{Alveolopalatalization} is an example of change in progress. The shortcoming of \mapref{map:16} is that the status the of the non-alveolopalatalizing places (white squares) is subject to change through time. Those markers depict places that were described without alveolopalatalization many years ago, but a closer examination of those same places today may reveal that the change from [ç]  to [ɕ] has already taken place.
\end{sloppypar}

\section{Discussion}\label{sec:10.6}

The present section considers three topics alluded to earlier, namely the origin and spread of alveolopalatalization (\sectref{sec:10.6.1}), the realization of the lenis nonsibilant fricative [ʝ] (\sectref{sec:10.6.2}), and changes involving underlying representations (\sectref{sec:10.6.3}).

\subsection{Origin and spread of alveolopalatalization}\label{sec:10.6.1}
\begin{sloppypar}
\isi{Alveolopalatalization} is a relatively recent phenomenon with its first attestations in the second half of the nineteenth century (\citealt{Herrgen1986}: 97ff.). To the best of my knowledge the earliest sources referring to the phenomenon are \citet[21]{Wahlenberg1877} for \ipi{Cologne}, \citet[281]{Trautmann1884} for the area south of \ipi{Leipzig}, \citet{Reis1892} for \ipi{Mainz}, and \citet[150]{Schmitz1893} for the area north of \ipi{Aachen}. \citet[281]{Wahlenberg1877} writes of the pronunciation of ch:
\end{sloppypar}

\begin{quote}
  ...mit harter, gutturaler Aussprache, nach a o ǫ u au und mit weicher, pala\-ta\-ler und dem sch liegender Aussprache nach e ę i ö ǫ̈ ü ei äu ...\smallskip\\
  “ ... with [a] hard, guttural pronunciation after a o ǫ u au and with [a] soft, palatal pronunciation close to [that of] sch after e ę i ö ǫ̈ ü ei äu ...ˮ
\end{quote}

A scrutiny of the literature cited throughout this chapter on alveolopalatalization reveals that the emergence of [ɕ] did not simply occur at one particular time and place (\isi{monogenesis}), but that it instead transpired at different places -- typically urban areas -- within the CG dialect area and at different times for any given area (\isi{polygenesis}). The dialects referred to here can therefore be thought of as \textsc{alveolopalatalizing} \textsc{islands}. To cite one of the sources cited above, \citet{Reis1892} observes the change to [ɕ]  (=⟦sch⟧) in the late nineteenth century pronunciation of \ipi{Mainz} German (\mapref{map:10}), but some of the related \il{Moselle Franconian}MFr varieties indicated on \mapref{map:10} in the neighborhood of \ipi{Mainz} (written during the same general time frame) made no mention of alveolopalatalization. Recall from the previous section that the \il{Rhenish Franconian}RFr varieties in \tabref{tab:10:1} are surrounded by other \il{Rhenish Franconian}RFr varieties without alveolopalatalization.

The clearest case of an \isi{alveolopalatalizing island} in Europe is the variety of \il{High Prussian}HPr described by \citet{Kuck1933}, which was once spoken in Kreis \ipi{Rosenberg} in West Prussia (\mapref{map:18}). In particular, \citet[148]{Kuck1933} observes that the fortis palatal fricative [ç] (=⟦χ⟧) is pronounced as ⟦š⟧, especially among young speakers. Significantly, Kreis \ipi{Rosenberg} appears to be unique for its area because alveolopalatalization has not been documented for other varieties of German once spoken in that general region (\chapref{sec:11}).

There are two additional examples of \isi{alveolopalatalizing islands} listed in \tabref{tab:10:1}. The first is the only LG variety known to me with alveolopalatalization (\ipi{Sępóno Krajeńskie}). The original source for that place \citep{Darski1973} consistently transcribes the modern reflex of historical [x] as [x] after back vowels and as [ɕ] after front vowels, e.g. [hɛʊx] ‘high’ vs. [rɛɕt] ‘right’. The second example is a cluster of three places (\il{Low Alemannic}LAlmc) in Alsace, namely Barr, \ipi{Benfeld}, and \ipi{Colmar} (\mapref{map:1}). The sources listed earlier for those three varieties are clear that alveolopalatalization is under way, especially among the younger generation of speakers.  That alveolopalatalization for Alsace is exceptional is clear from an examination of the maps in ALA. For example, Map 217 shows the realization of the \isi{etymological palatal} [ç] as alveolopalatal (⟦š⟧) for the word \textit{Hecht} ‘pike’ is restricted to a small area (Sainte-Marie-aux-Mines) to the northwest of \ipi{Colmar}.

Alveolopalatalization is not typical for the German-language islands discussed in this book, but two cases are known to me of alveolopalatalization within German-language islands in the United States. 

The first is the LG variety spoken in \ipi{Concordia}, Missouri (USA) described by \citet{Ballew1997}. According to that source (\citealt[57]{Ballew1997}), Concordia German has both [ç] and [x], but the former fricative tends to be indistinguishable from [ʃ] in the context after high front vowels. The two examples cited are \textit{Geschichte} ‘history’ (/kəʃɪçtə/ or /kəʃɪʃtə/) and \textit{durstig} ‘thirsty (/dɛstɪç/ or /dɛstɪʃ/). In the absence of phonetic evidence, it is not possible to know if Ballew’s [ʃ] is [ʃ] or [ɕ], but the important point is that this is an alveolopalatalizing island (restricted to the post-high front vowel context) in a German-language island geographically far removed from its point of origin. 

The second case is \ili{Texas German} (\citealt{Boas2009}). The earliest work on that dialect was an unpublished dissertation by Fred Eikel from 1954, which I was unable to find (see \citealt{PierceBoasGilbert2018} for discussion). However, a description of the phonology of that dialect was published twelve years later (\citealt{Eikel1966}). It is clear from \citet{Eikel1966} that Texas German -- referred to more specifically in that source as New Braunfels German -- had [x] and [ç], which were distributed as in StG (\citealt[258--260]{Eikel1966}). Eikel also observed that the lenis fricatives [ɣ ʝ] had a parallel distribution; hence, in present terms Texas German in 1966 and before had \isi{Velar Fronting-1} but no alveolopalatalization. The data from \citet{Eikel1966} are corroborated in many of the maps in the linguistic atlas for this area (LATG), which appeared in 1972. Significantly, in the introduction to that atlas the author (Glenn G. Gilbert) writes (p. 2): “For many speakers, [ç] and [ʃ] coalesce in all positions." That coalescence is expressed formally with Gilbert's phonological rule (6), which converts [ç] into a (nonanterior) sibilant fricative. The important point is that those speakers with coalescence retained Velar Fronting-1 with alveolopalatalization (Stage C).\footnote{In his earlier work on the German dialect of Kendall and Gillespie counties, Gilbert did not mention the coalescence referred to above (\citealt{Gilbert1963, Gilbert1964, Gilbert1970}). The palatal [ç] was also documented many years later (\citealt[115]{Roesch2012}) in a variety of Texas German which historically has no velar fronting (\ili{Texas Alsatian}). The case of Texas German can be contrasted with the most well-known and well-researched German-language island in the United States (\ili{Pennsylvania German}), where [ç] and [x] occur as positional variants without any sign of alveolopalatalization. See \citet[277]{Reed1947}, who describes a pattern for [ç] and [x] analogous to StG. Additional references include \citet[4]{Frey1942}, \citet[7]{Buffington1954}, and \citet[78--79; 91--93]{Kelz1971}. }

The investigation of alveolopalatalization from the sociolinguistic perspective also points to \isi{alveolopalatalizing islands}. See, for example, \citet[25ff.]{Auer2002}, \citet[38]{Wiese2012}, and \citet{JannedyWeirich2014} on the realization of [ʃ ç] as [ɕ] in various \is{ethnolect}ethnolects spoken in \ipi{Berlin}.

The upshot of all of the studies cited above is the following: \isi{Alveolopalatalization} did not occur in a single place and from there spread outwards in terms of space (and time). Instead, the evidence suggests that \isi{polygenesis} is the correct interpretation.

It has been asserted repeatedly in the literature that alveolopalatalization is an intergenerational change (\sectref{sec:2.5}). For example, \citet{Kuck1933} observes that alveolopalatalization in Kreis \ipi{Rosenberg} was initiated by young speakers. That alveolopalatalization involves intergenerational change is especially prominent in descriptions of \il{Upper Saxon}USax and \il{Thuringian}Thrn. For example, in his study of the \il{Thuringian}Thrn dialect spoken in \ipi{Gera} (\mapref{map:12}), \citet[61]{Dietrich1957} notes that [ç] shows the effects of alveolopalatalization among younger speakers (especially female). In one of his study on \il{Upper Saxon}USax, Große (1955: 49) writes: “Man kann sagen, daß die älteste Generation nur ganz selten, die mittlere occassionell, die jüngere schon mit vielen Vertretern usuell χ‘ artikuliert”. (“One can say that the oldest generation articulates χ (=[ç], T.A.H.) only rarely, the middle generation occasionally, and the younger generation quite often (lit. “with many representatives”)”.)

Although it is not possible to conclude from the works cited in \tabref{tab:10:1} that alveolopalatalization involves more than one stage, this conclusion can be reached on the basis of the CG dialects described in \sectref{sec:10.3} and \sectref{sec:10.4}. Those studies suggest that alveolopalatalization affected first the palatal allophone [ç] produced by velar fronting (Stage B) and only later [ʃ] (Stage C). As noted above, the data from \ipi{Frankfurt am Main}/\ipi{Montabaur} suggest that \isi{Delabialization} (/ʃ/ > /ɕ/) did not simply restructure every instance of /ʃ/ to /ɕ/ in one fell-swoop in a context-free fashion. Instead, the data from Féry’s speakers indicate that the change from /ʃ/ to /ɕ/ occurs only in the context after front unrounded vowels.

One can speculate that \isi{Delabialization} in all alveolopalatalizing dialects exhibits a gradual broadening of the context according to the \isi{rule generalization} model described in \sectref{sec:2.4.1}: The change occurs first after front unrounded vowels and only later is the change extended to all other contexts. The \ipi{Frankfurt am Main}/\ipi{Montabaur} data reflect the first stage and the remaining dialects discussed above the second stage. Future research on dialects currently undergoing alveolopalatalization may shed light on the incremental changes described here.

Stage C dialects (e.g. \ipi{Schlebusch}, \ili{Luxembourgish}, \ipi{Leipzig}) represent focal areas because they exhibit alveolopalatalization to its fullest extent. When alveolopalatalization was first phonologized in those places it reflected the more narrow Stage B dialects.

In any case, the evidence is clear that alveolopalatalization affected first [ç] and only later [ʃ] -- a generalization deriving support from the modern dialects discussed above. Additional evidence for my claim comes from unattested dialects. In particular, no dialect has been uncovered in the present survey in which alveolopalatalization affects only [ʃ] but not [ç]. The type of unattested synchronic system described here is depicted in \REF{ex:10:38a}. Likewise no dialect is known in which there is a three-way contrast among velar, palatal, and alveolopalatal (=\ref{ex:10:38b}). Note that the system in \REF{ex:10:38b} would be a dialect like the ones described in \chapref{sec:9} in which /x/ and /ç/ are phonemic together with \isi{Delabialization} of earlier /ʃ/ to /ɕ/.

\ea%38
\label{ex:10:38}
\begin{multicols}{2}
\ea\label{ex:10:38a}
\begin{forest} for tree = {fit=band}
[,phantom
  [/x/,calign=first [{[x]}] [{[ç]}]]   [/ɕ/ [{[ɕ]}]]
]
\end{forest}
\ex\label{ex:10:38b}\begin{forest}
[,phantom
   [/x/ [{[x]}]]  [/ç/ [{[ç]}]]      [/ɕ/ [{[ɕ]}]]
]
\end{forest}
\z 
\end{multicols}
\z 

If the systems in \REF{ex:10:38} are truly unattested then the conclusion is that alveolopalatalization affected first the palatal [ç] and only later on the postalveolar [ʃ], which is precisely the progression presupposed in the present chapter (recall \ref{ex:10:3}--\ref{ex:10:5}). Only future studies on alveolopalatalization in progress can lend further support to my observation.\footnote{{In a dialect I discuss below (\ipi{Dithmarschen}), I point out that a possible interpretation of the system of fortis fricatives is precisely the one depicted in \REF{ex:10:38a}.}}

\subsection{Realization of the lenis palatal fricative [ʝ] in German dialects}\label{sec:10.6.2}

Recall from \sectref{sec:10.3.1} that \ipi{Schlebusch} targets both /x/ and /ɣ/ for velar fronting but that of the two [coronal, dorsal] sounds created by that process only the fortis one is interpreted as a \isi{sibilant} by (\ref{ex:10:8a}). By contrast, the lenis palatal fricative ([ʝ]) fails to surface as alveolopalatal (*[ʑ]). It interesting to observe that the realization of [ʝ] as a nonsibilant holds for any [ʝ] in \ipi{Schlebusch}, regardless of the synchronic or diachronic source. In particular, there is the palatal [ʝ] deriving from /ɣ/ by velar fronting in words like [beːʝǝ] ‘bend-\textsc{inf}’ (=⟦bēːjǝ⟧) from \REF{ex:10:18b} as well as word-initial [ʝ] deriving from either \ili{WGmc} \textsuperscript{+}[j] in words like [ʝɔː] ‘yes’ (=⟦jǫ⟧) or from\ili{WGmc} \textsuperscript{+}[ɣ] in items like [ʝɛl] ‘yellow (=⟦jęl⟧); recall \REF{ex:10:12b}.

The same generalizations involving [ʝ] hold for the other two dialects discussed above with that sound. In \ipi{Cologne} (\sectref{sec:10.4.1}) the lenis fricative [ʝ] surfaces as a nonstrident sound in a word-internal onset, e.g. [fʏʝǝl] ‘bird-\textsc{pl}’ (=⟦fγjǝl⟧) from \REF{ex:10:34c}, and word-initially, e.g. [ʝǝzeːɕ] ‘face’ (=⟦jǝzeː£ʽ⟧) from \REF{ex:10:34a}. The facts are essentially the same in for Lxm [ʝ] alluded to in \sectref{sec:10.3.2}. See \citet{Gilles1999} for discussion.

There are two lenis palatal fricatives that need to be distinguished: (a) Palatal [ʝ] that is the modern reflex of an earlier velar (\ili{WGmc} \textsuperscript{+}[ɣ]), and (b) palatal [ʝ] that is the modern reflex of the palatal glide (\ili{WGmc} \textsuperscript{+}[j]). The [ʝ] in (a) appears to be immune to \isi{phonetic implementation} rules akin to the ones in (\ref{ex:10:6}--\ref{ex:10:8}) in all dialects discussed in the present book. That generalization holds for [ʝ] in word-initial position, as well as [ʝ] in a word-internal onset. Examples for both contexts were given above for \ipi{Schlebusch}. For additional dialects with [ʝ] the reader is referred to the case studies discussed below in Chapters~\ref{sec:11}--\ref{sec:12} and \ref{sec:14}.

One might suggest that there are CG dialects in which the [ʝ] from an earlier velar is in fact a \isi{sibilant} ([ʑ]) but that the linguists describing the dialects in question chose to ignore that detail. I consider that scenario to be unlikely. Authors of Ortsgrammatiken placed a great deal of emphasis on phonetic detail. Recall from \chapref{sec:1} that many of the authors of those works were well-versed in phonetics and also that phonological notions like phonemes and allophones had not yet been discovered. If historical [ç] is realized as a \isi{sibilant} and assigned a new phonetic symbol, why not do the same with historical [ʝ]? It is also important to stress that a sound similar to the lenis equivalent of [ɕ] was known to all of the authors of Ortsgrammatiken, namely the lenis counterpart to the postalveolar fricative [ʃ] (=[ʒ]), which is present in many loanwords from \ili{French}, e.g. \textit{Etage, Journal}.

It is surprisingly difficult to find descriptions of German dialects in which historical [ʝ] (<\ili{WGmc} \textsuperscript{+}[ɣ]) is realized as alveolopalatal. I tentatively consider this gap as systematic because the facts follow from the way the \isi{phonetic implementation} rules in (\ref{ex:10:6}--\ref{ex:10:8}) are stated. Future research might investigate whether or not there are dialects like the ones I have been unable to find.\footnote{\citet[369--370]{Schirmunski1962} observes that certain LG dialects realize [ʝ] (<\textsuperscript{+}[j]) as a \isi{sibilant} but his source \citep{Grimme1922} does not provide clear examples indicating that change. More recently, \citet[42]{GoltzWalker1989} note without comment that the \isi{etymological palatal} in NLG (their North Saxon) is “... often realized as the fricative [ʒ] or the \isi{affricate} [dʒ]ˮ.}

Several dialects are reported to have a lenis -- presumably \isi{sibilant} -- realization of the \isi{etymological palatal}. Nine such dialects (all LG) are known to me. I present data and a brief analysis for one of those dialects below and make passing reference to the other eight. In contrast to the alveolopalatalizing dialects discussed in the first part of this chapter very little is known about the dialects discussed below.

\citet{Kohbrok1901} describes a NLG dialect spoken in the county of \ipi{Dithmarschen} on the west coast of the German state of Schleswig-Holstein (\mapref{map:5}). The significance of \ipi{Dithmarschen} is that the modern reflex of the \isi{etymological palatal} is a lenis fricative Kohlbok represents as ⟦ž⟧, which he describes (pp. 15--16) as the voiced (lenis) equivalent of [ʃ] (=his ⟦š⟧). I interpret Kohlbok’s ⟦ž⟧ as the lenis alveolopalatal fricative and therefore transcribe it below as [ʑ].\footnote{\citet{Stammerjohann1914} offers a phonetic study of the sounds in the NLG community of \ipi{Burg} in the county of \ipi{Dithmarschen}. While he concurs that \ipi{Burg} possesses ⟦ž⟧, he stresses that the phonetic facts of the \ipi{Burg} variant of that sound are not exactly the same as they are for Kohbrok’s speakers. In particular, the tongue tip for ⟦ž⟧ lies closer to the alveolar ridge than it does for ⟦š⟧; \citet[67]{Stammerjohann1914}.}

The realization of \ili{WGmc} \textsuperscript{+}[j] in \ipi{Dithmarschen} as [ʑ] is illustrated for word-initial position before any type of vowel in (\ref{ex:10:39a}) or in a word-internal onset in (\ref{ex:10:39b}). [ʑ] does not surface in syllable-final position. The data in (\ref{ex:10:39c}, \ref{ex:10:39d}) demonstrate that \ipi{Dithmarschen} also has [x] in the context after a back vowel and [ç] after a front vowel. Significantly, the [ç] in \REF{ex:10:39d} is not realized as an alveolopalatal ([ɕ]). There is no lenis fricative [ʝ] (<\ili{WGmc} \textsuperscript{+}[ɣ]) in \ipi{Dithmarschen} because that historical sound either deleted or restructured to [g] (/g/) by \isi{g-Formation-1} (\sectref{sec:4.2}). Fortis postalveolar [ʃ] (=⟦š⟧) occurs initially and finally (in \ref{ex:10:39e}). That fricative derived historically from \ili{WGmc} \textsuperscript{+}[sk].\largerpage[-1]\pagebreak

\TabPositions{.15\textwidth, .33\textwidth, .5\textwidth, .75\textwidth}
\ea%39
\label{ex:10:39}\ipi{Dithmarschen} fricatives:
\ea\label{ex:10:39a} žōɑ \tab [ʑoːɐ] \tab Jahr \tab ‘year’ \tab 75\\
    žym \tab [ʑym] \tab ihr, euch \tab ‘you\textsc{{}-pl}’ \tab 71\\
    žyɡ \tab [ʑyg] \tab Joch \tab ‘yoke’ \tab 30\\
    ž\={ø}ɑn \tab [ʑøːɐn] \tab Jürgen \tab ‘(name)’ \tab 71
\ex\label{ex:10:39b} kɒužə \tab [kɒuʑə] \tab Koje \tab ‘berth’ \tab 75
\ex\label{ex:10:39c} ɑχdɑ \tab [ɑxdɐ] \tab hinter \tab ‘behind’ \tab 72\\
    doχtɑ \tab [doxtɐ] \tab Tochter \tab ‘daughter’ \tab 75\\
    hɒuχ \tab [hɒux] \tab hoch \tab ‘high’ \tab 75
\ex\label{ex:10:39d} ryx \tab [ryç] \tab Rücken \tab ‘back’ \tab 70\\
    rex \tab [reç] \tab recht \tab ‘right’ \tab 27\\
    stīx \tab [stiːç] \tab Steig \tab ‘hill-climbing’ \tab 32
\ex\label{ex:10:39e} šūɑ \tab [ʃuːɐ] \tab Schauer \tab ‘shower’ \tab 74\\
    šrĩm \tab [ʃrĩm] \tab schreiben \tab ‘write\textsc{{}-inf}’ \tab 74\\
    diš \tab [diʃ] \tab Tisch \tab ‘table’ \tab 74
    \z
\z 

As in all other LG varieties investigated in this book, \ili{WGmc} \textsuperscript{+}[j] (/j/) underwent \isi{Glide Hardening} to pre-\ipi{Dithmarschen} \textsuperscript{+}[ʝ] (/ʝ/). \ipi{Dithmarschen} is unique in that the new palatal fricative is now realized as [ʑ]. The phonological representation for the two fricatives in question ([ʝ] and [ʑ]) is identical, namely [coronal, dorsal]. The change from the former to the latter therefore did not involve phonology at all, but instead fell within the realm of \isi{phonetic implementation}. At the pre-\ipi{Dithmarschen} stage only simplex coronal fricatives were interpreted by \isi{phonetic implementation} as sibilants (by \ref{ex:10:6b}). The change from pre-\ipi{Dithmarschen} to \ipi{Dithmarschen} therefore involved the retention of \REF{ex:10:6b}, which is restated in \REF{ex:10:40b}, and the addition of a special provision for complex lenis fricatives in (\ref{ex:10:40a}).\footnote{{As noted earlier, virtually nothing is known about the phonology and phonetics of data like the ones in \REF{ex:10:39}. Hence, other analyses are conceivable. For example, one could analyze [ʃ] as [coronal, labial] and restate \REF{ex:10:40a} so that [coronal, labial] fricatives (both lenis and fortis) are interpreted as sibilants. It might also be the case that what I transcribe as [ʃ] is really alveolopalatal [ɕ], suggesting that \ipi{Dithmarschen} represents the unattested system in \REF{ex:10:38a}.} }

\ea%40
\label{ex:10:40}Phonetic Implementation:
\ea\label{ex:10:40a} Lenis complex fricatives ([coronal, dorsal]) are interpreted as sibilants ([ʑ]).
\ex\label{ex:10:40b} Simplex [coronal] fricatives are interpreted as sibilants ([s, z, ʃ]).
\z 
\z 

The original source for \ipi{Dithmarschen} also makes clear that the dialect possesses other simplex coronal fricatives: (a) a voiceless (“stimmloseˮ) dental fricative (=⟦ʃ⟧) occurring word-initially before a vowel (e.g. [siːd] (=⟦ʃīd⟧ ‘side’), and (b) a (nonstrident) lenis dental fricative [ð], e.g. [foːðɐ] ‘father’ (=⟦fōðɑ⟧), which is the modern reflex of \ili{WGmc} \textsuperscript{+}[d] in the context after a vowel, but only before the vocalized-r. It appears that [ð] is still an allophone of /d/ in the synchronic phonology. I leave open how to analyze those additional fricatives, but in any case those structures must be made immune to \REF{ex:10:40a}.

Eight additional varieties are known to me of LG dialects in which the \isi{etymological palatal} is realized as a lenis -- presumably alveolopalatal \isi{sibilant} -- fricative [ʑ]. Those places (together with \ipi{Dithmarschen}) are listed in \tabref{tab:10:2}, which I comment on below.

\begin{table}
\caption{Varieties of WLG and ELG in which \ili{WGmc} \textsuperscript{+}[j] is realized as a sibilant fricative ([ʑ]).\label{tab:10:2}}
\begin{tabular}{lll}
\lsptoprule
Place & Dialect & Source\\\midrule
\ipi{Burg} (\ipi{Dithmarschen}) & NLG & \citet{Kohbrok1901}, \citet{Stammerjohann1914}\\
\ipi{Bergenhusen} & NLG & \citet{Sievers1914}\\
\ipi{Heide} (\ipi{Dithmarschen}) & NLG & \citet{Jörgensen1928}\\
\ipi{Diepenau} & NLG & \citet{Schmeding1937}\\
\ipi{Altenwerder} & NLG & \citet{Höder2010}\\
\ipi{Lüneburger Wendland} & \il{Brandenburgish}Brb & \citet{Selmer1918}\\
\ipi{West Mecklenburg} & \il{Mecklenburgish-West Pomeranian}MeWPo & \citet{Kolz1914}\\
\ipi{South Mecklenburg} & \il{Mecklenburgish-West Pomeranian}MeWPo & \citet{Jacobs1925a,Jacobs1925b,Jacobs1926}\\
Kaarβen & \il{Mecklenburgish-West Pomeranian}MeWPo & \citet{Dützmann1932}\\
\lspbottomrule
\end{tabular}
\end{table}

The closest place listed in \tabref{tab:10:2} to \ipi{Dithmarschen} geographically is \ipi{Heide} (\citealt{Jörgensen1928}); \mapref{map:5}. In that work, Jörgensen consistently transcribes the  etymological lenis palatal fricative as ⟦ž⟧. A similar example is \ipi{Bergenhusen} (\citealt{Sievers1914}). An examination of the words listed in the historical part of that book with \ili{WGmc} \textsuperscript{+}[j] reveals that that sound has been replaced with a \isi{sibilant}. The facts are the same in \ipi{Diepenau} (\citealt{Schmeding1937}; \mapref{map:5}). According to that source (pp. 43--44), \ili{WGmc} \textsuperscript{+}[j] is regularly realized as ⟦ž⟧ in word-initial position, e.g. ⟦žǭ⟧ ‘yes’ (cf. \il{Standard German}StG \textit{ja}), ⟦žūxn⟧ ‘cheer\textsc{{}-inf}’ (cf. \il{Standard German}StG \textit{jauchzen}). \citet[7]{Höder2010} similarly notes that \ili{WGmc} \textsuperscript{+}[j] can be realized in \ipi{Altenwerder} as a \isi{sibilant} fricative in initial position. The ELG varieties in \tabref{tab:10:2} are depicted on \mapref{map:17}. For the \il{Brandenburgish}Brb variety of the Lüneburger Wedland, \citet[55--57]{Selmer1918} observes that \ili{WGmc} \textsuperscript{+}[j] is realized as ⟦ž⟧, which he refers to as the assibilated (“assibilierteˮ) realization of the \isi{etymological palatal}. The same generalization holds in the three varieties of \il{Mecklenburgish-West Pomeranian}MeWPo listed above. \citet[148]{Kolz1914} writes that for speakers in rural areas (his “Lingua vulgaris=Lv.”) \ili{WGmc} \textsuperscript{+}[j] is realized as a \isi{sibilant} fricative. According to \citet[123]{Jacobs1925b}, \ili{WGmc} \textsuperscript{+}[j] is regularly realized as ⟦ž⟧ in onset position (“Anlautˮ), e.g. ⟦žɑ̊˙⟧ ‘yes’, ⟦ho˙žɑ̊˙\={n}⟧ ‘yawn\textsc{{}-inf}’ (cf. MLG \textit{hojanen}).\footnote{\citet[130]{Jacobs1925b} gives one example in which the modern reflex of historical [ɣ] is ⟦ž⟧, namely ⟦brü˙žɑ̊˙m⟧ ‘bridegroom’ (cf. \il{Standard German}StG \textit{Bräutigam}). This appears to be an irregular form (\sectref{sec:12.8.3}), since historical [ɣ] is usually realized in \ipi{South Mecklenburg} as [g] between vowels.} Finally, in his list of consonants for Kaarβen, \citet[12]{Dützmann1932} lists no [j] (or fricative [ʝ]). In his discussion of the phonetics (p. 14), he remarks that the \isi{etymological palatal} (his ⟦ž⟧) is “formed like [š]ˮ. (“Es bildet sich wie das \textit{š}ˮ).\footnote{The dictionary for the Schleswig-Holstein dialect (SchlHWb) consistently transcribes [ʝ] as ⟦ž⟧ (=[ʒ]), e.g. \textit{Gicht} ‘gout’ (⟦žixt⟧) and \textit{jung} ‘young’ (⟦žuŋ⟧). Since SchlHWb is intended to reflect a large area, the implication is that the realization of [ʝ] as a \isi{sibilant} is much more widespread than what is suggested by the small list of places in \tabref{tab:10:2}.}

\subsection{Underlying representations}\label{sec:10.6.3}

Recall from the discussion of \ipi{Schlebusch} (\sectref{sec:10.3.1}) that there are two types of words with [ɕ] after a front vowel: Those in which [ɕ] is underlyingly /ɕ/ in (\ref{ex:10:14}) and those in which [ɕ] is underlyingly /x/ in (\ref{ex:10:17}). As noted earlier, the [ɕ] in the latter dataset does not alternate with another sound. Underlying and phonetic representations for representative examples from those two datasets are presented in \REF{ex:10:41a} and \REF{ex:10:41b} respectively. The Stage C column represents \ipi{Schlebusch} as it was described in 1935 by Rudolf Bubner. Stage B represents the pre-\ipi{Schlebusch} stage before \isi{Delabialization} restructured /ʃ/ to /ɕ/. I discuss Stage D below. In the final column I indicate the diachronic source of alveolopalatal [ɕ] in these items.

\ea%41
\label{ex:10:41}\NumTabs{5}
\begin{xlist}
\sn{} Stage B:    \tab    Stage C:      \tab  Stage D:     \tab
\ex\label{ex:10:41a}   /veʃ/  [veʃ] \tab    /veɕ/  [veɕ]  \tab  /veɕ/  [veɕ] \tab  ‘fish’ \tab  < [ʃ]
\ex\label{ex:10:41b}   /ex/    [eɕ] \tab    /ex/    [eɕ]  \tab  /eɕ/    [eɕ]  \tab  ‘I’   \tab  < [ç]
\end{xlist}
\z 

The examples here are drawn from a specific \il{Ripuarian}Rpn-speaking community \citep{Bubner1935}, although the issue discussed here holds for all Stage C varieties.

The underlying representations for pre-\ipi{Schlebusch} (Stage B) are justified because [ɕ] at that point was still an allophone of /x/ and /ʃ/ was uncontroversially a contrastive (phonemic) sound. At issue are the underlying representations at Stage C: How are post-1935 speakers of \ipi{Schlebusch} not knowledgeable of the history of their dialect able to deduce that surface [ɕ] is /ɕ/ in \REF{ex:10:41a} but /x/ in \REF{ex:10:41b}?

I argue that /veɕ/ ‘fish’ and /ex/ ‘I’ were correct for the first generation of Stage C speakers of \ipi{Schlebusch}. The first generation individuals referred to here were those speakers who were the first to restructure underlying representations like /veʃ/ to /veɕ/. However, once later generations were exposed to words like [veɕ] and [eɕ] it was inevitably the case that the Stage B (and first generation Stage C) underlying representation for /ex/ was restructured to /eɕ/. That modification occurs at Stage D. The reason for that restructuring is that those speakers were ignorant of the history of their dialect and that there was no evidence for analyzing [eɕ] as anything other than /eɕ/. This point aside, Stage D speakers inherited \isi{Velar Fronting-1} in order to account for [x]{\textasciitilde}[ɕ] alternations like the ones in \REF{ex:10:15}. The underlying and phonetic representations for a representative example for Pre-\ipi{Schlebusch} (Stage B) and \ipi{Schlebusch} (Stage C/D) are presented in \REF{ex:10:42}:

\ea%42
\NumTabs{5}
\label{ex:10:42}
\begin{xlist}
\sn{}     Stage B:            \tab  Stage C/D:       \tab
\ex\label{ex:10:42a}  /lɔx/       [lɔx]   \tab /lɔx/       [lɔx] \tab ‘hole’  \tab < [x]
\ex\label{ex:10:42b}  /løːx-ǝ/    [løːɕǝ] \tab /løːx-ǝ/  [løːɕǝ] \tab ‘hole-\textsc{pl}’ \tab < [ç]
\end{xlist}
\z 

Significantly, the restructuring of /ex/ ‘I’ to /eɕ/ by Stage D speakers in (\ref{ex:10:41b}) did not affect the underlying representations of alternating examples like the ones in \REF{ex:10:42}. That restructuring did not occur in items like [løːɕǝ] ‘hole-\textsc{pl}’, which continued to be analyzed with /x/ as /løːx-ǝ/ because of the related form with [x] (i.e. [lɔx]).

Recall from earlier chapters that underlying palatals which derived historically from velars -- palatal quasi-phonemes and phonemic palatals -- invariably occur in the context of a back vowel. The treatment of Stage D nonalternating morphemes in (\ref{ex:10:41b}) is significant because it reveals that there are also some dialects in which underlying palatals (/ɕ/) deriving from etymological velars also occur in the context of front vowels. The change from Stage C /x/ to Stage D /ɕ/ after a front vowel in (\ref{ex:10:41b}) may appear to involve a version of velar fronting, but closer examination reveals that the change in question was not phonological. First, the replacement of /x/ with /ɕ/ after front vowels failed to affect the /x/ in alternating examples in (\ref{ex:10:42b}). And second, the change from velar fricative to its fronted counterpart involved the restructuring of underlying representations, but no version of velar fronting in any of the dialects discussed in Chapters~\ref{sec:3}--\ref{sec:9} alters underlying representations. It is also possible that the change from /x/ to /ɕ/ in \REF{ex:10:41b} might not have affected all words like /eɕ/ ‘I’ at once, but instead that it occurred on a word-by-word basis. Since no evidence is present in any of the original sources for Stage C dialects which bears on this question I leave that possibility open.

\section{{Conclusion}}\label{sec:10.7}

This chapter has investigated alveolopalatalization ([ç ʃ] > [ɕ]), which is a common feature of CG dialects. It was argued above that the historical change from [ç ʃ] to [ɕ] involved two distinct changes, namely (a) the change from [ç] to [ɕ] (Stage B) followed by the change from [ʃ] to [ɕ] (Stage C). At Stage B [ɕ] was still an allophone of /x/ and had not yet merged with [ʃ] (/ʃ/). At Stage C the alveolopalatal fricative [ɕ] is phonemic (/ɕ/) because it contrasts with [x] (/x/) in the context after back vowels. The allophonic rule of velar fronting at Stage B was inherited at Stage C as a rule neutralizing the contrast between /x/ and /ɕ/ in the context after front vowels. Velar fronting at Stage B and at Stage C does not differ formally from the eponymous rule discussed for other dialects in earlier chapters: The feature [coronal] spreads from a front segment to a [dorsal] target (/x/), thereby producing a complex [coronal, dorsal] segment. That feature complex is interpreted as a \isi{sibilant} ([ɕ]) at Stage B and Stage C by \isi{phonetic implementation}.
\is{alveolopalatalization|)}
