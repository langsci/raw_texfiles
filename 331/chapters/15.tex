\chapter{Velar fronting islands}\label{sec:15}
\is{velar fronting island|(}

\section{Introduction}\label{sec:15.1}

By definition, a velar fronting island is a velar fronting place surrounded by areas in which velar fronting is absent. Two types can be distinguished. First, a velar fronting dialect of German might be attested in a German-language island. German-language islands originate when speakers of German emigrate to a new area where they are encircled by speakers of a different language (\citealt{Wiesinger1980}, \citealt{Wiesinger1983b}, \citealt[76--83]{Boas2009}, and \citealt{Putnam2011}). The reader is referred to the two case studies (\ili{Plautdietsch} and \ili{Transylvania Saxon}) discussed in \sectref{sec:9.5.2} and to \tabref{tab:c30} in Appendix \ref{appendix:c} for a list of German-language islands discussed in this book. Second, a (German) velar fronting island may be observable in a country where German is the dominant language. This is the case when the velar fronting variety of German is bounded by other varieties of German without velar fronting.

The purpose of this chapter is to document velar fronting islands known to me. For the first type I focus on some of those German-language islands in the areas to the east and south of modern-day German-speaking countries, namely the Czech Republic, Slovenia, and Northeast Italy. For the second type I consider specific places in Switzerland and Austria (\ipi{Tyrol}, \ipi{Vorarlberg}).

The velar fronting islands I discuss below can vary greatly in terms of size and robustness. On the one hand, certain islands are very small and might simply comprise a single village or town. On the other hand, some velar fronting islands are embedded in a large area with multiple villages and towns. Some of the dialects discussed below are either extinct or on the verge of extinction, while others are spoken by large numbers of speakers and show no signs at all of endangerment.

Velar fronting islands are important to document for more than one reason. First, they illustrate variation among some of the parameters discussed in previous chapters. Of those parameters, the one involving velar fronting triggers plays the most significant role below. A closer examination of those triggers reveals that they can consist of either all coronal sonorants -- referred to earlier as the default pattern -- or of some subset of the coronal sonorants. Since the enclaves discussed below phonologized velar fronting independently, it is precisely this type of variation that lends support to the historical stages proposed in \chapref{sec:12}. A surprising finding is that velar fronting is nonassimilatory (=Trigger Type F from \chapref{sec:14}) in several geographically distinct areas. Recall that that type of system, i.e. one with only palatals but no velars, is otherwise most robustly attested in \ipi{Nordösling} (\ipi{Luxembourg}) and in neighboring places in Germany and \ipi{East Belgium} (\sectref{sec:14.5}). A second reason for documenting velar fronting islands is to demonstrate that certain places possess patterns that are either rare or otherwise unattested. In the course of this chapter, I show that those patterns are compatible with the models adopted in this book.

It is important to clarify the nature of the data and sources I cite below. Some of those works give a large selection of data involving the distribution of velars and palatals that make it possible to precisely pinpoint the set of sounds that do and do not trigger the process. By contrast, other sources might give a prose statement indicating that velar fronting is present in a particular context and might (or might not) include only a small selection of data. Other sources -- in particular, linguistic atlases -- may give detailed maps indicating the geographic distribution of words with velars and palatals without giving any concrete examples.

It is not uncommon for two or more sources to describe the state of velar fronting in conflicting ways for the same place. For example, one source might state that a place (Town A) has no velar fronting at all, while another source might assert -- either directly in a prose statement or indirectly with data -- that (assimilatory) velar fronting is active for Town A, but another source might be clear that Town A has \isi{nonassimilatory velar fronting}. These conflicting claims should not be surprising because it is not the case that any one place always has a single version of velar fronting for all speakers. A more realistic view is that the inconsistency among speakers indicates that in any given place -- Town A in the hypothetical example given above -- there are non-velar fronting speakers and two types of velar fronting speakers (assimilatory and nonassimilatory).\footnote{{The situation I am describing is also documented in linguistic atlases. To cite one example, the informants for SSA for the town of Wangen im Allgäu had (assimilatory) velar fronting but the informants for that same place for VALTS did not (see \mapref{map:3}).}}

\begin{sloppypar}
The case studies investigated below are organized geographically. The first four sections concern themselves with velar fronting islands within German-language islands, namely \ipi{Iglau} and \ipi{Libinsdorf} in the Czech Republic (\sectref{sec:15.2}), \ipi{Schönhengst} in the Czech Republic (\sectref{sec:15.3}), \ipi{Giazza}/\ipi{Dreizehn Gemeinden} in Northeast Italy (\sectref{sec:15.4}), and \ipi{Gottschee} in Slovenia (\sectref{sec:15.5}). \sectref{sec:15.6} concerns itself with two velar fronting islands spoken in the Swiss canton of Grisons, namely \ipi{Obersaxen} and \ipi{Vals}. Since this chapter draws on data from the linguistic atlas of Switzerland (SDS). I devote \sectref{sec:15.7} to a discussion and interpretation of the symbols for dorsal fricatives in that source. \sectref{sec:15.8} evaluates the state of velar fronting in the German-speaking region of the Swiss canton of  Valais (\ipi{Upper Valais}) as well as in the neighboring German-language enclaves in Northwest Italy and the Swiss canton of Tessin. \sectref{sec:15.9} documents velar fronting in the Southwest \ipi{Bernese Oberland}, \sectref{sec:15.10} investigates the velar fronting islands in the isolated mountain valleys of \ipi{Tyrol}, and \sectref{sec:15.11} concerns itself with velar fronting varieties in a large region consisting of \ipi{East Switzerland}, \ipi{Liechtenstein}, and \ipi{Vorarlberg} (Austria). In \sectref{sec:15.12} I provide a summary of velar fronting islands and discuss the way in which they differ in terms of segments inducing the change.
\end{sloppypar}

\section{{Iglau} {and} {Libinsdorf}}\label{sec:15.2}

\ipi{Libinsdorf} (Czech: Karlov) is a small village in the Czech Republic situated about 116km southeast of Prague. The town was once a German-language island which was settled in 1789 by families from four North Bohemian villages. The German dialect of \ipi{Libinsdorf} is classified as \il{Upper Saxon}USax-North Bohemian \citep[915]{Wiesinger1983b}. According to the census of 1 December 1930, 20--50\% of the populace of \ipi{Libinsdorf} were ethnic Germans (SDA: Blatt 4).

The sound structure of the \ipi{Libinsdorf} dialect is described by \citet{Weinelt1940}. The data in that source indicate that palatal [ç] (=⟦χ⟧) occurs after any front vowel (=\ref{ex:15:1a}) or coronal sonorant consonant (=\ref{ex:15:1c}-{\ref{ex:15:1e}) and velar [x] (=⟦x⟧) after any back vowel (=\ref{ex:15:1b}). \ili{Czech} has [x] (/x/) but no corresponding palatal (\citealt{ŠimáčkováChládková2012}). That source makes no reference to a process fronting /x/ to [ç] in the \ili{Czech} language. \footnote{Some of the examples in \REF{ex:15:1} indicate that the surface dorsal fricative corresponds to an orthographic \textit{g}, e.g. [tswɑiç] in \REF{ex:15:1a}. No examples were found in \citet{Weinelt1940} in which those examples are followed by a vowel-initial suffix (cf. \il{Standard German}StG [tsvɑik] ‘branch’ vs. [tsvɑigə] ‘branch-\textsc{pl}’); hence, it cannot be known whether or not the underlying representations of such words contain /x/ or a lenis sound (e.g. /g/). The same comment holds for several other dialects posited in this chapter.}

\TabPositions{.2\textwidth, .4\textwidth, .55\textwidth, .75\textwidth}
\ea%1
\label{ex:15:1}Dorsal fricatives in \ipi{Libinsdorf}:
\ea\label{ex:15:1a} fiχtə \tab  [fiçtə] \tab Fichte \tab ‘spruce’ \tab 40\\
lẹ̄χt  \tab  [leːçt] \tab liegt \tab ‘lie-\textsc{3sg}’ \tab 38\\
kęχin \tab [kɛçin] \tab Köchin \tab ‘cook-\textsc{fem}’ \tab 41\\
tręχtə \tab [trɛçtə] \tab Trichter \tab ‘funnel’ \tab 44\\
laiχt \tab  [lɑiçt] \tab leicht \tab ‘easy’ \tab 39\\
tswaiχ \tab [tswɑiç] \tab Zweig \tab ‘branch’ \tab 43\\
\ex\label{ex:15:1b} pūxə \tab [puːxə] \tab Buche \tab ‘beech tree’ \tab 40\\
tōx  \tab  [toːx] \tab Dach \tab ‘roof’ \tab 37\\
haxt \tab [hɑxt] \tab Hecht \tab ‘pike’ \tab 41\\
knāxt \tab [knɑːxt] \tab Knecht \tab ‘vassal’ \tab 37\\
pauxwī \tab [pauxwiː] \tab Bauchweh \tab ‘stomach ache’ \tab 39
\ex\label{ex:15:1c} khirχə \tab [kʰirçə] \tab Kirche \tab ‘church’ \tab 38\\
m\k{u}rχl \tab [mʊrçl̩] \tab Morchel \tab ‘morel’ \tab 38\\
štarχ \tab [ʃtɑrç] \tab Storch \tab ‘stork’ \tab 38\\
harχn \tab [hɑrçn̩] \tab horchen \tab ‘hark-\textsc{inf}’ \tab 38\\
\ex\label{ex:15:1d} šelχ  \tab  [ʃelç] \tab schuldig \tab ‘guilty’ \tab 44
\ex\label{ex:15:1e} tsaumkhēnχ \tab [tsaumkʰeːnç] \tab Zaunkönig \tab ‘wren’ \tab 44
\z
\z

The patterning of velars and palatals in \REF{ex:15:1} is the default one described in previous chapters. Thus, velar fronting applies to any /x/ after a coronal sonorant:


\ea%2 
\label{ex:15:2}\isi{Velar Fronting-1}:\smallskip\\
\begin{forest}
[,phantom
  [\avm{[+son]} [\avm{[coronal]}, tier=word,name=target]]
  [\avm{[−son\\+cont]},name=parent [\avm{[dorsal]},tier=word]]
]
\draw [dashed] (parent.south) -- (target.north);
\end{forest}
\z 

The orthographic forms in (\ref{ex:15:1d}, \ref{ex:15:1e}) indicate that the final vowel was elided by the historical process of \isi{Syncope} (\chapref{sec:7}). No examples were found in \citet{Weinelt1940} in which a dorsal fricative occurs after [l] or [n] without a historically syncopated vowel, e.g. \il{Standard German}StG [zɔlç] ‘such’, [mɑnçmɑːl] ‘sometimes’.

\ipi{Iglau} (Czech: Jihlava) is a medium-sized Czech city about 114km southeast of Prague and 36km southwest of \ipi{Libinsdorf}; see \mapref{map:3}. The area in and around \ipi{Iglau} once formed a sizable German-language island (Iglauer Sprachinsel), the largest city of which was \ipi{Iglau}. According to the census of 1 December 1930, 80--90\% of the population of a large portion of the Iglauer Sprachinsel consisted of ethnic Germans (SDA: Blatt 4). The area was settled many centuries ago (between 1240 and 1260) by \il{North Bavarian}NBav speakers from the Upper Palatinate (Oberpfalz) and from ECG speakers from the Erzgebirge region \citep[909]{Wiesinger1983b}.

To the best of my knowledge, the most comprehensive source for the sound structure of the dialects once spoken in the towns of the Iglauer Sprachinsel is \citet{Stolle1969}. That work is a description of the historical changes affecting vowels in thirty-nine villages and towns in the Iglauer Sprachinsel. Although the author does not explicitly discuss the distribution of the ich-Laut and the ach-Laut, it is clear from Stolle’s phonetic transcriptions that those two sounds occur in all of the thirty-nine places in his study (Belegorte). The basic generalization for the entire area is that [ç] (=⟦χ⟧) occurs after any front vowel (=\ref{ex:15:3a}) and [x] (=⟦x⟧) after any back vowel (=\ref{ex:15:3b}). A number of words can be found in \citet{Stolle1969} with dorsal fricatives in the context after liquids (=\ref{ex:15:3c}, \ref{ex:15:3d}), which I discuss in greater detail below. The generalizations concerning the distribution of [ç] and [x] do not differ from place to place within the Iglauer Sprachinsel; hence, the data in \REF{ex:15:3} do not represent any one particular town. In the following discussion I therefore refer to all of the places in Stolle’s study collectively as \ipi{Iglau}.\footnote{{\ipi{Iglau} is also worthy of note because the etymological diphthong [ei] underwent \isi{Monophthongization} to [ɑː], as in [wɑːx] ‘soft’ in \REF{ex:15:3b}, (cf. \ili{MHG}} \textrm{\textit{weich}}\textrm{). \ipi{Iglau} therefore illustrates the completely transparent distribution of [ç] and [x], in contrast to the CG varieties discussed in \chapref{sec:9}, in which opaque [ç] surfaces in the same environment, e.g. \ipi{Wissenbach} [vɑːç] ‘soft’; recall \sectref{sec:9.2}.}}

\ea%3
\label{ex:15:3}Dorsal fricatives in \ipi{Iglau}:
\ea\label{ex:15:3a} iχ  \tab  [ɪç] \tab ich \tab ‘I’ \tab 92\\
giχt  \tab  [giçt] \tab Gicht \tab ‘gout’ \tab 46\\
hēχt \tab [heːçt] \tab Hecht \tab ‘pike’ \tab 77\\
štęχ  \tab  [ʃtɛç] \tab stechen \tab ‘sting\textsc{{}-inf}’ \tab 45\\
št\={ę}χ  \tab  [ʃtɛːç] \tab Steg \tab ‘footbridge’ \tab 45\\
tāeχ  \tab  [tɑːeç] \tab Teich \tab ‘pond’ \tab 133

\ex\label{ex:15:3b} nǫxt \tab [nɔxt] \tab Nacht \tab ‘night’ \tab 46\\
  pǫx  \tab  [pɔːx] \tab Bach \tab ‘stream’ \tab 45\\
  woxɒn \tab [woxɒn] \tab Wochen \tab ‘week-\textsc{pl}’ \tab 46\\
  nōx  \tab  [noːx] \tab noch \tab ‘still’ \tab 79\\
  wāx \tab [wɑːx] \tab weich \tab ‘soft’ \tab 153\\
  pāǫx \tab [pɑːɔx] \tab Bauch \tab ‘stomach’ \tab 136

\ex\label{ex:15:3c}  pęɒrx \tab [pɛɒrx] \tab Berg \tab ‘mountain’ \tab 48\\
d\k{u}ɒrx  \tab [dʊɒrx] \tab durch \tab ‘through’ \tab 48\\
tswęɒrx \tab [tswɛɒrx] \tab Zwerg \tab ‘dwarf’ \tab 71\\
lęɒrxŋ \tab [lɛɒrxŋ̍] \tab Lärche \tab ‘larch’ \tab 63\\
kįɒrxŋ \tab [kʰɪɒrxŋ̍] \tab Kirche \tab ‘church’ \tab 97\\
fįɒrxt \tab [fɪɒrxt] \tab fürchte \tab ‘fear-1\textsc{sg}’ \tab 105
\ex\label{ex:15:3d}    pîɫχ  \tab  [piːlʲç] \tab Bild \tab ‘picture’ \tab 96
    \z
\z 

I consider now the realization of /x/ after the rhotic (=\ref{ex:15:3c}) and the lateral (=\ref{ex:15:3d}) in that order.

The data in \REF{ex:15:3c} indicate that [x] consistently occurs after [r]. That realization is made explicit in his description of vocalic changes before /r/. For example, on p. 101 Stolle states that \ili{MHG} /u/ surfaces throughout the entire dialect area as ⟦\k{u}ɒrx⟧. That realization [rx] is significant because of its rarity among German dialects. Although a large part of Lower Bavaria is attested with [rx] sequences (\mapref{map:28}), the unmarked realization of /x/ after [r] in velar fronting areas is undoubtedly [ç]; see \mapref{map:21}, which shows the rarity of [rx]/[lx] sequences.

Words with a dorsal fricative preceded by the consonant [l] are rare in \ipi{Iglau} because that sound typically merges together with the preceding vowel by \isi{Liquid Vocalization} (\sectref{sec:3.5}, \sectref{sec:13.5.2}), e.g. the item in \REF{ex:15:3d} surfaces elsewhere in \ipi{Iglau} as [pyːç]. The realization given in \REF{ex:15:3d} with the secondarily palatalized lateral (=⟦ɫ⟧) represents two of the places in the north, namely Sehlenz and Langendorf. The occurrence of [x] after the rhotic and [ç] after the lateral suggests that the triggers for velar fronting must only include the latter but not the former. If so, the patterning of the ich-Laut and the ach-Laut in \ipi{Iglau} would be without precedent. I argue alternatively that the set of triggers for velar fronting throughout \ipi{Iglau} consists solely of front vocoids ([--consonantal, coronal]), as in \REF{ex:15:4}:

\ea%4
\label{ex:15:4}\isi{Velar Fronting-13}\smallskip\\
\begin{forest}
[,phantom
  [\avm{[−cons]} [\avm{[coronal]}, tier=word,name=target]]
  [\avm{[−son\\+cont]},name=source [\avm{[dorsal]}, tier=word]]
]
\draw [dashed] (target.north) -- (source.south);
\end{forest}
\z 

Given the context expressed in \REF{ex:15:4}, /x/ surfaces as palatal after /l/ in \REF{ex:15:3d} because the lateral is palatalized to [lʲ] in coda position (\isi{l-Palatalization}). As suggested by the phonetic transcription, [lʲ] consists of a lateral component ([l]) and a vocalic component ([ʲ]). Since the latter is featurally [--consonantal, coronal], any /x/ following that sound must therefore undergo velar fronting. Put differently, \isi{l-Palatalization} \isi{feeds} velar fronting.

The realization of /x/ in the context after [r] in \ipi{Libinsdorf} and \ipi{Iglau} are depicted on \mapref{map:33}. That map indicates the contrast between the unmarked pattern (represented by \ipi{Libinsdorf}) and the marked pattern (represented by thirty-nine small places in \ipi{Iglau}). Those two contrastive patterns are expressed directly in \isi{Velar Fronting-1} in (\ref{ex:15:3}) and Velar  Fronting-13 in (\ref{ex:15:4}).

\ip{Iglau}
\ip{Libinsdorf}
\begin{map}
% \includegraphics[width=\textwidth]{figures/VelarFrontingHall2021-img039.png}
\includegraphics[width=\textwidth]{figures/Map33_15.1.pdf}
\caption[{Iglau} and {Libinsdorf}]{{Iglau} and {Libinsdorf}. Squares indicate postsonorant velar fronting. The dark square \citep{Weinelt1940} indicates that velar fronting produces a palatal after [r], and the white squares \citep{Stolle1969} depict places where velar fronting fails to apply after [r].}
\label{map:33}
\end{map}

\section{Schönhengst}\label{sec:15.3}\largerpage

Up until 1945 the largest German-language island in the Czech Republic was the Schönhengster Sprachinsel in the Schönhengstgau (Czech: Hřebečsko), a historical region in Bohemia and Moravia. I refer to the Schönhengster Sprachinsel  henceforth simply as \ipi{Schönhengst}. As indicated on \mapref{map:9}, \ipi{Schönhengst} was situated in the modern-day Czech Republic, about 150km to the east of Prague. A close-up view of \ipi{Schönhengst} is depicted on \mapref{map:34}, which shows that it was separated from the German-speaking areas in the former province of \ipi{Silesia} (\ipi{Grafschaft Glatz}) by a small strip of land populated by Czech-speaking people. The largest cities of \ipi{Schönhengst} were Zwittau (Svitavy), Mährisch Trübau (Moravská Třebová), and Landskron (Lanškroun). According to the census of 1 December 1930, 80--100\% of the population of \ipi{Schönhengst} were ethnic Germans (SDA: Blatt 4). Most of those people were forced to leave \ipi{Schönhengst} after 1945.


According to \citet[909]{Wiesinger1983b} \ipi{Schönhengst} was settled over 800 years ago (between 1240 and 1290) by people coming primarily from the Upper \ipi{East Franconia} region (oberostfränkischer Raum), but also from Central Bavaria and \ipi{Silesia}/\ipi{North Moravia}. The various German dialects represented in \ipi{Schönhengst} are depicted on Blatt 5 in SDA.

Several descriptions for the \ipi{Schönhengst} dialect(s) point to an area in which velar fronting was active. As I demonstrate below, those works also indicate that the towns and villages of \ipi{Schönhengst}  differed from one another in terms of the segments that induced velar fronting; hence, \ipi{Schönhengst}  contrasted with \ipi{Iglau}, which had a uniform rule of velar fronting (=\ref{ex:15:4}). In the remainder of this section I discuss the status of velar fronting in \ipi{Schönhengst} according to \citet{Janiczek1911}, \citet{Graebisch1915}, \citet{Seemüller1908c}, and \citet{Benesch1969}, which I discuss in that order. The places described by those authors are indicated on \mapref{map:34}.\footnote{{That map also includes markers representing the following four works which only make passing reference to velar fronting: (a) \citet{Matzke1918} provides a phonetically transcribed text for the town of \ipi{Rathsdorf}. Although he does not transcribe dorsal fricatives with separate symbols, he states that velars (gutturals) and palatals surface after back vowels and front vowels respectively (p. 44). (b) \citet[21]{Appel1963} is clear that the ich-Laut and the ach-Laut are allophones of the same phoneme in Hilbetten, but he does not transcribe the difference between those two sounds with separate symbols.  (c) In his study of the consonants and vowels in the \il{Silesian}Sln dialects of \ipi{North Moravia} and the Adlergebirge, \citet{Weiser1937} indicates on his Map 2 and Map 8 that the palatal fricative occurs after [ɛ] in Schönwald/Lichtenstein. (d) \citet{Sandbach1922} is a study of place names in \ipi{Schönhengst}. That work provides phonetic transcriptions with separate symbols for velars and palatals and offers a short description of the phonetics of consonants and vowels. \mapref{map:34} indicates four of the place names in \citet{Sandbach1922} with a palatal fricative after a front vowel ([i] and [e]), namely Sichelsdorf (p. 8), Dittersbach (p. 16), Reichenau (p. 18), and Sternteich (p. 21).}}

\begin{map}
% \includegraphics[width=\textwidth]{figures/VelarFrontingHall2021-img040.png}
\includegraphics[width=\textwidth]{figures/Map34_15.2.pdf}
\caption[{Schönhengst}]{\ipi{Schönhengst}. Squares indicate some version of postsonorant velar fronting. 1=\citet{Seemüller1908c}, 2=\citet{Janiczek1911}, 3=\citet{Graebisch1915}, 4=\citet{Matzke1918}, 5=\citet{Sandbach1922}, 6=\citet{Weiser1937}, 7=\citet{Appel1963}, 8=\citet{Benesch1969}.}
\label{map:34}
\end{map}

\citet{Janiczek1911} investigates the vocalism in \ipi{Langenlutsch}, conveniently providing transcriptions with separate symbols for velar and palatal fricatives, namely [ç] (=⟦χ⟧) and [x] (=⟦x⟧). The data in \REF{ex:15:5a} show that the palatal surfaces after any front vowel, while the examples in \REF{ex:15:5b} demonstrate the occurrence of the velar after any back vowel.\largerpage

\ea%5
\label{ex:15:5}Dorsal fricatives in \ipi{Langenlutsch} (\ipi{Schönhengst}):
\ea\label{ex:15:5a} liχt  \tab  [lɪçt] \tab Licht \tab ‘light’ \tab 33\\
knęχt \tab [knɛçt] \tab Knecht \tab ‘vassal’ \tab 8\\
ɑiχ  \tab  [ɑiç] \tab ich \tab ‘I’ \tab 27
\ex\label{ex:15:5b} kūx  \tab  [kuːx] \tab Koch \tab ‘cook’ \tab 28\\
nǫxt \tab [nɔxt] \tab Nacht \tab ‘night’ \tab 29\\
dōx  \tab  [doːx] \tab Dach \tab ‘roof’ \tab 28\\
toxt  \tab  [tox] \tab Docht \tab ‘wick’ \tab 33\\
brɑux \tab [brɑux] \tab Bruch \tab ‘fracture’ \tab 28
\ex\label{ex:15:5c} štɑrx \tab [ʃtɑrəx] \tab stark \tab ‘strong’ \tab 41\\
furx \tab [fʊrəx] \tab Furche \tab ‘furrow’ \tab 41\\
khirx \tab [kʰɪrəx] \tab Kirche \tab ‘church’ \tab 41
    \z
\z 

Janiczek is clear that velar [x] also surfaces after [r], which is realized as the tongue-tip trill (p. 6); see \REF{ex:15:5c}. In his discussion of vowels in the context after /r/ plus labial or velar consonants (p. 41) Janiczek notes that there is a weak epenthetic vowel (“schwacher Sprossvokal”) between the rhotic and velar. He transcribes that vowel in some (but not in all) examples as ⟦\textsuperscript{e}⟧, which is his symbol for a short \isi{schwa} ([ə]). Janiczek writes (p. 41) that the epenthetic vowel is present in the context between [r] and [x] even though he does not always include it in his phonetic transcriptions.

The data in \REF{ex:15:5} point to the common pattern whereby /x/ surfaces as palatal after any front vowel. The significance of \ipi{Langenlutsch} is that the epenthetic vowel in \REF{ex:15:5c} is followed by velar [x] and not palatal [ç]. Recall from \sectref{sec:5.4} that \isi{Schwa Epenthesis} is very common among German dialects but that the overwhelming pattern is for the epenthetic vowel to be followed by the palatal fricative [ç]; see also \sectref{sec:12.8.1}. The palatal realization is a consequence of \isi{Schwa Fronting-2}: /Vlx/→{\textbar}Vləx{\textbar}→{\textbar}Vl\textbf{ə̟}x{\textbar}→[Vl\textbf{ə̟ç}]. The data in \REF{ex:15:5c} can be accounted for straightforwardly if \isi{Schwa Epenthesis} but not \isi{Schwa Fronting-2} is active: /Vlx/→[Vləx]. \ipi{Langenlutsch} is the only German dialect discovered in the present survey with an epenthetic vowel but without \isi{Schwa Fronting-2}.

From the formal perspective, \isi{Velar Fronting-13} (=\ref{ex:15:4}) is active in Langenlutch. Given that the set of triggers consists solely of front vowels, there is no interaction between that process and \isi{Schwa Fronting-2}.

\citet{Graebisch1915} gives a phonetically transcribed text in the \ipi{Rathsdorf} dialect. Velars occur after back vowels (=\ref{ex:15:6a}), the vocalized-r (=\ref{ex:15:6b}), and palatals after front vowel (=\ref{ex:15:6c}).

\ea%6
\label{ex:15:6}Dorsal fricatives in \ipi{Rathsdorf} (\ipi{Schönhengst}):
\ea\label{ex:15:6a} nochpr̥ \tab [noxpr̩] \tab Nachbar \tab ‘neighbor’\\
kǫchl \tab [kɔxl̩] \tab Küche \tab ‘kitchen’\\
rachen \tab [rɑxən] \tab rechnen \tab ‘calculate-\textsc{inf}’
\ex\label{ex:15:6b} kīəćh \tab [kiːəç] \tab Kirche \tab ‘church’
\ex\label{ex:15:6c} ićh  \tab  [iç] \tab ich \tab ‘I’\\
mećht \tab [meçt] \tab möchte \tab ‘would like-\textsc{1}\textsc{sg}’
\z
\z 

The interesting example is \REF{ex:15:6b}, which indicates that \isi{r-Vocalization} has applied (indicated as ⟦ə⟧) but not epenthesis (recall [kʰɪrəx] from \ref{ex:15:5c}). The occurrence of the palatal fricative after the vocalized-r is common throughout many of the areas discussed in previous chapters (including \il{Standard German}StG). However, the realization of /x/ as [ç] after the vocalized-r is an anomaly in this particular region because other places in \ipi{Schönhengst} discussed below have [x] in that context. There are two options regarding the analysis of [ç] in \REF{ex:15:6b}: (a) It is synchronically derived from /x/ on the basis of the /i/ preceding the vocalized-r (as in Lower Bavarian; \sectref{sec:13.5.2}), e.g. /kiːrx/→{\textbar}kiːəx{\textbar}→[kiːəç]; or (b) it is an underlying palatal /ç/, as in some of the dialects discussed in \chapref{sec:7}, as well as \il{Standard German}StG (\chapref{sec:17}). Option (a) can be shown to be correct if [x] but not [ç] were to surface after the vocalized-r when preceded by a back vowel. No such examples were found in \citet{Graebisch1915}. From the formal perspective both \isi{Velar Fronting-1} (=\ref{ex:15:2}) and \isi{Velar Fronting-13} (=\ref{ex:15:4}) are compatible with either (a) or (b).

\citet{Seemüller1908c} presents phonetically transcribed texts for speakers from \ipi{Altstadt}. Some data from that work are listed in \REF{ex:15:7}.

\ea%7
\label{ex:15:7}Dorsal fricatives in \ipi{Altstadt} (\ipi{Schönhengst}):
\ea\label{ex:15:7a} glaix \tab [glaiç] \tab gleich \tab ‘soon’\\
gəšixt \tab [gəʃiçt] \tab Geschichte \tab ‘story’\\
ʃlęxtɒ \tab [ʃlɛçtɐ] \tab schlechter \tab ‘bad-\textsc{infl}’
\ex\label{ex:15:7b} khūχlęfl \tab [kʰuːxlɛfl̥] \tab Kochlöffel \tab ‘wooden spoon’\\
nuχ \tab  [nux] \tab nach \tab ‘after’\\
toχtɒ \tab [toxtɐ] \tab Tochter \tab ‘daughter’\\
mǫχŋ \tab [mɔxŋ] \tab machen \tab ‘do-\textsc{inf}’
\ex\label{ex:15:7c} duɒχs \tab [duɐxs] \tab durchs \tab ‘through the’
\z
\z 

The items listed above show that the palatal (⟦x⟧) surfaces after a front vowel and the velar (⟦χ⟧) after a back vowel. Alstadt differs from \ipi{Langenlutsch} in that /r/ is vocalized in the former (=\ref{ex:15:7c}), after which [x] surfaces (cf. \ref{ex:15:6b} from \ipi{Rathsdorf}). The occurrence of [x] after the vocalized-r has been discussed earlier (e.g. \sectref{sec:3.5}, \sectref{sec:4.3}, \sectref{sec:13.5.2}). In short, the data in \REF{ex:15:7} are consistent with either \isi{Velar Fronting-1} (=\ref{ex:15:2}), which is bled by \isi{r-Vocalization} in \REF{ex:15:7c}, or \isi{Velar Fronting-13} (=\ref{ex:15:4}), which does not interact with \isi{r-Vocalization}.

\citet{Benesch1979} is without a doubt the most valuable source for velar fronting in \ipi{Schönhengst}. The book is devoted to the historical phonology of vowels and consonants (with separate symbols for velars and palatals). What is more, Benesch compares the sound structure of multiple places within \ipi{Schönhengst}, thereby providing a valuable source for how a rule type (velar fronting) can differ from place to place in a small area.

It is clear from the data provided by Benesch that all of the places within \ipi{Schönhengst} he discusses have some version of velar fronting \citep[144--145]{Benesch1979}. The basic generalization is unsurprising: [ç] (=⟦χ⟧) occurs after front vowels and [x] (=⟦x⟧) after back vowels. In the context after a consonant the predominant pattern is for [x] to surface after the coronal rhotic [r] throughout the area with the exception of \ipi{Mährisch Hermersdorf}, which has [ç]. Benesch (p. 144) writes “Nach r erscheint gewöhnlich x, nur H. (Z.G.) neigt in diesem Falle zur χ{}-Lautung”. (“After r usually only x occurs, but in [Mährisch] Hermersdorf (the Zwittauer region) it ([x]) tends to be pronounced in this context as χ”). In \REF{ex:15:8} I give a representative selection of data in Benesch’s transcription system with dorsal fricatives in the context after front vowels (=\ref{ex:15:8a}), back vowels (=\ref{ex:15:8b}), and [r] (=\ref{ex:15:8c}). The abbreviations in the six columns correspond to the six towns of Michelsdorf (Mi),  \ipi{Mährisch Hermersdorf}  (H.), \ipi{Vorder-Ehrnsdorf} (E.), \ipi{Augezd} (A.), Kornitz (K.), and Rehsdorf (Re.). Michelsdorf and Rehsdorf do not have dorsal fricatives after [r] because the latter sound is vocalized in coda position. As in \ipi{Altstadt} (=\ref{ex:15:7c}), /x/ surfaces as [x] after the vocalized-r in those two places, e.g. ⟦khīəx⟧ ‘church’.

\ea%8
\TabPositions{.2\textwidth, .3\textwidth, .4\textwidth, .5\textwidth, .6\textwidth, .7\textwidth}
% \NumTabs{7}
\label{ex:15:8}Dorsal fricatives in six places in \ipi{Schönhengst}:
\begin{xlist}
\sn[] \tab \tab Mi. \tab H. \tab E. \tab A. \tab Ko. \tab   Re.\\

\ex\label{ex:15:8a} ‘sting’ \tab štīχ \tab štęiχ \tab štɑiχ \tab št\textsuperscript{ō}iχ \tab štīχ \tab štɑiχ\\
‘cattle’ \tab fīχ \tab fęiχ \tab fɑiχ \tab f\textsuperscript{ö}īχ \tab fīχ\\
‘oak’ \tab ɑiχ \tab oiχ \tab oɒiχ \tab ǫiχ \tab  \tab tɑiχ\\
‘pond’ \tab tɑiχ \tab  \tab tɑiχ \tab tɑiχ \tab  \tab tɑiχ\\
‘bad’ \tab  \tab  \tab šlęχt \tab  \tab  \tab šlęχt\\
‘easy’ \tab lɑeχt \tab lɑeχt \tab lęχt \tab  \tab lęχt \tab lęχt\\
‘paint-\textsc{pret}’ \tab štrīχ \tab štręiχ \tab štrɑiχ \tab štrīχ \tab  \tab štrɑiχ
\ex\label{ex:15:8b} ‘weak’ \tab šwǫx \tab šwōx \tab šwōx \tab šwǫx \tab šwōx\\
‘wick’ \tab tọ̄xt \tab toxt \tab toxt \tab tōxt \tab toxt \tab toxt\\
‘shoe’ \tab šụ̄x \tab šiᵒx \tab šaux \tab šᵒūx \tab šūx \tab šaux\\
‘hose’ \tab šlɑux \tab  \tab  \tab  \tab šlɑux \tab šlɑx\\
\ex\label{ex:15:8c} ‘church’ \tab  \tab khiərχ \tab khiərx \tab khiərx \tab khiərx\\
‘through’ \tab  \tab duɒrχ \tab duɒrx \tab duɒrx\\
‘lark’ \tab  \tab lɑrx \tab lɑrx
\end{xlist} 
\z

Benesch also provides a number of maps. The most important ones for present purposes are Maps 11 and 14. The former depicts the realizations of /rx/ in \ipi{Schönhengst} for the word ‘church’. Map 14 for \textit{Köchin} ‘cook-\textsc{fem}’ show that the palatal occurs after a front vowel ([i] or [e]) throughout \ipi{Schönhengst}.

The distribution of dorsal fricatives in the town of \ipi{Rothmühl} \citep{Benesch1979} differs from the distribution of those sounds in the other six places listed in \REF{ex:15:8}. As indicated in \REF{ex:15:9}, palatal [ç] is restricted to the context after a front unrounded vowel (=\ref{ex:15:9b}), while velar [x] occurs after a back vowel (=\ref{ex:15:9a}), [r] (=\ref{ex:15:9d}), or a front rounded vowel (=\ref{ex:15:9c}).

\ea%9
\label{ex:15:9}Dorsal fricatives in \ipi{Rothmühl} (\ipi{Schönhengst}):
\ea\label{ex:15:9a} hūx \tab  [huːx] \tab hoch  \tab  ‘high’ \tab 75\\
wüox \tab  [wyox] \tab Woche  \tab  ‘week’ \tab 145\\
rōx  \tab   [roːx] \tab Rauch  \tab  ‘smoke’ \tab 145\\
tǫxt \tab  [tɔxt] \tab Docht  \tab  ‘wick’ \tab 150\\
braux \tab  [braux] \tab Brauch  \tab  ‘custom’ \tab 50\\
liɒxt \tab  [liɒxt] \tab Licht  \tab  ‘light’ \tab 58\\
rɑxt \tab  [rɑxt] \tab recht  \tab  ‘right’ \tab 16\\
\ex\label{ex:15:9b} štīχ \tab  [ʃtiːç] \tab Stich  \tab  ‘sting’ \tab 25\\
fīχ  \tab   [fiːç] \tab Vieh  \tab  ‘cattle’ \tab 103\\
štrīχ \tab [ʃtriːç] \tab strich  \tab  ‘paint-\textsc{pret}’ \tab 104\\
rɑiχ \tab  [rɑiç] \tab reich  \tab  ‘rich’ \tab 106\\
lɑeχt \tab [lɑeçt] \tab leicht  \tab  ‘easy’ \tab 47\\
reχtn̥ \tab  [reçtn̩] \tab richten  \tab  ‘judge\textsc{{}-inf}’ \tab 144\\
\ex\label{ex:15:9c} t\={ü}x  \tab   [tyːx] \tab Tuch  \tab  ‘towel’ \tab 62\\
š\={ü}x \tab  [ʃyːx] \tab Schuh  \tab  ‘shoe’ \tab 145\\
ɡər\={ü}x \tab  [gəryːx] \tab Geruch  \tab  ‘smell’ \tab 36\\
z\={ü}xŋ̥ \tab  [zyːxŋ̍] \tab suchen  \tab  ‘seach-\textsc{inf}’ \tab 139\\
\ex\label{ex:15:9d} khīərx \tab  [kʰiːərx] \tab Kirche  \tab  ‘church’ \tab 145\\
düɒrx \tab [dyɒrx] \tab durch  \tab  ‘through’ \tab 38, 89\\
khwɑrx \tab [kʰwɑrx] \tab quer  \tab  ‘across’ \tab 113
    \z
\z 

Front rounded vowels occur (as phonemes) throughout  \ipi{Schönhengst}, but they are rare in the context before dorsal fricatives. ⟦\={ü}⟧ (=[yː]) -- historically [uo] -- is the only front rounded vowel found before dorsal fricatives. Benesch describes that sound as equivalent to the long front rounded vowel [yː] in \il{Standard German}StG \textit{früh} ‘early’ (p. 5). The change from [uo] to [yː] occurred throughout the Rothmühler Gebiet \citep[61]{Benesch1979}; hence, the data in \REF{ex:15:9d} may hold for other towns in that area as well.

The data in \REF{ex:15:9} indicate that \ipi{Rothmühl}  has a rule of velar fronting which applies to /x/ in the context after front unrounded vowels (=Trigger Type A'{}' from \tabref{tab:12.29}). The restricted context is expressed below:

\ea%10
\label{ex:15:10}\isi{Velar Fronting-12}:\\
\begin{forest}
[,phantom
  [\avm{[−round]} [\avm{[coronal]},name=target,tier=word]]
  [\avm{[−son\\+cont]},name=source [\avm{[dorsal]}, tier=word]]
]
\draw [dashed] (target.north) -- (source.south);
\end{forest}
\z 

Recall from \sectref{sec:12.6.1} that the restriction of velar fronting triggers to front unrounded vowels is a very rare pattern which is otherwise only attested in two LG dialects. The only other example of Trigger Type A'{}' uncovered in the present survey is \ipi{South Mecklenburg} (\citealt{Jacobs1925a, Jacobs1925b, Jacobs1926}).

\section{{Giazza/Dreizehn} {Gemeinden}}\label{sec:15.4}

Several German-language islands are located in Northeast Italy (\mapref{map:35}). \citet[906]{Wiesinger1983b} identifies three Bav (Cimbrian) islands in that area: (a) \ipi{Dreizehn Gemeinden} (Thirteen Communities) in the province of Verona, (b) \ipi{Sieben Gemeinden} (Seven Communities) in the province of Vicenza, and (c) the communities of Folgaria, Lavarone, and Lucerna in the province of Trentino. According to \citet{Wiesinger1983b}, (a--c) were settled by speakers of Bav dialects (Cimbrian) beginning in the twelfth century.

\vfill
\begin{map}[H]
% \includegraphics[width=\textwidth]{figures/VelarFrontingHall2021-img041.png}
\includegraphics[width=\textwidth]{figures/Map35_15.3.pdf}
\caption[Northeast Italy]{Northeast Italy. Rectangles indicate the presence of some version of velar fronting (postsonorant and/or word-initial), and the circles show the absence of velar fronting. 1=\citet{Bacher1905}, 2=\citet{Schweizer1939}, 3=\citet{Mayer1971}, 4=\citet{Kranzmayer1981}, 5=\citet{Rowley1986}, 6=\citet{Tyroller2003}.}\label{map:35}
\end{map}\vfill\pagebreak

The sources for (b--c) indicate that there is no velar fronting, e.g. Luserna (\citealt{Bacher1905}, \citealt{Tyroller2003}), \ipi{Sieben Gemeinden} \citep{Kranzmayer1981}. The UG dialect of Fersentalerich (Mòcheno) spoken in \ipi{Fersental} \citep{Rowley1986} is likewise characterized by the absence of velar fronting. Recall that \sectref{sec:12.9.1} contained some remarks on coarticulatory fronting as described in \citet{Kranzmayer1981} and \citet{Rowley1986}.

Two sources for the \ipi{Giazza} (including \ipi{Dreizehn Gemeinden}) in (a) above indicate that velar fronting is active. The first of those sources is \citet{Mayer1971}, whose speakers have both [x] and [ç]. Mayer proposes a treatment of those sounds cast in traditional phonemic theory, according to which [x] and [ç] derive from /x/.  [h] is also included as an allophone of /x/ since it is restricted in its distribution to word-initial position before vowels, while [x] and [ç] only occur after a sonorant. As indicated below, [h] surfaces word-initially before vowels (=\ref{ex:15:11a}), while [x] occurs after a back vowel (=\ref{ex:15:11b}) and [ç] after a front vowel (=\ref{ex:15:11c}) or coronal sonorant consonant (=\ref{ex:15:11d}). The phonetic transcriptions in \REF{ex:15:11} are taken directly from \citet{Mayer1971}. The author is clear that [ç] surfaces after front vowels (“Vorder-Zungen-Vokaleˮ), although [i] is the only example Mayer gives for a front vowel preceding [ç].


\TabPositions{.2\textwidth, .4\textwidth, .6\textwidth, .75\textwidth}
\ea%11
\label{ex:15:11}Dorsal fricatives in \ipi{Giazza}/\ipi{Dreizehn Gemeinden}:
\ea\label{ex:15:11a}\relax [hurrt] \tab Hürde \tab ‘hurdle’ \tab 49
\ex\label{ex:15:11b}\relax [hǫax] \tab hoch \tab ‘high’ \tab 49\\
\relax[maxan] \tab machen \tab ‘do-\textsc{inf}’ \tab 49\\
\relax[foxlox] \tab Fuchsloch \tab ‘foxhole’ \tab 49\\
\relax[pruax] \tab Hose \tab ‘pants’ \tab 49\\
\relax[gəmaxt] \tab gemacht \tab ‘do-\textsc{part}’ \tab 52\\
\ex\label{ex:15:11c}\relax[niçt] \tab nicht \tab ‘not’ \tab 52\\
\relax[siçela] \tab Sichel \tab ‘sickle’ \tab 49
\ex\label{ex:15:11d}\relax[khalç] \tab Kalk \tab ‘lime’ \tab 49\\
\relax[starç] \tab stark \tab ‘strong’ \tab 49
\z 
\z 

The data in \REF{ex:15:11} display the default pattern whereby velar fronting occurs after a coronal sonorant. That pattern is expressed formally with \isi{Velar Fronting-1} (=\ref{ex:15:2}).

A second source for velar fronting in \ipi{Giazza} (including \ipi{Dreizehn Gemeinden}) is one predating \citet{Mayer1971} by over thirty years, namely \citet{Schweizer1939}. The latter work consists of a series of phonetically transcribed texts of varying length dealing with a wide variety of topics. The significance of those texts is that they can shed some light on the state of velar fronting in a German-language island in the early part of the twentieth century because they distinguish [ç] (=⟦x⟧) and [x] (=⟦χ⟧). Brief remarks on the phonetics of those two sounds are made in the section on phonetic symbols on p. 11. In the list of consonants on that page, Schweizer also includes the \isi{affricate} ⟦kχ⟧. Although he says nothing on p. 11 about its place of articulation, it is clear from the texts that both velar (⟦kχ⟧) and palatal (⟦kx⟧) affricates occur.

A comparison of the texts presented in Schweizer’s work indicates that they were based on the speech of many different informants. It is possible to draw this conclusion because the distribution of the dorsal fricatives in any one text can be shown to be slightly different from the distribution of the same sounds in another text. Unfortunately, Schweizer does not indicate where his informants are from; hence, it is not possible to make a statement on the precise geography of velar fronting in the Cimbrian language islands of Northeast Italy (in the area in and around \ipi{Giazza}).\footnote{{The linguistic atlas for this region (ZFSA) -- also authored by Bruno Schweizer -- provides a number of maps for the German-language islands of Northeast Italy, including \ipi{Sieben Gemeinden} and \ipi{Fersental}. As noted by Stefan Rabanus in the recent (2012) commentary (ZFSA: 25), \citegen{Schweizer1939} distinction between [x] and [ç] is not indicated on those maps. Rabanus opines in the commentary for Map 114 for} \textrm{\textit{Furche}} \textrm{‘furrow’ (p. 284) that Schweizer’s ⟦x⟧ can be interpreted as [ç].}}

I give a brief synopsis of the state of velar fronting in \citet{Schweizer1939} by comparing the distribution of velars (⟦χ⟧=[x], ⟦kχ⟧=[kx]) and palatals (⟦x⟧=[ç], ⟦kx⟧=[kç]) in three of his texts. Many of those texts are only a few sentences long, while others consist of between one and two pages. I have selected below three longer texts in order to ensure that enough tokens are present to draw generalizations on the occurrence of the dorsal sounds in question. The velars and palatals in the statistics summarized in \tabref{tab:fromex:15:12} include both fricatives and affricates. I consider the distribution of those sounds both word-initially and in postsonorant position. In both of those contexts I take into consideration the nature of the adjacent sound, where FV=front vowel, BV=back vowel, and CC=coronal sonorant consonant. There is no evidence that finer-grained distinctions are necessary, e.g. high front vowels vs. mid front vowels. The slash (/) indicates context, e.g. ‘P/BV’ for \tabref{tab:fromex:15:12}(a) means that the palatal is in word-initial position followed by a back vowel and for \tabref{tab:fromex:15:12}(b) that the palatal is situated after a back vowel. The number in each row in bold is the one that I interpret as an irregularity.

\begin{table}%12
\caption{Distribution of velars and palatals in three texts from \citet{Schweizer1939}. Wi.: Word-initial; Ps.: Postsonorant; P.: Palatal; V.: Velar.}
\label{tab:fromex:15:12}
\begin{tabular}{lcrrrrrr}
\lsptoprule
    & Text no. & P./BV & P./FV & P./CC & V./FV & V./BV & V./CC\\\midrule
a. Wi. & 31 & 33 & 0 & 0 &  \textbf{1} & 0 & 0\\
b. Ps. & 31 & 4 & 5 & 5 & 0 & \textbf{3} & 0\\\tablevspace
c. Wi. & 36 & 3 & 3 & 3 & 0 & 0 & 0 \\
d. Ps. & 36 &  \textbf{1}  & 11 & 0 & 0 & 5 & 0\\\tablevspace
e. Wi. & 38 & \textbf{2} & 5 & 1 & \textbf{2} & 36 & 0 \\
f. Ps. & 38 & 0 & 24 & 2 & \textbf{12}  & 20 & \textbf{1}\\
\lspbottomrule
\end{tabular}
\end{table}

Consider first the word-initial context. Since palatals occur in a number of tokens even before a back vowel in Text 31, it is fairly clear that this pattern reflects \isi{nonassimilatory velar fronting}. Examples in that context include ⟦kxôfft⟧ (=[kçɔfft]) ‘buy-\textsc{inf}’, ⟦kxuejer⟧ (=[kçuejer]) ‘shepherd-\textsc{pl} (for cows)’.  In \chapref{sec:14} I showed that that type of pattern involved the restructuring of historical velars as underlying palatals and that there is therefore no synchronic rule, e.g. [kçuejer] is /kçuejer/. Word-initial velar fronting in Text 36 is assimilatory because palatals are surfacing only in the context before coronal sonorants. Text 38 likewise appears to illustrate assimilatory velar fronting in word-initial position, although there are four irregularities.

In postsonorant position velar fronting is nonassimilatory in Text 31 (with three irregularities) but assimilatory in Text 36 (with one irregularity). Two examples from Text 36 are ⟦kxnêxt⟧ (=[kçnɛçt]) ‘vassal’ and ⟦maχen⟧ (=[mɑxən]) ‘do-\textsc{inf}’.  Text 38 may also reflect the assimilatory pattern, although it is interesting that the speaker(s) on which the data are based have a larger number of irregularities \REF{ex:15:13}. The assimilatory pattern described above is captured formally with \isi{Velar Fronting-1} (=\ref{ex:15:2}) or the mirror-image process for word-initial position, stated in \REF{ex:15:13}:

\ea%13
\label{ex:15:13}
Wd-Initial \isi{Velar Fronting-8}:\\
\begin{forest}
[,phantom
  [\avm{[−son\\+cont]},name=source [\avm{[dorsal]},tier=word]]
  [\avm{[+son]} [\avm{[coronal]},name=target,tier=word]]
]
\draw [dashed] (source.south) -- (target.north);
\node [left=1ex of source] {\textsubscript{wd} [};
\end{forest} 
\z
              
\section{{Gottschee}}\label{sec:15.5}



\ipi{Gottschee} was a German-language island in South Slovenia which corresponds roughly to the modern-day municipality of Kočevye (\mapref{map:36}). The area was settled between 1325 and 1360 by speakers of \il{South Bavarian}SBav from Upper Carinthia (Oberkärnten) and East \ipi{Tyrol} (Osttirol; \citealt{Wiesinger1983b}: 907--908).

\ip{Schönhengst}
\ip{Gottschee}
\begin{map}
% \includegraphics[width=\textwidth]{figures/VelarFrontingHall2021-img042.png}
\includegraphics[width=\textwidth]{figures/Map36_15.4.pdf}
\caption[{Gottschee}]{{Gottschee}. Places with velar fronting (postsonorant and/or word-initial) are indicated with squares and places without velar fronting with circles. 1=\citet{Tschinkel1908}, 2=\citet{Seemüller1909d}, 3=\citet{Wolf1982}, 4=\citet{Lipold1984}.}\label{map:36}
\end{map}

Several studies have investigated the sound structure of the German dialects of \ipi{Gottschee}. One of the earliest is \citet{Tschinkel1908}, who detected no velar fronting in the town of Lichtenbach (recall \sectref{sec:12.9.1}). A more recent work is \citet[37]{Wolf1982}, who is clear that there is no velar fronting in the area of \ipi{Suchener Tal}. Those works contrast with \citet{Seemüller1909d} for \ipi{Mitterdorf} and \citet{Lipold1984} for the entire \ipi{Gottschee} area because both of those studies indicate that velar fronting was active. In the remainder of this section I discuss the data from the latter two works.\footnote{Velar fronting is absent in the other former German-language island of Slovenia, namely \ipi{Zarz} (\citealt{Lessiak1959}; \mapref{map:3}). The \ili{Slovene} language possesses [x](/x/) but no corresponding palatal (\citealt{Greenberg2006}). There is also no allophonic process fronting /x/ in Slovene.}

\citet{Lipold1984} is an extremely valuable work on the sound structure of the dialects of \ipi{Gottschee}. That comprehensive study offers an in-depth synchronic treatment of the phonology of the entire area, concentrating specifically on the seven villages of Suchen, Hinterberg, Klindorf, Niedermösel, Reichenau, Rodine, and Hornberg. The book is accompanied with a tape recording of native speakers from those places -- recordings presented in written form on pp. 449--529 in phonemic transcriptions (/…/) and narrow phonetic ones ([…]). \citet{Lipold1984} contains copious data from all seven of the villages referred to above -- data indicating that those places had a version of velar fronting to be discussed below. The data in the seven places do not appear to differ from one another in any significant way with respect to the patterning of velars and palatals. I therefore concentrate on one particular place (Hinterberg) as a representative of all of \ipi{Gottschee}.

The material discussed below shows that the velar fricative ([x]), the velar stop ([k]), and the velar \isi{affricate} ([kx]) all have palatal allophones. The rule accounting for surface palatals (velar fronting) is triggered by all and only front vowels \citep[211--212]{Lipold1984}. \ipi{Gottschee} differs from other German dialects because it possesses \isi{central vowels} (distinct from \isi{schwa}) which contrast with front vowels and back vowels. For example, there are the two phonemic short front vowels /i e/, two phonemic short back vowels /u o/, and two phonemic short \isi{central vowels} ⟦ü ö⟧, which I retain in Lipold’s transcription system.\footnote{{The datasets presented below indicate that the reflexes of the \isi{central vowels} of \ipi{Gottschee} are often equivalent to front rounded vowels in \il{Standard German}StG (e.g. [y ø]) but that in other cases they correspond to \il{Standard German}StG back vowels (e.g. [u o]).}} In contrast to \il{Standard German}StG, there are no phonetically front rounded vowels like [y ø] \citep[123]{Lipold1984}. The contrast between front vs. central vs. back is captured in Lipold’s feature system with the two binary features [±front] and [±back]. In the present  framework I express the contrast with the two features [coronal] and [dorsal]. That system is given in \tabref{tab:fromex:15:14} for the six short vowels mentioned above, together with the short low back vowel /ɑ/.

\begin{table}%14
\caption{\label{tab:fromex:15:14}Distinctive features for vowels (Gottschee)}
\begin{tabular}{l *7{c}}
\lsptoprule
                & i & e & ü & ö & u & o & ɑ \\\midrule
\relax [coronal] & \ding{51} & \ding{51} &  &  &  &  &   \\
\relax  [dorsal] &  &  &  &  & \ding{51} & \ding{51} & \ding{51}  \\
\relax  [low] &  &  &  &  &  &  & +  \\
\relax  [high] & + & − & + & − & + & − &   \\
\lspbottomrule
\end{tabular}
\end{table}


In the inventory of vowels depicted in \tabref{tab:fromex:15:14} there are front (coronal) vowels, which contrast with back (dorsal) vowels and \isi{central vowels}, which are unmarked for [coronal] and [dorsal].\footnote{{An alternative to \tabref{tab:fromex:15:14} is to analyze the \isi{central vowels} as phonologically [coronal] and to adopt the feature [±round] to distinguish those sounds from front unrounded vowels. In that alternative approach, \isi{phonetic implementation} could capture the fact that ⟦ü ö⟧ are not the same vowels as [y ø] in other German dialects. The analysis of ⟦ü ö⟧ in \tabref{tab:fromex:15:14} can be tested by determining whether or not they pattern phonologically as front for processes other than velar fronting.}}

  Dorsal fricatives in Hinterberg do not occur word-initially, but dorsal affricates and stops do surface in that context: Palatal [cç] surfaces before front vowels (=\ref{ex:15:15a}) and the velar before \isi{central vowels} (=\ref{ex:15:15b}), back vowels (=\ref{ex:15:15c}), or [r] (=\ref{ex:15:15d}). The transcriptions in \REF{ex:15:15} are in Lipold’s system, which employs symbols very similar to the ones I have adopted in this book.

\ea%15
\label{ex:15:15}Word-initial dorsal affricates in Hinterberg
\ea\label{ex:15:15a} cçeːrts\textsuperscript{ɛ} \tab Kerze \tab ‘candle’ \tab 333\\
    cçeːrb\textsuperscript{ɛ} \tab Körbe \tab ‘basket-\textsc{pl}’ \tab 333\\
    cçepf\textsuperscript{ɛ} \tab Köpfe \tab ‘head-\textsc{pl}’ \tab 328
\ex\label{ex:15:15b} kxüxl̩ \tab Küche \tab ‘kitchen’ \tab 327\\
    kxüːts \tab kurz \tab ‘short’ \tab 331
\ex\label{ex:15:15c} kxaːfm̩ \tab kaufen \tab ‘buy-\textsc{inf}’ \tab 334
\ex\label{ex:15:15d} kxrüəkx \tab Krug \tab ‘jug’ \tab 335
\z
\z

The two stops [k] and [c] pattern like the affricates; hence, [c] surfaces before front vowels (=\ref{ex:15:16a}), and [k] before \isi{central vowels} (=\ref{ex:15:16b}), back vowels (=\ref{ex:15:16c}), or liquids (=\ref{ex:15:16d}).\footnote{{According to \citet{Lipold1984} the phonemic vowels of \ipi{Gottschee} have allophones, some of which are present in \REF{ex:15:16}, e.g. ⟦ę⟧ for /e/). The palatal segments in \ipi{Gottschee} occur in the context of all surface front vowels, including front vowels that are allophones}}


\TabPositions{.2\textwidth, .4\textwidth, .65\textwidth, .75\textwidth}
\ea%16
\label{ex:15:16}Word-initial dorsal stops in Hinterberg

\ea\label{ex:15:16a} cęsː\textsuperscript{ɛ} \tab Schultasche \tab ‘book bag’ \tab 315\\
    cęŋkx \tab Fußtritt \tab ‘kick’ \tab 315\\
\ex\label{ex:15:16b} kük\textsuperscript{ɛ} \tab Kuckuck \tab ‘cuckoo’ \tab 315\\
    ˈkürtαːt \tab nackt \tab ‘naked’ \tab 315\\
    ˈkölːər \tab Wamme \tab ‘dewlap’ \tab 315\\
\ex\label{ex:15:16c} kaːɪ̜f \tab Taschenmesser \tab ‘pocket knife’ \tab 315\\
    ˈkɔkaɪtsn̩ \tab gackern \tab ‘cluck-\textsc{inf}’ \tab 315\\
    kǫʃː  \tab  Wagenkorb \tab ‘basket’ \tab 315\\
\ex\label{ex:15:16d} krɔmp\textsuperscript{ɛ} \tab Krampen \tab ‘pick’ \tab 315\\
    klas\textsuperscript{ɛ} \tab Klasse \tab ‘class’ \tab 315\\
    \z
\z

Lipold likewise analyzes palatal [ɟ] and velar [g] as allophones word-initially (p. 370). I do not discuss those two stops because of the sparseness of the data containing them.

The data in \REF{ex:15:17} illustrate the distribution of velar and palatal fricatives in the context after a sonorant: [ç] surfaces after front vowels (=\ref{ex:15:17a}) and [x] after \isi{central vowels} (=\ref{ex:15:17b}), back vowels (=\ref{ex:15:17c}), or [r] (=\ref{ex:15:17d}).

\ea%17
\label{ex:15:17}Postsonorant dorsal fricatives in Hinterberg
\ea\label{ex:15:17a} ˈrɪçtαr \tab Richter \tab ‘judge’ \tab 301\\
    ˈesaɪç \tab Essig \tab ‘vinegar’ \tab 309\\
    gəˈbɪçt \tab Gewicht \tab ‘weight’ \tab 312\\
    glaːɪ̨ç \tab gleich \tab ‘soon’ \tab 313\\
    ˈʑleçtαr \tab schlechter \tab ‘worse-\textsc{infl}’ \tab 301\\
    \textsuperscript{u}ɔːẹç\textsuperscript{ɛ} \tab Eiche \tab ‘oak’ \tab 310\\
    buɔːẹç \tab weich \tab ‘soft’ \tab 312\\
    ʑlęçt \tab schlecht \tab ‘bad’ \tab 322
\ex\label{ex:15:17b}  vrüxt \tab Frucht \tab ‘fruit’ \tab 319\\
    ˈütrüxŋ \tab wiederkäuen \tab ‘chew cud-\textsc{inf}’ \tab 309\\
    gəˈvlöxtn \tab geflochten \tab ‘braid-\textsc{part}’ \tab 320\\
    gəˈvlöːxŋ \tab geflogen \tab ‘fly-\textsc{part}’ \tab 320\\
    rö̹xŋ \tab Roggen \tab ‘rye’ \tab 303\\
    bö̹xŋ \tab Wochen \tab ‘week-\textsc{pl}’ \tab 302\\
    bö̹x\textsuperscript{ɛ} \tab Woche \tab ‘week’ \tab 301\\
    lö̹x  \tab  Loch \tab ‘hole’ \tab 316
\ex\label{ex:15:17c} pruːxtl \tab gebracht \tab ‘bring-\textsc{part}’ \tab 313\\
    dɔx  \tab  Dach \tab ‘roof’ \tab 304\\
    bɔx\textsuperscript{ɛ} \tab Wache \tab ‘sentinel’ \tab 301\\
    ręːʌx \tab Reh \tab ‘deer’ \tab 316\\
    raxt  \tab  recht \tab ‘right’ \tab 316\\
    ɬaːx  \tab  Lauch \tab ‘leek’ \tab 316
\ex\label{ex:15:17d} düːrx \tab durch \tab ‘through’ \tab 312\\
    piːrx\textsuperscript{ɛ} \tab Birke \tab ‘birch tree’ \tab 313\\
    ʃtuːrx \tab stark \tab ‘strong’ \tab 332\\
    mr̩x\textsuperscript{ɛ} \tab Mähre \tab ‘old mare’ \tab 316\\
    vüːrx\textsuperscript{ɛ} \tab Furche \tab ‘furrow’ \tab 332\\
    ʑnǫːʌrxŋ \tab schnarchen \tab ‘snore-\textsc{inf}’ \tab 321
    \z
\z
    
The dataset in \REF{ex:15:18} illustrates the distribution of velar and palatal affricates in the context after a sonorant: [cç] occurs after front vowels (=\ref{ex:15:18a}), and [kx] after \isi{central vowels} (=\ref{ex:15:18b}), back vowels (=\ref{ex:15:18c}), or [r] (=\ref{ex:15:18d}).

\ea%18
\label{ex:15:18}Postsonorant dorsal affricates in Hinterberg
\ea\label{ex:15:18a} dɪcç\textsuperscript{ɛ} \tab dick \tab ‘fat’ \tab 312\\
    ęːʌbɪcç \tab ewig \tab ‘eternal’ \tab 310\\
    tuɔe̥cç \tab Teig \tab ‘dough’ \tab 300\\
    ʑmecçŋ \tab schmecken \tab ‘taste-\textsc{inf}’ \tab 321\\
    ʃtęcçŋ \tab stecken \tab ‘stick-\textsc{inf}’ \tab 323
\ex\label{ex:15:18b} rükx\textsuperscript{ɛ} \tab Rücken \tab ‘back’ \tab 300\\
    tükx \tab Tücke \tab ‘peril’ \tab 314\\
    ʑmükxŋ \tab schmiegen \tab ‘nuzzle-\textsc{inf}’ \tab 321\\
    lükx\textsuperscript{ɛ} \tab Lücke \tab ‘gap’ \tab 301\\
    bǫ̈kx \tab Bock \tab ‘buck’ \tab 302\\
    ʃtǫ̈kx \tab Stock \tab ‘stick’ \tab 323\\
    gəˈʃrǫ̈kxŋ \tab erschrocken \tab ‘scared-\textsc{part}’ \tab 323
\ex\label{ex:15:18c} vlakx \tab Fleck \tab ‘spot’ \tab 320\\
    ˈakxər \tab Äcker \tab ‘field-\textsc{pl}’ \tab 309\\
    ɬɔkx\textsuperscript{ɛ} \tab Lacke \tab ‘village pond’ \tab 301
\ex\label{ex:15:18d}  paːrkx \tab Berg \tab ‘mountain’ \tab 334
    \z
\z 

\citet[370]{Lipold1984} considers the palatal stops [c ɟ] to be allophones of /k g/ in postsonorant position, although the only example found for Hinterberg is the word [ˈglɪclɪç] ‘fortunate’ for [c] (p. 313).

The formal rules for Hinterberg are stated below for word-initial position (=\ref{ex:15:19a}) and postsonorant position (=\ref{ex:15:19b}). The triggers for both rules include all and only front vowels but not \isi{central vowels}, back vowels, or coronal consonants. The target segments for \REF{ex:15:19b} must minimally include the fricative /x/ and the \isi{affricate} /kx/. I opt for a broader set of targets, which also includes the stops /k/ and /g/. Although only one example was found for /k/ and no examples for /g/, I posit the broad set of targets on the basis of Lipold’s characterization of palatal stops as allophones in postsonorant position. For word-initial position (=\ref{ex:15:19a}) the targets must consist of all dorsal obstruents.

\ea%19
    \label{ex:15:19}
    \begin{multicols}{2}
    \ea \isi{Wd-Initial Velar Fronting-6}:\\\label{ex:15:19a}
     \begin{forest}
     [,phantom
         [\avm{[−son]},name=source [\avm{[dorsal]}]]
         [\avm{[−cons]} [\avm{[coronal]},name=target]]
     ]
     \draw [dashed] (source.south) -- (target.north);
     \node [left=1ex of source] {\textsubscript{wd} [};
     \end{forest}
    \ex \isi{Velar Fronting-8}:\\\label{ex:15:19b}
    \begin{forest}
     [,phantom
         [\avm{[−cons]} [\avm{[coronal]},name=target]]
         [\avm{[−son]},name=source [\avm{[dorsal]}]]
     ]
     \draw [dashed] (source.south) -- (target.north);
    \end{forest}
    \z 
    \end{multicols}
\z 

A second description for a \ipi{Gottschee} dialect is \citet{Seemüller1909d}, which is a very brief work consisting of phonetic transcriptions of the \isi{Wenkerbogen}  and other short texts for the \ipi{Mitterdorf} dialect. The transcriptions contain enough words with [ç] (=⟦x⟧) and [x] (=⟦χ⟧) to conclude that the village of \ipi{Mitterdorf} once had a synchronic rule of velar fronting. Consider the examples presented in \REF{ex:15:20}.\footnote{{\ipi{Mitterdorf} also possesses the corresponding lenis fricatives [ʝ] (=⟦γ⟧) and [ɣ] (=⟦ǥ⟧), which I do not discuss because the texts in \citet{Seemüller1909d} contain only a few items with those segments. (The two words found with ⟦γ⟧ occurred after the front vowels [ɪ] and [eː]). The texts in \citet{Seemüller1909d} also contain many words with velar stops ([k kʰ g]), which surface without change after front segments. None of the data presented in that source indicate that velar fronting is active in word-initial position.}} I retain the transcriptions in the original.

\ea%20
\label{ex:15:20}Dorsal fricatives in \ipi{Mitterdorf}:
\ea\label{ex:15:20a} ix  \tab  ich \tab ‘I’ \tab 25\\
    mīlix \tab Milch \tab ‘milk’ \tab 25\\
    gəšixtə \tab Geschichte \tab ‘story’ \tab 26\\
    entlix \tab endlich \tab ‘finally’ \tab 28\\
    tsēxnɑi \tab zehn \tab ‘ten’ \tab 25\\
    šlextə \tab schlechte \tab ‘bad-\textsc{infl}’ \tab 26\\
    dəroixŋ \tab erreichen \tab ‘reach-\textsc{inf}’ \tab 28\\
    lɑixtə \tab leichter \tab ‘easier’ \tab 28
\ex\label{ex:15:20b} bö̹χŋ \tab Wochen \tab ‘week-\textsc{pl}’ \tab 25\\
    nöχ  \tab  noch \tab ‘still’ \tab 25\\
    khöχlefl \tab Kochlöffel \tab ‘wooden spoon’ \tab 26
\ex\label{ex:15:20c} khūχŋ \tab Kuchen \tab ‘cake’ \tab 25\\
    gəprūχt \tab gebracht \tab ‘bring-\textsc{part}’ \tab 27\\
    toχtər \tab Tochter \tab ‘daughter’ \tab 25\\
    moχŋ \tab machen \tab ‘do-\textsc{inf}’ \tab 26\\
    rɑχt  \tab  recht \tab ‘right’ \tab 27\\
    hōɒχ \tab hoch \tab ‘high’ \tab 27\\
    hēɒχtər \tab höher \tab ‘higher’ \tab 27\\
    gəwīəχtət \tab gefürchtet \tab ‘fear-\textsc{inf}’ \tab 28
\ex\label{ex:15:20d} düɒrχs \tab durch \tab ‘through’ \tab 25
\ex\label{ex:15:20e} trökχnən \tab trockenen \tab ‘dry-\textsc{infl}’ \tab 25
\z
\z 

I posit that the features for vowels in \tabref{tab:fromex:15:14} also hold for Mitterwald. Thus, [ç] surfaces after front vowels (=\ref{ex:15:20a}) and [x] after \isi{central vowels} (=\ref{ex:15:20b}), back vowels (=\ref{ex:15:20c}), or [r] (=\ref{ex:15:20d}). One example was found with the velar \isi{affricate} in the context after a front rounded vowel (=\ref{ex:15:20e}), which is consistent with an analysis in which /k/ and /kx/ pattern the same way. The formal rule of velar fronting in \REF{ex:15:21} for \ipi{Mitterdorf} is \isi{Velar Fronting-13} (=\ref{ex:15:4}).

\section{{Grisons}}\label{sec:15.6}\il{Highest Alemannic|(}

In \sectref{sec:6.3} the dialect of \ipi{Obersaxen} was identified as a \il{Highest Alemannic}\is{Walser German}Walser variety of HstAlmc spoken in West Grisons  (Graubünden); \mapref{map:37}. As indicated on that map, \ipi{Obersaxen} is a German-language island because it is encircled by areas populated with speakers of \ili{Romansh}, a language with neither [ç] nor [x]; see \citet{Anderson2016}. There is no question that \ipi{Obersaxen} represents a velar fronting island because \ipi{Obersaxen} itself is a German-language island.


\begin{map}
% \includegraphics[width=\textwidth]{figures/VelarFrontingHall2021-img043.png}
\includegraphics[width=\textwidth]{figures/Map37_15.5.pdf}
\caption[Grisons]{Grisons. Velar fronting (postsonorant and/or word-initial) is depicted with a square and the absence of velar fronting with a circle. 1=\citet{Gröger1914a}, 2= \citet{Gröger1914b}, 3=\citet{Brun1918}, 4=\citet{Meinherz1920}, 5=\citet{Gröger1925}, 6=\citet{Kessler1931}, 7=\citet{Hotzenköcherle1934}. [Source for language borders: Kanton Graubünden in Wikipedia]}\label{map:37}
\end{map}

Recall the generalizations concerning velar fronting in \ipi{Obersaxen}: Velars (/x/ and /kx/) surface as palatal in word-initial position before a nonlow front vowel (\isi{Wd-Initial Velar Fronting-5}) and in postsonorant position after a nonlow front vowel (\isi{Velar Fronting-7}).

\citet[904--906]{Wiesinger1983b} identifies a number of other places in Grisons which are populated with speakers of \isi{Walser German}, but an examination of the sources for those varieties reveals that those places do not have velar fronting. Three examples indicated on \mapref{map:37} are \ipi{Nufenen} \citep{Gröger1914a}, \ipi{Mutten} \citep{Hotzenköcherle1934}, and \ipi{Schanfigg} \citep{Kessler1931}. A more remote (\il{South Bavarian}SBav) variety of German in Grisons without velar fronting is \ipi{Samnaun} \citep{Gröger1925}. (I discuss the status of velar fronting in the data from the linguistic atlas of Switzerland (SDS) in \sectref{sec:15.7}).

The closest place to \ipi{Obersaxen} with velar fronting is \isi{Walser German} variety of \ipi{Vals} \citep{Gröger1914b}. Like \ipi{Obersaxen}, \ipi{Vals} is a German-language island situated in a German-speaking area without velar fronting.\footnote{{\mapref{map:37} also indicates that there is a geographically more distant velar fronting place in North Grisons (\ipi{Maienfeld}; \citealt{Meinherz1920}) which was discussed in \sectref{sec:3.3}; see also \sectref{sec:15.11}.}}

\citet{Gröger1914b} is a phonetically transcribed text from a native speaker of the \ipi{Vals} variety of \il{Highest Alemannic}HstAlmc which reveals that velar fronting is active word-initially in (\ref{ex:15:21}) and in postsonorant position in (\ref{ex:15:22}). For both contexts the sound undergoing velar fronting is either the fricative /x/ or the \isi{affricate} /kx/. The items listed in \REF{ex:15:21} indicate that velar fronting is triggered by front vowels (including low front vowels) but not consonants. In postsonorant position the sounds inducing velar fronting are restricted to nonlow front vowels (\ref{ex:15:22b} vs. \ref{ex:15:22c}) or liquids (in \ref{ex:15:22d}). Recall that these generalizations for triggers are not the same as the ones for \ipi{Obersaxen}.\largerpage[2]


\TabPositions{.15\textwidth, .3\textwidth, .5\textwidth, .75\textwidth}
\ea%21
\label{ex:15:21}Word-initial dorsal fricatives and affricates in \ipi{Vals}:

\ea\label{ex:15:21a} xunt \tab [xunt] \tab kommt \tab ‘come-\textsc{3} \textsc{sg}’ \tab 45\\
    xo  \tab  [xo] \tab gekommen \tab ‘came-\textsc{part}’ \tab 41\\
    xoštə \tab [xoʃtə] \tab kosten \tab ‘cost\textsc{{}-inf}’ \tab 45\\
    xɑn  \tab  [xɑn] \tab kann \tab ‘be able-\textsc{3}\textsc{sg}’ \tab 46
\ex\label{ex:15:21b} xlepf \tab [xlepf] \tab Schläge \tab ‘blow-\textsc{pl}’ \tab 42\\
    xlīs  \tab  [xliːs] \tab kleiner \tab ‘small-\textsc{infl}’ \tab 43
\ex\label{ex:15:21c} xrummə \tab [xrʊmmə] \tab krumme \tab ‘bent-\textsc{infl}’ \tab 41\\
    kxrušt \tab [kxrʊʃt] \tab gekommen \tab ‘come-\textsc{part}’ \tab 46\\
    kxrɑt \tab [krɑːt] \tab gerade \tab ‘just’ \tab 43
\ex\label{ex:15:21d} kχent \tab [kçent] \tab gekannt \tab ‘know-\textsc{part}’ \tab 42\\
    χönə \tab [çønə] \tab können \tab ‘be able-\textsc{inf}’ \tab 43\\
    χætsər \tab [çætsər] \tab Ketzer \tab ‘heretic’ \tab 43
\z 
\ex%22
\label{ex:15:22}Postsonorant dorsal fricatives and affricates in \ipi{Vals}:
\ea\label{ex:15:22a} brūxə \tab [bruːxə] \tab brauchen \tab ‘need-\textsc{inf}’ \tab 43\\
     būx  \tab  [buːx] \tab Bauch \tab ‘stomach’ \tab 42\\
     lōx  \tab  [loːx] \tab Loch \tab ‘hole’ \tab 43\\
     dokxtər \tab [dokxtər] \tab Doktor \tab ‘doctor’ \tab 45\\
     bɑx  \tab  [bɑx] \tab Bank \tab ‘bench’ \tab 42\\
     kmɑxt \tab [kmɑxt] \tab gemacht \tab ‘do-\textsc{part}’ \tab 42\\
\ex\label{ex:15:22b} tsræxt \tab [tsræxt] \tab zurecht \tab ‘justifiably’ \tab 45\\
     mæxtɩɡə \tab [mæxtɪɡə] \tab mächtige \tab ‘powerful\textsc{{}-infl}’ \tab 44\\
     ræxt \tab [ræxt] \tab recht \tab ‘right’ \tab 46\\
     ks\={æ}xɩ \tab [ksæːxɪ] \tab sähe \tab ‘see-\textsc{3}\textsc{sg}.\textsc{subj}’ \tab 43
\ex\label{ex:15:22c} diχ  \tab  [ɪç] \tab you \tab ‘you-\textsc{acc}.\textsc{sg}’ \tab 44\\
     rükχte \tab [rʏkçte] \tab rückte \tab ‘move over-\textsc{pret} \tab 45\\
     fərštekχt \tab [fərʃtekçt] \tab versteckt \tab ‘hide-\textsc{part}’ \tab 42\\
\ex\label{ex:15:22d}  kwürkχt \tab [kwʏrkçt] \tab gewirkt \tab ‘seem\textsc{{}-part}’ \tab 44
\z 
\z 

The rules of velar fronting for Vals are stated in (\ref{ex:15:23}). \isi{Velar Fronting-2} accounts for (\ref{ex:15:22c}). (\ref{ex:15:22d}) requires \isi{Velar Fronting-3}, which is stated below in (\ref{ex:15:27b}).  Note that the set of triggers is not the same in word-initial and postsonorant position. Although that finding is rare among German dialects, it is not unattested (\sectref{sec:14.7}).

\ea%23
\label{ex:15:23}
\begin{multicols}{2}
\ea \isi{Velar Fronting-2}:\\\label{ex:15:23a}
    \begin{forest}
    [,phantom
        [\avm{[−low]} [\avm{[coronal]},name=target,tier=word]]
        [\avm{[−son\\+cont]},name=source [\avm{[dorsal]},tier=word]]
    ]
    \draw [dashed] (source.south) -- (target.north);
    \end{forest}
\ex \isi{Wd-Initial Velar Fronting-3}:\\\label{ex:15:23b}
    \begin{forest}
    [,phantom
        [\avm{[−son\\+cont]},name=source [\avm{[dorsal]},tier=word]]
        [\avm{[−cons]} [\avm{[coronal]},name=target,tier=word]]
    ]
    \draw [dashed] (source.south) -- (target.north);
    \node [left=1ex of source] {\textsubscript{wd} [};
    \end{forest}
\z 
\end{multicols}
\z 

As pointed out elsewhere in this book, one must take care in drawing conclusions on velar fronting based on a short text. Although \citet{Gröger1914b} leaves little doubt that velar fronting was active in \ipi{Vals} over a century ago, it is also possible that longer \ipi{Vals} texts from that time frame or from a later time period may reveal that the conclusions drawn here concerning triggers are in need of modification.\footnote{{Two inconsistencies in \citet{Gröger1914b} are: (i) One example indicates that [x] surfaces after a consonant ([l]), i.e. ⟦kwɑlxət⟧ ‘churn-}\textrm{\textsc{part}}\textrm{’ (cf. \ref{ex:15:22d}); (ii) One item has word-initial velar [kx] before [æ], i.e. (⟦kxærli⟧ (=[kxærli]) ‘fellow-\textsc{pl}’ (cf. the third example in \ref{ex:15:21d}). I assume that the inconsistencies here -- liquids sometimes do and sometimes do not trigger postsonorant fronting, low front vowels sometimes do and sometimes do not trigger word-initial fronting -- fall into the domain of irregularities documented for LG dialects (\sectref{sec:12.8.3}). Another set of examples in the text only appears to be irregular: If word-initial /x/ or /kx/ occur before a liquid then those obstruents surface as the corresponding palatals if the vowel following the liquid is nonlow and front, e.g. ⟦kχrümmɩ⟧ ‘state of being bent’. Several other examples were found in \citet{Gröger1914b} suggesting that nonlow front vowels but not liquids trigger word-initial fronting. The mirror-image generalization is true for postsonorant velar fronting in \ipi{Obersaxen} (\sectref{sec:6.3.2}).}}

It is interesting to consider the description of two of the non-velar fronting varieties of \isi{Walser German} referred to above. The first is \citet{Kessler1931}, who describes a dialect spoken in \ipi{Schanfigg}. Velar /x/ is realized throughout the region as [x] (=⟦x⟧), although the author notes that there are some isolated pockets where [ç] (=⟦χ⟧) occurs. \citet[105]{Kessler1931} writes:

\begin{quote}\sloppy
Das reibegeräusch von kx und x klingt, bes. in der nachbarschaft palataler vocale, bedeutend weniger velar und leiser als in den meisten der nördlichern Schweizermaa. Am stärksten fällt dies in Ar. auf. – Palatalen reibelaut höre ich ausnahmsweise in Lw. und von einer alten Frau in Cf.: χind, halmiχts (‘halmiges’) gras, iχχonnə ‘hinein gekonnt’, i χennə-nə níd ‘ich kenne ihn nicht’, betteχχɩ ‘bettdecke’ usw. ....

“The frication noise in kx und x sounds, especially in the neighborhood of palatal vowels, considerably less velar and quieter than in most of the more northern Swiss dialects. The most prominent [of these dialects] is Ar. [Arosa] – I hear the palatal fricative exceptionally in Lw. [Langwies] and from an old woman in Cf. [Calfreisen]: χind ‘child’, halmiχts (‘pertaining to a blade’) of grass, iχχonnə ‘able to go in’, i χennə-nə níd ‘I don’t know him’, betteχχɩ ‘blanket’ etc. ...ˮ
\end{quote}

\citet[316--317]{Hotzenköcherle1934} makes the same type of observation as Kessler for the speech of a single individual -- a woman approximately fifty years old -- in the non-velar fronting region of \ipi{Mutten}. Hotzenköcherle notes that his informant has the palatal (⟦χ⟧) realization in the context before and after high front vowels (⟦i ɩ⟧ and the long counterparts) and occasionally before mid front vowels (⟦e ē\textsuperscript{i}⟧).

The two passages are significant because they suggest that very small-scale velar fronting islands are attested at the level of the individual.

\section{{Interlude:} {The} {interpretation} {of} {symbols} {for} {dorsal} {fricatives} {in} {SDS}}\label{sec:15.7}\il{Highest Alemannic|)}

In the remainder of this chapter I draw on data from the linguistic atlas for Switzerland (SDS). Since that source adopts an unconventional set of symbols and categories for dorsal fricatives, it is essential that an interpretation for the transcription system in that source be put forth. As indicated above, this is the goal of the present section.\footnote{This section has benefited from discussions with Jürg Fleischer.}

The SDS terms and symbols for dorsal fricatives as well as my interpretation thereof are summarized in \tabref{tab:15.1} and commented on below. The SDS maps referred to in the remainder of this chapter for dorsal fricatives are listed in \tabref{tab:15.2}. Like SNiB (\tabref{tab:13.3}), SDS adopts a three-way system for classifying dorsal fricatives. That approach is summarized in \tabref{tab:15.1} with the three categories velar (⟦x⟧), palatal (⟦χ⟧), and prepalatal (⟦χ'⟧/⟦χ'{}'⟧)


\begin{table}
\caption{SDS symbols for dorsal fricatives}
\label{tab:15.1}
\fittable{\begin{tabular}{lll}
\lsptoprule
SDS term and symbol & Phonological Features & Probable phonetic  realization\\\midrule
prepalatal ⟦χ'⟧/⟦χ'{}'⟧ & [coronal, dorsal] & palatal ([ç]) or alveolopalatal ([ɕ])\\
palatal ⟦χ⟧ & [dorsal]  & \isi{prevelar} ([x̟]) or palatal ([ç])\\
            &  (or [coronal, dorsal]) & \\
velar ⟦x⟧ & [dorsal] & velar ([x]) or uvular ([χ])\\
\lspbottomrule
\end{tabular}}
\end{table}

\begin{table}
\caption{Maps from SDS with dorsal fricatives or affricates in word-initial or postsonorant position\label{tab:15.2}}
\begin{tabular}{ll}
\lsptoprule
Examples & Map no.\\\midrule
Kind ‘child’ & II 94\\
drücken ‘press-\textsc{inf}’ & II 95/96\\
trinken ‘drink-\textsc{inf}’ & II 97/98\\
getrunken ‘drink-\textsc{part}’ & II 99/100\\
tränken ‘soak-\textsc{inf}’ & II 101/102\\
Gestank ‘stench’ & II 103\\
Anke (Butter) ‘butter’ & II 104\\
Bank ‘bench’& II 105/106\\
Bänke ‘bench-\textsc{pl}’& II 105/106\\
Bänklein ‘bench-\textsc{dim}’ & II 105/106\\
melken ‘milk-\textsc{inf}’ & II 109\\
Chilche (Kirche) ‘church’ & II 110\\
Zeichen ‘sign’ & II 111\\
Speicher ‘attic’ & II 112\\
bache (backen) ‘bake-\textsc{inf}’ & II 183 \\
Rechen ‘rake’ & II 183\\
rauchen ‘smoke-\textsc{inf}’ & II 201\\
\lspbottomrule
\end{tabular}
\end{table}

The difficulty with the SDS “palatal” category in the first column of \tabref{tab:15.1} can be clearly seen in Map II 94 for \textit{Kind} ‘child’. This map shows the realization of the first sound in that word is “palatal” throughout many if not most parts of Switzerland. In fact, on the basis of this map one would have to conclude that the “palatal” fricative (⟦χ⟧) is far more prevalent than the velar fricative (⟦x⟧). If ⟦χ⟧ and ⟦x⟧ are truly equivalent to [ç] and [x] then the maps in SDS would therefore blatantly  contradict the claims made in the descriptive grammars of SwG dialects cited in \sectref{sec:12.3.1} and indicated on \mapref{map:2} with circles.

In order to understand the discrepancy between the traditional view of SwG /x/ as velar ([x]) or uvular ([χ]) and the one portrayed in SDS it is important to consider the following statement made in the introduction to that linguistic atlas \citep{Hotzenköcherle1962} in the passage on phonetic symbols (pp. 88--89, Footnote 7): “Die Grenze zwischen palatalem χ und velarem x ist praktisch in vielen~ Fällen schwer zu ziehen; χ deckt in unseren Materialien einen sehr~ weiten und insofern sehr fragwürdigen Bereich, während x~ausgesprochene Extremwerte fixiert und in diesem Sinn ...~ zuverlässiger sein dürfte”. (“The boundary between palatal χ and velar x is in practice difficult to draw; in our material χ covers a very broad and in this respect questionable area, while x depicts highly extreme values and ... may be more reliable”).

The preceding quote as well as the similar remarks on the transcriptions for the sounds representing \textit{ch} made on Map II 183 reveal that the authors of SDS consider “palatal” to be a dubious and unreliable realm that cannot be easily assigned a traditional phonetic category. In order to express a place of articulation that is unquestionably more front than ⟦χ⟧, SDS adopts a different category with a unique symbol, namely the prepalatal (“präpalatal”) place of articulation, which is transcribed as ⟦χ'⟧. Prepalatal also includes articulations even more front than ⟦χ'⟧, which are consequently transcribed as ⟦χ'{}'⟧.

As indicated in \tabref{tab:15.1}, I see SDS’s prepalatal ⟦χ'⟧/⟦χ'{}'⟧ as a phonologically front dorsal fricative, which translates into [coronal, dorsal] given the featural system adopted in this book. Thus, prepalatal ⟦χ'⟧/⟦χ'{}'⟧ can be thought of as the (fortis) sound produced by velar fronting represented in previous chapters with the phonetic symbol [ç]. By contrast, SDS’s ⟦x⟧ is phonologically a back dorsal fricative, which is analyzed in my featural system as a simplex [dorsal] sound.

It is not clear how ⟦χ'⟧/⟦χ'{}'⟧) and ⟦x⟧ are actually pronounced. I have provided traditional IPA symbols and diacritics in the final column of \tabref{tab:15.1}, which I comment on here.

Since the sounds traditionally transcribed as [ç] and [x] for varieties of German spoken in Germany can have more than one realization depending on the area and/or the speaker (recall \sectref{sec:1.5}, \sectref{sec:12.9}), it would not be unreasonable to assume that the same holds true for the SDS front dorsal ⟦χ'⟧/⟦χ'{}'⟧ and back dorsal ⟦x⟧; recall \tabref{tab:12.37}. For example, ⟦x⟧ might be pronounced by some speakers by raising the tongue dorsum to the soft palate (=velar [x]) and by others by raising the tongue dorsum to the uvula (=uvular [χ]). Likewise, some speakers might articulate ⟦χ'⟧/⟦χ'{}'⟧ by raising the front part of the dorsum to the hard palate (=palatal [ç]) and others by advancing the tongue body so that a \isi{sibilant} is produced (=[ɕ]), as in the CG dialects investigated in \chapref{sec:10}.

The phonetics of the two extremes (i.e. ⟦χ'⟧/⟦χ'{}'⟧ vs. ⟦x⟧) aside, the important point is that the former is a phonologically front dorsal and the latter a phonologically back dorsal. It is clear from the quote from SDS that the authors do not want to commit themselves as to the status of “palatal” ⟦χ⟧. The interpretation I adopt is that -- at least in the unmarked case -- ⟦χ⟧, like ⟦x⟧, is phonologically a back dorsal, which translates into a representation with a simplex [dorsal] feature. In order to express the fact that ⟦χ⟧ is more front than ⟦x⟧ from the point of view of phonetics, I hold that the unmarked realization of ⟦χ⟧ is a \isi{prevelar} (=[x̟] in a narrow transcription); recall \sectref{sec:12.9.1} and \tabref{tab:12.37}.

A second (marked) option for the realization of ⟦χ⟧ is that the articulation is interpreted phonologically as the same as ⟦χ'⟧/⟦χ'{}'⟧, namely a phonologically front dorsal (=\,[coronal, dorsal]).

Consider now the evidence in favor of my interpretation of the SDS symbols as described above: First, the symbols for prepalatal fricatives (⟦χ'⟧/⟦χ'{}'⟧) are present on SDS Map II 94 and II 183 for parts of \ipi{Upper Valais} (\sectref{sec:15.8}) and the Southwest \ipi{Bernese Oberland} (\sectref{sec:15.9}). Significantly, those prepalatals can be shown to be phonologically front dorsal fricatives on the basis of independent sources. Second, the analysis of ⟦χ⟧ in the unmarked case as phonologically on par with the phonologically back dorsal ⟦x⟧ is consistent with the prevalence of ⟦χ⟧ markers on SDS Map II 94 alluded to above (as well as the other maps in \tabref{tab:15.2}). Third, my interpretation of ⟦χ⟧ in the marked case as a phonologically front dorsal fricative ([coronal, dorsal]) makes sense because markers for ⟦χ⟧ can also be found in areas like the ones alluded to above in which velar fronting is active, i.e. \ipi{Upper Valais}, Southwest \ipi{Bernese Oberland}, as well as parts of \ipi{East Switzerland} (\sectref{sec:15.11}).

In \sectref{sec:15.6} I discussed two velar fronting varieties of \il{Highest Alemannic}HstAlmc Grisons, namely \ipi{Obersaxen} and \ipi{Vals}. The maps in SDS do not unambiguously (dis)confirm the presence of velar fronting in those two places. For \ipi{Obersaxen} the palatal fricative marker ⟦χ⟧ is present on Map II 94, and the palatal \isi{affricate} marker ⟦kχ⟧ is indicated on Map II 95/96 (for ⟦drįkχ\textsuperscript{æ}⟧ ‘press-\textsc{inf}’). By contrast, the prepalatal ⟦χ'⟧ is given in the list of data for Map II 97/98 for \textit{trinkt} ‘drink-\textsc{3sg}’ (⟦trīχ't⟧). For \ipi{Vals} the SDS maps show either ⟦χ⟧/⟦kχ⟧ or ⟦x⟧/⟦kx⟧.

The SDS data might confirm postsonorant velar fronting of /x/ and /kx/ for \ipi{Obersaxen} if the one prepalatal marker is considered representative and if the two palatal markers are interpreted as a front dorsal. The conclusion for \ipi{Vals} is not as obvious because there are no prepalatal markers indicated given for that place. I conclude that those palatal markers indicated front dorsals.

\section{Upper Valais, Northwest Italy, and Tessin}\label{sec:15.8}\il{Highest Alemannic|(}\largerpage

The canton of Valais (Wallis) in Southwest Switzerland is traditionally divided into three regions: Lower Valais (Unterwallis), Central Valais (Mittelwallis), and \ipi{Upper Valais} (Oberwallis). The former two are primarily \ili{French}-speaking, while \ipi{Upper Valais} is predominantly German-speaking. Most settlements in \ipi{Upper Valais} are located in the Rhône Valley between Siders and \ipi{Oberwald} -- including the side valleys --, although \ipi{Upper Valais} also extends as far south as \ipi{Zermatt} (\mapref{map:38}). Significantly, \ipi{Upper Valais} is a secluded area of Switzerland because the Rhône Valley is an Alpine valley, which is shut off from the German-speaking areas to the north of the Bernese Alps (Berner Alpen).

The German dialect spoken throughout \ipi{Upper Valais} is \il{Highest Alemannic}HstAlmc, a specific variety of which (\ipi{Visperterminen}) was discussed in \sectref{sec:6.2}. \citet{Wipf1910} is an invaluable source because it provides a detailed descriptive grammar of a velar fronting variety in a specific village in that region. Several additional sources for dorsal fricatives/affricates in \ipi{Upper Valais} are also known to me. Although those works do not compare with \citet{Wipf1910} in terms of quantity and depth of velar fronting data, they all provide valuable information concerning the extent to which velar fronting is active in other parts of \ipi{Upper Valais}. \mapref{map:38} indicates the places in that area referred to in the sources I discuss below. The map also includes a number of German-language (\il{Highest Alemannic}HstAlmc) islands  in Northwest Italy, as well as one \il{Highest Alemannic}HstAlmc variety in the \ili{Italian}-speaking canton of Tessin.\footnote{{Those (\il{Highest Alemannic}HstAlmc) German-language islands (\ipi{Issime}, \ipi{Gressoney}, \ipi{Alagna}, \ipi{Rima}, and \ipi{Macugnaga} in Italy and \ipi{Bosco Gurin} in Tessin) were settled during the \isi{Walser migrations} beginning in the thirteenth century \citep[903]{Wiesinger1983b}.} \textrm{\citet{Bohnenberger1913}, \citet{Jutz1931}, and \citet{Moulton1941} all observe that velar fronting -- phrased in their terms as the occurrence of [ç] and [x] as positional variants -- is common throughout \ipi{Upper Valais}. Bohnenberger in particular writes that the occurrence of the palatal fricative in the neighborhood of front vowels is typical for the entire region. Bohnenberger represents both sounds with the symbol ⟦χ⟧ and does not provide data from any particular place. \citet[208]{Jutz1931} refers only to \ipi{Visperterminen} as evidence that some South Almc dialects have [x] and [ç]. \citet[40]{Moulton1941} also observes that “Wallis dialects” have [x] and [ç] as positional variants, but his only example is \ipi{Visperterminen} (in addition to the Walser variety of \ipi{Obersaxen} discussed in \sectref{sec:15.6}).}}

\ip{Upper Valais}
\begin{map}
% \includegraphics[width=\textwidth]{figures/VelarFrontingHall2021-img044.png}
\includegraphics[width=\textwidth]{figures/Map38_15.6.pdf}
\caption[{Upper Valais}, Northwest Italy, and Tessin]{{Upper Valais}, Northwest Italy, and Tessin. Squares indicate some version of velar fronting (postsonorant and/or word-initial), and the circle represents the absence of velar fronting. 1=\citet{Wipf1910}, 2=\citet{Henzen1928, Henzen1932}, 3=\citet{Rübel1950}, 4=\citet{Schmid1969}, 5=\citet{Werlen1987}, 6=\citet{Russ2002}, 7=SDS.}\label{map:38}
\end{map}

In the remainder of this section I discuss data from additional sources for \il{Highest Alemannic}HstAlmc varieties in \ipi{Upper Valais}. I consider first those studies that focus on specific places and then turn to works that investigate the status of velar fronting in the region as a whole (including Northwest Italy and Tessin). The two major issues I address are: (a) The extent to which velar fronting is attested throughout the entire region and (b) the different requirements concerning the set of velar fronting triggers for those places with that rule.

\citet{Henzen1928} concerns himself with \isi{Vowel Reduction} in posttonic syllables and \citet{Henzen1932} with the morphology of the genitive. Both articles deal specifically with the dialect spoken in one of the side valleys of the Rhône Valley, namely the area in and around Blatten in the Lonza River Valley (\ipi{Lötschental}), about 20km to the northwest of \ipi{Visperterminen}. Henzen adopts the same phonetic transcriptions as in \citet{Wipf1910}, whereby ⟦x⟧=[x] and ⟦χ⟧=[ç]. The data in \REF{ex:15:24} and \REF{ex:15:25} have been drawn from the two articles referred to above. The pages in the final column refer to \citet{Henzen1928} and \citet{Henzen1932}, which are abbreviated as A and B respectively.\largerpage


\TabPositions{.14\textwidth, .31\textwidth, .5\textwidth, .79\textwidth}
\ea%24
\label{ex:15:24}Word-initial dorsal fricatives in \ipi{Lötschental}:

\ea xunt \tab [xunt] \tab kommt \tab ‘come-\textsc{3sg}’ \tab B: 98\\\label{ex:15:24a}
    xuæ \tab [xuæ] \tab Kuh \tab ‘cow’ \tab B:105\\
    xɑbus \tab [xɑbus] \tab Kohl \tab ‘cabbage’ \tab A: 116
\ex χind \tab [çind] \tab Kind \tab ‘child’ \tab B: 95\\\label{ex:15:24b}
    χiššini \tab [çiʃʃini] \tab Kissen \tab ‘pillow’ \tab A: 111\\
    χeštn \tab [çeʃtn̩] \tab Kosten \tab ‘cost-\textsc{pl}’ \tab B: 100\\
    χend \tab [çend] \tab Kinder \tab ‘child-\textsc{pl}’ \tab B: 110\\
    χɛs  \tab  [çɛːs] \tab Käse \tab ‘cheese’ \tab A: 139
\ex  χiæ  \tab  [çiæ] \tab Kühe \tab ‘cow-\textsc{pl}’ \tab B: 95\label{ex:15:24c}
\ex  xæxlæ \tab [xæxlæ] \tab Bergdohlen \tab ‘type of bird-\textsc{pl}’ \tab A: 111\label{ex:15:24d}
\ex  χlupf \tab [çlupf] \tab Furcht \tab ‘fear’ \tab A: 115\label{ex:15:24e}
\z 


\TabPositions{.14\textwidth, .31\textwidth, .5\textwidth, .79\textwidth}
\ex%25
\label{ex:15:25}Postsonorant dorsal fricatives in \ipi{Lötschental}:

\ea nox  \tab  [nox] \tab noch \tab ‘still’ \tab B: 105\\\label{ex:15:25a}
    lōx  \tab  [loːx] \tab Loch \tab ‘hole’ \tab A: 133\\
    bɑx  \tab  [bɑx] \tab Bach \tab ‘stream’ \tab A: 128\\
    sɑxx \tab [sɑxx] \tab Sache \tab ‘thing’ \tab B: 106
\ex iχ  \tab  [iç] \tab ich \tab ‘I’ \tab B: 95\\\label{ex:15:25b}
    riχr  \tab  [riçr̩] \tab reicher \tab ‘richer’ \tab B: 98\\
    līχ  \tab  [liːç] \tab Leiche \tab ‘body’ \tab B: 105\\
    teχtr \tab [teçtr̩] \tab Tochter \tab ‘daughter’ \tab B: 102\\
    oiχ  \tab  [oiç] \tab auch \tab ‘also’ \tab A: 132
\ex liæχpmæs \tab [liæçpmas] \tab Lichtmess \tab ‘Candlemass’ \tab B: 100\label{ex:15:25c}
\ex næxti(n) \tab [næxti(n)] \tab gestern abend \tab ‘yesterday evening’ \tab A: 112\\\label{ex:15:25d}
    suæxid \tab [suæxid] \tab sucht \tab ‘search-\textsc{3}\textsc{sg}’ \tab A: 112\\
    xæxlæ \tab [xæxlæ] \tab Bergdohlen \tab ‘type of bird-\textsc{pl}’ \tab A: 111\\
    dæxxri(n) \tab [dæxxxri(n)] \tab Dächern \tab ‘roof-\textsc{dat}.\textsc{pl}’ \tab A: 135
\ex milχ \tab [milç] \tab Milch \tab ‘milk’ \tab B: 104\\\label{ex:15:25e}
    lērχ  \tab  [leːrç] \tab Lärchbaum \tab ‘kind of tree’ \tab B: 103
\z 
\z 

The generalizations concerning triggers for word-initial and postsonorant position in \ipi{Lötschental} are not the same as in \ipi{Visperterminen}, where high front vowels are the sole triggers (=\isi{Wd-initial Velar Fronting-4} and \isi{Velar Fronting-6}). In \ipi{Lötschental} the triggers for word-initial and postsonorant position comprise the set of nonlow front vowels or liquids, i.e. (\ref{ex:15:26}) and (\ref{ex:15:27}). Note that /æ/ fails to induce fronting if it is a monophthong or the second component of the /uæ/ diphthong (=\ref{ex:15:25d}) but that /iæ/ does induce fronting (=\ref{ex:15:25c}). That /iæ/ is a velar fronting trigger is precisely the case in \ipi{Visperterminen}; recall the representations for vowels and diphthongs posited in \sectref{sec:6.2.1}.

\ea%26
    \label{ex:15:26}
    \begin{multicols}{2}
    \ea \isi{Wd-Initial Velar Fronting-1}:\\\label{ex:15:26a}
    \begin{forest}
        [,phantom
            [\avm{[−son\\+cont]},name=source [\avm{[dorsal]}]]
            [\avm{[−low]} [\avm{[coronal]},name=target]]
        ]
        \draw [dashed] (source.south) -- (target.north);
        \node [left=1ex of source] {\textsubscript{wd} [};
    \end{forest}
    \ex \isi{Wd-Initial Velar Fronting-2}:\\\label{ex:15:26b}
    \begin{forest}
        [,phantom
            [\avm{[−son\\+cont]},name=source [\avm{[dorsal]}]]
            [\avm{[+cons\\+son]} [\avm{[coronal]},name=target]]
        ]
        \draw [dashed] (source.south) -- (target.north);
        \node [left=1ex of source] {\textsubscript{wd} [};    
    \end{forest}
\z \end{multicols}
\ex%27
\label{ex:15:27}
\begin{multicols}{2}
\ea \isi{Velar Fronting-2}:\\\label{ex:15:27a}
    \begin{forest}
        [,phantom
            [\avm{[−low]} [\avm{[coronal]},name=target]]
            [\avm{[−son\\+cont]},name=source [\avm{[dorsal]}]]
        ]
        \draw [dashed] (source.south) -- (target.north);
    \end{forest}
\ex \isi{Velar Fronting-3}:\\\label{ex:15:27b}
    \begin{forest}
        [,phantom
            [\avm{[+cons\\+son]} [\avm{[coronal]},name=target]]
            [\avm{[−son\\+cont]},name=source [\avm{[dorsal]}]]
        ]
        \draw [dashed] (source.south) -- (target.north);
    \end{forest}
\z\end{multicols}
\z 

\REF{ex:15:26a} and \REF{ex:15:26b} together account for the fact that word-initial palatals occur either before a nonlow front vowel or before a sonorant consonant. \REF{ex:15:27a} and \REF{ex:15:27b} likewise express the mirror-image generalization for postsonorant position.

\citet{Schmid1969} investigates the dialect spoken in the village of \ipi{Bellwald}. Although the author does not provide extensive datasets, it is clear from the remarks on phonetic symbols and the phonetics of consonants that \ipi{Bellwald} has some version of velar fronting. Schmid (1969: XVI) posits a consonant chart with the three places of articulation for dorsal (“guttural”) fricatives and affricates from SDS (recall \sectref{sec:15.7}): Prepalatal (=⟦χ'{}'⟧/⟦kχ'{}'⟧), palatal-velar (=⟦χ'⟧/⟦kχ'⟧), and velar (=⟦χ⟧/⟦kχ⟧). Schmid (1969: XVII) even gives a clear statement on the pronunciation of the velar fricative:\largerpage[2]

\begin{quote}
  In \ipi{Bellwald}  wird von den zwei älteren Gewährsgruppen  der velare Rebelaut χ unmittelbar vor oder nach i oder e (und deren qualitativen und quantitativen Varianten) als dentaler Reibelaut š  gesprochen, vor oder nach einem Liquiden als palataler Reibelaut χ' χ'{}'.  Bei der jüngsten Gruppe ist in den gleichen Stellungen meist palatales χ', χ', selten dentales š zu hören.

  “The velar fricative χ is pronounced in \ipi{Bellwald} in the two groups of informants as a dental fricative š immediately before or after i or e (and their qualitative and quantitative variants) and as palatal fricative χ' χ'{}' before or after liquids. In the youngest group of informants palatal χ' χ'{}', but seldom dental š, can usually be heard in the same contexts. In general there is a tendency today for the soft palatal pronunciation χ'{}' ...ˮ
\end{quote}

I interpret velar fronting in \ipi{Bellwald} as follows: For older informants, the target segment is /x/, which shifts to a front dorsal ([coronal, dorsal]) fricative in the context of a coronal sonorant by \isi{Velar Fronting-1} (=\ref{ex:15:2}).\footnote{{\ipi{Bellwald} has low front vowels, but it is not clear from the source whether or not they induce velar fronting.}} If the coronal sonorant is a front vowel then the derived [coronal, dorsal] fricative surfaces as a \isi{sibilant} ([ɕ]), but if /x/ is adjacent to a liquid then it is realized as a nonsibilant ([ç]). In the speech of younger informants, /x/ is fronted to [coronal, dorsal] in the context of coronal sonorants and usually surfaces as a nonsibilant ([ç]), rarely as a \isi{sibilant} ([ɕ]).

It is interesting that the older generation of speakers has a \isi{sibilant} as the output (in the front vowel context) and that the younger generation has replaced the \isi{sibilant} with the nonsibilant [ç]. This is significant because the historical process of \isi{alveolopalatalization} described in \chapref{sec:10} documents precisely the reverse development: The nonsibilant [ç] is realized by the younger generation as a \isi{sibilant} ([ɕ]). To the best of my knowledge, \ipi{Bellwald} is the only variety of German which illustrates the historical change from \isi{sibilant} to nonsibilant. \ipi{Bellwald} is also unique in the sense that the output of velar fronting differs according to context: A \isi{sibilant} is created in the context of front vowels and a nonsibilant in the context of liquids.

\citet{Werlen1977} offers a detailed study of the sound structure of the \il{Highest Alemannic}HstAlmc variety spoken in and around \ipi{Brig} (now Brig-Gris) couched in early generative phonology. In his discussion of dorsal fricatives (pp. 187--191), Werlen adopts the SDS transcription system with separate symbols representing three categories of dorsal fricatives (and affricates). Throughout his book, Werlen refers to ⟦χ'⟧ as a “palatalized” ⟦χ⟧ and observes (p. 190) that the rule is a regional distinctive feature (“ein regional distinktives Merkmal”). An example of a place with velar fronting (Palatalization) is Ried-\ipi{Brig} \citep[328]{Werlen1977}. Werlen writes that all of his informants from that place have a strong palatal articulation (i.e. ⟦χ'⟧) for ⟦χ⟧. His examples are given in \REF{ex:15:28}. The five categories in (\ref{ex:15:28a}--\ref{ex:15:28e}) correspond to five different speakers. I give Werlen’s transcriptions in the first column, but I ignore a few of his diacritics for clarity.\largerpage[-1]



\TabPositions{.16\textwidth, .35\textwidth, .5\textwidth, .79\textwidth}

\ea%28
\label{ex:15:28}Prepalatal ⟦χ'⟧ in Ried-\ipi{Brig}:
\ea\label{ex:15:28a} kχ'ẹy bu\textsuperscript{u̯}te \tab keine Bauten \tab ‘no structure-\textsc{pl}’\\
    glị̄χ' \tab gleich \tab ‘same’
\ex\label{ex:15:28b} kχ'{}'ērt \tab gehört \tab ‘hear-\textsc{part}’
\ex\label{ex:15:28c} glị̄χ'\textsuperscript{α} \tab das Gleiche \tab ‘the same’\\
    felị̄χ't \tab vielleicht \tab ‘maybe’\\
    įχ' \tab ich \tab ‘I’
\ex\label{ex:15:28d} brüχ'{}'ya \tab braucht ja \tab ‘need-\textsc{3}\textsc{sg}’
\ex\label{ex:15:28e} įχ' \tab ich \tab ‘I’\\
     kχ'ērįχ' \tab höre ich \tab ‘hear-\textsc{1}\textsc{sg}’
     \z
\z 

Werlen posits a rule of Palatalization (velar fronting) with distinctive features (p. 328) which captures the occurrence of ⟦χ'⟧ in \REF{ex:15:28}. According to that rule, a target dorsal fricative is fronted when adjacent to a front ([--back]) vowel. The output of his rule is an “alveolar” fricative which appears to be identical featurally with the \isi{sibilant} ⟦š⟧ (p. 230). In the present system the data in \REF{ex:15:28} are consistent with either \isi{Velar Fronting-1} (=\ref{ex:15:2}) or \isi{Velar Fronting-13} (=\ref{ex:15:4}).

Ried-\ipi{Brig} contrasts with neighboring places which apparently only have ⟦χ⟧. Consider the discussion of Glis \citep[338]{Werlen1977}: Werlen observes that only one of his informants from that place palatalizes ⟦χ⟧ to ⟦χ'⟧. Although he does not state this point explicitly, the implication -- supported with his phonetic transcriptions -- is that the default case for Glis (and for the town of \ipi{Brig}) is that ⟦χ⟧ is realized as ⟦χ⟧ regardless of context. An examination of Werlen’s system of distinctive features (p. 23) reveals that his /χ/ phoneme is [+high] and [+back], which are precisely those features necessary to define the velar place of articulation (p. 226). My conclusion is that Werlen’s ⟦χ⟧ is not palatal, but velar ([x]); hence, Werlen’s speakers from the town of \ipi{Brig} do not have velar fronting. I return to the status of non-velar fronting varieties in \ipi{Upper Valais} below.

\citet{Rübel1950} concerns himself with the various \il{Highest Alemannic}HstAlmc terms relating to cattle breeding in \ipi{Upper Valais} (“Viezucht im Oberwallisˮ) from the perspective of dialectology and lexicography. As peripheral as the topic might sound for a book on the phonology of dorsal consonants, Rübel’s work is extremely valuable because the author presents cattle breeding terminology in phonetic transcription which clearly distinguishes places of articulation for dorsal fricatives. What is more, \citet{Rübel1950} does not draw his data from one specific locality, in contrast to \citet{Wipf1910}, \citet{Henzen1928, Henzen1932}, \citet{Schmid1969}, and \citet{Werlen1977}. Instead, Rübel lists copious examples from over 50 settlements interspersed along the Rhône Valley from Siders to \ipi{Oberwald} (including the side valleys) as well as towns and villages as far south as Saas-Grund and \ipi{Zermatt}. As such, the book sheds light on how velar fronting differs from place to place within a large region.

Rübel adopts a transcription system (p. XXX) similar to the one employed by SDS with the difference being that Rübel has four categories for dorsal fricatives: ⟦x⟧ for uvular (=[χ]), ⟦χ⟧ for velar (=[x]), ⟦χ'⟧ for palatal (=[ç]), and ⟦χ'{}'⟧ for prepalatal.\footnote{{According to Rübel ⟦χ⟧ corresponds to the (\il{Standard German}StG) ach-Laut and ⟦χ'⟧ to the ich-Laut (p. XXX, Footnote 2). In the same footnote he describes ⟦χ'{}'⟧ as a palatal colored h-sound (“palatal gefärbter Hauchlaut”).}} Impressionistically the uvular fricative is rare in the data provided. By contrast, the symbols for velar, palatal, and the prepalatal are common.\largerpage[-1]

In a surprising (but welcome) departure from his discussion of cattle breeding terminology, Rübel provides a short subsection on the realization of dorsal fricatives (pp. 12--13). In that passage he gives a statement similar to the one from \citet{Schmid1969} cited above, according to which the velar fricative is fronted to either ⟦χ'⟧ or ⟦χ'{}'⟧ in the context before or after i or e (including their qualitative and quantitative variants) or liquids. Like Werlen, Rübel writes of Palatalization (“Palatalisierung”), which is equivalent to velar fronting in the present framework. \citet[13]{Rübel1950} observes that the fronting (Palatalization) of velar to ⟦χ'{}'⟧ is particularly prevalent in the uppermost regions of Goms (the area around \ipi{Oberwald}), in the outer Visp Valley (the area between Visp and \ipi{Visperterminen}) and in \ipi{Lötschental}. (The specific places in those three areas are all indicated on \mapref{map:38}). A selection of data drawn from \citet[9]{Rübel1950} is presented in \REF{ex:15:29}, where my interpretation of his symbols is given in the second column.

\ea%29
\label{ex:15:29}
\ea\label{ex:15:29a} χ'rǫmə \tab [çrɔmə]\\
    χ'remə \tab [çremə]\\
\ex\label{ex:15:29b} χ'romo \tab [çromo]\\
    χ'reme \tab [çreme]\\
\ex\label{ex:15:29c} χrǫmu  \tab [xrɔmu]\\
    χreme  \tab [xreme]\\
\z 
\z 

The data illustrate singular vs. plural realizations for the noun \textit{Krommen} (unclear gloss), and the three different phonetic realizations in (\ref{ex:15:29a}--\ref{ex:15:29c}) correspond to the different villages in \ipi{Upper Valais}.\largerpage[-1]

Rübel’s observation concerning the places in \ipi{Upper Valais} where velar fronting (Palatalization) is most prevalent is important because it establishes that velar fronting is not limited to \ipi{Visperterminen}, \ipi{Lötschental}, \ipi{Bellwald}, and Ried-\ipi{Brig}, but instead that it is a rule that has diffused itself throughout most areas of \ipi{Upper Valais}.

The prevalence of velar fronting in this corner of Switzerland is confirmed by the presence of the many prepalatal markers (⟦χ'⟧) in that region on the SDS maps. Map II 94 for the word-initial dorsal fricative in \textit{Kind} ‘child’ was already commented on in \sectref{sec:15.7}, but several other maps in \tabref{tab:15.2} yield a similar picture. As I discuss below, the underlined sound(s) in the words listed in the first column of \tabref{tab:15.2} are realized as ⟦χ'⟧ or ⟦χ'{}'⟧ either word-initially or after a coronal sonorant in many places in \ipi{Upper Valais}. The vowels adjacent to the prepalatal markers on those maps are almost always front, although the back vowel context is clear from Map II 183 for \textit{bache (backen)} ‘bake-\textsc{inf}’ and Map II 201 for \textit{rauchen} ‘smoke-\textsc{inf}’.

The authors of SDS note in several places that (Upper) Valais is one of the few places in Switzerland where the prepalatal realization of dorsal fricatives and affricates is common. For example, in the commentary to Map II 201 they write that the prepalatal articulation is attested in numerous places throughout the western part of the \ipi{Bernese Oberland} (see \sectref{sec:15.9}), Valais, and (Northwest) Italy. (“ ... mit präpalataler Artikulation [zeichnen sich] zahlreiche Orte im westlichen Berner Oberland, im Wallis und im IT ... ˮ).\largerpage[-1]

On my \mapref{map:38} I indicate all of the places in \ipi{Upper Valais} where prepalatal markers occur in word-initial position on Map II 94. The data presented from that region on the other maps listed in \tabref{tab:15.2} reveal that all of the places with prepalatal markers for Map II 94 -- as well as many of the other villages and towns in \ipi{Upper Valais} -- also have some degree of velar fronting in postsonorant position. The extent to which velar fronting is present in any one place is determined by the number of prepalatal markers for the maps listed in \tabref{tab:15.2}. It is not the case that every village and town in \ipi{Upper Valais} consistently applies velar fronting, although it is interesting that few of the villages and towns present in \ipi{Upper Valais} in SDS has no prepalatal markers at all. However, the SDS maps in \tabref{tab:15.2} reveal that some places have significantly more prepalatal markers than palatal markers, while other places have many more palatal markers than prepalatal ones. In general it can be said that velar fronting is more consistent in the following areas: (a) Between \ipi{Grächen} and \ipi{Zermatt}, (b) \ipi{Simplon Dorf}, (c) between \ipi{Oberwald} and Grengiols, (d) in the German-speaking islands in Northwest Italy, and (e) \ipi{Bosco Gurin} (in Tessin). My conclusion concerning the prevalence of velar fronting in those five areas is especially clear on Map II 183 for \textit{bache (backen)} ‘bake-\textsc{inf}’. On that map the authors of SDS note in the commentary that the fricative in \textit{Rechen} ʽrake’ for \ipi{Zermatt}, \ipi{Oberwald}, \ipi{Simplon Dorf}, \ipi{Alagna}, and \ipi{Rima} is a ʽvery palatal ch.. ’ (“sehr palatales \ul{\textit{ch}} …”), where underlining is present in the original.

In \REF{ex:15:30} I list a representative selection of data from SDS from four places in \ipi{Upper Valais} and in \REF{ex:15:31} from three places in Northwest Italy and \ipi{Bosco Gurin}. It can be observed that ⟦χ'⟧ or ⟦χ'{}'⟧ occur predominantly in the context of high front vowels and after /l/, although a few examples listed below indicate the presence of prepalatals in the neighborhood of back vowels or back consonants like [ŋ].\footnote{The data listed in SDS often include multiple tokens for any one place, but only one example is included for each word in \REF{ex:15:30} and \REF{ex:15:31}. A horizontal line means either that there are no data for that particular example in that particular place or that the data given in SDS for that place contain [h] instead of a dorsal fricative or \isi{affricate}. The transcriptions given in \REF{ex:15:30} and \REF{ex:15:31} are the ones in SDS, although I omit some of the more exotic diacritics for clarity. SDS does not provide complete transcriptions for (\ref{ex:15:30h}--\ref{ex:15:30j}) and (\ref{ex:15:31h}, \ref{ex:15:31i}).}


\TabPositions{.15\textwidth, .3\textwidth, .5\textwidth, .7\textwidth}
\ea%30
\label{ex:15:30}Prepalatal fricatives and affricates in \ipi{Upper Valais} (SDS):
\begin{xlist}
\sn[]
\tab \tab  \ipi{Zermatt} \tab \ipi{Grächen} \tab \ipi{Oberwald} \tab  \ipi{Simplon Dorf}
\ex\label{ex:15:30a} drücken \tab trįkχ{\textquotesingle}e \tab trikχ{\textquotesingle}u \tab trįk{\textquotesingle}χ{\textquotedbl}ə \tab trikχ{\textquotesingle}u
\ex\label{ex:15:30b} trinken \tab trị̄χ{\textquotedbl}e \tab tr\~{ī}χ{\textquotesingle}u \tab t\textsuperscript{ə}rị̄χ{\textquotesingle}ə \tab  tr\~{ī}χ{\textquotesingle}u
\ex\label{ex:15:30c} getrunken \tab ɡitrüχ{\textquotedbl}e \tab ɡitr\~{ü}χ{\textquotesingle}u͈ \tab tr\~{ü}χ{\textquotesingle}ə \tab  ɡitrū\textsuperscript{i}χ{\textquotedbl}u
\ex\label{ex:15:30d} tränken \tab trê\textsuperscript{i}χ{\textquotedbl}e \tab tr\~{ê}\textsuperscript{i}χ{\textquotesingle}u \tab t\textsuperscript{ə}rêyχ{\textquotesingle}ə \tab  trê\textsuperscript{i}χ{\textquotesingle}u
\ex\label{ex:15:30e} Gestank \tab ɡštö\textsuperscript{ü}χ{\textquotedbl} \tab kštõ\textsuperscript{u}χ{\textquotedbl} \tab kštâyχ \tab kštâiχ{\textquotedbl}
\ex\label{ex:15:30f} Anke \tab ö\textsuperscript{ü}χ{\textquotedbl}e \tab \~{ö}\textsuperscript{ü}χ{\textquotesingle}o \tab ayχ{\textquotesingle}ə \tab åyχ{\textquotesingle}u
\ex\label{ex:15:30g} Bank \tab böüχ{\textquotedbl} \tab b\~{ö}\textsuperscript{ü}χ{\textquotesingle} \tab beyχ{\textquotesingle} \tab b\~{å}χ{\textquotedbl}
\ex\label{ex:15:30h} Speicher \tab īχ{\textquotedbl}  \tab  \~{ī}χ{\textquotedbl}  \tab  --- \tab iχ{\textquotesingle}
\ex\label{ex:15:30i} backen \tab χ{\textquotesingle}  \tab  χ{\textquotesingle}  \tab  χ{\textquotesingle}  \tab  ---\\
    Rechen \tab χ{\textquotedbl}  \tab  --- \tab χ{\textquotedbl}  \tab  χ{\textquotedbl}
\ex\label{ex:15:30j} rauchen \tab --- \tab --- \tab kχ{\textquotesingle} \tab kχ{\textquotedbl}
\end{xlist}
\ex%31
\label{ex:15:31}
Prepalatal fricatives and affricates in four German-language islands (SDS):
\begin{xlist}
\sn[]
\tab \tab \ipi{Alagna} \tab \ipi{Rima} \tab \ipi{Macugnaga} \tab \ipi{Bosco Gurin}
\ex\label{ex:15:31a} trinken \tab triŋχ{\textquotedbl}e \tab treŋχ{\textquotedbl}a \tab trĩŋχ{\textquotedbl}e  \tab  trîŋχ{\textquotesingle}æ
\ex\label{ex:15:31b} getrunken \tab druŋχ{\textquotedbl}e \tab ɡtraŋχ{\textquotedbl}d \tab ɡitruŋχ{\textquotedbl}es \tab tr\={ü}\textsuperscript{æ}χ{\textquotesingle}æ
\ex\label{ex:15:31c} tränken \tab dreŋχ{\textquotedbl}e \tab traŋχ{\textquotedbl}an \tab ---  \tab  trē\textsuperscript{ə}χ{\textquotesingle}æ
\ex\label{ex:15:31d} Anke \tab aŋɡχ{\textquotedbl}u \tab aŋχ{\textquotedbl}a \tab \~{å}ŋχ{\textquotedbl}e  \tab  ōχ{\textquotesingle}æ
\ex\label{ex:15:31e} Bank \tab baŋχ{\textquotedbl} \tab bāŋχ{\textquotesingle} \tab bãŋχ{\textquotedbl}  \tab  b\={å}χ{\textquotesingle}
\ex\label{ex:15:31f} melken \tab lχ{\textquotedbl}  \tab  lχ{\textquotedbl}  \tab  lχ{\textquotesingle}  \tab  lχ{\textquotesingle}
\ex\label{ex:15:31g} Chilche \tab χ{\textquotedbl}il{\textquotesingle}χ{\textquotedbl}a \tab χ{\textquotedbl}il{\textquotedbl}χ{\textquotedbl}o \tab χ{\textquotesingle}ilχ{\textquotedbl}u  \tab  χ{\textquotesingle}elχ{\textquotesingle}u
\ex\label{ex:15:31h} backen \tab χ{\textquotesingle}  \tab  --- \tab χ{\textquotesingle}χ{\textquotesingle}  \tab  χ{\textquotesingle}\\
    Rechen \tab χ{\textquotedbl}  \tab  χ{\textquotedbl}  \tab  ---  \tab  ---
\ex\label{ex:15:31i} rauchen \tab raikχ{\textquotesingle}-e \tab raukχ{\textquotedbl}-a \tab ---  \tab  ---
\end{xlist}
\z 

Since SDS does not offer a complete set of data for dorsal fricatives for any given place -- that is, a set of words in which /x/ and/or /kx/ occurs before or after all phonemic vowels as well as /l r n/ -- no definitive conclusions can be drawn concerning targets and triggers for velar fronting for any of the places listed in \REF{ex:15:30} or \REF{ex:15:31}. Based on the occurrence of prepalatals even in the neighborhood of back segments for some of the places listed above suggest that velar fronting is nonassimilatory (Trigger Type F; \chapref{sec:14}).

The data from SDS are important because they confirm the findings of \citet{Rübel1950} concerning the prevalence of velar fronting throughout the south and northeast of \ipi{Upper Valais}. Note the occurrence of the velar fronting markers from SDS on \mapref{map:38} coincide for the most part with the velar fronting markers from \citet{Rübel1950}.\footnote{Sources for one place (\ipi{Bosco Gurin}) do not agree on the status of velar fronting.  According to the SDS  maps, that place is characterized by \isi{nonassimilatory velar fronting}. However, in a more recent study devoted specifically to the sounds of \ipi{Bosco Gurin}, \citet[77]{Russ2002} is quite clear that there is no velar fronting. This example suggests that there are (and were) speakers with and without velar fronting in that particular place. I assume that SDS based its maps on those innovative speakers with velar fronting, while Russ based his treatment on conservative speakers without that rule.}

The data discussed up to this point have focused almost exclusively on the areas of \ipi{Upper Valais} to the northeast and south of Visp, but nothing has been said about the towns and villages along the Rhône River to the west of Visp, in particular between Raron and Siders (with the exception of \ipi{Lötschental}, which is a side valley). The maps in SDS indicate that velar fronting is active in this area, but only to a limited extent. Consider the case of Salgesh, which is the westernmost place in \ipi{Upper Valais} on the SDS maps. According to SDS Maps II 96, II 98, II 100, II 104, II 105 the palatal marker ⟦χ⟧ occurs for Salgesh, while the prepalatal ⟦χ'⟧ is present for that village on Maps II 103, II 109. The village of Agarn has even fewer prepalatal markers for those maps (2), while Inden and Turtmann both have 4 and Feschel 3. Map II 183 yields a similar picture: In the west (between Agarn and Raron) and in the general area around Ried-\ipi{Brig}, there is a predominance of velar markers ⟦x⟧, although that would not be surprising even in a velar fronting area because the velar occurs after a back vowel.

The data from \citet{Rübel1950} are similar to the ones from SDS: The places in \ipi{Upper Valais} with a word-initial velar [x] in \REF{ex:15:29c} extend roughly from Siders to Raron (but excluding the side valley of \ipi{Lötschental}).

My conclusion is that the sources available do not allow one to reach any kind of meaningful conclusion concerning the extent to which velar fronting is active in the western part of \ipi{Upper Valais}.

One striking feature of \mapref{map:38} is the absence of non-velar fronting places in \ipi{Upper Valais}. One of the reasons for this is that it is not clear how to interpret the palatal ⟦χ⟧ from SDS, which is adopted by some of the works cited above. It was noted above that \citegen{Werlen1977} treatment of the variety spoken in and around Brig-Gris is a non-velar fronting variety, at least for certain speakers. In contrast to Ried-\ipi{Brig}, which has velar fronting (Palatalization), there is no equivalent rule for Werlen’s other speakers (e.g. from the town of \ipi{Brig}). What is more, as noted earlier, his featural system treats ⟦χ⟧ as a velar. I tentatively conclude that there are non-velar fronting varieties in \ipi{Upper Valais}, but those places cannot be reliably identified based on the sources available at this time. Since it is difficult to know for sure whether or not the velar [x] is present at all in some of the places listed on my \mapref{map:38}, I do not attempt to indicate on that map those places with only front dorsals (=\isi{nonassimilatory velar fronting}) in contrast to Maps~\ref{map:40} and~\ref{map:41} below.

\ipi{Upper Valais} can be thought of as a sizable velar fronting island because it is almost completely surrounded by high mountains or areas where a \ili{Romance} language (\ili{Italian} or \ili{French}) is spoken. There is a small corridor in the northeastern part of \ipi{Upper Valais} (around \ipi{Oberwald}) which connects \ipi{Upper Valais} with the rest of German-speaking Switzerland, but the closest dialect -- \ipi{Urserental} in the canton of \ipi{Uri} ca. 23km to the northeast -- is one without velar fronting \citep{Abegg1910}. The distinction between Northeast Valais (velar fronting) with Southwest \ipi{Uri} (no velar fronting) is depicted well on SDS Map II 183, where the former is covered with markers for prepalatals and the latter with markers for velars.\il{Highest Alemannic|)}

\section{{Southwest} {Bernese} {Oberland}}\label{sec:15.9}\largerpage

The \ipi{Bernese Oberland} (Berner Oberland) is a large area in the southern part of the canton of \ipi{Bern} which corresponds to one of that canton’s five administrative divisions (Oberland). The places I discuss below with velar fronting (of /x/ and /kx/) are located in an area I refer to as Southwest \ipi{Bernese Oberland}, which is the region to the south(west) of Thun, as depicted on \mapref{map:39}. The German dialects in this region are classified as \il{Highest Alemannic}HstAlmc.\footnote{{The earliest work identifying this area as one with [x] and [ç] is \citet[63]{Moulton1941}. I discuss below the three places Moulton mentions as well as several others from sources not available to him. Moulton also considers \ipi{Grindelwald} (ca. 15km south of \ipi{Brienz}) to be a place with [x] and [ç]. I do not discuss \ipi{Grindelwald} because Moulton’s assessment is based solely on two words in word-initial position. Moulton does not discuss the extent to which the triggers for velar fronting can differ from place to place within the Southwest \ipi{Bernese Oberland}.}}

\ip{Bernese Oberland}
\begin{map}
% \includegraphics[width=\textwidth]{figures/VelarFrontingHall2021-img045.png}
\includegraphics[width=\textwidth]{figures/Map39_15.7.pdf}
\caption[Southwest {Bernese Oberland}]{Southwest \ipi{Bernese Oberland}. Squares indicate some version of velar fronting (postsonorant and/or word-initial), and circles indicate the absence of velar fronting. 1=\citet{Zahler1901} 2=\citet{Gröger1914c}, 3=\citet{Gröger1914d}, 4=\citet{Gröger1914e} 5=\citet{Stucki1917}, 6=\citet{Henzen1927}, 7=\citet{Marti1985}, 8=SDS, 9=SiWS.}\label{map:39}
\end{map}

I discuss first the sources for velar fronting in specific towns and villages in the Southeast \ipi{Bernese Oberland}, and then I turn to data from SDS.

\citet{Gröger1914e} provides a phonetically transcribed text from a native speaker from \ipi{Saanen} which indicates the realization of /x/ and /kx/ as the corresponding palatals in word-initial position (=\ref{ex:15:32}). The dataset in \REF{ex:15:33} indicates that the fronting of /x/ and the corresponding geminate /xx/ are also active after a coronal sonorant.

\TabPositions{.15\textwidth, .3\textwidth, .5\textwidth, .8\textwidth}
\ea%32
\label{ex:15:32}
Word-initial dorsal fricatives and affricates in \ipi{Saanen}:
\ea\label{ex:15:32a} xunšt \tab [xunʃt] \tab kommst \tab ‘come-\textsc{2}\textsc{sg}’ \tab 60\\
    kxobɩ \tab [kxobɪ] \tab Jakob \tab ‘name’ \tab 57\\
    xɑlbər \tab [xɑlbər] \tab Kälber \tab ‘cattle-\textsc{pl}’ \tab 58
\ex\label{ex:15:32b} χɩntsf\={ü}št \tab [çɪntsfyːʃt] \tab Kindsfäuste \tab ‘child’s fist-\textsc{pl}’ \tab 58\\
    χüə  \tab  [çʏə] \tab Kühe \tab ‘cow-\textsc{pl}’ \tab 57\\
    χænə \tab [çænə] \tab können \tab ‘be able-\textsc{inf}’ \tab 58
\ex χnæχtə \tab [çnæçtə] \tab Knechte \tab ‘vassal-\textsc{pl}’ \tab 58
\z 
\ex%33
\label{ex:15:33}Postsonorant dorsal fricatives in \ipi{Saanen}:
\ea\label{ex:15:33a}  kfɩəx \tab [kfɪəx] \tab Galtvieh \tab ‘young stock’ \tab 57\\
     mɑxxə \tab [mɑxxə] \tab machen \tab ‘do-\textsc{inf}’ \tab 58
\ex\label{ex:15:33b}  ɩχ  \tab  [ɪç] \tab ich \tab ‘I’ \tab 57\\
     kšɩχt \tab [kʃɪçt] \tab Geschichte \tab ‘story’ \tab 60\\
     ɡræχə \tab [græçə] \tab unclear gloss \tab ~  \tab  58
\ex\label{ex:15:33c}  mælχə \tab [mælçə] \tab melken \tab ‘milk-\textsc{inf}’ \tab 58
\z 
\z 

The data provided in \REF{ex:15:32} and \REF{ex:15:33} indicate that the set of triggers for velar fronting in both word-initial and postsonorant position is the class of coronal sonorants. The formal rules that account for these generalizations are \isi{Wd-Initial Velar Fronting-8} (=\ref{ex:15:13}) and \isi{Velar Fronting-1} (=\ref{ex:15:2}).

\citet{Gröger1914d} provides a phonetically transcribed text from a native speaker from \ipi{Frutigen} indicating the presence of velar fronting in word-initial (=\ref{ex:15:34}) and postsonorant position (=\ref{ex:15:35}).\largerpage[2]


\ea%34
\label{ex:15:34}Word-initial dorsal fricatives in \ipi{Frutigen}:
\ea\label{ex:15:34a} xūm \tab [xuːm] \tab kaum \tab ‘hardly’ \tab 57
\ex\label{ex:15:34b} χelɩ \tab [çelɪ]\tab unclear gloss \tab ~  \tab 56\\
    χönə \tab [çønə] \tab können \tab ‘be able-\textsc{inf}’ \tab 55\\
    χömɩ \tab [çømɪ] \tab käme \tab ‘come-\textsc{3}\textsc{sg}.\textsc{subj}’ \tab 55\\
    pχent \tab [pçent] \tab gekannt \tab ‘know-\textsc{part}’ \tab 57
\ex\label{ex:15:34c} χnöwwə \tab [çnøwwə] \tab niederknien \tab ‘kneel down-\textsc{inf}’ \tab 57
\z 
\ex%35
\label{ex:15:35}Postsonorant dorsal fricatives and affricates in \ipi{Frutigen}:
\ea\label{ex:15:35a} wuxxə \tab [wuxxə] \tab Woche \tab ‘week’ \tab 55\\
    tōxt  \tab  [toːxt] \tab gedünkt \tab ‘think-\textsc{part}’ \tab 55\\
    mɑxxə \tab [mɑxxə] \tab machen \tab ‘do-\textsc{inf}’ \tab 56
\ex\label{ex:15:35b} sɩχ  \tab  [sɪç] \tab sich \tab ‘reflexive pronoun’ \tab 56\\
    šlæχt \tab [ʃlæçt] \tab schlecht \tab ‘bad’ \tab 56\\
    ræχt \tab [ræçt] \tab recht \tab ‘right’ \tab 55\\
    kštekχt \tab [kʃtekçt] \tab gesteckt \tab ‘stick-\textsc{part}’ \tab 57
\z 
\z 

As in \ipi{Saanen}, the data in \REF{ex:15:34} and \REF{ex:15:35} from \ipi{Frutigen} indicate that velar fronting is induced by all coronal sonorants (=\isi{Wd-Initial Velar Fronting-8} and \isi{Velar Fronting-1}).

\citet{Zahler1901} provides a list of verb conjugations in the \il{Highest Alemannic}HstAlmc dialect of \ipi{St. Stephan}.  It is clear from Zahler’s data that [x] (=⟦x⟧) and [ç] (=⟦c⟧) are positional variants whose distribution is a function of an adjacent vowel. This can be seen in the three partial paradigms in \REF{ex:15:36} from Zahler (1901: 229, 231), which illustrate that [ç] surfaces in the context of a front vowel and [x] in the context of a back vowel.\footnote{{Zahler notes that some speakers have alternant pronunciations. For example, [ç] surfaces in the context of low front vowels for some informants, while others have [x] in that context, e.g. ⟦präcə⟧ ‘break-}\textrm{\textsc{inf}}\textrm{’ in \REF{ex:15:36b} vs. ⟦präxə⟧. Variation involving the status of low front vowels as velar fronting triggers has been made repeatedly in this book.}}

\ea%36
\label{ex:15:36}Dorsal fricatives in \ipi{St. Stephan}:
\ea\label{ex:15:36a} xoə \tab [xoə] \tab kommen \tab ‘come-\textsc{inf}’\\
    x\k{u}mə \tab [xʊmə] \tab komme \tab ‘come-\textsc{1}\textsc{sg}’\\
    x\k{u}mšt \tab [xʊmʃt] \tab kommst \tab ‘come-\textsc{2}\textsc{sg}’\\
    x\k{u}mt' \tab [xʊmt] \tab kommt \tab ‘come-\textsc{3}\textsc{sg}’\\
    cemə \tab [çemə] \tab kommen \tab ‘come-\textsc{1/3}\textsc{pl}’\\
    cemət' \tab [çemət] \tab kommt \tab ‘come-\textsc{2}\textsc{pl}’\\
    ceəmį \tab [çeəmɪ] \tab kam \tab ‘come-\textsc{pret}’\\
    xoə  \tab  [xoə] \tab gekommen \tab ‘come-\textsc{part}’\\
    xum \tab [xum] \tab komm \tab ‘come-\textsc{imp}.\textsc{sg}’\\
    cemət' \tab [çemət] \tab kommt \tab ‘come-\textsc{imp}.\textsc{pl}’
\ex\label{ex:15:36b} präcə \tab [præçə] \tab brechen \tab ‘break-\textsc{inf}’\\
    prįcə \tab [prɪçə] \tab breche \tab ‘break-\textsc{1}\textsc{sg}’\\
    prįcšt \tab [prɪçʃt] \tab brichst \tab ‘break-\textsc{2}\textsc{sg}’\\
    prįct' \tab [prɪçt] \tab bricht \tab ‘break-\textsc{3}\textsc{sg}’\\
    pr\k{u}xį \tab [prʊxɪ] \tab brach \tab ‘break-\textsc{pret}’\\
    proxə \tab [proxə] \tab gebrochen \tab ‘break-\textsc{part}’
\ex\label{ex:15:36c} šühə \tab [ʃyhə] \tab scheuchen \tab ‘shoo-\textsc{inf}’\\
    šüücšt \tab [ʃyːçʃt] \tab scheuchst \tab ‘shoo-\textsc{2}\textsc{sg}’\\
    šüüct' \tab [ʃyːçt] \tab streicht \tab ‘shoo-\textsc{3}\textsc{sg}’\\
    šüüct'į \tab [ʃyːçtɪ] \tab scheuchte \tab ‘shoo-\textsc{pret}’\\
    kšüüct' \tab [kʃyːçt] \tab gescheucht \tab ‘shoo-\textsc{part}’
\z 
\z 

The data in \citet{Zahler1901} include a number of verbs like the one in \REF{ex:15:36c} with alternations between [h] and [ç]; recall similar data and discussion from \ipi{Maienfeld} \citep{Meinherz1920} in \sectref{sec:3.3}. As in \ipi{Maienfeld}, the alternations involving [h] and [ç] require an underlying /x/ which shifts to [h] in onset position (between vowels) by \isi{Debuccalization}. If the vowel preceding that /x/ is front and if /x/ is parsed into the coda, then it surfaces as [ç], as in the final four examples in \REF{ex:15:36c}. Seen in this light, \isi{Debuccalization} in examples like [ʃyhə] (from /ʃyxə/) \isi{bleeds} velar fronting; hence, [ç] and [x] have a transparent distribution.

Additional data from \citet[231-233]{Zahler1901} reveal that the set of triggers for velar fronting in \ipi{St. Stephan} does not include nasalized vowels (=\ref{ex:15:37a}) or coronal sonorant consonants (=\ref{ex:15:37b}).

\ea%37
\label{ex:15:37}Postsonorant dorsal fricatives in \ipi{St. Stephan}:
\ea\label{ex:15:37a} hẽhə \tab [hẽhə] \tab hängen \tab ‘hang-\textsc{inf}’\\
    hẽxšt \tab [hẽxʃt] \tab hängst \tab ‘hang-\textsc{2}\textsc{sg}’\\
    hẽxt' \tab [hẽxt] \tab hängt \tab ‘hang-\textsc{3}\textsc{sg}’\\
    hẽxtį \tab [hẽxtɪ] \tab hing \tab ‘hang-\textsc{pret}’
\ex\label{ex:15:37b} melhə \tab [melhə] \tab melken \tab ‘milk-\textsc{inf}’\\
    mįlxə \tab [mɪlxə] \tab melke \tab ‘milk-\textsc{1}\textsc{sg}’\\
    mįlxšt \tab [mɪlxʃt] \tab melkst \tab ‘milk-\textsc{2}\textsc{sg}’\\
    mįlxt' \tab [mɪlxt] \tab melkt \tab ‘milk-\textsc{3}\textsc{sg}’
\z 
\z 

The items listed in \REF{ex:15:37a} are particularly significant because they require that the set of triggers for postsonorant velar fronting in \ipi{St. Stephan} be restricted to front [--nasal] vowels. This restriction is without precedent in German dialects and even from the cross-linguistic perspective it is rare, although it is attested in the West African language \ili{Fanti} (recall \sectref{sec:2.3.3}).\footnote{Zahler provides a number of other verbs confirming the same generalization, namely that [x] consistently fails to undergo velar fronting after a \isi{nasalized vowel}. In all of his examples the [x] that fails to front alternates with [h], cf. the infinitive [hẽhə] ‘hang-\textsc{inf} for the verb in (\ref{ex:15:37a}). I do not consider this fact to be of significance}}

\ea%38
\label{ex:15:38}\isi{Velar Fronting-14}:\\
\begin{forest}
[,phantom
    [\avm{[−cons\\−nasal]} [\avm{[coronal]},name=target]]
    [\avm{[−son\\+cont]},name=source [\avm{[dorsal]}]]
]
\draw [dashed] (target.north) -- (source.south);
\end{forest}
\z 

For word-initial position (=\ref{ex:15:36a}) there are no data with nasalized vowels following [x]/[ç]; hence, one cannot know whether or not that context requires a set of triggers consisting solely of front oral vowels. The data in \REF{ex:15:36a} are consistent with \isi{Wd-Initial Velar Fronting-8} (=\ref{ex:15:13}) or \isi{Wd-Initial Velar Fronting-3} (=\ref{ex:15:23b}).

\begin{sloppypar}
A more recent source for \ipi{St. Stephan} is the dictionary for the \ipi{Simmental} (SiWS), which focuses in particular on the dialect of \ipi{Simmental} spoken in that particular town (p. 5). In the pronunciation guide (p. 9) there is a brief statement concerning the pronunciation of dorsal fricatives:
\end{sloppypar}

\begin{quote}\textit{ch} sprechen manche Leute durchwegs als ach-Laut (hinten), andere fast ausschliesslich (selbst in Wörtern wie \textit{chlage}, \textit{chriege}) als ich-Laut (vorn), wieder andere als mittleres, am Gaumenbogen gebildetes \textit{ch}, während weitere je nach dem folgenden Konsonanten variieren: rauhes \textit{ch} bei \textit{Sach, Chù\-chi} (Küche), weiches bei \textit{Chüe} (Kühe), \textit{rìchtig}.

“Some people pronounce \textit{ch} consistently as an ach-Laut (back), others almost exclusively (even in words like  fast \textit{chlage}, \textit{chriege}) as an ich-Laut (front), and others as a middle \textit{ch} formed on the palatal arch, while others vary according to the following consonant: rough \textit{ch} in \textit{Sach, Chùchi} (Küche), soft in \textit{Chüe} (Kühe), \textit{rìchtig}ˮ.
\end{quote}

Since SiWS does not provide phonetic transcriptions it is not possible to compare the data in that source with the ones from \citet{Zahler1901}. However, the quote is revealing since it suggests that the dialect of \ipi{St. Stephan} is characterized by considerable variation. On the one hand, there are people without velar fronting, but on the other hand, there are individuals with that rule. Among the latter speakers, some apply velar fronting to produce palatal [ç] in the context of any sound (=\isi{nonassimilatory velar fronting}), while others restrict the occurrence of palatals to the context of front vowels (assimilatory velar fronting). Reference to the “middle \textit{ch}” suggests that for those speakers velars undergo coarticulatory velar fronting, which produces \isi{prevelars}. As noted below, \isi{nonassimilatory velar fronting} is also attested in the data from SDS for the \ipi{Bernese Oberland}.

\citet[245]{Henzen1927} provides some brief remarks on the realization of [x] in the Upper \ipi{Simmental} (Obersimmental), which is broadly defined as the region between Lenk and Boltigen. Henzen’s sparse set of examples in \REF{ex:15:39} shows that the palatal occurs in the context of front vowels and [x] in the context of back vowels.


\TabPositions{.15\textwidth, .3\textwidth, .5\textwidth, .7\textwidth}
\ea%39
\label{ex:15:39}Dorsal fricatives in Obersimmental:
\ea\label{ex:15:39a} χeəs \tab [çeəs] \tab Käse \tab ‘cheese’ \tab 245
\ex\label{ex:15:39b} ıχ \tab  [ɪç] \tab ich \tab ‘I’ \tab 245\\
    dıχ \tab  [dɪç] \tab dich \tab ‘you-\textsc{acc}.\textsc{sg}’ \tab 245
\ex\label{ex:15:39c} nɔx \tab  [nɔx] \tab noch \tab still \tab 245
\z 
\z 

Another (very terse) source for \ipi{Simmental} is \citet[26]{Panizzolo1982}, who remarks in passing that /x/ surfaces as palatal [ç]. One item is provided in that source for [ç] in word-initial position, namely orthographic \textit{ch} in \textit{Chäse} ‘cheese’ and two words for [ç] in postsonorant position, namely [ɑuç] ‘also’ and [dɔç] ‘however’. It is interesting to observe that the final two examples contain the palatal fricative in the context after a back vowel. If these data are representative (and if postsonorant velar fronting also applies after coronal sonorants) then Panizzolo’s variety of \ipi{Simmental} has \isi{nonassimilatory velar fronting}; recall the quote from SiWS given above.

The maps in SDS confirm that the Southwest \ipi{Bernese Oberland} is a velar fronting area; recall the quote from the commentary to Map II 201 given in the preceding section. That the region depicted on my \mapref{map:39} is a velar fronting area can be determined on the basis of the many prepalatal markers (⟦χ'⟧) for some of the places listed above as well as for other places in the same general vicinity. One such map is II 94 for \textit{Kind} ‘child’ with prepalatal markers for eight places in the Southwest \ipi{Bernese Oberland}. All of those villages and towns are depicted on my \mapref{map:39} with markers indicating velar fronting. I have also included on my map velar fronting markers for \ipi{Gsteig} and \ipi{Adelboden}, which are indicated with the ⟦χ'⟧ symbol in the commentary for Map II 94 for the similar word \textit{Korn} ‘grain’. I also include Kiental on my \mapref{map:39} as a velar fronting place because it is indicated on SDS Map II 183 with the prepalatal marker for \textit{backen (bache)} ‘bake-\textsc{inf}’.

In \REF{ex:15:40} I list four places from SDS in the Southwest \ipi{Bernese Oberland} along with the realization in those places of the five words in the first column. Those five words correspond to five of the maps in \tabref{tab:15.2}. The transcriptions are taken directly from SDS, although I have omitted a few of the diacritics for consonants and vowels for greater \isi{transparency}. For the words listed below I only list one of the tokens for each of the places listed in the top row. SDS does not provide full phonetic transcriptions for (\ref{ex:15:40d}, \ref{ex:15:40e}), but that source does indicate that the prepalatal ⟦χ'⟧ occurs in those places. The marker in SDS for \ipi{Gsteig} for (\ref{ex:15:40d}) does not indicate whether or not the dorsal fricative is palatal, prepalatal, or velar.


\TabPositions{.15\textwidth, .3\textwidth, .5\textwidth, .7\textwidth}
\ea%40
\label{ex:15:40}Dorsal fricatives and affricates in the Southwest \ipi{Bernese Oberland} (SDS):
\begin{xlist}
\sn[] \tab ~ \tab \ipi{Lauenen} \tab \ipi{Gsteig} \tab \ipi{Zwischenflüh} \tab \ipi{Adelboden}
\ex\label{ex:15:40a} drücken \tab drükχ{\textquotesingle}ə \tab drükχə \tab trükχ{\textquotesingle}ə  \tab  trükχ{\textquotesingle}
\ex\label{ex:15:40b} Anke \tab ãŋχ{\textquotesingle}ə \tab aŋχ{\textquotesingle}ə \tab aŋkχə  \tab  aŋkχ{\textquotesingle}ə
\ex\label{ex:15:40c} Bänklein \tab b\~{ē}kχ{\textquotesingle}lį \tab beŋkχ{\textquotesingle}lį \tab b\={ę}χ{\textquotesingle}lį  \tab  bēχ{\textquotesingle}lį
\ex\label{ex:15:40d} bache \tab χ{\textquotesingle}  \tab  --- \tab χ{\textquotesingle}    \tab χ{\textquotesingle}
\ex\label{ex:15:40e} rauchen \tab kχ{\textquotesingle}  \tab  kχ  \tab  kχ    \tab kχ
\end{xlist}
\z 

Note that there is some variation in the context of back sounds in (\ref{ex:15:40d}, \ref{ex:15:40e}), where both prepalatal and palatal markers occur. Example \REF{ex:15:40b} likewise illustrates that both prepalatal and palatal occur in the context after a (back) sound, namely the velar nasal preceded by a back vowel.

The sources cited above indicate that velar fronting is well-attested to various degrees in towns and villages confined to an area of about 35km from west to east and 25km from north to south. None of the works mentioned in this section give any indication that velar fronting is active outside of that small region, e.g. to the north of the Lower \ipi{Simmental} (Niedersimmental). The maps in SDS show only palatal markers (but no prepalatal markers) to the (north)west of \ipi{Saanen} (in Abländchen), to the north of \ipi{Zwischenflüh} (in Boltigen, Diemtigen, Reutigen, Faulensee, Aeschiried, and Reichenbach), and in the southwest (in Kandersteg).

The towns and villages in the small area I refer to as the Southwest \ipi{Bernese Oberland} can be thought of collectively as a velar fronting island. That region is bounded to the west by a different language (\ili{French}), and to the south by the Bernese Alps. The German-speaking area to the west in the neighboring canton of Freiburg (\ipi{Jaun}) and the part of Freiburg to the north of \ipi{Jaun} -- the \ipi{Sensebezirk} -- has no velar fronting; see \citet{Stucki1917} and \citet[20]{Henzen1927}. \citet{Marti1985} offers a description of the Bernese dialect between Thun and the parts of the canton of \ipi{Bern} to the north, but that source is clear that there is no velar fronting \citep[21]{Marti1985}. The absence of velar fronting is also attested in the town of Leissingen \citep{Gröger1914c} on the southeast shore of Lake Thun (Thunersee). No source is available for the places in the small passage of about 17km separating Leissingen from the Bernese Alps.

\section{{Tyrol}}\label{sec:15.10}

\ipi{Tyrol} is sometimes described as a region without velar fronting. For example, the dialect dictionary for that region (TiWb) classifies [x] (=⟦ch⟧) as a velar (i.e. guttural) fricative (I: p. xix). A similar assessment of the realization of the fortis dorsal fricative in Tyrolean can be found in \citet[96]{Luick1904}. More recently, \citet[73]{Gabriel1985} writes that the velar fricative is the usual pronunciation in West \ipi{Tyrol} (“In Westtirol, wo der velare Reibelaut die Regel ist …”).

While the absence of velar fronting is probably the norm for most of \ipi{Tyrol}, according to various remarks made in \citet{Schatz1903}, there are velar fronting islands in that region. Consider the following passage \citep[21]{Schatz1903}:

\begin{quote}
Der Reibelaut \textit{χ} ist wie alle Gaumenlaute nicht an eine bestimmte Artikulationsstelle gebunden, wie etwa der Lippenreibelaut \textit{f}. Nach Lauten, welche am harten Gaumen gebildet werden, wird auch \textit{χ} etwas weiter vorn gebildet, doch kennt das Inntal und Etschtal … nur mehr einen einzigen Gaumenreibelaut, der am weichen Gaumen gebildet wird. Dagegen hat in Nordtirol das \ipi{Ötztal}, Sill- und \ipi{Zillertal}, in Südtirol das Passeier, das obere Eisack- und Pustertal, das Iseltal … den ach-Laut und den ich-Laut, diesen nach palatalen Vokalen ....

“Like all other dorsal sounds, the fricative \textit{χ} is not bound to a particular place of articulation, as for example the labial fricative \textit{f}. After sounds produced on the hard palate, \textit{χ} has a slightly more advanced pronunciation, but Inntal and the Etchtal only have a single dorsal fricative, which is produced on the soft palate. By contrast, \ipi{Ötztal}, \ipi{Silltal}, \ipi{Zillertal} in North \ipi{Tyrol}, and Passeier(tal), Upper Eisacktal and Pustertal, Iseltal in South \ipi{Tyrol} … have the ach-Laut and the ich-Laut, the latter occurring after front vowels ...”
\end{quote}

According to the sources cited below, Schatz’s observation that velar fronting is active in various enclaves in \ipi{Tyrol} can be confirmed, although the data in those sources do not always agree that the triggers are restricted to front vowels. \mapref{map:40} indicates areas with and without velar fronting in \ipi{Tyrol} which are commented on below.\footnote{{I only consider here the status of velar fronting in secluded parts of \ipi{Tyrol} and therefore do not discuss urban areas. \ipi{Innsbruck} is indicated on \mapref{map:3} and \mapref{map:40} as a non-velar fronting variety on the basis of the phonetic transcriptions from one of \citegen{Moosmüller1991} speakers. On the other hand, her second speaker from \ipi{Innsbruck} clearly has (postsonorant) velar fronting.}}

\ip{Tyrol}
\begin{map}
% \includegraphics[width=\textwidth]{figures/VelarFrontingHall2021-img046.png}
\includegraphics[width=\textwidth]{figures/Map40_15.8.pdf}
\caption[{Tyrol}]{{Tyrol}. The white rectangle indicates assimilatory postsonorant velar fronting, shaded squares nonassimilatory postsonorant velar fronting and circles the absence of postsonorant velar fronting. Lined rectangles are potential velar fronting regions. 1=\citet{Schatz1897}, 2=\citet{Egger1909}, 3=\citet{Insam1936}, 4=\citet{Hathaway1979}, 5=\citet{Moosmüller1991}, 6=\citet{Kollmann2007}, 7=VALTS, 8=TSA.}\label{map:40}
\end{map}

I discuss first data from two Ortsgrammatiken and then I turn to the linguistic atlases for this region, namely VALTS and TSA. All of the sources and places described below are depicted on \mapref{map:40}.\largerpage[2]

\citet{Insam1936} discusses the broad area in and around \ipi{Meran}. In his discussion of phonetics (p. 12) Insam observes that the fortis dorsal fricative (his ⟦χ⟧) -- as well as the corresponding \isi{affricate} (his ⟦kχ⟧) -- can be realized as palatal (articulated on the hard palate) or velar (articulated on the soft palate) depending on both the phonological context and the place within the greater \ipi{Meran} region. Insam writes that the realization is palatal in the neighborhood of \textit{i, e}  in the valleys (“in den Tälern”), but that it is consistently realized as [x] in \ipi{Naturns}, and usually realized as [x] in \ipi{Meran}. It is clear from the discussion on p. 12 that one of the valleys he is referring to is \ipi{Passeiertal}. His data for \ipi{Naturns} (without velar fronting) and \ipi{Passeiertal} (with velar fronting) are presented in \REF{ex:15:41} and \REF{ex:15:42} respectively. Although Insam’s description implies that palatals only occur after front vowels, he provides several words with those segments in the context after back vowels, e.g. \REF{ex:15:42b}. If these data are representative, then \ipi{Passeiertal} illustrates Trigger Type F (\chapref{sec:14}). Other places with postsonorant velar fronting mentioned by \citet[49]{Insam1936} are Ulten and Hafling, although that source only provides a sparse set of data ([siççər] ‘certainly’, [ʃiɒç] ‘unattractive’).\largerpage[2]



\TabPositions{.15\textwidth, .3\textwidth, .5\textwidth, .8\textwidth}
\ea%41
\label{ex:15:41}Velar fricatives and affricates in \ipi{Naturns}:

\ea\label{ex:15:41a} šrękχ \tab [ʃrɛkx] \tab Schreck \tab ‘scare’ \tab 12\\
    glikχ \tab [glikx] \tab Glück \tab ‘fortune’ \tab 12\\
    šiχχər \tab [sixxər] \tab sicher \tab ‘certainly’ \tab 49
\ex\label{ex:15:41b} liɒkχ \tab [liɒkx] \tab Licht \tab ‘light’ \tab 12\\
    šiɒχ \tab  [ʃiɒx] \tab unschön \tab ‘unattractive’ \tab 49\\
    miɒχ \tab [miɒx] \tab würde machen \tab ‘would do-\textsc{1/3}\textsc{sg}’ \tab 12\\
    rokχ \tab [rokx] \tab Rock \tab ‘skirt’ \tab 12\\
    lukχ \tab [lukx] \tab Lücke \tab ‘gap’ \tab 12
\z
\ex%42
\label{ex:15:42}Palatal fricatives and affricates in \ipi{Passeiertal}:
\ea\label{ex:15:42a} šręk͡χ \tab [ʃrɛkç] \tab Schreck \tab ‘scare’ \tab 12\\
    glik͡χ \tab [glikç] \tab Glück \tab ‘fortune’ \tab 12\\
    šiχ͡χər \tab [siççər] \tab sicher \tab ‘certainly’ \tab 49
\ex\label{ex:15:42b} liɒk͡χ \tab [liɒkç] \tab Licht \tab ‘light’ \tab 12\\
    šiɒ͡χ \tab  [ʃiɒç] \tab unschön \tab ‘unattractive’ \tab 49\\
    miɒ͡χ \tab [miɒç] \tab würde machen \tab ‘would do-\textsc{1/3}\textsc{sg}’ \tab 12\\
    rok͡χ \tab [rokç] \tab Rock \tab ‘skirt’ \tab 12\\
    luk͡χ \tab [lukç] \tab Lücke \tab ‘gap’ \tab 12
    \z
\z 

Since velar affricates and fricatives are lacking in \ipi{Passeiertal} (in postsonorant position), I treat the palatals in that context as underlying palatals (/ç/, /kç/); recall \chapref{sec:14}.

A second velar fronting valley indicated on \mapref{map:40} is \ipi{Silltal}. \citet{Egger1909} describes the phonetics of consonants and vowels in that area. \citet[15]{Egger1909} stresses that dorsal (“gutturalˮ) fricatives, affricates, and stops can be articulated either on the hard palate in the context after front vowels or on the soft palate in the context of back segments. Since his data for the velar vs. palatal distinction are primarily fricatives (⟦x⟧=[x]; ⟦\.{x}⟧=[ç]), I ignore stops and affricates below. The data in \REF{ex:15:43} illustrate the pattern for postsonorant position:\largerpage

\ea%43
\label{ex:15:43}Dorsal fricatives in \ipi{Silltal}:
\ea\label{ex:15:43a}  pǫxxn̥ \tab [pɔxxn] \tab backen \tab ‘bake\textsc{{}-inf}’ \tab 16\\
     dọ̄x  \tab  [dɔːx] \tab Dach \tab ‘roof’ \tab 16\\
     ɑ̄xl  \tab  [ɑːxl] \tab kränklich \tab ‘sickly’ \tab 16
\ex\label{ex:15:43b}  p\={æ}x \tab [pæːx] \tab Pech \tab ‘misfortune’ \tab 16
\ex\label{ex:15:43c}  fī\.{x}  \tab  [fiːç] \tab Vieh \tab ‘cattle’ \tab 16\\
     mi\.{x}\.{x}l̥ \tab [miççl] \tab Michael \tab ‘(name)’ \tab 16\\
     šprǖ\.{x} \tab [ʃpryːç] \tab Spruch \tab ‘saying’ \tab 8\\
     wö\.{x}\.{x}ə \tab [wøççə] \tab unclear gloss  \tab ~ \tab  8
\ex\label{ex:15:43d}  mel\.{x}n̥ \tab [melçn] \tab melken \tab ‘milk\textsc{{}-inf}’ \tab 16\\
     wir\.{x}n̥ \tab [wirçn] \tab wirken \tab ‘seem\textsc{{}-inf}’ \tab 16
\z 
\z 

The words listed above show that velars occur after back vowels (=\ref{ex:15:43a}) or the low front vowel (=\ref{ex:15:43b}), while palatals surface after nonlow front vowels (=\ref{ex:15:43c}) or coronal sonorant consonants (=\ref{ex:15:43d}). \ipi{Silltal} therefore illustrates the relatively uncommon \isi{Velar Fronting-2} (=\ref{ex:15:23a}).

The maps from VALTS with words containing dorsal fricatives are listed in \tabref{tab:15.3}. The underlined sound(s) surface as dorsal affricates for Map III 5 and as dorsal fricatives in all other maps. The dorsal fricatives can be either in the context after a sonorant or in word-initial position.

\begin{table}
\caption{Maps from VALTS with dorsal fricatives/affricates in postsonorant position and word-initial position}
\label{tab:15.3}
\todo[inline]{check references to this table´}
\begin{tabular}{ll}
\lsptoprule
Examples & Map no.\\\midrule
Acker ‘field’ & III 41b\\
bücken ‘stoop-\textsc{inf}’ & III 41b\\
Decke ‘blanket’ & III 41b\\
bachen (=backen) ‘bake-\textsc{inf}’ & III 45a\\
Küche ‘kitchen’                      & III 45a\\
Rechen ‘rake’                        & III 45a\\
trocken ‘dry’ & III 45b\\
Mark ‘borderland’ & III 46\\
stark ‘strong’      & III 47\\
stärker ‘stronger’  & III 47\\
Birke ‘birch tree’ & III 48\\
Kalk ‘lime’ & III 49\\
melken ‘milk-\textsc{inf}’ & III 50\\
Molken ‘whey-\textsc{pl}’ & III 51\\
Wolke ‘cloud’ & III 52\\
Milch ‘milk’ & III 53\\
Floh ‘flea’    & III 59\\
Flöhe ‘flea-\textsc{pl}’  & III 59\\
Schuh ‘shoe’   & III 59\\
Schuhe ‘shoe-\textsc{pl}’ & III 59\\
Berg ‘mountain’ & III 5\\
Kind ‘child’ & III 40a\\
Kuh ‘cow’& III 40a\\
Kasten ‘box’ & III 40a\\
klein ‘small’ & III 40b\\
Knie ‘knee’ & III 40b\\
Kraut ‘herb’ & III 40b\\
Kitz ‘young goat’ & III 60a\\
kitzen ‘give birth to & III 60b\\
young goat-\textsc{inf}’ & III 60b\\
\lspbottomrule
\end{tabular}
\end{table}

Like SDS (\tabref{tab:15.1}), VALTS recognizes three places of articulation for dorsal fricatives/affricates \citep[74]{Gabriel1985}: ⟦χ⟧ (=palatal), ⟦x⟧ (=velar), and ⟦χ'⟧/⟦χ'{}'⟧ (=prepalatal or extreme prepalatal fricative (“präpalataler bzw. extrem präpala\-ta\-ler Reibelaut”)). I summarize the three categories in VALTS and my interpretation thereof in \tabref{tab:15.4}.\largerpage[2]

\begin{table}
\caption{\label{tab:15.4}VALTS symbols for dorsal fricatives and their probable interpretation}
\fittable{\begin{tabular}{lll}
\lsptoprule
VALTS term and symbol & Phonological Features & Phonetic realization\\\midrule
prepalatal ⟦χ'⟧/⟦χ'{}'⟧ & [coronal, dorsal] & [ç], [ç̟], [ɕ]\\
palatal ⟦χ⟧ & [dorsal] (or [coronal, dorsal]) & [x̟]  (or [ç])\\
velar ⟦x⟧ & [dorsal] & [x], [χ]\\
\lspbottomrule
\end{tabular}}
\end{table}

Given the maps from VALTS, the first area to consider is the one comprising the five velar fronting villages aligned along the Ötztaler Ache (in \ipi{Ötztal}): Umhausen, Längenfeld, Sölden, Obergurgl, and Vent. It is important to stress that those communities are isolated from all of surrounding villages given the mountainous terrain. For example, the closest place to Längenfeld in the west is St. Leonhard (Pitztal), but neither streets nor railways connect that place directly with Längenfeld or with any of the other velar fronting villages in \ipi{Ötztal}. The five velar fronting varieties of \ipi{Ötztal} are similarly cut off from the places to the south, e.g. Schnalstal in Italy (South \ipi{Tyrol}).

The velar fronting markers (lightly shaded squares) in \ipi{Ötztal} on \mapref{map:40} are indicated on the VALTS maps in \tabref{tab:15.3} with markers representing prepalatals (⟦χ'⟧/⟦χ'{}'⟧). There can be little doubt that the five velar fronting places in \ipi{Ötztal} collectively comprise a velar fronting island because they are in a secluded valley surrounded by places in which /x/ and /kx/ are consistently realized as velar.\footnote{\citet[71]{Kranzmayer1956} perceived of the prepalatal fricatives and affricates in \ipi{Ötztal} as sibilants. As indicated in \tabref{tab:15.4}, I see the \isi{sibilant} realization of ⟦χ'⟧/⟦χ'{}'⟧ as the alveolopalatal fricative ([ɕ]).}

Since the velar fronting island of \ipi{Ötztal} has prepalatal markers in postsonorant position after front vowels, liquids, and back vowels and in word-initial position before any sound, the data from VALTS suggest that this area is characterized by \isi{nonassimilatory velar fronting} (Trigger Type F; \chapref{sec:14}). No indication is given in VALTS that the five velar fronting places in \ipi{Ötztal}  have velar [x] or [kx]. If this is the correct interpretation of the maps from VALTS then historical /x/ and /kx/ have restructured to /ç/ and /kç/.{\interfootnotelinepenalty=10000\footnote{The conclusion drawn here is also consistent with the maps in VALTS for vowels not listed in \tabref{tab:15.3}. Since those maps are concerned with the modern reflexes of etymological vowels, it is not always clear from the markers what the sounds preceding or following those vowels are for any given place. However, in the maps for vowels followed by a dorsal fricative -- Map II 190a for \textit{Bach} ‘stream’ being a typical example -- the five velar fronting places in \ipi{Ötztal} (together with Moos in Passeier discussed below) are the only ones with markers for prepalatal fricatives.}}

Another valley to consider is \ipi{Passeiertal}, in South \ipi{Tyrol} (Italy); recall \REF{ex:15:42}. The VALTS maps in \tabref{tab:15.3} provide evidence that one particular place in \ipi{Passeiertal} (Moos in Passeier) is a velar fronting village because of the prevalence of prepalatal markers. This generalization holds for /ç/ (< /x/) in postsonorant and word-initial position, but not for the \isi{affricate} /kx/, which surfaces as [kx] in the example listed on Map III 5.

TSA includes a number of maps for words containing dorsal fricatives and affricates in postsonorant position. The words represented by those maps and the corresponding map number are listed in \tabref{tab:15.5}. The scope of that atlas subsumes both North \ipi{Tyrol} (Austria) and South \ipi{Tyrol} (Italy). The transcription system for TSA includes symbols for two velar fricatives/affricates: ⟦x⟧/⟦kx⟧ for voiceless lenis and ⟦χ⟧/⟦kχ⟧ for voiceless fortis (TSA I: 12). The corresponding lenis and fortis palatal sounds are expressed with the addition of the inverted breve diacritic (⟦~{̑ }\kern-4pt⟧) over the fricative symbol. In terms of place of articulation, TSA therefore differs from SDS and VALTS in the sense that it only has two place categories for dorsal fricatives and affricates, namely velar and palatal.

\begin{table}
\caption{Maps from TSA with dorsal fricatives in postsonorant position\label{tab:15.5}}
\begin{tabular}{ll}
\lsptoprule
Examples & Map no.\\\midrule
sehen ‘see-\textsc{inf}’ & 27\\
leihen ‘lend-\textsc{inf}’ & 28\\
aufhin ‘upwards’ & 29\\
Föhre ‘pine’ & 35\\
Truhe ‘chest’ & 36\\
Schuhe ‘shoe-\textsc{pl}’ & 37\\
Schmelhe ‘something small’ & 38\\
Floh ‘flea’& 39\\
hoch ‘high’ & 39\\
Kirche ‘church’ & 40\\
Lache ‘puddle’ & 41\\
Birke ‘birch tree’ & 46\\
wirken ‘seem\textsc{{}-inf}’ & 46\\
Milch ‘milk’ & 64\\
\lspbottomrule
\end{tabular}
\end{table}

An examination of the TSA maps listed above reveals that the typical dorsal place of articulation for the region as a whole is velar. However, several maps depict what appear to be velar fronting islands (recall the quote from \citealt{Schatz1903} at the beginning of this section). The difficulty with TSA is that it is not clear how to evaluate the palatal symbols. My interpretation thereof is summarized  in \tabref{tab:15.6}.

\begin{table}
\caption{\label{tab:15.6}TSA symbols for dorsal fricatives and their probable interpretation}
\begin{tabular}{lll}
\lsptoprule
TSA term and symbol & Phonological Features & Phonetic realization\\\midrule
palatal ⟦χ⟧ & [dorsal] (or [coronal, dorsal]) & [x̟] (or [ç])\\
velar ⟦x⟧ & [dorsal] & [x], [χ]\\
\lspbottomrule
\end{tabular}
\end{table}

On the one hand, it could be that ⟦χ⟧ corresponds to my palatal, e.g. [ç] for the fortis [coronal, dorsal] fricative. One area in \ipi{Tyrol} for which this interpretation is correct is \ipi{Ötztal}. Like the maps from VALTS, the ones from TSA -- in particular TSA Map 41-- indicate the palatal \isi{affricate} (⟦kχ⟧) in the area surrounding the five velar fronting places in \ipi{Ötztal} on my \mapref{map:40}. On the other hand, it is possible that the palatal symbols depicted on the maps in TSA do not represent my palatals, but instead \isi{prevelars}, which are phonologically simplex [dorsal] sounds; recall \tabref{tab:12.37}. A case in point is \ipi{Laurein} (\mapref{map:40}). Several of the maps in TSA suggest that \ipi{Laurein} has velar fronting because of the prevalence of palatal markers (TSA Maps 27, 35, 38, 40). However, as noted in \sectref{sec:12.9.1}, \citet[175]{Kollmann2007} shows that \ipi{Laurein} /x/ and /kx/ surface as \isi{prevelar}, which is not identical to the palatal articulation (ich-Laut) of \il{Standard German}StG. In terms of phonology, \ipi{Laurein} /x/ and /kx/ are simplex [dorsal] sounds that exhibit the effects of \isi{phonetic implementation} (\isi{gradient} fronting), not phonological (categorical) fronting. Recall from \largerpage\sectref{sec:15.7} and earlier in the present section that there was a similar difficulty involving the interpretation of “palatal” sounds in SDS and VALTS. In those two sources the problem was resolved by interpreting only the “prepalatal” symbols as phonologically front dorsals and by assigning the “palatal” markers two different interpretations. It can be seen in \tabref{tab:15.6} that the same strategy is adopted for TSA.

The conclusion is that the regions indicated on the maps in TSA with palatal fricatives and/or affricates can only be interpreted as potential velar fronting islands. I list below four of those valleys, all of which are indicated on \mapref{map:40}.


%%please move \begin{table} just above \begin{tabular
\begin{table}
\caption{Potential velar fronting areas in Tyrol on the basis of the maps in TSA\label{tab:15.7}}
\begin{tabular}{ll}
\lsptoprule
Place & TSA maps\\\midrule
\ipi{Zillertal} & 39, 40, 41, 46\\
Tauferer Tal & 27, 28, 29, 36, 39, 40, 46\\
Ultental & 27, 28, 35, 36, 38, 40, 46\\
Eisacktal & 36, 39\\
\lspbottomrule
\end{tabular}
\end{table}


\section{{East} {Switzerland,} {Liechtenstein,} {and} {Vorarlberg}}\label{sec:15.11}

The region investigated below is depicted on \mapref{map:41}.  It measures approximately 100km from east to west and 80km from north to south and consists of \ipi{East Switzerland}, parts of Southwest Germany (Swabia), the Austrian state of \ipi{Vorarlberg}, and the small nation of \ipi{Liechtenstein}. The area depicted on the map is bounded by Switzerland and Italy to the south, Germany to the north, Switzerland to the west, and Austria (\ipi{Tyrol}) to the east.

The region under discussion is intriguing because it consists of areas with velar fronting embedded within a larger, more conservative one which does not have that process. I discuss below the extent to which velar fronting places situated in this region can be thought of as a velar fronting island.

The places depicted on \mapref{map:41} can be classified into one of three groups: (a) areas with no velar fronting, (b) areas with velar fronting, and (c) potential velar fronting areas. I consider examples of (a--c) in order.\footnote{Several sources discussed below document velar fronting in \ipi{East Switzerland}. Unfortunately, the maps from SDS (\tabref{tab:15.2}) shed little light on this issue because most of the sounds in question are represented with palatal markers (⟦χ⟧) which, as discussed in \sectref{sec:15.7}, are difficult to interpret. The only place on the SDS maps in \ipi{East Switzerland} which has a significant number of prepalatal markers in \ipi{St. Antönien}, which I comment on below.}

\ip{East Switzerland}
\ip{Liechtenstein}
\ip{Vorarlberg}
\ip{Tyrol}
\begin{map}[ph]
% \includegraphics[width=\textwidth]{figures/VelarFrontingHall2021-img047.png}
\includegraphics[width=\textwidth]{figures/Map41_15.9.pdf}
\caption[{East Switzerland}, {Liechtenstein}, {Vorarlberg}, and West {Tyrol}]{{East Switzerland}, {Liechtenstein}, {Vorarlberg}, and West {Tyrol}. Circles indicate no postsonorant velar fronting, white squares (assimilatory) velar fronting, and diagonal squares (potential) velar fronting. 1=\citet{Vetsch1910}, 2=\citet{Hausknecht1911}, 3=\citet{Berger1913}, 4=\citet{Wiget1916}, 5=\citet{Meinherz1920}, 6=\citet{Jutz1922}, 7=\citet{Jutz1925}, 8=\citet{Trüb1951}, 9=\citet{Gabriel1963}, 10=\citet{BethgeBonnin1969}, 11=VALTS, 12=SDS.}\label{map:41}
\end{map}

\clearpage
\subsection{Areas with no velar fronting} 
In the eastern parts of \mapref{map:41} velars like /x/ surface as [x] regardless of context. Those places extend from the town of \ipi{Samnaun} (Switzerland) in the south to Oberstdorf, Sonthofen, and Thalkirchdorf (in Allgäu, Germany) in the north, as well as the numerous villages of Austria (West \ipi{Tyrol}) in between. The western part of \mapref{map:41} (Switzerland) is also characterized by an absence of velar fronting. This is clearly the case in the northwest from Lake Constance (Bodensee) extending south to the areas around \ipi{St. Gallen} and \ipi{Appenzell} and further south to \ipi{Toggenburg} (e.g. Krummenau, Wildhaus). Not depicted on \mapref{map:41} is the non-velar fronting area in the canton of \ipi{Glarus} described by \citet{Streiff1915} to the west of Walenstadt and Quarten.

The conclusion is that there is a relatively narrow central region between those two broad non-velar fronting areas on the periphery. The narrow region referred to here is characterized by velar fronting (or potential velar fronting) and forms -- roughly speaking -- a column of about 65km from east to west and 70km from north to south.\footnote{{I am aware of three studies for places in \ipi{Vorarlberg} documenting the absence of velar fronting within that column. Those three places are \ipi{Hohenems} \citep{Seemüller1909c}, \ipi{Nenzing} (\citealt{SchneiderMarte1910}), and \ipi{Lauterach} (\citealt{SchneiderMarte1910}). It is possible that the non-velar fronting areas depicted on \mapref{map:41} were once more extensive than they are in the present day.} }

\subsection{Velar fronting areas} 
Two velar fronting varieties are attested in Northeast Switzerland. The first is the \ipi{Rheintal} dialect in the canton of \ipi{St. Gallen} \citep{Berger1913}, which was discussed in \sectref{sec:3.4}. The second is the dialect spoken in \ipi{Appenzell} described by \citet{Vetsch1910}. This region subsumes the two cantons of \ipi{Appenzell} Innerrhoden and \ipi{Appenzell} Ausserrhoden, which are both completely surrounded by the canton of \ipi{St. Gallen}.\largerpage

According to \citet[16]{Vetsch1910}, the velar obstruents [k g x kx] can show some degree of coarticulatory fronting in the context before and after front vowels throughout the \ipi{Appenzell} region. However, in part of that area the velar fricative [x] -- including the corresponding geminate [xx] -- and the velar \isi{affricate} [kx] surface as palatal (=⟦χ χχ kχ⟧) in the neighborhood of front sounds. \citet[6]{Vetsch1910} calls the area with these palatal sounds Kurzenberg, which subsumes five municipalities (Gemeinden) of \ipi{Appenzell} Ausserrhoden (Heiden, Lutzenberg, Wolfhalden, Walzenhausen, Reute), as well as one municipality of \ipi{Appenzell} Innterrhoden (Oberegg). In the parts of \ipi{Appenzell} not belonging to Kurzenberg, dorsal fricatives and affricates surface as velar even in the context of front sounds. The velar fronting areas Vetsch calls Kurzenberg are situated roughly in the rectangle indicated on \mapref{map:41}.

The Kurzenberg examples in \REF{ex:15:44} show the distribution of the velar \isi{affricate} and its palatal counterpart. In word-initial position, [kx] surfaces a back vowel (=\ref{ex:15:44a}) and the palatal [kç] before a front vowel (=\ref{ex:15:44b}) or coronal sonorant consonant (=\ref{ex:15:44c}). The data in \REF{ex:15:44} are accounted for formally with \isi{Wd-Initial Velar Fronting-8} (=\ref{ex:15:13}).

\ea%44
\label{ex:15:44}Dorsal affricates in \ipi{Appenzell} (Kurzenberg):
\ea\label{ex:15:44a} kxɔštə \tab [kxɔʃtə] \tab kosten \tab ‘cost\textsc{{}-inf}’ \tab 160\\
    kxɑts \tab [kxɑts] \tab Katze \tab ‘cat’ \tab 160
\ex\label{ex:15:44b} kχištə \tab [kçiʃtə] \tab Kiste \tab ‘box’ \tab 160\\
    kχellə \tab [kçellə] \tab Kelle \tab ‘trowel’ \tab 160
\ex\label{ex:15:44c} kχrɔt \tab [kçrɔt] \tab Kröte \tab ‘toad’ \tab 160\\
    kχlɛbə \tab [kçlɛbə] \tab kleben \tab ‘stick-\textsc{inf}’ \tab 160\\
    kχnǖ \tab [kçnyː] \tab Knie \tab ‘knee’ \tab 160
    \z
\z

The data in \REF{ex:15:45} illustrate that the occurrence of postsonorant velars and palatals in Kurzenberg is a function of the preceding vowel. It can be seen here that velars occur after full back vowels (=\ref{ex:15:45a}) or after a diphthong ending in \isi{schwa} (=\ref{ex:15:45b}) and that palatals surface after front vowels (=\ref{ex:15:45c}). Note that the vowel preceding \isi{schwa} in \REF{ex:15:45b} is front. The only examples provided by Vetsch for category \REF{ex:15:45c} have high front vowels. The optionality involving tonic vowels ([y] vs. [yə]) illustrated in the final example in \REF{ex:15:45b} and \REF{ex:15:45c} shows the regularity of velar fronting: If the vowel is front ([y]) then /xx/ surfaces as palatal, but if it surfaces as a diphthong ending in a back vowel (\isi{schwa}), then /xx/ is realized as velar.

\ea%45
\label{ex:15:45}Dorsal fricatives in \ipi{Appenzell} (Kurzenberg):
\ea\label{ex:15:45a}  lɔxx \tab [lɔxx] \tab Loch \tab ‘hole’ \tab 161\\
     mɑxxə \tab [mɑxxə] \tab machen \tab ‘do-\textsc{inf}’ \tab 161\\
\ex\label{ex:15:45b}  štiəxx \tab [ʃtiəxx] \tab Stich \tab ‘sting’ \tab 102\\
     ksiəxt \tab [ksiəxt] \tab Gesicht \tab ‘face’ \tab 102\\
     trüəxxnə \tab [tryəxxnə] \tab trocknen \tab ‘dry-\textsc{inf}’ \tab 102
\ex\label{ex:15:45c}  līχt \tab  [liːçt] \tab leicht \tab ‘easy’ \tab 102\\
     siχχər \tab [siçər] \tab sicher \tab ‘certainly’ \tab 102\\
     trüχχnə \tab [tryççnə] \tab trocknen \tab ‘dry-\textsc{inf}’ \tab 161
\z 
\z 


Recall from \sectref{sec:3.4} that the set of velar fronting triggers for \ipi{Rheintal} is restricted to nonlow front vowels because phonologically [+low] sounds like /ɛ/ fail to induce fronting (=\isi{Velar Fronting-2} in \ref{ex:15:23a}). Since Vetsch does not provide the crucial data for /x/ in the context of vowels like /ɛ/ it is not possible to say whether or not \ipi{Appenzell} and \ipi{Rheintal} are the same or different in terms of triggers. In any case, the data in \REF{ex:15:45} can be captured with either \isi{Velar Fronting-1} (=\ref{ex:15:2}) or \isi{Velar Fronting-13} (=\ref{ex:15:4}).

One difference between the two neighboring dialects is the patterning of dorsal fricatives in the context after a diphthong consisting of a front vowel plus \isi{schwa}. As indicated in \REF{ex:15:45b} the velar fricative in \ipi{Appenzell} surfaces in that context. By contrast, in \ipi{Rheintal} the palatal surfaces in this environment (e.g. [liːəçt] ‘light’). The occurrence of the palatal was accounted for with \isi{Schwa Fronting-1} (\sectref{sec:3.4}), which is present in \ipi{Rheintal}, but absent in \ipi{Appenzell}.

The third velar fronting variety in \ipi{East Switzerland} is the one described by \citet{Meinherz1920}. Recall from \sectref{sec:3.3} that Meinherz’s dialect (\ipi{Maienfeld}) subsumes three velar fronting municipalities, namely \ipi{Maienfeld}, Fläsch and Malans. By contrast, the neighboring community of Jenins has no velar fronting. All of those places are indicated on \mapref{map:41}.

The fourth velar fronting area depicted on \mapref{map:41} is the one described by \citet{Jutz1925}, which comprises all of \ipi{Liechtenstein} and South \ipi{Vorarlberg}. It is clear from \citet{Jutz1925} that Liechtenstein-South \ipi{Vorarlberg} has both velar and palatal fricatives. \citet[26]{Jutz1925} writes: “Der Reibelaut χ wird im ganzen Gebiete zwischen den ɑχ- und iχ-Laut  unterschieden, von denen hier der velare mit χ, der palatale mit x bezeichnet wird”. (“The fricative χ is differentiated in the entire area between the ɑχ- and iχ-Laut, of which the velar is transcribed here with χ and the palatal with x”). At a later point (p. 207), Jutz makes it clear that the dialect also distinguishes palatal and velar affricates.

In word-initial position, the velar \isi{affricate} occurs before a back vowel (=\ref{ex:15:46a}) and the corresponding palatal before a front vowel (=\ref{ex:15:46b}) or a coronal sonorant consonant (=\ref{ex:15:46c}).\footnote{{Affricates are also attested in some parts of Liechtenstein-South \ipi{Vorarlberg} in postsonorant position, but I do not consider these data because of the irregularities referred to in \citet[207]{Jutz1925}.}} The distribution of velars and palatals in \REF{ex:15:46} can be captured formally with \isi{Wd-Initial Velar Fronting-8} (=\ref{ex:15:13}).

\ea\label{ex:15:46} Dorsal affricates in Liechtenstein-South \ipi{Vorarlberg}:
\ea\label{ex:15:46a} kχunt \tab [kxunt] \tab kommt \tab ‘come\textsc{{}-3sg}’ \tab 215
    kχoštə \tab [kxoʃtə] \tab kosten \tab ‘cost\textsc{{}-inf}’ \tab 207\\
    kχɑts \tab [kxɑts] \tab Katze \tab ‘cat’ \tab 207\\
\ex\label{ex:15:46b} kxīmmə \tab [kçiːmmə] \tab Keim \tab ‘germ’ \tab 207\\
    kxįfl \tab [kçɪfl̩] \tab Kiefer \tab ‘pine tree’ \tab 229\\
    kxǣr \tab [kçæːr] \tab Keller \tab ‘cellar’ \tab 223\\
    kxiərhə \tab [kçiərhə] \tab Kirche \tab ‘church’ \tab 224\\
\ex\label{ex:15:46c} kxr\={ę}ijə \tab [kçrɛːijə] \tab krähen \tab ‘crow-\textsc{inf}’ \tab 207\\
    kxlī \tab [kçliː] \tab klein \tab ‘small’ \tab 207\\
    kxnęxt \tab [kçnɛçt] \tab Knecht \tab ‘vassal’ \tab 207
    \z
\z 

The data in \REF{ex:15:47} illustrate the distribution of velar and palatal fricatives in postsonorant position. The velar surfaces after a back vowel (=\ref{ex:15:47a}) and the palatal after a front vowel (=\ref{ex:15:47b}), or a liquid (=\ref{ex:15:47c}). If the first part of a schwa-final diphthong is a front vowel then the dorsal fricative following that diphthong is palatal (=\ref{ex:15:47d}), but if the first component of a schwa-final diphthong is a back vowel then a dorsal fricative after that diphthong is velar (=\ref{ex:15:47e}). This is the default pattern which can be captured with \isi{Velar Fronting-1} (=\ref{ex:15:2}).

\ea%47
\label{ex:15:47}Dorsal fricatives in Liechtenstein-South \ipi{Vorarlberg}:
\ea\label{ex:15:47a}  rū̜χ  \tab  [rʊːx] \tab Rauch \tab ‘smoke’ \tab 209\\
     tɑχ  \tab  [dɑx] \tab Dach \tab ‘roof’ \tab 209\\
\ex\label{ex:15:47b}  glīx  \tab  [gliːç] \tab gleich \tab ‘same’ \tab 210\\
     ix  \tab  [iç] \tab ich \tab ‘I’ \tab 210\\
     štįx  \tab  [ʃtɪç] \tab Stich \tab ‘sting’ \tab 209\\
     flüxt \tab [flyçt] \tab flicht \tab ‘braid-\textsc{3}\textsc{sg}’ \tab 212\\
     ręxnə \tab [rɛçnə] \tab rechnen \tab ‘calculate-\textsc{inf}’ \tab 207\\
     ǣxərle \tab [æːçr̩li] \tab Eichhörnchen \tab ‘squirrel’ \tab 213
\ex\label{ex:15:47c}  melx \tab [melç] \tab Milch \tab ‘milk’  \tab   209\\
     štɑrx \tab [ʃtɑrç] \tab stark \tab ‘strong’  \tab  208
\ex\label{ex:15:47d}  tsīəxl \tab [tsiːəçli] \tab Zieche, dim \tab ‘cover-\textsc{dim}’ \tab 207\\
     nüəxtr \tab [nyəçtr̩] \tab nüchtern \tab ‘sober’ \tab 214\\
\ex\label{ex:15:47e}  būəχ \tab [buːəx] \tab Buch \tab ‘book’ \tab 209
\z 
\z 

To summarize: In postsonorant position and in word-initial position, velar fronting applies in the context of any coronal sonorant. The contrast between palatal and velar in (\ref{ex:15:47d}, \ref{ex:15:47e}) requires \isi{Schwa Fronting-1} to \isi{feed} postsonorant velar fronting, as in \ipi{Rheintal}.\footnote{{Jutz transcribes the palatal fricative occasionally after back vowels, e.g. ⟦prūxt⟧ ‘use-}\textrm{\textsc{part}}\textrm{’, ⟦fǭxt⟧ ‘catch-}\textrm{\textsc{3}\textsc{sg}}\textrm{’,  ⟦ǣnədɑxtsk⟧ ‘eighty-one’. These could be transcriptional errors. Alternatively, they might indicate that certain speakers have \isi{nonassimilatory velar fronting} (Trigger Type F; \chapref{sec:14}).}}

The fifth velar fronting place in the region depicted on \mapref{map:41} is the town of \ipi{Vandans} in \ipi{Vorarlberg} \citep{Jutz1922}. Jutz observes that \ipi{Vandans} possesses both velar and palatal fricatives and affricates. He writes (p. 276): “Von den Reibelauten bezeichnen χ und x das schriftdeutsche ch, doch mit dem Unterschiede, daβ eine Zweiteilung in den sogegannten ɑχ- und ix-Laut vorgenommen wurde…Diese beiden Laute werden in der Mundart von \ipi{Vandans} und Umgebung deutlich auseinandergehalten”.  (“Among the fricatives, χ and x depict written German ch with the difference that a distinction between the so-called ɑχ- and ix-sound was made…These two sounds are clearly distinguished in the dialect of \ipi{Vandans} and in the vicinity thereof”).

In word-initial position the velar \isi{affricate} occurs before back vowels (=\ref{ex:15:48a}), while the palatal \isi{affricate} surfaces before front vowels (=\ref{ex:15:48b}) or coronal sonorant consonants (=\ref{ex:15:48c}). The patterning of velars and palatals in \REF{ex:15:48} is expressed formally with \isi{Wd-Initial Velar Fronting-8} (=\ref{ex:15:13}).

\ea%48
\label{ex:15:48}Dorsal affricates in \ipi{Vandans}:
\ea\label{ex:15:48a} kχūə \tab [kxuːə] \tab Kuh \tab ‘cow’ \tab 290\\
    kχ\k{u}rts \tab [kxʊrts] \tab kurz \tab ‘short’ \tab 290\\
    kχɑts \tab [kxɑts] \tab Katze \tab ‘cat’ \tab 292
\ex\label{ex:15:48b} kxind \tab [kçind] \tab Kind \tab ‘child’ \tab 289\\
    kx\={į}rə \tab [kçɪːrə] \tab kehren \tab ‘sweep-\textsc{inf}’ \tab 289\\
    kxünɩg \tab [kçynɪg] \tab König \tab ‘king’ \tab 290\\
    kxü̜rpsə \tab [kçʏrpsə] \tab Kürbis \tab ‘pumpkin’ \tab 290\\
    kxessɩ \tab [kçessɪ] \tab Kessel \tab ‘kettle’ \tab 292
\ex\label{ex:15:48c} kxrumm \tab [kçrumm] \tab krumm \tab ‘bent’ \tab 292\\
    kxlębə \tab [kçlɛbə] \tab kleben \tab ‘stick-\textsc{inf}’ \tab 292\\
    kxlī \tab [kçliː] \tab klein \tab  ‘small’ \tab 296
    \z
\z 

The items listed in \REF{ex:15:49} reveal that velar fricatives (singleton and geminate) occur after any back vowel (=\ref{ex:15:49a}) and that palatals surface after any front vowel (=\ref{ex:15:49b}). The occurrence of palatal in \REF{ex:15:49c} and velar in \REF{ex:15:49d} can be accounted for with \isi{Schwa Fronting-1}, as in \ipi{Rheintal} (\sectref{sec:3.4}) and Liechtenstein-\ipi{Vorarlberg}.\footnote{{It is not clear whether or not [x] or [ç] surfaces after a consonant because Jutz has words illustrating both patterns, e.g. ⟦wærχχə⟧ ‘work-}\textrm{\textsc{inf}}\textrm{’ vs. ⟦f\k{u}rxtikt\={ü}r⟧ ‘terribly expensive’. The occurrence of the palatal \isi{affricate} before liquids in \REF{ex:15:48c} suggests that [ç] should be the expected dorsal fricative in the mirror image context (i.e. after liquids). A few of the examples in \citet{Jutz1922} have [x] after a back vowel, e.g. ⟦nɑxt⟧ ‘night’.}} The formal rule for \REF{ex:15:49} is \isi{Velar Fronting-1} (=\ref{ex:15:2}).\largerpage[-1]\pagebreak

\ea%49
\label{ex:15:49}Dorsal fricatives in \ipi{Vandans}:
\ea\label{ex:15:49a} rū̜χ  \tab  [rʊːx] \tab Rauch \tab ‘smoke’ \tab 292\\
    lo̜χχ \tab [lɔxx] \tab Loch \tab ‘hole’ \tab 292\\
    bɑχχ \tab [bɑxx] \tab Bach \tab ‘stream’ \tab 292
\ex\label{ex:15:49b} glīx  \tab  [gliːç] \tab gleich \tab ‘same’ \tab 292\\
    ix  \tab  [iç] \tab ich \tab ‘I’ \tab 292\\
    ksįxt \tab [ksɪçt] \tab Gesicht \tab ‘face’ \tab 292\\
    kr\={į}xt \tab [krɪːçt] \tab gerichtet \tab ‘judge-\textsc{part}’ \tab 289\\
    fēx  \tab  [feːç] \tab Vieh \tab ‘cattle’ \tab 292\\
    knęxt \tab [knɛçt] \tab Knecht \tab ‘vassal’ \tab 291
\ex\label{ex:15:49c} līəxt \tab [liːəçt] \tab Licht \tab ‘light’ \tab 292
\ex\label{ex:15:49d} pūəχ \tab [puːəx] \tab Buch \tab ‘book’ \tab 296
\z 
\z 

In sum,  word-initial velar fronting is triggered by all coronal sonorants and postsonorant velar fronting by front vowels.\footnote{In \ipi{Vandans}, the low front vowels [æ æː] are apparently restricted in their distribution to the context before liquids \citep[289]{Jutz1922}; hence, dorsal fricatives do not occur after those sounds. (No example was found with a word-initial dorsal \isi{affricate} before a low front vowel).}

\citet{BethgeBonnin1969} provide a phonetically transcribed text from a native speaker of the \ipi{Feldkirch} dialect (\ipi{Vorarlberg}). The text distinguishes velar fricatives ([x]) from palatal fricatives ([ç]). Although the number of words with those sounds is small, the generalization can be made that [x] surfaces after a back vowel ([ɑ ɑː ʊ]) and [ç] after a front vowel ([ɪ ʏ]). The text contains no examples of dorsal fricatives after sonorant consonants.

The one place in \ipi{East Switzerland} which is indicated in the SDS maps in \tabref{tab:15.2} with prepalatal symbols is the Walser settlement of \ipi{St. Antönien} in North Grisons. In \REF{ex:15:50} I give the SDS transcriptions for some of the words in that variety of German. On the basis of \REF{ex:15:50} I conclude that \ipi{St. Antönien} is a velar fronting variety of SwG, although not enough data are available to draw conclusions concerning the set of triggers.\largerpage

\ea%50
\label{ex:15:50}Prepalatal fricatives and affricates in \ipi{St. Antönien} (SDS):
\ea Kind \tab χ'{}'
\ex drücken \tab trükχ'ə
\ex Gestank \tab ŝt\={ą}χ'
\ex Bank \tab bẹχ'
\ex stinkt \tab štīχ't
\ex Speicher \tab īχ'
\z
\z 

Finally, I consider the status of velar fronting as indicated on the maps listed in \tabref{tab:15.3} from VALTS. Recall from \tabref{tab:15.4} that VALTS recognizes three places of articulation for dorsal sounds, namely velar (⟦x⟧), palatal (⟦χ⟧), and prepalatal (⟦χ'⟧/⟦χ'{}'⟧). Since it is not clear whether or not the palatal markers indicate phonologically [coronal, dorsal] sounds as opposed to phonologically simplex [dorsal] sounds that surface as phonetically fronted velars (\isi{prevelars}), I focus on those places with the prepalatal markers. An inspection of the maps from \tabref{tab:15.3} reveals the six velar fronting areas listed in \tabref{tab:15.8}. In the first column I list the area and in the second column villages and towns within that area. The first five of those areas are listed under the names for the respective valleys, while the sixth area is a specific town in \ipi{Liechtenstein}. In the third column I give the maps from VALTS which have prepalatal markers for the towns listed in the second column. Note that the final place listed in \tabref{tab:15.8} (\ipi{Triesenberg}) is part of a larger area (\ipi{Liechtenstein}) in which velar fronting is attested (recall \ref{ex:15:47} and \ref{ex:15:48}). The places listed in \tabref{tab:15.8} also have in common that they were settled by people from \ipi{Upper Valais} during the \isi{Walser Migrations} (\sectref{sec:6.3}; \citealt{Bohnenberger1913}, \citealt{Wiesinger1983b}: 902).

\begin{table}
\caption{Velar fronting areas in Vorarlberg/Liechtenstein on the basis of the maps in VALTS\label{tab:15.8}}
\begin{tabularx}{\textwidth}{lQl}
\lsptoprule
Area & Town/village & VALTS maps (volume III)\\\midrule
\ipit{Kleinwalsertal} & Mittelberg, Riezlern & 40a-b, 45a-b, 46, 47, 49--53\\
\ipit{Damülser Tal} & Damüls & 40a-b, 45a-b, 46, 47, 49--53\\
\ipit{Tal der Bregenzer Ache} & Schröcken & 40a-b, 45a-b, 53\\
\ipit{Großes Walsertal} & Sonntag, Blons, Fontanella, Raggal & 40a-b, 45a, 53\\
\ipit{Laternsertal} & Laterns & 45a-b, 53\\
\ipit{Liechtenstein} (Oberland) & \ipit{Triesenberg} & 45a-b, 46, 47, 49, 53\\
\lspbottomrule
\end{tabularx}
\end{table}

Since the velar fronting places listed above have prepalatals in postsonorant position after front vowels, liquids, and back vowels and in word-initial position before any sound, they are characterized by \isi{nonassimilatory velar fronting} (Trigger Type F; \chapref{sec:14}). No indication is given in VALTS that the velar fronting places in \tabref{tab:15.8} have velar [x]; thus, historical /x/ has restructured to /ç/.

\subsection{Potential velar fronting areas}\largerpage
\citet{Trüb1951} investigates the historical development of vowels in the SwG dialect spoken in the area of \ipi{Walensee-Seeztal} (to the west of \ipi{Liechtenstein}). In his charts for consonants (pp. xix--xx), Trüb classifies all dorsal stops and fricatives (fortis/lenis/long/short) -- his ⟦k ɡ χ⟧ -- as “palatal”, although he lists the equivalent nasal (⟦ŋ⟧) as  “velar”. In Footnote 1 (p. xx) he writes: “Das \textit{ch} unserer Landschaft wird im allgemeinen palatal gebildet, also weder präpalatal noch velar”. (“The \textit{ch} in our region is generally pronounced palatal, that is neither prepalatal nor velar”). Given this statement and the proximity of \ipi{Walensee-Seeztal} to the velar fronting areas to the immediate east, I consider it possible that velar fronting may be active in the region. However, given the brevity of the statement in Footnote 1, it is also possible that Trüb’s “palatals” may in fact be \isi{prevelars}; recall \citegen{Kollmann2007} conclusion concerning the realization of sounds like /x/ in \ipi{Laurein}.

\citet{Gabriel1963} investigates historical changes affecting vowels and the inflectional morphology in Vorarlberger \ipi{Rheintal}, a large region in Northwest \ipi{Vorarlberg} which subsumes \ipi{Dornbirn}, \ipi{Lustenau}, and \ipi{Hohenems}. In the section on the phonetics of consonants, \citet[79]{Gabriel1963} provides a one-page description of fricatives. In his transcription system (p. 45), ⟦x⟧ and ⟦χ⟧ represent voiceless lenis and voiceless fortis respectively. Gabriel provides a concise statement concerning the place of articulation of ⟦x⟧ and ⟦χ⟧ on p. 79: “x, χ  bezeichnet immer den ich-Laut”.   (“x, χ always denote the ich-Laut”). On the basis of that terse statement, it could be the case that (nonassimilatory) velar fronting was active historically in the region; however, it could also be the case that we are dealing with \isi{prevelars}. (In contrast to VALTS and SDS, Gabriel presupposes only two places of articulation for dorsal fricatives).

VALTS provides a wealth of data from most of the places listed on \mapref{map:41}. Recall that the velar fronting areas listed in \tabref{tab:15.8} all have prepalatal markers (⟦χ'⟧/⟦χ'{}'⟧) for the maps listed in \tabref{tab:15.3}. Those maps also indicate a number of places in \ipi{Vorarlberg} with palatal markers (⟦χ⟧). Two of those broad areas are indicated on my \mapref{map:41}. First, there is the region south of Lech and east of \ipi{Vandans}. Second, there is the area around Oberstaufen (Allgäu, Germany) extending south to the area around Mellau (\ipi{Vorarlberg}, Austria). Since /x/ is realized in these two regions as “palatal” it is possible that they are characterized by velar fronting, but it is also conceivable that the “palatals” represent phonetically fronted velars (\isi{prevelars}).

It is not easy to determine the status of the narrow -- but sizable -- velar fronting column depicted on \mapref{map:41}. On the one hand, it is possible that that column represents several different velar fronting enclaves (islands) that happen to be in the same general vicinity. On the other hand, it could be that the region as a whole is one large velar fronting area. Since the northernmost potential velar fronting region on \mapref{map:41} extends into an area in Southwest Germany with velar fronting (Swabia), the second interpretation suggests that the column is not a velar fronting island at all, but instead a velar fronting peninsula.

\section{{Summary}}\label{sec:15.12}

\tabref{tab:15.9} lists the places with postsonorant velar fronting discussed in this chapter. I include not only those places that are uncontroversially velar fronting islands but also some of the places discussed in \sectref{sec:15.11} that are probably parts of a large velar fronting peninsula. The modern-day countries are listed in the second column (AT = Austria, CH = Switzerland, CZ = Czech Republic, LI = Liechtenstein, IT = Italy, SL = Slovenia). I do not include any of the areas referred to as potential velar fronting areas, nor do I give those sources with a dataset that is too sparse to determine velar fronting triggers. For greater \isi{transparency} I summarize the triggers for postsonorant velar fronting in the final column of \tabref{tab:15.9} in lieu of the formal rules posited above. If velar fronting is induced by one or more consonant, then this information is stated in the final column. If not enough data are presented in the source to determine whether or not consonants serve as velar fronting triggers, then no reference to consonants is made in the final column. Most of the case studies summarized here only mention data involving liquids (/r l/) as triggers and omit /n/; hence, one can only speculate that the latter sound will always be a velar fronting trigger if one or more of the liquids do.\footnote{\tabref{tab:15.9} categorizes places only according to the triggers because the places discussed in this chapter do not display variation concerning the target segments. One exception is \ipi{Gottschee}, where according to \citet{Lipold1984} the targets for postsonorant and word-initial velar fronting consist of all velar obstruents.}

\begin{table}
\small
\caption{\label{tab:15.9}Velar fronting triggers (postsonorant) in velar fronting islands}
\begin{tabularx}{\textwidth}{Q@{~}llQ}
\lsptoprule
Place &  & Source & Velar fronting triggers\\
\midrule
\ipit{Libinsdorf} & CZ & \citet{Weinelt1940} & FV or /l r n/\\
\ipit{Iglau} & CZ & \citet{Stolle1969} & FV but not /r/\\
\ipit{Altstadt} & CZ & \citet{Seemüller1980c} & FV\\
\ipit{Langenlutsch} & CZ & \citet{Janiczek1911} & FV but not /r/\\
\ipit{Rathsdorf} & CZ & \citet{Graebisch1915} & FV\\
Michelsdorf, Rehsdorf & CZ & \citet{Benesch1979} & FV \\
\ipit{Mährisch Hermersdorf} & CZ & \citet{Benesch1979} & FV or /r/\\
\ipit{Vorder-Ehrnsdorf}, \ipi{Augezd}, Kornitz & CZ & \citet{Benesch1979} & FV but not /r/\\
\ipit{Rothmühl} & CZ & \citet{Benesch1979} & Front unrounded V but not /r/\\
\ipit{Giazza}\slash \ipit{Dreizehn Gemeinden} & IT & \citet{Schweizer1939} & FV or liquids (and back V for some speakers)\\
\ipit{Giazza}\slash \ipit{Dreizehn Gemeinden} & IT & \citet{Mayer1971} & FV or liquids\\
Hinterberg (and other places & SL & \citet{Lipold1984} & FV but not /r/\\
\ipit{Mitterdorf} & SL & \citet{Seemüller1909d} & FV but not /r/\\
\ipit{Vals} & CH & \citet{Gröger1914b} & Nonlow FV or liquids\\
\ipit{Obersaxen} & CH & \citet{Brun1918} & Nonlow FV but not liquids\\
\ipit{Visperterminen} & CH & \citet{Wipf1910} & High FV but not liquids\\
\ipit{Lötschental} & CH & \citet{Henzen1928} & Nonlow FV or liquids\\
\ipit{Upper Valais} & CH & \citet{Rübel1950} & FV or liquids\\
\ipit{Bellwald} & CH & \citet{Schmid1969} & FV or liquids\\
Ried-\ipit{Brig} & CH & \citet{Werlen1977} & FV \\
\ipit{St. Stephan} & CH & \citet{Zahler1901} & Front nonnasalized V\\
\ipit{Frutigen} & CH & \citet{Gröger1914d} & FV\\
\ipit{Saanen} & CH & \citet{Gröger1914e} & FV or /l/\\
\ipit{Silltal} & AT & \citet{Egger1909} & Nonlow FV or liquids\\
\ipit{Passeiertal} & IT & \citet{Insam1936} & FV, liquids, or back V\\
\ipit{Ötztal}, \ipit{Passeiertal} & AT; IT & VALTS & FV, liquids, or back V\\
\ipit{Appenzell} & CH & \citet{Vetsch1910} & FV\\
\ipit{Rheintal} & CH & \citet{Berger1913} & Nonlow FV or liquids\\
\ipit{Maienfeld} & CH & \citet{Meinherz1920} & FV or liquids\\
\ipit{Vandans} & AT & \citet{Jutz1922} & FV\\
Liechtenstein-South \ipit{Vorarlberg} & LI; AT & \citet{Jutz1925} & FV or liquids\\
\ipit{Feldkirch} & AT & \citet{BethgeBonnin1969}  & FV\\
\lspbottomrule
\end{tabularx}
\todo[inline]{I only put FV for front vowel for the time being. I also abbreviated the countries}
\end{table}


The significance of \tabref{tab:15.9} is that it lists a number of geographically disperse places with a wide variety of velar fronting triggers. In certain cases, the triggers represent common patterns, while in other cases they are either rare or otherwise unattested in German dialects. In the following summary I relate how those findings match up with the historical stages posited in \chapref{sec:12} and \chapref{sec:14}.\largerpage

The narrowest set of triggers is attested in \ipi{Visperterminen} (high front vowels but not coronal sonorant consonants), while a slightly broader one (nonlow front vowels but not coronal sonorant consonants) can be observed in  \ipi{Obersaxen}. \chapref{sec:13} demonstrates that the pattern for \ipi{Visperterminen} (Stage 2a) is the norm in Lower Bavaria; the restricted set of triggers for \ipi{Obersaxen} (Stage 2b) is attested outside of Switzerland and depicted on \mapref{map:20}. \ipi{Rothmühl} represents a restricted case of triggers that is otherwise only occurring in \ipi{South Mecklenburg} (front unrounded vowels; Stage 2a'{}'). According to one description of \ipi{St. Stephan}, the velar fronting triggers consist solely of front nonnasalized vowels. The latter pattern is the only one of its kind in German dialects and that it is also extremely rare outside of Germanic. The set of nonlow front vowels or liquids (Stage 2c) is attested as a trigger in \ipi{Vals}, \ipi{Lötschental}, \ipi{Silltal}, and \ipi{Rheintal}. The default pattern for German dialects (front vowels or liquids as postsonorant velar fronting triggers) is well-attested in the material investigated in the present chapter (Stage 2d). Finally, the \isi{nonassimilatory velar fronting} (Stage 2e) is well-documented for several places (e.g. \ipi{Ötztal}).

\tabref{tab:15.10} presents the velar fronting triggers for word-initial position for the places discussed in this chapter. That table shows that there is considerable variation concerning velar fronting triggers in word-initial position. For example, there is a narrow set of triggers in \ipi{Visperterminen} (Stage 2a), \ipi{Obersaxen} (Stage 2b), \ipi{Lötschental} and \ipi{Rheintal} (Stage 2c), South Vorarlberg-\ipi{Liechtenstein} (Stage 2d), and \ipi{Ötztal} (Stage 2e).

\begin{table}
\small
\caption{\label{tab:15.10}Velar fronting triggers (word-initial) in velar fronting islands}

\begin{tabularx}{\textwidth}{QllQ}
\lsptoprule
Place &  & Source & Velar fronting triggers\\\midrule
\ipit{Giazza}\slash \ipit{Dreizehn Gemeinden} & IT & \citet{Schweizer1939} & FV or liquids (and back V for some speakers)\\
Hinterberg (and other places) & SL & \citet{Lipold1984} & FV but not /r/\\
\ipit{Vals} & CH & \citet{Gröger1914b} & FV but not liquids\\
\ipit{Obersaxen} & CH & \citet{Brun1918} & Nonlow FV but not liquids\\
\ipit{Visperterminen} & CH & \citet{Wipf1910} & High FV but not liquids\\
\ipit{Lötschental} & CH & \citet{Henzen19281929,Henzen1932} & Nonlow FV or liquids\\
\ipit{Upper Valais} & CH & \citet{Rübel1950} & FV or liquids\\
\ipit{Bellwald} & CH & \citet{Schmid1969} & FV or liquids\\
Ried-\ipit{Brig} & CH & \citet{Werlen1977} & FV \\
\ipit{St. Stephan} & CH & \citet{Zahler1901} & FV\\
\ipit{Frutigen} & CH & \citet{Gröger1914d} & FV or /n/\\
\ipit{Saanen} & CH & \citet{Gröger1914e} & FV or /n/\\
\ipit{Ötztal}, \ipit{Passeiertal} & AT; IT & VALTS & FV, liquids, or back V\\
\ipit{Appenzell} & CH & \citet{Vetsch1910} & FV or /r, l, n/\\
\ipit{Rheintal} & CH & \citet{Berger1913} & Nonlow FV or liquids\\
\ipit{Vandans} & AT & \citet{Jutz1922} & FV or liquids\\
Liechtenstein-South \ipit{Vorarlberg} & LI; AT & \citet{Jutz1925} & FV\\
\lspbottomrule
\end{tabularx}
\end{table}


With the exception of \ipi{St. Stephan}, all of the historical stages described in Tables \ref{tab:15.9} and \ref{tab:15.10} are attested in the varieties of velar fronting discussed in Chapters \ref{sec:3}--\ref{sec:13}. The importance of velar fronting triggers for velar fronting islands is that -- as islands -- velar fronting must have phonologized in each place independently (\isi{polygenesis}). It is therefore remarkable that the places listed in Tables \ref{tab:15.9} and \ref{tab:15.10} confirm to the typologically attested generalizations discussed in \chapref{sec:12} and \chapref{sec:13}. For example, the segments inducing (assimilatory) velar fronting consist of a natural class drawn from the set of sounds referred to throughout this book as coronal sonorants. The attested natural classes for triggers listed in Tables \ref{tab:15.9} and \ref{tab:15.10} obey the \isi{Implicational Universal for Palatalization Triggers} without exception; hence, none of the unattested Trigger Types discussed in \sectref{sec:12.8.1} can be found among velar fronting islands.

The one unique case mentioned above (\ipi{St. Stephan}) is consistent with the \isi{rule generalization} approach adopted in this book. The set of velar fronting triggers in that place (front oral vowels) suggests that that natural class be assigned a unique Trigger Type with its own historical stage. All other velar fronting varieties of German discussed in this book fall into two groups: (a) those with only oral vowels and (b) those with oral vowels and nasalized vowels but where dorsal fricatives are absent after the latter sounds (e.g. \ipi{Visperterminen}). Since \ipi{St. Stephan} is the only velar fronting variety discovered in which dorsal fricatives occur in the context after front nasalized vowels it is not possible to know how rare or common that pattern is.
\is{velar fronting island|)}
