\chapter{When and where was velar fronting phonologized?}\label{sec:16}

\section{Introduction}\label{sec:16.1}

Although the preceding chapters have offered a diachronic treatment of velar fronting in a broad spectrum of German dialects, nothing at all has been said about how that change fits into the well-established stages in the history of German (Appendix~\ref{appendix:e}). In the present chapter I demonstrate how the linguistic evidence discussed in this book can shed light on when velar fronting was phonologized. That linguistic evidence is shown to be corroborated by philological evidence discussed in the earlier literature. I also discuss the extent to which the material from German dialects discussed in previous chapters can shed light on where velar fronting was phonologized.\largerpage

Establishing an accurate time frame for the \isi{phonologization} of velar fronting requires that I consider first word-initial position (\sectref{sec:16.2}) and then postsonorant position (\sectref{sec:16.3}). The question of where (geographically) velar fronting might have been phonologized can be found in \sectref{sec:16.4}. Next, I address the issue of \isi{directionality} as it relates to the earliest phonologized rule of velar fronting (\sectref{sec:16.5}). In \sectref{sec:16.6} I take a closer look at the historical model proposed in \sectref{sec:2.5} and show how it accounts for the general patterns discussed in Chapters~\ref{sec:3}–\ref{sec:13}. Finally, in \sectref{sec:16.7} I discuss the extent to which it is possible to make meaningful statements concerning why velar fronting was phonologized in a certain place and time but not in another place or time.

\section{{Word-initial} {position}}\label{sec:16.2}

Insight into the dating of the fronting of word-initial \ili{WGmc} \textsuperscript{+}[ɣ] can be adduced from those dialects in which that original velar is now an opaque palatal (Chapters~\ref{sec:7}--\ref{sec:8}). Recall that opaque palatals are underlying segments like /ç/ or /ʝ/ occurring in the context of a nonfront sound that was historically front ([coronal]). It was argued at length that opaque palatals were once palatal allophones of velars at the point in time before the original front trigger ([coronal]) was removed. Significantly, velar fronting must have been active before the elimination of the original [coronal] trigger. If the chronology of the latter change can be ascertained then it stands to reason that the dating of the originally allophonic rule of velar fronting can be inferred as well.\footnote{{The other etymological velar in word-initial position is \ili{WGmc}} \textrm{\textsuperscript{+}}\textrm{[k], including the} \textrm{\textsuperscript{+}}\textrm{[k] after a \isi{sibilant} in \ili{WGmc}} \textrm{\textsuperscript{+}}\textrm{[sk]. The evidence discussed below concerns the dating of the fronting of word-initial} \textrm{\textsuperscript{+}}\textrm{[ɣ], but that evidence cannot be extended to the fronting of the fortis velar. The only dialects discovered in which the modern reflex of \ili{WGmc}} \textrm{\textsuperscript{+}}\textrm{[k] is an opaque palatal at the left edge of a word are \il{East Pomeranian}EPo (Kreis \ipi{Konitz}; \sectref{sec:11.5}) and \il{High Prussian}HPr (\ipi{Reimerswalde}; \sectref{sec:11.6}). Since the sound changes responsible for creating the underlying palatal in those places were specific to those particular dialects, no evidence is available to my knowledge to determine the chronology of velar fronting.}}

A plethora of dialects was discussed earlier in which the reflex of \ili{WGmc} \textsuperscript{+}[ɣ] is palatal in word-initial position before front vowels or before \isi{schwa} in the \textit{ge}{}- prefix (e.g. [çə]/[ʝə]; cf. \il{Standard German}StG [gə]) but velar before full back vowels. Examples include \il{Eastphalian}Eph (\ipi{Dorste}, \sectref{sec:4.4}; \ipi{Eilsdorf}, \sectref{sec:8.3}; \ipi{Dingelstedt am Huy}, \sectref{sec:8.4}), \il{Westphalian}Wph (\ipi{Elspe} and \ipi{Schieder-Schwalenberg}, \sectref{sec:7.2}), as well as several LG and CG varieties spoken in the northeast of pre-1945 Germany in \chapref{sec:11}. In such dialects, the realization of an etymological velar as palatal before \isi{schwa} follows if that palatal was created by velar fronting when \isi{schwa} was still [i]. The chronology of the sound change producing \isi{schwa} from \isi{full vowels} like [i] (\isi{Vowel Reduction}) can be ascertained to a fair degree of accuracy on the basis of orthographic evidence.\footnote{{I do not discuss other (dialect-specific) sound changes that led to the development of opaque palatals in word-initial position because the dating of those changes is not as well-established, e.g. \isi{r-Deletion} in \ipi{Reinhausen} (\sectref{sec:7.2}).} } The assimilatory fronting of \ili{WGmc} \textsuperscript{+}[ɣ] before an etymological [i] in word-initial position is most prevalent in LG. However, since much more is known on the time frame for \isi{Vowel Reduction} in HG, I discuss first that evidence before I consider parallel data from LG.\largerpage[2]

The earliest attested stages of HG were \ili{OHG} (750–1050) and \ili{MHG} (1050–1350). In \ili{OHG} the prefix referred to above was rendered orthographically as \textit{ga}{}-, \textit{gi-}, and \textit{ge-}, whose vowels I interpret as [ɑ], [i], and [ə] respectively \citep[73--74]{Braune2004}. In general it can be said that \textit{ga}{}- was significantly more common in early \ili{OHG}, but that \textit{gi}{}- and then later \textit{ge}{}- established themselves. By the end of the ninth century, \textit{gi}{}- was the most common realization in all \ili{OHG} dialects, and in late \ili{OHG} \textit{ge}{}- had become more and more prevalent. By early \ili{MHG} \textit{ge}{}- was the sole realization \citep[108]{Paul2007}. The frequency of the three realizations of \textit{ga}{}-, \textit{gi-}, and  \textit{ge-} depended on the dialect of \ili{OHG}. For example, \textit{gi}{}- was first attested in CG (Franconian) dialects of \ili{OHG} at the beginning of the ninth century.\footnote{{The scholarly literature on the realization of the} \textrm{\textit{ge}}\textrm{{}- prefix in early Gmc is vast; some of those studies are cited in the standard reference grammar of \ili{OHG} \citep[73--74]{Braune2004}. I do not attempt to summarize those works here. It needs to be stressed that my treatment concerns itself only with the \ili{OHG} progenitor of modern-day} \textrm{\textit{ge}}\textrm{{}- in dialects referred to above where the original velar is now a palatal and the vowel is \isi{schwa}. The important point is that realizations like [ʝə] only make sense if the vowel of the prefix was once a trigger for velar fronting, e.g. [ɣi].}}

A similar development from [i] to [ə] in the \textit{ge}{}- prefix can be observed in the earliest stages of LG, namely \ili{OSax} (800--1150) and \ili{MLG} (1150--1600). Given the paucity of textual evidence from \ili{OSax}, not as much is known concerning the progression from \textit{gi}{}- to \textit{ge}{}- in specific \ili{OSax} dialects as in \ili{OHG} varieties (see \citealt{King1965} for some discussion on \ili{OSax}). The most significant generalization is that in \ili{OSax} the two most common realizations of the prefix in question were \textit{gi}{}- and \textit{ge}{}-, where the former was far more common than the latter \citep[42]{Holthausen1900}. The transition from \textit{gi}{}- to \textit{ge}{}- appears to have been completed by the onset of \ili{MLG} \citep[125]{Lasch1914}.

In light of the developments discussed above it can be concluded that the fronting of word-initial \ili{WGmc} \textsuperscript{+}[ɣ] began when the vowel of the \textit{ge}{}- prefix was still [i], meaning that velar fronting must have been phonologized before that [i] (/i/) was restructured to \isi{schwa} (/ə/). The conclusion is that the \isi{phonologization} of velar fronting in word-initial position began no later than late-\ili{OHG}/\ili{OSax} and that the opaque in word-initial position was in place by early \ili{MHG}/\ili{MLG}. It is difficult to establish a precise century during \ili{OHG}/\ili{OSax} when velar fronting was phonologized, although it can be said with a fair degree of certainty that velar fronting in word-initial position -- or after a sonorant (\sectref{sec:16.3}) -- was not inherited from \ili{WGmc} because one other \ili{WGmc} language (\ili{Dutch}) fails to have it.\footnote{{Sound changes resembling velar fronting occurred independently in other branches of Gmc (\ili{NGmc} and \ili{WGmc}). As I point out in Appendix~\ref{appendix:i} there are significant differences between those changes (\isi{Velar Palatalization}) and velar fronting in HG/LG; hence, it could not have been the case that velar fronting was inherited from \ili{WGmc}.}}

\tabref{tab:16:1} summarizes the status of the word-initial palatal deriving historically from \ili{WGmc} \textsuperscript{+}[ɣ] in HG (CG) and LG respectively. I assume here that the initial palatal in the prefix in the first column is a lenis fricative, although other palatal realizations are possible depending on the dialect, e.g. fortis fricative ([ç]) or stop ([c] or [ɟ] in \ipi{West Mecklenburg}, \ipi{Sebnitz}, and \ipi{Seifhennersdorf}; recall \chapref{sec:11}).\footnote{On occasion, one encounters statements in some of the literature which maintain that velar fronting in \il{Standard German}StG arose during or shortly after \ili{MHG}. An examination of those sources reveals that such claims are based solely on speculation. For example, \citet[107]{Penzl1975} asserts that [x] developed a palatal allophone after front vowels and sonorant consonants in Late \ili{MHG}, but he gives no evidence. \citet[63]{Cercignani1979} uncritically accepts Penzl’s claim, which is also adopted in textbooks \citep[288]{Schmidt2007}. \citet[85]{Russ1982} opines that velar fronting “… is probably not very old, since it does not exist in all German dialects ...”.}

\begin{table}
\caption{Chronology of word-initial velar fronting of \ili{WGmc} \textsuperscript{+}[ɣ] in CG (a) and LG (b)\label{tab:16:1}}
\begin{tabular}{llll}
\lsptoprule
   & Realization of \textit{ge}- & Time frame &   Status of palatal from WGmc \textsuperscript{+}[ɣ]:\\\midrule
a. & [ʝi] (from /ɣi/) & ca. 750--1000  & Allophone \\
   & [ʝə] (from /ʝə/) & after ca. 1000 & Palatal quasi-phoneme \\\tablevspace
b. & [ʝi] (from /ɣi/) & ca. 800--1150  & Allophone \\
   & [ʝə] (from /ʝə/) & after ca. 1200 & Palatal quasi-phoneme\\
\lspbottomrule
\end{tabular}
\end{table}

The palatal allophone referred to in \tabref{tab:16:1} ([ʝ]) and corresponding velar ([ɣ]) were in complementary distribution. This implies that the etymological glide (\ili{WGmc} \textsuperscript{+}[j]) was still realized as a glide and that \isi{Glide Hardening} (\sectref{sec:4.2}) had not yet been phonologized; recall the case studies discussed in \chapref{sec:8}. \citet{Hall2014a} discusses the chronology of \isi{Glide Hardening} in LG at length, since that change was an important component of the development from glide ([j]) to lenis velar fricative ([ɣ]) after short vowels, traditionally referred to as \isi{Schärfung}. Hall’s conclusion is that \isi{Glide Hardening} must have been active in \ili{OSax} because the change from [j] to [ɣ] was completed by the end of the \ili{OSax} era.

The chronology of the fronting of word-initial \ili{WGmc} \textsuperscript{+}[ɣ] as summarized in \tabref{tab:16:1}  only holds for those dialects listed above in which the modern reflex of that historical velar is palatal before front vowels or before \isi{schwa} but velar before full back vowels. In some dialects \ili{WGmc} \textsuperscript{+}[ɣ] is realized in word-initial position as palatal before front vowels and velar before all back vowels, including \isi{schwa} (e.g. \ipi{Soest}, \sectref{sec:4.3}). In a very common pattern exemplified primarily by CG dialects, \ili{WGmc} \textsuperscript{+}[ɣ] is realized as palatal in word-initial position before any sound (recall Stage 2e dialects discussed in \chapref{sec:14}). The \ipi{Soest} pattern was argued in \sectref{sec:7.4} to involve the same chronology as the one depicted in \tabref{tab:16:1}. On the basis of the \isi{rule generalization} model, the extension of velar fronting triggers to the broadest context (word-initially before all sounds) must have postdated the change from velar to palatal before all and only front vowels in \tabref{tab:16:1}.

A number of commentators have noted that there is strong orthographic evidence from earlier stages of German that \ili{WGmc} \textsuperscript{+}[ɣ] had a palatal variant before front vowels. That evidence is significant because it lends independent support to the chronology proposed above. I only present a brief overview of the philological facts here since they are discussed in much greater detail in \citet{VanderHoek2010} and references cited therein. The philological evidence is strongest in \ili{OSax}: In that language the letter used to represent \ili{WGmc} \textsuperscript{+}[j] was the same as the letter used for \ili{WGmc} \textsuperscript{+}[ɣ], e.g. \textit{giung} (<\ili{WGmc} \textsuperscript{+}[juŋg] ‘young’). Likewise when \ili{WGmc} \textsuperscript{+}[ɣ] appeared before \textit{i} or \textit{e}, it is not unusual to find a spelling \textit{i} or \textit{hi} (or zero), e.g. \textit{ieldan} (cf. \textit{geldan} ‘pay\textsc{{}-inf}’). Philological evidence for the fronting of word-initial \ili{WGmc} \textsuperscript{+}[ɣ] in \ili{OHG} is scanty, although those facts hold for northern (Franconian) dialects which were presumably the progenitor of the CG dialects referred to in \tabref{tab:16:1}(a). In those \ili{OHG} dialects the letter representing \ili{WGmc} \textsuperscript{+}[ɣ] is \textit{j} before front vowels, e.g. \il{Ripuarian}Rpn \textit{iechose}, which is apparently a proper name.

\section{{Postsonorant} {position}}\label{sec:16.3}

The dating of velar fronting in word-initial position was established on the basis of the chronology of a sound change creating opaque palatals (\isi{Vowel Reduction}). A number of changes were discussed earlier (\chapref{sec:7} and \chapref{sec:9}) that produced opaque palatals in postsonorant position, but unlike \isi{Vowel Reduction}, most of those changes were dialect-specific and not pan-German developments. Since no historical evidence is known to me on the dating of those sound changes (\isi{r-Retraction}, \isi{Vowel Retraction}, \isi{Syncope}), I do not discuss them and leave this issue open for further research provided data becomes available. One might hope that \isi{Vowel Reduction} could provide clues on the dating of postsonorant velar fronting, but only a very small number of dialects discussed earlier have opaque palatals created by that change. One example (\ipi{Wermelskirchen} in \sectref{sec:7.3}) is the word [iːvəç] (/iːvəç/) ‘eternal’ (cf.\ili{OHG} \textit{ēwīg} and \il{Standard German}StG [eːvɪç] with the unreduced front vowel [ɪ]). That item from \ipi{Wermelskirchen} suggests that velar fronting was phonologized before \isi{Vowel Reduction}, but \isi{Vowel Reduction} could have postdated \ili{OHG} in that type of word because it was specific to a particular CG dialect.

Two reliable linguistic arguments can be adduced for the dating of velar fronting in postsonorant position. The first of those arguments comes from the findings from \sectref{sec:14.7}: If velar fronting is attested in word-initial position then the same process is also present in postsonorant position in the same dialect for the same target segment. That implication is exceptionless in the present survey of German dialects. The reason for the absence of dialects with velar fronting in word-initial position but no fronting in postsonorant position was attributed to history: Velar fronting began in postsonorant position and then spread geographically to such a degree that an extension of velar fronting to word-initial position was only possible if that dialect already had postsonorant velar fronting.

The consequence of the findings from \sectref{sec:14.7} is that the fronting of \ili{WGmc} \textsuperscript{+}[ɣ] in postsonorant position must have already been phonologized in the dialects discussed in \sectref{sec:16.2} (summarized in \tabref{tab:16:1}) prior to the \isi{phonologization} of the fronting of \ili{WGmc} \textsuperscript{+}[ɣ] in word-initial position. Recall that the type of dialect referred to here was particularly prevalent in LG, i.e. \il{Eastphalian}Eph (\ipi{Dorste}, \sectref{sec:4.4}; \ipi{Eilsdorf}, \sectref{sec:8.3}; \ipi{Dingelstedt am Huy}, \sectref{sec:8.4}), \il{Westphalian}Wph (\ipi{Elspe} and \ipi{Schieder-Schwalenberg}, \sectref{sec:7.2}). The generous time frame for HG and LG in \tabref{tab:16:1} leaves plenty of breathing room for velar fronting in both contexts: Postsonorant fronting of \ili{WGmc} \textsuperscript{+}[ɣ] may have been phonologized at the beginning of the ninth century and then spread geographically over the next hundred years at which point the change was extended to word-initial position.

The second linguistic argument for establishing a time frame for the \isi{phonologization} of postsonorant velar fronting pertains to the fortis fricative [x]. Recall that postsonorant [x] has two main progenitors, namely \ili{WGmc} \textsuperscript{+}[x] (for HG and LG) and \ili{WGmc} \textsuperscript{+}[k] (for HG). The challenge in this case is clear: The velar and palatal reflexes of [x] are both spelled the same way in the earliest attested HG and LG branches (cf. \il{Standard German}StG \textit{ch} for [x] and [ç]). Hence, there is no philological evidence telling us when \textit{ch} first started being realized as palatal after front vowels. However, linguistic evidence can prove beneficial. Recall from \sectref{sec:2.3} and \sectref{sec:12.8.2} the following exceptionless implication:

\eanoraggedright%1
\label{ex:16:1}
      \textsc{\isi{Implicational Universal for Velar Fronting Targets-2}}:\\
      If a lenis sound undergoes velar fronting then the corresponding fortis sound does as well.
\z 

That implication accounts for the synchronic fact that there are dialects in which the targets for velar fronting are fortis (/x/) and lenis (/ɣ/) sounds (Target Type M/Stage 2bb), or fortis (/x/) but not lenis (Target Type L/Stage 2aa). Significantly, there is no dialect where a lenis velar (/ɣ/) undergoes fronting but the corresponding fortis sound (/x/) does not.

\REF{ex:16:1} can tell us something about when the fronting of postsonorant [x] was phonologized, although that evidence only holds for certain dialects. Consider the many Target Type M dialects referred to in \chapref{sec:12} which have no velar fronting in word-initial position. In that type of system it can be concluded that postsonorant velar fronting was phonologized first with the /x/ target and that the change only later extended to /ɣ/. However, no conclusions can be drawn concerning when the postsonorant fronting of /x/ was phonologized. More revealing are Stage 2bb dialects with velar fronting in word-initial position. Representative examples were mentioned above, namely LG, i.e. \il{Eastphalian}Eph (\ipi{Dorste}, \ipi{Eilsdorf}, \ipi{Dingelstedt am Huy}) and \il{Westphalian}Wph (\ipi{Elspe}, \ipi{Schieder-Schwalenberg}), as well as LG and CG varieties spoken in the northeast of pre-1945 Germany. In those places it can be deduced that word-initial \ili{WGmc} \textsuperscript{+}[ɣ] was fronted allophonically during \ili{OHG}/\ili{OSax} and that the fronting of /ɣ/ (from \ili{WGmc} \textsuperscript{+}[ɣ]) in postsonorant position had been phonologized before then. Significantly, it can also be concluded on the basis of \REF{ex:16:1} that the fronting of postsonorant [x] (< \ili{WGmc} \textsuperscript{+}[x]/\textsuperscript{+}[k]) had been phonologized even before the fronting of postsonorant [ɣ] (< \ili{WGmc} \textsuperscript{+}[ɣ]). 

The conclusions concerning the general time frame for velar fronting are summarized in \REF{ex:16:2}:

\eanoraggedright\label{ex:16:2}
\eanoraggedright\label{ex:16:2a} In many LG (and some CG) varieties, the fronting of \ili{WGmc} \textsuperscript{+}[ɣ] must have been phonologized in word-initial position in \ili{OHG}/\ili{OSax}; see \tabref{tab:16:1}. In those same dialects, the allophonic palatal had become an underlying (opaque) palatal by early \ili{MHG}/\ili{MLG};
\ex\label{ex:16:2b} In the dialects referred to in \REF{ex:16:2a}, the fronting of /ɣ/ (< \ili{WGmc} \textsuperscript{+}[ɣ]) must have been phonologized in postsonorant position even before it was phonologized in word-initial position;
\ex\label{ex:16:2c} In the dialects referred to in \REF{ex:16:2b} the fronting of /x/ (< \ili{WGmc} \textsuperscript{+}[x]/\textsuperscript{+}[k]) must have been phonologized in postsonorant position even before the fronting of  /ɣ/ (< \ili{WGmc} \textsuperscript{+}[ɣ]) in postsonorant position.
\z 
\z 

\section{{Remarks} {on} {geography}}\label{sec:16.4}

It was noted in \chapref{sec:12} and \chapref{sec:14} that the areal distribution for the various velar fronting patterns appears somewhat haphazard and does not always give a clear indication of whether or not there are (or were) isoglosses separating the postulated historical stages. Nevertheless, the material on German dialects discussed in previous chapters does give some clues concerning the relative age of velar fronting in certain areas with respect to others.

One point needs to be stressed at the outset: The presence of \isi{velar fronting islands} only makes sense if velar fronting had more than place of origin. Few definitive conclusions can be reached on the \isi{focal area}(s) for velar fronting in Germany and Austria. On the one hand, one could adopt \isi{monogenesis} and claim that there was only one original place where velar fronting was phonologized. On the other hand, since \isi{velar fronting islands} are well-attested in Switzerland/\ipi{Tyrol} among other places (\chapref{sec:15}), there is no principled reason why \isi{polygenesis} could not be correct for Germany.

That point aside, there is agreement in the literature that  sound change begins in a \isi{focal area} and then spreads both temporally and geographically from that point of origin (\sectref{sec:2.4.1} and \sectref{sec:16.6} below). Spreading can involve more than one factor, but the two that are most significant for velar fronting are the triggers and/or targets, which gradually expand in the \isi{focal area} to include more and more segments. The original change in the \isi{focal area} also spreads geographically in the sense that outlying areas adopt it. Significantly, the change is active the longest in the \isi{focal area}, and it is there where it reaches its most general form in terms of the number of triggers/targets. However, in some of the outermost areas the change never progresses to the more general contexts in the \isi{focal area}. The important point is that the \isi{focal area} for velar fronting -- the place where that process is phonologized -- is that place where the set of triggers/targets is most general.\largerpage

On the basis of the various velar fronting patterns discussed in previous chapters, one generalization is that velar fronting must be quite old in CG varieties of \ili{OHG} but much more recent in LG (\ili{OSax}). \mapref{map:42} indicates the major dialect areas of \ili{OHG} and \ili{OSax} referred to here. Consider the following four pieces of evidence.\footnote{{A fifth difference between the two dialects is \isi{alveolopalatalization}, which is well-attested in CG but not in LG. I do not consider \isi{alveolopalatalization} because that change began much later than \ili{OHG}/\ili{OSax}; recall the discussion in \sectref{sec:10.6.1}.}}

\begin{enumerate}[label=(\Alph*)]
\item In postsonorant position those CG/LG varieties with a narrow set of triggers are not common. By contrast, WCG (\il{Ripuarian}Rpn, \il{Moselle Franconian}MFr) has the broad set of triggers (coronal sonorants) without exception (\tabref{tab:12.9}). The narrow triggers in those LG (\il{Westphalian}Wph) places and the broad triggers in WCG (\il{Ripuarian}Rpn, \il{Moselle Franconian}MFr) make sense if velar fronting in postsonorant position was present longer in WCG (\il{Ripuarian}Rpn, \il{Moselle Franconian}MFr) than in LG (\il{Westphalian}Wph).
\item In postsonorant position, LG (\il{Westphalian}Wph) dialects have a strong preference for a narrow set of targets (/x/ but not /ɣ/; Stage 2aa=Target Type L), but the more inclusive set of targets (/x/ and /ɣ/; Stage 2bb=Target Type M) are more prevalent in WCG (\il{Ripuarian}Rpn, \il{Moselle Franconian}MFr). In fact, no Stage 2aa dialect was found among \il{Ripuarian}Rpn/\il{Moselle Franconian}MFr dialects in the survey given in \chapref{sec:12}. Compare \tabref{tab:12.15} for \il{Westphalian}Wph with \tabref{tab:12.9} for \il{Ripuarian}Rpn/\il{Moselle Franconian}MFr.
\item In word-initial position, LG (\il{Westphalian}Wph, \il{Eastphalian}Eph) dialects are well-attested with a narrow set of triggers (\tabref{tab:12.16} for \il{Westphalian}Wph and \tabref{tab:12.18} for \il{Eastphalian}Eph). By contrast, velar fronting in many WCG varieties (e.g. \il{Ripuarian}Rpn) exhibits the broadest possible set of triggers (Stage 2e); see \tabref{tab:14:2}. Recall that Stage 2e is the change from velar to palatal as a nonassimilatory change. The dichotomy between broad vs. narrow triggers in word-initial position suggests that velar fronting has been present longer in WCG (\il{Ripuarian}Rpn) and is of more recent origin in LG (\il{Westphalian}Wph, \il{Eastphalian}Eph).
\item Within the \il{Westphalian}Wph dialect continuum certain communities can be identified in relatively close proximity which represent the incremental assimilatory stages for velar fronting triggers in word-initial and postsonorant position; recall the discussion of \il{Westphalian}Wph in \sectref{sec:12.5.2}. The significance of those \il{Westphalian}Wph places is that they point to a region where velar fronting was phonologized relatively late (in contrast to other areas). In particular, the \isi{focal area} for the change was not in the \il{Westphalian}Wph region. Instead, that change was phonologized elsewhere and then spread geographically from that \isi{focal area} to the \il{Westphalian}Wph region thereby leaving relics in the modern era.
\end{enumerate}

\begin{map}
\includegraphics[width=\textwidth]{figures/Map42_16.1.pdf}
\caption[Continental \ili{West Germanic} languages (ca. ninth century)]{Continental West Germanic languages (ca. ninth century). Adapted from \citet[209]{MeinekeSchwerdt2001}.}\label{map:42}
\end{map}

On the basis of (A)-(C) it can be concluded that velar fronting has been active for a long time in many varieties of WCG, but the status of velar fronting in UG varieties of \ili{OHG} is not as clear. The shift of \ili{WGmc} \textsuperscript{+}[ɣ] to palatal in word-initial position did not occur in UG because that original fricative was restructured to [g] (/g/), which was not a velar fronting target. And since the change from \ili{WGmc} \textsuperscript{+}[ɣ] to [g] also occurred in postsonorant position in UG, it is difficult to find UG varieties in which the target for postsonorant velar fronting is anything other than /x/; hence, UG is not one of the dialects referred to in \REF{ex:16:2}. No conclusions at all can be drawn concerning when /x/ first developed a palatal allophone in UG, although the data discussed for Lower Bavaria from SNiB points to a fairly recent date (\sectref{sec:13.4}).\footnote{\textrm{Conclusions concerning the status of NLG are also tentative. In contrast to \il{Westphalian}Wph/\il{Eastphalian}Eph, NLG exhibits the broadest possible set of triggers for assimilatory fronting in postsonorant position (\tabref{tab:12.14}). That table also reveals that the prevalent pattern for NLG is that velar fronting has a broad set of targets (Stage 2bb=Target Type M). It is conceivable that there was also a \isi{focal area} for velar fronting for NLG varieties of \ili{OSax}, but since this topic is purely speculative I do not pursue it further.}}

\section{{Directionality} {revisited}}\label{sec:16.5}

The typological literature on \isi{Velar Palatalization} makes extensive reference to a \isi{directionality} parameter (\sectref{sec:2.3.5}). Recall that \isi{directionality} is not an issue for velar fronting in German dialects because postsonorant velar fronting always applies from left-to-right (progressively), cf. \il{Standard German}StG [kuːxən] ‘cake’ vs. [kʏçə] ‘kitchen’. Since the vowel to the right of the target is \isi{schwa}, speakers do not have the option of applying velar fronting regressively. However, in the \isi{velar fronting island} of \ipi{Visperterminen} (\sectref{sec:6.2}) the rule creating \isi{schwa} (\isi{Vowel Reduction}) never occurred; hence, there are many native words (or assimilated loanwords) in which a potential velar fronting trigger is to the right of a target (e.g. [xuxxi] ‘kitchen’). The conclusion is that when velar fronting was phonologized in \ipi{Visperterminen} speakers had a choice between two directions and that -- for whatever reason -- they opted for the one direction and not the other.

\ipi{Visperterminen} is not unique. Since \isi{Vowel Reduction} affected any unstressed \isi{full vowel}, there must have been many words in \ili{OHG}/\ili{OSax} with a velar fronting target (/x/) situated between a (stressed) back vowel and an (unstressed) front vowel, i.e. words containing sequences like [ɑxi], [ɑxe] etc. If so, could velar fronting have been phonologized as a regressive assimilation in some dialects of \ili{OHG}/\ili{OSax}? In the remainder of this section I argue that the answer must have been negative and that when velar fronting was phonologized it applied progressively in every variety.

To illustrate this point, consider the items from \ili{OHG} in the first column of \REF{ex:16:3}. These \ili{OHG} examples consist of a partial paradigm for a verb in (\ref{ex:16:3}a), a noun in (\ref{ex:16:3}b), and an adjective in (\ref{ex:16:3}c). The verb, noun, and adjective presented here have in common that the bare stem ends in [x] and that at least one of the inflectional suffixes begin with a front vowel.

\ea%3
\TabPositions{.25\textwidth, .45\textwidth, .6\textwidth, .7\textwidth}
    \label{ex:16:3}
\begin{xlist}
\sn{}    OHG \tab Stage A \tab Stage B\\
\ex suochu [suoxu] \tab [suoxu] \tab [zuːxə] \tab ‘seek\textsc{{}-1sg}’\\
    suochis [suoxis] \tab [suoçis] \tab [zuːçst] \tab  ‘seek\textsc{{}-2sg}’\\
    suochit [suoxit] \tab [suoçit] \tab [zuːçt] \tab ‘seek\textsc{{}-3sg}’\\
\ex bah [bɑx] \tab [bɑx] \tab [bɑx] \tab ‘stream’\\
    bahes [bɑxes] \tab [bɑçes] \tab [bɑçəs] \tab ‘stream\textsc{{}-gen.sg}’
\ex hōh [hoːx] \tab [hoːx] \tab [hoːx] \tab ‘high’\\
    hōhēr [hoːxeːr] \tab [hoːçeːr] \tab [hoːçɐ] \tab ‘high\textsc{{}-masc.sg}’\\
    hōhiu [hoːxiu] \tab [hoːçiu] \tab [hoːçə] \tab ‘high\textsc{{}-fem.sg}’
\end{xlist}
\z 

The interesting examples are the ones in the first column with front vowels in the suffix. If there had been \ili{OHG} dialects in which velar fronting was phonologized as a rule spreading the frontness feature from right-to-left (regressively), then those early dialects must have been realized phonetically as in the Stage A column above. If the inflectional suffixes underwent the same changes as in \il{Standard German}StG (\isi{Vowel Reduction}, \isi{Syncope}, \isi{r-Vocalization}), then Stage A could have conceivably developed into Stage B.

No dialect in the present survey has anything resembling Stage B. Although that hypothetical dataset has a contrast between [x] and [ç] after a back vowel, Stage B is nothing at all like the dialects discussed in \chapref{sec:9} with phonemic palatals. The focus of that chapter was on dialects with a contrast between [x] and [ç] after a back vowel, where the back vowel \textit{before} [ç] was historically front, e.g. minimal pairs in \ipi{Wissenbach} (\sectref{sec:9.2}) like [dɑx] ‘roof’ (cf. \il{Standard German}StG \textit{Dach}) vs. [dɑç] ‘dike’ (cf. \il{Standard German}StG \textit{Deich}). Recall that velar fronting is still active in dialects like \ipi{Wissenbach} to account for regular [x]{\textasciitilde}[ç] alternations in morphological paradigms. But Stage B in \REF{ex:16:3} represents an entirely different type of system than the one discussed in \chapref{sec:9} because it contains many stems with [x]{\textasciitilde}[ç] alternations after back vowels. From the synchronic perspective Stage B requires a rule fronting /x/ to palatal in morphologically-defined contexts, e.g. in the second and third person singular of verbs or in the genitive singular of nouns. In the typological literature on \isi{Velar Palatalization} referred to throughout the present book, the observation has been made that Palatalization rules can apply in some languages in such morphological contexts. Thus, from the cross-linguistic perspective, Stage B in \REF{ex:16:3} might be conceivable. However, no dialect investigated in the present book exhibits that pattern.

More than one explanation for the lack of Stage B dialects is possible. Here are two: (a) Stage B is no longer attested in modern dialects, although it was present at an earlier stage. The cells in the Stage B paradigms with [ç] underwent a later analogical change to [x], thereby producing the pattern found in \il{Standard German}StG, e.g. [zuːçst], [zuːçt] > [zuːxst], [zuːxt]. (b) Stage B is not attested in modern dialects, nor was it ever attested at any earlier stage. The reason for that gap is that velar fronting was phonologized consistently as a progressive assimilation in all German dialects without exception.

Explanation (a) relies on the assumption that there was an analogical change of [ç] to [x], but it cannot account for the fact that every Stage B dialect changed into the familiar \il{Standard German}StG-type pattern without exception and that there are no relics preserving that Stage B system. Although \isi{analogy} has undeniably played an important role in the history of German, explanation (a) also cannot account for the fact that stem allomorphy is quite persistent among verbs and nouns in \il{Standard German}StG as well as German dialects. Thus, explanation (a) begs the question of why [ç] would change the deviant [x] to eliminate stem allomorphy when stem allomorphy is elsewhere so robustly attested?

I contend that the only conceivable reason for the lack of Type B systems is (b). The generalization from \sectref{sec:2.3.5} is repeated in \REF{ex:16:4}:

\eanoraggedright%4
\label{ex:16:4}Directionality of Velar Fronting: If a target for velar fronting is situated after a sonorant and before a vowel then the trigger for velar fronting is always the sonorant to the immediate left of that velar sound.
\z 

\REF{ex:16:4} is admittedly little more than a statement of what is true, but it explains nothing. Put differently, why is it that German dialects described from 1860 to the present in Germany (including the pre-1945 borders) exhibit variation for targets and triggers as well as limited variation concerning outputs, but no variation at all with respect to \isi{directionality}? One could argue that \REF{ex:16:4} makes sense if velar fronting were phonologized only once (\isi{monogenesis}), in which case the progressive direction was simply inherited when than original rule spread outwards from the original \isi{focal area}. However appealing that explanation might be, it cannot account for \isi{velar fronting islands}, which phonologized velar fronting with slightly different triggers but with the same progressive direction. Regrettably, the proper explanation for \REF{ex:16:4} cannot be offered.\footnote{{One might attempt to argue that velar fronting was phonologized consistently in the left-to-right direction because other rules active in German dialects at that time also involved the progressive spreading of a feature. This is an appealing idea; however, it is counterexemplified by the most well-known rule of \ili{OHG}/\ili{OSax}, namely \isi{i-Umlaut}, which spreads the features of frontness and height from /i/ to the left, e.g. OHG [gɑst] ‘guest’ vs. [gesti] ‘guest-\textsc{pl}’. A brief glance at the sound changes for \ili{OHG} in \citet{Braune2004} does not reveal any clear candidates for regular progressive spreadings.}}

\section{{The} {historical} {model}}\label{sec:16.6}

In \sectref{sec:2.5} I posited a historical model (\figref{fig:2.3}), and in Chapters~\ref{sec:3}--\ref{sec:14} I demonstrated in a series of detailed case studies how those data fit into the various stages proposed in the model. The point of this section is to provide a brief summary of the most important patterns involving velar fronting and to demonstrate how they exemplify the model I have proposed.

On the basis of the patterning of velar fronting in HG/LG dialects much can be inferred about the nature of Stage 2 and Stage 3. As noted earlier, Stage 1 has not been taken into consideration because the original sources for velar fronting do not provide the necessary data (e.g. the degree to which [x] is gradiently fronted based on the nature of the adjacent \isi{vocoid}). I make first a few speculative remarks on the nature of Stage 1, especially in light of the claims I advanced in the earlier part of this chapter on the time frame for the \isi{phonologization} of velar fronting. The bulk of this section is devoted to a discussion of Stage 2 and Stage 3.

\subsection{Stage 1}
  This is the point at which the phonological rule of velar fronting is absent. Stage 1 is therefore represented by any language where velar sounds do not undergo a categorical fronting in the context of front sounds.

  The nature of Stage 1 needs to be further refined in light of the findings presented in the preceding chapters. First, velar fronting involves left-to-right (progressive) spreading in every HG/LG variety that has that process (\sectref{sec:16.5}). Second, when velar fronting phonologizes at Stage 2, the target for that change is the fortis fricative /x/, and the triggers are high front vowels like /i/. These three properties together mean that Stage 1 in the context of the present book can be defined specifically as any dialect/language which possesses sequences like /ix/ ([ix]), where the velar does not undergo categorical fronting. An example of a modern Gmc language that can be classified as Stage 1 is \ili{Dutch}.

  As noted earlier, it is common for velars to be articulated in a slightly more forward position along the palate in the neighborhood of front vowels than in the neighborhood of back vowels. However, this is the coarticulatory (phonetic) fronting of velars and not the categorical change characterized by velar fronting. It has been stressed throughout this book that velar fronting is phonological and not phonetic; hence, the Stage 1 languages under discussion may have the coarticulatory fronting of /x/ after /i/.\footnote{{Recall from \sectref{sec:12.9.1} that several sources for UG varieties suggest that there is coarticulatory fronting of velars like /x/ in the context after /i/.}}

  In the preceding chapters I have documented a number of Stage 1 LG\slash HG varieties. Many of those lects are located along the \ili{Dutch} border, but a surprising finding in the present book is that \isi{non-velar fronting islands} are attested as well, e.g. Kreis \ipi{Stolp} (\mapref{map:18}).

  I claim that there was an earlier point in the history of Gmc (Stage 1) when the phonological rule of velar fronting was not present. Since velar fronting was phonologized at an early stage (\ili{OHG}/\ili{OSax}), I conjecture that Stage 1 was represented by the \ili{WGmc} language.

\subsection{Stage 2} 
  I hypothesize that the earliest stages of \ili{OHG}/\ili{OSax} were characterized by Stage 1 coarticulatory fronting of /x/ in the context after /i/. At Stage 2 (also \ili{OHG}/\ili{OSax}) that \isi{gradient} phonetic process was phonologized. Put differently, at Stage 2 the difference between phonetically fronted /x/ in the context after /i/ and back /x/ in the context of vowels like /u/ at Stage 1 was exaggerated to the point where speakers perceived of the two articulations as different sounds: Palatal [ç] and velar [x]. The \isi{phonologization} of velar fronting occurred sometime during the time frame discussed earlier in this chapter for \ili{OHG}/\ili{OSax}.

  Stage 2 was characterized by the reinterpretation by the younger generation of the \isi{gradient} coarticulatory fronting from Stage 1 of the older generation as a categorical process relating two distinct articulations. Thus, the change from Stage 1 to Stage 2 was intergenerational.

  Since velars and palatals did not contrast at Stage 2, those segments stood in an allophonic relationship: [ç] and [x] were associated with one phoneme (/x/), whose realization as palatal was expressed formally with a specific version of velar fronting. That rule spread the feature [coronal] from a high front vowel to a following /i/, thereby producing [ç]. Hence, \isi{phonologization} (Stage 2) involved the addition of a phonological rule into the Phonology component depicted in \tabref{tab:fromfig:representationallevels}. Once in the grammar that synchronic process remained active until it was modified in light of the various changes involving triggers and targets discussed below.

  The change from Stage 1 to Stage 2 is depicted in \tabref{tab:16:2}. Stage 2 is given as Stage 2a because it was defined in terms of a narrow set of triggers, as described below. I give sample underlying and phonetic representations in the second column. Note that the underlying representations for Stage 1 are acquired without change by the following generations of speakers (Stage 2a).\footnote{{In the case studies discussed above a total of fourteen versions for postsonorant velar fronting are posited (Appendix~\ref{appendix:d}). The set of narrow triggers at Stage 2a in \tabref{tab:16:2} therefore suggests that the correct version of velar fronting is Vel-Fr-6, discussed in \sectref{sec:6.2.2}. In the remainder of this section I continue to discuss the expansion of triggers in the \isi{rule generalization} model in terms of segments (/i/, /e/ etc.), but these generalizations can easily be translated into one of the formal rules posited earlier.}}

\begin{table}
\caption{Change from Stage 1 to Stage 2a\label{tab:16:2}}
\fittable{
\begin{tabular}{lll}
\lsptoprule
Stage & Underlying and phonetic representations & Triggers for velar fronting\\\midrule
 1 & /ix/ [ix], /ex/ [ex] /ɑx/ [ɑx] &  No rule\\
2a & /ix/ [iç], /ex/ [ex] /ɑx/ [ɑx] &  /i/\\
\lspbottomrule
\end{tabular}
}
\end{table}

In the intergenerational, listener-based approach described in \sectref{sec:2.5}, Stage 1 and Stage 2a represent a speaker and a listener respectively. The former utters a word containing [ix] -- where the fricative shows the effects of coarticulatory fronting ([x̟]), -- but the listener acquiring the language misperceives that prevelar as palatal. The change from Stage 1 (speaker) to Stage 2a (listener) involves not only the emergence of a new pronunciation ([ç]), but also the interpretation of that new sound as a phonological unit. The listener does this by relating the new palatal ([ç]) with the other dorsal fricative ([x]) as allophones, whose distribution is expressed with the newly acquired rule of velar fronting.

As stressed throughout this book, velar fronting was phonologized in more than one place. A moment’s reflection reveals that this scenario also implies a temporal dimension. Imagine the younger generation of speakers in a particular place (P\textsubscript{1}) acquiring the rule of velar fronting at Stage 2a at a particular point in time (T\textsubscript{1}). If \isi{phonologization} happens in a different place (P\textsubscript{2}), and if there is no contact between speakers of P\textsubscript{1} and P\textsubscript{2} because they are separated by hundreds of kilometers then it is unlikely that velar fronting in P\textsubscript{2} is also phonologized precisely at T\textsubscript{1}. What this suggests is that the \isi{phonologization} of velar fronting began in certain places during \ili{OHG}/\ili{OSax} but that the process of \isi{phonologization} in other places (P\textsubscript{2}, P\textsubscript{2} …) must have continued on into the future as well (e.g. \ili{MHG}/\ili{MLG}).

I describe now how \isi{rule generalization} occurred with the rule of velar fronting as it was originally phonologized (\tabref{tab:16:2}).

When velar fronting made the jump from Stage 1 to Stage 2a it affected only a single velar segment (/x/), it was triggered by a narrow set of triggers (/i/), and the output was palatal ([ç]). What is more, velar fronting was phonologized as a progressive assimilation meaning that the trigger was to the immediate left of the target.

The gradual expansion of targets and triggers is depicted abstractly for targets and triggers in \figref{fig:2.2}. In \figref{fig:16:1} I modify \figref{fig:2.2} in order to show how the set of triggers expanded in time and space for velar fronting. The three Trigger Types depicted here were defined in \tabref{tab:14:1}. Recall from that table that there are a number of other stages which correspond to expanded sets of triggers. I focus here only on three stages indicated below, although the same principles hold for the additional stages.

\begin{figure}
  \begin{tikzpicture}
  \matrix (matrix) [nodes={minimum width=1cm, minimum height=1cm}]
    {
      \node{A}; &[20mm] & \node{B}; & \\
      \node[draw, fill=black] {}; & \node[draw, fill=black] {}; & \node[overlay, draw, fill=black!50, yshift=-5mm] {}; & \node[draw] {};\\
      \node[draw, fill=black!50] {}; & \node[draw, fill=black!50] {}; &  \node[overlay, draw, yshift=-5mm] {};& \\
      \node[draw] {}; & \node[draw] {}; & & \\[7mm]
      \node (P1a) {P\textsubscript{1}}; & \node (P1b) {P\textsubscript{1}}; & \node (P2) {P\textsubscript{2}}; & \node (P3) {P\textsubscript{3}};\\
    };
      \draw[-{Triangle[]}] (P1a.north) -- ++(0,5mm);
      \draw[-{Triangle[]}] (P1b.north) -- ++(0,5mm);
      \draw[-{Triangle[]}] (P2.north)  -- ++(0,5mm);
      \draw[-{Triangle[]}] (P3.north)  -- ++(0,5mm);
      \draw[-{Triangle[]}] (matrix.south west) -- (matrix.south east) node [midway, below] {space};
      \draw[-{Triangle[]}] (matrix.south west) -- (matrix.north west) node [midway, sloped, above] {time};
  \matrix (legend) [right=.5cm of matrix, nodes={minimum width=5mm, minimum height=5mm}]
    {
                      & \node{Stage}; & \node{Velar fronting of /x/};\\
      \node[draw, fill=black] {}; & \node {2c'}; & \node{after /i/, /e/, /æ/};\\
      \node[draw, fill=black!50] {}; & \node {2b}; & \node {after /i/, /e/};\\
      \node[draw] {}; & \node {2a}; & \node{after /i/};\\
    }; 
  \end{tikzpicture}
\caption{Rule generalization in time and space for velar fronting with three Trigger Types\label{fig:16:1}}
\end{figure}

In \figref{fig:16:1} I compare three contexts for velar fronting, namely after all high front vowels (represented by /i/), after all nonlow front vowels (represented by /i/, /e/), and after all front vowels (represented by /i/, /e/, /æ/). Low front vowels (/æ æː/) were phonemicized by the onset of \ili{MHG}/\ili{MLG} (ca. 1050). Recall that many modern LG/HG dialects possess at least one low front vowel.

Consider first column A, which illustrates how velar fronting (Vel Fr) spread temporally: Phonologization occurred in a particular place (P\textsubscript{1}) for the target (/x/) and the narrow trigger (/i/). Stage 2a is depicted with the white square. At some later point in time (Stage 2b), Vel Fr generalized in P\textsubscript{1} to include all high front and mid front vowels (/i/, /e/), which is depicted with the gray square. Next, Vel Fr was generalized in P\textsubscript{1} further at a later period in time (Stage 2c') by applying after all front vowels (/i/, /e/, /æ/). This point is illustrated with the black square.

The \isi{rule generalization} model means that varieties of HG/LG where Vel Fr applies after all front vowels were preceded by a stage in which the triggers were nonlow front vowels, which was preceded by a stage when the triggers were the high front vowels. It is not possible to provide direct evidence for this type of temporal spread because it would require a description of a dialect spoken at a particular place (P) at a particular time (T) and another description of the same variety spoken in P at a time earlier or later than T. Although that type of direct evidence is lacking, there is indirect evidence for the progression of triggers as depicted in \figref{fig:16:1}. See in particular \chapref{sec:13} and \mapref{map:28}, which document places throughout Lower Bavaria which represent the three historical stages depicted in \figref{fig:16:1}.

According to the \isi{rule generalization} model the addition of triggers and targets proceeds not only temporally (column A of \figref{fig:16:1}), but also in terms of space (column B). As shown under column B, Vel Fr was phonologized in P\textsubscript{1} for the target /x/ and the trigger /i/, defined as Stage 2a and depicted with a white square, and at a later point in time Vel Fr generalized its triggers to attain Stage 2b in P\textsubscript{1} (gray square). At some point when Vel Fr was active at Stage 2a in P\textsubscript{1}, Vel Fr also spread geographically to P\textsubscript{2}. When Vel Fr was phonologized in P\textsubscript{2} its triggers were defined narrowly as Stage 2a (white square). At the top of column B it can be seen that Vel Fr was generalized further in P\textsubscript{1} to attain Stage 2c' (black square) and that Vel Fr also spread temporally to P\textsubscript{2} by attaining the targets and/or triggers representing Stage 2b (gray square). At some point Vel Fr was then phonologized with the narrow set of triggers (white square) in a third place (P\textsubscript{3}).

The gradual increase in the number of triggers meant that each stage resulted in a modification of the rule of velar fronting that was active for that synchronic stage. When a new stage was attained, the younger generation reanalyzed the earlier rule by generalizing the number of triggers. For example, speakers at Stage 2a had underlying and phonetic representations like the ones in \tabref{tab:16:3} with a rule of velar fronting applying only after high front vowels. The younger generation (Stage 2b) inherited the same underlying generations, but then extended the rule so that it applied after all and only nonlow front vowels. The next generation (Stage 2c') consequently inherited the same underlying representations from Stage 2b, but then generalized the context of the rule (after all front vowels) and therefore the /x/ in sequences like /ix/, /ex/, /æx/ (but not the /x/ in /ɑx/) was realized as [ç].

\begin{table}
\caption{Change from Stage 2a to Stage 2b to Stage 2c'\label{tab:16:3}}
\fittable{
\begin{tabular}{lll}
\lsptoprule
Stage & Underlying and phonetic representations  & Triggers for velar fronting\\\midrule
2a & /ix/ [iç], /ex/ [ex], /æx/ [æx], /ɑx/ [ɑx]  & /i/\\
2b & /ix/ [iç], /ex/ [eç], /æx/ [æx], /ɑx/ [ɑx]  & /i/, /e/\\
2c' & /ix/ [iç], /ex/ [eç], /æx/ [æç], /ɑx/ [ɑx]  & /i/, /e/, /æ/\\
\lspbottomrule
\end{tabular}
}
\end{table}

Each of the three stages in \tabref{tab:16:3} represents a slightly different synchronic system. That point is expressed in the final column, which lists the triggers that need to be expressed formally in the synchronic rule of velar fronting for that stage. For example, the Stage 2a rule spreads [coronal] from a [+high] segment to /x/, but the next generation of speakers who expand the set of targets to the one for Stage 2b have a rule spreading [coronal] from a [--low] sound to /x/. The next generation of speakers then acquires a rule spreading [coronal] from all front vowels to /x/.

\figref{fig:16:1} depicts the expansion of triggers for postsonorant velar fronting with /x/ as the sole target segment. Velar fronting also involved a gradual expansion of target segments. Thus, the first velar to serve as target was /x/, the second was /ɣ/, and the third was the set of noncontinuants (/k g ŋ/). \tabref{tab:16:4} lists underlying and phonetic representations for sequences consisting of a high front vowel (/i/) followed by the fortis velar fricative (/x/), the corresponding lenis (/ɣ/) and the three velar noncontinuants (/k/, /g/, /ŋ/). It can be seen in the second column below that velar fronting is phonologized at Stage 2aa because that is the stage in which /x/ is the sole target segment. At Stage 2bb the target consists of all and only velar fricatives, and at Stage 2cc of all velar consonants.

\begin{table}
\caption{Change from Stage 2aa to Stage 2bb to Stage 2cc\label{tab:16:4}}
\fittable{\begin{tabular}{lll}
\lsptoprule
 Stage & Underlying and phonetic representations & Targets for velar fronting\\\midrule
2aa & /ix/ [iç], /iɣ/ [iɣ], /ik/ [ik], /ig/ [ig], /iŋ/ [iŋ] & /x/\\
2bb & /ix/ [iç], /iɣ/ [iʝ], /ik/ [ik], /ig/ [ig], /iŋ/ [iŋ] & /x/, /ɣ/\\
2cc & /ix/ [iç], /iɣ/ [iʝ], /ik/ [ic], /ig/ [iɉ], /iŋ/ [iɲ] & /x/, /ɣ/, /k/, /g/, /ŋ/\\
\lspbottomrule
\end{tabular}}
\end{table}

The set of target segments for the individual stages is expressed formally with features in the various versions of velar fronting. For example, for speakers representing Stage 2aa velar fronting spreads [coronal] to [+consonantal, --sonorant, +continuant, +fortis, dorsal], but the next generation extends the targets at Stage 2bb to [+consonantal, --sonorant, +continuant, dorsal] and then the later generation at Stage 2cc to [+consonantal, --sonorant, dorsal].

The spread from /x/ to additional target sounds as depicted in \tabref{tab:16:4} proceeded temporally as well as spatially. Evidence for these three stages comes from HG/LG dialects: Many varieties are attested in which /x/ is the sole trigger, but a number of varieties are attested in the same general areas where the targets are broader (\mapref{map:23}). The broadest set of targets (Stage 2cc) is attested in a small number of dialects spoken in the eastern areas of pre-1945 Germany (\mapref{map:18}).

I have described how the \isi{rule generalization} model can be applied to the triggers and targets for (postsonorant) velar fronting, but it needs to be stressed that the spread from a narrow to broad set of triggers (\tabref{tab:16:3}) and the spread from a narrow to a broad set of targets (\tabref{tab:16:4}) did not always match up. Put differently, when \isi{phonologization} occurs, Stage 2a for triggers goes hand in hand with Stage 2aa for targets, but some dialects extend the set of triggers at a faster rate than the set of targets. This point accounts for the fact that many varieties of HG/LG are attested with the narrowest set of targets (/x/) but with the broadest set of triggers (coronal sonorants); see \chapref{sec:12} for examples.

Earlier on in this chapter I discussed the connection between postsonorant velar fronting and word-initial velar fronting. The conclusion (\sectref{sec:16.3}) is that the former must have preceded the latter. Thus, the \isi{phonologization} of velar fronting with /x/ as the target and front vowels like /i/ as the triggers and the gradual increase in the number of triggers occurred before word-initial velars succumbed to \isi{phonologization}.

\begin{sloppypar}
The word-initial velar which served as the target for velar fronting went through the same stages for triggers and targets as depicted above for postsonorant position. \tabref{tab:16:5} illustrates the most common pattern for word-initial velar fronting: At Stage 1, \ili{WGmc} \textsuperscript{+}/ɣ/ exhibited coarticulatory fronting in word-initial position before /i/. The younger generation of speakers interpreted that fronted velar (Stage 2aa) as a palatal ([ʝ]) and therefore a specific version of word-initial velar fronting was acquired by those speakers. Underlying and phonetic representations are given below. Recall from \tabref{tab:16:4} that the target for Stage 2aa in postsonorant position is /x/; however, dialects displaying the pattern in \tabref{tab:16:5} have no /x/ in word-initial position; hence, /ɣ/ is the only dorsal fricative in that context. At Stage 2cc the younger generation of speakers extends the set of targets to include velar stops as well; in the dialects referred to here /k/ is the only noncontinuant in word-initial position.
\end{sloppypar}

\begin{table}
\caption{Change from Stage 2aa to Stage 2bb to Stage 2cc (word-initial)\label{tab:16:5}}
\fittable{
\begin{tabular}{lll}
\lsptoprule
Stage & Underlying and phonetic representations   & Targets for Velar fronting\\\midrule
1 & /ɣi/ [ɣi], /ki/ [ki]  & No targets\\
2aa & /ɣi/ [ʝi], /ki/ [ki] & /ɣ/ \\
2bb & /ɣi/ [ʝi], /ki/ [ci] & /ɣ/, /k/\\
\lspbottomrule
\end{tabular}
}
\end{table}

At Stage 2aa and 2bb the synchronic rule of word-initial velar fronting differs slightly in order to express the target segments. Thus, [coronal] spreads to [+consonantal, --sonorant,                     +continuant, dorsal] at Stage 2aa and to [+consonantal, --sonorant, dorsal] at Stage 2bb.

At Stage 2 the synchronic rule of velar fronting interacts transparently with synchronic and diachronic rules changing those targets and triggers. This means that velar fronting could be fed or bled by another rule (synchronically or diachronically); recall \figref{fig:2.6}. This transparent relationship holds during the expansion of targets and triggers as described above; see \REF{ex:16:5}. The underlying and phonetic representations here do not depict specific words, but instead entire classes of words. /i/ represents high front vowels, /e/ mid front vowels, and /eɑ/ a diphthong ending in a back vowel.

\protectedex{\ea%5
    \label{ex:16:5}
    \begin{multicols}{2}
\ea \label{ex:16:5a}\begin{tabular}[t]{@{}llll@{}}
              & /iɣ/  & /iɣ-ə/          &  /ix/       \\   
    Fnl Fort  &  ix   & {}-{}-{}-{}-{}- & {}-{}-{}-{}-\\ 
    Vel Fr    &  iç   & {}-{}-{}-{}-{}- & iç          \\
              &  [iç] & [iɣə]           & [iç]        \\
    \end{tabular}
\ex \label{ex:16:5b}\begin{tabular}[t]{@{}llcll@{}}
    /ix/ & /ex/ & > & /ix/ & /eɑx/\\
    \relax [iç] & [eç] &   & [iç] & [eɑx]\\
    \end{tabular}
\z  \end{multicols}
\z} 

(\ref{ex:16:5a}) illustrates the most common synchronic \isi{feeding} relationship. In that type of system (e.g. \ipi{Soest}, \sectref{sec:4.3}), there are two phonemic velar fricatives (/x/, /ɣ/), but only the fortis fricative /x/ serves as a target for velar fronting; hence, the synchronic rule of velar fronting illustrates Stage 2aa for targets. As shown in \REF{ex:16:5a}, \isi{Final Fortition} (Fnl Fort) \isi{feeds} velar fronting (Vel Fr). This example shows that the target for velar fronting could be either an underlying fortis velar fricative or a fortis velar fricative derived by \isi{Final Fortition}. In this example regular morphophonemic alternations of the type [x]{\textasciitilde}[ɣ] imply that \isi{Final Fortition} is synchronically active.

(\ref{ex:16:5b}) depicts a \isi{bleeding} relationship. The two examples to the left of the wedge show that velar fronting is active as a synchronic rule at Stage 2. At a later point (to the right of the wedge) a sound change replaces a front vowel with a diphthong ending in a back vowel. Since there are no alternations between [e] and [eɑ] that change is diachronic, meaning that it restructures underlying representations. Significantly, after the change from /e/ to /eɑ/ the /x/ in /eɑx/ surfaces as velar [x] and not as palatal [ç] because the second part of the diphthong /eɑ/ is not a trigger for velar fronting. In this example the change from /e/ to /eɑ/ \isi{bleeds} velar fronting. The historical \isi{bleeding} relationship discussed here is well attested in many varieties of HG and LG.

\subsection{Stage 3}
\begin{sloppypar}
The transparent relationship between velar fronting and other processes described above for Stage 2 can change into an opaque relationship. Stage 3 is the cover term for velar fronting when velar fronting is opaque. Two types of \isi{opacity} are attested: (a) some velars surface unexpectedly as velars in the context of velar fronting (\isi{underapplication}); or (b) some palatals deriving historically from velars occur unexpectedly in the back vowel context (\isi{overapplication}).
\end{sloppypar}

As discussed in Chapters~\ref{sec:5}–\ref{sec:11}, \isi{underapplication} and \isi{overapplication} are each manifested in two ways. For \isi{underapplication}, the two options are: (aa) velar fronting is counterfed synchronically by another process, or (ab) \isi{neutral vowels} emerge. For \isi{overapplication} the two historical paths are: (ba) the emergence of palatal quasi-phonemes, or (bb) the emergence of phonemic palatals that contrast with velars. In all four cases the change from Stage 2 to Stage 3 is intergenerational; hence, the older generation has velar fronting, which interacts transparently with other rules, and the younger generation acquires the opaque forms.

I consider the four scenarios described above in order:

\begin{itemize}
\item[(aa):] In this system there is a synchronic rule (Rule X) that creates new target segments which can potentially undergo velar fronting. Since those new velars fail to undergo that process, velar fronting is counterfed by Rule X. In the case studies exemplifying (aa) discussed in \chapref{sec:5} both velar fronting and Rule X are active synchronically. In \REF{ex:16:6} I focus on a dialect in which Rule X is \isi{Final Fortition}. Stage 3 is depicted to the right of the wedge in \REF{ex:16:6}. That opaque system is the outgrowth of the transparent system in \REF{ex:16:5}, repeated in \REF{ex:16:6} to the left of the wedge.

\ea \label{ex:16:6}\begin{tabular}[t]{@{}llllcllll@{}}
             &  \multicolumn{3}{c}{Stage 2}          & >  &          &     \multicolumn{3}{c}{Stage 3}           \\
             & /iɣ/ & /iɣ-ə/          &  /ix/        &    &          & /iɣ/         & /iɣ-ə/       & /ix/        \\
    Fnl Fort & ix   & {}-{}-{}-{}-{}- & {}-{}-{}-{}- &    & Vel Fr   & {}-{}-{}-{}- & {}-{}-{}-{}- & iç          \\
    Vel Fr   & iç   & {}-{}-{}-{}-{}- & iç           &    & Fnl Fort & ix           & {}-{}-{}-{}- & {}-{}-{}-{}-\\
             & [iç] & [iɣə]           & [iç]         &    &          & [ix]         & [iɣə]        & [iç]        \\
    \end{tabular}
\z 

Sequences like [ix] at Stage 3 illustrate \isi{underapplication} \isi{opacity} because \isi{Final Fortition} \isi{counterfeeds} velar fronting.

\item[(ab):] In this scenario a historical process (Rule Y) creates new front vowels which can potentially serve as triggers for velar fronting. Since those new front vowels fail to induce velar fronting, the latter process is counterfed historically by Rule Y. In the case studies discussed in \chapref{sec:6} illustrating (ab), Rule Y is no longer active synchronically. Instead, it restructures underlying representations for a younger generation of speakers. The emergence of the \isi{neutral vowel} /øix/ at Stage 3 is illustrated in \REF{ex:16:7}. The non-\isi{neutral vowel} /ei/ is included for comparison.

\ea \label{ex:16:7}\begin{tabular}[t]{@{}llcllcll@{}}
 \multicolumn{2}{c}{Stage 1} & & \multicolumn{2}{c}{Stage 2} & & \multicolumn{2}{c}{Stage 3}\\
  /oux/  &  /eix/ &  > &  /oux/ &  /eix/ & > &  /øix/ &  /eix/\\
 \relax [oux]  &  [eix] &    & [oux]  & [eiç]  &   & [øix]  & [eiç] \\
  \end{tabular}
\z 

The important point is that surface sequences like [øix] illustrate \isi{underapplication} \isi{opacity}. From the synchronic perspective, velar fronting at Stage 2 is inherited by Stage 3 speakers, but those speakers also acquire the unique representation for \isi{neutral vowels} whereby the /i/ in /øi/ is no longer [coronal].

\item[(ba):] In this type of dialect a historical process (Rule Z) eliminates triggers for velar fronting, but that change fails to \isi{bleed} velar fronting. An example of Rule Z is the change from a front vowel to \isi{schwa} (/ə/) in an unstressed syllable (\isi{Vowel Reduction}). In \REF{ex:16:8} I illustrate a system that is common (\chapref{sec:7}). At Stage 2 velar fronting is active in word-initial position. When \isi{Vowel Reduction} changes unstressed vowels -- including crucially unstressed front vowels like /i/ -- to /ə/ the palatal remains even though \isi{schwa} would be expected to be preceded by [x]. Ellipsis in the first example at Stage 2 and Stage 3 means that there is a part of the word containing a stressed vowel.

\ea%8
    \label{ex:16:8}\begin{tabular}[t]{@{}lll c lll@{}}
    \multicolumn{3}{c}{Stage 2:}   & & \multicolumn{3}{c}{Stage 3:}\\
    /xi.../ & /xe/ & /xɑ/ & > & /çə.../  & /xe/ &   /xɑ/\\
\relax [çi...] & [çe] & [xɑ] &   &  [çə...] & [çe] & [xɑ]  \\
    \end{tabular}
\z 

\begin{sloppypar}
In this type of example \isi{Vowel Reduction} \isi{counterbleeds} velar fronting. From the synchronic perspective speakers at Stage 3 acquire underlying representations like the ones to the right of the wedge. The phonetic palatal [ç] at Stage 3 is clearly an underlying palatal synchronically (/ç/) because its original trigger has been eliminated. That underlying palatal is a quasi-phoneme because [ç] and [x] never contrast in the context before \isi{schwa}.
\end{sloppypar}

\item[(bb):] In this type of dialect there is a historical process (Rule Z) which eliminates triggers for velar fronting, but that change does not \isi{bleed} velar fronting. An example of Rule Z attested in the dialects discussed in \chapref{sec:9} is the replacement of a diphthong ending in a front vowel with a back monophthong (/ɑi/ > /ɑ/).

\ea%9
    \label{ex:16:9}\begin{tabular}[t]{@{}lll c lll@{}}
    \multicolumn{3}{c}{Stage 2:}  &  & \multicolumn{3}{c}{Stage 3:}\\
     /ɑx/ & /ix/ & /ɑix/ & > &  /ɑx/&   /ix/ &  /ɑç/\\
 \relax    [ɑx] & [iç] & [ɑiç] &   & [ɑx] &  [iç]  & [ɑç] \\
    \end{tabular}
\z 

Synchronically the younger generation of speakers acquires underlying representations like the ones to the right of the wedge. The palatal must be treated as an underlying sound (/ç/) because the earlier trigger is no longer present.
\end{itemize}

The two \isi{overapplication} outcomes (ba and bb) do not imply that velar fronting is lost at Stage 3. First, in a dialect in which [x] and [ç] (< [x]) only contrast in the context of one or more back vowel, [ç] can be synchronically derived from /x/ in the context of front vowels. Second, there are still regular morphophonemic alternations triggered by \isi{Umlaut} represented by \il{Standard German}StG [bɑx] ‘stream’ vs. [bɛçə] ‘stream-\textsc{pl}’. Even though \isi{Umlaut} alternations like [ɑ]{\textasciitilde}[ɛ] are irregular, if a stem has a front vowel alternant and if that front vowel is followed by a dorsal fricative which is a trigger for velar fronting then the dorsal fricative surfaces as palatal. This generalization is true for all dialects without exception. The transition from Stage 2 to Stage 3 in \REF{ex:16:9} therefore entails two changes. First, the original palatal allophone for the older generation is now a \isi{phonemic palatal} for the younger generation. And second, velar fronting undergoes the change from an allophonic process (Stage 2) to a \isi{neutralization} (Stage 3). Likewise in varieties with the \isi{palatal quasi-phoneme} /ç/ the change from Stage 2 to Stage 3 involves a reinterpretation of velar fronting from an allophonic rule to a \isi{quasi-neutralization}.

One of the parameters mentioned earlier (output of velar fronting) is not indicated in \figref{fig:2.3}. Recall from \chapref{sec:10} that there are two different outcomes for a /x/ target: nonsibilant palatal [ç] and \isi{sibilant} alveolopalatal [ɕ]. \isi{Alveolopalatalization} involves two modifications to the Stage 2 system with the allophones [x] and [ç]. First, [ç] is realized for innovative speakers as the new allophone [ɕ] which is phonetically and phonologically distinct from postalveolar [ʃ] (/ʃ/). Second, [ɕ] and [ʃ] merge for the next generation to [ɕ], which is phonemic (/ɕ/) because it contrasts with [x] (/x/) in the context after a back vowel. That merger does not exhibit \isi{opacity} because the new phoneme /ɕ/ in the context after a back vowel does not derive historically from a velar (but instead from the coronal [ʃ]). The three stages for \isi{alveolopalatalization} are depicted in \tabref{tab:16:6}. Stage 2 is the same as Stage A.

\begin{table}
\caption{Alveolopalatalization\label{tab:16:6}}
\begin{tabular}{ll}
\lsptoprule
Stage & Underlying and phonetic representations \\\midrule
2 (=A) & /ix/ [iç], /ɑx/ [ɑx], /iʃ/ [iʃ], /ɑʃ/ [ɑʃ]  \\
B & /ix/ [iɕ], /ɑx/ [ɑx], /iʃ/ [iʃ], /ɑʃ/ [ɑʃ] \\
C & /ix/ [iɕ], /ɑx/ [ɑx], /iɕ/ [iɕ], /ɑɕ/ [ɑɕ] \\
\lspbottomrule
\end{tabular}
\end{table}

It is argued that \isi{alveolopalatalization} ([ç ʃ] > [ɕ]) is not expressed in terms of phonological rules; hence the realization of /x/ as [ç] at Stage 2/Stage A is captured formally with the same rule of velar fronting as the realization of /x/ as [ɕ] at Stage B. That the output of velar fronting is realized first as a nonsibilant and then only later as a \isi{sibilant} is expressed not in the phonology, but instead with rules of \isi{phonetic implementation}.

\section{{Velar} {fronting} {and} {the} {actuation} {problem}}\label{sec:16.7}

One question not addressed above is why velar fronting failed to phonologize in other languages/dialects with /x/. Put differently: Why was velar fronting phonologized at one particular time (ca. twelve hundred years ago) and in one particular place (modern-day Germany) but not at another time or in another place? The question posed here is a very general one that not only pertains to velar fronting but to any type of change. \citet{WeinreichHerzog1968} call it the \textsc{actuation} \textsc{problem}, which they phrase as follows (p. 102): “What features can account for the actuation of changes? Why do changes in a structural feature take place in a particular language at a particular time, but not in other languages with the same feature, or in the same language at other times?”\footnote{For recent discussion of the \isi{actuation problem} the reader is referred to \citet{Walkden2017}. See also \citet[401]{Janda2005}, who discusses briefly the \isi{actuation problem} with respect to the fronting of velars before front vowels, i.e. \isi{Velar Palatalization} as described in Appendix~\ref{appendix:i}.}

Nine years before Weinrich, Labov and Herzog published their article, \citet{Lüdtke1959} pondered the \isi{actuation problem} with respect to velar fronting. In particular, he made a proposal for why velar fronting was phonologized in Germany and not in the Netherlands: Lüdke observed that German (=LG/CG) has a phonemic lenis /ʝ/, -- in present terms, the \isi{etymological palatal} -- which served as a catalyst for the creation of fortis [ç] by velar fronting. The reason the fronting of velars after front vowels was not phonologized in the Netherlands is that \ili{Dutch} has a palatal glide /j/ (< \ili{WGmc} \textsuperscript{+}/j/), but no /ʝ/. Since there is no palatal fricative phoneme in the \ili{Dutch} system (\citealt{Gussenhoven1992}, \citealt{Booij1995}, \citealt{Verhoeven2005}) there was no precondition for the \isi{phonologization} of velar fronting.

Lüdke’s proposal is an attractive one, but it is not consistent with my claim that \ili{WGmc} \textsuperscript{+}/ɣ/ underwent velar fronting to the palatal fricative allophone [ʝ] in word-initial position before \isi{Glide Hardening} created the phoneme /ʝ/ from \ili{WGmc} \textsuperscript{+}/j/ (\sectref{sec:4.2}). One might respond that my claim concerning the time frame for \isi{Glide Hardening} is not correct. This may be the case; however, there is a deeper reason for why it is difficult to successfully account for the geography of velar fronting given the type of approach advocated by Lüdke. In particular, the truly difficult question is why that change failed to phonologize in the H(st)Almc and \il{South Bavarian}SBav regions of Switzerland and Austria (\ipi{Tyrol}). Those dialects are similar to \ili{Dutch} in the sense that they possess the palatal glide /j/ (<\ili{WGmc} \textsuperscript{+}/j/) and not the corresponding fricative. Assuming for the sake of argument that there is an independent reason for why velar fronting failed to phonologize in Switzerland and Austria (\ipi{Tyrol}), there remain two unresolved questions: (a) Why was velar fronting phonologized throughout UG (\il{Low Alemannic}LAlmc, \il{Swabian}Swb, \il{East Franconian}EFr, NBa, MBav) in South Germany and Austria?, and (b) why was velar fronting phonologized in a number of places (\chapref{sec:15}) independently from one another? The reason why these two questions are difficult to answer is that whatever structural feature one proposes for the non-velar fronting varieties of H(st)Almc and \il{South Bavarian}SBav, that same structural feature is most likely present in all of the places in (b) and in many of the places in (a).
