\chapter*{Notes on maps}

This book contains forty-six maps. Several of those maps are called locator maps because they indicate with a series of markers the location of the varieties of High German and Low German discussed in this book and show where those markers are in terms of the major dialect regions from Appendix~\ref{appendix:a}. Others are called areal distribution maps because they depict various properties involving velar fronting spatially. \mapref{map:42} portrays Old Saxon and Old High German dialect areas in around the ninth century and \mapref{map:43} the various modern realizations of diminutive suffixes. \mapref{map:44} indicates all major dialect areas for High German and Low German, and \mapref{map:45} is a historical map which illustrates the administrative divisions in pre-World War I Germany.

The borders between the major dialect areas from Appendix~\ref{appendix:a} on the locator maps correspond to the ones on the maps in \citet{WiesingerRaffin1982}. No attempt has been made to include the transition zones separating those major dialect areas (\citealt{Wiesinger1983a}: Map 47.4) on any of the maps in this book.

\begin{sloppypar}
Country borders on many of the areal distribution maps and on locator Maps~\ref{map:17} and~\ref{map:18} depict pre-World War I country borders because a significant number of sources were written during that era or before.
\end{sloppypar}

Since many of the markers indicated on the maps reflect sources from the late nineteenth century to the present day the reader must be cautioned not interpret those maps as the correct depiction of the state of velar fronting at any one point in time.
