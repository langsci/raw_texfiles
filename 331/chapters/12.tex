\chapter{Targets, triggers, and rule generalization}\label{sec:12}

\section{Introduction}\label{sec:12.1}

Rule generalization (\sectref{sec:2.4.1}) postulates that change begins with a highly restricted trigger and/or target in which phonetic conditions are particularly favorable and then progressively spreads through time and space to include more general triggers and/or targets. Recall how that model was applied in \sectref{sec:11.9.1} to German dialects with an expanded set of velar fronting target segments (noncontinuants) which developed out of a narrower set (fricatives).

In this chapter I apply the model of \isi{rule generalization} to a larger selection of German dialects. It is argued that velar fronting in both postsonorant and word-initial position was originally induced by a narrow set of front segments and that the target segment was likewise restricted to a single velar (fortis) fricative. Later stages expanded the set of triggers to include more and more front sounds, while the set of targets analogously increased to include the lenis velar and then finally velar noncontinuants. The spread from a narrow set of triggers/targets to a larger one occurred both spatially and temporally. Rule generalization is depicted abstractly in \figref{fig:2.2}.

In order to successfully implement the \isi{rule generalization} model it is necessary to provide an in-depth discussion of attested triggers and targets for velar fronting in word-initial and postsonorant position for a selection of varieties of German dialects. In principle, those varieties should be well-distributed geographically and should also represent all of the subdivisions of German dialects (Appendix~\ref{appendix:a}). To achieve that end I consider over two hundred fifty varieties of German that meet those criteria. That number includes most of the places discussed in the preceding chapters, as well as many others.

It is not feasible to provide detailed case studies for all of the works cited below. The discussion in the present chapter is therefore necessarily superficial, although care has been taken to classify those varieties in terms of targets and triggers that is consistent with the way in which those dialects are described in the original sources.

Since the focus below is on the set of targets and the set of triggers I do not discuss other aspects of velar fronting investigated in previous chapters. Hence, velar fronting may be allophonic in some dialects (Chapters~\ref{sec:3}--\ref{sec:4}), while in others there may be palatal quasi-phonemes (\chapref{sec:7}) or phonemic palatals (Chapters~\ref{sec:8}--\ref{sec:10}). It is also conceivable that velar fronting is counterfed by another process in the synchronic phonology (\chapref{sec:5}).

In \sectref{sec:12.2} I introduce a methodology that enables all dialects to be classified into a small number of trigger types and target types and in \sectref{sec:12.3} I present a survey of triggers/targets for velar fronting in German dialects (Appendix~\ref{appendix:a}) based on that methodology. \sectref{sec:12.4} considers the areal distribution of triggers and targets, and \sectref{sec:12.5} matches the trigger/target types with a series of incremental historical stages. In doing so, I demonstrate that there are certain regions where the postulated stages are represented by dialects described in the latter nineteenth century. \sectref{sec:12.6} discusses a small number of German dialects with very rare requirements governing triggers. \sectref{sec:12.7} investigates how the present treatment sheds light on the typological literature on \isi{Velar Palatalization}. In \sectref{sec:12.8} I discuss three additional properties of velar fronting: the \isi{adjacency} of its triggers and targets (\sectref{sec:12.8.1}), its domain (\sectref{sec:12.8.2}), and the status of irregularities (\sectref{sec:12.8.3}). \sectref{sec:12.9} considers the ways in which velars like /x/ are realized in the phonetics if they do not undergo velar fronting. A brief conclusion is provided in \sectref{sec:12.10}.

\section{Preliminary discussion}\label{sec:12.2}

\subsection{Velar fronting triggers}\label{sec:12.2.1}

The preceding chapters have demonstrated that the set of triggers for velar fronting differ minimally from dialect to dialect. The way in which those triggers can vary is reflected in the different versions for those fronting processes, expressed featurally (Appendix~\ref{appendix:d}). The data presented in Chapters~\ref{sec:3}--\ref{sec:11} reveal that there are five triggers for velar fronting which account for virtually all of the dialects discussed.\footnote{{The set of triggers for Kreis \ipi{Rummelsburg} (front tense vowels) from \sectref{sec:11.5} is omitted from the present discussion because it does not involve the height dimension. That unique example is discussed in \sectref{sec:12.6.2} in the context of other rare dialects with \isi{nonheight features} defining the triggers for velar fronting.}} Those triggers are given in \REF{ex:12:1}. In the notation adopted here and below, HFV=high front vowels, MHV=mid front vowels, LFV=low front vowels, and CC=coronal sonorant consonants.\largerpage[-1]\pagebreak

\ea%1
\label{ex:12:1}Five attested triggers for velar fronting:
\begin{multicols}{5}\raggedcolumns
\ea\label{ex:12:1a} \avm{\{HFV\}}\columnbreak
\ex\label{ex:12:1b} \avm{\{HFV\\MFV\}}\columnbreak
\ex\label{ex:12:1c} \avm{\{HFV\\MFV\\CC\}}\columnbreak
\ex\label{ex:12:1d} \avm{\{HFV\\MFV\\LFV\}}\columnbreak
\ex\label{ex:12:1e} \avm{\{HFV\\MFV\\LFV\\CC\}}
\z
\end{multicols}
\z

\tabref{tab:12.1} refines the way in which the triggers in \REF{ex:12:1} are to be interpreted. In the first column I indicate with letters the names of the Trigger Types, which are defined in the second and third columns.

\begin{table}
\caption{Definition of Trigger Types\label{tab:12.1}}
\begin{tabular}{llll}
\lsptoprule
Type & Trigger & \multicolumn{2}{c}{Presence in fronting context}\\\cmidrule(lr){3-4}
     &         & Present & Not present\\\midrule
A & HFV & MFV, LFV, CC\\
B & HFV, MFV & LFV, CC\\
C & HFV, MFV, CC & LFV\\
D & HFV, MFV, LFV & CC \\
E & HFV, MFV, LFV, CC & \\
AA & HFV, MFV & LFV & CC\\
BB & HFV, MFV & CC & LFV\\
CC & HFV, MFV, CC &  & LFV\\
DD & HFV, MFV, LFV &  & CC\\
EE & HFV, MFV &  & LFV, CC\\
\lspbottomrule
\end{tabular}
\end{table}

For Trigger Type A, the sole set of segments inducing fronting are high front vowels (HFV), but other front segments do not serve as triggers. In order to determine whether or not Trigger Type A is the correct one, it is therefore crucial that front segments other than high front vowels (\{MFV, LFV, CC\}) occur in the context of the velar that undergoes fronting. Trigger Type B holds if fronting is induced by \{HFV, MFV\} but not by \{LFV, CC\}, Trigger Type C if the context for fronting is \{HFV, MFV, CC\} but not \{LFV\}, and Trigger Type D if fronting is induced by \{HFV, MFV, LFV\} but not by \{CC\}. If the context for fronting is the entire set of front segments then Trigger Type E holds.

In many dialects the set of triggers involves gaps. The attested patterns with certain front segments absent in the context of velar fronting are illustrated in the last five rows in \tabref{tab:12.1}, where the gaps in question are indicated with the segment type listed in the final column. If a segment type is listed in the final column this means that either: (a) that segment type is entirely absent in the dialect, (b) that segment type is present in the dialect but not in the context for fronting, or (c) that it cannot be determined on the basis of the original source whether or not that segment type induces fronting.

Consider Trigger Type BB as an example: A target segment (e.g. /x/) undergoes fronting in the context of \{HFV, MFV\}; since \{CC\} is present in the context for fronting it can be safely concluded that \{CC\} is not included in the set of triggers. By contrast, \{LFV\} does not occur in the context for velar fronting. Because of that gap it cannot be known with certainty whether or not \{LFV\} is a trigger.

A few remarks are in order concerning the three Trigger Types where \{CC\} is not present in the fronting context (AA, DD, EE). In several varieties discussed earlier (e.g. \ipi{Sörth}; \sectref{sec:5.4}) \isi{schwa} intervenes between a coronal consonant and the fronted velar (e.g. [rəç] from /rx/), but a sequence of coronal consonant plus dorsal fricative is not attested without \isi{schwa} (e.g. [rç]). Recall from \sectref{sec:5.4} that processes of \isi{schwa} epenthesis and \isi{schwa} fronting are active in such systems, e.g. /kɛrx/→{\textbar}kɛrǝx{\textbar}→{\textbar}kɛrə̟x{\textbar}→[kɛrə̟ç]; cf. \il{Standard German}StG \textit{Kirche} ‘church’. Likewise some dialects (\chapref{sec:6} and \chapref{sec:11}) were shown to require that coalescence \isi{feeds} velar fronting, e.g. /milx/→{\textbar}m\textbf{il}x{\textbar}→[m\textbf{ilç}]; cf. \il{Standard German}StG \textit{Milch} ‘milk’. In the present chapter I only consider \{CC\} to be a trigger for velar fronting if that consonant and the velar are adjacent; this assumption means that dialects in which \isi{schwa} epenthesis and \isi{schwa} fronting or coalescence are active are classified as one of the three Trigger Types where \{CC\} is absent from the fronting context (AA, DD, EE). See \sectref{sec:12.8.1} for a synopsis of German dialects where a sound intervenes between the target and trigger.

The Trigger Types listed in \tabref{tab:12.1} all treat coronal sonorant consonants (\{CC\}) uniformly. Put differently, if /r/ is a trigger for velar fronting, then /l/ and /n/ will be as well. This assumption is confirmed in the case studies discussed throughout this book, although it is conceivable that there are systems in which the consonants described by \{CC\} should be treated individually.

The conclusions concerning \tabref{tab:12.1} are important because they either support or refute claims made in the literature on velar fronting in German or in the cross-linguistic literature on \isi{Velar Palatalization} (\sectref{sec:2.3}). Three patterns and the corresponding Trigger Types are listed in \REF{ex:12:2}:

\ea%2
\label{ex:12:2}
\ea\label{ex:12:2a} Trigger Types indicating that \{MFV, LFV\} does not induce fronting: A
\ex\label{ex:12:2b} Trigger Types indicating that \{LFV\} does not induce fronting: A, B, C, AA
\ex\label{ex:12:2c} Trigger Types indicating that \{CC\} does not induce fronting: A, B, D, BB
\z 
\z 

I discuss the relevance of \REF{ex:12:2} for my analysis of \isi{rule generalization} in German dialects in \sectref{sec:12.4.1} and for typology in \sectref{sec:12.7}.

\subsection{Velar fronting targets}\label{sec:12.2.2}

In the velar fronting varieties discussed in Chapters~\ref{sec:3}--\ref{sec:10} the targets are restricted to one or both of the velar fricatives listed in \REF{ex:12:3a}. In \chapref{sec:11} it was demonstrated that a number of places with velar fronting select their targets from the expanded list of velar consonants listed in \REF{ex:12:3b}.\footnote{I do not consider the \isi{affricate} /kx/, which serves as a target segment in various UG dialects, because that sound behaves the same way as the corresponding fricative /x/.}

\ea%3
\label{ex:12:3}Targets for velar fronting:
\ea\label{ex:12:3a} /x/, /ɣ/ 
\ex\label{ex:12:3b} /x/, /ɣ/, /k/, /g/, /ŋ/
\z
\z 

The segments in \REF{ex:12:3} can either be underlying velars or velars created from other synchronic rules. For example, the latter situation obtains in dialects for a target fortis velar fricative, which can be either underlying /x/ or {\textbar}x{\textbar} derived from /ɣ/ by \isi{Final Fortition}.

\tabref{tab:12.2} defines the way in which the targets in \REF{ex:12:3} are to be interpreted. In the first column I indicate with letter names of the Target Types, which are defined below. /g/ is enclosed in parentheses in the first four Target Types because that sound is absent in many dialects.

\begin{table}
\caption{Definition of Target Types\label{tab:12.2}}
\begin{tabular}{llll}
\lsptoprule
Type & Target & \multicolumn{2}{c}{Presence in fronting context}\\\cmidrule(lr){3-4}
     &         & Present & Not present\\\midrule
L & /x/ & /ɣ k (g) ŋ/\\
M & /x ɣ/ & /k (g) ŋ/\\
LL & /x/ & /k (g) ŋ/ & /ɣ/ \\
MM & /ɣ/ & /k (g) ŋ/ & /x/\\\midrule
Type & \multicolumn{3}{l}{Target drawn from}\\\midrule
N & \multicolumn{3}{l}{/x ɣ k g ŋ/}\\
\lspbottomrule
\end{tabular}
\end{table}

Target Types L and M are well-attested in the data presented earlier: They have in common that only fricatives serve as triggers. For Target Type L, the sole fricative undergoing fronting is /x/, but for Target Type M both /x/ and /ɣ/ serve as targets for that change.

\tabref{tab:12.2} lists two Target Types with potential targets absent from the velar fronting environment, namely Target Type LL and Target Type MM. Those systems hold if only one of the two dorsal fricatives is present in the fronting context but not the other. The segments listed in the final column mean either: (a) that the segment type (/x/ or /ɣ/) is absent entirely in the dialect, (b) that the segment type (/x/ or /ɣ/) is present in the dialect but not in the context for fronting, or (c) that it cannot be determined on the basis of the original source whether or not the segment in question (/x/ or /ɣ/) is present as a target. 

The final row in \tabref{tab:12.2} describes many of the dialects discussed in \chapref{sec:11} with the expanded set of target segments listed in \REF{ex:12:3b}. Target Type N clearly obtains if all velar consonants listed in \REF{ex:12:3b} undergo velar fronting. However, Target Type N also holds if only a subset of the velar consonants serve as targets. For example, several dialects were discussed in \chapref{sec:11} in which the set of sounds undergoing velar fronting in word-initial position consist of /ɣ k/, but other velar sounds (e.g. /x ŋ/) do not occur in that context. Target Type N also obtains if the set of undergoers includes only /k g/, but other velars do not occur in that context.\footnote{{Target Type N is admittedly a grab bag category because it does not differentiate the individual manner types among the sounds in \REF{ex:12:3b}. The drawback with Target Type N is that dialects classified as such cannot be properly interpreted without additional discussion. For example, if /ɣ k/ serve as targets (Target Type N), then one cannot know for sure whether or not both /x/ and /ŋ/ also undergo fronting. It is demonstrated below that this type of information is important in able to draw conclusions regarding velar fronting from the typological perspective (\sectref{sec:12.7}). A similar point holds for the way in which triggers and targets spread in terms of time and space (\sectref{sec:12.5}).} }

The classification in \tabref{tab:12.2} makes it possible to reach conclusions concerning the types of sounds that can or cannot undergo velar fronting. Two patterns and the corresponding Target Types are listed in \REF{ex:12:4}:

\ea%4
\label{ex:12:4}
\ea\label{ex:12:4a} Target Types indicating that /ɣ/ does not undergo fronting: L
\ex\label{ex:12:4b} Target Types indicating that /k (g) ŋ/ do not undergo fronting: L, M, LL, MM
\z 
\z 

The significance of \REF{ex:12:4} for my analysis of \isi{rule generalization} in German dialects is discussed in \sectref{sec:12.4.2}.

It is assumed above that the set of triggers are the same for any two target segments. This generalization is correct for most of the varieties investigated in previous chapters, although there are some systems attested in which one target segment (e.g. /x/) has a different set of triggers than for another target segments (e.g. /ɣ/). One example discussed in this chapter is the \il{Rhenish Franconian}RFr variety spoken in \ipi{Beerfelden} (\sectref{sec:12.7.1}). In \sectref{sec:12.3} I focus on those varieties of German where the triggers are the same for all target segments. Some discussion of varieties of German in which the triggers for /x/ are not the same as the triggers for /ɣ/ can be found in \chapref{sec:14}.

\section{{Survey} {of} {triggers} {and} {targets} {for} {velar} {fronting} {in} {German} {dialects}}\label{sec:12.3}

In the following paragraphs I classify a representative selection of HG and LG varieties in terms of the Target Types and Trigger Types defined in \sectref{sec:12.2}. The discussion is organized into subsections corresponding to the major dialects presented in Appendix~\ref{appendix:c}. All of the places cited in this section can be found on the respective locator maps (Maps~\ref{map:1}--\ref{map:12} and Maps~\ref{map:17}--\ref{map:18}).

\subsection{High and Highest Alemannic}\label{sec:12.3.1}\il{High Alemannic|(}\il{Highest Alemannic|(}

In H(st)Almc there is a single dorsal fricative (/x/); some varieties also possess the corresponding \isi{affricate} (/kx/). Those velars are realized consistently as velar (e.g. [x]) in the overwhelming majority of H(st)Almc varieties; recall, for example, \ipi{Glarus} (\citealt{Streiff1915}; \sectref{sec:3.3}). Additional H(st)Almc varieties of Switzerland in which [x]/[kx] are described as velar include \ipi{Kerenzen} \citep[17]{Winteler1876}, \ipi{Urserental} in the canton of \ipi{Uri} \citep[9]{Abegg1910}, Kesswil in the canton of Thurgau \citep[8]{Enderlin1910}, \ipi{St. Gallen} \citep[16]{Hausknecht1911}, \ipi{Entlebuch} in the canton of Lucern (\citealt{Schmid1915}: 14, 17), \ipi{Jaun} in the canton of Freiburg \citep[21]{Stucki1917}, the Berner Seeland \citep[11]{Baumgartner1922}, the \ipi{Zürcher Oberland} \citep[18]{Weber1923}, the \ipi{Sensebezirk} and the Southeast \ipi{Seebezirk} in the canton of Freiburg \citep[20]{Henzen1927}, Unterschächental in the canton of \ipi{Uri} \citep[20]{Clauss1929}, \ipi{Schaffhausen} \citep[8-9]{Wanner1941}, \ipi{Brienz} in the canton of \ipi{Bern} \citep[37]{Schultz1951}, and \ipi{Zürich} (\citealt{FleischerSchmid2006}: 244). The same generalization holds for \il{High Alemannic}HAlmc varieties spoken in the German state of Baden-Württemberg. For example, \citet[9-10]{Kaiser1910} writes that there is no palatal fricative in \ipi{Todtmoos-Schwarzenbach} and that [x] surfaces even after front vowels (“Ein palatales \textit{ch}, das in der nhd. Gemeinsprache nach den hellen Vokalen eintritt, kennt die Mundart nicht, indem auch nach den hellen Vokalen stets gutturales \textit{x} stehtˮ). \citet[56]{Beck1926} similarly observes that [x] is velar in every context (“\textit{x} ist in jeder Stellung velarˮ) in the \ipi{Markgräflerland}. A similar assessment is made for [x] in \ipi{Jestetten} by \citet{Keller1963}.

The varieties of H(st)Almc discussed in previous chapters (\sectref{sec:3.3}, \sectref{sec:3.4}, \chapref{sec:6}) with some version of velar fronting are therefore exceptions to the general pattern. In all of those places the target for that process is /x/ (and /kx/, if present). Variation among the velar fronting varieties of SwG involves the sets of sounds comprising the triggers.

In \tabref{tab:12.3}, I list the four velar fronting dialects of SwG discussed in previous chapters. In this and in all subsequent tables I give the corresponding Target Type and Trigger Type in the first two columns and the place where the dialect in question is/was spoken and the source in the third and fourth columns respectively. In the heading for each table I give the historical source for the target velar segment. For more detailed information concerning the location of the places listed in \tabref{tab:12.3} and in all subsequent tables in \sectref{sec:12.3} the reader is referred to Appendix~\ref{appendix:c}. As indicated in the heading below, the Trigger and Target Types hold for velar fronting in postsonorant position.

\begin{table}
\caption{Targets and triggers for (postsonorant) velar fronting in H(st)Almc (<\ili{WGmc} \textsuperscript{+}[k x])\label{tab:12.3}}
\begin{tabular}{llll}
\lsptoprule
Target & Trigger & Place & Source\\\midrule
LL & A &  \ipit{Visperterminen} & \citet{Wipf1910}\\
LL & B &  \ipit{Obersaxen} & \citet{Brun1918}\\
LL & C &  \ipit{Rheintal} & \citet{Berger1913}\\
LL & E &  \ipit{Maienfeld} & \citet{Meinherz1920}\\
\lspbottomrule
\end{tabular}
\end{table}

It can be seen that all four SwG varieties have the same Target Type. As indicated in the second column of \tabref{tab:12.3}, the differences among those four places is the Trigger Type: In \ipi{Visperterminen} the trigger is the set of high front vowels, in \ipi{Obersaxen} it is the high front vowels and the mid front vowels, in \ipi{Rheintal} it is the high front vowels, mid front vowels, and coronal sonorant consonants, and in \ipi{Maienfeld} it is the set of all front vowels and coronal sonorant consonants.

\tabref{tab:12.4} summarizes the targets and triggers for those SwG dialects discussed earlier with word-initial velar fronting.


\begin{table}
\caption{Targets and triggers for (word-initial) velar fronting in H(st)Almc (<\ili{WGmc} \textsuperscript{+}[k])\label{tab:12.4}}

\begin{tabular}{llll}
\lsptoprule
Target & Trigger & Place & Source\\\midrule
LL & A &  \ipit{Visperterminen} & \citet{Wipf1910}\\
LL & B &  \ipit{Obersaxen} & \citet{Brun1918}\\
LL & C &  \ipit{Rheintal} & \citet{Berger1913}\\
\lspbottomrule
\end{tabular}
\end{table}

Again, the three varieties listed here have the same Target Type, and they differ only in terms of the types of segments that trigger the fronting of a (word-initial) velar.

In \chapref{sec:15} I discuss the Trigger Types from additional H(st)Almc velar fronting areas in Switzerland and Austria (\ipi{Vorarlberg}, West \ipi{Tyrol}). In that chapter I also consider data from the linguistic atlases from those regions (SDS, VALTS).\il{High Alemannic|)}\il{Highest Alemannic|)}

\subsection{Low Alemannic and Swabian}\label{sec:12.3.2}

\subsubsection{General remarks}

In the southwesternmost varieties of \il{Low Alemannic}LAlmc, velars (/x/) are realized as velars regardless of the nature of the adjacent sound. For example, in the \il{Low Alemannic}LAlmc dialect spoken in Basel, \citet[69]{Heusler1888} makes it clear that [x] has no palatal variant (“Das \textit{χ} ... ist in jeder Stellung velar, nie der ‘ich-Laut’ˮ). A similar statement can be found in descriptions of \il{Low Alemannic}LAlmc dialects spoken in Germany (Baden-Württemberg) in and around \ipi{Freiburg im Breisgau}, e.g. \citet[43]{Ehret1911} and \citet[50]{Eckerle1936}, and in Alsace (Elsass), e.g. \citet[8]{Mankel1886} for \ipi{Münsterthal} and \citet[61]{Henry1900} for \ipi{Colmar}. The reader is referred to Map 7 in \citet{Klausmann1985a,Klausmann1985b}, which indicates the places in and around \ipi{Freiburg im Breisgau} where velar fronting is and is not active. In ALA a number of maps are given for words with [x] and [ç] in Alsace. An examination of those maps indicates that \ipi{Colmar} (my \mapref{map:1}) is the approximate boundary between velar fronting (to the north) and no velar fronting (to the south). A few places indicating the presence and absence of velar fronting from the maps in ALA are indicated on my \mapref{map:1}. E.M. \citet{Hall1991, Hall1991b} investigates the \il{Low Alemannic}LAlmc/\il{Swabian}Swb varieties in  a broad area to the south of \ipi{Villingen-Schwenningen}. Map 22 in that source shows that velar fronting is active in places to the east and west of \ipi{Villingen-Schwenningen}, but not in the places to the south. Representative examples of two velar fronting places (\ipi{Tuningen} and \ipi{Urach}) and two non-velar fronting places (\ipi{Titisee-Neustadt} and \ipi{Donaueschingen}) are indicated on my \mapref{map:1}.\footnote{{One of the first linguists to discuss the distribution of [x] and [ç] in terms of geography was \citet{Fischer1895}. In that work, Fischer observes (pp. 68-69) that the dorsal (“gutturalˮ) fricative is consistently realized as [x] after any type of vowel in SwG but that north of the Alps (“[n]ördlich der Albˮ) the same fricative is articulated as velar after back vowels and palatal after front vowels. Fischer similarly observes (pp. 63-64) that postvocalic [g] is pronounced [ç]/[x] according to context (“je nach der Lautumgebungˮ) in the north and northeast (of the \il{Swabian}Swb dialect region) and that [g] has the palatal articulation ([j]) after front vowels or [r] in the northwest (Rhineland). The latter generalization is depicted on his Map 20. Fischer writes (p. 68) that one of the reasons he was unable to determine a clear isogloss separating velar fronting areas from non-velar fronting ones -- to use my terminology -- is that north of the Alps dorsal fricatives have more than two places of articulation (recall \sectref{sec:1.5}).} } The conclusions concerning the presence vs. absence of velar fronting in Southwest Germany are also consistent with the maps in SSA. For example, the map for words like \textit{Sichel}  ‘sickle’ in Volume 2 indicates a large region with [x] after a front vowel. That area extends to the south and west of \ipi{Freiburg im Breisgau} and to the south and southeast of \ipi{Villingen-Schwenningen}.

Another area characterized by the absence of velar fronting is the eastern part of the \il{Swabian}Swb dialect region. Consider first the assessment of \citet[8]{Moser1936}, who concerns himself with the dialect spoken in the \ipi{Staudengebiet} (southwest of Augsburg). Moser writes that the palatal articulation does not occur in the dialect, even in the neighborhood of front vowels. (“Die palatale Artikulation findet sich in unserer Mda. nicht, selbst nicht in der Nachbarschaft heller Vokale wie i, e ...”). The same observation is made by \citet[46]{König1970} for \ipi{Graben}, ca. 25 south of Augsburg.\footnote{{König’s observation holds for the speech of the elderly. He adds that [x] is realized as palatal in unstressed syllables -- presumably only after front segments -- at faster rates of speech and by younger speakers.}} In fact, the assessment of Moser and König concerning the realization of [x] holds for a much larger expanse. \citet{Ibrom1971} investigates the broad region between Augsburg and Donauwörth and observes that the one dorsal fricative is realized consistently as [x] (p. 252-254). The maps in volume 7.2 of SBS provide similar data for the broad region between Augsburg and Ulm (see my \mapref{map:1}).

As indicated on \mapref{map:1}, the southeast part of the \il{Swabian}Swb dialect area both velar fronting as well as the absence of velar fronting are attested. For example, SBS Map 173 for \textit{Sichel} ‘sickle’ indicates the realization [sɪxl̩] (=⟦s\k{i}xl̥⟧) in \ipi{Ebersbach} (near Kaufbeuren). The maps in VALTS also reveal the absence of velar fronting in Wangen im Allgäu. By contrast, the SSA map for \textit{Sichel}  ‘sickle’ alluded to above indicates velar fronting for Niederwangen, and \citet{BausingerRuoff1959} show that velar fronting is attested for Beuren.\ip{Beuren (Allgäu)}

The conservative (non-velar fronting) places described above should not detract from the predominant pattern for \il{Low Alemannic}LAlmc/\il{Swabian}Swb, whereby velars like /x/ are realized as the corresponding palatals ([ç]) in the context after front sounds. Velar fronting is not attested in \il{Low Alemannic}LAlmc/\il{Swabian}Swb in word-initial position. In all of the sources cited here the sole target for (postsonorant) velar fronting fronting is /x/ (Target Type LL). Recall from earlier discussion that this means /x/ is the sole target for velar fronting because the corresponding lenis sound /ɣ/ is absent. As a general rule, velar fronting applies after all coronal sonorants, although a few varieties are attested with a narrower set of triggers. In particular, some systems possess a low front vowel (/æ/); in one set of dialects that sound serves as a trigger for velar fronting, but in another set it does not.

I make a few brief comments below on some of the \il{Low Alemannic}LAlmc/\il{Swabian}Swb varieties with uncommon triggers. \tabref{tab:12.5} provides a summary of the Trigger Types and Target Types for the \il{Low Alemannic}LAlmc/\il{Swabian}Swb sources I have consulted. A complex case of triggers varying within a small area is discussed after \tabref{tab:12.5}.

\subsubsection{Low Alemannic}
\ipi{Rheinbischofsheim} has the full set of velar fronting triggers (Trigger Type E), e.g. [hɛçt] ‘pike’, [blæç] ‘tin’, [dmelç] ‘the milk’ vs. [nɑːxt] ‘night’. In \ipi{Forbach} and in a number of communities to the east of \ipi{Freiburg im Breisgau} (Glottertal, Elztal, St. Peter, St. Märgen, Gütenbach) there are no low front vowels and epenthesis prohibits /x/ from occurring next to a consonant (=Trigger Type EE). Examples from \ipi{Forbach} include [liçt] ‘light’, [gnɛːçt] ‘vassal’ vs. [hoːx] ‘high’.

In \ipi{Oberschopfheim} the set of triggers for fronting consists solely of nonlow front vowels (Trigger Type AA), e.g. [heçt] ‘pike’ vs. [blæx] ‘tin’, [nɑːxt] ‘night’. For the areas in the \ipi{Ortenaukreis} investigated by \citet{Kilian1935} the facts are essentially the same, e.g. [sɪçl̩] ‘sickle’, [deːçtər] ‘daughter’ vs. [dræːxdər] ‘funnel’, [noːxt] ‘night’. Kilian notes that words like [dræːxdər] are realized with [ç] in communities in which [æː] corresponds to [ɛː], i.e. [drɛːçdər].

\subsubsection{Swabian}
One variety with the broadest set of triggers (Trigger Type E) is \ipi{Erdmannsweiler} (\sectref{sec:3.2}), e.g. [fræç] ‘impudent’, [knæːçt] ‘vassal’ [kalç] ‘lime’ vs. [lɑxə] ‘laugh\textsc{{}-inf}’. \ipi{Freudenstadt} is representative of Trigger Type CC, e.g. [ʃtɛçə] ‘sting’, [rɛːçt] ‘right’, [milç] ‘milk’ vs. [kʰoxə] ‘cook\textsc{{}-inf}’, and \ipi{Memmingen} of Trigger Type EE, e.g. [dɛçlə] ‘roof-\textsc{dim}’ vs. [nɑxt] ‘night’.\footnote{{The phonetic transcriptions provided in the \il{Swabian}Swb dialect dictionary (SchwWb) point to Target Type L and Trigger Type CC.} }

A narrow set of triggers (Target Type AA) is attested in \ipi{Bavendorf} \citep[49]{Schöller1939}. Although he does not provide separate symbols for [ç] and [x], Schöller gives a clear statement concerning the distribution of those sounds. In particular, \citet[49]{Schöller1939} writes that [ç] occurs after front vowels (⟦e, i, ö, ü ei⟧) and [x] after back vowels (⟦ɑ ɑ̈ o u⟧). The important point to observe in this passage is that [x] surfaces after ⟦ɑ̈⟧, which is Schöller’s symbol for [ɛ] (p. 9). If [ɛ] is phonologically [+low] -- the dialect has no phonetic [æ] -- then the set of triggers for velar fronting consists of all nonlow front vowels (Trigger Type AA).

\begin{table}
\caption{Targets and triggers for (postsonorant) velar fronting in \il{Low Alemannic}LAlmc and \il{Swabian}Swb (<\ili{WGmc} \textsuperscript{+}[k x]).\label{tab:12.5}}
\begin{tabular}{llll}
\lsptoprule
Target & Trigger & Place & Source\\
\midrule
LL & E &  \ipit{Reutlingen}         & \citet{Wagner1889}  \\
   &   &  \ipit{Rheinbischofsheim}&    \citet{Weik1913} \\
   &   &  \ipit{Erdmannsweiler}   &    \citet{Besch1961}\\
% \tablevspace
LL & AA &  \ipit{Oberschopfheim} & \citet{Schwend1900}\\
   &    &  \ipit{Ortenaukreis}   & \citet{Kilian1935}  \\
   &    &  \ipit{Bavendorf}      & \citet{Schöller1939}\\
% \tablevspace
LL & CC &  \ipit{Horb am Neckar}           & \citet{Kauffmann1887, Kauffmann1890}                     \\
   &     &  \ipit{Münsingen}               & \citet{Bopp1890}                          \\
   &     &  \ipit{Oberweier}               & \citet{Wasmer1915,Wasmer1916, Wasmer1916b}             \\
   &     &  \ipit{Herrenberg}              & \citet{Zinser1933}                        \\
   &     &  \ipit{Freudenstadt} \ipi{Stuttgart}  & \citet{Baur1967}                          \\
   &     &  \ipit{Breisgau}                & \citet{Frey1975}                          \\
   &     &  \ipit{Tuningen}, \ipi{Urach}         & \citet{Klausmann1985a,Klausmann1985b}     \\
   &     &                         & E.M. \citet{Hall1991, Hall1991b}\\
% \tablevspace
LL & EE &  \ipit{Forbach}                & \citet{Heilig1897}          \\
   &    &  \ipit{Ries}                   &    \citet{Schmidt1898}      \\
   &    &  \ipit{Pforzheim}              &    \citet{Sexauer1927}      \\
   &    &  \ipit{Freiburg im Breisgau}   &    \citet{Eckerle1936}      \\
   &    &  \ipit{Dreistammesecke}        &    \citet{Nübling1938}      \\
   &    &  \ipit{Blaesheim}              &    \citet{Philipp1965}      \\
   &    &  \ipit{Memmingen}              &    \citet{Hufnagl1967}      \\
   &    & Kreis \ipi{Balingen}         &    \citet{BethgeBonnin1969} \\
   &    &  \ipit{Mulhouse}               &    \citet{BethgeBonnin1969} \\
   &    &  \ipit{Metzeral}               &    \citet{Zeidler1978}      \\
   &    &  \ipit{Mittelbaden}            &    \citet{Schrambke1981}\\
\lspbottomrule
\end{tabular}
\end{table}

\citet{Haag1898} describes a set of \il{Swabian}Swb dialects spoken in the vicinity of \ipi{Villingen-Schwenningen}. The following passage \citep[82]{Haag1898} is significant because it illustrates the way in which the distribution of [x] and [ç] can vary depending on both time and place (as well as religion). In the varieties discussed below, there is no phonetically low vowel (i.e. [æ]), but [ɛ] (/ɛ/) is assumed to be phonologically low and front. Due to the intricacies described below, I do not include it in \tabref{tab:12.5}.

\begin{quote}
\textit{Ch} behält in allen Stellungen seine gutturale Artikulation mit Entschiedenheit nur noch im Südwesten: ex, rextə, riix, biixtə, ɛxt, štɛxxə, migləx \&; nach l mit Gleitvokal ə: : miləx, khiləxə, khaləx \&. Sonst geht es nach palatalen Vokalen in ç über: iç, riçtə, riiç, biiçtə, ɛçt, štɛççə \& . Nach Liquiden bleibt es guttural: šnarxlə, khalx, milx, khirxə, štrolx \&; mit Entschiedenheit freilich nur noch in der älteren Generation und im Osten; die jüngere, besonderes Protestanten, hat ç übernommen; Tuttlingen-Neuhausen, verschobene ç-Insel, auch noch: štaarç, štɔarç, mɛlçə \&. In der Nordwestecke, hinter 25, ausschliesslich ç: šnarçlə, khalç, milç khirç, duiç für durch \&, Gleitvokal i leicht angedeutet. Von Westen her dringt die vordere Artikulation mehr und mehr ein. Dass diese auf dem Hauptgebiet neu ist, lehrt die Zwischenstufe zwischen x und ç, die, vor allem im Südosten, für letzteres gilt, und im Munde Aeltere immer wieder mit reinem x wechselt: ix, rixtə, rɛxxtə \&, weshalb hier eine feste Grenzlegung schwierig ist; ferner im Osten Heuberg, Bära, IIart bis zum Albrand, wo fast reine gutturale Artikulation herrscht: gleix, feixt, reixbax. -- Im Nordosten, in katholischen Gemeinden, bleibt Guttural nach ɛ: ɛxxtə, rɛxxə; protestantische ɛççtə, rɛççə.

“\textit{Ch} retains its guttural articulation in all positions only in the Southwest: ex, rextə, riix, biixtə, ɛxt, štɛxxə, migləx etc.; after \textit{l} with its epenthetic vowel ə: miləx, khiləxə, khaləx etc.  Otherwise it [\textit{ch}] changes into ç after a front vowel: iç, riçtə, riiç, biiçtə, ɛçt, štɛççə etc. After liquids it remains guttural: šnarxlə, khalx, milx, khirxə, štrolx etc.; resolutely of course only in the older generation and in the East; the younger generation, especially Protestants, has adopted ç; Tuttlingen-Neuhausen, an advanced ç-island, also has: štaarç, štɔarç, mɛlçə etc.  In the Northwest corner [there is] exclusively ç: šnarçlə, khalç, milç khirç, duiç for durch etc., epenthetic vowel i implied. From the West the front articulation is occurring more and more. That this is new in the main area is indicated by the intermediate sound between x and ç which, above all in the Southwest, holds for the latter and alternates in the mouth of the elderly more and more with pure x: ix, rixtə, rɛxxtə \& ; for this reason a clear boundary is difficult to determine; furthermore in the East Heuberg, Bära, IIart up to Albrand, where a pure guttural articulation is still retained: gleix, feixt, reixbax. -- In the Northeast, in Catholic parishes, the guttural is retained after ɛ: ɛxxtə, rɛxxə; for Protestants: ɛççtə, rɛççəˮ.
\end{quote}

On the basis of this passage it is possible to draw the following conclusions: (a) In the southwest there are speakers of conservative non-velar fronting (Stage 1) varieties; (b) there are many speakers (i.e. those belonging to the older generation and those in the east) who have both [ç] and [x]; [ç] occurs for those speakers after front vowels but not after liquids (Trigger Type D or BB); (c) other speakers (e.g. young Protestants) have [ç] and [x], but the former sound occurs after front vowels and liquids (Trigger Type E or CC), and (d) a distinction can be drawn between Protestants ([ɛ] is a trigger for velar fronting) and Catholics ([ɛ] is not a trigger).

\subsection{Bavarian and East Franconian}\label{sec:12.3.3}
\subsubsection{General remarks}
The periphery of the Bav dialect continuum -- in particular, \il{South Bavarian}SBav in North \ipi{Tyrol} (Austria) and South \ipi{Tyrol} (Italy) -- is characterized by the absence of velar fronting; thus, /x/ surfaces consistently as [x]. This point is clear from the many \il{South Bavarian}SBav non-velar fronting varieties indicated on \mapref{map:3}. Within the \il{South Bavarian}SBav dialect area velar fronting is active in certain places indicated on \mapref{map:3}, including \ipi{Graz} \citep{Moosmüller1991} and in several isolated mountain valleys of \ipi{Tyrol} (e.g. \citealt{Egger1909}). I discuss those velar fronting enclaves in greater detail in \sectref{sec:15.10}.

Non-velar fronting areas are well-attested throughout \il{Central Bavarian}CBav. This point is clear from \citet{Ibrom1971}, who investigates a large area in the northwest of the \il{Central Bavarian}CBav dialect area between Augsburg and Aichach, including a large part of Northeast \il{Swabian}Swb (recall \sectref{sec:12.3.2}). On the basis of his study he concludes that the one dorsal fricative (/x/) is realized either as uvular (p. 256) or velar (p. 257, 259). That assessment is confirmed on the basis of several maps in SBS, which documents non-velar fronting from Aichach to the south along the Lech River through Grafrath and Weilheim (both indicated on \mapref{map:3}).

Outside of the areas discussed in \citet{Ibrom1971} there is little doubt that (postsonorant) velar fronting predominates in both urban and rural areas in the \il{Central Bavarian}CBav. The occurrence of velar fronting in the context after sonorants in Southeast Germany can be confirmed by examining the sources cited above, which are indicated on my \mapref{map:3}. That velar fronting is the norm for \il{Central Bavarian}CBav is also evident from the maps in the linguistic atlases for this region (SNiB for Lower Bavaria, SOB for Upper Bavaria). This point is discussed in greater detail in \chapref{sec:13}. It is nevertheless noteworthy that the maps in Volumes 3 and 4 in SNiB and some of the other sources cited below suggest that velar fronting is absent in certain places. Thus, although velar fronting predominates in \il{Central Bavarian}CBav, there is nothing out of the ordinary for some communities to articulate [x] even after front vowels. Such places can therefore be thought of as \textsc{{non-velar} \textsc{fronting} \textsc{islands}} \textsc{--} conservative enclaves which have preserved the original (\ili{WGmc}) system with /x/ and no palatal allophone.\footnote{The absence of velar fronting was also noted by V. M. Schirmunski over ninety years ago in a (NBav) German-language island in Jamburg (Ukraine) \citep[255]{Schirmunski1931}.}

Velar fronting in \il{North Bavarian}NBav/\il{East Franconian}EFr has a similar status. The sources cited in this book from those dialect areas (indicated on \mapref{map:4}) as well as the maps in the linguistic atlases for these dialect areas (SUF, SMF, SNOB, SBS) point to a region characterized by velar fronting. However, it is also not unusual to find non-velar fronting enclaves, especially in \il{North Bavarian}NBav. A few representative examples from SNOB and SBS are indicated on my \mapref{map:4}.

As in Almc, the typical pattern for \il{North Bavarian}NBav/\il{Central Bavarian}CBav/\il{East Franconian}EFr is that the sole target for postsonorant fronting is the one fricative /x/ because /ɣ/ (< \ili{WGmc} \textsuperscript{+}[ɣ]) is absent (Target Type LL). Velar fronting is typically induced by all front vowels and (if present) coronal sonorant consonants. Since low front vowels are often nonoccurring, Trigger Type CC is the most widely attested. In those varieties with no low front vowels the presence of \isi{Schwa Epenthesis} (\sectref{sec:5.4}) means that Trigger Type EE is also well-attested.

The Trigger Types and Target Types of all \il{North Bavarian}NBav/\il{South Bavarian}SBav/\il{East Franconian}EFr varieties discussed in this book are given in Tables \ref{tab:12.6} and \ref{tab:12.7}. I discuss below a few systems from this area that are typical, but I focus primarily on those patterns that are unique for the region.

\subsubsection{Central and North Bavarian}
Unmarked Target Type LL is represented by all varieties surveyed with the exception of \ipi{Kallmünz} (Target Type M). In that \il{North Bavarian}NBav variety, etymological [ɣ] has the palatal allophone [ʝ]; e.g. [gʃiçt] ‘history’, [næːʝɐl] ‘nail-\textsc{dim}’ vs. [wox] ‘week’, [lɑːɣɐ] ‘situation’. That pattern is shown below to be the unmarked one for most of CG, but it is rare for UG.

Trigger Type CC is represented by \ipi{Erdmannsdorf}/\ipi{Zillertal}, e.g. [køːç] ‘porridge’, [mɪlç] ‘milk’ vs. [mɔxn̩] ‘do-\textsc{inf}’ and Trigger Type EE by \ipi{Marchfeld},  [lɑːɪçt] ‘light’, [gŋɛːçt] ‘vassal’ vs. [brɑːuxŋ] ‘need-\textsc{inf}’. \ipi{Bergstetten} possesses low front vowels, which serve as triggers (Trigger Type E), e.g. [ʃtiːç] ‘sting’, [tsæːç] ‘tough’, [møːlç] ‘milk’ vs. [troːx] ‘trough’. \ipi{Bergstetten} can be contrasted with \ipi{Großberghofen}, where low front vowels do not induce velar fronting (Trigger Type AA), e.g. [rɪçtn̩] ‘judge-\textsc{inf}’, [ɛçt] ‘genuine’ vs. [brɔxt] ‘bring-\textsc{part}’, [ɑxt] ‘eight’, [tsæxɐ] ‘tear’. A similar pattern obtains in Kallmüntz (Trigger Type DD).

The variety of \il{Central Bavarian}CBav described by \citet{Maier1965} deserves special comment because the data in that source reveal that the dialect is unique for its area. Maier investigates a broad \il{Central Bavarian}CBav-speaking region in Upper Bavaria which is bounded in the south by Austria, to the east by the Inn River, and to the west by the Isar River. In his description of the ch-sound (“ch-Lautˮ), \citet[4]{Maier1965} observes that the region exhibits variation concerning the realization of the sound(s) depicted by those letters. The generalization is that in one corner of the larger region -- defined as Kiefersfelden, Oberaudorf, and Niederaudorf -- the only dorsal fricative is velar [x] (=⟦x⟧) even in the context after front vowels, e.g. [tixtə] ‘capable’, [ʃlɛxt] ‘bad’, [ɔxt] ‘eight’. Hence, those three places can therefore be thought of as \isi{non-velar fronting islands}. By contrast, in the other areas within the region -- e.g. Maier’s Isarwinkel, defined in terms of the villages and towns he lists on p. 1 -- there are two dorsal fricatives, namely [x] and [ç] (=⟦χ⟧). The data in \citet{Maier1965} reveal that the triggers for postsonorant velar fronting in Isarwinkel (\mapref{map:3}) consist of all and only front vowels (including the low front vowels [æ]), but not the coronal consonants. Thus, Isarwinkel represents Trigger Type D, which is rare among German dialects. Examples from the original source include the following: [ti{ҫt{ə}] ‘capable’, [{ʃ}l{ɛ}ҫt] ‘bad’, [ræç] ‘smoke’ vs. [kru{ː}x] ‘smell’, [t{ɒ}xt{ɐ}] ‘daughter’,} [bɔːx] ‘stream’, [lɛrx] ‘lark’, [milx] ‘milk’.

One \il{North Bavarian}NBav variety is attested in which only front vowels but not coronal consonants are triggers for velar fronting (\ipi{Eisendorf}; \citealt{Seemüller1908b}). According to the data from that source (=Trigger Type BB), [ç] surfaces after front vowels (no low front vowel is attested) and [x] after back vowels and liquids ([l r]), e.g. [iç] ‘I’, [ʃlɛçt] ‘bad’ vs. [woxɒ] ‘week-\textsc{pl}’, [dmyːlx] ‘the milk’, [bɑrx] ‘mountain-\textsc{pl}’.

\subsubsection{East Franconian}\il{East Franconian|(}
Unmarked Target Type LL is represented by all varieties surveyed with the exception of \ipi{Schmalkalden} (Target Type M) and \ipi{Schefflenz} (Target Type L). Representative examples from \ipi{Schmalkalden} are [lɛçt] ‘light’, [giːʝə] ‘violin’ vs. [buːx] ‘book’, [boːɣə] 'bow'. \ipi{Schefflenz} is unusual in that it possesses both /x/ and /ɣ/, and yet only the former serves as a trigger for velar fronting, e.g. [iːç] ‘I’, [brɛçə] ‘break\textsc{{}-inf}’, [mɑnç] ‘many’ vs. [lɑxə] ‘laugh-\textsc{inf}’, [fouɣl̩] ‘bird’, [fɛɣl̩] ‘bird-\textsc{pl}’, [iːɣl̩] ‘hedgehog’. That pattern (Target Type L) is otherwise restricted to LG (\il{Westphalian}Wph), e.g. \ipi{Soest} (\sectref{sec:4.3}).

Unmarked Trigger Type CC is represented by \ipi{Bonnland}, e.g. [ɛçə] ‘oak tree’, [lɛrçə] ‘lark’ vs. [joux] ‘yoke’. In a few varieties low front vowels can be shown to induce velar fronting, e.g. \ipi{Waldau} (Trigger Type E) [brɛç] ‘break-\textsc{inf}’, [væːç] ‘soft’, [blæç] ‘pale’, [ɛlç] ‘elk’ vs. [lɑx] ‘laugh-\textsc{inf}’. By contrast, \ipi{West Central Franconia} (Trigger Type C) and \ipi{Schmalkalden} (Trigger Type AA) both possess low front vowels, which do not serve as triggers, e.g. \ipi{Schmalkalden} [knæːxt] ‘vassal’, [sæɣə] ‘blessing’ (together with data from that variety given above); \ipi{West Central Franconia} [liçt] ‘light’, [milç] ‘milk’ vs. [buːx] ‘book’, [hɔːx] ‘high’, [ʃlæxt] ‘bad’.

\begin{table}
\small
\caption{Targets and triggers for (postsonorant) velar fronting in \il{South Bavarian}SBav, \il{Central Bavarian}CBav, and \il{North Bavarian}NBav (<\ili{WGmc} \textsuperscript{+}[k x])\label{tab:12.6}}
\begin{tabularx}{\textwidth}{llQQ}
\lsptoprule
Target & Trigger  & Place & Source\\
\midrule
M & DD &  \ipit{Kallmünz} & \citet{Götz1987}\\
LL & BB &  \ipit{Eisendorf} & \citet{Seemüller1908b}\\
LL & DD &  \ipit{Vienna} & \citet{Moosmüller1987}\\
  &    &  \ipit{Salzburg} & \citet{Moosmüller1991}\\
LL & D & Isarwinkel & \citet{Maier1965}\\
LL & E &  \ipit{Bergstetten} & \citet{Dozauer1967}\\
LL & AA &  \ipit{Großberghofen} & \citet{Gladiator1971}\\
LL & CC &  \ipit{Erdmannsdorf}/\ipi{Zillertal}    & \citet{Siebs1906}       \\
   &    &  \ipit{Nürnberg}                   &\citet{Gebhardt1907}      \\
   &    & Eggerland                        &\citet{Eichhorn1908}      \\
   &    &  \ipit{Hausruckviertel}            &\citet{Mindl19241925}     \\
   &    &  \ipit{Untereichenbach}            &\citet{Hain1936}          \\
   &    &  \ipit{Freutsmoos}                 &\citet{Kufner1957}        \\
   &    &  \ipit{Munich}                     &\citet{Kufner1961}        \\
   &    &  \ipit{Asch}                       &\citet{Gütter1962a}       \\
   &    &  \ipit{Schönbach}                  &\citet{Gütter1962b}       \\
   &    &  \ipit{Lauterbach}                 &\citet{Gütter1963a}       \\
   &    &  \ipit{Rezat-Altmühl}              &\citet{Schödel1967}       \\
   &    & Kreis \ipi{Schwabach}            &\citet{BethgeBonnin1969}  \\
   &    & Kreis \ipi{Wunsiedel}            &\citet{BethgeBonnin1969}  \\
   &    &  \ipit{Windischeschenbach}         &\citet{Denz1977}          \\
   &    &  \ipit{West Hungary}               &\citet{Manherz1977}       \\
   &    &  \ipit{Eslarn}                     &\citet{Bachmann2000}      \\
   &    &  \ipit{Ramsau am Dachstein}        &\citet{Noelliste2017}\\
LL & EE &  \ipit{West Bohemia}             & \citet{Gradl1895}       \\
   &    &  \ipit{Vienna}                     &\citet{Gartner1900}       \\
   &    &  \ipit{Loosdorf}                   &\citet{Seemüller1908a}    \\
   &    &  \ipit{St. Georgen} an der Gusen   &\citet{Seemüller1909a}    \\
   &    &  \ipit{Pilgersham}                 &\citet{Seemüller1909b}    \\
   &    &  \ipit{Marchfeld}                  &\citet{Pfalz1911}         \\
   &    &  \ipit{Neckenmarkt}                &\citet{Bíró1918}          \\
   &    &  \ipit{Upper Austria}              &\citet{Haasbauer1924}     \\
   &    &  \ipit{Böhmerwald}                 &\citet{Kubitschek1926}    \\
   &    & Central Bavarian                 &\citet{Kufner1960}        \\
   &    &  \ipit{Graslitz}                   &\citet{Gütter1963b}       \\
   &    &  \ipit{Munich}                     &\citet{BethgeBonnin1969}  \\
   &    &  \ipit{Hallertau}                  &\citet{Zehetner1978}      \\
   &    &  \ipit{Graz}                       &\citet{Moosmüller1991}\\
\lspbottomrule
\end{tabularx}
\end{table}

\begin{table}
\caption{Targets and triggers for (postsonorant) velar fronting in \il{East Franconian}EFr (<\ili{WGmc} \textsuperscript{+}[k x ɣ])}
\label{tab:12.7}

\begin{tabularx}{\textwidth}{llQQ}
\lsptoprule
Target  & Trigger  & Place & Source\\\midrule
L & CC &  \ipit{Schefflenz} & \citet{Roedder1936}\\
M & AA &  \ipit{Schmalkalden} & \citet{Kaupert1914}\\
LL & C &  \ipit{West Central Franconia} & \citet{Diegritz1971}\\
LL & E &  \ipit{Waldau}            & \citet{Bock1965}   \\
   &   &  \ipit{Vogtland} (Trieb)   &  \citet{Gerbet1908}\\
   &   &  \ipit{Kleinschmalkalden}  &  \citet{Dellit1913}\\
   &   &  \ipit{Suhl}               &  \citet{Kober1962}\\
LL & CC &  \ipit{Schöneck}        & \citet{Hedrich1891}     \\
   &    &  \ipit{Pfersdorf}           &\citet{HertelHertel1902}  \\
   &    &  \ipit{Wachbach}            &\citet{Dietzel1908}       \\
   &    &  \ipit{Bamberg}             &\citet{Batz1911}          \\
   &    &  \ipit{Frankenland}         &\citet{Heilig1912}        \\
   &    &  \ipit{Bonnland}            &\citet{MSchmidt1912}      \\
   &    & Rot-tal                   &\citet{Knupfer1912}       \\
   &    &  \ipit{Frankenwald}         &\citet{Werner1961}        \\
   &    &  \ipit{Gaisbach}            &\citet{Sander1916}        \\
   &    &  \ipit{East Franconia}      &\citet{Steger1968}        \\
   &    &  \ipit{Spessart}            &\citet{Hirsch1971}        \\
   &    &  \ipit{Obermainraum}        &\citet{Trukenbrod1973}    \\
   &    &  \ipit{Heilbronn}           &\citet{Jakob1985}         \\
   &    &  \ipit{Weingarts}           &\citet{Schnabel2000}\\
LL & EE &  \ipit{Klein-Allmerspann}&     \citet{Blumenstock1911}       \\
   &    & Fichtelgebirge & \citet{Meinel1932}\\
\lspbottomrule
\end{tabularx}
\end{table}


In sum, the material cited above points to a region with a clearly unmarked pattern which is disrupted only by a few outliers described above. However, the reader is referred to \chapref{sec:13}, which shows on the basis of data from SNiB that the most uncommon set of velar fronting triggers documented in the present chapter (high front vowels) is the norm in the villages and towns of Upper Bavaria.\il{East Franconian|)}

\subsection{West Central German}\label{sec:12.3.4}
\subsubsection{General remarks}
In the WCG dialect region a few conservative non-velar fronting places are situated in the northwest of the \il{Ripuarian}Rpn/LFr dialect continuum along the Dutch/Belgian border (\mapref{map:8}). One such LFr variety is the one described by \citet{Ramisch1908} for the area south of \ipi{Geldern}. Two non-velar fronting \il{Ripuarian}Rpn places in \ipi{East Belgium} are Kreis \ipi{Eupen} \citep{Welter1929} and \ipi{Montzen} \citep{Welter1933}, e.g. Kreis \ipi{Eupen} [lɪiɣə] ‘lie-\textsc{inf}’, [lyxt] ‘lie-\textsc{3sg}’. Further south, a cluster of non-velar fronting \il{Moselle Franconian}MFr varieties are indicated on \mapref{map:10}, but those places will be shown in \sectref{sec:14.6.3} to have lost the once historically active rule of velar fronting. By contrast, in the non-velar fronting varieties described by \citet{Ramisch1908} and \citet{Welter1929,Welter1933} velar fronting was never active historically.

The linguistic atlases for WCG give no indication of non-velar fronting places, even along the country borders separating German (WCG) from other languages. In ALLG there are a number of maps for German Lorraine (Deutsch-Lothringen), which includes the area between Thionville and Sarrebourg (see my \mapref{map:10}). Map 7 (\textit{lachen} ‘laugh-\textsc{inf}’) and Map 116 (\textit{hauch} ‘breath’) in ALLG indicate consistent realizations with [x] after various back vowels, while Map 160 (\textit{streicheln} ‘pet-\textsc{inf}’), Map 200 (\textit{bleich} ‘pale’), and Map 269 (\textit{Milch} ‘milk’) reveal that [ç] appears after any coronal sonorant throughout the region without exception. MRhSA likewise gives no indication that there are conservative enclaves without velar fronting, even among those places along the borders with France, \ipi{Luxembourg}, and Belgium.

Two places in the northeastern part of the \il{Ripuarian}Rpn dialect area had not yet phonologized velar fronting in the late nineteenth century: \ipi{Mülheim an der Ruhr} \citep{Maurmann1889} and \ipi{Remscheid} \citep{Holthausen1885, Holthausen1885b}. It is fairly clear from those descriptions that the ich-Laut is absent. \citet[10]{Maurmann1889} indicates this by placing his velar consonants [k g x ɣ ŋ] in a separate column from his one palatal ([ʝ]), although his description of the phonetics (p. 11) suggests that there is some coarticulatory fronting of velars. \citet{Holthausen1885} is very clear that velar fronting was not active at that time. He writes (p. 406): “x ist -- \textit{ch} in \textit{acht}, vor und nach palatalen vocalen wird seine bildungsstelle -- wie auch dies des \textit{k} -- ein klein wenig nach vorn verschoben, ohne dass jedoch die palatale articulation des \textit{ch} in \textit{ich} erreicht würde”.  (“\textit{x} is -- \textit{ch} in \textit{acht}, its place of articulation before and after front vowels -- like that of k -- advanced  slightly towards the front without reaching the palatal articulation of the \textit{ch} in \textit{ich}”). That quote and the discussion in that article suggest that there is some coarticulatory fronting of postsonorant and word-initial /x/ but that velar fronting had not yet been phonologized. Both \ipi{Mülheim an der Ruhr} and \ipi{Remscheid} appear to be \isi{non-velar fronting islands} because they are surrounded by dorsal fronting varieties. For example, \ipi{Remscheid} is located about 5km from \ipi{Wermelskirchen} to the south and 5km from Ronsdorf to the north, but both of those places have velar fronting (see \citealt{Holthaus1887} and \citealt{Hasenclever1905} on \mapref{map:8}).

Those non-velar fronting varieties aside, the generalization is that postsonorant velar fronting is present throughout the WCG area. In contrast to several of the H(st)Almc and \il{Westphalian}Wph varieties discussed earlier, velar fronting in WCG is not active synchronically in word-initial position. The diachronic change whereby word-initial \textsuperscript{+}[ɣ] surfaces as the corresponding palatal ([ʝ]) before any type of sound is a common change throughout CFr (=\il{Ripuarian}Rpn/\il{Moselle Franconian}MFr). That topic is postponed until \chapref{sec:14} because it illustrates that velar fronting can apply as a nonassimilatory change.

In postsonorant position the set of targets for WCG consists of /x/ and -- if present -- /ɣ/  (Target Type M). See \citet[7]{Viëtor1875} for an  early description of colloquial speech (Umgangssprache) specifically in RFr where this broad set of targets is presupposed. Many of the varieties below are classified as Target Type LL because /ɣ/ is absent. In only a few rare cases does /x/ but not /ɣ/ undergo velar fronting (Target Type L). This contrasts with the typical pattern for Wpf (\sectref{sec:12.3.6}). The difference between Target Type M for CG (\il{Ripuarian}Rpn) and Target Type L for ELG (\il{Westphalian}Wph) was already recognized in 1915 by Otto Lobbes (\citealt{Lobbes1915}; \mapref{map:8}). He writes (pp. 17-18) of the difference between \il{Ripuarian}Rpn and Wpf (separated by the Uerdingen line):

\begin{quote}
Im Rip. haben wir … eine palatale stimmhafte Spirans (j), die auch inlautend nach palatalen Vocalen steht, während nach velarem Vocal die velare stimmhafte Spirans (γ) eintritt. … Dagegen weichen die Mdaa. nördlich und östlich der Ürdinger Linie erheblich von den ripuarischen Mdaa. … An Stelle des stimmhaften palatalen Reibelautes … der rip. Mdaa. haben wir im Inlaut de[n] stimmhafte[n] velare[n] Spirant(en) (γ), der aber auch nach palatalem Vocal seinen velaren Charackter beibehält.

“In Ripuarian we have a palatal voiced fricative (j), which also stands word-internally after front vowels, while the velar voiced fricative (γ) occurs after back vowels ... By contrast, the dialects north and east of the Uerdingen line deviate considerably from the Ripuarian dialects ... In place of the voiced palatal fricative ... of the Ripuarian dialects we have word-internally the voiced velar fricative (γ), which also retains its velar character even after front vowelsˮ.
\end{quote}

In terms of triggers the clear pattern for WCG dialects is for all coronal sonorants to induce the change from velar to palatal, including low front vowels (if present). Recall that the broadest set of triggers is reflected with Trigger Type E. In a small number of cases identified below, low front vowels are present (/æ/), but they fail to trigger velar fronting (=Trigger Type C, AA, or CC).

The present findings concerning targets and triggers for postsonorant velar fronting are also documented in dialect dictionaries. Three dictionaries for \il{Ripuarian}Rpn (KWb and WbKM for \ipi{Cologne} and WbUS for the Lower Sieg region (die untere Sieg) -- a large area in and around Bonn -- document Trigger Type E and Target Type M. (KWb provides a description on pp. 15-17 of how to pronounce the spelling of \textit{ch} and \textit{g}. KWb and WbUS give phonetic transcriptions for each lexical entry with different symbols for the lenis palatal and velar fricatives). More extensive (multiple volume) dialect dictionaries provide details concerning the pronunciation of words in specific places. One example is SHesWb for South Hesse, which provides phonetic transcriptions with separate symbols for velars and palatals; recall from \sectref{sec:9.5} that SHesWb encodes the [x] vs. [ç] contrast. In that source, multiple phonetic transcriptions corresponding to specific places in the broad region are provided for any given word. In SHesWb, [x] regularly occurs after back vowels (e.g. \textit{Loch} ‘hole’) and [ç] after front vowels and liquids (e.g. \textit{Licht} ‘light’, \textit{Dolch} ‘dagger’); since no low front vowel is present this pattern corresponds to Target Type C. The same source reveals that some places are attested with Target Type M (e.g. \textit{Lager} ‘camp’ with [ɣ] and \textit{fegen} ‘sweep-\textsc{inf}’ with [ʝ]) and others with the rare Target Type L (e.g. \textit{Lager} ‘camp’ and \textit{fegen} ‘sweep-\textsc{inf}’ with [ʝ]). A second multiple volume dialect dictionary for WCG is RWb for the Rheinland. Like SHesWb, RWb provides phonetically transcribed words in numerous places for any given word. The area defined by RWb includes LFr, \il{Ripuarian}Rpn, \il{Rhenish Franconian}RFr, and \il{Moselle Franconian}MFr. Segments inducing velar fronting imply that Trigger Type C is typical for the region, although a closer scrutiny of RWb may reveal different Trigger Types. Target Type LL and M are typical for RWb, although it is clear from the occurrence of [ɣ] in a word-internal onset after a front vowel that the same source also recognizes Target Type L.

I discuss first the three Hes dialects (\il{East Hessian}EHes, \il{North Hessian}NHes, \il{Central Hessian}CHes) and provide a summary of the targets and triggers in \tabref{tab:12.8}. I then consider LFr, CFr (\il{Moselle Franconian}MFr/\il{Ripuarian}Rpn), and \il{Rhenish Franconian}RFr and give a summary in \tabref{tab:12.9}.

\subsubsection{Central Hessian, North Hessian, and East Hessian}
According to all sources consulted, velar fronting affects both /x/ and /ɣ/ (Target Type M) or only /x/ if /ɣ/ is not present (Target Type LL). Target Type M is represented by \ipi{Oberellenbach}, e.g. [iʝəl] ‘hedgehog’, [sæːʝn̩] ‘say{}-\textsc{inf}’, [blæç] ‘tin’, [mɛlç] ‘milk’, [ærʝər] ‘anger’ vs. [βoːɣə] ‘scale’, [kɔx] ‘cook’. No Hes variety has been uncovered with /x/ and /ɣ/ in which only /x/ triggers velar fronting (Target Type L). 

If a low front vowel is present then that vowel typically induces velar fronting, e.g. (Trigger Type E) \ipi{Oberellenbach} (see above). A second example is \ipi{Central Vogelsberg} [ʃlɛçt] ‘bad’, [bræçə] ‘break-\textsc{inf}’, [mɛlç] ‘milk’ vs. [nox] ‘still’. \ipi{Atzenhain}/\ipi{Grünberg}  (\sectref{sec:9.2}) represents a variety in which low front vowels fail to induce velar fronting (Trigger Type AA), e.g. [gəsiçt] ‘face’, [breːç] ‘break-\textsc{inf}’  vs. [bux] ‘book’, [nɔːxt] ‘night’, [blæx] ‘tin’.

\begin{table}
\small
\caption{Targets and triggers for (postsonorant) velar fronting in \il{East Hessian}EHes, \il{North Hessian}NHes, and \il{Central Hessian}CHes (<\ili{WGmc} \textsuperscript{+}[k x ɣ])\label{tab:12.8}}
\begin{tabularx}{\textwidth}{llQQ}
\lsptoprule
Target & Trigger & Place & Source\\\midrule
M & C &  \ipit{Rauschenberg} & \citet{Bromm1936}\\
M & E &  \ipit{Schlierbach}  & \citet{Schaefer1907}\\
  &   &  \ipit{Oberellenbach} & \citet{Hofmann1926}\\
M & CC &  \ipit{Bad Salzungen}                   & \citet{Hertel1888}    \\
  &    &  \ipit{Friedberg}                        &\citet{Reuss1907}       \\
  &    &  \ipit{Amtshausen}                       &\citet{Hackler1914}     \\
  &    &  \ipit{Niederhessen}                     &\citet{Hofmann1940}     \\
  &    &  \ipit{Battenberg}                       &\citet{Martin1942}      \\
  &    &  \ipit{Bad Wildungen}                    &\citet{Martin1942}      \\
  &    &  \ipit{Siegerland}/\ipi{Eichsfeld}       &\citet{Möhn1962}        \\
  &    &  \ipit{Marburg}                          &\citet{Spenter1964}     \\
  &    &  Holzhausen                             &\citet{Arend1991}\\
M & EE &  \ipit{Frankfurt am Main} & \citet{BethgeBonnin1969}\\
LL & E &  \ipit{Bad Hersfeld}     & \citet{Salzmann1888}\\
  &    &  \ipit{Hanau}               &\citet{Urff1926}      \\
  &    &  \ipit{Werra-Fuldaraum}     &\citet{Weber1959}     \\
  &    &  \ipit{Schlitzerland}       &\citet{Krafft1969}    \\
  &    &  \ipit{Central Vogelsberg}  &\citet{Hasselbach1971}\\
  &    &  \ipit{Central Hesse}       &\citet{Hasselberg1979}\\
LL & AA &  \ipit{Atzenhain}/\ipi{Grünberg} & \citet{Knauss1906}\\
LL & CC & Pfahlgraben           & \citet{Faber1912} \\
   &    & Kreis \ipi{Alsfeld}   &\citet{Heidt1922}  \\
   &    &  \ipit{Wetterfeld}      &\citet{Schudt1927} \\
   &    &  \ipit{Fulda}           &\citet{Noack1938}  \\
   &    &  \ipit{Hintersteinau}   &\citet{Müller1958a}\\
   &    &  \ipit{Kassel}          &\citet{Müller1958b}\\
   &    &  \ipit{Fulda}           &\citet{Dingeldein1995}\\
LL & DD &  \ipit{Naunheim}       & \citet{Leidolf1891} \\
   &    &  \ipit{Rhöntal}          &\citet{Glöckner1913} \\
   &    &  \ipit{Fuldaer Land}     &\citet{Wegera1977}\\
LL & EE &  \ipit{Weidenhausen}         & \citet{Friebertshäuser1961}\\
   &    & Königsstein im Taunus        & \citet{Schnellbacher1963}  \\
   &    &  \ipit{Erbstadt}               & \citet{Schudt1970}         \\
   &    &  \ipit{Bad Salzschlirf}        & \citet{Post1985}\\
\lspbottomrule
\end{tabularx}
\end{table}

Not included in \tabref{tab:12.8} is the \il{East Hessian}EHes region around \ipi{Bad Hersfeld} \citep{Martin1957}, which nicely illustrates the way in which triggers and targets can differ from place to place. Throughout the area, /x/ surfaces as [ç] after a nonlow front vowel or consonant (e.g. [liçt] ‘light’, [mɛlç] ‘milk’) and [x] after a back vowel (e.g. [ɔːxt] ‘eight’). After [æː] the dorsal fricative is realized as [ç] to the east of \ipi{Bad Hersfeld} and as [x] to the west, e.g. [ʃlæːçt] vs. [ʃlæːxt] ‘bad’, [flæːçt] vs. [flæːxt] ‘braid’. The [x] realization is attested in Kirchheim (Reckerode, Rotterode, Gershausen) and Nieraula (Kleba, Niederjossa, Hattenbach), while [ç] occurs in Wölfershausen, Unterneurode, Hillartshausen and Wehrshausen (\citealt{Martin1957}: 31, 100). Martin’s regional variety also elucidates the distinction between Trigger Type M and Trigger Type LL: In the north, /ɣ/ is realized as [ɣ] in a word-internal onset after a back vowel (e.g. [ʃvɔːɣəɐ] ‘brother-in-law’), but in the south (where [x] is realized after [æː]), /ɣ/ has restructured to /x/, e.g. [ʃvɔːxəɐ] ‘brother-in-law’.

\subsubsection{Low Franconian} Few descriptions of LFr are available, and hence it is not possible to say what the typical Target Type and Trigger Type are for that dialect area. Two varieties are attested with Target Type L, namely Kalkar (\sectref{sec:8.2}) and Homberg. The sources for those places also reveal that low front vowels are velar fronting triggers (Trigger Type D), e.g. Kalkar [pleçt] ‘duty’, [flæçtə] ‘braid’ vs. [kloxt] ‘gap’, [ʀɛːɣə] ‘rain’. The facts are similar in Kleve (Trigger Type DD).

\subsubsection{Central Franconian and Rhenish Franconian}\il{Moselle Franconian|(}
Target Type M is the norm for CFr and \il{Rhenish Franconian}RFr. A \il{Moselle Franconian}MFr variety illustrating Target Type M (\ipi{Sörth}) was discussed earlier (\sectref{sec:5.4}). A second \il{Moselle Franconian}MFr variety is \ipi{Sehlem}, e.g. [knɛːçt] ‘vassal, [hæːçəl] ‘hackle’, [ʃpiːʝəl] ‘mirror’, vs. [vɔx] ‘week’, [frɑːɣə] ‘ask\textsc{{}-inf}’, and a Target Type M from \il{Rhenish Franconian}RFr is Zaisenhausen, e.g. [prɛçə] ‘break\textsc{{}-inf}’, [iːʝl̩] ‘hedgehog’ vs. [lɑxə] ‘laugh\textsc{{}-inf}’, [froːɣə] ‘ask\textsc{{}-inf}’.\largerpage

\begin{sloppypar}
Typical for CFr and \il{Rhenish Franconian}RFr is the nonoccurrence of dorsal fricatives after consonants due to \isi{Schwa Epenthesis}, e.g. (\il{Ripuarian}Rpn) \ipi{Dülken} [veːç] ‘path’, [bøyʝən] ‘bend\textsc{{}-inf}’ vs. [frɑxt] ‘freight’, [drɑːɣə] ‘carry\textsc{{}-inf}’ and [foləç] ‘consequence’ (from /folɣ/). A similar pattern obtains in \ipi{Wermelskirchen} (\il{Ripuarian}Rpn), although that dialect allows palatals to surface after liquids in a word-internal onset, e.g. [ʃprɛçən] ‘speak\textsc{{}-inf}’, [fɛːʝən] ‘sweep\textsc{{}-inf}’, [folʝən] ‘follow\textsc{{}-inf}’ vs. [lɔx] ‘hole’, [fuːɣəl] ‘bird’.
\end{sloppypar}

Two \il{Rhenish Franconian}RFr varieties unique to their area are \ipi{Mönchzell} and \ipi{Heppenheim} because they both exhibit Target Type L, e.g. \ipi{Mönchzell} [blɛç] ‘tin’, [ʀæçt] ‘right’, [feʝə] ‘sweep\textsc{{}-inf}’, [folʝə] ‘follow\textsc{{}-inf}’ vs. [wox] ‘week’, [kʊɣl̩] ‘ball’, [flɪɣl̩] ‘wing’; \ipi{Heppenheim} [ʃlɛːç] ‘bad’, [fɛnçl̩] ‘fennel’, vs. [foɣl̩] ‘bird’, [ʃtɑiɣə] ‘climb\textsc{{}-inf}’, [lɑxə] ‘laugh\textsc{{}-inf}’.

\begin{longtable}{llll}
\caption{Targets and triggers for (postsonorant) velar fronting in LFr, \il{Moselle Franconian}MFr, \il{Ripuarian}Rpn, and \il{Rhenish Franconian}RFr (<\ili{WGmc} \textsuperscript{+}[k x ɣ])\label{tab:12.9}\il{Low Franconian}}\\
\lsptoprule Target & Trigger & Place & Source\\\midrule\endfirsthead
\midrule Target & Trigger & Place & Source\\\midrule\endhead
\endfoot\lspbottomrule\endlastfoot
L & E &  \ipit{Mönchzell} & \citet{Reichert1914}\\
L & CC &  \ipit{Heppenheim} & \citet{Seibt1930}\\
L & DD &  \ipit{Homberg}& \citet{Meynen1911}\\
  &     &  \ipit{Kalkar} & \citet{Hanenberg1915}\\
LL & CC &  \ipit{Werden} & \citet{Koch1879}\\
LL & DD &  \ipit{Ober-Flörsheim} &  \citet{Haster1908}\\
   &    &  \ipit{Saarhölzbach} &   \citet{Thies1912}\\
LL & EE & Kreis \ipi{Moers} & \citet{BethgeBonnin1969}\\
M & E &  \ipit{Kenn}   &    \citet{Thomé1908}\\
 &    & Beuren\ip{Beuren (Trier)} &   \citet{Peetz1989}\\
M & CC &  \ipit{Cologne}      & \citet{Wahlenberg1877}       \\
  &     &  \ipit{Wermelskirchen}&   \citet{Hasenclever1905}    \\
  &     &  \ipit{Sörth}         &   \citet{Hommer1910}         \\
  &     &  \ipit{Vianden}       &   \citet{Engelmann1910}      \\
  &     &  \ipit{Schelsen}      &   \citet{Greferath1922}      \\
  &     &  \ipit{Speyer}        &   \citet{Waibel1932}         \\
  &     &  \ipit{Gleuel}        &   \citet{Heike1970}          \\
  &     &  \ipit{Krefeld} &          \citet{Bister-Broosen1989}\\
M & DD &  \ipit{Sehlem} & \citet{Ludwig1906}\\
M & EE &  \ipit{Handschuhsheim}           &  \citet{Lenz1900}                   \\
  &    &  \ipit{Aegidienberg}              &    \citet{Müller1900}               \\
  &    &  \ipit{Erftgebiet}                &    \citet{Münch19041970}            \\
  &    &Zaisenhausen              &    \citet{Wanner1907,Wanner1908}    \\
  &    &  \ipit{Laubach}                   &    \citet{Wimmert1910}              \\
  &    &  \ipit{Dülken}                    &    \citet{Frings1913}               \\
  &    &  \ipit{Düsseldorf}                &    \citet{Zeck1921}                 \\
  &    &  \ipit{Seelscheid}                &    \citet{Mackenbach1924}           \\
  &    &  \ipit{Plankstadt}                &    \citet{Treiber1931}              \\
  &    &  \ipit{Pfungstadt}                &    \citet{Grund1935}                \\
  &    &  \ipit{Schlebusch}                &    \citet{Bubner1935}               \\
  &    &Kreis Wittlich            &    \citet{BethgeBonnin1969}         \\
  &    &  \ipit{Oftersheim}                &    \citet{Liébray1969}              \\
  &    &  \ipit{Großrosseln}               &    \citet{Pützer1988}               \\
  &    &  \ipit{Horath} (Hunsrück)         &    \citet{Reuter1989}               \\
  &    &Lxm                      &    \citet{Gilles1999}               \\
LL & CC &  \ipit{Southeast Palatinate}   & \citet{Heeger1896}      \\
   &    &  \ipit{Siegerland}           &  \citet{Martin1922}     \\
   &    &  \ipit{Lubeln}               &  \citet{Reuter1903}     \\
   &    &  \ipit{Merzig}               &  \citet{Tarral1903}     \\
   &    &  \ipit{Warmsroth}            &  \citet{Fuchs1903}      \\
   &    &  \ipit{Zell im Mümlingtal}   &  \citet{Freiling1929}   \\
   &    &   South Palatinate     &  \citet{Karch1980}\\
LL & DD &  \ipit{Arel}         & \citet{Bertrang1921} \\
   &    &  \ipit{Echternach}     &  \citet{Palgen1931}  \\
   &    &  \ipit{Kleve}           & \citet{Stiebels2013}\\
LL & EE &  \ipit{Niederembt} &   \citet{Grass1920}    \\
   &    &Saarlouis   &     \citet{Lehnert1926}\\
   &    &  \ipit{Saarbrücken} &     \citet{Kuntze1932} \\
   &    &  \ipit{Ittersdorf} &     \citet{Pallier1934}\\
\end{longtable}\il{Low Franconian}

According to all of the sources for CFr and \il{Rhenish Franconian}RFr surveyed, if a low front vowel is present then it serves as a trigger for velar fronting (Trigger Type D, E, DD). In this respect, CFr and \il{Rhenish Franconian}RFr differ from Hes ones.\il{Moselle Franconian|)}

\subsection{East Central German}\label{sec:12.3.5}
\subsubsection{General remarks}
The present survey has failed to discover any references to non-velar fronting enclaves in the ECG dialect region; hence, all places discussed below have some version of postsonorant velar fronting. Word-initial velar fronting occurs in \il{High Prussian}HPr (\sectref{sec:11.7}) and the two \il{Silesian}Sln varieties referred to above (\sectref{sec:11.4}), but that type of system is otherwise unattested in this area. In \il{North Upper Saxon-South Markish}NUSax-SMk it is common for the modern reflex of \ili{WGmc} \textsuperscript{+}[ɣ] to be realized as the corresponding palatal, but this development is not discussed until \chapref{sec:14}.

One nearly exceptionless generalization holding for the sources cited is that if /x ɣ/ are present, then both sounds undergo (postsonorant) velar fronting (Target Type M). One ECG dialect has been found -- commented on below -- possessing /x ɣ/, in which only /x/ undergoes fronting (Target Type L). Recall from \chapref{sec:11} that two varieties of \il{Silesian}Sln (\ipi{Sebnitz} and \ipi{Seifhennersdorf}) as well as \il{High Prussian}HPr (\ipi{Reimerswalde}) have a broad set of targets for velar fronting because it includes velar stops and the velar nasal (Target Type N).

As a general rule, the velar fronting triggers subsume all front vowels, including low front vowels (if present), and coronal sonorant consonants (Trigger Type E). A small number of varieties commented on below have been discovered in which low front vowels fail to trigger velar fronting. A pattern that is even more rare is one in which only front vowels but not coronal sonorant consonants induce velar fronting. A few such places (Trigger Type BB) have been identified and are discussed below.

I consider now the individual groupings within ECG, beginning with \il{Thuringian}Thrn, \il{Upper Saxon}USax, and \il{North Upper Saxon-South Markish}NUSax-SMk, and then I turn to \il{Silesian}Sln and \il{High Prussian}HPr. The generalizations concerning targets and triggers are summarized in Tables \ref{tab:12.10}--\ref{tab:12.13}.

\subsubsection{Thuringian, Upper Saxon, and North Upper Saxon-South Markish}\largerpage[2]
Target Type M is represented by \ipi{Leinefelde} (\il{Thuringian}Thrn), e.g. [zɪçl̩] ‘sickle’, [tsɛʝn̩] ‘goat’, [plæç] ‘tin’, [næːçr̩] ‘closer’, [pʊrç] ‘castle’ vs. [lɔx] ‘hole’, [loːɣr̩] ‘camp’. Since historical \textsuperscript{+}[ɣ] restructured to [x] (/x/) in most varieties of \il{Thuringian}Thrn/\il{Upper Saxon}USax, Target Type LL is the predominant pattern, e.g. \ipi{Sondershausen} (\il{Thuringian}Thrn) [brɛçə] ‘break\textsc{{}-inf}’, [blæç] ‘tin’, [forçə] ‘furrow’ vs. [buːx] ‘book’. A rare case of Target Type M for \il{Upper Saxon}USax is attested in \ipi{Salzfurtkapelle}, e.g. [bleç] ‘tin’, [iːʝəl] ‘hedgehog’, [balʝən] ‘scuffle\textsc{{}-inf}’, [zorʝə] ‘worry’ vs. [brauxən] ‘need\textsc{{}-inf}’, [krɑːɣən] ‘collar’. The one case of Target Type L known to me is \ipi{Eisenach} (\il{Thuringian}Thrn), e.g. [ɛç] ‘I’ vs. [pɑxt] ‘lease’, [boːɣn̩] ‘bow’, [beːɣn̩] ‘bow-\textsc{pl}’.

Trigger Type E is illustrated by \ipi{Zschorlau} (\il{Upper Saxon}USax), e.g. [heç] ‘height’, [læçt] ‘light’, [dolç] ‘dagger’ vs. [rɑːx] ‘smoke’. Four sources consulted make it clear that there are low front vowels that fail to induce velar fronting (Trigger Type C). That pattern is reflected in \il{Upper Saxon}USax (\ipi{Vorerzgebirge}) and \il{Thuringian}Thrn (\ipi{Buttelstedt}, Southwest Thuringian, \ipi{Eichsfeld}), e.g. \ipi{Vorerzgebirge} [knɛçt] ‘vassal’, [doːrç] ‘through’ vs. [nuːx] ‘still’, [nɑxt] ‘night’, [wæːxŋ] ‘because of’. The [æː] in the latter example is described in the original source as low front vowel (“überhelles \textit{a} tiefster Mittelzungenvokal …”); \citet[43]{Bergmann1965}.{\interfootnotelinepenalty=10000\footnote{According to the transcriptions provided in the dictionary for \il{Upper Saxon}USax (ObersWb), it is evident that the target for velar fronting is /x/ (Target Type LL) and that the process is induced by coronal sonorants (Trigger Type CC).}}

One unique place in this region where velar fronting fails to apply after [r] (Target Type BB) is \ipi{Itzgrund}, which occupies the southern corner of the \il{Thuringian}Thrn dialect region (\mapref{map:12}). \citet[128]{Spangenberg1989} notes that the entire \il{Thuringian}Thrn region is characterized by (postsonorant) velar fronting. He writes that after a consonant [ç] typically occurs but that in \ipi{Itzgrund} [x] is commonly realized after the rhotic. (“Nach Kons. erscheint wie in der StSpr. ç, doch im Itzgr begegnet nach r wie in der Mda auch häufig x …”). The three examples Spangenberg gives are [dʊrx] ‘through’, [ʃnɑːrxt] ‘snore-3\textsc{sg}’, and [kirxŋkoːr] ‘church choir’.

%%please move \begin{table} just above \begin{tabular
\begin{table}
\caption{Targets and triggers for (postsonorant) velar fronting in \il{Thuringian}Thrn (<\ili{WGmc} \textsuperscript{+}[k x ɣ])\label{tab:12.10}}
\begin{tabularx}{\textwidth}{llQQ}
\lsptoprule
Target & Trigger & Place & Source\\\midrule
L & E &  \ipit{Eisenach} & \citet{Flex1893}\\
M & E &  \ipit{North Thuringia}  & \citet{Schultze1874}   \\
  &   &  \ipit{Stiege}         & \citet{Liesenberg1890} \\
  &   &  \ipit{Leinefelde}     & \citet{Hentrich1905}   \\
  &   &   Honsteinisch   & \citet{Rudolph1924}\\
LL & C &  \ipit{Buttelstedt}          & \citet{KürstenBremer1910}        \\
   &   &  \ipit{Southwest Thuringia}  & \citet{Kürsten1910,Kürsten1911}  \\
   &   &  \ipit{Eichsfeld}            &  \citet{Hentrich1920}\\
LL & E &  \ipit{Bad Frankenhausen}   & \citet{Frank1898}      \\
   &   &  \ipit{Sondershausen}       &\citet{Schirmer1932}    \\
   &   &  \ipit{Unterellen}          &\citet{Spangenberg1962} \\
   &   & Dudenrodt, \ipi{Netra}    &\citet{Guentherodt1982} \\
   &   &  \ipit{Barchfeld}           &\citet{Weldner1991}\\
LL & BB &  \ipit{Itzgrund} & \citet{Spangenberg1989}\\
LL & CC &  \ipit{Osterland}         & \citet{Trebs1899}  \\
   &    &  \ipit{Altenburg}         & \citet{Daube1906}  \\
   &    &  \ipit{Niddawitzhausen}   & \citet{Rasch1912}  \\
   &    & Weidenhain        & \citet{Krug1969}   \\
   &    &  \ipit{Ludwigsstadt}      & \citet{Harnisch1987}\\
\lspbottomrule
\end{tabularx}
\end{table}

\begin{table}
\caption{Targets and triggers for (postsonorant) velar fronting in \il{Upper Saxon}USax and \il{North Upper Saxon-South Markish}NUSax-SMk (<\ili{WGmc} \textsuperscript{+}[k x ɣ])\label{tab:12.11}}
\begin{tabularx}{\textwidth}{llQQ}
\lsptoprule
Target & Trigger & Place & Source\\
\midrule
M & CC &  \ipit{Salzfurtkapelle}      & \citet{Schönfeld1958}                \\
  &    &  \ipit{Friedersdorf}           &\citet{Seibicke1967}                 \\
  &    &  \ipit{Grassau}                &\citet{Stellmacher1973}              \\
  &    &  \ipit{Berlin}                 &\citet{Schönfeld1986}, \citet{BethgeBonnin1969}\\
M & EE &  \ipit{Aken} (Elbe) & \citet{Bischoff1935}\\
LL & C &  \ipit{Vorerzgebirge} & \citet{Bergmann1965}\\
LL & E &  \ipit{Leipzig}      & \citet{Albrecht1983}  \\
   &   &  \ipit{Zwickau}        & \citet{Philipp1897}    \\
   &   &  \ipit{Zschorlau}      & \citet{Lang1906}       \\
   &   &Meißnisch            & \citet{Große1955}      \\
   &   &  \ipit{Dresden}        & \citet{Fleischer1961}  \\
   &   &Erzgebirge           & \citet{Goepfert1878}\\
LL & CC &  \ipit{Greiz}               & \citet{Hertel1887}    \\
   &   &  \ipit{Brüx}                    &\citet{Hausenblas1898}  \\
   &   &  \ipit{Dubraucke}               &\citet{Goessgen1902}    \\
   &   &  \ipit{Schokau}                 &\citet{Pompé1907}       \\
   &   &  \ipit{Northwest Bohemia}       &\citet{Hausenblas1914}  \\
   &   &  \ipit{West Lausitz}            &\citet{Protze1957}      \\
   &   &South Upper Saxon             &\citet{Becker1969}      \\
   &   &  \ipit{Wittenberg}              &\citet{Langner1977}\\
\lspbottomrule
\end{tabularx}
\end{table}

\subsubsection{Silesian and High Prussian}
Target Type M is represented by \ipi{Kieslingswalde} (\il{Silesian}Sln), e.g. [lɪçt] ‘light’, [rɛçtɐ] ‘judge’, [keːʝl̩] ‘pin’, [mɛlç] ‘milk’ vs. [hoːɣl̩] ‘hail’, [nɔx] ‘still’, and Target Type LL by Reichenberg (\il{Silesian}Sln), e.g. [raeç] ‘rich’, [mɑnçə] ‘some\textsc{{}-infl}’ vs. [vɔxə] ‘week’. \ipi{Sebnitz} and \ipi{Seifhennersdorf} (\sectref{sec:11.4}) deviate from the other \il{Silesian}Sln dialects because they possess the broadest set of targets for postsonorant fronting (Target Type N).

The generalizations concerning targets and triggers for velar fronting in \il{Silesian}Sln are for the most part consistent with the maps in SchlSA (although recall the discussion in \sectref{sec:9.5} on the occurrence of [ç] after a back vowel). A closer examination of  SchlSA’s Map 6 reveals that there are parts of the \il{Silesian}Sln dialect region where the word \textit{Kirche} ‘church’ is realized with [x] after the coronal consonant [r], e.g. \ipi{Hohenelbe} ([kɛrx]), \ipi{Grulich} ([kɛrxə], [kɑrxə]), and \ipi{Bärn} ([kɪrx]). All three places are indicated on my \mapref{map:9}. SchlSA’s Map 26 for \textit{leuchten} ‘glow-\textsc{inf}’ reveals that velar fronting is active throughout \il{Silesian}Sln -- including the three aforementioned places -- because [ç] is present after a front vowel (i.e. [lɛçd̥n̩]).

\begin{table}
\caption{Targets and triggers for (postsonorant) velar fronting in \il{Silesian}Sln and \il{High Prussian}HPr (<\ili{WGmc} \textsuperscript{+}[k x ɣ ŋ])}
\label{tab:12.12}
\begin{tabularx}{\textwidth}{llQQ}
\lsptoprule
Target & Trigger & Place & Source\\
\midrule
M & CC &  \ipit{Kieslingswalde}       & \citet{Pautsch1901}  \\
  &    &  Schlesische Mundart & \citet{vonUnwert1908}\\
  &    &  Kreis \ipi{Hirschberg}    & \citet{Graebisch1912a}\\
  &    &  Alt-Waltersdorf     & \citet{Graebisch1912b}\\
  &    &  \ipit{North Moravia}       & \citet{Weiser1937}   \\
  &    &  Kreis \ipi{Jauer}         & \citet{Halbsguth1938}\\
  &    &  \ipit{Grafschaft Glatz}    & \citet{Blaschke1966}\\

M & EE &  \ipit{Kunewald} & \citet{Giernoth1917}\\
N & EE &  \ipit{Seifhennersdorf} & \citet{Michel1891}      \\
  &    &  \ipit{Sebnitz}        & \citet{Meiche1898}      \\
  &    &  \ipit{Reimerswalde}   & \citet{KuckWiesinger1965}\\
LL & E &  \ipit{Römerstadt} & \citet{Rieger1935}\\
LL & BB &  \ipit{Hohenelbe}, \ipi{Grulich}, \ipi{Bärn} & SchlSA\\
LL & CC &  \ipit{Lehmwasser}       & \citet{Hoffmann1906}  \\
   &    &  Reichenberg     &\citet{Kämpf1920}      \\
   &    &  \ipit{East Bohemia}    &\citet{Festa1925}      \\
   &    &  \ipit{Kay}             &\citet{Messow1965}\\
LL & EE & Groβ-Schönau & \citet{Wenzel1919}\\
\lspbottomrule
\end{tabularx}
\end{table}

As noted above, word-initial velar fronting is restricted to two varieties of \il{Silesian}Sln and one variety of \il{High Prussian}HPr.

\begin{table}
\caption{Targets and triggers for (word-initial) velar fronting in \il{Silesian}Sln and \il{High Prussian}HPr (<\ili{WGmc} \textsuperscript{+}[k x ɣ ŋ])\label{tab:12.13}}

\begin{tabularx}{\textwidth}{llQQ}
\lsptoprule
Target  & Trigger  & Place & Source\\
\midrule
N & EE &  \ipit{Sebnitz}          & \citet{Meiche1898}       \\
  &    &  \ipit{Seifhennersdorf}  &  \citet{Michel1891}      \\
  &    &  \ipit{Reimerswalde}     &  \citet{KuckWiesinger1965}\\
\lspbottomrule
\end{tabularx}
\end{table}

In sum, ECG is a region with a consistent pattern whereby /x/ -- and /ɣ/ if present -- undergo fronting to the respective palatals after any coronal sonorant. That unified picture is disrupted by a few places referred to above which have a broad set of targets as well as several enclaves where fronting is induced by only a subset of coronal sonorants.

\subsection{West Low German}\label{sec:12.3.6}
\subsubsection{General remarks}
As noted in \sectref{sec:4.2}, non-velar fronting varieties of WLG (NLG) are attested in the far western part of Lower Saxony, i.e. \ipi{Lathen}, e.g. [zyxtə] ‘sigh\textsc{{}-inf}’, [riːɣə] ‘row’. \il{Westphalian}Wph varieties with no velar fronting include \ipi{Grafschaft Bentheim} \citep[13]{Rakers1944} and \ipi{Ostbevern} \citep{Holtmann1939}. A more recent example is discussed by \citet{Brandes2011} for the area between Breckerfeld, Hagen, and Iserlohn. He writes (p. 242): “Das reine palatale [ç] existiert in UG nicht.” (“The pure palatal [ç] does not exist in the area under investigation.”) In the context after a front vowel, Brandes transcribes the fricative in question as “[ç/x]”, which is intended to reflect the fact that it has an articulation between [x] and [ç]. I interpret this to mean that /x/ undergoes phonetic fronting to a prevelar; see \sectref{sec:12.9.1}. In all of these non-velar fronting places, \ili{WGmc} \textsuperscript{+}[ɣ] is also preserved as a velar fricative in word-initial position regardless of the nature of the following sound, a pattern that is also reflected in \ipi{Borken} (\citealt{Herdemann1921}) and \ipi{Gütersloh} \citep{Wix1921}. Note that velar fronting is active in postsonorant position in both \ipi{Borken} and \ipi{Gütersloh}.\footnote{According to the section on pronunciation (p. 377) in the dictionary for the \il{Westphalian}Wph dialect (WphWb), the \il{Standard German}StG ich-Laut is absent in most varieties of \il{Westphalian}Wph. This is a peculiar assertion, since it is blatantly contradicted by the studies on \il{Westphalian}Wph cited throughout this book. A more realistic statement can be found in WMlWb. That source states clearly (p. 25) that Westmünsterland -- roughly speaking, the area between Bocholt and Vreden (\mapref{map:6}) -- is characterized by postsonorant velar fronting of /x/. By contrast, WMlWb notes that the modern reflex of \ili{WGmc} \textsuperscript{+}[ɣ] in word-initial position is a velar fricative even in the context before front vowels (p. 25).}

In all other sources consulted for WLG dialects there is some version of postsonorant velar fronting and -- in some places -- word-initial velar fronting. In contrast to HG, WLG shows much more variation concerning targets and triggers. For example, several places (nearly all \il{Westphalian}Wph) are attested with the rare target Type L. \il{Westphalian}Wph is also important because it exhibits variation in the types of segments that can serve as triggers, i.e. rare Trigger Types A and B are both attested. \il{Westphalian}Wph and \il{Eastphalian}Eph also contrast with HG in the sense that some version of word-initial velar fronting can be shown to be synchronically active.

I discuss the three WLG groupings separately, beginning with NLG. The generalizations concerning targets and triggers are summarized in Tables \ref{tab:12.14}--\ref{tab:12.18}.

\subsubsection{North Low German}\il{North Low German|(}
Velar fronting is active throughout this region, but only in postsonorant position. Typical for NLG is Target Type LL because historical /ɣ/ has restructured to /g/; recall \ipi{Altengamme} from \sectref{sec:4.2}. The set of triggers subsumes all front vowels -- including low front vowels (if present) and coronal sonorant consonants (if present). In many varieties /g/ spirantizes in coda position, surfacing as [x] or [ç] depending on the nature of the preceding sound.\footnote{{This description of the distribution of [x]/[ç] and [g] in NLG matches the one for Hamburg as presented in HaWb (Volume 2: 231). The examples cited in the survey of NLG (“Niedersächsisch") in \citet[37--38]{Stellmacher1981} point to an area where postsonorant velar fronting is active.} }

Target Type LL is attested in \ipi{Altengamme} (\sectref{sec:4.2}).  e.g. [slɛç] ‘bad’, [fɛlç] ‘wheel rim’ vs. [ɑx] ‘eight’. \ipi{Diepenau} illustrates Target Type M, e.g. [flɛçtn̩] ‘braid\textsc{{}-inf}’, [fɔlʝn̩] ‘follow\textsc{{}-inf}’ vs. [lɑxn̩] ‘laugh\textsc{{}-inf}’, [ʒɔːɣn̩] ‘hunt\textsc{{}-inf}’, while \ipi{Jadebusen} is the only NLG example of rare Target Type L, e.g. [zœːɣ] ‘sow’, [leːiɣt] ‘lie\textsc{{}-2pl}’ vs. [ɛçt] ‘genuine’.

\ipi{Oldenburg} exemplifies the entire range of triggers (Trigger Type E), e.g. [dɪçt] ‘tight’, [lɛçt] ‘light’, [dæːç] ‘hard-working’ vs. [lʊxt] ‘air’.

\begin{table}
\caption{Targets and triggers for (postsonorant) velar fronting in NLG (<\ili{WGmc} \textsuperscript{+}[k x ɣ])\label{tab:12.14}}

\begin{tabularx}{\textwidth}{llQQ}
\lsptoprule
Target  & Trigger & Place & Source\\\midrule
L & EE &  \ipit{Hollenstedt} & \citet{Götze1922}\\
  &    & Jade        & \citet{Götze1922} \\
  &    &  \ipit{Jadebusen}   & \citet{Schmidt-Brockhoff1943}\\
M & E &  \ipit{Diepenau} & \citet{Schmeding1937}\\
M & EE &  \ipit{Badbergen}   & \citet{Vehslage1908}\\
  &    &  \ipit{Bergenhusen}& \citet{Sievers1914}\\
LL & CC &  \ipit{Bleckede} & \citet{Rabeler1911}\\
LL & E &  \ipit{Oldenburg} &  \citet{vorMohr1904}\\
LL & CC &  \ipit{Altengamme}                & \citet{Larsson1917}\\
   &    &  \ipit{Finkenwärder}          & \citet{Kloeke1914} \\
   &    &  \ipit{Grambkermoor} (Bremen) & \citet{Bollmann1942}\\
LL & EE & Kreis Herzogtum \ipi{Lauenburg}& \citet{Heigener1937}    \\
   &    &  \ipit{Hemmelsdorf}              & \citet{Pühn1956}        \\
   &    &  \ipit{Harburg}                  & \citet{Keller1961}      \\
   &    & Kreis \ipi{Kiel}               & \citet{BethgeBonnin1969}\\
   &    &  \ipit{Oldenburger Ammerland}    & \citet{Mews1971}\\
   &    &  \ipit{Nordstrand}    & \citet{Willkommen1999}\\
\lspbottomrule
\end{tabularx}
\end{table}\il{North Low German|)}

\subsubsection{Westphalian}\largerpage
In contrast to NLG, Target Types L and M are both well-attested. Target Type M for the entire range of triggers (Trigger Type E) is represented by \ipi{Elspe} (\sectref{sec:7.2}), and \ipi{Borken} (\sectref{sec:4.3}); e.g. \ipi{Borken} [zɛç] ‘say\textsc{{}-3sg}’, [fæːʝən] ‘sweep\textsc{{}-inf}’, [bɛrç] ‘mountain’, [zɛʝʝən] ‘say\textsc{{}-inf}’ vs. [trɔx] ‘trough’. Target Type L is exemplified by four varieties discussed in previous chapters, namely \ipi{Soest} (\sectref{sec:4.3}), \ipi{Adorf} (\sectref{sec:4.3}), \ipi{Schieder-Schwalenberg} (\sectref{sec:7.2}), and \ipi{Rhoden} (\sectref{sec:5.2}).

The triggers for postsonorant velar fronting can consist of all coronal sonorants (Trigger Type E) or a more restricted subset.  One example of the latter is \ipi{Rhoden} (Trigger Type AA), e.g. [lɛçt] ‘light’ vs. [ʃlæxt] ‘bad’.

The \ipi{Byfang} data (within Vest Recklinghausen) discussed in \citet{Hellberg1936} point to a variety with the rare Trigger Type B (and Target Type L). In \ipi{Byfang} there is no low front vowel ([æ]), but front lax [ɛ] patterns phonologically as [+low]. Representative examples include [tɑiçəl] ‘brick’ (from /tɑixəl/), [tyːç] ‘stuff’ (from /tyːɣ/) vs. [ɑxtər] ‘behind’, [kɔɣəl] ‘ball’, [liɣən] ‘lie\textsc{{}-inf}’, [rɛx] ‘quite’, and [ʃlɛxt] ‘bad’. Examples like [pɔːlbœrɣər] ‘someone whose family has been living in a community over several generations’ and [bɛrx] ‘mountain’ indicate that the set of triggers does not include coronal sonorant consonants.

\ipi{Plettenberg} displays the rare Trigger Type A. In that dialect, /x/ regularly undergoes fronting to [ç] after [i], e.g. [biçtə] ‘confession’, [filiçtə] ‘maybe’. After back vowels, coronal consonants and nonhigh front vowels, [x] occurs, e.g. [tuxt] ‘breeding’, [nox] ‘still’, [æxtr̩] ‘behind’, [nœxtə] ‘vicinity’, [lext] ‘light’, [rɛxt] ‘justice’, [biɛrx] ‘mountain’. The set of segments undergoing fronting consists solely of /x/ because /ɣ/ surfaces as [ɣ] in a word-internal onset, even after [i], e.g. [niɣə] ‘new’. The facts involving velar fronting in \ipi{Plettenberg} are discussed and further refined in \sectref{sec:12.6.1}.\largerpage

\begin{table}
\caption{Targets and triggers for (postsonorant) velar fronting in \il{Westphalian}Wph (<\ili{WGmc} \textsuperscript{+}[k x ɣ])\label{tab:12.15}}
\begin{tabularx}{\textwidth}{llQQ}
\lsptoprule
Target  & Trigger & Place & Source\\
\midrule
L & A &  \ipit{Plettenberg} & \citet{Gregory1934}\\
L & B & Vest Recklinghausen (\ipi{Byfang}) & \citet{Hellberg1936}\\
L & AA &  \ipit{Rhoden}  & \citet{Martin1925}\\
  &    &  \ipit{Willingen}& \citet{Martin1942}\\
L & CC &  \ipit{Schieder-Schwalenberg} &\citet{Böger1906}\\
  &    &  \ipit{Altenluenne} &           \citet{Borchert1955}\\
L & DD &  \ipit{Adorf} & \citet{Collitz1899}\\
L & E &  \ipit{Sudeck}            &  \citet{Martin1942} \\
  &   &   Kreis \ipi{Tecklenburg} &  \citet{BethgeBonnin1969}\\
L & EE &  \ipit{Soest}        & \citet{Holthausen1886} \\
  &     &  \ipit{Paderborn}     &  \citet{Brand1914}     \\
  &     &  \ipit{Freienhagen}   &  \citet{Martin1942}    \\
  &     &  \ipit{Laer}          &  \citet{Niebaum1974}   \\
  &     &  \ipit{Müschede}      &  \citet{NiebaumTeepe1976}\\
M & E &  \ipit{Elspe}   &\citet{Arens1908}\\
  &   &  \ipit{Borken} &  \citet{Herdemann1921}\\
M & EE &  \ipit{Münster} & \citet{Keller1961}\\
M & CC &  \ipit{Lippe}        & \citet{Hoffmann1887}   \\
  &    &  \ipit{Hiddenhausen} & \citet{Schwagmeyer1908}\\
  &    &  \ipit{Nienberge}    & \citet{Seymour1970}    \\
  &    &  \ipit{Reelkirchen}  & \citet{Stellmacher1972}\\
\lspbottomrule
\end{tabularx}
\end{table}

One source not listed in \tabref{tab:12.15} is \citegen{Schulte1941} survey of the \il{Westphalian}Wph varieties spoken in the \ipi{Southeast Sauerland}.\footnote{{The work is not cited in \tabref{tab:12.15} because it is difficult to determine the correct set of targets and triggers for any one community. Nevertheless, as I point out below \citegen{Schulte1941} study represents a microcosm of the \il{Westphalian}Wph region.} } Several generalizations can be made from that source that corroborate the facts from other \il{Westphalian}Wph dialects. Schulte has both [ç] (=⟦χ⟧) and [x] (=⟦x⟧), and -- not surprisingly -- [ç] but never [x] occurs after high front vowels and [x] but not [ç] after back vowels, regardless of the village, e.g. [nɪç] ‘not’ vs. [mɑːxn̩] ‘do\textsc{{}-inf}’. By contrast, dorsal fricatives occurring after the front vowels [ɛ œ] can vary according to regions between [x] and [ç]. For example, \citet[26]{Schulte1941} observes that \textit{schlechten} ‘bad\textsc{{}-infl}’ is realized as [ʃlɛçtn̩] in some communities and as [ʃlɛxtn̩] in others (Wenden, Hilmicke, Altendorf). Other items include [rɛxt] ‘right’ and [frɛxən] ‘impudent\textsc{{}-infl}’. Occasional examples in the original source also suggest that there is variation concerning the postconsonantal context, e.g. [ʃtœrçə] ‘stork-\textsc{pl}’ is realized as [ʃtœrxə] in the northern regions (recall the [rx] sequences from \ipi{Byfang}). Schulte’s study is also important because it corroborates the two patterns for targets of postsonorant fronting described above for other varieties of \il{Westphalian}Wph: Communities within the \ipi{Southeast Sauerland} can have either Target Type L or Target Type M; \citet[61-62]{Schulte1941}. For example, the words \textit{Brücke} ‘bridge’ and \textit{Rücken} ‘back’ can be realized as [bryɣə]/[bryʝə] and [riɣn̩]/[ryʝn̩] respectively. According to Schulte, pronunciation with [ɣ] (=⟦γ⟧) is typical for northern regions and [ʝ] (=⟦j⟧) in parts of the west.

Velar fronting in word-initial position in \il{Westphalian}Wph (<\ili{WGmc} \textsuperscript{+}[ɣ]) typically exemplifies Target Type LL because /x/ (<\ili{WGmc} \textsuperscript{+}[ɣ]) is the only dorsal fricative present in that context. (Recall from \sectref{sec:4.3} that \il{Westphalian}Wph -- represented by \ipi{Soest} -- underwent \isi{Wd-Initial /ɣ/-Fortition}). See \citet[66]{Jellinghaus77} and \citet[44--46]{Niebaum1977} for general discussion for Wph. A pattern that is uncommon for \il{Westphalian}Wph is attested by \ipi{Kirchspiel Courl} (Target Type MM). In that place, \ili{WGmc} \textsuperscript{+}[ɣ] is retained as /ɣ/ in word-initial position and is not restructured to /x/. /ɣ/ surfaces as [ɣ] before a back vowel and as [ʝ] before a front vowel or coronal consonant, e.g. [ɣɔːən] ‘go\textsc{{}-inf}’, [ʝɛet] ‘go\textsc{{}-3sg}’, [ʝrɑf] ‘grave’. This pattern represents Target Type MM because /ɣ/ is the only velar serving as the target for word-initial velar fronting.

Word-initial velar fronting shows variation concerning triggers. For example, in  certain parts of the \ipi{Plettenberg} region (\sectref{sec:12.6.1}), word-initial fronting of /x/ to [ç] is triggered by high front vowels (Trigger Type A), e.g. [çiətn̩] ‘eat\textsc{{}-part}’ vs. [xelt] ‘money’. By contrast, \ipi{Soest} illustrates Trigger Type BB, e.g. \ipi{Soest} [çɪstɑn] ‘yesterday’, [çɛɔs] ‘goose’ vs. [xuət] ‘good’, [xlʏkə] ‘fortune’. The same pattern obtains in the text provided (in phonetic transcription) for \ipi{Laer} \citep[155-177]{Niebaum1974}, e.g. [çɪft] ‘poison’ cs. [xɑːniks] ‘nothing at all’, [xlɑtiːs] ‘black ice’. In \ipi{Schieder-Schwalenberg} (\sectref{sec:7.2}) the set of triggers for word-initial velar fronting consists of all coronal sonorant consonants (Trigger Type CC), e.g. [çistəʀn] ‘yesterday’, [çelt] ‘money’, [çlɑs] ‘glas’ vs. [xɑfəl] ‘fork’. Note that /x/ also surfaces as [x] before the uvular rhotic, e.g. [xʀɑf] ‘grave’. \ipi{Elspe} (\sectref{sec:7.2}) exhibits the entire range of triggers (Trigger Type E), e.g. [çɛlt] ‘money’, [çæftǝ] ‘give\textsc{{}-subj}’, [çlɔftə] ‘believe\textsc{{}-pret}’ vs. [xɔlt] ‘gold’. In contrast to \ipi{Schieder-Schwalenberg}, the rhotic in \ipi{Elspe} is coronal [r], before which the palatal fricative occurs, e.g. [çrɛɔt] ‘big’.

\begin{table}
\caption{Targets and triggers for (word-initial) velar fronting in \il{Westphalian}Wph (<\ili{WGmc} \textsuperscript{+}[ɣ])\label{tab:12.16}}
\begin{tabularx}{\textwidth}{llQQ}
\lsptoprule
Target & Trigger & Place & Source\\\midrule
LL & A &  \ipit{Plettenberg} & \citet{Gregory1934}\\
LL & E &  \ipit{Elspe} & \citet{Arens1908}\\
LL & BB &  \ipit{Soest} &       \citet{Holthausen1886}    \\
   &    &  \ipit{Laer} &           \citet{Niebaum1974}\\
LL & CC &  \ipit{Schieder-Schwalenberg} &  \citet{Böger1906}   \\
   &    &  \ipit{Nienberge} &            \citet{Seymour1970}\\
MM & CC &  \ipit{Kirchspiel Courl} & \citet{Beisenherz1907}\\
\lspbottomrule
\end{tabularx}
\end{table}

\pagebreak

Not reflected in \tabref{tab:12.16} is postsibilant [x] (<\ili{WGmc} \textsuperscript{+}[sk]). In general, that [x] surfaces as velar even if a front vowel follows (see \citealt{Hall2021} for extensive discussion). This is the pattern attested in \ipi{Diemelsee}, e.g. [ʃxiːp] ‘ship’, \ipi{Plettenberg}, e.g. [sxiɛp] ‘ship’, \ipi{Gütersloh}, e.g. [ʃxyt] ‘shoot-\textsc{3sg}’, and \ipi{Laer}, e.g. [sxøin] ‘beautiful’. In \ipi{Elspe}, [x] surfaces as [ç] after word-initial [s] if a front vowel or coronal consonant follows that sound, e.g. [ʃçyt] ‘shoot-\textsc{3sg}’, [ʃçrɑpn̩] ‘scrape\textsc{{}-inf}’ vs. [ʃxʊɣn̩] ‘dread\textsc{{}-inf}’. By contrast, in \ipi{Borken} the /x/ in question surfaces as [ç] before a front vowel and as [x] when followed by a back vowel or coronal consonant, e.g. [sçip] ‘ship’, [sçæːməl] ‘stool’ vs. [sxɑp] ‘cupboard’, [sxrubbm̩] ‘scrub\textsc{{}-inf}’. In the same dialect word-initial [ɣ] does not undergo fronting, e.g. [ɣæːl] ‘yellow’. Hence, for \ipi{Borken} we have target Type L and Trigger Type D, but only for /x/ following a word-initial \isi{sibilant}.

\subsubsection{Eastphalian}\largerpage
Typical for this region is Target Type M, although Target Type L is also well-represented. Target Type M is attested in four places discussed earlier, namely \ipi{Magdeburger Börde} (\sectref{sec:4.4}), \ipi{Eilsdorf} (\sectref{sec:8.3}), \ipi{Dorste} (\sectref{sec:4.4}), and \ipi{Dingelstedt am Huy} (\sectref{sec:8.4}). Target Type L is exemplified by \ipi{Meinersen} (\sectref{sec:4.3}), \ipi{Börßum} (\sectref{sec:4.3}), and \ipi{Lesse} (\sectref{sec:8.3}), e.g. \ipi{Meinersen} [dɑxt] ‘wick’, [slɛçt] ‘bad’ vs. [vɑːɣn̩] ‘car’, [geːɣn̩] ‘around’; \ipi{Börßum} [lʊxt] ‘air’, [bɪçtə] ‘confession’, [mɑrçt] ‘market’ vs. [zeːɣn̩] ‘say{}-\textsc{inf}’.

In contrast to \il{Westphalian}Wph, the set of triggers for postsonorant velar fronting in all but one of the \il{Eastphalian}Eph sources consists of all coronal sonorants (if present). An \il{Eastphalian}Eph variety illustrating Trigger Type E is \ipi{Lesse}, e.g. [slɛçt] ‘bad’, [væːç] ‘way’, [bɑlç] ‘brat’ vs. [lɑxn̩] ‘laugh-\textsc{inf}’, [brʏɣə] ‘bridge’, [fɔːɣl̩] ‘bird’.

The one place in the \il{Eastphalian}Eph region characterized by a more restricted set of triggers (and targets) is the area around \ipi{Celle}, documented in ACeM. It is clear from the maps in that source that velar fronting is active in postsonorant position. For example, the map for \textit{wenig} ‘few’ on p. 133 shows realizations throughout the region with the symbol for a fortis palatal fricative after the front vowel [i]. Other examples discussed in that source reveal the occurrence of [x] after back vowels (e.g. pp. 44-45). Several items ACeM document the occurrence of the lenis velar fricative [ɣ] in the context after back vowels and front vowels (e.g. [ɣ] after [ɑi] and before another vowel in the word \textit{fliegen} ‘fly-\textsc{inf}’, p. 139); hence the region is characterized by Target Type L. The area in and around \ipi{Celle} is worthy of note because of the context after a coronal sonorant consonant. According to the map for \textit{Berg} ‘mountain’ (p. 59), the final segment is pronounced as [x] after [r] throughout the entire region, while [..rç..] is restricted to the town of Nordburg (ca. 15km southeast of \ipi{Celle}). The realization of /x/ as [x] after [r] is also attested in the same area for the word \textit{durch} ‘through’, p, 217, although [ç] (from /x/) also occurs to the north. I conclude that the region around \ipi{Celle} was characterized by Target Type L and the rare Trigger Type BB.

\begin{table}
\caption{Targets and triggers for (postsonorant) velar fronting in \il{Eastphalian}Eph (<\ili{WGmc} \textsuperscript{+}[x ɣ])\label{tab:12.17}}
\begin{tabularx}{\textwidth}{llQQ}
\lsptoprule
Target & Trigger & Place & Source\\\midrule
L & E &  \ipit{Lesse} & \citet{Löfstedt1933}\\
L & BB &  \ipit{Celle} &  {ACeM}\\
L & EE & Kreis \ipi{Hannover} & \citet{BethgeBonnin1969}\\
L & CC &  \ipit{Meinersen} &\citet{Bierwirth1890}\\
  &     &  \ipit{Börßum} &                     \citet{Heibey1891}\\
M & CC &  \ipit{Eilsdorf}                 &     \citet{Block1910}   \\
  &    &  \ipit{Dorste}                  &     \citet{Dahlberg1937}\\
  &    &  \ipit{Emmerstedt}              &     \citet{Brugge1944}  \\
  &    &  \ipit{Göddeckenrode}/\ipi{Isingerode} &     \citet{Lange1963}\\
M & EE &  \ipit{Magdeburger Börde}    & \citet{Roloff1902}\\
  &      &  \ipit{Dingelstedt am Huy}    & \citet{Hille1939}\\
LL & CC &  \ipit{Braunschweig}         & \citet{Pahl1943}           \\
   &    &  \ipit{Mascherode}            & \citet{BethgeFlechsig1958} \\
   &    &Kreis \ipi{Wolfenbüttel}    & \citet{BethgeBonnin1969}\\
\lspbottomrule
\end{tabularx}
\end{table}

Word-initial velar fronting is not present in many varieties of \il{Eastphalian}Eph because historical /ɣ/ was restructured to /g/ ([g]) by \isi{g-Formation-1} (\sectref{sec:4.2}), e.g. \ipi{Börßum} (\sectref{sec:4.3}), [gluːɔbm̩] ‘believe-\textsc{inf}’, [guːɔt] ‘good’, [gæl] ‘yellow’. This generalization also holds for the region around \ipi{Celle} in the maps (pp. 221, 223, 291) provided in ACeM. In those places where \ili{WGmc} \textsuperscript{+}[ɣ] is retained as /ɣ/ in word-initial position, velar fronting applies (Target Type MM), e.g. \ipi{Lesse} [ʝeːm̩] ‘give\textsc{{}-inf}’, [ɣɑf] ‘give\textsc{{}-pret}’. That pattern is much more prevalent in \il{Eastphalian}Eph than in \il{Westphalian}Wph, which prefers Target Type LL. Recall from \sectref{sec:8.5} that \ipi{Dingelstedt am Huy} has alternations in word-initial position between [g] (before a back vowel or consonant) and [ʝ] (before a front vowel). Those alternations derive from /ɣ/, which surfaces as [ʝ] by word-initial velar fronting (Target Type MM, Trigger Type B). That same type of example is also attested in \ipi{Cattenstedt} (\il{Eastphalian}Eph). Target Type LL is attested in \ipi{Dorste} (Trigger Type BB), e.g. [çɛlt] ‘money’ vs. [xlɑs] ‘glass’, [xɔt] ‘God’. A similar pattern obtains in \ipi{Kamschlaken}, e.g. [çift] ‘poison’, [çeːm] ‘give-\textsc{inf}’ vs. [xɑlə] ‘gall bladder’, [xlygə] ‘fortune’. \ipi{Reinhausen} (\sectref{sec:7.2}) is also Target Type LL, although that dialect shows that coronal sonorant consonants also induce fronting (Trigger Type C), e.g. [çɛlt] ‘money’, [çliːk] ‘same’ vs. [xɔt] ‘God’.\largerpage

\begin{table}
\caption{Targets and triggers for (word-initial) velar fronting in \il{Eastphalian}Eph (<\ili{WGmc} \textsuperscript{+}[ɣ])\label{tab:12.18}}
\begin{tabularx}{\textwidth}{llQQ}
\lsptoprule
Target & Trigger & Place & Source\\\midrule
LL & BB &  \ipit{Dorste}   &  \citet{Dahlberg1937}\\
   &   &  \ipit{Kamschlaken} &   \citet{Göschel1973}\\
LL & C &  \ipit{Reinhausen} & \citet{Jungandreas1926,Jungandreas1927}\\
MM & BB &  \ipit{Magdeburger Börde}         & \citet{Roloff1902}     \\
   &    &  \ipit{Eilsdorf}                    & \citet{Block1910}      \\
   &    &  \ipit{Cattenstedt}                 & \citet{Damköhler1919}  \\
   &    &  \ipit{Lesse}                       & \citet{Löfstedt1933}   \\
   &    &  \ipit{Dingelstedt am Huy}          & \citet{Hille1939}      \\
   &    &  \ipit{Braunschweig}                & \citet{Pahl1943}       \\
   &    &  \ipit{Göddeckenrode}/ \ipi{Isingerode}   & \citet{Lange1963}\\
\lspbottomrule
\end{tabularx}
\end{table}

\citet{Hassel1942} offers an overview of the \il{Eastphalian}Eph dialect spoken in towns and villages in the area south of Göttingen in the Werra Valley (\ipi{Werratal}). That study is significant because it shows with maps that two different Trigger Types are attested directly next to one another. Hassel observes that \ili{WGmc} \textsuperscript{+}[ɣ] is realized in word-initial position in the north of the \ipi{Werratal} as [x] before back vowels and as [ç] before front vowels (e.g. [xolt] ‘gold’ vs. [çistərn] ‘yesterday’). In the context before a consonant there are two attested outcomes: In one cluster of towns the realization is [x], and in others it is [ç], e.g. [xliːk]/[çliːk] ‘fortune’; see \citet[65-67]{Hassel1942}. In terms of the present classification those places with the realization [xliːk] have Trigger Type BB and those with the pronunciation [çliːk] Trigger Type CC.

As indicated in the heading for \tabref{tab:12.18}, the target segment for word-initial velar fronting derives historically from \ili{WGmc} \textsuperscript{+}[ɣ]. No variety of \il{Eastphalian}Eph has been found in which word-initial velar fronting affects the original velar in \ili{WGmc} \textsuperscript{+}[sk] clusters \citep{Hall2021}.

None of the \il{Eastphalian}Eph dialects in the present study are attested which have a word-initial dorsal fricative that always surfaces as velar regardless of the nature of the following sound; recall the \il{Westphalian}Wph dialect once spoken in \ipi{Gütersloh} \citep{Wix1921}.

\subsection{East Low German}\label{sec:12.3.7}
\subsubsection{General remarks}\largerpage
Velar-fronting is active throughout this region, although previous chapters have documented various places within ELG where that rule is characterized by various quirks. One anomaly not mentioned earlier is the \il{East Pomeranian}EPo variety described by \citet{Stritzel1937}, in the region surrounding the town of \ipi{Lauenburg} (Kreis \ipi{Lauenburg} and Kreis \ipi{Stolp}; \mapref{map:18}). Stritzel’s material contains an oddity otherwise unattested in \il{East Pomeranian}EPo. In particular, \citet[55]{Stritzel1937} documents a small enclave where [x] surfaces consistently as [x] regardless of the nature of the preceding sound. He writes: “Der NW der Landschaft hat die Eigenart, jedes palatale χ der angrenzenden Mda. als gutturales x zu sprechen”. (“The northwest of the region has the peculiarity of pronouncing every palatal χ in the bordering dialects as guttural x”). According to Stritzel’s Maps 16 and 21, those non-fronting varieties in the northwest occur in a number of communities in Kreis \ipi{Stolp}, while velar fronting areas include Kreis \ipi{Lauenburg} and Kreis \ipi{Bütow} (\mapref{map:18}). Examples include [nɑːxt]/[nɔːxt] ‘night’, [liːxt] ‘light’ and [ʃlɛxt] ‘bad’ (where [x] derives from /x/) as well as [krɪxt] ‘get\textsc{{}-3sg}’, [zɛxt] ‘say\textsc{{}-3sg}’, where [x] derives from /ɣ/.

\begin{sloppypar}
This one conservative \isi{non-velar fronting island} aside, postsonorant velar fronting is active throughout ELG. Recall from \chapref{sec:11} that word-initial velar fronting is also attested in various places in this region. In terms of segments undergoing postsonorant velar fronting, there is a clear preference for Target Type M, but \chapref{sec:11} documented several varieties with a broader set of target segments (Target Type N). One  rare pattern for this area is Target Type L, which is only attested in two places in the present survey (see below).
\end{sloppypar}

In the sources cited here the triggers for postsonorant velar fronting consist of all front vowels and coronal consonants, but one variety commented on below is attested in which coronal consonants (/r/) fail to induce postsonorant velar fronting.

I consider consider first \il{Brandenburgish}Brb/\il{Mecklenburgish-West Pomeranian}MeWPo and summarize the generalizations concerning targets and triggers in Tables \ref{tab:12.19} and \ref{tab:12.20}. I conclude this section by summarizing the patterns attested for \il{East Pomeranian}EPo and \il{Low Prussian}LPr.{\interfootnotelinepenalty=10000\footnote{The one ELG subdivision I do not discuss is \il{Central Pomeranian}CPo. There is general agreement in the literature on German dialectology that \il{Central Pomeranian}CPo is a region not quite the same as the neighboring ones (e.g. \citealt{Wiesinger1983a}, \citealt{Schönfeld1989}), but there is sadly a dearth of detailed studies on the structure of \il{Central Pomeranian}CPo (see \citealt{WiesingerRaffin1982}: 379-380). The only sources for \il{Central Pomeranian}CPo indicated on \mapref{map:17} are \citet{Brose1955} and \citet{Prowatke1973}. On the basis of the phonetic transcriptions in both of those works it can be concluded that postsonorant velar fronting is active for the target /x/ and that the triggers are front vowels. The modern reflex of \ili{WGmc} \textrm{\textsuperscript{+}}\textrm{[ɣ] for Brose’s speakers is [g] in word-initial and postsonorant position. \citet[77]{Prowatke1973} observes that \ili{WGmc}} \textrm{\textsuperscript{+}}[ɣ] is often realized as [ʝ] in word-initial position.}}

\begin{table}[h]
\caption{Targets and triggers for (word-initial) velar fronting in \il{Brandenburgish}Brb and \il{Mecklenburgish-West Pomeranian}MeWPo (<\ili{WGmc} \textsuperscript{+}[k ɣ])\label{tab:12.20}}
\begin{tabularx}{\textwidth}{llQQ}
\lsptoprule
Target  & Trigger & Place & Source\\\midrule
N & EE &  \ipit{West Mecklenburg} & \citet{Kolz1914}\\
MM & BB &  \ipit{Neumark}     & \citet{Teuchert1907a,Teuchert1907b}   \\
   &     &  \ipit{Neu-Golm}   &  \citet{Siewert1912}\\
\lspbottomrule
\end{tabularx}
\end{table}

\begin{table}
\caption{Targets and triggers for (postsonorant) velar fronting in \il{Brandenburgish}Brb and \il{Mecklenburgish-West Pomeranian}MeWPo (<\ili{WGmc} \textsuperscript{+}[x ɣ])\label{tab:12.19}}
\begin{tabularx}{\textwidth}{llQQ}
\lsptoprule
Target  & Trigger & Place & Source\\\midrule
L & EE &  \ipit{Lüneburger Wendland} & \citet{Selmer1918}\\
  &    &  \ipit{Rebenstorf} (Lübbow) & \citet{Götze1922}\\
M & E &  \ipit{Jerichower Land} & \citet{Bathe1932}\\
M & AA &  \ipit{Hinzdorf} (Wittenberge) & \citet{Bretschneider1951}\\
M & CC &  \ipit{Neumark}    & \citet{Teuchert1907a,Teuchert1907b} \\
  &    &Warte               &  \citet{Teuchert1907c}              \\
  &    &  \ipit{Besten}              &  \citet{Siewert1907}                \\
  &    &  \ipit{Prenden}             &  \citet{Seelmann1908}               \\
  &    &  \ipit{Neu-Golm}            &  \citet{Siewert1912}                \\
  &    &  \ipit{South Stargard}      &  \citet{Teuchert1934}               \\
  &    &  \ipit{Tempelfelde}         &  \citet{Schönfeld1989}\\
M & EE &  \ipit{Magdeburg}&   \citet{Krause1895}      \\
  &    &  Kaarβen &       \citet{Dützmann1932}\\
N & EE &  \ipit{West Mecklenburg} & \citet{Kolz1914}\\
LL & BB &  \ipit{Wolgast} & \citet{Warnkross1912}\\
LL & CC &  \ipit{Stargard}      & \citet{Blume1933, Blume1933b, Blume1933c, Blume1933d}        \\
   &    &  \ipit{Arendsee}    &  \citet{Törnqvist1949}   \\
   &    &  \ipit{Heckelberg}  &  \citet{Teuchert1964}    \\
   &    &  \ipit{Schollene}   &  \citet{Schönfeld1965}   \\
   &    &   Kreis \ipi{Wismar}&  \citet{BethgeBonnin1969}\\
LL & DD &  \ipit{South Mecklenburg} & \citet{Jacobs1925a,Jacobs1925b,Jacobs1926}\\
LL & EE &  \ipit{Ivenack-Stavenhagen}           &  \citet{Holst1907}    \\
   &    &  \ipit{Barth}                         &   \citet{GSchmidt1912}\\
   &    & Kreise Arnswalde, Friedeberg  &   \citet{Seelmann1913}\\
   &    &  \ipit{Greifswald}, \ipi{Schwerin} &            \citet{Prowatke1973}\\
\lspbottomrule
\end{tabularx}
\end{table}


\subsubsection{Brandenburgish and Mecklenburgish-West Pomeranian}
Target Type M for postsonorant velar fronting is represented by \ipi{Neu-Golm} (\il{Brandenburgish}Brb), e.g. [heːçtə] ‘height’, [bɑlç] ‘bellows’, [bɛːlʝə] ‘bellows-\textsc{pl}’ vs. [lɑxn̩] ‘laugh\textsc{{}-inf}’, [foːɣl̩] ‘bird’. The two attested cases of Target Type L are found in the westernmost region of this dialect area, namely in the \ipi{Rebenstorf} and \ipi{Lüneburger Wendland}, e.g. \ipi{Lüneburger Wendland} [myç] ‘mosquito’, [rɛçt] ‘right’ vs. [lɑxn̩] ‘laugh\textsc{{}-inf}’, [mɑɣɐ] ‘lean’, [nɛːɣəl] ‘nail’.\footnote{In the pronunciation guide to TeWb for the \ipi{Teltow} dialect there is a clear description of the realization of velar fricatives as palatal, which corresponds to Target Type M and Trigger Type~E (pp. 300--301).}

Front vowels (including low front vowels if present) induce velar fronting, e.g. \ipi{South Mecklenburg} [vɛç] ‘path’, [væːç] ‘paths’ vs. [tʊxt] ‘breeding’, [oːx] ‘eye’.  The rare case of Target Type BB is attested in \ipi{Wolgast}, e.g. e.g. [pliːçt] ‘duty’, [zɛç] ‘say-\textsc{part}’, [brøːç] ‘bridge’ vs. [dox] ‘day’, [bɑlx] ‘brat-\textsc{dat}.\textsc{sg}’.

\citegen{Bretschneider1951} description of the \il{Brandenburgish}Brb variety of \ipi{Hinzdorf} (Wittenberge) is significant because her discussion of the velar and palatal fricatives reveals that there is a low front vowel which does not trigger velar fronting (=Trigger Type AA). She writes (p. 97): “Zu beachten ist besonders, daß überoffenes e, mit ä bezeichnet, als gutturaler Laut dem ach-Laut verbunden ist ...”. (“Attention should be paid to [the fact that] the over-open e, indicated as ä, is connected with the guttural sound, the ach-Laut”). For example, the ⟦ch⟧ in ⟦sächt⟧ ‘say-\textsc{3} \textsc{sg}’ is phonetically [x].

In word-initial position \ili{WGmc} \textsuperscript{+}[ɣ] is realized in some \il{Brandenburgish}Brb varieties as [ʝ] in the context before front vowels  (Target Type BB). The more common change from velar to [ʝ] in word-initial position before all segments (including consonants and back vowels) for \il{Brandenburgish}Brb and other dialect regions is discussed at length in \chapref{sec:14}. Word-initial fronting (Target Type MM, trigger Type BB) is attested in \ipi{Neu-Golm} and \ipi{Neumark} (both \il{Brandenburgish}Brb); e.g. \ipi{Neu-Golm} [gɑns] ‘goose’ vs. [ʝenzə] ‘goose-\textsc{pl}’; \ipi{Neumark} [goːn] ‘go\textsc{{}-inf}’ vs. [ʝeːst] ‘go\textsc{{}-2sg}’. The complex pattern of word-initial velar fronting in \ipi{West Mecklenburg} was discussed at length in \sectref{sec:11.3}. 

\subsubsection{East Pomeranian and Low Prussian} 
This area is diverse in terms of variation for targets and triggers (\sectref{sec:11.5}, \sectref{sec:11.6}). Velar fronting places in this region typically select the target segments from the set of velar consonants in \REF{ex:12:3b}; recall \tabref{tab:11.1}. 

\begin{table}
\caption{Targets and triggers for (postsonorant) velar fronting in \il{East Pomeranian}EPo and \il{Low Prussian}LPr (<\ili{WGmc} \textsuperscript{+}[k x ɣ ŋ])\label{tab:12.21}}
\begin{tabular}{llll}
\lsptoprule
Target & Trigger & Place & Source\\\midrule
M & CC &  \ipit{Königsberg}  & \citet{Mitzka1919}\\
  &    & Kreis \ipi{Schlawe}   & \citet{Mahnke1931}\\
  &    &  \ipit{Mandtkeim}       &\citet{Bink1953} \\
N & C &  \ipit{Kamnitz} & \citet{Tita19211965}\\
N & E & Kreis \ipi{Bütow}   & \citet{Mischke1936}\\
  &   &  \ipit{Willuhnen} &  \citet{Natau1937}\\
N & CC & Kreis \ipi{Konitz} & \citet{Semrau1915a,Semrau1915b}\\
  &    &  \ipit{Sępóno Krajeńskie} & \citet{Darski1973}\\
N & EE &  \ipit{Lauenburg} & \citet{Pirk1928}\\
\lspbottomrule
\end{tabular}
\end{table}

\begin{table}
\caption{Targets and triggers for (word-initial) velar fronting in \il{East Pomeranian}EPo and \il{Low Prussian}LPr (<\ili{WGmc} \textsuperscript{+}[k x ɣ ŋ])\label{tab:12.22}}
\begin{tabularx}{\textwidth}{llQQ}
\lsptoprule
Target & Trigger & Place & Source\\\midrule
N & D &  \ipit{Lauenburg}   &  \citet{Pirk1928}        \\
  &   &  \ipit{Kamnitz}   &    \citet{Tita19211965}  \\
  &   &  \ipit{Willuhnen} &    \citet{Natau1937}\\
N & BB & Kreis \ipi{Konitz}     &     \citet{Semrau1915a,Semrau1915b}   \\
  &      &  \ipit{Sępóno Krajeńskie} &     \citet{Darski1973}\\
MM & BB & Kreis \ipi{Bütow}         &   \citet{Mischke1936}  \\
   &    &     Kreis \ipi{Rummelsburg} &   \citet{Mischke1936}\\
MM & CC &  \ipit{Königsberg}    & \citet{Mitzka1919}  \\
   &    &  \ipit{Mandtkeim}   &  \citet{Bink1953}\\
\lspbottomrule
\end{tabularx}
\end{table}


The dictionary for the Pommern (Pomerania) dialect (PWb) provides a brief statement on the realization of \ili{WGmc} \textsuperscript{+}[ɣ] in word-initial position (Volume 1: 891) in a broad area defined as the former province of Pomerania (\mapref{map:45}). According to that statement, the etymological lenis velar fricative is typically realized as a palatal ([ʝ]) in the context before front vowels (=Trigger Type D or BB). In another area, \ili{WGmc} \textsuperscript{+}[ɣ] is pronounced palatal before front vowels and [d] before liquids (=Trigger Type C or CC). The change from [ʝ] to [d] before [l r] necessitates a separate rule. No mention is made in PWb of velar noncontinuants serving as targets; hence, Target Type MM (and not target Type N) holds for all areas with word-initial velar fronting.

\section{{Areal} {distribution} {of} {trigger} {and} {target} {types}}\label{sec:12.4}

I present four maps below which indicate the distinction between various velar fronting triggers (\sectref{sec:12.4.1}) and targets (\sectref{sec:12.4.2}). An examination of those maps should indicate the difficulty of drawing isoglosses separating Trigger/Target Types. In contrast to well-known textbook examples in which targets and triggers for other changes correspond to discreet areas separated by large distances (e.g. \isi{High German Consonant Shift}), the areal distribution for the various velar fronting patterns does not always give a clean picture. The way in which the German dialects discussed below shed light on where velar fronting was phonologized is delayed until \sectref{sec:16.4}.

\subsection{Velar fronting triggers}\label{sec:12.4.1}

As indicated in \REF{ex:12:2}, there are three Trigger Types that have not been recognized in the small literature on velar fronting in German dialects (e.g. \citealt{Herrgen1986}, \citealt{Robinson2001}), namely the restriction of fronting to the context of either (a) high front vowels, (b) nonlow front vowels, or (c) front vowels to the exclusion of front (coronal) consonants. I consider each in turn in light of the present survey.

The high front vowel context is the rarest of all Trigger Types, since it is attested only two geographically noncontiguous varieties of German (in both word-initial position and postsonorant position), namely \ipi{Visperterminen} (\il{Highest Alemannic}HstAlmc) and \ipi{Plettenberg} (\il{Westphalian}Wph).

Although the set of nonlow front vowels as triggers is robustly attested, that type of dialect is considerably less preferred to those in which all front vowels trigger fronting. \mapref{map:12:20} indicates the two types of dialect referred to here for the postsonorant context.

\begin{map}[ph]
% \includegraphics[width=\textwidth]{figures/VelarFrontingHall2021-img026.png}
\includegraphics[width=\textwidth]{figures/Map20_12.1.pdf}
  \caption[Areal distribution of low front vowels as velar fronting triggers]{Areal distribution of low front vowels as velar fronting triggers. Varieties of High German and Low German in which low front vowels do not serve as triggers for postsonorant velar fronting are indicated with white squares. Varieties in which low front vowels serve as triggers are indicated with black squares.}
  \label{map:12:20}\label{map:20}
\end{map}\clearpage

\mapref{map:12:20} reveals that those places in which low front vowels do not induce fronting (white squares) are clustered primarily in the west, from as south as Switzerland to as far north as Rhineland and Lower Saxony. Recall from \REF{ex:12:2b} that this pattern reflects Trigger Types A, B, C, and AA.\footnote{{Included among the white squares is one variety (\il{East Hessian}EHes) discussed in \sectref{sec:12.3.4} that I did not place in any of the tables, namely the area to the west of \ipi{Bad Hersfeld} \citep{Martin1957}.}} The more numerous and geographically well-distributed dialects are those in which all front vowels (including low front vowels) serve as triggers (black squares). Those dialects display Trigger Types D, E, and DD.

\chapref{sec:13} assesses the state of velar fronting in Lower Bavaria on the basis of data from 221 villages, towns, and cities drawn from a linguistic atlas (SNiB). It is demonstrated in that chapter that the places within Lower Bavaria can differ according to Trigger Type. In particular, it is shown that the rarest Trigger Type referred to above (high front vowels) is the most common one, while the one with the largest set of triggers (all front vowels) is the rarest.

A dichotomy can be drawn between those dialects in which coronal sonorant consonants (e.g. /l/, /r/) do or do not trigger velar fronting (recall \ref{ex:12:2c}). The type of dialect  in which such sounds fail to trigger velar fronting in postsonorant position is rare; the present survey has uncovered fifteen; see Trigger Types A, B, D, and BB in the tables presented earlier.\footnote{{Three of those places are not listed in the tables. See the discussion in \sectref{sec:12.3.2} on the \il{Swabian}Swb varieties discussed by \citet{Haag1898} and in \sectref{sec:12.3.6}  on the \il{Westphalian}Wph ones by \citet{Schulte1941}. The third is \ipi{Mühlingen} \citep{Müller1911}. Due to various complexities the data from that \il{Swabian}Swb variety cannot be discussed until \sectref{sec:14.3.2}.}} By contrast, the inclusion of coronal sonorant consonants among the triggers for postsonorant velar fronting is clearly the unmarked pattern (attested in 95 varieties of German). \mapref{map:12:21} depicts those rare varieties in which coronal sonorant consonants do not serve as triggers.

\begin{map}[p]
% \includegraphics[width=\textwidth]{figures/VelarFrontingHall2021-img027.png}
\includegraphics[width=\textwidth]{figures/Map21_12.2.pdf}
\caption[Areal distribution of coronal sonorant consonants as triggers for postsonorant velar fronting]{Areal distribution of coronal sonorant consonants as triggers for postsonorant velar fronting. Varieties of High German and Low German in which coronal sonorant consonants (e.g. /l/, /r/) do not serve as triggers are indicated with white squares.}
\label{map:12:21}\label{map:21}
\end{map}

As a general rule, the realization of /x/ as [ç] after coronal consonants is the one documented in dialect dictionaries, regardless of region. The only exception to my knowledge is the dictionary for \ipi{Dortmund} (DoWb), which provides a clear statement (p. XVIII) inferring that [ç] occurs after front vowels and [x] after back vowels or consonants ([l]). \ipi{Dortmund} is indicated on \mapref{map:6}.

For further discussion on the status of consonants like [l] and [r] as triggers for postsonorant velar fronting the reader is referred to \sectref{sec:13.5.2} and \chapref{sec:15}.

Word-initial position illustrates the opposite distribution: 22 varieties are attested in which velar fronting is not induced by coronal sonorant consonants (=Trigger Types A, B, D, BB), but only eight have been discovered in which those segments do serve as a trigger. The areal distribution of those two types of dialect are indicated on \mapref{map:12:22} for word-initial position.

\begin{map}[p]
% \includegraphics[width=\textwidth]{figures/VelarFrontingHall2021-img028.png}
\includegraphics[width=\textwidth]{figures/Map22_12.3.pdf}
\caption[Areal distribution of coronal sonorant consonants as triggers for word-initial velar fronting]{Areal distribution of coronal sonorant consonants as triggers for word-initial velar fronting. Varieties of High German and Low German in which coronal sonorant consonants (e.g. /l/, /r/) do not serve as triggers are indicated with white squares. Varieties in which coronal sonorant consonants serve as triggers are indicated with black squares.}
\label{map:12:22}\label{map:22}
\end{map}\clearpage\largerpage[2]

\mapref{map:12:22} reveals that most of the rare varieties where sounds like /l/ are triggers (black squares) are clustered in the west central region of Germany. The more common pattern (white squares) is well-attested in Central/North Germany.

\subsection{Velar fronting targets}\label{sec:12.4.2}

There are systems in which only fortis /x/ but not lenis /ɣ/ undergoes velar fronting (Target Type L) as well as ones in which both /x ɣ/ serve as targets for that process (Target Type M).  \mapref{map:23} indicates the areal distribution of both types of dialect for postsonorant velar fronting. As indicated there, Target Type L is well-attested (twenty-two), although Target Type M is far more common (forty-three).

\begin{map}[p]
% \includegraphics[width=\textwidth]{figures/VelarFrontingHall2021-img029.png}
\includegraphics[width=\textwidth]{figures/Map23_12.4.pdf}
\caption[Areal distribution of velar fricatives as targets for velar fronting]{Areal distribution of velar fricatives as targets for velar fronting. Varieties of High German and Low German in which /x/ but not /ɣ/ serve as targets for postsonorant velar fronting (Target Type L) are indicated with white squares. Varieties in which both /x/ and /ɣ/ serve as targets (Target Type M) are indicated with black squares.}
\label{map:12:23}\label{map:23}
\end{map}

It can be observed that Target Type L (white squares) is well-represented in the central and northern parts of Germany (WLG) with only a few attestations further south. Target Type M (black squares) reveals a much broader distribution among German dialects (\il{Westphalian}Wph, \il{Eastphalian}Eph, \il{Silesian}Sln, \il{Moselle Franconian}MFr, \il{Rhenish Franconian}RFr, \il{East Pomeranian}EPo, \il{Low Prussian}LPr).

German dialects in which velar fronting affects the entire class of velar consonants (Target Type N) are clustered in the northeast of 1914 Germany. It was noted in \chapref{sec:11} that that type of dialect (Target Type N) can be contrasted with communities in the same region with a more restricted set of segments undergoing velar fronting (Target Type M). Those two types of systems are plotted on \mapref{map:19}.

\section{{Rule} {generalization}}\label{sec:12.5}

I consider first the way in which the attested Target Types and Trigger Types match up with historical stages (\sectref{sec:12.5.1}) and then illustrate how those stages are reflected in certain clusters of dialects spoken in the same region (\sectref{sec:12.5.2}). A more in-depth discussion of how the various Target Types and Trigger Types can shed light on the relative age of velar fronting in certain regions is presented in \chapref{sec:16}. \chapref{sec:13} looks at velar fronting throughout Lower Bavaria, showing how the three attested Trigger Types in that region can be interpreted historically in terms of \isi{rule generalization}.

\subsection{Triggers, targets, and historical stages}\label{sec:12.5.1}

In \tabref{tab:12.23} I repeat the Trigger Types listed in the first five rows of \tabref{tab:12.1} and show how they correspond to the historical stages referred to throughout the remainder of this book. It is demonstrated here that Stage 2 is subdivided into a series of incremental stages defined according to Trigger Type.

\begin{table}
\caption{Trigger Types and the corresponding historical stages\label{tab:12.23}}
\begin{tabular}{lll}
\lsptoprule
Type & Trigger & Stage\\\midrule
A & HFV & 2a\\
B & HFV, MFV & 2b\\
C & HFV, MFV, CC & 2c\\
D & HFV, MFV, LFV & 2c'\\
E & HFV, MFV, LFV, CC & 2d\\
\lspbottomrule
\end{tabular}
\end{table}


Stages 2a, 2b, 2c, 2d proceed chronologically in that order. Stage 2c{}' is coterminous with Stage 2c. The reason is that Stage 2b includes only \{HFV, MFV\} as triggers, at which point there is the option of expanding those triggers to include \{CC\} (=Trigger Type C=Stage 2c) or \{LFV\} (=Trigger Type D=Stage 2c{}').

The Trigger Types in the final five rows of \tabref{tab:12.1} have in common that each one has at least one segment type not present in the context for velar fronting and hence there is indeterminacy concerning how those Trigger Types fit into the historical stages in \tabref{tab:12.24}. For example, for Trigger Type AA, the coronal consonant trigger (\{CC\}) is not present in the fronting context; hence, that Trigger Type could be either Stage 2b (\{HFV, MFV\}) or Stage 2c (\{HFV, MFV, CC\}). In \tabref{tab:12.24} the five Trigger Types referred to here are matched to the historical stages from \tabref{tab:12.23}.

\begin{table}
\caption{Possible Trigger Types and the corresponding historical stages\label{tab:12.24}}
\begin{tabular}{ll}
\lsptoprule
Trigger & Stage\\\midrule
AA & 2b or 2c\\
BB & 2b or 2c'\\
CC & 2c or 2d\\
DD & 2c' or 2d\\
EE & 2b, 2c, 2c', or 2d\\
\lspbottomrule
\end{tabular}
\end{table}

In \tabref{tab:12.25} I present Target Types L, M, and N from \tabref{tab:12.3} and the corresponding historical stages. Due to gaps described earlier, the two remaining Target Types (LL/MM) cannot be unambiguously classified into one of the three stages listed in the final column of \tabref{tab:12.25}. The possible historical stages for Target Types LL/MM are listed in \tabref{tab:12.26}.

\begin{table}
\caption{Target Types and the corresponding historical stages\label{tab:12.25}}
\begin{tabular}{lll}
\lsptoprule
Type & Target & Stage\\\midrule
L & /x/ & 2aa\\
M & /x/, /ɣ/ & 2bb\\
N & /x/, /ɣ/, /k/, /g/, /ŋ/ & 2cc\\
\lspbottomrule
\end{tabular}
\end{table}

\begin{table}
\caption{Possible Target Types and the corresponding historical stages\label{tab:12.26}}
\begin{tabular}{ll}
\lsptoprule
Target & Stage\\\midrule
LL & 2aa or 2bb\\
MM & 2bb or 2cc\\
\lspbottomrule
\end{tabular}
\end{table}

The historical stages for triggers (\tabref{tab:12.23}) and targets (\tabref{tab:12.25}) are independent of one another. This point is illustrated in \figref{fig:12:1}.

\begin{figure}[H]
\hfill\begin{tabular}[t]{cc}
\multicolumn{2}{c}{Triggers:}\\
\multicolumn{2}{c}{Stage 2a} \\ 
 \multicolumn{2}{c}{↓}       \\  
\multicolumn{2}{c}{Stage 2b} \\  
     ↙      &    ↘           \\  
  Stage 2c & Stage 2c'       \\  
  ↘      &    ↙ \\
\multicolumn{2}{c}{Stage 2d}
\end{tabular}\hfill\begin{tabular}[t]{c}
 Targets:\\
Stage 2aa\\
 ↓ \\
Stage 2bb\\
 ↓ \\
Stage 2cc\\
 \end{tabular}\hfill\hbox{}
\caption{Historical stages for triggers and targets\label{fig:12:1}}
\end{figure}

Stage 2aa (/x/ as the sole target) could cooccur with any one of the stages for triggers, as could Stage 2bb (/x ɣ/) and Stage 2cc (/x ɣ k g ŋ/).

\subsection{Historical stages in selected areas}\label{sec:12.5.2}

Velar fronting was phonologized first in the context of high front vowels (Stage 2a), at which point that group of triggers gradually expanded. The same progression occurred in targets from narrow to broad (Stage 2aa > Stage 2bb > Stage 2cc). As demonstrated below certain regions can be identified in which all or some of the historical stages for triggers and targets are reflected in specific communities in relatively close proximity.

Consider first the SwG dialects (\sectref{sec:12.3.1}), which exemplify the various historical stages corresponding to different Trigger Types for both word-initial and postsonorant position. For word-initial position (\tabref{tab:12.4}) the stages are: \ipi{Visperterminen} (Stage 2a), \ipi{Obersaxen} (Stage 2b), and \ipi{Rheintal} (Stage 2c) and for postsonorant position (\tabref{tab:12.3}) they are  \ipi{Visperterminen} (Stage 2a), \ipi{Obersaxen} (Stage 2b), \ipi{Rheintal} (Stage 2c), and \ipi{Maienfeld} (Stage 2d).

For the aforementioned SwG varieties the progression from a narrow set of triggers to a broader one transpired along the time dimension, but not along the place dimension. The reason Stage 2a did not extend to Stage 2b in terms of geography is that the Stage 2 SwG dialects listed above (\isi{velar fronting islands}) are separated from one another by large distances. Given the distances among the four velar fronting SwG communities the implication is that velar fronting was phonologized independently in each of the four places (\isi{polygenesis}); hence, there were four distinct focal areas in Switzerland.

A careful scrutiny of certain regions on the locator maps presented in previous chapters reveals places in close proximity representing the various historical stages. I consider two such LG dialect clusters, namely \il{Westphalian}Wph and \il{East Pomeranian}EPo, in that order.

A reexamination of \mapref{map:6} for \il{Westphalian}Wph is instructive because it reveals a number of varieties which represent the various Trigger Types. Seven of those \il{Westphalian}Wph varieties are listed in \REF{ex:12:10}. I also include here \ipi{Grafschaft Bentheim} as representative of Stage 1. The significant point is that the velar fronting varieties are all located within an area of about 100km from north to south and 80km from east to west. As illustrated in \REF{ex:12:10}, six of the places indicated on \mapref{map:6} represent distinct historical stages for (word-initial) velar fronting Trigger Types. (In that context there is no variation in Target Type, since the only sound undergoing velar fronting is /x/). The word-initial velar fricative referred to here derived historically from \ili{WGmc} \textsuperscript{+}[ɣ], although a similar set of stages also involved the reflexes of \textsuperscript{+}[k] in \ili{WGmc} \textsuperscript{+}[sk] clusters. In \sectref{sec:14.2.2} I expand \REF{ex:12:10} by adding an additional dialect.

\ea%10
\label{ex:12:10}Historical stages for triggers for (word-initial) velar fronting (\il{Westphalian}Wph):\smallskip\\
\begin{tabular}[t]{@{}ll@{}}
Stage 1:   &  \ipit{Grafschaft Bentheim}\\
Stage 2a:  &  \ipit{Plettenberg}        \\
Stage 2b:  & (\ipi{Soest}, \ipi{Laer})      \\
Stage 2c:  & (\ipi{Nienberge})        \\
Stage 2c': & (\ipi{Borken})           \\
Stage 2d:  &  \ipit{Elspe}              \\
\end{tabular}
\z 

\begin{sloppypar}
\ipi{Plettenberg} represents the rare high front vowel trigger corresponding to Stage 2a, while \ipi{Elspe} possesses the broadest set of triggers (Stage 2d). There are no clear-cut examples of dialects representing Stage 2b, 2c, or 2c', although there are several potential ones, four of which are presented above in parentheses. What is clear from \REF{ex:12:10} is that there is a cluster of \il{Westphalian}Wph dialects in which coronal sonorant consonants (\{CC\}) do not belong to the set of triggers (\ipi{Plettenberg}, \ipi{Soest}, \ipi{Laer}, \ipi{Borken}), while other varieties fronting is induced by some subset of the front vowels or coronal sonorant consonants (\ipi{Elspe}).
\end{sloppypar}

In \REF{ex:12:11} I list the same \il{Westphalian}Wph varieties for postsonorant velar fronting. I also include \ipi{Byfang}, which represents Stage 2b. Note that the seven velar fronting varieties in \REF{ex:12:11} represent two distinct historical stages for Target Types.

\ea%11
\label{ex:12:11}Historical stages for triggers and targets for (postsonorant) velar fronting (\il{Westphalian}Wph):\smallskip\\
\begin{tabular}[t]{@{}l@{~}l l@{~}@{}l@{}}
Stage 1:   &  \multicolumn{3}{@{~}l}{\ipit{Grafschaft Bentheim}}\\
Stage 2a:  &  \ipit{Plettenberg}    & Stage 2aa: &  \ipit{Plettenberg}, \ipi{Byfang}, \ipi{Soest}, \ipi{Laer}\\
Stage 2b:  &  \ipit{Byfang}         & Stage 2bb: &  \ipit{Borken}, \ipi{Nienberge}, \ipi{Elspe}        \\
Stage 2c:  &   (\ipi{Nienberge})                                                 \\
Stage 2c': &  (\ipi{Soest}, \ipi{Laer})                                                \\
Stage 2d:  &  \ipit{Elspe}, \ipi{Borken}                                                 \\
\end{tabular}
\z 

A comparison of the places listed in \REF{ex:12:10} and \REF{ex:12:11} with \mapref{map:6} reveals that they are situated in the same region, although it is not possible to say that one particular place is immediately adjacent to another one which represents the immediately following historical stage.

A similar cluster of \il{East Pomeranian}EPo varieties (\sectref{sec:12.3.7}, listed in \tabref{tab:12.21}) is depicted in \REF{ex:12:12} for postsonorant velar fronting. All of the places listed here are located in an area of between approximately 80km from north to south and 80km from east to west on \mapref{map:18}.

\ea%12
\label{ex:12:12}Historical stages for triggers and targets for (postsonorant) velar fronting (\il{East Pomeranian}EPo):
\begin{tabular}[t]{@{}l@{~}l l@{~}l@{}}
  Stage 1:   & Kreis \ipi{Stolp}              \\
  Stage 2a:  & ---                      \\
  Stage 2b:  & ---                      & Stage 2bb: &  Kreis \ipi{Schlawe}                  \\
  Stage 2c:  &  \ipit{Kamnitz}, (Kreis \ipi{Schlawe}) & Stage 2cc: &  \ipit{Kamnitz}, \ipi{Lauenburg}, \\
             &                                        &            & Kreis \ipi{Bütow}\\
  Stage 2c': &  \ipit{Lauenburg}                \\
  Stage 2d:  & Kreis \ipi{Bütow}              \\
\end{tabular}
\z 

Recall from \sectref{sec:12.3.7} that Kreis \ipi{Stolp} is a rare example of a \isi{non-velar fronting island}. The stages for Trigger Types are well-represented in this region, although there are two gaps (Stages 2a and 2b). Kreis \ipi{Schlawe} exemplifies Stage 2bb (which is rare for that region), while \ipi{Kamnitz}, \ipi{Lauenburg}, and Kreis \ipi{Bütow} illustrate the more common Stage 2cc.

\section{{Nonheight} {features} {as} {triggers}}\label{sec:12.6}
\begin{sloppypar}
The data from German dialects presented in Chapters~\ref{sec:3}--\ref{sec:11} provide overwhelming evidence that variation among front vocalic triggers involves the vowel height dimension alone. In this section I discuss those rare cases in which velar fronting is triggered by nonheight features, namely \isi{rounding} (\sectref{sec:12.6.1}), \isi{tenseness} (\sectref{sec:12.6.2}), and \isi{stress} (\sectref{sec:12.6.3}). I speculate below on how these deviant systems fit into the \isi{rule generalization} model. One \isi{nonheight feature} I do not discuss is [nasal], which is shown in \sectref{sec:15.9} to be relevant in defining velar fronting triggers in a SwG dialect of the Southwest \ipi{Bernese Oberland}.
\end{sloppypar}
\subsection{Rounding}\label{sec:12.6.1}\is{Rounding|(}

Consider once again the \il{Westphalian}Wph dialect once spoken in the region around \ipi{Plettenberg} (\citealt{Gregory1934}; \mapref{map:6}). It was noted in \sectref{sec:12.3.6} that \ipi{Plettenberg} displays the rare Trigger Type A (=Stage 2a) for both postsonorant velar fronting and word-initial velar fronting. That assessment requires further refinement on the basis of the material presented in the original source. Enough data are provided in \citet{Gregory1934} to safely conclude that the high front vowel [i] serves as a trigger for both postsonorant fronting and word-initial fronting. It can also be deduced from that source that nonhigh front vowels do not serve as triggers. However, within the high front vowel category, Gregory’s material includes not only unrounded [i] (=⟦i⟧) but also the rounded vowel [y] (=⟦ü⟧).\footnote{{Among high vowels \ipi{Plettenberg} has a length contrast, i.e. [i y u] vs. [iː yː uː] (=\,⟦ī ǖ ū⟧), but high lax vowels found in other dialects ([ɪ ʏ ʊ]) are absent. No examples were found in the original source in which long high front vowels surface in the neighborhood of /x/.}} The complication is that [y] fails to serve as a trigger for velar fronting in both postsonorant and word-initial position. Consider first postsonorant fronting, which applies after [i] in \REF{ex:12:13a}. [x] surfaces after nonhigh front vowels in (\ref{ex:12:13c}), back vowels in (\ref{ex:12:13d}), and coronal sonorant consonants in (\ref{ex:12:13e}). Example \REF{ex:12:13f} shows that [x] (<\ili{WGmc} \textsuperscript{+}[sk]) also surfaces after an obstruent. Crucially, [x] and not [ç] occurs after [y] in (\ref{ex:12:13b}). The [ç] and [x] in \REF{ex:12:13} derive historically from velars (\ili{WGmc} \textsuperscript{+}[ɣ] or \textsuperscript{+}[x]).

\TabPositions{.15\textwidth, .33\textwidth, .5\textwidth, .75\textwidth}
\ea%13
\label{ex:12:13}Postsonorant dorsal fricatives in \ipi{Plettenberg} (from /x/):
\ea\label{ex:12:13a} filiχtə \tab [filiçtə] \tab vielleicht \tab ‘maybe’ \tab 22\\
    biχtə \tab [biçtə] \tab Beichte \tab ‘confession’ \tab 22\\
    slōpəriχ \tab [sloːpəriç] \tab schläfrig \tab ‘sleepy’ \tab 21\\
    xəsxiχtə \tab [xəsxiçtə] \tab Geschichte \tab ‘history’ \tab 30
\ex\label{ex:12:13b} füxn̥ \tab [fyxn̩] \tab Fichten \tab ‘spruce-\textsc{pl}’ \tab 37\\
    ʒəhüxtə \tab [ɣəhyxtə] \tab Dickicht \tab ‘thicket’ \tab 35
\ex\label{ex:12:13c} kröxn̥ \tab [krøxn̩] \tab husten \tab ‘cough-\textsc{inf}’ \tab 16\\
    döxtr̥ \tab [døxtr̩] \tab Töchter \tab ‘daughter-\textsc{pl}’ \tab 16\\
    nö̜xtə \tab [nœxtə] \tab  Nähe \tab ‘vicinity’ \tab 37\\
    wié̜x \tab [viɛx] \tab Weg \tab ‘path’ \tab 13\\
    lęxn̥ \tab [lɛxn̩] \tab lagen \tab ‘lie-\textsc{pret}.\textsc{pl}’ \tab 21\\
    ręxt \tab [rɛxt] \tab  Recht \tab ‘justice’ \tab 35\\
    lext \tab [slext] \tab Licht \tab ‘light’ \tab 29\\
    knext \tab [knext] \tab Knecht \tab ‘vassal’ \tab 35\\
    æxtr̥ \tab [æxtr̩] \tab  hinter \tab ‘behind’ \tab 35
\ex\label{ex:12:13d} tuxt \tab [tuxt] \tab  Zucht \tab ‘breeding’ \tab 37\\
    doxtr̥ \tab [doxtr̩] \tab Tochter \tab ‘daughter’ \tab 37
\ex\label{ex:12:13e} bięrx \tab [biɛrx] \tab  Berg \tab ‘mountain’ \tab 37
\ex\label{ex:12:13f} tüsxr̥ \tab [tysxr̩] \tab zwischen \tab ‘between’ \tab 15
\z 
\z 

In word-initial position the same generalization holds: Velar fronting affects /x/ (<\ili{WGmc} \textsuperscript{+}[ɣ]), which surfaces as [ç] before [i] (see \ref{ex:12:14a}), and as [x] before nonhigh front vowels in (\ref{ex:12:14c}), consonants in (\ref{ex:12:14d}), and most significantly [y] in (\ref{ex:12:14b}). The data in \REF{ex:12:14e} reveal that word-initial [sx] (<\ili{WGmc} \textsuperscript{+}[sk]) surfaces as [sx] even if [i] follows.\footnote{In some varieties of \ipi{Plettenberg} referred to in the original source, \ili{WGmc} \textsuperscript{+}[ɣ] is inherited without change as [ɣ]. That sound fails to undergo fronting even before [i], e.g. [ɣelt]  ‘money’, [ɣistr̩n] ‘yesterday’. The contrast between \REF{ex:12:14a} and \REF{ex:12:14e} suggests that word-initial velar fronting in \ipi{Plettenberg} must specify that the target segment (/x/) is at the left edge of the word.}

\ea%14
\label{ex:12:14}Word-initial dorsal fricatives in \ipi{Plettenberg}:
\ea\label{ex:12:14a} χistr̥n \tab [çistr̩n] \tab gestern \tab ‘yesterday’ \tab 13\\
    χié̜wn̥ \tab [çiɛvn̩] \tab  geben \tab ‘give\textsc{{}-inf}’ \tab 13\\
    χié̜tn̥ \tab [çiɛtn̩] \tab  gegessen \tab ‘eat\textsc{{}-part}’ \tab 30
\ex\label{ex:12:14b} xüt \tab [xyt] \tab  gieβt \tab ‘water\textsc{{}-3sg}’ \tab 30\\
    xüö̜tə \tab [xyœtə] \tab  Grütze \tab ‘groat’ \tab 19
\ex\label{ex:12:14c} xeld \tab [xelt] \tab  Geld \tab ‘money’ \tab 13
\ex\label{ex:12:14d} xrié̜p \tab [xriɛp] \tab  Griff \tab ‘handle’ \tab 14
\ex\label{ex:12:14e} sxié̜mn̥ \tab [sxiɛmn̩] \tab schämen \tab ‘be ashamed\textsc{{}-inf}’ \tab 12\\
    sxié̜p \tab [sxiɛp] \tab Schiff \tab ‘ship’ \tab 14\\
    sxiufkɑ̄r \tab [sxiufkɑːr] \tab Schubkarre \tab ‘wheelbarrow’ \tab 25
\z 
\z 

The data in \REF{ex:12:13} and \REF{ex:12:14} can be expressed by incorporating the feature [--round] in the set of velar fronting triggers:

\ea%15
\label{ex:12:15}
\begin{multicols}{2}\raggedcolumns
\ea \isi{Velar Fronting-11}:\\\label{ex:12:15a}
    \begin{forest}
    [,phantom
        [\avm{[−round\\+high]} [\avm{[coronal]},name=target]]
        [\avm{[−son\\+cont]},name=parent [\avm{[dorsal]}]]
    ]
    \draw [dashed] (parent.south) -- (target.north);
    \end{forest}
\ex \isi{Wd-Initial Velar Fronting-7}:\\\label{ex:12:15b}
    \begin{forest}
    [,phantom
        [\avm{[−son\\+cont]},name=parent [\avm{[dorsal]}]]
        [\avm{[−round\\+high]} [\avm{[coronal]},name=target]]
    ]
    \node [left=1ex of parent] {\textsubscript{wd} [};
    \draw [dashed] (parent.south) -- (target.north);
    \end{forest}
\z 
\end{multicols}
\z 

The two rules in \REF{ex:12:15} are unique to \ipi{Plettenberg}; however, the pattern discussed in the case study discussed below derives support from several German dialects.

Diachronically the \ipi{Plettenberg} data suggest that the historical stages for Trigger Types proposed in \tabref{tab:12.23} need to be further refined. In particular, I claim that Stage 2a can be preceded by a stage in which only high front unrounded vowels (HFUV) but not high front rounded vowels serve as triggers. I refer to that stage as Stage 2a' (=Trigger Type A') in \tabref{tab:12.27}.\largerpage

\begin{table}
\caption{Trigger Types and the corresponding historical stages for Plettenberg\label{tab:12.27}}
\begin{tabular}{lll}
\lsptoprule
Type & Trigger & Stage\\\midrule
A' & HFUV ([i]) & 2a'\\
A & HFV ([i y]) & 2a\\
B & HFV, MFV & 2b\\
C & HFV, MFV, CC & 2c\\
D & HFV, MFV, LFV & 2c'\\
E & HFV, MFV, LFV, CC & 2d\\
\lspbottomrule
\end{tabular}
\end{table}


Data from \ipi{South Mecklenburg} (\citealt{Jacobs1925a, Jacobs1925b, Jacobs1926}; \sectref{sec:11.3}, \mapref{map:17}) lend further support to the claim that dialects can draw a distinction between front rounded and front unrounded vowels as triggers for velar fronting. However, the material presented below from that dialect suggest that there can be an additional stage intervening between Stage 2a{}' and Stage 2a. As noted earlier, Jacobs provides a wealth of material collected in a broad region in \ipi{South Mecklenburg} indicating that [x] surfaces after a back vowel and [ç] after any front unrounded vowel. In contrast to \ipi{Plettenberg}, [x] surfaces for many speakers after front rounded vowels regardless of height (=\ref{ex:12:16a}). Doublets are provided for many tokens (=\ref{ex:12:16b}); according to Jacobs, the ones with [ç] occur in the northwest and the ones with [x] in the south. The [x] and [ç] in \REF{ex:12:16} derive historically from velars (\ili{WGmc} \textsuperscript{+}[x] or \textsuperscript{+}[ɣ]). The formal rule expressing the fact that the set of triggers is restricted to front unrounded vowels is stated in \REF{ex:12:17}.\footnote{The distribution of dorsal fricatives in \REF{ex:12:16} and the formal rule in \REF{ex:12:17} are shown to be attested in one German-language island (\sectref{sec:15.3}).}

\TabPositions{.17\textwidth, .38\textwidth, .55\textwidth, .75\textwidth}

\ea%16
\label{ex:12:16}\ipi{South Mecklenburg} [x] and [ç]:
\ea\label{ex:12:16a} lü̜xt \tab [lyxt] \tab  Laterne \tab ‘lantern’ \tab  1925b: 121  \\
    žü̜xt \tab [ʒyxt] \tab zweifelhafte  \tab ‘questionable \tab 1925b: 121 \\
         \tab        \tab Flüssigkeit  \tab liquid’\\
    zœ̄x\tab [zœːx] \tab Sau \tab ‘sow’ \tab  1925b      \\
    h\={œ}x \tab [hœːx] \tab  Freude \tab ‘joy’ \tab  1925b: 111   \\
    brö̜xt \tab [brøxt] \tab brachte \tab ‘bring\textsc{{}-pret}’ \tab 1925b: 133
\ex\label{ex:12:16b} mö̜xt, mü̜χt \tab [møxt], [myçt] \tab mochte \tab ‘like\textsc{{}-pret}’ \tab 1926: 129    \\
    brüx, brü·χ \tab [bryx], [bry·ç] \tab Brücke \tab ‘bridge’ \tab 1926: 129  \\
    mü̜x, mü·χ \tab [myx], [my·ç] \tab  \ipi{Mücke} \tab ‘mosquito’ \tab   1926: 129 \\
    rü̜x, rü·χ \tab [ryx], [ryç] \tab Rücken \tab ‘back’ \tab  1926: 129              \\
    trü̜x, trü·χ \tab [tryx], [try·ç] \tab zurück \tab ‘back’ \tab  1926: 129
\z 
\ex%17
\label{ex:12:17}\isi{Velar Fronting-12}\\
\begin{forest}
[,phantom
    [\avm{[−round]} [\avm{[coronal]},name=target,tier=word]]
    [\avm{[−son\\+cont]},name=source [\avm{[dorsal]},tier=word]]
]
\draw [dashed] (source.south) -- (target.north);
\end{forest}
\z 

The data from South Mecklenburg suggests that speakers with [x] after a front vowel preserve an earlier historical stage in which the triggers for velar fronting were high front unrounded vowels (HFUV) and mid front unrounded vowels (MFUV). This means that Stage 2b was preceded by Stage 2a' (as in \ipi{Plettenberg}), followed by Stage 2a'{}' (= Trigger Type A'{}', consisting of front unrounded vowels).

\begin{table}
\caption{Trigger Types and the corresponding historical stages for South Mecklenburg\label{tab:12.28}}
\begin{tabular}{lll}
\lsptoprule
Type & Trigger & Stage\\\midrule
A' & HFUV ([i]) & 2a'\\
A'{}' & HFUV, MFUV ([i e]) & 2a'{}'\\
B & HFV, MFV ([i y e ø] & 2b\\
C & HFV, MFV, CC & 2c\\
D & HFV, MFV, LFV & 2c'\\
E & HFV, MFV, LFV, CC & 2d\\
\lspbottomrule
\end{tabular}
\end{table}

Due to the rarity of Trigger Type A' and A'{}', it is not clear whether or not all German dialects begin at Stage 2a' and proceed to Stage 2a'{}', or if dialects have the option of beginning at Stage 2a' (as in \ipi{Plettenberg}) or Stage 2a.\is{Rounding|)}

\subsection{Tenseness}\label{sec:12.6.2}\is{Tenseness|(}\largerpage[2]

Recall from \sectref{sec:11.5} that the \il{East Pomeranian}EPo variety once spoken in Kreis \ipi{Rummelsburg} (\citealt{Mischke1936}; \mapref{map:18}) is unique among German dialects in the sense that the triggers for velar fronting are restricted to front tense vowels (/i e æ/) or coronal sonorant consonants. After front lax vowels (/ɪ ɛ/) and back vowels, underlying velars /x ɣ/ surface as velar.

From the diachronic perspective it is not clear how the set of triggers for velar fronting in \ipi{Rummelsburg} translates into the historical stages proposed above. I describe here a possible scenario: \ipi{Rummelsburg} represents a point (Stage 2d') whereby high front tense vowels (HFTV), mid front tense vowels (MFTV), low front tense vowels (LFTV) and CC trigger velar fronting. As indicated in \tabref{tab:12.29}, Stage 2d' preceded Stage 2d. Given the \isi{rule generalization} model adopted in the present study, one might expect the set of triggers for \ipi{Rummelsburg} to be narrower at an earlier stage. Since the triggers in question refer crucially to [+tense] front vowels, it would be consistent with the present approach to further restrict those triggers along the height dimension; hence, the triggers for velar fronting in pre-\ipi{Rummelsburg} stages might have been more restricted groupings of front [+tense] vowels, three of which are indicated in \tabref{tab:12.29}: Stage 2c'{}' (front tense vowels are triggers), Stage 2b' (nonlow front tense vowels are triggers), and Stage 2a'{}'{}' (high front tense vowels are triggers).

\begin{table}
\caption{Trigger Types and the corresponding historical stages for Rummelsburg\label{tab:12.29}}
\begin{tabular}{lll}
\lsptoprule
Type & Trigger & Stage\\\midrule
A'{}'{}' & HFTV ([i]) & 2a'{}'{}'\\
B' & HFTV, MFTV ([i e]) & 2b'\\
C' & HFTV, MFTV, LFTV ([i e æ]) & 2c'{}'\\
D' & HFTV, MFTV, LFTV, CC ([i e æ r]) & 2d'\\
E & HFV, MFV, LFV, CC & 2d\\
\lspbottomrule
\end{tabular}
\end{table}

\pagebreak
Since \ipi{Rummelsburg} is unique, the tentative proposal sketched above can only be evaluated once similar case studies from German dialects or other languages become known.\is{Tenseness|)}

\subsection{Stress}\label{sec:12.6.3}\is{Stress|(}

In the \il{Moselle Franconian}MFr variety of \ipi{Sörth} in Westerwald (\citealt{Hommer1910}; \sectref{sec:5.4}; \mapref{map:10}), the reflex of \ili{WGmc} \textsuperscript{+}[ɣ] in word-initial position (in the \textit{ge}{}- prefix) is an underlying palatal (/ʝ/) before \isi{schwa} in (\ref{ex:12:18a}), but before other sounds the original word-initial velar is retained as the velar stop [g]. The examples in \REF{ex:12:18} are representative. Note that [g] occurs before front vowels in (\ref{ex:12:18b}), back vowels in (\ref{ex:12:18c}) or consonants in (\ref{ex:12:18d}).

\ea%18
\label{ex:12:18}\relax[ʝ] (from /ʝ/) and [g] (from /g/) in \ipi{Sörth}:
\ea\label{ex:12:18a} jǝlɑ̄xt \tab [ʝǝlɑːxt] \tab gelacht \tab ‘laugh\textsc{{}-part}’ \tab 24
\ex\label{ex:12:18b} ɡiwǝl \tab [giwǝl] \tab Giebel \tab ‘gable’ \tab 10\\
    ɡē̜lən \tab [gɛːlǝn] \tab gelten \tab ‘be valid\textsc{{}-inf}’ \tab 22
\ex\label{ex:12:18c} ɡōt \tab [goːt] \tab gut \tab ‘good’ \tab 24
\ex\label{ex:12:18d} ɡrūs \tab [gruːs] \tab groβ \tab ‘large’ \tab 24
\z 
\z 

Since the \isi{schwa} in [ʝǝ] was originally [i] (cf. \ili{OHG}, \ili{MHG} \textit{gi}{}-) it appears that historical [ɣ] fronted to palatal in word-initial position before that particular vowel. This assumption is consistent with Stage 2a: Velar fronting applied word-initially before high front vowels. The problem is that the change from velar to palatal did not occur in words like [giwǝl] ‘gable’ in \REF{ex:12:18b}. Note that the [i] in that type of example can also be traced back to [i] in earlier stages of German, cf. \ili{MHG} \textit{gibel}, \ili{OHG} \textit{gibil}.

There was neither a qualitative nor quantitative difference between the [i] in \ili{MHG} \textit{gi}{}- and the [i] in the first syllable of words like MHG \textit{gibel}. The only difference between the two instantiations of [i] is that the one in \textit{gibel} was stressed, while the one in \textit{gi}{}- was not. The conclusion is that the set of triggers for the first stage of (word-initial) velar fronting in dialects like \ipi{Sörth} was an unstressed high front vowel.

Note that the data from \ipi{Sörth} contrast with the more common pattern whereby all front vowels (or a subset thereof) trigger fronting, regardless of whether or not the front vowels in question are stressed or unstressed (=Trigger Types A-E). One example discussed earlier (\sectref{sec:8.4}) is the \il{Eastphalian}Eph dialect once spoken in \ipi{Dingelstedt am Huy} (\citealt{Hille1939}; \mapref{map:7}): A word-initial velar (<\ili{WGmc} \textsuperscript{+}[ɣ]) surfaces as palatal before any original front vowel, e.g. [ʝɛlt] ‘money’ (cf. \ili{OSax} \textit{gelt}), [ʝɑːiʝǝ] ‘violin’ (cf. \ili{MHG} \textit{gīge}), [ʝǝzɪçtǝ] ‘face’ (cf. \ili{OHG} \textit{gisiht}) vs. [guːt] ‘good’ (cf. \ili{OSax} \textit{gōd}), [glɑːs] ‘glass’ (cf. \ili{OSax} \textit{glas}).

\ipi{Sörth} is not an isolated example. According to the phonetically transcribed texts in \citet{CornelissenEtAl1989} there are four towns in the same general area of Westerwald as \ipi{Sörth} which display the same pattern. The data in \REF{ex:12:19} are from one of those places (\ipi{Birken}). I retain the original transcriptions which indicate that [ʝ] (=⟦J⟧/⟦j⟧) occurs only before \isi{schwa} (=⟦ę⟧) in \REF{ex:12:19a} and [g] (=⟦G⟧/⟦g⟧) before front vowels (=\ref{ex:12:19b}), full back vowels (=\ref{ex:12:19c}), or consonants (=\ref{ex:12:19d}). \citet{CornelissenEtAl1989} also indicate that the pattern in \REF{ex:12:19} is the same in Friesenhagen, Flammersfeld, and Morsbach.

\ea%19
\label{ex:12:19}\relax[ʝ] (from /ʝ/) and [g] (from /g/) in \ipi{Birken}:
\ea\label{ex:12:19a} Jędicht \tab Gedicht \tab ‘poem’\\
    Jęschwistęr \tab Geschwister \tab ‘sibling’\\
    jęschlacht \tab geschlachtet \tab ‘slaughter-\textsc{part}’
\ex\label{ex:12:19b} ɡing \tab ging \tab ‘go-\textsc{pret}’\\
    Gänse \tab Gänse \tab ‘goose-\textsc{pl}’
\ex\label{ex:12:19c} gǫǫn \tab gehen \tab ‘go-\textsc{inf}’\\
    gǫǫre \tab gute \tab ‘good-\textsc{infl}’\\
    gaantsęn \tab ganzen \tab ‘whole-\textsc{infl}’\\
\ex\label{ex:12:19d} ɡlööf \tab glaube \tab ‘believe-\textsc{1sg}’
\z 
\z 

On the basis of the sources discussed above I conclude that there must have been a stage preceding Stage 2a for word-initial position -- at least, in parts of Westerwald. At that point (Stage 2a{}'{}'{}'{}'), \ili{WGmc} \textsuperscript{+}[ɣ] underwent velar fronting in the narrow context before an unstressed [i]. \tabref{tab:12.30} situates that stage with some of the other ones posited above (HUFV=High unstressed front vowel).

\begin{table}
\caption{Trigger Types and the corresponding historical stages for word-initial position\label{tab:12.30}}
\begin{tabular}{lll}
\lsptoprule
Type & Trigger & Stage\\\midrule
A'{}'{}'{}' & HUFV (unstressed [i]) & 2a'{}'{}'{}'\\
A & HFV & 2a\\
B & HFV, MFV & 2b\\
C & HFV, MFV, CC & 2c\\
D & HFV, MFV, LFV & 2c'\\
E & HFV, MFV, LFV, CC & 2d\\
\lspbottomrule
\end{tabular}
\end{table}


\mapref{map:24} depicts the five places discussed above representing Stage 2a'{}'{}'{}' in Westerwald. The map also indicates those places in the same area which represent Stage 1 for word-initial position. Stage 1 means that the place of articulation of the original velar (\ili{WGmc} \textsuperscript{+}[ɣ]) is retained as velar ([g]). For comparison, \mapref{map:24} also indicates a very common pattern discussed in greater detail in \chapref{sec:14} whereby \ili{WGmc} \textsuperscript{+}[ɣ] is realized as palatal in word-initial position before any type of segment (Stage 2e).

\begin{map}
% \includegraphics[width=\textwidth]{figures/VelarFrontingHall2021-img030.png}
\includegraphics[width=\textwidth]{figures/Map24_12.5.pdf}
\caption[Westerwald]{Westerwald. Circles represent the absence of velar fronting in word-initial position (\ili{WGmc} \textsuperscript{+}[ɣ] is realized as [g]). White squares indicate the realization of \ili{WGmc} \textsuperscript{+}[ɣ] as [ʝ] in word-initial position before \isi{schwa} ([ə]) and as velar ([g]) in the context before all other vowels as well as consonants (liquids). Lightly shaded squares indicate the realization of \ili{WGmc} \textsuperscript{+}[ɣ] as a palatal fricative in word-initial position before any type of vowel or consonant (liquid). Sources: \citet{Hommer1910} for \ipi{Sörth} and \citet{CornelissenEtAl1989} for all other places.}\label{map:24}
\end{map}

\mapref{map:24} only documents those places discussed in \citet{CornelissenEtAl1989} that are to the north and east of the Rhine River. It is possible that the aforementioned source might also contain evidence of Stage 2a'{}'{}'{}' places in other areas.\footnote{{The areas on \mapref{map:24} to the south and west of the Rhine River fall within the domain of MRhSA. Maps 381 for} \textrm{\textit{Garten}} \textrm{‘garden’ and 382 for} \textrm{\textit{grün}} \textrm{‘garden’ in that source document the [ʝ] realization consistent with Stage 2d, as well as the [g] realization for Stage 1. MRhSA notes on Map 381 that no comparable map is published for the word} \textrm{\textit{gebissen}} \textrm{‘bite-}\textrm{\textsc{part}}\textrm{’ (cf. \il{Standard German}StG [gəbɪsən]) because the areal distribution for palatal and velar in that word is almost identical with the areal distribution of [ʝ] and [g] in} \textrm{\textit{Garten}}. \textrm{Map 73 in volume 4 of WDU indicates that the initial consonant of} \textrm{\textit{gefallen}} \textrm{‘please someone-}\textrm{\textsc{inf}}\textrm{’ (cf. \il{Standard German}StG [gəfalən]) is realized as a palatal in an area of West Central Germany that includes the Westerwald. One cannot conclude that Map 73 provides independent evidence for Stage 2a'{}'{}'{}' because WDU does not provide maps showing the realization of historical velars in word-initial position in other contexts, i.e. before full front and back vowels and consonants (liquids).}}

An independent source for a very different part of Germany \citep{Kieser1963} documents Stage 2a'{}'{}'{}'. \citet{Kieser1963} investigates the modern realizations of word-initial \ili{WGmc} \textsuperscript{+}[ɣ] in \ipi{South Brandenburg} (\mapref{map:12}). According to that source there is a broad area in which \ili{WGmc} \textsuperscript{+}[ɣ] is realized as [ʝ] in word-initial position, but only in the context before \isi{schwa} (see \mapref{map:30} below).\is{Stress|)}

\section{{Significance} {of} {triggers} {and} {targets} {for} {typology}}\label{sec:12.7}

The present survey of German dialects draws several conclusions concerning triggers and targets, some of which derive support in the cross-linguistic work on \isi{Velar Palatalization} (\sectref{sec:2.3}). I consider triggers (\sectref{sec:12.7.1}) and targets (\sectref{sec:12.7.2}) in that order.

\subsection{Velar fronting triggers}\label{sec:12.7.1}

\subsubsection{Vowel height}
The most significant finding in the present study is that the front vocalic triggers for velar fronting vary along the height dimension. The generalization is expressed as the implication in \REF{ex:12:20}, which is motivated on the basis of a wide selection of typologically diverse languages (from \citealt{Bateman2007}: 64), based on earlier studies by \citet[37]{Neeld1973} and \citet[177]{Chen1973}. See also \citet{Kochetov2011}:

\eanoraggedright%20
\label{ex:12:20}
  \textsc{\isi{Implicational Universal for Palatalization Triggers}}:\\
  If lower front vowels trigger Palatalization, then so will higher front vowels.
\z
\REF{ex:12:20} is exceptionless for the German dialects discussed in this book. No counterexamples from German dialects are known to the present writer.

The \isi{Implicational Universal for Palatalization Triggers} accounts for the fact that several dialects are attested in which nonlow front vowels serve as triggers but the low front vowels do not, e.g. sequences like [iç] and [eç] (with velar fronting) vs. ones like [æx] (without velar fronting). Significantly, none of the sources cited above have the reverse, i.e. sequences like [æç] (with velar fronting) vs. ones like [ix] and [ex] (without velar fronting). Apparent counterexamples discussed above are those dialects in which a front vowel traditionally described as mid fails to trigger fronting, while other vowels in the mid front range do, e.g. a sequence like [ɛx] (without velar fronting) vs. ones like [eç] and [iç] (with velar fronting). One example discussed earlier involves the fronting of word-initial /kx/ in the \il{High Alemannic}HAlmc variety of \ipi{Rheintal} (\citealt{Berger1913}; \sectref{sec:3.4}; \mapref{map:2}) before [i y øː eə] but not before [ɛː ɛə]. Recall that this is not a true counterexample to \REF{ex:12:20} because [ɛː] and the first component of [ɛə] are phonologically [+low], in contrast to [i y øː] and the first part of [eə], which are [--low].

Given the three types of front vowels that can function as triggers (\{HFV, MFV, LFV\}), the \isi{Implicational Universal for Palatalization Triggers} accounts for the fact that four logically-possible triggers are unattested:

\begin{table}
\caption{Unattested Trigger Types involving vocalic triggers\label{tab:12.31}}
\begin{tabular}{lll}
\lsptoprule
Type & Trigger & Present in context for fronting\\\midrule
R & MFV & HFV\\
S & LFV & HFV, MFV\\
T & HFV, LFV & MFV\\
U & MFV, LFV & HFV\\
\lspbottomrule
\end{tabular}
\end{table}

All four Trigger Types in \tabref{tab:12.31} violate the \isi{Implicational Universal for Palatalization Triggers}. For example, velar fronting is triggered by mid front vowels for Trigger Type R, but not for the high front vowels. The fact that \REF{ex:12:20} derives cross-linguistic support suggests that the four unattested Trigger Types in \tabref{tab:12.31} are not simply accidental gaps.

\subsubsection{Nonheight features} The material presented from German dialects also supports the finding from \citet[62]{Bateman2007} that velar fronting is only rarely sensitive to nonheight features. Recall from the earlier discussion that this generalization cannot be a universal without exceptions because the language Fanti (\ili{Niger-Congo}; Ghana) is attested in which only front oral vowels serve as triggers for velar fronting. Although the German dialects discussed in Chapters~\ref{sec:3}--\ref{sec:11} provide overwhelming support for Bateman’s observation, there is a small number of dialects in which velar fronting triggers are partially defined in terms of \isi{nonheight features}. I consider the three \isi{nonheight features} referred to in that earlier section in turn and conclude this section by discussing the status of coronal sonorant consonants as velar fronting triggers.

\subsubsubsection{Rounding}\is{Rounding|(}
The data from \ipi{Plettenberg} in \REF{ex:12:13} and \REF{ex:12:14} and \ipi{South Mecklenburg} in \REF{ex:12:16} show that the triggers for velar fronting make a distinction between front rounded and front unrounded vowels. The cross-linguistic studies cited earlier on Palatalization find no correlation between that process (regardless of whether or not the target is a coronal or a velar) and (un)\isi{rounding} of vocalic triggers (\citealt{Bhat1978}, \citealt{Bateman2007,Bateman2011}, \citealt{Kochetov2011}).


The claim that front unrounded vowels are more favorable triggers for Palatalization than front rounded vowels is discussed at length in \citet{Neeld1973}. The example he discusses is the fronting of velar [g] to postalveolar [ʒ] (=⟦ž⟧) before [i] but not before [y] in the history of \ili{French}, e.g. [reʒim] ‘regime’ vs. [regylarite] ‘regularity’, where [ʒ] and [g] both derive from earlier [g].

\citet[61]{Bhat1978} too notes that velar fronting in \ili{French} apparently failed to take place before front rounded vowels. However, he suggests that the failure of a front rounded vowel to trigger the fronting of a velar may not be because of the roundedness of the trigger but instead because the trigger is not sufficiently front. One might be inclined to apply this proposal to the velar fronting data from \ipi{Plettenberg} and \ipi{South Mecklenburg}, but since no data from those varieties is available corroborating the claim that vowels such as [y] are slightly more retracted than ones like [i], the proposal must remain open for further study.\is{Rounding|)}

\subsubsubsection{Tenseness}\is{Tenseness|(}
The set of vocalic triggers for velar fronting in the now extinct \il{East Pomeranian}EPo community of \ipi{Rummelsburg} (\citealt{Mischke1936}; \sectref{sec:11.5}, \mapref{map:18}) is restricted to [+tense] front vowels. That variety is not only unique for German dialects; it is apparently unprecedented from the cross-linguistic perspective as well. See the literature cited earlier (\citealt{Bhat1978}, \citealt{Bateman2007,Bateman2011}, \citealt{Kochetov2011}), in which no reference is made to languages restricting the set of triggers for Palatalization processes along the \isi{tenseness} dimension. Although that typological literature suggests that \ipi{Rummelsburg} stands alone in the languages of the world, it is interesting to consider the way in which that dialect corroborates the conclusions drawn by \citet{Ćavar2007} in her analysis of palatalized consonants in \ili{Polish}. Ćavar argues that there is a direct correlation between \isi{tenseness} -- expressed in her treatment with the feature \is{ATR@[ATR]}[ATR] -- and Palatalization. In particular, she shows that the positive value of that feature occurs before palatalized consonants (i.e. alveolopalatals like [ɕ]) as well as secondarily palatalized velars (e.g. [k\textsuperscript{j}]). For example, most consonants of \ili{Polish} have secondarily palatalized allophones in the context of [i] and [j], but the vowel [ɨ] never induces that process. In Ćavar’s treatment, \isi{secondary palatalization} is guaranteed by a constraint specifying that the vocalic trigger and  the target (e.g. /k/) must share the \isi{tenseness} feature [+ATR]. Since [i] and [j] are [+ATR] and [ɨ] is [--ATR], the correct prediction is made that [k\textsuperscript{j}i] but not [k\textsuperscript{j}ɨ] surfaces.\footnote{One could alternatively argue that [i] and [ɨ] are distinguished not by \is{ATR@[ATR]}[ATR], but instead by a frontness feature (e.g. [back]). I do not attempt to evaluate the merits of Ćavar’s proposal here and choose to leave that question open for further research.}\is{Tenseness|)}

\subsubsubsection{Stress}\is{Stress|(}
Data from several varieties of German in Westerwald were discussed in \sectref{sec:12.6.3} indicating that the trigger for velar fronting fronting in word-initial position is restricted to an unstressed high front vowel ([i]). Studies on the typology of Palatalization observe that \isi{stress} can be a conditioning factor for that process, although the conclusion from that research is that stressed syllables rather than unstressed syllables favor the fronting of velars. For example, \citet[55]{Bhat1978} cites \ili{Uzbek} (\ili{Turkic}; Uzbekistan), \ili{Eastern Armenian} (\ili{Indo-European}; Armenia), \ili{Sindhi} (\ili{Indo-Aryan}; Pakistan), \ili{Common Samoyed} (\ili{Uralic}; North Eurasia) and \ili{Sirionó} (\ili{Tupian}; Bolivia) as languages in which velars are fronted (palatalized in Bhat's terms) before stressed (front) vowels. No language is cited in \citet{Bhat1978} in which velar fronting is triggered by an unstressed vowel, although Bhat does show that unstressed vowels tend to trigger Palatalization (i.e. raising) of alveolar sounds.\is{Stress|)}

\subsubsection{Coronal sonorant consonants}
The typological literature cited above has little to say on the topic of consonants as triggers. The few languages in which true consonants -- as opposed to glides like [j] -- trigger Palatalization (regardless of the type of target) involve long distance spreading and are therefore very different from the type of velar fronting under investigation in the present book. That finding from the typological literature suggests that the front segments triggering Palatalization include vowels (e.g. [i]) but not consonants (e.g. [l], [r], [n]). Although that assessment appears to be implicit in much of the cross-linguistic work cited earlier, it is clearly incorrect for German because the predominant pattern for (postsonorant) velar fronting is that the triggers consist of both front vowels or coronal sonorant consonants, in particular /l r n/.


Given that /l r n/ can trigger velar fronting, the present treatment predicts that -- in principle -- those sounds alone could trigger that process. To see this, consider once again the four front segment types that can function as triggers (\{HFV\}, \{MFV\}, \{LFV\}, \{CC\}) as well as their logical combinations. Given the three categories for vocalic triggers (\{HFV\}, \{MFV\}, \{LFV\}), there are seven logical combinations, three of which are attested (Trigger Types A, B, D) and four of which are not (Trigger Types R-U from \tabref{tab:12.31}). Eight logically-possible triggers involve \{CC\} and front vowels (\{HFV\}, \{MFV\}, \{LFV\}). Two of those eight are attested, namely Trigger Type C (\{HFV, MFV, CC\}) and Trigger Type E (\{HFV, MFV, LFV, CC\}). The remaining six are listed in the first six rows of \tabref{tab:12.32}. In the final two rows I list the two logical combinations of \{CC\} and front vowels in the case that \{LFV\} is absent.

\begin{table}
\caption{Trigger Types involving vocalic and consonantal triggers\label{tab:12.32}}
\begin{tabular}{llll}
\lsptoprule
Type & Trigger & Present in context for fronting & Stage\\\midrule
V & CC & HFV, MFV, LFV & 2a'{}'{}'{}'{}'\\
W & HFV, CC & MFV, LFV & 2a'{}'{}'{}'{}'{}'\\
X & MFV, CC & HFV & {}-{}-{}-\\
Y & LFV, CC & HFV, MFV & {}-{}-{}-\\
Z & HFV, LFV, CC & MFV & {}-{}-{}-\\
ZZ & MFV, LFV, CC & HFV & {}-{}-{}-\\\tablevspace
VV & CC & HFV, MFV & 2aa'{}'{}'{}'\\
WW & HFV, CC & MFV & 2aa'{}'{}'{}'{}'\\
\lspbottomrule
\end{tabular}
\end{table}

Trigger Types X, Y, Z, and ZZ are correctly predicted to be nonoccurring because they violate the \isi{Implicational Universal for Palatalization Triggers} in \REF{ex:12:20}. By contrast, there is no reason why Trigger Types V and W (and their equivalents VV and WW without low front vowels) should not occur. In the remainder of this section I demonstrate that this is the correct prediction for Trigger Type VV. Trigger Types V, W, and WW, while not attested, are predicted to be possible in principle.

Two examples are known to me for Trigger Type VV. The first is the \il{Rhenish Franconian}RFr variety of \ipi{Beerfelden} (\citealt{Wenz1911}; \mapref{map:10}). The data from that source indicate that both [ç] and [x] occur in postvocalic position. Wenz transcribes both segments with the same symbol (=⟦χ⟧), but he is clear that they represent palatal [ç] after a front vowel and velar [x] after a back vowel, e.g. [ɪç] ‘I’ (=⟦ìχ⟧) vs. [bʊx] ‘book’ (=⟦bùχ⟧). (\ipi{Beerfelden} has no low front vowels). There are no examples of either [ç] or [x] after a consonant because the crucial examples contain an epenthetic vowel, e.g. [mɪlɪç] ‘milk’ (=⟦mìlìχ⟧). The significance of \ipi{Beerfelden} can be seen in the distribution of [ɣ]/[ʝ], which display a pattern distinct from their fortis counterparts. As indicated below, velar [ɣ] (=⟦γ⟧), surfaces after a front vowel and before a vowel in (\ref{ex:12:21a}), after a back vowel and before a vowel or syllabic liquid in (\ref{ex:12:21b}), and after [i] from /r/ in (\ref{ex:12:21c}). By contrast, palatal [ʝ] (=⟦j⟧) occurs after a coronal consonant (always [l]) and before a vowel in (\ref{ex:12:21d}).

\ea%21
\label{ex:12:21}\relax[ɣ] and [ʝ] in \ipi{Beerfelden}:
\ea\label{ex:12:21a} bîγǝ \tab [bɪːɣǝ] \tab biegen \tab ‘bend\textsc{{}-inf}’ \tab 35\\
    féγǝ \tab [feːɣǝ] \tab fegen \tab ‘sweep\textsc{{}-inf}’ \tab 35
\ex\label{ex:12:21b} fòγl \tab [fɔɣl̩] \tab Vogel \tab ‘bird’ \tab 35\\
    fróγǝ \tab [froːɣǝ] \tab fragen \tab ‘ask\textsc{{}-inf}’ \tab 35\\
    sâγǝ \tab [sɑːɣǝ] \tab sagen \tab ‘say\textsc{{}-inf}’ \tab 35
\ex\label{ex:12:21c} bòiγǝ \tab [bɔiɣǝ] \tab borgen \tab ‘borrow\textsc{{}-inf}’ \tab 35\\
    gòiγl \tab [gɔiɣl̩] \tab Gurgel \tab ‘throat’ \tab 35\\
\ex\label{ex:12:21d} fòljǝ \tab [fɔlʝǝ] \tab folgen \tab ‘follow\textsc{{}-inf}’ \tab 35\\
    fèljǝ \tab [fɛlʝǝ] \tab Radfelge \tab ‘wheel rim’ \tab 35
\z 
\z 

From the synchronic perspective, \ipi{Beerfelden} has Target Type M (=Stage 2bb) because both /x/ and /ɣ/ undergo postsonorant fronting; however, the triggers differ for those two targets: For /x/ we have Trigger Type EE, but for /ɣ/ it is Trigger Type VV (\tabref{tab:12.32}).

Synchronically /ɣ/ (<\ili{WGmc} \textsuperscript{+}[ɣ]) undergoes fronting to [ʝ] in the \{CC\} context (i.e. after [l]). The data in \REF{ex:12:21} can be accommodated in the \isi{rule generalization} approach endorsed here given the stages in \tabref{tab:12.33}. The first two stages are the ones unique to \ipi{Beerfelden} (see the final two rows of \tabref{tab:12.32}), while the final three are the same as ones proposed earlier.

The second variety known to me for Trigger Type VV is the Hes dialect described by \citet{Gommermann1975}. That dissertation provides a detailed description of the historical phonology and morphology for speakers living in Milwaukee, Wisconsin (USA), whose ancestors came originally from an area south of Fulda (\mapref{map:11}). From there, the progenitors of this presumably EHes variety emigrated to the town of Mucsi (Hungary) and then later to the United States. \citet[105--106; 108--109]{Gommermann1975} shows that velar and palatal fricatives (both fortis and lenis) surface. Both /x/ and /ɣ/ serve as targets for postsonorant velar fronting (=Target Type M), but -- as in Beerfelden -- /ɣ/ undergoes fronting only in the context after liquids, while an epenthetic vowel intervenes between a liquid and /x/, e.g. ⟦moːɣə⟧ ‘stomach’, ⟦gədsoːuɣə⟧ ‘pull-\textsc{part}’, ⟦viːɣə⟧ ‘scale’, ⟦ẹːɣə⟧ ‘harrow’,  ⟦gnaːiçd⟧ ‘vassal’, ⟦ɪç⟧ ‘I’, ⟦buːx⟧ ‘book’,  ⟦kʰʊxə⟧ ‘cake’, ⟦gɒljə⟧ ‘gallows’, ⟦ęorjṛ⟧ ‘anger’, ⟦mɛlɪç⟧ ‘milk’ (cf. StG [mɪlç]), ⟦ʃdʊrɪç⟧ ‘stork’ (cf. StG [ʃtɔʀç]). 

\begin{table}
\caption{Alternate Trigger Types and historical stages for Beerfelden /ɣ/\label{tab:12.33}}
\begin{tabular}{lll}
\lsptoprule
Type & Trigger & Stage\\\midrule
VV & CC & 2aa'{}'{}'{}'\\
WW & HFV, CC & 2aa'{}'{}'{}'{}'\\
B & HFV, MFV, CC & 2b\\
D & HFV, MFV, LFV & 2cˈ\\
E & HFV, MFV, LFV, CC & 2d\\
\lspbottomrule
\end{tabular}
\end{table}

The proposal in \tabref{tab:12.33} is consistent with the approach to \isi{rule generalization} described above where there is a progression of triggers from specific to general.

\subsection{Velar fronting targets}\label{sec:12.7.2}\largerpage[2]

A significant finding in the present study is that the targets for velar fronting in German dialects obey the implication in \REF{ex:12:22} without exception. Recall from \sectref{sec:2.3} that \REF{ex:12:22} derives strong support from both phonetics and typology in a wide variety of languages. No counterexamples are known.

\eanoraggedright%22
\label{ex:12:22}
\textsc{\isi{Implicational Universal for Velar Fronting Targets-2}}:\\
If a lenis sound undergoes velar fronting then the corresponding fortis sound does as well.
\z
\REF{ex:12:22} accounts for the fact that there are dialects in which the targets for velar fronting are fortis (/x/) and lenis (/ɣ/) sounds (Target Type M), or fortis (/x/) but not lenis (Target Type L). Significantly, there is no dialect in which a lenis velar (/ɣ/) undergoes fronting but the corresponding fortis sound (/x/) does not. Target Types L and M are restated in \tabref{tab:12.34} in a slightly simplified form as well as nonoccurring Target Type M{}'.
 

\begin{table}
\caption{Unattested and attested Target Types\label{tab:12.34}}
\begin{tabular}{lll}
\lsptoprule
Type & Target & Present in fronting context\\\midrule
L & /x/ & /ɣ/\\
M & /x ɣ/ & {}-{}-{}-{}-{}-\\
M' & /ɣ/ & /x/\\
\lspbottomrule
\end{tabular}
\end{table}

\textcite[56ff.]{Bateman2007} observes that the most common targets for Palatalizations are obstruents (stops, fricatives) as opposed to sonorants (e.g. /ŋ/). The generalization also holds for velar fronting in German dialects, although the only sonorant target for velar fronting in the material discussed above is [ŋ]. Only a small number of dialects exhibit the fronting of a velar nasal (\chapref{sec:11}); however, of those dialects with that change, velar stops and velar fricatives also undergo fronting. It is possible to posit an exceptionless implication (“If a velar sonorant undergoes fronting then so does a velar obstruentˮ); however, that statement is not particularly meaningful given the small number of dialects where [ŋ] undergoes fronting.

Bateman also observes that languages with stops as targets outnumber those with fricatives, although she concedes that there are also many languages in which fricatives but not stops serve as targets. The present study demonstrates that the latter situation is the norm for German dialects (recall the \isi{Implicational Universal for Velar Palatalization Targets-1} from \sectref{sec:2.3.2}). Thus, there are many dialects in which only fricatives (/x/) but not stops (/k/) undergo velar fronting (Target Types L and M), and there is also a small but not insignificant group of dialects in which both stops and fricatives undergo fronting (Target Type N). However, no dialect has been found in which only velar stops but not velar fricatives undergo fronting. Recall from \sectref{sec:11.8} that a historical explanation was offered to account for the strong preference of velar fricatives over velar stops as targets for velar fronting in German dialects.

\section{Additional properties of velar fronting}\label{sec:12.8}\largerpage

\subsection{Adjacency of targets and triggers}\label{sec:12.8.1}

In almost all case studies discussed in Chapters~\ref{sec:3}--\ref{sec:12} the velar fronting target either immediately follows the trigger (postsonorant velar fronting) or immediately precedes it (word-initial velar fronting). In a small number of systems, the trigger and target are separated by another sound (referred to below as Q). In the present section I provide a synopsis of those dialects and the patterns they represent.

\tabref{tab:12.35} lists four patterns expressed in earlier chapters with the corresponding rules. Those patterns involve either the spreading of [coronal] from the trigger in the third column to the target (Q) in the second column by a version of \isi{schwa} fronting, or by the merger of the [coronal] feature of a front vowel with the [coronal] feature of an adjacent consonant (Q) by one of two processes of coalescence.\footnote{{In contrast to the two \isi{schwa} fronting processes, neither rule of coalescence has a target or a trigger. The front vowel referred to in the third column of \tabref{tab:12.35} represents the trigger for velar fronting, and the adjacent consonant in the second column corresponds to Q.} } The segments in bold in the fourth column share the same [coronal] feature. The four patterns are classified as one of the types listed in the first column. Note that the three rules listed for Types PP, QQ, and RR \isi{feed} velar fronting in the postsonorant context, while the coalescence indicated for Type SS \isi{feeds} word-initial velar fronting. In the penultimate column of \tabref{tab:12.35} I list the places discussed in earlier chapters with the four patterns. I comment on the target segment /v/ for Type SS below.

\begin{sidewaystable}\small
\caption{Processes \isi{feeding} velar fronting. IS: Intervening sound.\label{tab:12.35}\il{High Alemannic}\il{Mecklenburgish-West Pomeranian}\il{Silesian}\il{High Prussian}\il{East Pomeranian}}
\begin{tabularx}{\textwidth}{ll>{\raggedright}p{\widthof{front vowel}}QlQ>{\raggedright\arraybackslash}p{\widthof{\il{Mecklenburgish-West Pomeranian}MeWPo}}>{\raggedright\arraybackslash}p{\widthof{§10.3.1}}}
\lsptoprule
Type & IS (Q) & Trigger & Pattern & Rule & Place & Dialect & Sec.\\\midrule
PP & \isi{schwa} & nonlow front vowel & /iəx/ → {\textbar}\textbf{iə̟}x{\textbar} → [\textbf{iə̟ç}]

/uəx/ → [uəx] & \isi{Schwa Fronting-1} &  \ipit{Rheintal} & HAlmc & \sectref{sec:3.4}\\
QQ & \isi{schwa} & liquid & /Vlx/ → {\textbar}Vləx{\textbar} → {\textbar}V\textbf{lə̟}x{\textbar} → [V\textbf{lə̟ç}] & \isi{Schwa Fronting-2} & Many places & HG & \sectref{sec:5.4}, \sectref{sec:10.3.1}\\
RR & liquid & front vowel & /ilx/ → {\textbar}\textbf{il}x{\textbar} → [\textbf{ilç}]\newline /ɑlx/ → [ɑlx] & \isi{Coalescence-1} &  \ipit{Visperterminen}    & HAlmc     & \sectref{sec:6.2} \\
   &         &            &                                                                                 &                     &  \ipit{Obersaxen}           & HAlmc     & \sectref{sec:6.3} \\
   &         &            &                                                                                 &                     &  \ipit{West Mecklenburg}    & MeWPo     & \sectref{sec:11.3}\\
   &         &            &                                                                                 &                     &  \ipit{Wolgast}             & MeWPo     & \sectref{sec:11.3}\\
   &         &            &                                                                                 &                     &  \ipit{Sebnitz}             & Sln       & \sectref{sec:11.4}\\
   &         &            &                                                                                 &                     &  \ipit{Reimerswalde}        & HPr       & \sectref{sec:11.7}\\
SS & liquid or /v/ & front vowel & /gli/ → {\textbar}g\textbf{li}{\textbar} → [\textbf{ɟli}]\newline /glɑ/ → [glɑ] & \isi{Coalescence-2} &  \ipit{West Mecklenburg}     &    MeWPo    & \sectref{sec:11.3}     \\
   &         &            &                                                                                  &                           &  \ipit{Sebnitz}               &    Sln    &   \sectref{sec:11.4}       \\
   &         &            &                                                                                  &                           &  \ipit{Seifhennersdorf}       &    Sln    &   \sectref{sec:11.4}       \\
   &         &            &                                                                                  &                           &  Kreis \ipi{Konitz}          &   EPo      & \sectref{sec:11.5}\\
\lspbottomrule
\end{tabularx}
\end{sidewaystable}

The one dialect listed above exemplifying Type PP is \ipi{Rheintal}. Recall from \sectref{sec:3.4} that the front vowel trigger and the target \isi{schwa} together form a diphthong. Type PP can be contrasted with the pattern whereby a velar surfaces as velar after a diphthong consisting of a front vowel plus \isi{schwa}, e.g. \ipi{Ramsau am Dachstein} (\sectref{sec:3.5}). Some of the dialects discussed in this chapter in passing also reflect Type PP, although no data were presented. Two \il{Westphalian}Wph examples are \ipi{Laer} (\citealt{Niebaum1974}; \sectref{sec:12.3.6}) and \ipi{Müschede} (\citealt{NiebaumTeepe1976}). The original sources are clear concerning the facts involving the distribution of [x] and [ç] after diphthongs ending in \isi{schwa}, e.g. \ipi{Laer} [dri·əç] ‘wear-\textsc{pret}’ vs. [vu·əx] ‘weigh-\textsc{pret}’. \citet[62-63]{Niebaum1974} writes: “Dies zeigt, dass bei diesen Diphthongen für die Auswahl der Reibelautvariante jeweils die erste Diphthong-komponente entscheidend ist”. (“This shows that for the selection of the fricative variant [i.e. [x] or [ç], T.A.H.] the first component of the diphthong is crucial”).

For Type QQ, \isi{schwa} is epenthesized and then undergoes \isi{Schwa Fronting-2}. The two varieties of German discussed above exemplifying \isi{schwa} epenthesis-cum-\isi{schwa} fronting are \ipi{Sörth} and \ipi{Schlebusch}, although it was also noted in the context of the former dialect that that type of system is quite common among HG dialects.

Types RR and SS are interesting from the point of dialectology and historical phonology because they are attested in areas that are not geographically contiguous. Thus, Type RR can be observed in Switzerland, Mecklenburg-Vorpommern and East Prussia, and Type SS in Mecklenburg-Vorpommern, West Prussia, and East Prussia (but not in Switzerland). It is also important to stress that not all varieties of German described in Mecklenburg-Vorpommern, East Prussia, West Prussia, and Switzerland represent Type RR and/or Type SS. For example, \ipi{West Mecklenburg} and \ipi{Wolgast} represent Type RR, but \ipi{South Mecklenburg} does not (because /x/ is realized as [x] after a liquid, even if the liquid is preceded by a front vowel). The conclusion is that the two coalescence processes -- like velar fronting -- can arise independently in noncontiguous areas (\isi{polygenesis}).

The leftmost segment (Q) of \isi{Coalescence-2} is a (coronal) liquid, as indicated in \tabref{tab:12.35}. However, Q can also be /v/, e.g. \ipi{West Mecklenburg} [cveə] ‘across’ (from /kveə/). Words like [cveə] pose a potential problem for the present treatment because /v/ is not a coronal consonant and therefore does not fit the structural description of \isi{Coalescence-2}, as formalized in \sectref{sec:11.3}. \ipi{West Mecklenburg} is not an isolated example because the same generalization involving [v] holds for the other dialects exemplifying Type SS. One possible alternative analysis is to reject \isi{Coalescence-2} and to posit that the trigger and target for velar fronting in Type SS systems need not be adjacent. Velar fronting can then spread across a liquid if liquids are not specified for coronality, and spreading can likewise occur across a labial (/v/) without incurring a violation of the line-crossing constraint in nonlinear phonology. However, that reanalysis may pose a problem for various cross-linguistic generalizations involving \isi{adjacency} (\citealt{Odden1994} as well as work by later authors). A more attractive option in my view is to analyze the /v/ referred to above not as an obstruent, but instead as a sonorant, i.e. as the glide-like (approximant) sound /ʋ/. See Appendix~\ref{appendix:h} and \citet{Hall2014c} for a discussion of similar data from \il{Westphalian}Wph.

Given the processes of \isi{schwa} fronting for Types PP and QQ it can be said that the trigger and target for velar fronting are adjacent on the surface. The reason is that the fronted \isi{schwa} is a (derived) front vowel, and front vowels are triggers for velar fronting. For Type RR and SS the situation is different because the trigger for velar fronting (front vowel) is not adjacent to the target velar even after coalescence merges the [coronal] feature of Q with [coronal] of the trigger. The same point holds if Q is /v/. Future work may want to consider Type RR and SS dialects in light of Bateman’s conclusion that the trigger and target for \isi{Velar Palatalization} are always adjacent \citep[77-82]{Bateman2007}.

\subsection{Domain of velar fronting}\label{sec:12.8.2}

In every case study discussed in this book the trigger and target for velar fronting belong to the same word. There is no evidence from any dialect that those two sounds can span a word boundary as in the rule of \is{Flapping (American English)}Flapping for \ili{American English} (\sectref{sec:2.2.1}); hence, nothing suggests that velar fronting has the status of a phrase level (postlexical) rule in any German dialect. In the models referred to in \sectref{sec:2.2.1}, i.e. \isi{Lexical Phonology and Morphology} (e.g. \citealt{Kiparsky1982b}, \citealt{KaisseShaw1985}, \citealt{Mohanan1986}, \citealt{HargusKaisse1993}) and \isi{Stratal Optimality Theory} (\citealt{KaisseMcMahon2011}, \citealt{Bermúdez-Otero2015}) velar fronting must be classified a word level (lexical) rule.

It is possible to test whether or not velar fronting applies across words (postlexically): One needs to consider a sequence of two lexical items (“Word Aˮ and “Word Bˮ), where Word A ends in a segment that serves as a trigger for velar fronting (e.g. /i/), and Word B begins with a target for velar fronting (e.g. /x/). If velar fronting is active with /i/ as a trigger and /x/ as a target, then the rule would be expected to apply to the sequence described above if velar fronting is a phrase level rule. In most of the dialects investigated in the present book there are no sequences such as /i/ plus /x/ in connected speech because the target velar segment (/x/) does not occur word-initially. In those velar fronting dialects with a word-initial target velar (/x/), that sound systematically fails to undergo velar fronting even after an appropriate trigger at the end of a preceding word. As a representative example, consider \ipi{Rheintal} (\sectref{sec:3.4}). In that dialect, /x/ and /kx/ regularly undergo fronting after a nonlow front vowel or a coronal sonorant consonant. The same velars also undergo fronting in word-initial position if the same triggers follow. One can test whether or not postsonorant velar fronting is a phrasal rule in \ipi{Rheintal} by considering a sequence of Word A and Word B, where Word A ends in a velar fronting trigger and Word B begins with [k] or [kx]. Fortuitously, several examples of that structure are present in the texts provided by \citet[188-191]{Berger1913}. One example from the \ipi{Rheintal} variety is the phrase [i kxɑmmər] ‘in the room’ (=⟦i kxɑmmər⟧), with a velar after a front vowel trigger. On the basis of that type of example one can conclude that velar fronting in \ipi{Rheintal} is a word level (lexical) rule (i.e. target and trigger belong to the same word) and not a phrase level (postlexical) rule (i.e. target and trigger can span a word-boundary).\footnote{{In the hypothetical example described above, the postlexical rule of velar fronting applies from left-to-right (progressively). Since no German dialect is attested in which postsonorant velar fronting applies regressively from a trigger to a target belonging to the same word (\sectref{sec:2.3.5}, \sectref{sec:6.5.2}, \sectref{sec:16.5}) it should come as no surprise that regressive spreading across a word boundary is also not attested; cf. \il{Standard German}StG [buːxɪst] ‘book is’ vs. *[buːçɪst].}}

Velar fronting is a word level rule in those dialects like \ipi{Rheintal} for which evidence is available, although I do not provide additional examples here.

The conclusion is that velar fronting is word-bounded, but it also needs to be stressed that the trigger and target for velar fronting never span a morpheme boundary, a generalization that is true without exception for all German dialects with that rule. Put differently, the trigger and target for velar fronting (word-initial and postsonorant) always belong to the same morpheme. As a representative example, consider the MStGm words [lɑxən] ‘laugh\textsc{{}-inf}’ (from /lɑx-ən/) and [ʀiːçən] ‘smell\textsc{{}-inf}’ (from /ʀiːx-ən/), in which the morpheme boundary is situated after the dorsal fricative of the stem and before the schwa-initial suffix. In those examples the vocalic trigger is tautomorphemic with the velar target. By contrast, there are no suffixes beginning with a velar fronting target (/x/) that could potentially undergo the rule and surface as palatal after a stem ending in a front vowel trigger, e.g. hypothetical morphologically-complex items like /liː{}-xə/ and /nɑ-xə/, which would presumably surface as [liːçə] and [nɑxə] respectively. The famous example involving the occurrence of \is{chen@\textit{-chen}}-\textit{chen} ([çən]) even after a stem ending in a back vowel is not a counterexample because the initial sound of that suffix is an underlying palatal (/ç/) and not an underlying velar (/x/); see \sectref{sec:17.3.2} for discussion.

Since the trigger and target for velar fronting never span a morpheme boundary there is no German dialect in which the fronting of velars displays the kind of opaque effects typical of lexical rules discussed in the literature on \isi{Lexical Phonology and Morphology} and \isi{Stratal Optimality Theory}. For example, there is no German dialect in which certain suffixes trigger velar fronting but others do not, cf. \textit{national} vs. \textit{nationhood}, in which \is{Trisyllabic Laxing (English)}Trisyllabic Laxing is induced by the presence of -\textit{al}, but not by -\textit{hood}. Recall the discussion on stem level rules and word level rules from \sectref{sec:2.2.1}.

A recent model couched in the theory of \isi{Stratal Optimality Theory} postulates a mechanism of historical change called \isi{domain narrowing} (\sectref{sec:5.5.1}), which proposes that rules are phonologized at the end of the grammar and then gradually work their way up into smaller domains, e.g. the change from (a) phrasal level rule to a word level rule, and (b) word level rule to a stem level rule (\citealt{Bermúdez-Otero2007,Bermúdez-Otero2015,Ramsammy2015}). Since phrase level velar frontings are not attested, this book offers no evidence for (a). And since there is no evidence for the distinction between word level suffixes and stem level suffixes (cf. \ili{English}  {}-\textit{hood} vs. -\textit{al}), no dialect supporting (b) either.

\subsection{Status of irregular forms}\label{sec:12.8.3}

The sources cited in this book rarely state explicitly that velar fronting (synchronic or diachronic) is regular. However, it can be said that the descriptions for HG give no indication at all that the distribution of velars and palatals has idiosyncrasies modern linguists call \isi{lexical exceptions}. That generalization holds not only for those HG regions in which velar fronting has the status of an allophonic rule, but also for those HG localities identified in Chapters~\ref{sec:7}--\ref{sec:10} in which that process is a \isi{neutralization} (or \isi{quasi-neutralization}). The present section considers first the aberrant items in the \il{Westphalian}Wph variety of \ipi{Rhoden}  that were characterized earlier as irregular forms (\sectref{sec:5.2}) and then takes a closer look at them in the context of other varieties with similar data. I show below that the anomalous forms do not fit the profile of \isi{lexical exceptions} as that term is usually understood.

Given the regularity of velar fronting in HG it is interesting to recall that there are several lexical items in the LG (\il{Westphalian}Wph) variety of \ipi{Rhoden} \citep{Martin1925}, which unexpectedly contain [x] after a front vowel trigger. In \ipi{Rhoden}, velar fronting converts the target /x/ to palatal [ç] after a nonlow front vowel, but a small number of word were transcribed in the original source with [x] after high front vowels, e.g. [gəʃxɪxtə] ‘history’, [fʏxtə] ‘humidity’. Items like these are clearly surprising because a segment that belongs to the targets for velar fronting ([x]) stands after a segment that belongs to the triggers ([ɪ ʏ]).

\begin{sloppypar}
\ipi{Rhoden} is not unique: A number of descriptive grammars for LG present enough data to safely conclude that velar fronting is active, but those sources also note that [x] can occur unexpectedly in the context of front vowels; those vowels are typically lax ([ɪ ʏ ɛ œ]). The occurrence of [x] after front (lax) vowels is documented in the following quotes from original sources. The first one \citep{Martin1925} was already given in \sectref{sec:5.2}, but the others were not mentioned earlier:
\end{sloppypar}

\citet[14]{Martin1925} on the \il{Westphalian}Wph variety in \ipi{Rhoden} (\mapref{map:6}):

\begin{quote} 
...hört man sehr oft \textit{x} ... nach palatalen Vocalen

  “ ...one hears very often [x] after a front vowelˮ
\end{quote}

\citet[23]{Kloeke1914} on the NLG variety of \ipi{Finkenwärder} (\mapref{map:5}):

\begin{quote}
Es ist mir öfter aufgefallen, dass in schneller und schlaff artikulierter Rede das [χ] (=[ç]) wie [x] gesprochen wird, so sagt man [vɛx] weg, fort, [zɛx] gesagt statt [vɛχ] und [zɛχ]. Diese Aussprache scheint nur auf nachlässiger Artikulation zu beruhen, denn wenn ich das Wort noch einmal aus\-zu\-sprech\-en bat, wurde immer [χ] gesprochen.

“I have often observed that [χ] (=[ç]) is pronounced as [x] in fast and sloppily articulated speech; for example, [vɛx] ‘away, gone’, [zɛx] ‘say-\textsc{pret}’ are uttered instead of [vɛχ] and [zɛχ]. This pronunciation appears to be based solely on careless articulation, because [χ] was always uttered when I requested that the word be repeatedˮ.
\end{quote}

\citet[24]{Seelmann1908} on the \il{Brandenburgish}Brb variety of \ipi{Prenden} (\mapref{map:14}):\footnote{{In the transcription system of Seelmann and Holst (see below), the front vowels ⟦i⟧ and ⟦e⟧ in the irregular items are lax ([ɪ ɛ]).}}

\begin{quote}
Mnd. Ch erscheint nach palatalen Vokalen und nach Liquiden als χ, nach gutturalen Vokalen als x ... In gleicher Weise scheiden die meisten nd. Dialekte beide Laute, jedoch nicht alle. In mecklenburgischen Dörfern kann man sehr oft nixt, rext u.ä sprechen hören.

  “Middle Low German \textit{Ch} occurs after palatal vowels and liquids as \textit{χ}, and after guttural vowels as \textit{x} ... Most Low German dialects divide the sounds the same way, however not all [dialects]. In Mecklenburgian villages one can quite often hear nixt, rext being utteredˮ.
\end{quote}

\citet[156]{Holst1907} on the \il{Mecklenburgish-West Pomeranian}MeWPo variety of \ipi{Ivenack-Stavenhagen} (\mapref{map:14}):

\begin{quote}
  Im ursprünglichen Auslaut ist \textit{g} stimmloser Reibelaut geworden, und zwar gewöhnlich \textit{ich}{}- oder ach-Laut (χ – x), je nach dem vorhergehenden Vokal (dax = Tag, (ik) sēχ. = ich sah. … Doch kommt auch öfter ... ach-Laut für zu erwartenden ich-Laut vor (vex = Weg, nix = nicht).

  “In an original coda position \textit{g} has become a voiceless fricative, that is, the usual \textit{ich}{}- or \textit{ach}{}-sound depending on the preceding vowel (dax = Tag, (ik) sēχ. = I saw. … However, the \textit{ach}{}-sound often occurs in place of the expected \textit{ich}{}-sound (vex = Weg, nix = nicht)ˮ.
\end{quote}

\begin{sloppypar}
More recently, \citet[208]{Lauf1996} observes that [x] often occurs in \il{Westphalian}Wph colloquial speech (“westphälische Umgangsspracheˮ) after mid front vowels ([møːxlɪç] ‘possible’) and [l] (e.g. [mɪlx] ‘milk’).
\end{sloppypar}

It is clear from other sources that [x] can occur unexpectedly after the front rounded lax vowel [œ], although those data are often presented without comment. One example is the description of the \il{Westphalian}Wph variety of \ipi{Gütersloh} (\citealt{Wix1921}; \mapref{map:6}). According to the material given in that source (pp. 80-81) it can be concluded that [x] (=⟦x⟧) occurs after a back vowel and [ç] (=⟦χ⟧) after a front vowel. While [ç] is consistently transcribed after high front vowels and mid front unrounded vowels and [x] after back vowels, Wix is not consistent in the way he transcribes dorsal fricatives in the context after [œ]. Thus, he has [brœçtə] ‘bring\textsc{{}-opt}’ with his symbol for the palatal after his symbol for [œ] on p. 98, but the same word is given as [brœxtə] on p. 40. A second example of a word with [x] after [œ] is [kœxən] ‘cough-\textsc{inf}’ (p. 31).

The irregularity of velar fronting in the context after [œ] is also documented in the \il{Westphalian}Wph variety of \ipi{Lüdenscheid} (\citealt{Frebel1957}; \mapref{map:6}). The author provides a clear description of the distribution of dorsal fricatives on p. 34 suggesting that [x] (=⟦x⟧) surfaces after back vowels and [ç] (=⟦χ⟧) after front vowels. Among the words with [ç] is [lœçtə] ‘lamp’, illustrating the occurrence of [ç] after [œ], but on the same page he gives [kœxən] ‘cough\textsc{{}-inf}’, with [x] after the same vowel.

In \tabref{tab:12.36} I provide a list of the LG varieties cited above together with representative examples of irregular forms. I also include examples from LG dialects not mentioned earlier. The data from most of the sources below come from phonetically transcribed texts of individual speakers. This type of source is advantageous because it eliminates the possibility that data from different speakers are being intermingled. As indicated on \mapref{map:25}, all of the places listed in the first column of \tabref{tab:12.36} are in the same general region in North Germany.


\begin{table}
\caption{Selection of LG velar fronting varieties with irregularities (word-initial and/or postsonorant)\label{tab:12.36}}
\small
\begin{tabularx}{\textwidth}{lllQ}
\lsptoprule
Place/Region: & Dialect & Source: & Irregularities:\\
\midrule
\ipi{Finkenwärder} & NLG & \citet{Kloeke1914} & [vɛx] ‘away’, p.23\\
\ipi{Borgstede} & NLG & \citet{Feyer1939} & [lɪx] ‘lie\textsc{{}-3sg}’ p. 39,

[nɪx] ‘not’ p. 31\\
\ipi{Baden} & NLG & \citet{Feyer1941} & [lɪx] ‘lie\textsc{{}-3sg}’, p. 89\\
\ipi{Gütersloh} & \il{Westphalian}Wph & \citet{Wix1921} & [kœxən] ‘cough\textsc{{}-inf}’, p. 31\\
\ipi{Rhoden} & \il{Westphalian}Wph & \citet{Martin1925} & [gəʃxɪxtə] ‘history’,  p. 188;

[fʏxtə] ‘humidity’, p. 36\\
\ipi{Lüdenscheid} & \il{Westphalian}Wph & \citet{Frebel1957} & [kœxən] ‘cough\textsc{{}-inf}’\\
\ipi{Riesenbeck} & \il{Westphalian}Wph & \citet{Bethge1970} & [xrɑɔdə]  ‘straight’, p. 50;

[xøŋ] ‘go-\textsc{pret}’, p. 30\\
\ipi{Laer} & \il{Westphalian}Wph & \citet{Niebaum1974} & [sɛxs] ‘six’, p. 163\\
\ipi{Prenden} & \il{Brandenburgish}Brb & \citet{Seelmann1908} & [nɪxt] ‘not’, p. 24\\
\ipi{Ivenack-Stavenhagen} & \il{Mecklenburgish-West Pomeranian}MeWPo & \citet{Holst1907} & [vɛx] ‘path’, p. 156\\
\ipi{Schwerin} & \il{Mecklenburgish-West Pomeranian}MeWPo & \citet{Teuchert1927} & [dœrx] ‘trough’, p. 9\\
\ipi{Ratzeburg} & \il{Mecklenburgish-West Pomeranian}MeWPo & T\&S (1933)\footnote{\citep{TeuchertSchmitt1933}} & [brœxt] ‘bring-\textsc{part}’, p. 10\\
\ipi{Rostock} & \il{Mecklenburgish-West Pomeranian}MeWPo & T\&S (1933) & [mɪtbrœxt]

‘bring along-\textsc{part}’, p. 9\\
\ipi{Lank} & \il{Mecklenburgish-West Pomeranian}MeWPo & T\&S (1933) & [zɛx] ‘say-\textsc{imp}.\textsc{sg}’, p. 18\\
\ipi{South Stargard} & \il{Mecklenburgish-West Pomeranian}MeWPo & \citet{Teuchert1934} & [brœxt] ‘bring-\textsc{part}’,

[ʃlɪxt] ‘bad’, p. 13\\
\ipi{Bristow} & \il{Mecklenburgish-West Pomeranian}MeWPo & \citet{Schönfeld1989} & [bəzœxt] ‘visit-\textsc{part}’, p. 99\\
\lspbottomrule
\end{tabularx}
\end{table}

The items presented in the final column of \tabref{tab:12.36} are referred to here as “irregularities” and not as “\isi{lexical exceptions}” because they do not behave as the kind of \isi{lexical exceptions} discussed in the literature on phonology. Consider my own informal definition: A word (W\textsubscript{a}) is a \isi{lexical exception} to rule R if there is a string of segments (XYZ) in W\textsubscript{a} which satisfies the structural description of R, but R does not apply. That definition succeeds in characterizing a textbook case of exceptionality, namely the word \textit{obesity} (cf. \textit{obese}), which fails to undergo the \ili{English} rule of \is{Trisyllabic Laxing (English)}Trisyllabic Laxing (\sectref{sec:2.2.1}, \sectref{sec:12.8.2}), cf. \textit{sincerity} (cf. \textit{sincere}).\footnote{{There is a large body of research investigating the status of exceptions in phonology as well as other components of grammar (e.g. \citealt{Zonneveld1978}, \citealt{Wolf2011}, \citealt{SimonWiese2011}). The question of how to define what is meant by exception is dealt with in works such as these.}} Note crucially that the definition posited here presupposes that R consistently fails to affect XYZ in every occurrence of W\textsubscript{a}. Thus, the word \textit{obesity} consistently fails to undergo \is{Trisyllabic Laxing (English)}Trisyllabic Laxing for any given speaker in any given utterance.


Given this definition it is doubtful if any of the irregular forms from \tabref{tab:12.36} is a true \isi{lexical exception}. The reason is that in the sources cited the /x/ in the word in question (W\textsubscript{a}) fails to undergo velar fronting (=R) in some instances in a given text but in other instances (i.e. a few pages later in the same text for the same speaker), R correctly applies to the /x/ in W\textsubscript{a}. As a representative example, it was noted above that the irregular realization of the word [brœxtə] ‘bring\textsc{{}-opt}’ in \ipi{Gütersloh} is also realized with the expected pronunciation [brœçtə] \citep{Wix1921}. The quote given above from \citet[23]{Kloeke1914} for the NLG variety of \ipi{Finkenwärder} likewise implies that a speaker who utters an irregular form (e.g. [vɛx] ‘away’) might also pronounce the same word with the expected pronunciation (i.e. [vɛç]).

\begin{map}
% \includegraphics[width=\textwidth]{figures/VelarFrontingHall2021-img031.png}
\includegraphics[width=\textwidth]{figures/Map25_12.6.pdf}
\caption[Areal distribution of velar fronting varieties of Low German]{Areal distribution of velar fronting varieties of LG with irregularities involving the fluctuation between regular [ç] and unexpected [x] after coronal triggers.}\label{map:25}
\end{map}

Three questions can be posed: (A) Why do all of these examples in \tabref{tab:12.36} involve LG varieties in the same area of North Germany?; (B): If [x] is adjacent to a front vowel then why is that vowel typically lax?; (C) If the irregularities given in \tabref{tab:12.36} do not fit the standard profile of \isi{lexical exception}, then what are they?

Concerning (A): It is important not to lose sight of the fact that speakers with the items listed in \tabref{tab:12.36} lived in an area (North Germany) at a time (first part of the twentieth century) when the triggers (and targets) for velar fronting still differed from place to place. Recall from \sectref{sec:12.3.6} that the dialects spoken in that area (WLG) displayed considerably more variation with respect to velar fronting triggers than HG. This means that both during and after the \isi{acquisition} of velar fronting the speakers referred to in \tabref{tab:12.36} -- in contrast to speakers of the typical variety of HG -- must have been exposed to speakers with different versions of velar fronting (Appendix D).

Concerning (C): Given the diversity of velar fronting triggers in North Germany I contend that the irregularities listed above simply reflect the fact that many speakers in that area are influenced by the speech of individuals with alternate realizations of dorsal fricatives. For example, a speaker (P\textsubscript{1}) who acquires a broad set of triggers (after all front vowels) might have contact with a speaker (P\textsubscript{2}) with a restricted set (e.g. after all front unrounded vowels). P\textsubscript{1} pronounces sequences like [iç], [œç], [ɑx] in their own speech, but P\textsubscript{2} has the pronunciation [iç], [œx], [ɑx]. When P\textsubscript{1} utters an occasional word with [œx] this is simply an indication that their speech has been influenced by the speech of P\textsubscript{2}.\footnote{{The fact that the irregularities listed above consist of a only small set of words is a consequence of the relatively short length of the phonetically transcribed texts in the sources cited. The prediction is that speakers with irregular forms in all likelihood have more irregular forms that were not documented in the sources cited.}}

Concerning (B): As described in \sectref{sec:11.5} one LG (\il{East Pomeranian}EPo) variety (Kreis \ipi{Rummelsburg}) had a set of triggers restricted to front lax vowels. In \sectref{sec:12.7.2} I suggested that the triggers for Kreis \ipi{Rummelsburg} occupies a unique historical stage, namely Stage 2d'. I summarize the final three Trigger Types and historical stages from \tabref{tab:12.29} in \REF{ex:12:23}:

\ea%23
\label{ex:12:23}
Three Trigger Types/Historical Stages:\smallskip\\
\begin{tabular}[t]{@{}lll@{}}
C' & HFTV, MFTV, LFTV ([i e æ]) & 2c'{}'\\
D' & HFTV, MFTV, LFTV, CC ([i e æ r]) & 2d'\\
E  & HFV, MFV, LFV, CC & 2d\\
\end{tabular}
\z

The occurrence of [x] after front lax vowels for a speaker with a broad set of triggers (P\textsubscript{1}) indicates that their speech has been influenced by a speaker (P\textsubscript{2}) with a narrow one (Trigger Types C' and D').

A potential weakness with the present proposal is that Kreis \ipi{Rummelsburg} is the only dialect uncovered in this book with a set of velar fronting triggers defined according to \isi{tenseness}. What is more, Kreis \ipi{Rummelsburg} is geographically further to the east that even the easternmost marker on \mapref{map:24}. This may be true, but it is also conceivable that in an earlier time frame Trigger Types C' and D' were much more widespread in Northwest Germany and that those restricted sets of triggers were simply not recorded in the descriptive literature for WLG dialects.

A final question is whether or not it is correct to say that irregularities like the ones identified for LG are absent from HG. I would not make that claim. However, I do contend that it is difficult to find HG dialects akin to the LG ones discussed above because velar fronting is older in HG than in LG (\chapref{sec:16}). Given its age, velar fronting has had more time to diffuse itself in HG regions by adopting the full set of triggers (Trigger Type E). Seen in that light, there may have once been many HG varieties with irregular forms, but those aberrant items were eventually eliminated through time. In fact, I demonstrate below (\chapref{sec:13}) that Lower Bavaria has many velar fronting places with irregular forms. Not surprisingly, the villages and towns of Lower Bavaria differ in terms of what types of front vowels serve as triggers.

\section{{The} {fate} {of} {velars} {that} {do} {not} {undergo} {velar} {fronting}}\label{sec:12.9}
\subsection{General remarks}

The topic of this book is the categorical change from velar to palatal (or alveolopalatal), which is what I refer to as velar fronting. The goal of the present section is to discuss velars such as /x/ which do not undergo some version of that phonological process. The velars referred to here are present in dialects with or without velar fronting. In dialects with velar fronting, /x/ remains a simplex [dorsal] in the back vowel context. In dialects without velar fronting, /x/ is phonologically [dorsal] regardless of context. However, from the point of view of phonetics, all simplex [dorsal] segments can be pronounced in more than one way.

In \sectref{sec:12.9.1} I discuss dialects in which velars (/x/) show phonetic fronting. In \sectref{sec:12.9.2} I turn to systems in which velars surface either as velars or in which they are retracted to uvulars. The discussion focuses on the dorsal fricative /x/, although similar generalizations can be made concerning other velars (e.g. /k/, /g/); recall \sectref{sec:11.2}.

In the first column of \tabref{tab:12.37} I give the categories for dorsal fricatives employed in the present book and the corresponding phonetic symbols for fortis fricatives. In the second column I give the analysis of those four articulations in terms of the features posited earlier.

%%please move \begin{table} just above \begin{tabular
\begin{table}
\caption{Realization of /x/ in phonetics and phonology\label{tab:12.37}}
\begin{tabular}{ll}
\lsptoprule
Phonetic realization & Phonological features\\\midrule
Palatal ([ç]) & [coronal, dorsal] \\
Prevelar ([x̟]) & [dorsal] \\
Velar ([x]) & [dorsal]\\
Uvular ([χ]) & [dorsal]\\
\lspbottomrule
\end{tabular}
\end{table}

The realization of simplex [dorsal] sounds as fronted (\isi{prevelar}) or retracted (uvular) is not expressed in phonological features because it is a function of \is{phonetic rule}phonetic rules.

Evidence for the phonetics of /x/ I summarize below is based on statements made in descriptive grammars of German dialects. Future work may want to conduct phonetic studies in order to (dis)confirm some of those claims.

\subsection{Phonetic fronting}\label{sec:12.9.1}

The phonetic fronting of velars to the sound I refer to as \isi{prevelar} is well-attested in the literature on German dialects. The examples I discuss below are drawn from non-velar fronting (UG) varieties in Switzerland, Austria, and Northeast Italy.

A number of sources provide a description of the pronunciation of velars which suggest that they are fronted in the context of front vowels by \isi{coarticulation}. Consider the following examples:

\citet{Kurath1965} describes the \il{South Bavarian}SBav dialect spoken in \ipi{St. Ruprecht bei Villach} (\mapref{map:3}) in the Austria state of Carinthia (\ipi{Kärnten}). In the following passage \citet[32]{Kurath1965} discusses the complementary distribution of [h] and [x], which he considers to be allophones, as well as the phonetics of [x]:

\begin{quote}
Im Anlaut, wo er [=x] nur vor Vokal vorkommt, wird er als [h] aus\-ge\-sproch\-en, in anderen Stellungen als ein velarer Reibelaut [x]. Nach hintervokalen und nach dem velaren r ... ist [x] aus\-ge\-sproch\-en hintergaumig, nach Vordervokalen und nach n, l mittelgaumig (nicht vordergaumig wie im Bühnenhochdeutschen) ….

“In a [word-initial] onset, where it only occurs before vowels, [x] is pronounced as [h]; in other positions [it is pronounced] as a velar fricative. After back vowels and after the velar r it is a markedly back-palate sound, after front vowels it is a mid-palate sound (not a front-palate sound as in Standard German) …ˮ
\end{quote}

The point is that the velar fricative [x] in St. Ruprecht has a gradiently fronted variant -- Kurath’s mid-palate sound, which is the equivalent of my \isi{prevelar} -- in the context after coronal sonorants. Significantly, the \isi{prevelar} is not articulated as palatal [ç] (“front-palate soundˮ in Kurath’s terms).

\citet{Rowley1986} is a detailed description of the phonetics and phonology of the German-language (UG) island of \ipi{Fersental} in Northeast Italy (\mapref{map:3} and \mapref{map:35} below). The dorsal obstruents in Fersentalerisch (Mòcheno) are classified as velar (p. 65). \citet[143]{Rowley1986} provides the following remark on the pronunciation of velar fricatives [ɣ] and [x]):\footnote{The fricative [x] is analyzed as an allophone of /h/ and [ɣ] as an allophone of /g/.}

\begin{quote}
Nach vorderen Vokalen /i/, /e/, /ɛ/ usw. wird das [x] bzw. [ɣ] etwas weiter nach vorne gebildet als nach hinteren Vokalen /o/, /u/ usw. Manchmal nä\-hert sich die Aussprache dem hdt. [ç]: [mïlx+] …\textit{Milch}.

“After front vowels /i/, /e/, /ɛ/ etc., [x] and [ɣ] are produced somewhat more front than after back vowels /o/, /u/ etc. Sometimes the pronunciation approaches standard German [ç]: [mïlx+] …Milchˮ.
\end{quote}

As in St. Ruprecht, [x] undergoes coarticulatory fronting (to \isi{prevelar}) in Fersentalerisch, but neither variety has the phonological process of velar fronting.

The coarticulatory fronting of velars has been documented for other German-language (UG) islands as well. Two examples are \ipi{Gottschee} (Bav) in modern-day Slovenia (\citealt{Tschinkel1908}: 26) and the Cimbrian variety of the \ipi{Sieben Gemeinden} (“Seven Communitiesˮ) in and around Asiago in Northeast Italy (\citealt{Kranzmayer1956}: 50). Both places are indicated on \mapref{map:3}, \mapref{map:35}, and \mapref{map:36}.

Similar examples from SwG dialects can be found in the descriptive literature as well. One case known to me is implied in the statement made by \citet[11]{Baumgartner1922} on the realization of /x/ in the \il{High Alemannic}HAlmc variety spoken in the area around \ipi{Bern}. On that page he notes that his dialect only has velar [x], which can be slightly fronted in the context of front sounds. (“Unsere Mundarten kennen nur das velare x … In palataler Umgebung setzt der Laut ganz wenig vorn ein … ˮ).

The \il{South Bavarian}SBav variety of \ipi{Laurein} (\citealt{Kollmann2007}; \mapref{map:3}) possesses a single fortis dorsal fricative (< \ili{WGmc} \textsuperscript{+}[k x]) as well as the corresponding \isi{affricate}. According to \citet[175]{Kollmann2007} those sounds surface as \isi{prevelar} (“prävelarˮ), which is very close (but not identical) to the palatal articulation (ich-Laut) of \il{Standard German}StG. The reason \ipi{Laurein} is different from the other UG varieties discussed above is that it does not involve coarticulatory fronting. The reason is that /x/ and /kx/ are realized as \isi{prevelar} in the context of any type of sound, including front vowels.

The requirement that \ipi{Laurein} /x/ and /kx/ be articulated as phonetically front\-ed velars (prevelars) -- but not as palatals -- is due to a dialect-specific rule of \isi{phonetic implementation}. Recall from \sectref{sec:2.2.1} that \isi{phonetic implementation} is necessary to specify fine-grained place and manner distinctions for consonants that are not necessary in the phonology. For example, a rule of \isi{phonetic implementation} is required to indicate the exact place of articulation of [coronal] stops (/t d/) as alveolar or dental. A similar rule is present in \ipi{Laurein}; namely that the one requiring that /x/ surfaces as a \isi{prevelar} (but not as a palatal, velar, or uvular).\largerpage

\ipi{Laurein} is the clearest case to my knowledge of a variety of German in which /x/ undergoes small-scale (phonetic) fronting even in the neighborhood of back sounds. Based on the terse statement regarding the realization of /x/ in the \il{Low Alemannic}LAlmc dialect spoken in northwest Switzerland (\mapref{map:1}), \citet[30]{Schläpfer1956} implies that the same kind of \isi{phonetic implementation} is attested in that variety as well. (“χ (=[x]) wird normalerweise palatal … gebildet”). However, Schläpfer points out in a footnote on the same page that the fronted articulation is not as front as \il{Standard German}StG [ç].


\subsection{Phonetic retraction}\label{sec:12.9.2}\is{phonetic retraction|(}
An underlying segment like /x/ can surface as velar ([x]) or it can have a retracted realization as uvular ([χ]); recall \sectref{sec:1.5} and \tabref{tab:12.37}. I discuss below examples from the descriptive literature on German dialects for those two articulations ([x] and [χ]). 

Even in a well-researched language like \il{Standard German}StG the term “velar” is often misused. Anyone knowledgeable about \il{Standard German}StG knows that the ach-Laut after a low back vowel (e.g. in words like \textit{Bach} ‘stream’ and \textit{machen} ‘do-\textsc{inf}’) is uvular and not velar (\citealt{Kohler1990b, Kohler1990a}), but there is nevertheless a tradition of referring to the sound as “velar” and transcribing it as [x].\footnote{The realization of the ach-Laut after the vowel [ɑ] as uvular is particularly clear in the x-ray tracing (Tafel 11)  in \citet{Wängler1981}. Wängler refers to that realization of the ach-Laut as “velar-postdorsal”.} One example of a work that follows this tradition is the most well-known pronouncing dictionary for \il{Standard German}StG (\citealt{Mangold2005}: 44, 46, 52).

It is conceivable that there is a system in which [x] and [χ] are both present and that the two are contextually determined. If there were such a German dialect, then the uvular realization would be a consequence of \isi{coarticulation}. To the best of my knowledge, no German dialect has been described in this manner, although the reader is referred to the discussion in \sectref{sec:1.5} on the contextually-determined distribution of [x] and [χ] in \il{Standard German}StG.

In the descriptive literature referred to throughout this book, the ach-Laut ([x]) is usually classified as a phonetic velar. By contrast, other authors observe that /x/ is pronounced as uvular ([χ]) regardless of the neighboring sounds. Sometimes [x] and [χ] are assumed to be free variants; in other cases the difference between [x] and [χ] is shown to be a function of geography.  In all such studies /x/ is not a product of \isi{coarticulation}, but instead of \isi{phonetic implementation}. One study mentioned earlier (\sectref{sec:12.3}) is \citet{Ibrom1971}, who notes that /x/ is pronounced as velar ([x]) in the \il{Swabian}Swb variety between Augsburg and Donauwörth (\mapref{map:1}) but as either velar ([x]) or uvular ([χ]) in the \il{Central Bavarian}CBav varieties between Augsburg and Aichach (\mapref{map:3}). According to \citet{Stein-Meintker2000}, the uvular [χ] pronunciation is the most common realization of /x/ in \ipi{Garmisch-Partenkirchen} (\il{Central Bavarian}CBav; \mapref{map:3}). In \ipi{Zürich} German (\il{High Alemannic}HAlmc; \mapref{map:2}), \citet[244]{FleischerSchmid2006} observe that /x/ (and its lenis counterpart) are in free variation with the corresponding uvulars. The final example mentioned here is \citet[100-101]{Hove2002}, who demonstrates that there are three ways of realizing /x/ in SwG, namely velar, velar with a slightly lowered tongue body, or uvular.\is{phonetic retraction|)}

\section{{Conclusion}}\label{sec:12.10}

The aim of this chapter has been to summarize the attested targets and triggers for velar fronting -- in both word-initial and postsonorant position -- in a large selection of German dialects representing all subdivisions of German dialects from Appendix~\ref{appendix:a}. The study has determined that dialects can be classified according to the generality of targets and triggers; hence, some varieties have a very specific set of triggers and/or targets, while others have a much broader set. The findings have been argued to support the model of \isi{rule generalization}, according to which language change (velar fronting) began with a small set of targets and triggers and then expanded through time and space to include more and more target segments and more and more triggers.

In the following chapter I consider data drawn from a linguistic atlas (SNiB) which provides evidence that velar fronting is active throughout Lower Bavaria. I demonstrate in \chapref{sec:13} that three of the historical stages posited in the present chapter on the basis of velar fronting triggers are attested in Lower Bavaria, namely Stage 2a (after high front vowels), Stage 2b (after nonlow front vowels, and Stage 2c' (after all front vowels).
