\addchap{\lsAbbreviationsTitle}


\section*{Dialects (Varieties) of German and historical stages of German}
{\small
\begin{tabularx}{.5\textwidth}[t]{@{}lQ}
Almc & Alemannic\\
Bav & Bavaria\\
Brb & Brandenburgish\\
CBav & Central Bavarian\\
CFr & Central Franconian\\
CG & Central German\\
CHes & Central Hessian\\
CPo & Central Pomeranian\\
ECG & East Central German\\
EFr & East Franconian\\
EGmc & East Germanic\\
EHes & East Hessian\\
ELG & East Low German\\
ENHG & Early New High German\\
Eph & Eastphalian\\
EPo & East Pomeranian\\
Gmc & Germanic\\
Go & Gothic\\
HAlmc & High Alemannic\\
Hes & Hessian\\
HG & High German\\
HPr & High Prussian\\
HstAlmc & Highest Alemannic\\
IE & Indo-European\\
LAlmc & Low Alemannic\\
LFr & Low Franconian\\
LG & Low German\\
LPr & Low Prussian\\
LRG & Lower Rhine German\\
Lxm & Luxembourgish\\
ME & Middle English\\
MeWPo & Mecklenburgish-West Pomeranian\\
\end{tabularx}\begin{tabularx}{.5\textwidth}[t]{lQ@{}}
MFr & Moselle Franconian\\
MHG &  Middle High German\\
NBav & North Bavarian\\
NGmc & North Germanic\\
NHes & North Hessian\\
NLG & North Low German\\
NUSax-SMk & North Upper Saxon-South Markish\\
OE & Old English\\
OHG & Old High German\\
ON & Old Norse\\
OSax & Old Saxon\\
PGmc & Proto-Germanic\\
RFr & Rhenish Franconian\\
Rpn & Ripuarian\\
SBav & South Bavarian\\
Sln & Silesian\\
StAG & Standard Austrian German\\
StG & Standard German\\
StSwG & Standard Swiss German\\
Swb & Swabian\\
SwG & Swiss German\\
Thrn & Thuringian\\
Tyr & Tyrolean\\
UG & Upper German\\
USax & Upper Saxon\\
WCG & West Central German\\
WGmc & West Germanic\\
WLG & West Low German\\
Wph & Westphalian
\end{tabularx}}\clearpage

\section*{Grammatical categories}
\begin{multicols}{3}
\begin{tabbing}
infl~~ \= inflected\kill
1     \>  first person\\
2     \>  second person\\
3     \>  third person\\
acc   \>   accusative\\
adj   \>   adjective\\
dat   \>   dative\\
dim   \>   diminutive\\
fem   \>   feminine\\
gen   \>   genitive\\
imp   \>   imperative\\
infl  \>   inflected\\
obl   \>  oblique\\
gen   \>  genitive\\
loc   \>  locative\\
nom   \>  nominative\\
part  \>   participle\\
pl    \>  plural\\
pret  \>  preterite\\
sg    \>  singular\\
subj  \>  subjunctive\\
wd    \>  word
\end{tabbing}
\end{multicols}

\section*{Features}
\begin{multicols}{3}
\begin{tabbing}
cons \= consonantal\kill
cons  \>  consonantal\\
cont  \>  continuant\\
cor   \>  coronal\\
dors  \>  dorsal\\
nas   \>  nasal\\
per   \>  peripheral\\
son   \>  sonorant
\end{tabbing}
\end{multicols}

\section*{Other abbreviations}
\begin{multicols}{2}
\begin{tabbing}
NUSax-S \= North Upper Saxon-South Markish~~ \kill
Ba    \>  back sound\\
BV    \>  back vowel\\
C     \>  consonant\\
CC    \>  coronal sonorant consonant\\
Fr    \>  front sound\\
FV    \>  front vowel\\
HBV   \>   high back vowel\\
HFTV  \>  high front tense vowel\\
HFUV  \>  high front unrounded    vowel\\
HFV   \>  high front vowel\\
LBV   \>  low back vowel\\
LFTV  \>  low front tense vowel\\
LFV   \>  low front vowel\\
MBV   \>  mid back vowel\\
MFTV  \>  mid front tense vowel\\
MFV   \>  mid front vowel\\
Pa    \>  palatal\\
V     \>  vowel\\
Ve    \>  velar
\end{tabbing}
\end{multicols}

\section*{Symbols}
\begin{tabbing}
\_\_\_ C\textsubscript{0} ]\textsubscript{${\sigma}$} \= coda position\kill
\relax [ ... ]  \>   phonetic representation\\
/ ... /  \>  underlying  representation\\
⟦ ... ⟧  \>  phonetic representation  in an original source\\
{\textbar} ... {\textbar}  \>  intermediate synchronic phonological representation\\
\textsubscript{wd}[ A   \>    A is at the left edge of a   word\\
${\sigma}$  \>  syllable\\
C\textsubscript{0}  \>  zero or more consonants\\
\_\_\_ C\textsubscript{0} ]\textsubscript{${\sigma}$}  \>  coda position of a syllable\\
\{A B\}  \>  the set of A and B\\
A > B  \>  A is realized in the next historical stage as B\\
B < A  \>  \\
A {\textasciitilde} B  \>  A and B are   morphophonemic alternants\\
A → B   \>  A is realized as B as the  output of a synchronic  phonological rule\\
 A . B  \>  syllable boundary between A and B\\
 A - B  \>  morpheme boundary between A and B\\
\~{A}  \>  A is nasalized\\
A̟  \>  A is fronted\\
A̠  \>  A is retracted\\
ˈA  \>  A is stressed\\
\textsuperscript{+}A  \>  A is reconstructed\\
\textsuperscript{*}A  \>  A is ungrammatical\\
Aʰ   \> A is aspirated\\
x̟   \> \isi{prevelar} fricative\\
\end{tabbing}


\tabref{tab:abbv:1} (p.~\pageref{tab:abbv:1}) lists the phonetic symbols for those consonants and glides referred to in this book. The rhotics [r] and [ʀ] can surface as trills, taps or approximants. The distinction between those articulations is not important and is therefore not expressed in phonetic representations. The fricatives [s z ʃ ʒ ɕ ʑ] and the affricates [ts tʃ dʒ] are sibilants. All other sounds -- in particular the fricatives [ç ʝ x ɣ] and the \isi{affricate} [kç] -- are nonsibilants. [h ɦ] are assumed to be obstruents. Many of the obstruents in \tabref{tab:abbv:1} occur in pairs, e.g. [p b], [f v], [ç ʝ], [x ɣ]. I refer to the first sound in each pair (e.g. [p f ç x]) as fortis and the second (e.g. [b v ʝ ɣ]) as lenis.

\begin{table}
\caption{\label{tab:abbv:1}Phonetic symbols for consonants and glides}
\begin{tabular}{lcccccccccc} 
\lsptoprule
& \rotatebox{90}{bilabial} & \rotatebox{90}{labio-dental} & \rotatebox{90}{dental} & \rotatebox{90}{alveolar} & \rotatebox{90}{post-alveolar} & \rotatebox{90}{alveolo-palatal} & \rotatebox{90}{palatal} & \rotatebox{90}{velar} & \rotatebox{90}{uvular} & \rotatebox{90}{glottal}\\\midrule
stop & p b &  &  & t d &  &  & c ɉ & k g &  & \\
\isi{affricate} & pf &  &  & ts & tʃ dʒ &  & kç & kx &  & \\
fricative & β & f v & θ ð & s z & ʃ ʒ & ɕ ʑ & ç ʝ & x ɣ & χ ʁ & h  ɦ\\
nasal & m &  &  & n &  &  & ɲ & ŋ &  & \\
lateral &  &  &  & l &  &  &  &  &  & \\
rhotic &  &  &  & r &  &  &  &  & ʀ & \\
glide & w & ʋ &  &  &  &  & j &  &  & \\
\lspbottomrule
\end{tabular}
\end{table}

\tabref{tab:abbv:2} (p.~\pageref{tab:abbv:2}) lists the phonetic symbols for the most important vowels referred to in the following chapters. \isi{Full vowels} are all vowels listed above with the exception of \isi{schwa} ([ə]). The glide in diphthongs is not indicated with a diacritic, e.g. [ai] in \textit{Zeit} ‘time’ (as opposed to [ai̯]). The sound [ɐ] is not included in the chart for vowels. It represents the phonetically upper-low glide/vowel in words like \textit{Tier} ‘animal’ and \textit{Vater} ‘father’, often referred to as the vocalized-r. The distinction between phonetically back vowels and phonetically central vowels is not important and hence all nonfront vowels are classified as back. The difference between vowel pairs like [i] and [ɪ] is referred to throughout this book in terms of \isi{tenseness} (e.g. [i] is tense and [ɪ] is lax]). The tense-lax pairs [e ɛ] and [ø œ] are assumed to be mid, although the lax members ([ɛ œ]) are phonologically low in some dialects. If there is only one low back vowel then that vowel is transcribed as [ɑ]. The symbol [a] is used to represent a low back vowel distinct from [ɑ]; those two articulations are assumed to differ in terms of \isi{tenseness}.

\begin{table}
\caption{\label{tab:abbv:2}Phonetic symbols for vowels}
\begin{tabular}{lcccc}
\lsptoprule
     & \multicolumn{2}{c}{front} & \multicolumn{2}{c}{back}\\\cmidrule(lr){2-3}\cmidrule(lr){4-5}
     &  unrounded &  rounded &  unrounded &  rounded\\\midrule
high & i ɪ & y ʏ &  & u ʊ\\
mid  & e ɛ & ø œ & ə ʌ & o ɔ\\
low  & æ &  & a ɑ & ɒ\\
\lspbottomrule
\end{tabular}
\end{table}
