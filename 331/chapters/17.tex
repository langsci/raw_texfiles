\chapter{Velar Fronting in Standard German}\label{sec:17}\il{Standard German|(}

\begin{epigraphs}
\qitem{Ihnen beiden verschiedenen Lauten des \textit{ch} … weiss ich keine schicklicheren Namen zu geben, als wenn ich jenen den Achlaut, diesen aber den Ichlaut nenne.\footnotemark{}}{Gottfried August \citet[131]{Bürger1798}}

\end{epigraphs}
\footnotetext{“I do not know a more fitting name to give the two different sounds of \textit{ch} … than if I call the one the ach-Laut and the other the ich-Laut".}


\section{{Introduction}}\label{sec:17.1}

Previous chapters have scrutinized the status of velar fronting in a broad selection of regional varieties of German. The goal of the present chapter is to discuss the patterning of the ich-Laut and the ach-Laut in \il{Standard German}StG and to demonstrate that the distribution of those sounds reflects patterns encountered in previous chapters. \sectref{sec:17.2} presents a representative selection of data and an analysis thereof, and \sectref{sec:17.3} concludes by considering three of the research questions from \sectref{sec:1.4.4} in light of the treatment of \il{Standard German}StG. \sectref{sec:17.2} also includes a few brief remarks on the distribution of [ç] and [x] in the standard German language of Austria (\ili{StAG}) and shows how \il{Standard German}StG differs from StAG.

\section{{Data} {and} {analysis}}\label{sec:17.2}

\il{Standard German}StG (\citealt{Siebs1969}, \citealt{Krech1982}, \citealt{Mangold2005}) has the phonemic front vowels  /iː ɪ yː ʏ eː ɛ ɛː øː œ/, the phonemic back vowels /uː ʊ oː ɔ ɑː ɑ ə/, and the three phonemic diphthongs /ɑi ɔy ɑu/. The literature cited in \sectref{sec:1.2} has focused almost exclusively on the data described below.

The two dorsal fricatives are [x] and [ç]. Lenis [ɣ] is not a surface sound, although there is a synchronically derived {\textbar}ɣ{\textbar} (from /g/), as in LRG (\sectref{sec:5.3.1}). There is no lenis palatal fricative ([ʝ]).\footnote{{A long-standing debate in the literature is whether or not the initial sound in words like} \textrm{\textit{ja}} \textrm{‘yes’ is a fricative ([ʝ]) or a glide ([j]). In contrast to many of the LG and CG varieties discussed in the present book, \il{Standard German}StG does not have alternations between [ɣ] and [ʝ] indicating that the latter sound patterns phonologically like a fricative. I treat the \il{Standard German}StG sound represented by} \textrm{\textit{j}} \textrm{henceforth as the glide ([j]). See \citet{Wiese1996a} and \citet{Hall2007} for discussion and formal treatments.}}

The patterning of [x] and [ç] is expressed for postsonorant position in \REF{ex:17:1}.\footnote{{Neither of those sounds occur in word-initial position in the native lexicon. The basic generalizations concerning the patterning of word-initial [x] and [ç] in loanwords is unclear and is therefore not discussed in the present chapter. See Appendix~\ref{appendix:g} and \citet{Robinson2001} for elaboration.}}

\ea\label{ex:17:1} \begin{forest}
[,phantom
  [/x/,name=x [{[x]}]]    
  [/ç/ [{[ç]},name=c]]
]
\draw (x.south) -- (c.north);
\end{forest}
\z
     

The patterning of [ç] and [x] can be summarized as follows: (A) [ç] -- but not [x] -- surfaces after a front vowel but not after a phonemic back vowel, and [x] -- but never [ç] -- occurs after a phonemic back vowel but not after a front vowel, (B) [ç] surfaces after the two coronal sonorant consonants [n l], but [x] never does, (C) [ç] -- but never [x] -- occurs after the back vowel [ɐ] or after the dorsal consonant [ʀ], both of which derive from /ʀ/, and (D)  [ç] -- but never [x] -- is the realization of \textit{ch} in the diminutive suffix -\textit{chen} regardless of the nature of the preceding sound. I demonstrate below that [ç] and [x] in (A)-(B) derive from /x/ by velar fronting, while the [ç] in (C)-(D) is an underlying palatal (/ç/). As discussed below, the contexts described in (C) and (D) involve (historical) \isi{overapplication} \isi{opacity} because [ç] (from an earlier velar) was historically preceded by a front ([coronal]) sound.

The items listed below exemplify generalization (A): [x] surfaces after phonemic back vowels in (\ref{ex:17:2}a and \ref{ex:17:3}a) and [ç] after front vowels in (\ref{ex:17:2}b and \ref{ex:17:3}b). The dorsal fricatives in \REF{ex:17:2} are in coda position, but the same sounds are in intervocalic position in \REF{ex:17:3}. The data in \REF{ex:17:2} and \REF{ex:17:3} together therefore show that the syllable cannot be a factor in the distribution of [x] and [ç]. [x ç] in examples like the ones in \REF{ex:17:2} and \REF{ex:17:3} are the modern realizations of historical fortis velars (\ili{WGmc} *[k x]).\footnote{{There are several accidental gaps. For example, no native words are attested in which a dorsal fricative occurs after [eː], although [ç] surfaces after short [e] in the nonnative word} \textrm{\textit{Mechanik}} \textrm{‘mechanics’. After [oː] and before a vowel, [x] is apparently only attested in the toponym} \textrm{\textit{Bochum}}\textrm{. The only word to my knowledge with a dorsal fricative ([ç]) following [øː] is the realization of the morpheme} \textrm{\textit{hoch}} \textrm{‘high’ with an umlauted stem vowel (i.e. [høːç-] in [høːçst] ‘extreme’). Finally, no dorsal fricatives occur after [ə].}}\pagebreak\largerpage

\TabPositions{.2\textwidth, .4\textwidth, .6\textwidth, .75\textwidth}
\ea%2
\label{ex:17:2}Postvocalic dorsal fricatives (from /x/) in the coda:
\ea 
\relax [tuːx] \tab Tuch \tab ‘towel’\\
\relax [bʊxt] \tab Bucht \tab ‘bay’\\
\relax [hoːx] \tab hoch \tab ‘high’\\
\relax [kɔx] \tab Koch \tab ‘cook’\\
\relax [nɑːx] \tab nach \tab ‘after’\\
\relax [bɑx] \tab Bach \tab ‘stream’\\
\relax [bɑux] \tab Bauch \tab ‘stomach’
\ex \relax [ziːç] \tab siech \tab ‘ailing’\\
\relax [lɪçt] \tab Licht \tab ‘light’\\
\relax [gəʀʏçt] \tab Gerücht \tab ‘rumor’\\
\relax [gəʃpʀɛːç] \tab Gespräch \tab ‘conversation’\\
\relax [ʀɛçt] \tab recht \tab ‘right’\\
\relax [høːçst] \tab höchst \tab ‘extreme’\\
\relax [vœç.nə.ʀɪn] \tab Wöchnerin \tab ‘woman in childbed’\\
\relax [ʀɑiç] \tab Reich \tab ‘empire’\\
\relax [ɔyç] \tab euch \tab ‘you-\textsc{dat/acc}.\textsc{pl}’
  \z
\ex%3
\label{ex:17:3}Postvocalic dorsal fricatives (from /x/) before a vowel:
\ea \relax [kuːxən] \tab Kuchen \tab ‘cake’\\
\relax [bəɑnʃprʊxən] \tab beanspruchen \tab ‘claim\textsc{{}-inf}’\\
\relax [knɔxən] \tab Knochen \tab ‘bone’\\
\relax [ʃprʀɑːxə] \tab Sprache \tab ‘language’\\
\relax [mɑxən] \tab machen \tab ‘do\textsc{{}-inf}’\\
\relax [tɑuxən] \tab tauchen \tab ‘dive\textsc{{}-inf}’\\
\ex \relax [ʀiːçən] \tab riechen \tab ‘smell\textsc{{}-inf}’\\
\relax [møːklɪçə] \tab mögliche \tab ‘possible\textsc{{}-infl}’\\
\relax [flyːçə] \tab Flüche \tab ‘curse-\textsc{pl}’\\
\relax [kʏçə] \tab Küche \tab ‘kitchen’\\
\relax [gəmɛːçɐ] \tab Gemächer \tab ‘chamber-\textsc{pl}’\\
\relax [lœçɐ] \tab Löcher \tab ‘hole-\textsc{pl}’\\
\relax [ɑiçə] \tab Eiche \tab ‘oak tree’\\
\relax [kɔyçən] \tab keuchen \tab ‘gasp\textsc{{}-inf}’
    \z
\z 


The distribution of [x] and [ç] as in \REF{ex:17:2} and \REF{ex:17:3} is also reflected in many morphophonemic alternations like the one in \REF{ex:17:4}: [x] surfaces after a back vowel in the morphologically underived word (e.g. singular noun) and [ç] after the corresponding front vowel (via \isi{Umlaut}) in the morphologically derived word (e.g. plural noun). As in \REF{ex:17:2} and \REF{ex:17:3}, [x ç] in examples like these derived historically from \ili{WGmc} *[k] or *[x].

\ea%4
\label{ex:17:4}\relax [x]{\textasciitilde}[ç] alternations (from /x/):
\ea
\relax [buːx]  \tab  Buch \tab ‘book’\\
\relax [byːçɐ]  \tab  Bücher \tab ‘book-\textsc{pl}’
\ex
\relax [lɔx]  \tab  Loch \tab ‘hole’\\
\relax [lœçɐ]  \tab  Löcher \tab ‘hole-\textsc{pl}’
\ex 
\relax [bɑx]  \tab  Bach \tab ‘stream’\\
\relax [bɛçə]  \tab  Bäche \tab ‘stream-\textsc{pl}’
\z 
\z

The data in (\ref{ex:17:2}--\ref{ex:17:4}) are captured by analyzing the dorsal fricatives as /x/, which surfaces as palatal after a front vowel by \isi{Velar Fronting-1}:

\ea%5
\label{ex:17:5}
\isi{Velar Fronting-1}:\\
\begin{forest}
[,phantom
    [\avm{[+son]} [\avm{[coronal]},name=target,tier=word]]
    [\avm{[−son\\+cont]},name=source [\avm{[dorsal]},tier=word]]
]
\draw [dashed] (source.south) -- (target.north);
\end{forest}
\z 

A second source for the surface (coda) palatal fricative [ç] can be seen in (\ref{ex:17:6a}, \ref{ex:17:6b}). These words illustrate an alternation between [g] and [ç] after the vowel [ɪ]: The alternant with [ç] occurs in coda position and the one with [g] before a vowel. The [g]{\textasciitilde}[k] alternations in \REF{ex:17:6c} show that coda /g/ -- like all other voiced obstruents -- undergoes \isi{Final Fortition} to [k] after any vowel other than [ɪ]. The [g]{\textasciitilde}[ç] alternations in (\ref{ex:17:6a}, \ref{ex:17:6b}) are analyzed in the literature cited earlier with an underlying /g/ that spirantizes to [ɣ] in the coda after the vowel [ɪ] by \isi{g-Spirantization-2} in (\ref{ex:17:7}); cf. \isi{g-Spirantization-1}, which applies in the context after all vowels (\sectref{sec:4.2}). Alternating [g] and [ç] in examples like the ones in \REF{ex:17:6} derived historically from \ili{WGmc} *[ɣ].\footnote{According to \citet{Mangold2005}, the stem-final sound in words like the ones in (\ref{ex:17:6a}, \ref{ex:17:6b}) is realized as [k] -- and not as the expected [ç] -- in the context after [ɪ] and before a morpheme containing [ç], e.g. \textit{königlich} [køːnɪk.lɪç] ‘royal’. I do not discuss this type of example because it is not directly related to the topic of velar fronting.}

\ea%6
\label{ex:17:6}[g]{\textasciitilde}[ç] alternations (from /g/):
 \ea \label{ex:17:6a}
\relax [køːnɪç] \tab König \tab ‘king’\\
\relax [køːnɪgə] \tab Könige \tab ‘king-\textsc{pl}’\\
\ex\label{ex:17:6b}
\relax [leːdɪç] \tab ledig \tab ‘single’\\
\relax [leːdɪgə] \tab ledige \tab ‘single\textsc{{}-infl}’
\ex \label{ex:17:6c}
\relax [tɑk] \tab Tag \tab ‘day’\\
\relax [tɑigə] \tab Tage \tab ‘day-\textsc{pl}’
    \z
\ex%7
\label{ex:17:7}\isi{g-Spirantization-2}:\smallskip\\
\avm{[−son\\−cont\\−fortis\\dorsal]}  → [+cont] / ɪ {\longrule}\ C\textsubscript{0} ]\textsubscript{${\sigma}$}
\z 

In examples like \textit{König} and \textit{ledig} in (\ref{ex:17:6a}, \ref{ex:17:6b}) \isi{g-Spirantization-2} produces a derived coda {\textbar}ɣ{\textbar} which shifts to {\textbar}x{\textbar} via \isi{Final Fortition} and then surfaces as [ç] by \isi{Velar Fronting-1}. Hence, surface [ç] in \il{Standard German}StG can derive from /x/ in \REF{ex:17:2}-\REF{ex:17:4} or from /g/ in (\ref{ex:17:6a}, \ref{ex:17:6b}). See \citet[228]{Hall1992}, \citet[207; 211--212]{Wiese1996a}, \citet{Robinson2001}, \citet{ItoMester2002}, and \citet{Glover2011, Glover2014} for formal treatments of \isi{g-Spirantization-2} in \il{Standard German}StG.\footnote{{\isi{Final Fortition} \isi{counterbleeds} \isi{g-Spirantization-2}, otherwise the final segment a word like /køːnɪg/ would shift to {\textbar}k{\textbar} and \isi{bleed} \isi{g-Spirantization-2}. As in \ipi{Altengamme} (\sectref{sec:4.2}), the type of \isi{counterbleeding} relationship between \isi{Final Fortition} and spirantization described here does not involve \isi{opacity}.} }

 A potential drawback with \isi{g-Spirantization-2} involves [g]{\textasciitilde}[ç] alternations after the diphthong /ɑi/, e.g. [tɑik] ‘dough’ vs. [tɑigɪç] ‘doughy’. If the second part of /ɑi/ is analyzed as /ɪ/ (e.g. \citealt{Hall1992}, \citealt{Wiese1996a}), then the incorrect prediction is made that the /g/ should surface as [ç] in coda position in words like [tɑik] (from /tɑɪg/). I argue that the /ɪ/ which serves as the vocalic trigger for \isi{g-Spirantization-2} is phonologically [--tense] because it contrasts with the [+tense] vowel /iː/. The second part of the diphthong /ɑi/ is not marked for \isi{tenseness} because there is no contrast between a diphthong ending in [i] and one ending in [ɪ]. Given this treatment, the /g/ in a word like /tɑig/ is correctly predicted not to spirantize. The reader is referred to \citet{Noelliste2017}, who applies that type of treatment to the diphthongs of \ipi{Ramsau am Dachstein}, and to \sectref{sec:13.5.1} for a discussion of the diphthongs in \il{Central Bavarian}CBav varieties of Lower Bavaria.

The words in \REF{ex:17:8} exemplify the occurrence of [ç] after the two sonorant coronal consonants [l n]; recall generalization (B). The [ç] in examples like these is the modern realization of a historical fortis velar (\ili{WGmc} *[k x]).

\ea%8
\label{ex:17:8}Postconsonantal dorsal fricatives (from /x/):
\ea\relax [mœnç] \tab Mönch \tab ‘monk’
\ex\relax [ɛlç]  \tab Elch  \tab ‘moose’
\z
\z 

Palatal [ç] in items like the ones in \REF{ex:17:8} is precisely what one would expect given that the set of triggers for \isi{Velar Fronting-1} consists of all coronal sonorants and that /l n/ are both [coronal] and [+sonorant]. Hence, surface [ç] after /l n/ derives from /x/.

Palatal [ç] -- but not velar [x] -- surfaces after dorsal /ʀ/, which is realized optionally in the phonetic representation the consonant [ʀ] or as the vowel [ɐ]; recall generalization (C). Representative examples are presented in \REF{ex:17:9a}. The same [ʀ]/[ɐ] variants occur after any short vowel and before an optional coda consonant; see \REF{ex:17:9b}. After any long vowel, /ʀ/ surfaces as [ɐ]; see \REF{ex:17:9c}. The literature in which data like these are discussed include \citet[36]{Moulton1962}, \citet{Hall1993}, \citet[54]{Mangold2005}, \citet[253ff.]{Wiese1996a}, and \citet{Glover2014}. The [ç] in words like the ones in \REF{ex:17:9a} derived historically from a fortis velar fricative (\ili{WGmc} *[x] or *[xx]). The significance of the examples in \REF{ex:17:9a} is that they involve (historical) \isi{overapplication} \isi{opacity} because the palatal (from an earlier velar) surfaces after a back sound.


\TabPositions{.25\textwidth, .4\textwidth, .6\textwidth, .75\textwidth}
\ea%9
\label{ex:17:9}[ʀ] and [ɐ] (from /ʀ/):
\ea\label{ex:17:9a} \relax [dʊʀç], [dʊɐç] \tab durch \tab ‘through’\\
    \relax [kɪʀ.çə], [kɪɐ.çə] \tab Kirche \tab ‘church’
\ex\label{ex:17:9b} \relax 
    \relax [ɪʀt], [ɪɐt] \tab irrt \tab ‘be mistaken\textsc{{}-3sg}’\\
    \relax [ɪ.ʀən] \tab irren \tab ‘be mistaken\textsc{{}-inf}’
\ex\label{ex:17:9c} \relax [tyːɐ] \tab Tür \tab ‘door’\\
    \relax [tyː.ʀən] \tab Türen \tab ‘door-\textsc{pl}’
\z 
\z

I analyze the sound underlying [ʀ]/[ɐ] in \REF{ex:17:9} as /ʀ/, which surfaces as [ɐ] by \REF{ex:17:10}. I do not attempt to capture the optionality of that process after short vowels -- a condition that accounts for the variant pronunciations in (\ref{ex:17:9a}, \ref{ex:17:9b}). The target (/ʀ/) is [+consonantal, +sonorant, --nasal, dorsal], and the output ([ɐ]) is [--consonantal, +sonorant, --nasal, dorsal]; hence, \isi{r-Vocalization} only changes [±consonantal]; see \textcites[57]{Hall1992}[]{Hall1993}, \citet[256]{Wiese1996a}, and \citet{Glover2014}.

\ea%10
\label{ex:17:10}\isi{r-Vocalization}:\smallskip\\
\avm{[+cons\\+son\\−nasal\\dorsal]} → [--cons] / {\longrule} C\textsubscript{0} ]\textsubscript{${\sigma}$}
\z 

Since the trigger \isi{Velar Fronting-1} in \REF{ex:17:5} bears the frontness feature ([coronal]), that process cannot apply after /ʀ/, which is [dorsal]. It is precisely for that reason that I analyze [ç] in the context after a rhotic as an underlying palatal (quasi-phoneme), e.g. /dʊʀç/ and /kɪʀçə/ for \REF{ex:17:9a}. One might attempt to argue that /x/ can produce [ç] after /ʀ/ if the latter sound is analyzed phonologically as [coronal], but that treatment was considered and rejected for various regional dialects in \sectref{sec:7.4.2}. For further discussion see \sectref{sec:17.3.1}.

Recall from \chapref{sec:7} that several varieties of German are attested in which the \isi{palatal quasi-phoneme} occurs in the context of various back sounds, including the vocalized-r. It was demonstrated in that chapter that there was an earlier historical stage in which dorsal /ʀ/ was coronal (/r/), and that the earlier /r/ triggered the shift from /x/ to [ç] by velar fronting, which at that point was an allophonic rule. All surface palatals at that earlier stage were derived from /x/, but when the old front segment /r/ became back (/ʀ/) by \isi{r-Retraction} (\sectref{sec:3.4}), the surface palatal was quasi-phonemicized in that one context. Given that development it is not surprising that \il{Standard German}StG has [ç] after a back (dorsal) sound because that back sound used to be front.

The \il{Standard German}StG words with the diminutive suffix -\textit{chen} presented in \REF{ex:17:11} indicate that that the initial sound in that suffix consistently surfaces as [ç], regardless of whether or not it occurs after a stem ending in a back vowel in (\ref{ex:17:11a}), a front vowel in (\ref{ex:17:11b}), or a consonant in (\ref{ex:17:11c}). The initial fricative in that suffix is a historical velar (\ili{WGmc} *[x]). The most significant example is the one (\ref{ex:17:11a}), since palatal [ç] otherwise never occurs after a front vowel; hence, example (\ref{ex:17:11a}) exemplifies (historical) \isi{overapplication} \isi{opacity}. The examples in \REF{ex:17:11} illustrate generalization (D) stated earlier.


\TabPositions{.2\textwidth, .4\textwidth, .6\textwidth, .75\textwidth}
\ea%11
\label{ex:17:11}\il{Standard German}StG -\textit{chen} (/-çən/):
\ea\relax [tɑuçən]  \tab  Tauchen   \tab ‘rope\textsc{{}-dim}’  (cf. [tau] Tau ‘rope’)\label{ex:17:11a}
\ex\relax [ɑiçən]   \tab   Eichen   \tab ‘egg\textsc{{}-dim}’  (cf. [ɑi] Ei ‘egg’)\label{ex:17:11b}
\ex\relax [hʏntçən] \tab   Hündchen \tab ‘dog\textsc{{}-dim}’  (cf. [hʊnt] Hund ‘dog’)\label{ex:17:11c}
\z
\z 

Note that there are examples of minimal pairs, e.g. [tɑuxən] ‘dive-\textsc{inf}’ (from \ref{ex:17:3}b) vs. [tɑuçən] ‘rope-\textsc{dim}’ (from \ref{ex:17:11a}).

I follow \citet{Robinson2001} in analyzing the initial segment of \textit{-chen} as an underlying palatal (/ç/). Hence, a word like [tɑuçən] ‘rope\textsc{{}-dim}’ is underlyingly /tɑu-çən/. The underlying palatal drives support on the basis of the history of the \textit{-chen} suffix, as discussed below in \sectref{sec:17.3.2}.

The occurrence of [ç] after the vocalized-r in (\ref{ex:17:9}) and after back vowels in (\ref{ex:17:11a}) points to surface opacity in \il{Standard German}StG. By contrast, the distribution of [ç] and [x] in \ili{StAG} is transparent (\citealt{hildenbrandt2013}, \citealt{MoosmüllerBrandstätter2015}). In \ili{StAG} [ç] surfaces after a front vowel and [x] after a back vowel, including the vocalized-r, e.g. [kiɐxɛ] ‘church’. Since -\textit{chen} does not occur in StAG, there are no words where [ç] surfaces after a back vowel. (I mention two additional differences between \il{Standard German}StG and \ili{StAG} here: First, in StAG there are no alternations between [g] and [ç], as in (\ref{ex:17:6}); cf. StAG [køːnɪk] ‘king’, [køːnɪgə] ‘king-\textsc{pl}’. Second, [ç] is realized as [k] in StAG in word-initial position in loanwords, e.g. StAG [kemiː] ‘chemistry’. See Appendix~\ref{appendix:g} for discussion). 

\section{{Discussion}}\label{sec:17.3}

I consider three of the research questions posed earlier (\sectref{sec:1.4.4}) that have been discussed intensively in the literature on the synchronic phonology of German. The literature referred to here concerns itself primarily with \il{Standard German}StG, although the same questions are also relevant for many of the dialects investigated in preceding chapters. In \sectref{sec:17.3.1}, I consider and reject the proposal that the rhotic ([ʀ]/[ɐ]) is an articulation conducive to velar fronting. In \sectref{sec:17.3.2} I defend the treatment proposed above with an underlying palatal in \is{chen@\textit{-chen}}\textit{-chen}. Finally, in \sectref{sec:17.3.3} I discuss the question of whether or not the rule relating [ç] and [x] derives the palatal from the velar or the velar from the palatal and argue in favor of the former treatment.

\subsection{/ʀ/ is not a phonetically natural environment for [ç]}\label{sec:17.3.1}

In his discussion of the distribution of German [x] and [ç], \citet[78--81]{Robinson1992} cites some of the phonetics literature -- in particular \citet{Ulbrich1972} -- suggesting that surface vocalized-r ([ɐ]) is phonetically a front vowel. According to the material collected by Ulbrich, the [ɐ] in the context after a short vowel and before a palatal fricative (e.g. in a word like [dʊɐç] ‘through’ from \ref{ex:17:9a}) is further forward than the [ɐ] in other contexts. Robinson’s conclusion is that [ɐ] is a “phonetically natural environment for [ç]”.

Since his (pan-dialectal) equivalent of \isi{Velar Fronting-1} spreads [coronal] from a sonorant sound to a following /x/, Robinson concludes that [ɐ] should therefore be analyzed phonologically as [coronal].\footnote{{In fact, it is not entirely clear from the passage in Ulbrich that [ɐ] can be considered a front vowel from the point of view of phonetics. Robinson’s translation of the passage in question is ‘[ɐ] tends...a great deal toward [ə] or [ɪ]’, but [ə] is central and not front.}} Robinson emphasizes that the occurrence of a palatal after [ɐ] is the expected realization of /x/. One could rephrase Robinson’s position in the present framework by asserting that the occurrence of [ç] after [ɐ] is transparent, although Robinson eschews the latter term. In any case I reject his interpretation and argue instead that palatal [ç] after [ɐ] exemplifies \isi{opacity} and not \isi{transparency}. I therefore analyze the palatal in words like [dʊɐç] ‘through’ as an underlying palatal (quasi-phoneme) and not as a palatal derived from /x/. Two arguments can be levelled against Robinson’s treatment, which I consider in turn.

First, there are German dialects with some version of velar fronting after front vowels, but /x/ surfaces in those dialects without change as [x] after [ɐ]. Data from two of those dialects (from \sectref{sec:3.5} and \sectref{sec:4.3} respectively) are repeated in \REF{ex:17:12}. As discussed earlier, the realization of /x/ as [x] in examples like these is the expected (i.e. transparent) realization because the sound preceding /x/ is [dorsal] and not [coronal]. Recall from \sectref{ex:17:2} that [x] surfaces after the vocalized-r in \ili{StAG} as well.

\ea%12
\label{ex:17:12}Velar [x] (from /x/) after [ɐ] (from /ʀ/) in \ipi{Soest} (a) and \ipi{Ramsau am Dachstein} (b):
\ea\relax [bɛːɐx]  \tab Berg    \tab ‘mountain’\\
   \relax [tvɛːɐx] \tab Zwerg   \tab  ‘dwarf’
\ex\relax [ʃtɔɐx]  \tab Storch  \tab  ‘stork’\\
   \relax [kiɐxŋ̩] \tab  Kirche \tab   ‘church’
    \z
\z 

Robinson does not discuss dialects like the ones in \REF{ex:17:12}. If [ɐ] were a front (i.e. [coronal]) vowel in \il{Standard German}StG (as per Robinson), then it is not clear how he would analyze the dialects in \REF{ex:17:12}. One could speculate that the [ɐ] in that type of dialect is phonetically further back than the [ɐ] in \il{Standard German}StG (and perhaps phonologically [dorsal] as well), but this strategy stands in clear contrast to the implicit claim in \citet{Robinson2001} that his treatment holds for all German dialects. In any case, I hold that the burden of proof lies on the shoulders of linguists who claim that there are dialects with a coronal [ɐ] and those with a dorsal [ɐ].

Second, and most important, it is not clear how Robinson’s treatment actually works. According to his analysis, the [coronal] sound [ɐ] derives from /ʀ/, which is he analyzes as a singleton [dorsal]; see \citet[113]{Robinson2001}. His equivalent of \isi{Velar Fronting-1} spreads [coronal] from a sonorant to a following dorsal fricative, although he sees the target segment as [+high] and not [dorsal]. In any case, underlying /x/ correctly surfaces as the corono-dorsal fricative [ç] after a front vowel, as in my own treatment. However, Robinson never says how /ʀ/ changes from [dorsal] to [coronal] in words like [dʊɐç] ‘through’ and [kɪɐçə] ‘church’. Since Robinson sees every instance of [ɐ] is [coronal] and not simply the [ɐ] before [ç], the change from [dorsal] to [coronal] needs to occur in a context-free fashion. One can speculate that the featural change described here is a part of \isi{r-Vocalization} (which Robinson never formalizes), but if so, we have no explanation for why the vocalization of a consonant should also entail the change in place.

None of these problems hold for the present analysis. As noted above, \isi{Velar Fronting-1} correctly fails to affect the /x/ in examples like the ones in \REF{ex:17:12} and therefore surfaces without change as [x]. The dorsal fricative in \il{Standard German}StG examples like [dʊɐç] ‘through’ and [kɪɐçə] ‘church’ cannot be /x/, otherwise [x] would be the expected surface realization. The surface palatal fricative in examples like those is therefore an underlying palatal (quasi-phoneme). If it is true that [ɐ] is further forward before [ç] than in other contexts, then this is due to \isi{phonetic implementation} and is not an articulation that a phonological analysis can or should account for. Put differently, the fronted [ɐ] in words like [dʊɐç] ‘through’ is a consequence of [ç] and not the other way around.

\subsection{Status of -\textit{chen}}\label{sec:17.3.2}\is{chen@\textit{-chen}|(}

It was noted in chapter 1 that the analysis of [-çən] in words like [tauçən] ‘rope-\textsc{dim}’ in \REF{ex:17:11a} is moot for most of the dialects discussed in the present book because those dialects do not have [-çən] or any variant of that suffix with [ç]. See also \citet[64--70]{Robinson2001}, who bases his remarks on the maps in \citet{Tiefenbach1987}. See \mapref{map:43}.

\begin{map}
% \includegraphics[width=\textwidth]{figures/VelarFrontingHall2021-img049.png}
\includegraphics[width=\textwidth]{figures/Map43_17.1.pdf}
\caption[Diminutive suffixes in High and Low German]{Diminutive suffixes in High and Low German. Adapted from \citet{Tiefenbach1987}.}\label{map:43}
\end{map}

For example, LG dialects have a [k]-initial diminutive that is some variant of [\mbox{-kən}], while UG varieties have an [l]-initial variant of [-lɑin], the latter of which also occurs in \il{Standard German}StG, e.g. \textit{Kindlein} ‘child-\textsc{dim}’; cf. \textit{Kind} ‘child’. Not surprisingly, those patterns are reflected in the original sources cited earlier. For example, in the \il{Highest Alemannic}HstAlmc dialect of \ipi{Visperterminen} (\sectref{sec:6.2}), \citet[168--172]{Wipf1910} discusses at length the following realizations of the diminutive in her dialect: [-i], [-li], [ji], [-tsi], [-tʃi] and [-ki], but no mention is made of a variant with [ç]. The same point holds for the \il{Westphalian}Wph dialect of \ipi{Soest} (\sectref{sec:4.3}), where the diminutive appears to be consistently realized as [kn̩]; see \citet{Holthausen1886}.

These points aside, it is undeniably the case that [-çən] -- or a similar variant with [ç] --  occurs in many of the other dialects investigated in the preceding chapters, in particular CG dialects, on which \il{Standard German}StG is based. Some of the CG sources cited earlier list examples with -\textit{chen}, while others do not. In \REF{ex:17:13} I give examples from three of the former dialects. In each item, -\textit{chen} surfaces with [ç] even after stems ending in non-front segments:


\TabPositions{.12\textwidth, .25\textwidth, .4\textwidth, .55\textwidth}

\ea%13
\label{ex:17:13}\relax[-çən] after nonfront sounds in CG dialects:
\ea kœpçən \tab [kœpçən] \tab Tasse     \tab ‘cup-\textsc{dim}’      \tab \citet[86]{Hasenclever1905}
\ex kibχən \tab [kibçən] \tab Kuh, dim  \tab ‘cow-\textsc{dim}’   \tab \citet[151]{Hofmann1926}
\ex beɡχən \tab [begçən] \tab Bock, dim \tab  ‘buck-\textsc{dim}’ \tab \citet[21]{Schirmer1932}
    \z
\z 

The problem that has been discussed at length in the theoretical literature cited in \sectref{sec:1.2} is how to account for the opaque palatal in -\textit{chen} after a stem ending in a back vowel in \il{Standard German}StG (as in \ref{ex:17:11a}), although the same issue holds for the realization of that suffix after nonfront segments in other varieties of German, as in \REF{ex:17:13}.

As stated above, I  hold that the initial segment in the diminutive suffix [-çən] in \il{Standard German}StG is an underlying palatal (/ç/). The same analysis can be applied to dialect data like the ones in \REF{ex:17:13}. Since the target segment for velar fronting is by definition a velar that process cannot affect the /ç/ in /-çən/, which therefore surfaces as [çən] even after nonfront sounds. An analysis of the initial segment in [-çən] as an underlying velar /x/ with a separate rule applying only at the left edge of a morpheme is hardly credible for the simple reason that the rule required would only apply in a single morpheme.\pagebreak\largerpage

The underlying palatal /ç/ in [-çən] is a direct consequence of the history of that suffix. The \ili{MHG} reflex of [-çən] was -\textit{ichen} \citep[171]{Seebold2011}. The reader is also referred to the extensive discussion of the German diminutive suffixes in \citet[475--488]{Schirmunski1962}. Since the dorsal fricative represented by \textit{ch} followed the front vowel \textit{i}, it was realized as the palatal fricative [ç] at the point where velar fronting became phonologized (=Stage 2 in the historical model described in \sectref{sec:2.5}). When the initial vowel [i] in -\textit{ichen} was elided, [ç] came to stand after any stem, even if that stem ended in a back vowel. At that point, the original allophone [ç] changed into /ç/, as indicated in \REF{ex:17:14}. I give the underlying and phonetic representations for both historical stages. I include only the relevant features for /i/ and /x/, namely [coronal] and [dorsal]:

\ea%14
    \label{ex:17:14}
    \begin{tikzpicture}[baseline=(matrix-1-1.base)]
    \matrix (matrix) [matrix of nodes, column sep=0.75cm]
      {
        /i  & x  & ə &  n/ &[2mm] > &[2mm]  /ç & ə & n/\\
      \relax  [i  & ç  & ə &  n] &[2mm] > &[2mm] \relax [ç  & ə & n]\\
      };
      \node (c1) [below=10pt of matrix-2-1.base,xshift=-10pt] {[\textsc{coronal}]};
      \node (d1) [below=10pt of matrix-2-2.base] {[\textsc{dorsal}]};
      \node (c2) [below=10pt of matrix-2-6.base] {[\textsc{coronal}]};
      \node (d2) [below=10pt of matrix-2-7.base,xshift=10pt] {[\textsc{dorsal}]};
      \draw (matrix-2-1.south) -- ++(0pt,-4pt) -- (matrix-2-2.south) -- (d1.north);
      \path (matrix-2-6.south) edge (c2.north)
                               edge (d2.north);
    \end{tikzpicture}
\z 

To the left of the wedge the dorsal fricative is underlyingly /x/, which surfaces as [ç] by some version of velar fronting. The result of that spreading operation is the creation of a synchronically derived complex segment which is [coronal] and [dorsal]. When the initial /i/ was elided the feature [coronal] was retained on the newly-created underlying segment /ç/.\is{chen@\textit{-chen}|)}

\subsection{Velar to palatal or palatal to velar?}\label{sec:17.3.3}

An issue dealt with at length in the literature on \il{Standard German}StG phonology is whether or not the rule relating [ç] and [x] derives the former from the latter or the latter from the former (\sectref{sec:1.2}, \sectref{sec:7.4.3}). The same question can be posed with respect to the velars and palatals in the velar fronting dialects discussed in the present book. The two options referred to here are stated in \REF{ex:17:15}, where (\ref{ex:17:15a}, \ref{ex:17:15b}) apply in the postsonorant context and (\ref{ex:17:15c}, \ref{ex:17:15d}) word-initially. In \REF{ex:17:15}, [x] and [ç] are understood to be representative for any type of velar and palatal respectively.\footnote{{From the historical perspective, (\ref{ex:17:15b}, \ref{ex:17:15d}) are uncontroversially correct, but the debate described below holds for the synchronic phonology. If \REF{ex:17:15a} and/or \REF{ex:17:15c} can be shown to be correct synchronically, then \isi{rule inversion} must have taken place; recall \ipi{Neuendorf} (\sectref{sec:8.5}).}}

\ea\label{ex:17:15}
\ea  /ç/ → [x] / ...     \label{ex:17:15a}
\ex /x/ → [ç] / ...  \label{ex:17:15b}
\ex /ç/ → [x] / \textsubscript{Wd}[ ...\label{ex:17:15c}
\ex /x/ → [ç] / \textsubscript{Wd}[ ...\label{ex:17:15d}
\z 
\z 

Compare, for example, the treatment proposed for \il{Standard German}StG above, which adopts \REF{ex:17:15b}, with the one in \REF{ex:17:16} and \REF{ex:17:17}, which presupposes \REF{ex:17:15a}. Variants of \REF{ex:17:15a} for \il{Standard German}StG have been proposed in a number of the works cited earlier (e.g. \citealt{Wurzel1980}, \citealt{MeinholdStock1982}, \citealt{Hall1989}).


\TabPositions{.1\textwidth, .15\textwidth, .4\textwidth}

\ea%16
\label{ex:17:16}Underlying /ç/ in \il{Standard German}StG (rejected):
\ea /tuːç/ \tab →  \tab  [tuːx]         \tab  ‘scarf’
\ex /lɪçt/ \tab →  \tab  [lɪçt]         \tab ‘light’
\ex /dʊʀç/ \tab →  \tab  [dʊʀç], [dʊɐç] \tab ‘through’
\z
\ex%17
\label{ex:17:17}Hypothetical alternative to velar fronting (rejected):\\
\phonrule{/ç/}{[x]}{\{back vowels\}{\longrule}}
\z 

The consequence of the treatment in \REF{ex:17:16} and \REF{ex:17:17} is that it must require a special provision for  the occurrence of [ç] in the diminutive suffix [çən] after a back vowel; recall [tɑuçən] ‘rope-\textsc{dim}’ from \REF{ex:17:11a}.

The “velar to palatal” approach in (\ref{ex:17:15b}, \ref{ex:17:15d}) was uncritically adopted for \il{Standard German}StG as well as  the German dialects discussed in Chapters~\ref{sec:3}--\ref{sec:15}, but it is important to consider what the proposed treatment for those varieties might look like if velars were being derived from palatals, as in (\ref{ex:17:15a}, \ref{ex:17:15c}). Although one variety was discussed earlier in which the “palatal to velar” change in word-initial position (=\ref{ex:17:15c}) is the only possible one (\ipi{Neuendorf} in \sectref{sec:8.5}), it is demonstrated below that in the overwhelming number of dialects -- including \il{Standard German}StG -- the “velar to palatal” analysis is correct.

There are three reasons why a rule changing a palatal to a velar either leads to treatments that are far less explanatory than ones with a velar changing to a palatal or does not even work on technical grounds. (The unique case of \ipi{Neuendorf} is discussed at the end of this section). For convenience, I refer henceforth to the “palatal to velar” treatment in (\ref{ex:17:15a}, \ref{ex:17:15c}) as the Pa→Ve Analysis.

The first argument against the Pa→Ve Analysis pertains to the dialects discussed in Chapters~\ref{sec:8}--\ref{sec:10} and many of the varieties in \chapref{sec:11}. Those dialects have in common that velars (e.g. [x], [ɣ]) and palatals (e.g. [ç], [ʝ]) contrast in the context of the same back sounds. As demonstrated in those chapters, velar fronting is still active synchronically as a rule neutralizing the palatal vs. velar contrast to palatal in the context of front segments. That type of dialect is important because the Pa→Ve Analysis does not even work technically. As a representative example, consider \ipi{Schlebusch} (\sectref{sec:10.3.1}): [x] occurs only after a back vowel, but [ɕ] surfaces after a front vowel, coronal sonorant consonant, or back vowel. On the basis of these generalizations it was demonstrated that velar fronting applies to /x/ in the context after a coronal sonorant. For example, /x/ surfaces as [ɕ] in [løːɕǝ] ‘hole-\textsc{pl}’ (from /løːx-ǝ/), but /x/ is realized without change as [x] in [lɔx] ‘hole’ (from /lɔx/). It was noted in \sectref{sec:10.3.1} that one does not even have the option of analyzing such data with an underlying /ɕ/ which surfaces as [x] after a back vowel, as in (\ref{ex:17:15a}). The reason is that there are many morphemes with nonalternating [ɕ] after a back vowel which would incorrectly undergo the rule, e.g. [vrɔɕ] ‘frog (from /vrɔɕ/)’ (cf. [vrøɕ] ‘frog-\textsc{pl}’ from /vrøɕ/).

In \tabref{tab:17:1} I provide a list of dialects investigated in Chapters~\ref{sec:8}--\ref{sec:11} in which the Pa→Ve Analysis does not work (as in \ipi{Schlebusch}) because velars and the corresponding palatals contrast in the neighborhood of the same back vowel. The examples in the final three rows refer to word-initial position, while the remaining ones refer to postsonorant position. The velars and palatals in question are listed in the final column. I do not attempt to list all of the dialects investigated in Chapters~\ref{sec:8}--\ref{sec:11} involving word-initial [ʝ] and [ɣ]/[g] because that is an extremely common pattern.\largerpage

\begin{table}
\caption{Pa→Ve Analysis not possible after a sonorant or word-initially \label{tab:17:1}}
\begin{tabular}{lll}
\lsptoprule
Place/Region & Section & Sounds\\\midrule
\ipit{Wissenbach}         & \sectref{sec:9.2} &   [ç] and [x]          \\
\ipit{Langenselbold}      & \sectref{sec:9.2} &             \\
\ipit{Weidenhausen}       & \sectref{sec:9.2} &             \\
\ipit{Ebsdorf}            & \sectref{sec:9.2} &  \\
\ipit{Atzenhain}/\ipi{Grünberg} & \sectref{sec:9.2} &             \\
\ipit{Zell im Mümlingtal} & \sectref{sec:9.3} &             \\
Heppenhaim         & \sectref{sec:9.3} &             \\\midrule
\ipit{Schlebusch}        &  \sectref{sec:10.3} & [ɕ] and [x]  \\
\ili{Luxembourgish}& \sectref{sec:10.3}               &              \\
\ipit{Leipzig}& \sectref{sec:10.3}                     &              \\
\ipit{Cologne}& \sectref{sec:10.4}                     & \\
\ipit{Frankfurt am Main}/\ipi{Montabaur}& \sectref{sec:10.4} &              \\\midrule
Kreis \ipi{Bütow} & \sectref{sec:11.5} & [ɲ] and [ŋ]\\
\ipit{Lauenburg} & \sectref{sec:11.5} & [c] and [k]\\
Kreis \ipi{Konitz} & \sectref{sec:11.5} & [ç ɲ] and [k ŋ]\\
\ipit{Reimerswalde} & \sectref{sec:11.7} & [c ɟ] and [k g]\\
\midrule
Many dialects & \sectref{sec:8}, \sectref{sec:10}, \sectref{sec:11} & [ʝ] and [ɣ]/[g]\\
Kreis \ipi{Konitz} & \sectref{sec:11.5} & [ç] and [k]\\
\ipit{Reimerswalde} & \sectref{sec:11.7} & [c ɟ] and [k g]\\
\lspbottomrule
\end{tabular}
\end{table}

Recall from \tabref{tab:10:1} that there are many CG varieties like  \ipi{Schlebusch}, \ili{Luxembourgish}, \ipi{Leipzig}, \ipi{Cologne}, \ipi{Frankfurt am Main}/\ipi{Montabaur} that could be added to the \tabref{tab:17:1}.

The second reason for calling the Pa→Ve Analysis into question is that the alternative rules involved often require disjunctions in which one of the contexts is clearly ad hoc. As a representative example consider the distribution of word-initial [x] and [ç] in \ipi{Soest} (\sectref{sec:4.3}): Recall that [x] surfaces in that variety before back vowels or sonorant consonants and [ç] before front vowels. The correct rule therefore converts /x/ to palatal in word-initially before a front vowel. If /ç/ were taken as basic then the rule would create [x] in word-initial position before (a) back vowels or (b) sonorant consonants (/l n ʀ/). The problem is that context (b) is an arbitrary list of sounds that fails to express the assimilatory nature of the rule. In \tabref{tab:17:2} I list some of the dialects investigated in Chapters~\ref{sec:3}--\ref{sec:11} which, like \ipi{Soest}, require an awkward disjunction given the Pa→Ve Analysis. In the final column I list the arbitrary contexts that would be required if the velar is derived from the palatal.

\begin{table}
\caption{Disjunctions with an ad hoc context assuming the Pa→Ve Analysis\label{tab:17:2}}
\begin{tabularx}{\textwidth}{llQ}
\lsptoprule
Place/Region & Section & Ad hoc context\\\midrule
\ipit{Rheintal}   & \sectref{sec:3.4}  & /ç/→[x] in context of low front vowels  \\
\ipit{Rhoden}  & \sectref{sec:5.2}     & \\
\ipit{Kamnitz} & \sectref{sec:11.5}    &                                       \\
\midrule
\ipit{Soest}   & \sectref{sec:4.3}  & \multirow{2}{8cm}{/ç/→[x] word-initially before a sonorant consonant}\\
\ipit{Dorste} & \sectref{sec:4.4}  & \\
\midrule
\ipit{Obersaxen} & \sectref{sec:6.3} & /ç kç/→[x kk] in context of low front vowels and /ʏu/\\
\midrule
\ipit{Visperterminen} & \sectref{sec:6.2} & /ç kç/→[x kk] in context of nonlow front vowels and \isi{neutral vowels}\\
\midrule
Kreis \ipi{Rummelsburg} & \sectref{sec:11.5} & /ç ʝ/→[x ɣ] after front lax vowels\\
\midrule
Rauchenberg & \sectref{sec:7.2} &  {/ç/→[x] after any back vowel other than /ɑː/}\\
 \ipi{Rhöntal} & &  \\
\lspbottomrule
\end{tabularx}
\end{table}

A deeper generalization is expressed in \tabref{tab:17:3}, which lists four of the Trigger Types discussed in \chapref{sec:12} and shows the connection between those Trigger Types and the kind of ad hoc contexts required. For example, the Pa→Ve Analysis for any dialect with Trigger Type A requires palatals to be realized as velar in the context of nonhigh front vowels or coronal sonorant consonants. The additional problematic Trigger Types and the corresponding disjunctions are listed in \tabref{tab:17:3} as well.

\begin{table}
\caption{Connection between Trigger Type and ad hoc contexts necessary given the Pa→Ve Analysis\label{tab:17:3}}
\begin{tabular}{ll}
\lsptoprule
Trigger Type & Ad hoc disjunction\\\midrule
A & Nonhigh front vowel or coronal sonorant consonant\\
B & Nonlow front vowel or coronal sonorant consonant\\
C/AA & Nonlow front vowel\\
D/BB & Coronal sonorant consonant\\
\lspbottomrule
\end{tabular}
\end{table}

The reader may recall that disjunctions were posited in several varieties discussed in the previous chapters; however, in contrast to the problematic ones in Tables~\ref{tab:17:2} and~\ref{tab:17:3}, the disjunctions in the present analysis all involve assimilations. Consider as a representative example, the distribution of velars ([x] and [kx]) and palatals ([ç] and [kç] in \ipi{Rheintal} \sectref{sec:3.4}). In that section it was shown that the velars surface in the context of (a) nonlow front vowels, or (b) coronal sonorant consonants, captured formally with two versions of velar fronting (both assimilatory). By contrast, an alternative given the P→V Analysis requires the two contexts: (a) back vowels, or (b) nonlow front vowels, but the (b) context is ad hoc.

The third reason for rejecting the P→V Analysis is that in a number of dialects there is a [dorsal] segment serving as a target for velar fronting that is derived synchronically from a [dorsal] nontarget segment. The derived sound in question ({\textbar}x{\textbar}) can have more than one synchronic source, namely: (a) /ɣ/ (by \isi{Final Fortition}), (b) /g/ (by some version of g-Spirantization and \isi{Final Fortition}), or (c) /ʀ/ (by \isi{Laryngeal Assimilation-2} or \isi{Final Fortition}). The problem for the Pa→Ve Analysis is that the type of dialect referred to here requires a rule fronting the derived velar {\textbar}x{\textbar} which would be required alongside the rule creating [x] from an underlying palatal; see \citet{Glover2014}, who makes the same point for \il{Standard German}StG. Consider \ipi{Soest} as a representative example. Alternations from that dialect between [ɣ] and [ç] in words like [stui.ɣə] ‘climb\textsc{{}-inf}’ vs. [stɪçst] ‘climb\textsc{{}-2sg}’ require an underlying velar /ɣ/ which surfaces as [ɣ] after a vowel in a word-internal onset (in [stui.ɣə] from /stuiɣ-ə/). That velar undergoes \isi{Final Fortition} to {\textbar}x{\textbar} in coda position and then velar fronting to [ç] after a front vowel (in [stɪçst] from /stɪɣ-st/). If the Pa→Ve Analysis is adopted to capture the complementary distribution of [x] and [ç] not deriving from /ɣ/, e.g. [nɪçtə] ‘niece’ /nɪçtə/ and  [lʊxt] ‘air’ /lʊçt/, then the rule backing /ç/ to [x] would be unable to front the derived {\textbar}x{\textbar} to [ç]. In \tabref{tab:17:4} I list in the third column the three types of derived velars referred to above and a selection of some of the corresponding dialects from Chapters~\ref{sec:3}--\ref{sec:5} in the first column. Note that \ipi{Soest} has Target Type L discussed in \chapref{sec:12}; hence, that one dialect is simply one representative example of a significantly larger set of dialects. Impressionistically many CG varieties not discussed in the present book have some version of g-spirantization; hence, the two examples \ipi{Altengamme} and LRG are simply two representative instances of a much larger sample of German dialects.

\begin{table}
\caption{Dialects with a derived velar ({\textbar}x{\textbar}) which undergoes fronting\label{tab:17:4}}

\begin{tabular}{lll}
\lsptoprule
Place/Region & Section & Source for derived velar\\\midrule
\ipi{Soest} & \sectref{sec:4.3} & {\textbar}x{\textbar} from /ɣ/\\\midrule
\ipi{Altengamme} & \sectref{sec:4.2} & {\textbar}x{\textbar} from /g/\\
LRG & \sectref{sec:5.3} & \\\midrule
\ipi{Upper Austria} & \sectref{sec:3.6}\\
\ipi{Erdmannsdorf} & \sectref{sec:5.3} & {\textbar}x{\textbar} from /ʀ/\\
LRG  & \sectref{sec:5.3} \\
\lspbottomrule
\end{tabular}
\end{table}

\il{Standard German}StG can be included in the list of dialects with {\textbar}x{\textbar} derived from /g/. Recall from \REF{ex:17:9} that there are examples involving [g]{\textasciitilde}[ç] alternations like [køːnɪç] ‘king’ vs. [køːnɪgə] ‘king-\textsc{pl}’. That type of word requires that /g/ shift to the corresponding fricative (i.e. {\textbar}ɣ{\textbar} by \isi{g-Spirantization-2} and to {\textbar}x{\textbar} by \isi{Final Fortition}), which then surfaces as [ç] by velar fronting.

There is a small set of dialects discussed earlier in which the relationship between velars ([x]) and palatals ([ç]) is potentially free from the three problems discussed above. In that type of system, velars and palatals fulfill the following three conditions: (a) they are in complementary distribution, (b) the palatals occur in the context of all front vowels (and not a subset thereof), and (c) there are no derived velars that undergo fronting to palatal. Potential examples are listed in \tabref{tab:17:5}. The dialects listed from \chapref{sec:3} are Almc or \il{Central Bavarian}CBav varieties attested in South Germany, Switzerland, and Austria and ones from \chapref{sec:7} are \il{Eastphalian}Eph-speaking areas once spoken in North Germany. Consider \ipi{Erdmannsweiler} as a representative example. In that dialect [ç] surfaces after a front vowel or coronal sonorant consonant and [x] after a back vowel. The velar fronting treatment proposed in \sectref{sec:3.2} could be replaced with a Pa→Ve Analysis given in the final column of \tabref{tab:17:5}. Note that this is only a potential example of a dialect in which a P→V Analysis works technically because the dialect does not possess low front vowels like [æ]. Since that vowel is not present in \ipi{Erdmannsweiler} one cannot know for sure if [ç] or [x] surfaces after that sound. If [ç] surfaced after [æ] then \ipi{Erdmannsweiler} would be a true example of a dialect in which the Pa→Ve Analysis works technically, but if [x] surfaced after [æ] then the Pa→Ve Analysis would require an ad hoc disjunction (“palatal shifts to velar after a low front vowelˮ). The same indeterminacy holds for \ipi{Maienfeld}, \ipi{Ramsau am Dachstein}, \ipi{Reinhausen}, and \ipi{Schieder-Schwalenberg}. By contrast, \ipi{Elspe} possesses [æ], before which [ç] occurs; hence, the facts from word-initial position in \ipi{Elspe} represent the only clear-cut case in which the Pa→Ve Analysis works technically. Additional examples of dialects like \ipi{Elspe} are ones in which (a--c) are fulfilled which (like \ipi{Elspe}) represent Trigger Type E.

\begin{table}
\caption{Dialects in which the Pa→Ve Analysis is technically possible  or required \label{tab:17:5}}
\begin{tabularx}{\textwidth}{llQ}
\lsptoprule
Place/Region & Section & Alternative rule\\\midrule
\ipi{Erdmannsweiler} & \sectref{sec:3.2}\\
\ipi{Maienfeld} & \sectref{sec:3.3} & /ç/→[x] after a back vowel\\
\ipi{Ramsau am Dachstein}  & \sectref{sec:3.5} & \\\midrule
\ipi{Elspe} & \sectref{sec:7.2} & \multirow{2}{6cm}{/ç/→[x] word-initially before a [dorsal] vowel}\\
\ipi{Reinhausen} & & \\\midrule
\ipi{Schieder-Schwalenberg} & \sectref{sec:7.2} & /ç/→[x] word-initially before a [dorsal] sonorant\\
\midrule
\ipi{Neuendorf} & \sectref{sec:8.5} & /ç/→[x] word-initially before a [dorsal] vowel\\
\lspbottomrule
\end{tabularx}
\end{table}

The only example of a German dialect uncovered in the present book in which the relationship between velars and palatals actually requires a rule converting an underlying palatal to velar (as in \ref{ex:17:15a}, \ref{ex:17:15c}) is \ipi{Neuendorf} (\sectref{sec:8.5}). The correct rule for that dialect (\isi{Wd-Initial Palatal Retraction}) is stated in prose form in the final column of \tabref{tab:17:5}. Recall from \sectref{sec:8.5} that \isi{Wd-Initial Palatal Retraction} in \ipi{Neuendorf} had a peculiar history: In particular, it was the product of \isi{rule inversion} from a pre-\ipi{Neuendorf} system with velar fronting. That earlier fronting operation reverted to \isi{Wd-Initial Palatal Retraction} by the elimination of one of the [coronal] triggers (\isi{r-Deletion}). It was also mentioned in passing in that earlier chapter (\sectref{sec:8.6}) that it is notoriously difficult to find unambiguous examples of “palatal to velar”  assimilations in any natural language. (In fact, I have found none). That kind of cross-linguistic evidence suggests that it would be misguided to propose a reanalysis of the velar fronting operations for the dialects in \tabref{tab:17:4} as in the final column.

In sum, the relationship between velars and palatals in the overwhelming number of German dialects investigated in this book require a rule fronting the velar to the palatal (and not the reverse). That generalization also holds for \il{Standard German}StG, which has a derived velar ({\textbar}x{\textbar}) like the dialects listed in \tabref{tab:17:4}. The only case in which a dialect actually requires a rule backing a palatal to a velar, that type of system emerged via \isi{rule inversion}.\il{Standard German|)}
