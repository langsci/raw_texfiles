\chapter{Velar noncontinuants as targets}\label{sec:11}

\section{Introduction}\label{sec:11.1}

The focus of the present chapter lies in German dialects in which the set of velar fronting targets includes at least one velar noncontinuant in addition to at least one velar fricative (/ç/ or /ʝ/). Velar noncontinuants are defined here as velar stops (/k g/) and the velar nasal (/ŋ/). When those sounds undergo fronting, the corresponding palatals are created, namely [c ɉ ɲ]. The investigation is oriented towards those palatal noncontinuants in native words which derived from either etymological velars or from new velars created by independent changes. It is demonstrated below that the historical rule of velar fronting is active synchronically, although the version of that process can differ depending on the type of segments that serve as targets and/or triggers.

In terms of area, the dialects investigated are -- for the most part -- situated in the northeast of pre-1945 Germany (\mapref{map:45} in Appendix~\ref{appendix:b}), a region comprising the former provinces of East Pomerania (Ostpommern), Posen, West Prussia (Westpreußen), and East Prussia (Ostpreußen). From the perspective of dialect affiliation, the varieties in question belong to ELG (\il{East Pomeranian}EPo, \il{Low Prussian}LPr) and ECG (\il{High Prussian}HPr). Three places outside of the region described above are attested in which velar noncontinuants serve as triggers for fronting. Those three outliers are (a) one variety of ELG (\il{Mecklenburgish-West Pomeranian}MeWPo) in the far west of the modern-day German state of Mecklenburg-Vorpommern and (b) two ECG varieties (both \il{Silesian}Sln) in the southeast of the modern-day German state of Saxony (Sachsen).

The material presented below is significant because it provides evidence from dialects described in the modern era for two distinct stages of velar fronting: A first stage with a narrow set of targets (fricatives) and a later stage with an expanded set (velar consonants).

Since most of the places discussed below were situated in the eastern realm of the German-speaking world prior to 1945, they were therefore coterritorial with \ili{Slavic} languages which possess consonants phonetically similar to [c ɉ ɲ]. Although the change from velar noncontinuants to the corresponding palatals was uncontroversially endemic to German, I suggest that contact with \ili{Slavic} languages probably played a role in their \isi{phonologization}.

As indicated in the title of this chapter, dialects are investigated below with an expanded set of target segments for velar fronting. However, this chapter also considers the extent to which velar fronting triggers can differ depending on dialect. The generalizations concerning targets, triggers, and outputs are stated here:

\begin{description}
\item[Targets:] These segments can consist of some subset of the class of velar consonants (/x ɣ k g ŋ/). In some places that set of target sounds can be broad (velar consonants), and in others narrow (velar fricatives).

\item[Triggers:] These sounds can vary from place to place. Many varieties have the broadest set of triggers (coronal sonorants), while others have a narrower set (e.g. front vowels, nonlow front vowels).

\item[Outputs:]\sloppy In the dialects described below the target sound does not change its manner of articulation when fronted; hence, the manner of the target sound is the same as the manner  of the output (after velar fronting). This means that the velar fricatives /x ɣ/ surface as palatal fricatives -- alveolopalatalization is not a typical feature of this area -- and that the velar nasal /ŋ/ surfaces as the palatal nasal. Generally speaking, the same statement holds for stops, so /k g/ surface as the corresponding palatals ([c ɉ]). For one variety discussed below in \sectref{sec:11.5} /k g/ are realized as palatal fricatives when fronted; however, it is demonstrated in that section that velar fronting only alters place (velar→palatal) and that the change in continuancy (stop→fricative) is the result of a separate process.
\end{description}

It has been observed (e.g. \citealt{Mitzka1943}: 125) that the fronted realization of velar stops /k g/ in \il{East Pomeranian}EPo can be affricates (e.g. [tʃ dʒ]). I do not dispute that observation, although it needs to be stressed that the \isi{affricate} realization is not well-documented in the sources cited below. It is possible that velar fronting is simply responsible for shifting /k g/ to palatal stops and the realization of those palatal stops as affricates is due to \isi{phonetic implementation} (\sectref{sec:2.2.1}). It is also conceivable that the change from /k g/ to affricates is accomplished in the phonology and not in the phonetics; if so, that interpretation suggests that the change reflects an instance of the broader set of outputs characterized by \isi{Velar Palatalization} (\sectref{sec:2.3.1}). Since the data discussed below do not allow one to decide which of the two interpretations is correct, I simply leave the question open.\footnote{{Recall from Appendix~\ref{appendix:i} that the historical process usually referred to as \isi{Velar Palatalization} typically has affricates as output sounds in \ili{Slavic}, \ili{Romance}, \ili{North Germanic}, and \ili{West Germanic} (\ili{OE}, \ili{OFr}).} }

Since the output parameter does not play a significant role, I concentrate below on triggers/targets. The trigger/target parameters are important because they shed light on the way in which velar fronting spread through time and space -- a topic dealt with at greater length in \chapref{sec:12}.

The sounds that constitute the set of targets for velar fronting consist not only of historical velars, but also of velars created from etymologically non-velar sounds. The two changes referred to are presented in \REF{ex:11:1}:

\ea%1
\label{ex:11:1}
\ea\label{ex:11:1a}\isi{Wd-Initial Nasal Place Assimilation}:   [n] > [ŋ] / \textsubscript{wd}[k {\longrule}
\ex\label{ex:11:1b}\isi{Velarization}:         [nd nt] > [ŋ] / {\longrule}
\z 
\z 

\isi{Wd-Initial Nasal Place Assimilation} creates [kŋ] clusters that are realized as [kn] in other dialects (cf. MStGm [knɔxən] ‘bone’). That new [kŋ] sequence is a potential target for velar fronting if that process applies in word-initial position. I make the noncrucial assumption here that postvocalic velar nasal plus velar stop sequences ([ŋk]) were inherited (from \ili{WGmc} \textsuperscript{+}[ŋk]) and that there never was a stage in which [nk] was attested. \isi{Velarization} is the name for the change from alveolar to velar depicted in \REF{ex:11:1b}; see \citet[395--400]{Schirmunski1962} and \citet{Werlen1983}, who use the traditional term “\isi{gutturalization}". For example, the cluster [nt] preserved in \il{Standard German}StG words like \textit{unten} [ʊntən] ‘under’ is realized in velarizing dialects as [ŋ]. That sound is a potential target for velar fronting provided that a front segment precedes it.\footnote{{In many dialects \isi{Velarization} only applies after high vowels like /i u/. The target segments can also include the singleton velar nasal as well as velar stops. These are unessential details and are therefore not discussed. The reader is referred to \citet[Chapter 8]{Streck2012}, who shows that Velarization is much more widespread geographically than suggested above.} }

Many of the dialects discussed in this chapter possess underlying palatal noncontinuants (palatal quasi-phonemes and/or phonemic palatals), i.e. /c ɉ ɲ/. All dialects have the \isi{etymological palatal} (/ʝ/). Since the following case studies are quite complex, I attempt to economize by referring on occasion to underlying palatals without specifying the type of palatal and only make passing reference to the distinction between palatal quasi-phonemes, phonemic palatals, and etymological palatals.

I economize in another way as well. In particular, given the large number of targets and triggers it is not feasible to provide a sample word for each phonemic vowel in the neighborhood of every target segment for word-initial and postsonorant position in each of the dialects investigated. The correct context for each case study was determined on the basis of the data in the original sources; hence, I typically provide only one or two examples representing a particular context (e.g. a word containing /ex/ for all front vowels before /x/). I likewise do not provide a complete set of phonemic vowels for every case study.

In places with phonemic palatals those sounds exhibit the Contrast Type B system discussed at length in Chapters~\ref{sec:8}--\ref{sec:10}. As illustrated in \REF{ex:11:2}, such dialects possess a contrast between velars and palatals in the context of back vowels (represented as [ɑ]), but in the context of front vowels (represented as [i]) only palatals occur.

\ea Contrast Type B in word-initial (=\ref{ex:11:2a}) and postsonorant (=\ref{ex:11:2b}) position:\label{ex:11:2}
\begin{multicols}{2}
\ea\label{ex:11:2a}
\fbox{
\begin{tabular}[t]{@{}ll@{}}
   [\textsc{pa} i...] & [\textsc{pa} ɑ…] \\
 & [\textsc{ve} ɑ…]                      \\
\end{tabular}
}
\ex\label{ex:11:2b}
\fbox{
\begin{tabular}[t]{@{}ll@{}}
 [...i \textsc{pa}...] & [...ɑ \textsc{pa}...]\\
 & [...ɑ \textsc{ve}…]\\
\end{tabular}
}
\z
\end{multicols}
\z 

Phonemic palatal noncontinuants -- as well as palatal noncontinuant quasi-phonemes -- can arise the same way as their fricative counterparts. For example, an original velar like [k] (/k/) in the context of a front vowel can develop a palatal allophone ([c]) which is realized at a later stage as an underlying palatal (/c/) when the original front vowel trigger is eliminated by changes discussed in previous chapters (\isi{Vowel Reduction}, \isi{Vowel Retraction}, \isi{Syncope}).

Many of the varieties discussed below have another quirk in common: Velar fronting can occur even when a segment intervenes between the target and trigger, e.g. the velar (/k/) after a front vowel plus liquid sequence (/il/) is realized as palatal ([ilc]), but the velar remains a velar if a back vowel precedes the liquid (/ɑlk/→[ɑlk]). Recall from \chapref{sec:6} that examples like these are also attested in two \il{Highest Alemannic}HstAlmc varieties. It was argued in that chapter that velar fronting is fed by a change merging the coronal feature of a front vowel with the coronal feature of the liquid (\isi{Coalescence-1}). A mirror-image process for word-initial position is shown to be active in some varieties as well.

In \sectref{sec:11.2} I provide some general remarks on the status of velar noncontinuants as targets outside of the area investigated in the present chapter. In \sectref{sec:11.3} and \sectref{sec:11.4} I discuss those systems in \il{Mecklenburgish-West Pomeranian}MeWPo and Sil. The bulk of the material discussed below is devoted to a description and brief analysis of those varieties once spoken in East Pomerania and Posen (\sectref{sec:11.5}) and East Prussia (\sectref{sec:11.6}, \sectref{sec:11.7}). A summary of the findings and the relevance for palatalization typology is presented in \sectref{sec:11.8}. In \sectref{sec:11.9} I discuss the extension of velar fronting targets historically in the \isi{rule generalization} model and the connection between the development of palatal noncontinuants and the existence of \ili{Slavic} loanwords containing sounds phonetically similar to [c ɉ ɲ]. In \sectref{sec:11.10} I consider the areal distribution of German dialects with a broader set of targets. I provide a brief conclusion in \sectref{sec:11.11}.

\section{{General} {remarks} {on} {velar} {noncontinuants} {as} {targets}}\label{sec:11.2}

The set of targets for velar fronting in all German dialects discussed in previous chapters consists of velar fricatives only. That /k g ŋ/ do not have palatal allophones in the neighborhood of front vowels is also implicit in the literature on \il{Standard German}StG, although that presupposition is rarely stated explicitly.

In some of the late nineteenth and early twentieth century descriptive work on German dialects, velar noncontinuants like [k] and [ŋ] are described as having palatal variants in the neighborhood of front vowels, even in regions outside of the ones investigated below. I make no attempt to document the kind of grammar referred to here. Instead, I cite one representative example (NLG), namely \ipi{Greetsiel} in the far western part of the German state of Lower Saxony (Niedersachsen; \citealt{Hobbing1879}; \mapref{map:5}). Hobbing’s work is an articulatory phonetic description of the consonants and vowels in which he states (p. 24) clearly that [k] can have two articulations (reflected in two distinct symbols): ⟦k\textsuperscript{1}⟧ in the neighborhood of front vowels and ⟦k⟧ in the neighborhood of back vowels.

It is difficult to know with certainty whether or not the palatal stop ⟦k\textsuperscript{1}⟧ is phonological ([c] as an allophone of /k/ created by velar fronting) or simply the byproduct of phonetics, i.e. a \isi{prevelar} which is a consequence of \isi{coarticulation}. I assume here that the latter is the correct interpretation, although it would be also consistent with the theme of this book to analyze ⟦k\textsuperscript{1}⟧ as phonological. Thus, I assume that the fronted realization of [k] in dialects like \ipi{Greetsiel} is a phonetic variant on par with the fronted [k] in \ili{English} words like \textit{keep}. I speculate that the reason Hobbing as well as many of his contemporaries included the fronted realization of sounds like [k] in their grammars is that there was no distinction at that time between phonetics (which was already well-established in the late nineteenth century in Germany) and phonology (which did not yet exist). Since the concept of phonemes and allophones lay a number of years in the future, phonetically-trained linguists like Hobbing had no alternative but to treat the palatal realization of [k] on par with segments that are uncontroversially phonemes.

In contrast to dialects like \ipi{Greetsiel}, palatal noncontinuants in the dialects discussed below are phonological and not phonetic. The reason for my conclusion is that the segments in question display the same degrees of \isi{phonologization} as the corresponding fricatives [ç ʝ] by occurring as palatal quasi-phonemes or even as contrastive sounds (phonemic palatals).

\section{{Mecklenburgish-West} {Pomeranian}}\label{sec:11.3}\il{Mecklenburgish-West Pomeranian|(}

\citet{Kolz1914} describes a \il{Mecklenburgish-West Pomeranian}MeWPo variety spoken in the northwest corner of Landkreis Nordwestmecklenburg (\mapref{map:17}). Kolz refers to his variety as the \ipi{West Mecklenburg} dialect (“Westmecklenburgischer Dialektˮ).

The dorsal consonants of \ipi{West Mecklenburg} are listed in \REF{ex:11:3a} and \REF{ex:11:3b} for word-initial and postsonorant position respectively. Kolz adopts a wide array of phonetic symbols and diacritics expressing laryngeal distinctions. The relevant symbols for velars and palatals are [x ɣ]=⟦x g⟧, [ç ʝ]=⟦χ ɡ⟧, [k g]=⟦k ɡ⟧, [c ɟ]=⟦c ɡ⟧. The lenis palatal fricative [ʝ] is transcribed in two ways depending on the etymological source: ⟦ɡ⟧ (< WGmc \textsuperscript{+}[ɣ]) and ⟦j⟧ (< WGmc \textsuperscript{+}[j]). [ŋ] and [ɲ] are both rendered as ⟦\ExtraChars{ᵰ}⟧.

\ea%3
\label{ex:11:3}
\ea\label{ex:11:3a} \begin{forest} for tree = {fit=band}
  [,phantom
   [/ʝ/ [{[ʝ]}]]   [/k/,calign=first [{[k]}] [{[c]}]]  [/g/ [{[g]}]]    [/ɟ/  [{[ɟ]}]]   
   ] 
   \end{forest}
\ex\label{ex:11:3b} \begin{forest} for tree = {fit=band}
     [,phantom
       [/x/,calign=first [{[x]}] [{[ç]}]]      
       [/ɣ/,calign=first [{[ɣ]}] [{[ʝ]}]]             
       [/k/ [{[k]}] ]
       [/c/ [{[c]}] ]
       [/g/,calign=first [{[g]}] [{[ɟ]}]]            
       [/ŋ/,calign=first [{[ŋ]}] [{[ɲ]}]]
     ]
   \end{forest}
\z 
\z 
    
Consider first the stops [k c g ɟ] (<\ili{WGmc} \textsuperscript{+}[ɣ k]) in word-initial position. In that context the velars [k g] never contrast with the corresponding palatals ([c ɟ]): [k g] occur before a full back vowel in (\ref{ex:11:4a}, \ref{ex:11:4e}) or a consonant followed by a full back vowel in (\ref{ex:11:4b}, \ref{ex:11:4f}) and [c ɟ] before a front vowel in (\ref{ex:11:4c}, \ref{ex:11:4g}) or a consonant followed by a front vowel in (\ref{ex:11:4d}, \ref{ex:11:4h}). [k g] never occur before a front vowel or a consonant plus front vowel sequence. [ɟ] (<\ili{WGmc} \textsuperscript{+}[ɣ]) also surfaces before \isi{schwa} in (\ref{ex:11:4i}) and [ʝ] (<\ili{WGmc} \textsuperscript{+}[j]) before any vowel in (\ref{ex:11:4j}). 

\begin{map}
% \includegraphics[width=\textwidth]{figures/VelarFrontingHall2021-img023.png}
\includegraphics[width=\textwidth]{figures/Map17_11.1.pdf}
  \caption[Mecklenburgish-West Pomeranian, Brandenburgish, and Central Pomeranian]{Mecklenburgish-West Pomeranian (\il{Mecklenburgish-West Pomeranian}MeWPo), Brandenburgish (\il{Brandenburgish}Brb), and Central Pomeranian (\il{Central Pomeranian}CPo). Squares indicate postsonorant velar fronting. 1=\citet{Holst1907}, 2= \citet{GSchmidt1912}, 3=\citet{Warnkross1912}, 4=\citet{Kolz1914}, 5=\citet{Jacobs1925a, Jacobs1925b, Jacobs1926}, 6=\citet{Teuchert1927} (\ipi{Rehna}), 7=\citet{Teuchert1927} (\ipi{Schwerin}), 8=\citet{Dützmann1932}, 9=\citet{TeuchertSchmitt1933} (\ipi{Ratzeburg}), 10=\citet{TeuchertSchmitt1933} (\ipi{Rostock}), 11=\citet{TeuchertSchmitt1933} (\ipi{Lank}), 12=\citet{Blume1933, Blume1933b, Blume1933c, Blume1933d}, 13=\citet{Teuchert1934}, 14=\citet{BethgeBonnin1969}, 15=\citet{Prowatke1973} (\ipi{Greifswald}), 16=\citet{Prowatke1973} (\ipi{Schwerin}), 17=\citet{Schönfeld1989} (Teterow), 18=\citet{Krause1895}, 19=\citet{Krause1896}, 20=\citet{Siewert1907}, 21=\citet{Teuchert1907a, Teuchert1907b}, 22=\citet{Teuchert1907c}, 23=\citet{Seelmann1908}, 24=\citet{Siewert1912}, 25=\citet{Seelmann1913}, 26=\citet{Hildebrand1913}, 27=\citet{Selmer1918}, 28=\citet{Götze1922}, 29=\citet{Teuchert1930},  30=\citet{Bathe1932}, 31=\citet{Bathe1937}, 32=\citet{Törnqvist1949}, 33=\citet{Bretschneider1951}, 34=\citet{Teuchert1964}, 35=\citet{Bathe1965}, 36=\citet{Gebhardt1965}, 37=\citet{Schönfeld1965}, 38-\citet{Schönfeld1989}  (\ipi{Tempelfelde}), 39=\citet{Brose1955}, 40=\citet{Prowatke1973}.}\label{map:17}
\end{map}

\TabPositions{.15\textwidth, .33\textwidth, .5\textwidth, .75\textwidth}
\ea%4
\label{ex:11:4}Word-initial dorsal obstruents:
\ea\label{ex:11:4a} kus \tab [kʊs] \tab  Kuss \tab ‘kiss’ \tab 135
\ex\label{ex:11:4b} krum \tab [krʊm] \tab krumm \tab ‘bent’ \tab 127\\
    knu\textsuperscript{ə}dn \tab [knuədn̩] \tab Knorren \tab ‘gnarl’ \tab 67
\ex\label{ex:11:4c} cind \tab [cɪnt] \tab  Kind \tab ‘child’ \tab 17
\ex\label{ex:11:4d} cli·f \tab [clif] \tab Klette \tab ‘burr’ \tab 127\\
    cneχt \tab [cneçt] \tab Knecht \tab ‘vassal’ \tab 28
\ex\label{ex:11:4e} ɡɔ·bl \tab [gɔ·bl̩] \tab Gabel \tab ‘fork’ \tab 129
\ex\label{ex:11:4f} ɡrɑm \tab [grɑm] \tab böse \tab ‘angry’ \tab 124\\
    ɡnɑ·dn \tab [gnɑdn̩] \tab knarren \tab ‘creak\textsc{{}-inf}’ \tab 59
\ex\label{ex:11:4g} ɡelt \tab  [ɟelt] \tab  Geld \tab ‘money’ \tab 27
\ex\label{ex:11:4h} ɡlīnt \tab [ɟliːnt] \tab  Lattenzaun \tab ‘picket fence’ \tab 21
\ex\label{ex:11:4i} ɡəsiχt \tab [ɟəsɪçt] \tab  Gesicht \tab  ‘face’ \tab 17
\ex\label{ex:11:4j} ju\ExtraChars{ᵰ}k \tab [ʝʊŋk] \tab jung \tab ‘young’ \tab 15
    \z
\z 

After a sonorant, velar fricatives ([x ɣ]) and their palatal counterparts ([ç ʝ]) are allophones: The velars occur after a back vowel in (\ref{ex:11:5a}, \ref{ex:11:5e}) and the palatals after a front vowel in (\ref{ex:11:5c}, \ref{ex:11:5f}). [x] also occurs after a liquid preceded by a back vowel in (\ref{ex:11:5b}) and [ç] after a liquid preceded by a front vowel in (\ref{ex:11:5d}). No parallel example like \REF{ex:11:5d} was found for [ʝ]. Velar stops ([k g]) and their palatal counterparts ([c ɟ]) display a parallel distribution in (\ref{ex:11:5g}--\ref{ex:11:5n}).\footnote{\label{fn:11:3}The velar stop [g] in (\ref{ex:11:5k}, \ref{ex:11:5l}) and the palatal stop [ɟ] in (\ref{ex:11:5n}) are followed by the (syllabic) velar nasal [ŋ] and the (syllabic) palatal nasal [ɲ] respectively. Examples like these suggest that the place features of a syllabic nasal spread from the place features of a preceding obstruent (\isi{Progressive Nasal Place Assimilation}). Since that process is independent of velar fronting it is not discussed here; see \citet{Hall2020}, who shows that \isi{Progressive Nasal Place Assimilation} is active in several WGmc languages.} The dorsal sounds referred to above ([x ç ɣ ʝ k c g ɟ]) are all modern reflexes of velars (\ili{WGmc} \textsuperscript{+}[ɣ k] or \textsuperscript{+}[gg]). The items in (\ref{ex:11:5o}, \ref{ex:11:5p}) show that nasal plus stop sequences (<\ili{WGmc} \textsuperscript{+}[ŋk]) are homorganic. After a front vowel, the nasal and stop are palatal, and after a back vowel they are both velar; the distinction between the two nasals in examples like these is clear from the original source \citep[147]{Kolz1914}: “as \ExtraChars{ᵰ} vor gutturalem Verschlusslaut … ist … erhalten als palatales \ExtraChars{ᵰ} vor palatalem, als velares \ExtraChars{ᵰ} vor velarem Verschlusslautˮ.  (“\ili{Old Saxon} \ExtraChars{ᵰ} … is palatal \ExtraChars{ᵰ} before palatal stops and velar \ExtraChars{ᵰ} before velar stopsˮ). The [c] in (\ref{ex:11:5q}) occurs in the context after a historically elided front vowel (by \isi{Syncope}; recall \chapref{sec:7}).\largerpage

\ea%5
\label{ex:11:5}Postsonorant dorsal consonants:
\ea\label{ex:11:5a}  tuxt \tab [tʊxt] \tab  Zucht \tab ‘breeding’ \tab 68
\ex\label{ex:11:5b}  tɑlx \tab [tɑlx] \tab Talg \tab ‘tallow’ \tab 52
\ex\label{ex:11:5c}  liχt \tab [lɪçt] \tab  Licht \tab ‘light’ \tab 15
\ex\label{ex:11:5d}  fel·χ \tab [felç] \tab Felge \tab ‘wheel rim’ \tab 27
\ex\label{ex:11:5e}  fogl \tab [fɔɣl̩] \tab Vogel \tab ‘bird’ \tab 15
\ex\label{ex:11:5f}  flɛ·ɡl \tab [flɛ·ʝl̩] \tab Flegel \tab ‘boor’ \tab 15
\ex\label{ex:11:5g}  rók \tab [rok] \tab  Rauch \tab ‘smoke’ \tab 127
\ex\label{ex:11:5h}  kɑlk \tab [kɑlk] \tab Kalk \tab ‘lime’ \tab 45
\ex\label{ex:11:5i}  dic \tab [dɪc] \tab  dick \tab ‘fat’ \tab 17
\ex\label{ex:11:5j}  melc \tab [melc] \tab Milch \tab ‘milk’ \tab 24
\ex\label{ex:11:5k}  bɑɡ\ExtraChars{ᵰ} \tab [bɑgŋ̍] \tab backen \tab ‘bake\textsc{{}-inf}’ \tab 43
\ex\label{ex:11:5l}  bɑlɡ\ExtraChars{ᵰ} \tab [bɑlgŋ̍] \tab Balken \tab ‘beam’ \tab 53
\ex\label{ex:11:5m}  eɡ\textsuperscript{ə} \tab [eɟə] \tab Egge \tab ‘harrow’ \tab 28
\ex\label{ex:11:5n}  m\.{e}lɡ\ExtraChars{ᵰ} \tab [mɛlɟɲ̍] \tab melken \tab ‘milk\textsc{{}-inf}’ \tab 35
\ex\label{ex:11:5o}  di\ExtraChars{ᵰ}c \tab [dɪɲc] \tab Ding \tab ‘thing’ \tab 125
\ex\label{ex:11:5p}  lɑ\ExtraChars{ᵰ}k \tab [lɑŋk] \tab lang \tab ‘long’ \tab 125
\ex\label{ex:11:5q}  \'{v}ɛdc \tab [vɛdc] \tab Enterrich \tab ‘gander’ \tab 33
\z 
\z 


The initial stops in (\ref{ex:11:4a}--\ref{ex:11:4h}) are underlying velars (/k g/), which surface as the corresponding palatals before a front vowel in (\ref{ex:11:4c}--\ref{ex:11:4g}) by the specific version of velar fronting stated in \REF{ex:11:6}. In (\ref{ex:11:4d}, \ref{ex:11:4h}), the [coronal] feature of the front vowel and the [coronal] feature of the preceding sonorant consonant undergo \REF{ex:11:7b}, which \isi{feeds} \REF{ex:11:6}, e.g. /gliːnt/→{\textbar}g\textbf{liː}nt{\textbar}→[\textbf{ɟliː}nt], where the segments in bold reflect the application of \REF{ex:11:7b} and \REF{ex:11:6}. The word-initial consonant in (\ref{ex:11:4i}) is a \isi{palatal quasi-phoneme} (/ɟ/), and in (\ref{ex:11:4j}) it is the \isi{etymological palatal} (/ʝ/).\footnote{As stated in \REF{ex:11:7b} the leftmost segment of \isi{Coalescence-2} is a coronal sonorant consonant, e.g. the /l/ in [ɟliːnt] ‘picket fence’ from (\ref{ex:11:4h}). Data not presented above show that \REF{ex:11:6} also affects a word-initial velar before the labial [v], e.g. [cveə] ‘across’ (from /kveə/). I do not discuss this complication here; see \sectref{sec:12.8.1}.}\largerpage

\ea%6
\label{ex:11:6}\isi{Wd-Initial Velar Fronting-6}:

\begin{forest}
[,phantom
   [\avm{[−son]},name=parent [\avm{[dorsal]},tier=word]]
   [\avm{[−cons]} [\avm{[coronal]},tier=word,name=target]]
]
\draw [dashed] (parent.south) -- (target.north);
\node [left=1ex of parent] {\textsubscript{wd}[};
\end{forest}
\ex%7
\label{ex:11:7}
\ea \isi{Coalescence-1}:\\\label{ex:11:7a}
\begin{forest}
[,phantom
  [\avm{[−cons]} [\avm{[coronal]},tier=word]]
  [\avm{[+cons\\+son]} [\avm{[coronal]},tier=word]]
  [→]
  [\avm{[−cons]},base=bottom
    [,phantom]
    [\avm{[coronal]},name=coronal]
  ]
  [\avm{[+cons\\+son]},name=cons,base=top]
]
\draw (coronal.north) -- (cons.south);
\end{forest}
\ex \isi{Coalescence-2}:\\\label{ex:11:7b}
\begin{forest}
[,phantom
  [\avm{[+cons\\+son]} [\avm{[coronal]},tier=word]]
  [\avm{[−cons]} [\avm{[coronal]},tier=word]]
  [→]
  [\avm{[+cons\\+son]},base=top
    [,phantom]
    [\avm{[coronal]},name=coronal]
  ]
  [\avm{[−cons]},base=bottom,name=cons]
]
\draw (coronal.north) -- (cons.south);
\end{forest}
\z 
\z

All postvocalic palatals in \REF{ex:11:5} derive from the corresponding velars by the mirror-image of \REF{ex:11:6}, stated in \REF{ex:11:8}. If the target sound (/x ɣ k g/) follows a liquid, then it surfaces as the corresponding palatal if the vowel preceding the liquid is front, otherwise it is velar (cf. \ref{ex:11:5b} vs. \ref{ex:11:5d}; \ref{ex:11:5h} vs. \ref{ex:11:5j}; \ref{ex:11:5l} vs. \ref{ex:11:5n}); recall \ipi{Visperterminen} and \ipi{Obersaxen} (\chapref{sec:6}). Front vowel plus liquid sequences in (\ref{ex:11:5d}, \ref{ex:11:5j}, \ref{ex:11:5n}) share [coronal] by \isi{Coalescence-1} (=\ref{ex:11:7a}). That merged [coronal] feature spreads to a following velar by \isi{Velar Fronting-8}, thereby creating a palatal. In postvocalic nasal plus stop clusters in (\ref{ex:11:5o}, \ref{ex:11:5p}) the sequence (/ŋk/) has a single place feature dominating [dorsal]. If the vowel preceding /ŋk/ is front then [coronal] spreads from that vowel to the right by \isi{Velar Fronting-8}, e.g. /dɪŋk/→[d\textbf{ɪɲc}]. The final segment in (\ref{ex:11:5q}) is an underlying palatal (quasi-phoneme), i.e. /c/.

\ea\label{ex:11:8}\isi{Velar Fronting-8}:\\
\begin{forest}
[,phantom
    [\avm{[−cons]} [\avm{[coronal]},name=target]]           
    [\avm{[−son]},name=parent [\avm{[dorsal]}]]
]
\draw [dashed] (target.north) -- (parent.south);
\end{forest}
\z 

Kolz’s variety of \ipi{West Mecklenburg} is unique for its region in more than one way. First, the target segments for all fronting operations consist of velar consonants, but the corresponding targets in neighboring places are restricted to one (/x/) or two (/x ɣ/) velar fricatives. Second, velar fronting word-initially and after a sonorant is fed by one of the coalescence processes, but in all but one of the sources discussed here, coalescence is absent. Third, there are underlying palatal stops (quasi-phonemes) in \ipi{West Mecklenburg}, but such palatals are absent in the dialects discussed below. I conclude this section by discussing briefly the status of velar fronting in some of the other places in the \il{Mecklenburgish-West Pomeranian}MeWPo region. All of these places are indicated on \mapref{map:17}.

\begin{sloppypar}
Consider first \citegen{Teuchert1927} phonetic transcriptions of native speakers from two places close geographically to the area investigated by \citet{Kolz1914}, namely \ipi{Rehna} and \ipi{Schwerin}. On the basis of the material in \citet{Teuchert1927} it can be safely concluded that coronal sonorants are the triggers for postsonorant fronting and that /x/ is the sole target for (postsonorant) velar fronting. Velar fronting does not occur word-initially. Significantly, \citet{Teuchert1927} gives no evidence that noncontinuants undergo velar fronting. The same generalizations hold for the phonetic transcriptions of native speakers from \ipi{Ratzeburg}, \ipi{Rostock}, and \ipi{Lank} from \citet{TeuchertSchmitt1933}.\largerpage

None of the other sources for \il{Mecklenburgish-West Pomeranian}MeWPo indicate that velar noncontinuants serve as targets for velar fronting: In a series of detailed studies, \citet{Jacobs1925a, Jacobs1925b, Jacobs1926} investigates the dialects spoken in the south of Mecklenburg-Vorpommern (“\ipi{South Mecklenburg}”) between \ipi{Lübz} and Hagenow (\sectref{sec:10.6.2}; \sectref{sec:12.7.1}). \citet{Jacobs1925b} presents copious data indicating that the set of targets for velar fronting is the velar fricative [x] (<\ili{WGmc} \textsuperscript{+}[x ɣ]), e.g. [vɛç] ‘path’ (=⟦vęχ⟧), [væːç], ‘path-\textsc{pl}’ (=⟦vä̅χ⟧) vs. [tʊxt] ‘breeding’ (=⟦tųxt⟧), [oːx] ‘eye’ (=⟦ōx⟧). However, there is no indication in \citet{Jacobs1925a, Jacobs1925b, Jacobs1926} that [k g] have palatal variants after front vowels.\footnote{{A brief statement can be found in \citet[47]{Jacobs1925a} asserting that [ŋ] has a fronted variant after a front vowel, though that type of example is not discussed further.} } That /x/ is the only target for velar fronting is clear in descriptions of \ipi{Ivenack-Stavenhagen} \citep{Holst1907}, e.g. [bryç] ‘bridge’ (=⟦brüχ⟧) vs. [nɔx] ‘still’ (=⟦nox⟧) and \ipi{Wolgast} \citep{Warnkross1912}, e.g. [brøːç] ‘bridge’ (=⟦br\^{ö}χ⟧) vs. [dox] ‘day’ (=⟦dox⟧). Neither \citet{Holst1907} nor \citet{Warnkross1912} mention a fronted realization of [k g ŋ].\footnote{{The velar fronting targets in Kaarβen \citep{Dützmann1932} are /x ɣ/, e.g. [nɪç] ‘not’ (=⟦n\k{i}χ⟧), [zœːʝ] ‘sow’ (=⟦z\^{œ}:γ⟧) vs. [lɑxn̥] ‘laugh}\textrm{\textsc{{}-inf}}\textrm{’ (=⟦lɑxn̥⟧), [dɑoɣ] ‘day’ (=⟦dɑ̊o:γ⟧). (⟦γ⟧ represents either velar [ɣ] or palatal [ʝ]). \citet[12]{Dützmann1932} has palatal and velar stops as well as palatal and velar nasals, but he does not discuss the distribution of those sounds. The same point holds for \ipi{Barth} (\citealt{GSchmidt1912}), where /x/ is the sole target for fronting, e.g. [tyːç] ‘stuff’ (=⟦tȳç⟧) vs. [oːx] ‘eye’ (=⟦ōx⟧).} }
\end{sloppypar}

Among the dialects discussed in the preceding paragraph \ipi{South Mecklenburg} (\citealt{Jacobs1925a, Jacobs1925b, Jacobs1926}) is the only one in which \isi{Coalescence-1} is clearly not active, cf. [fɛlx] ‘wheel rim’ (=⟦fęl̅x⟧). That type of example is not mentioned in \citet{Holst1907} or \citet{Dützmann1932} and therefore one cannot know for certain whether or not \isi{Coalescence-1} is present. By contrast, \ipi{Wolgast} is a dialect with \isi{Coalescence-1}, cf. [balx] ‘brat’ (=⟦bɑlx⟧) vs. [telç] ‘branch’ (=⟦telχ⟧).\footnote{{That type of example might also be attested in \ipi{Barth}:  \citet{GSchmidt1912} mentions [felç] ‘wheel rim’ (=⟦felç⟧). However, no examples in that source have a back vowel followed by /lx/.} } Finally, none of the sources cited above appears to have palatal quasi-phonemes.\largerpage

\begin{sloppypar}
Velar noncontinuants do not serve as targets for velar fronting in those NLG varieties spoken in Lower Saxony or Schleswig-Holstein which border \ipi{West Mecklenburg}. The closest of those dialects to \ipi{West Mecklenburg} for which a source is available is the NLG variety of \ipi{Hemmelsdorf} (\citealt{Pühn1956}; \mapref{map:5}), but that work is clear that the sole target for velar fronting is /x/, e.g. [knɛç] ‘vassal’ (=⟦knęχ⟧) vs. [hoːx] ‘high’ (=⟦hōx⟧) and that velar stops and the velar nasal surface without change even after front vowels. The same point holds for the NLG variety of Kreis Herzogtum \ipi{Lauenburg} (\citealt{Heigener1937}; \mapref{map:5}), e.g. [lɪçt] ‘light’ (=⟦licd⟧) vs. [ɑxt] ‘eight’ (=⟦ɑχd⟧), and (NLG) \ipi{Bleckede} (\citealt{Rabeler1911}; \mapref{map:5}), e.g. [gəziçt] ‘face’ (=⟦ɡəzixd⟧) vs. [hoːx] ‘high’ (=⟦ ̔ōχ⟧). No examples were found in any of the aforementioned sources for words consisting of a back vowel plus liquid followed by /x/ which could potentially shed light on whether or not a \isi{Coalescence-1} is active. Likewise no palatal quasi-phonemes were found in any of the sources cited above.\il{Mecklenburgish-West Pomeranian|)}
\end{sloppypar}

\section{Silesian}\label{sec:11.4}\il{Silesian|(}

\citet{Meiche1898} describes the \il{Silesian}Sln variety of \ipi{Sebnitz} (\mapref{map:9}). The patterning of dorsal consonants is depicted in \REF{ex:11:9}.

\ea%9
\label{ex:11:9}
\ea\label{ex:11:9a}
  \begin{forest} for tree = {fit=band}
   [,phantom
    [/ʝ/ [{[ʝ]}] ]
    [/k/ [{[k]}] ]
    [/c/ [{[c]}] ]
    [/k\textsuperscript{h}/,calign=first [{[k\textsuperscript{h}]}] [{[c\textsuperscript{h}]}]]
    [/ŋ/,calign=first [{[ŋ]}] [{[ɲ]}]]
    ]       
  \end{forest}
\ex\label{ex:11:9b}
  \begin{forest} for tree = {fit=band}
   [,phantom
       [/x/ [{[x]}] ]
       [/ç/ [{[ç]}] ]
       [/k/,calign=first [{[k]}] [{[c]}] ]
       [/k\textsuperscript{h}/,calign=first [{[k\textsuperscript{h}]}] [{[c\textsuperscript{h}]}]]          
       [/ŋ/,calign=first [{[ŋ]}]   [{[ɲ]}] ]
   ]
  \end{forest}
\z 
\z 

Meiche refers to the lenis and fortis contrast among stops in terms of aspiration, which is the way in which I transcribe the difference between lenis and fortis sounds, e.g. ⟦g⟧ and ⟦k⟧ are depicted below as [k] and [k\textsuperscript{h}] respectively. An added complication not discussed here is that the aspirated sounds (e.g. [k\textsuperscript{h}]) only occur initially before vowels but not before consonants. Palatal stops are rendered in the original source either with separate symbols or with diacritics making them distinct from the corresponding velars, e.g. [c c\textsuperscript{h}] =⟦gʹ c⟧ and [ʝ ç]=⟦j χ⟧. [ɲ] and the [ŋ] are transcribed as ⟦η⟧ and ⟦\ExtraChars{ᵰ}⟧ respectively. There are four qualities among low vowels. One is front (=⟦ɑ̇⟧=[æ]), while three are back (⟦ɑ⟧=[ɑ], ⟦ɑ̊⟧=[a], ⟦ɒ⟧=[ɒ]).

In word-initial position velars never contrast with the corresponding palatals: [k k\textsuperscript{h}] (<\ili{WGmc} \textsuperscript{+}[ɣ k]) occur before a full back vowel in (\ref{ex:11:10a}, \ref{ex:11:10e}) or a liquid followed by a full back vowel in (\ref{ex:11:10b}), and [c c\textsuperscript{h}] before a front vowel in (\ref{ex:11:10c}, \ref{ex:11:10f}) or a liquid followed by a front vowel in (\ref{ex:11:10d}). The examples in (\ref{ex:11:10g}, \ref{ex:11:10h}) illustrate that the original nasal (\ili{WGmc} \textsuperscript{+}[n]) has undergone \isi{Wd-Initial Nasal Place Assimilation} in (\ref{ex:11:1a}). The derived velar sequence ([kŋ]<\ili{WGmc} \textsuperscript{+}[kn]) surfaces as velar if a back vowel follows those clusters in (\ref{ex:11:10g}) and as palatal if a front vowel follows in the first example in (\ref{ex:11:10h}). [c] (<\ili{WGmc} \textsuperscript{+}[ɣ]) surfaces before \isi{schwa} in (\ref{ex:11:10i}) and [ʝ] (<\ili{WGmc} \textsuperscript{+}[j]) before any vowel in (\ref{ex:11:10j}).

\ea%10
\label{ex:11:10}Word-initial dorsal consonants:
\ea\label{ex:11:10a} gɑ̊st \tab [kast] \tab Gast \tab ‘guest’ \tab 88
\ex\label{ex:11:10b} gloɒs \tab [kloɒs] \tab Glas \tab ‘glass’ \tab 88
\ex\label{ex:11:10c} gʹȩdər \tab [cɛtər] \tab Götter \tab ‘God-\textsc{pl}’ \tab 43
\ex\label{ex:11:10d} gʹlygʹə \tab [clʏcə] \tab Glück \tab ‘fortune’ \tab 45
\ex\label{ex:11:10e} kū \tab [k\textsuperscript{h}uː] \tab  Kuh \tab ‘cow’ \tab 90
\ex\label{ex:11:10f} cęnər \tab [c\textsuperscript{h}ɛnəʀ] \tab  keiner \tab ‘none\textsc{{}-masc.sg}’ \tab 90
\ex\label{ex:11:10g} g\ExtraChars{ᵰ}ɑ̄dṇ \tab [kŋɑːtn̩] \tab  kneten \tab ‘kneed\textsc{{}-inf}’ \tab  90
\ex\label{ex:11:10h} gʹηīə \tab  [cɲiːə] \tab  Knie \tab ‘knee’ \tab 91\\
    gʹη̄ɑ̇̄χt \tab [cɲæːçt] \tab  Knecht \tab ‘vassal’ \tab 90
\ex\label{ex:11:10i} gʹəbūrt \tab [cəpuːʀt] \tab Geburt \tab ‘birth’ \tab  88
\ex\label{ex:11:10j} jumər \tab  [ʝʊməʀ] \tab Jammer \tab ‘lament’ \tab 31 
    \z
\z 

Velar vs. palatal contrasts are also absent in postsonorant position. In that context [x] surfaces after back vowels in (\ref{ex:11:11a}) and [ç] after front vowels in (\ref{ex:11:11b}) or coronal sonorant consonants in (\ref{ex:11:11c}, \ref{ex:11:11d}). The form in \REF{ex:11:11d} exemplifies a difference from \ipi{West Mecklenburg} (cf. \ref{ex:11:5b} vd. \ref{ex:11:5d}). However, the same conclusion cannot be drawn concerning the distribution of velar and palatal stops: [k k\textsuperscript{h}] occur after a back vowel in (\ref{ex:11:11e}, \ref{ex:11:11i}) or after a liquid preceded by a back vowel in (\ref{ex:11:11f}, \ref{ex:11:11j}) and the corresponding palatals [c c\textsuperscript{h}] after a front vowel in (\ref{ex:11:11g}, \ref{ex:11:11k}) or a liquid preceded by a front vowel in (\ref{ex:11:11h}). The data in (\ref{ex:11:11l}, \ref{ex:11:11m}) illustrate the patterning of the velar nasal and the palatal nasal is precisely as in \ipi{West Mecklenburg} (cf. \ref{ex:11:5o}, \ref{ex:11:5p}). Many of the examples containing [ç] listed in the original source occur after a historically elided front vowel in (\ref{ex:11:11n} via \isi{Syncope}) or after the historical coronal rhotic /r/ in (\ref{ex:11:11o} via \isi{r-Retraction}, recall \chapref{sec:7}). The dorsal consonants referred to in the present paragraph derived historically from velars (\ili{WGmc} \textsuperscript{+}[x ɣ k]). The place of articulation of the syllabic nasal in (\ref{ex:11:11d}, \ref{ex:11:11h}) is determined by \isi{Progressive Nasal Place Assimilation} (\fnref{fn:11:3}).

\TabPositions{.2\textwidth, .4\textwidth, .6\textwidth, .8\textwidth}
\ea%11
\label{ex:11:11}Postsonorant dorsal consonants:
\ea\label{ex:11:11a}  nɑxt \tab [nɑxt] \tab  Nacht \tab ‘night’ \tab 27
\ex\label{ex:11:11b}  hɑ̇χt \tab [hæçt] \tab  Hecht \tab ‘pike’ \tab 57
\ex\label{ex:11:11c}  milχ \tab [mɪlç] \tab  Milch \tab ‘milk’ \tab 37
\ex\label{ex:11:11d}  gɑ̊lχη \tab [kalçɲ̍] \tab  Galgen \tab ‘gallows’ \tab 88
\ex\label{ex:11:11e}  flugs \tab [flʊks] \tab  flugs \tab ‘quickly’ \tab 88
\ex\label{ex:11:11f}  fɑ̊lgə \tab [falkə] \tab Falke \tab ‘falcon’ \tab 29
\ex\label{ex:11:11g}  dygʹə \tab [tʏcə] \tab dick \tab ‘fat’ \tab 91\\
     undərwɑ̇̄gʹs \tab [untəʀvæːcs] \tab  unterwegs \tab ‘underway’ \tab 88
\ex\label{ex:11:11h}  mɑ̇lgʹη \tab [mælcɲ̍] \tab melken \tab ‘milk\textsc{{}-inf}’ \tab 35
\ex\label{ex:11:11i}  sɑ̄k \tab [sɑːk\textsuperscript{h}] \tab  Sack \tab ‘sack’ \tab 91
\ex\label{ex:11:11j}  fulk \tab [fʊlk\textsuperscript{h}] \tab Volk \tab ‘people’ \tab 91
\ex\label{ex:11:11k}  drɑ̇̄c\tab [dʀæːc\textsuperscript{h}] \tab  Dreck \tab ‘dirt’ \tab 91
\ex\label{ex:11:11l}  zwɑ\ExtraChars{ᵰ}k \tab [tsvɑŋk\textsuperscript{h}] \tab Zwang \tab ‘compulsion’ \tab 91
\ex\label{ex:11:11m}  diηc \tab [tɪɲc\textsuperscript{h}] \tab Ding \tab ‘thing’ \tab 91
\ex\label{ex:11:11n}  kɑ̄fχ \tab  [kɑːfç] \tab  Käfig \tab  ‘cage’ \tab 34
\ex\label{ex:11:11o}  mɑ̊rχt \tab [maʀçt] \tab  Markt \tab ‘market’ \tab 91
    \z 
\z\todo{Zwang has a glyph problem}

For word-initial position, palatal stops ([c c\textsuperscript{h}]) in pre-vocalic position (=\ref{ex:11:10c}, \ref{ex:11:10f}) derive from the corresponding velars (/k k\textsuperscript{h}/) by \isi{Wd-Initial Velar Fronting-6}. \isi{Coalescence-2} merges the [coronal] feature of the front vowel and the preceding liquid in (\ref{ex:11:10d}), and the fronting of the velar preceding that liquid is accomplished with \isi{Wd-Initial Velar Fronting-6}, e.g. /klʏcə/→{\textbar}k\textbf{lʏ}cə{\textbar}→[\textbf{clʏ}cə]. The homorganic nasal plus stop sequences in (\ref{ex:11:10g}, \ref{ex:11:10h}) have a single [dorsal] feature (/kŋ/). If /kŋ/ is followed by a front vowel, then the feature [coronal] spreads to the left by \isi{Wd-Initial Velar Fronting-6}, e.g. /kŋiːə/→[\textbf{cɲiː}ə]. The initial consonant in (\ref{ex:11:10i}, \ref{ex:11:10j}) is an underlying palatal, i.e. the \isi{palatal quasi-phoneme} /c/ in (\ref{ex:11:10i}), and the \isi{etymological palatal} /j/ in (\ref{ex:11:10j}).

After a front vowel (in \ref{ex:11:11b}, \ref{ex:11:11g}, \ref{ex:11:11k}), palatal stops and fricatives derive from velars (/x k kʰ/) by \isi{Velar Fronting-8}, and after a back vowel those velars surface without change as [x k kʰ] in (\ref{ex:11:11a}, \ref{ex:11:11e}, \ref{ex:11:11i}). If a front vowel is followed by a liquid in (\ref{ex:11:11c}, \ref{ex:11:11h}) then \isi{Coalescence-1} applies, e.g. /mɪlx/→{\textbar}m\textbf{ɪl}x{\textbar}; /mælkŋ̍{\textbar}→{\textbar}m\textbf{æl}kŋ̍{\textbar}. If the liquid is preceded by a back vowel in (\ref{ex:11:11d}) then the feature [coronal] from the liquid spreads to /x/ by \isi{Velar Fronting-1} in (\ref{ex:11:12}), e.g. /kalxŋ̍/→{\textbar}ka\textbf{lç}ŋ̍{\textbar}. Since the target for \REF{ex:11:12} is a velar fricative, spreading occurs in \REF{ex:11:11d} but not in (\ref{ex:11:11f}, \ref{ex:11:11j}), e.g. /fʊlk\textsuperscript{h}/→[fʊlk\textsuperscript{h}]. The merged [coronal] feature in (\ref{ex:11:11c}, \ref{ex:11:11h}) spreads to the following velar by either \isi{Velar Fronting-1} in \REF{ex:11:12} or \isi{Velar Fronting-8}, thereby creating a palatal.

\ea%12
\label{ex:11:12}\isi{Velar Fronting-1}:\\
\begin{forest}
[,phantom
    [\avm{[+son]} [\avm{[coronal]},name=target,tier=word]]
    [\avm{[−son\\+cont]},name=parent [\avm{[dorsal]},tier=word]]
]
\draw [dashed] (target.north) -- (parent.south);
\end{forest}
\z 

Nasal plus stop clusters in (\ref{ex:11:11l}, \ref{ex:11:11m}) bear a single [dorsal] feature in the underlying representation. If a front vowel precedes that cluster in (\ref{ex:11:11m}) then the feature [coronal] of the front vowel spreads to the left by \isi{Velar Fronting-8}, thereby creating [ɲc\textsuperscript{h}]. [ç] in (\ref{ex:11:11n}, \ref{ex:11:11o}) is an underlying palatal (quasi-phoneme), i.e. /ç/.

\citet{Michel1891} describes the \il{Silesian}Sln variety of \ipi{Seifhennersdorf} (\mapref{map:9}). That dialect possesses velar and palatal fricatives [x ɣ] (=⟦χ ʒ⟧) and [ç ʝ] (=⟦ȷ j⟧), velar and palatal stops [k\textsuperscript{h} k] (=⟦kh k⟧) and [c\textsuperscript{h} c] (=⟦ch c⟧), the velar nasal [ŋ] (=⟦\ExtraChars{ᵰ}⟧),\todo{glyph} and the palatal nasal ([ɲ]=⟦ŋ⟧). The distribution of those sounds is illustrated in \REF{ex:11:13}.\footnote{\label{fn:11:8}In his description of the neighboring dialect spoken in Groβschönau (see below), \citet[2--3]{Wenzel1919} refers to the dialect spoken in \ipi{Seifhennersdorf} as “de[m] merkwürdigsten aller Dialekte der Oberlausitzˮ. (“The most peculiar of all dialects of the Oberlausitzˮ). At the time he wrote those words (in 1919) he considered both \ipi{Seifhennersdorf} and \ipi{Sebnitz} to be already archaic (“bereits historischˮ).}

\ea%13
\label{ex:11:13}
\ea\label{ex:11:13a} \begin{forest} for tree = {fit=band}
[,phantom
    [/ʝ/ [{[ʝ]}]]  
    [/k/ [{[k]}]]    
    [/c/ [{[c]}]]   
    [/k\textsuperscript{h}/,calign=first [{[k\textsuperscript{h}]}] [{[c\textsuperscript{h}]}]]            
    [/ŋ/,calign=first [{[ŋ]}] [{[ɲ]}]]
]               
\end{forest}
\ex\label{ex:11:13b} \begin{forest} for tree = {fit=band}
[,phantom
   [/x/ [{[x]}]]  
   [/ç/ [{[ç]}]]   
   [/ɣ/,calign=first [{[ɣ]}] [{[ʝ]}]]          
   [/k/,calign=first [{[k]}] [{[c]}]]             
   [/ŋ/,calign=first [{[ŋ]}] [{[ɲ]}]]
]
\end{forest}
\z 
\z 

Velars never contrast with the corresponding palatals. In word-initial position [k\textsuperscript{h} k] occur before a full back vowel in (\ref{ex:11:14a}, \ref{ex:11:14e}) or a consonant followed by a full back vowel in (\ref{ex:11:14b}) and [c\textsuperscript{h} c] before a front vowel in (\ref{ex:11:14c}, \ref{ex:11:14f}) or a consonant followed by a front vowel in (\ref{ex:11:14d}).\footnote{{Most of Michel’s examples belonging to category \REF{ex:11:14d} have [i] after the liquid. In some of his data the initial sound is transcribed as velar (⟦k⟧) if the post-liquid vowel is nonhigh, e.g. [knæçt] ‘vassal’ (=⟦knɑχt⟧). It is therefore possible that the set of triggers for the process of word-initial velar fronting described below consists of nonhigh front vowels.} } The stops referred to here ([k\textsuperscript{h} k c\textsuperscript{h} c]) derived from historical velars (\ili{WGmc} \textsuperscript{+}[ɣ k]). A stop plus nasal sequence (<\ili{WGmc} \textsuperscript{+}[kn]) via \isi{Wd-Initial Nasal Place Assimilation} in (\ref{ex:11:1a}) surfaces as velar before a back vowel in (\ref{ex:11:14g}) and palatal before a front vowel in (\ref{ex:11:14h}). Palatal [c] (<\ili{WGmc} \textsuperscript{+}[ɣ]) occurs before \isi{schwa} in (\ref{ex:11:14i}) and [ʝ] (<\ili{WGmc} \textsuperscript{+}[j]) before any vowel in (\ref{ex:11:14j}).

\ea%14
\label{ex:11:14}Word-initial dorsal obstruents:
\ea\label{ex:11:14a} kut \tab [kʊt] \tab  gut \tab ‘good’ \tab 57
\ex\label{ex:11:14b} klɑͅs \tab [klɑs] \tab Glas \tab ‘glass’ \tab 7
\ex\label{ex:11:14c} cęstɑ̆n \tab [cɛstɐn] \tab  gestern \tab ‘yesterday’ \tab 57
\ex\label{ex:11:14d} cliŋcĕ \tab [clɪɲcə] \tab  Klinke \tab ‘handle’ \tab 50
\ex\label{ex:11:14e} khɑ̄lt \tab [k\textsuperscript{h}ɑːlt] \tab  kalt \tab ‘cold’ \tab 55
\ex\label{ex:11:14f} chind \tab [c\textsuperscript{h}ɪnt] \tab  Kind \tab ‘child’ \tab 55
\ex\label{ex:11:14g} k\ExtraChars{ᵰ}outn \tab [kŋoutn̩] \tab Knoten \tab ‘node’ \tab 13
\ex\label{ex:11:14h} cŋī \tab [cɲiː] \tab Knie \tab ‘knee’ \tab 20
\ex\label{ex:11:14i} cĕbūrt \tab [cəpuːrt] \tab Geburt \tab ‘birth’ \tab  11
\ex\label{ex:11:14j} ĭumɑ \tab [ʝʊmɐ] \tab Jammer \tab ‘lament’ \tab 42
\z
\z 

In postsonorant position [x ɣ] occur after back vowels and [ç ʝ] after front vowels or coronal sonorant consonants in (\ref{ex:11:15a}--\ref{ex:11:15g}). [x ɣ ç ʝ] in these examples derive from historical velars (\ili{WGmc} \textsuperscript{+}[x k ɣ]). [k] and [c] have a distribution that mirrors their fricative counterparts in (\ref{ex:11:15h}--\ref{ex:11:15j}). On the basis of (\ref{ex:11:15d}, \ref{ex:11:15g}, \ref{ex:11:15j}) it can be deduced that \isi{Coalescence-1} is not active in the phonology of \ipi{Seifhennersdorf}. The dorsal stops ([k c]) in (\ref{ex:11:15h}--\ref{ex:11:15j}) derive from etymological velars (\ili{WGmc} \textsuperscript{+}[ɣ k] or \textsuperscript{+}[gg]). The clusters [ŋk ɲc] (<\ili{WGmc} \textsuperscript{+}[ŋk]) surface after back vowels and front vowels respectively in (\ref{ex:11:15k}, \ref{ex:11:15l}). Example (\ref{ex:11:15m}) indicates that [ç] (<\ili{WGmc} \textsuperscript{+}[ɣ]) is also present after a historically elided front vowel (by \isi{Syncope}).

\TabPositions{.15\textwidth, .33\textwidth, .5\textwidth, .8\textwidth}
\ea%15
\label{ex:11:15}Postsonorant dorsal consonants:
\ea\label{ex:11:15a} woχĕ \tab [vɔxə] \tab  Woche \tab ‘week’ \tab 56
\ex\label{ex:11:15b} hɑȷet \tab  [hæçt] \tab  Hecht \tab ‘pike’ \tab 6
\ex\label{ex:11:15c} mylȷ \tab  [mʏlç] \tab  Milch \tab ‘milk’ \tab 46
\ex\label{ex:11:15d} molȷ \tab [mɔlç] \tab Molch \tab ‘newt’ \tab 46
\ex\label{ex:11:15e} ouʒĕ \tab [ouɣə] \tab Auge \tab ‘eye’ \tab 57
\ex\label{ex:11:15f} ęjĕ \tab [ɛʝə] \tab Egge \tab ‘harrow’ \tab 57
\ex\label{ex:11:15g} foljĕ \tab [fɔlʝə] \tab Folge \tab ‘consequence’ \tab 58
\ex\label{ex:11:15h} pflūk \tab [pfluːk] \tab  Pflug \tab ‘plow’ \tab 57
\ex\label{ex:11:15i} mycĕ \tab [mʏcə] \tab  \ipi{Mücke} \tab ‘mosquito’ \tab 57
\ex\label{ex:11:15j} khɑͅlc \tab [k\textsuperscript{h}ɑlc] \tab Kalk \tab ‘lime’ \tab 45
\ex\label{ex:11:15k} fɑͅ\ExtraChars{ᵰ}k \tab [fɑŋk] \tab fang \tab ‘catch\textsc{{}-imp.sg}’ \tab 6
\ex\label{ex:11:15l} tiŋc \tab [tɪɲc] \tab Ding \tab ‘thing’ \tab 50
\ex\label{ex:11:15m} chęfȷ \tab [k\textsuperscript{h}ɛfç] \tab  Käfig \tab ‘cage’ \tab 42
\z 
\z\todo{glyph}

For word-initial position, palatal stops in pre-vocalic position (in \ref{ex:11:14c}, \ref{ex:11:14f}) derive synchronically from the corresponding velars (/k kʰ/) by \isi{Wd-Initial Velar Fronting-6}. In \REF{ex:11:14d} \isi{Coalescence-2} merges the [coronal] feature of the front vowel and the coronal feature of the preceding liquid, thereby \isi{feeding} fronting, e.g. /klɪŋkə/→{\textbar}k\textbf{lɪ}ŋkə{\textbar}→{\textbar}\textbf{clɪ}ŋkə{\textbar}. The nasal plus stop sequences in (\ref{ex:11:14g}, \ref{ex:11:14h}) are underlying velar (/kŋ/). If the [dorsal] feature of /kŋ/ is followed by a front vowel, then its [coronal] feature spreads to the left by \isi{Wd-Initial Velar Fronting-6}, e.g. /kŋiː/→[\textbf{cɲiː}]. The word-initial consonants in (\ref{ex:11:14i}, \ref{ex:11:14j}) are underlying palatals, i.e. /ʝ c/.

After a sonorant, palatals derive from velars by \isi{Velar Fronting-9} in (\ref{ex:11:16}). Given the broad set of triggers (i.e. coronal sonorants), \REF{ex:11:16} spreads [coronal] from a front vowel in (\ref{ex:11:15b}, \ref{ex:11:15f}, \ref{ex:11:15i}) or liquid in (\ref{ex:11:15c}, \ref{ex:11:15g}, \ref{ex:11:15j}) to a following velar (/x ɣ k/). In examples (\ref{ex:11:15k}, \ref{ex:11:15l}) the nasal stop clusters (/ŋk/) bear one [dorsal] feature. If a front vowel precedes that cluster in (\ref{ex:11:15l}) then the feature [coronal] of that front vowel spreads to the right by \REF{ex:11:16}, e.g. /tɪŋk/→[t\textbf{ɪɲc}]. The final segment in (\ref{ex:11:15m}) is an underlying palatal (quasi-phoneme), i.e. /ç/.

\ea%16
\label{ex:11:16}\isi{Velar Fronting-9}:\\
\begin{forest}
[,phantom
  [\avm{[+son]} [\avm{[coronal]},name=target]]
  [\avm{[+cons]},name=source [\avm{[dorsal]}]]
]
\draw [dashed] (source.south) -- (target.north);
\end{forest}
\z 

Note the difference between \ipi{Seifhennersdorf} and \ipi{Sebnitz}: In the former variety, palatals derive from velars after front vowels and sonorant consonants alike. However, in \ipi{Sebnitz} the choice of velar vs. palatal is determined by the vowel preceding the liquid, but only in the case of palatal stops (recall \ref{ex:11:11h}, \ref{ex:11:11j}), but not palatal fricatives (recall \ref{ex:11:11c}, \ref{ex:11:11d}).

Like \ipi{West Mecklenburg}, the two \il{Silesian}Sln varieties described above are unique in more than one way. In particular, none of the neighboring communities are reported to have velar noncontinuants as targets for velar fronting. The \il{Silesian}Sln variety closest geographically to \ipi{Sebnitz} and \ipi{Seifhennersdorf} for which a description is available is \citet{Wenzel1919} (\mapref{map:9}; recall \fnref{fn:11:8}). It is clear from that source that the set of targets for velar fronting consists solely of /x/ and that velar noncontinuants do not have palatal realizations, e.g. [lɪçt] ‘light’ (=⟦liχt⟧) vs. [lɑxŋ̍]  ‘laugh\textsc{{}-inf}’ (=⟦lɑχ\ExtraChars{ᵰ}⟧). The original sources for the \il{Upper Saxon}USax varieties spoken in \ipi{Schokau} (\citealt{Pompé1907}; \mapref{map:12}) and the broad area in \ipi{West Lausitz} (\citealt{Protze1957}; \mapref{map:12}) devote considerable discussion to the phonetics of consonants and vowels. It is clear from both sources that the sole target for velar fronting is /x/ but that velar noncontinuants do not have a palatal realization. No examples were found in any of the aforementioned sources for words consisting of a back vowel plus liquid followed by /x/ which could potentially shed light on whether or not \isi{Coalescence-2} is active.

\il{Silesian}Sln dialects located further away from \ipi{Sebnitz} and \ipi{Seifhennersdorf} are not reported to have noncontinuants as targets for velar fronting either. See in particular the varieties referred to in \sectref{sec:5.3.2} (\mapref{map:9}), namely Kreis \ipi{Jauer} \citep{Halbsguth1938}, \ipi{Kieslingswalde} (Kreis Habelschwerdt; \citealt{Pautsch1901}), and the supraregional \il{Silesian}Sln dialect described by \citet{vonUnwert1908}.\footnote{SchlSA makes no reference to palatal noncontinuants either. For example, on Map 51 for  \textit{kein}  ‘none’, all of the realizations begin with the velar [k] (cf. \ref{ex:11:10f} with [c]). In the introduction to that atlas the list of consonants (p. 5) includes palatal fricatives (both lenis and fortis), but only velar stops. In a separate chart on the same page there is a symbol for a palatalized (“palatalisiert[e]ˮ) velar nasal, but no tokens with that segment were found in the maps.}\il{Silesian|)}

\section{East Pomeranian}\label{sec:11.5}\il{East Pomeranian|(}

\citet{Mischke1936} describes the \il{East Pomeranian}EPo dialects spoken once in Kreis \ipi{Bütow} and Kreis \ipi{Rummelsburg}, which I consider in that order (\mapref{map:18}). The synchronic distribution of dorsal consonants in Kreis \ipi{Bütow} is depicted in \REF{ex:11:17}.

\begin{map}
% \includegraphics[width=\textwidth]{figures/VelarFrontingHall2021-img024.png}
\includegraphics[width=\textwidth]{figures/Map18_11.2.pdf}
\caption[East Pomeranian, Low Prussian, and High Prussian]{East Pomeranian (\il{East Pomeranian}EPo), Low Prussian (\il{Low Prussian}LPr), and High Prussian (\il{High Prussian}HPr). Squares indicate postsonorant velar fronting, and the circle indicates the absence of postsonorant velar fronting. 1=\citet{Teuchert1913}, 2=\citet{Semrau1915a, Semrau1915b}, 3=\citet{Pirk1928}, 4=\citet{Mahnke1931}, 5=\citet{Kühl1932}, 6=\citet{Mischke1936} (Kreis \ipi{Bütow}), 7=\citet{Mischke1936} (Kreis \ipi{Rummelsburg}), 8=\citet{Stritzel1937} (Kreis \ipi{Lauenburg}), 9=\citet{Stritzel1937} (Kreis \ipi{Stolp}), 10=\citealt{Tita1921}, 11=\citet{Darski1973}, 12=\citet{Kuck1933}, 13=\citet{KuckWiesinger1965}, 14=\citet{Wagner1912}, 15=\citet{Mitzka1919}, 16=\citet{Mitzka1922}, 17=\citet{Natau1937}, 18=\citet{Bink1953}, 19=\citet{Tessmann1966}.}\label{map:18}
\end{map}

\ea%17
\label{ex:11:17}
\ea\label{ex:11:17a}
\begin{forest}  for tree = {fit=band}
[,phantom
  [/ɣ/ [{[ɣ]}]]  
  [/ʝ/ [{[ʝ]}]]   
  [/x/ [{[x]}]]   
  [/k/,calign=first [{[k]}] [{[c]}]]
]
\end{forest}
\ex\label{ex:11:17b}
\begin{forest} for tree = {fit=band}
[,phantom
     [/x/,calign=first [{[x]}] [{[ç]}]]                                             
     [/ɣ/,calign=first [{[ɣ]}] [{[ʝ]}]]
     [/k/,calign=first [{[k]}] [{[c]}]]
     [/g/,calign=first [{[g]}] [{[ɟ]}]]
     [/ŋ/ [{[ŋ]}] ]
     [/ɲ/ [{[ɲ]}] ]
]
\end{forest}
\z 
\z 

[x] surfaces word-initially before a consonant in (\ref{ex:11:18a}), but not before a vowel. [ɣ] and [ʝ] in (\ref{ex:11:18b}--\ref{ex:11:18d}) exemplify Contrast Type B in \REF{ex:11:2a}. The vowel [ɑːi] in \REF{ex:11:18c} was historically front (cf. \ili{OSax} \textit{giotan}). An example of a [ɣ]{\textasciitilde}[ʝ] alternation is listed in \REF{ex:11:18e}. The initial segment in (\ref{ex:11:18a}--\ref{ex:11:18e}) derived historically from a velar (\ili{WGmc} \textsuperscript{+}[ɣ]). Palatal [ʝ] (<\ili{WGmc} \textsuperscript{+}[j]) occurs before any type of vowel in (\ref{ex:11:18f}). Word-initial dorsal stops ([k c]) stand in an allophonic relationship: [k] (=⟦k⟧) surfaces before back vowels or consonants in (\ref{ex:11:18g}) and [c] (=⟦k´⟧) before front vowels in (\ref{ex:11:18h}). No data are given for a word beginning with a dorsal stop followed by a liquid plus front vowel; hence, it cannot be determined if \isi{Coalescence-2} is active. The formal rule of fronting of velars in word-initial position (see below) is triggered by all front vowels in contrast to the fronting process in neighboring Kreis \ipi{Rummelsburg}.

\ea%18
\label{ex:11:18}Word-initial dorsal fricatives:
\ea\label{ex:11:18a} xrɑf \tab [xrɑf] \tab Grab \tab ‘grave’ \tab 39\\
    xl\k{i}k \tab [xlɪk] \tab Glück \tab ‘fortune’ \tab 39\\
    xnōuʒə \tab [xnoːuɣə] \tab nagen \tab ‘gnaw\textsc{{}-inf}’ \tab 39
\ex\label{ex:11:18b} ʒɑųt \tab [ɣɑːut] \tab gut \tab ‘good’ \tab 39\\
    ʒųlt \tab [ɣʊlt] \tab Gold \tab ‘gold’ \tab 39
\ex\label{ex:11:18c} jɑ̄\k{i}tə \tab [ʝɑːitə] \tab gieβen \tab ‘water\textsc{{}-inf}’ \tab 39
\ex\label{ex:11:18d} j\k{i}rtl \tab [ʝɪrtl̩] \tab Gürtel \tab ‘belt’ \tab 39
\ex\label{ex:11:18e} ʒɑ̊· \tab [ɣɑ·] \tab gehen \tab ‘go\textsc{{}-inf}’ \tab 64\\
    jiŋ \tab [ʝiŋ] \tab ging \tab ‘go\textsc{{}-pret}’ \tab 64
\ex\label{ex:11:18f} jo·pə \tab [ʝo·pə] \tab Joppe \tab ‘jacket’ \tab 40\\
    jǟ·kə \tab [ʝæːkə] \tab jucken \tab ‘itch\textsc{{}-inf}’ \tab 40
\ex\label{ex:11:18g} kōukə \tab [koːukə] \tab kochen \tab ‘cook\textsc{{}-inf}’ \tab 38\\
    kno·p \tab [kno·p] \tab Knopf \tab ‘button’ \tab 38
\ex\label{ex:11:18h} k´i·k´ə \tab [ci·cə] \tab gucken \tab ‘look\textsc{{}-inf}’ \tab 38
    \z
\z 

In the context after a sonorant, velar obstruents do not contrast with the corresponding palatals. Thus, velars ([x ɣ k g]) surface after a back vowel in (\ref{ex:11:19a}, \ref{ex:11:19c}, \ref{ex:11:19f}, \ref{ex:11:19i}) and palatals ([ç ʝ c ɟ]) after a front vowel in (\ref{ex:11:19b}, \ref{ex:11:19d}, \ref{ex:11:19g}, \ref{ex:11:19j}) or coronal sonorant consonant in (\ref{ex:11:19e}, \ref{ex:11:19h}). The palatals and velars referred to in (\ref{ex:11:19a}--\ref{ex:11:19j}) derive historically from velars (\ili{WGmc} \textsuperscript{+}[x ɣ k]).

\ea%19
\label{ex:11:19}Distribution of postsonorant dorsal fricatives:
\ea\label{ex:11:19a} jūx \tab [ʝuːx] \tab euer \tab ‘your\textsc{{}-pl}’ \tab 26\\
    dǫxtə(r) \tab [dɔxtə(r)] \tab Tochter \tab ‘daughter’ \tab  12
\ex\label{ex:11:19b} n\k{i}χ \tab [nɪç] \tab  nicht \tab ‘not’ \tab 35\\
    flēχ \tab [fleːç] \tab Floh \tab ‘flea’ \tab 24\\
    ǟχ \tab [æːç] \tab stumpf \tab ‘blunt’ \tab 10
\ex\label{ex:11:19c} būʒə \tab [buːɣə] \tab bauen \tab ‘build\textsc{{}-inf}’ \tab 26
\ex\label{ex:11:19d} štījə \tab [ʃtiːʝə] \tab steigen \tab ‘climb\textsc{{}-inf}’ \tab 20\\
    lǟjə \tab [læːʝə] \tab legen \tab ‘place\textsc{{}-inf}’ \tab 10
\ex\label{ex:11:19e} mǫrjə \tab [mɔrʝə] \tab morgen \tab ‘tomorrow’ \tab 13\\
    bɑlχ \tab [bɑlç] \tab Balg \tab ‘brat’ \tab 48
\ex\label{ex:11:19f} klɑ̄ųk \tab [klɑːuk] \tab klug \tab ‘clever’ \tab 38
\ex\label{ex:11:19g} ɑ̄\k{i}k´ \tab [ɑːic] \tab Eiche \tab ‘oak tree’ \tab 38
\ex\label{ex:11:19h} mälk´ \tab [mælc] \tab Milch \tab ‘milk’ \tab 38
\ex\label{ex:11:19i} bɑɡdə \tab [bɑgdə] \tab backte \tab ‘bake\textsc{{}-pret}’ \tab 39
\ex\label{ex:11:19j} b\k{i}ɡ´ə \tab [bɪɟə] \tab picken \tab ‘pick\textsc{{}-inf}’ \tab 39
\ex\label{ex:11:19k} tųŋ \tab [tʊŋ] \tab Zunge \tab ‘tongue’ \tab 15\\
    ɑŋəs \tab [ɑŋəs] \tab anders \tab ‘different’ \tab 36
\ex\label{ex:11:19l} b\k{i}ŋ´əl \tab [bɪɲəl] \tab Bengel \tab ‘rascal’ \tab 32
\ex\label{ex:11:19m} drɑŋ´k \tab [drɑɲc] \tab Trank \tab ‘drink’ \tab 32
    \z
\z 

The distribution of [ŋ] in (\ref{ex:11:19k}) and [ɲ] in (\ref{ex:11:19l}, \ref{ex:11:19m}) exemplifies Contrast Type B in \REF{ex:11:2b}. Historical [nd] sequences shifted to [ŋ] via \isi{Velarization} in (\ref{ex:11:1b}), e.g. the second example under (\ref{ex:11:19k}). Historical [ŋ] surfaces as [ŋ] after a back vowel in the first example in (\ref{ex:11:19k}) and as [ɲ] after a front vowel in (\ref{ex:11:19l}). Palatal [ɲ] in (\ref{ex:11:19m}) was historically [ŋ] (cf. \il{Standard German}StG [trɑŋk]).

The initial sound in (\ref{ex:11:18c}, \ref{ex:11:18f}) is an underling palatal (/ʝ/). In all other examples in dataset \REF{ex:11:18} the leftmost sound is an underlying velar (/x ɣ k/) which surfaces as palatal before front vowels by \isi{Wd-Initial Velar Fronting-6}. The postsonorant dorsal consonants in (\ref{ex:11:19a}--\ref{ex:11:19l}) are underlyingly velar (/x ɣ k g ŋ/) which surface as palatal after a coronal sonorant by \isi{Velar Fronting-9}. The postvocalic nasal plus stop sequence in (\ref{ex:11:19m}) is underlyingly palatal. Since the back vowel in that example was also etymologically back (cf. \ili{MHG} \textit{tranc}), the \isi{phonemicization} of palatal /ɲ/ in that word was probably a consequence of \isi{analogy} (\sectref{sec:8.6.2}), cf. [drɪŋkə] ‘drink-\textsc{inf}’ (=⟦dr\k{i}ŋ´kə⟧).\footnote{No words were found in the original source in which [ŋ] and [ɲ] alternate, although I consider that gap to be accidental. Examples in which \isi{Velarization} (=\ref{ex:11:1b}) applies after a front vowel which could potentially \isi{feed} velar fronting are apparently absent.}

Consider now the patterning of dorsal obstruents in Kreis \ipi{Rummelsburg}  (\citealt{Mischke1936}; \mapref{map:18}) in \REF{ex:11:20}. Note that [g] is absent entirely. All instances of [g] in Kreis \ipi{Bütow} are realized as [k] in Kreis \ipi{Rummelsburg}, e.g. Kreis \ipi{Bütow}  [bɑgdə] backte ‘bake-\textsc{pret}’ (=\ref{ex:11:19i}) vs. Kreis \ipi{Rummelsburg} [bɑːkdə] (=⟦bɑkdə⟧).

\ea%20
\label{ex:11:20}
\begin{multicols}{2}
\ea\label{ex:11:20a}\begin{forest}
      [,phantom
        [/ɣ/ [{[ɣ]}]]   
        [/ʝ/ [{[ʝ]}]] 
        [/x/ [{[x]}]]
      ]              
    \end{forest}
\ex\label{ex:11:20b}\begin{forest} for tree = {fit=band}  
   [,phantom
       [/x/,calign=first [{[x]}] [{[ç]}]]         
       [/ɣ/,calign=first [{[ɣ]}] [{[ʝ]}]]
   ]
      \end{forest}
\z 
\end{multicols}
\z 

A significant difference between Kreis \ipi{Bütow} and Kreis \ipi{Rummelsburg} is that the former dialect possesses palatal noncontinuants (recall \ref{ex:11:17}), but the latter does not. This point is clear in the description of the reflexes of \ili{MLG} [k] in \citet[38--39]{Mischke1936}. For example, Mischke transcribes the Kreis \ipi{Bütow} realization of  ⟦k´i·k´ə⟧ ‘look-\textsc{inf}’ in \REF{ex:11:18h} with palatal stops, but the same word is rendered with velar stops (⟦ki·kə⟧) in Kreis \ipi{Rummelsburg}. Likewise palatal [ɲ] in Kreis \ipi{Bütow} is absent in Kreis \ipi{Rummelsburg}, which is decidedly velar (“ausgesprochen gutturalˮ; \citealt{Mischke1936}: 32).

A second significant difference between Kreis \ipi{Bütow} and Kreis \ipi{Rummelsburg} is the set of triggers for postsonorant velar fronting. Kreis \ipi{Rummelsburg} has the phonemic monophthongs  in \tabref{tab:fromex:11:21}. All phonemic vowels are included here with the exception of placeless \isi{schwa} (/ə/). The three-way length distinction among certain vowels is ignored.\largerpage[1.5]

\begin{table}[H]%21
\caption{\label{tab:fromex:11:21} Distinctive features for vowels (Kreis Rummelsburg)}
\begin{tabular}{lccccccccccc}
\lsptoprule
         & i i· i & ɪ & e e· & ɛ & æː æ· & uː u· u & ʊ & oː o· & ɔ & ɑː ɑ & aː a·\\\midrule
\relax [coronal] & \ding{51} & \ding{51} & \ding{51} & \ding{51} & \ding{51} &  &  &  &  &  & \\
\relax [dorsal] &  &  &  &  &  & \ding{51} & \ding{51} & \ding{51} & \ding{51} & \ding{51} & \ding{51}\\
\relax [high] & + & + & {}-- & {}-- & {}-- & + & + & {}-- & {}-- & {}-- & {}--\\
\relax [tense] & + & {}-- & + & {}-- & + & + & {}-- & + & {}-- & + & {}--\\
\relax [low] &  &  & {}-- &  & + &  &  &  &  &  & \\
\lspbottomrule
\end{tabular}
\end{table}

Among front vowels, /i i· i ɪ/ are [+high] and /eː e· ɛ æː æ·/ are [--high]. Within both groups, the split is then made between [+tense] and [--tense]. In the [coronal, --high, +tense] category, [±low] distinguishes /eː e·/ from /æː æ·/. Within each of the three [coronal, +tense] columns, length units distinguish the individual members. The same procedure assigns the features listed above to the [dorsal] vowels. It is demonstrated below that [±tense] is crucial in defining the set of triggers for postsonorant fronting.\largerpage[2]

\citet{Mischke1936} lists seven diphthongs; the ones important for my treatment are the two ending in a front vowel, which he transcribes as ⟦ɑ\k{i} ei⟧. Note that the second component of ⟦ɑ\k{i}⟧ is rendered with the traditional symbol for a lax vowel, while the second part of ⟦ei⟧ with the traditional symbol for a tense vowel. I treat the second part of both diphthongs as phonologically [+tense] (=[ɑːi ei]) because their right edges behave as [+tense] vowels. As in \il{Standard German}StG, no word in Kreis \ipi{Rummelsburg} can end in a lax vowel. For example, there are words ending in [i·] but not [ɪ], e.g. [fri·] ‘free’ (=⟦fri·⟧); \citet[20]{Mischke1936}. Significantly, there are words ending in both [ei] and [ɑːi], e.g. [ʃnɑːi] ‘snow’ (=⟦šnɑ\k{i}⟧), [dei] ‘you{}-\textsc{dat}.\textsc{sg}’ (=⟦dei⟧); \citet[17, 20]{Mischke1936}. The existence of words like those suggests that the second component of the diphthongs [ei] and [ɑːi] is phonologically [+tense].

 The patterning of dorsal fricatives in word-initial position is the same as in the related variety of Kreis \ipi{Bütow} (recall \ref{ex:11:18}). What is important is the distribution of [x ɣ] and their palatal counterparts in postsonorant position. The following datasets demonstrate that the velars never contrast with the corresponding palatals. In \REF{ex:11:22} it can be seen that [ç] surfaces after a front [+tense] monophthong in (\ref{ex:11:22a}) and [x] after a front [--tense] monophthong in (\ref{ex:11:22b}) or a back vowel in (\ref{ex:11:22c}). The historical reflex of the postvocalic dorsal fricatives in \REF{ex:11:22} and below is a velar sound (\ili{WGmc} \textsuperscript{+}[ɣ x k]). Examples like [liçt] ‘light’ with a short front tense vowel [i] in \REF{ex:11:22a} are important because they show that the trigger for fronting is the \isi{tenseness} feature and not a feature for length.

\TabPositions{.13\textwidth, .28\textwidth, .55\textwidth, .8\textwidth}
\ea%22
\label{ex:11:22}\relax[ç] and [x] (from /x/):
\ea\label{ex:11:22a} miːχt \tab [miːçt] \tab möchte \tab ‘would like-\textsc{3}\textsc{sg}’ \tab 15\\
    liχt \tab [liçt] \tab Licht \tab ‘light’ \tab 12\\
    flēχ \tab [fleːç] \tab Floh \tab ‘flea’ \tab 24\\
    dre·χ \tab [dre·ç] \tab trocken \tab ‘dry’ \tab 40\\
    zǟχ \tab [zæːç] \tab Sau \tab ‘sow’ \tab 25
\ex\label{ex:11:22b} n\k{i}x \tab  [nɪx] \tab nicht \tab ‘not’ \tab 35\\
    m\k{i}xəl \tab [mɪxəl] \tab Michel \tab ‘(name)’ \tab 12\\
    tręxlə \tab [trɛxlə] \tab Trichter \tab ‘funnel’ \tab 35
\ex\label{ex:11:22c} jūx \tab [ʝuːx] \tab euer \tab ‘your-\textsc{pl}’ \tab 26\\
    ru·x \tab [ru·x] \tab rauh \tab ‘rough’ \tab 26\\
    rōx \tab [roːx] \tab Ruhe \tab ‘quiet’ \tab 23\\
    dǫxtə(r) \tab [dɔxtə(r)] \tab Tochter \tab ‘daughter’ \tab  12\\
    blɑ̊·x \tab [bla·x] \tab blau \tab ‘blue’ \tab 34\\
    šlɑ̄x \tab [ʃlɑːx] \tab  schlackiges Wetter \tab ‘wet weather’ \tab 8
\z 
\z 

[ʝ] and [ɣ] have the same distribution as their fortis counterparts: [ʝ] occurs after a front [+tense] monophthong in (\ref{ex:11:23a}) and [ɣ] after a back monophthong in (\ref{ex:11:23b}). There are a number of gaps that I consider to be accidental, e.g. there are apparently no short front tense monophthongs before [ʝ] and no short back vowels before [ɣ].

\ea%23
\label{ex:11:23}\relax[ʝ] and [ɣ] (from /ɣ/):
\ea\label{ex:11:23a} lījə \tab [liːʝə] \tab leihen \tab ‘lend-\textsc{inf}’ \tab 17\\
    bējə \tab [beːʝə] \tab biegen \tab ‘bend-\textsc{inf}’ \tab 24\\
    brǟjə \tab [bræːʝə] \tab Gehirn \tab ‘brain’ \tab 19
\ex\label{ex:11:23b} būʒə \tab [buːɣə] \tab bauen \tab ‘build-\textsc{inf}’ \tab 26\\
    kōʒə \tab [koːɣə] \tab kauen \tab ‘chew-\textsc{inf}’ \tab 23\\
    rōʒə \tab [roːɣə] \tab ruhen \tab ‘rest-\textsc{inf}’ \tab 23\\
    mɑ̊ʒə \tab [zaːɣə] \tab Magen \tab ‘stomach’ \tab 16\\
    ɑʒərə \tab [ɑːɣərə] \tab ärgern \tab ‘annoy-\textsc{inf}’ \tab 11
\z 
\z 

[ç] (/ɣ/) occurs after a [+tense] monophthong in (\ref{ex:11:24a}) and [x] (/ɣ/) after a [--tense] monophthong in (\ref{ex:11:24b}) or a back vowel in (\ref{ex:11:24c}). [x ç] in these examples derives historically from a velar (\ili{WGmc} \textsuperscript{+}[ɣ]). As indicated in the first row of \REF{ex:11:24}, I assume that /ɣ/ is the underlying sound for [x ç] in the synchronic phonology, although it is also possible that the original lenis sound (\ili{WGmc} \textsuperscript{+}/ɣ/) restructured to /x/ in those words where there is no longer a lenis alternant. Underlying /ɣ/ remains velar in (\ref{ex:11:24b}, \ref{ex:11:24c}) and shifts to palatal in \REF{ex:11:24a} by the fronting rule I posit below. In both sets of examples, the underlying lenis sound undergoes \isi{Final Fortition} in coda position.\largerpage[2]

\ea%24
\label{ex:11:24}\relax[ç] and [x] (from /ɣ/):
\ea\label{ex:11:24a} fli·χt \tab  [fli·çt] \tab Flügel \tab ‘wing’ \tab 15\\
    ti·χ \tab [ti·ç] \tab Zeug \tab ‘stuff’ \tab 27\\
    twi·ntiχ \tab [twintiç] \tab zwanzig \tab ‘twenty’ \tab 12\\
    šte·χ \tab [ʃte·ç] \tab stieg \tab ‘climb-\textsc{pret}’ \tab 41
\ex\label{ex:11:24b} zęxt \tab [zɛxt] \tab sagt \tab ‘say-\textsc{3sg}’ \tab 9
\ex\label{ex:11:24c} dro·x \tab  [dro·x] \tab trog \tab ‘deceive-\textsc{pret}’ \tab 40\\
    zɑ̊x \tab [zaːx] \tab Säge \tab ‘saw’ \tab 16\\
    dɑx \tab [dɑx] \tab Tag \tab ‘day’ \tab 8
\z 
\z 

Palatals occur after a diphthong whose second member is [+tense] in (\ref{ex:11:25a}) and velars elsewhere in (\ref{ex:11:25b}):

\ea%25
\label{ex:11:25}Palatals (from /x ɣ/) after a diphthong:
\ea\label{ex:11:25a} dɑ̄\k{i}χ \tab [dɑːiç] \tab Teig \tab ‘dough’ \tab 17\\
    tɑ̄\k{i}jəl \tab [tɑːiʝəl] \tab Ziegel \tab ‘clay brick’ \tab 18\\
    šteijə \tab [ʃteiʝə] \tab steigen \tab ‘climb-\textsc{inf}’ \tab 20\\
\ex\label{ex:11:25b} m\k{i}·əx \tab [mɪəx] \tab \ipi{Mücke} \tab ‘mosquito’ \tab 40
    l\k{i}·əʒə \tab [lɪəɣə] \tab liegen \tab ‘lie-\textsc{inf}’ \tab 40\\
    ę·əx \tab [ɛəx] \tab stumpf \tab ‘blunt’ \tab 10\\
    lę·əʒə \tab [lɛəɣə] \tab legen \tab ‘place-\textsc{inf}’ \tab 10\\
    bǫ·əx \tab [bɔəx] \tab Eber \tab ‘boar’ \tab 13\\
    mǫ·əʒə \tab [mɔəɣə] \tab morgen \tab ‘tomorrow’ \tab 13\\
    plɑ̄ųx \tab [plɑːux] \tab Pflug \tab ‘plow’ \tab 22
\z 
\z 

After a coronal sonorant consonant ([r l n]) palatals surface, as in \REF{ex:11:26}. The realization of /x ɣ/ as palatal after [r l n] is not conditioned by the type of vowel preceding that consonant; hence, \isi{Coalescence-1} is not present in the phonology of this dialect.

\ea%26
\label{ex:11:26}Palatals (from (from /x ɣ/) after a coronal consonant:
\ea\label{ex:11:26a} lųrχ \tab [lʊrç] \tab schlechter Kaffee \tab ‘bad coffee’ \tab 29\\
    d\k{i}rχ \tab [dɪrç] \tab durch \tab ‘through’ \tab 29\\
    ɑ̄rjərə \tab [ɑːrʝərə] \tab ärgern \tab ‘annoy-\textsc{inf}’ \tab 29
\ex\label{ex:11:26b} bɑlχ \tab [bɑlç] \tab Kind \tab ‘child’ \tab 48
\ex\label{ex:11:26c} fęnχt \tab [fɛnçt] \tab voriges \tab ‘previous-\textsc{infl}’ \tab  28
\z 
\z 

Postsonorant palatal fricatives in (\ref{ex:11:22a}, \ref{ex:11:23a}, \ref{ex:11:24a}, \ref{ex:11:25a}) derive from the corresponding velars after a front [+tense] vowel by \REF{ex:11:27a} and after a consonant in (\ref{ex:11:26}) by \REF{ex:11:27b}; recall \sectref{sec:3.4}. Since [±tense] is distinctive for vowels but not for consonants the two rules cannot be collapsed into one.

\ea%27
\label{ex:11:27}\begin{multicols}{2}
\ea \isi{Velar Fronting-10}:\\\label{ex:11:27a}
\begin{forest}
[,phantom
    [\avm{[+tense]} [\avm{[coronal]},name=target,tier=word]]
    [\avm{[−son\\+cont]},name=source [\avm{[dorsal]},tier=word]]
]
\draw [dashed] (target.north) -- (source.south);
\end{forest}
\ex \isi{Velar Fronting-3}:\\\label{ex:11:27b}
\begin{forest}
[,phantom
    [\avm{[+cons\\+son]} [\avm{[coronal]},name=target,tier=word]]
    [\avm{[−son\\+cont]},name=source [\avm{[dorsal]},tier=word]]
]
\draw [dashed] (target.north) -- (source.south);
\end{forest}
\z 
\end{multicols}
\z 

\citet{Tita1921} discusses the \il{East Pomeranian}EPo dialect once spoken in the town of \ipi{Kamnitz} (\mapref{map:18}). That author does not consider whether or not the velar nasal [ŋ] has a palatal realization. The dorsal obstruents for \ipi{Kamnitz} are listed in \REF{ex:11:28}. The dialect does not have [g].

\ea%28
\label{ex:11:28}
\ea\label{ex:11:28a}
\begin{forest} 
[,phantom [/ɣ/ [{[ɣ]}]]   [/ʝ/ [{[ʝ]}]]  [/k/ [{[k]}] [{[c]}]]]
\end{forest}        
\ex\label{ex:11:28b}
\begin{forest} for tree = {fit=band}
[,phantom
     [/x/,calign=first [{[x]}] [{[ç]}]]           
     [/ɣ/,calign=first [{[ɣ]}] [{[ʝ]}]]            
     [/k/ [{[k]}] ]
     [/c/ [{[c]}] ]
]
\end{forest}
\z 
\z 

The phonemic front vowels are /i ɪ eː ɛː ɛ/, the phonemic back vowels are /u ʊ oː o ɔː ɔ ǝ ɑː ɑ/, and the phonemic diphthongs are /ɑi ɛi ɑu ɛu/. I demonstrate below that velar fronting is active in postsonorant position and that it requires /ɛ/ -- but not its long counterpart /ɛː/ -- to be analyzed phonologically as [+low]; recall \ipi{Rheintal} (\sectref{sec:3.4}). The distinctive features for the phonemic vowels (excluding placeless \isi{schwa}) are presented in \tabref{tab:fromex:11:29}.

\begin{table}
\caption{\label{tab:fromex:11:29} Distinctive features for vowels (Kamnitz)}
\begin{tabular}{lccccccccccc}
\lsptoprule
                 & i & ɪ & eː & ɛː & ɛ & u & ʊ & oː o & ɔː ɔ & ɑː ɑ\\\midrule
\relax [coronal] & \ding{51} & \ding{51} & \ding{51} & \ding{51} & \ding{51} &  &  &  &  & \\
\relax [dorsal] &  &  &  &  &  & \ding{51} & \ding{51} & \ding{51} & \ding{51} & \ding{51}\\
\relax [low] & {}-- & {}-- & {}-- & {}-- & + & {}-- & {}-- & {}-- & {}-- & +\\
\relax [high] & + & + & {}-- & {}-- &  & + & + & {}-- & {}-- & \\
\relax [tense] & + & {}-- & + & {}-- &  & + & {}-- & + & {}-- & \\
\lspbottomrule
\end{tabular}
\end{table}

Front vowels and back vowels are [coronal] and [dorsal] respectively. Within those two groups, the feature values [+low] and [--low] are assigned, and then within the two [--low] groups, the vowels are marked as [±high] and [±tense].

\begin{sloppypar}
\ipi{Kamnitz} exhibits Contrast Type B in \REF{ex:11:2a} for word-initial [ɣ] and [ʝ]. [ɣ] (<\ili{WGmc} \textsuperscript{+}[ɣ]) occurs before a consonant in (\ref{ex:11:30a}) or any back vowel with the exception of [ɑi] or [ə] in (\ref{ex:11:30b}), but never before a front vowel. The original velar (\textsuperscript{+}[ɣ]) is now realized as a palatal [ʝ] before the back vowels [ɑi] or [ə] in (\ref{ex:11:30c}) or front vowels in (\ref{ex:11:30d}). As in many other dialects, the original velar now participates in [ɣ]{\textasciitilde}[ʝ] alternations in (\ref{ex:11:30e}). Palatal [ʝ] (<\ili{WGmc} \textsuperscript{+}[j]) occurs before any type of vowel in (\ref{ex:11:30f}). Note that [ɣ] and [ʝ] contrast before the same back vowel in examples like [ɣɔːn] ‘yarn’ in (\ref{ex:11:30b}) vs. [ʝɔːɣə] ‘hunt-\textsc{inf}’ in (\ref{ex:11:30f}). \citet[57]{Tita1921} observes that [k] (⟦k⟧) is realized as velar or palatal depending on the context. On the basis of his data it can be concluded that [k] occurs before a consonant in (\ref{ex:11:30g}) or any back vowel in (\ref{ex:11:30h}) and [c] before any front vowel in (\ref{ex:11:30i}).
\end{sloppypar}

\ea%30
\label{ex:11:30}Word-initial dorsal obstruents:
\ea\label{ex:11:30a} γlik \tab [ɣlik] \tab gleich \tab ‘soon’ \tab 49
\ex\label{ex:11:30b} ɣǭn \tab [ɣɔːn] \tab Garn \tab ‘yarn’ \tab 64
\ex\label{ex:11:30c} jɑitə \tab [ʝɑitə] \tab  gieβen \tab ‘water-\textsc{inf}’ \tab 60\\
    jənɑitə \tab [ʝənɑitə] \tab genieβen \tab ‘enjoy-\textsc{inf}’ \tab 52
\ex\label{ex:11:30d} j\k{i}lə \tab [ʝɪlə] \tab gelten \tab ‘be valid-\textsc{inf}’ \tab 60\\
    j\={ę}l \tab [ʝɛːl] \tab gelb \tab ‘yellow’ \tab 60\\
    jęsəl \tab [ʝɛsəl] \tab Gänschen \tab ‘goose-\textsc{dim}’ \tab 60
\ex\label{ex:11:30e} ɣɑst \tab [ɣɑst] \tab  Gast \tab ‘guest’ \tab 59\\
    jęst \tab [ʝɛst] \tab Gäste \tab ‘guest-\textsc{pl}’ \tab 59
\ex\label{ex:11:30f} jųŋk \tab  [ʝʊŋk] \tab jung \tab ‘young’ \tab 64\\
    jǭγə \tab [ʝɔːɣə] \tab jagen \tab ‘hunt-\textsc{inf}’ \tab 64
\ex\label{ex:11:30g} kręuts \tab [krɛuts] \tab Karausche \tab ‘crucian carp’ \tab 57
\ex\label{ex:11:30h} kōl \tab [koːl] \tab Kohl \tab ‘cabbage’ \tab 57
\ex\label{ex:11:30i} kēl \tab [ceːl] \tab Kerl \tab ‘fellow’ \tab 57\\
    kęinə \tab [cɛinə] \tab keimen \tab ‘germinate-\textsc{inf}’ \tab 57
\z 
\z 

In postsonorant position, [x ɣ] and [ç ʝ] never contrast. The generalization is that [x] occurs after a back vowel in (\ref{ex:11:31a}) or [ɛ] in (\ref{ex:11:31b}), while [ç] surfaces after front vowels other than [ɛ] in (\ref{ex:11:31c}) or a coronal sonorant consonant in (\ref{ex:11:31d}). The same generalizations hold for [ɣ ʝ] in (\ref{ex:11:31e}--\ref{ex:11:31g}) and for [x ç] derived historically from \ili{WGmc} \textsuperscript{+}[ɣ] in (\ref{ex:11:31h}--\ref{ex:11:31k}). \isi{Umlaut} alternations of the type [ɔ]{\textasciitilde}[ɛ] provide further support that [x] occurs after [ɛ] (see \ref{ex:11:31l}). Velar [k] and palatal [c] (both from \ili{WGmc} \textsuperscript{+}[k]) never contrast; the former occurs after back vowels in (\ref{ex:11:31m}) and the former after front vowels in (\ref{ex:11:31n}). No examples were found in the original source with [k] or [c] after a coronal sonorant consonant. At least one example (\ref{ex:11:31o}) has palatal [c] (< \ili{WGmc} \textsuperscript{+}[k]) in the context after a historically syncopated front vowel (which is visible in the \il{Standard German}StG orthography).\footnote{\citet{Tita1921} does not provide an example for [c] after [ɛ], although he does give the one item [vɛlc] ‘which’ (=⟦węlk⟧), in which [c] (=⟦k⟧) occurs after the sequence [ɛl]. Two treatments suggest themselves for the fronting of /k/ in [vɛlc]: (a) velar fronting is triggered by /l/ (recall \ref{ex:11:27b}), or (b) the fronting of /k/ is indirectly triggered by the vowel /ɛ/: \isi{Coalescence-1} merges the [coronal] feature for /ɛ/ and /l/, and then velar fronting spreads [coronal] from any front vowel to a velar stop. Since no additional examples are provided I leave this question open.}

\ea%31
\label{ex:11:31}Postsonorant dorsal obstruents:
\ea\label{ex:11:31a} hōx \tab  [hoːx] \tab high \tab ‘high’ \tab 62\\
    tǭx \tab [tɔːx] \tab zähe \tab ‘tough’ \tab 62
\ex\label{ex:11:31b} dęxt \tab [dɛxt] \tab Docht \tab ‘wick’ \tab 62\\
    ręxt \tab [rɛxt] \tab recht \tab ‘right’ \tab 43\\
    fręx \tab [frɛx] \tab frech \tab ‘impudent’ \tab 62
\ex\label{ex:11:31c} l\k{i}χt \tab [lɪçt] \tab leicht \tab ‘light’ \tab 49\\
    tēχt \tab [teːçt] \tab zehnte \tab ‘tenth’ \tab 76
\ex\label{ex:11:31d} d\k{i}rχ \tab [dɪrç] \tab durch \tab ‘through’ \tab 62
\ex\label{ex:11:31e} truɣə \tab [truɣə] \tab trauen \tab ‘trust-\textsc{inf}’ \tab 50\\
    bǭɣə \tab [bɔːɣə] \tab Bogen \tab ‘bow’ \tab 59
\ex\label{ex:11:31f} krijə \tab [kriʝə] \tab kriegen \tab ‘get-\textsc{inf}’ \tab 49\\
    t\={ę}jəl \tab [tɛːʝəl] \tab Zügel \tab ‘rein’ \tab 48
\ex\label{ex:11:31g} bɑljə \tab [bɑlʝə] \tab streiten \tab ‘argue-\textsc{inf}’ \tab 60
\ex\label{ex:11:31h} nɑux \tab [nɑux] \tab genug \tab ‘enough’ \tab 49
\ex\label{ex:11:31i} vęx \tab [vɛx] \tab Weg \tab ‘path’ \tab 43
\ex\label{ex:11:31j} tiχ \tab [tiç] \tab Zeug \tab ‘stuff’ \tab 52
\ex\label{ex:11:31k} tęlχ \tab  [tɛlç] \tab Zweig \tab ‘branch’ \tab 43
\ex\label{ex:11:31l} trǫx \tab [trɔx] \tab Trog \tab ‘trough’ \tab 45\\
    tręx \tab [trɛx] \tab Tröge \tab ‘trough-\textsc{pl}’ \tab 73
\ex\label{ex:11:31m} rǫk \tab [rɔk] \tab Rock \tab ‘skirt’ \tab 57
\ex\label{ex:11:31n} zɑikə \tab [zɑicə] \tab suchen \tab ‘search-\textsc{inf}’ \tab 57
\ex\label{ex:11:31o} morętk \tab [morɛtc] \tab Meerrettich \tab ‘horseradish’ \tab 57
\z 
\z 

The set of targets for the fronting of velars in word-initial position in \REF{ex:11:30} is /ɣ k/, and the set of triggers consists of front vowels but not coronal consonants. Synchronically derived [ʝ] is situated before a front vowel in (\ref{ex:11:30d}) and the second word in (\ref{ex:11:30e}). The palatal allophone [c] in (\ref{ex:11:30i}) derives from /k/. The fronting of word-initial /ɣ k/ is accomplished by \isi{Wd-Initial Velar Fronting-6}. All other instances of [ʝ] are underlying palatals (\ref{ex:11:30c}, \ref{ex:11:30f}).

For postsonorant position, the target segments are /ɣ x k/ and the triggers are (a) the front [--low] vowels or (b) coronal sonorant consonants. Fronting in (a) and (b) is accomplished with \REF{ex:11:32} and \REF{ex:11:27b} respectively.

\ea%32
\label{ex:11:32}\isi{Velar Fronting-2}:\\
\begin{forest}
[,phantom
   [\avm{[−low]} [\avm{[coronal]},tier=word,name=target]]
   [\avm{[−son\\+cont]},name=source [\avm{[dorsal]},tier=word]]
]
\draw [dashed] (target.north) -- (source.south);
\end{forest}
\z 

The distribution of dorsal obstruents in the town of \ipi{Lauenburg} (\citealt{Pirk1928}; \mapref{map:18}) is depicted in \REF{ex:11:33}. Word-initial [g] is an allophone of /ɣ/. I comment on the status of dorsal nasals below.

\ea%33
\label{ex:11:33}
\ea\label{ex:11:33a}
   \begin{forest}
    [,phantom
        [/ɣ/ [{[g]}]]   [/ʝ/ [{[ʝ]}]]  [/k/,calign=first [{[k]}]  [{[c]}]]
    ]        
    \end{forest}
\ex\label{ex:11:33b}\begin{forest} for tree = {fit=band}
    [,phantom
        [/x/,calign=first [{[x]}] [{[ç]}]]            
        [/ɣ/,calign=first [{[ɣ]}] [{[ʝ]}]]             
        [/k/ [{[k]}]]   
        [/c/ [{[c]}]]   
        [/g/ [{[ɟ]}]]
    ]
    \end{forest}
\z 
\z 

As illustrated in (\ref{ex:11:34a}--\ref{ex:11:34f}), \ipi{Lauenburg} exhibits Contrast Type B in \REF{ex:11:2a} for word-initial [g] and [ʝ]. The post-palatal back vowel in \REF{ex:11:34c} was historically front and shifted to a back vowel by either \isi{Vowel Retraction} or \isi{Vowel Reduction}. [g ʝ] in (\ref{ex:11:34a}--\ref{ex:11:34e}) derive from a historical velar (\ili{WGmc} \textsuperscript{+}[ɣ]). Palatal [ʝ] (<\ili{WGmc} \textsuperscript{+}[j]) in \REF{ex:11:34f} stands before any type of vowel.  [c] and [k] never contrast: The latter surfaces before a back vowel in (\ref{ex:11:34g}) or consonant in (\ref{ex:11:34h}) and the former before a front vowel in (\ref{ex:11:34i}). [k]{\textasciitilde}[c] alternations are attested in (\ref{ex:11:34j}). I interpret Pirk’s ⟦ɑ⟧ as a low front vowel ([æ]) because it is one of the umlauted (fronted) realizations of back vowels.

\ea%34
\TabPositions{.15\textwidth, .3\textwidth, .45\textwidth, .7\textwidth}
\label{ex:11:34}Word-initial dorsal obstruents:
\ea\label{ex:11:34a} ɡot \tab [gɔt] \tab Gott \tab ‘God’ \tab 8
\ex\label{ex:11:34b} ɡlɑ̊t \tab [glɑt] \tab glatt \tab ‘smooth’ \tab 22\\
    ɡlękʹ \tab [glɛc] \tab Glück \tab ‘fortune’ \tab 9
\ex\label{ex:11:34c} ɡɑ̊ršt \tab [ʝɑrʃt] \tab Gerste \tab ‘barley’ \tab 8\\
    ɡəšɑ̄inə \tab [ʝəʃɑːinə] \tab geschehen \tab ‘happen-\textsc{inf}’ \tab 10
\ex\label{ex:11:34d} ɡistrə \tab [ʝɪstrə] \tab gestern \tab ‘yesterday’ \tab 7\\
    ɡęlt \tab [ʝɛlt] \tab Geld \tab ‘money’ \tab 8
\ex\label{ex:11:34e} ɡɑ̄us \tab [gɑːus] \tab  Gans \tab ‘goose’ \tab 19\\
    ɡɑnz \tab [ʝænz] \tab Gänse \tab ‘goose-\textsc{pl}’ \tab 19
\ex\label{ex:11:34f} ɡɑ̊\textsuperscript{u}r \tab [ʝɑur] \tab Jahr \tab ‘year’ \tab 10
\ex\label{ex:11:34g} kɑ̄u \tab [kɑːu]  \tab  Kuh \tab ‘cow’ \tab 18
\ex\label{ex:11:34h} krīɡə \tab [kriːʝə] \tab kriegen \tab ‘get-\textsc{inf}’ \tab 10
\ex\label{ex:11:34i} kʹint \tab [cɪnt] \tab  Kind \tab ‘child’ \tab 7\\
    kʹɑstər \tab [cæstər] \tab Küster \tab ‘sexton’ \tab 8
\ex\label{ex:11:34j} kop \tab [kɔp] \tab Kopf \tab ‘head’ \tab 14\\
    kʹɑp \tab [cæp] \tab Köpfe \tab ‘head-\textsc{pl}’ \tab 14
\z 
\z 

Velar fricatives [x ɣ] never contrast with the corresponding palatals in postsonorant position: The velars occur after back vowels in (\ref{ex:11:35a}, \ref{ex:11:35c}) and the palatals [ç ʝ] after front vowels in (\ref{ex:11:35b}, \ref{ex:11:35d}). No examples were found in which dorsal fricatives occur after consonants. The lenis palatal stop [ɟ] is the reflex of an earlier geminate (\textsuperscript{+}[gg]) after a front vowel in (\ref{ex:11:35e}); cf. \ili{OSax} \textit{hruggi} ‘back’. No examples are provided in the original source for modern reflexes of a phonetic [g] (<\ili{WGmc} \textsuperscript{+}[gg] after back vowels). The relationship between [k] and [c] is not the same as the relationship between the other velar and palatal pairs discussed above. Velar [k] occurs after back vowels in (\ref{ex:11:35f}) but never after front vowels, and [c] can be found in many items after a front vowel in (\ref{ex:11:35g}). [k]{\textasciitilde}[c] alternations in (\ref{ex:11:35h}) are also attested. However, palatal [c] also occurs in a context other than after a front vowel in diminutives in (\ref{ex:11:35i}) and at the right edge of nouns and certain verbs in (\ref{ex:11:35j}). [k c] in the examples referred to here derive from \ili{WGmc} \textsuperscript{+}[k].


\ea%35
\TabPositions{.12\textwidth, .30\textwidth, .5\textwidth, .82\textwidth}
\label{ex:11:35}Postsonorant dorsal obstruents:
\ea\label{ex:11:35a} doxtər \tab [dɔxtər] \tab  Tochter \tab ‘daughter’ \tab 8
\ex\label{ex:11:35b} knɑχt \tab [knæçt] \tab  Knecht \tab ‘vassal’ \tab 18
\ex\label{ex:11:35c} zūʓə \tab [zuːɣə] \tab  saugen \tab ‘suck-\textsc{inf}’ \tab 16
\ex\label{ex:11:35d} lɑ̄iɡə \tab [lɑːiʝə]  \tab  lügen \tab ‘lie-\textsc{inf}’ \tab 16
\ex\label{ex:11:35e} riɡʹə \tab [rɪɟə]  \tab   Rücken \tab ‘back’ \tab 9\\
    zɑɡʹə \tab [zæɟə] \tab  sagen \tab ‘say-\textsc{inf}’ \tab 8
\ex\label{ex:11:35f} bɑ̄uk \tab [bɑːuk] \tab  Buch \tab ‘book’ \tab 12
\ex\label{ex:11:35g} ękʹ \tab  [ɛc]  \tab  ich \tab ‘I’ \tab 17
\ex\label{ex:11:35h} bok \tab [bok] \tab  Bock \tab ‘buck’ \tab 14\\
    bɑkʹ \tab [bæc] \tab Böcke \tab ‘buck-\textsc{pl}’ \tab 14
\ex\label{ex:11:35i} ɑ̄ikʹskʹə \tab [ɑːicscə]  \tab  Eiche, dim \tab ‘oak{}-\textsc{dim}’ \tab 18\\
    buŋkskʹə \tab [bʊŋkscə]  \tab  Käfer, dim \tab ‘bug{}-\textsc{dim}’ \tab 18\\
    hęltkʹəs \tab [hɛltcəs] \tab Holzäpfel \tab ‘crab apple-\textsc{pl}’ \tab 14\\
    kręlkʹəs \tab [krɛlcəs] \tab Pellkartoffeln \tab ‘potato-\textsc{pl} in the skin’ \tab 40
\ex\label{ex:11:35j} mɑlkʹ \tab [mælc] \tab Milch \tab ‘milk’ \tab 8\\
    mɑlkʹə \tab [mælcə] \tab melken \tab ‘milk{}-\textsc{inf}’ \tab 8\\
    mulk \tab  [mʊlk] \tab  melkte \tab ‘milk{}-\textsc{pret}’ \tab 30\\
    mulkʹə \tab [mʊlcə] \tab  melkten \tab ‘milk{}-\textsc{pret}.\textsc{pl}’ \tab 30\\
    molkʹə \tab [mɔlcə] \tab  gemolken \tab ‘milk{}-\textsc{part}’ \tab 30
\z 
\z 

The word-initial sound in (\ref{ex:11:34a}, \ref{ex:11:34b}, \ref{ex:11:34d}, \ref{ex:11:34e}) is velar /ɣ/, which surfaces as [ʝ] before a front vowel by \isi{Wd-Initial Velar Fronting-6}. Elsewhere (before a back vowel or consonant) that /ɣ/ is realized as [g] by \isi{g-Formation-2} (\sectref{sec:8.4}). In the context before a back vowel in (\ref{ex:11:34c}, \ref{ex:11:34f}), the word-initial [ʝ] is an underlying palatal (/ʝ/). The word-initial sound in (\ref{ex:11:34g}--\ref{ex:11:34j}) is /k/, which surfaces as [c] before a front vowel by \isi{Wd-Initial Velar Fronting-6} and otherwise as [k].

In postsonorant position velar /x ɣ/ are realized as palatal after a front vowel in (\ref{ex:11:35b}, \ref{ex:11:35d}) by \isi{Velar Fronting-8}. I analyze the stop in \REF{ex:11:35e} as an allophone of /g/, which surfaces as palatal [ɟ] by \isi{Velar Fronting-8}. The same process creates [c] from /k/ in (\ref{ex:11:35f}--\ref{ex:11:35h}). The [c] in (\ref{ex:11:35i}, \ref{ex:11:35j}) is an underlying palatal (/c/).\footnote{{\citet{Pirk1928} does not comment on whether or not [ŋ] has a palatal realization in the neighborhood of front vowels. A few words in his grammar suggest that the sound transcribed as ⟦ŋ⟧ is phonetically [ɲ] after a front vowel because the palatal stop (and not the velar stop) follows that nasal, e.g. [pɪɲcstə]  ‘Pentecost’ (=⟦piŋkʹstə⟧). I tentatively conclude that /ŋ/ is one of the targets for velar fronting. This suggests that the [ɲc] in words like ⟦piŋkʹstə⟧ is underlyingly /ŋk/, which surfaces as [ɲc] by \isi{Velar Fronting-8}.}} \footnote{{It is not clear what the generalization is involving the alternations in (\ref{ex:11:35j}), but the occurrence of [k] and [c] after a back vowel plus [l] suggests that /c/ is a \isi{phonemic palatal} because it contrasts with /k/. One could argue that the occurrence of [k] or [c] after a liquid in word-final position is a consequence of the stem vowel, i.e. [k] if that vowel is back and [c] if it is front.  However, after [r], only [c] surfaces, even if the vowel preceding that [r] is back, e.g [mɑrcə] ‘notice{}-}\textrm{\textsc{inf}}\textrm{’ (=⟦mɑ̊rkʹə⟧). (The same generalization holds for the palatal fricatives [ç ʝ], e.g. [bɑrç] ‘mountain’ (=⟦bɑ̊rχ⟧), [bɑrʝə] ‘mountain-\textsc{pl}’ (=⟦bɑ̊rɡə⟧)). It is conceivable that the set of triggers for postsonorant velar fronting includes all front vowels and /r/, but not /l/. If this were the correct treatment, it would be the only case in the present study in which only /r/ but not /l/ serves as trigger. Alternatively, there may be words not mentioned in \citet{Pirk1928} containing [k] after [r], which would contrast with [c], as in (\ref{ex:11:35j}) for the context after /l/. I leave this question open.}}

Two varieties of \il{East Pomeranian}EPo that are the essentially the same in terms of velar fronting are the ones once spoken in close proximity, namely \ipi{Sępóno Krajeńskie} (\citealt{Darski1973}; \mapref{map:18}) and Kreis \ipi{Konitz} (\citealt{Semrau1915a,Semrau1915b}; \mapref{map:18}). I describe below the latter variety.\footnote{\label{fn:11:15}One difference between the two varieties is the nature of the velar fronting outputs. Recall from \sectref{sec:10.5} that \ipi{Sępóno Krajeńskie} is the only known LG variety with \isi{alveolopalatalization}. The discussion in \citet{Semrau1915a,Semrau1915b} does not provide a clear indication that her variety can also be so classified. The transcriptions in \citet{Darski1973} indicate that the output for velar fronting for a velar stop target is a (\isi{sibilant}) \isi{affricate} (e.g. historical [k] is realized as [tɕ] in the context of front segments).}

The phonemic dorsal consonants for Kreis \ipi{Konitz} are depicted in \REF{ex:11:36}. In that system there are velar and palatal fricatives ([ɣ]/[ʝ] and [x]/[ç]), velar and palatal nasals ([ŋ]/[ɲ]), and the velar stop [k]. There is no palatal stop ([c]) corresponding to [k].

\ea%36
\label{ex:11:36}
\ea\label{ex:11:36a} \begin{forest}
    [,phantom
     [/ɣ/ [{[g]}]]   
     [/ʝ/ [{[ʝ]}]]   
     [/ç/ [{[ç]}]]    
     [/k/  [{[k]}]]
    ]
    \end{forest}
\ex\label{ex:11:36b} \begin{forest}
    [,phantom
       [/x/ [{[x]}]]     
       [/ç/ [{[ç]}]]
       [/ɣ/ [{[ɣ]}]]  
       [/ʝ/ [{[ʝ]}]]   
       [/k/ [{[k]}]]    
       [/ŋ/ [{[ŋ]}]]   
       [/ɲ/ [{[ɲ]}]]
     ]  
 \end{forest}
\z 
\z 

In Semrau’s system [k x g ɣ ŋ] correspond to ⟦k x ɡ ʓ ŋ⟧. For palatals, [ç]=⟦c⟧ or ⟦tc⟧ depending on the etymological source: ⟦c⟧ is historically a fricative (<\textsuperscript{+}[x] or \textsuperscript{+}[ɣ]) and ⟦tc⟧ a historical stop (<\textsuperscript{+}[k]) The dialect-specific sound change from [k] to [ç] is shown to be active synchronically. [ʝ] (=⟦ᵈj⟧) is described as a voiced lenis palatal fricative (“palataler Reibelaut, stimmhafte lenisˮ). As implied by the raised “dˮ, [ʝ]  (=⟦ᵈj⟧) can be realized as an \isi{affricate} in some places within Kreis \ipi{Konitz} (recall \fnref{fn:11:15}). Semrau is clear that her ⟦tc⟧ is a voiceless fortis palatal fricative (“palataler Reibelaut, stimmlose fortisˮ), which is rendered in my transcription as [ç].

Word-initial position exemplifies Contrast Type B in \REF{ex:11:2a} for [k] (/k/) and [ç] (/ç/) as well as [g] (/ɣ/) and [ʝ] (/ʝ/): Velar [k g] surface before a back vowel in (\ref{ex:11:37a}, \ref{ex:11:37b}), but never before a front vowel, while palatal [ç ʝ] occur before front vowels in (\ref{ex:11:37c}, \ref{ex:11:37d}) or back vowels in (\ref{ex:11:37e}, \ref{ex:11:37f}, \ref{ex:11:37p}). In word-initial position before a consonant, [k g] only surface if the vowel following the consonant is back in (\ref{ex:11:37g}, \ref{ex:11:37h}), while [ç ʝ] can surface if the stem vowel is front in (\ref{ex:11:37i}, \ref{ex:11:37j}) or back in (\ref{ex:11:37k}, \ref{ex:11:37l}). The back stem vowel in (\ref{ex:11:37e}, \ref{ex:11:37f}, \ref{ex:11:37k}, \ref{ex:11:37l}) was etymologically front (e.g. [ʝɑːʃt] ‘barley’; cf. \ili{OSax} \textit{gersta}); [çnɑi] ‘knee’; cf. \ili{OSax} \textit{knio}). Regular alternations involving [k]{\textasciitilde}[ç] and [g]{\textasciitilde}[ʝ] are attested in word-initial position (in \ref{ex:11:37m}--\ref{ex:11:37o}). [k ç] in \REF{ex:11:37} derived historically from \ili{WGmc}\textsuperscript{ +}[k], [ʝ] in (\ref{ex:11:37p}) from \ili{WGmc} \textsuperscript{+}[j] and [g ʝ] in the remaining examples from  \ili{WGmc} \textsuperscript{+}[ɣ].

\ea%37
\label{ex:11:37}Dorsal obstruents in word-initial position
\ea\label{ex:11:37a} kǫp \tab [kɔp] \tab Kopf \tab ‘head’ \tab 192
\ex\label{ex:11:37b} ɡɑɑv \tab [gɑːv] \tab Garbe \tab ‘sheaf’ \tab 194
\ex\label{ex:11:37c} tc\k{i}n \tab [çɪn] \tab Kinn \tab ‘chin’ \tab 193
\ex\label{ex:11:37d} ᵈjelt \tab [ʝelt] \tab Geld \tab ‘money’ \tab 194
\ex\label{ex:11:37e} tcǫǫtcən \tab [çɔːçən] \tab  Küche \tab ‘kitchen’ \tab 193
\ex\label{ex:11:37f} ᵈjɑɑšt \tab [ʝɑːʃt] \tab  Gerste \tab ‘barley’ \tab 195
\ex\label{ex:11:37g} knųt \tab [knʊt] \tab  Flachsknoten \tab ‘flax knot’ \tab 185
\ex\label{ex:11:37h} ɡroot \tab [groːt] \tab groβ \tab ‘large’ \tab 195
\ex\label{ex:11:37i} tcleet \tab [çleːt] \tab Kleid \tab ‘dress’ \tab 193
\ex\label{ex:11:37j} ᵈjrüͅt \tab [ʝrʏt] \tab Grütze \tab ‘groat’ \tab 195
\ex\label{ex:11:37k} cnɑi \tab [çnɑi] \tab  Knie \tab ‘knee’ \tab 195
\ex\label{ex:11:37l} ᵈjrooiə \tab [ʝroːiə] \tab grüne \tab ‘green-\textsc{infl}’ \tab 195
\ex\label{ex:11:37m} kǫǫ’f \tab [kɔːf] \tab Korb \tab ‘basket’ \tab 194\\
    tcö̜ö̜’v \tab [çœːv] \tab Körbe \tab ‘basket-\textsc{pl}’ \tab 194
\ex\label{ex:11:37n} krɑɑnts \tab [krɑːnts] \tab Kranz \tab ‘wreath’ \tab 193\\
    tcr\k{i}nnts \tab [çrɪnts] \tab Kränze \tab ‘wreath-\textsc{pl}’ \tab 193
\ex\label{ex:11:37o} ɡɑɑs \tab [gɑːs] \tab Gans \tab ‘goose’ \tab 194\\
    ᵈjęę’z \tab [ʝɛːz] \tab Gänse \tab ‘goose-\textsc{pl}’ \tab 195
\ex\label{ex:11:37p} ᵈjum \tab [ʝum] \tab Junge \tab ‘boy’ \tab 196
\z 
\z 

Contrast Type B is also attested in postsonorant position (=\ref{ex:11:2b}). In that context velars [k x ɣ] surface after a back vowel in (\ref{ex:11:38a}--\ref{ex:11:38c}) but never after a front vowel or coronal sonorant consonant. The palatals [ç ʝ] occur after a front vowel in (\ref{ex:11:38d}, \ref{ex:11:38e}), coronal sonorant consonant in (\ref{ex:11:38f}, \ref{ex:11:38g}) or back vowel in (\ref{ex:11:38h}, \ref{ex:11:38i}). The [ç] in \REF{ex:11:38h} was once preceded by a coronal sonorant consonant (cf. \ili{MHG} \textit{arc} ‘bad’). Example (\ref{ex:11:38i}) illustrates that palatal [ʝ] can occur after a back vowel. Alternations in postsonorant position between [k]{\textasciitilde}[ç] in (\ref{ex:11:38j}) and [ɣ]{\textasciitilde}[ʝ]/[ç] in (\ref{ex:11:38k}) are common. All dorsal stops and fricatives referred to above derive from historical velars (\ili{WGmc} \textsuperscript{+}[ɣ k]).\largerpage[2]

\ea%38
\label{ex:11:38}Dorsal obstruents in postsonorant position:
\ea\label{ex:11:38a} brukǝ \tab [brukǝ] \tab brauchen \tab ‘need-\textsc{inf}’ \tab 193
\ex\label{ex:11:38b} dɑxt \tab [dɑxt] \tab Docht \tab ‘wick’ \tab 196
\ex\label{ex:11:38c} fǭʓəl \tab [fɔːɣəl] \tab Vogel \tab ‘bird’ \tab 194
\ex\label{ex:11:38d} slęct \tab [slɛçt] \tab schlecht \tab ‘bad’ \tab 196\\
    f\={ę}tcə \tab [fɛːçə] \tab Ferkel \tab ‘piglet’ \tab 193
\ex\label{ex:11:38e} r\={ę}ᵈjənə \tab [rɛːʝənə] \tab regnen \tab ‘rain-\textsc{inf}’ \tab 149
\ex\label{ex:11:38f} bɑɑltcə \tab [bɑːlçə] \tab Balken \tab ‘beam’ \tab 194\\
    vųų’ltcə \tab [vʊːlçə] \tab Wolken \tab ‘cloud-\textsc{pl}’ \tab 251
\ex\label{ex:11:38g} müͅ\textsuperscript{} rᵈjəl \tab [mʏrʝəl] \tab Mergel \tab ‘marl’ \tab 195
\ex\label{ex:11:38h} bǫǫtc \tab [bɔːç] \tab Borke \tab ‘bark’ \tab  194\\
    ɑɑc \tab [ɑːç] \tab arg \tab ‘bad’ \tab 196\\
\ex\label{ex:11:38i}  zǫǫᵈj \tab [zɔːʝ] \tab Sau \tab ‘sow’ \tab 195\\
\ex\label{ex:11:38j}    br\={ę}tcə \tab [breːçə] \tab brechen \tab ‘break-\textsc{inf}’ \tab 242\\
    brętcst \tab [brɛçst] \tab brichst \tab ‘break-\textsc{2sg}’ \tab 242\\
    brętct \tab [brɛçt] \tab bricht \tab ‘break-\textsc{3sg}’ \tab 242\\
    brɑk \tab [brɑk] \tab brach \tab ‘break-\textsc{pret}.\textsc{3sg}’ \tab 242\\
    breetcst \tab [breːçst] \tab brachst \tab ‘break-\textsc{pret}.\textsc{2sg}’ \tab 242
\ex\label{ex:11:38k} drǭʓən \tab [drɔːɣən] \tab tragen \tab ‘carry-\textsc{inf}’ \tab 195\\
    drö̜ct \tab [drœçt] \tab trägt \tab ‘carry-\textsc{3sg}’ \tab 195\\
    fǭʓəl \tab [fɔːɣəl] \tab Vogel \tab ‘bird’ \tab 194\\
    fö̜ö̜ᵈjəls \tab [fœːʝəls] \tab Vögel \tab ‘bird-\textsc{pl}’ \tab 252
\z 
\z 

The (\ref{ex:11:2b}) contrast also holds for nasals: [ŋ] surfaces only after a back vowel in (\ref{ex:11:39a}), but never after a front vowel or consonant, and [ɲ] after a front vowel in (\ref{ex:11:39b}) or a back vowel in (\ref{ex:11:39c}). Note the near minimal pair [zuŋə] ‘sing-\textsc{part}’ in \REF{ex:11:39a} vs. [fuɲə] ‘find-\textsc{part}’ in \REF{ex:11:39c}. [ŋ]{\textasciitilde}[ɲ] alternations as in \REF{ex:11:39d} are common.

\ea%39
\label{ex:11:39}Dorsal nasals in postsonorant position:
\ea\label{ex:11:39a} slɑŋ \tab [slɑŋ] \tab Schlange \tab ‘snake’ \tab 201\\
    zuŋə \tab [zuŋə] \tab gesungen \tab ‘sing-\textsc{part}’ \tab 171
\ex\label{ex:11:39b} fi\={ŋ}ə \tab [fɪɲə] \tab Finger \tab ‘finger’ \tab 201\\
    i\={ŋ} \tab [ɪɲ] \tab Ende \tab ‘end’ \tab 202
\ex\label{ex:11:39c} hu\={ŋ}ət \tab [huɲət] \tab hundert \tab ‘hundred’ \tab 202\\
    fu\={ŋ}ə \tab [fuɲə] \tab gefunden \tab ‘find-\textsc{part}’ \tab 171
\ex\label{ex:11:39d} tvi\={ŋ}ə \tab [tviɲə] \tab zwingen \tab ‘force-\textsc{inf}’ \tab 241\\
    tvuŋk \tab [tvuŋk] \tab zwang \tab ‘force-\textsc{pret}’ \tab 241
\z 
\z 

Palatal [ɲ] in \REF{ex:11:39b} derived historically from [ŋ] by velar fronting. That velar could be either an original [ŋ] (e.g. [fɪɲə] < [fɪŋə]) or a new [ŋ] created by \isi{Velarization} (=\ref{ex:11:1b}), e.g. [ɪɲ] < [ɪŋ] < [ɪnd]). It is not clear from the original source what triggered the change from [ŋ] to [ɲ] in \REF{ex:11:39c}.

The word-initial palatal before a back vowel in (\ref{ex:11:37e}, \ref{ex:11:37f}, \ref{ex:11:37p}) or before a liquid followed by a back vowel (\ref{ex:11:37k}, \ref{ex:11:37l}) is an underlying palatal (/ʝ/ or /ç/). Word-initial [g]{\textasciitilde}[ʝ] alternations in (\ref{ex:11:37o}) are accounted for with an underlying velar (/ɣ/) that surfaces as [ʝ] before a front vowel by \isi{Wd-Initial Velar Fronting-6} and as [g] in the elsewhere case by \isi{g-Formation-2} (\sectref{sec:8.4}); recall \ipi{Lauenburg} in \REF{ex:11:33a}. Nonalternating [ʝ] before a front vowel in (\ref{ex:11:37d}) is likewise analyzed as /ɣ/. In (\ref{ex:11:37j}) the [coronal] feature of the front vowel and of the preceding sonorant consonant merge to a single instantiation of [coronal] by \isi{Coalescence-2}. The latter process \isi{feeds} \isi{Wd-Initial Velar Fronting-6}, thereby creating [ʝ]. Word-initial velars before back vowels in (\ref{ex:11:37a}, \ref{ex:11:37b}) or before back vowels separated by a consonant in (\ref{ex:11:37g}, \ref{ex:11:37h}) are underlying velars (/k ɣ/). As described above, /ɣ/ is realized as [g] by \isi{g-Formation-2}.

Kreis \ipi{Konitz} is the only dialect uncovered in the present survey with regular [k]{\textasciitilde}[ç] alternations. The sound underlying [k]{\textasciitilde}[ç] alternations for word-initial position (in \ref{ex:11:37m}, \ref{ex:11:37n}) is /k/, which undergoes fronting to the corresponding palatal ({\textbar}c{\textbar}) by \isi{Wd-Initial Velar Fronting-6}. That palatal stop surfaces as [ç] by \REF{ex:11:40}. The change from stop to fricative is stated without a context because any \isi{derived palatal} stop ({\textbar}c{\textbar}) is realized as the corresponding fricative, regardless of whether or not it is word-initial or postsonorant (see below). Given the distribution of [k] and [ç], I analyze [ç] in nonalternating examples like \REF{ex:11:37c} as /k/ as well. Example (\ref{ex:11:37i}) is accounted for formally as (\ref{ex:11:37j}) described in the preceding paragraph.

\ea%40
\label{ex:11:40}\isi{c-Spirantization}:\smallskip\\
\avm{[−son\\−cont\\coronal\\dorsal] → [+cont]}
\z 

After a back vowel (=\ref{ex:11:38h}, \ref{ex:11:38i}) [ç ʝ] are underlying palatals (/ç ʝ/). All other postsonorant dorsal obstruents in \REF{ex:11:38} are underlying velars (/k x ɣ/), which shift to the corresponding palatals after a coronal sonorant in (\ref{ex:11:38d}--\ref{ex:11:38g}, \ref{ex:11:38j}, \ref{ex:11:38k}) by \isi{Velar Fronting-9}. The \isi{derived palatal} ({\textbar}c{\textbar}) from /k/ surfaces as [ç] by \isi{c-Spirantization}. The nasal (/ŋ/) in (\ref{ex:11:39a}, \ref{ex:11:39b}, \ref{ex:11:39d}) bears the [dorsal] feature and surfaces as palatal after a front vowel in (\ref{ex:11:39b}) by \isi{Velar Fronting-9}. In the context after back vowels in (\ref{ex:11:39c}), [ɲ] is an underlying palatal (/ɲ/).\footnote{The patterning of velars and palatals in Kreis Konitz is essentially the same in the variety of \ili{Zipser German} in \ipi{Hobgarten} (modern-day Slovakia), as described by \citet{Gréb1921}. Zipser German was a German-language island which developed from the CG dialect spoken by the people who originally settled that region at the onset of the thirteenth century. The area is a \isi{velar fronting island} because it is completely surrounded by a language with [x] (/x/) with no [ç] realization (\ili{Slovak}); see \citet{HanulíkováandHamann2010}. It is interesting to observe that descriptions of other varieties of Zipser German have a more restricted set of velar fronting targets and triggers than in Hobgarten. For example,  postsonorant velar fronting only affects /x/ after a coronal sonorant in the town of \ipi{Leibitz}, as described by \citet{Lumtzer1894, Lumtzer1896}, e.g. ⟦nęχ⟧ ‘not’, ⟦knæ̂χt⟧ ‘vassal’, ⟦hɑrχn̥⟧ ‘hark-\textsc{inf}’ vs. ⟦ɡəbrǫxt⟧ ‘bring-\textsc{part}’. By contrast, the phonetically transcribed entries in the dictionary for \ipi{Dobschau} (WbMD) and the phonetically transcribed texts in  \citet{Kövi1911} for \ipi{Käsmark} point to a narrower set of velar fronting triggers, namely the front vowels but not the coronal sonorant consonants, e.g. ⟦iχ⟧ ‘I’ vs. ⟦mɑnx⟧ ‘many’, ⟦pox⟧ ‘stream’. These facts are expressed formally with \isi{Velar Fronting-13}, which is discussed in \sectref{sec:13.3.4} and \sectref{sec:15.2}.}

Although velar noncontinuants typically pattern together with the velar fricatives as targets for velar fronting in \il{East Pomeranian}EPo, other \il{East Pomeranian}EPo varieties have a narrower set of targets. One dialect in which velar noncontinuants fail to serve as triggers for velar fronting was mentioned above, namely Kreis \ipi{Rummelsburg} (=\ref{ex:11:20}), which contrasts with the broad set of targets in the neighboring variety once spoken in Kreis \ipi{Bütow} in (\ref{ex:11:17}). A second example not mentioned earlier is Kreis \ipi{Schlawe} (\citealt{Mahnke1931}; \mapref{map:18}). As in \ipi{Lauenburg} (=\ref{ex:11:33}), Kreis \ipi{Schlawe} has a version of velar fronting that shifts /ɣ/ to [ʝ] in word-initial position before a front vowel; before a back vowel or consonant, /ɣ/ surfaces as [g] by \isi{g-Formation-2}, e.g. [ʝɛlt] ‘money’ (=⟦\^{g}elt⟧) vs. [gɑːǝ] ‘go-\textsc{inf}’ (=⟦gɑ̊ǝ⟧). In postsonorant position, the two velars that undergo fronting to palatal are /x ɣ/, e.g. [lɑxǝ] ‘laugh-\textsc{inf}’ (=⟦lɑxǝ⟧) vs. [ʃlɛçt] ‘bad’ (=⟦šleχt⟧); [oːɣ] ‘eye’ (=⟦ōʒ⟧) vs. [ʃwiːʝǝ] ‘be silent-\textsc{inf}’ (=⟦swī\^{g}ǝ⟧). \citet[35]{Mahnke1931} makes no reference to a palatal realization of [k], noting that \ili{MLG} [k] is preserved in all positions as [k] (=⟦k⟧). No mention is made of a palatal realization of [ŋ]. From the formal perspective, the set of targets for velar fronting in the Kreis \ipi{Schlawe} variety is restricted to velar fricatives (=\isi{Wd-Initial Velar Fronting-3} in \sectref{sec:4.3} and \isi{Velar Fronting-1}).\il{East Pomeranian|)}

\section{{Low} {Prussian}}\label{sec:11.6}

In several varieties of \il{Low Prussian}LPr it is clear from the original sources that the targets for velar fronting (both word-initial and postsonorant) consist of velar fricatives and velar noncontinuants. In some sources for \il{Low Prussian}LPr the palatal realization of sounds like /k/ and /ŋ/ is simply commented on but not expressed with separate phonetic symbols, but other sources provide distinct symbols for velars and palatals and therefore enable one to draw conclusions concerning the triggers and targets for velar fronting. I consider data from one \il{Low Prussian}LPr variety and then conclude by discussing briefly some of the other sources for this dialect area.

\citet{Natau1937} describes the \il{Low Prussian}LPr dialects once spoken in the northeastern part of East Prussia, concentrating on the small village of \ipi{Willuhnen} (\mapref{map:18}). That dialect has the phonemic dorsal sounds and their realizations depicted in \REF{ex:11:41}. Among those sounds are the two palatal noncontinuants [c ɲ]. Natau transcribes those sounds with the same phonetic symbol for the corresponding velars (⟦k⟧ and ⟦ŋ⟧), but he gives very clear statements regarding the velar vs. palatal distribution (p. 31--32). The stop [g] (=⟦g⟧) is present as the reflex of \ili{WGmc} \textsuperscript{+}[gg], e.g. ⟦rigə⟧ ‘back’‚ but Natau does not discuss whether or not his ⟦g⟧ is velar or palatal after a front vowel.

\ea%41
\label{ex:11:41}
\ea\label{ex:11:41a} \begin{forest} for tree = {fit=band}
  [,phantom
    [/ɣ/ [{[ɣ]}]] [/ʝ/ [{[ʝ]}]] [/k/,calign=first [{[k]}] [{[c]}]]
  ]
  \end{forest}
\ex\label{ex:11:41b} \begin{forest} for tree = {fit=band,calign=first}
  [,phantom
    [/ɣ/ [{[ɣ]}]   [{[ʝ]}]]  [/x/ [{[x]}]   [{[ç]}]]   [/k/ [{[k]}]    [{[c]}]] [/ŋ/ [{[ŋ]}]    [{[ɲ]}]]
  ]
  \end{forest}
\z 
\z 

Word-initial position (=\ref{ex:11:42a}--\ref{ex:11:42g}) illustrates Contrast Type B in \REF{ex:11:2a} for [ɣ] and [ʝ]. The initial palatal in \REF{ex:11:42f} was historically velar (\ili{WGmc} \textsuperscript{+}[ɣ]) and the one in \REF{ex:11:42g} was the \isi{etymological palatal}. Velar and palatal stops [k]/[c] stand in an allophonic relationship. In word-initial position [k] occurs before any back vowel in (\ref{ex:11:42h}) or consonant in (\ref{ex:11:42i}) and [c] before any front vowel in (\ref{ex:11:42j}); \citet[31]{Natau1937}. Examples (\ref{ex:11:42b}, \ref{ex:11:42i}) illustrate that \isi{Coalescence-2} is not active in this dialect.

\ea%42
\label{ex:11:42}Word-initial dorsal obstruents:
\ea\label{ex:11:42a}  γuldə      \tab [ɣʊldə]    \tab Gulden  \tab ‘guilder’ \tab 15
\ex\label{ex:11:42b}  γrīs       \tab [ɣriːs]    \tab grau    \tab ‘gray’    \tab 35
\ex\label{ex:11:42c}  jēərn      \tab [ʝeːərn]   \tab gern    \tab ‘gladly’  \tab 56\\
     j\={æ}əršt \tab [ʝæːərʃt]  \tab Gerste  \tab ‘barley’  \tab 34
\ex\label{ex:11:42d}  γast   \tab [ɣɑst]         \tab Gast    \tab ‘guest’   \tab 57\\
     jæst   \tab [ʝæst]         \tab Gäste   \tab ‘guest-\textsc{pl}’  \tab 57
\ex\label{ex:11:42e}  γrōət  \tab [ɣroːət]       \tab groß    \tab ‘large’   \tab 56\\
     γretər \tab [ɣrɛtər]       \tab größer  \tab ‘larger’  \tab 56
\ex\label{ex:11:42f}  jəkoft \tab [jəkɔft]       \tab gekauft \tab ‘buy-\textsc{part}’ \tab 21
\ex\label{ex:11:42g}  juŋ    \tab [ʝʊŋ]          \tab Junge   \tab ‘boy’    \tab 55
\ex\label{ex:11:42h}  korf   \tab [kɔrf]         \tab Korb    \tab ‘basket’ \tab 31
\ex\label{ex:11:42i}  kreb   \tab [krɛb]         \tab Krippe  \tab ‘crib’   \tab 31
\ex\label{ex:11:42j}  ken    \tab [cɛn]          \tab Kinn    \tab ‘chin’   \tab 31\\
     kæp    \tab [cæp]          \tab Köpfe   \tab ‘head-\textsc{pl}’  \tab 31
\z 
\z 

In postsonorant position [x ɣ] only occur after a back vowel in (\ref{ex:11:43a}, \ref{ex:11:43d}) and [ç ʝ] after a front vowel in (\ref{ex:11:43b}, \ref{ex:11:43e}) or liquid in (\ref{ex:11:43c}, \ref{ex:11:43f}). [k] and [c] have a parallel distribution in (\ref{ex:11:43g}--\ref{ex:11:43i}). As in \ipi{West Mecklenburg} (\sectref{sec:11.3}) and \ipi{Sebnitz} (\sectref{sec:11.4}), [ŋk] occurs after a back vowel and [ɲc] after a front vowel.\footnote{{\citet[26]{Natau1937} transcribes the diminutive suffix as ⟦kə⟧, but he does not say whether or not the ⟦k⟧ is phonetically [k] or [c], e.g. ⟦kīəlkə⟧ ‘wedge-}\textrm{\textsc{dim}}\textrm{’. In certain \il{East Pomeranian}EPo varieties, the consonant in that suffix is realized consistently as [c], regardless of the nature of the preceding sound (e.g. \ipi{Lauenburg}; recall \ref{ex:11:35i}).} } From the historical perspective, [x ç] derive from \ili{WGmc} \textsuperscript{+}[x], [ɣ ʝ] from \ili{WGmc} \textsuperscript{+}[ɣ], [k c] from \ili{WGmc} \textsuperscript{+}[k], and [ŋ ɲ] from \ili{WGmc} \textsuperscript{+}[ŋ].

\ea%43
\label{ex:11:43}Postsonorant dorsal consonants:
\ea\label{ex:11:43a}  brux    \tab  [brʊx]    \tab Bruch   \tab ‘fracture’               \tab 32
\ex\label{ex:11:43b}  hæχt    \tab  [hæçt]    \tab Hecht   \tab ‘pike’                   \tab 32
\ex\label{ex:11:43c}  štorχ   \tab  [ʃtɔrç]   \tab Storch  \tab ‘stork’                  \tab 32
\ex\label{ex:11:43d}  frǭəɣə  \tab  [frɔːəɣə] \tab fragen  \tab ‘ask-\textsc{inf}’       \tab 36
\ex\label{ex:11:43e}  ne·ijə  \tab  [neiʝə]   \tab neigen  \tab ‘incline-\textsc{inf}’   \tab 21
\ex\label{ex:11:43f}  zorjd   \tab  [zɔrʝd]   \tab sorgte  \tab ‘care for-\textsc{pret}’ \tab 36
\ex\label{ex:11:43g}  dak     \tab  [dɑk]     \tab Dach    \tab ‘roof’               \tab 31
\ex\label{ex:11:43h}  ek      \tab  [ɛc]      \tab ich     \tab ‘I’                  \tab 32
\ex\label{ex:11:43i}  molkə   \tab  [mɔlcə]   \tab Molke   \tab ‘whey’               \tab 32
\ex\label{ex:11:43j}  baŋk    \tab  [bɑŋk]    \tab Bank    \tab ‘bank’               \tab 32
\ex\label{ex:11:43k}  driŋkə  \tab  [drɪɲcə]  \tab trinke  \tab ‘drink-\textsc{1sg}’ \tab 32
\z 
\z 

From the formal point of view, the initial sound in (\ref{ex:11:42a}--\ref{ex:11:42e}) is /ɣ/, and in (\ref{ex:11:42h}--\ref{ex:11:42j}) it is /k/. Those velars surface as palatal in word-initial position before a front vowel by \isi{Wd-Initial Velar Fronting-6}. In the context before a back vowel the initial sound in (\ref{ex:11:42f}, \ref{ex:11:42g}) is an underlying palatal (/ʝ/). In postsonorant position (=\ref{ex:11:43}), /x ɣ k/ shift to the corresponding palatals after a coronal sonorant by \isi{Velar Fronting-9}. 

In Willuhnen, velar noncontinuants and velar fricatives serve as targets for velar fronting. However, other LPr varieties once spoken in the same region (East Prussia) have a narrower set of targets. Two very similar varieties are the ones described by \citet{Bink1953} in and around the village of \ipi{Mandtkeim} and \citet{Mitzka1919} for \ipi{Königsberg} (\mapref{map:18}). It is clear from both sources that the palatal fricatives [ç ʝ] are allophones of the corresponding velars in postsonorant position, e.g. \ipi{Königsberg} [vʊxt] ‘impact’ (=⟦wuxt⟧) vs. [kɪç] ‘kitchen’ (=⟦kĭχ⟧); [tuːɣənt] ‘virtue’ (=⟦tûγǝnt⟧) vs. [kriːʝə] ‘get-\textsc{inf}’ (=⟦krîjǝ⟧). However, neither Mitkza nor Bink give any indication that there are palatal stops or a palatal nasal. From the formal perspective the set of target segments for velar fronting consists solely of velar fricatives (/ɣ x/). This requirement suggests that the correct rule is \isi{Velar Fronting-1}.\footnote{The narrow set of velar fronting targets is also attested in the LG dialect once spoken to the northeast of Königsberg (\mapref{map:18}), in modern-day Estonia and Latvia. That variety is known as \ili{Baltic German} (\citealt{Sallmann1872, Mitzka1923b, Mitzka1923, Masing1926, Deeters1939}). The data presented in those sources reveal that the Baltic German region was a \isi{velar fronting island} because the area was completely surrounded by languages without velar fronting (\ili{Latvian} and \ili{Estonian}). In Baltic German, velar fronting applied word-initially and in postsonorant position for dorsal fricative targets (/x ɣ/). In postsonorant position the triggers are coronal sonorants. Representative examples from \citet{Mitzka1923b} are ⟦lījən⟧ ‘lie-\textsc{inf}’, ⟦niχ⟧ ‘not’,  ⟦berjə⟧ ‘mountain-\textsc{pl}’, ⟦foljən⟧ ‘follow-\textsc{inf}’, ⟦tūγənt⟧ ‘virtue', and ⟦jɑ̄xt⟧ ‘hunt’. That  pattern is captured formally with \isi{Velar Fronting-1}. In word-initial position, /x/ does not occur, but /ɣ/ surfaces as the corresponding palatal before a front vowel and as [g] before a back vowel or consonant, e.g. ⟦jên⟧ ‘go-\textsc{inf}’ vs. ⟦ɡolt⟧ ‘Gold’, ⟦ɡrɑ̄pən⟧ ‘iron pot', ⟦ɡrikən⟧ ‘buckwheat’. Recall the Eph pattern represented by \ipi{Dingelstedt am Huy} from \sectref{sec:8.4}, which was expressed with \isi{Wd-Initial Velar Fronting-3} and \isi{g-Formation-2}.}

Several sources for EPr indicate that the alternation between [ɣ] and [ʝ] in word-initial position is different than in Willuhnen (recall \ref{ex:11:42}). For example, in his discussion of the various subdivisions of the EPr dialect area, \citet{Ziesemer1924} observes that in the area to the north of \ipi{Königsberg}, word-initial [ʝ] (< \ili{WGmc} \textsuperscript{+}[ɣ]) occurs before a front vowel or before a liquid followed by a front vowel (e.g. ⟦jenz⟧ ‘goose-\textsc{pl}’, ⟦jrēwə⟧ ‘greave-\textsc{pl}’), while [ɣ] surfaces before a back vowel or before a liquid followed by a back vowel (e.g. ⟦γǭnə⟧ ‘go-\textsc{inf}’, ⟦γlɑs⟧ ‘glass’). \citet{Ziesemer1924} does not say whether or not word-initial [k] has a palatal realization. The same pattern involving  word-initial [ɣ] and [ʝ] (< \ili{WGmc} \textsuperscript{+}[ɣ]) is reflected in the material presented in \citet{Wagner1912} for \ipi{Alt-Thorn}, \citet{Mitzka1919} for \ipi{Königsberg}, and \citet{Tessmann1966} for \ipi{Bieberstein} bei Barten. General descriptions of EPr either give clear statements expressing the distribution of word-initial [ɣ] and [ʝ] (< \ili{WGmc} \textsuperscript{+}[ɣ]), or they present data with distinct symbols so that the generalizations concerning their patterning can be deduced, e.g. \citet{Förstemann1850}, \citet{Fischer1896}, \citet{Kantel1900}, and \citet{Betcke1924}. Note that consonants (liquids) in onset position are transparent to velar fronting, as in West Meckenburg (\sectref{sec:11.3}), Sebnitz (\sectref{sec:11.4}), and Kreis Konitz (\sectref{sec:11.5}). See also HPr (\sectref{sec:11.7}) below. From the formal perspective, the transparency of liquids in word-initial onsets requires a version of velar fronting (\isi{Wd-Initial Velar Fronting-3}) and \isi{Coalescence-2}, as in the aforementioned case studies.

The patterning of velars and palatals in word-initial and postsonorant position throughout the EPr dialect area is summarized in \citet[84--87]{Schönfeldt1977}. That source confirms that there are two types of dialect defined according to how word-initial [ɣ] and [ʝ] (<\ili{WGmc} \textsuperscript{+}[ɣ]) are distributed. First, there are places like \ipi{Willuhnen} -- representing the eastern region in general --, where [ɣ] surfaces before back vowels or liquids, even if the vowel following the liquid is front, while [ʝ] (<\ili{WGmc} \textsuperscript{+}[ɣ]) occurs only before a front vowel (henceforth “Pattern P"). Second, there are the remaining regions, which obey liquid transparency, as described in the preceding paragraph (henceforth “Pattern Q"). Pattern Q is expressed in the statements \citet[84]{Schönfeldt1977} gives for the distribution of fortis velar and palatal stops in word-initial and postsonorant position. According Schönfeldt's survey of the dialects once spoken in East Prussia and West Prussia, Pattern Q is the norm, and Pattern P is geographically restricted to a few places in the east, e.g. Willuhnen.

The two types of system described here have been known among dialectologists since the early 1840s. \citet[30--32]{Lehmann1842} observes that for his speakers of LPr word-initial /g/ is realized as palatal ([ʝ]) before front vowels or /r l n/, but only if a front vowel follows those consonants (Pattern Q). Lehmann writes: “Das \textit{g} wird vor \textit{a, o} und \textit{u} sowie vor Konsonanten richtig augesprochen, nähert sich dagegen vor \textit{e, i, ä, ö} und \textit{ü}, ferner vor \textit{l, n} und \textit{r}, wenn auf diese ein \textit{e, i} oder Umlaut folgt, ganz dem \textit{j}”.  For Lehmann’s speakers there is no fronting of the other word-initial velar ([k]). Lehmann's system can be contrasted with the LPr variety described by \citet[29--30]{Gortzitza1841}, who is clear that for his speakers, all word-initial velars (/g k x/) are fronted to palatals before a front vowel but not before a back vowel or /r l n/ (Pattern P). Gortzitza lists several dozen words (in StG orthography) which illustrate the distribution of palatal (“Gaumlaut”) and velar (“Kehllaut”) sounds. For example, the palatal fricative ([ʝ]) occurs in words like \textit{Geld} ‘money’ and \textit{Gift} ‘poison’, while the corresponding velar ([g]) is realized in words like \textit{Gans} ‘goose’, \textit{Gott} ‘God’, \textit{Glück} ‘fortune’, and \textit{Glas} ‘glass’. The same pattern obtains for word-initial [k], with the palatal segment (“Gaumlaut") attested in words like \textit{Kinn} ‘chin’ and \textit{Kerze} ‘candle’ and the velar (“Kehllaut") in items like \textit{Kalk} ‘lime’, \textit{Kuh} ‘cow’, \textit{Klang} ‘sound’, and \textit{Krieg} ‘war’ (\citealt{Gortzitza1841}: 24-25). The same distribution of velars and palatals applies to [x] and [ç] in loanwords, e.g. palatal (“Gaumlaut”) in \textit{Chemie} ‘chemistry’ and \textit{Cherub} ‘cherub’, but velar (“Kehllaut”) in \textit{Chaos} ‘chaos’ and \textit{Christ} ‘Christian’ (\citealt{Gortzitza1841}: 29-30). 

From the formal point of view, the data in \citet{Gortzitza1841} representing Pattern P point to \isi{Wd-Initial Velar Fronting-6}. By contrast liquid transparency characterized by Pattern Q as described by \citet{Lehmann1842} requires \isi{Wd-Initial Velar Fronting-6} and \isi{Coalescence-2}, as in \ipi{West Mecklenburg}.\footnote{\citet{Lehmann1842} does not discuss the distribution  of velars and palatals in postsonorant position. \citet{Gortzitza1841} provides many concrete examples of words illustrating that velars ([k g x]) occur after a back vowel and the corresponding palatals after a front vowel or liquid even if a back vowel precedes the liquid. (Gortzitza does not discuss the status of the velar nasal; hence it is not possible to know if his speakers fronted that sound to palatal after coronal sonorants). The data in that source therefore point to \isi{Velar Fronting-6}.}

Another variety of LPr with liquid transparency for word-initial position (Pattern Q) is \ili{Plautdietsch} (\sectref{sec:9.5.2}). For example, in his treatment of \ipi{Chortitza}, \citet{Quiring1928} has ⟦jelt⟧ ‘money' and ⟦jlek⟧ ‘fortune’ vs. ⟦γɑust⟧ ‘guest' and ⟦γrunt⟧ ‘reason'. It is clear from Quiring's description of [k] on p. 68 that there is a palatal allophone, although he does not indicate this in his phonetic transcriptions. That palatal stop occurs word-initially before a front vowel or before a consonant followed by a front vowel; elsewhere the velar stop surfaces. The facts of Chortitza therefore suggest that Wd-Initial Velar Fronting-6 and Coalescence-2 are active in the synchronic phonology, as in \ipi{West Mecklenburg}.

In postsonorant position, liquids are not transparent in Chortitza (recall \ref{ex:11:45} for \ipi{Willuhnen}). Thus, [x] surfaces after back vowels and [ç] after coronal sonorants (front vowels or liquids) regardless of the nature of the vowel preceding the liquids (recall the discussion on the phonemic contrast between /x/ and /ç/ in Plautdietsch in \sectref{sec:9.5.2}). The examples cited earlier with [ç] after coronal sonorants are ⟦liχt⟧ ‘light', ⟦treχtɑ⟧ ‘funnel’, ⟦horχst⟧ ‘hark-\textsc{2sg}' and with [x] after back vowels is the example ⟦doxt⟧ ‘think-\textsc{pret}. The same pattern holds for the lenis counterparts [ɣ] and [ʝ], e.g. ⟦krîjən⟧ ‘get-\textsc{inf}’ and ⟦borjən⟧ ‘borrow-\textsc{inf}’ vs. ⟦hȫoγəl⟧ ‘hail'. The velar stops [k] and [g] surface as velar after back vowels and as palatals after front vowels or liquids, regardless of the nature of the vowel preceding the liquid. (Palatal stops are not indicated in Quiring's phonetic transcriptions). The velar nasal surfaces after back vowels, but it is realized as palatal after front vowels; see \citet[75]{Quiring1928}. From the formal perspective, Chortitza requires \isi{Velar Fronting-9} in (\ref{ex:11:16}).

\section{{High} {Prussian}}\label{sec:11.7}\il{High Prussian|(}

I discuss first a specific variety in detail. At the end of this section I turn to additional sources for HPr.

The dorsal consonants of the \il{High Prussian}HPr variety once spoken in \ipi{Reimerswalde} (\citealt{KuckWiesinger1965}; \mapref{map:18}) have the distribution depicted in \REF{ex:11:44}:

\ea%44
\label{ex:11:44}
\ea\label{ex:11:44a}\begin{forest} for tree = {fit=band} 
   [,phantom
    [/ʝ/  [{[ʝ]}]]   [/k/ [{[k]}]]   [/c/ [{[c]}]]  [/g/ [{[g]}]]  [/ɟ/ [{[ɟ]}]]   
   ] 
   \end{forest}     
\ex\label{ex:11:44b}\begin{forest} for tree = {fit=band}
    [,phantom
      [/x/,calign=first [{[x]}]   [{[ç]}]]          [/ʝ/ [{[ʝ]}]]    [/k/ [{[k]}]]     [/c/ [{[c]}]]     [/g/ [{[g]}]]     [/ɟ/ [{[ɟ]}]]   [/ŋ/,calign=first [{[ŋ]}] [{[ɲ]}]]
    ]
    \end{forest}
\z 
\z 

The word-initial examples in (\ref{ex:11:45a}--\ref{ex:11:45m}) exhibit Contrast Type B in \REF{ex:11:2a} for [k g] (<\ili{WGmc} \textsuperscript{+}[k ɣ]) and the corresponding stops [c ɟ]. [ʝ] occurs before any type of vowel. The palatal in (\ref{ex:11:45n}) derives from \ili{WGmc} \textsuperscript{+}[j] and the one in (\ref{ex:11:45o}) from \ili{WGmc} \textsuperscript{+}[ɣ].

\ea%45
\label{ex:11:45}Word-initial dorsal obstruents:              
\ea\label{ex:11:45a}  kū        \tab [kuː]       \tab Kuh       \tab ‘cow’   \tab  130
\ex\label{ex:11:45b}  ɡǭrə      \tab [gɔrə]      \tab Garn      \tab ‘yarn’  \tab  144
\ex\label{ex:11:45c}  \'{k}en   \tab [cen]       \tab Kinn      \tab ‘chin’  \tab  137
\ex\label{ex:11:45d}  ɡęlt      \tab [ɟɛlt]      \tab Geld      \tab ‘money’ \tab  143
\ex\label{ex:11:45e}  \'{k}ɑ̄vɒ  \tab [cɑːvɐ]     \tab Käfer     \tab ‘bug’   \tab 137\\
     \'{k}ɑen  \tab [cɑen]      \tab Keim      \tab ‘germ’  \tab 124
\ex\label{ex:11:45f}  ɡɑ̄ršt     \tab [ɟɑːrʃt]    \tab Gerste    \tab ‘barley’    \tab 144\\
     ɡɑest     \tab [ɟɑest]     \tab Geist     \tab ‘intellect’ \tab 128
\ex\label{ex:11:45g}  krɑot     \tab [krɑot]     \tab Kraut     \tab ‘herb’      \tab 137\\
     klǭɡǝ     \tab [klɔːgǝ]    \tab klagen    \tab ‘complain-\textsc{inf}’ \tab 137\\
     knǫpǝ     \tab [knɔpǝ]     \tab Knoten    \tab ‘knot’     \tab 137
\ex\label{ex:11:45h}  \'{k}rɑ̄fs \tab [crɑːfs]    \tab Krebs     \tab ‘crab’     \tab 139\\
     \'{k}lɑe  \tab [clɑe]      \tab Kleie     \tab ‘bran’     \tab 124
\ex\label{ex:11:45i}  ɡrɑp      \tab [grɑp]      \tab Grab      \tab ‘grave’    \tab  141
\ex\label{ex:11:45j}  ɡrɑbələ   \tab [ɟrɑbələ]   \tab greifen   \tab ‘grasp-\textsc{inf}’ \tab 140\\
     ɡlɑeχ     \tab [ɟlɑeç]     \tab gleich    \tab ‘soon’    \tab 139
\ex\label{ex:11:45k}  ɡlek      \tab [ɉlek]      \tab Glück     \tab ‘fortune’ \tab 143\\
     \'{k}nepə \tab [cnepə]     \tab knüpfen   \tab ‘tie{}-\textsc{inf}’ \tab 137
\ex\label{ex:11:45l}   kǫp      \tab [kɔp]       \tab Kopf      \tab ‘head’               \tab 119\\
     \'{k}ęp   \tab [cɛp]       \tab Köpfe     \tab ‘head-\textsc{pl}’              \tab 120
\ex\label{ex:11:45m}  klųk  \tab [klʊk]      \tab klug      \tab ‘clever’             \tab 131\\
     \'{k}līɡɒ \tab [cliːɟɐ]    \tab klüger    \tab ‘more clever’     \tab  131
\ex\label{ex:11:45n}  jūɡənt    \tab [ʝuːgənt]   \tab Jugend    \tab ‘youth’           \tab  117\\
     j\={ę}nɒ  \tab [ʝɛːnɐ]     \tab jener     \tab ‘that\textsc{{}-masc.sg}’ \tab 146
\ex\label{ex:11:45o}  jəblēivə  \tab [ʝəbleːivə] \tab geblieben \tab ‘stay-\textsc{part}’    \tab  143
\z 
\z 

In postsonorant position, [x] and [ç] never contrast: [x] occurs after back vowels in (\ref{ex:11:46a}) and [ç] after front vowels in (\ref{ex:11:46b}) or coronal sonorant consonants in (\ref{ex:11:46c}, \ref{ex:11:46d}). [k g] and [c ɟ] illustrate Contrast Type B in \REF{ex:11:2b}; see (\ref{ex:11:46e}--\ref{ex:11:46m}).  Alternations between velar and palatal stops are well-attested in (\ref{ex:11:46n}). In contrast to [k], [g] never surfaces after a consonant, but [ʝ] does in (\ref{ex:11:46o}). [ʝ] surfaces after a historically elided front vowel in (\ref{ex:11:46p}). No examples were found in the original source for [x] after [uː], which is not a common vowel in the dialect.

\ea%46
\label{ex:11:46}Postsonorant dorsal obstruents:
\ea\label{ex:11:46a} vǫx        \tab [vɔx]    \tab Woche     \tab ‘week’ \tab 119\\
    dɑx        \tab [dɑx]    \tab Dach      \tab ‘roof’ \tab 142
\ex\label{ex:11:46b} r\k{i}χə   \tab [rɪçə]   \tab riechen   \tab ‘smell-\textsc{inf}’ \tab 130\\
    hęχt       \tab [hɛçt]   \tab Hecht     \tab ‘pike’    \tab 118
\ex\label{ex:11:46c} štǫrχ      \tab [ʃtɔrç]  \tab Storch    \tab ‘stork’   \tab 139
\ex\label{ex:11:46d} melχ       \tab [melç]   \tab Milch     \tab ‘milk’    \tab 115
\ex\label{ex:11:46e} krųk   \tab [krʊk]   \tab Krug      \tab ‘jug’     \tab 131
\ex\label{ex:11:46f} mɑ̄ɡə       \tab [mɑːgə]  \tab Magen     \tab ‘stomach’ \tab 122
\ex\label{ex:11:46g} re\'{k}    \tab [rec]    \tab Rücken    \tab ‘back’    \tab 117
\ex\label{ex:11:46h} šp\k{i}ɡəl \tab [ʃpɪɟəl] \tab Spiegel   \tab ‘mirror’  \tab 144
\ex\label{ex:11:46i} vɑ̄\'{k}    \tab [vɑːc]   \tab Weg       \tab ‘path’    \tab 147
\ex\label{ex:11:46j} flɑ̄ɡə      \tab [flɑːɟə] \tab pflegen   \tab ‘care for-\textsc{inf}’ \tab 121
\ex\label{ex:11:46k} štɑrk      \tab [ʃtɑːrk] \tab stark     \tab ‘strong’                \tab 122
\ex\label{ex:11:46l} vęr\'{k}   \tab [vɛrc]   \tab Werk      \tab ‘work’                  \tab 121\\
    męl\'{k}ə  \tab [mɛlcə]  \tab melken    \tab ‘milk-\textsc{inf}’ \tab 121
\ex\label{ex:11:46m} bɑ̄r\'{k}   \tab [bɑːrc]  \tab Berg      \tab ‘mountain’          \tab 121
\ex\label{ex:11:46n} zɑoɡə      \tab [zɑogə]  \tab saugen    \tab ‘suck-\textsc{inf}’  \tab 124\\
    zɑe\'{k}st \tab [zɑecst] \tab säugst    \tab ‘suck-\textsc{2sg}’ \tab 125
\ex\label{ex:11:46o} mǫrjə      \tab [mɔrʝə]  \tab morgen    \tab ‘tomorrow’ \tab 119\\
    fęnj       \tab [fɛnʝ]   \tab Pfennige  \tab ‘penny-\textsc{pl}’  \tab 143
\ex\label{ex:11:46p} lɑ̄vχ       \tab [lɑːvç]  \tab lebendig  \tab ‘lively’   \tab 121\\
    rūχ        \tab [ruːç]   \tab ruhig     \tab ‘quiet’    \tab 132\\
    rūjɒ       \tab [ruːʝɐ]  \tab ruhiger   \tab ‘more quiet’ \tab 143
\z 
\z 

The velar nasal and the palatal nasal stand in an allophonic relationship: [ŋ] only surfaces after a back vowel in (\ref{ex:11:47a}) and [ɲ] only after a front vowel in (\ref{ex:11:47b}--\ref{ex:11:47d}). The palatal nasal has two historical sources: \ili{WGmc} \textsuperscript{+}[ŋ] by velar fronting in (\ref{ex:11:47b}) or \ili{WGmc} \textsuperscript{+}[nd] by \isi{Velarization} from (\ref{ex:11:1b}) in (\ref{ex:11:47c}). [ŋ]{\textasciitilde}[ɲ] alternations are attested in (\ref{ex:11:47e}).

\ea%47
\label{ex:11:47}Dorsal nasals in postsonorant position:
\ea\label{ex:11:47a} tsųŋ \tab [tsʊŋ] \tab Zunge \tab ‘tongue’ \tab 116
\ex\label{ex:11:47b} e\'{ŋ}əl \tab [eɲəl] \tab Engel \tab ‘angel’ \tab 149
\ex\label{ex:11:47c} l\k{i}\'{ŋ} \tab  [lɪɲ] \tab Linde \tab ‘linden tree’ \tab 115
\ex\label{ex:11:47d} hų\k{i}\'{ŋ}t \tab [hʊɪɲt] \tab Hund \tab ‘dog’ \tab 142
\ex\label{ex:11:47e} jəfųŋə \tab [ʝəfʊŋə] \tab gefunden \tab ‘find-\textsc{part}’ \tab 116\\
    f\k{i}\'{ŋ}ə \tab [fɪɲə] \tab finden \tab ‘find-\textsc{inf}’ \tab 115
\z 
\z 

In word-initial position, underlying velar stops (/k g/) surface as palatal ([c ɟ]) before a front vowel in (\ref{ex:11:45c}, \ref{ex:11:45d}, \ref{ex:11:45l}) by \isi{Wd-Initial Velar Fronting-6}. If /k g/ are followed by a liquid plus front vowel in (\ref{ex:11:45k}, \ref{ex:11:45m}) then the feature [coronal] of that front vowel merges with the [coronal] feature of the liquid by \isi{Coalescence-2}, which then \isi{feeds} \isi{Wd-Initial Velar Fronting-6}. Word-initial palatal stops are underlyingly palatal (/c ɟ/) before a back vowel in (\ref{ex:11:45e}, \ref{ex:11:45f}) or before a consonant followed by a back vowel in (\ref{ex:11:45h}, \ref{ex:11:45j}). [ʝ] is likewise an underlying palatal in (\ref{ex:11:45o}).

In postsonorant position the allophones [x] and [ç] in (\ref{ex:11:46a}--\ref{ex:11:46d}) derive from /x/, which is realized as palatal [ç] after a coronal sonorant by \isi{Velar Fronting-1}. For the nasal allophones [ŋ ɲ] in \REF{ex:11:47} the underlying sound is /ŋ/, which surfaces as [ɲ] after a front vowel by \isi{Velar Fronting-8}. The latter process also accounts for the realization of /k g/ as [c ɟ] after a front vowel in (\ref{ex:11:46g}, \ref{ex:11:46h}, \ref{ex:11:46n}). If /k g/ are preceded by a front vowel plus liquid sequence in (\ref{ex:11:46l}) then \isi{Coalescence-1} merges [coronal] from the front vowel and the liquid, thereby \isi{feeding} Vel-Fr-8, e.g. /vɛrk/→{\textbar}v\textbf{ɛr}k{\textbar}→[v\textbf{ɛrc}]. In postsonorant position palatal stops are underlying sounds (/c ɟ/) after back vowels in (\ref{ex:11:46i}, \ref{ex:11:46j}) and after consonants preceded by back vowels in (\ref{ex:11:46m}). [ç ʝ] are likewise underlying sounds (quasi-phonemes /ç ʝ/) in (\ref{ex:11:46p}). 

Additional sources for varieties of HPr are \citet{Stuhrmann1896}, \citet{Ziesemer1924}, \citet{Kuck1927, Kuck1933}, and \citet{Tessmann1969}. The data presented in those works were drawn from various towns and villages in the HPr dialect area. There is no significant difference for any of those authors concerning the distribution of [x] (=⟦x⟧) and [ç] (=⟦χ⟧), which surface as predictable positional variants. The sources also agree that HPr has both velar and palatal stops, although separate phonetic symbols for the palatal sounds are rarely given, e.g. \citet[123]{Ziesemer1924}. It is interesting to consider what \citet[116]{Tessmann1969} writes on this topic. He justifies his use of a single symbol for velars and palatals because the distinction between those two places of articulation is predictable based on context, suggesting that they are allophones. However, on the same page Tessmann notes that separate symbols for the palatal series is only necessary “in special cases" (“in besonderen Fällen"). Those instances where Tessmann has separate symbols for palatal stops (⟦k’⟧ and ⟦g’⟧) are precisely the ones where the palatals occur in the context of a back vowel that was historically front, e.g. ⟦k’ɑ̄fər⟧ ‘bug’ (cf. StG \textit{Käfer}), ⟦wɑ̄g’⟧ ‘path’ (cf. StG \textit{Weg}),  ⟦bɑ̄rg’⟧ ‘mountain’ (cf. StG \textit{Berg}). A parallel example from \citet{Ziesemer1924} is ⟦kɑinə⟧ ‘germinate-\textsc{inf}’, which the author gives as a word with an initial fortis palatal fricative; recall the parallel example from Reimerswalde in (\ref{ex:11:45e}). In present terms, the palatal stops in the aforementioned examples are phonemic.\il{High Prussian|)}



\section{Summary}\label{sec:11.8}

There are two clearly identifiable patterns for the set of velar fronting targets for both word-initial and postsonorant position, namely (A) the broad group consisting of all velar consonants, or (B) the narrow set of sounds comprising all and only velar fricatives.

Pattern (A) holds for postsonorant velar fronting in \ipi{West Mecklenburg}, \ipi{Sebnitz}, \ipi{Seifhennersdorf}, Kreis \ipi{Bütow}, \ipi{Kamnitz}, Laueburg, Kreis \ipi{Konitz}, \ipi{Willuhnen}, and \ipi{Reimerswalde}, while the narrow group of target sounds in pattern (B) is attested in Kreis \ipi{Rummelsburg}, Kreis \ipi{Schlawe} (\sectref{sec:11.5}), and \ipi{Königsberg} (\sectref{sec:11.6}). There are no clear-cut cases in which the set of targets consists of velar fricatives and velar stops but not the velar nasal (recall the discussion of targets in the typological literature discussed in \sectref{sec:2.3.2}). \ipi{Kamnitz} is a potential example, but this conclusion cannot be definitive because the original source is not clear on whether or not the velar nasal has a palatal variant after front vowels. In one dialect mentioned earlier (\ipi{Bleckede}; \sectref{sec:11.3}) both /x/ and /ɣ/ appear after a sonorant, but only /x/ undergoes fronting. In a number of localities there is a single dorsal fricative as a target (/x/) with velar noncontinuants not undergoing fronting, i.e. \ipi{South Mecklenburg}, \ipi{Ivenack-Stavenhagen}, \ipi{Wolgast}, \ipi{Hemmelsdorf}, Kreis Herzogtum \ipi{Lauenburg} from \sectref{sec:11.3} and Groβschönau, \ipi{Schokau}, \ipi{West Lausitz} from \sectref{sec:11.4}. Since /ɣ/ is not present in those dialects it cannot be known whether or not the set of targets consist of all velar fricatives or only /x/, as in \ipi{Bleckede}. In Kaarβen and \ipi{Barth} (\sectref{sec:11.3}) it cannot be determined whether or not all velar consonants are undergoing fronting (=A) or only the fricatives (=B). Note that \ipi{Barth} is also a potential example of the \ipi{Bleckede} system where only /x/ but not /ɣ/ undergoes fronting.

The predominant pattern for word-initial position is for any velar consonant present in that context to undergo fronting (=A), i.e. \ipi{West Mecklenburg}, \ipi{Sebnitz}, \ipi{Seifhennersdorf}, Kreis \ipi{Bütow}, \ipi{Kamnitz}, \ipi{Lauenburg}, Kreis \ipi{Konitz}, \ipi{Willuhnen}, \ipi{Reimerswalde}.

The broadest context (coronal sonorant consonants) is well-attested in postsonorant position, i.e. \ipi{Seifhennersdorf}, Kreis \ipi{Bütow}, Kreis \ipi{Konitz}, \ipi{Willuhnen}. However, those triggers are not attested in all dialects. First, the set of triggers for a number of varieties listed above consists of front vowels but crucially not the coronal sonorant consonants. That narrow set of triggers is particularly well-attested in word-initial position, e.g. \ipi{Kamnitz}, Kreis \ipi{Lauenburg}, \ipi{Willuhnen}. Second, two dialects are documented with an even narrower group of triggers for postsonorant fronting: nonlow front vowels (\ipi{Kamnitz}) and front tense vowels (Kreis \ipi{Rummelsburg}). Nonlow triggers are attested in German dialects outside of the region investigated in this chapter, i.e. \ipi{Rheintal} (\sectref{sec:3.4}), \ipi{Rhoden} (\sectref{sec:5.2}), \ipi{Obersaxen} (\sectref{sec:6.3}). However, the front tense vowel context is otherwise without precedent in German dialects (see \sectref{sec:12.7.1}).

Several dialects discussed in the present chapter exhibit the effects of \is{Coalescence-1}Co\-a\-les\-cence-1 or \isi{Coalescence-2}. As noted earlier, places with one of those processes are situated in the same area as the ones in which they are absent. For example, in \ipi{West Mecklenburg}, the /x/ after a sequence of back vowel plus liquid surfaces as [x], but after a front vowel plus liquid as [ç] (=\ref{ex:11:5}b,d). By contrast in \ipi{South Mecklenburg} /x/ surfaces in both contexts as [ç]. See \sectref{sec:12.8.1} for further discussion of how the two processes of coalescence fit into German dialects as a whole.

\section{{Velar} {noncontinuant} {targets} {viewed} {historically} }\label{sec:11.9}

I consider first (\sectref{sec:11.9.1}) the historical interpretation of the two types of dialect referred to in \sectref{sec:11.8} and then the influence non-Gmc language on that development (\sectref{sec:11.9.2}).

\subsection{Extension of velar fronting targets}\label{sec:11.9.1}

The two patterns referred to in \sectref{sec:11.8} -- broad targets (A) and narrow targets (B) -- mirror two distinct historical stages. In particular, velar fronting was originally phonologized with a smaller set of targets (B), and later the set of targets was expanded to include all velar consonants (A); recall the \isi{rule generalization} model from \sectref{sec:2.4.1}. That historical progression supports the implication in \REF{ex:11:48} (from \sectref{sec:2.3.2}), which German dialects obey without exception.\footnote{{In \REF{ex:11:48} and below I juxtapose velar fricatives (/x ɣ/) with velar stops (/k g/). It may be possible to propose a similar generalization for the velar nasal (/ŋ/).} }

\ea%48
\label{ex:11:48}\textsc{\isi{Implicational Universal for Velar Fronting Targets-1}}:\\
    If a velar stop (/k g/) undergoes velar fronting then the corresponding fricative (/x ɣ/) does as well.
\z

As stated above, \REF{ex:11:48} correctly predicts that there are dialects in which velar stops and velar fricatives serve as targets (A) as well as dialects where only velar fricatives undergo fronting (B). However, the same implication precludes dialects in which only velar stops undergo the change but velar fricatives in the same context fail to exhibit fronting. The final clause in the preceding sentence (“...velar fricatives in the same context...”) is important because there are dialects in which velar fronting targets stops, but velar fricatives are not present in that context. One example is \ipi{West Mecklenburg}, in which velar fronting targets /k g/ in word-initial position (=\ref{ex:11:4}). Significantly, neither /x/ nor /ɣ/ occur word-initially.

Although \REF{ex:11:48} holds without exception for German, it cannot be universally valid (recall the discussion on the typology of \isi{Velar Palatalization} targets in \sectref{sec:2.3.2}). Since there are many languages where velar stops undergo fronting/\isi{Velar Palatalization} but not the velar fricatives, it should come as no surprise that there is no phonetic motivation for \REF{ex:11:48}.

The reason \REF{ex:11:48} is correct for German is due to the history of velar fronting targets, as described above: The first targets historically were velar fricatives, while the velar noncontinuants were only added to that set at a later stage. The narrow targets have been in the language for such a long time that that version of velar fronting has had time to diffuse geographically through virtually all of modern-day Germany and most of Austria; hence, there are very few places in Germany and Austria where velar fronting could be phonologized with only velar noncontinuants as the sole targets.

\subsection{Influence from non-Germanic languages}\label{sec:11.9.2}

The palatal noncontinuants investigated in this chapter ([c ɟ ɲ]) derived historically from the corresponding velars by some version of velar fronting. That assessment is not controversial because the original velars are preserved in other dialects. For example, in \ipi{Reimerswalde} (\sectref{sec:11.7}), the initial sound in the native German word ‘money’ is palatal ([ɟ]), i.e. [ɟɛlt], but that palatal surfaces in other dialects as velar ([g]), e.g. \il{Standard German}StG [gɛlt]. Sounds like [c ɟ ɲ] therefore have the same history as the palatal fricatives [ç ʝ] in the dialects discussed in previous chapters in the sense that both sets of sounds arose via some version of velar fronting. Those noncontinuants that are now palatal quasi-phonemes or phonemic palatals were once allophones of velars in the neighborhood of front segments that served as triggers for velar fronting. When those front sounds elided or shifted to back sounds, the palatal noncontinuant allophones were encoded directly in underlying representations.

Most dialects with expanded targets are coterritorial with at least one \ili{Slavic} language, in particular either \ili{Polish} and/or \ili{Kashubian} (both \ili{West Slavic}). \ili{Slavic} languages possess phonemic sounds that are similar phonetically to [c ɟ ɲ]. Although the palatal noncontinuants discussed below had an endogenous (German-internal) history whose emergence is structural (phonological), I suggest that social factors (contact with \ili{Slavic} languages) probably played a role in their \isi{phonologization} as well.\footnote{{This question has been discussed in the literature for a number of years; a representative example of the type of publication that was common over one century ago is \citet{Gréb1921} for \ili{Zipser German}, who discusses the findings of \citet{Semrau1915a, Semrau1915b} at length (\sectref{sec:11.5}). Another linguist who draws a correlation between palatal stops in ELG dialects and \ili{Slavic} languages is \citet[120--124]{Mitzka1959}. See also the discussion in \citet[92--98]{Siemens2012} on Plautdietsch (\sectref{sec:9.5.2} and \sectref{sec:11.6}).} }

Before discussing the \ili{Slavic} influence on German dialects, consider the way in which palatal noncontinuants arose in native German words. As a representative example, I provide three items in \REF{ex:11:49} from \ipi{Reimerswalde} illustrating the development of \ili{WGmc} \textsuperscript{+}[k] in word-initial position. These three concrete examples from one particular variety are representative of the palatal noncontinuants in the other varieties discussed above. Stage 1 represents the point where velar fronting was absent and /k/ surfaced without change as [k], although phonetic (coarticulatory) fronting is also assumed to have been present. At Stage 2, velar fronting was phonologized. That process applied in the context before front vowels, as indicated in the first and third examples. At Stage 3, \isi{Vowel Retraction} restructured the vowel /iː/ to /ɑe/, a change that triggered the restructuring of the original /k/ to the phoneme /c/ at Stage 3. The change from /yː/ to /iː/ in the final example is assumed to have postdated the change from /iː/ to /ɑe/. Velar fronting continued to operate before front vowels, as in the final two examples.

\ea%49
\label{ex:11:49}
\begin{tabular}[t]{@{}llll@{}}
 \relax /kiːn/        &  /kuː/        &    /kyː/      &        \\
 \relax [kiːn]        &   [kuː]       &  [kyː]        & Stage 1\\\tablevspace
 \relax /kiːn/        &  /kuː/        &    /kyː/      &         \\
 \relax [ciːn]        &   [kuː]       &  [cyː]        & Stage 2\\\tablevspace
 \relax /cɑen/        &  /kuː/        &    /kiː/      &        \\
 \relax [cɑen]        &   [kuː]       &  [ciː]        & Stage 3\\\tablevspace
 \relax \textit{Keim} &  \textit{Kuh} & \textit{Kühe} & \il{Standard German}StG  \\
 \relax ‘germ’        &  ‘cow’        &  ‘cow-\textsc{pl}’       &
\end{tabular}
\z 

Examples like [cɑen] ‘germ’ show that the etymological front vowel /iː/ (cf. \ili{OHG} \textit{kīmo}) has left its trace in the form of the palatal [c] (/c/), which was formerly a positional variant of /k/.\footnote{The historical derivations in {\REF{ex:11:49} are intended to illustrate that velar fronting affected velar noncontinuants like [k] in native words. Some of the examples discussed in the present chapter reveal that velar fronting also applied in loanwords; however, examples like those involve loanwords that have been well-integrated into the language, e.g. the word ‘head’ (and its plural) \il{Standard German}StG [kɔpf]{\textasciitilde}[kœpfə] from (\ref{ex:11:45l}) was originally borrowed from Latin} \textrm{\textit{cūpa, cuppa}}.}

Why did velar fronting affected velars like /k/ predominantly in those German-speaking areas coterritorial with \ili{Slavic} languages? There was unarguably contact between speakers of \ili{Slavic} languages and speakers of the German dialects examined in this chapter, and I claim that this contact probably played a role in the extension of velar fronting to velar noncontinuants.

There is more than one way in which language-contact might have played out. I describe a possible scenario which involves the \isi{acquisition} of \ili{Slavic} loanwords, although variations on the same theme are also conceivable. It needs to be stressed that the progression of changes described here is highly speculative. First, not all of the original sources cited earlier discuss \ili{Slavic} loanwords, and second -- even in those works where that type of \isi{loanword} is included -- not all of them contain palatal noncontinuants. Consider now the following three historical stages. WeSl designates a West \ili{Slavic} language (see discussion below) and EaGm those varieties of ELG and ECG with palatal noncontinuants.

\begin{description}
\item[Stage P:]   EaGm had velar fronting, which only affected velar fricatives (/x ɣ/); velar noncontinuants may have been subject to coarticulatory (phonetic) fronting;
\item[Stage Q:]   WeSl loanwords with palatal noncontinuants were acquired by speakers of EaGm;
\item[Stage R:]   The presence of palatal noncontinuants in WeSl loanwords in EaGm served as a catalyst for the extension of the set of triggers for velar fronting from velar fricatives to all velar consonants.
\end{description}

Stage P corresponds to Stage 2 in \REF{ex:11:49} and Stage R to Stage 3. Stage Q therefore represents a point not depicted above between Stage 2 and Stage 3.

Linguistic evidence points to a \isi{phonologization} of velar fronting in WCG at a very early date, namely around the ninth century (\chapref{sec:16}). The northeastern parts of pre-1945 Germany discussed in this chapter were originally populated by \ili{Slavic} peoples, and German-speaking settlers only entered that region via the \is{Ostsiedlung}\textsc{ostsiedlung} in a series of waves beginning in the eleventh century; see  \citet[135ff.]{Hirt1925}, \citet[180ff.]{Bach1950}, \citet{Mitzka1959}, and \citet[169ff.]{Bach1970}. I speculate that many of those settlers brought Stage P/Stage 2 velar fronting with them in that migration eastwards.

One difficulty involving Stage Q is that the historical rule of velar fronting was phonologized many centuries ago, and for that reason it is not clear what the nature of the palatal noncontinuants in loanwords was at that point in time. For example, modern \ili{Polish} and modern \ili{Kashubian} (\mapref{map:44}) have no \isi{phonemic palatal} stops ([c ɟ]), although they both possess alveolopalatal affricates ([tɕ dʑ]). Did the WeSl loanwords at Stage Q contain [tɕ dʑ], or perhaps an earlier reflex of those sounds ([c ɟ])? Could [c ɟ] have been present in loanwords from a \ili{Baltic} language?\footnote{Three candidates are \ili{Latvian}, \ili{Lithanian} (both East \ili{Baltic}), and  the extinct West \ili{Baltic} language \ili{Old Prussian}. Latvian contrasts /c ɟ/ and /k g/ (\citealt{Urek2016}). Both Lithuanian (\citealt{Augustaitis1964}) and Old Prussian (\citealt{Schmalstieg1964}) contrast /k g/ and the secondarily palatalized sounds /kʲ gʲ/.} Regardless of how one answers these questions, the point is that loanwords in EaGm referred to here served as a signal to speakers that palatal noncontinuants are sounds distinct from the corresponding velars.

Consider now the connection between the \isi{acquisition} of loanwords (Stage Q) and the extension of VeFr to velar noncontinuants (Stage R). Stage Q is clearly a sufficient condition for Stage R, because there are many varieties discussed above with the broad set of targets that possess WeSl loanwords. However, it remains unclear whether or not Stage Q is a necessary condition for Stage R.

As noted earlier, not all original sources for EaGm dialects discuss WeSl loanwords, so this question will ultimately need to remain open for further study. Some evidence that there is a direct correlation between the influence of WeSl (which might be deduced on the basis of the sheer number of \ili{Slavic} loanwords) and the broader set of targets for velar fronting in postsonorant position can be observed in two neighboring \il{East Pomeranian}EPo varieties discussed earlier (\sectref{sec:11.5}): Kreis \ipi{Bütow} (with a broad set of targets) and Kreis \ipi{Rummelsburg} (with a narrow set of targets): The source for both dialects \citep[73]{Mischke1936} notes that Kreis \ipi{Bütow} has more \ili{Slavic} loanwords than Kreis \ipi{Rummelsburg} (“B.M. [=Kreis \ipi{Bütow}] hat mehr slaw. Lehnwörter als R.M. (=Kreis \ipi{Rummelsburg})ˮ). In Kreis \ipi{Schlawe} (narrow set of targets), \citet[83]{Mahnke1931} similarly observes that the number of \ili{Slavic} loanwords is relatively very small (“verhältnismässig sehr geringˮ).\footnote{{A potential argument against a necessary connection between Stage Q and Stage R is posed by Kreis \ipi{Konitz} (\sectref{sec:11.5}). The author of the orginal source \citep[144]{Semrau1915a} stresses that even though her \il{East Pomeranian}EPo dialect is surrounded by \ili{Polish}-speaking communities, there was no comingling of the two languages (“keinerlei Vermischung [hat] stattgefundenˮ).}} \footnote{{One \il{East Pomeranian}EPo variety discussed earlier (\ipi{Lauenburg}; \citealt{Pirk1928}) lists the nativized realization of a small number of \ili{Slavic} loanwords containing [c] (=⟦kʹ⟧). Those examples are significant because the [c] realization corresponds to [k] in the donor language (\ili{Polish}), e.g. ⟦borōvkʹə⟧ ‘blueberry’ < Polish} \textrm{\textit{borówka}}\textrm{. Since there is no evidence that the final vowel in the \ili{Polish} example was ever front, it appears that speakers of the \ipi{Lauenburg} dialect treat [c] as a sound whose distribution is governed by the phonology of \ipi{Lauenburg}. Recall from (\ref{ex:11:35i}) that [c] (and not [k]) surfaces before \isi{schwa}. See \citet[157]{Jacobs1996} for similar examples involving [l}\textrm{\textsuperscript{j}}\textrm{] in \ili{Central Yiddish} loanwords from \ili{Polish}.}}

Recall from \sectref{sec:2.4.1} that sound change begins in a \isi{focal area} and then spreads both temporally and geographically from that point of origin. As pointed out in that earlier section, the \isi{focal area} is the place where that process has the most general set of triggers/targets. The implication is that dialects like \ipi{Reimerswalde} in \REF{ex:11:49} with an expanded set of target segments (all velar consonants) must have been a \isi{focal area}. This is a possible interpretation, although the role of loanwords suggests that there might be an alternative. In particular, dialects like \ipi{Reimerswalde} might have an expanded set of targets not because they are older than dialects with a narrow set of targets (fricatives) but instead because their speakers had a greater exposure to loanwords. Since there are two conceivable interpretations for dialects like \ipi{Reimerswalde} with a broad set of target segment I do not consider this issue further.\largerpage

In sum, the emergence of palatal noncontinuants in German dialects once spoken in the east clearly had a structural (phonological) justification, but in all likelihood a social one as well (loanwords from \ili{Slavic} languages). These two factors therefore provide evidence for \isi{polycausality}, as described briefly in \sectref{sec:2.4.4}. As noted in that section, my analysis of the \isi{phonemicization} of [c ɟ ɲ] in LG varieties once spoken in the eastern parts of pre-1945 Germany strongly resembles the treatment for the \isi{phonemicization} of lenis (voiced) fricatives in the history of \ili{English} ([v z ð]); \citet[142]{RingeEska2013} and \citet[91--93]{Minkova2014} and \sectref{sec:8.6.1}. The literature on this topic is in agreement that one of the reasons for \isi{phonemicization} was the occurrence of \ili{French} loanwords with those sounds. For example, in \ili{OE} the two fricatives [f] and [v] were allophones, where the latter occurred between lenis sounds (e.g. intervocalically) and the former in the elsewhere case (e.g. word-initially). \citet{Minkova2014} notes that there was an influx of over 800 \ili{French} loanwords beginning with [v] after the eleventh century (in \ili{ME}) that were not adapted with [f]; this means that there were now minimal pairs involving the inherited (Gmc) [f] and the new [v] in loanwords, e.g. \textit{fēle} ‘many’ (cf. \il{Standard German}StG [fiːl]) vs. \textit{vēle} ‘veal’ (< Old \ili{French}). Loanwords with intervocalic [f] which failed to surface as [v] were also attested in ME, e.g. \textit{sacrifice} < \ili{French} \textit{sacrifice}.

\section{{Areal} {distribution} {of} {palatal} {noncontinuants}}\label{sec:11.10}

This chapter has taken a close look at the phonology of German dialects in which at least one velar noncontinuant serves as a target for velar fronting. It is possible to talk about those targets in more than one way. First, one could draw a distinction between velar fronting targets in word-initial position and postsonorant position. Second, one could classify those varieties in which the palatal noncontinuant outputs of velar fronting are allophones (synchronically \isi{derived palatals}) vs. those in which the output sounds are underlying palatal noncontinuants (recall \ref{ex:11:2}). Third, one could ask whether or not the output for a target velar stop is itself a stop, or an \isi{affricate}. Instead of giving a series of tables in which such distinctions are made individually, I simply give one (\tabref{tab:11.1}), which includes all of the relevant studies discussed in this chapter in addition to a few others once spoken in the same region. All of these places are indicated on \mapref{map:19}.

\begin{table}
\caption{Selection of LG/HG varieties in which velar noncontinuants serve as targets for velar fronting in postsonorant/word-initial position\label{tab:11.1}}
\begin{tabular}{lll}
\lsptoprule
Place & Dialect & Source\\\midrule
\ipi{West Mecklenburg} & \il{Mecklenburgish-West Pomeranian}MeWPo & \citet{Kolz1914}\\
\ipi{Seifhennersdorf} & \il{Silesian}Sln & \citet{Michel1891}\\
\ipi{Sebnitz} & \il{Silesian}Sln & \citet{Meiche1898}\\
\ipi{Putzig} (Posen) & \il{East Pomeranian}EPo & \citet{Teuchert1913}\\
Kreis \ipi{Konitz} & \il{East Pomeranian}EPo & \citet{Semrau1915a,Semrau1915b}\\
\ipi{Kamnitz} & \il{East Pomeranian}EPo & \citet{Tita1921}\\
\ipi{Lauenburg} & \il{East Pomeranian}EPo & \citet{Pirk1928}\\
Kreis \ipi{Bütow} & \il{East Pomeranian}EPo & \citet{Mischke1936}\\
\ipi{Sępóno Krajeńskie} & \il{East Pomeranian}EPo & \citet{Darski1973}\\
\ipi{Reimerswalde} & \il{High Prussian}HPr & \citet{KuckWiesinger1965}\\
\ipi{Alt-Thorn} & \il{Low Prussian}LPr & \citet{Wagner1912}\\
\ipi{Danziger Nehrung} & \il{Low Prussian}LPr & \citet{Mitzka1922}\\
\ipi{Willuhnen} & \il{Low Prussian}LPr & \citet{Natau1937}\\
\ipi{Bieberstein} bei Barten & \il{Low Prussian}LPr & \citet{Tessmann1966}\\
\lspbottomrule
\end{tabular}
\end{table}

All of the sources listed above make it clear that the targets for velar fronting must also include velar noncontinuants even though some authors (e.g. \citealt{Teuchert1913}) do not give separate symbols for those sounds (e.g. [k] vs. [c]). In that type of source no conclusions can be drawn concerning the triggers for velar fronting; hence, I do not discuss them further.

\begin{map}
% \includegraphics[width=\textwidth]{figures/VelarFrontingHall2021-img025.png}
\includegraphics[width=\textwidth]{figures/Map19_11.3.pdf}
\caption[Areal distribution of velar noncontinuant targets]{Areal distribution of velar noncontinuant targets. Low Prussian, High Prussian, East Pomeranian, Mecklenburgish-West Pomeranian, and Silesian varieties with at least one velar noncontinuant as target for word-initial and/or postsonorant velar fronting are indicated with white squares. Varieties in the same general area in which velar fronting (word-initial and/or postsonorant) targets consist only of fricatives are indicated with black squares.}\label{map:19}
\end{map}

One point stressed throughout this chapter is that the more general targets characterized by the places in \tabref{tab:11.1} might be narrower for other places in the same region. This point is made clear in \mapref{map:19}, which includes all of the varieties listed in \tabref{tab:11.1} (white squares) as well as the varieties in the same area discussed in this chapter where the target for velar fronting consists only of fricatives (black squares).

\section{{Conclusion}}\label{sec:11.11}

This chapter has examined German dialects in which the set of targets for velar fronting consists of velar fricatives like (/x ɣ/) as well as velar noncontinuants (/k g ŋ/). Thus, in contrast to the dialects discussed in Chapters~\ref{sec:3}--\ref{sec:10} velar fronting in the case studies investigated in the present chapter has a broader set of targets. It was also demonstrated the original palatal allophones of velar noncontinuants (i.e. [c ɉ ɲ]) can have an opaque history and either quasi-phonemicize or phonemicize (i.e. /c ɉ ɲ/) according to the same paths described in Chapters~\ref{sec:7}--\ref{sec:10}.
