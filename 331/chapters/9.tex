\chapter{Phonemicization of palatals (part 2)}\label{sec:9}

\section{{Introduction}}\label{sec:9.1}

The present chapter probes dialects in which velars and the corresponding palatals contrast in postsonorant position. The case studies discussed below all have in common that the velar vs. palatal contrast occurs after certain back vowels, but not after front vowels, where only the palatal surfaces. That system is the mirror-image of the word-initial one referred to in \chapref{sec:8} as Contrast Type B. The two Contrast Type B systems investigated below for postsonorant position are depicted in \REF{ex:9:1}, where [i] and [ɑ] are cover symbols for front vowels and back vowels respectively. The dorsal fricatives in \REF{ex:9:1a} are fortis [x ç] and the ones in \REF{ex:9:1b} are the lenis counterparts ([ɣ ʝ]).
 
\ea%1
\label{ex:9:1}Contrast Type B:
\begin{multicols}{2}
\ea\label{ex:9:1a}
\fbox{
\begin{tabular}[t]{ll}
[...iç...] & [...ɑç...]\\
           & [...ɑx…]\\
\end{tabular}
}
\ex\label{ex:9:1b}
\fbox{
\begin{tabular}[t]{ll}
[...iʝ...] & [...ɑʝ...]\\
           & [...ɑɣ…]\\
\end{tabular}
}
\z
\end{multicols}
\z 

The palatal vs. velar contrasts in \REF{ex:9:1} are the consequence of the \isi{phonemic split} depicted in \REF{ex:9:2}:

\ea%2
\label{ex:9:2}Phonemic split in postsonorant position (\isi{Vowel Retraction}):
\begin{forest} for tree = {fit=band}
[,phantom
 [\textsc{/Ve/}  [\textsc{[Ve]}]]              
 [>]       
 [\textsc{/Ve/},calign=first  [\textsc{[Ve]}]    [\textsc{[Pa]}]]
 [>]       
 [\textsc{/Ve/}  [\textsc{[Ve]}]] 
 [\textsc{/Pa/} [\textsc{[Pa]}]]
]
\end{forest}
\z 

At Stage 1 the velar is realized as velar regardless of the nature of the preceding sound. At Stage 2 the same velar develops a palatal allophone in the context after coronal sonorants (or some subset thereof). The palatal allophone at Stage 2 is then phonemicized (/ç/ or /ʝ/) at Stage 3 when a front vowel triggering the palatal allophone at Stage 2 was restructured to a back vowel by \isi{Vowel Retraction} (Chapters~\ref{sec:7}--\ref{sec:8}). As a result of that change, the velar and the palatal contrast in the context after certain back vowels. Note that the opaque post-back vowel palatal exemplifies the historical \isi{overapplication} of velar fronting; recall \figref{fig:2.12}.

As described in \chapref{sec:8}, the \isi{phonemicization} of the palatal at Stage 3 did not lead to the loss of velar fronting. Instead, that rule remains in that system as a rule of \isi{neutralization} applying in the context after front segments.

In \sectref{sec:9.2} and \sectref{sec:9.3} I discuss several Contrast Type B varieties of \il{Central Hessian}CHes and \il{Rhenish Franconian}RFr illustrating the \isi{phonemic split} depicted in \REF{ex:9:2}. In \sectref{sec:9.4} I consider two questions, namely the status of Contrast Type A and Contrast Type C systems attested in word-initial position (\chapref{sec:8}) for postsonorant position and the relationship between the quasi-phonemicization of palatals and phonemic palatals. In \sectref{sec:9.5} I discuss the areal distribution of German dialects with a contrast between postsonorant velar and palatal fricatives. The chapter concludes in \sectref{sec:9.6}.

\section{{Central} {Hessian}}\label{sec:9.2}\il{Central Hessian|(}

Contrasts between [x] and [ç] after certain back vowels (=Contrast Type B in \ref{ex:9:1}) are attested in several varieties of \il{Central Hessian}CHes, a point stressed throughout the survey of Hessian vocalism in R. D. \citet[30-34]{Hall1973}. In this section I consider five representative varieties.\footnote{{The palatal vs. velar contrast referred to above is also commented on in the recent survey of Hes dialects in \citet[447]{BirkenesFleischer2019}. I only discuss oral vowels below, concentrating primarily on monophthongs. Nasalized vowels are ignored because not enough data are presented in the original sources where those vowels are followed by dorsal fricatives to arrive at conclusions concerning the distribution of the latter sounds. The occurrence of dorsal fricatives after \isi{schwa} (/ə/) and diphthongs are not considered in detail because palatals and velars typically do not contrast after those vocalic sounds.} }

\citet{Kroh1915} describes the dialect spoken in \ipi{Wissenbach} (\mapref{map:11}), which has the phonemic front vowels /iː i eː e ɛ/ and back vowels /uː u oː o ɔː ɔ ɑː ɑ/. Velars ([x ɣ]) contrast with the corresponding palatals ([ç ʝ]). The four phonemic dorsal fricatives are listed in \REF{ex:9:3}. Not depicted here is [g], which is phonemic (/g/) because it contrasts with both [ɣ] and [x]. In word-initial position the only dorsal fricative that surfaces is the \isi{etymological palatal} [ʝ].

\ea%3
\label{ex:9:3}
  \begin{forest}
  [,phantom
  [/x/ [{[x]}]]  [/ç/ [{[ç]}]]  [/ɣ/ [{[ɣ]}]] [/ʝ/ [{[ʝ]}]]
  ]
  \end{forest}
\z

Although velars contrast with the corresponding palatals after certain back vowels, only palatals occur after a coronal sonorant.\largerpage

The examples in \REF{ex:9:4} indicate that [x] (=⟦x⟧) surfaces after a back vowel. No examples were found in \citet{Kroh1915} in which [x] occurs after [uː oː o]. These are accidental gaps.


\TabPositions{.1\textwidth, .25\textwidth, .5\textwidth, .8\textwidth}
\ea%4
\label{ex:9:4}\ipi{Wissenbach} [x] (from /x/):
\ea\label{ex:9:4a}  šbrux                    \tab [ʃprux] \tab Spruch             \tab  ‘saying’           \tab 110\\
      hǭx                     \tab [hɔːx]  \tab zweizinkige Hacke  \tab ‘two-pronged hoe’   \tab  86\\
      fǫxə                    \tab [fɔxə]  \tab fauchen            \tab ‘hiss-\textsc{inf}’ \tab   92\\
      ɑ̄x                      \tab [ɑːx]   \tab auch               \tab ‘also’ \tab 95\\
      dɑx                     \tab [dɑx]   \tab Dach               \tab ‘roof’ \tab 70\\
\ex\label{ex:9:4b}   blę\textsuperscript{ɑ}x \tab [blɛɑx] \tab Blech              \tab  ‘tin’ \tab 76
\z 
\z

The items in \REF{ex:9:4a} have in common that the back vowel before [x] is etymologically back, while the diphthong [ɛɑ] in \REF{ex:9:4b} was etymologically front (e.g. \ili{MHG} \textit{blech}). For most of the examples given below the nature of the stem vowel (front vs. back) can be inferred from \il{Standard German}StG spelling.

The data in \REF{ex:9:5} exemplify the occurrence of the opaque palatal [ç] (=⟦χ⟧) after a back vowel ([ɑː ɑ ɔː ɔ]). The back vowels in the first column all derived historically from front vowels.\footnote{\citet[729]{Behaghel1911} may have been the first linguist to observe that the ich-Laut occurs in certain varieties of Hes after back vowels ([ɑ]) that derived historically from diphthongs ending in a front vowel ([ɑi]).}

\ea%5
\label{ex:9:5}\ipi{Wissenbach} [ç] (from /ç/):
\ea\label{ex:9:5a} blɑ̄χ  \tab [blɑːç] \tab bleich    \tab ‘pale’ \tab 94\\
    wɑ̄χ   \tab [vɑːç]  \tab  weich    \tab ‘soft’ \tab 94
\ex\label{ex:9:5b} ɡlɑχ  \tab [glɑç]  \tab  gleich   \tab ‘same’ \tab 89\\
    dɑχ   \tab [dɑç]   \tab Deich     \tab ‘dike’ \tab 89
\ex\label{ex:9:5c} ɑχ    \tab [ɑç]    \tab  ich      \tab ‘I’  \tab 81\\
    mɑχ   \tab [mɑç]   \tab  mich     \tab ‘me-\textsc{acc.sg}’ \tab 81
\ex\label{ex:9:5d} šǭχə  \tab [ʃɔːçə] \tab scheuchen \tab ‘shoo-\textsc{inf}’ \tab 97\\
\ex\label{ex:9:5e} lǫχdə \tab [lɔçtə] \tab Leuchte   \tab ‘light’ \tab 97
\z
\z 

The items listed in \REF{ex:9:4a} and \REF{ex:9:5} illustrate a contrast between [x] (/x/) and [ç] (/ç/) after the back vowels [ɑː ɑ ɔː ɔ]. Minimal pairs are not uncommon, e.g. [dɑx] ‘roof’ vs. [dɑç] ‘dike’.

Additional items illustrating the occurrence of opaque [ç] after [ɑ] are provided in \REF{ex:9:6}. Unlike the words in (\ref{ex:9:5b}, \ref{ex:9:5c}), the original tonic vowel in \REF{ex:9:6} was back (cf. \textsc{MHG} [ɑ]). However, I show below in \REF{ex:9:16} that there is evidence that the original back vowel shifted to a diphthong ending in a front vowel ([ɑi]) before reducing to the monophthong [ɑ].\largerpage[-1]\pagebreak

\ea%6
    \label{ex:9:6}\ipi{Wissenbach} [ç] (from /ç/):\\
    \begin{xlist}
      \sn
    mɑχd  \tab [mɑçt]  \tab macht  \tab ‘do-\textsc{3sg}’ \tab 74\\
    mɑχst \tab [mɑçst] \tab machst \tab ‘do-\textsc{2sg}’   \tab 74\\
    hɑχəl \tab [hɑçəl] \tab Hechel \tab ‘hatchel’           \tab 74\\
    \end{xlist}
\z 

The following examples exemplify [x]{\textasciitilde}[ç] alternations in singular vs. plural pairs. Note that the stem vowels in \REF{ex:9:7} are back in both the singular and the plural. Significantly, the dorsal fricative is [ç] in the plural even though the preceding vowel is back. It is clear from the original source that back stem vowels in the singular regularly undergo fronting (\isi{Umlaut}) before -\textit{er} plurals if the consonant following that vowel is not an original velar, e.g. ⟦flǫs⟧ ‘raft’ {\textasciitilde} ⟦flęsər⟧ ‘raft-\textsc{pl}’ \citep[123--124]{Kroh1915}. However, if the consonant after the original back stem vowel is a velar (e.g. [x]) then its fronted counterpart was once a diphthong ending in a front vowel which was later deleted, e.g. [dɑxər] > [dɑiçər] > [dɑçər], as noted above for \REF{ex:9:6}.

\TabPositions{.15\textwidth, .3\textwidth, .45\textwidth, .65\textwidth}
\ea%7
\label{ex:9:7}\ipi{Wissenbach} [x]{\textasciitilde}[ç] alternations (from /x/ or /ç/):
\ea\label{ex:9:7a} dɑx    \tab [dɑx]    \tab Dach       \tab ‘roof’   \tab 70\\
    dɑχər  \tab [dɑçər]  \tab Dächer     \tab ‘roof-\textsc{pl}’  \tab 74
\ex\label{ex:9:7b} lǫx    \tab [lɔx]    \tab Loch       \tab ‘hole’   \tab 81\\
    lǫχər  \tab [lɔçər]  \tab   Löcher   \tab ‘hole-\textsc{pl}’  \tab 83
\ex\label{ex:9:7c} šdrǫx  \tab [ʃtrɔx]  \tab Strauch    \tab ‘shrub’  \tab 92\\
    šdrǫχə \tab [ʃtrɔçə] \tab  Sträucher \tab ‘shrub-\textsc{pl}’ \tab 92
  \z
\z 

As indicated above, I analyze the dorsal fricatives as either /x/ or /ç/, e.g. /dɑx/ ‘roof’ and      /dɑç-ər/ ‘roof-\textsc{pl}’ for \REF{ex:9:7a}.

A very different set of [x]{\textasciitilde}[ç] alternations is presented in \REF{ex:9:8}. Observe that the stem vowel is back before [x] and front before [ç] (via \isi{Umlaut}).


\ea%8
\label{ex:9:8}\ipi{Wissenbach} [x]{\textasciitilde}[ç] alternations (from /x/):
\ea\label{ex:9:8a} bux        \tab [bux]      \tab Buch     \tab ‘book’                  \tab 90\\
    biχər      \tab [biçər]    \tab Bücher   \tab ‘book-\textsc{pl}’                 \tab 91
\ex\label{ex:9:8b} brǭxdə     \tab [brɔːxtə]  \tab brachte  \tab ‘bring-\textsc{pret}’ \tab 87\\
    br\={ę}χdə \tab [brɛːçtə]  \tab brächte  \tab ‘bring-\textsc{subj}’ \tab 87
\ex\label{ex:9:8c} rɑ̄x        \tab [rɑːx]     \tab Rauch    \tab ‘smoke’                 \tab 95\\
    rǭiχərn    \tab [rɔːiçərn] \tab räuchern \tab ‘smoke-\textsc{inf}’    \tab 96
   \z
\z 

I analyze the underlying sound in the [x]{\textasciitilde}[ç] alternations in \REF{ex:9:8} as /x/, which surfaces as [ç] after a front vowel by \isi{Velar Fronting-1}, which is reproduced in \REF{ex:9:9}:

% \TabPositions{.15\textwidth, .3\textwidth, .45\textwidth, .8\textwidth}
\ea%9
\label{ex:9:9}\isi{Velar Fronting-1}:\\
\begin{forest}
[,phantom
   [\avm{[+son]} [\avm{[coronal]},tier=word,name=target]]
   [\avm{[−son\\+cont]},name=parent [\avm{[dorsal]},tier=word]]            
]
\draw [dashed] (parent.south) -- (target.north);
\end{forest}
\z 

The sound underlying the [x]{\textasciitilde}[ç] alternations in \REF{ex:9:8} must be velar /x/ in the synchronic phonology and not palatal /ç/. If \REF{ex:9:9} were replaced with a \isi{neutralization} retracting /ç/ to [x] after a back vowel then that process would incorrectly affect the /ç/ after [ɑː ɑ ɔː ɔ] in words like the ones in (\ref{ex:9:5}--\ref{ex:9:7}), e.g. [dɑçər] (/dɑç-ər/) ‘roof-\textsc{pl}’.

There is no contrast between [x] and [ç] after a coronal sonorant. \is{Velar Fronting-1}Velar Front\-ing-1 is  therefore a \isi{neutralization} because the contrast between [x] and [ç] is suspended in favor of [ç] after any front vowel, e.g. [i ɛː ɔːi] in \REF{ex:9:8}. The \isi{neutralization} property crucially differentiates \isi{Velar Fronting-1} in \ipi{Wissenbach} from the fronting processes discussed in earlier chapters which relate noncontrasting [x] and [ç].

The data in \REF{ex:9:10} illustrate that [ç] -- but never [x] -- also occurs after a coronal sonorant in morphemes that have no [x] alternant. The front vowels in \REF{ex:9:10a} and coronal sonorant consonants like [l] in \REF{ex:9:10b} were historically front (coronal) sounds, as suggested by the \il{Standard German}StG forms in the third column. No examples were found in in \citet{Kroh1915} in which [ç] occurs after [iː eː e]. These gaps are accidental.

\ea%10
\label{ex:9:10}\ipi{Wissenbach} [ç] (from /x/):
\ea\label{ex:9:10a}   jiχd  \tab [ʝiçt]  \tab Gicht    \tab ‘gout’ \tab 79\\
      šlęχd \tab [ʃlɛçt] \tab schlecht \tab ‘bad’  \tab 76
\ex\label{ex:9:10b}   melχ  \tab [melç]  \tab Milch    \tab ‘milk’ \tab 119
\z 
\z 

I analyze the underlying sound in nonalternating morphemes like the ones in \REF{ex:9:10} as velar (/x/). The reason palatal /ç/ is not the underlying sound is that there is no contrast between palatals and the corresponding velars after coronal sonorants (recall \ref{ex:9:1a}). Put differently, dorsal fricatives are predictable palatal in the context after a coronal sonorant.

The items in \REF{ex:9:11} illustrate the occurrence of [ɣ] after a back vowel (which was also historically back), while the data in \REF{ex:9:12} reveal that there is also an opaque [ʝ], which surfaces after the back vowel [ɑː]. The back vowel in question ([ɑː]) derived historically from a front vowel (cf. \ili{MHG} [ei]). The [ɣ] and [ʝ] in \REF{ex:9:11} and \REF{ex:9:12} are modern reflexes of \ili{WGmc} \textsuperscript{+}[ɣ].

\ea%11
\label{ex:9:11}\ipi{Wissenbach} [ɣ] (from /ɣ/):\\
    \begin{xlist}
    \sn
    mǭɣə \tab [mɔːɣə] \tab Magen \tab ‘stomach’ \tab 120\\
    ɑ̄γ   \tab [ɑːɣ]   \tab Auge  \tab ‘eye’     \tab 120
    \end{xlist}
\ex%12
\label{ex:9:12}\ipi{Wissenbach} [ʝ] (from /ʝ/):\\
    \begin{xlist}
    \sn
    ɑ̄jə   \tab [ɑːʝə]   \tab eigen  \tab ‘own’    \tab 94\\
    dɑ̄jiχ \tab [dɑːʝiç] \tab teigig \tab ‘doughy’ \tab 94\\
    rɑ̄jər \tab [rɑːʝər] \tab Reiher \tab ‘heron’  \tab 94\\
    \end{xlist}
\z 

Significantly, [ʝ] contrasts with [ɣ], which also surfaces after the same two back vowels, e.g. in the minimal pair [ɑːɣə] ‘eye-\textsc{pl}’ vs. [ɑːʝə] ‘own’.

Many morphemes exhibit [g]{\textasciitilde}[ʝ] alternations, as in \REF{ex:9:13}. The [g] and [ʝ] in words like these derived historically from \ili{WGmc} \textsuperscript{+}[ɣ].

\ea%13
\TabPositions{.15\textwidth, .3\textwidth, .45\textwidth, .7\textwidth}
\label{ex:9:13}\ipi{Wissenbach} [g]{\textasciitilde}[ʝ] alternations (from /ɣ/):
\ea\label{ex:9:13a} bədruɡ   \tab [bədrug]   \tab betrog    \tab ‘cheat-\textsc{pret}’    \tab 121\\
    bədreijə \tab [bədreiʝə] \tab betrügen  \tab ‘cheat-\textsc{inf}’       \tab 121
\ex\label{ex:9:13b} šwiɡ     \tab [ʃvig]     \tab schwieg   \tab ‘be silent-\textsc{pret}’ \tab 121\\
    šwɑijə   \tab [ʃvɑiʝə]   \tab schweigen \tab ‘be silent-\textsc{inf}’   \tab 121
 \z
\z 

The sound underlying [g]{\textasciitilde}[ʝ] alternations is /ɣ/, which surfaces as [g] in coda position by \REF{ex:9:14} and as [ʝ] in a word-internal onset (by \isi{Velar Fronting-1}).\footnote{{The reason /g/ cannot be the underlier in \REF{ex:9:13} is that the rule of spirantization required to convert that sound to a fricative would incorrectly affect /g/ in words like [vɑigə] ‘wake-}\textrm{\textsc{inf}}\textrm{’ (=⟦wɑ}\textrm{\textsuperscript{i}}\textrm{ɡə⟧). It is clear from the original source that the occurrence of [ɣ] and [ʝ] in postvocalic position is more involved than what is implied here; I refrain from providing details because that discussion would detract from the velar vs. palatal contrasts, which are the main concern in the present chapter.} }

\ea%14
\label{ex:9:14}\isi{g-Formation-3}: \smallskip\\

\avm{[−son\\+cont\\−fortis\\dorsal]} → [--cont] /   {\longrule} C\textsubscript{0} ]\textsubscript{${\sigma}$}
\z 

Palatal [ʝ] (<\ili{WGmc}  \textsuperscript{+}[ɣ]) -- but never [ɣ] -- surfaces after a front vowel in (\ref{ex:9:15a}) or coronal sonorant consonant in (\ref{ex:9:15b}).\largerpage

\ea%15
\label{ex:9:15}\ipi{Wissenbach} [ʝ] (from /ɣ/):
\ea\label{ex:9:15a} bl\={ę}jə \tab [plɛːʝə] \tab pflegen \tab ‘care for-\textsc{inf}’ \tab  76\\
    rējəl     \tab [reːʝəl] \tab Regel   \tab ‘rule’                  \tab  77\\
    ēj        \tab [eːʝ]    \tab Egge    \tab ‘harrow’                \tab 120
\ex\label{ex:9:15b} foljə     \tab [folʝə]  \tab folgen  \tab ‘follow-\textsc{inf}’   \tab  81
  \z
\z 

As indicated above, the underlying dorsal fricative in words like the ones in \REF{ex:9:15} is analyzed as velar (/ɣ/).

The occurrence of palatal fricatives after back vowels is the consequence of a \isi{phonemic split} triggered by \isi{Vowel Retraction} (=\ref{ex:9:2}). In \REF{ex:9:16} I provide seven representative examples (from \ref{ex:9:4a}, \ref{ex:9:5a}, \ref{ex:9:6}, \ref{ex:9:7a}, and \ref{ex:9:8a}). Consider first the items in the first four columns. It is shown here that the velars and palatals in those words derived from an earlier stage in which the fricatives in question were allophones (=Stage 2). The most significant example involves the /x/ in [blɑːç] ‘pale’, which surfaced as [ç] at Stage 2 because it was preceded by the front vowel [ei]. When \isi{Vowel Retraction} restructured underlying representations (e.g. /ei/ > /ɑː/) at Stage 3, contrasts between the newly created (opaque) phoneme /ç/ in words like [blɑːç] and the inherited phoneme /x/ in words like [ɑːx] ‘also’ emerged after back vowels such as [ɑː]. The example [blɑːç] therefore illustrates that the historical process eliminating front vowels (\isi{Vowel Retraction}) counterbled \isi{Velar Fronting-1}.\largerpage

\eabox{\label{ex:9:16}\resizebox{.91\textwidth}{!}{\begin{tabular}[t]{@{} *{8}{l} @{}}
 \relax /ɑːx/    & /bux/           &  /bix-ər/        &  /bleix/         & /hɑxəl/         &  /dɑx/        & /dɑix-ər/       &        \\
 \relax [ɑːx]    &[bux]            & [bixər]          &  [bleix]         & [hɑxəl]         &   [dɑx]       &  [dɑixər]       & Stage 1\\\tablevspace
 \relax /ɑːx/    & /bux/           &  /bix-ər/        &   /bleix/        &  /hɑixəl/       &  /dɑx/        & /dɑix-ər/       &        \\
 \relax [ɑːx]    & [bux]           & [biçər]          &   [bleiç]        &  [hɑiçəl]       &   [dɑx]       &  [dɑiçər]       & Stage 2\\\tablevspace
 \relax  /ɑːx/   &  /bux/          &   /bix-ər/       &  /blɑːç/         & /hɑçəl/         &   /dɑx/       &  /dɑç-ər/       &        \\
 \relax [ɑːx]    & [bux]           & [biçər]          &  [blɑːç]         & [hɑçəl]         &    [dɑx]      &   [dɑçər]       & Stage 3\\\tablevspace
   \textit{auch} &  \textit{Buch}  &  \textit{Bücher} &  \textit{bleich} & \textit{Hechel} & \textit{Dach} & \textit{Dächer} & \il{Standard German}StG \\
  ‘also’         &  ‘book’         &   ‘book-\textsc{pl}’        &  ‘pale’          & ‘hatchel’       & ‘roof’        &  ‘roof-\textsc{pl}’        &        \\
\end{tabular}}}


Phrased in terms of the listener-driven model described in \sectref{sec:2.5}, a speaker utters [bleiç] (from /bleix/) at Stage 2. The listener misperceives the diphthong as [ɑː] but correctly hears the palatal [ç]. This results in the new (Stage 3) pronunciation [blɑːç]. Most importantly, the listener concludes that the Stage 3 underlying representation contains a palatal (/blɑːç/) because that fricative contrasts with the corresponding velar ([x]) after the same vowel.

The same explanation for the occurrence of [ç] after a back vowel holds for the examples in (\ref{ex:9:5b}--\ref{ex:9:5e}). The original front stem vowel in those items underwent \isi{Vowel Retraction} to a back vowel ([ɑ ɔ ɔː]), but only after the original front stem vowel had created a palatal allophone. The reader is referred to R. D. \citet{Hall1973}, who discusses qualitative shifts among vowels in Hessian varieties.

It can be observed in \REF{ex:9:16} that the allophonic rule of \isi{Velar Fronting-1} at Stage 2 became a rule of \isi{neutralization} at Stage 3. At that point the process neutralized the contrast between velar and palatal to the latter after front vowels in words like [biçər] (/bix-ər/) ‘book-\textsc{pl}’.

The example [hɑçəl] ‘hatchel’ in \REF{ex:9:16} is different from [blɑːç] ‘pale’ because its original stem vowel was back (cf. \ili{MHG} [ɑ]). As indicated above, there is evidence that the original back vowel ([ɑ] /ɑ/) shifted to a diphthong ending in  a front vowel ([ɑi] /ɑi/) and later restructured to a back vowel ([ɑ] /ɑ/) by \isi{Vowel Retraction}. As discussed by \citet[74]{Kroh1915}, the change I dub \isi{Back Vowel Diphthongization} (e.g. [ɑ] /ɑ/ > [ɑi] /ɑi/ for ‘hatchel’) occurred in the context before velar consonants (/x g k ŋ/), where it is retained as [ɑi] before velar noncontinuants ([g k ŋ]), e.g. [hɑiks] ‘witch’; cf. \ili{OHG} [hɑgzussɑ]. The restructuring of the new diphthong [ɑi] (/ɑi/) to the monophthong [ɑ] (/ɑ/) by \isi{Vowel Retraction} only occurred in the context before a palatal.\footnote{{The back vowel in the singular forms in (\ref{ex:9:7b}, \ref{ex:9:7c}) likewise shifted to a diphthong ending in [i] by \isi{Back Vowel Diphthongization}, which was later deleted (\citealt{Kroh1915}: 83, 92--93).} } [hɑçəl] ‘hatchel’ and [dɑx] ‘roof’ illustrate that \isi{Back Vowel Diphthongization} only affected a monophthong ([ɑ]) before [x] (/x/) if the latter sound was in an original open syllable (e.g. [dɑ.xər]). In a syllable closed by one consonant (e.g. [dɑx]), the monophthong failed to diphthongize and is retained as [ɑ].\footnote{{In a syllable closed by two consonants the original vowel ([ɑ]) lowered and rounded to [ɔ], e.g.  [mɔxt] ‘power’ (cf. \il{Standard German}StG [mɑxt]).}} As noted earlier in \REF{ex:9:7}, the umlauted vowel in -\textit{er} plurals in the \il{Central Hessian}CHes dialect of \ipi{Wissenbach} was [ɑi] before a velar. The second component of that diphthong was deleted at Stage 3 by \isi{Vowel Retraction}, thereby creating a \isi{phonemic palatal}.

From a formal point of view, the palatal in words like [blɑːç] ‘pale’, [hɑçəl]  ‘hatchel’, and [dɑçər] ‘roof-\textsc{pl}’ arose just as palatal quasi-phonemes (\chapref{sec:7}): The frontness feature ([coronal]) of the second component of the earlier diphthong ([ei] or [ɑi]) was simultaneously linked to the following palatal sound ([ç]). When those diphthongs were restructured to back monophthongs by \isi{Vowel Retraction}, the feature [coronal] was not deleted, but instead remained anchored to the palatal, which had been phonemicized. Note that the underlying /x/ in the first two examples in \REF{ex:9:16} was inherited without change as /x/ at Stage 3.

I now consider four additional varieties of \il{Central Hessian}CHes which are structurally similar to \ipi{Wissenbach}. In all of those dialects the contrast alluded to arose via \isi{Vowel Retraction}, as depicted in \REF{ex:9:2} and \REF{ex:9:16}.\largerpage

\citet{Friebertshäuser1961} describes the dialect spoken in and around \ipi{Weidenhausen} (\mapref{map:11}). That source lists twenty-seven monophthongs, but not all of those vocalic elements are phonemic in the same community. In \ipi{Weidenhausen} the two fricatives [x] (=⟦x⟧) and [ç] (=⟦χ⟧) are phonemic as in (\ref{ex:9:3}) because they contrast after certain back vowels.\footnote{\label{fn:9:6}\ipi{Weidenhausen} also possesses the \isi{etymological palatal} [ʝ] in word-initial position. The lenis velar [ɣ] is absent entirely. Palatal [ʝ] (<\ili{WGmc} \textsuperscript{+}[ɣ]) surfaces after a coronal sonorant and before a vowel, but \citet[24]{Friebertshäuser1961} also includes one example in which that sound occurs after a back vowel, i.e. [fʊʝəl] ‘bird-\textsc{pl}’ (=⟦f\k{u}jəl⟧). The velar stop [g] contrasts with palatal [ʝ] in postvocalic position, although many words exhibit alternations between [ʝ] and [g]. I leave open how to analyze that array of facts in a synchronic treatment.}

As illustrated in \REF{ex:9:17}, [x] occurs after back vowels that were also historically back. By contrast, the examples in \REF{ex:9:18} show that the opaque palatal [ç] surfaces after a back vowel ([ɑ ɔ]) that was historically front. The change from front vowel to back vowel was accomplished by \isi{Vowel Retraction}; recall the parallel examples from \ipi{Wissenbach} in (\ref{ex:9:5b}, \ref{ex:9:5c}, \ref{ex:9:5e}). The [x] and [ç] in \REF{ex:9:17} and \REF{ex:9:18} derived historically from velar sounds (\ili{WGmc} \textsuperscript{+}[k x ɣ]). Note that \ili{WGmc} \textsuperscript{+}[ɣ] (/ɣ/) restructured to fortis [x] (/x/), e.g. [kuːxəl] ‘ball’ and [boːxə] ‘bow’.


\ea%17
\label{ex:9:17}\ipi{Weidenhausen} [x] (from /x/):\\
\begin{xlist}
\sn
kūxəl    \tab [kuːxəl] \tab Kugel \tab ‘ball’  \tab 16\\
d\k{u}x  \tab [dʊx]    \tab Tuch  \tab ‘towel’ \tab 18\\
bōxə     \tab [boːxə]  \tab Bogen \tab ‘bow’   \tab 15\\
pǫxt     \tab [pɔxt]   \tab Pacht \tab ‘lease’ \tab 15\\
dɑx      \tab [dɑx]    \tab Dach  \tab ‘roof’  \tab 11\\
r\={ę}ɑx \tab [rɛːɑx]  \tab Rauch \tab ‘smoke’ \tab 20\\
\end{xlist}
\ex %18
\label{ex:9:18}\ipi{Weidenhausen} [ç] (from /ç/):
\ea\label{ex:9:18a} dɑχ   \tab  [dɑç]   \tab Teich   \tab ‘pond’ \tab 18\\
    lɑχ   \tab  [lɑç]   \tab Leiche  \tab ‘body’ \tab 18
\ex\label{ex:9:18b} ɑχ    \tab  [ɑç]    \tab ich     \tab ‘I’    \tab 14\\
    dɑχ   \tab  [dɑç]   \tab dich    \tab ‘you-\textsc{acc}.\textsc{sg}’ \tab 14
\ex\label{ex:9:18c} lǫχdə \tab  [lɔçtə] \tab Leuchte \tab ‘light’                        \tab 21\\
    fǫχd  \tab  [fɔçt]  \tab feucht  \tab ‘damp’                         \tab 21
   \z
\z 

The examples listed above are important because they show contrasts between [x] and [ç] after [ɔ] in [pɔxt] ‘lease’ vs. [fɔçt] ‘damp’ and after [ɑ] in [dɑx] ‘roof’ vs. [dɑç] ‘pond’.\footnote{\citet[63--64]{Friebertshäuser1961} notes that palatal [ç] occurs after the long low back vowel [ɑː], which derived historically from \ili{WGmc} \textsuperscript{+}[e] in closed syllables, e.g. [ʃlɑːçt] ‘bad’ (=⟦šlɑ̄χd⟧; cf. \ili{MHG} \textit{sleht}). The [ç] in that type of example is clearly an underlying palatal (/ç/). I interpret that [ç] as a quasi-phoneme and not as a \isi{phonemic palatal} because no example was found in the original source where [x] surfaces after [ɑː].}

\ipi{Weidenhausen} also contrasts [x] and [ç] (<\ili{WGmc}  \textsuperscript{+}[k x] or  \textsuperscript{+}[ɣ]) after the diphthong [ɔːə], as in \REF{ex:9:19a} vs. \REF{ex:9:19b}. Unlike the words in \REF{ex:9:18}, the diphthong in \REF{ex:9:19} was etymologically back (cf. \ili{MHG} [ɑ]). Recall from \REF{ex:9:16} that in the related variety spoken in \ipi{Wissenbach}, \ili{MHG} [ɑ] (/ɑ/) underwent a shift to a diphthong ending in a front vowel ([ɑi] /ɑi/) which then monophthongized to [ɑ] (/ɑ/) before [ç]. I posit that there was a similar development in \ipi{Weidenhausen}; hence, the diphthong deriving from historical [ɑ] ended in a front vowel, which triggered \isi{Velar Fronting-1}, thereby creating [ç]. Assuming that the diphthong in question was [ɔːi] (/ɔːi/), the change to [ɔːə] (/ɔːə/) in \REF{ex:9:19a} triggered the \isi{phonemicization} of /ç/. The change from a diphthong ending in a front vowel to one ending in \isi{schwa} is a specific example of \isi{Vowel Retraction}.

\ea%19
\label{ex:9:19}\ipi{Weidenhausen} [x] (from /x/) and [ç] (from /ç/)
\ea\label{ex:9:19a}
nǭəχd   \tab [nɔːəçt]   \tab Nacht   \tab ‘night’ \tab 11\\
ɡəmǭəχd \tab [gəmɔːəçt] \tab gemacht \tab ‘do-\textsc{part}’ \tab 11
\ex\label{ex:9:19b} 
mǭəxə  \tab [mɔːəxə]   \tab Magen   \tab ‘stomach’ \tab  24\\
ɡrǭəxə \tab [grɔːəxə]  \tab Kragen  \tab ‘collar’  \tab 24
  \z
\z 

The examples in \REF{ex:9:19b} differ from the ones in \REF{ex:9:19a} in that the dorsal fricatives in the former examples derived from \ili{WGmc} \textsuperscript{+}[ɣ]. It is not clear why [x] and not [ç] occurs in \REF{ex:9:19b}. One possibility is that when velar fronting was first phonologized the trigger was restricted to [+fortis] sounds. Given that restriction, [ɣ] surfaced in a word-internal onset even after front vowels. \ili{WGmc}  \textsuperscript{+}[ɣ] was then restructured to [x] (/x/) at a later point. Since the details are not crucial for the present analysis, I do not discuss this issue.

The data in \REF{ex:9:20} exemplify [x]{\textasciitilde}[ç] alternations. Note that the stem vowel in the words as they were transcribed by Friebertshäuser in 1961 are back in both the singular and the plural but that the plural form has an (opaque) palatal fricative [ç]; recall the parallel examples from \ipi{Wissenbach} in \REF{ex:9:7}.

\ea%20
\label{ex:9:20}\ipi{Weidenhausen} [x]{\textasciitilde}[ç] alternations (from /x/ or /ç/):
\ea\label{ex:9:20a} b\k{u}x   \tab [bʊx]   \tab Buch        \tab ‘book’   \tab  22\\
    b\k{u}χər \tab [bʊçər] \tab Bücher      \tab ‘book-\textsc{pl}’  \tab  22
\ex\label{ex:9:20b} šdrǫx     \tab [ʃtrɔx] \tab Strauch     \tab ‘shrub’  \tab  34\\
    šdrǫχ     \tab [ʃtrɔç] \tab Sträucher \tab ‘shrub-\textsc{pl}’ \tab  34
   \z
\z 

I analyze the dorsal fricatives in \REF{ex:9:20} as either /x/ or /ç/, e.g.  /ʃtrɔx/ ‘shrub’ and /ʃtrɔç/ ‘shrub-\textsc{pl}’ for \REF{ex:9:20b}.

Many words exhibit [x]{\textasciitilde}[ç] alternations triggered by a stem vowel mutation. The examples in \REF{ex:9:21a} illustrate that the vowel mutation in question can be \isi{Umlaut}, while the items in \REF{ex:9:21b} show that dialect-specific vowel changes could also trigger the occurrence of [x] after a back vowel that was etymologically front. [x ç] in these examples derived historically from a velar sound (\ili{WGmc} \textsuperscript{+}[k]).

\ea%21
\label{ex:9:21}\ipi{Weidenhausen} [x]{\textasciitilde}[ç] alternations (from /x/):
\ea\label{ex:9:21a} fl\k{u}xə \tab [flʊxə] \tab fluchen \tab ‘curse-\textsc{inf}’  \tab 18\\
    fl\k{i}χ  \tab [flɪç]  \tab Flüche  \tab ‘curse-\textsc{pl}’              \tab 18
\ex\label{ex:9:21b} šd\k{i}əx \tab [ʃtɪəx] \tab Stich   \tab ‘sting’               \tab 22\\
    šd\k{i}χ  \tab [ʃtɪç]  \tab Stiche  \tab ‘sting-\textsc{pl}’              \tab 22
  \z
\z 

The underlying sound in the [x]{\textasciitilde}[ç] alternations in \REF{ex:9:21} is /x/, which fronts to [ç] after a front vowel by \isi{Velar Fronting-1}.

As in \ipi{Wissenbach}, there is no contrast between [x] and [ç] after a coronal sonorant. The data in \REF{ex:9:22} illustrate that [ç] (but never [x]) occurs in that context. The front vowels in \REF{ex:9:22a} and the coronal sonorant consonants like [l] in (\ref{ex:9:22b}) were historically front (coronal) sounds.

\ea%22
\label{ex:9:22}\ipi{Weidenhausen} [ç] (from /x/):
  \ea\label{ex:9:22a} r\k{i}χə  \tab [rɪçə]   \tab riechen \tab ‘smell-\textsc{inf}’ \tab 20\\
      dsēχə     \tab [tseːçə] \tab Zeichen \tab ‘sign’ \tab 19\\
      fęχdə     \tab [fɛçtə]  \tab fechten \tab ‘fence-\textsc{inf}’ \tab 13\\
  \ex\label{ex:9:22b} m\k{i}lχ  \tab [mɪlç]   \tab Milch   \tab ‘milk’ \tab 14
  \z
\z 

I adopt underlying representations for words like the ones in \REF{ex:9:22} with /x/.

\citet{Bender1938} describes a \il{Central Hessian}CHes variety spoken in and around \ipi{Marburg}, focusing in particular on the town of \ipi{Ebsdorf} (\mapref{map:11}). The author lists twenty-six monophthongs (p. 14), but it is not clear how many of those sounds are phonemic in any one community. On the basis of the material in that source, it appears that \ipi{Ebsdorf} has the phonemic front and back vowels /i eː e ɛː ɛ/ and /u oː o ɔː ɔ ɑː ɑ/ respectively. \ipi{Ebsdorf} has the four dorsal fricatives [x ç ɣ ʝ]. [x] (=⟦x⟧) and [ç] (=⟦χ⟧) are phonemic because they contrast after certain back vowels, whereas [ɣ] and [ʝ] stand in an allophonic relationship:

\ea%23
    \label{ex:9:23}
    \begin{forest}
    [,phantom
          [/x/ [{[x]}]]  [/ç/ [{[ç]}]]   [/ɣ/,calign=first [{[ɣ]}]   [{[ʝ]}]]
    ]
    \end{forest}
\z 

The data in \REF{ex:9:24} exemplify the occurrence of [x] after a back vowel, while the examples in \REF{ex:9:25} reveal that the opaque palatal [ç] surfaces after the back vowel [ɑ]. [x ç] in \REF{ex:9:24} and \REF{ex:9:25} derive from an etymological velar sound (\ili{WGmc} \textsuperscript{+}[k x]). Note that [x] and [ç] contrast after [ɑ], e.g. [bɑx] ‘stream’ vs. [tɑç] ‘pond’. As in \ipi{Wissenbach}, \ipi{Ebsdorf} [ɑ] in examples like the ones in \REF{ex:9:25a} derived historically from a front vowel (cf. \ili{MHG} [iː]). The original stem vowel in \REF{ex:9:25b} was back (cf. \ili{MHG} [ɑ]), which underwent \isi{Back Vowel Diphthongization} to [ɑi] and then \isi{Vowel Retraction} to [ɑ] before [ç]; see the discussion in \REF{ex:9:16} involving the \ipi{Wissenbach} data in \REF{ex:9:6} and the parallel examples from \ipi{Weidenhausen} in \REF{ex:9:19}. The vowel in \REF{ex:9:24} was etymologically back.\largerpage

\ea%24
\label{ex:9:24}\ipi{Ebsdorf} [x] (from /x/):\\
\begin{xlist}
  \sn
bux  \tab [bux] \tab Buch \tab ‘book’ \tab 24\\
nōx  \tab [noːx] \tab nach \tab ‘after’ \tab 23\\
wox  \tab [wox] \tab Woche \tab ‘week’ \tab 20\\
nǭxt \tab [nɔːxt] \tab Nacht \tab ‘night’ \tab 16\\
kǫxə \tab [kɔxə] \tab kochen \tab ‘cook-\textsc{inf}’ \tab 20\\
bɑx  \tab [bɑx] \tab Bach \tab ‘stream’ \tab 15
\end{xlist}
\ex%25
\label{ex:9:25}\ipi{Ebsdorf} [ç] (from /ç/):\\
\ea\label{ex:9:25a} ɡlɑχ \tab [glɑç] \tab gleich \tab ‘soon’ \tab 24\\
    tɑχ \tab [tɑç] \tab Teich \tab ‘pond’ \tab 24
\ex\label{ex:9:25b} hɑχəl \tab [hɑçəl] \tab Hechel \tab ‘hatchel’ \tab 17
   \z
\z 


As in the other varieties of \il{Central Hessian}CHes discussed above, the contrast between velar [x] (/x/) and palatal [ç] (/ç/) arose via a \isi{phonemic split} triggered by \isi{Vowel Retraction} (=\ref{ex:9:2}).\footnote{Palatal [ç] (<\ili{WGmc} {\textsuperscript{+}}{[k]) also occurs after a consonant in words like [hobç] ‘hawk’ (=⟦hobχ⟧). The palatal in that type of example was quasi-phonemicized (/ç/) when the original front vowel preceding it was syncopated (cf. \ili{MHG} \textit{habech, habich}). [ç] (<\ili{WGmc}} {\textsuperscript{+}}{[x]) -- but not [x] -- also occurs in \ipi{Ebsdorf} after the back vowel [ɑː], which is the reflex of \ili{WGmc}} \textsuperscript{+}[e] in a closed syllable, e.g. [ʃlɑːçt] ‘bad’ (=⟦šlɑχt⟧). That palatal is a quasi-phoneme (/ç/), as in \ipi{Weidenhausen}.}

A representative example illustrating [x]{\textasciitilde}[ç] (<\ili{WGmc} \textsuperscript{+}[k x]) alternations triggered by an umlauted stem vowel is presented in \REF{ex:9:26}. The underlying velar in that alternation surfaces as palatal by \isi{Velar Fronting-1}. Morphemes containing a nonalternating palatal [ç] after coronal sonorants are listed in \REF{ex:9:27}.

\ea%26
\label{ex:9:26}\ipi{Ebsdorf} [x]{\textasciitilde}[ç] alternations (from /x/):\\
\begin{xlist}
  \sn
dux \tab [dux] \tab Tuch \tab ‘towel’ \tab 24\\
diχər \tab [diçər] \tab Tücher \tab ‘towel-\textsc{pl}’ \tab 25
\end{xlist}
\ex%27
\label{ex:9:27}\ipi{Ebsdorf} [ç] (from /x/):
\ea\label{ex:9:27a} fliχt \tab  [fliçt] \tab Pflicht \tab ‘duty’ \tab 19\\
    keχ \tab [keç] \tab Küche \tab ‘kitchen’ \tab 22\\
    \={ę}χə \tab [ɛːçə] \tab Eiche \tab ‘oak tree’ \tab 32\\
    bęχər \tab [bɛçər] \tab Becher \tab ‘cup’ \tab 18\\
\ex\label{ex:9:27b} melχ \tab [milç] \tab Milch \tab ‘milk’ \tab 19
   \z
\z 

The data in \REF{ex:9:28} illustrate the postsonorant distribution of [ɣ], which only occurs after a back vowel in (\ref{ex:9:28a}) and [ʝ], which only surfaces after a coronal sonorant in (\ref{ex:9:28b}, \ref{ex:9:28c}). Both fricatives in question derive from an etymological velar (\ili{WGmc} \textsuperscript{+}[ɣ]). The palatal in examples like these derives synchronically from /ɣ/ by \isi{Velar Fronting-1}.

\ea%28
\label{ex:9:28}\ipi{Ebsdorf} [ɣ] and [ʝ] (from /ɣ/):
\ea\label{ex:9:28a} mǫɣə \tab [mɔːɣə] \tab Magen \tab ‘stomach’ \tab 33\\
    ɑ̄ɣə \tab [ɑːɣə] \tab Auge \tab ‘eye’ \tab 33
\ex\label{ex:9:28b} sējə \tab [seːʝə] \tab Säge \tab ‘saw’ \tab 33\\
    lęjə \tab [lɛʝə] \tab legen \tab ‘place-\textsc{inf}’ \tab 17
\ex\label{ex:9:28c} mǫrjə \tab [mɔrʝə] \tab morgen \tab ‘tomorrow’ \tab 33
   \z
\z 

Note that \isi{Velar Fronting-1} has a different status depending on the trigger: For /x/ the rule functions as a \isi{neutralization}, but for /ɣ/ it continues to be an allophonic process (as it was for /x/ at Stage 2).

\citet{Knauss1906} describes the \il{Central Hessian}CHes variety spoken in the neighboring localities of \ipi{Atzenhain} and \ipi{Grünberg} (\mapref{map:11}). \ipi{Atzenhain}/\ipi{Grünberg} possesses the front vowels /i eː e ɛ ɛː æ/ and the back vowels /u o ɔː ɑ/. Note the presence of the low front vowel [æ] (<\ili{WGmc} \textsuperscript{+}[e]), which is absent in the \il{Central Hessian}CHes varieties discussed above. [x] (=⟦χ⟧) and [ç] (=⟦c⟧) are phonemic because they contrast after one of the phonemic back vowels ([ɑ]). The only lenis palatal fricative is [ʝ], which appears to have a distribution as in \ipi{Weidenhausen} (see \fnref{fn:9:6}).

In both \ipi{Atzenhain} and \ipi{Grünberg} [x] surfaces after a back vowel which is historically back in (\ref{ex:9:29}), while the opaque palatal [ç] occurs after the back vowel [ɑ] which derived historically from a front vowel (cf. \ili{MHG} [iː]) in (\ref{ex:9:30a}). In \ipi{Grünberg} [ç] also occurs after [ɑː] in (\ref{ex:9:30b}, \ref{ex:9:30c}), whose progenitor was a diphthong whose both components were front. The changes affecting the original vowels in \REF{ex:9:30} are specific examples of \isi{Vowel Retraction}. A sample [x]{\textasciitilde}[ç] alternation in which the stem vowel is back before both sounds is presented in \REF{ex:9:31}. The fricatives ([x ç]) in (\ref{ex:9:29}--\ref{ex:9:31}) derived historically from a velar sound (\ili{WGmc} \textsuperscript{+}[k x]).

\ea%29
\label{ex:9:29}\ipi{Atzenhain}/\ipi{Grünberg} [x] (from /x/):\\
\begin{xlist}
  \sn
buχ \tab [bux] \tab Buch \tab ‘book’ \tab 74\\
loχ \tab [lox] \tab Loch \tab ‘hole’ \tab 58\\
dɑχ \tab [dɑx] \tab Dach \tab ‘roof’ \tab 28\\
ɑ̄χ \tab  [ɑːx] \tab auch \tab ‘also’ \tab 70\\
\end{xlist}
\z 

\ea%30
\label{ex:9:30}\ipi{Atzenhain}/\ipi{Grünberg} [ç] (from /ç/):
\ea\label{ex:9:30a} bɑcd \tab [bɑçt] \tab Beichte \tab ‘confession’ \tab  57 \\
    ɡlɑc \tab [glɑç] \tab gleich \tab ‘same’ \tab 57
\ex\label{ex:9:30b} blɑ̄cə \tab [blɑːçə] \tab bleichen \tab ‘bleach-\textsc{inf}’ \tab  68
\ex\label{ex:9:30c} rɑ̄cn̥ \tab [rɑːçn̩] \tab räuchern \tab ‘smoke-\textsc{inf}’ \tab  68
    \z
\ex%31
\label{ex:9:31}\ipi{Grünberg} [x]{\textasciitilde}[ç] alternations (from /x/ and /ç/):\\

\begin{xlist}
  \sn
rɑ̄χ \tab  [rɑːx] \tab Rauch \tab ‘smoke’ \tab 70\\
rɑ̄cn̥ \tab [rɑːçn̩] \tab räuchern \tab ‘smoke-\textsc{inf}’ \tab  71
\end{xlist}
\z 

[x]{\textasciitilde}[ç] (<\ili{WGmc} \textsuperscript{+}[k x]) alternations triggered by the quality of the preceding vowel (via \isi{Umlaut}) are presented in \REF{ex:9:32}. The palatal in that type of example derives from the velar by the rule of fronting posited below.

\ea%32
\label{ex:9:32}\ipi{Atzenhain}/\ipi{Grünberg} [x]{\textasciitilde}[ç] alternations (from /x/):
\ea\label{ex:9:32a}   buχ \tab [bux] \tab Buch \tab ‘book’ \tab 74\\
      bicər \tab [biçər] \tab Bücher \tab ‘book-\textsc{pl}’ \tab 74
\ex\label{ex:9:32b}   nǭχd \tab [nɔːxt] \tab Nacht \tab ‘night’ \tab 32\\
      nęcd \tab [nɛçt] \tab Nächte \tab ‘night-\textsc{pl}’ \tab 41
\ex\label{ex:9:32c}   dɑχ \tab [dɑx] \tab Dach \tab ‘roof’ \tab 28\\
      dęcr̥ \tab [dɛçr̩] \tab Dächer \tab ‘roof-\textsc{pl}’ \tab 45
\z 
\z 

As indicated in \REF{ex:9:33}, the distribution of dorsal fricatives after front vowels is not the same as in the other \il{Central Hessian}CHes varieties mentioned above: [x] surfaces after the low front vowel [æ] in (\ref{ex:9:33a}); see R. D. \citet[18]{Hall1973} for discussion. By contrast, [ç] occurs after a nonlow front vowel in (\ref{ex:9:33b}) or a coronal sonorant consonant in (\ref{ex:9:33c}). Velar [x] never surfaces after nonlow front vowels, nor does palatal [ç] occur after [æ]. The dorsal fricatives in all of these examples derive from velars (\ili{WGmc}  \textsuperscript{+}[k x]).

\ea%33
\label{ex:9:33}\ipi{Atzenhain}/\ipi{Grünberg} [x] and [ç] (from /x/):
\ea\label{ex:9:33a}   blæχ \tab [blæx] \tab Blech \tab ‘tin’ \tab 47
\ex\label{ex:9:33b}   ɡəsicd \tab [gəsiçt] \tab Gesicht \tab ‘face’ \tab 53\\
      brẹ̄c \tab [breːç] \tab brechen \tab ‘break-\textsc{1sg}’ \tab 52\\
      šdec \tab [ʃdeç] \tab Stiche \tab ‘sting-\textsc{pl}’ \tab 54\\
      šl\={ę}cd \tab [ʃlɛːçt] \tab schlecht \tab ‘bad’ \tab 48\\
      ɑic \tab [ɑiç] \tab ich \tab ‘I’ \tab 56
\ex\label{ex:9:33c}   melc \tab [melç] \tab Milch \tab ‘milk’ \tab 56
\z 
\z 

The data in \REF{ex:9:33} require the set of triggers for fronting to consist of nonlow front vowels. The rule required is \isi{Velar Fronting-2} (\sectref{sec:3.4}), which is reproduced in \REF{ex:9:34}:

\ea%34
\label{ex:9:34}\isi{Velar Fronting-2}:\\
\begin{forest}
[,phantom
  [\avm{[−low]} [\avm{[coronal]},tier=word,name=target]]
  [\avm{[−son\\+cont]},name=parent [\avm{[dorsal]},tier=word]]
]
\draw [dashed] (parent.south) -- (target.north);
\end{forest}
\z 

In a short (four page) summary of his dissertation of 1921, \citet{Siemon1922} describes the \il{Central Hessian}CHes variety of \ipi{Langenselbold}, near \ipi{Hanau} (\mapref{map:11}). The data in that source indicate that \ipi{Langenselbold} possesses front vowels (/i iː eː e ɛ ɛː/), back vowels (/u uː o oː ɔ ɔː ɑ ɑː/) and several diphthongs. Enough crucial examples in \citet{Siemon1922} are provided to conclude that this \il{Central Hessian}CHes variety has both velar [x] (=⟦χ⟧) and palatal [ç] (=⟦c⟧). Those fricatives are both phonemic (=\ref{ex:9:1a}) because they contrast after one of the phonemic back vowels ([ɑː]). (The historical lenis fricatives \ili{WGmc} \textsuperscript{+}[ɣ] and \textsuperscript{+}[ʝ] have merged with their fortis counterparts).

The data from \ipi{Langenselbold} presented in (\ref{ex:9:35}--\ref{ex:9:38}) are very similar to the examples in the neighboring \il{Central Hessian}CHes varieties discussed earlier. The words in \REF{ex:9:35} indicate that [x] surfaces after back vowels that are historically back. The two examples in \REF{ex:9:36} reveal that [ç] surfaces after a back vowel ([ɑː]) which was etymologically a diphthong ending in a front vowel. Note that [x] and [ç] contrast in the context after [ɑː], e.g. [ɑːx] ‘also’ vs. [vɑːç] ‘soft’; hence, they are both phonemic, as indicated in the headings for the two datasets. The [x ç] in all of the examples presented below derived historically from a velar sound (\ili{WGmc} \textsuperscript{+}[k x]).\largerpage[-2]

\ea%35
\label{ex:9:35}\ipi{Langenselbold} [x] (from /x/):\\
    \begin{xlist}
    \sn
    hūx    \tab [huːx] \tab hoch \tab ‘high’ \tab 140\\
    wuxə   \tab  [vuxə] \tab Woche \tab ‘week’ \tab 142\\
    nǭxd  \tab  [nɔːxt] \tab Nacht \tab ‘night’ \tab 140\\
    kǫxə   \tab  [kɔxə] \tab kochen \tab ‘cook-\textsc{inf}’ \tab 142\\
    bɑx    \tab [bɑx] \tab Bach \tab ‘stream’ \tab 139\\
    ɑ̄x     \tab [ɑːx] \tab auch \tab ‘also’ \tab 140\\
    flǫuxə \tab [flɔuxə] \tab fluchen \tab ‘curse-\textsc{inf}’ \tab 140
    \end{xlist} 
\z
\pagebreak
\ea%36
\label{ex:9:36}\ipi{Langenselbold} [ç] (/from /ç/):\\
    \begin{xlist}
    \sn
    ɡlɑ̄χə \tab  [glɑːçə] \tab gleichen \tab ‘resemble-\textsc{inf}’ \tab 140\\
    wɑ̄χ \tab [vɑːç] \tab weich \tab ‘soft’ \tab 140
    \end{xlist}
\z 

The additional data reveal that there are morphemes with [x]{\textasciitilde}[ç] alternations in (\ref{ex:9:37}) as well as nonalternating words in which [ç] surfaces after a front vowel or coronal sonorant consonant in (\ref{ex:9:38}).

\ea%37
\label{ex:9:37}\ipi{Langenselbold} [x]{\textasciitilde}[ç] alternations (/from /x/):
\ea\label{ex:9:37a}  fuxəl \tab [fuxəl] \tab Vogel \tab ‘bird’ \tab 140
\ex\label{ex:9:37b}  fiχəl \tab [fiçəl] \tab Vögel \tab ‘bird-\textsc{pl}’ \tab 139
\z 
\ex%38
\label{ex:9:38}\ipi{Langenselbold} [ç] (/from /x/):
\ea\label{ex:9:38a}  liχd \tab [liçt] \tab Licht \tab ‘light’ \tab 139\\
     knē̜χd \tab [knɛːçt] \tab Knecht \tab ‘vassal’ \tab 141\\
     šbręχə \tab [ʃprɛçə] \tab sprechen \tab ‘speak-\textsc{inf}’ \tab 142\\
     ɑiχ \tab [ɑiç] \tab ich \tab ‘I’ \tab 142
\ex\label{ex:9:38b}  kę\textsuperscript{r}χ \tab [kɛrç] \tab Kirche \tab ‘church’ \tab 139
\z 
\z 

As indicated in the headings for \REF{ex:9:37} and \REF{ex:9:38}, the dorsal fricatives in these words are underlyingly /x/. That sound is realized as [ç] after a coronal sonorant by \isi{Velar Fronting-1}.

The five places discussed above are very different from other \il{Central Hessian}CHes varieties in which velar and palatal fricatives do not contrast. For example, in \ipi{Naunheim} (\citealt{Leidolf1891}; \mapref{map:11}) [x] and [ç] stand in complementary distribution: [x] only surfaces after a back vowel, e.g. [tsʊxt] ‘breeding’ (=⟦tsŭcd⟧) and [ç] after a front vowel, e.g. [dɪçt] ‘tight’ (=⟦dĭçd⟧). The reason [ç] does not surface after back vowels is that \isi{Vowel Retraction} did not occur, cf. \ipi{Naunheim} [blɑiç] ‘pale’ (=⟦blā\textsuperscript{j}ç⟧; recall \ref{ex:9:5a}), [loiçtə] ‘light’ (=⟦loiçdə⟧; recall \ref{ex:9:5e}). Examples like [ʃlæçt] ‘bad’ (=⟦šlæçd⟧) indicate that the triggers for \isi{Velar Fronting-1} in \ipi{Naunheim} subsume all front vowels and not simply nonlow front vowels as in \ipi{Atzenhain}\slash\ipi{Grünberg}. A \il{Central Hessian}CHes dialect in closer proximity to the four velar vs. palatal contrasting varieties discussed above is the one spoken in \ipi{Schlierbach} (\citealt{Schaefer1907}; \mapref{map:11}). As in \ipi{Naunheim}, no \isi{Vowel Retraction} occurred and hence there are no contrasts between velars and palatals, which stand in complementary distribution.\il{Central Hessian|)}

\section{{Rhenish} {Franconian}}\label{sec:9.3}\il{Rhenish Franconian|(}

Two varieties of \il{Rhenish Franconian}RFr are discussed below which exhibit Contrast Type B (=\ref{ex:9:1}a) in postsonorant position between [x] (/x/) and [ç] (/ç/). Since the sources have data very similar to the ones presented in \sectref{sec:9.2} for \il{Central Hessian}CHes I do not discuss the \il{Rhenish Franconian}RFr material in as much detail as the \il{Central Hessian}CHes varieties.

\citet{Freiling1929} describes the variety of \ipi{Zell im Mümlingtal} in the Oldenwald (\mapref{map:10}). \ipi{Zell im Mümlingtal} has a number of phonemic front vowels (/iː i eː e ɛː ɛ/), phonemic back vowels (/uː u oː o ɔː ɔ ɑː ɑ/) as well as several diphthongs. A representative dataset is presented in \REF{ex:9:39}. The words in \REF{ex:9:39a} indicate that [x] surfaces after a back vowel that is etymologically back. The items presented in \REF{ex:9:39b} show that [ç] surfaces after the one back vowel [ɑː], which derived historically from a diphthong ending in a front vowel (cf. \ili{MHG} [ei]). As in the \il{Central Hessian}CHes varieties discussed above, [x] and [ç] contrast in the context after the back vowel [ɑː]; hence, [x] (=⟦x⟧) and [ç] (=⟦χ⟧) are both phonemic and illustrate Contrast Type B. (As in \ipi{Langenselbold}, historical [ɣ] and [ʝ] have merged with their fortis counterparts). The items listed in \REF{ex:9:39c} show that there are [x]{\textasciitilde}[ç] alternations, and the data in \REF{ex:9:39d} reveal that [ç] -- but never [x] -- surfaces after a front segment. The dorsal fricatives in (\ref{ex:9:39c}, \ref{ex:9:39d}) is underlyingly /x/ and is realized as [ç] after any front segment by \isi{Velar Fronting-1}.

\ea%39
\label{ex:9:39}\ipi{Zell im Mümlingtal} [x] and [ç]:
\ea\label{ex:9:39a} wux \tab [vux] \tab Woche \tab ‘week’ \tab 75\\
    koxə\textsuperscript{} \tab [koxə] \tab kochen \tab ‘cook-\textsc{inf}’ \tab 75\\
    nǫxd \tab [nɔxt] \tab Nacht \tab ‘night’ \tab 10\\
    lɑxə \tab [lɑxə] \tab lachen \tab ‘laugh-\textsc{inf}’ \tab 75\\
    rɑ̄x \tab [rɑːx] \tab Rauch \tab ‘smoke’ \tab 35\\
    ɑ̄xə \tab [ɑːxə] \tab Augen \tab ‘eye-\textsc{pl}’ \tab 35\\
\ex\label{ex:9:39b} ɑ̄χ \tab  [ɑːç] \tab Eiche \tab ‘oak tree’ \tab 33\\
    wɑ̄χ \tab [vɑːç] \tab weich \tab ‘soft’ \tab 33\\
\ex\label{ex:9:39c} nǫxd \tab [nɔxt] \tab Nacht \tab ‘night’ \tab 10\\
    nęχd \tab [nɛçt] \tab Nächte \tab ‘night-\textsc{pl}’ \tab 12
\ex\label{ex:9:39d} siχə\textsuperscript{r} \tab [siçər] \tab sicher \tab ‘certainly’ \tab 74\\
    beχ \tab [beç] \tab Bäche \tab ‘stream-\textsc{pl}’ \tab 74\\
    ɡnē̜χd \tab [knɛːçt] \tab Knecht \tab ‘vassal’ \tab 16
   \z
\z

\citet{Seibt1930} describes the dialect of \ipi{Heppenheim} (\mapref{map:10}). He lists nineteen monophthongs, but it is probably not the case that all of those sounds are phonemic. On the basis of that source, \ipi{Heppenheim} has phonemic front vowels (/iː i eː e ɛː ɛ æ/) and back vowels (/uː u oː o ɔː ɔ ɑː ɑ/) as well as several diphthongs. \ipi{Heppenheim} has the four dorsal fricatives [x ç ɣ ʝ]. [x] (=⟦x⟧) and [ç] (=⟦χ⟧) are phonemic because they contrast after certain back vowels, whereas [ɣ] and [ʝ] stand in complementary distribution; see \REF{ex:9:23}.

The examples in \REF{ex:9:40a} reveal that [x] surfaces after back vowels that are etymologically back. [ç] surfaces after the one back vowel [ɑː] in (\ref{ex:9:40b}), which derives historically from a diphthong ending in a front vowel. Since [x] and [ç] contrast after [ɑː] those two fricatives are phonemic. A representative example of a morpheme exhibiting [x]{\textasciitilde}[ç] alternations is given in \REF{ex:9:40c}, and the words in (\ref{ex:9:40d}, \ref{ex:9:40e}) show that the palatal but never the velar occurs after coronal sonorants. The final set of examples indicates that the lenis dorsal fricative [ɣ] (/ɣ/) surfaces after a front vowel in (\ref{ex:9:40f}) or back vowel in (\ref{ex:9:40g}).

\ea%40
\label{ex:9:40}\ipi{Heppenheim} [x] (from /x/):
\ea\label{ex:9:40a} bux \tab [bux] \tab Buch \tab ‘book’ \tab 30\\
    doxdə\textsuperscript{r} \tab [doxtər] \tab Tochter \tab ‘daughter’ \tab 58\\
    ǫxd \tab [ɔxt] \tab acht \tab ‘eight’ \tab 58\\
    lɑxə \tab [lɑxə] \tab lachen \tab ‘laugh-\textsc{inf}’ \tab 58\\
    rɑ̄x \tab [rɑːx] \tab Rauch \tab ‘smoke’ \tab 33\\
\ex\label{ex:9:40b} sɑ̄χə \tab [sɑːçə] \tab seichen \tab ‘piss-\textsc{inf}’ \tab 32\\
\ex\label{ex:9:40c} nǭxd \tab [nɔːxt] \tab Nacht \tab ‘night’ \tab 68\\
    nē̜χd \tab [nɛːçt] \tab Nächte \tab ‘night-\textsc{pl}’ \tab 68
\ex\label{ex:9:40d} khiχ \tab [kʰiç] \tab Küche \tab ‘kitchen’ \tab 30\\
    šlē̜χd \tab [ʃlɛːçt] \tab schlecht \tab ‘bad’ \tab 58\\
    fęχdə \tab [fɛçtə] \tab fechten \tab ‘fence-\textsc{inf}’ \tab 19\\
    rɑiχ \tab [rɑiç] \tab reich \tab ‘rich’ \tab 57
\ex\label{ex:9:40e} fęrχdə \tab [fɛrçtə] \tab fürchten \tab ‘fear-\textsc{inf}’ \tab 45
\ex\label{ex:9:40f} fę̄γə \tab [fɛːɣə] \tab fegen \tab ‘sweep-\textsc{inf}’ \tab 56\\
    šdɑiγə \tab [ʃtɑiɣə] \tab steigen \tab ‘climb-\textsc{inf}’ \tab 56
\ex\label{ex:9:40g} foγl \tab [foɣl̩] \tab Vogel \tab ‘bird’ \tab 56\\
    nɑγl \tab [nɑɣl̩] \tab Nagel \tab ‘nail’ \tab 56
    \z
\z 

The dorsal fricatives in (\ref{ex:9:40c}--\ref{ex:9:40e}) are underlyingly /x/, which surfaces as [ç] after a coronal sonorant. The target segment must be specified as [+fortis] to ensure that only /x/ but not /ɣ/ is affected; hence, the rule for \ipi{Heppenheim} is \isi{Velar Fronting-4} (\sectref{sec:4.3}, \sectref{sec:7.2}).\il{Rhenish Franconian|)}

\section{Discussion}\label{sec:9.4}

I discuss first the status of Contrast Type A and Contrast Type C systems attested in word-initial position (\chapref{sec:8}) for postsonorant position (\sectref{sec:9.4.1}) and second the question of whether or not the quasi-phonemicization of palatals is a necessary prerequisite for the \isi{phonemicization} of palatals (\sectref{sec:9.4.2}).

\subsection{Velar vs. palatal contrasts}\label{sec:9.4.1}

All of the case studies discussed in this chapter have in common that they exemplify Contrast Type B (=\ref{ex:9:1}), which involves a palatal vs. velar contrast after one or more back vowel, but in the context of front vowels, only palatals surface. The present survey of German dialects has failed to uncover Contrast Type A or Contrast Type C (as described in \chapref{sec:8}) in postsonorant position, as in \REF{ex:9:41}:

\ea \label{ex:9:41}  Nonoccurring contrasts:
\ea Contrast Type A:\label{ex:9:41a}\smallskip\\

\fbox{
\begin{tabular}[t]{@{}ll@{}}
\relax [...iç...] & [...ɑç...]\\
\relax [...ix...] & [...ɑx...]\\
\end{tabular}
}
\fbox{
\begin{tabular}[t]{@{}ll@{}}
\relax [...iʝ...] & [...ɑʝ...]\\
\relax [...iɣ...] & [...ɑɣ...]\\
\end{tabular}
}
\ex Contrast Type C:\label{ex:9:41b}\smallskip\\
\fbox{
\begin{tabular}[t]{@{}ll@{}}
\relax [...iç...] &            \\
\relax [...ix...] & [...ɑx...] \\
\end{tabular}
}
\fbox{
\begin{tabular}[t]{@{}ll@{}}
\relax [...iʝ...] & \\
\relax [...iɣ...] & [...ɑɣ...]\\
\end{tabular}
}
  \z
\z 

In \REF{ex:9:41a} velars and palatals contrast after back vowels and front vowels, but in \REF{ex:9:41b} that contrast occurs only after front vowels but not after back vowels, where only the velar surfaces. I speculate here on the absence of the two systems depicted in \REF{ex:9:41}.

Consider first \REF{ex:9:41a}. There is more than one way in which a system involving a contrast between [ɣ] and [ʝ] after front and back vowels might arise. One way would require the following developments: (a) Etymological \textsuperscript{+}[ɣ] is inherited without change as [ɣ] after a back vowel, (b) etymological \textsuperscript{+}[ɣ] surfaces as [ʝ] after a back vowel derived from an earlier front vowel (by \isi{Vowel Retraction}), and (c) \ili{WGmc} \textsuperscript{+}[j] undergoes \isi{Glide Hardening} in a word-internal onset after front vowels and back vowels. Recall from \sectref{sec:9.2} that (a) and (b) are well-attested, e.g. in \ipi{Wissenbach} examples \REF{ex:9:11} and \REF{ex:9:12}. That point aside, it is difficult to find examples for (c) because the \isi{etymological palatal} glide was typically either deleted in postvocalic position, or it merged together with the preceding vowel to form a diphthong (Appendix~\ref{appendix:f}).

Consider now \REF{ex:9:41b}. Recall from \chapref{sec:8} that the mirror-image of \REF{ex:9:41b} involving [x] and [ç] in word-initial position is attested in a single village. In that place the velar vs. palatal contrast before a front vowel arose when \isi{r-Deletion} eliminated the /r/ between a word-initial velar (/x/) and a front vowel. A deletion process affecting a postvocalic /r/ is attested in German dialects (e.g. in the \il{Rhenish Franconian}RFr varieties discussed by \citealt{Karch1981}; see \sectref{sec:9.5}). If velar fronting applies after front vowels but not after coronal consonants, and if /r/ were elided between any vowel (including front vowels) and velar sounds (including /x/), then the surface sequence of front vowel plus velar fricative ([ix] /ix/) would be created, e.g. a sequence like [ɪrx] (/ɪrx/) > [ɪx] (/ɪx/) in a word like \textit{Kirche} (cf. \ipi{Erdmannsweiler} [kʰerç] /kʰerx/ from \sectref{sec:3.2}). Although the deletion of a postvocalic /r/ is not at all uncommon in German dialects the scenario just described would be difficult to document because only a small number of German dialects restrict velar fronting to the context after front vowels but not after coronal consonants like /r/ (see \chapref{sec:12}).

\subsection{Relationship between phonemic palatals and palatal quasi-phonemes}\label{sec:9.4.2}

The dialects discussed in \chapref{sec:7} all have in common that a Stage 2 allophonic rule of velar fronting developed into a Stage 3 system with a \is{palatal quasi-phoneme}palatal qua\-si-pho\-neme, but none of those dialects also possess phonemic palatals. The question is whether or not the quasi-phonemicization of palatals is a necessary prerequisite for the \isi{phonemicization} of palatals; see \REF{ex:9:42a}. Recall from \sectref{sec:7.4.4} that this is the historical progression predicted by \citet{Kiparsky2015}. Alternatively, quasi-phonemicization and \isi{phonemicization} might not be directly related, in which case a system involving allophony could develop into either one, as depicted in \REF{ex:9:42b}.

\ea%42
    \label{ex:9:42}
\ea \label{ex:9:42a}  
    \hspace*{1.8cm}\begin{forest} for tree = {edge=->}
      [Allophony  
        [Quasi-phonemicization 
          [Phonemicization]
        ]
      ]       
      \end{forest}                  
\ex \label{ex:9:42b}  
     \begin{forest} for tree = {edge=->}
      [Allophony
        [Quasi-phonemicization]
        [Phonemicization]
      ]
      \end{forest}
\z 
\z 

Most of the dialects discussed in this book with phonemic palatals also possess palatal quasi-phonemes, a system that can be accommodated with either \REF{ex:9:42a} or \REF{ex:9:42b}. This is true for word-initial position in \ipi{Eilsdorf} (\sectref{sec:8.3}) and \ipi{Dingelstedt am Huy} (\sectref{sec:8.4}) as well as in the LG varieties discussed below in \chapref{sec:11}.

I tentatively suggest that \REF{ex:9:42b} is the correct path. The reason \REF{ex:9:42a} cannot always be correct is that there is at least one example of a dialect with phonemic palatals but no palatal quasi-phonemes, namely the \il{Central Hessian}CHes dialect of \ipi{Wissenbach} (\sectref{sec:9.2}). One could speculate that \ipi{Wissenbach} once had a \isi{palatal quasi-phoneme} before the velar vs. palatal contrasts emerged and that the original \isi{palatal quasi-phoneme} fell together with the new contrastive palatals, thereby obscuring its historical origin. That scenario is a plausible one, and for that reason I ultimately leave open for further research whether or not \REF{ex:9:42b} is the correct path.

\section{Areal distribution of postsonorant phonemic palatals}\label{sec:9.5}

The case studies discussed in this chapter have in common that they contrast velars and palatals in postsonorant position. I consider below the additional dialects known to me with this contrast. I discuss first those varieties of German spoken in Germany (\sectref{sec:9.5.1}), and then I turn to two velar fronting islands (\sectref{sec:9.5.2}).

\subsection{Germany}\label{sec:9.5.1}

Sources documenting a contrast between postvocalic /x/ and /ç/ in dialects spoken in Germany are listed in \tabref{tab:9.1}.

\begin{table}
\caption{Varieties of German with phonemic palatals (< \ili{WGmc} +[k~x]) in postsonorant position\label{tab:9.1}}
\begin{tabular}{lll}
\lsptoprule
Place & Dialect & Source\\\midrule
\ipi{Großen-Buseck} & \il{Central Hessian}CHes & \citet{WagnerHorn1900}\\
\ipi{Atzenhain}/\ipi{Grünberg} & \il{Central Hessian}CHes & \citet{Knauss1906}\\
\ipi{Wissenbach} & \il{Central Hessian}CHes & \citet{Kroh1915}\\
\ipi{Selters bei Weilburg} & \il{Central Hessian}CHes & \citet{Schwing1921}\\
\ipi{Langenselbold} & \il{Central Hessian}CHes & \citet{Siemon1922}\\
\ipi{Wetterfeld} & \il{Central Hessian}CHes & \citet{Schudt1927}\\
\ipi{Ebsdorf} & \il{Central Hessian}CHes & \citet{Bender1938}\\
\ipi{Weidenhausen} & \il{Central Hessian}CHes & \citet{Friebertshäuser1961}\\
Mittelhessisch & \il{Central Hessian}CHes & \citet{Hasselberg1979}\\
\ipi{Ober-Flörsheim} & \il{Rhenish Franconian}RFr & \citet{Haster1908}\\
\ipi{Kaulbach} & \il{Rhenish Franconian}RFr & \citet{Christmann1927}\\
\ipi{Zell im Mümlingtal} & \il{Rhenish Franconian}RFr & \citet{Freiling1929}\\
\ipi{Heppenheim} & \il{Rhenish Franconian}RFr & \citet{Seibt1930}\\
Area south of \ipi{Mainz} & \il{Rhenish Franconian}RFr & \citet{Karch1981}\\
\ipi{Merzig} & \il{Moselle Franconian}MFr & \citet{Fuchs1903}\\
\ipi{Dudenrode} & \il{Thuringian}Thrn & \citet{Guentherodt1982}\\
\ipi{Neuendorf} & \il{Eastphalian}Eph & \citet{Schütze1953}\\
\lspbottomrule
\end{tabular}
\end{table}

\tabref{tab:9.1} includes the seven \il{Central Hessian}CHes/\il{Rhenish Franconian}RFr case studies discussed above as well as works not discussed earlier, which I comment on below.  All of these places are listed on the maps for the respective dialect areas.

In their discussion of the inflectional morphology of verbs in \ipi{Großen-Buseck}, \citet{WagnerHorn1900} list examples like [ʃlɑçǝ] ‘creep-\textsc{inf}’ (cf. \il{Standard German}StG \textit{schleichen}) vs. [ʃtrɑçǝ] ‘paint-\textsc{inf}’ (cf. \il{Standard German}StG \textit{streichen}) with [ç] after the back vowel [ɑ] that derived historically from [ɑi]. Significantly, they also include items like [mɑxǝ] ‘do-\textsc{inf}’ (cf. \il{Standard German}StG \textit{machen}), where [x] occurs after [ɑ]. In a short excerpt from his dissertation, \citet{Schwing1921} describes the historical phonology of the (\il{Central Hessian}CHes) area around \ipi{Selters bei Weilburg}, noting the existence of contrasts between [x] and [ç] in words like [ɑːxə] ‘eye’ (cf. \il{Standard German}StG \textit{Auge}) vs. [tsɑːçələ] ‘draw-\textsc{inf}’ (cf. \il{Standard German}StG \textit{zeichnen}). The same type of contrast can be found in the material presented in \citet{Schudt1927} for \ipi{Wetterfeld}, e.g. [ɑːx] ‘also’ (cf. \il{Standard German}StG \textit{auch}) vs. [blɑːç] ‘pale’ (cf. \il{Standard German}StG \textit{bleich}), as well as in \citet{Christmann1927} for \ipi{Kaulbach}, e.g. [rɑːxə] ‘smoke-\textsc{inf}’ (cf. \il{Standard German}StG \textit{rauchen}) vs. [rɑːçə] ‘be sufficient-\textsc{inf}’ (cf. \il{Standard German}StG \textit{reichen}) and \citet{Haster1908} for \ipi{Ober-Flörsheim}, e.g. [rɑːxən] ‘smoke-\textsc{inf}’ vs. [rɑːçən] ‘be sufficient-\textsc{inf}’. \ipi{Merzig} \citep{Fuchs1903} is geographically further removed from the others varieties listed in \tabref{tab:9.1}. Like the dialects listed above, historical [ei] is now realized as [ɑː] in \ipi{Merzig}; hence, there are contrastive pairs like [rɑːxən] ‘smoke-\textsc{inf}’ (cf. \il{Standard German}StG \textit{rauchen}) vs. [blɑːçən] ‘bleach-\textsc{inf}’ (cf. \il{Standard German}StG \textit{bleichen}). Similar examples involving a contrast between [x] and [ç] after the same back vowel can be found in the data in \citet{Hasselberg1979}, which were drawn from a number of places in \ipi{Central Hesse}. \citet{Karch1981} is the description of the sound structure of five towns just south of \ipi{Mainz}, namely \ipi{Wackernheim}, \ipi{Nackenheim}, \ipi{Alzey}, \ipi{Wallertheim}, and Bechtheim. \citet[23]{Karch1981} writes that /x/ and /ç/ must be separate phonemes because they contrast after certain back vowels, e.g. [dɑx] ‘roof’ (cf. \il{Standard German}StG \textit{Dach}) vs. [dɑç] ‘through’ (cf. \il{Standard German}StG \textit{durch}). In contrast to all of the other studies listed in \tabref{tab:9.1}, phonemic /ç/  arose when a postsonorant rhotic deleted, cf. [dɑç] ‘through’ < [dʊrç]. The original source for the one ECG dialect listed above \citep{Guentherodt1982} provides phonetic transcriptions for three speakers from \ipi{Dudenrode} and observes (p. 46) that [ç] and [x] contrast after the one low vowel (short and long), e.g. [ʃlaxt-] ‘slaughter-\textsc{vb} \textsc{stem}’ (cf. \il{Standard German}StG \textit{schlachten}) vs. [ʃlaçt] ‘bad’ (cf. \il{Standard German}StG \textit{schlecht}). The status of velars and palatals in word-initial position in the one LG variety (\il{Eastphalian}Eph) cited above (\ipi{Neuendorf}) was discussed in \sectref{sec:8.5}. The original source for that dialect \citep{Schütze1953} gives examples of contrasts between velar [x] and palatal [ç] in the context after [ɑː], e.g. [dɑːç] ‘dough’ (cf. \il{Standard German}StG \textit{Teig}) vs. [plɑːx] ‘plow’ (cf. \il{Standard German}StG \textit{Pflug}).

The places listed above with a palatal vs. velar contrast can be complemented with data from linguistic atlases. Consider the following two examples:

Map 4 of ThürDA depicts the various realizations of the word \textit{Egge} ‘harrow’ in the state of Thuringia. An examination of that map reveals that there is a small part of west Thuringia with a palatal fricative ([ç] or [ʝ]) after the back vowel [ɑː] -- a point that is stressed with an exclamation point after the back vowel plus palatal sequence in the commentary to Map 4 in Volume 1 (p. 32). Since [x] surfaces after back vowels (including [ɑː]) throughout the area, those places with words containing [ɑːç] illustrate a contrast between [ç] and [x].

A second example for the palatal vs. velar contrast comes from SchlSA, which depicts an area far removed spatially from the places listed in \tabref{tab:9.1}, namely the former province of \ipi{Silesia} (Schlesien). Map 26 from that source depicts the realizations of the word \textit{leuchten} ‘glow-\textsc{inf}’. The initial vowel in that word (<~\ili{MHG} [yː]) is either a front monophthong or a diphthong ending in [i] in (\ref{ex:9:43a}) or a back monophthong in (\ref{ex:9:43b}). Significantly, the fricative in \REF{ex:9:43b} is always realized as palatal [ç]. As illustrated on my \mapref{map:9}, velar fronting after coronal sonorants (or a subset thereof) is the norm throughout \ipi{Silesia} (\sectref{sec:12.3.5}).\footnote{{The symbol [a] in SchlSA is categorized as central (p. 5). On p. 13 of the introduction, G. Bellmann comments on how remarkable (“[b]emerkenswertˮ) it is that [ç] occurs after a back vowel in items like the ones in \REF{ex:9:43b}. The realization  [lɔçd̥n̩] was attested just to the east of Grunlich and the variant [lɔçd̥a] about 70km southwest of Gleiwitz (see my \mapref{map:9}). By contrast, the markers indicating [laçd̥n̩] are much more numerous, being interspersed with the transparent realizations in \REF{ex:9:43a} in a broad area in between Görlitz and Breslau.}}

\ea%43
\label{ex:9:43}
\ea\label{ex:9:43a}\relax  [lɛçd̥n̩], [luiçd̥n̩], [lɔiçd̥n̩]
\ex\label{ex:9:43b}\relax  [laçd̥n̩], [lɔçd̥n̩]
\z 
\z 

Although the SchlSA does not provide a map with [x] after back vowels, it is clear from all of the descriptions of \il{Silesian}Sln dialects I have consulted (Appendix~\ref{appendix:c}, \tabref{tab:appendix:c19}) that words of that structure are common; hence, the places in \ipi{Silesia} where \REF{ex:9:43b} were once attested can be safely assumed to be areas where [ç] and [x] contrasted after back vowels.

Most of the places listed in \tabref{tab:9.1} are \il{Central Hessian}CHes varieties situated within the same general vicinity in the German state of Hesse, although a few of the \il{Rhenish Franconian}RFr/\il{Moselle Franconian}MFr outliers and the \il{Silesian}Sln varieties in \REF{ex:9:43b} indicate that contrasts between [x] and [ç] are not restricted to that specific area. No other areas with phonemic /x/ and /ç/ in German-speaking countries are known to the present writer. 

\mapref{map:15} depicts all of the places listed in \tabref{tab:9.1} as well as those \il{Silesian}Sln varieties with the pronunciations listed in \REF{ex:9:43b}.

\begin{map}
% \includegraphics[width=\textwidth]{figures/VelarFrontingHall2021-img021.png}
\includegraphics[width=\textwidth]{figures/Map15_9.1.pdf}
 \caption[Areal distribution of postsonorant velar vs. palatal contrasts]{Areal distribution of postsonorant velar vs. palatal contrasts. High German (Central German) and Low German (Eastphalian) varieties with a contrast between a fortis velar [x] (/x/) and a fortis palatal [ç] (/ç/) (< \ili{WGmc} \textsuperscript{+}[k] or \textsuperscript{+}[x]) after a back vowel are indicated with white squares.}\label{map:15}
\end{map}

The contrast between [x] and [ç] in the context after back vowels is also documented in dialect dictionaries. A case in point is SHesWb for the south part of Hesse, which provides phonetic transcriptions with separate symbols for velars (⟦x⟧=[x]) and palatals (⟦χ⟧=[ç]). In SHesWb, multiple phonetic transcriptions corresponding to specific places in the broad region are provided for any given word. The regular pattern whereby [x] occurs after back vowels and [ç] after front vowels and liquids is clear from many common words, e.g. \textit{Loch} ‘hole’, \textit{Licht} ‘light’, \textit{Dolch} ‘dagger’. The important point is that words like the ones discussed earlier which contain an etymological front vowel now realized as back are transcribed with the symbol for the palatal fricative, e.g. \textit{bleich} ‘pale’ (⟦blɑχ⟧), \textit{Deich} ‘dike’ (⟦dɑχ⟧).

\subsection{Velar fronting islands}\label{sec:9.5.2}

Contrasts between /x/ ([x]) and /ç/ ([ç]) after the same back vowel are also attested in two German-language islands, namely \ili{Plautdietsch} and \ili{Transylvania Saxon}. Both illustrate the notion of a \isi{velar fronting island} because velar fronting is active, as in the dialects discussed in \sectref{sec:9.2} and \sectref{sec:9.3}. I provide below some brief discussion of the two aforementioned German language islands.\footnote{There are probably additional German-language islands that could be added to this list. A lesser-known example is mentioned here: \citet{SokolskajaSinder1930} investigate a CHes colony consisting of six villages in North Ukraine which was founded in the eighteenth century. Their data show that the ich-Laut and the ach-Laut are close to being positional variants with the exception of the context after [ɑː], in which case the palatal ([ç]=⟦χ⟧) can occur, e.g. ⟦vɑ̄χ⟧ ‘soft' (cf. StG \textit{weich}.)}

\subsubsection{Plautdietsch}\il{Plautdietsch|(}

Plautdietsch (also known as \ili{Mennonite Low German}) is a LPr variety spoken by the ancestors of the people who emigrated from West Prussia to Russia and Ukraine beginning at the end of the eighteenth century \citep{Siemens2012}. It is currently spoken in a number of countries in Europe (e.g. Russia, Ukraine), Asia (e.g. Russia, Kazakhstan), North America (e.g. Canada, the United States, Mexico), and South America (e.g. Brazil, Argentina). There is an extensive body of research documenting the varieties of Plautdietsch. I do not attempt to summarize that research here; the interested reader is referred to \citet{Siemens2012} and \citet{CoxTucker2013}. For discussion of Plautdietsch in the larger context of LPr see see \sectref{sec:11.6}.

The sources for Plautdietsch I have consulted agree that there are two fortis dorsal fricatives ([x] and [ç]) and that those two sounds contrast in the context after certain back vowels. Some authors say explicitly that the two fricatives in question are phonemes, while others imply that this is the case with the examples cited. See, for example, \citet{Quiring1928} for \ipi{Chortitza} (South Russia), \citet{Goerzen1952}, \citet{Lehn1957}, and \citet{CoxTucker2013} for Canada, \citet{Mierau1964} for Indiana (USA), \citet{Moelleken1966} and \citet{Brandt1992} for Mexico, \citet{Jedig1966} and \citet{Nieuweboer1999} for the Altai region between Russia and Kazakhstan, and \citet{teVeldeVosburg2021} for Kansas and Oklahoma (USA). \citet{Loewen1988} discusses the phonemes of Plautdietsch and their connection to the orthography, while \citet{Naiditch2005} and \citet{Siemens2012} give diachronically-oriented descriptions of Plautdietsch in general. Most of those sources also provide examples from the inflectional morphology in which [x] and [ç] alternate, meaning that some version of velar fronting applies synchronically as a rule of neutralization.\footnote{The historical process of velar fronting in Plautdietsch is usually referred to in the literature as “Palatalization”. In my view, that sound change fronted first the velar fricatives (/x ɣ/) and then later on the velar stops (/k g/) and the velar nasal (/ŋ/). By contrast, \citet[92--98]{Siemens2012} apparently does not consider the fronting of /x ɣ/ to fall within the domain of (Velar) Palatalization, since he does not mention those fricatives in his discussion of that sound change. See \chapref{sec:11} for discussion of German dialects like Plautdietsch with a broad set of velar fronting targets. The phonemicization of /ç/ in Plautdietsch was the direct result of a change fronting a back vowel (see below for discussion). That same change resulted in the phonemicization of palatal stops (/c ɉ/) and the palatal nasal (/ɲ/) (see \chapref{sec:11}.)}

As a representative example, I consider the patterning of velar [x] and palatal [ç] in Chortitza.  According to the copious data provided in the original source \citep{Quiring1928}, only [ç] (=⟦χ⟧) occurs after coronal sonorants (e.g. ⟦liχt⟧ ‘light', ⟦treχtɑ⟧ ‘funnel’, ⟦horχst⟧ ‘hark-\textsc{2sg}'), and  only [x] (=⟦x⟧) surfaces after back vowels (with the exception of [ɑ]), e.g. ⟦doxt⟧ ‘think-\textsc{pret}. In the context after [ɑ] the two fricatives contrast: ⟦ɑxt⟧ ‘eight', ⟦lɑxən⟧ ‘laugh-\textsc{inf}’, ⟦nɑxt⟧ ‘night' vs. ⟦blɑχ⟧ ‘tin’, ⟦frɑχ⟧ ‘impudent’, ⟦knɑχt⟧ ‘knecht’, ⟦šlɑχt⟧ ‘bad’. The reason for the occurrence of the ich-Laut after [ɑ] is that that vowel was originally [e]; recall the examples from Wissenbach in \REF{ex:9:4} and \REF{ex:9:5} above. In the section on inflectional morphology, Quiring provides a number of examples in which [x] and [ç] alternate within the same paradigm, e.g. ⟦liχt⟧ ‘lie-\textsc{3sg}’ vs. ⟦lɑx⟧ ‘lie-\textsc{pret}’. That type of example requires an underlying velar which shifts to palatal after a front vowel. (In this case the underlying velar is /ɣ/, as in the case studies described in \sectref{sec:9.2}). As in  Weidenbach, Chortitz also possesses alternating examples where [x] and [ç] both occur after a back vowel ([ɑ]), e.g. ⟦lox⟧ ‘hole’ vs. ⟦lɑχɑ⟧ ‘hole-\textsc{pl}'; recall \REF{ex:9:20}.{\interfootnotelinepenalty=10000\footnote{\citet{Buchheit1978} describes the variety of Plautdietsch spoken in parts of Nebraska (USA). In contrast to all of the works on Plautdietsch cited above, \citet[73]{Buchheit1978} contends that [x] and [ç] are allophones of the same phoneme (/x/) because they stand in complementary distribution: [x] after back vowels and [ç] after front vowels.}}\il{Plautdietsch|)}

\subsubsection{Transylvania Saxon}\largerpage\il{Transylvania Saxon|(}

\ili{Transylvania Saxon} (Siebenbürgisch-Sächsich) is the traditional name for the German dialect spoken in \ipi{Transylvania} (Siebenbürgen), which is a large region in Central Romania (see \mapref{map:9.2}). Despite its name, Transylvania Saxon is a MFr dialect, which is spoken by the descendants of emigrants from modern-day West Central Germany beginning in the twelfth century.\pagebreak

\begin{map}
\includegraphics[width=\textwidth]{figures/Map15a_9.2.pdf}
\caption[German-language islands in Romania.]{German-language islands in Romania. Source: Wikipedia.\label{map:9.2}}
\end{map}

There is no question that Transylvania is an area where velar fronting is the norm. This conclusion is clear from Map 12 (for \textit{euch} ‘you-\textsc{acc/dat.pl}), Map 22 (\textit{recht} ‘right'), Map 33 (for \textit{sprechen} ‘speak-\textsc{inf}') and Map 58 (for \textit{liegen} ‘lie-\textsc{inf}') in the linguistic atlas for Transylvania German (SDSA). The same conclusion can be drawn from the detailed descriptions of the historical phonology in specific towns of Transylvania, e.g. \ipi{Mediasch} \citep{Scheiner1887}, \ipi{Bistritz} \citep{Kisch1893, Klein1927}, and \ipi{Schäßburg} \citep{Bruch1966}. Those sources all indicate that postvocalic [x] and [ç] (and their lenis counterparts) do not contrast because they occur after back vowels and front vowels respectively.\largerpage

One variety of Transylvania Saxon is different, namely the one spoken in Burgberg (ca. 20km to the northeast of Hermannstadt). According to the Burgberg data provided by \citet{Maurer1959}, only [ç] (=⟦χ⟧) (but not [x]=⟦x⟧) occurs after coronal sonorants (e.g. ⟦laiχt⟧ ‘light', ⟦šęχ⟧ ‘shoe’,  ⟦durχ⟧ 'through'), while  only [x] (but not [ç]) surfaces after back vowels (with the exception of [ɑː]), e.g. ⟦dox⟧ ‘however’, ⟦əuxt⟧ ‘eight’, ⟦laxən⟧ ‘laugh-\textsc{inf}’. [x] and [ç] contrast in Burgberg in the context after ⟦ā⟧ (=[ɑː]), e.g. ⟦šprāx⟧ ‘language’ vs. ⟦āχt⟧ ‘eight’. Palatal [ç] arose after the long low back vowel (e.g. in words like ⟦āχt⟧) because the original short low vowel (⟦a⟧) fronted and raised to ⟦ē⟧ and then retracted and lowered to ⟦ā⟧; see \citet[12]{Maurer1959}. Thus, ⟦ē⟧ triggered the change from /x/ to the earlier allophone [ç], which was then phonemicized to /ç/ when ⟦ē⟧ shifted to ⟦ā⟧. The ⟦ā⟧ in words like ⟦šprāx⟧ ‘language’ did not undergo the change to ⟦ē⟧ because that process of fronting and raising only affected an original short vowel. As in all of the case studies discussed in this chapter, Burgberg has alternations in the inflectional morphology between velar [x] and palatal [ç], meaning that some version of velar fronting operates as a neutralization.\footnote{Historical /k/ and /g/ also exhibit fronting in postsonorant position and word-initially. The fronting of velar stops is referred to in the earlier literature as “Palatalisierung”, e.g. \citet[75]{Maurer1959}.}

\begin{sloppypar}
The phonemicization of /ç/ in \ipi{Burgberg} is attested in other places in Transylvania as well. This conclusion is clear from the material presented in \citet{Scheiner1922}, which investigates the historical development of vowels in Southeast Transylvania. According to the data provided in that source \citep[61, 63]{Scheiner1922}, [ç] (=⟦χ⟧) occurs in the context after [ɑ] in a number of towns and villages in the general area around Kronstadt (see \mapref{map:9.2}). For example, the towns of Zeiden, Nussbach, and Schirkronyen all have ⟦krɑχn⟧  ‘crawl-\textsc{inf}’. That type of example is akin to the cases documented in this chapter because the historical rule of velar fronting overapplies. However, there are also cases attested in the same area (around Kronstadt) involving the historical underapplication of velar fronting, i.e. [x] occurs after a front vowel that was historically back. To cite one example, \citet[61]{Scheiner1922} lists five towns with [x] after [e], e.g. ⟦bex⟧ 'book’ (cf. StG [buːx]). I do not attempt to document the cases of opacity in Transylvania Saxon and instead leave this undertaking open for future research.\footnote{I have not been able to detect evidence for the phonemicization of /ç/ on the basis of the maps in SDSA. However, Map 58 (for \textit{fliegen} ‘fly-\textsc{inf}') indicates a few isolated pockets in Transylvania (e.g. the area around Mühlbach) where the lenis fricative [ʝ] occurs after the low back vowel [ɑ]. A contrast between velars and palatals is not attested in the other German-Language islands in Romania, i.e. in \ipi{Sathmar} in the northwest and the \ipi{Banat} region in the southwest (see \mapref{map:9.2}). \citet{Barba1982, Wolf1987, Dama1991}, and \citet{Mileck1997} all agree that [ç] and [x] have a transparent distribution in \ili{Banat Swabian}, expressed in the present treatment  with \isi{Velar Fronting-1}. There is likewise no evidence for a contrast between velar and palatal fricatives in Sathmar (\citealt{Moser1937}).}
\end{sloppypar}\il{Transylvania Saxon|)}

\section{Conclusion}\label{sec:9.6}

The case \il{Central Hessian}CHes/\il{Rhenish Franconian}RFr studies discussed in this chapter have in common that velars and the corresponding palatals contrast in the context after certain back vowels. That type of contrast was the result of a \isi{phonemic split} triggered by \isi{Vowel Retraction}, after which a new contrast arose between velar and palatal after a back vowel. The consequence is that the original rule of velar fronting ceased to operate as an allophonic operation and became a rule of \isi{neutralization} which only applied in the context after coronal sonorants (or a subset thereof).

In \chapref{sec:10} I consider a set of German dialects that is similar to the ones discussed in the present chapter in the sense that phonemic palatals now stand in contrast with the corresponding velars after back vowels. In contrast to the systems examined above, the ones I investigate in the following chapter have in common that the phonemic palatals are realized in the phonetics as sibilants (i.e. as alveolopalatal [ɕ]).
