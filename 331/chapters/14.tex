\chapter{The nonassimilatory fronting of velars}\label{sec:14}
\is{nonassimilatory velar fronting|(}
\section{Introduction}\label{sec:14.1}

The synchronic and diachronic processes fronting velar segments in word-initial and postsonorant position investigated in previous chapters are uncontroversially assimilatory. As documented above, the generalization is that velar fronting is induced by front (coronal) sonorants or some subset thereof. Regardless of how one captures the fronting of velars in a formal model, that process spreads the fronting feature from the triggers to the appropriate targets in an assimilatory fashion, thereby creating palatals.

\begin{sloppypar}
Considerable evidence from the literature on German dialects points to the nonassimilatory fronting of historical velars to palatals. An example is the change from an etymological velar (\ili{WGmc} \textsuperscript{+}[ɣ]) to palatal in word-initial position before back vowels, e.g. [ʝɑbəl] ‘fork’ (cf. \il{Standard German}StG [gɑbəl]). The change in question was not an assimilation because it occurred before all or some back vowels, in addition to coronal sonorants, e.g. [ʝiːsən] ‘water-\textsc{inf}’ (cf. \il{Standard German}StG [giːsən]). The nonassimilatory change referred to here affected historical velar fricatives in word-initial position in many varieties of CG and in some varieties of LG. The analogous nonassimilatory change in postsonorant position likewise shifted historical velar fricatives to palatals and is attested primarily in WCG.
\end{sloppypar}

Some earlier studies have discussed the motivation for \isi{nonassimilatory velar fronting} in word-initial position. For example, \citet[1707]{Scheutz2005} and more recently \citet[10]{Hinskens2021} both note that the change from an original velar to palatal in words like [ʝɑbəl] ‘fork’ mentioned above was the extension of the assimilatory change from velar to palatal only in the context before a front vowel in items like [ʝiːsən] ‘water-\textsc{inf}’. In present terms, the change from velar to palatal before back vowels involved \isi{rule generalization}, as defined earlier.

In this chapter I adopt the \isi{rule generalization} approach endorsed by Scheutz and Hinskens, but I propose a much more fine-grained treatment. For example, I consider the change from velar to palatal in word-initial position before coronal sonorant consonants, e.g. [ʝroːs] ‘large’ (cf. \il{Standard German}StG [groːs]) as well as the same change in postsonorant position after all types of segments, namely front vowels, coronal sonorant consonants, and back vowels. The main idea of the present chapter is that the nonassimilatory change in question is the final historical stage defined in terms of velar fronting triggers.

In \sectref{sec:14.2} I discuss the nonassimilatory fronting of historical velars in word-initial position. \sectref{sec:14.3} considers dialects in which velars fronted to palatals after only a subset of back vowels, and \sectref{sec:14.4} examines those varieties in which the same fronting processes transpired after all back vowels. \sectref{sec:14.5} looks at the state of \isi{nonassimilatory velar fronting} in a cluster of \il{Moselle Franconian}MFr dialects in the area of \ipi{Nordösling} (North \ipi{Luxembourg}). \sectref{sec:14.6} investigates several questions related to the nonassimilatory fronting of etymological velars that arise in the course of the chapter, and \sectref{sec:14.7} considers various issues relating to the connection between velar fronting in word-initial and postsonorant position. The chapter concludes in \sectref{sec:14.8}.

\section{{Word-initial} {velar} {fronting}}\label{sec:14.2}

\subsection{Introduction}\label{sec:14.2.1}

The nonassimilatory change from velar to palatal in word-initial position to be documented below is depicted in \REF{ex:14:1c}. The velar referred to here could be \ili{WGmc} \textsuperscript{+}[ɣ] or the \textsuperscript{+}[x], which developed out of the \textsuperscript{+}[k] in \ili{WGmc} \textsuperscript{+}[sk] clusters.\footnote{{Recall from earlier chapters that a small number of dialects fronted \ili{WGmc}} \textrm{\textsuperscript{+}}\textrm{[k] in word-initial position (e.g. \il{High Alemannic}HAlmc and ELG). That type of fronting is assimilatory in all varieties investigated earlier. No dialect is known which exhibits the nonassimilatory fronting of word-initial \ili{WGmc}} \textrm{\textsuperscript{+}}\textrm{[k].} } The fronting process in \REF{ex:14:1c} is not an assimilation because it occurred regardless of the nature of the following sound; in particular, it transpired before front vowels (FV), coronal consonants (CC), and (crucially) back vowels (BV). In all dialects investigated in previous chapters with velar fronting in word-initial position that change is assimilatory, as in (\ref{ex:14:1a}--\ref{ex:14:1b}).

\ea%1
\label{ex:14:1}
Velar fronting (word-initial):
\ex\label{ex:14:1a} velar > palatal / \textsubscript{wd} [ (C) {\longrule}{\longrule} FV
\ex\label{ex:14:1b} velar > palatal / \textsubscript{wd} [ (C) {\longrule}{\longrule} FV, CC
\ex\label{ex:14:1c} velar > palatal / \textsubscript{wd} [ (C) {\longrule}{\longrule} FV, CC, BV
\z 

(\ref{ex:14:1a}) and (\ref{ex:14:1b}) were expressed formally in previous chapters as the spreading of [coronal] to a target. The nonassimilatory change in \REF{ex:14:1c} cannot be captured in the same way because back vowels are not [coronal]. I do not provide a formal rule for \REF{ex:14:1c}, although that process would have to be stated as one which adds (but does not spread) the frontness feature [coronal]. Since the formal rule applies before any segment it would be sufficient to state the change with the word-initial context without reference to any segmental triggers at all. Seen in that light, it is not true that the change in \REF{ex:14:1c} is “triggeredˮ by all back vowels. I continue to refer to the back vowels in \REF{ex:14:1c} as triggers for the sake of comparison with \REF{ex:14:1a} and \REF{ex:14:1b}.

I assume that \REF{ex:14:1c} -- as well as the mirror-image development in postsonorant position discussed in \sectref{sec:14.3} and \sectref{sec:14.4} -- affected underlying representations, e.g. \ili{WGmc} \textsuperscript{+}/ɣ/ shifted to /ʝ/. Alternatively, one could argue that nonassimilatory changes like the one in \REF{ex:14:1c} did not alter underlying representations, but instead remain in the respective dialects as synchronic rules, e.g. \ili{WGmc} \textsuperscript{+}/ɣ/ was inherited without change as /ɣ/, which then surfaced at that later stage as [ʝ] by the synchronic reflex of \REF{ex:14:1c} in word-initial position; see \sectref{sec:14.6.2}.

I contend that \REF{ex:14:1c} represented the final stage in the \isi{rule generalization} scenario described in \chapref{sec:12} for word-initial position. Trigger Types A-E and the corresponding historical stages proposed earlier are listed in \tabref{tab:14:1}. In the final row (Trigger Type F=Stage 2e) I include the change from velar to palatal in the elsewhere case, i.e. before back vowels, which represents \REF{ex:14:1c}.

\begin{table}
\caption{Trigger Types and the corresponding historical stages\label{tab:14:1}}
\begin{tabular}{lll}
\lsptoprule
Type & Trigger & Stage\\\midrule
A & HFV & 2a\\
B & HFV, MFV & 2b\\
C & HFV, MFV, CC & 2c\\
D & HFV, MFV, LFV & 2c'\\
E & HFV, MFV, LFV, CC & 2d\\
F & HFV, MFV, LFV, CC, BV & 2e\\
\lspbottomrule
\end{tabular}
\end{table}

One could argue that Stage 2e  {}-- like the changes from Stage 2a to 2b etc. -- should be broken down into a series of substages defined according to the height of the vocalic trigger: On that view, the first nonassimilatory change -- call it Stage 2e' -- occurs before a high back vowel, later (Stage 2e'{}') before a high back vowel or a mid back vowel, and finally (Stage 2e'{}'{}') before all back vowels. No evidence from is known to me which supports the decomposition of Stage 2e in that manner for word-initial position. However, I demonstrate in \sectref{sec:14.3} and \sectref{sec:14.4} that the nonassimilatory change from velar to palatal after a sonorant proceeded in an incremental fashion according to vowel height.

The change from velar to palatal in word-initial position at Stage 2e occurred before all back vowels in every example meeting its structural description. None of the studies described below for Stage 2e dialects provides evidence for \isi{lexical exceptions}. For example, if an original velar shifted to palatal before [u o ɑ], then the change occurred in every item with those three vowels; there were no aberrant items beginning with an unshifted velar followed by any one of [u o ɑ]. The nonassimilatory change in \REF{ex:14:1c} can therefore be thought of as a Neogrammarian-style development. It is conceivable that the across-the-board change referred to here began its life as a \isi{lexical diffusion} type change that applied sporadically, on a word-by-word basis, but as yet no evidence is available to my knowledge for that claim.

In \REF{ex:14:2} I state the four diachronic stages affecting \ili{WGmc} \textsuperscript{+}[ɣ] in word-initial position. In the headings for (\ref{ex:14:2a}--\ref{ex:14:2d}) I list four of the historical stages given in \tabref{tab:14:1}. The initial segment in the three sequences listed in phonetic representation for each of the stages corresponds to the attested realization of that original velar fricative \textsuperscript{+}[ɣ]. The symbols  “[i]ˮ, “[l]ˮ and  “[ɑ]ˮ represent the natural classes of front vowels, coronal sonorant consonants and back vowels respectively. The initial sound in some of the three sequences in (\ref{ex:14:2a}--\ref{ex:14:2d}) underwent \isi{Wd-Initial ɣ-Fortition} (\sectref{sec:4.3}).

\ea%2
\label{ex:14:2}Modern reflexes of \ili{WGmc} \textsuperscript{+}[ɣ] in word-initial position:
\begin{multicols}{4}\raggedcolumns
\ea    Stage 1:\\\label{ex:14:2a}
  \relax  [ɣi   ɣl   ɣɑ]\\     
  \relax  [xi   xl   xɑ]\\  
\ex  Stage 2b/2c':\\\label{ex:14:2b}
 \relax  [ʝi   ɣl   ɣɑ]\\
 \relax  [çi  xl   xɑ]\\
 \relax  [ʝi  çl   ʝɑ]    
   
\ex Stage 2c/d:\\\label{ex:14:2c}
 \relax [ʝi    ʝl   ɣɑ]\\ 
 \relax [çi   çl   xɑ]\\

\ex Stage 2e:\\\label{ex:14:2d}
 \relax [ʝi   ʝl   ʝɑ]\\
 \relax [çi  çl  çɑ]\\
\z 
\end{multicols}
\z 

At Stage 1 (=\ref{ex:14:2a}) the original velar \textsuperscript{+}[ɣ] is retained as a velar. Not taken into consideration in \REF{ex:14:2a} is the default pattern -- represented in \il{Standard German}StG and UG -- whereby word-initial \textsuperscript{+}[ɣ] is realized as a stop ([g]) regardless of the nature of the following sound (by \isi{g-Formation-1} from \sectref{sec:4.2}). \REF{ex:14:2b} is the point where the change to palatal is induced by front vowels but not by coronal consonants (=\ref{ex:14:1a}). \REF{ex:14:2c} shows the assimilatory change whereby \textsuperscript{+}[ɣ]  undergoes fronting before a front vowel or a coronal sonorant consonant (=\ref{ex:14:1b}). The assimilatory changes in (\ref{ex:14:2b}, \ref{ex:14:2c}) were examined from the point of view of \isi{rule generalization} in \chapref{sec:12}. Stage 2e (=\ref{ex:14:2d}) reflects the most advanced fronting stage -- the nonassimilatory one -- because original \textsuperscript{+}[ɣ] is realized as palatal before front vowels, coronal consonants and back vowels.

The reflexes of \ili{WGmc} \textsuperscript{+}[sk] clusters in word-initial position as described in late-nineteenth and early twentieth century sources are summarized in \REF{ex:14:3}; see also \citet{Hall2021}. The symbols “[i]ˮ, “[r]ˮ and “[ɑ]ˮ are cover symbols representing front vowels, the coronal rhotic consonant, and back vowels respectively. The symbol “[s]ˮ is similarly a cover symbol for a \isi{sibilant} fricative ([s] or [ʃ]).\footnote{{The developments depicted in (\ref{ex:14:3a}--\ref{ex:14:3d}) only hold for word-initial position because the reflexes of \ili{WGmc}} \textrm{\textsuperscript{+}}\textrm{[sk] in word-internal and word-final position in the dialects discussed below were either [sk], [s], or [ʃ], but never [sx]/[sç]; \citet{Hall2021}.}} \REF{ex:14:3e} represents the final stage (reflected in \il{Standard German}StG); it is not directly relevant to the present chapter, but it is included for completeness.

\TabPositions{.15\textwidth, .33\textwidth, .5\textwidth, .75\textwidth}
\ea%3
\label{ex:14:3}Reflexes of \ili{WGmc} \textsuperscript{+}[sk] clusters in word-initial position:
\ea\label{ex:14:3a}\relax [sxi   sxr   sxɑ] \tab (=Stage 1)
\ex\label{ex:14:3b}\relax [sçi   sxr   sxɑ] \tab (=Stage 2c')
\ex\label{ex:14:3c}\relax [sçi   sçr   sxɑ] \tab (=Stage 2c or 2d)
\ex\label{ex:14:3d}\relax [sçi   sçr   sçɑ] \tab (=Stage 2e)
\ex\label{ex:14:3e}\relax [ʃi    ʃr     ʃɑ]
\z
\z

The patterns depicted in \REF{ex:14:3a} and especially the assimilatory ones in (\ref{ex:14:3b}, \ref{ex:14:3c}) were discussed in \chapref{sec:12}. Pattern \REF{ex:14:3d} exhibits the historical stage involving the nonassimilatory change from [sx] to [sç] before any type of sound.

\subsection{Data and discussion}\label{sec:14.2.2}

In (\ref{ex:14:4}--\ref{ex:14:12}) I present data from nine varieties of German reflecting the four stages depicted in \REF{ex:14:2}. The respective heading provides information concerning place, dialect classification, source, and respective stages. For each data set I provide one or two representative examples for the reflex of \ili{WGmc} \textsuperscript{+}[ɣ] in the three contexts (a) before front vowels, (b) before sonorant consonants, and (c) before back vowels. For comparison, I also include one word possessing the modern reflex of \ili{WGmc} \textsuperscript{+}[j], which underwent \isi{Glide Hardening} to [ʝ] in all varieties discussed below. The most important examples for purposes of this chapter are given in (\ref{ex:14:10}--\ref{ex:14:12}), which illustrate the nonassimilatory change from velar to palatal in word-initial position in \REF{ex:14:1c}.\largerpage[2]

\ea%4
\label{ex:14:4}
  \ipi{Kalkar} (LFr; \citealt{Hanenberg1915}; \sectref{sec:8.2}; \mapref{map:8}, Stage 1):
\ea\label{ex:14:4a}  ɡēͅrn \tab [ɣɛːrn] \tab gern \tab ‘gladly’ \tab 192
\ex\label{ex:14:4b}  ɡrōnd \tab [ɣʀoːnt] \tab Grund \tab ‘reason’ \tab 195
\ex\label{ex:14:4c}  ɡūt \tab [ɣuːt] \tab gut \tab ‘good’ \tab 216
\ex\label{ex:14:4d}  jɑxt \tab [ʝɑxt] \tab Jagd \tab ‘hunt’ \tab 209
\z 
\ex%5
\label{ex:14:5}\ipi{Gütersloh} (\il{Westphalian}Wph; \citealt{Wix1921}; \sectref{sec:4.3}; \mapref{map:6}, Stage 1):
\ea\label{ex:14:5a} xeln \tab [xeln̩] \tab gelten \tab ‘be valid\textsc{{}-inf}’ \tab 88
\ex\label{ex:14:5b} xlɑs \tab [xlɑs] \tab Glas \tab ‘glass’ \tab 12
\ex\label{ex:14:5c} xolt \tab [xɔlt] \tab Gold \tab ‘gold’ \tab 27
\ex\label{ex:14:5d} jǫuɒ \tab [ʝɔuɐ] \tab Jahr \tab ‘year’ \tab 77
\z 
\ex%6
\label{ex:14:6}\ipi{Eilsdorf} (\il{Eastphalian}Eph; \citealt{Block1910}; \sectref{sec:8.3}; \mapref{map:7}, Stage 2b/2c{}'):
\ea\label{ex:14:6a} jęlt \tab [ʝɛlt] \tab Geld \tab ‘money’ \tab 342
\ex\label{ex:14:6b} ʒlɑs \tab [ɣlɑs] \tab Glas \tab ‘glass’ \tab 340
\ex\label{ex:14:6c} ʒuut \tab [ɣuːt] \tab gut \tab  ‘good’ \tab 342
\ex\label{ex:14:6d} jɑ̊ɑ̊ \tab  [ʝɑː] \tab  ja  \tab  ‘yes’ \tab 338
\z
\ex%7
\label{ex:14:7} \ipi{Soest} (\il{Westphalian}Wph; \citealt{Holthausen1886}; \sectref{sec:4.3}; \mapref{map:6}, Stage 2b/2c{}'):
\ea\label{ex:14:7a} cist\textit{ɑ}n \tab  [çɪstɐn] \tab gestern \tab ‘yesterday’ \tab 44
\ex\label{ex:14:7b} xlykə \tab [xlʏkə] \tab Glück \tab  ‘fortune’ \tab  84
\ex\label{ex:14:7c} xuət \tab [xuət]  \tab  gut \tab ‘good’ \tab 88
\ex\label{ex:14:7d} jɑͅͅͅ \tab [ʝɔː] \tab ja \tab ‘yes’  \tab  43
\z 
\ex%8
\label{ex:14:8}\ipi{Kirchspiel Courl} (\il{Westphalian}Wph; \citealt{Beisenherz1907}; \mapref{map:6}, Stage 2c/2d):
\ea\label{ex:14:8a} ʓĭɛl \tab [ʝɪɛl] \tab gelb \tab ‘yellow’ \tab 28\\
    ʓɛẹ̆t \tab [ʝɛet] \tab geht \tab ‘go\textsc{{}-3sg}’ \tab 15
\ex\label{ex:14:8b} ʓrɑf \tab [ʝrɑf] \tab  Grab \tab ‘grave’ \tab 40\\
    ʓlīən \tab [ʝliːən] \tab  glitten \tab ‘slide\textsc{{}-pret.pl}’ \tab 40
\ex\label{ex:14:8c} ɡɑrvə \tab [ɣɑrvə] \tab Garbe \tab ‘sheaf’ \tab 16
\ex\label{ex:14:8d} (im) jɔ̄ərə \tab [ʝɔːərə] \tab (im) Jahre \tab ‘(in the) year’ \tab 24
\z 
\ex%9
\label{ex:14:9}\ipi{Elspe} (\il{Westphalian}Wph; \citealt{Arens1908}; \sectref{sec:7.2}; \mapref{map:6},  Stage 2c/2d):
\ea\label{ex:14:9a} χelt \tab [çɛlt] \tab Geld \tab ‘money’ \tab 31
\ex\label{ex:14:9b} χreŏt \tab [çrɛɔt] \tab groβ \tab ‘large’ \tab 89
\ex\label{ex:14:9c} xolt \tab [xɔlt] \tab Gold \tab ‘gold’ \tab 66
\ex\label{ex:14:9d} jōa \tab [ʝoːɐ] \tab Jahr \tab ‘year’ \tab 28
\z
\ex%10
\label{ex:14:10}\ipi{Schlebusch} (\il{Ripuarian}Rpn; \citealt{Bubner1935}; \sectref{sec:10.3.1}; \mapref{map:8}, Stage 2e):
\ea jęl \tab [ʝɛl] \tab gelb \tab ‘yellow’ \tab 72\\\label{ex:14:10a}
    jīːh\={ø}ːzəš \tab [ʝiːhøːzəʃ] \tab jähzornig \tab ‘irascible’ \tab 72
\ex jlɑt \tab [ʝlɑt] \tab glatt \tab ‘smooth’ \tab 72\\\label{ex:14:10b}
    jrɑs \tab [ʝrɑs] \tab Gras \tab ‘grass’ \tab 72
\ex jɑs \tab [ʝɑs] \tab Gast \tab ‘guest’ \tab 72\\\label{ex:14:10c}
    jəpęk \tab [ʝəpɛk] \tab Gepäck \tab ‘luggage’ \tab 72
\ex jǭ \tab [ʝɔː] \tab ja \tab ‘yes’ \tab 88\\\label{ex:14:10d}
    jets \tab [ʝets] \tab jetzt \tab ‘now’ \tab 88
\z
\ex%11
\label{ex:14:11}Kreis \ipi{Lippe} (\il{Westphalian}Wph; \citealt{Hoffmann1887}; \sectref{sec:7.2}; \mapref{map:6}, Stage 2e):
\ea\label{ex:14:11a} χæust \tab [çæust] \tab Geist \tab ‘intellect’ \tab 23
\ex\label{ex:14:11b} χnɑidiχ \tab [çnɑidiç] \tab gnädig \tab ‘merciful’ \tab 32\\
    χlet \tab [çlet] \tab Glied \tab ‘member’ \tab 17\\
    χrunt \tab [çʀunt] \tab Grund \tab ‘reason’ \tab 20
\ex\label{ex:14:11c} χɑus \tab  [çɑus] \tab Gans \tab ‘goose’ \tab 3\\
    χəwolt \tab [çəvolt] \tab Gewalt \tab ‘violence’ \tab 14
\ex\label{ex:14:11d}  juŋk \tab [ʝuŋk] \tab jung  \tab  ‘young’ \tab 20
    \z
\ex%12
\label{ex:14:12}Mansfeld (\il{Thuringian}Thrn; \citealt{Hennemann1901}; \mapref{map:12}, Stage 2e):
\ea\label{ex:14:12a} j\={æ}l \tab [ʝæːl] \tab gelb \tab ‘yellow’ \tab 20
\ex\label{ex:14:12b} χrɑ̄s \tab [çʀɑːs] \tab Gras \tab ‘grass’ \tab 40\\
    χlɑs \tab [çlɑt] \tab Glas \tab ‘glass’ \tab 40
\ex\label{ex:14:12c} jōrtṇ \tab [ʝoːʀtn̩] \tab Garten \tab ‘garden’ \tab 17\\
    jənɑuə \tab [ʝənɑuə] \tab genau \tab ‘exactly’ \tab 35
\ex\label{ex:14:12d} joxṇ \tab [ʝoxn̩] \tab jagen \tab ‘hunt\textsc{{}-inf}’ \tab 39
\z
\z 

In (\ref{ex:14:13}--\ref{ex:14:16}) I present data from four \il{Westphalian}Wph varieties illustrating the stages depicted in (\ref{ex:14:3a}--\ref{ex:14:3d}) for \ili{WGmc} \textsuperscript{+}[sk] in word-initial position \citep{Hall2021}. The LG dialect in \REF{ex:14:17} represents a stage postdating Stage 2e, whereby \ili{WGmc} \textsuperscript{+}[sk] is consistently realized as [ʃ] (=\ref{ex:14:3e}). Note that the dorsal (uvular) rhotic in \REF{ex:14:13b} and \REF{ex:14:14b} shows the effects of \isi{r-Retraction} (\sectref{sec:3.5}) from [r] to [ʀ].

\ea%13
\label{ex:14:13}\ipi{Adorf} (\il{Westphalian}Wph; \citealt{Collitz1899}; \mapref{map:6}, Stage 1):
\ea šɧīp \tab [ʃxiːp] \tab Schiff \tab ‘ship’ \tab 45\label{ex:14:13a}
\ex šɧrå \tab [ʃxʀɑ]  \tab mager  \tab ‘lean’ \tab 79\label{ex:14:13b}
\ex šɧou \tab [ʃxou]  \tab Schuh  \tab ‘shoe’ \tab 29\label{ex:14:13c}
\z
\ex%14
\label{ex:14:14}\ipi{Soest} (\il{Westphalian}Wph; \citealt{Holthausen1886}; \sectref{sec:4.3}; \mapref{map:6}, Stage 2c{}'):
\ea\label{ex:14:14a} scylic \tab [sçʏlɪç] \tab schuldig \tab ‘guilty’ \tab  43\\
    scèpm \tab [sçɛpm̩] \tab schöpfen \tab ‘ladle-\textsc{inf}’ \tab  43
\ex\label{ex:14:14b} sxʀuĭvə \tab [sxʀuivə] \tab schreibe \tab ‘write\textsc{{}-1sg}’ \tab  43\\
    sxʀiʓn \tab [sxʀɪɣn̩] \tab schreien \tab ‘scream\textsc{{}-inf}’ \tab  62
\ex\label{ex:14:14c} sxult \tab [sxʊlt] \tab Schuld \tab ‘fault’ \tab  15\\
    sxɑ̖̄ͅp \tab [sxɔːp] \tab Schaf \tab ‘sheep’ \tab  43
  \z
\ex%15
\label{ex:14:15}\ipi{Elspe} (\il{Westphalian}Wph; \citealt{Arens1908}; \sectref{sec:7.2}; \mapref{map:6}, Stage 2d):
\ea\label{ex:14:15a} šχyt \tab [ʃçyt] \tab schieβt \tab ‘shoot\textsc{{}-3sg}’ \tab 97\\
    šχelə \tab  [ʃçɛlə] \tab Schale \tab ‘bowl’ \tab 33
\ex\label{ex:14:15b} šχrɑpn \tab [ʃçrɑpn̩] \tab schaben \tab ‘scrape\textsc{{}-inf}’ \tab 27
\ex\label{ex:14:15c} šxuɡn \tab [ʃxʊɣn̩] \tab scheuen \tab ‘dread\textsc{{}-inf}’ \tab 96\\
    šxɑ̄p \tab [ʃxɑːp] \tab Schrank \tab ‘cabinet’ \tab 23
\z
\ex%16
\label{ex:14:16}Kreis \ipi{Lippe} (\il{Westphalian}Wph; \citealt{Hoffmann1887}; \sectref{sec:7.2}; \mapref{map:6}, Stage 2e):
\ea\label{ex:14:16a} sχoin \tab [sçoin] \tab schön \tab ‘beautiful’ \tab 3
\ex\label{ex:14:16b} šrị̄ƀən \tab [ʃʀiːβən] \tab geschrieben \tab ‘write\textsc{{}-part}’ \tab 17
\ex\label{ex:14:16c} sχẹ̄u \tab [sçeːu] \tab Schuh \tab ‘shoe’ \tab 3
  \z
\ex%17
\label{ex:14:17}\ipi{Diepenau} (NLG; \citealt{Schmeding1937}; \mapref{map:5}):
\ea\label{ex:14:17a} šulαn \tab [ʃulɐn] \tab Schulter \tab ‘shoulder’ \tab 14
\ex\label{ex:14:17b} šāp \tab [ʃɑːp]  \tab  scharf \tab ‘sharp’ \tab 19
  \z
\z 

The closest variety found for stage 2e is Kreis \ipi{Lippe} in (\ref{ex:14:16}). \citet[3]{Hoffmann1887} notes that in the year 1887 the realization [sç] was rapidly being replaced with [ʃ] and that one hears [sç] only in the speech of the elderly. In that dialect there are apparently no [sç] sequences before the rhotic consonant -- even in the speech of the elderly -- because \ili{WGmc} \textsuperscript{+}[sk] underwent coalescence to [ʃ] in that context, as in \REF{ex:14:16b}.

In contrast to \ili{WGmc} \textsuperscript{+}[sk] clusters, many varieties of German are attested which exhibit the change from \textsuperscript{+}[ɣ] in word-initial position before any sound, as in \REF{ex:14:10}-\REF{ex:14:12}. A list of those varieties from the original sources cited in this book is given in \tabref{tab:14:2}. The final column indicates the palatal realization before a front vowel (“[i]ˮ), coronal sonorant consonant (“[l]ˮ), or back vowel (“[ɑ]ˮ); recall \REF{ex:14:2}.\footnote{According to the source for \ipi{Friedersdorf} \citep[37]{Seibicke1967} the palatal in the context before a consonant exhibits a strong tendency to surface as [ʝ] and not as [ç]. Speakers who have that realization therefore display pattern \REF{ex:14:2a} and not \REF{ex:14:2c}. The same point holds for Weidenhain \citep[39]{Krug1969}.} Many \il{Brandenburgish}Brb and \il{North Upper Saxon-South Markish}NUSax-SMk varieties exemplify Stage 2e, several of which are listed in \tabref{tab:14:2}. The location of those places can be found on \mapref{map:12} and \mapref{map:17}.

\begin{longtable}{lll}
\caption{Velar fronting (word-initial) of \ili{WGmc} \textsuperscript{+}[ɣ] (=Stage 2e)\label{tab:14:2}}\\
\lsptoprule Place & Source & Pattern\\\midrule\endfirsthead
\midrule Place & Source & Pattern\\\midrule\endhead
\endfoot\lspbottomrule\endlastfoot
\il{Ripuarian}Rpn\\\midrule
\ipi{Krefeld}  &  \citet{Röttsches1877}, & [ʝi   ʝl   ʝɑ]\\
               & \citet{Bister-Broosen1989} & \\
\ipi{Aachen}  &  \citet{Jardon1891} & [ʝi   ʝl   ʝɑ]\\
\ipi{Aegidienberg}  &  \citet{Müller1900} & [ʝi   ʝl   ʝɑ]\\
\ipi{Erftgebiet}  &  \citealt{Münch1904} & [ʝi   ʝl   ʝɑ]\\
\ipi{Wermelskirchen}  &  \citet{Hasenclever1905} & [ʝi   ʝl   ʝɑ]\\
\ipi{Cologne}  &  \citet{Müller1912} & [ʝi   ʝl   ʝɑ]\\
\ipi{Dülken}  &  \citet{Frings1913} & [ʝi   ʝl   ʝɑ]\\
\ipi{Niederembt}  &  \citet{Grass1920} & [ʝi   ʝl   ʝɑ]\\
\ipi{Düsseldorf}  &  \citet{Zeck1921} & [ʝi   ʝl   ʝɑ]\\
\ipi{Schelsen}  &  \citet{Greferath1922} & [ʝi   ʝl   ʝɑ]\\
\ipi{Seelscheid}  &  \citet{Mackenbach1924} & [ʝi   ʝl   ʝɑ]\\
\ipi{Eckenhagen}, \ipi{Berghausen}  &  \citet{Branscheid1927} & [ʝi   ʝl   ʝɑ]\\
\ipi{Schlebusch}  &  \citet{Bubner1935} & [ʝi   ʝl   ʝɑ]\\
\ipi{Aachen}  &  \citet{Welter1938} & [ʝi   ʝl   ʝɑ]\\
\ipi{Burscheid}  &  \citet{Heinrichs1978} & [ʝi   ʝl   ʝɑ]\\
\ipi{Rimburg}  &  \citet{Hinskens1992} & [ʝi   ʝl   ʝɑ]\\
\ipi{Gleuel}  &  \citet{Heike1970} & [ʝi   ʝl   ʝɑ]\\
\ipi{Moresnet}  &  \citet{Jongen1972} & [ʝi   ʝl   ʝɑ]\\
\ipi{Burg-Reuland}  &  \citet{Hecker1972} & [ʝi   ʝl   ʝɑ]\\
\ipi{Niederbachem}, \ipi{Oberbachem}  &  \citet{Fuss2001} & [ʝi   ʝl   ʝɑ]\\\midrule
\il{Moselle Franconian}MFr\\\midrule

\ipi{Prüm} & \citet{Büsch1888} & [ʝi   ʝl   ʝɑ]\\
\ipi{Ihren}, \ipi{Sellerich}, \ipi{Weinsheim} & \citet{Meyers1913, Meyers1913b} & [ʝi   ʝl   ʝɑ]\\
Elsenborn & \citet{Hecker1972},  & [ʝi   ʝl   ʝɑ]\\
          & \citet{CajotBeckers1979} & \\\midrule
\il{North Upper Saxon-South Markish}NUSax-SMk\\\midrule

\ipi{Aken} (Elbe) & \citet{Bischoff1935} & [ʝi   ʝl   ʝɑ]\\
\ipi{Grassau}  &  \citet{Stellmacher1973} & [ʝi   ʝl   ʝɑ]\\
\ipi{South Brandenburg}  &  \citet{Kieser1963} & [ʝi   ʝl   ʝɑ]\\
\ipi{Friedersdorf}  &  \citet{Seibicke1967} & [ʝi  çl   ʝɑ]\\
Weidenhain  &  \citet{Krug1969} & [ʝi  çl   ʝɑ]\\
\ipi{Wittenberg}  &  \citet{Langner1977} & [ʝi   ʝl   ʝɑ]\\
\ipi{Berlin}  &  \citet{Schönfeld1986} & [ʝi   ʝl   ʝɑ]\\\midrule
\il{Upper Saxon}USax\\\midrule
\ipi{Saalkreis} &  \citet{Bremer1909} & [ʝi   ʝl   ʝɑ]\\
\ipi{Salzfurtkapelle} & \citet{Schönfeld1958} & [ʝi   ʝl   ʝɑ]\\\midrule
\il{Thuringian}Thrn\\\midrule
\ipi{Stiege} &  \citet{Liesenberg1890} & [ʝi   ʝl   ʝɑ]\\
Mansfeld  & \citet{Hennemann1901} & [ʝi  çl   ʝɑ]\\
Südharz & \citet{Rudolph1924} & [ʝi   ʝl   ʝɑ]\\\midrule
\il{Brandenburgish}Brb\\\midrule
\ipi{Magdeburg} &  \citet{Krause1895} & [ʝi   ʝl   ʝɑ]\\
Kreis \ipi{Jerichow} I  &  \citet{Krause1896} & [ʝi   ʝl   ʝɑ]\\
\ipi{Besten}  &  \citet{Siewert1907} & [ʝi   ʝl   ʝɑ]\\
\ipi{Prenden}  &  \citet{Seelmann1908} & [ʝi   ʝl   ʝɑ]\\
\ipi{Strodehne} (Havelaue)  &  \citet{Hildebrand1913} & [ʝi   ʝl   ʝɑ]\\
\ipi{Jerichower Land}  &  \citet{Bathe1932} & [ʝi   ʝl   ʝɑ]\\
\ipi{Kleinwusterwitz}  &  \citet{Bathe1937} & [ʝi   ʝl   ʝɑ]\\
\ipi{Heckelberg}  &  \citet{Teuchert1964} & [ʝi   ʝl   ʝɑ]\\
\ipi{Schollene}  &  \citet{Gebhardt1965}, \citet{Schönfeld1965} & [ʝi   ʝl   ʝɑ]\\\midrule
\il{Westphalian}Wph\\\midrule
Kreis \ipi{Lippe} &  \citet{Hoffmann1887} & [çi  çl  çɑ]\\
\ipi{Hiddenhausen} & \citet{Schwagmeyer1908} & [çi  çl  çɑ]\\
\end{longtable}

The earliest attestation of Stage 2e among my sources is \citet{Rovenhagen1860} for \ipi{Aachen}. He writes (p. 8): “The breathing sound j (engl. y)  … [is] … in most cases a substitute for g; thus at the beginning of words g has always this sound … this pronunciation is common in the \ipi{Berlin} etc.  vulgar speaking ...ˮ

In \sectref{sec:12.5.2} I presented a cluster of \il{Westphalian}Wph dialects which represent several historical stages involving the trigger for the fronting of word-initial \ili{WGmc} \textsuperscript{+}[ɣ]. In \REF{ex:14:18} I reproduce those dialects and include Kreis \ipi{Lippe} from (\ref{ex:14:16}) and \tabref{tab:14:2} for Stage 2e.\pagebreak

\ea%18
\label{ex:14:18}Historical stages for triggers for (word-initial) velar fronting (\il{Westphalian}Wph) for WGmc \textsuperscript{+}[ɣ]:\\
  \begin{tabular}[t]{@{}l@{~}l@{}}
  Stage 1:   &  \ipi{Grafschaft Bentheim}\\
  Stage 2a:  &   \ipi{Plettenberg}       \\
  Stage 2b:  &   (\ipi{Soest}, \ipi{Laer})     \\
  Stage 2c:  &     (\ipi{Nienberge})     \\
  Stage 2c': &    (\ipi{Borken})         \\
  Stage 2d:  &   \ipi{Elspe}             \\
  Stage 2e:   &  Kreis \ipi{Lippe}        \\
  \end{tabular}
\z 

Recall that parentheses in \REF{ex:14:18} indicate that the dialect in question cannot be unambiguously classified as a particular Target Type, e.g. \ipi{Soest} could be either Stage 2b or Stage 2'.

\subsection{Areal distribution of the reflexes of \ili{WGmc} \textsuperscript{+}[ɣ] in word-initial position}\label{sec:14.2.3}

Stage 2e dialects for word-initial position have been discussed at length in the literature on German dialectology, although to the best of my knowledge no one has proposed the historical stages in \tabref{tab:14:1}. Before presenting my own map, I consider briefly some of the findings in dialectology on the fronting of historical \textsuperscript{+}[ɣ] in word-initial position.

An inspection of the earlier literature on the modern realizations of \ili{WGmc} \textsuperscript{+}[ɣ] reveals that the change from that sound to palatal in word-initial position (=Stage 2e) has an areal distribution akin to the one suggested in by the works listed in \tabref{tab:14:2}. One such work is \citet{Diederichs1884}, who provides a list of places in Germany and indicates how \ili{WGmc} \textsuperscript{+}[ɣ] is realized in those places word-initially, word-medially, and word-finally. Among those places are several in North and Central Germany with [ʝ] in initial position (=Stage 2e), but also a few with a velar before a back vowel and a palatal before a front vowel (recall the \il{Eastphalian}Eph dialects discussed in \sectref{sec:8.3}, \sectref{sec:8.4}). A second work is KDSA. In particular, Map 80 (for \textit{Gänse} ‘goose-\textsc{pl}’), Map 81 (for \textit{Garten} ‘garden’), and Map 95 (for \textit{glaube} ‘believe-\textsc{1sg}’) indicate the areas in pre-1914 Germany where the initial sound (an etymological velar) is realized as \textit{j} (=[ʝ]). The dialect regions on those maps correspond to the ones reflected in the second column of \tabref{tab:14:2}.

Stage 2e for word-initial position has been discussed in works focusing on a specific region. One dialect area particularly well-known for the change in question is CFr (=\il{Ripuarian}Rpn and \il{Moselle Franconian}MFr). The extent of that change in \il{Moselle Franconian}MFr is evident from Maps 381 and 382 in volume 4 of MRhSA, which depicts the realization of the original velar as palatal or alveolopalatal in the words \textit{Garten} ‘garden’ and \textit{grün} ‘green’. The (alvelo)palatal realization is the dominant pronunciation to the west of Koblenz and north of the Mosel River (see \il{Moselle Franconian}MFr region on \mapref{map:10}). According to \citet[397--398]{Cornelissen2000} the change from original lenis velar to palatal fricative in word-initial position (=Stage 2e) is typical for the \il{Ripuarian}Rpn variety in and around \ipi{Cologne} extending north to the \isi{Uerdinger Line}, the approximate boundary between \il{Ripuarian}Rpn and LFr (\mapref{map:8}).  \citet[49--50]{vandeWijngaard2007} likewise documents that change in the \il{Ripuarian}Rpn areas in the Netherlands (Limburg), especially around Kerkrade. The phonetic transcriptions of various dialogues from informants throughout the \il{Ripuarian}Rpn/\il{Moselle Franconian}MFr region presented in \citet{CornelissenEtAl1989} similarly reveal the extent of Stage 2e. Several places from that source in the \il{Ripuarian}Rpn dialect region are indicated on \mapref{map:8}.

The change from word-initial \ili{WGmc} \textsuperscript{+}[ɣ] to a palatal fricative is also well-documented in the literature on ECG (\il{Thuringian}Thrn, \il{North Upper Saxon-South Markish}NUSax-SMk) and ELG (\il{Brandenburgish}Brb). Three detailed case studies documenting that change are \citet{Hankel1913}, \citet{Kieser1963}, and \citet{Bathe1965}. The former author discusses data collected in a number of communities (\il{Thuringian}Thrn) in the northeastern part of the state of Thuringia (\mapref{map:12}). \citet{Kieser1963} focuses on the realization of \ili{WGmc} \textsuperscript{+}[ɣ] as palatal in a number of \il{North Upper Saxon-South Markish}NUSax-SMk-speaking towns in \ipi{South Brandenburg} (\mapref{map:12}). \citet{Bathe1965} likewise documents the same change, concentrating on \il{Brandenburgish}Brb varieties in a broad area in western Brandenburg (\mapref{map:17}).  All three authors demonstrate that the contexts for the change from velar to palatal can differ from village to village within a small area. A closer examination of that small-scale variation confirms the stages posited above; thus, \ili{WGmc} \textsuperscript{+}[ɣ] shifts to palatal before front vowels in some towns and villages (=\ref{ex:14:2a}), before front vowels and coronal consonants in others (=\ref{ex:14:2b}), and before any type of segments in other places (=\ref{ex:14:2c}).\largerpage

Stage 2e for word-initial position (< \ili{WGmc} \textsuperscript{+}[ɣ]) is also well-attested in a number of dialect dictionaries for the dialect areas in \tabref{tab:14:2}. For \il{Ripuarian}Rpn, two dictionaries for the \ipi{Cologne} dialect (NKSS, WbKM) provide a brief statement in the pronunciation guide (NKSS Volume 1: 265; WbKM: 17) that word-initial \textit{g} is articulated as \textit{j} (=[ʝ]). KWb gives phonetic transcriptions for all lexical entries beginning with \textit{g} as [ʝ]. Also for \il{Ripuarian}Rpn, the dictionaries for Neunkirchen-\ipi{Seelscheid}  (NSSS), the Lower Sieg (WbUS), \ipi{Aachen} (AaWb) and \ipi{Dremmen} (DrWb) list all words beginning with [g] in \il{Standard German}StG as j-initial, e.g. \textit{Jeld} ‘money’ (cf. \il{Standard German}StG [gɛlt]), \textit{Jlaas} ‘glass’ (cf. \il{Standard German}StG [glɑs]), and \textit{Jold} ‘gold’ (cf. \il{Standard German}StG [gɔlt]). For \il{Ripuarian}Rpn and \il{Moselle Franconian}MFr, RWb includes among words with initial \textit{g} such as \textit{gut} ‘good’, \textit{gießen} ‘water-\textsc{inf}’, and \textit{Glück} ‘fortune’ the  realization ⟦j⟧ (=[ʝ]). Finally, for \il{Brandenburgish}Brb, the dictionary for \ipi{Teltow} (TeWb), provides a clear statement to the effect that historical \textsuperscript{+}[ɣ] is realized in word-initial position as a lenis palatal fricative [ʝ] before vowels and consonants alike (p. 300).

\mapref{map:29} depicts the modern realization of \ili{WGmc} \textsuperscript{+}[ɣ] in word-initial position representing three historical stages: No velar fronting (=Stage 1), velar fronting as an assimilatory change (Stage 2a-d), and velar fronting as a nonassimilatory change (=Stage 2e). For Stage 1 I only include those dialects mentioned earlier (\sectref{sec:12.3}) in which \ili{WGmc} \textsuperscript{+}[ɣ] is realized as a velar fricative ([ɣ] or [x]); hence, I ignore the prevalent pattern reflected in UG whereby that original sound is now realized as a velar stop ([g]). For Stage 2e I list all of the places listed in \tabref{tab:14:2}. For those localities where velar fronting applies as an assimilatory change I do not attempt to distinguish the five incremental steps discussed in \chapref{sec:12} (summarized in \tabref{tab:14:1}). Those varieties are listed in Tables \ref{tab:12.13}, \ref{tab:12.16}, \ref{tab:12.18}, \ref{tab:12.20}, and \ref{tab:12.22} in \sectref{sec:12.3}.\largerpage

\begin{map}
% \includegraphics[width=\textwidth]{figures/VelarFrontingHall2021-img035.png}
\includegraphics[width=\textwidth]{figures/Map29_14.1.pdf}
\caption[Areal distribution of the realization of West Germanic {\textsuperscript{+}[ɣ]} in word-initial position]{Areal distribution of the realization of \ili{WGmc} \textsuperscript{+}[ɣ] in word-initial position. Circles are varieties of High German and Low German with no word-initial velar fronting (Stage 1), white squares are varieties with word-initial velar fronting as an assimilatory change (Stage 2a-d), and dark squares are varieties with word-initial velar fronting as a nonassimilatory change (Stage 2e). The velars and palatals referred to for Stages 2a-e can be either fortis ([x ç]) or lenis ([ɣ ʝ]).}\label{map:29}
\end{map}

It can be seen on \mapref{map:29} that Stage 1 varieties are restricted to the far western regions of modern-day Germany, including German-speaking parts of Belgium. The numerous Stage 2e varieties belong overwhelmingly to CG. By contrast, velar fronting as an assimilation word-initially is a common pattern for LG.

It is difficult -- although not impossible -- to project isoglosses onto \mapref{map:29} separating those areas where velar fronting applies as an assimilation (white square) vs. those places where the change is nonassimilatory (black square). I hypothesize that many centuries ago -- but some time after velar fronting had been phonologized in word-initial position -- the white square areas were much more prominent and black square areas were rare. At that earlier point in time I claim that it would have been possible to discern isoglosses separating the four assimilatory stems (Stage 2a-Stage 2d) from \tabref{tab:14:1}.

\subsection{Word-initial velar fronting before all vowels}\label{sec:14.2.4}

The treatment proposed in this chapter asserts that the assimilatory process of velar fronting (=\ref{ex:14:19a}, \ref{ex:14:19b}) applies historically before the corresponding nonassimilatory process (=\ref{ex:14:19c}). Nothing has been said up to this point about the change in \REF{ex:14:19d}, which applies before front vowels and back vowels but not before coronal consonants. That development poses a potential problem because it includes a nonassimilatory change (velar > palatal before a back vowel) but not an assimilatory one (velar > palatal before a sonorant coronal consonant).

\ea%19
\label{ex:14:19}Velar fronting (word-initial):
\ea\label{ex:14:19a} velar > palatal / \textsubscript{wd} [ {\longrule}{\longrule} FV
\ex\label{ex:14:19b} velar > palatal / \textsubscript{wd} [ {\longrule}{\longrule} FV, CC
\ex\label{ex:14:19c} velar > palatal / \textsubscript{wd} [ {\longrule}{\longrule} FV, CC, BV
\ex\label{ex:14:19d} velar > palatal / \textsubscript{wd} [ {\longrule}{\longrule} FV, BV
    \z
\z 

The historical change in \REF{ex:14:19d} is attested in more than one region; hence, the goal of this section is to explain why it is compatible with the present treatment of velar fronting.

As a representative example of \REF{ex:14:19d}, consider \citegen{Kieser1963} study of the pronunciation of word-initial \textit{g} in \ipi{South Brandenburg} (\il{North Upper Saxon-South Markish}NUSax-SMk; \mapref{map:12}). Kieser shows that that broad region displays more than one pattern (=Trigger Types or historical stages in the present framework). Most significant is the area between Marxdorf and Rothstein and further to the east in the area around Deutsch Sorno. Those areas are characterized by the change from \ili{WGmc} \textsuperscript{+}[ɣ] in word-initial position to palatal (⟦j⟧=[ʝ]) in the context before front vowels (=\ref{ex:14:20a}) or back vowels (=\ref{ex:14:20b}), but not before coronal sonorant consonants, where the original velar is retained as a velar stop (=\ref{ex:14:20c}). I retain Kieser’s original transcriptions because it is not clear how some of his phonetic symbols and diacritics match up with the ones adopted in this book.

\ea%20
\label{ex:14:20}Nonassimilatory velar fronting:
\ea\label{ex:14:20a} j\k{i}ḅ’ \tab gib \tab ‘give-\textsc{imp}.\textsc{sg}’\\
    jęrnə \tab gerne \tab ‘gladly’
\ex\label{ex:14:20b} jɑns \tab Gans \tab ‘goose’\\
    jūḍ’ \tab gut \tab ‘good’
\ex\label{ex:14:20c} glɑi \tab sogleich \tab ‘immediately’\\
    \.{g}rīnəs glɑ̊s \tab grünes Glas \tab ‘green-\textsc{infl} glass’
    \z
\z 

The data in \REF{ex:14:20} can be accommodated in the present framework by taking phonotactics into consideration. I argue that the pattern in \REF{ex:14:20} obtains because its speakers have adopted a condition governing the type of complex onset that is (not) allowed. In \REF{ex:14:21} I give a list of the complex (two-member) onset clusters attested in \il{Standard German}StG (e.g. \citealt{Hall1992}, \citealt{Wiese1996a}). It is not possible to present the onset clusters for the dialect in \REF{ex:14:20} because the source cited does not give them. However, the data presented in \citet{Kieser1963} suggest that the basic generalization is the same in \il{Standard German}StG and in \REF{ex:14:21}: A complex onset can consist of an obstruent plus liquid (=\ref{ex:14:21}a), an obstruent plus nasal (=\ref{ex:14:21}b), an obstruent plus [v] (=\ref{ex:14:21}c), or a \isi{sibilant} plus stop (=\ref{ex:14:21}d).\largerpage[2]

\ea%21
\label{ex:14:21}\begin{tabular}[t]{@{} *{9}{l} @{}} 
a. & pl  & bl &     &    & kl & gl & fl & ʃl \\
   & pʀ  & bʀ & tʀ  & dʀ & kʀ & gʀ & fʀ & ʃʀ \\
   & pfl &    &     &    &    &    &    &    \\
   & pfʀ &    &     &    &    &    &    &    \\
b. &     &    &     &    & kn & gn \\
c. & tsv & kv & ʃv       \\
d. & ʃp  & ʃt & sk       \\
\end{tabular}
\z 

All of the clusters in \REF{ex:14:21} have in common that the individual members are simplex segments in the sense that they bear only one of the place features [labial], [coronal], [dorsal]. For example, /pl/ in (\ref{ex:14:21}a) consists of /p/, which is [labial], and /l/, which is [coronal], and /tsv/ in (\ref{ex:14:21}c) is made up of the \isi{affricate} /ts/, which is [coronal], and the fricative /v/, which is [labial].

By contrast, there are no complex onsets containing a featurally complex consonant which bears more than one of the three features [labial], [coronal], [dorsal]. In the featural approach described in \chapref{sec:2} the only complex segments in this sense of the word are palatals, which are both [dorsal] and [coronal]. The following condition holds for the dialect in \REF{ex:14:20} on the type of complex onset allowed:

\ea%22
\label{ex:14:22}
  \textsc{\isi{Condition on Complex Onsets}}:\\
  Segments with more than one of the features [labial], [coronal], [dorsal] are not allowed in a complex onset.
\z 

Speakers of the dialect in \REF{ex:14:20} have adopted the \isi{Condition on Complex Onsets} in \REF{ex:14:22} on the basis of the occurring onset clusters in \REF{ex:14:21}. Given that condition, there cannot be clusters which contain a palatal because palatals are both [coronal] and [dorsal]. This means that speakers who incorporated \REF{ex:14:22} into their grammar could not have applied velar fronting to the initial velar (=\ili{WGmc} \textsuperscript{+}[ɣ]) in (\ref{ex:14:21}c), otherwise a cluster would be created like [ʝl ʝr ʝn çr çl çn],  which violates \REF{ex:14:22}.\footnote{{The historical rule of velar fronting referred to here is shown in \chapref{sec:16} to have applied in \ili{OHG}/\ili{OSax}. The basic generalizations concerning the onset clusters of \il{Standard German}StG in \REF{ex:14:21} also held for earlier stages of German. See in particular \citet[167]{Moulton1972}, who lists obstruent plus liquid/nasal clusters for \ili{PGmc} which were similar to the ones in (\ref{ex:14:21}a, \ref{ex:14:21}b) in the sense that each member was either [labial], [coronal], or [dorsal]. At that early stage in the language there were also clusters containing} \textrm{\textsuperscript{+}}\textrm{/w/, which was presumably [labial] and [dorsal]. However, the} \textrm{\textsuperscript{+}}\textrm{/w/ in} \textrm{\textsuperscript{+}}\textrm{/wr wl/ onset clusters of \ili{PGmc} was deleted in the earliest stages of \ili{OHG} \citep[108]{Braune2004}. \ili{OHG} also possessed onset clusters consisting of an obstruent plus} \textrm{\textsuperscript{+}}\textrm{/w/ which were the historical precursors of the clusters in (\ref{ex:14:21}c), e.g.} \textrm{\textsuperscript{+}}\textrm{/tw dw/. It is possible that at this early stage the} \textrm{\textsuperscript{+}}\textrm{/w/ in such clusters was [--consonantal], which would then escape \REF{ex:14:22} if that condition only held for onset clusters that were [+consonantal].} }

Given that the change in \REF{ex:14:19d} is attested, I assign it a unique Trigger Type (=E') and a unique historical Stage (=2d'), which I list in \tabref{tab:14:3} together with four other stages for word-initial position.

\begin{table}
\caption{Trigger Types and the corresponding historical stages for word-initial position\label{tab:14:3}}
\begin{tabular}{lll}
\lsptoprule
Type & Trigger & Stage\\\midrule
A'{}'{}'{}' & HUFV (unstressed [i]) & 2a'{}'{}'{}'\\
D & HFV, MFV, LFV & 2c'\\
E & HFV, MFV, LFV, CC & 2d\\
E' & HFV, MFV, LFV, BV & 2d'\\
F & HFV, MFV, LFV, CC, BV & 2e\\
\lspbottomrule
\end{tabular}
\end{table}

Stage 2d{}' is coterminous with Stage 2d. Thus, there are two possible developments: (i) Stage 2c' > Stage 2d by the addition of coronal sonorant consonants (CC) to the set of triggers, or (ii) Stage 2c' > Stage 2d' by incorporating back vowels (BV) among the segments inducing the change.

\mapref{map:30} depicts the three places mentioned above in \ipi{South Brandenburg} (Marxdorf, Rothstein, Deutsch Sorno) which illustrate Stage 2d'. The same map also includes those places listed in \citet{Kieser1963} characterized by Stage 2a'{}'{}'{}' (recall \sectref{sec:12.6.3} and \mapref{map:24}), Stage 2c', and Stage 2e.\footnote{{\citet{Kieser1963} discusses six areas (Grenzzonen), four of which (I-IV) match up with the historical stages on \mapref{map:30}: I (=Stage 2e), II (=Stage 2d'), III (=Stage 2c'), and IV (=Stage 2a'{}'{}'{}'). I do not include on \mapref{map:30} those places further to the south (Grenzzone V), where \ili{WGmc}} \textrm{\textsuperscript{+}}\textrm{[ɣ] surfaces as a palatal fricative (⟦j⟧ or ⟦χ⟧) in word-initial position before \isi{schwa} (<[i]), but only if the consonant following \isi{schwa} is velar ([g]). If the post-\isi{schwa} consonant is anything other than a velar then \ili{WGmc}} \textrm{\textsuperscript{+}}\textrm{[ɣ] is realized in those places as velar ([g]), e.g. ⟦\.{g}ə-mɑxd’⟧ ‘do-}\textrm{\textsc{part}}\textrm{’ vs. ⟦χə-\.{g}ōfd’⟧ ‘buy-}\textrm{\textsc{part}}\textrm{’. I also do not include on \mapref{map:30} those areas even further to the south (in Saxony) illustrating the retention of the original velar place of articulation (\ili{WGmc}} \textrm{\textsuperscript{+}}\textrm{[ɣ]) as velar ([g]), i.e. Stage 1 (=Grenzzone VI).}}

\ip{South Brandenburg}
\begin{map}
% \includegraphics[width=\textwidth]{figures/VelarFrontingHall2021-img036.png}
\includegraphics[width=\textwidth]{figures/Map30_14.2.pdf}
\caption[{South Brandenburg}]{{South Brandenburg}. Velar fronting of \ili{WGmc} \textsuperscript{+}[ɣ] in word-initial position illustrates four historical stages:  Stage 2e (before any type of segment), Stage 2c' (before all front vowels), Stage 2d' (before front vowels and back vowels), and Stage 2{}'{}'{}'{}'  (before [ə] < unstressed \textsuperscript{+}[i]). Data have been drawn from \citet{Kieser1963}.}
\label{map:30}
\end{map}

I mention here two other places in Germany where \REF{ex:14:19d} occurred. The first can be observed in a number of the phonetically transcribed texts in \citet{CornelissenEtAl1989}. In their discussion of the dialect features of that broad area of West Central Germany those authors note that some places in Westerwald are characterized by the following pattern (p. 39). I retain here the original transcriptions, whereby [ʝ] corresponds to ⟦J⟧/⟦j⟧.

\ea\label{ex:14:23}
\ea\label{ex:14:23a} Jędicht \tab Gedicht  \tab ‘poem’\\
    jęṣoot \tab  gesagt  \tab  ‘say-\textsc{part}’
\ex\label{ex:14:23b} joof    \tab gab      \tab ‘give-\textsc{pret}’
\ex\label{ex:14:23c} klöövęn \tab glaubten \tab ‘believe-\textsc{1/3}\textsc{pl}.\textsc{part}’\\
    kruęs   \tab groß     \tab ‘large’
\z 
\z 

Recall that \mapref{map:24} depicts places in Westerwald which represent three of the historical stages posited in this book with special reference to word-initial position. The data in \REF{ex:14:23} suggest that there are other places in that same area not depicted on \mapref{map:24} which represent Trigger Type E' (=Stage 2d').

A second example of \REF{ex:14:19d} is a small area to the west of the Elbe River (in the \il{Eastphalian}Eph/\il{Brandenburgish}Brb dialect area), which is indicated on a map in the dialect dictionary for that region (MiElWb: 1087--1090). In particular, that map depicts the places where the reflex of word-initial \textsuperscript{+}[ɣ] is palatal ([ʝ]) before front and back vowels but is retained as velar before a consonant.

\section{{Velar} {fronting} {after} {a} {subset} {of} {back} {vowels}}\label{sec:14.3}

\subsection{Introduction}\label{sec:14.3.1}

The nonassimilatory fronting of historical velars also occurred in the context after a sonorant. In contrast to word-initial position, the developments depicted in \REF{ex:14:24} are attested in the context of high back vowels (HBV), mid back vowels (MBV), and low back vowels (LBV). The velar undergoing  \REF{ex:14:24} could be \ili{WGmc} \textsuperscript{+}[k], \textsuperscript{+}[x] or \textsuperscript{+}[ɣ], depending on the dialect.

\ea%24
\label{ex:14:24}Nonassimilatory velar fronting:
\ea\label{ex:14:24a} velar > palatal / HBV {\longrule}{\longrule}
\ex\label{ex:14:24b} velar > palatal / HBV, MBV {\longrule}{\longrule}
\ex\label{ex:14:24c} velar > palatal / HBV, MBV, LBV {\longrule}{\longrule}
\z
\z 

The changes depicted in \REF{ex:14:24} represent the final phases in the \isi{rule generalization} scenario described in \chapref{sec:12} for postsonorant position. The first five rows of \tabref{tab:14:4} list Trigger Types A-E and the corresponding historical stages. In the final three rows, Trigger Type F/Stage 2e from \tabref{tab:14:1} is decomposed into three separate stages defined according to vowel height. Those three stages applied in the chronological order given below.

\begin{table}
\caption{Trigger Types for front and back segments and the corresponding historical stages\label{tab:14:4}}
\begin{tabular}{lll}
\lsptoprule
Type & Trigger & Stage\\\midrule
A & HFV & 2a\\
B & HFV, MFV & 2b\\
C & HFV, MFV, CC & 2c\\
D & HFV, MFV, LFV & 2c'\\
E & HFV, MFV, LFV, CC & 2d\\\tablevspace
F' & HBV & 2e'\\
F'{}' & HBV, MBV & 2e'{}'\\
F'{}'{}' & HBV, MBV, LBV & 2e'{}'{}'\\
\lspbottomrule
\end{tabular}
\end{table}

In the remainder of this section I focus on \il{Swabian}Swb, \il{Eastphalian}Eph, and \il{Central Hessian}CHes varieties which document the changes in (\ref{ex:14:24a}, \ref{ex:14:24b}). In \sectref{sec:14.4} I discuss dialects that support the general development in \REF{ex:14:24c}.\footnote{{The reflex of \ili{WGmc}} \textrm{\textsuperscript{+}}\textrm{[ɣ] in a word-internal onset is a palatal glide ([j]) in many varieties, especially UG. In the present section I restrict my discussion to sources in which that etymological velar is a fricative. An example of a source I do not consider (\il{Low Alemannic}LAlmc) is \ipi{Ottenheim} (\citealt{Heimburger1887}; \mapref{map:1}). Heimburger states that [x] occurs after front vowels and [ç] after back vowels and that the palatal glide [j] (< \ili{WGmc}} \textrm{\textsuperscript{+}}\textrm{[ɣ]) surfaces in a word-internal onset in the context after front vowels, liquids, and back vowels. The essential facts are the same in the \il{Rhenish Franconian}RFr dialect of \ipi{Spessart} (\citealt{Lauinger1929}; \mapref{map:10}), which I likewise do not consider below.}}

\subsection{Data and discussion }\label{sec:14.3.2}

\citet{Strohmaier1930} describes the \il{Swabian}Swb dialect spoken in and around \ipi{Blaubeuren} (\mapref{map:1}). The author is clear that the dialect possesses [ç] and [x], which he transcribes with the same symbol (⟦x⟧). Of interest is Strohmaier’s (1930: 94--95) description of the distribution of those two dorsal sounds, which I cite below. The important part of this passage is the final sentence, which I have italicized.

\begin{quote}
Die Unterscheidung zwischen gutturaler und palataler Spirans erfolgt nach denselben Gesichtspunkten wie sie schon bei Bopp (S. 16) und bei Keinath (S. 86) aufgeführt worden sind. Nach den dunklen Vokalen \textit{a} und \textit{o} und den unechten Diphthongen \textit{ūə} und \textit{īə} ist die Spirans guttural … Gutturales \textit{x} tritt auch auf in \textit{miləx}, \textit{dswiləx} Zwilch, soweit es nicht geschwunden ist. \textit{Nach} i, u, ẹ, əi, əu \textit{erscheint MHD} ch \textit{dagegen als palatale Spirans}.

“The difference between guttural and palatal fricative is a consequence of the same factors already discussed by Bopp (p. 16) and Keinath (p. 86). After the back vowels \textit{a} and \textit{o} and the pseudo-diphthongs \textit{ūə} and \textit{īə} the fricative is guttural. Guttural \textit{x} also occurs in \textit{miləx}, \textit{dswiləx} Zwilch, unless it was elided. \textit{By contrast, after} i, u, ẹ, əi, əu \textit{MHG} ch \textit{occurs as a palatal fricative}ˮ.
\end{quote}

What is surprising is that [ç] surfaces after both the back vowel [u] and the diphthong [əu]. Aside from that one quirk, the distribution of [x] and [ç] is precisely what one would expect: The velar occurs after a back vowel -- or a diphthong whose second element is back -- and the palatal after a front vowel.

One way of coming to grips with Strohmaier’s surprising description of dorsal fricatives in \ipi{Blaubeuren} is to either deny the facts or question the reliability of the source. As simple and tempting as that strategy might be, it is weak because -- as I make clear below -- several other varieties of German are described in which [ç] patterns with front vowels and high back vowels like [u].

\citet{Müller1911} is a historical description of the sounds in the \il{Swabian}Swb dialect spoken in \ipi{Mühlingen} (\mapref{map:1}). In contrast to \citet{Strohmaier1930}, \citet{Müller1911} does not provide a clear statement concerning the distribution of [ç] and [x], the only two dorsal fricatives in the dialect. However, the correct generalizations concerning the distribution of those sounds can be inferred from Müller’s data because he has two distinct symbols distinguishing velar [x] (his ⟦x⟧) and palatal [ç] (his bold ⟦\textbf{x}⟧). The dialect has four front monophthongs (/i ɪ e ɛ/), seven back monophthongs (/u ʊ o ɔ ɑ ɑː ə/), diphthongs ending in a front vowel (/iɛ ɛɪ/), several diphthongs ending in \isi{schwa} (/iːə uːə ɔə ɛːə/), and the diphthong /ɑu/.

The data in \REF{ex:14:25a} reveal that [x] occurs after the four back monophthongs [o ɔ ɑ ɑː] and after all of the diphthongs ending in \isi{schwa}, and the items in \REF{ex:14:25c} show that [x] occurs after a coronal consonant ([l]). The examples in \REF{ex:14:25b} exemplify the distribution of [ç], which surfaces after all of the front monophthongs and diphthongs ending in a front vowel. The [x] and [ç] in all examples derive from \ili{WGmc} \textsuperscript{+}[k] or \textsuperscript{+}[x].

\TabPositions{.15\textwidth, .33\textwidth, .5\textwidth, .75\textwidth}
\ea%25
\label{ex:14:25}Dorsal fricatives in \ipi{Mühlingen}:
\ea lox  \tab  [lox] \tab Loch \tab ‘hole’ \tab 25\\\label{ex:14:25a}
    dōxt \tab [doːxt] \tab Docht \tab ‘wick’ \tab 56\\
    mɑxt \tab [mɑxt] \tab Macht \tab ‘power’ \tab 56\\
    šn\={ɑ}xlə \tab [ʃnɑːxlə] \tab schnarchen \tab ‘snore\textsc{{}-inf}’ \tab 56\\
    \={ɑ}əx \tab [ɑːəx] \tab Arche \tab ‘ark’ \tab 58\\
    līəxə \tab [liːəxə] \tab Heu aus dem Heustock rupfen\\
    \tab \tab  ‘pick\textsc{{}-inf} hay from hayrick’ \tab 55\\
    būəx \tab [buːəx] \tab Buch \tab ‘book’ \tab 50\\
    wōəx \tab [vɔəx] \tab weich \tab ‘soft \tab 45\\
    wɛ̄əxə \tab [vɛːəxə] \tab angestrengt arbeiten\\
    \tab \tab  ‘work\textsc{{}-inf} intensely’ \tab 58
\ex fi\textbf{x}t \tab [fiçt] \tab feucht \tab ‘damp’ \tab 56\\\label{ex:14:25b}
    wɪ\textbf{x}tɪk \tab [vɪçtɪk] \tab wichtig \tab ‘important’ \tab 56\\
    še\textbf{x}lɪ \tab [ʃeçlɪ] \tab kleiner Heuhaufen\\
    \tab \tab  ‘small-\textsc{infl} haystack’ \tab 55\\
    fɛ\textbf{x}də \tab [fɛçdə] \tab fechten \tab ‘fence\textsc{{}-inf}’ \tab 56\\
    bl\textsuperscript{i}ɛ\textbf{x} \tab [bliɛç] \tab Blech \tab ‘tin’ \tab 14\\
    rɛɪ\textbf{x} \tab [rɛɪç] \tab reich \tab ‘rich’ \tab 37
\ex kʽɑlx \tab [kʰɑlx] \tab Kalk \tab ‘lime’ \tab 54\\\label{ex:14:25c}
    dmɪlx \tab [dmɪlx] \tab die Milch \tab ‘the milk’ \tab 63
\z 
\z 

\ipi{Mühlingen} differs from all dialects discussed up to this point -- with the exception of Strohmaier’s variety of \ipi{Blaubeuren} -- in the sense that palatal [ç] surfaces after a high back vowel, namely [ʊ] in (\ref{ex:14:26a}), [u] in (\ref{ex:14:26b}) and [au] in (\ref{ex:14:26c}). As in \REF{ex:14:25}, the [x] and [ç] in \REF{ex:14:26} derive from \ili{WGmc} \textsuperscript{+}[k] or \textsuperscript{+}[x].

\ea%26
\label{ex:14:26}Dorsal fricatives in \ipi{Mühlingen}:
\ea fʊ\textbf{x}dlə \tab [fʊçdlə] \tab fuchteln \tab ‘wave about\textsc{{}-inf}’ \tab 56\\\label{ex:14:26a}
    sʊ\textbf{x}t \tab [sʊçt] \tab Art von Krankheit\\
    \tab \tab  ‘type of sickness’ \tab 27\\
    ɡrʊ\textbf{x} \tab [grʊç] \tab Geruch \tab ‘smell \tab 62\\
    kʽʊ\textbf{x}ɪ \tab [kʰʊçɪ] \tab Küche \tab ‘kitchen’ \tab 55\\
    fʊ\textbf{x}dsɛ̅ə \tab [fʊçdsɛːə] \tab fünfzehn \tab ‘fifteen’ \tab 59\\
    fʊ\textbf{x}dsk \tab [fʊçdsk] \tab fünfzig \tab ‘fifty’ \tab 59\\
\ex fru\textbf{x}bɑr \tab [fruçbɑr] \tab fruchtbar \tab ‘fertile’ \tab 66\label{ex:14:26b}
\ex bau\textbf{x} \tab [bɑuç] \tab Bauch \tab ‘stomach’ \tab 41\\\label{ex:14:26c}
    bau\textbf{x}we \tab [bɑuçve] \tab Bauchweh \tab ‘stomach ache’ \tab 41\\
    brau\textbf{x}ə \tab [brɑuçe] \tab brauchen \tab ‘need\textsc{{}-inf}’ \tab 41\\
    hau\textbf{x}ə \tab [hɑuçe] \tab hauchen \tab ‘aspirate\textsc{{}-inf}’ \tab 41\\
    šlau\textbf{x} \tab [ʃlɑuç] \tab Schlauch \tab ‘hose’ \tab 41\\
    rau\textbf{x} \tab [rɑuç] \tab Rauch \tab ‘smoke’ \tab 47
\z 
\z 

In sum, [x] and [ç] in \ipi{Mühlingen} stand in complementary distribution: [ç] after front vowels or coronal consonants and high back vowels and [x] after all other back vowels. From the historical perspective, the nonassimilatory change in \REF{ex:14:24a} was active.

\citet{Dreher1919} describes the \il{Swabian}Swb variety spoken in and around \ipi{Liggersdorf} (\mapref{map:1}). The dialect possesses four front monophthongs (/iː ɪ ɛː ɛ/), seven back monophthongs (/uː ʊ ɔː o ɑː ɑ ə/), two diphthongs ending in a front vowel (/ei ɛi/), several diphthongs ending in \isi{schwa} (e.g. /ɔːə ɛə/), and the diphthong /əu/. The only two dorsal fricatives are [x] and [ç].

The data presented in the original source reveal the following generalizations: (a) Palatal [ç] (=⟦χ⟧) occurs after the front monophthongs or diphthongs ending in a front vowel in (\ref{ex:14:27a}), a coronal sonorant consonant in (\ref{ex:14:27b}), or a high back monophthong, i.e. either [uː] in (\ref{ex:14:27c}) or [ʊ] in (\ref{ex:14:27d}).

\ea%27
\label{ex:14:27}Dorsal fricatives in \ipi{Liggersdorf}:
\ea\label{ex:14:27a} bīχdə \tab [biːçdə] \tab Beichte \tab ‘confession’ \tab 38\\
    kwĭχd \tab [kvɪçd] \tab Gewicht \tab ‘weight’ \tab 23\\
    sē̜χə \tab [sɛːçə] \tab verstohlen schauen\\
    \tab \tab  ‘look\textsc{{}-inf} sneakily’ \tab 35\\
    hĕͅχl̥ \tab [hɛçl̩] \tab Hechel \tab ‘hatchel’ \tab 20\\
    leiχt \tab [leiçt] \tab leicht \tab ‘easy’ \tab 37\\
    rẹ̆iχ \tab [rɛiç] \tab  reich \tab ‘rich’ \tab 37
\ex\label{ex:14:27b} khĭərχə \tab [kʰɪərçə] \tab Kirche \tab ‘church’ \tab 25\\
    milχ \tab [milç] \tab Milch \tab ‘milk’ \tab 52
\ex\label{ex:14:27c} bfūχə \tab [bfuːçə] \tab fauchen \tab ‘hiss\textsc{{}-inf}’ \tab 75\\
    būχ \tab [buːç] \tab Bauch \tab ‘stomach’ \tab 39\\
    khūχə \tab [kʰuːçə] \tab hauchen \tab ‘aspirate\textsc{{}-inf}’ \tab 39
\ex\label{ex:14:27d} tsŭχd \tab [tsʊçd] \tab Zucht \tab ‘breeding’ \tab  74\\
    sŭχd \tab [sʊçd] \tab Sucht \tab ‘addiction’ \tab  28\\
    trŭχə \tab [trʊçə] \tab Truhe \tab ‘chest’ \tab 28\\
    khŭχi \tab [kʰʊçi] \tab Küche \tab ‘kitchen’ \tab 30
\z 
\z 

The author is consistent in transcribing her symbol for [ç] in each of the four contexts in \REF{ex:14:27}, even after high back vowels in (\ref{ex:14:27c}, \ref{ex:14:27d}). \citet[74]{Dreher1919} herself realizes that there is a significant generalization concerning the context for [ç], which she describes as after front (“hellˮ) vowels. Apparently Dreher considers ⟦ŭ ū⟧ to be front.\footnote{{The type of vowel described here appears to have been recognized in the literature on German dialectology. See in particular the chart for vowels in \citet[1]{Wiesinger1970a}, which is based on the one proposed by \citet{SchmittWiesinger1964}. In that system there are two categories of sounds I call “back”, namely velar rounded (“velar gerundet”) and palato-velar (“palato-velar”); significantly, vowels in the latter category are considered to be centralized (“zentralisiert”). In the system proposed by Wiesinger, there are two distinct sets of symbols, e.g. ⟦u⟧ is velar rounded and ⟦}\textrm{\textbf{u}}\textrm{⟧ is palato-velar. Several of the case studies in \citet{Wiesinger1970a} dealing with \il{Low Alemannic}LAlmc have the centralized back vowel.} }

In \REF{ex:14:28} I provide data with dorsal fricatives in the remaining contexts listed above, namely after the one diphthong ending in a high back vowel ([əu]) in \REF{ex:14:28a}, diphthongs ending in \isi{schwa} in (\ref{ex:14:28b}, \ref{ex:14:28c}), mid back vowels [ɔː ɔ] in \REF{ex:14:28d}, and the low back vowel [ɑ] in \REF{ex:14:28e}. Note that for each of the first five categories some words are attested with [x] (=⟦x⟧) and others with [ç]. One token was found with [ç] after a low back vowel, i.e. [lɑçə] ‘laugh\textsc{{}-inf}’ (=⟦lɑχə⟧). I assume that [x] is the unmarked pronunciation for the dorsal fricative after [ɑ] and comment on that one exceptional item below.\largerpage

\ea%28
\label{ex:14:28}Dorsal fricatives in \ipi{Liggersdorf}:
\ea\label{ex:14:28a} šləux \tab [ʃləux] \tab Schlauch \tab ‘hose’ \tab 39\\
    rəuχ \tab [rəuç] \tab Rauch \tab ‘smoke’ \tab 41
\ex\label{ex:14:28b} blǭəx \tab [blɔːəx] \tab bleich \tab ‘pale’ \tab 75\\
    ǭəxr̥ \tab [ɔːəxr̩] \tab Eichhorn \tab ‘squirrel’ \tab 44\\
    tsǭəxə \tab [tsɔːəxə] \tab Zeichen \tab ‘sign’ \tab 45\\
    sǭəχ \tab [sɔːəç] \tab Harn \tab ‘urine’ \tab 45\\
    ɡlǭəχ \tab [glɔːəç] \tab Gelenk \tab ‘joint’ \tab 44
\ex\label{ex:14:28c} šdĕͅəxə \tab [ʃdɛəxə] \tab stechen \tab ‘sting\textsc{{}-inf}’ \tab 96\\
    bĕͅəx \tab [bɛəx] \tab Pech \tab ‘misfortune’ \tab 75\\
    fĕͅəχdə \tab [fɛəçdə] \tab betteln \tab ‘beg\textsc{{}-inf}’ \tab  21\\
    sĕͅəχtsē̜ə \tab [sɛəçtsɛːə] \tab sechzehn \tab ‘sixteen’ \tab  21
\ex\label{ex:14:28d} šǭxə \tab [ʃɔːxə] \tab Heuhaufen \tab ‘haystack’ \tab 75\\
    fəlọ̆χərə \tab [fəlɔçərə] \tab in die Erde vergraben \\
    \tab \tab ‘bury\textsc{{}-inf} in the ground’ \tab 26\\
    brǭχət \tab [brɔːçət] \tab Brachmonat \tab ‘fallow month’ \tab 33\\
    fədǭχt \tab [fədɔːçt] \tab Verdacht \tab ‘suspicion’ \tab  33
\ex\label{ex:14:28e} bɑx \tab [bɑx] \tab Bach \tab ‘stream’ \tab 16
\z 
\z 

I hypothesize that the [x] pronunciation in (\ref{ex:14:28a}--\ref{ex:14:28d}) represents one group of speakers (Variety A) and that the [ç] realization characterizes a separate set of speakers (Variety B). The generalizations are the following: For Variety A, [ç] only surfaces after a coronal sonorant (=\ref{ex:14:27a}, \ref{ex:14:27b}) or a high back monophthong (=\ref{ex:14:27c}, \ref{ex:14:27d}) and [x] after any other back vowel (=[x] realization in \ref{ex:14:28}). For Variety B, [ç] surfaces after a coronal sonorant (=\ref{ex:14:27a}, \ref{ex:14:27b}) or any nonlow back vowel, including diphthongs ending in a nonlow vowel (=\ref{ex:14:27c}, \ref{ex:14:27d} and the [ç] realization in \ref{ex:14:28a}--\ref{ex:14:28d}), while [x] occurs after a low back vowel (=\ref{ex:14:28e}). This suggests that \REF{ex:14:24a} was the version of velar fronting that was active for Variety A -- with the added provision that the high back vowel be a monophthong -- and that the more general context in \REF{ex:14:24b} was the one that held for Variety B. It is difficult to interpret the item [lɑçə] ‘laugh\textsc{{}-inf}’ referred to above. That token might be a mistranscription, or it could indicate that for some speakers the more general change in \REF{ex:14:24c} has transpired (or was in the process of transpiring in 1919).\footnote{{An alternative interpretation is that Variety A and Variety B are present in the grammar of a single individual. It is not possible to know for sure whether or not this is true for \ipi{Liggersdorf}, but this is clearly the correct interpretation for the \il{Swabian}Swb speakers of Beuren\ip{Beuren (Allgäu)} investigated by \citet{BausingerRuoff1959}; see \mapref{map:3}. Bausinger \& Ruoff provide phonetically transcribed texts for four speakers of Beuren.\ip{Beuren (Allgäu)} All four have palatal [ç] after front segments and [x] after back vowels, but the first three speakers also have several instances of [ç] in the context after back vowels (e.g. [brəuçɐ] ‘need-}\textrm{\textsc{inf}}\textrm{’, [nɔçər] ‘afterward’). These speakers have internalized both assimilatory velar fronting, whereby /x/ surfaces as palatal after any coronal sonorant, as well as some version of \isi{nonassimilatory velar fronting} from \REF{ex:14:24}. Since the assimilatory pattern corresponds to the one for \il{Standard German}StG (\chapref{sec:17}), the Beuren\ip{Beuren (Allgäu)} speakers appear to be \isi{diglossic}.}}

\citet{Jarfe1929} describes the \il{Eastphalian}Eph dialect once spoken in \ipi{Ramlingen} (\mapref{map:7}), which has front vowels (/i iː y yː e eː ɛː ø øː/), back vowels (/u uː o oː ɑ ɑː ə/) and three diphthongs (/ɑi ɑu oi/), as well as the two dorsal fricatives [x] and [ç].

The following data indicate that [ç] (=⟦χ⟧) surfaces after front vowels in (\ref{ex:14:29a}--\ref{ex:14:29g}) or coronal sonorant consonants in (\ref{ex:14:29h}). There are copious examples of words like these, which are consistently transcribed with Jarfe’s symbol for [ç] and never with the symbol for [x] (=⟦x⟧).

\ea%29
\label{ex:14:29}Dorsal fricatives in \ipi{Ramlingen}:
\ea\label{ex:14:29a} twīχ \tab [tviːç] \tab Zweig \tab ‘branch’ \tab 25
\ex\label{ex:14:29b} liχtə \tab [liçtə] \tab leicht \tab ‘light’ \tab 26\\
    tjiχt \tab [tjiçt] \tab Gicht \tab ‘gout’ \tab 18\\
    liχt \tab [liçt] \tab liegt \tab ‘lie\textsc{{}-3sg}’ \tab 19
\ex\label{ex:14:29c} t\={y}χ \tab [tyːç] \tab Zeug \tab ‘stuff’ \tab 30
\ex\label{ex:14:29d} dyχdiχ \tab [dyçdiç] \tab tüchtig \tab ‘capable’ \tab 22
\ex\label{ex:14:29e} weχ \tab [veç] \tab Weg \tab ‘path’ \tab 17\\
    sleχt \tab [sleçt] \tab schlecht \tab ‘bad’ \tab 17
\ex\label{ex:14:29f} høχtə \tab [høçtə] \tab Höhe \tab ‘height’ \tab 30\\
    krøχəln \tab [krøçəln] \tab husten \tab ‘cough\textsc{{}-inf}’ \tab 21
\ex\label{ex:14:29g} n\={ø̜}χtə \tab [nøːçtə] \tab Nähe \tab ‘vicinity’ \tab 24
\ex\label{ex:14:29h} dolχ \tab [dolç] \tab Dolch \tab ‘dagger’ \tab 11
\z 
\z 

The data in \REF{ex:14:29} are consistent with Jarfe’s (\citeyear{Jarfe1929}: 11) assertion that the palatal occurs after front vowels and consonants (“nach hellen Vokalen und Konsonantenˮ). However, the data in \citet{Jarfe1929} containing back monophthongs or diphthongs ending in a back vowel indicate that [ç] is not limited to the coronal sonorant environment. In fact, in the overwhelming number of items with nonlow back vowels followed by a dorsal fricative, that fricative is represented with Jarfe’s symbol for [ç]. That being said, some items can be found in which nonlow back vowels are followed by [x]. In (\ref{ex:14:30a}--\ref{ex:14:30e}) I provide representative examples for words with five nonlow back vowels (including the diphthong [ɑu]) followed by a dorsal fricative. Data like the ones in \REF{ex:14:30f} indicate that [x] surfaces after low back vowels, although I found one token with [ç] in that context, namely [dinzədɑç] ‘Tuesday’ (=⟦dinzədɑχ⟧).

\TabPositions{.22\textwidth, .48\textwidth, .64\textwidth,  .84\textwidth}
\ea%30
\label{ex:14:30}Dorsal fricatives in \ipi{Ramlingen}:
\ea\label{ex:14:30a} tjūχən \tab [tjuːçən] \tab  kreischen \tab ‘screech\textsc{{}-inf}’ \tab 27
\ex\label{ex:14:30b} plɑuχ \tab [plɑuç] \tab Pflug \tab ‘plow’ \tab 26\\
    ɡənɑuχ, ɡənɑux \tab [gənɑuç], [gənɑux] \tab genug \tab ‘enough’ \tab 26  
\ex\label{ex:14:30c} slōχ \tab [sloːç] \tab schlug \tab ‘strike\textsc{{}-pret}’ \tab 26\\
    lōχ \tab [loːç] \tab log \tab ‘lie\textsc{{}-pret}’ \tab 29\\
    tōχ, tōx \tab  [toːç] [toːx] \tab  zog \tab ‘pull\textsc{{}-pret}’ \tab  11
\ex\label{ex:14:30d} nox \tab [nox] \tab  noch \tab ‘still’ \tab 19\\
    trox \tab [trox] \tab Trog \tab ‘trough’ \tab 20\\
    soχt \tab [soçt] \tab sucht \tab ‘search\textsc{{}-3sg}’ \tab 26\\
    hoχtīt \tab [hoçtiːt] \tab Hochzeit \tab ‘wedding’ \tab 29
\ex\label{ex:14:30e} tǭx \tab [tɔːx] \tab zähe \tab ‘tough’ \tab 23\\
    wɑidǭχ \tab [vɑidɔːç] \tab Schmerzen \tab ‘pain-\textsc{pl}’ \tab 28
\ex\label{ex:14:30f} ɑxt \tab [ɑxt] \tab acht \tab ‘eight’ \tab 14\\
    dɑxt \tab [dɑxt] \tab Docht \tab ‘wick’ \tab 23\\
    slɑx \tab [slɑx] \tab Schlag \tab ‘blow’ \tab 14\\
    b\={ɑ}x \tab [bɑːx] \tab Berg \tab ‘mountain’ \tab 17
\z
\z 

I posit that there are two groups of speakers: Variety A and Variety B. For Variety A, [ç] only occurs after coronal sonorants (=\ref{ex:14:29}) and [x] only after back vowels (=[x] realization in \ref{ex:14:30}), but for Variety B, [ç] surfaces after coronal sonorants (=\ref{ex:14:29}) or nonlow back vowels (=[ç] realization in \ref{ex:14:30a}--\ref{ex:14:30e}), and [x] after low back vowels (=\ref{ex:14:30f}). Thus, nonassimilatory fronting has not yet affected Variety A, but for Variety B, \REF{ex:14:24b} obtains. Since the [dinzədɑç] ‘Tuesday’ example mentioned above appears to be an isolated example it is difficult to know whether or not this item was simply mistranscribed or if there is a third group of speakers for which \REF{ex:14:24c} occurred or was in the process of occurring in 1929.\footnote{{It is also conceivable that both Variety A and Variety B are present in the grammar of a single individual. Since Variety A corresponds to the \il{Standard German}StG pattern, this alternative interpretation points to \isi{diglossia}: Variety A is the \il{Standard German}StG rule of velar fronting, and Variety B is a nonassimilatory version, which is the local dialect.}}

Recall from \sectref{sec:9.2} the \il{Central Hessian}CHes variety spoken in \ipi{Wissenbach} (\citealt{Kroh1915}; \mapref{map:11}), which has the phonemic front oral vowels /i e ɛ/ and back oral vowels /u o ɔ ɑ ə/; most of those vowels can surface as short or long. As described earlier, in that dialect \ili{WGmc} \textsuperscript{+}[k] and \textsuperscript{+}[x] regularly neutralized to [ç] after a coronal sonorant, although \isi{Monophthongization} (/ei/ > /ɑː/) later led to the development of the palatal phoneme /ç/. \ipi{Wissenbach} is significant in the context of this chapter because of the development of \ili{WGmc} \textsuperscript{+}[ɣ]. That sound regularly shifted to [ʝ] after a coronal sonorant in (\ref{ex:14:31b}, \ref{ex:14:31c}) and was retained as [ɣ] after a low back vowel in (\ref{ex:14:31a}). However, in the context after a nonlow back vowel,  \ili{WGmc} \textsuperscript{+}[ɣ] shifted to [ʝ] by \REF{ex:14:24b}; see \REF{ex:14:31d}. Examples like the one in (\ref{ex:14:31d}) illustrate that \ili{WGmc} \textsuperscript{+}[ɣ] failed to undergo \REF{ex:14:24b} after [ɔː] which derived historically from a low back vowel (cf. \ili{MHG} [ɑ]). Note the contrast between [ɣ] and [ʝ] after [ɔː] and before \isi{schwa}.


\TabPositions{.2\textwidth, .4\textwidth, .6\textwidth,  .8\textwidth}
\ea%31
\label{ex:14:31}Dorsal fricatives in \ipi{Wissenbach}:
\ea\label{ex:14:31a} ɑ̄γ \tab [ɑːɣ]  \tab  Auge \tab ‘eye’  \tab  120\\
    ɑ̄γə \tab [ɑːɣə] \tab Augen \tab ‘eye-\textsc{pl}’  \tab  120
\ex\label{ex:14:31b} bl\={ę}jə \tab [plɛːʝə] \tab pflegen \tab ‘care for\textsc{{}-inf}’ \tab 76\\
    rējəl \tab [reːʝəl] \tab Regel \tab ‘rule’  \tab  77\\
    ēj \tab [eːʝ]  \tab  Egge \tab ‘harrow’  \tab  120
\ex\label{ex:14:31c} foljə \tab [folːʝə] \tab folgen \tab ‘follow\textsc{{}-inf}’ \tab 81\\
    bǭjə \tab [bɔːʝə] \tab Bogen \tab ‘bow’  \tab  82\\
    ɡəflǭjə \tab [gəflɔːʝə] \tab geflogen \tab ‘fly\textsc{{}-part}’ \tab 120
\ex\label{ex:14:31d} mǭγə \tab [mɔːɣə] \tab Magen \tab ‘stomach’ \tab 71
\z
\z 

There are no examples listed in \citet{Kroh1915} in which [ɣ] surfaces after a high back vowel like [u] or after mid back vowels other than [ɔː]. The reason for those gaps is that the original \ili{MHG} vowels neutralized to other sounds or deleted.

\section{{Velar} {fronting} {after} {all} {sonorants}}\label{sec:14.4}

\subsection{Introduction}\label{sec:14.4.1}

The nonassimilatory change examined below involves the fronting of a \ili{WGmc} velar sound in postsonorant position after any type of vowel (=\ref{ex:14:24c}). The velar under discussion can be \ili{WGmc} \textsuperscript{+}[ɣ],\textsuperscript{ +}[k], and/or \textsuperscript{+}[x], depending on the dialect.

The reflexes attested in the material cited below for \ili{WGmc} \textsuperscript{+}[ɣ] and \textsuperscript{+}[x]/\textsuperscript{+}[k] in postsonorant position are summarized in \REF{ex:14:32}. I comment on those stages in greater detail below. The dorsal fricative in the three sequences given in phonetic representation in each row correspond to the attested realization of that original velar. The symbols “[i]ˮ, “[l]ˮ and “[ɑ]ˮ represent the natural classes of front vowels, coronal sonorant consonants and back vowels respectively. As noted earlier, the coronal sonorant consonants referred to here are [l] and/or [r] depending on dialect. [n] is also attested, although the number of those examples is relatively small, and many of the sources do not include those examples. As indicated below, the four categories in \REF{ex:14:32a} and \REF{ex:14:32b} are argued to correspond to four distinct historical stages.

\ea%32
\settowidth\jamwidth{(=Stage 2c/\ref{ex:14:2}d)}
\columnsep=-6em\label{ex:14:32}\begin{multicols}{2}\raggedcolumns
\ea Postsonorant \textsuperscript{+}[ɣ]:\label{ex:14:32a}
\ea \relax [iɣə  lɣə   ɑɣə] 
\ex \relax [iʝə   lɣə   ɑɣə]
\ex \relax [iʝə   lʝə   ɑɣə]
\ex \relax [iʝə   lʝə   ɑʝə]
\z
\ex Postsonorant \textsuperscript{+}[x], \textsuperscript{+}[k]:\label{ex:14:32b}
\ea \relax [ixə   lxə   ɑxə]  \jambox{(=Stage 1)}
\ex \relax [içə   lxə   ɑxə]  \jambox{(=Stage 2c')}
\ex \relax [içə   lçə   ɑxə]  \jambox{(=Stage 2c/2d)}
\ex \relax [içə   lçə   ɑçə]  \jambox{(=Stage 2e'{}'{}')}
\z
\z 
\end{multicols}
\z 

At Stage 1 (=\ref{ex:14:32a}i and \ref{ex:14:32b}i) historical velars are retained as velar. When velar fronting is phonologized it does so first in the context after high vowels (Stage 2a) and then after high vowels and mid vowels (Stage 2b), two changes not depicted above. The next incremental change is Stage 2c' (=\ref{ex:14:32a}ii and \ref{ex:14:32b}ii), whereby historical velars are realized as palatal after a front vowel but not after a coronal sonorant consonant, and elsewhere as velar. At Stage 2c/2d (=\ref{ex:14:32a}iii and \ref{ex:14:32b}iii) the velar changes to palatal after a front vowel (or a nonlow front vowel) or coronal consonant but is retained as velar after a back vowel. Those assimilatory changes were examined from the point of view of \isi{rule generalization} in \chapref{sec:12}. Stage 2e'{}'{}' (=\ref{ex:14:32a}iv and \ref{ex:14:32b}iv) reflects the most advanced fronting stage -- the nonassimilatory one -- because original velars are realized as palatal after front vowels, sonorant consonants and (crucially) back vowels. As indicted earlier in \tabref{tab:14:4}, Stage 2e'{}'{}' is the third and final nonassimilatory change after Stage 2e' and 2e'{}', which are not depicted in \REF{ex:14:32}.

There are solid descriptions for a number of varieties exhibiting Stage 2e'{}'{}' (=\ref{ex:14:32a}iv and \ref{ex:14:32b}iv) to varying degrees. In several of those varieties, the lenis velar fricative regularly shows \isi{nonassimilatory velar fronting} (=\ref{ex:14:32a}iv), while the fortis fricative only shows that change to a limited extent (=\ref{ex:14:32b}iv). The neighboring \il{Moselle Franconian}MFr varieties in \ipi{Luxembourg}, Belgium, and Germany discussed in \sectref{sec:14.5} display the nonassimilatory shift from any velar fricative to palatal, regardless of the historical source (=\ref{ex:14:32a}iv and \ref{ex:14:32b}iv).

Most of the dialects discussed below exhibit the historical merger of various back vowels as well as the deletion of etymological [ɣ] in intervocalic position. A consequence of those developments is that there are gaps involving [ʝ], which is attested only after a subset of the phonemic back vowels. For example, [ʝ] surfaces in some dialects only after long nonhigh back vowels like [oː ɑː], but there are no examples with [ʝ] after the corresponding short vowels ([ɔ ɑ]) or high back vowels like [u uː]. In the type of dialect described here I assume that \ili{WGmc} \textsuperscript{+}[ɣ] underwent fronting after all back vowels (=\ref{ex:14:24c}), although a weaker position is that the change only occurred after a subset of the back vowels (=\ref{ex:14:24a}, \ref{ex:14:24b}).

\subsection{Data and discussion}\label{sec:14.4.2}

The data in (\ref{ex:14:33a}--\ref{ex:14:33c}) from the \il{Westphalian}Wph variety once spoken in \ipi{Soest} (\sectref{sec:4.3}) illustrate Stage 1 for \ili{WGmc} \textsuperscript{+}[ɣ] (=\ref{ex:14:32a}i) and the data in (\ref{ex:14:33d}, \ref{ex:14:33e}) the assimilatory change (Stage 2c/2c{}'/2d) for \ili{WGmc} \textsuperscript{+}[x] (=\ref{ex:14:32b}ii or \ref{ex:14:32b}iii).


\TabPositions{.15\textwidth, .35\textwidth, .5\textwidth, .82\textwidth}
\ea%33
\label{ex:14:33}Dorsal fricatives in \ipi{Soest}:
\ea\label{ex:14:33a} liʓə \tab [lɪɣə] \tab liege \tab ‘lie\textsc{{}-1sg}’ \tab 44\\
    lèʓə \tab [lɛɣə] \tab lege \tab ‘place-\textsc{1sg}’ \tab 44
\ex\label{ex:14:33b} bɑlʓə \tab [bɑlɣə] \tab  Balge \tab ‘brat-\textsc{dat}.\textsc{sg}’ \tab 44
\ex\label{ex:14:33c} vɑ̄ʓn \tab [vɑːɣn̩] \tab Wagen \tab ‘car’ \tab 45\\
    ròʓə \tab [ʀɔɣə] \tab Roggen \tab ‘rye’ \tab 44
\ex\label{ex:14:33d} trèct\textit{ɑ} \tab [tʀɛçtɐ] \tab Trichter \tab ‘funnel’ \tab 14
\ex\label{ex:14:33e} dɑxtə \tab [dɑxtə] \tab dachte \tab ‘think\textsc{{}-pret}’ \tab 44
\z 
\z 

By contrast, the \il{Eastphalian}Eph dialect of \ipi{Eilsdorf} (\sectref{sec:8.3}) in \REF{ex:14:34} represents Stage 2c/2d for \ili{WGmc} \textsuperscript{+}[ɣ] (=\ref{ex:14:32a}iii) and Stage 2c/2d for \ili{WGmc} \textsuperscript{+}[x] (=\ref{ex:14:32b}iii). The examples in \REF{ex:14:33} and \REF{ex:14:34} were discussed earlier and therefore require no comment.

\ea%34
\label{ex:14:34}Dorsal fricatives in \ipi{Eilsdorf}:
\ea\label{ex:14:34a} l\k{i}jən \tab [lɪʝən] \tab liegen \tab ‘lie\textsc{{}-inf}’ \tab 342\\
    fęęjən \tab [fɛːʝən] \tab  fegen \tab ‘sweep\textsc{{}-inf}’ \tab 342
\ex\label{ex:14:34b} mǫrjən \tab  [mɔrʝən] \tab morgen \tab ‘tomorrow’ \tab 342\\
    feljə \tab  [felʝə] \tab Radfelge \tab ‘wheel rim’ \tab 342
\ex\label{ex:14:34c} fǫʒəl \tab [fɔɣəl] \tab Vogel \tab ‘bird’ \tab 342\\
    swɑ̊ɑ̊ʒər \tab [swɑːɣər] \tab Schwager \tab ‘brother-in-law’ \tab 342
\ex\label{ex:14:34d} b\k{i}ct \tab [bɪçt] \tab Beichte \tab ‘confession’ \tab 341
\ex\label{ex:14:34e} fr\k{u}xt \tab [frʊxt] \tab Frucht \tab ‘fruit’ \tab 341
\z 
\z 


The remaining datasets display the nonassimilatory developments in \REF{ex:14:24}. Consider first material from the two (\il{North Hessian}NHes) places in (\ref{ex:14:35}--\ref{ex:14:36}). The words listed in \REF{ex:14:35} from \ipi{Loshausen-Zella} (\citealt{Schoof1913a, Schoof1913b, Schoof1913c}; \mapref{map:11}) exhibit the change from \ili{WGmc} \textsuperscript{+}[ɣ] to [ʝ] after a front vowel (=\ref{ex:14:35a}), coronal sonorant consonant (=\ref{ex:14:35b}), or back vowel (=\ref{ex:14:35c}). The items provided in (\ref{ex:14:35d}--\ref{ex:14:35f}) show the modern reflexes of \ili{WGmc} \textsuperscript{+}[k x] in postsonorant position: [ç] surfaces after a front vowel (=\ref{ex:14:35d}) or a coronal sonorant consonant (=\ref{ex:14:35e}) and [x] after a back vowel (=\ref{ex:14:35f}). In the context after [u] the original source \citep[209]{Schoof1913c} provides several examples like the ones in \REF{ex:14:35g}, which show that [u] can be followed by either the velar or (surprisingly) the palatal. The phonemic front vowels in this variety are /i ɪ e ɛ ø æ/ and the phonemic full back vowels /u o ɔ ɑ/; most of those vowels can surface as short or long.\footnote{{I do not include /ə/ among the back vowels of \ipi{Loshausen-Zella} or in any of the dialects listed below because that vowel fails to occurs in the context before dorsal fricatives. I likewise do not include diphthongs among the phonemic vocalic sounds.}}

\ea%35
\label{ex:14:35}Dorsal fricatives in \ipi{Loshausen-Zella}:
\ea\label{ex:14:35a} laijə \tab [lɑiʝə] \tab liegen \tab ‘lie\textsc{{}-inf}’ \tab 207\\
    ööjə \tab [øːʝə] \tab Augen \tab ‘eye-\textsc{pl}’ \tab 207\\
    sääjə \tab [sæːʝə] \tab Segen \tab ‘blessing’ \tab 207
\ex\label{ex:14:35b} šwäljə \tab [ʃvælʝə] \tab schwelgen \tab ‘wallow\textsc{{}-inf}’ \tab 207\\
    mǫrjə \tab [mɔrʝə] \tab morgen \tab ‘tomorrow’ \tab 207
\ex\label{ex:14:35c} ɡəfloojə \tab [gəfloːʝə] \tab geflogen \tab ‘fly\textsc{{}-part}’ \tab 207\\
    frɑ̊ɑ̊j \tab [frɑːʝ] \tab fragen \tab ‘ask\textsc{{}-inf}’ \tab 207\\
\ex\label{ex:14:35d} ricə \tab [riçə] \tab riechen \tab ‘smell\textsc{{}-inf}’ \tab 209\\
    rööcərn \tab [røːçərn] \tab rauchen \tab ‘smoke\textsc{{}-inf}’ \tab 209\\
    šlääct \tab [ʃlæːçt] \tab schlecht \tab ‘bad’ \tab 209
\ex\label{ex:14:35e} šilcə \tab [ʃilçə] \tab schielen \tab ‘squint\textsc{{}-inf}’ \tab 210
\ex\label{ex:14:35f} hoox \tab [hoːx] \tab hoch \tab ‘high’ \tab 209\\
    šbrɑ̊ɑ̊x \tab [ʃprɑːx] \tab Sprache \tab ‘language’ \tab 209
\ex\label{ex:14:35g} bux, buc \tab [bux], [buç] \tab Bauch \tab ‘stomach’ \tab 209
\z 
\z 

Schoof does not say much about the items in \REF{ex:14:35g}, other than the fact that this optionality sometimes (“zuweilenˮ) exists. Three points are in need of clarification: First, Schoof’s examples all involve words with [u], but he does not state explicitly that the optionality only holds after that one vowel. Second, it is not clear whether or not the same optionality holds for [x]/[ç] after [u] in all words with that vowel. Third, we cannot know for sure how to interpret the optionality itself. Recall that I accounted for examples in \ipi{Liggersdorf} in (\ref{ex:14:28a}--\ref{ex:14:28d}) and \ipi{Ramlingen} in (\ref{ex:14:30b}--\ref{ex:14:30e}) where the velar and palatal occur after the same back vowel by postulating that the two pronunciations reflect two different sets of speakers (varieties). I hypothesize that the two realizations in \REF{ex:14:35g} are likewise speaker-dependent; hence, some speakers have [bux], while others have [buç].

The data in \REF{ex:14:35} point to two different stages depending on the target velar fricative: Stage 2e'{}'{}' (=\ref{ex:14:24c}) for [ɣ] (<\ili{MHG} \textsuperscript{+}[ɣ]) and Stage 2e' (=\ref{ex:14:24a}) for [x] (<\ili{MHG} \textsuperscript{+}[x k]), but only for those speakers with the pronunciation [buç] ‘stomach’.

In the \il{North Hessian}NHes dialect of \ipi{Blankenheim} (\citealt{Dittmar1891}; \mapref{map:11}) \ili{WGmc} \textsuperscript{+}[ɣ] is realized as palatal [ʝ] after a front vowel (=\ref{ex:14:36a}), coronal sonorant consonant (=\ref{ex:14:36b}), or back vowel (=\ref{ex:14:36c}). By contrast, [x] (<\ili{WGmc} \textsuperscript{+}[k x]) undergoes assimilatory velar fronting in the context after a coronal sonorant (=\ref{ex:14:32b}iii), as in (\ref{ex:14:36d}, \ref{ex:14:36e}), and otherwise surfaces as velar after a back vowel, as in \REF{ex:14:36f}. The dialect possesses the phonemic front vowels /i ɪ y e ɛ/ and the phonemic back vowels /u o ɔ ɑ/. Most of those sounds can surface as short or long. Due to historical neutralizations of various vowels referred to above no examples involve [ʝ] in the context after high and low back vowels like [u ɑ].

\ea\label{ex:14:36} Dorsal fricatives in \ipi{Blankenheim}:
\ea\label{ex:14:36a} îjəl \tab [iːʝəl] \tab Igel \tab ‘hedgehog’ \tab 42\\
    sê·jəl \tab [seːʝəl] \tab Segel \tab ‘sail’ \tab 42
\ex\label{ex:14:36b} ɡɑljən \tab [gɑlʝən] \tab Galgen \tab ‘gallows’ \tab 42\\
    mórjən \tab [mɔʀʝə] \tab morgen \tab ‘tomorrow’ \tab 42
\ex\label{ex:14:36c} fôjəl \tab [foːʝəl] \tab Vogel \tab ‘bird’ \tab 42\\
    frô·j \tab [fʀoːʝ] \tab fragen \tab ‘ask\textsc{{}-inf}’ \tab 43
\ex\label{ex:14:36d} síçəl \tab [sɪçəl] \tab Sichel \tab ‘sickle’ \tab 44\\
    slêç.d \tab [sleːçt] \tab schlecht \tab ‘bad’ \tab 44
\ex\label{ex:14:36e} ke·rç \tab [kerç] \tab Kirche \tab ‘church’ \tab 44
\ex\label{ex:14:36f} bûc \tab [buːx] \tab Buch \tab ‘book’ \tab 29\\
    no.cd \tab [nɔxt] \tab Nacht \tab ‘night’ \tab 44\\
    šɑcdəl \tab [ʃɑxtəl] \tab Schachtel \tab ‘box’ \tab 44
    \z
\z 

The data in \REF{ex:14:36} exemplify Stage 2e'{}'{}' (=\ref{ex:14:24c}) for [ɣ] (<\ili{MHG} \textsuperscript{+}[ɣ]) and Stage 2c/2d (=\ref{ex:14:32b}iii) for [x] (<\ili{MHG} \textsuperscript{+}[x k]).

\ipi{Kirchspiel Courl} (\citealt{Beisenherz1907}; \mapref{map:6}) illustrates the \isi{nonassimilatory velar fronting} of \ili{WGmc} \textsuperscript{+}[ɣ]. Recall from \REF{ex:14:8} that \ili{WGmc} \textsuperscript{+}[ɣ] underwent fronting to [ʝ] in word-initial position before a front vowel or coronal sonorant consonant. In postsonorant position, \ili{WGmc} \textsuperscript{+}[ɣ] shifted to [ʝ] after a front vowel or a sequence of front vowel plus \isi{schwa} (=\ref{ex:14:37a}), coronal sonorant consonant (=\ref{ex:14:37b}), or back vowel (=\ref{ex:14:37c}). The examples in (\ref{ex:14:37d}--\ref{ex:14:37f}) illustrate the assimilatory fronting of \ili{WGmc} \textsuperscript{+}[x] in postsonorant position. The phonemic front vowels are /i ɪ y e ɛ ø æ/, and the phonemic back vowels are /u o ɔ ɑ/. Due to dialect-specific processes of Diphthongization there do not appear to be examples of back monophthongs in the context before [ʝ]. The data in \REF{ex:14:37} show that Stage 2e'{}'{}' (=\ref{ex:14:32a}iv) holds for [ɣ] (<\ili{MHG} \textsuperscript{+}[ɣ]) and Stage 2c' (=\ref{ex:14:32b}ii) for [x] (<\ili{MHG} \textsuperscript{+}[x]).

\ea%37
\label{ex:14:37}Dorsal fricatives in \ipi{Kirchspiel Courl}:
\ea\label{ex:14:37a} ĭʓl \tab [ɪʝl̩] \tab  Igel \tab ‘hedgehog’ \tab 39\\
    nīəʓn \tab  [niːəʝn̩] \tab neun \tab  ‘nine’ \tab  39\\
\ex\label{ex:14:37b} mĭɛrʓl \tab [mɪɛrʝl̩] \tab Mergel \tab ‘marl’ \tab 34
\ex\label{ex:14:37c} būɒʓn \tab [buːɒʝn̩] \tab  Bogen \tab ‘bow’ \tab 65\\
    d\={y}əʓn \tab [dyːəʝn̩] \tab taugen \tab ‘be good for sth\textsc{{}-inf}’ \tab 70
\ex\label{ex:14:37d} bictə \tab [biçtə] \tab Beichte \tab ‘confession’ \tab 56\\
    rɛct \tab [rɛçt] \tab  Recht \tab ‘justice’ \tab 2
\ex\label{ex:14:37e} bŭɒrx \tab [buɐrx] \tab geschnittenes Schwein\\
\tab \tab \tab ‘sliced-\textsc{infl} pig’ \tab 62
\ex\label{ex:14:37f} fuxt \tab [fuxt]  \tab  (no gloss) \tab  \tab 61\\
    doxt \tab [doxt] \tab Docht \tab ‘wick’ \tab 23\\
    dɑx \tab [dɑx] \tab  dachte \tab ‘think\textsc{{}-pret}’ \tab 23
    \z 
\z

\ipi{Aachen} (\citealt{Welter1938}: 13; \mapref{map:8}) appears to be a dialect in transition from Stage 2c/2d to Stage 2e{}'{}'{}' for \ili{WGmc} \textsuperscript{+}[ɣ]. Recall from \tabref{tab:14:2} that \ili{WGmc} \textsuperscript{+}[ɣ] undergoes the nonassimilatory change to palatal in word-initial position. The dialect possesses phonemic front vowels (/i y e ɛ ø/) and phonemic full back vowels (/u o ɔ ɑ/), which Welter transcribes with tone as well as more than one degree of length.

In postsonorant position the reflex of \ili{WGmc} \textsuperscript{+}[ɣ] is palatal [ʝ] after a front vowel (or a front vowel plus \isi{schwa}) and before a vowel (=\ref{ex:14:38a}) or after an original coronal consonant and before a vowel (=\ref{ex:14:38b}). (The pre-rhotic \isi{schwa} in \ref{ex:14:38b} is epenthetic; recall \sectref{sec:5.4}). As in \ipi{Schlebusch} (\sectref{sec:10.3.1}), \ili{WGmc} \textsuperscript{+}[k x] assimilate to a \isi{sibilant} fricative (alveolopalatal [ɕ]) after a front vowel (=\ref{ex:14:38e}) and otherwise surface as [x] (=\ref{ex:14:38f}). Welter writes that \ili{WGmc} \textsuperscript{+}[ɣ] is normally (“normalerweiseˮ) realized as [ɣ] after a back vowel and before a vowel but that after [ɑː] it is occasionally (“gelegentlichˮ) replaced with the palatal in (\ref{ex:14:38c}). The example in \REF{ex:14:38d} reveals that \ili{WGmc} \textsuperscript{+}[ɣ] after other back vowels and before a vowel is optionally replaced with the glide [u̯]. Significantly, there do not appear to be words listed in the original source containing back vowels other than [ɑː] or [oə] after which [ɣ] could potentially occur. This suggests that [ɣ] only occurs in the context after [ɑː] or [oə] and before another vowel and that [ɣ] is replaced with [ʝ] after any back vowel and before another vowel because it deletes after the only other back vowel. Thus, Stage 2c/2d is replaced with Stage 2e{}'{}'{}' for \ili{WGmc} \textsuperscript{+}[ɣ]. By contrast, \ili{WGmc} \textsuperscript{+}[k x] exhibit Stage 2c/2d (=\ref{ex:14:32b}iii).

\TabPositions{.22\textwidth, .49\textwidth, .6\textwidth, .82\textwidth}
\ea%38
\label{ex:14:38}Dorsal fricatives in \ipi{Aachen}:
\ea\label{ex:14:38a} lý·j.ə \tab [lyʝə]  \tab  lügen \tab ‘lie\textsc{{}-inf}’ \tab 13\\
    vẹ́·ə.jə \tab [veəʝə]  \tab  fegen \tab ‘sweep\textsc{{}-inf}’ \tab 13
\ex\label{ex:14:38b} \'{ę}·r.əjər \tab [ɛrəʝər]  \tab  Ärger \tab ‘anger’ \tab 13
\ex\label{ex:14:38c} dr\={ɑ}:γə, dr\={ɑ}:jə \tab  [drɑːɣə], [drɑːʝə]  \tab   tragen \tab ‘carry\textsc{{}-inf}’ \tab 13\\
    z\={ɑ}:γə, z\={ɑ}:jə \tab  [zɑːɣə], [zɑːʝə]  \tab   sagen \tab ‘say\textsc{{}-inf}’ \tab 13
\ex\label{ex:14:38d} vr\d{ó}·ə.γə, vrǭ:u̯ə  \tab  [vroəɣə], [vrɔːu̯ə]  \tab  fragen \tab ‘ask\textsc{{}-inf}’ \tab 13
\ex\label{ex:14:38e} rî:š \tab  [riːɕ]  \tab  reich \tab ‘rich’ \tab 13\\
    vø.š(t) \tab [vøɕt]  \tab  feucht \tab ‘damp’ \tab 15
\ex\label{ex:14:38f} štrû:x \tab [ʃtruːx]  \tab  Strauch \tab ‘shrub’ \tab 13\\
    lɑ·xt.ə \tab [ʃlɑxtə]  \tab  schlachten\\
    \tab \tab \tab ‘slaughter\textsc{{}-inf}’ \tab 15
    \z 
\z

\citet{Braun1906} discusses a number of places (\il{East Franconian}EFr) in the general vicinity of \ipi{Heilbronn} (\mapref{map:4}). The author observes that the distinction between [x] and [ç] is not nearly as well-defined as in the standard language (“nicht stark ausgeprägtˮ) and consequently transcribes the fortis dorsal fricatives in his material with [ç]. Some representative examples illustrating the occurrence of palatal [ç] can be observed in \REF{ex:14:39}. What these examples suggest is that [ç] can have any historical source, i.e. \ili{WGmc} \textsuperscript{+}[k x] in (\ref{ex:14:39a}, \ref{ex:14:39b}) or \ili{WGmc} \textsuperscript{+}[ɣ] in (\ref{ex:14:39c}--\ref{ex:14:39g}).\pagebreak

\ea%39
\label{ex:14:39}Dorsal fricatives in \ipi{Heilbronn}:
\ea\label{ex:14:39a} gsiçd \tab [gsiçt]  \tab  Gesicht \tab ‘face’ \tab 12\\
    rɛçd \tab [rɛçt]  \tab  recht \tab ‘right’ \tab 12
\ex\label{ex:14:39b} buuç \tab [buːç]  \tab  Buch \tab ‘book’ \tab 12\\
    doç \tab [doç]  \tab  doch \tab ‘however’ \tab 12
\ex\label{ex:14:39c} fliiʒə, fliiçə \tab [fliːɣə], [fliːçə] \tab Fliege \tab ‘fly’ \tab 13
\ex\label{ex:14:39d} fooʒəl, flooçəl \tab  [foːɣəl], [foːçəl]  \tab   Vogel \tab ‘bird’ \tab 13\\
    maaʒər, maaçər \tab [mɑːɣər], [mɑːçər]  \tab  mager \tab ‘lean’ \tab 13
\ex\label{ex:14:39e} seçd \tab [seçt]  \tab  sagt \tab ‘say\textsc{{}-3sg}’ \tab 13
\ex\label{ex:14:39f} taaç  \tab  [tɑːç]  \tab  Tag \tab ‘day’ \tab 13
\ex\label{ex:14:39g} bɛrç \tab [bɛrç]  \tab  Berg \tab ‘mountain’ \tab 14
    \z
\z 

I interpret the optionality in (\ref{ex:14:39c}, \ref{ex:14:39d}) as speaker-dependent; hence, some speakers have the pronunciation with [ɣ] and others with [ç]. Significantly, the change from \ili{WGmc} \textsuperscript{+}[ɣ] or \textsuperscript{+}[k x] to [ç] occurred after any type of sound. Note in particular the occurrence of [ç] after back vowels (=\ref{ex:14:39b}, \ref{ex:14:39d}, \ref{ex:14:39f}). The phonemic front vowels in this dialect are /i e ɛ/ and the phonemic full back vowels /u o ɑ/, which surface as short or long as well as nasalized or oral.

One could take the data in \REF{ex:14:39} at face value and conclude that the dialect has fronted \ili{WGmc} \textsuperscript{+}[ɣ k x] to [ç] after any type of sound (=Stage 2e{}'{}'{}'). Alternatively, one might interpret Braun’s comments concerning the distinction between [x] and [ç] not as a complete merger to [ç], but instead as a \isi{near-merger}. If correct, that would mean that the two fricatives are still phonetically distinct, even though Braun decided to transcribe them with the same phonetic symbols. If the latter interpretation is on the right track then \ipi{Heilbronn} represents a transitional dialect on its way to becoming fully-fledged Stage 2e{}'{}'{}'.

\subsection{Areal distribution of nonassimilatory velar fronting after a sonorant}\label{sec:14.4.3}

\tabref{tab:14:5} provides a list of all dialects discussed in this chapter involving some version of \isi{nonassimilatory velar fronting} in \REF{ex:14:24} for the postsonorant context. All of those varieties are indicated in \mapref{map:31}. I also include the places discussed in \sectref{sec:14.5} below. In \tabref{tab:14:5} I do not indicate the velar fronting targets for the stages indicated in the final column.

In \chapref{sec:15} I discuss data from linguistic atlases for various places in \ipi{Vorarlberg} and \ipi{Tyrol} which have palatal [ç] but no velar [x] (=Stage 2e'{}'{}'). I do not list those places in \tabref{tab:14:5}, nor do I include them on \mapref{map:31}. Discussion of those areas is delayed until \chapref{sec:15}, which considers their status as velar fronting islands.

All of the places indicated on \mapref{map:31} are situated in the western part of Germany (and \ipi{Luxembourg}/Belgium). Those varieties occupy various points along a broad vertical column extending from the area just north of Switzerland to a point to the northwest of \ipi{Hannover} in Lower Saxony. Although the twelve varieties are found in the same broad region, there is considerable space separating most of them.

\begin{table}
\caption{Nonassimilatory velar fronting of \ili{WGmc} velars in postsonorant position (=\ref{ex:14:19a}--\ref{ex:14:19c})\label{tab:14:5}}
\begin{tabular}{llll}
\lsptoprule
Place & Dialect & Source & Stage\\\midrule
\ipi{Mühlingen} & \il{Swabian}Swb & \citet{Müller1911} & 2e{}'\\
\ipi{Blaubeuren} & \il{Swabian}Swb & \citet{Strohmaier1930} & 2e{}'\\
\ipi{Liggersdorf} & \il{Swabian}Swb & \citet{Dreher1919} & 2e{}'/2e{}'{}'\\
\ipi{Ramlingen} & \il{Eastphalian}Eph & \citet{Jarfe1929} & 2d/2e{}'{}'\\
\ipi{Wissenbach} & \il{Central Hessian}CHes & \citet{Kroh1915} & 2e'{}'\\
\ipi{Nordösling} & \il{Moselle Franconian}MFr & \citet{Bruch1952} & 2e'{}'{}'\\
\ipi{Burg-Reuland} & \il{Moselle Franconian}MFr & \citet{Hecker1972} & 2e'{}'{}'\\
\ipi{Lützkampen}, \ipi{Dahnen} & \il{Moselle Franconian}MFr & {MRhSA} & 2e'{}'{}'\\
\ipi{Aachen} & \il{Ripuarian}Rpn & \citet{Welter1938} & 2e'{}'{}'\\
\ipi{Loshausen-Zella} & \il{North Hessian}NHes & \citet{Schoof1913a,Schoof1913b,Schoof1913c} & 2e'{}'{}'\\
\ipi{Blankenheim} & \il{North Hessian}NHes & \citet{Dittmar1891} & 2e'{}'{}'\\
\ipi{Kirchspiel Courl} & \il{Westphalian}Wph & \citet{Beisenherz1907} & 2e'{}'{}'\\
\ipi{Heilbronn} & \il{East Franconian}EFr & \citet{Braun1906} & 2e'{}'{}'\\
\lspbottomrule
\end{tabular}
\end{table}

\begin{map}[p]
% \includegraphics[width=\textwidth]{figures/VelarFrontingHall2021-img037.png}
\includegraphics[width=\textwidth]{figures/Map31_14.3.pdf}
\caption[Areal distribution of nonassimilatory velar fronting in postsonorant position]{Areal distribution of {nonassimilatory velar fronting} in postsonorant position. Varieties of High German and Low German in which postsonorant velar fronting is a nonassimilatory change (Stage 2e', 2e'{}', 2e'{}'{}') are indicated with squares.}\label{map:31}
\end{map}\is{nonassimilatory velar fronting}
\clearpage

\section{{Nonassimilatory} {velar} {fronting} {in} {Nordösling}}\label{sec:14.5}\il{Moselle Franconian|(}\il{Luxembourgish|(}

The case studies discussed in \sectref{sec:14.4} have in common that some velar fricatives undergo fronting to palatals, but other velar fricatives remain and therefore surface as such. In the present section I discuss a set of dialects in the northwest corner of the \il{Moselle Franconian}MFr region (\mapref{map:10}) which have in common that they do not possess velar fricatives because those sounds underwent \isi{nonassimilatory velar fronting} (or underwent \isi{g-Formation-1} to [g]). I discuss first the variety of North Lxm spoken in \ipi{Nordösling} \citep{Bruch1952} followed by the \il{Moselle Franconian}MFr variety of \ipi{Burg-Reuland} in the southeastern tip of Belgium \citep{Hecker1972}. I conclude by showing that the same pattern is attested in data from MRhSA for two German villages in the same area.

The data in \REF{ex:14:40} from North Lxm (\ipi{Nordösling}) can be contrasted with the material analyzed earlier in \sectref{sec:10.3.2} from Central, South, and East Lxm discussed by \citet{Gilles1999}. \ipi{Nordösling} is a region in North \ipi{Luxembourg} in the canton of Clerf. The words listed in \REF{ex:14:40} reveal that \ili{WGmc} \textsuperscript{+}[ɣ] shifted to palatal [ʝ] between vowels if the first vowel is front (=\ref{ex:14:40a}) or back (=\ref{ex:14:40b}). No examples were found in \citet{Bruch1952} illustrating the environment after a consonant. \ili{WGmc} \textsuperscript{+}[ɣ] in coda position shifted to palatal ([ç]) after a front vowel (=\ref{ex:14:40c}) or back vowel (=\ref{ex:14:40d}). \ili{WGmc} \textsuperscript{+}[k x] are similarly realized as [ç] (/ç/) after a front vowel (=\ref{ex:14:40e}) or back vowel (=\ref{ex:14:40f}). The change to palatal can even be observed for historical sources other than the ones mentioned above, e.g. an original glide (=\ref{ex:14:40g}).{} \footnote{{The phonemic front vowels in this dialect are /i e ɛ/ and the phonemic back vowels /u o ɔ ɑ/; most of those vowels surface as short or long.} \textrm{Since there is no significant difference between the phonetic symbols in the original source and the ones I employ in the present book, I simply give the former in the first column of \REF{ex:14:40}. Two minor differences are that Bruch’s ⟦j⟧ depicts the palatal fricative [ʝ] and that ⟦aː⟧ is the low back vowel [ɑ].} }  In sum, the words listed below illustrate that \ipi{Nordösling} exhibits Stage 2e'{}'{}' for \ili{WGmc} \textsuperscript{+}[ɣ x k]. No velar fricatives occur word-initially because WGmc \textsuperscript{+}[ɣ] underwent \isi{g-Formation-1} and therefore surfaces as [g].  There were no independent sound changes introducing velar fricatives in word-initial position. That velars changed to palatals even after back vowels is stated clearly in the original source (\citealt{Bruch1952}: 35, 36: “nach velaren wie nach palatalen Vokalenˮ).


\TabPositions{.2\textwidth, .4\textwidth, .6\textwidth, .8\textwidth}
\ea%40
\label{ex:14:40}Palatal fricatives in \ipi{Nordösling}:
\ea\label{ex:14:40a} lɛːjən \tab legen \tab ‘place\textsc{{}-inf}’ \tab 35\\
    fleijən \tab fliegen \tab ‘fly\textsc{{}-inf}’ \tab 35
\ex\label{ex:14:40b} tujənt \tab Tugend \tab ‘virtue’ \tab 35\\
    fujəl \tab Vogel \tab ‘bird’ \tab 24\\
    mɔːjər \tab mager \tab ‘lean’ \tab 35
\ex\label{ex:14:40c} fleiç \tab Fliege \tab ‘fly’ \tab 35\\
    lɛ·ç \tab Lage \tab ‘situation’ \tab 35\\
    vɛːç \tab Weg \tab ‘path’ \tab 35
\ex\label{ex:14:40d} tsuç \tab Zug \tab ‘train’ \tab 35\\
    vɔːç \tab Waage \tab ‘scale’ \tab 35\\
    taːç \tab Tag \tab ‘day’ \tab 35\\
    fouç \tab Fuge \tab ‘seam’ \tab 35
\ex\label{ex:14:40e} zeçər \tab sicher \tab ‘certainly’ \tab 23\\
    ʃpɛːçt \tab Specht \tab ‘woodpecker’ \tab 22
\ex\label{ex:14:40f} koːçən \tab kochen \tab ‘cook\textsc{{}-inf}’ \tab 12\\
    lɔːç \tab Loch \tab ‘hole’ \tab 23\\
    ʃwaːç \tab schwach \tab ‘weak’ \tab 21\\
    aːçt \tab acht \tab ‘eight’ \tab 21\\
    baːçən \tab backen \tab ‘bake\textsc{{}-inf}’ \tab 36\\
    kouç \tab Kuchen \tab ‘cake’ \tab 36\\
    hɑːuçən \tab hauchen \tab ‘aspirate\textsc{{}-inf}’ \tab 28
\ex\label{ex:14:40g} blɔːç \tab blau \tab ‘blue’ \tab 32\\
    grɔːç \tab grau \tab ‘gray’ \tab 32
\z 
\z 

\begin{sloppypar}
The pattern depicted in dataset \REF{ex:14:40} is confirmed by independent sources. First, according to LSA, palatal fricatives are attested after front and back vowels throughout North \ipi{Luxembourg}. Some examples of words containing  [ç] after a back vowel from that source are [nɑːçt] ‘night’ (Map 25), [voçən] ‘week-\textsc{pl}’ (Map 61), [kɑçən] ‘cook\textsc{{}-inf}’ (Map 64), and [luːçt] ‘air’ (Map 82). Second, \citet{Gilles1999} collected data throughout \ipi{Luxembourg}, including the area in and around \ipi{Nordösling}. He concludes that the palatal fricative [ç] now surfaces for his informants as alveolopalatal [ɕ] after front and back vowels alike. Examples of words in his survey from \ipi{Nordösling} include [nɑːɕt] ‘night’ and [brɑːɕt] ‘bring-\textsc{part}’.\footnote{Some of the maps in LSA suggest that \isi{nonassimilatory velar fronting} is attested outside of \ipi{Nordösling}. For example, Map 25 for \textit{Nacht} ‘night’ indicates that the realization as [nuǝçt] occurs throughout Central Lxm.}
\end{sloppypar}

As in \ipi{Nordösling}, etymological velar fricatives (\ili{WGmc} \textsuperscript{+}[ɣ x]) in the region in and around \ipi{Burg-Reuland} in \ipi{East Belgium} in the province of Liège (Lüttich) have been consistently replaced with their fronted counterparts.\largerpage

The data in \REF{ex:14:41} from \citet{Hecker1972} are representative for the area around \ipi{Burg-Reuland}; see also \citet[197]{CajotBeckers1979}. As shown in \REF{ex:14:41a}, historical \textsuperscript{+}[ɣ] underwent \isi{nonassimilatory velar fronting} to [ʝ] (=⟦j⟧) in word-initial position (Stage 2e). In the context between sonorants, original \textsuperscript{+}[ɣ] likewise shifted to [ʝ] after a front vowel in (\ref{ex:14:41b}), but -- more significantly -- the same change took place after any back vowel in (\ref{ex:14:41c}). The same generalization holds for historical \textsuperscript{+}[x], which now surfaces as the corresponding alveolopalatal fricative [ɕ] (=⟦š⟧). The change from \textsuperscript{+}[x] to [ɕ] occurred in the context after a front vowel in (\ref{ex:14:41d}), but also after any back vowel in (\ref{ex:14:41e}). The data in \REF{ex:14:41f} show that an original \textsuperscript{+}[ɣ] in the context after a back vowel and before another vowel is alveolopalatal [ɕ] and not [ʝ], and the items in \REF{ex:14:41g} indicate that an original fortis velar stop is now [ɕ] even though a back vowel precedes that sound. The words in \REF{ex:14:41h} contain a historical \textsuperscript{+}[ʃ], which merged together with historical [ɣ x] to alveolopalatal [ɕ].\footnote{The data in \citet{Hecker1972} reveal that many instances of historical velar fricatives deleted, but I do not consider those examples here. The diacritic ⟦ˊ⟧ in some of the items listed in the first column of \REF{ex:14:41} represents a distinct tonal contour which I ignore in my transcriptions in the second column. The phonemic front vowels of \ipi{Burg-Reuland} are /i e ɛ æ/ and the phonemic back vowels /u o ɔ ɑ/. Those eight vowels surface as short or long. There are a few gaps (e.g. no example was found with [ʝ] between a nonhigh front vowel and another vowel), but they are not deemed significant.}


\TabPositions{.15\textwidth, .3\textwidth, .5\textwidth, .82\textwidth}
\ea%41
\label{ex:14:41}Alveolopalatal/palatal fricatives in \ipi{Burg-Reuland}:
\ea\label{ex:14:41a} juːˊt \tab [ʝuːt] \tab gut \tab ‘good’ \tab 65\\
    jeːˊl \tab [ʝeːl] \tab gelb \tab ‘yellow’ \tab 97\\
    jraːs \tab [ʝrɑːs] \tab Gras \tab ‘grass’ \tab 65
\ex\label{ex:14:41b} fliːˊje \tab [fliːʝə] \tab fliegen \tab ‘fly\textsc{{}-inf}’ \tab 65
\ex\label{ex:14:41c} plǫːje \tab [plɔːʝə] \tab plagen \tab ‘afflict\textsc{{}-inf}’ \tab 65\\
    kla:ːje \tab [klɑːʝə] \tab klagen \tab ‘complain-\textsc{inf}’ \tab 106
\ex\label{ex:14:41d} bräːše \tab [bræːɕə] \tab brechen \tab ‘break\textsc{{}-inf}’ \tab 62\\
    reš \tab [reɕ] \tab reich \tab ‘rich’ \tab 62
\ex\label{ex:14:41e} fluːˊše \tab [fluɕə] \tab fluchen \tab ‘curse\textsc{{}-inf}’ \tab 60\\
    kǫːše \tab [kɔɕə] \tab kochen \tab ‘cook\textsc{{}-inf}’ \tab 62\\
    štrǫše \tab [ɕtrɔɕə] \tab streichen \tab ‘paint\textsc{{}-inf}’ \tab 118\\
    oš \tab [oɕ] \tab auch \tab ‘also’ \tab 62\\
    baːš \tab [bɑːɕ] \tab Bach \tab ‘stream’ \tab 62
\ex\label{ex:14:41f} doːše \tab [doːɕə] \tab taugen \tab ‘be good for sth-\textsc{inf}’ \tab 132
\ex\label{ex:14:41g} dreše \tab [dreɕə] \tab trocken \tab ‘dry’ \tab 117\\
    baːše \tab [bɑːɕə] \tab backen \tab ‘bake-\textsc{inf}’ \tab 104
\ex\label{ex:14:41h} šlaŋ \tab [ɕlɑŋ] \tab Schlange \tab ‘snake’ \tab 39\\
    fläš \tab [flæɕ] \tab Flasche \tab ‘bottle’ \tab 62\\
    biːšt \tab [biːɕt] \tab Bürste \tab ‘brush’ \tab 82\\
    touše \tab [touɕə] \tab tauschen \tab ‘exchange-\textsc{inf}’ \tab 115
\z 
\z\todo{check accent}

Unlike the alveolopalatalizing dialects discussed in \chapref{sec:10} there are no alternations between [x] and [ɕ] in \ipi{Burg-Reuland} which would motivate a synchronic rule neutralizing the contrast between /x/ and /ɕ/. The reason for this gap is that \ipi{Burg-Reuland} has no /x/.

The pattern for the \il{Moselle Franconian}MFr variety of \ipi{Burg-Reuland} depicted in \REF{ex:14:41} stands in contrast with the system of velars and (alveolo)palatals in neighboring \il{Ripuarian}Rpn varieties of \ipi{East Belgium} discussed in \citet{Hecker1972}, e.g. Elsenborn, Wallerode, Recht, St. Vith, Manderfeld (=\citealt{Hecker1972} on \mapref{map:8}). Consider the data in \REF{ex:14:42} from Elsenborn, ca. 20 km north of \ipi{Burg-Reuland}:

\ea%42
\label{ex:14:42}Alveolopalatal/palatal and velar fricatives in Elsenborn:

\ea\label{ex:14:42a} bräːše \tab [bræːɕə] \tab brechen \tab ‘break\textsc{{}-inf}’ \tab 62\\
    riš \tab [riɕ] \tab reich \tab ‘rich’ \tab 62
\ex\label{ex:14:42b} fluxe \tab [fluxə] \tab fluchen \tab ‘curse\textsc{{}-inf}’ \tab 60\\
    kǫːxe \tab [kɔxə] \tab kochen \tab ‘cook\textsc{{}-inf}’ \tab 62\\
    baːx \tab [bɑːx] \tab Bach \tab ‘stream’ \tab 62
\ex\label{ex:14:42c} maxen \tab [mɑxə] \tab machen \tab ‘do-\textsc{inf}’ \tab 61\\
    mešt \tab [meɕt] \tab macht \tab ‘do-\textsc{3sg}’ \tab 61
\z 
\z 

These data reveal that Elsenborn retains historical velars after back vowels. \citet{Hecker1972} points out that the areas of \ipi{East Belgium} north of \ipi{Burg-Reuland} now have alternations involving [x] and [ɕ] as in \REF{ex:14:42c} which motivate a synchronic process of velar fronting (as in \ipi{Schlebusch}; \sectref{sec:10.3.1}).

An examination of the maps in the fourth volume of MRhSA indicates that the two German villages \ipi{Lützkampen} and \ipi{Dahnen} in Rhineland-Palatinate exhibit a pattern that is essentially the same as the one in \REF{ex:14:41} for \ipi{Burg-Reuland}. Both of those villages fall into the broad alveolopalatalizing region; hence, historical [ç] is now [ɕ]. That change occurred after coronal sonorants -- evident in the maps for \textit{ich} ‘I’ and \textit{Kirche} ‘church’ -- but most significantly after historically back vowels. Etymological \textsuperscript{+}[ɣ] likewise underwent velar fronting to palatal [ʝ] after a sonorant and before a vowel or to alveolopalatal [ɕ] in the coda even after back vowels. The data in \REF{ex:14:43} have been drawn from MRhSA. They have in common that the change from \textsuperscript{+}[ɣ] to [ʝ]/[ɕ] occurred after back vowels.\footnote{{I have simplified the phonetic representations in \REF{ex:14:43} from the original source by ignoring diacritics capturing low-level phonetic detail (half-length in vowels, slight aspiration in fortis stops) and tone contours. The sound [ʆ] is the alveolopalatal \isi{sibilant} I transcribe as [ɕ].}} As in \ipi{Nordösling}, the modern reflex of \ili{WGmc} \textsuperscript{+}[ɣ] in \ipi{Lützkampen} and \ipi{Dahnen} is [g].

\ea%43
\label{ex:14:43}Alveolopalatal/palatal fricatives in \ipi{Lützkampen} (in a) and \ipi{Dahnen} (in b):
\ea\label{ex:14:43a}\relax [βɔʆ] \tab Waage \tab ‘scale’ \tab Map 384\\
    \relax [kʊʝəl] \tab Kugel \tab ‘ball’ \tab Map 387
\ex\label{ex:14:43b}\relax [b̥lʊʆ] \tab Pflug \tab ‘plow’ \tab Map 392\\
    \relax [nɑːʆd̥] \tab Nacht \tab ‘night’ \tab Map 338\\
    \relax [lʊːʆd̥] \tab Luft \tab ‘air’ \tab Map 399
\z 
\z 

The risk of drawing conclusions solely on the basis of a sparse set of examples drawn from a linguistic atlas is that the maps might not reflect deeper generalizations concerning the dialect as a whole. Nevertheless, on the basis of the data in MRhSA and the close proximity of \ipi{Lützkampen} and \ipi{Dahnen} to \ipi{Nordösling} and \ipi{Burg-Reuland}, I assume -- unless evidence can be adduced to the contrary -- that the two German villages in question have no velar fricatives.

\mapref{map:32} contrasts assimilatory vs. \isi{nonassimilatory velar fronting} in the places discussed in the present section. The markers in Belgium correspond to the towns and villages indicated on the map in \citet[146]{Hecker1972} showing the realization of /x/ as /x/ or /ɕ/ (=⟦š⟧) in \textit{machen} ‘do-\textsc{inf}’, and the ones in \ipi{Luxembourg} are the locations of the informants for LSA (Belegorte) which are in the \ipi{Nordösling} region described by \citet{Bruch1952}. The markers in Germany are the ones indicated in MRhSA (Belegorte).

\ip{Nordösling}
\ip{Luxembourg}
\ip{East Belgium}
\begin{map}
% \includegraphics[width=\textwidth]{figures/VelarFrontingHall2021-img038.png}
\includegraphics[width=\textwidth]{figures/Map32_14.4.pdf}
\caption[{Luxembourg} ({Nordösling}),  {East Belgium}, and West Central Germany (Rhineland-Palatinate)]{{Luxembourg} ({Nordösling}), {East Belgium}, and West Central Germany (Rhineland-Palatinate). Shaded squares indicate \isi{nonassimilatory velar fronting} and white squares assimilatory velar fronting (alveolopalatalization). 1=\citet{Hecker1972}, 2=LSA, 3=MRhSA.}
\label{map:32}
\end{map}

The detailed descriptions cited above for {Nordösling} and \ipi{Burg-Reuland} are the only ones uncovered up to this point in the present survey possessing (alveolo)palatal fricatives but no corresponding velars. (For examples attested outside of this area see Chapter~\ref{sec:15}). The data from MRhSA for \ipi{Lützkampen}/\ipi{Dahnen} suggests that the same generalization is also true for those places. What is more, since closely related varieties of German spoken in the same area display the unmarked assimilatory pattern of (alveolo)palatals after front vowels and velars after back vowels (=Stage 2c/2d), the inescapable conclusion is that \ipi{Nordösling}, \ipi{Burg-Reuland}, and \ipi{Lützkampen}/\ipi{Dahnen} -- indicated on \mapref{map:32} with lightly shaded squares -- exhibited the loss of velar fronting (\sectref{sec:14.6.3}).\il{Moselle Franconian|)}\il{Luxembourgish|)}

\section{Discussion}\label{sec:14.6}

I turn to several unresolved issues. First, I consider and reject a possible alternative treatment to the one presupposed in the present chapter (\sectref{sec:14.6.1}). Second, I discuss the status of the nonassimilatory changes in \REF{ex:14:1c} and \REF{ex:14:19} as synchronic rules (\sectref{sec:14.6.2}). Third, I discuss the topic of \isi{rule loss} in light of the \ipi{Nordösling}/\ipi{Burg-Reuland}/\ipi{Lützkampen}/\ipi{Dahnen} data presented above (\sectref{sec:14.6.3}). Finally, I provide further remarks on accounting for unattested Trigger Types (\sectref{sec:14.6.4}).

\subsection{Alternative approach}\label{sec:14.6.1}

This chapter asserts that the fronting of velars in the neighborhood of one or more back vowel by \REF{ex:14:1c} and \REF{ex:14:24} is nonassimilatory because back vowels do not bear the frontness feature ([coronal]). One could alternatively argue that the basic premise is incorrect and that the back vowels inducing fronting are marked phonologically for the frontness feature, in which case velars would be expected to surface as palatals in the neighborhood of those vowels. It is demonstrated below that an alternative analysis along these lines is flawed.

I apply the alternative analysis to \ipi{Schlebusch} (=\ref{ex:14:10}), which is exemplifies the change from an original velar to palatal in word-initial position before any kind of sound (=\ref{ex:14:1c}). I focus on \ipi{Schlebusch}, although the same argument can be extended to any of the other dialects listed in \tabref{tab:14:2}.

The alternative analysis for \ipi{Schlebusch} is depicted in \REF{ex:14:44} and \REF{ex:14:45}. To the right of the wedge in \REF{ex:14:44} I give the phonetic representation for three words from \REF{ex:14:10}. To the left of the wedge I give the reconstructed example with [ɣ] or [ʝ] at the point before \ili{WGmc} \textsuperscript{+}[ɣ] shifted to palatal [ʝ] before a back vowel (=Stage 2d). Front vowels like /ɛ/ and liquids like /l/ are simplex [coronal] sounds, as indicated below. Since a velar fronts to palatal before any back vowel, all back vowels must be analyzed as phonologically coronal. This treatment might be plausible if back vowels are phonetically (and phonologically) central, which might translate into a treatment whereby vowels like /ɑ/ are complex (=[coronal, dorsal]), as in \REF{ex:14:44c}. Given the features in \REF{ex:14:44}, rule \REF{ex:14:45} is triggered by all coronal sonorants; hence, that rule is assimilatory and \REF{ex:14:1c} never occurred at all.


\TabPositions{.1\textwidth, .15\textwidth, .2\textwidth, .3\textwidth, .4\textwidth, .7\textwidth}
\ea%44
\label{ex:14:44}
\ea\label{ex:14:44a}\textsuperscript{+}[ʝɛl]  \tab > \tab [ʝɛl] \tab /ɛ/ \tab = [coronal]         \tab ‘yellow’
\ex\label{ex:14:44b}\textsuperscript{+}[ʝlɑt] \tab > \tab [jlɑt]\tab /l/ \tab = [coronal]         \tab ‘smooth’
\ex\label{ex:14:44c}\textsuperscript{+}[ɣɑs]  \tab > \tab [ʝɑs] \tab /ɑ/ \tab = [coronal, dorsal] \tab ‘guest’
\z 
\ex%45
\label{ex:14:45}
      velar > palatal / \textsubscript{wd} [ {\longrule}{\longrule} [coronal, +sonorant]
\z 


The problem with analyzing all back vowels (/u uː o oː ɔ ɔː ɑ ɑː/) phonologically as coronal is that those same back vowels do not trigger the fronting of a following velar. For example, as noted earlier in \sectref{sec:10.3.1}, \ipi{Schlebusch} /x ɣ/ both undergo assimilatory fronting after front vowels (\isi{Velar Fronting-1}), thereby accounting for alternating forms like [ruxǝ]{\textasciitilde}[ryɕ] ‘smell-\textsc{inf}{\textasciitilde}smell-\textsc{3sg}’. \isi{Velar Fronting-1} affects /x/ after any coronal sonorant; hence, the /x/ in the word [ruxǝ] (/ruxǝ/) fails to undergo it because the vowel /u/ is back and hence not [coronal]. If the vowels /u uː o oː ɔ ɔː ɑ ɑː/ are phonologically [coronal] to account for the fronting of a velar preceding that sound, then one would incorrectly expect a velar after that sound to be fronted as well.

The upshot is that the alternative analysis described above cannot work for the dialects in \tabref{tab:14:2}. I do not discuss any of the dialects illustrating the changes depicted in \REF{ex:14:24} for postsonorant position, but potential problems arise in those varieties where \textsuperscript{+}[ɣ] and \textsuperscript{+}[x]/\textsuperscript{ +}[k] do not behave in a consistent manner in the context after back vowels.

\subsection{Status of nonassimilatory velar fronting in the synchronic phonology}\label{sec:14.6.2}\largerpage

Stress in this chapter has been placed on the historical nonassimilatory process of fronting, both word-initially and after a sonorant. One question not discussed earlier is the status of the nonassimilatory fronting of velars in the synchronic grammar. Two positions suggest themselves here, which I refer to below as Analysis A and Analysis B. For Analysis A, the nonassimilatory fronting of velars restructured underlying representations and hence that change is no longer present in the synchronic grammar. By contrast, for Analysis B the nonassimilatory change in question did not alter underlying representations but instead remains active in the grammar as a synchronic process.

As I point out below, Analysis B in its strongest form cannot be correct for the change from an assimilatory process fronting velars to one of the nonassimilatory changes in \REF{ex:14:1c} or \REF{ex:14:24}. However, there is good reason for believing that a variant of Analysis B holds for the initial (assimilatory) stages of velar fronting. To illustrate that point I consider as a representative example data in \REF{ex:14:46} for velar fronting in word-initial position in the \il{Westphalian}Wph variety of \ipi{Elspe} described by \citet{Arens1908}; recall \sectref{sec:7.2}. The example discussed here concerns a change in the set of triggers, but the same point holds for a change in the set of target segments.



\TabPositions{.2\textwidth, .3\textwidth, .4\textwidth, .7\textwidth}
\ea\label{ex:14:46}
\ea\label{ex:14:46a}\relax  [çɛlt]  /xɛlt/  \tab < \tab \textsuperscript{+}[çɛlt]  /xɛlt/  \tab ‘money’
\ex\label{ex:14:46b}\relax  [çrɛɔt] /xrɛɔt/ \tab < \tab \textsuperscript{+}[xrɛɔt] /xrɛɔt/ \tab ‘large’
\ex\label{ex:14:46c}\relax  [xɔlt] /xɔlt/   \tab < \tab \textsuperscript{+}[xɔlt] /xɔlt/   \tab ‘gold’
\z 
\z 

The phonetic representations in the first column of \REF{ex:14:46} are the ones representing \ipi{Elspe} in 1908. Since there was no contrast between [ç] in [x] in word-initial position at that stage, those two sounds derive from /x/. The set of triggers for velar fronting (word-initial) in 1908 subsumed high, mid and low front vowels as well as coronal sonorant consonants (=Stage 2d). The immediately preceding stage could be either Stage 2c (high front vowels, mid front vowels or coronal consonants) or Stage 2c' (high, mid or low front vowels). I assume the latter, although the choice between the two is not crucial. The phonetic representations for the three examples are reconstructed for Stage 2c'. The crucial example is the one in \REF{ex:14:46b}: At Stage 2c' the set of triggers for word-initial velar fronting consisted of all front vowels, but not the consonants. The important point is that the change from Stage 2c' to Stage 2d did not involve a change in underlying representations. In \REF{ex:14:46a}, \REF{ex:14:46b} -- and crucially \REF{ex:14:46c} -- /x/ is simply inherited without change. Rule generalization therefore describes the relationship between a synchronic process at one stage with the same synchronic process at the next stage.

A treatment like the one in \REF{ex:14:46} for \isi{nonassimilatory velar fronting} fails in those places where velars and palatals contrast, e.g. in the \il{Central Hessian}CHes variety spoken in \ipi{Wissenbach} \citep{Kroh1915} from \REF{ex:14:31}. In that dataset it was demonstrated that \ili{WGmc} \textsuperscript{+}[ɣ] shifted to [ʝ] after nonlow back vowels deriving historically from nonlow back vowels (=\ref{ex:14:24b}), e.g. [bɔːʝə] ‘bow’ (< pre-\il{Central Hessian}CHes \textsuperscript{+}[bɔːɣə]; cf. \ili{MHG} \textit{boge}). According to Analysis B, \REF{ex:14:24b} did not alter underlying representations; hence, a pre-\il{Central Hessian}CHes underlying representation like /bɔːɣə/ was also present in the dialect as it was described in 1915 as /bɔːɣə/, in which case \REF{ex:14:24b} applied as a synchronic process. Analysis B fails for the \ipi{Wissenbach} variety because \REF{ex:14:24b} incorrectly applies to the /ɣ/ after [ɔː] (< [ɑ]), e.g. [mɔːɣə] (/mɔːɣə/) ‘stomach’ (=\ref{ex:14:31d}).

In some of those varieties where velars and palatals never contrast, Analysis B can be shown to be highly questionable at best. Consider as a representative example the Stage 2e'{}'{}' variety of \ipi{Nordösling} from (\ref{ex:14:40}). In that dataset it was shown that \ili{WGmc} velars (\textsuperscript{+}[x ɣ k]) are realized as palatal in postsonorant position (=\ref{ex:14:24c}). The phonetic representation in the first column of \REF{ex:14:47a} is the pronunciation described in the original source \citep{Bruch1952}. Given the \isi{rule generalization} approach described above, the nonassimilatory stages (Stage 2e'{}'{}' < Stage 2e'{}' < Stage 2e') were preceded by a stage in which velars front in an assimilatory manner after coronal sonorants (=Stage 2d). The examples in the first column of \REF{ex:14:47a} are reconstructed to the right of the wedge as a Stage 2d dialect. Since the set of triggers subsume front sounds, the synchronic rule referred to here is an assimilation spreading the frontness feature [coronal] (\isi{Velar Fronting-1}). According to Analysis A, \isi{Velar Fronting-1} was active until \REF{ex:14:24c} altered /x ɣ/ in postsonorant position to /ç ʝ/. Significantly, Alternative A implies that the restructuring of underlying representations to palatals led to the loss of the earlier rule of \isi{Velar Fronting-1}. Consider now Alternative B, which is sketched in \REF{ex:14:47b}. According to that treatment, \REF{ex:14:19c} did not alter underlying representations; hence, /x ɣ/ are inherited into the dialect as it was described in 1952 with /x ɣ/. Alternative B does not necessitate \isi{rule loss}; instead, it involves the reanalysis of a rule of assimilation triggered by coronal sonorants (\isi{Velar Fronting-1}) to a nonassimilatory change which applies synchronically after any type of segment (=\ref{ex:14:24c}).

\ea%47
\TabPositions{.25\textwidth, .3\textwidth, .6\textwidth}
\label{ex:14:47}
\ea Analysis A for \ipi{Nordösling}:\label{ex:14:47a}\\
\relax [aːçt]  /aːçt/        \tab < \tab \textsuperscript{+}[aːxt]     /aːxt/ \tab ‘eight’\\
\relax [ʃpɛːçt]   /ʃpɛːçt/   \tab < \tab \textsuperscript{+}[ʃpɛːçt]   /ʃpɛːxt/ \tab  ‘woodpecker’\\
\relax [fuʝəl]    /fuʝəl/    \tab < \tab  \textsuperscript{+}[fuɣəl]  /fuɣəl/ \tab ‘bird’\\
\relax [lɛːʝən]     /lɛːʝən/ \tab < \tab  \textsuperscript{+}[lɛːʝən]    /lɛːɣən/ \tab ‘place-\textsc{inf}’\\
\ex Analysis B for \ipi{Nordösling}:\label{ex:14:47b}\\
\relax [aːçt]   /aːxt/    \tab < \tab \textsuperscript{+}[aːxt]  /aːxt/ \tab ‘eight’\\
\relax [ʃpɛːçt] /ʃpɛːxt/  \tab < \tab \textsuperscript{+}[ʃpɛːçt]   /ʃpɛːxt/ \tab  ‘woodpecker’\\
\relax [fujəl]  /fuɣəl/   \tab < \tab  \textsuperscript{+}[fuɣəl]  /fuɣəl/ \tab ‘bird’\\
\relax [lɛːʝən] /lɛːɣən/  \tab < \tab  \textsuperscript{+}[lɛːʝən]    /lɛːɣən/ \tab ‘place-\textsc{inf}’
\z 
\z 

It is important to stress that \ipi{Nordösling} possesses neither [x] nor [ɣ]. If Analysis B were adopted then the question is why speakers would continue to analyze the palatal in words like [aːçt] ‘eight’ as a sound they do not have, i.e. /x/. Put differently: How are language learners not knowledgeable about the history of the \ipi{Nordösling} variety able to deduce that an underlying representation for words like [aːçt] is /aːxt/ (as per Analysis B) and not /aːçt/?

The weakness described above for Analysis B holds for those dialects which no longer possess the surface velars in question, but it is not clear what the status is of the nonassimilatory changes in \REF{ex:14:1c} and \REF{ex:14:24} in those varieties which do possess the corresponding velars. Consider as a representative example the \il{Westphalian}Wph variety Kreis \ipi{Lippe} (=\ref{ex:14:11}). One could make a case that word-initial [ç] in the items like [çɑus] ‘goose’ is underlyingly /x/ -- as per Analysis B -- and that \REF{ex:14:1c} is active synchronically, thereby deriving [ç]. That type of treatment is not subject to the criticisms described in the preceding paragraph for \ipi{Nordösling} because Kreis \ipi{Lippe} does possess the velar in question (i.e. [x] /x/) in postsonorant position, e.g. [dɑxt] ‘wick’ (\sectref{sec:7.2}).

The assumption made in the present book is that the nonassimilatory changes in \REF{ex:14:1c} and \REF{ex:14:24} altered underlying representations, as in Analysis A. I contend that this is the correct treatment even for dialects like Kreis \ipi{Lippe}, although I concede that an Analysis B-type treatment for that type of dialect may work technically.

\subsection{Rule loss}\label{sec:14.6.3}

As indicated in \REF{ex:14:47a}, the present-day \ipi{Nordösling} system in \REF{ex:14:40} possesses two phonemic palatals (/ç ʝ/) and since the corresponding velars -- underlying /x ɣ/ and surface [x ɣ] -- are absent, the dialect cannot have a synchronic process of velar fronting. However, related varieties of Lxm \citep{Gilles1999} possess an assimilatory version of that rule (\isi{Velar Fronting-1}; \sectref{sec:10.3.2}). The implication is that \ipi{Nordösling} once had the same system as Lxm, as illustrated to the right of the wedge in \REF{ex:14:47a}. The nonassimilatory change in \REF{ex:14:24c} affected underlying representations in \ipi{Nordösling}, in which case the originally synchronic rule of fronting which continues to be active in Lxm (\isi{Velar Fronting-1}) was lost. I consider briefly the way in which the \ipi{Nordösling} variety bears the issue of \isi{rule loss} in historical phonology. Recall that the facts of \ipi{Nordösling} are mirrored in the closely related \il{Moselle Franconian}MFr variety of \ipi{Burg-Reuland} spoken in the southeastern tip of Belgium and \ipi{Lützkampen}/\ipi{Dahnen} in West Central Germany.

The change in \REF{ex:14:47a} involves the replacement of every /ɣ/ with /ʝ/ and every /x/ with /ç/ -- changes accomplished by \REF{ex:14:24c}, which restructured underlying representations. I interpret this change as a generational one: The earlier generation of speakers had underlying representations like /lɛːɣən/ ‘place-\textsc{inf}’ and /fuɣəl/ ‘bird’, which were then restructured by a later generation as /lɛːʝən/ and  /fuʝəl/. That new generation of innovative speakers represents the informants in \citet{Bruch1952}.

Rule loss has been discussed for a number of years in the framework of generative grammar (e.g. \citealt{King1969}, \citealt{Hock1986}, \citealt{RingeEska2013}). One topic investigated in that earlier literature concerns the location of \isi{rule loss} as a stage in the \isi{life cycle of a rule} (\sectref{sec:2.5}). A  number of linguists cited in that section argued that \isi{rule loss} is the endpoint in a long series of stages, whereby loss only occurs after a rule has become morphologized and then lexicalized. In a synthesis of much of the previous work on the \isi{life cycle of a rule}, \citet{Hyman2013} gives the following stages:\footnote{{Hyman defines “morphologizationˮ as the loss of a phonological condition on an alternation, while “lexicalizationˮ means that specific morphemes have to be marked as undergoing or not undergoing an alternation. It is not clear how morphologization translates into the present model, and hence I eschew that term below. To the best of my knowledge there is no variety of German in which velar fronting has been lexicalized.} }

\ea%48
\label{ex:14:48}
phonetic > phonologized > phonemicized > morphologized > lexicalized > loss
\z 

However, there is no evidence from \ipi{Nordösling} -- or \ipi{Burg-Reuland}\slash\ip{Lützkampen}Lützkamp\-en\slash\ipi{Dahnen} -- for a phonemicized stage (i.e. one with phonemic palatals) or a lexicalized version of velar fronting. Instead, the facts point to a situation in which an allophonic rule (\isi{Velar Fronting-1}) -- presumably Hyman’s phonologized stage in \REF{ex:14:48} -- is simply lost without any intermediate stage. What is more, the nonassimilatory change in \REF{ex:14:24c} appears to have been abrupt in the sense that every velar was replaced with the corresponding palatal at the same time. No evidence can be found in the original source for \isi{lexical diffusion}. The interpretation of \isi{rule loss} as an abrupt change is consistent with the treatment of the loss of \isi{schwa} apocope in \ili{Yiddish} proposed by \citet{RingeEska2013} but is not compatible with the analysis of the loss of Final Devoicing in \ili{Yiddish} as lexically gradual endorsed by \citet[268--269]{Hock1986}. The present treatment of rule loss is likewise very different from the lexically gradual approach to rule loss in \ili{West Frisian} discussed by \citet{Tiersma1980}.\largerpage

\subsection{Accounting for unattested trigger types}\label{sec:14.6.4}

The approach to \isi{rule generalization} adopted here presupposes that the nonassimilatory changes in \REF{ex:14:1c} and \REF{ex:14:24} can only become active after velars have been fronted assimilatorily by the stages in \tabref{tab:14:1}. That assumption -- a nonassimilatory change follows an assimilation -- accounts for the two unattested Trigger Types listed in \tabref{tab:14:6}.


\begin{table}
\caption{Unattested Trigger Types for the fronting of velars\label{tab:14:6}}
\begin{tabularx}{.8\textwidth}{XXl}
\lsptoprule
Type & Trigger & Present in context for fronting\\\midrule
G & BV & FV, CC\\
H & BV, CC & FV\\
\lspbottomrule
\end{tabularx}
\end{table}

The reason Trigger Type G represents an unattested system is that fronting cannot begin as a nonassimilatory change by being conditioned solely by back vowels. Instead, the change begins as an assimilation in a specific context conducive to fronting (FV). The change from velar to palatal could therefore not have begun applying before back vowels because that context does not involve an assimilation. A similar explanation holds for the absence of Trigger Type H, in which the context most conducive to velar fronting (FV) is absent, while the one least conducive to fronting (BV) is.

No dialect of German has been discovered in the present survey which has any of the four Trigger Types listed in \tabref{tab:14:7}.\largerpage[2]

\begin{table}
\caption{Unattested Trigger Types involving vocalic triggers\label{tab:14:7}}
\begin{tabularx}{.8\textwidth}{XXl}
\lsptoprule
Type & Trigger & Present in context for fronting\\\midrule
R' & MBV & HBV\\
S' & LBV & HBV, MBV\\
T' & HBV, LBV & MBV\\
U' & MBV, LBV & HBV\\
\lspbottomrule
\end{tabularx}
\end{table}

In a hypothetical dialect with Trigger Type R{}', a velar fronts to palatal in the context of a mid back vowel, but not before a high back vowel. Trigger Type S{}' represents a system involving the shift from velar to palatal in the context of a low back vowel, but fronting is not induced by mid or high back vowels. Trigger Type T{}' and U{}' only include a subset of back vowels as triggers.{\interfootnotelinepenalty=10000\footnote{There are dialects known to me which apparently represent Trigger Type R{}' and Trigger Type S{}'. One source (\citealt{Weber1959}; \mapref{map:11}) for an \il{East Hessian}EHes variety provides a large selection of data pointing to Trigger Type S{}'. \citet{Krafft1969} and \citet[29]{Post1985} make the same observation for the related \il{East Hessian}EHes varieties of \ipi{Schlitzerland} and \ipi{Bad Salzschlirf}. The facts (from \citealt{Weber1959}) are drawn from a number of cities and towns in a broad region (\ipi{Werra-Fuldaraum}): [ç] surfaces after all front vowels, coronal consonants, and the diphthong [ɔɑ] and [x] after all back sounds with the exception of [ɔɑ], e.g. [lɪçt] ‘light’, [tsoɪçt] ‘breeding’, [nɔɑçt] ‘night’ vs. [hoːx] ‘high’, [kɔxə] ‘cake’, [gəruːx]/[gərʊx]/[gərɔx] ‘smell’. The diphthong in examples like [nɔɑçt] derives from etymological [ɑ]. In \sectref{sec:5.2} I discussed similar examples from \ipi{Weidenhausen} (CFr) and argued that the original back vowel underwent a change to a diphthong ending in a front vowel, which then triggered the change from velar to palatal. I hold that the same explanation holds for the data in \citet{Weber1959}, and probably \citet{Krafft1969} and \citet{Post1985} as well.}}

Significantly, the nonoccurring Trigger Types listed in \tabref{tab:14:7} are parallel to the ones discussed in \tabref{tab:12.31}. For example, no dialect is attested in which mid front vowels but not high front vowels trigger fronting, nor are dialects attested in which only low front vowels but not high or mid front vowels condition the rule. Those earlier gaps were accounted for by appealing to the \isi{Implicational Universal for Palatalization Triggers}, which is repeated in \REF{ex:14:49}. If \REF{ex:14:49} is rephrased as in \REF{ex:14:50}, the gaps in \tabref{tab:14:8} can be accounted for.

\eanoraggedright%49
\label{ex:14:49}
  \textsc{\isi{Implicational Universal for Palatalization Triggers}}:\\
  If lower front vowels trigger Palatalization, then so will higher front vowels.
\ex%50
\label{ex:14:50}
  \textsc{\isi{Implicational Universal for Palatalization Triggers}} (revised):\\
  If lower vowels trigger Palatalization, then so will higher vowels.
\z 


The status of \REF{ex:14:50} for non-Gmc languages is unclear because the typological studies cited earlier (\citealt{Chen1973}, \citealt{Bhat1978}, \citealt{Bateman2007,Bateman2011,Bateman2007}, \citealt{Kochetov2011}) do not discuss nonassimilatory \isi{Velar Palatalization} (velar fronting).\footnote{{\citet[64]{Bateman2007} recognizes the existence of Palatalizations triggered by high back vowels (/u/), but languages with that change raise an alveolar (e.g. /s/) to a postalveolar ([ʃ]) in the context of high vowels only (i.e. /i/ and /u/). That type of change therefore entails the assimilation of a height feature, which is very different from velar fronting as discussed in the present chapter. \citet{Bateman2007} has no examples involving the fronting of a velar in the context of all back vowels.} }

\section{{Connection} {between} {word-initial} {and} {postsonorant} {velar} {fronting}}\label{sec:14.7}

In many dialects investigated above velar fronting is active after a sonorant and word-initially. However, the two rules are not always mirror-images of one another because they can differ in terms of the factors identified earlier (targets, triggers, \isi{opacity}). That the triggers and targets for word-initial velar fronting and postsonorant velar fronting for any one dialect are not always the same can be ascertained by comparing those targets and triggers in some of the tables presented in \chapref{sec:12}. Likewise the presence vs. absence of \isi{opacity} need not be identical in word-initial and postsonorant position. To cite one example, in \ipi{Dorste} (\sectref{sec:4.4}) the \isi{palatal quasi-phoneme} /ʝ/ occurs word-initially before \isi{schwa}, but [ʝ] in postsonorant position (from /ɣ) is an allophone which only surfaces after a coronal sonorant. Finally, there is the case of \ipi{Neuendorf} (\sectref{sec:8.5}), in which an underlying palatal undergoes retraction to velar in the context of back vowels in word-initial position, while an underlying velar surfaces as a palatal after coronal sonorants. What all of these examples suggest is that the rules relating velars and palatals word-initially and after a sonorant in any given dialect are independent of one another and can therefore have a life of their own.

A question not discussed above is whether or not a correlation holds between the presence or absence of velar fronting word-initially and after a sonorant. Given the two variables “word-initiallyˮ and “after a sonorantˮ four logical types of dialect obtain (\tabref{tab:14:8}). As indicated there, Type AAA are those varieties with some version of velar fronting in word-initial position but not in postsonorant position, while Type BBB indicates the mirror-image. Type CCC represents dialects alluded to in the preceding paragraph with some version of velar fronting in both contexts, and Type DDD are those dialects with no version of velar fronting.

\begin{table}
\caption{Four types of dialect\label{tab:14:8}}
\begin{tabular}{lll}
\lsptoprule
Type & Description & Dialects attested\\\midrule
AAA & Word-initially only & 0\\
BBB & After a sonorant only & many\\
CCC & Word-initially and after a sonorant & many\\
DDD & {}-{}-{}-{}-{}-{}-{}-{}-{}-{}- & many\\
\lspbottomrule
\end{tabular}
\end{table}

The present survey shows that Types BBB and CCC are robustly attested. The same can be said for Type DDD, although I have only made sporadic reference to those non-velar fronting varieties without attempting to compile a more exhaustive list. The most significant finding in the present study is that Type AAA is not attested.\footnote{{The careful reader may have observed that two varieties were referred to in \sectref{sec:12.3} with word-initial velar fronting, but those same varieties were not discussed in the postsonorant context, namely \ipi{Kirchspiel Courl} (\il{Westphalian}Wph) and \ipi{Reinhausen} (\il{Eastphalian}Eph). However, the sources for those dialects (\citealt{Beisenherz1907} and \citealt{Jungandreas1926,Jungandreas1927} respectively) are clear that velar fronting is also active in postsonorant position.}}

Two interpretations for the absence of Type AAA suggest themselves: First, one could argue that Type AAA represents a systematic gap, in which case such dialects would be considered impossible. Second, the gap could be accidental, meaning that Type AAA -- although clearly dispreferred in German -- could in principle occur. I adopt the second interpretation.

The analysis adopted here is consistent with the fact that there is no phonetic and/or phonological reason why a dialect could not have velar fronting in word-initial position but lack that process in postsonorant position. In languages other than German that type of example should be attested, and in fact this is precisely the case in \ili{Afrikaans} (Appendix~\ref{appendix:i}). I see unattested Type AAA German dialects as a consequence of the history of velar fronting in German: I argue in \chapref{sec:16} that velar fronting was first phonologized in the postsonorant context and was later extended to the word-initial environment in some dialects (Type CCC) but not others (Type BBB). Phonologization of velar fronting in the postsonorant consonant occurred throughout a very large area (virtually all of Germany and most of Austria), while word-initial velar fronting occurred in that broad region only in North and Central Germany. Significantly, those areas of North and Central Germany that developed word-initial velar fronting already had velar fronting in postsonorant position. Given that historical progression the absence of Type AAA dialects can be thought of as a historical accident. However, the present book has uncovered a number of dialects that exhibit highly marked patterns, e.g. nonheight features as triggers (\sectref{sec:12.7}); hence, it would not be inconceivable that there is a marked variety yet undiscovered in which velar fronting is not active at all for many speakers (Type DDD), although some innovative speakers in that same area phonologized velar fronting in word-initial position only (Type AAA).

\section{{Conclusion}}\label{sec:14.8}

In the present chapter I investigated processes of velar fronting in word-initial position and in postsonorant position which are not assimilatory. The claim defended above is that the nonassimilatory fronting of velars can only occur from the historical perspective after the assimilatory fronting of velars. Support for my hypothesis can be found not only in the patterning of dorsal fricatives one finds in German dialects, but also in the unattested patterns.

The survey of velar fronting in German dialects is nearly complete. In the following chapter I consider the status of velar fronting in German-language islands.
\is{nonassimilatory velar fronting|)}
