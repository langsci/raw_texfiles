\chapter{Allophony (part 2)}\label{sec:4}

\section{Introduction}\label{sec:4.1}

The present chapter investigates the allophonic distribution of velars like [x] and palatals like [ç] in three varieties of WLG. In contrast to the UG dialects discussed in \chapref{sec:3} the WLG varieties considered below possess one or two lenis dorsal fricatives, namely velar [ɣ] and palatal [ʝ], in addition to [x] and [ç]. Three systems are compared (System A-C), which are defined according to the target segments for postsonorant velar fronting. In System A the set of targets comprises /x/ as well as the lenis dorsal fricative ({\textbar}ɣ{\textbar}) produced from an underlying /g/ in coda position. That synchronically derived fricative {\textbar}ɣ{\textbar} surfaces in coda position as [ç] after a coronal sonorant and as [x] after a back vowel. In System B /x/ surfaces as [ç] after a coronal sonorant, but /ɣ/ is realized as [ɣ] in a word-internal onset (e.g. between vowels) even if the segment preceding /ɣ/ is a front vowel. However, /ɣ/ is realized as [x] or [ç] in coda position after a back or front vowel respectively. In System C [ʝ] and [ɣ] are positional variants (as are [x] and [ç]); hence, the two palatals [ç ʝ] derive synchronically from the corresponding velars (/x ɣ/) by a version of velar fronting. The conclusion is that velar fronting differs according to the target segments: In System B the target is /x/ (but not /ɣ/), in System C the target consists of both /x/ and /ɣ/, and in System A it cannot be determined if the target consists of the fortis velar fricative only (/x/ and {\textbar}x{\textbar} from /ɣ/) or /x/ and the derived lenis sound {\textbar}ɣ{\textbar} before it hardens to {\textbar}x{\textbar}. The triggers for velar fronting consist of coronal sonorants in System A-C, although it is demonstrated below that the rule fronting /x/ in a word-initial onset to [ç] in System B and System C is triggered only by front vowels but not by coronal sonorant consonants.

In all three dialects the lenis palatal fricative [ʝ] (/ʝ/) surfaces in word-initial position before front vowels and back vowels. That sound was referred to in \sectref{sec:2.4.3} as the \isi{etymological palatal} because it derived historically from the homorganic glide (\ili{WGmc} \textsuperscript{+}[j]). Since the [ʝ] in question never derived historically from a velar sound, its occurrence in the context of back vowels does not involve \isi{opacity}, i.e. the \isi{overapplication} of velar fronting.

The purely transparent distribution of palatals (in the neighborhood of front sounds) and velars (in the neighborhood of back sounds) holds regardless of the historical source of the triggers for velar fronting. For example, velars like [x] occur not only in the context of back segments that were historically back but also when adjacent to back sounds that were historically front (\isi{Vowel Retraction}, \isi{r-Retraction}). Likewise palatals like [ç] surface in the context of front sounds that were etymologically front as well as front sounds that were etymologically back (\isi{Vowel Fronting}). The sounds undergoing velar fronting included not only underlying velars but also new velars created by independent changes.

The effect retractions and frontings had on the triggers for the fronting of velars was discussed in \chapref{sec:3}. In \REF{ex:4:1} and \REF{ex:4:2} I exemplify the formal aspects of those changes. \REF{ex:4:1} depicts retraction, where the phonetic symbols “iˮ and “ɑˮ represent front and back sounds and “xˮ and “çˮ a velar and a palatal. \REF{ex:4:1a} depicts postsonorant position and \REF{ex:4:1b} word-initial position. Retraction (i.e. \isi{Vowel Retraction}/\isi{r-Retraction}) is expressed in \REF{ex:4:1} as /i/ > /ɑ/. The multiple link between the two features [coronal] and [dorsal] to the left of the wedge in the phonetic representation is created by the synchronic rule of velar fronting. Since the front sound in the trigger of fronting is replaced with a back sound (after the wedge) it can be said that the processes of retraction \isi{bleeds} fronting.

\ea\label{ex:4:1}
  \ea\label{ex:4:1a}
  \begin{tikzpicture}[baseline={(matrix-1-1.base)}]
    \matrix (matrix) [matrix of nodes, nodes in empty cells]
      { /i & x/ &[10mm] &[10mm] /ɑ & x/\\[5mm]
        \relax{[\textsc{coronal}]} & {[\textsc{dorsal}]} & & {[\textsc{dorsal}]} & {[\textsc{dorsal}]}\\
                             &                     & > &                   & \\
        \relax {[i} & ç] & & {[ɑ} & x]\\[5mm]
        \relax {[\textsc{coronal}]} & {[\textsc{dorsal}]} & & {[\textsc{dorsal}]} & {[\textsc{dorsal}]}\\
      };
      \foreach \i in {1,2,4,5} \draw (matrix-1-\i) -- (matrix-2-\i);
      \foreach \i in {1,2,4,5} \draw (matrix-4-\i) -- (matrix-5-\i);
      \draw (matrix-4-2.south) -- (matrix-5-1.north);
  \end{tikzpicture}
  \ex\label{ex:4:1b}
  \begin{tikzpicture}[baseline={(matrix-1-1.base)}]
    \matrix (matrix) [matrix of nodes, nodes in empty cells]
      { /x & i/ &[10mm] &[10mm] /x & ɑ/\\[5mm]
        \relax{[\textsc{dorsal}]} & {[\textsc{coronal}]} & & {[\textsc{dorsal}]} & {[\textsc{dorsal}]}\\
                             &                     & > &                   & \\
        \relax {[ç} & i] & & {[x} & ɑ]\\[5mm]
        \relax {[\textsc{dorsal}]} & {[\textsc{coronal}]} & & {[\textsc{dorsal}]} & {[\textsc{dorsal}]}\\
      };
      \foreach \i in {1,2,4,5} \draw (matrix-1-\i) -- (matrix-2-\i);
      \foreach \i in {1,2,4,5} \draw (matrix-4-\i) -- (matrix-5-\i);
      \draw (matrix-4-1.south) -- (matrix-5-2.north);
      \node [left=1mm of matrix-1-1.center, overlay] {\textsubscript{wd}[};
      \node [left=1mm of matrix-1-4.center, overlay] {\textsubscript{wd}[};
  \end{tikzpicture} 
\z
\z


In \REF{ex:4:2} I illustrate how the change shifting back sounds to front sounds (\isi{Vowel Fronting}) affected the fronting of velars in postsonorant position in (\ref{ex:4:2a}) and word-initial position in (\ref{ex:4:2b}). Velar fronting is not present to the left of the wedge in \REF{ex:4:2a} or \REF{ex:4:2b} because the back segment is not a trigger. When that back sound is restructured as front (/ɑ/ > /i/) the velar then fronts to palatal; hence, \isi{Vowel Fronting} \isi{feeds} the fronting of velars.

\ea \label{ex:4:2}
  \ea \label{ex:4:2a} \begin{tikzpicture}[baseline={(matrix-1-1.base)}]
    \matrix (matrix) [matrix of nodes, nodes in empty cells]
      { /ɑ & x/ &[10mm] &[10mm] /i & x/\\[5mm]
        {[\textsc{dorsal}]} & {[\textsc{dorsal}]} & & \relax{[\textsc{coronal}]} & {[\textsc{dorsal}]}\\
                             &                     & > &                   & \\
        \relax {[ɑ} & x] & & {[i} & ç]\\[5mm]
        \relax {[\textsc{dorsal}]} & {[\textsc{dorsal}]} & & {[\textsc{coronal}]} & {[\textsc{dorsal}]}\\
      };
      \foreach \i in {1,2,4,5} \draw (matrix-1-\i) -- (matrix-2-\i);
      \foreach \i in {1,2,4,5} \draw (matrix-4-\i) -- (matrix-5-\i);
      \draw (matrix-4-5.south) -- (matrix-5-4.north);
  \end{tikzpicture}
  \ex\label{ex:4:2b}
  \begin{tikzpicture}[baseline={(matrix-1-1.base)}]
    \matrix (matrix) [matrix of nodes, nodes in empty cells]
      { /x & ɑ/ &[10mm] &[10mm] /x & i/\\[5mm]
        \relax{[\textsc{dorsal}]} & {[\textsc{coronal}]} & & {[\textsc{dorsal}]} & {[\textsc{dorsal}]}\\
                             &                     & > &                   & \\
        \relax {[x} & ɑ] & & {[ç} & i]\\[5mm]
        \relax {[\textsc{dorsal}]} & {[\textsc{dorsal}]} & & {[\textsc{dorsal}]} & {[\textsc{coronal}]}\\
      };
      \foreach \i in {1,2,4,5} \draw (matrix-1-\i) -- (matrix-2-\i);
      \foreach \i in {1,2,4,5} \draw (matrix-4-\i) -- (matrix-5-\i);
      \draw (matrix-4-4.south) -- (matrix-5-5.north);
      \node [left=1mm of matrix-1-1.center, overlay] {\textsubscript{wd}[};
      \node [left=1mm of matrix-1-4.center, overlay] {\textsubscript{wd}[};
  \end{tikzpicture} 
  \z
\z


I discuss three WLG varieties (corresponding to System A-C referred to above), namely NLG (\sectref{sec:4.2}), \il{Westphalian}Wph (\sectref{sec:4.3}), and \il{Eastphalian}Eph (\sectref{sec:4.4}). In \sectref{sec:4.5} I provide some discussion, and in \sectref{sec:4.6} I conclude.

\section{{North} {Low} {German}}\label{sec:4.2}\il{North Low German|(}

In the dialect discussed below (System A) a velar target -- either /x/ or the spirantized realization of /g/ -- surfaces as palatal after a coronal sonorant and elsewhere (after a back vowel) as velar.

\citet{Larsson1917} describes a NLG dialect spoken in \ipi{Altengamme} (\mapref{map:5}). The dialect has phonemic front vowels (/iː ɪ yː ʏ eː ɛ øː œ/), back vowels (/uː ʊ oː o ɔ ɑ ə/), diphthongs ending in a front vowel (/ɔɪ ɑɪ/), and diphthongs ending in a back vowel (/øʊ ɔʊ ɑʊ/). There are three dorsal fricatives [x ç ʝ]. The lenis velar [ɣ] does not occur on the surface, although that sound ({\textbar}ɣ{\textbar}) is created by a rule spirantizing /g/. It is clear from the discussion of the phonetics in the original source \citep[11--12]{Larsson1917} that [ʝ] is a fricative and not a glide.

\begin{map}
% \includegraphics[width=\textwidth]{figures/VelarFrontingHall2021-img007.png}
\includegraphics[width=\textwidth]{figures/Map5_4.1.pdf}
  \caption[North Low German]{North Low German (NLG). Squares indicate postsonorant velar fronting, and circles indicate the absence of postsonorant velar fronting. 27 is a variety of High German spoken in \ipi{Kiel}. 1=\citet{Hobbing1879}, 2=\citet{Kohbrok1901}, 3=\citet{vorMohr1904}, 4=\citet{Schönhoff1908}, 5=\citet{Vehslage1908}, 6=\citet{Rabeler1911}, 7=\citet{Kloeke1914}, 8=\citet{Stammerjochen1914}, 9=\citet{Sievers1914}, 10=\citet{Larsson1917}, 11=\citet{Götze1922} (\ipi{Hollenstedt}), 12=\citet{Götze1922} (Jade), 13=\citet{Jörgensen1928}, 14=\citet{Heigener1937}, 15=\citet{Schmeding1937}, 16=\citet{Feyer1939}, 17=\citet{Feyer1941}, 18=\citet{Bollmann1942}, 19=\citet{Schmidt-Brockhoff1943}, 20=\citet{Pühn1956}, 21=\citet{vonEssen1958}, 22=\citet{Keller1961}, 23=\citet{BethgeBonnin1969} (\ipi{Kiel}), 24=\citet{Mews1971}, 25=\citet{Willkommen1999}, 26=\citet{Höder2010}, 27=\citet{Glover2011, Glover2014}}
  \label{fig:4.1}\label{map:5}
\end{map}

As shown in \REF{ex:4:3}, \ipi{Altengamme} has two underlying dorsal fricatives: /x/ and /ʝ/. The dialect also has a phonemic /g/ which I include in \REF{ex:4:3} because it participates in morphophonemic alternations with [x] and [ç]. The sounds in \REF{ex:4:3a} occur word-initially and the ones in \REF{ex:4:3b} after a sonorant. Appendix~\ref{appendix:h} provides a list of the contrastive consonants for LG dialects like \ipi{Altengamme}.

\ea \label{ex:4:3}
\begin{multicols}{2}\raggedcolumns
\ea \label{ex:4:3a} \begin{forest}
      [,phantom
        [/ʝ/ [{[ʝ]}]]
        [/g/ [{[g]}]]
      ]                    
\end{forest}
\columnbreak\ex \label{ex:4:3b}\begin{forest}
  [,phantom
    [/x/, calign=first [{[x]}] [{[ç]}]]
    [/g/ [{[g]}]]
  ]
\end{forest}
\z 
\end{multicols}
\z 

The formalism in \REF{ex:4:3} expresses traditional phonemes and allophones only; hence, it is not intended to capture morphophonemic alternations between two or more underlying segments, e.g. between [g] and [x]/[ç] alluded to above.

The only context in which [ʝ] surfaces is word-initial. That sound is the \isi{etymological palatal} because it is the modern realization of an earlier palatal glide (\ili{WGmc} \textsuperscript{+}[j]). It is clear from the appendix in \citet{Larsson1917} that there are no constraints on the type of vowel following [ʝ]. For example, [ʝ] can occur before a back vowel in (\ref{ex:4:4a}) or front vowel in (\ref{ex:4:4b}). Word-initial [ʝ] contrasts with [g] (<\ili{WGmc} \textsuperscript{+}[ɣ]), which likewise surfaces before any back vowel in (\ref{ex:4:4c}) or front vowel in (\ref{ex:4:4d}). Singular-plural pairs like [gɑs]{\textasciitilde}[gɛs] ‘guest{\textasciitilde}guest-\textsc{pl}’ show that [g] does not alternate with [ʝ] before a front vowel (cf. data from \ipi{Dingelstedt am Huy} in \sectref{sec:8.4}). [ʝ] does not surface in a word-internal onset (e.g. between vowels) because (i) \ili{WGmc} \textsuperscript{+}[j] in that context either deleted or turned into another sound, and (ii) there were no sound changes that introduced new instances of [ʝ] in a word-internal onset. By contrast, [g] (<\ili{WGmc} \textsuperscript{+}[ɣ]) surfaces in a word-internal onset after a back vowel in (\ref{ex:4:4e}) or front vowel in (\ref{ex:4:4f}). [k]{\textasciitilde}∅ alternations in (\ref{ex:4:4g}) are captured synchronically with an underlying /g/ that surfaces as [k] by in coda position by \isi{Final Fortition} (see below), e.g. /lɑŋg/→[lɑŋk] or by a process deleting /g/ before a vowel, e.g. /lɪŋg-r/→[lɪŋɐ].

\TabPositions{.15\linewidth, .3\linewidth, .45\linewidth, .6\linewidth}
\ea%4
  Word-initial [ʝ] (from /ʝ/) and [g] (from /g/): \label{ex:4:4}
  \ea\label{ex:4:4a} jamɑ  \tab [ʝɑmɐ] \tab Jammer \tab ‘lament’          \tab 11
  \ex\label{ex:4:4b} jȳ    \tab [ʝyː]  \tab ihr    \tab ‘you-\textsc{pl}’ \tab 79
  \ex\label{ex:4:4c} gas   \tab [gɑs]  \tab Gast   \tab ‘guest’           \tab 87
  \ex\label{ex:4:4d} gɪf   \tab [gɪf]  \tab Gift   \tab ‘poison’          \tab 114
  \ex\label{ex:4:4e} mōgɑ  \tab [moː.gɐ]    \tab  mager   \tab ‘lean’     \tab 88
  \ex\label{ex:4:4f} zēgl̩ \tab   [zeː.gl̩] \tab    Segel \tab   ‘sail’     \tab   88
  \ex\label{ex:4:4g} laŋk  \tab [lɑŋk]      \tab  lang    \tab ‘long’     \tab 120\\
                     lɪŋɑ  \tab [lɪŋɐ]      \tab  länger  \tab ‘longer’   \tab  120
  \z
\z 

Velar [x] only occurs after a back vowel in (\ref{ex:4:5a}) and palatal [ç] after a front vowel in (\ref{ex:4:5b}) or sonorant consonant in (\ref{ex:4:5c}).\footnote{{[ç] does not occur after [r] because the latter sound either deletes or merges together with a preceding vowel before a labial or velar \citep[42--48]{Larsson1917}. The two sounds in the [nç] sequence present in other dialects are separated by a vowel, e.g. \ipi{Altengamme} [mɑnɪçmɔʊl] ‘sometimes’; cf. \il{Standard German}StG [mɑnçmɑːl]. There are a few gaps involving long vowels (e.g. [eː]) in (\ref{ex:4:5a}, \ref{ex:4:5b}), which are accidental because they occur in the words with [g] alternations introduced below in \REF{ex:4:7}.} } From the synchronic perspective, [x ç] in \REF{ex:4:5} are the realization of the phoneme /x/. \ipi{Altengamme} [x ç] in \REF{ex:4:5} have several diachronic sources (\ili{WGmc} \textsuperscript{+}[x ɣ gg f]), all of which restructured to /x/. The original [ɣ] in words like [foːx] in \REF{ex:4:5a} and [fɛlç] in \REF{ex:4:5c} is synchronically /x/ and not /g/ because there is no alternant with [g]. Examples like these therefore differ from the alternating examples discussed below.

\TabPositions{.15\linewidth, .3\linewidth, .45\linewidth, .7\linewidth}
\ea%5
    \label{ex:4:5}[x] and [ç] (from /x/):
\ea \label{ex:4:5a}
buxt   \tab [bʊxt]  \tab Bucht   \tab ‘bay’                \tab 109\\
fōx    \tab [foːx]  \tab Vogt    \tab ‘reeve’              \tab 113\\
nɔx    \tab [nɔx]   \tab noch    \tab ‘still’              \tab 123\\
axtɑ   \tab [ɑxtɐ]  \tab hinter  \tab ‘behind’             \tab  84\\
høux   \tab [høʊx]  \tab hoch    \tab ‘high’               \tab  86
\ex \label{ex:4:5b} 
bɪχ    \tab [bɪç]   \tab Beichte \tab ‘confession’         \tab 108\\
bryχ   \tab [brʏç]  \tab Brücke  \tab ‘bridge’             \tab 109\\
fɛχən  \tab [fɛçən] \tab fechten \tab ‘fence-\textsc{inf}’ \tab  86
\ex \label{ex:4:5c} 
fɛlχ   \tab [fɛlç]  \tab Felge   \tab ‘wheel rim’          \tab 112
\z
\z

As in all of the dialects discussed in this book, the front vowel in Umlaut-induced alternations regularly conditions the occurrence of [ç], e.g. [pœç] ‘frog-\textsc{pl}’ (cf. [pɔx] ‘frog’).

Velar /x/ in \REF{ex:4:5} surfaces as palatal [ç] after a coronal sonorant by \isi{Velar Fronting-1} (\sectref{sec:3.2}), which is repeated in \REF{ex:4:6}. Elsewhere (after back vowels) /x/ is realized as [x].

\ea%6
    \isi{Velar Fronting-1}:\label{ex:4:6}\\
    \begin{forest}
     [,phantom
       [\avm{[+son]} [\avm{[coronal]},name=coronal,tier=word]]
       [\avm{[−son\\+cont]},name=parent [\avm{[dorsal]},tier=word]]
     ]
     \draw [dashed] (parent.south) -- (coronal.north);
    \end{forest}
\z 

Since \ipi{Altengamme} does not have /ɣ/ there is no reason for the target segment (/x/) to be specified for a laryngeal feature ([+fortis]). However, underlying /ʝ/ in words like the ones in (\ref{ex:4:4a}, \ref{ex:4:4b}) is a complex (corono-dorsal) fricative marked [−fortis]. /ʝ/ must bear that feature to make its representation distinct from the corono-dorsal structure for the \isi{derived palatal} [ç]. This assumption concerning features holds not only for \ipi{Altengamme} but for all other dialects with /x/ and /ʝ/.

A second source for [x] and [ç] can be observed in \REF{ex:4:7}. These items illustrate a regular alternation between [g] and [x] after a back vowel in (\ref{ex:4:7a}) or between [g] and [ç] after a front vowel in (\ref{ex:4:7b}--\ref{ex:4:7d}). The original source suggests that there are no constraints on the type of back vowel in \REF{ex:4:7a} or front vowel in (\ref{ex:4:7b}--\ref{ex:4:7d}) that occur before these dorsal sounds. The [g] in \REF{ex:4:7} is in a word-internal onset, as reflected in the syllable boundaries in the phonetic representations. [g x ç] in \REF{ex:4:7} derived historically from \ili{WGmc} \textsuperscript{+}[ɣ] (/ɣ/).

\ea%7
\relax \label{ex:4:7} [g]{\textasciitilde}[x]/[ç] alternations (from /g/):
\ea \label{ex:4:7a}
frōɡŋ \tab [froː.gŋ] \tab fragen \tab ‘ask-\textsc{inf}’ \tab 113\\
frōx  \tab [froːx]   \tab Frage  \tab ‘question’         \tab 113

\ex \label{ex:4:7b}
flåɪgŋ \tab [flɔɪ.gŋ] \tab  fliegen \tab ‘fly\textsc{{}-inf}’ \tab 113\\
flăɪχ  \tab [flɑɪç]   \tab  Fliege  \tab ‘fly’                \tab 113

\ex \label{ex:4:7c}
drēgŋ \tab [dreː.gŋ]  \tab tragen \tab ‘carry-\textsc{inf}’   \tab 40\\
drēχ  \tab [dreːç]    \tab trage  \tab ‘carry-\textsc{1sg}’   \tab 40\\
drɪχs \tab [drɪçs]    \tab trägst \tab ‘carry-\textsc{2sg}’ \tab 40
                                                               
\ex \label{ex:4:7d} 
låɪgŋ   \tab [lɔɪ.gŋ] \tab  lügen \tab ‘lie-\textsc{inf}’     \tab 66\\
lyχs    \tab [lʏçs]   \tab  lügst \tab ‘lie-\textsc{2sg}’    \tab 77\\
l\={ø}χ \tab [løːç]   \tab  Lüge  \tab ‘lie’                  \tab 121
\z
\z 

The coda /g/ in \REF{ex:4:7} undergoes \isi{Final Fortition} in (\ref{ex:4:8a}) and \isi{g-Spirantization-1} in (\ref{ex:4:8b}).\footnote{{I have been unable to find examples in which [g] alternates with [ç] after a consonant ([l]). \REF{ex:4:5c} appears to be such an example, but as noted above, the dorsal fricative in that item does not have an alternant with [g]. \isi{Final Fortition} derives independent support from fortis vs. lenis alternations, e.g. [grɑs] ‘grass’ vs. [grɔʊ.zn̩] ‘graze-}\textrm{\textsc{inf}}\textrm{’. The reason that such alternations derive from a lenis sound (/z/) which undergoes fortition in the coda and not from a fortis sound (/s/) which lenites in the onset is that there are items like [lɑɪ.sn̩] ‘afford-}\textrm{\textsc{inf}}\textrm{’ in which [s] (from /s/) surfaces in a word-internal onset.}} \isi{g-Spirantization-1} does not affect [+fortis] /k/, which surfaces in coda position without change, e.g. [lɔk] ‘hole’ (from /lɔk/).

\ea%8
    \label{ex:4:8}
\ea\label{ex:4:8a}\isi{Final Fortition}:\\\relax
            [−sonorant] → [+fortis] / \_\_\_ C\textsubscript{0} ]\textsubscript{${\sigma}$}  ${}$
\ex\label{ex:4:8b}\isi{g-Spirantization-1}:\smallskip\\
 \avm{[−son\\−cont\\−fortis\\dorsal]} →  \avm{[+cont]} / \avm{[−cons]} \_\_\_ C\textsubscript{0} ]\textsubscript{${\sigma}$}
\z 
\z 

On the basis of the data in \citet{Larsson1917} the set of vocalic ([--consonantal]) triggers for \isi{g-Spirantization-1} is the entire natural class of front vowels.

\isi{Final Fortition} (Fnl For) and \isi{Velar Fronting-1} (Vel Fr-1) create transparent outputs, as shown in \REF{ex:4:9a} for [dreːç] (/dreːg/) from \REF{ex:4:7c}. The /g/ in that type of example is parsed as a coda and therefore shifts to {\textbar}ɣ{\textbar} by \isi{g-Spirantization-1} (g-Spir-1). That derived {\textbar}ɣ{\textbar} hardens to [x] and then surfaces as [ç] by \isi{Velar Fronting-1}.

 \ea \label{ex:4:9}
 \begin{multicols}{2}\raggedcolumns 
 \ea  \label{ex:4:9a}  
      \begin{tabular}[t]{@{}ll@{}}
                      & /dreːg/\\ 
             g-Spir-1 & dreːɣ  \\                
             Fnl Fort & dreːx  \\                
             Vel Fr-1 & dreːç  \\                
                      & [dreːç]\\                  
                      & ‘carry-\textsc{1sg}’
     \end{tabular}
\columnbreak    
\ex   \label{ex:4:9b}
      \begin{tabular}[t]{@{}ll@{}}
               & /dreːg/\\
      g-Spir-1 &  dreːɣ \\
      Vel Fr-1 &  dreːʝ \\
      Fnl Fort &  dreːç \\
               & [dreːç]
       \end{tabular}
\z
\end{multicols}
\z

Alternatively, {\textbar}ɣ{\textbar} undergoes \isi{Velar Fronting-1} to {\textbar}ʝ{\textbar} and then \isi{Final Fortition} to [ç], as in \REF{ex:4:9b}. \REF{ex:4:9} illustrates that \isi{Final Fortition} and \isi{Velar Fronting-1} are not ordered.\footnote{{\isi{Final Fortition} \isi{counterbleeds} \isi{g-Spirantization-1} in either scenario, otherwise the underlying /g/ in a word like /dreːg/ would shift to {\textbar}k{\textbar} in the coda and \isi{bleed} \isi{g-Spirantization-1}. However, the \isi{counterbleeding ordering} described here does not involve opaque \isi{overapplication} effects (recall \sectref{sec:2.2.4}).} }

As noted in \REF{ex:4:4}, word-initial palatal [ʝ] (/ʝ/) derived historically from the corresponding glide (\ili{WGmc} \textsuperscript{+}[j]), while [g] (/g/) is the reflex of \ili{WGmc} \textsuperscript{+}[ɣ]. The changes affecting those original sounds are stated in \REF{ex:4:10}:

\ea%10
    \label{ex:4:10}
    \begin{multicols}{2}\raggedcolumns
\ea \isi{Glide Hardening}:\label{ex:4:10a}\\
    WGmc \textsuperscript{+}/j/ > /ʝ/ \textsubscript{${\sigma}$}[ \_\_\_\_
\columnbreak
\ex \isi{g-Formation-1}:\label{ex:4:10b}\\
    WGmc \textsuperscript{+}/ɣ/ > /g/
\z 
\end{multicols}
\z 

\isi{Glide Hardening} is a very general change in LG and CG; see \citet{Hall2014a} and Appendix~\ref{appendix:f}. As observed in \citet{Hall2014a} that change affected all glides and not simply \ili{WGmc} \textsuperscript{+}[j]. \ili{WGmc} \textsuperscript{+}[ɣ] is realized as the corresponding stop ([g]) throughout UG and in many CG and LG varieties (=\ref{ex:4:10b}). \ipi{Altengamme} represents dialects where every instance of \ili{WGmc} \textsuperscript{+}[ɣ] shifted to [g]; other dialects discussed below in \chapref{sec:8} only affect \ili{WGmc} \textsuperscript{+}[ɣ] in certain contexts but not others (e.g. word-initially).

The distribution of velars and palatals in \ipi{Altengamme} holds regardless of the historical source of the triggers for \isi{Velar Fronting-1}. Thus, dialect-specific sound changes shifting original back vowels to front vowels (\isi{i-Umlaut} as an example of \isi{Vowel Fronting}) fed \isi{Velar Fronting-1}, e.g. [pœç] ‘frog-\textsc{pl}’ (cf. [pɔx] ‘frog’). The formal change in that type of example is depicted in \REF{ex:4:2a}. The change from an etymological front vowel to a back vowel (\isi{Vowel Retraction}) appears not to be attested in \ipi{Altengamme}.

The pattern described above differs from what is found in other varieties of NLG, especially those in the vicinity of the \ili{Dutch} border. For example, in \ipi{Lathen} (\citealt{Schönhoff1908}; \mapref{map:5}) there is a contrast between the \isi{etymological palatal} [ʝ] (<\ili{WGmc} \textsuperscript{+}[j]) and velar [ɣ] (<\ili{WGmc} \textsuperscript{+}[ɣ]) in word-initial position (\sectref{sec:8.2}). Since /ɣ/ surfaces consistently as [ɣ] even before front vowels there is no velar fronting in word-initial position. In that same variety velar fronting is also absent in postsonorant position, since [x] (<\ili{WGmc} \textsuperscript{+}[x]) and [ɣ] (<\ili{WGmc} \textsuperscript{+}[ɣ]) surface as velars even after front vowels. Thus, \ipi{Lathen} mirrors \ili{Dutch}, e.g. [zɛx] ‘say-\textsc{1sg}’.\footnote{Here and below I transcribe the Dutch fricative in question broadly as [x]. See \citet{Gussenhoven1992}, \citet{CollinsMees2003}, and \citet{Verhoeven2005} for discussion of its phonetic realization.}

In those varieties of NLG with postsonorant fronting that process is characterized by the broad set of triggers, as in \ipi{Altengamme}. Hence, [ç] (from /x/ or /g/) only surfaces after a coronal sonorant and [x] only after a back vowel, e.g. \ipi{Oldenburg} (\citealt{vorMohr1904}), \ipi{Finkenwärder} \citep{Kloeke1914}, Kreis Herzogtum \ipi{Lauenburg} \citep{Heigener1937}, \ipi{Grambkermoor} \citep{Bollmann1942} and \ipi{Hemmelsdorf} \citep{Pühn1956}. All of these places as well as other ones for similar NLG dialects are indicated on \mapref{map:5}.\il{North Low German|)}

\section{Westphalian}\label{sec:4.3}\il{Westphalian|(}

\il{Westphalian}Wph represents a branch of LG which exhibits little consistency with respect to the distribution of dorsal fricatives. I discuss below a late nineteenth century description of the dialect once spoken in a single town. However, it will be clear in the ensuing chapters that other \il{Westphalian}Wph communities exhibit a very different pattern. The variation involving dorsal fricatives referred to here can be observed throughout the \il{Westphalian}Wph-speaking region over a time frame of approximately ninety years (1886--1974), after which the dialect has essentially become moribund.

The data discussed below have been drawn from the \il{Westphalian}Wph dialect once spoken in the town of \ipi{Soest} ([zoːst]; \citealt{Holthausen1886}; \mapref{map:6}).

The phonemic monophthongs consist of the front vowels /ɪ ɛː ɛ ʏ œː œ/ and the back vowels /ʊ ɔː ɔ ɑː ɑ ə/. \citet[7]{Holthausen1886} lists a total of twenty-one diphthongs. Of those sounds, three have a second element that is front (/ui ɔe ɑe/), while eighteen have a second element that is back. The dorsal fricatives discussed below only occur in the context of six of the diphthongs ending in a back vowel (/iə iːə yə uə iu ɛo/).

\ipi{Soest} has four dorsal fricatives: [x ç] and [ɣ ʝ], whose relationship is expressed in \REF{ex:4:11a} for word-initial position and in \REF{ex:4:11b} for the context after a sonorant. In contrast to \ipi{Altengamme}, the dialect has no [g].

\ea%11
    \label{ex:4:11}
    \begin{multicols}{2}\raggedcolumns
\ea \begin{forest}
    [,phantom
      [/x/,calign=first [{[x]}]   [{[ç]}]]
      [/ʝ/ [{[ʝ]}] ]
    ]
   \end{forest}\label{ex:4:11a}
\columnbreak\ex 
    \begin{forest}
    [,phantom
      [/x/, calign=first [{[x]}]     [{[ç]}]]
      [/ɣ/ [{[ɣ]}]]
    ]
    \end{forest}\label{ex:4:11b}
\z 
\end{multicols}
\z 

As indicated in \REF{ex:4:11}, [ç] and [x] stand in complementary distribution both word-initially and after a sonorant. [ɣ] and [ʝ] likewise never contrast because the latter only occurs initially and the former only after a sonorant. \ipi{Soest} represents System B referred to in \sectref{sec:4.1}.

\begin{map}
% \includegraphics[width=\textwidth]{figures/VelarFrontingHall2021-img008.png}
\includegraphics[width=\textwidth]{figures/Map6_4.2.pdf}
  \caption[Westphalian]{Westphalian (\il{Westphalian}Wph). Squares indicate postsonorant velar fronting and circles the absence of postsonorant velar fronting. 1=\citet{Holthausen1886}, 2=\citet{Hoffmann1887}, 3=\citet{Collitz1899}, 4=\citet{Böger1906}, 5=\citet{Beisenherz1907}, 6=\citet{Arens1908}, 7=\citet{Schwagmeyer1908}, 8=\citet{Brand1914}, 9=\citet{Herdemann1921}, 10=\citet{Wix1921}, 11=\citet{Götz1922} (\ipi{Behringhausen}), 12=\citet{Götz1922} (\ipi{Schinkel}), 13=\citet{Martin1925}, 14=\citet{Gregory1934}, 15=\citet{Hellberg1936}, 16=\citet{Holtmann1939}, 17=\citet{Schulte1941}, 18=\citet{Martin1942} (\ipi{Willingen}), 19=\citet{Martin1942} (\ipi{Sudeck}), 20=\citet{Martin1942} (\ipi{Freienhagen}), 21=\citet{Rakers1944}, 22=\citet{Borchert1955}, 23=\citet{Frebel1957}, 24=\citet{Keller1961}, 25=\citet{BethgeBonnin1969} (Kreis \ipi{Tecklenburg}), 26=\citet{Seymour1970}, 27=\citet{Bethge1970}, 28=\citet{Stellmacher1972}, 29=\citet{Niebaum1974, Niebaum1982}, 30=\citet{NiebaumTeepe1976}, 31=\citet{Brandes2011}.}
  \label{fig:4.2}\label{map:6}
\end{map}

In word-initial position, [x] (=⟦x⟧) surfaces either before a back vowel in (\ref{ex:4:12a}) or a sonorant consonant in (\ref{ex:4:12c}), while [ç] (=⟦c⟧) occurs before a front vowel in (\ref{ex:4:12b}). The sonorant consonant after [x] in \REF{ex:4:12c} is either a liquid ([l] or [ʀ]) or the nasal [n]. \citet[9]{Holthausen1886} describes the rhotic consonant as a dorsal fricative (‘gutturaler Engelaut’). Word-initial [x] surfaces before a consonant regardless of the quality of the vowel following that consonant; in particular, that vowel can be either front (first example in \ref{ex:4:12c}) or back (second two examples). There are a few gaps in the data set below (e.g. no [x] before [ɔː]), which I consider to be accidental. Word-initial [x] and [ç] in examples like these derived historically from \ili{WGmc} \textsuperscript{+}[ɣ]; see \citet[44]{Holthausen1886}.\footnote{{The \il{Standard German}StG cognate verb (infinitive) for [xʀuinə] ‘cry-}\textrm{\textsc{1sg}}\textrm{’ in \REF{ex:4:12c} is} \textrm{\textit{greinen}} \textrm{[gʀɑinən] ‘whine-}\textrm{\textsc{inf}}\textrm{’. The historical precursor for [xnɑːɣn̩] ‘gnaw-}\textrm{\textsc{inf}}\textrm{’ in \REF{ex:4:12c} is \ili{OSax}} \textrm{\textit{gnagan}}.}

\ea%12
\TabPositions{.15\linewidth, .3\linewidth, .6\linewidth, .933\linewidth}
Distribution of word-initial [x] and [ç] (from /x/):\label{ex:4:12}
\ea\label{ex:4:12a}  xuət              \tab [xuət]    \tab gut                 \tab‘good’              \tab 88\\
     xòt               \tab [xɔt]     \tab geht                \tab‘go-\textsc{3sg}’ \tab 73\\
     xɑ̄                \tab [xɑː]     \tab gar                 \tab‘even’              \tab 42\\
     xɑst              \tab [xɑst]    \tab Gast                \tab‘guest’             \tab 44\\
     xədult            \tab [xədʊlt]  \tab Geduld              \tab‘patience’          \tab 15
\ex\label{ex:4:12b}  cist\textit{ɑ}n   \tab [çɪstɐn]  \tab gestern             \tab‘yesterday’         \tab 44\\
     cymln             \tab [çʏml̩n]  \tab  weinerlich sprechen\tab ‘speak whiningly-\textsc{inf}’\tab  44\\
     cèst              \tab [çɛst]    \tab Hefe                \tab‘yeast’                        \tab 43 \\
     co̤\textit{ɑ}tə   \tab  [çœɐtə]  \tab  Grütze             \tab ‘groat’                      \tab  44\\
     cèŏs              \tab [çɛɔs]    \tab Gans                \tab‘goose’                        \tab 44 \\
     c\={ę}\textit{ɑ}n \tab [çɛːɐn]   \tab gern                \tab‘gladly’                       \tab  44\\
     ciəntn            \tab [çiəntn̩] \tab  dort               \tab ‘there’                       \tab   43
\ex\label{ex:4:12c}   xlykə            \tab [xlʏkə]   \tab Glück               \tab‘fortune’                      \tab  84\\
      xruĭnə           \tab [xʀuinə]  \tab weine               \tab‘cry\textsc{{}-1sg}’  \tab 44   \\
      xnɑ̄ʓn            \tab [xnɑːɣn̩] \tab  nagen              \tab ‘gnaw\textsc{{}-inf}’  \tab 44 
\z
\z 

The complementary distribution of word-initial [x] and [ç] also holds after a word-initial consonant (always [s]), as in \REF{ex:4:13}: [x] surfaces before a back vowel in (\ref{ex:4:13a}) or consonant (always [ʀ]), in (\ref{ex:4:13c}) and [ç] before a front vowel in (\ref{ex:4:13b}). The [sx sç] in these examples derived etymologically from \ili{WGmc} \textsuperscript{+}[sk].

\ea%13
Distribution of word-initial [sx] and [sç] (from /sx/):\label{ex:4:13}

\ea\label{ex:4:13a} sxult \tab [sxʊlt] \tab Schuld \tab ‘guilt’ \tab 15\\
    sxɑ̖̄ͅp \tab  [sxɔːp]\tab  Schaf \tab  ‘sheep’\tab  43

\ex\label{ex:4:13b} scylic \tab [sçʏlɪç] \tab  schuldig \tab ‘guilty’ \tab  43\\
    scèpm  \tab [sçɛpm̩] \tab  schöpfen \tab  ‘ladle-\textsc{inf}’ \tab  43

\ex\label{ex:4:13c} sxruĭvə \tab [sxʀuivə] \tab schreibe \tab ‘write-\textsc{1sg}’  \tab 43\\
    sxriʓn  \tab [sxʀɪɣn̩] \tab schreien \tab  ‘scream-\textsc{inf}’\tab   62
\z 
\z 

Holthausen’s discussion of inflectional morphology includes copious examples of regular Umlaut-induced alternations between [x] and [ç] in word-initial position, e.g. [xɑst] ‘guest’ vs. [çɛstə] ‘guest-\textsc{pl}’ in which [x]/[ç] are the reflexes of \ili{WGmc} \textsuperscript{+}[ɣ] and [sxɑp] ‘cabinet’ vs. [sçɛpə] ‘cabinet-\textsc{pl}’, where [sx sç] are the reflexes of \ili{WGmc} \textsuperscript{+}[sk].

[x] and [ç] in \REF{ex:4:12}-\REF{ex:4:13} are surface realizations of underlying /x/ in word-initial onset position. In that context, /x/ surfaces as [ç] by \REF{ex:4:14} and elsewhere (before a back vowel or /ʀ/) as [x]. Since there is no /ɣ/ in word-initial position that could potentially undergo \REF{ex:4:14} there is no reason to specify that its target be marked for a laryngeal feature.

\ea%14
      \isi{Wd-Initial Velar Fronting-3}:\\\label{ex:4:14}
      \begin{forest}
      [,phantom
        [\avm{[−son\\+cont]},name=parent [\avm{[dorsal]},tier=word]]
        [\avm{[−cons]} [\avm{[coronal]},name=coronal,tier=word]]          
      ]
      \draw [dashed] (parent.south) -- (coronal.north);
      \node [left=1mm of parent.west] {\textsubscript{wd} [ (C)};
      \end{forest}
\z 

As indicated above, [coronal] spreads leftward from a front vowel. The feature [--consonantal] in the trigger ensures that /x/ fails to shift to [ç] before coronal consonants like [l] and [n] (cf. \ref{ex:4:12c}).

The data below show that the \isi{etymological palatal} (<\ili{WGmc} \textsuperscript{+}[j]) surfaces in word-initial position before a back vowel in (\ref{ex:4:15a}) or a front vowel in (\ref{ex:4:15b}).\footnote{{Example \REF{ex:4:15b} is rare because word-initial [ʝ] shifted to [ç] before a front vowel \citep[43]{Holthausen1886}. Apparently the word [ʝiuxn̩] ‘cheer-}\textrm{\textsc{inf}}\textrm{’ was an exception to that change.}} As noted earlier, the velar counterpart to [ʝ] (i.e. [ɣ]) never surfaces in word-initial position. [ʝ] in \REF{ex:4:15} is an underlying palatal (/ʝ/).

\ea%15
  Word-initial [ʝ] (from /ʝ/):\label{ex:4:15}
\ea jɑͅͅͅ    \tab [ʝɔː]   \tab ja    \tab ‘yes’  \tab  43\\
    juŋk  \tab [ʝʊŋk]  \tab jung  \tab ‘young’ \tab 43 \label{ex:4:15a}
\ex jiŭxn \tab [ʝiuxn̩] \tab jauchzen \tab ‘cheer-\textsc{inf}’ \tab  43\label{ex:4:15b}
\z
\z 

The data in \REF{ex:4:16} illustrate that [x] and [ç] do not contrast in postvocalic position: [x] surfaces after a back vowel in \REF{ex:4:16a} and [ç] after a front vowel in \REF{ex:4:16b}. \citet{Holthausen1886} also provides many examples exhibiting Umlaut-induced alternations between [x] and [ç], e.g. [dɔxtɐ] ‘daughter’ vs. [dœçtɐ] ‘daughter-\textsc{pl}’. In contrast to some of the dialects discussed above and below, /x/ does not occur after a consonant, although I consider that gap to be accidental. As indicated below, the dorsal fricatives in \REF{ex:4:16} are underlyingly /x/. The diachronic source for [x]/[ç] in (\ref{ex:4:16a}, \ref{ex:4:16b}) is \ili{WGmc} \textsuperscript{+}[x]. The additional examples in \REF{ex:4:16c} show that the diachronic source for /x/ can be a sound other than /x/. In particular, the [x ç] in those items derived historically from \ili{WGmc} \textsuperscript{+}[f] by a change affecting LG (\isi{x-Formation}); cf. \ili{OSax} \textit{luft} ‘air’, \ili{MHG} \textit{niftel(e)} ‘niece’. (\isi{x-Formation} is an example of a change that increased the number of potential targets; recall Rule W from \tabref{tab:2.wxyz}).

\ea%16
  Postvocalic [x] and [ç] (from /x/):\label{ex:4:16}
  \ea\label{ex:4:16a} sòxtə           \tab [sɔxtə]  \tab suchte   \tab ‘search-\textsc{pret}’\tab  44\\
      lɑxən           \tab [lɑxən]  \tab lachen   \tab ‘laugh-\textsc{inf}’    \tab  44
  \ex\label{ex:4:16b} dyctic          \tab [dʏçtɪç] \tab tüchtig  \tab ‘capable’               \tab  44\\
      kröcn           \tab [kʀœçn̩] \tab  husten  \tab  ‘cough-\textsc{inf}’   \tab   44\\
      trèct\textit{ɑ} \tab [tʀɛçtɐ] \tab Trichter \tab ‘funnel’                \tab  14\\
      fröctn          \tab [fʀœçtn̩]\tab  fürchten\tab  ‘fear\textsc{{}-inf}’  \tab   44
  \ex\label{ex:4:16c} luxt            \tab [lʊxt]   \tab Luft     \tab ‘air’                   \tab  44\\
      nicte           \tab [nɪçtə]  \tab Nichte   \tab ‘niece’                 \tab  44
  \z
\z 

A series of sound changes ensured that [x]/[ç] occur after a short vowel and usually before [t] but not after a long vowel. First, historical \textsuperscript{+}[x] deleted in contexts other than before [t]; second, long front monophthongs shortened (and laxed) to [ɪ ʏ ɛ œ] before \textsuperscript{+}[x]; and third, \textsuperscript{+}[x] deleted in word-internal position before a vowel. As a result of those changes there are now no native words in \ipi{Soest} in which [ç] (or [x]) are situated in word-internal onset position.

As indicated in \REF{ex:4:17}, velar [ɣ] (=⟦ʓ⟧) surfaces after a sonorant and before a syllabic nasal or vowel. In the second column, [ɣ] stands in a word-internal onset preceded by a back vowel (\ref{ex:4:17a}), front vowel (\ref{ex:4:17b}), or consonant (\ref{ex:4:17c}). It is shown below that underlying /ʀ/ can occur before /ɣ/, but that the former sound regularly vocalizes to [ɐ] in that position. The [ɣ] in \REF{ex:4:17} derives historically from one of several dorsal sounds (\ili{WGmc} \textsuperscript{+}[ɣ gg j]); see \citet{Hall2014a} for discussion.

\ea%17
      Postsonorant [ɣ] (from /ɣ/):\label{ex:4:17}
 \ea\label{ex:4:17a} vɑ̄ʓn           \tab [vɑː.ɣn̩] \tab  Wagen \tab  ‘car’  \tab  45\\
     ròʓə           \tab [ʀɔ.ɣə]   \tab Roggen \tab ‘rye’   \tab 44\\
     rę\textit{ɑ}ʓn \tab [ʀeɐ.ɣn̩] \tab  Regen \tab  ‘rain’ \tab   44\\
     tīəʓn          \tab [tiːə.ɣn̩]\tab  gegen \tab  ‘against’\tab   44
 \ex\label{ex:4:17b} bryʓə          \tab [bʀʏ.ɣə]  \tab Brücke \tab ‘bridge’  \tab 44\\
     liʓə           \tab [lɪ.ɣə]   \tab liege  \tab ‘lie-\textsc{1sg}’  \tab 44\\
     drèʓn          \tab [dʀɛ.ɣn̩] \tab  drehen\tab  ‘turn-\textsc{inf}’\tab  34\\
     ruĭʓə          \tab [ʀui.ɣə]  \tab Reihe  \tab ‘row’  \tab  44
 \ex\label{ex:4:17c}  bɑlʓə         \tab [bɑl.ɣə]  \tab Balge  \tab ‘brat-\textsc{dat}.\textsc{sg}’  \tab  44
 \z
\z 

The items listed in \REF{ex:4:17b} are significant because they show that the palatal counterpart of [ɣ] (i.e. [ʝ]) does not occur even after a front vowel.\footnote{\textrm{Since \ipi{Soest} has no surface [g], [ɣ]{\textasciitilde}[g] alternations motivating a synchronic rule of g-Spirantization (recall \ref{ex:4:8b}) are absent; see \citet[43]{Holthausen1886}. The post-nasal [k] in words like [diŋk] ‘thing’ arguably derives from an underlying representation /diŋg/, whereby /g/ undergoes \isi{Final Fortition} to [k]; recall the parallel examples from \ipi{Altengamme} in \REF{ex:4:4g}.}}

The dorsal fricatives in \REF{ex:4:16} derive from /x/, which surfaces as [ç] after a front vowel by \REF{ex:4:18} and as [x] in the elsewhere case (after a back vowel). The examples in \REF{ex:4:17} show that the target for fronting cannot be the natural class of dorsal fricatives ([--sonorant, +continuant, dorsal]) because /ɣ/ is unaffected. As noted earlier, there are no examples in the original source in which /x/ occurs after a consonant. I assume that the trigger for fronting is [+sonorant], although it would alternatively be possible to posit that the trigger is [--consonantal], as in \isi{Wd-Initial Velar Fronting-3} in \REF{ex:4:14}. Since \isi{Velar Fronting-4} only affects /x/ but not /ɣ/, surface [ɣ] after front vowels as in \REF{ex:4:17b} does not exemplify \isi{opacity}.

\ea%18
      \isi{Velar Fronting-4}:\label{ex:4:18}\\
      \begin{forest}
       [,phantom
         [\avm{[+son]} [\avm{[coronal]},name=coronal,tier=word]]
         [\avm{[−son\\+cont\\+fortis]},name=parent [\avm{[dorsal]},tier=word]]
       ]
       \draw [dashed] (parent.south) -- (coronal.north);
      \end{forest}
\z 

It was noted above that [x]/[ç] in \REF{ex:4:16} only occur after a short vowel. There is no reason to specify that the trigger in \REF{ex:4:18} be restricted to the context after a short vowel because there are no data in which /x/ is present after a long vowel.

\ipi{Soest} has regular alternations between fortis and lenis fricatives (and fortis and lenis stops). Since fricatives are the focus of the present study, I only concentrate on those alternations here, e.g. [lius] ‘louse’ vs. [lui.zə] ‘louse-\textsc{pl}’ for [s]{\textasciitilde}[z] and [slɑx] ‘blow’ vs. [sleɐ.ɣə] ‘blow-\textsc{pl}’, [viːəx] ‘weigh-\textsc{imp}.\textsc{sg}’ vs. [veɐ.ɣə] ‘weigh-\textsc{1sg}’ for [x]{\textasciitilde}[ɣ]. Those alternations require an underlying lenis sound (e.g. /z ɣ/) that surfaces as fortis in coda position by \isi{Final Fortition} (in \ref{ex:4:8a}); see \citet[75, 76]{Holthausen1886}. Morphemes with nonalternating fortis fricatives preclude analyzing [x]{\textasciitilde}[ɣ] alternations with a rule leniting underlying fortis sounds, e.g. [kʏ.sn̩] ‘kiss-\textsc{inf}’, [lɑ.xn̩] ‘laugh-\textsc{inf}’.

The examples presented above show that postsonorant [x] has two synchronic sources: /x/ in words like the ones \REF{ex:4:16} and /ɣ/ in alternating words like [slɑx] ‘blow’ (cf. [sleɐ.ɣə] ‘blow-\textsc{pl}’) mentioned in the preceding paragraph. The {\textbar}x{\textbar} derived from /ɣ/ regularly shifts to palatal [ç] in coda position after a front vowel, as in \REF{ex:4:19}. \citet{Holthausen1886} lists many strong verbs, nouns and adjectives like these exhibiting alternations along laryngeal and place dimensions (i.e. [ɣ]{\textasciitilde}[x]{\textasciitilde}[ç]).

\ea\label{ex:4:19}Place and laryngeal alternations (from /ɣ/):

\ea\label{ex:4:19a}  stuĭʓn  \tab [stui.ɣn̩] \tab  steigen \tab  ‘climb-\textsc{inf}’   \tab  61\\
     sticst  \tab [stɪçst]   \tab steigst  \tab ‘climb-\textsc{2sg}’    \tab 61\\
     stòĕc   \tab [stɔeç]    \tab stieg    \tab ‘climb-\textsc{pret}’ \tab 61
\ex\label{ex:4:19b}  flɑĕʓn  \tab [flɑe.ɣn̩] \tab  fliegen \tab  ‘fly-\textsc{inf}’     \tab  63\\
      flycst \tab [flʏçst]   \tab fliegst  \tab ‘fly-\textsc{2sg}’       \tab 63\\
      flèŏx  \tab [flɛox]    \tab flog     \tab ‘fly-\textsc{pret}’    \tab 63
\z 
\z 

\isi{Final Fortition} and \isi{Velar Fronting-4} together produce transparent outputs. The /ɣ/ in \REF{ex:4:19} surfaces as [ɣ] in onset position, e.g. [stui.ɣn̩] ‘climb-\textsc{inf}’ and [flɑe.ɣn̩] ‘fly-\textsc{inf}’. In the coda, /ɣ/ shifts to {\textbar}x{\textbar}, which surfaces as [x] after a back vowel (e.g. [flɛɔx] ‘fly-\textsc{pret}’) and as [ç] after a front vowel via \isi{Velar Fronting-4} (Vel Fr-4). \isi{Final Fortition} therefore creates a new {\textbar}x{\textbar} which forms the input to \isi{Velar Fronting-4}, as in \REF{ex:4:20a} for /stɪɣ-st/ ‘climb-\textsc{2sg}’ (from \ref{ex:4:19a}). The word /fʀœxt-n̩/ ‘fear-\textsc{inf}’ (from \ref{ex:4:16b}) is a representative example of /x/ after a front vowel for comparison. The relationship between \isi{Final Fortition} and \isi{Velar Fronting-4} is a \isi{feeding} one, cf. \REF{ex:4:20a}. The reverse ordering in \REF{ex:4:20b} shows that \isi{Final Fortition} cannot \isi{counterfeed} \isi{Velar Fronting-4}. See also \sectref{sec:5.2}, in which I discuss a different \il{Westphalian}Wph variety in which the \isi{counterfeeding} relationship between the two rules in question is correct.

\ea%20
    \label{ex:4:20}
\ea\label{ex:4:20a}\begin{tabular}[t]{@{} p{2cm}p{2.5cm}p{3cm} @{}}
         & /stɪɣ-st/             &  /fʀœxt-n̩/\\                
Fnl For  & stɪx-st   &  ---     \\
Vel Fr-4 &   stɪçst  &  fʀœçtn̩   \\            
         & [stɪçst]  & [fʀœçtn̩]  \\
         & ‘climb-\textsc{2sg}’  & ‘fear-\textsc{inf}’               
    \end{tabular}
\ex\label{ex:4:20b}
\begin{tabular}[t]{@{} p{2cm}p{2.5cm}p{3cm} @{}}
         &  /stɪɣ-st/            &  /fʀœxt-n̩/ \\
Vel Fr-4 &    ---                &   fʀœçtn̩   \\
 Fnl For &  stɪx-st              &   ---    \\
         & *[stɪxst]             &  [fʀœçtn̩]  \\
    \end{tabular}
\z 
\z 

The \isi{feeding} relationship depicted in \REF{ex:4:20a} is a specific example of the hypothetical Dialect A from \figref{fig:2.6}.

As in \ipi{Ramsau am Dachstein} (\sectref{sec:3.5}), \ipi{Soest} has many alternations involving the consonantal rhotic (dorsal [ʀ]) and the vocalized-r ([ɐ]). A discussion of the realization of [ʀ] in the coda can be found in \citet[42]{Holthausen1886}.\footnote{{Note the similarity between Holthausen’s symbol for the vocalized-r (⟦}\textrm{\textit{ɑ}}\textrm{⟧) and his symbol for the short low back vowel (i.e. ⟦ɑ⟧). The discussion of the phonetics of vowels in that source indicates that the two vowels in question are distinct \citep[7]{Holthausen1886}. I transcribe Holthausen’s vocalized-r henceforth as [ɐ].}}

\ea%21
  \relax [ʀ]{\textasciitilde}[ɐ] alternations (from /ʀ/):\label{ex:4:21}
\ea\label{ex:4:21a}   \={ę}rə            \tab [ɛːʀə]   \tab ihre   \tab ‘her-\textsc{infl}’    \tab 25\\
      \={ę}\textit{ɑ}    \tab [ɛːɐ]    \tab ihr    \tab ‘her’                  \tab 25
\ex\label{ex:4:21b}   h\={œ}rə           \tab [hœːʀə]  \tab höre   \tab ‘hear-\textsc{1sg}’    \tab 28\\
      h\={œ}\textit{ɑ}st \tab [hœːɐst] \tab hörst  \tab ‘hear-\textsc{2sg}’    \tab 28
\ex\label{ex:4:21c}   t\={ę}rə           \tab [tɛːʀə]  \tab zehre  \tab ‘feed on-\textsc{1sg}’ \tab 74\\
      t\={ę}\textit{ɑ}st \tab [tɛːɐst] \tab zehrst \tab ‘feed on-\textsc{2sg}’ \tab 74\\
  \z
\z 

The data in \REF{ex:4:22} illustrate that /ɣ/ surfaces as velar ([x]) in coda position after [ɐ]:

\ea%22
  Velar [x] (from /x/) after [ɐ] (from /ʀ/):\\\label{ex:4:22}
b\={ę}\textit{ɑ}x  \tab [bɛːɐx]  \tab Berg  \tab ‘mountain’ \tab  44\\
tv\={ę}\textit{ɑ}x \tab [tvɛːɐx] \tab Zwerg \tab ‘dwarf’    \tab  24\\
bǫ\textit{ɑ}x      \tab [boːɐx]  \tab Borg  \tab ‘barrow’   \tab  44
\z 

\ipi{Soest} /ʀ/ surfaces as [ɐ] in coda position by \isi{r-Vocalization} in \REF{ex:4:23} and elsewhere (in the onset) as [ʀ]. Example \REF{ex:4:17c} indicates that \ipi{Soest} does not vocalize coda /l/ as \ipi{Ramsau am Dachstein}.

\ea%23
    \isi{r-Vocalization}:\\\label{ex:4:23}
    \avm{[+cons\\+son\\−nasal\\dorsal]} → [−cons] / \_\_\_ C\textsubscript{0} ]\textsubscript{${\sigma}$}
\z 

\isi{r-Vocalization} only alters the feature [±consonantal]; hence, the derived sound [ɐ] -- like the input /ʀ/ -- is also [dorsal]. Since [ɐ] is phonologically [dorsal] the occurrence of [x] after that sound is precisely what one would expect in a dialect where [x] and [ç] have a transparent distribution.

The significant point concerning the history of dorsal fricatives in \ipi{Soest} is that sound changes converting original front sounds to back sounds (\isi{Vowel Retraction}) or the reverse (\isi{Vowel Fronting}) had no effect on the distribution of velar and palatal allophones in the neighborhood of those front/back sounds. I consider first word-initial position and then the context after a sonorant.

In the dialect of \ipi{Soest} as it was described in 1886, word-initial [x]/[ç] developed out of \ili{WGmc} \textsuperscript{+}[ɣ] and word-initial [sx]/[sç] from \ili{WGmc} \textsuperscript{+}[sk]. Palatal [ç] occurs before front vowels that were historically front (=\ref{ex:4:24d}) and before front vowels that were etymologically back (=\ref{ex:4:24c}). The surface velar [x] likewise occurs before etymological back vowels (=\ref{ex:4:24a}, \ref{ex:4:24e}) and before vowels that were originally front (=\ref{ex:4:24b}). The reconstructed forms to the right of the wedge are my own; the forms in the third column represent Stage 1. It is assumed that \ili{WGmc} \textsuperscript{+}[ɣ] and \ili{WGmc} \textsuperscript{+}[sk] had already changed to [x] (/x/) and [sx] (/sx/) respectively. The second column represents the point where velar fronting (=\isi{Wd-Initial Velar Fronting-3}) was phonologized but before certain front vowels had changed to \isi{schwa} in (\ref{ex:4:24b}).


\TabPositions{.13\linewidth, .31\linewidth, .5\linewidth, .67\linewidth, .8\linewidth}
\ea%24
    \label{ex:4:24}
\ea\label{ex:4:24a}  [xɑst]   \tab<  \textsuperscript{+}[xɑst]     \tab<  \textsuperscript{+}[xɑst]     \tab‘guest’      \tab cf. OSax \textit{gast}  
\ex\label{ex:4:24b}  [xədʊlt] \tab<  \textsuperscript{+}[çidʊlt]   \tab<  \textsuperscript{+}[xidʊlt]   \tab‘patience’   \tab cf. OSax \textit{githuld}
\ex\label{ex:4:24c}  [çɛɔs]   \tab<  \textsuperscript{+}[çɛɔns]    \tab<  \textsuperscript{+}[xɑs]      \tab‘goose’      \tab cf. MLG \textit{gōs}
\ex\label{ex:4:24d}  [çɪstɐn] \tab<  \textsuperscript{+}[çɪstɐn]   \tab<  \textsuperscript{+}[xɪstɐn]   \tab‘yesterday’  \tab cf. MLG \textit{gisteren} 
\ex\label{ex:4:24e}  [sxʊlt]  \tab<  \textsuperscript{+}[sxʊlt]    \tab<  \textsuperscript{+}[sxʊlt]    \tab‘guilt’      \tab cf. OSax \textit{skuld} 
\ex\label{ex:4:24f}  [sçʏlɪç] \tab<  \textsuperscript{+}[sçʏldɪç]  \tab<  \textsuperscript{+}[sxʊldɪx]  \tab‘guilty’     \tab     
\z
\z 

\begin{sloppypar}
The two historical changes that introduced /x/ into word-initial onsets are stated in (\ref{ex:4:25a}, \ref{ex:4:25b}). \isi{Wd-Initial ɣ-Fortition} and \isi{k-Spirantization} were general sound changes affecting many LG varieties. The vocalic modifications in \REF{ex:4:24} exemplifying \isi{Vowel Fronting} include \isi{i-Umlaut} in (\ref{ex:4:24f}) and a change specific to LG in (\ref{ex:4:24c}), i.e. /ɑ/ > /ɛɔ/. Note that the latter change is classified as \isi{Vowel Fronting} because the part of the new diphthong that is front (/ɛ/) is the one adjacent to the dorsal fricative. \isi{Vowel Fronting} in word-initial position is depicted in \REF{ex:4:2b}. The change whereby unstressed \isi{full vowels} surfaced as \isi{schwa} in (\ref{ex:4:24b}) is presented in \REF{ex:4:25c}; recall \sectref{sec:2.4.3}. If the original vowel is front (=\ref{ex:4:24b}) then \isi{Vowel Reduction} can be classified as a particular type of \isi{Vowel Retraction}. \isi{Vowel Reduction} is a major sound change that affected virtually all LG and HG dialects.
\end{sloppypar}

\ea%25
    \label{ex:4:25}
\ea Wd-Initial ɣ-Fortition:\\\label{ex:4:25a}
    /ɣ/ > /x/ \textsubscript{wd}[ \_\_\_\_ 

\ex \isi{k-Spirantization}:\\\label{ex:4:25b}
    /k/ > /x/ \textsubscript{wd}[ s \_\_\_\_
 
\ex \isi{Vowel Reduction}:\\\label{ex:4:25c}
/\{ unstressed vowel \}/ >  /ə/
\z 
\z 

In \REF{ex:4:26} I give historical derivations for three examples from \REF{ex:4:24}. At Stage 1 Velar fronting has not yet been phonologized, but \isi{Wd-Initial ɣ-Fortition} and \isi{k-Spirantization} had already introduced new instances of [x] (/x/). Stage 2 represents the dialect as it was described by Ferdinand Holthausen in 1886. The intermediate stage (“Pre-Soestˮ) corresponds to the second column in \REF{ex:4:24}: This is the point after \isi{Wd-Initial Velar Fronting-3} had been phonologized but before \isi{Vowel Reduction} had restructured unstressed \isi{full vowels} to \isi{schwa}.

\ea%26
    \label{ex:4:26}
    \begin{tabular}[t]{@{} llll @{}}                                     
        /xɑst/        &  /xidʊlt/          & /xɪstɐn/         &          \\
  \relax[xɑst]        &   [xidʊlt]         & [xɪstɐn]         & Stage 1  \\\tablevspace
        /xɑst/        &  /xidʊlt/          & /xɪstɐn/         &          \\
  \relax[xɑst]        &   [çidʊlt]         & [çɪstɐn]         & Pre-\ipi{Soest}\\\tablevspace
        /xɑst/        &  /xədʊlt/          & /xɪstɐn/         &          \\
  \relax[xɑst]        &   [xədʊlt]         & [çɪstɐn]         & Stage 2  \\\tablevspace
        \textit{Gast} &  \textit{Geduld}   & \textit{gestern} & \il{Standard German}StG   \\
        ‘guest’       &  ‘patience’        & ‘yesterday’      &          \\
    \end{tabular}
\z 

The word [xədʊlt] requires comment. The initial fricative in that type of example was a surface palatal at the Pre-\ipi{Soest} stage, prior to \isi{Vowel Reduction}. That stage ([çidʊlt]) is not attested in any modern German dialect, although the material presented in the LG dialects discussed in \chapref{sec:7} and \chapref{sec:8} makes it clear that there must have been that earlier stage (cf. \ili{OSax} \textit{giduld}). Since the vowel following /x/ in [xədʊlt] is \isi{schwa}, the fricative /x/ cannot front to palatal and therefore surfaces as [x]. The change from a Pre-\ipi{Soest} sequence like /xi/ to Stage 2 /xǝ/ involved \isi{Vowel Reduction}, which deleted the feature [coronal] from the front vowel (/i/). The significance of that development is discussed in Chapters~\ref{sec:7} and~\ref{sec:8} where it is shown that there are other German dialects in which the initial sound surfaces as palatal [ç] before \isi{schwa}.

The reflexes of \ili{WGmc} \textsuperscript{+}[ɣ] in postsonorant position are given for three representative words in \REF{ex:4:27}. The reconstructions are my own.

\ea%27
\label{ex:4:27}
\TabPositions{.15\linewidth, .2\linewidth, .35\linewidth, .55\linewidth, .8\linewidth}
\ea\label{ex:4:27a}  \relax [stui.ɣn̩] \tab  < \tab  \textsuperscript{+}[stiɣɑn]\tab  ‘climb-\textsc{inf}’  \tab  cf. OSax \textit{stīgan}\tab  (from \ref{ex:4:19a})
\ex\label{ex:4:27b}  \relax [stɪçst]   \tab <  \tab \textsuperscript{+}[stiɣst] \tab ‘climb-\textsc{2sg}’  \tab                          \tab (from \ref{ex:4:19a})
\ex\label{ex:4:27c}  \relax [bɛːɐx]    \tab <  \tab \textsuperscript{+}[berɣ]   \tab ‘mountain’            \tab cf. OSax \textit{berg}   \tab (from \ref{ex:4:22})
\z 
\z 

Example \REF{ex:4:27c} deserves comment. As in \ipi{Ramsau am Dachstein} (\sectref{sec:3.5}), the [dorsal] rhotic derived historically from the corresponding [coronal] sound /r/. Evidence for that assumption is that the [coronal] sound is retained as [r] (/r/) in many closely related LG dialects discussed throughout this book. My treatment in \REF{ex:4:27c} is also consistent with the description of the rhotic consonant in the original source, where \citet[43]{Holthausen1886} observes that the original realization ([r]) is retained in the villages surrounding \ipi{Soest}. Given the earlier stage with [r] (/r/), \isi{r-Retraction} (\sectref{sec:3.5}) must have restructured the [coronal] rhotic in \ipi{Soest} to the [dorsal] rhotic.

The three examples from \REF{ex:4:27} are illustrated in \REF{ex:4:28}. Stage 2B is the \ipi{Soest} dialect of 1886. Recall from \REF{ex:4:20} that both \isi{Velar Fronting-4} and \isi{Final Fortition} are present. It is assumed here that \isi{Final Fortition} was already active at Stage 1; see \sectref{sec:5.5.2} for discussion.\largerpage[2]

\ea%28
\label{ex:4:28}
\begin{tabular}[t]{@{} llll @{}}
  /stuiɣ-n/            &   /stɪɣ-st/            &  /bɛːrɣ/      &           \\
\relax [stui.ɣn̩]            &   [stɪxst]            &    [bɛːrx]    & Stage 1   \\\tablevspace
  /stuiɣ-n/            &   /stɪɣ-st/            &    /bɛːrɣ/    &           \\
\relax  [stui.ɣn̩]            &   [stɪçst]            &    [bɛːrç]    & Stage 2A  \\\tablevspace
  /stuiɣ-n/            &   /stɪɣ-st/            &    /bɛːʀɣ/    &           \\
\relax  [stui.ɣn̩]            &   [stɪçst]            &    [bɛːɐx]    & Stage 2B  \\\tablevspace
 \textit{steigen}      &  \textit{steigst}     & \textit{Berg} &   \il{Standard German}StG\\
 ‘climb-\textsc{inf}’  &  ‘climb-\textsc{2sg}’ &   ‘mountain’  &         \\
  \end{tabular}
\z 


Stage 2A represents the point where \isi{Velar Fronting-4} and \isi{Final Fortition} were first active together in the synchronic phonology. Crucially, the rhotic in words like [bɛːrç] ‘mountain’ was still coronal. Since the set of triggers for Velar Front\-ing-4 includes all coronal sonorants, the dorsal fricative in words like [bɛːrç] surfaced as palatal at Stage 2A. Evidence for Stage 2A comes from dialects spoken roughly at the same time as the one \citet{Holthausen1886} describes in \ipi{Soest} where the [rç] sequence is preserved, cf. [bɑrç] ‘mountain’ in \ipi{Dorste} in \REF{ex:4:34a} below.

Stage 2B (=\ipi{Soest} in 1886) represents the point where \isi{r-Retraction} restructured /r/ to /ʀ/. The change from /r/ to /ʀ/ is a specific example of retraction, which was depicted above for vowels in \REF{ex:4:1a}: In the first column of \REF{ex:4:28}, the feature [coronal] is linked to both sounds (e.g. [r] and [ç]) as a consequence of \isi{Velar Fronting-4}. When /r/ restructured to /ʀ/ by \isi{r-Retraction} [coronal] was replaced with [dorsal], as illustrated to the right of the wedge in \REF{ex:4:1a}. The consequence of that change is that the surface palatal [ç] depicted before the wedge reverted back to [x] (after the wedge).

Each variety of \il{Westphalian}Wph needs to be assessed individually because there are few generalizations concerning the patterning of dorsal fricatives that hold for that entire branch of WLG. One might conclude that there is nothing at all unusual about the transparent patterning of dorsal fricatives in \ipi{Soest}, but this impression is not correct when one compares the \ipi{Soest} system with other \il{Westphalian}Wph (and \il{Eastphalian}Eph) ones. The conclusion is that the system of velar and palatal fricatives in \ipi{Soest} is more the exception than the rule.

\ipi{Soest} has processes fronting velars in word-initial position and after a sonorant. However, some \il{Westphalian}Wph varieties are attested with no velar fronting in word-initial position; hence, /x/ (<\ili{WGmc} \textsuperscript{+}[ɣ] or after initial \textsuperscript{+}[s]) surfaces as [x] even before front vowels, e.g. \ipi{Gütersloh} (\citealt{Wix1921}; \mapref{map:6}). In \il{Westphalian}Wph varieties with either word-initial or postsonorant velar fronting the targets and triggers for those processes are not necessarily the same as the targets and triggers for the same processes in \ipi{Soest}. In word-initial onset position \ipi{Soest} /x/ surfaces as [ç] before a narrow set of sounds (front vowels but not before coronal consonants). That pattern is essentially the same in \ipi{Laer} (\citealt{Niebaum1974,Niebaum1982}; \mapref{map:6}), but in \ipi{Elspe} (\citealt{Arens1908}; \mapref{map:6}) the set of triggers is broader (all coronal sonorants). For postsonorant position the sole target for fronting in \ipi{Soest} is /x/ (but not /ɣ/). That narrow set of targets is attested in other varieties of \il{Westphalian}Wph, e.g. \ipi{Adorf} (\citealt{Collitz1899}; \mapref{map:6}) and \ipi{Laer} (\citealt{Niebaum1974, Niebaum1982}; \mapref{map:6}). The same point holds for \il{Eastphalian}Eph, e.g. \ipi{Meinersen} (\citealt{Bierwirth1890}; \mapref{map:7}) and \ipi{Börßum} (\citealt{Heibey1891}; \mapref{map:7}), as well as several varieties of WCG (\sectref{sec:12.3.4}). That pattern can be contrasted with other varieties with a broader set of targets (i.e. /x/ and /ɣ/; \sectref{sec:4.4}). 

The patterning of [x]/[ç] from /x/ is allophonic in \ipi{Soest}, but that type of system can be contrasted with ones in which [ç] has been quasi-phonemicized to /ç/, e.g. \ipi{Elspe} (\citealt{Arens1908}; \mapref{map:6}) and \ipi{Schieder-Schwalenberg} (\citealt{Böger1906}; \mapref{map:6}). Examples like these are discussed in detail in \chapref{sec:7}.\il{Westphalian|)}

\section{{Eastphalian}}\label{sec:4.4}\il{Eastphalian|(}

The present section investigates the complementary distribution of velar fricatives ([x ɣ]) with the corresponding palatals ([ç ʝ]) in an \il{Eastphalian}Eph variety. The significance of this case study is that the target segments for postsonorant fronting consist of both /x/ and /ɣ/ and not just /x/, as in \ipi{Soest} (System C referred to in \sectref{sec:4.1}).

The data discussed below are drawn from the \il{Eastphalian}Eph dialect once spoken in and around the town of \ipi{Dorste} (\citealt{Dahlberg1934,Dahlberg1937}; \mapref{map:7}). See also \citegen{Mackel1939} phonetic transcriptions of a speaker from \ipi{Dorste} (Osterode am Harz).

\begin{map}
% \includegraphics[width=\textwidth]{figures/VelarFrontingHall2021-img009.png}
\includegraphics[width=\textwidth]{figures/Map7_4.3.pdf}
  \caption[Eastphalian]{Eastphalian (\il{Eastphalian}Eph). Squares indicate postsonorant velar fronting. 1=\citet{Bierwirth1890}, 2=\citet{Heibey1891}, 3=\citet{Roloff1902}, 4=\citet{Block1910}, 5=\citet{Damköhler1919}, 6=\citet{Jungandreas1926}, \citet{Jungandreas1927}, 7=\citet{Jarfe1929}. 8= \citet{Löfstedt1933}, 9=\citet{Dahlberg1934, Dahlberg1937}, 10=\citet{Mackel1939} (Osterode am Harz), 11=\citet{Mackel1939} (Hildesheim), 12=\citet{Hille1939}, 13=\citet{Hassel1942}, 14=\citet{Pahl1943}, 15=\citet{Brugge1944}, 16=\citet{Schütze1953}, 17=\citet{BethgeFlechsig1958}, 18=\citet{Lange1963}, 19=\citet{BethgeBonnin1969}, 20=\citet{BethgeBonnin1969}, 21=\citet{Göschel1973}, 22=ACeM.}
  \label{fig:4.3}\label{map:7}
\end{map}

The phonemic monophthongs consist of the front vowels /iː ɪ yː ʏ eː ɛː ɛ œ/ and the back vowels /uː ʊ ɔ ɑː ɑ ə/. I omit Dahlberg’s ⟦\={œ}⟧ because no example was found in that source with a dorsal fricative in the neighborhood of that vowel. Of the twelve diphthongs listed in the original source I only consider the eight which occur in the context of dorsal fricatives. Those diphthongs end in a front vowel (/ʊɪ ɔɪ ɑɪ/) or back vowel (/iːə uːə eːə ɛʊ ɑʊ/).

\ipi{Dorste} possesses the four dorsal fricatives [x ç ɣ ʝ], whose relationship is depicted word-initially \REF{ex:4:29a} and after a sonorant in \REF{ex:4:29b}. There are no contrasts between velar and the corresponding palatal.

\ea\label{ex:4:29}
\begin{multicols}{2}
\ea\label{ex:4:29a}
   \begin{forest}
   [,phantom
    [/x/,calign=first [{[x]}]   [{[ç]}]]              
    [/ʝ/ [{[ʝ]}]]    
   ]
    \end{forest}
\ex\label{ex:4:29b}
   \begin{forest}
    [,phantom
     [/x/,calign=first [{[x]}]     [{[ç]}]]                
     [/ɣ/,calign=first [{[ɣ]}]       [{[ʝ]}]]
    ]
    \end{forest}
\z 
\end{multicols}
\z 

The word-initial pattern in \REF{ex:4:29a} is the same as in \ipi{Soest}. The difference between \ipi{Soest} and \ipi{Dorste} is in postsonorant position, as indicated in \REF{ex:4:29b}, cf. \REF{ex:4:11b}.

Word-initial [x] (=⟦x⟧) surfaces either before a back vowel in (\ref{ex:4:30a}) or a sonorant consonant in (\ref{ex:4:30c}) and [ç] (=⟦χ⟧) before a front vowel in (\ref{ex:4:30b}); see \citet[15]{Dahlberg1937}. The coronal sonorant consonant after [x] in \REF{ex:4:30c} is either a liquid ([l]/[r]) or nasal [n]. [r] is coronal because it is articulated with the tongue tip (“Zungenspitzen-r”: \citealt{Dahlberg1937}: 5). Word-initial [x] surfaces before a consonant regardless of the quality of the vowel following that consonant; in particular, it can be either front (as in the final example in \ref{ex:4:30c}) or back (as in the first two examples). Word-initial [x ç] in \REF{ex:4:30} derived historically from \ili{WGmc} \textsuperscript{+}[ɣ] by Wd-Initial ɣ{}-Fortition (in \ref{ex:4:25a}).\footnote{{In contrast to \ipi{Soest}, there are no words containing [x] or [ç] after a word-initial \isi{sibilant} (recall  \ref{ex:4:13}). The corresponding examples in \ipi{Dorste} are realized as [ʃ] (/ʃ/), e.g. [ʃɑːf] ‘sheep’ (cf. \ipi{Soest} [sxɔːp]).} \textrm{Dahlberg’s ⟦ē̜⟧ is described as a vowel corresponding to \il{Standard German}StG} \textrm{\textit{spät}} \textrm{‘late’ \citep[13]{Dahlberg1934} and is therefore transcribed as [ɛː]. His ⟦ɛ⟧ expresses a vowel quality between [eː] (=⟦ē⟧) and [ɛː] (=⟦ē̜⟧). I transcribe ⟦ɛ⟧ below as [ɛː] because my treatment does not hinge on the fine-grained vowel qualities described in the original source. For the same reason, I transcribe Dahlberg’s ⟦e̜⟧ and ⟦ë̜⟧ both as [ɛ].}}

\ea%30
\TabPositions{.133\linewidth, .295\linewidth, .6\linewidth, .95\linewidth}
Distribution of word-initial [x] and [ç] (from /x/):\label{ex:4:30}
\ea\label{ex:4:30a}
      xūətə  \tab [xuːətə]  \tab  Gosse  \tab ‘gutter’    \tab 73  \\
      xu̜nst \tab  [xʊnst]  \tab   Gunst \tab  ‘favor’    \tab  73 \\
      xo̜t   \tab  [xɔt]    \tab   Gott  \tab  ‘God’      \tab  15 \\
      xɑ̣̄l  \tab  [xɑːl]   \tab  geil  \tab  ‘horny'     \tab  73 \\
      xɑst   \tab [xɑst]    \tab  Gast   \tab ‘guest’     \tab 73  \\
      xɑi̜st \tab  [xɑɪst]  \tab   Geist \tab  ‘intellect’\tab  72 \\
      xau̜t  \tab  [xɑʊt]   \tab   gut   \tab  ‘good’     \tab  15
\ex\label{ex:4:30b}
      χījn̥      \tab [çiːʝn̩]     \tab  gegen             \tab  ‘against’                       \tab  15   \\
      χi̜stə(r)n \tab [çɪstə(r)n]  \tab gestern            \tab ‘yesterday’                      \tab 74    \\
      χü̜stə     \tab [çʏstə]      \tab keine Milch gebend \tab ‘not giving\textsc{{}-part} milk’\tab 74    \\
      χē̜və      \tab [çɛːvə]      \tab gäbe               \tab ‘give-\textsc{subj}’             \tab  74   \\
      χe̜lt      \tab [çɛlt]       \tab Geld               \tab ‘money’                          \tab  73   \\
      χö̜rtl̥     \tab [çœrtl̩]    \tab   Gürtel          \tab    ‘belt’                         \tab    74 \\
      χēərn     \tab [çeːərn]   \tab  gern                \tab ‘gladly’                          \tab 74    \\
      χë̜u̜s    \tab   [çɛʊs]   \tab    Gans              \tab   ‘goose’                         \tab   73  \\
      χīən      \tab [çiːən]    \tab  jäten               \tab ‘weed-\textsc{inf}’               \tab 74    
\ex\label{ex:4:30c}   xlɑs      \tab [xlɑs]     \tab  Glas                \tab ‘glass’                           \tab 15   \\
      xrɑs      \tab [xrɑs]     \tab  Gras                \tab ‘grass’                           \tab   15 \\
      xnë̜u̜ʓn̥ \tab    [xnɛʊɣn̩]   \tab  nagen           \tab     ‘gnaw-\textsc{inf}’               \tab   73
\z 
\z

\citet{Dahlberg1937} gives many morphemes with [x]{\textasciitilde}[ç] alternations triggered by \isi{Umlaut}, e.g. [xɑst] ‘guest’ vs. [çɛstə] ‘guest-\textsc{pl}’. The two sounds [x] and [ç] in those alternating examples and in \REF{ex:4:30} are surface realizations of the phoneme /x/, which surfaces as [ç] in word-initial position if the following segment is a front vowel (by \isi{Wd-Initial Velar Fronting-3} from \ref{ex:4:14}).\largerpage[-3]

The \isi{etymological palatal} ([ʝ] < \ili{WGmc} \textsuperscript{+}[j]) surfaces in word-initial position before a back vowel in (\ref{ex:4:31a}) or front vowel in (\ref{ex:4:31b}). The velar counterpart to [ʝ] (i.e. [ɣ]) never occurs in word-initial position.\footnote{[ʝ] (/ʝ/) also derived historically from \ili{WGmc} \textsuperscript{+}[ɣ] in word-initial position before \isi{schwa}, e.g. [ʝəvɑlt] ‘violence’ (=⟦jəvɑlt⟧), cf.\ili{OSax} \textit{giwald}. The [ʝ] (/ʝ/) in that type of example is a \isi{palatal quasi-phoneme}, which is discussed in other German dialects in detail in \chapref{sec:7}. No \il{Eastphalian}Eph dialect in the present study is completely free from \isi{opacity}, although the one opaque palatal in \ipi{Dorste} is extremely limited in its occurrence. Note that the corresponding examples in \ipi{Soest} have a velar ([x]), e.g. [xədʊlt] ‘patience’ in \REF{ex:4:12a}; cf. \ili{OSax} {\textit{giduld}}.}

\ea\label{ex:4:31}%31
 Word-initial [ʝ] (from /ʝ/):
 \TabPositions{.15\linewidth, .33\linewidth, .5\linewidth, .775\linewidth}
\ea\label{ex:4:31a} ju̜ŋə   \tab [ʝʊŋə]    \tab  Junge \tab  ‘boy’    \tab 76\\
    jɑmə(r) \tab[ʝɑmə(r)ə] \tab Jammer \tab ‘lament’  \tab 76
\ex\label{ex:4:31b}
jü̜kn̥\tab  [ʝʏkn̩]  \tab  jucken \tab  ‘itch-\textsc{inf}’ \tab 76\\
jö̜k  \tab [ʝœk]   \tab  euch   \tab  ‘you-\textsc{acc/dat}.\textsc{pl}’ \tab 76
\z 
\z

In postsonorant position [x] and [ç] stand in an allophonic relationship: [x] surfaces after a back vowel in (\ref{ex:4:32a}) and [ç] after a front vowel in (\ref{ex:4:32b}) or a coronal sonorant consonant in (\ref{ex:4:32c}). The nonoccurrence of [x ç] after phonemic vowels other than the ones listed below or after /l n/ is accidental. Due to Umlaut-induced vowel changes there are many [x]{\textasciitilde}[ç] alternations, e.g. [hɛʊx] ‘high’ vs. [hœçst] ‘highest’. [x ç] in alternating examples like these and [x ç] in \REF{ex:4:32} derive synchronically from /x/ by \isi{Velar Fronting-1}. Historically the dorsal fricatives in (\ref{ex:4:32a}--\ref{ex:4:32c}) derive from \ili{WGmc} \textsuperscript{+}[x]. As in \ipi{Soest} (=\ref{ex:4:16c}), /x/ can also derive historically from /f/ by \isi{x-Formation} (=\ref{ex:4:32d}).


\TabPositions{.133\linewidth, .295\linewidth, .5  \linewidth, .75\linewidth, .95\linewidth}

\ea%32
      Postsonorant [x] and [ç] (from /x/):\label{ex:4:32}
\ea\label{ex:4:32a} do̜x\tab  [dɔx]\tab   doch \tab  ‘however’ \tab  15\\
    ɑxt \tab [ɑxt] \tab  acht  \tab ‘eight’    \tab 64

\ex\label{ex:4:32b} bi̜χtə  \tab [bɪçtə]  \tab Beichte   \tab ‘confession’          \tab 16\\
    lü̜χtn̥ \tab  [lʏçtn̩]\tab   leuchten\tab   ‘glow-\textsc{inf}’ \tab   80\\
    fu̜i̜χt \tab  [fʊɪçt] \tab  feucht   \tab  ‘damp’               \tab  72

\ex\label{ex:4:32c} fo̜rχt \tab [fɔrçt] \tab  Furcht \tab ‘fear’    \tab 71
\ex\label{ex:4:32d} lu̜xt  \tab [lʊxt]  \tab  Luft   \tab ‘air’     \tab 16\\
    e̜χt   \tab [ɛçt]   \tab  echt   \tab ‘genuine’ \tab 16
   \z
\z 

Recall from the discussion after \REF{ex:4:16} that there are historical reasons for why the dorsal fricatives in LG items like these only occur after a short vowel (typically followed by [t]).

[ɣ] (=⟦ʓ⟧) and [ʝ] (=⟦j⟧) have a distribution that parallels the one involving [x] and [ç] (see \ref{ex:4:29b}). As indicated in \REF{ex:4:33}, velar [ɣ] surfaces in a word-internal onset after a back vowel in (\ref{ex:4:33a}) and palatal [ʝ] in a word-internal onset after a front vowel in (\ref{ex:4:33b}) or coronal sonorant consonant in (\ref{ex:4:33c}). In the overwhelming majority of examples like these [ɣ ʝ] are the reflexes of \ili{WGmc} \textsuperscript{+}[ɣ]. As noted below, [ɣ ʝ] in \REF{ex:4:33} derive synchronically from velar /ɣ/.\footnote{{The fricative in words like [drɛː.ʝn̩] ‘turn-}\textrm{\textsc{inf}}\textrm{’ in \REF{ex:4:33b} is the reflex of \ili{WGmc}} \textrm{\textsuperscript{+}}\textrm{[j]. That glide shifted to velar /ɣ/ after a front vowel by a regular change that affected LG dialects \citep{Hall2014a}. Recall from \REF{ex:4:17b} that the /ɣ/ in question is preserved as [ɣ] in \ipi{Soest} (e.g. [dʀɛ.ɣn̩] ‘turn-}\textrm{\textsc{inf}}\textrm{’).} }

\ea%33
\label{ex:4:33}
   Postsonorant [ɣ] and [ʝ] (from /ɣ/):

\ea\label{ex:4:33a} būʓn̥  \tab [buː.ɣn̩] \tab Bogen \tab  ‘bow’ \tab 16\\
    ë̜u̜ʓə \tab  [ɛʊ.ɣə]  \tab Auge  \tab ‘eye’  \tab 70

\ex\label{ex:4:33b} hīʝn̥       \tab  [hiː.ʝn̩] \tab hegen \tab ‘foster-\textsc{inf}’   \tab  75\\
    bǖjl̥       \tab  [byː.ʝl̩] \tab Bügel \tab ‘clamp’                 \tab  68\\
    fɛ̄jn̥       \tab  [fɛː.ʝn̩] \tab fegen \tab ‘sweep-\textsc{inf}’    \tab  75\\
    dr\={ę}jn̥  \tab  [drɛː.ʝn̩]  \tab  drehen  \tab ‘turn-\textsc{inf}’\tab  69\\
    mo̜i̜jətə    \tab  [mɔɪ.ʝətə] \tab  Mägde   \tab ‘maidservant-\textsc{pl'}     \tab  15\\
    dru̜i̜ʝə     \tab  [drʊɪ.ʝə]  \tab  trocken \tab ‘dry’              \tab  69\\
    flɑi̜ʝə     \tab  [flɑɪ.ʝə]   \tab Fliege   \tab‘fly’               \tab 71

\ex\label{ex:4:33c} fe̜ljə \tab [fɛl.ʝə] \tab  Felge  \tab ‘wheel rim’ \tab 16\\
    mo̜rjə \tab [mɔr.ʝə] \tab  morgen \tab  ‘tomorrow’ \tab 81
   \z
\z 

In postsonorant position palatals occur only after coronal sonorants and velars after back vowels; hence, there are no contrastive sequences like [ix] vs. [ɑç] (or [iɣ] vs. [ɑʝ] before a vowel).

The examples in \REF{ex:4:33} illustrate an important difference between \ipi{Dorste} and \ipi{Soest}: In the latter dialect, [ɣ] -- but not [ʝ] -- surfaces in word-internal onset position after back vowels, front vowels and liquids (recall \ref{ex:4:17}). By contrast, in \ipi{Dorste} the two fricatives [ɣ] and [ʝ] -- like [x] and [ç] -- stand in complementary distribution in postsonorant position. Thus, the set of targets for postsonorant fronting consists of /x/ but not /ɣ/ in \ipi{Soest}, but in \ipi{Dorste} it consists of /x/ and /ɣ/. The different targets are captured formally with two different fronting processes: \isi{Velar Fronting-4} for \ipi{Soest} and \isi{Velar Fronting-1} for \ipi{Dorste}.

The examples in \REF{ex:4:34} exhibit an alternation involving laryngeal features in (\ref{ex:4:34a}, \ref{ex:4:34b}) and both place and laryngeal features in (\ref{ex:4:34c}, \ref{ex:4:34d}); see \citet[34]{Dahlberg1937}.\largerpage[-2]

\ea\label{ex:4:34}%34
\TabPositions{.15\linewidth, .3\linewidth, .45\linewidth, .8\linewidth}
Place and laryngeal alternations (from /ɣ/):
\ea\label{ex:4:34a}  bɑrχ  \tab [bɑrç]  \tab  Berg  \tab ‘mountain’  \tab 65\\
     bɑrjə \tab [bɑrʝə] \tab  Berge \tab ‘mountain-\textsc{pl}’ \tab 16

\ex\label{ex:4:34b} ve̜χ  \tab [vɛç]  \tab Weg  \tab ‘path’  \tab 34\\
    ve̜jə \tab [vɛʝə] \tab Wege \tab ‘path-\textsc{pl}’ \tab 34

\ex\label{ex:4:34c} slɑx      \tab  [slɑx]      \tab Schlag     \tab  ‘blow’  \tab  34\\
    slë̜u̜ʓəs \tab    [slɛʊɣəs] \tab   Schlages \tab    ‘blow-\textsc{gen}.\textsc{sg}’ \tab 34\\
    slɛjə     \tab  [slɛːʝə]    \tab Schläge    \tab   ‘blow-\textsc{pl}’ \tab  34
    
\ex\label{ex:4:34d}  dǖjn̥   \tab  [dyːʝn̩]\tab   taugen \tab    ‘be good for sth-\textsc{inf}’\tab  32\\
     dǖjə    \tab [dyːʝə]  \tab tauge    \tab ‘id.-\textsc{1sg}’               \tab 32\\
     dö̜χst  \tab  [dœçst] \tab  taugst  \tab   ‘id.-\textsc{2sg}’             \tab 32\\
     do̜xtə  \tab  [dɔxtə] \tab  taugte  \tab   ‘id.-\textsc{pret}’            \tab 32
    \z
\z 

The fortis vs. lenis alternations in \REF{ex:4:34} are accounted for with \isi{Final Fortition} (\ref{ex:4:8a}) and the surface palatals with \isi{Velar Fronting-1}. As shown below, the outputs are transparent because the two rules in question are unordered (cf. \ipi{Altengamme} in \ref{ex:4:9}). The word [vɛç] ‘path’ (\ref{ex:4:34b}) is representative of words ending in a front vowel followed by /ɣ/. The word [ɛçt] ‘genuine’ (\ref{ex:4:32d}) illustrates the behavior of /x/ after a front vowel for comparison.

\ea%35
    \label{ex:4:35}
    \begin{multicols}{2}
\ea \begin{tabular}[t]{@{} lll @{}}  
         &  /vɛɣ/ &   /ɛxt/    \\
Fnl For  &  vɛx   &   -----    \\
Vel Fr-1 &  vɛç   &  ɛçt       \\   
         & [vɛç]  &  [ɛçt]     \\     
         & ‘path’ &   ‘genuine’\\
    \end{tabular}\label{ex:4:35a}
\ex \begin{tabular}[t]{@{} lll @{}}
           &  /vɛɣ/ & /ɛxt/\\
  Vel Fr-1 &  vɛʝ   & ɛçt  \\
   Fnl For &  vɛç   &  ----- \\
           & [vɛç]  & [ɛçt]\\
 \end{tabular}\label{ex:4:35b}
\z 
\end{multicols}
\z 

The relationship between \isi{Velar Fronting-1} and \isi{Final Fortition} in \REF{ex:4:35} can be compared with the ones in \REF{ex:4:20} for \ipi{Soest}, in which \isi{Final Fortition} \isi{feeds} \isi{Velar Fronting-4}.

As in all of the other German dialects discussed above, velars fronted to palatal in \ipi{Dorste} regardless of the etymological source of the segments serving as triggers. Thus, palatals surface in the neighborhood of front vowels that were etymologically front in (\ref{ex:4:36a}, \ref{ex:4:37a}) or back in (\ref{ex:4:36b} , \ref{ex:4:37b}) and velars in the neighborhood of back vowels that were etymologically back in (\ref{ex:4:36c}, \ref{ex:4:37c}) or front in (\ref{ex:4:36d}). \REF{ex:4:36b} and \REF{ex:4:37b} illustrate dialect-specific examples of \isi{Vowel Fronting} and \REF{ex:4:36d} of \isi{Vowel Retraction}; no parallel case was found for \isi{Vowel Retraction} in the postsonorant context. Note that the change from back monophthong to a diphthong in \REF{ex:4:36b} and \REF{ex:4:37b} is classified as \isi{Vowel Fronting} on the basis of the location of the front vowel component of that diphthong. Hence, [oː] > [ɛʊ] involves \isi{Vowel Fronting} because the [ɛ] component is adjacent to the dorsal fricative, but [o] > [ʊɪ] is likewise \isi{Vowel Fronting} because the front component [ɪ] is adjacent to the dorsal fricative. The changes involving \isi{Vowel Retraction} and \isi{Vowel Fronting} in \ipi{Dorste} are depicted in \REF{ex:4:1} and \REF{ex:4:2} respectively. The reconstructed examples in the second column below are my own.

\ea%36
   \TabPositions{.15\linewidth, .2\linewidth, .35\linewidth, .55\linewidth, .8\linewidth}
    \label{ex:4:36}
\ea\label{ex:4:36a} \relax [çɛlt] \tab < \tab  \textsuperscript{+}[xɛld] \tab ‘money’ \tab cf. OSax \textit{geld} \tab (from \ref{ex:4:30b})
\ex\label{ex:4:36b} \relax [çɛʊs] \tab < \tab  \textsuperscript{+}[xoːs] \tab ‘goose’ \tab cf. MLG \textit{gōs}   \tab (from \ref{ex:4:30b})
\ex\label{ex:4:36c} \relax [xɑst] \tab < \tab  \textsuperscript{+}[xɑst] \tab ‘guest’ \tab cf. OSax \textit{gast} \tab (from \ref{ex:4:30a})
\ex\label{ex:4:36d} \relax [xɑːl] \tab < \tab  \textsuperscript{+}[xeːl] \tab ‘horny’  \tab cf. OSax \textit{gēl}  \tab (from \ref{ex:4:30a})

\z 
\ex%37
\TabPositions{.15\linewidth, .2\linewidth, .35\linewidth, .55\linewidth, .8\linewidth}
    \label{ex:4:37}
\ea\label{ex:4:37a} \relax [fɛːʝn̩] \tab  <\tab   \textsuperscript{+}[fɛːɣn̩]\tab   ‘sweep-\textsc{inf}’ \tab   cf. OSax \textit{fegon}\tab  (from \ref{ex:4:33b})
\ex\label{ex:4:37b} \relax [drʊɪʝə] \tab < \tab \textsuperscript{+}[droɣə]   \tab ‘dry’                  \tab cf. MLG \textit{droge}   \tab(from \ref{ex:4:33b})
\ex\label{ex:4:37c} \relax [ɛʊɣə]   \tab < \tab \textsuperscript{+}[ɛʊɣə]    \tab ‘eye’                  \tab cf. OSax \textit{ōga}    \tab(from \ref{ex:4:33a})
\z 
\z 

\begin{sloppypar}
The broad set of targets for postsonorant fronting (/x ɣ/) and the full range of triggers for that change (coronal sonorants) were exemplified above for \ipi{Dorste}. The same pattern is reflected in other \il{Eastphalian}Eph varieties, e.g. \ipi{Magdeburger Börde} (\citealt{Roloff1902}; \mapref{map:7}), \ipi{Eilsdorf} (\citealt{Block1910}; \mapref{map:7}), \ipi{Emmerstedt} (\citealt{Brugge1944}; \mapref{map:7}), and \ipi{Göddeckenrode}\slash \ipi{Isingerode} (\citealt{Lange1963}; \mapref{map:7}). However, as noted in \sectref{sec:4.2}, there are also \il{Eastphalian}Eph-speaking communities like \ipi{Meinersen} and \ipi{Börßum} with the narrow set of targets for postsonorant velar fronting (/x/), as in \ipi{Soest}.
\end{sloppypar}

The synchronic fronting of word-initial /x/ to [ç] as described above for \ipi{Dorste} is not a general feature of \il{Eastphalian}Eph because \isi{Wd-Initial ɣ-Fortition} in (\ref{ex:4:25a}) did not affect that entire dialect region. Instead, \ili{WGmc} \textsuperscript{+}[ɣ] underwent either \is{g-Formation-1}g-For\-ma\-tion-1 in (\ref{ex:4:10b}) or a more specific change from /ɣ/ to /g/ in word-initial position only. The realization of /x/ as [x] or [ç] after a word-initial [s] as in \ipi{Soest} (=\ref{ex:4:13}) is attested neither in \ipi{Dorste}, nor in other varieties of \il{Eastphalian}Eph, where the realization is [ʃ], as in \il{Standard German}StG; see \citet{Hall2021} for extensive discussion.

Velar fricatives are in complementary distribution with the corresponding palatals in \ipi{Dorste}, but other varieties of \il{Eastphalian}Eph are attested in which velar vs. palatal contrasts occur in word-initial position. According to one pattern, \ili{WGmc} \textsuperscript{+}[ɣ] is realized as [ɣ] word-initially before back vowels. Since \isi{Glide Hardening} in (\ref{ex:4:10a}) ensured that word-initial \textsuperscript{+}[j] is realized as [ʝ] (/ʝ/) before any vowel, that type of dialect now has contrasts between velars ([ɣ] /ɣ/) and palatals ([ʝ] /ʝ/) in word-initial position before back vowels (e.g. \citealt{Block1910}; \mapref{map:7}). Examples like these are discussed in \chapref{sec:8}.\il{Eastphalian|)}

\section{Discussion}\label{sec:4.5}

The material investigated in this chapter and in \chapref{sec:3} reveals variation involving targets and triggers for velar fronting. In \sectref{sec:4.5.1} I summarize the synchronic facts discussed above and consider briefly how they are accommodated in the \isi{rule generalization} model (\sectref{sec:2.4.1}). In all of the case studies discussed up to this point palatal [ç] is the derived allophone of /x/ occurring in the context of a front segment, but the word-initial \isi{etymological palatal} [ʝ] (/ʝ/) occurs before front and back vowels. In \sectref{sec:4.5.2} I discuss the phonological motivation for the emergence of those underlying palatals.

\subsection{Interim assessment of targets, triggers, and rule generalization}\label{sec:4.5.1}

The various versions of velar fronting posited above do not have a consistent set of targets and/or triggers. For example, the target for postsonorant fronting consists solely of /x/ in \ipi{Soest} but of /x ɣ/ in \ipi{Dorste}. The triggers for the fronting of /x/ in word-initial position in \ipi{Soest} and \ipi{Dorste} is the set of front vowels, but in another \il{Westphalian}Wph variety alluded to earlier (\ipi{Elspe}; \citealt{Arens1908}) the fronting of word-initial /x/ is induced by all coronal sonorants. Postsonorant velar fronting in \ipi{Rheintal} (\sectref{sec:3.4}) occurs in the context of nonlow front vowels or coronal sonorant consonants, but the set of triggers for the same process in a number of other varieties discussed above consists of all coronal sonorants.

In this book I apply \isi{rule generalization} to velar fronting in German dialects. When that process was first phonologized the change was triggered by a highly restricted set of front segments, and the target segment was likewise restricted to a single velar; recall \figref{fig:2.4} and \figref{fig:2.5}. In terms of time, the set of triggers expanded to include more and more front sounds, while the set of targets analogously increased to include a greater number of velars. In terms of space, velar fronting spread outwards from the focal areas; when this occurred, the process had the narrow set of targets and triggers. Recall how \isi{rule generalization} was depicted abstractly in \figref{fig:2.2}.

A preliminary list of changes from specific to general targets/triggers is given in \REF{ex:4:38}. HFV=high front vowels, MFV=mid front vowels, LFV=low front vowels, and CC=coronal sonorant consonants

\ea%38
    \label{ex:4:38}
          Changes in targets (in a) and triggers (in b):
\ea \label{ex:4:38a} /x/  >  /x ɣ/  > /x ɣ k g ŋ/
\ex \label{ex:4:38b}\avm{\{HFV\}} >  \avm{\{ FHV\\MFV \}} > \avm{\{ HFV \\ MFV \\CC\}} > \avm{\{ HFV \\ MFV \\ LFV \\CC \}}
\z 
\z

Synchronic evidence for the historical stages depicted in \REF{ex:4:38} comes in the form of dialects described in late nineteenth century up to the present representing those stages. For example, postsonorant velar fronting in \ipi{Soest} exemplifies the /x/ stage and \ipi{Dorste} the /x ɣ/ stage. In terms of time, the original set of targets expanded from pre-\ipi{Dorste} (/x/ only) to \ipi{Dorste} (/x/ and /ɣ/). Dialects with the largest set of targets (/x ɣ k g ŋ/) are discussed in \chapref{sec:11}.

That /ɣ/ was not an original target follows from the \isi{Implicational Universal for Velar Fronting Targets-2} (\sectref{sec:2.3.2}): “If a lenis sound undergoes velar fronting then the corresponding fortis sound does as wellˮ. Given that exceptionless generalization, /ɣ/ cannot be the target segment unless /x/ is.

The set of triggers for velar fronting at the point when the process was first phonologized likewise consisted of a small number of sounds most conducive to fronting and then gradually expanded to include a larger number of sounds; see \REF{ex:4:38b}. The final stage in \REF{ex:4:38b} represents the default set of triggers (all coronal sonorants), which can be observed in several varieties discussed above. The penultimate stage in \REF{ex:4:38b} is represented by \ipi{Rheintal}, while the antepenultimate one (HFV, MFV) is perhaps reflected in word-initial velar fronting in \ipi{Soest} and \ipi{Dorste}.\footnote{{The conclusion is inconclusive because neither \ipi{Soest} nor \ipi{Dorste} possess low front vowels (e.g. /æ/); hence, one cannot know for certain whether or not low front vowels in either of those varieties belongs to the set of triggers.}}

The progression from high front vowels to high and mid front vowels to all front vowels is a consequence of the \isi{Implicational Universal for Palatalization Triggers} (\sectref{sec:2.3.3}): “If lower front vowels trigger Palatalization, then so will higher front vowelsˮ. No dialect is attested which fails to obey that hierarchy.

Variation in terms of space (regional dialects) directly reflects changes along the temporal dimension. In particular, dialects with a more restricted set of triggers/targets preserve an earlier historical stage than dialects with the full set of triggers/targets, which represent a later stage. Regions where velar fronting had the greatest set of targets/triggers (e.g. /x ɣ/ in \ipi{Dorste}) represent places where velar fronting has been active longer than those places where velar fronting exhibits a narrower set of targets/triggers (e.g. /x/ in \ipi{Soest}). The reason is that velar fronting has been present longer in the focal areas than in outlying areas and that the change has therefore had more time to expand the number of targets and triggers.

\subsection{Emergence of the underlying palatal /ʝ/ via Glide Hardening}\label{sec:4.5.2}\is{Glide Hardening|(}

In all of the studies discussed up to this point the palatal fricative [ç] has a highly restricted distribution in the sense that it only occurs when adjacent to coronal sonorants (or some subset thereof). The limited occurrence of [ç] is captured formally by treating that sound as an allophone of a velar produced by velar fronting. The same generalization holds for the postsonorant palatal [ʝ] in \ipi{Dorste}, which was shown to be a realization of the velar /ɣ/.

One of the challenges in this book is to account for the occurrence of opaque palatals like [ç] occurring in the neighborhood of back sounds. Reference to such dialects was made at various points in previous chapters. One generalization true for the dialects discussed below is that the back vowels adjacent to opaque palatals were etymologically front. Thus, the backing of those front vowels by \isi{Vowel Retraction} and/or \isi{r-Retraction} follows if velar fronting was active synchronically at the stage before either retraction process occurred. Thus, retraction resulted in a reassociation of the feature [coronal] from the front vowel trigger to the adjacent dorsal fricative, thereby creating either a \isi{palatal quasi-phoneme} or a \isi{phonemic palatal} (/ç/ or /ʝ/).

At issue is the word-initial lenis palatal fricative [ʝ] in all three of the LG varieties discussed in the present chapter. That sound is an underlying palatal (/ʝ/) because it occurs before front vowels and back vowels. Since the /ʝ/ in question was never the product of assimilatory fronting from an earlier velar it does not have an opaque history. The important point is that [ʝ] (/ʝ/) emerged in the back vowel context even in dialects which otherwise ban fortis palatals in that environment. Given this, what is the phonological reason for the emergence of that [ʝ] (/ʝ/), especially in the context of back vowels?

As noted above, the palatal fricative under discussion is the modern reflex of an earlier palatal glide (\ili{WGmc} \textsuperscript{+}[j]) by \isi{Glide Hardening} (=\ref{ex:4:10a}). The motivation for that change is syllable structure, since it only affected glides in onset position, while glides in the nucleus or coda were immune. It has long been known that languages impose sonority-based constraints on onset and coda segments. A version of the \isi{Sonority Hierarchy} is posited in \REF{ex:4:39}, which is similar to the one proposed by \citet{Clements1990} with the exception that rhotics like /r/ are analyzed in \REF{ex:4:39} as more sonorous than laterals. See \citet{Vennemann1982}, \citet{Strauss1982}, \citet{Wiese1988}, \citet{EisenbergRamersVater1992},  \citet{Wiese1996a}, \citet{Grijzenhout1998}, and \citet{Hall2002} for discussion of sonority in \il{Standard German}StG. \citet{Hall2011b} examines sonority in a \il{Highest Alemannic}HstAlmc variety spoken in \ipi{Visperterminen} (\sectref{sec:6.2}), and \citet{Noelliste2019} gives a language-specific sonority hierarchy for Bav. \citet{Parker2011} proposes a very fine-grained version of \REF{ex:4:39} on the basis of cross-linguistic evidence.

\ea%39
    \label{ex:4:39}Sonority hierarchy:\\
    \begin{tikzpicture}
    \matrix (matrix) [matrix of nodes]
      {
        Vowels         & Glides  &  Rhotics  &  Laterals  &  Nasals  &  Obstruents\\
        more sonorous  &         &           &            &          &less sonorous\\
      };
      \draw[{Triangle[]}-{Triangle[]}] (matrix.east) -- (matrix.west);
    \end{tikzpicture}
\z 

Languages tend to prefer less sonorous sounds in the onset and more sonorous sounds in the coda. Thus, there is general agreement in phonology that glides (as sonorants) make for poor onsets. This cross-linguistic generalization has been  captured formally in various ways, e.g., the \isi{Head Law} of \citet{Vennemann1988} in the \isi{Preference Law} framework, the \isi{Sonority Dispersion Principle} \citep{Clements1990}, or the various \isi{Margin Hierarchies} in \isi{Optimality Theory} (e.g. \citealt{PrinceSmolensky2004}, \citealt{Clements1997}, \citealt{Smith2003}, and \citealt{Hall2011b} to name a few). The reason why glides and not other sonorants (i.e. nasals or liquids) are singled out for hardening in \REF{ex:4:10a} can be found by considering the sonority of these sounds: According to most versions of the \isi{Sonority Hierarchy} (as in \ref{ex:4:39}), glides are more sonorous than nasals or liquids. For this reason, syllables like [ja] and [wa] are worse than ones like [na], [la] and [ra] because of the relatively shallow rise in sonority from glide to following vowel.

In sum, the preference for fricatives as opposed to glides in an onset was prioritized over the requirement that palatal fricatives be banned in the context of back vowels.

Since there is solid phonological motivation for \isi{Glide Hardening} it should not come as a surprise that that change -- or something very similar -- is independently attested both within and outside of Gmc; see the discussion in \citet{Hall2014a}. In early Gmc there was a sound change traditionally referred to as Verschärfung (literally ‘sharpening’) -- otherwise known as \isi{Holtzmann’s Law} (\citealt{Polomé1949}, \citealt{Kurylowicz1967}, \citealt{Suzuki1990}, \citealt{DavisIverson1996}, and \citealt{Page1999}) --, which was responsible for the shift of \ili{PGmc} singleton glides \textsuperscript{+}[j] and \textsuperscript{+}[w] after a short vowel and before a vowel to a geminate obstruent in \ili{East Germanic} (\ili{Go}) or \ili{NGmc} (\ili{ON}). Two examples discussed in the works cited earlier are \ili{PGmc} \textit{\textsuperscript{+}}\textit{twa-jē} > Go \textit{twaddjē}, ON \textit{tveggja}, OHG \textit{zweiio} ‘two-\textsc{gen}’ and PGmc \textit{\textsuperscript{+}}\textit{trewa-s} > Go \textit{triggws}, ON \textit{tryggr}, OHG \textit{triuwi} ‘true’. Following \citet{Page1999}, Verschärfung involved the following two stages: \textsuperscript{+}[VG\textsubscript{a}V] > \textsuperscript{+}[VG\textsubscript{a}G\textsubscript{a}V] > [VO\textsubscript{a}O\textsubscript{a}V]. Stage 1 converted a singleton glide (G\textsubscript{a}) into a geminate glide (G\textsubscript{a}G\textsubscript{a}) after a short vowel and before a vowel, while the Stage 2 changed that geminate glide into a geminate obstruent (O\textsubscript{a}O\textsubscript{a}). The relevant part is Stage 2, which involved the exceptionless shift of a geminate glide to a geminate obstruent. That change was similar to my process of \isi{Glide Hardening} as stated in \REF{ex:4:10a}, but it differed from the latter change because Stage 2 of Verschärfung could not have been motivated as an avoidance of glides in the onset. The reason is that Go [j] did not harden to an obstruent in word-internal onset position. For example, Go \textit{bidjan} ‘ask-\textsc{inf}’ was syllabified [bid.jɑn], and yet, there is no evidence that the [j] hardened to an obstruent.

These differences aside, it is undeniably the case that \isi{Glide Hardening} independent of Verschärfung has been attested throughout the history of Gmc. For example, \citet[174, 183]{Seebold1982} discusses the change from \ili{Indo-European} (IE) \textsuperscript{+}[w] to \ili{PGmc}/\ili{WGmc} \textsuperscript{+}[k] or \textsuperscript{+}[g], e.g. \il{Indo-European}IE \textsuperscript{+}daiwēr ‘brother-in-law’ > PGmc \textsuperscript{+}taikur (cf. OHG \textit{zeihhur}) and IE \textsuperscript{+}juwnti ‘youth’ > \ili{WGmc} \textsuperscript{+}jugunþi (cf. OHG \textit{jugund}). This change was similar to \isi{Glide Hardening} in \REF{ex:4:10a} because the target segment was a glide and the output was an obstruent.

Examples of changes in later stages of German show directly or indirectly that \isi{Glide Hardening} was involved. For example, in \ili{MHG}, the glide [w] regularly shifted to [b] after liquids in \ili{ENHG}, e.g. MHG [narwə] > \ili{ENHG} [narbə] ‘scar’, MHG [gelwər] > \ili{ENHG} [gelbər] ‘yellow-\textsc{infl}’; \citet[368]{Schirmunski1962}.

A number of non-Gmc languages are attested with \isi{Glide Hardening}. For example, \citet{HarrisKaisse1999} investigate alternations between the palatal glide [j] and the lenis nonanterior coronal fricative [ʒ] -- ⟦ž⟧ in the original source -- in \ili{Argentinian Spanish}, e.g. lé[j] ‘law’ vs. lé.[ž]-es ‘laws’. Harris \& Kaisse account for data like these with a rule they dub Coronalization (p. 146), which converts the glide /j/ into the lenis nonanterior coronal fricative in syllable-initial position. Like \isi{Glide Hardening} in \REF{ex:4:10a}, \is{Coronalization (Spanish)}Coronalization creates a fricative from a glide in onset position. \citet{BaltazaniTopintzi2016} discuss Glide Hardening in \ili{Greek} (and in other languages) at length. They demonstrate that \ili{Greek} has a rule of Glide Hardening which targets glides in onset position in CjV sequences, turning them into consonants.\footnote{Other instances of \isi{Glide Hardening} probably involve the assimilation (spreading) of a major class feature ([+consonantal]); see \citet{Kaisse1992}. For example, \citet{Kamprath1986} discusses \isi{Glide Hardening} in \ili{Bergüner Romansh} (Switzerland), which converts the glides /j/ and /w/ into into a velar stop in the context before another consonant. The near mirror-image process is attested in \ili{Cypriot Greek} \citep{Newton1972b}.}\is{Glide Hardening|)}

\section{{Conclusion}}\label{sec:4.6}

In this chapter I examined the transparent (allophonic) distribution of velars and palatals in three varieties of WLG, which were defined according to the target segments for postsonorant velar fronting.

The occurrence of palatals and velars holds regardless of the historical source of the triggers and targets for velar fronting. Thus, velars like [x] occur not only in the context of back segments that were historically back but also when adjacent to back sounds that were historically front. Palatals like [ç] likewise surface in the context of front sounds that were etymologically front as well and in the neighborhood of front sounds that were etymologically back. The sounds undergoing velar fronting include not only underlying (etymological) velars but also new velars created from non-velars by independent changes.

The transparent distribution of velars and palatals in the present chapter -- and in \chapref{sec:3} -- can be contrasted with the opaque distribution of those sounds discussed in the following five chapters. In Chapters~\ref{sec:5}–\ref{sec:6} I consider velar fronting dialects in which some instances of a velar ([x]) occur in the context of front vowels, indicating that velar fronting underapplies. In Chapters~\ref{sec:7}–\ref{sec:9} I consider velar fronting dialects with some instances of a palatal (e.g. [ç]) occurring in the context of a back sound that was etymologically front. In that type of dialect, the historical process of velar fronting overapplies.
