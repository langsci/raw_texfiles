\chapter{The status of [x] and [ç] in loanwords}\label{appendix:g}

Dorsal fricatives in nonnative words occur either word-initially or after a sonorant. The purpose of this appendix is to introduce some of the data and to provide brief remarks on the difficulties they pose for a potential analysis.

\section{Word-initial position}
There are no native words of \il{Standard German}StG beginning with [x] or [ç]; the historical reasons for that gap are discussed in Appendix~\ref{appendix:f}. Word-initial [x] or [ç] discussed in the literature referred to in \sectref{sec:1.1} therefore all involve loanwords like the ones in \REF{ex:appendix:g:1}. Representative examples of words with [ç] are listed in \REF{ex:appendix:g:1a} and ones with [x] in \REF{ex:appendix:g:1b}. The pronunciation in the first column is the one found in \citet{Mangold2005}.

\TabPositions{.175\textwidth, .35\textwidth, .55\textwidth, .75\textwidth}
\ea \label{ex:appendix:g:1}
\ea  \label{ex:appendix:g:1a}
\relax  [çemiː]   \tab  Chemie   \tab  ‘chemistry’\\
    \relax  [çiːnɑ]   \tab  China    \tab ‘China’\\
    \relax  [çɑrɪsmɑ] \tab Charisma  \tab   ‘charisma’\\
    \relax  [çolɛsteriːn] \tab Cholesterin  \tab  ‘cholesterol’\\
\ex  \label{ex:appendix:g:1b}
\relax [xɔtɛk] \tab Chotek  \tab ‘Chotek’\\
    \relax [xɛp]   \tab Cheb    \tab ‘Cheb’  \\
    \relax [xʊntɑ] \tab Junta   \tab ‘junta’ \\
\z
\z

According to one school of thought, words like the ones in \REF{ex:appendix:g:1a} are integrated (assimilated) loanwords, while the ones in \REF{ex:appendix:g:1b} are non-integrated (unassimilated). That approach therefore sees the palatal [ç] as the only acceptable pronunciation in word-initial position, while initial [x] can be ignored because it lies on the extreme periphery of the German lexicon. Some of the authors who accept a variant of that view include \citet[956]{Wurzel1980}, \citet[3]{Hall1989}, \citet[210]{Wiese1996a}, and \citet[232, Footnote 3]{Noske1997}, although other names could be added to that list as well.

The theoretical literature cited above almost invariably treats loanwords like the ones in \REF{ex:appendix:g:1a} on par with native words. In \sectref{sec:1.2} I describe briefly one such approach to \il{Standard German}StG dorsal fricatives, according to which the data in \REF{ex:appendix:g:1a} are crucial in determining whether or not the underlying dorsal fricative in postsonorant position in fully native words is /x/ or /ç/. The argument is that by including the data in \REF{ex:appendix:g:1a}, [ç] occurs in a wider set of contexts than [x], since the former occurs after front vowels, after sonorant consonants, or word-initially, while the latter surfaces only after back vowels. The implication is that surface [x] should be derived from the segment with the wider distribution, namely /ç/. Given that approach, velar fronting regularly creates [x] from /ç/ after a back vowel, and in word-initial position /ç/ surfaces without change as [ç]. On this approach the surface [x] in \REF{ex:appendix:g:1b} is ignored because it is present in unassimilated words.

\citet{Robinson2001} criticizes the approach described above -- correctly in my view -- on the grounds that the decision to classify a \isi{loanword} as integrated or non-integrated is arbitrary. He writes (p. 58): “…it cannot honestly be said that any of the analyses I have looked at [regarding data like the ones in (\ref{ex:appendix:g:1}), T.A.H.] give any independent criteria for what constitutes a fully integrated \isi{loanword} in German (that is, one which in the relevant respects adheres to German phonological patterns)”.

The nature of the word-initial fricative in \REF{ex:appendix:g:1} can vary depending on the dialect/speaker. For example, many speakers substitute the [ç] in \REF{ex:appendix:g:1a} with either [ʃ] or [k]. \citet[222]{Noske1997} gives the examples in \REF{ex:appendix:g:2}, which can be taken to be representative for some speakers. It needs to be stressed that speakers with the [ʃ] or [k] pronunciation in \REF{ex:appendix:g:2} will have [x] and [ç] as predictable positional variants in postsonorant position; hence, the examples with [ʃ] or [k] in \REF{ex:appendix:g:2} cannot be interpreted as an across-the-board avoidance of dorsal fricatives.

\ea \label{ex:appendix:g:2}
\begin{tabular}[t]{@{}lll@{}}
\relax [çiːʀʊɐk], [kiːʀʊɐk], [ʃiːʀʊɐk] & Chirurg & ‘surgeon’\\
\relax [çemiː], [kemiː], [ʃemiː] & Chemie &  ‘chemistry’\\
\relax [çiːnɑ], [kiːnɑ], [ʃiːnɑ] & China  & ‘China’\\
\relax [çɑʀɪsmɑ], [kɑʀɪsmɑ], [ʃɑʀɪsmɑ] & Charisma  & ‘charisma’\\
\relax [çoːlɛsteʀiːn], [koːlɛsteʀiːn], [ʃoːlɛsteʀiːn] & Cholesterin & ‘cholesterol’\\
\end{tabular}
\z

The three pronunciations in \REF{ex:appendix:g:2} are sometimes interpreted as belonging to different dialects. For example, according to \citet[254]{Pilch1966}, the pronunciation with [ç] is preferred for northeast German speakers (“Nordostdeutscheˮ), while speakers in the northwest prefer [ʃ] and speakers in the south [k]. Recall from \sectref{sec:17.2} that [k] is typical for \ili{StAG}.

Many speakers have yet another realization of word-initial dorsal fricatives like the ones in \REF{ex:appendix:g:1}. Consider first the variety of German spoken in the city of \ipi{Kiel} (\mapref{map:5}) described by \citet{Glover2014}. As indicated in \REF{ex:appendix:g:3}, Glover’s speakers have a very different pattern than the one in \il{Standard German}StG \REF{ex:appendix:g:1}. In particular, \ipi{Kiel} has no word-initial [x]; hence, the \il{Standard German}StG examples in \REF{ex:appendix:g:1b} are realized with the stop [k] or the glide [j]; see (\ref{ex:appendix:g:3b}, \ref{ex:appendix:g:3c}). It needs to be stressed that the pronunciation in \REF{ex:appendix:g:3c} holds for speakers with no knowledge of \ili{Spanish}. Significantly, the only word-initial dorsal fricative acceptable to Glover’s speakers is [ç], but only before a front vowel; see \REF{ex:appendix:g:3a}.

\ea%3
    \label{ex:appendix:g:3}
\ea \label{ex:appendix:g:3a}
\relax  [çemiː]       \tab Chemie    \tab ‘chemistry’  \\
   \relax  [çiʀʊɐk]      \tab Chirurg   \tab  ‘surgeon’   \\
\ex \label{ex:appendix:g:3b}
\relax  [kɑːʀɪsmɑ]    \tab  Charisma \tab    ‘charisma’\\
   \relax  [kolɛsteʀiːn] \tab   Cholesterin  \tab  ‘cholesterol’\\
\ex \label{ex:appendix:g:3c}
\relax  [jʊntɑ]       \tab   Junta        \tab  ‘junta’
\z
\z

From a formal point of view, the word-initial [ç] can be analyzed as a word-initial allophone of /x/; the [k] in \REF{ex:appendix:g:3b} derives synchronically from /k/ and the glide in [j] from the corresponding vowel (/i/), although the treatment of glides is peripheral to the analysis of dorsal fricatives.\footnote{{Impressionistically, I can confirm the data in (\ref{ex:appendix:g:3a}, \ref{ex:appendix:g:3b}) on the basis of numerous discussions with native speakers through the years. I recall many speakers who express extreme aversion to pronouncing [ç] in word-initial position before a back vowel (e.g. in the final two words in \ref{ex:appendix:g:1a}). Those speakers invariably pronounce those words with [k]. My view on the initial sound in (\ref{ex:appendix:g:3b}) is shared by \citet[32]{Rapp1841}, who opines that a [ç] in word-initial position before a back vowel -- his examples are \textit{Chaos, Character, Cholera} -- would sound “abominable" (“abscheulich").} }

According to \citet{HoveHaas2009}, the distribution of postsonorant [x] and [ç] in \ili{StSwG} is as in \il{Standard German}StG: [x] after a back vowel and [ç] after coronal sonorants. In word-initial position, [ç] occurs before a front vowel in (\ref{ex:appendix:g:4a}), but before a back vowel in (\ref{ex:appendix:g:4b}) or consonant in (\ref{ex:appendix:g:4c}), either [x] or [k] occurs. Thus, word-initial /x/ in \ili{StSwG} shows a fronting to palatal [ç] in \REF{ex:appendix:g:4a} by a version of velar fronting. (The variant pronunciation with [k] derives synchronically from /k/).

\ea%4
\TabPositions{.33\textwidth, .5\textwidth}\label{ex:appendix:g:4}
\ea\label{ex:appendix:g:4a}
    \relax [çemiː]    \tab Chemie    \tab ‘chemistry’\\
    \relax [çiʁʊʁgiː] \tab Chirurgie \tab  ‘surgery’
\ex\label{ex:appendix:g:4b}
    \relax [xaːɔs], [kaːɔs] \tab Chaos  \tab  ‘chaos’\\
    \relax [xaʁaktəʁ], [xaʁaktəʁ] \tab Charakter \tab ‘character’
\ex\label{ex:appendix:g:4c}
    \relax [xʁoːm], [kʁoːm] \tab Chrom  \tab  ‘cholesterol’
\z
\z

A similar generalization concerning word-initial dorsal fricatives holds for the data discussed in \citet{Jessen1988}, although he accepts both [x] and [ç] in word-initial position in his speech. Jessen argues that the two sounds stand in an allophonic relationship in word-initial position, where the choice between the two is determined by the following vowel: [ç] before a front vowel, as in (\ref{ex:appendix:g:2}a) and [x] before a back vowel, as in (\ref{ex:appendix:g:2}b). The rule he posits relating [x] and [ç] is bidirectional and therefore applies postvocalically in words like \textit{mich} [mɪç] ‘me-\textsc{acc}’ and \textit{Krach} [kʀɑx] ‘noise’ and progressively in word-initial position, as in \REF{ex:appendix:g:2}. Word-initial [ç] before a back vowel in words like \textit{Charon} [çɑːʀɔn] ‘\ili{Greek} mythological figure’ and \textit{Chauke} [çɑukə] ‘Germanic tribe’ (cf. Latin \textit{Chauci}) are treated as exceptions \citep[391]{Jessen1988}.

Although there is disagreement in the literature concerning the status of words like the ones in (\ref{ex:appendix:g:1b}) vs. (\ref{ex:appendix:g:1b}), there is a general consensus that the examples cited in the pronouncing dictionaries in which a dorsal fricative appears in word-initial position before a consonant are truly unacceptable. This generalization is true for both [ç], as in (\ref{ex:appendix:g:5}a) and [x], as in (\ref{ex:appendix:g:5}b). The examples in \REF{ex:appendix:g:5} were drawn from \citet{Mangold2005}. However, recall from \REF{ex:appendix:g:4c} that some speakers of \ili{StSwG} have [x] in that context.

\ea \label{ex:appendix:g:5}
\ea  chtonisch \tab [çtoːnɪʃ] \tab ‘underground’\\
     chrysander \tab [çʀyzandɐ] \tab ‘(name)’
\ex  Chmel \tab [xmɛl] \tab ‘(name)’\\
     Chrobak \tab [xʀoːbak] \tab ‘(name)’
\z
\z

See \citet[60]{Robinson2001}, who remarks in a footnote that he omits from his discussion the pronunciations of word-initial \textit{ch} before a consonant because they have typically not played a role in the analysis of word-initial [x] and [ç].

The observation made in the works cited above is that the status of word-initial dorsal fricatives in loanwords depends to a large extent on geography. This is precisely the conclusion drawn by AADG and WDU, which provide maps illustrating the pronunciation of word-initial \textit{ch} in several of the words listed above. For example, according to AADG, the initial sound in the word \textit{Charisma} is realized as [kʰ] throughout almost all of Germany and Austria and as [x] throughout most of Switzerland. Of the six hundred sixty-nine  speakers involved in the survey, only two had the [ç] realization prescribed in the pronouncing dictionaries. WDU Map 112 in Volume 2 likewise depicts the areal distribution of the initial sound in the word \textit{China}.

\section{Postsonorant position}
Four representative examples of loanwords containing postsonorant dorsal fricatives are presented in \REF{ex:appendix:g:6}. The pronunciation indicated here is the one for \il{Standard German}StG \citep{Mangold2005}. These examples show the same pattern described earlier for dorsal fricatives in native words: [x] surfaces after a back vowel in (\ref{ex:appendix:g:6a}) and [ç] after a front vowel in (\ref{ex:appendix:g:6}b) or sonorant consonant in (\ref{ex:appendix:g:6}c). Since I make some reference below to \isi{stress} I include the diacritic in \REF{ex:appendix:g:6} and below.

\TabPositions{.25\textwidth, .5\textwidth}
\ea%6
\label{ex:appendix:g:6}
\ea  \label{ex:appendix:g:6a}
\relax [mɑzoˈxɪsmʊs] \tab   Masochismus \tab ‘masochism’\\
\ex \relax [ˈɛço]  \tab   Echo \tab ‘echo’\\
\ex \relax [kolˈçoːzə] \tab Kolchose \tab ‘kolkhoz’\\
    \relax [tutɑnˈçɑːmon] \tab Tutanchamon \tab ‘Tutanchamon’\\
\z
\z

In a very small number of works discussed below the observation has been made that some speakers have an alternate pronunciation for the item listed in (\ref{ex:appendix:g:6a}). That example and a few other words are presented in \REF{ex:appendix:g:7}. Note that palatal [ç] occurs in some items after a back vowel.

\ea%7
\label{ex:appendix:g:7}
\ea\label{ex:appendix:g:7a}
\relax [mɑzoːˈçɪsmʊs]  \tab  Masochismus \tab ‘masochism’\\
    \relax [ˈmɑzoːx]       \tab  Masoch      \tab ‘Masoch’
\ex\label{ex:appendix:g:7b}
\relax [ɔynuːˈçɪsmʊs]  \tab  Eunuchismus \tab ‘eunuchism’\\
    \relax [ɔyˈnuːx]       \tab  Eunuch      \tab ‘eunuch’
\ex\label{ex:appendix:g:7c}
\relax [hypoːˈçɔndɐ]   \tab  Hypochonder \tab ‘hypochondriac’
\z
\z

The data in \REF{ex:appendix:g:7} are drawn from the first publication to my knowledge in which the alternate pronunciation for words like the one in \REF{ex:appendix:g:6a} is discussed, namely \citet[308]{Kenstowicz1994}. That author attributes the examples in \REF{ex:appendix:g:7} to an unpublished manuscript \citep{Moltmann1990}. Kenstowicz has an exercise involving the distribution of German [x] and [ç] which includes not only some of the familiar examples involving [x] and [ç] in native words but also the words in \REF{ex:appendix:g:7}. Note that the items in \REF{ex:appendix:g:7a} and \REF{ex:appendix:g:7b} show an alternation between [x] and [ç].\footnote{{Kenstowicz has incomplete transcriptions which only include the vowel plus dorsal fricative sequence (i.e. “[oːx]ˮ for the first example in \ref{ex:appendix:g:7a} and “[uːç]ˮ for the first example in \ref{ex:appendix:g:7b}). No transcription is provided for the item in \REF{ex:appendix:g:7c}, other than [ç].}} A more recent treatment of examples like the ones in \REF{ex:appendix:g:7} is \citet{Taylor2010}. 

One of the reasons why the alternate pronunciation (e.g. [mɑzoˈçɪsmʊs] in \ref{ex:appendix:g:7a} vs. [mɑzoˈxɪsmʊs] in \ref{ex:appendix:g:6a}) is difficult to assess is that it is not clear what the data are one is supposed to be analyzing. The problem is that neither Kenstowicz nor the final source I discuss below provides a complete set of data. Some of the factors any analysis needs to consider are: (a) \isi{stress}, (b) the nature of the vowel following the dorsal fricative, (c) the nature of the vowel preceding the dorsal fricative, and (d) syllabification.

On the basis of the words in \REF{ex:appendix:g:7}, one might hypothesize that the dorsal fricative is realized as [ç] before a stressed syllable. Since feet in German are trochaic \citep{Féry1998}, one could argue that speakers with the pronunciation in \REF{ex:appendix:g:7} have a rule deriving [ç] from /x/ in foot-initial position. The prediction would therefore be that /x/ surfaces as [x] after a back vowel if the fricative is not foot-initial, as in (\ref{ex:appendix:g:6}b). The problem is that Kenstowicz does not include that type of example in his exercise; hence, one cannot know if the analysis is correct.

A second published treatment of the [mɑzoːˈçɪsmʊs]-type data in \REF{ex:appendix:g:7} is \citet[711]{Merchant1996}. He lists -- in addition to the familiar examples involving [x] and [ç] in native words -- the six words in \REF{ex:appendix:g:8}. The phonetic transcriptions are the ones given in that source; I include the diacritic for \isi{stress} for reference. Merchant includes neither the item in \REF{ex:appendix:g:7c} nor the ones in (\ref{ex:appendix:g:6}b, \ref{ex:appendix:g:6}c).

\ea \label{ex:appendix:g:8}
\ea \label{ex:appendix:g:8a}
\relax [mɑzoːˈ.çɪst] \tab Masochist \tab ‘masochist’\\
    \relax [ˈmɑzoːx] \tab Masoch \tab ‘Masoch’
\ex \label{ex:appendix:g:8b}
    \relax [oɪnuːˈçɪsmus] \tab Eunuchismus \tab ‘eunuchism’\\
    \relax [ɔyˈnuːx] \tab Eunuch \tab ‘eunuch’\\
    \relax [ɔynuːçɪˈziːrən] \tab eunuchisieren \tab ‘make-\textsc{pl} into a eunuch’
\ex  \label{ex:appendix:g:8c}
\relax [paroːˈçiː] \tab Parochie \tab ‘parish’
\z
\z

The third item in \REF{ex:appendix:g:8b} is the only one that speaks against the foot-based analysis referred to above. Merchant argues that the dorsal fricative is realized as [ç] in syllable-initial position. Thus, a word like the first one in \REF{ex:appendix:g:8a} is parsed [mɑ.zoː.çɪst]. By contrast, the realization of the dorsal fricative is [x] after a back vowel and before a vowel if that dorsal fricative is ambisyllabic, e.g. the [x] in a (native) word like \textit{rauchen} [raʊxən] ‘smoke-\textsc{inf}’.

A drawback with the analysis of Merchant is that it relies on analyzing certain intervocalic consonants as ambisyllabic (e.g. the [x] in [raʊxən] ‘smoke-\textsc{inf}’) for which there is no independent evidence at all. To be clear: It has been proposed in the literature on \il{Standard German}StG that certain intervocalic consonants are ambisyllabic, but those studies agree that ambisyllabic consonants are  situated between a short vowel and another vowel \citep{Wiese1996a}. The analysis of the [x] in a word like [raʊxən] ‘smoke-\textsc{inf}’ as ambisyllabic therefore derives no independent support. The reader is also referred to studies arguing against ambisyllabic consonants in German \citep{Jensen2000}.




