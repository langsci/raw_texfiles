\chapter{Family tree for Germanic languages}\label{appendix:e}

A number of proposals have been made for the classification of Germanic languages; see \citet{Robinson1992} for some useful discussion and references. There is widespread agreement that the original language (Proto-Germanic) had three branches: West Germanic, North Germanic, and East Germanic. Those three groupings are depicted in the family tree below. A number of scholars have proposed that West and North Germanic derived from an earlier \ili{Northwest Germanic} group. The reader is referred to \citet[22ff.]{Fulk2018} for an assessment of the arguments for the Northwest Germanic grouping and general discussion (including many useful references) of the Germanic language family tree.

The most significant branches for present purposes are the ones culminating in High German (HG) and Low German (LG). The dates for the HG branch given below are in accordance with the ones usually assumed in the scholarly literature; see, for example, \citet[9--10]{Paul2007}. The distinction among the early stages of the LG branch is not as clear cut as it is for HG. I adopt henceforth the stages and dates in \citet{Foerste1957}. A useful summary of the dates for the HG and LG branches can be found in \citet[16--22]{Schmidt2007}. 

\begin{figure}
\caption{Germanic languages}
\small
\begin{forest} forked edges, for tree={grow'=east}
[{{Proto-\\Germanic}}
    [{East Germanic}
        [{Gothic}]
    ]
    [{{West\\Germanic}}
        [{{Old English}}
            [{{Middle English}}
                [{English}, tier=contemporary]
            ]
        ]
        [{{Old\\Frisian}}
            [{North Frisian}, tier=contemporary]
            [{West Frisian}, tier=contemporary]
            [{Saterland Frisian}, tier=contemporary]
        ]
        [{{Old\\Saxon}}
            [{{Middle\\ Low German}}
                [{Low\\German}, tier=contemporary]
            ]
        ]
        [{{Old Low Franconian}}
            [{Dutch}, tier=contemporary, grow=280
                [{Afrikaans}]
            ]
        ]
        [{{Old\\ High\\German}}
            [{Middle\\High\\ German}
                [{{Early\\New\\High\\German}}
                    [{High\\German}, tier=contemporary]
                ]
                [{Yiddish}, tier=contemporary]
            ]
        ]
    ]
    [{{North Germanic}}
        [{{Old Norse}}]
    ]
]
\end{forest}\\
\fittable{
\begin{tabular}{*{2}{l@{ }c}}
\lsptoprule
High German:                  &                       & &  Low German:\\\midrule
{Old High German} (OHG):        &  750--1050            & {Old Saxon} (OSax):       & 800--1150    \\
{Middle High German} (MHG):     &  1050--1350           & Middle Low German (MLG) & 1150--1600   \\
Early New High German (ENHG): & 1350--1650            & Low German:             & 1600--present\\
High German:                  & 1650--present                                                  \\
\lspbottomrule
\end{tabular}
}
\il{East Germanic}
\il{Gothic}
\il{Old English}
\il{Middle English}
\il{English}
\il{Old Frisian}
\il{North Frisian}
\il{West Frisian}
\il{Saterland Frisian}
\il{Old Saxon}
\il{Middle Low German}
\il{Old Low Franconian}
\il{Dutch}
\il{Afrikaans}
\il{Old High German}
\il{Middle High German}
\il{Early New High German}
\il{Yiddish}
\end{figure}
