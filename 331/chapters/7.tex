\chapter{Quasi-phonemicization of palatals}\label{sec:7}
\is{palatal quasi-phoneme|(}
\section{Introduction}\label{sec:7.1}

In many German dialects palatal sounds (e.g. [ç]) occur in the context of front vowels and certain back sounds ([Bk]) and velars (e.g. [x]) in the context of all back sounds with the exception of [Bk]. Palatal ([ç]) and velar ([x]) do not contrast because they stand in complementary distribution. All instances of palatals ([ç]) in the context of front vowels derive -- both synchronically and diachronically -- from the corresponding velar, but opaque palatals in the context of [Bk] are quasi-phonemes (/ç/). Significantly, palatal quasi-phonemes were once palatal allophones deriving from velars in the neighborhood of a front vowel (e.g. [ç] from /x/). When that original front vowel was eliminated, the palatal allophone was quasi-phonemicized to /ç/. This chapter investigates German dialects with palatal quasi-phonemes.

The way in which quasi-phonemes (opaque palatals) arise historically is illustrated in \REF{ex:7:1}: Stage 1 (far left) depicts a system without velar fronting, and Stage 2 (middle) represents a system in which velar fronting is phonologized as a rule creating a palatal allophone ([\textsc{Pa}]). Stage 3 (far right) is one in which a quasi-phoneme is present (/\textsc{Pa}/). In \chapref{sec:16} I discuss the time frame for the developments depicted in \REF{ex:7:1} and show how those changes fit into the early stages of German (Appendix~\ref{appendix:e}).

\ea%1
    \label{ex:7:1}
    \begin{forest}
    [/\textsc{Ve}/ [{[\textsc{Ve}]}]]
    \end{forest}
    >
    \begin{forest}
    [/\textsc{Ve}/,calign=first [{[\textsc{Ve}]}] [{[\textsc{Pa}]}] ]
    \end{forest}
    >       
    \begin{forest}
    [,phantom
        [/\textsc{Ve}/ [{[\textsc{Ve}]}] ]   
        [/\textsc{Pa}/ [{[\textsc{Pa}]}] ]
    ]
    \end{forest}  
\z 

This chapter focuses on two types of palatal [\textsc{Pa}] at Stage 3, although that distinction is not expressed in \REF{ex:7:1}: (a) The synchronically \isi{derived palatal} [\textsc{Pa}], which is the surface manifestation of underlying /\textsc{Ve}/, and (b) the underlying palatal quasi-phoneme [\textsc{Pa}] (/\textsc{Pa}/), which by definition cannot be synchronically derived from a velar. \isi{Derived palatals} are situated in the context for velar fronting (e.g. after front vowels), while velars like [\textsc{Ve}] surface in the elsewhere case. Palatal quasi-phonemes like [\textsc{Pa}] (/\textsc{Pa}/) are found neither in the front vowel context, nor in the elsewhere context for velars. Thus, velars and palatals in the dialects described below do not contrast.

As indicated below, the palatal allophone at Stage 2 -- depicted in (2) and (3) with the symbol [ç] -- is quasi-phonemicized (/ç/) at Stage 3 when one of the triggers for velar fronting (e.g. /i/) is eliminated. Those opaque palatals therefore exemplify a historical \isi{overapplication} of velar fronting. \REF{ex:7:2} illustrates quasi-phonemicization in the context after a sonorant (a front coronal) and \REF{ex:7:3} word-initially (before a front coronal). In \REF{ex:7:2} and \REF{ex:7:3} I indicate both the underlying representation and the phonetic representation.

\ea\label{ex:7:2}
  \ea \label{ex:7:2a} \begin{tikzpicture}[baseline={(matrix-1-1.base)}]
    \matrix (matrix) [matrix of nodes, nodes in empty cells]
      { /i & x/ &[5mm] &[5mm] /ɑ & ç/\\[5mm]
        {[\textsc{coronal}]} & {[\textsc{dorsal}]} & & {[\textsc{dorsal}]} & {[\textsc{coronal}]} & {[\textsc{dorsal}]}\\
                             &                     & > &                   & \\
        \relax {[i} & ç] & & {[ɑ} & ç]\\[5mm]
        \relax {[\textsc{coronal}]} & {[\textsc{dorsal}]} & & {[\textsc{dorsal}]} & {[\textsc{coronal}]} & {[\textsc{dorsal}]}\\
      };
      \foreach \i in {1,2,4,5} \draw (matrix-1-\i) -- (matrix-2-\i);
      \foreach \i in {1,2,4,5} \draw (matrix-4-\i) -- (matrix-5-\i);
      \draw (matrix-1-5.south) -- (matrix-2-6.north);
      \draw (matrix-4-5.south) -- (matrix-5-6.north);
      \draw (matrix-4-2.south) -- (matrix-5-1.north);
  \end{tikzpicture}
  \ex\label{ex:7:2b}
  \begin{tikzpicture}[baseline={(matrix-1-1.base)}]
    \matrix (matrix) [matrix of nodes, nodes in empty cells]
      { /i & x/ &[5mm] &[5mm] / & ç/\\[5mm]
        \relax{[\textsc{coronal}]} & {[\textsc{dorsal}]} & & & {[\textsc{coronal}]} & {[\textsc{dorsal}]}\\
                             &                     & > &                   & \\
        \relax {[i} & ç] & & {[} & ç]\\[5mm]
        \relax {[\textsc{coronal}]} & {[\textsc{dorsal}]} & & & {[\textsc{coronal}]} & {[\textsc{dorsal}]}\\
      };
      \foreach \i in {1,2,5} \draw (matrix-1-\i) -- (matrix-2-\i);
      \foreach \i in {1,2,5} \draw (matrix-4-\i) -- (matrix-5-\i);
      \draw (matrix-1-5.south) -- (matrix-2-6.north);
      \draw (matrix-4-5.south) -- (matrix-5-6.north);
      \draw (matrix-4-2.south) -- (matrix-5-1.north);
%       \node [left=1mm of matrix-1-1.center, overlay] {\textsubscript{wd}[};
%       \node [left=1mm of matrix-1-4.center, overlay] {\textsubscript{wd}[};
  \end{tikzpicture} 
  \z
\ex\label{ex:7:3}
  \ea \label{ex:7:3a} \begin{tikzpicture}[baseline={(matrix-1-1.base)}]
    \matrix (matrix) [matrix of nodes, nodes in empty cells]
      { /x & i/ &[5mm] &[5mm] /ç & & ɑ/\\[5mm]
        {[\textsc{dorsal}]} & {[\textsc{coronal}]} & & {[\textsc{dorsal}]} & {[\textsc{coronal}]} & {[\textsc{dorsal}]}\\
                             &                     & > &                   & \\
        \relax {[ç} & i] & & {[ç} & & ɑ]\\[5mm]
        \relax {[\textsc{dorsal}]} & {[\textsc{coronal}]} & & {[\textsc{dorsal}]} & {[\textsc{coronal}]} & {[\textsc{dorsal}]}\\
      };
      \foreach \i in {1,2,4,6} \draw (matrix-1-\i) -- (matrix-2-\i);
      \foreach \i in {1,2,4,6} \draw (matrix-4-\i) -- (matrix-5-\i);
      \draw (matrix-1-4.south) -- (matrix-2-5.north);
      \draw (matrix-4-4.south) -- (matrix-5-5.north);
      \draw (matrix-4-1.south) -- (matrix-5-2.north);
      \node [left=1mm of matrix-1-1.center, overlay] {\textsubscript{wd}[};
      \node [left=1mm of matrix-1-4.center, overlay] {\textsubscript{wd}[};
      \end{tikzpicture}
  \ex\label{ex:7:3b}
  \begin{tikzpicture}[baseline={(matrix-1-1.base)}]
    \matrix (matrix) [matrix of nodes, nodes in empty cells]
      { /x & i/ &[5mm] &[5mm] /ç  & /\\[5mm]
        \relax{[\textsc{dorsal}]} & {[\textsc{coronal}]} &  & {[\textsc{dorsal}]} & {[\textsc{coronal}]}\\
                             &                     & > &                   & \\
        \relax {[ç} & i] & & {[ç} &  {]}\\[5mm]
        \relax {[\textsc{dorsal}]} & {[\textsc{coronal}]} &  & {[\textsc{dorsal}]} & {[\textsc{coronal}]}\\
      };
      \foreach \i in {1,2,4} \draw (matrix-1-\i) -- (matrix-2-\i);
      \foreach \i in {1,2,4} \draw (matrix-4-\i) -- (matrix-5-\i);
      \draw (matrix-1-4.south) -- (matrix-2-5.north);
      \draw (matrix-4-4.south) -- (matrix-5-5.north);
      \draw (matrix-4-2.south) -- (matrix-5-1.north);
%       \node [left=1mm of matrix-1-1.center, overlay] {\textsubscript{wd}[};
%       \node [left=1mm of matrix-1-4.center, overlay] {\textsubscript{wd}[};
  \end{tikzpicture} 
  \z
\z

The structure to the left of the wedge in \REF{ex:7:2} and \REF{ex:7:3} illustrates the stage in which velar fronting is present as an allophonic rule (=Stage 2). At that point, velar fronting spreads [coronal] from the front segment (/i/) to an adjacent velar (/x/), thereby creating a palatal, i.e. a structure with both [coronal] and  [dorsal]. The quasi-phoneme /ç/ is present to the right of the wedge (=Stage 3) in \REF{ex:7:2} and \REF{ex:7:3}: In \REF{ex:7:2a} and \REF{ex:7:3a} the velar fronting trigger (/i/) is restructured to a back vowel (/ɑ/), a change requiring that the trigger lose [coronal] and acquire [dorsal]. Crucially, the [coronal] feature in question is not deleted entirely, but instead it remains linked to the palatal. Since that palatal can no longer be derived synchronically from an adjacent front sound, it is present in the underlying representation. In \REF{ex:7:2b} and \REF{ex:7:3b} the palatal is quasi-phonemicized when the velar fronting trigger (/i/) deletes.

As depicted in \REF{ex:7:2} and \REF{ex:7:3}, palatal quasi-phonemes emerge when a sound that serves as trigger for velar fronting is no longer present. From the formal perspective, the change involves the deletion of the feature that is propagated in velar fronting, which in the present treatment is [coronal]. The quasi-phonemes in the case studies described below can arise from any of the four changes listed in \REF{ex:7:4}, all of which restructure underlying representations. The first three sound changes were introduced in preceding chapters; \isi{Syncope} is discussed below. The changes in \REF{ex:7:4} all have in common that they decrease the number of potential triggers for velar fronting (=Rule Z in \tabref{tab:2.1}).

\ea%4
\label{ex:7:4}Sound changes which can delete [coronal]:
\begin{multicols}{2}\raggedcolumns
\ea\label{ex:7:4a} Vowel Retraction:\smallskip\\
  / \avm[attributes=\normalfont]{\{ front\\vowel \}} / >
  / \avm[attributes=\normalfont]{\{ back\\vowel \}} /
\ex\label{ex:7:4b} \isi{r-Retraction}:\smallskip\\
  /r/ > /ʀ/
\ex\label{ex:7:4c} Vowel Reduction:\smallskip\\
 / \avm[attributes=\normalfont]{\{ unstressed\\vowel \}} / > /ə/
\ex\label{ex:7:4d} \isi{Syncope}:\smallskip\\
 / \avm[attributes=\normalfont]{\{ unstressed\\vowel \}} / > ∅
\z
\end{multicols}
\z 


Vowel Retraction (\sectref{sec:3.2}) in \REF{ex:7:4a} is a cover term for the change from a front vowel to a back vowel. A formally similar change to \REF{ex:7:4a} is \isi{r-Retraction} (\sectref{sec:3.5}) in \REF{ex:7:4b}, which is responsible for the change from coronal /r/ to dorsal (uvular) /ʀ/. \isi{Vowel Reduction} (\sectref{sec:4.3}) in \REF{ex:7:4c} is the change from any unstressed \isi{full vowel} to \isi{schwa}. Recall that \isi{full vowels} bear place features, while \isi{schwa} does not; hence, the change in \REF{ex:7:4c} involves the deletion of place features, including crucially [coronal] if the vowels in question are front. Although \isi{Vowel Reduction} affected the vowel in both prefixes and in suffixes, the examples discussed below involve primarily the former, in particular the deletion of historical [i] in the \textit{ge}{}- ([gə]) prefix of \il{Standard German}StG (cf. \ili{OHG} \textit{gi}{}-, \ili{OSax} \textit{gi}{}-).\footnote{{That prefix is also attested in early \ili{OHG} as} \textrm{\textit{gɑ}}\textrm{{}-. Since the vowel [i] (but not the vowel [ɑ]) serves as a trigger for velar fronting, I conclude that the realization with [ɑ] could not have been the one from which [ə] derives in the dialects I discuss below; see \sectref{sec:16.2} for discussion.}} \isi{Syncope} in \REF{ex:7:4d} entails the deletion of any vowel in an unstressed syllable. Significantly, if the vowel elided by \REF{ex:7:4d} is front (e.g. /i/), then [coronal] is lost. In the examples discussed below, \isi{Syncope} affected a front vowel in the weak member of a trochaic foot (e.g. the second syllable in \il{Standard German}StG [ˈhɑːbɪçt] ‘hawk’) or a front vowel in certain suffixes, e.g. the denominal adjective-forming -\textit{ig} ([ɪç]) (cf. \ili{OHG} -\textit{ig}).

In the remainder of this chapter I present a series of brief case studies from German dialects possessing quasi-phonemes, i.e. either /ç/, /ʝ/ or both sounds. Those dialects can have the underlying and surface fricatives depicted in \REF{ex:7:5a} and/or \REF{ex:7:5b}.

\ea\label{ex:7:5}
\begin{multicols}{2}
\ea\label{ex:7:5a}\begin{forest}
    [,phantom
       [/x/ [{[x]}]]   [/ç/  [{[ç]}]]
    ]  
    \end{forest}              
\ex\label{ex:7:5b}\begin{forest}
    [,phantom
     [/ɣ/ [{[ɣ]}]]     [/ʝ/  [{[ʝ]}]]
    ]
     \end{forest}
\z 
\end{multicols}
\z 

In some systems the palatal quasi-phonemes depicted in \REF{ex:7:5} can be found word-initially, in other systems they are attested in postsonorant position, and yet in others they occur in both contexts. The historical triggers for quasi-phonemes can be a coronal sonorant in any one of the changes listed in \REF{ex:7:4}.

Data are presented in \sectref{sec:7.2} and \sectref{sec:7.3} from WLG and CG varieties with palatal quasi-phonemes. In \sectref{sec:7.4} I discuss and reject various alternative treatments. \sectref{sec:7.5} provides some discussion of the areal distribution of palatal quasi-phonemes. The chapter concludes in \sectref{sec:7.6}.\largerpage[-2]

\section{{West} {Low} {German}}\label{sec:7.2}\largerpage[-2]

\citet{Arens1908} describes the \il{Westphalian}Wph dialect of \ipi{Elspe} (\mapref{map:6}). In that variety, [x] and [ç] do not contrast in word-initial position. In the context before a full back vowel, [x] occurs in (\ref{ex:7:6a}), while [ç] surfaces before a front vowel in (\ref{ex:7:6b}) or a coronal sonorant consonant in (\ref{ex:7:6c}). As suggested by the \il{Standard German}StG orthography in the third column, [x] and [ç] in (\ref{ex:7:6a}--\ref{ex:7:6c}) derived historically from \ili{WGmc} \textsuperscript{+}[ɣ]. The same complementary distribution of [x] and [ç] holds for [sx sç] clusters (<\ili{WGmc} \textsuperscript{+}[sk]) in (\ref{ex:7:6d}--\ref{ex:7:6f}). Most significantly, the items listed in \REF{ex:7:6g} illustrate that palatal [ç] (<\ili{WGmc} \textsuperscript{+}[ɣ]) occurs before \isi{schwa}.

\TabPositions{.15\textwidth, .33\textwidth, .5\textwidth, .75\textwidth}
\ea%6
\label{ex:7:6}\relax[x] and [ç] in a word-initial onset in \ipi{Elspe}:
\ea\label{ex:7:6a} xolt          \tab [xɔlt]      \tab Gold         \tab  ‘gold’                \tab  66    \\
    xɑrvə         \tab [xɑrvə]     \tab Garbe        \tab  ‘sheaf’               \tab  24     \\
    x\={ɑ}an      \tab [xɑːɐn]     \tab Garten       \tab  ‘garden’              \tab  25     \\
\ex\label{ex:7:6b} χīəẓn         \tab [çiːǝʝn̩]   \tab  gegen       \tab   ‘against’            \tab   43    \\
    χistan        \tab [çɪstan]    \tab gestern      \tab  ‘yesterday’           \tab  62     \\
    χyt           \tab [çʏt]       \tab gieβt        \tab  ‘water-\textsc{3sg}’ \tab  97     \\
    χɛ̄əštə        \tab [çɛːǝʃstǝ]  \tab Gerste       \tab  ‘barley’              \tab  38     \\
    χelt          \tab [çɛlt]      \tab Geld         \tab  ‘money’               \tab  31     \\
    χɒftə         \tab [çæftǝ]     \tab gäbe         \tab  ‘give-\textsc{subj}’  \tab  60     \\
\ex\label{ex:7:6c} χreŏt         \tab [çrɛɔt]     \tab groβ         \tab  ‘large’                  \tab  89  \\
    χloftə        \tab [çlɔftə]    \tab glaubte      \tab  ‘believe-\textsc{pret}’ \tab  89  \\
\ex\label{ex:7:6d} šxuɡn         \tab [ʃxʊɣn̩]    \tab  scheuen     \tab   ‘dread-\textsc{inf}’    \tab   96 \\
    šx\={ɑ}p      \tab [ʃxɑːp]     \tab Schrank       \tab  ‘cabinet’                \tab  23  \\
\ex\label{ex:7:6e} šχyt          \tab [ʃçʏt]      \tab schieβt      \tab  ‘shoot-\textsc{3sg}’    \tab  97  \\
    šχelə         \tab [ʃçɛlə]     \tab Schale       \tab  ‘bowl’                   \tab  33  \\
\ex\label{ex:7:6f} šχrɑpn        \tab [ʃçrɑpn̩]   \tab  schaben     \tab   ‘scrape-\textsc{inf}’   \tab   27 \\
\ex\label{ex:7:6g} χəv\={ɑ}a     \tab [çəvɑːɐ]    \tab gewahr       \tab  ‘aware’                  \tab  25  \\
    χəzelšop      \tab [çəzɛlʃop]  \tab Gesellschaft \tab ‘society’                 \tab  68  \\
    χəf\={ø}alək  \tab [çəføːɐlək] \tab gefährlich   \tab ‘dangerous’               \tab  57  \\
    \z
\z

/x/ in a word-initial onset surfaces as [ç] before a coronal sonorant in (\ref{ex:7:6b}, \ref{ex:7:6c}, \ref{ex:7:6e}, \ref{ex:7:6f}) by \REF{ex:7:7}, otherwise /x/ is realized as [x] in (\ref{ex:7:6a}, \ref{ex:7:6d}).\pagebreak

\ea%7
\label{ex:7:7}\isi{Wd-Initial Velar Fronting-6}:\\
    \begin{forest}
    [,phantom
      [\avm{[−son\\+cont]},name=parent [\avm{[dorsal]},tier=word]]
      [\avm{[+son]} [\avm{[coronal]},name=target,tier=word]]
    ]
    \draw [dashed] (parent.south) -- (target.north);
    \node [left=1ex of parent.west]{\textsubscript{wd}[  (C)};
    \end{forest}
\z 

Word-initial [ç] in (\ref{ex:7:6g}) is a quasi-phoneme /ç/ because it does not contrast with the corresponding velar in the context before \isi{schwa} and because it derived historically from the palatal allophone [ç] of the velar /x/. The change from the original /i/ to /ə/ in the initial syllable was due to \isi{Vowel Reduction} (=\ref{ex:7:4c}), e.g. [çəvɑːɐ] ‘aware’ <  \textsuperscript{+}[xivɑːɐ]; cf. \ili{OSax} \textit{giwar}. The latter change led to the \isi{overapplication} of the historical precursor of \REF{ex:7:7}.

Since \isi{Wd-Initial Velar Fronting-6} produces a sound ([ç]) that is present in underlying representations as a quasi-phoneme, that process is neither an allophonic rule, nor is it a \isi{neutralization}. Instead, \isi{Wd-Initial Velar Fronting-6} is a \isi{quasi-neutralization} in dialects like \ipi{Elspe}.

\begin{sloppypar}
In the \il{Eastphalian}Eph dialect of \ipi{Reinhausen} (\citealt{Jungandreas1926,Jungandreas1927}; \mapref{map:7}), [x ç] (<\ili{WGmc} \textsuperscript{+}[ɣ]) stand in complementary distribution in word-initial position. In his discussion of word-initial [x ç] Jungandreas observes that the velar [x] surfaces before back vowels (“vor velaren Vokalenˮ) and the palatal [ç] before front vowels (“vor palatalen Vokalenˮ); see (\ref{ex:7:8a}, \ref{ex:7:8b}). The author also notes that the palatal occurs before the two liquids, as in (\ref{ex:7:8c}, \ref{ex:7:8d}). Significantly, the symbol ⟦r⟧ in the original source represents a uvular (=dorsal) sound (“Wgerm. \textit{r} ist als Zäpfchen-r erhaltenˮ; \citealt{Jungandreas1926}: 288).  Jungandreas was aware of the anomalous nature of the palatal in \REF{ex:7:8d} in noting that its occurrence before [ʀ] is an indication that the rhotic was once pronounced as coronal (“ ... ein Zeichen übrigens, dass \textit{r} früher mit der Vorderzunge artikuliert wurdeˮ).
\end{sloppypar}

\ea%8
\label{ex:7:8}Word-initial dorsal fricatives in \ipi{Reinhausen}:
\ea\label{ex:7:8a} xūl\tab [xuːl]\tab Gaul\tab ‘horse’\tab 291\\
 xǫt\tab [xɔt]\tab Gott\tab ‘God’\tab 291\\
\ex\label{ex:7:8b} χęlt\tab [çɛlt]\tab Geld\tab ‘money’\tab 291\\
 χēm\tab [çeːm]\tab geben\tab ‘give-\textsc{inf}’\tab 291\\
\ex\label{ex:7:8c} χlīk\tab [çliːk]\tab gleich\tab ‘same’\tab 291\\
\ex\label{ex:7:8d} χrunt\tab [çʀunt]\tab Grund\tab ‘reason’\tab 291
\z
\z

Palatal [ç] derives from /x/ in \REF{ex:7:8b} by \isi{Wd-Initial Velar Fronting-6}, but the opaque [ç] in \REF{ex:7:8d} is a quasi-phoneme (/ç/) because it does not contrast with [x] in the context before [ʀ] and because it derived historically from the allophone [ç] of /x/. Note that the quasi-phonemicization of /ç/ was a consequence of the change from the coronal rhotic /r/ to /ʀ/ by \isi{r-Retraction} in (\ref{ex:7:4b}). The \isi{palatal quasi-phoneme} /ç/ before the dorsal rhotic [ʀ] (/ʀ/) in [çʀunt] ‘reason’ in \REF{ex:7:8d} can be compared with the synchronically \isi{derived palatal} [ç] (from /x/) before the coronal rhotic [r] (/r/) in \REF{ex:7:6c} [çrɛɔt] ‘large’ from \ipi{Elspe}.


\citet{Böger1906} describes the \il{Westphalian}Wph variety of the region in and around the town of \ipi{Schieder-Schwalenberg} (\mapref{map:6}). In word-initial position [x] (=⟦ɧ⟧) occurs before a back vowel in (\ref{ex:7:9a}) or the dorsal (uvular) consonant [ʀ] in (\ref{ex:7:9b}) and [ç] (=⟦χ⟧) before a front vowel in (\ref{ex:7:9c}), coronal sonorant consonant in (\ref{ex:7:9d}), or \isi{schwa} in (\ref{ex:7:9e}). The diachronic source for [x] and [ç] in the aforementioned examples is \ili{WGmc}  \textsuperscript{+}[ɣ]. In postsonorant position, velar [x] surfaces after a back vowel in (\ref{ex:7:10a}) and palatal [ç] after a front vowel in (\ref{ex:7:10b}). Both [x] and [ç] in (\ref{ex:7:10a}, \ref{ex:7:10b}) derive from \ili{WGmc}  \textsuperscript{+}[x]. Velar [ɣ] (=⟦ɡ⟧) surfaces in a word-internal onset after any vowel or sonorant consonant in (\ref{ex:7:10c}), or as [ç] in coda position after a coronal sonorant consonant in (\ref{ex:7:10d}). [ç] also surfaces in coda position after dorsal [ʀ] =⟦r⟧ in (\ref{ex:7:10e}).

\ea%9
\label{ex:7:9}Word-initial dorsal fricatives in \ipi{Schieder-Schwalenberg}:
\ea\label{ex:7:9a} ɧafəl \tab [xɑfəl] \tab Gabel \tab ‘fork’ \tab 151\\
    ɧōən \tab [xoːən] \tab gehen \tab ‘go\textsc{{}-inf}’ \tab 151\\
\ex\label{ex:7:9b} ɧraf \tab [xʀɑf] \tab Grab \tab ‘grave’ \tab 151\\
     ɧröte \tab [xʀøtə] \tab Gröβe \tab ‘size’ \tab 152\\
\ex\label{ex:7:9c} χistərn \tab [çistəʀn] \tab gestern \tab ‘yesterday’ \tab 151\\
    χelt \tab [çelt] \tab Geld \tab ‘money’ \tab 150\\
\ex\label{ex:7:9d} χlas \tab [çlɑs] \tab Glas \tab ‘glass’ \tab 151\\
     χnaidiχ \tab [çnɑidiç] \tab gnädig \tab ‘merciful’ \tab  151\\
\ex\label{ex:7:9e} χədult \tab [çədult] \tab Geduld \tab ‘patience’ \tab 150\\
   χəfōr \tab [çəfoːʀ] \tab Gefahr \tab ‘danger’ \tab 150\\
\z
\ex%10
\label{ex:7:10}Postsonorant dorsal fricatives in \ipi{Schieder-Schwalenberg}:
\ea\label{ex:7:10a}  luɧt     \tab [luxt] \tab Licht \tab ‘light’ \tab 157\\
     naɧt     \tab [nɑxt] \tab Nacht \tab ‘night’ \tab 158\\
\ex\label{ex:7:10b}  liχt     \tab [liçt] \tab leicht \tab ‘light’ \tab 156\\
     lüχtǝn   \tab [lyçtǝn] \tab leuchten \tab ‘glow\textsc{{}-inf}’ \tab 157\\
\ex\label{ex:7:10c}  jiuɡǝnt  \tab [ʝiuɣǝnt] \tab Jugend \tab ‘youth’ \tab 153\\
     möɡǝn    \tab [møɣǝn] \tab mögen \tab ‘like-\textsc{inf}’ \tab 158\\
     ärɡǝrn   \tab [ɛʀɣǝrn] \tab ärgern \tab ‘annoy-\textsc{inf}’ \tab 145\\
\ex\label{ex:7:10d}  talχ     \tab [tɑlç] \tab Talg \tab ‘tallow’ \tab 165\\
\ex\label{ex:7:10e}  ɑrχ      \tab [ɑʀç] \tab arg \tab ‘bad’ \tab 144\\
     ōərχ     \tab [oːəʀç] \tab artig \tab ‘well-behaved’ \tab 159
\z
\z 

Word-initial /x/ surfaces as [ç] before a coronal sonorant by \isi{Wd-Initial Velar Fronting-6} in (\ref{ex:7:9c}, \ref{ex:7:9d}), otherwise /x/ is realized as [x], in (\ref{ex:7:9a}, \ref{ex:7:9b}). As in \ipi{Elspe} (\ref{ex:7:6g}), word-initial opaque [ç] before \isi{schwa} in (\ref{ex:7:9e}) is a quasi-phoneme (/ç/) which arose when the original front vowel /i/ was restructured to /ə/ by \isi{Vowel Reduction} (=\ref{ex:7:4c}). After a coronal sonorant in (\ref{ex:7:10b}, \ref{ex:7:10d}), palatals derive from the corresponding velars (/x ɣ/) by \isi{Velar Fronting-4} in (\ref{ex:7:11}), otherwise those velars surface without change as velar in (\ref{ex:7:10a}--\ref{ex:7:10c}).

\ea%11
\label{ex:7:11}\isi{Velar Fronting-4}:\\
\begin{forest}
 [,phantom
   [\avm{[+son]} [\avm{[coronal]},tier=word,name=target]]
   [\avm{[−son\\+cont\\+fortis]},name=source [\avm{[dorsal]},tier=word]]
 ]
 \draw [dashed] (source.south) -- (target.north);
\end{forest}
\z 

The \isi{palatal quasi-phoneme} /ʝ/ occurs after /ʀ/, surfacing as [ç] in coda position in (\ref{ex:7:10e}). That quasi-phoneme arose historically when coronal /r/ was realized as uvular /ʀ/ by \isi{r-Retraction} (=\ref{ex:7:4b}).

In \tabref{extab:7:12} I provide historical derivations for representative examples for word-initial [x ç] from \ipi{Reinhausen} in \tabref{extab:7:12}a and \ipi{Schieder-Schwalenberg} in \tabref{extab:7:12}b. To save space I do not include examples in which the original velar occurred before a back vowel.

% \ea%12
\ip{Rheinhausen}
  \ip{Schieder-Schwalenberg}
\begin{table}
\small
\subfigure[Reinhausen (=\ref{ex:7:8})]{
  \begin{tabular}[t]{@{}l@{~}l@{~}ll}
 \relax  /xɛlt/  &   /xliːk/ &   /xrunt/ &          \\
 \relax  [xɛlt]  &   [xliːk] &   [xrunt] &  Stage 1 \\\tablevspace
 \relax  /xɛlt/  &   /xliːk/ &  /xrunt/  &          \\
 \relax  [çɛlt]  &   [çliːk] &   [çrunt] &  Stage 2 \\\tablevspace
 \relax  /xɛlt/  &   /xliːk/ &   /çʀunt/ &          \\
 \relax  [çɛlt]  &   [çliːk] &   [çʀunt] & Stage 3  \\\tablevspace
 \textit{Geld}  &   \textit{gleich} & \textit{Grund} & \il{Standard German}StG\\
  ‘money’ &  ‘same’ & ‘reason’ \\
  \end{tabular}
  }
\subfigure[Schieder-Schwalenberg (=\ref{ex:7:9})]{
  \begin{tabular}[t]{l@{~}l@{~}l@{~}l @{}}
 /xelt/  & /xlɑs/   & /xrɑf/  & /xidult/  \\{}
 [xelt]  &  [xlɑs]  &  [xrɑf] &   [xidult]\\\tablevspace
 /xelt/  & /xlɑs/   & /xrɑf/  & /xidult/  \\{}
 [çelt]  &  [çlɑs]  &  [çrɑf] &   [çidult]\\\tablevspace
 /xelt/  &  /xlɑs/  &  /xʀɑf/ &  /çədult/ \\{}
  [çelt] &  [çlɑs]  &  [xʀɑf] &  [çədult] \\\tablevspace
 \textit{Geld} & \textit{Glas} & \textit{Grab}  &   \textit{Geduld}  \\
 ‘money’ & ‘glass’ & ‘grave’ & ‘patience’\\
  \end{tabular}
  }
  \caption{\label{extab:7:12} Historical derivations  for word-initial [x ç] from Reinhausen  and Schieder-Schwalenberg}
\end{table}\todo{positioning stages}

Consider first \tabref{extab:7:12}a. At Stage 2 the two fricatives [x] and [ç] stood in an allophonic relationship, but when /r/ was restructured to /ʀ/ at Stage 3 by \isi{r-Retraction}, the word-initial opaque fricative [ç] in examples like [çʀunt] ‘reason’ was quasi-phonemicized to /ç/. Examples like [çɛlt] ‘money’ and [çliːk] ‘same’ demonstrate that \isi{Wd-Initial Velar Fronting-6} remains active synchronically at Stage 3. In \tabref{extab:7:12}b [x ç] were allophones at Stage 2. In words like [xʀɑf] ‘grave’ \isi{r-Retraction} restructured /r/ to /ʀ/ at Stage 3, but the original /x/ was not quasi-phonemicized (in contrast to the /x/ in \ipi{Reinhausen} [çʀunt] ‘reason’ in \tabref{extab:7:12}a). Instead, underlying /x/ in [xʀɑf] was retained as /x/ at Stage 3. The final example in \tabref{extab:7:12}b illustrates the quasi-phonemicization of /ç/ when \isi{Vowel Reduction} restructured /i/ to /ə/ at Stage 3.

The \il{Westphalian}Wph variety of Kreis \ipi{Lippe} (\citealt{Hoffmann1887}; \mapref{map:6}) has the four dorsal fricatives [x ɣ ç ʝ], whose postsonorant distribution is exemplified below. See \sectref{sec:14.2.2} for discussion of word-initial position, where only palatals but not velars surface. In postsonorant position [x] (<\ili{WGmc} \textsuperscript{+}[x] or \textsuperscript{+}[f]) surfaces after a back vowel in (\ref{ex:7:13a}) and [ç] (<\ili{WGmc} \textsuperscript{+}[x]) after a front vowel in (\ref{ex:7:13b}). Velar [ɣ] surfaces in a word-internal onset after a back vowel in (\ref{ex:7:13c}), and palatal [ʝ] occur in a word-internal onset after a front vowel in (\ref{ex:7:13d}) or coronal sonorant consonant in (\ref{ex:7:13e}). [ɣ ʝ] in those examples derive historically from \ili{WGmc} \textsuperscript{+}[ɣ] or \textsuperscript{+}[gg]. Regular alternations involving the four dorsal fricatives permeate the inflectional system in (\ref{ex:7:13f}). The example [dreːux] ‘carry\textsc{{}-pret}’ in (\ref{ex:7:13f}) shows that the second part of the diphthong and not the first determines the place of the following dorsal fricative. Opaque palatals (quasi-phonemes) surface after dorsal [ʀ] in (\ref{ex:7:13g}) and the diphthong [æu] in (\ref{ex:7:13h}).\footnote{{It is clear from \citet[5]{Hoffmann1887} that the one rhotic surfaces as a uvular consonant (/ʀ/) even in coda position; hence, \isi{r-Vocalization} (\sectref{sec:4.3}) is not active in the dialect. As the phonetic symbol ⟦æu⟧ in the original source suggests, the first element in that diphthong is front, and the second one is back (in Hoffmann’s terms “Gutturalˮ; see \citealt{Hoffmann1887}: 11--13).} } No example was found in the original source for [ç] after [æu], a gap I consider to be accidental.

\ea%13
\label{ex:7:13}Postsonorant dorsal fricatives in Kreis \ipi{Lippe}:
\ea\label{ex:7:13a} luxt \tab [lʊxt] \tab Luft \tab ‘air’ \tab 19\\
dɑxt \tab [dɑxt] \tab Docht \tab ‘wick’ \tab 44\\
\ex\label{ex:7:13b}  liχt \tab [lɪçt] \tab  leicht \tab ‘light’ \tab 44\\
füχtə \tab [fʏçtə] \tab  Fichte \tab ‘spruce’ \tab 46\\
reχt \tab [ʀɛçt] \tab  Recht \tab ‘justice’ \tab 15\\
\ex\label{ex:7:13c}   bọ̄ʒə \tab [boːɣən] \tab Bogen \tab ‘bow’ \tab 19\\
wɑ̄ʒən \tab [vɑːɣən] \tab Wagen \tab ‘car’ \tab 50\\
auʒə \tab [ɑuɣə] \tab Auge \tab ‘eye’ \tab 26\\
\ex\label{ex:7:13d}   χījən \tab [çiːʝǝn] \tab gegen \tab ‘against’ \tab 15\\
brüjə \tab [brʏʝə] \tab Brücke \tab ‘bridge’ \tab 4\\
\ex\label{ex:7:13e}  χɑljən \tab [çɑlʝən] \tab Galgen \tab ‘gallows’ \tab 14\\
\ex\label{ex:7:13f}  drẹ̄ux \tab [dreːux] \tab trug \tab ‘carry\textsc{{}-pret}’ \tab 24\\
drẹ̄jən \tab [dreːʝən] \tab tragen \tab ‘carry\textsc{{}-inf}’ \tab 4\\
ʽɑux \tab [hɑux] \tab hoch \tab ‘high’ \tab 25\\
ʽoijər \tab [hoiʝər] \tab höher \tab ‘higher’ \tab 26\\
\ex\label{ex:7:13g}  forχt \tab [fɔʀçt] \tab  Furcht \tab ‘fear’ \tab  18\\
sorjə \tab [sɔʀʝə] \tab Sorge \tab ‘sorrow’ \tab 18\\
\ex\label{ex:7:13h}  æujən \tab  [æuʝən] \tab  eigen \tab ‘own’ \tab 23\\
læujən \tab  [læuʝən] \tab lägen \tab ‘lie\textsc{{}-subj}’ \tab 32\\
læujən \tab  [læuʝən] \tab lügen \tab ‘lie\textsc{{}-inf}’ \tab 28
\z 
\z 


Hoffmann lists no examples in which a velar fricative surfaces after [æu]. It will become clear below that there is a historical reason for that gap. Significantly, [æu] has a relatively free distribution and is therefore phonemic (/æu/) in the dialect as it was described in 1886. For example, there are no restrictions concerning the place or manner of articulation of any consonants to the left or right of [æu], e.g. [bʀæuf] ‘letter’, [væuk] ‘soft’, [æutə] ‘eat\textsc{{}-subj}’. What is more, [æu] contrasts with other diphthongs and monophthongs, e.g. in the context before [p] in [dæup] ‘deep’ vs. [knɑup] ‘button’.

The velars /x ɣ/ surface as palatals [ç ʝ] after a coronal sonorant in (\ref{ex:7:13b}, \ref{ex:7:13d}--\ref{ex:7:13f}) by \isi{Velar Fronting-1} in (\ref{ex:7:14}), otherwise those underlying velars are realized as [x ɣ] in (\ref{ex:7:13a}, \ref{ex:7:13c}, \ref{ex:7:13f}).

\ea%14
\label{ex:7:14}\isi{Velar Fronting-1}:\\
\begin{forest}
   [,phantom
     [\avm{[+son]} [\avm{[coronal]},name=target,tier=word]]
     [\avm{[−son\\+cont]},name=source [\avm{[dorsal]},tier=word]]
   ]
   \draw [dashed] (target.north) -- (source.south);
\end{forest}
\z 

\begin{sloppypar}\relax
[ç ʝ] are quasi-phonemes (/ç ʝ/) in (\ref{ex:7:13g}, \ref{ex:7:13h}), e.g. /fɔʀçt/ ‘fear’, /æuʝən/ ‘own’. Those underlying (opaque) palatals arose historically from front ([coronal]) sounds to their immediate left. The historical /r/ in \REF{ex:7:13g} restructured to the [dorsal] rhotic (/ʀ/) via \isi{r-Retraction} (=\ref{ex:7:4b}). The diphthong [æu] in \REF{ex:7:13h} was a front vowel at an earlier stage which shifted to [æu] (/æu/) by \isi{Vowel Retraction} (=\ref{ex:7:4a}). In particular, [æu] is the reflex of earlier [eː] (/eː/), which itself derived form one of three vowels: [eː], [ɑː], [io] (all present in \ili{OSax}), e.g. [æuʝən] ‘own’ (cf. OSax \textit{ēgan}), [læuʝə] ‘lie-\textsc{subj}’ (cf. OSax \textit{lāgīn}), and [læuʝən] ‘lie\textsc{{}-inf}’ (cf. OSax \textit{liogan}). The three original vowels [eː ɑː io] merged to the front vowel [eː] (/eː/), which later shifted to [æu]; \citet[62--63]{Hoffmann1887}. That all instances of modern [æu] were once a front monophthong ([eː] /eː/) derives additional support from the survey of LG dialects presented in \citet{Sarauw1921}, who provides a list of the modern reflexes of the \ili{OSax} vowels in question in eighteen LG communities (p. 145). According to that chart, the modern reflexes are either front monophthongs (typically [eː]) or diphthongs whose second member is a front vowel (e.g. [ɑi], [ei]) in every LG variety with the exception of the one described by \citet{Hoffmann1887}. What this suggests is that [æu] was at one point a front vowel and that the change to [æu] was a very recent shift because it only occurred in the Kreis \ipi{Lippe} variety and nowhere else.
\end{sloppypar}

\section{{Central} {German}}\label{sec:7.3}

\citet{Hasenclever1905} describes the \il{Ripuarian}Rpn dialect of \ipi{Wermelskirchen} (\mapref{map:8}). See \sectref{sec:14.2.2} for discussion of word-initial position, where only palatals but not velars surface. In postsonorant position [x ɣ] (=⟦χ g⟧) surface after a back vowel in (\ref{ex:7:15a}, \ref{ex:7:15c}) and [ç ʝ] after a front vowel or coronal sonorant consonant in (\ref{ex:7:15b}, \ref{ex:7:15d}, \ref{ex:7:15e}). Opaque palatals (quasi-phonemes) surface after the dorsal (uvular) rhotic in (\ref{ex:7:15f}) or \isi{schwa} in (\ref{ex:7:15g}). \citet[10]{Hasenclever1905} states that the sound he transcribes as ⟦r⟧ is uvular (=dorsal) and not coronal.  The dorsal fricatives in \REF{ex:7:15} derived historically from velars (\ili{WGmc} \textsuperscript{+}[ɣ x k]).

\ea%15
\label{ex:7:15}Dorsal fricatives in \ipi{Wermelskirchen}:
\ea\label{ex:7:15a} lɑχən \tab [lɑxən] \tab lachen \tab ‘laugh\textsc{{}-inf}’ \tab 51\\
\ex\label{ex:7:15b} diçtə \tab [diçtə] \tab dicht \tab ‘dense’ \tab 51\\
ʃprɛçən \tab [ʃprɛçən] \tab sprechen \tab ‘speak\textsc{{}-inf}’ \tab 51\\
\ex\label{ex:7:15c} fūgəl \tab [fuːɣǝl] \tab Vogel \tab ‘bird’ \tab 47\\
zɑgən \tab [zɑɣǝn] \tab sagen \tab ‘say-\textsc{inf}’ \tab 47\\
\ex\label{ex:7:15d} fɛ̄jən \tab [fɛːʝən] \tab fegen \tab ‘sweep-\textsc{inf}’ \tab 47\\
\ex\label{ex:7:15e} foljən \tab [fɔlʝən] \tab folgen \tab ‘follow-\textsc{inf}’ \tab 47\\
\ex\label{ex:7:15f} ɛrjər \tab [ɛʀʝər] \tab Ärger \tab ‘anger’ \tab 47\\
\ex\label{ex:7:15g} īːvəç \tab [iːvəç] \tab ewig \tab ‘eternal’ \tab 83\\
\z
\z 

\ipi{Wermelskirchen} /x ɣ/ shift to the corresponding palatals after a coronal sonorant by \isi{Velar Fronting-1}. The two contexts in which quasi-phonemes occur are: (i) after /ʀ/ in (\ref{ex:7:15f}), and (ii) after /ə/ in (\ref{ex:7:15g}). The original palatal allophones were quasi-phonemicized in (i) when /r/ was restructured to /ʀ/ by \isi{r-Retraction} in (\ref{ex:7:4b}), and in (ii) when front vowels shifted to \isi{schwa} (/ə/) by \isi{Vowel Reduction} in (\ref{ex:7:4c}). The vowel in the -\textit{ig} ([əç]) suffix in \REF{ex:7:15g} derived historically from [i] (cf. OHG -\textit{ig}).

In the \il{Moselle Franconian}MFr variety of \ipi{Echternach} (\citealt{Palgen1931}; \mapref{map:10}) velar [x] surfaces after back vowels in (\ref{ex:7:16a}) and palatal [ç] after front vowels in (\ref{ex:7:16b}). In intervocalic position historical [ɣ ʝ] (<\ili{WGmc} \textsuperscript{+}[ɣ]) elided, although a few rare words preserve [ɣ] if the preceding vowel is back in (\ref{ex:7:16c}). Palatal [ʝ] is regularly retained after a coronal sonorant consonant (i.e. [l] in \ref{ex:7:16d}). Significantly, the two palatals [ç ʝ] also surface after the vocalized-r in (\ref{ex:7:16e}, \ref{ex:7:16f}). \citet[6]{Palgen1931} observes that the one rhotic (/ʀ/) is articulated on the uvula (“Zäpfchen-rˮ) in initial position and that it is vocalized in coda position. That sound is transcribed in the original source as ⟦ɒ⟧, which I render as [ɐ], as in all other German dialects with that sound (recall \chapref{sec:2} and \chapref{sec:3}).

\ea%16
\label{ex:7:16}Dorsal fricatives in \ipi{Echternach}:
\ea\label{ex:7:16a} vox \tab [vox] \tab Woche \tab ‘week’ \tab 45\\
hǫux \tab [hɔux] \tab Hauch \tab ‘breath’ \tab 27\\
\ex\label{ex:7:16b} rīχtən \tab [ʀiːçtən] \tab richten \tab ‘judge\textsc{{}-inf}’ \tab 18\\
brēχən \tab [bʀeːçən] \tab brechen \tab ‘break\textsc{{}-inf}’ \tab 45\\
šläχt \tab [ʃlæçt] \tab schlecht \tab ‘bad’ \tab 21\\
\ex\label{ex:7:16c} mōɣən \tab [moːɣǝn] \tab Magen \tab ‘stomach’ \tab 49\\
\ex\label{ex:7:16d} ɡɑljən \tab [gɑlʝən] \tab Galgen \tab ‘gallows’ \tab 49\\
\ex\label{ex:7:16e} kīɒχ \tab [kiːɐç] \tab Kirche \tab ‘church’ \tab 18\\
\ex\label{ex:7:16f} z·ō.ɒχ \tab [zoːɐç] \tab Sorge \tab ‘sorrow’ \tab 49\\
zōɒjən \tab [zoːɐʝən] \tab sorgen \tab ‘care for\textsc{{}-inf}’ \tab 49\\
\z
\z 

Palatals ([ç ʝ]) in (\ref{ex:7:16b}, \ref{ex:7:16d}) derive from velars (/x ɣ/) by \isi{Velar Fronting-1}, otherwise they surface as [x ɣ] in (\ref{ex:7:16a}, \ref{ex:7:16c}). The palatals [ç] and [ʝ] in (\ref{ex:7:16e}, \ref{ex:7:16f}) are quasi-phonemes (/ç ʝ/), which arose via \isi{r-Retraction} in (\ref{ex:7:4b}). Thus, the original rhotic was coronal [r] (/r/), which was restructured to [ʀ] (/ʀ/). From the synchronic perspective, /ʀ/ in the dialect as it was described in 1931 surfaces as [ɐ] in coda position by \isi{r-Vocalization}:\largerpage[-1]\pagebreak

\ea%17
\label{ex:7:17}\isi{r-Vocalization}:\smallskip\\
\avm{[+cons\\+son\\−nasal\\dorsal] → [−cons]} / \_\_\_\_ C\textsubscript{0} ] \textsubscript{${\sigma}$}
\z 

Recall from \sectref{sec:3.5} and \sectref{sec:4.3} that \isi{Liquid Vocalization} (and the more specific process of \isi{r-Vocalization}) produce the back vowel [ɐ] in other dialects, e.g. \ipi{Soest} (\il{Westphalian}Wph), \ipi{Ramsau am Dachstein} (\il{Central Bavarian}CBav). A significant difference between those earlier case studies and \ipi{Echternach} is that dorsal fricatives to the right of the vocalized-r are realized in \ipi{Echternach} as palatals and not as velars, cf. [ʃtɔɐx] ‘sorrow’ (from /ʃtɔʀx/) in \ipi{Ramsau am Dachstein} and [bɛːɐx] ‘mountain’ (from /bɛːʀx/) in \ipi{Soest}. The occurrence of palatal fricatives after the vocalized-r in \ipi{Echternach} has a parallel in \il{Standard German}StG, which is discussed in greater detail in \sectref{sec:17.3.1}.

The distribution of [x ç] (< \ili{WGmc} \textsuperscript{+}[k x]) in the \il{North Hessian}NHes variety attested in \ipi{Loshausen} (\citealt{Corell1936}; \mapref{map:11}) is illustrated in \REF{ex:7:18}. The velar surfaces after a back vowel in (\ref{ex:7:18a}) and the palatal after a front vowel in (\ref{ex:7:18b}) or a coronal sonorant consonant in (\ref{ex:7:18c}). The item listed in \REF{ex:7:18d} shows that the opaque palatal [ç] surfaces after a noncoronal consonant. \ipi{Loshausen} also possesses the palatal fricative [ʝ], whose distribution is not discussed here.

\begin{map}
% \includegraphics[width=.8\textwidth]{figures/VelarFrontingHall2021-img017.png}
\centering\includegraphics[width=.8\textwidth]{figures/Map11_7.1.pdf}
\caption[East Hessian, Central Hessian, and North Hessian]{East Hessian (\il{East Hessian}EHes), Central Hessian (\il{Central Hessian}CHes), and North Hessian (\il{North Hessian}NHes). Squares indicate postsonorant velar fronting. 1=\citet{Hertel1888}, 2=\citet{Salzmann1888}, 3=\citet{Glöckner1913}, 4=\citet{Noack1938}, 5=\citet{Martin1957}, 6=\citet{Müller1958a}, 7=\citet{Weber1959}, 8=\citet{Krafft1969}, 9=\citet{Wegera1977}, 10=\citet{Post1985}, 11=\citet{Schwarz1992}, 12=\citet{Dingeldein1995}, 13=\citet{Leidolf1891}, 14=\citet{WagnerHorn1900}, 15=\citet{Knauss1906}, 16=\citet{Schaefer1907}, 17=\citet{Reuss1907}, 18=\citet{Freund1910}, 19=\citet{Faber1912}, 20=\citet{Kroh1915}, 21=\citet{Rauh1921}, 22=\citet{Schwing1921}, 23=\citet{Siemon1922}, 24=\citet{Urff1926}, 25=\citet{Schudt1927}, 26=\citet{Bender1938}, 27=\citet{Friebertshäuser1961}, 28=\citet{Schnellbacher1963}, 29=\citet{Spenter1964}, 30=\citet{BethgeBonnin1969}, 31=\citet{Schudt1970}, 32=\citet{Hasselbach1971}, 33=\citet{Hasselberg1979}, 34=\citet{Féry2017}, 35=\citet{Dittmar1891}, 36=\citet{Schoof1913a, Schoof1913b, Schoof1913c}, 37=\citet{Hackler1914}, 38=\citet{Heidt1922}, 39=\citet{Hofmann1926}, 40=\citet{Bromm1936}, 41=\citet{Corell1936}, 42=\citet{Hofmann1940}, 43=\citet{Martin1942} (\ipi{Battenberg}), 44=\citet{Martin1942} (\ipi{Bad Wildungen}), 45=\citet{Müller1958b}, 46=\citet{Möhn1962}, 47=\citet{Arend1991}.}\label{map:11}
\end{map}


\ea%18
\label{ex:7:18}Dorsal fricatives in \ipi{Loshausen}:
\ea\label{ex:7:18a} ọ̄xt \tab [oːxt] \tab acht \tab ‘eight’ \tab 141\\
lɑxə \tab [lɑxə] \tab lachen \tab ‘laugh\textsc{{}-inf}’ \tab 141\\
\ex\label{ex:7:18b}  ẹ̄χəl \tab [eːçəl] \tab Eichel \tab ‘acorn’ \tab 134\\
rȩ̄χt \tab [rɛːçt] \tab recht \tab ‘right’ \tab 134\\
\ex\label{ex:7:18c} mẹlχ \tab [melç] \tab Milch \tab ‘milk’ \tab 134\\
lęrχ \tab [lɛrç] \tab Lerche \tab ‘lark’ \tab 134\\
\ex\label{ex:7:18d} hǫbχ \tab [hɔpç] \tab Habicht \tab ‘hawk’ \tab 134\\
\z
\z 

Palatal [ç] in (\ref{ex:7:18b}, \ref{ex:7:18c}) is derived from /x/ by \isi{Velar Fronting-1}, and the opaque [ç] in \REF{ex:7:18d} is a quasi-phoneme (/ç/), which arose when the original front vowel before [ç] was eliminated via \isi{Syncope} (=\ref{ex:7:4d}).

\citet{Hofmann1926} is a historical grammar and dictionary documenting the \il{North Hessian}NHes community of \ipi{Oberellenbach} (\mapref{map:11}). The examples in (\ref{ex:7:19a}--\ref{ex:7:19e}) show the basic pattern whereby the palatals [ç ʝ] surface after a coronal sonorant and the velars [x ɣ] after a back vowel. In these examples, [x ç] are the reflexes of \ili{WGmc}  \textsuperscript{+}[k x] and [ɣ ʝ] of \ili{WGmc}  \textsuperscript{+}[ɣ]. The examples in (\ref{ex:7:19f}, \ref{ex:7:19g}) exemplify the occurrence of the quasi-phoneme /ç/, which occurs after a noncoronal consonant in \REF{ex:7:19f} and word-initially before \isi{schwa} in (\ref{ex:7:19g}).

\ea%19
\label{ex:7:19}Dorsal fricatives in \ipi{Oberellenbach}:
\ea\label{ex:7:19a} būx  \tab [buːx] \tab Buch \tab ‘book’ \tab  73\\
kǫx \tab [kɔx] \tab Koch \tab ‘cook’ \tab 145\\
lɑxən \tab [lɑxən] \tab lachen \tab ‘laugh\textsc{{}-inf}’ \tab 153\\
\ex\label{ex:7:19b} eχ  \tab [eç] \tab ich \tab ‘I’ \tab 129\\
l\={ø}χ \tab [løːç] \tab  Lauch \tab ‘leek’ \tab 19\\
blɑ̜χ \tab [blæç] \tab Blech \tab ‘tin’ \tab 68\\
\ex\label{ex:7:19c} wōʒə \tab [βoːɣə] \tab Waage \tab ‘scale’ \tab 27\\
\ex\label{ex:7:19d} ijəl \tab [iʝəl] \tab Igel \tab ‘hedgehog’ \tab 27\\
bējən \tab [beːʝən] \tab biegen \tab ‘bend\textsc{{}-inf}’ \tab 27\\
sɑ̜̜̄jṇ \tab [sæːʝn̩] \tab sagen \tab ‘say\textsc{{}-inf}’ \tab 27\\
\ex\label{ex:7:19e} męlχ \tab [mɛlç] \tab Milch \tab ‘milk’ \tab 168\\
ɑ̜rjər \tab [ærʝər] \tab Ärger \tab ‘anger’ \tab 54\\
\ex\label{ex:7:19f} hǫbχ \tab [hɔpç] \tab Habicht \tab ‘hawk’ \tab 24\\
\ex\label{ex:7:19g} jəsønt \tab [ʝəsønt] \tab gesund \tab ‘healthy’ \tab 106
\z
\z 

The opaque palatal in \REF{ex:7:19f} originally stood before a front vowel and was quasi-phonemicized when that segment underwent \isi{Syncope} (=\ref{ex:7:4d}). The \isi{schwa} in \REF{ex:7:19g} likewise derived historically from the front vowel [i] (/i/); the palatal that stood before that sound was quasi-phonemicized when the original /i/ underwent \isi{Vowel Reduction} to [ə] (/ə/) (=\ref{ex:7:4c}). Note that word-initial \ili{WGmc} \textsuperscript{+}[ɣ] shifted to [ʝ] only before [i], which was later realized as \isi{schwa}; before any other sound, \ili{WGmc}  \textsuperscript{+}[ɣ] surfaces as [g], e.g. [geːʝən] ‘around’ (=⟦gējən⟧). The palatals in (\ref{ex:7:19b}, \ref{ex:7:19d}, \ref{ex:7:19e}) are derived from the corresponding velars by \isi{Velar Fronting-1}, while the opaque sounds in (\ref{ex:7:19f}, \ref{ex:7:19g}) are palatal quasi-phonemes (/ç ʝ/).

The \il{North Hessian}NHes variety attested in and around \ipi{Rauschenberg} (\citealt{Bromm1936}; \mapref{map:11}) possesses the four dorsal fricatives [x ç ɣ ʝ]. The postsonorant distribution of the velar and palatal articulations is exemplified in \REF{ex:7:20}: [x] surfaces after back vowels with the exception of the long low vowel [ɑː] in (\ref{ex:7:20a}) and [ç] after front vowels in (\ref{ex:7:20b}). Historical [ɑː] (/ɑː/) was regularly replaced with [ɔ] (/ɔ/), e.g. [ʃtɔxə] ‘sting-\textsc{pret}’ (cf. MHG [stɑːx]). Velar [ɣ] likewise occurs after any phonemic vowel with the exception of [ɑː] in (\ref{ex:7:20c}), while its palatal counterpart surfaces in a word-internal onset after a front vowel in (\ref{ex:7:20d}) or coronal sonorant consonant in (\ref{ex:7:20e}). The items listed in \REF{ex:7:20f} show that an opaque palatal [ç] surfaces after [ɑː] (/ɑː/). As indicated in the \il{Standard German}StG orthography in the third column, the [ɑː] (/ɑː/) in the latter examples derived historically from the front vowel [e] (/e/). Parallel examples with opaque [ʝ] were not found in the original source.

\ea%20
\label{ex:7:20}Dorsal fricatives in \ipi{Rauschenberg}:
\ea\label{ex:7:20a} bux \tab [bux] \tab Buch \tab ‘book’ \tab 23\\
hōx \tab [hoːx] \tab hoch \tab ‘high’ \tab 20\\
nǭxd \tab [nɔːxt] \tab  Nacht \tab  ‘night’ \tab 9\\
lǫx \tab [lɔx] \tab Loch \tab ‘hole’ \tab 23\\
mɑxə \tab [mɑxə] \tab machen \tab ‘do\textsc{{}-inf}’ \tab 23\\
\ex\label{ex:7:20b} liχd \tab  [liçt] \tab Licht \tab ‘light’ \tab 20\\
ɡəseχd \tab [gəseçt] \tab Gesicht \tab ‘face’ \tab 30\\
ds\={ę}χə \tab [tsɛːçə] \tab Zeichen \tab ‘sign’ \tab 18\\
bręχə \tab [brɛçə] \tab brechen \tab ‘break\textsc{{}-inf}’ \tab 23\\
ręiχ \tab [rɛiç] \tab reich \tab ‘rich’ \tab 23\\
\ex\label{ex:7:20c} foɣəl \tab [foɣəl] \tab  Vogel \tab ‘bird’ \tab  25\\
ǭɣə \tab [ɔːɣə] \tab Auge \tab ‘eye’ \tab 19\\
\ex\label{ex:7:20d} flijə \tab  [fliʝə] \tab fliegen \tab ‘fly\textsc{{}-inf}’ \tab 21\\
wējə \tab [veːʝə] \tab Wege \tab ‘path-\textsc{pl}’ \tab 25\\
f\={ę}jə \tab [fɛːʝə] \tab fegen \tab ‘sweep\textsc{{}-inf}’ \tab 25\\
\ex\label{ex:7:20e} foljə \tab [folʝə] \tab  folgen \tab ‘follow\textsc{{}-inf}’ \tab 25\\
\ex\label{ex:7:20f} rɑ̄χd \tab [rɑːçt] \tab  recht \tab ‘right’ \tab 11\\
šlɑ̄χd \tab [ʃlɑːçt] \tab  schlecht \tab ‘bad’ \tab 11\\
ɡnɑ̄χd \tab [knɑːçt] \tab  Knecht \tab ‘vassal’ \tab 11\\
\z
\z 

/x ɣ/ surface as the corresponding palatals after a front vowel in (\ref{ex:7:20b}, \ref{ex:7:20d}) or coronal sonorant consonant in (\ref{ex:7:20e}) by \isi{Velar Fronting-1}, otherwise (i.e. after a back vowel), they are realized as [x ɣ] in (\ref{ex:7:20a}, \ref{ex:7:20c}). The palatal fricative [ç] is a quasi-phoneme (/ç/) after the one back vowel [ɑː] (/ɑː/) in \REF{ex:7:20f}. As noted above, that opaque palatal (quasi-phoneme) arose when the etymological front vowel preceding it ([e] /e/) restructured to [ɑː] (/ɑː/) by \isi{Vowel Retraction} (=\ref{ex:7:4a}).

In the \il{East Hessian}EHes variety documented in the communities of the Rhön Valley (\ipi{Rhöntal}; \citealt{Glöckner1913}; \mapref{map:11}) the two dorsal fricatives [x ç] exhibit a pattern of distribution represented by the data in \REF{ex:7:21}:  [x] surfaces after a back vowel in (\ref{ex:7:21a}) and [ç] after a front vowel in (\ref{ex:7:21b}) or coronal sonorant consonant in (\ref{ex:7:21c}). It is clear from the original source that ⟦a⟧ and ⟦aa⟧ represent low front vowels (=[æ æː]) and that ⟦ɑ⟧ and ⟦ɑɑ⟧ are low back vowels (=[ɑ ɑː]). The most significant examples are the ones in (\ref{ex:7:21d}, \ref{ex:7:21e}), which reveal that palatal [ç] surfaces after the long low back vowel [ɑː]. [x] or [ç] surface optionally after [ɑː] derived historically from [e] in (\ref{ex:7:21d}), but only [ç] occurs after the [ɑː] deriving from earlier [ei] in (\ref{ex:7:21e}). The optionality in \REF{ex:7:21d} is speaker-dependent.\footnote{{Glöckner gathered his data from speakers in a variety of communities living in a broad region; hence, the speaker-dependent variation referred to here is probably a factor of geography. [ɣ] does not occur in the dialect as it was described in 1913; historical} \textrm{\textsuperscript{+}}\textrm{[ɣ] (/ɣ/) restructured to /x/, which regularly underwent fronting in words like [rɛːçl̩] ‘rule’ and otherwise surfaces as [x] in items like [foːxl̩] ‘bird’. A small number of items in the original source contain palatal [ʝ] in a word-internal onset after a front vowel, but I do not take these examples into consideration below.} }


\TabPositions{.2\textwidth, .4\textwidth, .6\textwidth, .8\textwidth}
\ea%21
\label{ex:7:21}Dorsal fricatives in the \ipi{Rhöntal}:
\ea\label{ex:7:21a} ɡərūx \tab [gəruːx] \tab Geruch \tab ‘smell’ \tab 31\\
brux \tab [brux] \tab brauchen \tab ‘need\textsc{{}-inf}’ \tab 43\\
fōxl̥ \tab [foːxl̩] \tab Vogel \tab ‘bird’ \tab 28\\
bǭx \tab [bɔːx] \tab Buch \tab ‘book’ \tab 92\\
kǫx \tab [kɔx] \tab kochen \tab ‘cook\textsc{{}-inf}’ \tab 29\\
sɑx \tab [sɑx] \tab Sache \tab ‘thing’ \tab 91\\
boux \tab [boux] \tab Bauch \tab ‘stomach’ \tab 44\\
\ex\label{ex:7:21b} iχ \tab  [iç] \tab ich \tab ‘I’ \tab 92\\
wīχ \tab [βiːç] \tab Wiege \tab ‘cradle’ \tab 24\\
füχd \tab [fyçt] \tab Feuchte \tab ‘humidity’ \tab 46\\
rẹ̄χl̥ \tab [reːçl̩] \tab Regel \tab ‘rule’ \tab 21\\
lẹχd \tab [leçt] \tab Licht \tab ‘light’ \tab 51\\
swöχ \tab [sβøç] \tab Schwäche \tab ‘weakness’ \tab 18\\
fraχ \tab  [fræç] \tab frech \tab ‘impudent’ \tab 92\\
šlaaχd \tab [ʃlæːçt] \tab schlecht \tab ‘bad’ \tab 21\\
\ex\label{ex:7:21c} ɡwaarχ \tab [kβæːrç] \tab quer \tab ‘across’ \tab 21\\
\ex\label{ex:7:21d} blɑɑχ, blɑɑx \tab [blɑːç], [blɑːx] \tab Blech \tab ‘tin’ \tab 22\\
bɑɑχ, bɑɑx \tab [bɑːç], [bɑːx] \tab Pech \tab ‘misfortune’ \tab 22
\ex\label{ex:7:21e} wɑɑχ \tab [βɑːç] \tab weich \tab ‘soft’ \tab 58
\z
\z 

I account for the optionality in \REF{ex:7:21d} as follows: I postulate two groups of speakers (Variety A and Variety B). For speakers of Variety A [x] occurs after all back vowels with the exception of [ɑː] (=\ref{ex:7:21a}), and [ç] surfaces after coronal sonorant consonants (=\ref{ex:7:21b}, \ref{ex:7:21c}) or after [ɑː] (=\ref{ex:7:21e} and the [ç] realization in \ref{ex:7:21d}). Speakers of Variety B have [x] after all back vowels, including [ɑː] (=\ref{ex:7:21a} and the [x] realization in \ref{ex:7:21d}) and [ç] after coronal sonorant consonants (=\ref{ex:7:21b}, \ref{ex:7:21c}) or [ɑː] (=\ref{ex:7:21e}). For Variety B the two fricatives [ç] and [x] contrast after [ɑː]. I do not discuss that type of example because similar case studies are dealt with at length in \chapref{sec:8} (for word-initial position) and in \chapref{sec:9} (for postsonorant position).

For Variety A the two fricatives [x] and [ç] do not contrast. As in \ipi{Rauschenberg} (recall \ref{ex:7:20f}) there is a historical reason for the nonoccurrence of [ç] after [ɑː]: First, etymological [ɑː] (/ɑː/) was replaced by [ɔː] (/ɔː/); e.g. [nɔːx] ‘after’ (cf. MHG \textit{nāch}), or [ɔ] (/ɔ/); [dɔxt] ‘wick’ (cf. MHG \textit{tāht}). Second, the vowel [ɑː] (/ɑː/) in the dialect as it was described in 1913 derived historically from a front vowel, namely [e] (/e/) in \REF{ex:7:21d} and [ei] (/ei/) in \REF{ex:7:21e}.

Palatal [ç] after a front vowel in (\ref{ex:7:21b}) or coronal sonorant consonant in (\ref{ex:7:21c}) derives synchronically from /x/ by \isi{Velar Fronting-1} and otherwise surfaces as [x] in (\ref{ex:7:21a}). The opaque palatal fricative [ç] is a quasi-phoneme (/ç/) for those Variety A speakers who have that sound after the back vowel [ɑː] in \REF{ex:7:21d} and for the examples with [ç] in \REF{ex:7:21e}. As noted above, the quasi-phoneme /ç/ arose when the etymological front vowel preceding it (/e/ or /ei/) restructured to [ɑː] (/ɑː/) by \isi{Vowel Retraction} (=\ref{ex:7:4a}).\footnote{\citet{Glöckner1913} also includes a number of examples in which palatal [ç] (=⟦χ⟧) surfaces after the diphthong [ɔə] (=⟦ǫə⟧, e.g. ⟦ǫəχd⟧ ‘eight’). Since the diphthong in question consists of two back vowels there is an apparent conundrum because the etymological vowel ([ɑ]) was back, e.g. ⟦ǫəχd⟧ (cf. MHG \textit{aht}). I hold that the original vowel [ɑ] (/ɑ/)  underwent a restructuring to a diphthong ending in a front vowel (e.g. [ɔi]/[ɔɪ]), at which point the /x/ following that vowel surfaced as a palatal allophone by \isi{Velar Fronting-1}. When the front vowel in that diphthong was restructured to one ending in \isi{schwa} ([ɔə] /ɔə/), the following palatal was quasi-phonemicized. Evidence for the intermediate stage whereby [ɑ] (/ɑ/) changed to a diphthong ending in a front vowel is attested in the \il{Central Hessian}CHes variety spoken in \ipi{Weidenhausen} (\citealt{Friebertshäuser1961}; \mapref{map:11}) discussed in \sectref{sec:9.2}.}

In the \il{Thuringian}Thrn dialect of \ipi{Sondershausen} (\citealt{Schirmer1932}; \mapref{map:12}) the two dorsal fricatives [x] and [ç] never contrast. As illustrated in (\ref{ex:7:22a}, \ref{ex:7:22b}), the velar occurs after a back vowel and the palatal after a front vowel. [ɣ ʝ] do not occur in postsonorant position because the historical source for those sounds (\ili{WGmc} \textsuperscript{+}[ɣ]) was restructured to /x/, which is realized as [x] after a back vowel, e.g. [duːxənt] ‘virtue’ (=⟦dūxənt⟧), and as [ç] after a coronal sonorant, e.g. [iːçəl] ‘hedgehog’ (=⟦īχəl⟧). Two contexts for quasi-phonemes are (i) after a noncoronal consonant in (\ref{ex:7:22c}) or (ii) in word-initial position before \isi{schwa} in (\ref{ex:7:22d}).

\ea%22
\label{ex:7:22}Dorsal fricatives in \ipi{Sondershausen}:
\ea\label{ex:7:22a} būx \tab  [buːx] \tab Buch \tab ‘book’ \tab 65\\
lox \tab [lox] \tab Loch \tab ‘hole’ \tab 65\\
dɑ̊x \tab [dɑx] \tab Dach \tab ‘roof’ \tab 65\\
\ex\label{ex:7:22b} rīχ \tab  [ʀiːç] \tab reich \tab ‘rich’ \tab 65\\
bręχə \tab [bʀɛçə] \tab  brechen \tab ‘break\textsc{{}-inf}’ \tab 65\\
blæχ \tab [blæç] \tab Blech \tab ‘tin’ \tab 65
\ex\label{ex:7:22c} ɡ\={æ}fχ \tab [kæːfç] \tab Käfig \tab ‘cage’ \tab 14
\ex\label{ex:7:22d} χəsiχtə \tab [çəsiçtə] \tab Gesicht \tab ‘face’ \tab 18
 \z
\z 

The item in \REF{ex:7:22c} exemplifies the deletion of an etymological front vowel (\isi{Syncope}), while the one in \REF{ex:7:22d} shows the effects of \isi{Vowel Reduction}. The historical source for word-initial [ç] in \REF{ex:7:22d} is \ili{WGmc} \textsuperscript{+}[ɣ]. In word-initial position before any sound other than \isi{schwa}, that etymological fricative is realized as [g], e.g. [gift] ‘poison’.

\begin{map}
% \includegraphics[width=\textwidth]{figures/VelarFrontingHall2021-img018.png}
\includegraphics[width=\textwidth]{figures/Map12_7.2.pdf}
\caption[Thuringian, Upper Saxon, and North Upper Saxon-South Markish]{Thuringian (\il{Thuringian}Thrn), Upper Saxon (\il{Upper Saxon}USax), and North Upper Saxon-South Markish (\il{North Upper Saxon-South Markish}NUSax-SMk). Squares indicate postsonorant velar fronting. 1=\citet{Schultze1874}, 2=\citet{Liesenberg1890}, 3=\citet{Flex1893}, 4=\citet{Frank1898}, 5=\citet{Trebs1899}, 6=\citet{Hennemann1901}, 7=\citet{Hentrich1905}, 8=\citet{Daube1906}, 9=\citet{KürstenBremer1910}, 10=\citet{Kürsten1910,Kürsten1911}, 11=\citet{Rasch1912}, 12=\citet{Hankel1913}, 13=\citet{Hentrich1920}, 14=\citet{Rudolph1924}, 15=\citet{Schirmer1932}, 16=\citet{Dietrich1957}, 17=\citet{Spangenberg1962}, 18=\citet{Spangenberg1974,Spangenberg1989}, 19=\citet{Guentherodt1982} (\ipi{Dudenrode}), 20=\citet{Guentherodt1982} (\ipi{Netra}), 21=\citet{Harnisch1987}, 22=\citet{Weldner1991}, 23=\citet{Spangenberg1998}, 24=\citet{Goepfert1878}, 25=\citet{Albrecht1983}, 26=\citet{Hertel1887}, 27=\citet{Philipp1897}, 28=\citet{Hausenblas1898}, 29=\citet{Lang1906}, 30=\citet{Pompé1907}, 31=\citet{Bremer1909}, 32=\citet{Hausenblas1914}, 33=\citet{Große1955}, 34=\citet{Große1957}, 35=\citet{Protze1957}, 36=\citet{Schönfeld1958}, 37=\citet{Fleischer1961}, 38=\citet{Bergmann1965}, 39=\citet{Becker1969}, 40=\citet{BethgeBonnin1969}, 41=\citet{KahnWeise2013}, 42=\citet{Goessgen1902}, 43=\citet{Bischoff1935}, 44=\citet{Kieser1963}, 45=\citet{Seibicke1967}, 46=\citet{Krug1969}, 47=\citet{BethgeBonnin1969}, 48=\citet{Stellmacher1973}, 49=\citet{Langner1977}, 50=\citet{Schönfeld1986}, 51=\citet{Schönfeld2001}.}\label{map:12}
\end{map}

\section{{Discussion}}\label{sec:7.4}

I consider and reject three alternative treatments for the case studies given in this chapter, all of which have in common that they eschew quasi-phonemes and treat the rules relating velars and palatals as allophonic operations and not as quasi-neutralizations (\sectref{sec:7.4.1}--\sectref{sec:7.4.3}). Those alternative treatments qualify as straw man analyses, although it needs to be stressed that formally similar treatments have been applied to independent sets of examples in German as well as other languages. I conclude this section (\sectref{sec:7.4.4}) by considering and rejecting \citegen{Kiparsky2015} claim that quasi-phonemes arise before the original conditioning factor was eliminated.

In all of the case studies discussed in this chapter the \isi{palatal quasi-phoneme} is adjacent to a noncoronal segment, namely before or after the dorsal rhotic (/ʀ/), after a full back vowel (e.g. /æu/, /ɑ/), or before or after \isi{schwa} (/ə/). I refer below to the noncoronal segments adjacent to quasi-phonemes as NCSs.

\subsection{Analysis A: Counterbleeding opacity}\label{sec:7.4.1}\is{counterbleeding order}

According to this alternative treatment NCSs are phonologically [coronal]. That [coronal] feature then spreads from a NCS to the adjacent velar fricative (/x/ or /ɣ/) by some version of velar fronting, and a later operation deletes [coronal] from the NCS. The treatment described here can potentially be applied to any of the NCSs referred to above. As a representative example, I consider the diphthong /æu/ of Kreis \ipi{Lippe} from (\ref{ex:7:13h}). In the treatment depicted in \REF{ex:7:23a}, [æu] is analyzed as a diphthong ending in a [coronal] sound in the underlying representation (/æi/), which shifts to [æu] by a rule I refer to as /i/-Retraction (/æi/→[æu]). Kreis \ipi{Lippe} does not possess the surface diphthong [æi].\largerpage

\ea%23
\label{ex:7:23}Alternative treatment for Kreis \ipi{Lippe} (rejected):\smallskip\\
\begin{multicols}{2}\raggedcolumns
\ea\label{ex:7:23a}
\begin{tabular}[t]{@{}ll@{}}
               & /æiɣən/ \\
Vel Fr-1       &  æiʝən  \\
/i/-Retraction &  æuʝən  \\
               &  [æuʝən]\\
               &  ‘own’  \\
\end{tabular}\columnbreak
\ex\label{ex:7:23b}
\begin{tabular}[t]{@{}ll@{}}
                 &  /æiɣən/\\
 /i/-Retraction  &  /æuɣən/\\
  Vel Fr-1       &  ---    \\
                 & *[æuɣən]\\
                 & \\
  \end{tabular}
\z
\end{multicols}
\z

Observe that the correct output in \REF{ex:7:23a} can only be obtained if /i/-Retraction \isi{counterbleeds} \isi{Velar Fronting-1} (Vel Fr-1) in the synchronic phonology. The relationship is \isi{counterbleeding} because the reverse ordering in \REF{ex:7:23b} requires /i/-Retraction to \isi{bleed} \isi{Velar Fronting-1}. Note too that the \isi{counterbleeding} ordering involves an \isi{overapplication} of \isi{Velar Fronting-1} because the front vowel trigger for [ç] in the phonetic representation is not present on the surface.

Although no study to my knowledge has proposed the specific treatment in \REF{ex:7:23a} to the Kreis \ipi{Lippe} data, many phonologists endorse similar analyses for phenomena in other languages. Examples in early generative phonology are easy to come by, e.g. \citet{ChomskyHalle1968} and many other authors writing during the 1970s. More recently, \citet{Calabrese2005} proposes a derivational model with \isi{counterbleeding} orderings involving \isi{overapplication}. For example, in his treatment of \ili{Icelandic}, \citet[38--41]{Calabrese2005} follows \citet{Anderson1981} in deriving palatal stops from underlying velars in the context before front vowels (\isi{Velar Palatalization}). For example, /k/ surfaces as [c] in [cɪftɑ] ‘marry\textsc{{}-inf}’ (from /kɪftɑ/) but as [k] in [kouːmyr] ‘palate’ (from /kouːmyr/). In order to account for the occurrence of palatal stops before the diphthong [ɑiː], Calabrese analyzes that diphthong as a low front monophthong. Given that treatment, [c] is derived from /k/ before that low front monophthong because Velar Palatalization is ordered before the change from a low front monophthong to [ɑiː] \isi{Low Vowel Diphthongization}). For example, the /ɑ/ in /kɑl-i/ ‘freeze-\textsc{subj}.\textsc{1}\textsc{sg}’ shifts to the low front vowel {\textbar}æ{\textbar} by \isi{Umlaut}, which \isi{feeds} \isi{Velar Palatalization}, at which point \isi{Low Vowel Diphthongization} applies, i.e. /kɑl-i/→{\textbar}kæl-i{\textbar}→{\textbar}cæl-i{\textbar}→{\textbar}cɑil-i{\textbar}, which is ultimately realized as [cʰɑiːli]. The important point is that \isi{Low Vowel Diphthongization} \isi{counterbleeds} Velar Palatalization.

Recall from \sectref{sec:5.4} that there is no evidence in the present survey on German dialects that any version of velar fronting is counterbled synchronically by another rule. A significant finding of \chapref{sec:5} is that \isi{opacity} involving the fronting of velars in German dialects involves synchronic \isi{counterfeeding} orders but not synchronic \isi{counterbleeding} orders. That point aside, I also question the wisdom behind rules like /i/-Retraction in \REF{ex:7:23a}, which exemplify an \textsc{absolute} \textsc{neutralization} (see \citealt{KaisseShaw1985} and other authors in the \isi{Lexical Phonology and Morphology} framework). What the treatment in \REF{ex:7:23a} amounts to is \isi{Velar Fronting-1} overapplying because it is counterbled by a rule of \isi{absolute neutralization} whose sole purpose is to undo a representation that never surfaces. The advantage of the present treatment is that it does not require \isi{counterbleeding} \isi{opacity} involving rules eliminating fictional segments, as in \REF{ex:7:23a}.

\subsection{Analysis B: NCCs are permanently [coronal]}\label{sec:7.4.2}
\begin{sloppypar}
The objections to \isi{counterbleeding} \isi{opacity} could potentially be mitigated by adopting a treatment in which the NCS in question is underlyingly [coronal] and remains [coronal] throughout the phonological component. I apply Analysis B to the NCS /ʀ/ because that type of treatment has been made in the published literature discussed below. The arguments I level against analyzing /ʀ/ in that way can be extended to a treatment of other NCSs as well.
\end{sloppypar}

Two representations for /ʀ/ according to Analysis B are presented in \REF{ex:7:24}. Both of those structures have in common that the surface dorsal sound [ʀ] is analyzed as phonologically [coronal] in the underlying representation. \REF{ex:7:24a} is a singleton coronal segment, while \REF{ex:7:24b} is a complex corono-dorsal sound.

\ea%24
\label{ex:7:24}Alternative representations for /ʀ/ (rejected):
\begin{multicols}{2}\raggedcolumns
\ea Singleton coronal:\\\label{ex:7:24a}
  \begin{forest}     
  [\avm{[+cons\\+son\\−nasal]}
     [\avm{[coronal]}]
  ]
  \end{forest}    
\ex   Complex corono-dorsal:\\\label{ex:7:24b}
      \begin{forest}
        [\avm{[+cons\\+son\\−nasal]}
           [\avm{[coronal]}]
           [\avm{[dorsal]}]
        ]
      \end{forest}
\z
\end{multicols} 
\z

Given either representation in \REF{ex:7:24} the surface palatal [ç] in the context of /ʀ/ can be analyzed as velar /x/ and not as the quasi-phoneme /ç/. For example, the word [çʀunt] ‘reason’ from \ipi{Reinhausen} in \REF{ex:7:8d}, which I analyze as underlyingly /çʀunt/, can be reanalyzed according to Analysis B as /xʀunt/. The /x/ in that type of example surfaces as [ç] by \isi{Wd-Initial Velar Fronting-6} because the liquid is [coronal], as in \REF{ex:7:24}.

Representations similar to the ones in \REF{ex:7:24} have been posited for surface dorsal liquids in both the cross-linguistic literature and in the literature on German phonology. For example, \citet{Blevins1994} examines the phonological patterning of the velar (dorsal) lateral /ʟ/ in several \ili{Trans-New Guinean} languages spoken in Papua New Guinea. She shows that [ʟ] alternates with simplex (alveolar) coronals such as [t] and [l] in languages such as \ili{Yagaria}, \ili{Kuman}, and \ili{Kanite}. Within Gmc, \citet{Hall2009b} presents material from the \il{South Bavarian}SBav variety spoken in \ipi{Imst} (\citealt{Schatz1897}; \mapref{map:3}), in which the dorsal rhotic [ʀ] patterns phonologically with the alveolar coronal stop [d]. Both Blevins and Hall argue that the phonetically dorsal liquids in question are phonologically [coronal], as in \REF{ex:7:24a}.

The structure in \REF{ex:7:24b} is akin to the universal representation for liquids proposed by \citet{WalshDickey1997}. \citet{Glover2014} argues that \il{Standard German}StG /ʀ/ is underlyingly underspecified for place features and that default rules create the complex co\-ro\-no-dor\-sal structure in \REF{ex:7:24b}.

Although it might seem appealing to adopt a treatment whereby palatals are created from /x/ in the context of /ʀ/, the disadvantages both representations in \REF{ex:7:24} have is that they do not derive independent support. The argument for the treatment of /ʀ/ as [coronal] in \il{Standard German}StG is based solely on the occurrence of the surface palatal fricative [ç] after that sound. Significantly, there is no evidence that \ipi{Reinhausen} /ʀ/ is coronal if /ʀ/ is situated in any context other than word-initial position after [ç]. Citing the analysis of \citet{Hall1995}, \citet{Glover2014} argues that phonotactic evidence from \il{Standard German}StG corroborates an analysis of /ʀ/ as [coronal]. However, the phonotactic evidence referred to here only holds for postvocalic consonant clusters where /ʀ/ occupies the first slot. No phonotactic evidence supports \REF{ex:7:24} for /ʀ/ in a context other than the first slot in a sequence of two postvocalic consonants, e.g. word-initial /ʀ/, /ʀ/ between vowels etc. A more serious drawback with that type of argumentation is that same phonotactics involving postvocalic consonant clusters hold in dialects like \ipi{Schieder-Schwalenberg}, where /ʀ/ demonstratively patterns as a noncomplex (singleton) [dorsal] fricative (recall \ref{ex:7:9b}).

An advocate of either \REF{ex:7:24a} or \REF{ex:7:24b} might claim that the quasi-phonemes in my analysis do not have independent support either, but this contention is not correct. Quasi-phonemes are surface palatals that must be analyzed as underlying palatals because they do not appear in a context where they can be derived by any version of velar fronting. The highly specific contexts for quasi-phonemes (e.g. word-initial position before /ʀ/ in \ipi{Reinhausen} in \ref{ex:7:8d}) derive diachronic support: The reason quasi-phonemes appear synchronically only when adjacent to certain NCSs like /ʀ/ is that those NCSs were once phonologically [coronal] at an earlier historical stage. That feature was then transferred to the dorsal fricative and created the new quasi-phoneme when the original [coronal] trigger lost the feature [coronal] by sound change (=\ref{ex:7:4}). But the same point cannot be made for the representations in \REF{ex:7:24} because the feature [coronal] in those two structures is present regardless of whether or not those representations are adjacent to a dorsal fricative.

\subsection{Analysis C: Underlying palatals but no underlying velars}\label{sec:7.4.3}

A final alternative to palatal quasi-phonemes is to maintain that all instances of surface velars and surface palatals derive from underlying palatals. Analysis C is illustrated in \REF{ex:7:25} with six representative words from \ipi{Elspe} from \REF{ex:7:6} and the alternative rule in \REF{ex:7:26}.

\ea\label{ex:7:25}Alternative analysis for \ipi{Elspe} (rejected):
\ea\label{ex:7:25a}/çɔlt/   \tab → \tab  [xɔlt]   \tab‘gold’
\ex\label{ex:7:25b}/çɪstan/ \tab → \tab [çɪstan]  \tab ‘yesterday’
\ex\label{ex:7:25c}/çrɛɔt / \tab → \tab [çrɛɔt]   \tab‘large’
\ex\label{ex:7:25d}/ʃçɑːp/  \tab → \tab [ʃxɑːp]   \tab‘cabinet’
\ex\label{ex:7:25e}/ʃçɛlə/  \tab →  \tab[ʃçɛlə]   \tab ‘bowl’
\ex\label{ex:7:25f}/çəvɑːɐ/ \tab  → \tab  [çəvɑːɐ]\tab  ‘aware’
\z 
\z
\ea %26
\label{ex:7:26}\isi{Wd-Initial Palatal Retraction} (rejected):\smallskip\\
/ç/ → [x] / \textsubscript{wd}[ (C) {\longrule}{\longrule} back vowel
\z 

Since the sound triggering \isi{Wd-Initial Palatal Retraction} is phonologically back (i.e. [dorsal]), the underlying palatal /ç/ in (\ref{ex:7:25a}, \ref{ex:7:25d}) undergoes it and correctly surfaces as [x]. /ç/ in the neighborhood of \isi{schwa} in \REF{ex:7:25f} fails to undergo \isi{Wd-Initial Palatal Retraction} given the representation of \isi{schwa} that is placeless and therefore correctly surfaces as [ç]. Finally, /ç/ before front sounds in (\ref{ex:7:25b}, \ref{ex:7:25c}, \ref{ex:7:25e}) surfaces without change as [ç].

Although the alternative treatment for \ipi{Elspe} in \REF{ex:7:25} and \REF{ex:7:26} works technically, I reject it because it cannot be extended successfully to other German dialects with quasi-phonemes. As a representative example, consider the reanalysis in \REF{ex:7:27} and \REF{ex:7:28} of the realization of [ɣ ʝ] in Kreis \ipi{Lippe} from (\ref{ex:7:13}). Note that there is an underlying palatal /ʝ/, but no /ɣ/.

\ea%27
\label{ex:7:27}Alternative analysis for Kreis \ipi{Lippe} (rejected):\\

\ea\label{ex:7:27a}/ɑuʝə/  \tab →  \tab [ɑuɣə]   \tab ‘eye’
\ex\label{ex:7:27b}/spɪʝən/\tab  → \tab [spɪʝən]\tab    ‘spout-\textsc{inf}’
\ex\label{ex:7:27c}/sɔʀʝə/ \tab →  \tab[sɔʀʝə]  \tab ‘sorrow’
\ex\label{ex:7:27d}/æuʝən/ \tab →  \tab[æuʝən]  \tab  ‘own’
\z 
\ex%28
\label{ex:7:28}Palatal Retraction (rejected):\\
  /ʝ/ → [ɣ] / back vowel {\longrule}{\longrule}
\z 

\isi{Palatal Retraction} in \REF{ex:7:28} correctly produces [ɣ] in example \REF{ex:7:27a}, but that same process cannot account for the palatal after [ʝ] in \REF{ex:7:27d}.

[ç] and [ʝ] in the examples discussed in the present chapter -- regardless of variety -- are uncontroversially the product of historical rules that fronted etymological velars. Seen in this light, the proposed diachronic treatment whereby underlying palatals (quasi-phonemes) emerge in the neighborhood of back sounds that were once front is not controversial. Analysis C is an attempt to eschew \isi{opacity} (=palatal quasi-phonemes) by analyzing all instances of dorsal fricatives as underlying palatal in a synchronic treatment. Since velar fronting (and not \isi{Palatal Retraction}) was uncontroversially the correct historical process for all German dialects, Analysis C presupposes that \isi{rule inversion} (\citealt{Vennemann1972}, \citealt{McCarthy1991}, \citealt{Blevins2004}, \citealt{Hall2009a}) has taken place in every variety of German with quasi-phonemes. Although a rule of \isi{Palatal Retraction} for word-initial position akin to the one in \REF{ex:7:26} is posited for an \il{Eastphalian}Eph variety in \sectref{sec:8.5}, it is the only one of its nature discovered in the present survey of velar fronting in German dialects.

Analysis C is directly related to one of the controversial research questions discussed in the literature on the distribution of dorsal fricatives in \il{Standard German}StG (\sectref{sec:1.2}): Do the two sounds [x] and [ç] derive synchronically from /x/ or /ç/? Since that question can only be addressed after all case studies of German dialects have been presented, I delay discussion until \sectref{sec:17.3.3}. In that section I demonstrate that the evidence is overwhelming that palatals derive from velars and not the other way around. That conclusion cannot be reconciled with Analysis C.

\subsection{\citegen{Kiparsky2015} treatment of quasi-phonemes}\label{sec:7.4.4}

\citet{Kiparsky2015} offers an analysis of \isi{i-Umlaut} in the history of German that relies crucially on the notion of vocalic quasi-phonemes. It is instructive to consider his analysis because his quasi-phonemes are argued to possess a property that cannot be extended to the palatal quasi-phonemes endorsed in this chapter.

Kiparsky’s concern is how and why phonemes originate (\isi{phonemicization}) and why they are sometimes lost (merger). In his treatment of the \isi{phonemicization} of front rounded vowels in the history of German summarized below, Kiparsky makes crucial use of the notion of quasi-phoneme. In Kiparsky’s system, quasi-phonemes are defined in terms of two binary parameters, which he dubs “contrastivenessˮ and “distinctivenessˮ. Contrastiveness relates to whether or not the distribution of the sounds in question is contextually predictable; distinctiveness is a perceptual notion which refers to whether or not native speakers regard the sounds in question as phonetically different. The traditional definition of phonemes requires that the sounds in question to be both contrastive (contextually unpredictable) and distinctive (perceived as different), while traditional allophones are neither contrastive (because they are in complementary distribution) nor distinctive (because native speakers are typically unaware of the difference between allophones of a given phoneme).

The two properties referred to above predict the existence of two types of sounds that are unexpected in traditional phonemic theory. First, there may be sounds that are distinctive without being contrastive (quasi-phonemes) and second, there could be sounds that are contrastive without being distinctive (near contrasts). The four logical possibilities are summarized in \tabref{tab:fromex:7:29}.

According to Kiparsky, the change from allophones to phonemes depicted in \REF{ex:7:1} involves an intermediate stage, namely quasi-phonemes. His claim is illustrated in the following example from the history of German.

The historical rule of \isi{i-Umlaut} (Chapters~\ref{sec:3}--\ref{sec:4}) fronted back vowels before [i] or [j] in the following syllable. At a later stage the two triggers were eliminated: [j] was lost, while [i] -- like all other unstressed vowels -- was restructured to \isi{schwa} ([ə] /ə/) by \isi{Vowel Reduction}. The examples in (\ref{ex:7:30b}, \ref{ex:7:30c}) -- adapted from \citet{Kiparsky2015} -- illustrate the effects of both \isi{i-Umlaut} and \isi{Vowel Reduction}. Example \REF{ex:7:30a} shows that the original back vowel (/uo/ [uo]) is retained without change when there is no suffix present.


\TabPositions{.15\textwidth, .3\textwidth, .35\textwidth, .4\textwidth, .6\textwidth, .8\textwidth}
\ea%30
\NumTabs{6}
\label{ex:7:30}
                 \tab  OHG \tab     \tab     MHG   \\
\ea\label{ex:7:30a}\relax [huot]  \tab (/huo/)    \tab > ~~ [huot]  \tab (/huo/)    \tab ‘hat’
\ex\label{ex:7:30b}\relax [huote] \tab (/huot-e/) \tab > ~~ [huotə] \tab (/huot-ə/) \tab ‘hat-\textsc{dat}.\textsc{sg}’
\ex\label{ex:7:30c}\relax [hyeti] \tab (/huot-i/) \tab > ~~ [hyetə] \tab (/hyet-ə/) \tab ‘hat-\textsc{pl}’
\z 
\z 

The most important example is \REF{ex:7:30c}: That item illustrates that the trigger for \isi{i-Umlaut} (/i/) was restructured to /ə/, at which point the conditioning environment for \isi{i-Umlaut} was no longer present. In the traditional literature (e.g. \citealt{Twaddell1938}) it is assumed that the loss of the conditioning environment for the earlier allophonic rule of \isi{i-Umlaut} triggered the \isi{phonemicization} of originally allophonic front vowels like [ye] (from /uo/) to phonemic front vowels (/ye/) in \ili{MHG}. That change is depicted in \REF{ex:7:31}. Note that [ye] and [uo] contrasted in the context before \isi{schwa} in \ili{MHG}.\largerpage

\ea%31
    \label{ex:7:31}
    \begin{tabular}[t]{@{}lll@{}}
 \relax /huot-i/ &      /huot-e/  &           \\
 \relax [huoti]  &  [huote]       &  pre-OHG  \\\tablevspace
 \relax /huot-i/ &      /huot-e/  &           \\
 \relax [hyeti]  &  [huote]       &   \ili{OHG}     \\\tablevspace
 \relax /hyet-ə/ &      /huot-ə/  &           \\
 \relax [hyetə]  &  [huotə]       &  \ili{MHG}      \\
 \relax  ‘hat-\textsc{pl}’  &     ‘hat\textsc{{}-dat.sg}’\\
   \end{tabular}
\z 

\begin{table}%29
\caption{\label{tab:fromex:7:29}Distinctiveness and contrastivity \citep{Kiparsky2015}}
\begin{tabular}{lll}
\lsptoprule
             & contrastive & noncontrastive\\\midrule
distinctive & phoneme & quasi-phoneme\\
nondistinctive & near contrast & allophone\\
\lspbottomrule
\end{tabular}
\end{table}

The question \citet{Kiparsky2015} ponders (see also \citealt{Liberman1991}) is why the underlying representation for \ili{OHG} [ye] (/uo/) was not retained as /uo/ after the conditioning environment was eliminated, in which case the (nominative) plural \ili{MHG} /huot-ə/ would have surfaced as *[huotə]. Since that type of change did not occur, Kiparsky proposes that the future front vowel phonemes (e.g. [ye]) were quasi-phonemicized -- they became perceptually distinctive, as depicted in \tabref{tab:fromex:7:29} -- prior to the loss of the conditioning environment (\isi{Vowel Reduction}); see also the discussion in \citet[409ff.]{Janda2005}. The approach envisioned by Kiparsky would therefore reanalyze the historical progression in \REF{ex:7:31} as in \REF{ex:7:32}:\largerpage

\ea%32
    \label{ex:7:32}
    \begin{tabular}[t]{@{}lll@{}}
 \relax /huot-i/ &  /huot-e/ &                                        \\
 \relax [huoti]  &  [huote]  &  pre-OHG                               \\\tablevspace
 \relax /huot-i/ &  /huot-e/ &                                        \\
 \relax [hyeti]  &  [huote]  &  ([ye] and [uo] are allophones)        \\\tablevspace
 \relax /hyet-i/ &  /huot-e/ &                                        \\
 \relax [hyeti]  &  [huote]  &  ([ye] /ye/ is a quasi-phoneme)        \\\tablevspace
 \relax /hyet-ə/ &  /huot-ə/ &                                        \\
 \relax [hyetə]  &  [huotə]  &  ([ye] /ye/ and [uo] /uo/ are phonemes)\\
 \relax ‘hat-\textsc{pl}’   &   ‘hat\textsc{{}-dat.sg}’\\
 \end{tabular}
\z 

Note that three of the categories from \tabref{tab:fromex:7:29}, namely allophones, quasi-phonemes, and phonemes, are related historically in the sense that allophones become quasi-phonemes, which in turn become phonemes.

There are several differences between Kiparsky’s quasi-phonemes and my own. Recall from \sectref{sec:2.4.3} that I defined palatal quasi-phonemes as phonemic palatals which posses two properties: (a) they do not contrast with the corresponding velar, and (b) they have an opaque (\isi{counterbleeding}) history. For example, in the data from \ipi{Elspe} in \REF{ex:7:6}, word-initial [ç] in the context before \isi{schwa} ([ə]) is an underlying palatal (/ç/) that does not contrast with [x] and which has an opaque history schematized in \REF{ex:7:3a}. Although property (a) is the same in my treatment and in the one proposed by Kiparsky, property (b) is not because Kiparsky’s quasi-phonemes have a transparent history, as illustrated in \REF{ex:7:32}. That same historical derivation illustrates that Kiparsky’s quasi-phonemes occupy an intermediate stage between allophones and phonemes. However, as I point out below in \sectref{sec:9.4.2}, it is not always the case that phonemic palatals without palatal quasi-phonemes.\footnote{{The distinctness property as defined above (recall Table \ref{tab:fromex:7:29}) does not play a role in the present treatment of quasi-phonemes; in fact, I do not discuss this issue in any chapter of this book because it is not addressed in the original sources I cite.} }

The most significant difference between Kiparsky’s quasi-phonemes and my own is that palatal quasi-phonemes in my analysis must emerge \textit{after} the conditioning environment for velar fronting is lost and not before. The reason Kiparsky's treatment cannot be extended to my palatal quasi-phonemes that those palatals can indeed revert back to their original velars in other dialects. More than one example illustrating reversion were given earlier. For example, as shown in \tabref{extab:7:12}, the loss of the conditioning environment for velar fronting caused the original velar /x/ to become quasi-phonemicized to /ç/ before /ʀ/ (< /r/) in \ipi{Reinhausen}, but the palatal allophone [ç] (/x/) in \ipi{Schieder-Schwalenberg} reverted back to [x] /x/ in the same context. The same point can be illustrated by comparing the distribution of word-initial [x] and [ç] in two \il{Westphalian}Wph dialects discussed earlier, namely \ipi{Soest} (\sectref{sec:4.3}) and \ipi{Elspe} (\sectref{sec:7.1}). Three representative words from those two dialects are given for \ipi{Soest} in (\ref{ex:7:33}) and \ipi{Elspe} in (\ref{ex:7:34}). The three categories represent the three contexts “before a full back vowelˮ, “before schwaˮ, and “before a front vowelˮ.



\TabPositions{.2\textwidth, .4\textwidth, .6\textwidth, .8\textwidth}
\ea\label{ex:7:33}
\ea\label{ex:7:33a}\relax [xuət] \tab /xuət/ \tab gut \tab ‘good’
\ex\label{ex:7:33b}\relax [xədʊlt] \tab /xədʊlt/ \tab Geduld \tab ‘patience’
\ex\label{ex:7:33c}\relax [çɪstɐn] \tab /xɪstɐn/ \tab gestern \tab ‘yesterday’
\z 
\ex\label{ex:7:34}
\ea\label{ex:7:34a}\relax [xɔlt] \tab /xɔlt/ \tab Gold \tab ‘gold’
\ex\label{ex:7:34b}\relax [çəvɑːɐ] \tab /çəvɑːɐ/ \tab gewahr \tab ‘aware’
\ex\label{ex:7:34c}\relax [çɪstan] \tab /xɪstan/ \tab gestern \tab ‘yesterday’
\z 
\z 

The difference between the two dialects is the context before \isi{schwa}, which was originally a front vowel (/i/). Compare now three historical stages for \ipi{Soest} in (\ref{ex:7:35}a) and \ipi{Elspe} in (\ref{ex:7:35}b):

\ea%35
    \label{ex:7:35}
\begin{tabular}[t]{@{} *{7}{l} @{}}
a.&   /xɪstɐn/  &    /xidʊlt/  & b. & /xɪstan/       &  /xivɑːɐ/  &        \\
  &   [xɪstɐn]  &  [xidʊlt]    &    &  [xɪstan]      &[xivɑːɐ]    & Stage 1\\\tablevspace
  &   /xɪstɐn/  &    /xidʊlt/  &    &  /xɪstan/      &   /xivɑːɐ/ &        \\
  &   [çɪstɐn]  &  [çidʊlt]    &    &  [çɪstan]      &[çivɑːɐ]    & Stage 2\\\tablevspace
  &   /xɪstɐn/  &    /xədʊlt/  &    &  /xɪstan/      &   /çəvɑːɐ/ &        \\
  &  [çɪstɐn]   & [xədʊlt]     &    &  [çɪstan]      &[çəvɑːɐ]    & Stage 3\\
  & ‘yesterday’ &   ‘patience’ &    &    ‘yesterday’ &  ‘aware’   &        \\
\end{tabular}
\z 

At Stage 2 \isi{Vowel Reduction} had not yet reduced \isi{full vowels} to \isi{schwa}. Velar fronting was phonologized and therefore created the palatal [ç] before any front vowel. At Stage 3 \isi{Vowel Reduction} restructured unstressed vowels like /i/ to \isi{schwa} (/ə/). Significantly, the elimination of the conditioning environment -- the creation of /ə/ -- led to the emergence of the \isi{palatal quasi-phoneme} /ç/ in \ipi{Elspe}, but not in \ipi{Soest}, where the original underlying sound (/x/) is retained.\footnote{{Kiparsky's quasi-phonemes have been viewed critically in some of the recent literature (\citealt{RenwickLadd2016}). Those authors correctly point out that the notion of contrast is much more nuanced than what Kiparsky’s four-way classification in \tabref{tab:fromex:7:29} suggests. In particular, they demonstrate that the mid vowels of \ili{Italian} illustrate marginal contrasts that cannot be easily categorized in Kiparsky's terms. I do not discuss \citet{RenwickLadd2016} because it is not clear that their criticisms of Kiparsky's system in \tabref{tab:fromex:7:29} can be extended to the quasi-phonemes in my own analysis.}}

An issue relating to \REF{ex:7:35} is how those historical changes can be interpreted given a historical model whereby change is intergenerational and listener-driven (\sectref{sec:2.5}).

Consider first (\ref{ex:7:35}a), where transparent Stage 2 remains transparent at Stage 3: The speaker (P\textsubscript{1}) utters [çidʊlt] (from /xidʊlt/), but the listener (P\textsubscript{2}) misperceives the first vowel as \isi{schwa} ([ə]). (S)he therefore alters the pronunciation by replacing [i] with [ə]. P\textsubscript{2} correctly perceives the first segment in [çidʊlt] as a palatal, but (s)he also knows that [ç] does not occur before full back vowels. None of the words P\textsubscript{2}  has acquired begin with a palatal fricative before \isi{schwa}. (S)he assumes that there is an exceptionless ban on words beginning with [ç] followed by any back vowel -- full back vowels and \isi{schwa} -- and therefore substitutes the palatal fricative with the other dorsal fricative allophone [x]. Hence, the earlier pronunciation ([çidʊlt]) undergoes two modifications ([xədʊlt]), but only one of those modifications is the result of misperception. Equally important is that P\textsubscript{2} alters the underlying representation to one retaining the earlier /x/, while adopting the new vowel \isi{schwa}, i.e. /xədʊlt/.

Consider now (\ref{ex:7:35}b), where transparent Stage 2 becomes opaque Stage 3: The speaker (P\textsubscript{1}) utters [çivɑːɐ] ‘aware’ (from /xivɑːɐ/), but the listener (P\textsubscript{2}) misperceives the first vowel as \isi{schwa} and therefore alters the pronunciation by replacing [i] with [ə]. P\textsubscript{2} also correctly perceives the palatal fricative as palatal; hence, [ç] is retained. On the basis of other words P\textsubscript{2} has acquired (s)he knows that [ç] does not occur word-initially before full back vowels, but the word-initial back vowels in all of the words acquired do not include \isi{schwa}. P\textsubscript{2} therefore concludes that the fricative [ç] in word-initial position before \isi{schwa} is phonotactically legal because [çə] does not contrast with [xə]. The word is therefore pronounced as [çəvɑːɐ], but more significantly, P\textsubscript{2} posits a new underlying representation, namely /çəvɑːɐ/.

\section{Areal distribution of palatal quasi-phonemes}\label{sec:7.5}

The survey of German dialects in this chapter reveals that underlying palatals (quasi-phonemes) are well-attested in LG and CG. The case studies presented in Chapters~\ref{sec:8}--\ref{sec:10} indicate that palatal quasi-phonemes also occur in varieties of German not mentioned in the present chapter, e.g. \il{East Pomeranian}EPo and \il{Low Prussian}LPr (both ELG) as well as \il{Upper Saxon}USax and \il{High Prussian}HPr (both ECG). \tabref{tab:7:1} lists the varieties of German with palatal quasi-phonemes discussed in  \sectref{sec:7.2} and \sectref{sec:7.3} as well as a few additional ones I have found in the sources provided. The places listed below that have not been discussed are depicted on \mapref{map:9}, \mapref{map:10}, and \mapref{map:18}. All of the places in the first column of \tabref{tab:7:1} are plotted on \mapref{map:13}.

\begin{table}
\caption{\label{tab:7:1}Varieties of LG (upper box) and CG (lower box) with palatal quasi-phonemes (< \ili{WGmc} \textsuperscript{+}[k x ɣ]) in word-initial and/or postsonorant position.}
\begin{tabular}{lll}
\lsptoprule
Place & Dialect & Source\\\midrule
Kreis \ipi{Lippe} & \il{Westphalian}Wph & \citet{Hoffmann1887}\\
\ipi{Schieder-Schwalenberg} & \il{Westphalian}Wph & \citet{Böger1906}\\
\ipi{Elspe} & \il{Westphalian}Wph & \citet{Arens1908}\\
\ipi{Eilsdorf} & \il{Eastphalian}Eph & \citet{Block1910}\\
\ipi{Reinhausen} & \il{Eastphalian}Eph & \citet{Jungandreas1926,Jungandreas1927}\\
\ipi{Dorste} & \il{Eastphalian}Eph & \citet{Dahlberg1934,Dahlberg1937}\\
\ipi{Dingelstedt am Huy} & \il{Eastphalian}Eph & \citet{Hille1939}\\
\ipi{West Mecklenburg} & \il{Mecklenburgish-West Pomeranian}MeWPo & \citet{Kolz1914}\\
\ipi{Wermelskirchen} & \il{Ripuarian}Rpn & \citet{Hasenclever1905}\\
\ipi{Warmsroth} & \il{Moselle Franconian}MFr & \citet{Martin1922}\\
\ipi{Echternach} & \il{Moselle Franconian}MFr & \citet{Palgen1931}\\
\ipi{Rhöntal} & \il{East Hessian}EHes & \citet{Glöckner1913}\\
\ipi{Selters bei Weilburg} & \il{Central Hessian}CHes & \citet{Schwing1921}\\
\ipi{Marburg} & \il{Central Hessian}CHes & \citet{Spenter1964}\\
\ipi{Oberellenbach} & \il{North Hessian}NHes & \citet{Hofmann1926}\\
\ipi{Loshausen} & \il{North Hessian}NHes & \citet{Corell1936} \\
\ipi{Rauschenberg} & \il{North Hessian}NHes & \citet{Bromm1936}\\
\ipi{Seifhennersdorf} & \il{Silesian}Sln & \citet{Michel1891}\\
\ipi{Sebnitz} & \il{Silesian}Sln & \citet{Meiche1898}\\
\ipi{Bad Frankenhausen} & \il{Thuringian}Thrn & \citet{Frank1898}\\
\ipi{Buttelstedt} & \il{Thuringian}Thrn & \citet{KürstenBremer1910}\\
Southeast Thuringia & \il{Thuringian}Thrn & \citet{Kürsten1910,Kürsten1911}\\
\ipi{Vorerzgebirge} & \il{Upper Saxon}USax & \citet{Bergmann1965}\\
\ipi{Reimerswalde} & \il{High Prussian}HPr & \citet{KuckWiesinger1965}\\
\lspbottomrule
\end{tabular}
\end{table}

\begin{map}
% \includegraphics[width=\textwidth]{figures/VelarFrontingHall2021-img019.png}
\includegraphics[width=\textwidth]{figures/Map13_7.3.pdf}
\caption[Areal distribution of palatal quasi-phonemes]{Areal distribution of palatal quasi-phonemes. High German and Low German varieties with palatal quasi-phonemes (< \ili{WGmc} \textsuperscript{+}[k x ɣ]) in word-initial or post-sonorant position are indicated with white squares.}\label{map:13}
\end{map}

I make no claim that the list of places in \tabref{tab:7:1} is anywhere near being complete. Thus, a closer scrutiny of the sources listed in Appendix~\ref{appendix:b} should reveal many additional varieties that could be added to \tabref{tab:7:1} and included on a future revision of \mapref{map:13}.\footnote{One way of finding additional examples is to conduct an in-depth investigation of the regions affected by the sound changes listed in \REF{ex:7:4}. However, it needs to be stressed that the set of German dialects with palatal quasi-phonemes is not identical to the set of German dialects with the sound changes deleting the frontness feature discussed in this chapter. The reason is that those changes do not automatically result in the emergence of an underlying palatal because they can revert back to the original velar, as demonstrated in \tabref{extab:7:12}, \REF{ex:7:33}, and \REF{ex:7:34}.}

It is not difficult to find examples of underlying palatals (quasi-phonemes) like the ones discussed above in linguistic atlases. I conclude this section by considering the occurrence of underlying palatals due to \isi{Syncope} in (\ref{ex:7:4d}) and due to changes eliminating the frontness feature of an original /r/. I focus on two specific linguistic atlases illustrating the occurrence of sounds such as /ç/ in those two contexts.

According to ThürDA, there is an area in Thuringia possessing coda clusters with [ç], which arose via \isi{Syncope} in \REF{ex:7:4d}; recall the word [kæːfç] ‘cage’ (cf. \il{Standard German}StG [kɛːfɪç]) in \ipi{Sondershausen} (\il{Thuringian}Thrn) from \REF{ex:7:22c}, as well as the similar items from \il{Central Hessian}CHes in \REF{ex:7:18d} and \REF{ex:7:19f}. According to Map 26 in ThürDA (for the word \textit{Friedhof} ‘cemetery’) there are parts of central and west Thuringia where that word is realized as \textit{kirfich} or \textit{kerwich}, but to the south the word undergoes two modifications: First, the second vowel is syncopated, and second, the final two consonants metathesize, i.e. \textit{kirfich} > \textit{kirfch} > \textit{kirchf}. The commentary for Map 26 (Volume 2: 136) makes it clear that the consonant cluster is [ʀfç] after \isi{Syncope} and [ʀçf] after Metathesis. The same commentary notes the similarity between those clusters and clusters of obstruent plus [ç] in words, such as \textit{teigig} ‘doughy’ (⟦dēɡχ⟧=[deːgç]) and \textit{Teppich} ‘carpet’ (⟦dębχ⟧=[dɛbç]). The important point is that the palatal fricative [ç] in all of the examples mentioned must be an underlying palatal (/ç/) because it is not preceded by a coronal sonorant.\largerpage[-1]

Underlying palatals in the context of an original /r/ are documented in MRhSA for the \il{Moselle Franconian}MFr/\il{Rhenish Franconian}RFr dialect area. Since /r/ is one of the segments serving as a trigger for velar fronting any change that eliminates its frontness feature can induce the restructuring of an adjacent velar (/x/) to palatal (/ç/). Underlying palatals are documented on Map 224 in Volume 3 of MRhSA for \textit{durch} ‘through’. In \REF{ex:7:36} I list four places on that map with the respective phonetic representations. The reason I have chosen those particular places is that these are the ones listed in that source with [ç] after a back vowel.

\ea\label{ex:7:36}Palatal ([ç]) after an etymological /r/:
\ea\label{ex:7:36a}Barweiler  \tab [d̥uç]
\ex\label{ex:7:36b}Wendelsheim\tab   [d̥ɔəç]
\ex\label{ex:7:36c}Kuhardt\tab   [d̥ɔəç]
\ex\label{ex:7:36d}Ilbesheim\tab   [d̥oəç]
\z 
\z

The data listed above indicate that /x/ has been restructured to /ç/ because the original /r/ was either deleted in (\ref{ex:7:36a}) or converted into a diphthong consisting of back vowels in (\ref{ex:7:36b}--\ref{ex:7:36d}). It is not possible to say whether or not the /ç/ in \REF{ex:7:36} is a quasi-phoneme or a \isi{phonemic palatal} (\chapref{sec:9}), but either way it is an underlying palatal and not a palatal synchronically derived from /x/.\footnote{{A skeptic might attempt to argue that the \isi{schwa} in (\ref{ex:7:36b}--\ref{ex:7:36d}) is phonetically (and phonologically) front, in which case [ç] as opposed to [x] would be the expected outcome of velar fronting. I reject any treatment along those lines because there is no independent evidence in any German dialect for a phonemic front ([coronal]) \isi{schwa}. It is possible that the \isi{schwa} in (\ref{ex:7:36b}--\ref{ex:7:36d}) is phonetically not as retracted as the \isi{schwa} before other sounds, but this is a consequence of \isi{coarticulation} (phonetics). See \sectref{sec:17.3.1} for my rejection of an analysis of the vocalized-r in \il{Standard German}StG as a front sound.}}

The data from ThürDA and MRhSA should give the reader a feeling for the pervasiveness of palatal quasi-phonemes. \isi{Syncope} is a change that is well-attested throughout CG, and changes eliminating the frontness feature of an earlier /r/ are well-documented in HG and LG. For these reasons it should not be difficult to find additional attestations of underlying palatals like the ones discussed in this chapter.

\section{{Conclusion}}\label{sec:7.6}\largerpage

What the case studies discussed in this chapter have in common is that they possess opaque palatals (quasi-phonemes), which by definition are underlying palatals that cannot be derived synchronically from other sounds. From the historical perspective, palatal quasi-phonemes were once palatal allophones of underlying velars that were restructured to underlying segments when historical processes eliminated the front vowels that originally served as triggers for velar fronting. Palatals in the neighborhood of back sounds that were originally front exemplify the \isi{underapplication} of the historical process of velar fronting.

In the following two chapters I consider German dialects with underlying palatals that differ from quasi-phonemes because they contrast with the corresponding velars. Those contrastive (i.e. phonemic) palatals can arise historically in more than one way, although it is demonstrated below that the four changes listed in \REF{ex:7:4} are instrumental in their development.

\is{palatal quasi-phoneme|)}
