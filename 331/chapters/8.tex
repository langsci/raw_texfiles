\chapter{Phonemicization of palatals (part 1)}\label{sec:8}

\section{Introduction}\label{sec:8.1}

This chapter investigates dialects in which velars and the corresponding palatals contrast in word-initial position. Those contrasting dorsal sounds are captured directly in the underlying representation with phonemic velars (e.g. /x/) and phonemic palatals (e.g. /ç/). As described below, there is more than one way in which palatals were phonemicized (recall \sectref{sec:2.4.3}, \sectref{sec:2.5}).

The following case studies are organized into three distinct types, defined both synchronically and diachronically. In certain varieties (Contrast Type A) the sounds in question contrast before back vowels and front vowels, but in others (Contrast Type B) that velar vs. palatal contrast occurs before back vowels but not before front vowels, where only the palatal surfaces. In yet another system (Contrast Type C) the velar and the palatal contrast before front vowels, but before back vowels only the velar surfaces. It is argued below that velars and palatals are all phonemic in Contrast Type A-C (either /ɣ ʝ/ or /x ç/). The distribution of word-initial velars and palatals for the three systems are depicted in \REF{ex:8:1}, where [i] and [ɑ] are cover symbols for front vowels and back vowels respectively.

\ea%1
\label{ex:8:1}

\ea\label{ex:8:1a}
Contrast Type A: \smallskip\\
\fbox{
\begin{tabular}{@{}ll@{}}
 \textsubscript{wd}[ [ʝi...] & \textsubscript{wd}[ [ʝɑ…]  \\
 \textsubscript{wd}[ [ɣi...] & \textsubscript{ wd}[ [ɣɑ…] 
\end{tabular}
}

\ex\label{ex:8:1b}
Contrast Type B:\smallskip\\
\fbox{
\begin{tabular}{@{}ll@{}}
 \textsubscript{wd}[ [ʝi...] & \textsubscript{wd}[ [ʝɑ…]\\ 
 & \textsubscript{wd}[ [ɣɑ…]                          
\end{tabular}
}

\ex\label{ex:8:1c}
Contrast Type C:\smallskip\\
\fbox{
\begin{tabular}{@{}ll@{}} 
\textsubscript{wd}[ [çi...] & \\
\textsubscript{wd}[ [xi...] & \textsubscript{wd}[ [xɑ…]
\end{tabular}
}
\z 
\z 

The three systems in \REF{ex:8:1} are not equally common among German dialects. There is no question that \REF{ex:8:1b} represents the default case, which is represented by many descriptions of LG varieties spoken throughout North Germany (including the pre-1945 regions in the east; see \chapref{sec:11}). \REF{ex:8:1a} is not nearly as well-attested as \REF{ex:8:1b}, although it can be found in more than one variety in the neighborhood of the Dutch border. By contrast, \REF{ex:8:1c} is restricted to a single \il{Eastphalian}Eph village. As such, it deserves special attention because it shows how a unique system can develop as the result of a dialect-specific change introducing potentially new front vowel triggers.

Word-initial palatal vs. velar contrasts as in \REF{ex:8:1} came about in more than one way, the first of which is exemplified in \REF{ex:8:2}. In \ili{WGmc} there was a contrast between the lenis velar fricative \textsuperscript{+}[ɣ] (/ɣ/) and the palatal glide \textsuperscript{+}[j] (/j/). That \ili{WGmc} system is depicted to the left of the wedge in \REF{ex:8:2}. At that stage \textsuperscript{+}[ɣ] occurred before front vowels, back vowels, or consonants and \textsuperscript{+}[j] before front vowels or back vowels but not before consonants (Appendix~\ref{appendix:f}). The original fricative vs. glide contrast was altered to a contrast between the lenis velar fricative [ɣ] (/ɣ/) and the lenis palatal fricative [ʝ] (/ʝ/) by \isi{Glide Hardening} (\sectref{sec:4.2}, restated in \ref{ex:8:3}). When that restructuring occurred, a contrast arose in word-initial position between a velar fricative /ɣ/ and the corresponding palatal fricative /ʝ/ (=\ref{ex:8:1a}).

\ea%2
\label{ex:8:2}Phonemicization of /ʝ/ in word-initial position (\isi{Glide Hardening}):\\
\begin{forest} for tree = {fit=band}   
[,phantom
[/ɣ/ [{[ɣ]}]]  [/j/ [{[j] }]]      [>]  [/ɣ/ [{[ɣ]}]] [/ʝ/ [{[ʝ]}]]
]
\end{forest}

\ex%3
    \label{ex:8:3}\isi{Glide Hardening}:\\
  /j/ > /ʝ/  \textsubscript{${\sigma}$} [ {\longrule}
\z 

In the system depicted to the right of the wedge in \REF{ex:8:2}, [ɣ] and [ʝ] contrast before a back vowel or front vowel (=\ref{ex:8:1a}). Word-initial velar fronting is absent in that system; what is more, there never was a stage in which that process was active in its history. The occurrence of the palatal [ʝ] before a back vowel and the velar [ɣ] before a front vowel does not imply \isi{opacity} because velar fronting was never active in word-initial position.

The way in which the velar vs. palatal contrast in \REF{ex:8:2} emerged is very different from the developments that led to Contrast Type B and Contrast Type C. As indicated in \REF{ex:8:4} those two systems arose by a \isi{phonemic split}:

\ea%4
\label{ex:8:4}
\begin{forest} for tree = {fit=band}   
[,phantom
     [/\textsc{Ve}/  [{[\textsc{Ve}]}] ] 
   [>]      
      [/\textsc{Ve}/,calign=first  [{[\textsc{Ve}]}]  [{[\textsc{Pa}]}]]         
   [>]   
   [/\textsc{Ve}/ [{[\textsc{Ve}]}]]   
   [/\textsc{Pa}/ [{[\textsc{Pa}]}]]
]
\end{forest}
\z 

Consider first Contrast Type B, depicted in \REF{ex:8:5a} and \REF{ex:8:5b}. As indicated in the heading below, the change in both \REF{ex:8:5a} and \REF{ex:8:5b} involved the \isi{phonemic split} of the two allophones [ɣ] and [ʝ], as in \REF{ex:8:4}.

\ea%5
\label{ex:8:5}
    Phonemic splits in word-initial position triggered by merger (\isi{Glide Hardening}):
\ea  \label{ex:8:5a}      
\begin{forest} for tree = {fit=band}   
  [,phantom 
    [/ɣ/ [{[ɣ]}]]  
    [/j/  [{[j]}]]
    [>]   
    [/ɣ/,calign=first [{[ɣ]}]  [{[ʝ]}]]
    [/j/  [{[j]}]]   
    [>]       
    [/ɣ/  [{[ɣ]}]]     
    [/ʝ/ [{[ʝ]}]]
   ]
  \end{forest}
\ex \label{ex:8:5b}      
\begin{forest} for tree = {fit=band}   
  [,phantom 
  [/ɣ/ [{[ɣ]}]]   
  [/j/ [{[j]}]]      
  [>]   
  [/ɣ/,calign=first  [{[ɣ]}]    [{[ʝ]}]] 
  [/j/ [{[j]}]]    
  [>]       
  [/ɣ/ [{[ɣ]}]]    
  [/ʝ/ [{[ʝ]}]]  
  [>]  
  [/ɣ/ [{[g]}]]       
  [/ʝ/ [{[ʝ]}]]                                                          
  ]
\end{forest}
\z 
\z 

In \REF{ex:8:5a}, \ili{WGmc} \textsuperscript{+}[ɣ] (/ɣ/) shifted to the corresponding palatal [ʝ] in the context before a coronal sonorant (or some subset thereof) via velar fronting (after the first wedge). Since that change occurred before \ili{WGmc} \textsuperscript{+}[j] underwent \isi{Glide Hardening}, velar fronting was still an allophonic rule relating the positional variants [ɣ] and [ʝ]; see \chapref{sec:16} for discussion on the time frame for the changes in \REF{ex:8:5a}. When \isi{Glide Hardening} merged [ʝ] (/ʝ/) with the [ʝ] allophone of /ɣ/ (after the second wedge) a contrast between /ɣ/ and /ʝ/ emerged before a back vowel (as in \ref{ex:8:1b}). That new palatal does not exemplify \isi{overapplication} \isi{opacity} because it was not the product of velar fronting.

It is shown below that velar fronting was not lost after the \isi{phonemicization} of the palatal in \REF{ex:8:5a}. Instead, velar fronting remained in that system as a rule of \isi{neutralization} creating [ʝ] from /ɣ/ before a front vowel. The palatal in that type of example exemplifies the synchronically \isi{derived palatal} discussed in \sectref{sec:2.4.3}.

A variant of \REF{ex:8:5a} is depicted in \REF{ex:8:5b}. As in \REF{ex:8:5a}, word-initial velar fronting was active as an allophonic rule at the stage before \isi{Glide Hardening} transpired (after the first wedge). After \isi{Glide Hardening} merged the new /ʝ/ with the earlier allophone [ʝ], [ɣ] (/ɣ/) was realized as [g] (after the third wedge). As in \REF{ex:8:5a}, word-initial velar fronting remains active in \REF{ex:8:5b} as a rule of \isi{neutralization}, thereby creating a \isi{derived palatal} before a front vowel.

Changes other than a merger can trigger the \isi{phonemic split} in \REF{ex:8:4}. Consider \REF{ex:8:6}, which shows Contrast Type C:\pagebreak

\ea%6
\label{ex:8:6}Phonemic split in word-initial position (\isi{r-Deletion}):\\
\begin{forest} for tree = {fit=band}   
  [,phantom
   [/ɣ/ [{[ɣ]}]]      
   [>]       
   [/x/ [{[x]}]]       
   [>]         
   [/x/,calign=first [{[x]}]  [{[ç]}]]            
   [>]       
   [/x/ [{[x]}]]    
   [/ç/ [{[ç]}]]                                                        
  ]
\end{forest}
\z 

\ili{WGmc} \textsuperscript{+}[ɣ] (/ɣ/) underwent \isi{Wd-Initial ɣ-Fortition} (\sectref{sec:4.3}) to [x] (/x/), at which point the new [x] developed a palatal allophone before a front vowel by velar fronting (after the second wedge). In word-initial position before a consonant, \ili{WGmc} \textsuperscript{+}[ɣ] was likewise realized as [x], but that fricative did not shift to palatal [ç] because the set of triggers for word-initial velar fronting consisted solely of front vowels. The crucial examples involve \ili{WGmc} \textsuperscript{+}[ɣ] before [r], e.g. sequences like \textsuperscript{+}[ɣri] (where [i] represents any front vowel) and [ɣrɑ] (where [ɑ] represents any back vowel). In word-initial [xr] clusters, the rhotic was elided by (\ref{ex:8:7}) regardless of the nature of the following vowel.

\ea%7
\label{ex:8:7}
  \isi{r-Deletion}:\\
  /r/ > ∅  / \textsubscript{wd}[ C {\longrule}
\z 

As a consequence of \REF{ex:8:7}, sequences like [xri] (/xri/) and [xrɑ] (/xrɑ/) were restructured to [xi] (/xi/) and [xɑ] (/xɑ/) respectively. The result was that [x] and [ç] contrast in word-initial position before a front vowel, but before a back vowel only [x] surfaces, as in (\ref{ex:8:1c}). Note that the pre-front vowel [x] exemplifies the historical \isi{underapplication} of velar fronting. As discussed at length below, \isi{r-Deletion} led directly to \isi{rule inversion}. This means that the historical rule fronting a word-initial /x/ to [ç] was reanalyzed as a synchronic rule converting a palatal (/ç/) to the corresponding velar ([x]) before a back vowel.

A \isi{phonemic split} between a word-initial velar and palatal as in (\ref{ex:8:4}) can also occur when the front vowel triggering the original palatal allophone undergoes a qualitative change to a back vowel by \isi{Vowel Retraction} (\sectref{sec:7.1}). That development (attested in \ref{ex:8:1b} dialects) is depicted schematically in \REF{ex:8:8}. The number of words exemplifying the type of change here it is very small (\sectref{sec:8.6.2}). By contrast, the mirror-image \isi{phonemic split} of velar and palatal in postvocalic position is well-attested in copious examples (\chapref{sec:9}). Recall that \isi{Vowel Retraction} is also responsible for the emergence of quasi-phonemes (\chapref{sec:7}).

\ea%8
    \label{ex:8:8}Phonemic splits in word-initial position (\isi{Vowel Retraction}):
\ea \label{ex:8:8a}
  \begin{forest} for tree = {fit=band}   
  [,phantom
  [/ɣ/ [{[ɣ]}]]   
  [/j/ [{[j]}]]     
  [>]   
  [/ɣ/,calign=first  [{[ɣ]}] [{[ʝ]}]]           
  [>]         
  [/ɣ/ [{[ɣ]}]]      
  [/ʝ/  [{[ʝ]}]]
  ]
  \end{forest}                                       
\ex \label{ex:8:8b}     
  \begin{forest} for tree = {fit=band}   
  [,phantom
  [/ɣ/ [{[ɣ]}]]   
  [/j/ [{[j]}]]      
  [>]   
  [/x/,calign=first   [{[x]}] [{[ç]}]]          
  [>]        
  [/x/ [{[x]}]]
  [/ç/ [{[ç]}]]
  ]
  \end{forest}
\z 
\z 

The original \ili{WGmc} system in \REF{ex:8:8} led to one in which velar and palatal fricatives were allophones (after the second wedge). When one or more front vowel triggering the palatal allophones shifted to a back vowel by \isi{Vowel Retraction}, the earlier palatal remained palatal before the new back vowel, thereby creating a contrast between velar and palatal, as in (\ref{ex:8:1b}). Note that the opaque pre-back vowel palatal exemplifies the historical \isi{overapplication} of velar fronting.

\sectref{sec:8.2} focuses on a case study (from LFr) with a word-initial [ɣ] vs. [ʝ] contrast before back vowels and front vowels (Contrast Type A). In \sectref{sec:8.3} and \sectref{sec:8.4} I examine \il{Eastphalian}Eph varieties (Contrast Type B), in which the word-initial velar vs. palatal contrast is attested before a back vowel. \sectref{sec:8.5} investigates an \il{Eastphalian}Eph variety (Contrast Type C) in which [x] and [ç] contrast in word-initial position before a front vowel. \sectref{sec:8.6} provides some discussion and \sectref{sec:8.7} an assessment of the areal distribution of word-initial phonemic palatals. Concluding remarks can be found in \sectref{sec:8.8}.

\section{{Low} {Franconian}}\label{sec:8.2}\il{Low Franconian|(}

Two very similar varieties of LFr are described by \citet{Meynen1911} for \ipi{Homberg} and \citet{Hanenberg1915} for \ipi{Kalkar}. Both places are indicated on \mapref{map:8}. I restrict my discussion below to the \ipi{Kalkar} variety, although the one for \ipi{Homberg} is essentially the same.

The phonemic front and back vowels for \ipi{Kalkar} are /i e ɛ ɛː æ y ø œ/ and /u o ɔ ɑ ə/ respectively, most of which can occur as either short or long. I interpret Hanenberg’s ⟦ɛ⟧ as the low vowel [æ] because it occupies a place in his vowel chart lower than his ⟦ę⟧ (=my [ɛ]). \ipi{Kalkar} possesses the four dorsal fricatives [x ç ɣ ʝ], of which only [ɣ ʝ] occur initially. The distribution of those sounds is expressed in \REF{ex:8:9}:

\ea%9
\label{ex:8:9}
    \begin{forest}
          [,phantom
          [/ɣ/ [{[ɣ]}]]  
          [/ʝ/ [{[ʝ]}]]
          ]
    \end{forest}
\z 

The significance of \ipi{Kalkar} is that the velar and corresponding palatal in \REF{ex:8:9} contrast in word-initial position before any kind of vowel (=\ref{ex:8:1a}).

The data in \REF{ex:8:10} exemplify the distribution of word-initial [ɣ] and [ʝ], both of which derived from \ili{WGmc} \textsuperscript{+}[ɣ]. Hanenberg’s symbol ⟦ɡ⟧ represents a lenis (voiced) velar fricative (“stimmhafter, gutturaler Reibelautˮ), and his ⟦j⟧ depicts a lenis (voiced) palatal fricative (“stimmhafter, palataler Reibelautˮ). [ɣ] occurs word-initially before back vowels in (\ref{ex:8:10a}), front vowels in (\ref{ex:8:10b}), or consonants in (\ref{ex:8:10c}). The [ɣ] in all of these examples is inherited without change from \ili{WGmc} \textsuperscript{+}[ɣ]. The \isi{etymological palatal} ([ʝ]) surfaces before back vowels in (\ref{ex:8:10d}) or front vowels in (\ref{ex:8:10e}). As in many other dialects, the \isi{etymological palatal} is rare before a front vowel.

\TabPositions{.15\textwidth, .33\textwidth, .5\textwidth, .75\textwidth}
\ea%10
\label{ex:8:10}Word-initial dorsal fricatives in \ipi{Kalkar}:\\
\ea\label{ex:8:10a}   ɡūt     \tab [ɣuːt]     \tab   gut    \tab ‘good’  \tab 216\\
      ɡoͅlt   \tab  [ɣɔlt]    \tab    Gold   \tab ‘gold’ \tab  216\\
      ɡədei̯ə \tab  [ɣədeiə]  \tab    gedeihen \tab ‘thrive\textsc{{}-inf}’ \tab 217
\ex\label{ex:8:10b}   ɡøͅn    \tab  [ɣœn]     \tab    gehe  \tab ‘go\textsc{{}-1sg}’ \tab 211\\
      ɡēͅrn   \tab  [ɣɛːrn]   \tab    gern  \tab ‘gladly’   \tab   192\\
      ɡɛlt    \tab [ɣælt]     \tab   Geld  \tab ‘money’     \tab  192
\ex\label{ex:8:10c}   ɡlik    \tab [ɣlik]     \tab   gleich  \tab ‘soon’    \tab  198\\
      ɡrōnd   \tab [ɣʀoːnt]   \tab   Grund  \tab ‘reason’   \tab  195
\ex\label{ex:8:10d}   joͅmər  \tab  [ʝɔmər]   \tab    Jammer  \tab ‘lament’ \tab   209\\
      jɑxt    \tab [ʝɑxt]     \tab   Jagd  \tab ‘hunt’      \tab  209
\ex\label{ex:8:10e}   jøͅkə   \tab  [ʝœkə]    \tab    jucken  \tab ‘itch\textsc{{}-inf}’ \tab 209
 \z
\z 

\ipi{Kalkar} contrasts [ɣ] and [ʝ] before front and back vowels alike. In fact, it is not difficult to find examples in which the two fricatives occur before the same vowel, e.g. [ɣɔlt] ‘gold’ vs. [ʝɔmər] ‘lament’. From the synchronic perspective, both velar and palatal are phonemic, as depicted in \REF{ex:8:9}.

As illustrated in \REF{ex:8:2}, word-initial contrasts like the ones in \REF{ex:8:10} arose historically from an earlier stage in which the fricative [ɣ] (<\ili{WGmc} \textsuperscript{+}[ɣ]) contrasted with the palatal glide [j] (<\ili{WGmc} \textsuperscript{+}[j]). When the latter sound underwent \isi{Glide Hardening}, the contrast between /ɣ/ and /ʝ/ emerged.

Phonemicization as in \REF{ex:8:2} is also attested in other varieties of German spoken in the same general region, two examples of which are presented in \REF{ex:8:11} and \REF{ex:8:12}. The first three categories in both of those datasets exemplify the contexts for [ɣ] before back vowels, front vowels, and coronal consonants respectively. \REF{ex:8:11d} and \REF{ex:8:12d} are items with the \isi{etymological palatal} ([ʝ]). Both sources cited below are clear that the respective word-initial sounds in \REF{ex:8:11} and \REF{ex:8:12} represent lenis velar and palatal fricatives. Like \ipi{Kalkar}, the two varieties below can be classified as Contrast Type A (=\ref{ex:8:1a}), although [ʝ] is unstable before a front vowel.

\ea%11
    \label{ex:8:11}\ipi{Lathen} (NLG; \citealt{Schönhoff1908}; \mapref{map:5}):
\ea \label{ex:8:11a}ʓou̯t \tab [ɣout] \tab gut \tab ‘good’ \tab 183
\ex \label{ex:8:11b}ʓ\={œ}vn \tab [ɣœːvn̩] \tab geben \tab ‘give\textsc{{}-inf}’ \tab 195
\ex \label{ex:8:11c}ʓlyk \tab [ɣlyk] \tab Glück \tab ‘fortune’ \tab 175
\ex \label{ex:8:11d}jɔ̄ \tab [ʝɔː] \tab  ja \tab ‘yes’ \tab 155
\z
\ex %12
\label{ex:8:12}\ipi{Montzen} (\il{Ripuarian}Rpn; \citealt{Welter1933}; \mapref{map:8}):
\ea \label{ex:8:12a}ɣɑ̄:də \tab [ɣɑːdə] \tab Garten \tab ‘garden’ \tab 18
\ex \label{ex:8:12b}γȩ̄:.lt \tab [ɣɛːlt] \tab Geld \tab ‘money’ \tab 18
\ex \label{ex:8:12c}γru·ə.t \tab [ɣruət] \tab groß \tab ‘large’ \tab 18
\ex \label{ex:8:12d}jǭ:r \tab [ʝɔːr] \tab Jahr \tab ‘year’ \tab 23
\z
\z 

One difference between \ipi{Lathen} and \ipi{Montzen} on the one hand and \ipi{Kalkar}\slash\ipi{Homberg} on the other is that only \ipi{Kalkar}/\ipi{Homberg} have velar fronting in postsonorant position. By contrast, in both \ipi{Lathen} and \ipi{Montzen} the velars [x] and [ɣ] surface after front and back vowels. The pattern described here for \ipi{Kalkar}\slash\ipi{Homberg} is also attested in the \il{Ripuarian}Rpn variety of Ronsdorf (\citealt{Holthaus1887}; \mapref{map:8}), although velar fronting only affects the fortis fricative in postsonorant position.\il{Low Franconian|)}

\section{{Eastphalian} {(part} {1)}}\label{sec:8.3}\il{Eastphalian|(}

\citet{Block1910} describes the \il{Eastphalian}Eph dialect of \ipi{Eilsdorf} (\mapref{map:7}). The phonemic front and back vowels are /iː ɪ eː e ɛː ɛ yː ʏ œː œ/ and /uː ʊ ɔ ɑː ɑ ə/ respectively.\footnote{{I omit the vowel \citet[327]{Block1910} describes as an overshort open i-sound (“überkurzer offener i-Lautˮ), which appears to be a variant pronunciation of [ə].}} The diphthongs ending in a front vowel are /oi ɑi/ and the ones ending in a back vowel are /ɑu oːǝ øːǝ eːǝ/. \ipi{Eilsdorf} has the four dorsal fricatives [x ç ɣ ʝ]. In contrast to the related \il{Eastphalian}Eph variety spoken in \ipi{Dingelstedt am Huy} (\sectref{sec:8.4}), \ipi{Eilsdorf} possesses no [g]. The only dorsal fricatives occurring word-initially are [ɣ ʝ], which contrast as in the varieties discussed in \sectref{sec:8.2}, e.g. \ipi{Kalkar}. The word-initial dorsal sounds have the distribution depicted in \REF{ex:8:9}.\footnote{\label{fn:8:2}In postsonorant position the four dorsal fricatives of \ipi{Eilsdorf} are [x ç ɣ ʝ]. The two palatals surface after coronal sonorants and the two velars after back vowels, as in \il{Eastphalian}Eph variety of \ipi{Dorste} (\sectref{sec:4.4}). In the context after \isi{schwa}, [ʝ] is a \isi{palatal quasi-phoneme} (/ʝ/), e.g. [brɛdəʝɑm] ‘groom’ (=⟦brędəjɑm⟧).}

The examples in \REF{ex:8:13} exemplify the occurrence of velar [ɣ] in word-initial position before a full back vowel in (\ref{ex:8:13a}) or a coronal consonant in (\ref{ex:8:13b}). The coronal (apical) rhotic (“Zungenspitzen-rˮ) is realized consistently as [r], regardless of whether or not it occurs in the onset or in the coda. The coronal consonant referred to here ([n l r]) can be followed by any type of vowel. Gaps involving the phonemic vowels listed above after word-initial [ɣ] are accidental. The word-initial sound in \REF{ex:8:13} derived from \ili{WGmc} \textsuperscript{+}[ɣ].

\TabPositions{.15\textwidth, .33\textwidth, .55\textwidth, .8\textwidth}
\ea%13
    \label{ex:8:13}
          Word-initial [ɣ] (from /ɣ/):
\ea \label{ex:8:13a}ʒuut \tab [ɣuːt] \tab gut \tab ‘good’ \tab 342\\
      ʒ\k{u}ln \tab  [ɣʊln] \tab Gulden \tab ‘guilder’ \tab 349\\
      ʒǫrts \tab  [ɣɔrts] \tab Gottfried \tab ‘(name)’ \tab 342\\
      ʒɑt \tab  [ɣɑt] \tab Loch \tab ‘hole’ \tab 342\\
      ʒɑ̊ɑ̊n \tab [ɣɑːn] \tab gehen \tab ‘go\textsc{{}-inf}’ \tab 342\\
      ʒɑit \tab [ɣɑit] \tab geht \tab ‘go\textsc{{}-3sg}’ \tab 335\\
      ʒɑus \tab [ɣɑus] \tab Gans \tab ‘goose’ \tab  342
\ex   \label{ex:8:13b}ʒlɑs \tab  [ɣlɑs] \tab Glas \tab ‘glass’ \tab 340\\
      ʒleezər \tab [ɣleːzər] \tab Gläser \tab ‘glass-\textsc{pl}’ \tab 333\\
      ʒriis \tab [ɣriːs] \tab Greis \tab ‘old man’ \tab 340\\
      ʒn\k{i}tə \tab [ɣnɪtə] \tab  kleine \ipi{Mücke} \tab ‘small mosquito’ \tab 342
   \z
\z 

The examples in \REF{ex:8:14} reveal that palatal [ʝ] surfaces in word-initial position before a back vowel. The orthography indicates that the [ʝ] in question (\isi{etymological palatal}) is the modern reflex of the \ili{WGmc} palatal glide \textsuperscript{+}[j].\footnote{\label{fn:8:3}[ʝ] (<\ili{WGmc} \textsuperscript{+}[ɣ]) -- but never [ɣ] -- occurs in word-initial position before \isi{schwa}, e.g. [ʝəzʊnt] ‘healthy’ (=⟦jəz\k{u}nt⟧). As in a number of case studies discussed in \chapref{sec:7}, the palatal in that type of example represents the \isi{palatal quasi-phoneme} /ʝ/. The presence of [ʝ] before a back vowel does not imply that \ipi{Eilsdorf} represents \REF{ex:8:1a} because there is no contrast between [ʝ] and [ɣ] before \isi{schwa}. One very general question concerning all dialects with a contrast between a velar and the corresponding palatal is whether or not the a \isi{palatal quasi-phoneme} is always present in those systems. If so, this suggests that the \isi{quasi-phonemicization} of palatals is a necessary prerequisite for the \isi{phonemicization} of palatals. Since this question can only be addressed after all case studies involving \isi{phonemicization} have been investigated I delay discussion until \sectref{sec:9.4.2}.}

\ea%14
\label{ex:8:14}Word-initial [ʝ] (from /ʝ/):\\
j\k{u}ŋk \tab [ʝʊŋk] \tab jung \tab ‘young’ \tab 338\\
jɑmər \tab [ʝɑmər] \tab Jammer \tab ‘lament’ \tab 338\\
jɑ̊ɑ̊ \tab  [ʝɑː] \tab ja \tab  ‘yes’ \tab 338\\
jɑuln \tab [ʝɑuln̩] \tab jaulen \tab ‘yowl\textsc{{}-inf}’ \tab 338
\z 

From the synchronic perspective, \ipi{Eilsdorf} exemplifies \REF{ex:8:1b}. Block does not list words beginning with the \isi{etymological palatal} followed by a front vowel, although the [ʝ] deriving from \ili{WGmc} \textsuperscript{+}[ɣ] in the context before a front vowel are present in some of the examples discussed below.

As indicated in \REF{ex:8:13a} and \REF{ex:8:14},  [ɣ] and [ʝ] contrast in word-initial position before a full back vowel. Note that some of these items illustrate the contrast between [ʝ] and [ɣ] holds before the same vowels, e.g. [ɣʊln] ‘guilder’ vs. [ʝʊŋk] ‘young’.  On the basis of contrasts like these, [ʝ] and [ɣ] are both phonemic (/ʝ/ and /ɣ/).

In word-initial position before a front vowel, [ʝ] (<\ili{WGmc} \textsuperscript{+}[ɣ]) surfaces in many items, as in (\ref{ex:8:15}). The front vowels in examples like these were also etymological front vowels. From the synchronic perspective palatal [ʝ] in \REF{ex:8:15} does not alternate with [ɣ].

\ea%15
\label{ex:8:15}Word-initial [ʝ] (from /ɣ/):\\
  jiir \tab [ʝiːr] \tab Gier \tab ‘greed’ \tab 342\\
  j\k{i}stərn \tab [ʝɪstərn] \tab gestern \tab ‘yesterday’ \tab 342\\
  jüüstə \tab [ʝyːstə] \tab unfruchtbar \tab ‘barren’ \tab 342\\
  jęlt \tab [ʝɛlt] \tab Geld \tab ‘money’ \tab 342\\
  jeel \tab [ʝeːl] \tab gelb \tab ‘yellow’ \tab 342
\z 

As indicated above, I analyze the underlying representation for word-initial [ʝ] before a front vowel in \REF{ex:8:15} as velar (/ɣ/), which undergoes \isi{Wd-Initial Velar Fronting-3} (\sectref{sec:4.3}), repeated in \REF{ex:8:16}. Velar /ɣ/ (as opposed to palatal /ʝ/) is justified in \REF{ex:8:15} because [ɣ] and [ʝ] never contrast before a front vowel, as in (\ref{ex:8:1b}). The word-initial palatals in \REF{ex:8:15}, together with the ones discussed in \REF{ex:8:17} below, exemplify \isi{derived palatals} (\sectref{sec:2.4.3}). \isi{Wd-Initial Velar Fronting-3} applies as a \isi{neutralization} in \REF{ex:8:15} because the contrast between [ɣ] and [ʝ] is suspended in favor of [ʝ] in word-initial position before a front vowel. The \isi{neutralization} property crucially differentiates \isi{Wd-Initial Velar Fronting-3} in \ipi{Eilsdorf} from the fronting processes discussed in other varieties of German in previous chapters. In those earlier (LG) case studies, the fronting of velars relates [x] and [ç] which do not contrast, e.g. \ipi{Soest} (\sectref{sec:4.3}), where [x] and [ç] are allophones, and in \ipi{Elspe} (\sectref{sec:7.2}), where the complementary distribution of word-initial [x] and [ç] is disrupted by the occurrence of the \isi{palatal quasi-phoneme} [ç] (/ç/).

\ea%16
\label{ex:8:16}\isi{Wd-Initial Velar Fronting-3}:\\
\begin{forest}
[,phantom
  [\avm{[−son\\+cont]},name=parent [\avm{[dorsal]},tier=word]]
  [\avm{[−cons]} [\avm{[coronal]},name=target,tier=word]]
]
\draw [dashed] (parent.south) -- (target.north);
\node [left=1ex of parent] {\textsubscript{wd} [};
\end{forest}
\z 

Since fortis dorsal fricatives do not occur word-initially, it is not necessary to specify that the target for \isi{Wd-Initial Velar Fronting-3} be marked for a laryngeal feature. However, the trigger for that process is restricted to front vowels only. Were the trigger the set all coronal sonorants, then word-initial /ɣ/ would incorrectly surface as palatal in \REF{ex:8:13b}.

Many words are attested with \isi{Umlaut} alternations. The significance of those examples is that if the word begins with a lenis dorsal fricative, then that sound is realized as velar ([ɣ]) before a back vowel and as palatal ([ʝ]) before a front vowel. Representative examples are presented in \REF{ex:8:17}. As indicated below, the sound underlying that alternation is /ɣ/, which shifts to the corresponding palatal by \REF{ex:8:16}.\footnote{{I ignore the idiosyncrasies in these \isi{Umlaut} alternations (e.g. [ɑu] alternates with [œ]) because they are not relevant for my analysis. The important point is that the stem vowel in the second word in the two pairs is front and not back.} }

\ea%17
\label{ex:8:17}Word-initial [ɣ]{\textasciitilde}[ʝ] alternations (from /ɣ/):

\ea\label{ex:8:17a}  ʒɑ̊ɑ̊rə \tab  [ɣɑːrə] \tab Garten \tab ‘garden’ \tab 342\\
     jęrtnęęr \tab [ʝɛrtnɛːr] \tab Gärtner \tab ‘gardener’ \tab 342
\ex\label{ex:8:17b}  ʒɑus \tab  [ɣɑus] \tab Gans \tab ‘goose’ \tab 342\\
     jø̜səln \tab [ʝœsəln] \tab kleine Gänse \tab ‘small goose-\textsc{pl}’ \tab 342
\ex\label{ex:8:17c}  ʒɑ̊ɑ̊f \tab  [ɣɑːf] \tab gab \tab ‘give\textsc{{}-pret}’ \tab 329\\
     jeeəbm \tab [ʝeːəbm̩] \tab geben \tab ‘give\textsc{{}-inf}’ \tab 342
   \z
\z 

The sound underlying the [ɣ]{\textasciitilde}[ʝ] alternation in \REF{ex:8:17} cannot be palatal (/ʝ/). If \REF{ex:8:16} were replaced with a \isi{neutralization} converting word-initial /ʝ/ to [ɣ] before a back vowel then that process would incorrectly affect the /ʝ/ in words like [ʝʊŋk]  (/ʝʊŋk/) from \REF{ex:8:14}. The significance of the underlying velar is discussed in greater detail in \sectref{sec:17.3.3}.

The system described by Block in 1910 in which [ɣ] (/ɣ/) and [ʝ] (/ʝ/) contrast in word-initial position was the outgrowth of an earlier stage in which /ɣ/ surfaced as [ɣ] before any vowel; see Stage 1 in \REF{ex:8:18} for four representative examples showing that \ipi{Eilsdorf} illustrates pattern \REF{ex:8:5a}. At some point during Stage 1, those velars succumbed to a coarticulatory (phonetic) fronting, which was then phonologized as an allophonic rule (\isi{Wd-Initial Velar Fronting-3}) at Stage 2. At that point, word-initial [ɣ] surfaced in the neighborhood of a back vowel or consonant and word-initial [ʝ] in the neighborhood of a front vowel. When \isi{Glide Hardening} altered underlying representations (Stage 3), contrasts between the new \isi{phonemic palatal} /ʝ/ and the inherited velar phoneme /ɣ/ emerged in word-initial position before a full back vowel, as in the first and third example in \REF{ex:8:18}.

\ea%18
    \label{ex:8:18}
    \begin{tabular}[t]{@{} lllll @{}}
   /jɑː/       &   /ɣuːt/       &        /ɣɑːf/           &     /ɣeːəbm̩/         &           \\
 \relax  [jɑː]       &   [ɣuːt]       &       [ɣɑːf]            &    [ɣeːəbm̩]          &  Stage 1  \\\tablevspace
  /jɑː/        &  /ɣuːt/        &       /ɣɑːf/            &    /ɣeːəbm̩/          &           \\
 \relax [jɑː]        &  [ɣuːt]        &      [ɣɑːf]             &   [ʝeːəbm̩]           & Stage 2   \\\tablevspace
   /ʝɑː/       &   /ɣuːt/       &        /ɣɑːf/           &     /ɣeːəbm̩/         &           \\
 \relax  [ʝɑː]       &   [ɣuːt]       &       [ɣɑːf]            &    [ʝeːəbm̩]          &  Stage 3  \\\tablevspace
   \textit{ja} & \textit{gut}   &   \textit{gab}          &  \textit{geben}       &    \il{Standard German}StG \\
  ‘yes’        & ‘good’         &  ‘give\textsc{{}-pret}’ & ‘give\textsc{{}-inf}’ &           \\
  \end{tabular}
\z 

The word [ɣuːt] ‘good’-- representative of data set \REF{ex:8:13} -- shows that a word-initial velar remains velar in the phonetic representation and in the underlying representation at all three stages. The example [ʝɑː] ‘yes’-- representative of data set \REF{ex:8:14} -- reveals that a new phoneme entered the language at Stage 3 (/ʝ/). That sound is a \isi{phonemic palatal} because it contrasts with /ɣ/ before full back vowels. The words [ɣɑːf] ‘give\textsc{{}-pret}’ and [ʝeːəbm̩] ‘give\textsc{{}-inf}’ typify [ɣ]{\textasciitilde}[ʝ] alternations, as in (\ref{ex:8:17}). At Stage 2, /ɣ/ surfaced as [ʝ] before a front vowel in words like [ʝeːəbm̩] because the front vowel in /eːə/ belonged to the set of triggers for fronting. At Stage 3, the underlying representation for those alternating pairs did not change; hence, /ɣ/ from Stage 2 was inherited as /ɣ/ at Stage 3. Original /ɣ/ was likewise inherited in the nonalternating examples in \REF{ex:8:15}.

The \il{Eastphalian}Eph pattern for word-initial dorsal fricatives in (\ref{ex:8:13}--\ref{ex:8:15}) is attested elsewhere in that dialect region. A representative example is the \il{Eastphalian}Eph variety of \ipi{Lesse} (\citealt{Löfstedt1933}; \mapref{map:7}), which is about 40 km from \ipi{Eilsdorf}. Löfstedt uses the same symbol for [ɣ] and [ʝ], although he is clear that the distribution of the two is a function of the following vowel. The data in \REF{ex:8:19} suggest that \ipi{Lesse} exemplifies \REF{ex:8:1b}, although it appears that word-initial [ʝ] is unstable before a front vowel \citep[51]{Löfstedt1933}.

\ea%19
\label{ex:8:19}Word-initial dorsal fricatives in \ipi{Lesse}:
\ea\label{ex:8:19a} ʓolt \tab [ɣolt] \tab Gold \tab ‘gold’ \tab 56
\ex\label{ex:8:19b} ʓēbm̥ \tab [ʝeːbm̩] \tab geben \tab ‘give\textsc{{}-inf}’ \tab 56
\ex\label{ex:8:19c} ɡlɑ̊̄s \tab [glɑːs] \tab Glas \tab ‘glass’ \tab 56
\ex\label{ex:8:19d} jɑ̊̄r \tab  [ʝɑːr] \tab Jahr \tab ‘year’ \tab 51
\z
\z 

As in \ipi{Eilsdorf}, \ipi{Lesse} has a contrast between [ɣ] and [ʝ] in word-initial position before a full back vowel. Note that \ili{WGmc} \textsuperscript{+}[ɣ] is realized as the velar stop [g] before a consonant in (\ref{ex:8:19c}) and not as [ɣ], as in \REF{ex:8:19b}.

To summarize, the data described above for word-initial dorsal fricatives in \ipi{Eilsdorf} (and \ipi{Lesse}) represent one pattern for \il{Eastphalian}Eph (see \sectref{sec:8.4} for another pattern). That system is also well-represented in varieties of ELG discussed in \chapref{sec:11}, e.g. \ipi{Willuhnen} (\il{Low Prussian}LPr; \citealt{Natau1937}; \mapref{map:18}), Kreis \ipi{Bütow} and Kreis \ipi{Rummelsburg} (\il{East Pomeranian}EPo; \citealt{Mischke1936}; \mapref{map:18}).

\section{{Eastphalian} {(part} {2)}}\label{sec:8.4}

\citet{Hille1939} describes the \il{Eastphalian}Eph dialect of \ipi{Dingelstedt am Huy} (\mapref{map:7}). The phonemic front and back vowels are /iː ɪ eː ɛː ɛ yː ʏ øː/ and /uː ʊ oː ɔ ɑː ɑ ə/ respectively. The phonemic diphthongs ending in a front vowel are /ɑːi oːy ʏø ɪe/ and the ones ending in a back vowel are /ɑːu ʊo/. \ipi{Dingelstedt am Huy} has the four dorsal fricatives [x ɣ ç ʝ], in addition to the stop [g], which is demonstrated below to be an allophone of /ɣ/. In word-initial position only [ʝ] and [g] surface, which both contrast before back vowels; that word-initial system is depicted in \REF{ex:8:20}. In postsonorant position [x ɣ ç ʝ] pattern as in \ipi{Eilsdorf} (\fnref{fn:8:2}).

\ea%20
  \label{ex:8:20}
  \begin{forest}
  [,phantom
    [/ɣ/ [{[g]}]]  
    [/ʝ/ [{[ʝ]}]]
  ]
  \end{forest}
\z 

Word-initial [g] (<\ili{WGmc} \textsuperscript{+}[ɣ]) surfaces before a full back vowel in (\ref{ex:8:21a}) or a consonant in (\ref{ex:8:21b}). Recall from \REF{ex:8:13} that words like the ones in \REF{ex:8:21} are pronounced in \ipi{Eilsdorf} with an initial [ɣ]. The absence of items beginning with [g] followed by [oː ɔ] is accidental.

\ea%21
\label{ex:8:21}Word-initial [g] (from /ɣ/):
\ea\label{ex:8:21a} ɡūt    \tab [guːt]   \tab gut    \tab ‘good   \tab 30\\
    ɡus    \tab [gus]    \tab Guss   \tab ‘gush’  \tab 119\\
    ɡɑst   \tab [gɑns]   \tab ganz   \tab ‘quite’ \tab  101\\
    ɡɑ̄ist  \tab [gɑːist] \tab Geist  \tab ‘intellect’ \tab 64
\ex\label{ex:8:21b} ɡlɑ̄s   \tab [glɑːs]  \tab Glas   \tab ‘glass’     \tab 64\\
    ɡlükkə \tab [glʏkə]  \tab Glücke \tab ‘fortune\textsc{{}-pl}’ \tab 66\\
    ɡrɑ̄s   \tab [gʀɑːs]  \tab Gras   \tab ‘grass’ \tab 64
    \z
\z 

In word-initial position before a back vowel, palatal [ʝ] (/ʝ/) likewise can occur, as in (\ref{ex:8:22}). The [ʝ] in examples like these is the \isi{etymological palatal}. As in \ipi{Eilsdorf} (\fnref{fn:8:3}), in the context before \isi{schwa}, [ʝ] (<\ili{WGmc} \textsuperscript{+}[ɣ]) is present as a \isi{palatal quasi-phoneme}, e.g. [ʝədɑŋkə] ‘thought’ (=⟦jədɑŋkə⟧).

\ea \label{ex:8:22}  Word-initial [ʝ] (from /ʝ/):\\
jū \tab [ʝuː] \tab euer \tab ‘your\textsc{{}-pl}’ \tab 53\\
juŋk \tab [ʝʊŋk] \tab jung \tab ‘young’ \tab 27\\
jɑmmər \tab [ʝɑməʀ] \tab Jammer \tab ‘lament’ \tab 21\\
jɑ \tab [ʝɑː] \tab ja \tab ‘yes’ \tab 101
\z 

Palatal [ʝ] -- but never [g] -- surfaces in word-initial position before a front vowel in (\ref{ex:8:23}). The [ʝ] in these examples derives historically from \ili{WGmc} \textsuperscript{+}[ɣ]. As indicated here, I analyze the initial sound in \REF{ex:8:23} as an underlying velar.\footnote{{There are two words listed in the glossary of the original source  \citep[115--127]{Hille1939} in which the \isi{etymological palatal} occurs before a front vowel, namely [ʝiː] ‘her’ (=⟦jī⟧) and [ʝɪedəʀ] ‘every}\textrm{\textsc{{}-masc.sg}}.\textrm{’(=⟦j\k{i}edər⟧).}} The [ʝ] in these examples is a nonalternating palatal (like the corresponding \ipi{Eilsdorf} items in \ref{ex:8:15}).

\ea%23
\label{ex:8:23}Word-initial [ʝ] (from /ɣ/) in nonalternating words:\\
  j\={ę}rn̥ \tab [ʝɛːʀn̩] \tab gären \tab ‘ferment\textsc{{}-inf}’ \tab 42\\
  jelt \tab [ʝɛlt] \tab Geld \tab ‘money’ \tab 24\\
  j\k{i}ejən \tab [ʝɪeʝən] \tab gegen \tab ‘against’ \tab  18
\z 

In word-initial position before a full back vowel, [g] and [ʝ] contrast. This is illustrated in the examples presented above in \REF{ex:8:21a} vs. \REF{ex:8:22}, e.g. [guːt] ‘good’ vs. [ʝuː] ‘your\textsc{{}-pl}’. Items like these show that the contrast between word-initial [ʝ] and word-initial [g] holds before the same full back vowels. \ipi{Dingelstedt am Huy} represents \REF{ex:8:1b}, where [ɣ] in \REF{ex:8:1b} corresponds to [g].

The treatment of word-initial sequences like [gi] as systematic gaps is supported by alternating pairs like the ones in \REF{ex:8:24}. The first word in each pair begins with [g] followed by a full back vowel and the second word shows the fronting of that back vowel to a front vowel via \isi{Umlaut}. The important point is that the dorsal fricative is realized as [ʝ] before a front vowel.

\ea%24
\label{ex:8:24}Word-initial [g]{\textasciitilde}[ʝ] alternations (from /ɣ/):\\
\ea\label{ex:8:24a} ɡɑst \tab [gɑst] \tab Gast \tab ‘guest’ \tab 52\\
    jestə \tab [ʝɛstə] \tab Gäste \tab ‘guest-\textsc{pl}’ \tab 52
\ex\label{ex:8:24b} ɡɑ̄us \tab  [gɑːus] \tab  Gans \tab ‘goose’ \tab 52\\
    jössələ \tab [ʝœsələ] \tab Gänseküken \tab ‘goose chick’ \tab 52
  \z
\z 

I analyze the word-initial consonant in \REF{ex:8:23} and \REF{ex:8:24} as an underlying velar (/ɣ/). That sound shifts to [ʝ] before a front vowel by \isi{Wd-Initial Velar Fronting-3} in (\ref{ex:8:16}) and elsewhere surfaces as [g] (see below for discussion). The trigger for fronting must be the class of front vowels and not the class of coronal sonorants, otherwise word-initial /ɣ/ would incorrectly surface as [ʝ] before sounds like /l/ and /r/ in (\ref{ex:8:21b}). As in \ipi{Eilsdorf}, the distribution of velars and palatals necessitates an underlying velar which surfaces as palatal and not an underlying palatal which is realized as velar. If the alternations in \REF{ex:8:24} were analyzed in the synchronic phonology with an underlying /ʝ/ which retracts to {\textbar}ɣ{\textbar} (→[g]) before a back vowel, then the /ʝ/ in \REF{ex:8:22} would incorrectly be affected as well.

It was noted above that \isi{Wd-Initial Velar Fronting-3} creates [ʝ] from /ɣ/ before a front vowel. Word-initial /ɣ/ in the elsewhere case (i.e. before a back vowel or consonant) surfaces as [g] by \isi{g-Formation-2} in \REF{ex:8:25}. \isi{g-Formation-2} applies at the left edge of a word and not at the left edge of a syllable. The latter context cannot be correct because the /ɣ/ in a word-internal onset does not surfaces as [g], e.g. [fɔ.ɣəl] ‘bird’ (=⟦foǥǥəl⟧).

\ea%25
\label{ex:8:25}\isi{g-Formation-2}:\smallskip\\
\avm{[−son\\+cont\\−fortis\\dorsal]}  → [−cont] / \textsubscript{wd}[  {\longrule}
\z 

\isi{Wd-Initial Velar Fronting-3} (Wd-In Vel Fr-3) and \isi{g-Formation-2} (g-Form-2) have a very different status in the synchronic phonology. Since the former eliminates the contrast between underlying velar and underlying palatal to the latter, it is a \isi{neutralization}. However, \isi{g-Formation-2} applies to any word-initial /ɣ/ that has not undergone \isi{Wd-Initial Velar Fronting-3}. That type of /ɣ/ can be present in words that alternate with [ʝ], as in \REF{ex:8:24}, or in words that have no such alternation (e.g. in [glʏkə] ‘fortune\textsc{{}-pl}’ from /ɣlʏkə/ in \ref{ex:8:21b}). \isi{g-Formation-2}  is therefore an allophonic rule. As indicated in \REF{ex:8:26a}, \isi{Wd-Initial Velar Fronting-3} \isi{bleeds} \isi{g-Formation-2} in the second example.

\ea%26
\label{ex:8:26}
\ea\label{ex:8:26a} \begin{tabular}[t]{@{} lll @{}}
                 & /ɣɑst/ & /ɣɛst-ə/\\
  Wd-In Vel Fr-3 & --- &   ʝɛst-ə\\
  g-Form-2       & gɑst &  ---\\
                 & [gɑst]  &  [ʝɛstə]\\
                 & ‘guest’ & ‘guest-\textsc{pl}’\\
\end{tabular}
\ex\label{ex:8:26b} \begin{tabular}[t]{@{} lll @{}}
                 & /ɣɑst/ & /ɣɛst-ə/ \\
  g-Form-2       &  gɑst  & gɛst-ə   \\
  Wd-In Vel Fr-3 &  ---   &  ---     \\
                 & [gɑst] &  *[gɛstə]\\
\end{tabular}
\z 
\z 

Were \isi{g-Formation-2} to precede \isi{Wd-Initial Velar Fronting-3} (see \ref{ex:8:26b}), then the incorrect output would be obtained in the second example. Note that the ordering in \REF{ex:8:26b} is not \isi{counterbleeding}. Instead, \isi{Wd-Initial Velar Fronting-3} \isi{bleeds} \isi{g-Formation-2}; hence, those two processes stand in a transparent (\isi{mutually bleeding}) relationship (\sectref{sec:2.2.4}).

In \REF{ex:8:27} I provide three representative examples illustrating the development of dorsal sounds in word-initial position (as depicted in \ref{ex:8:5b}). The first three stages are the same as the three stages presented earlier for \ipi{Eilsdorf}: Stage 1 represents the point where velars are phonologically [ɣ] even in the neighborhood of front sounds. Stage 2 depicts the point in the history of LG before \isi{Glide Hardening}, in which [ɣ] and [ʝ] stood in an allophonic relationship. At that stage the palatal surfaced word-initially only before a front vowel and the velar elsewhere. When \isi{Glide Hardening} restructured the initial palatal to the phoneme /ʝ/, \isi{Wd-Initial Velar Fronting-3} operated as a \isi{neutralization} (Stage 3A). The difference between \ipi{Dingelstedt am Huy} and \ipi{Eilsdorf} can be observed at Stage 3B: The former dialect is more innovative than the latter because it added \isi{g-Formation-2}.

\ea%27
    \label{ex:8:27}\begin{tabular}[t]{@{} llll @{}}
  \relax /jɑː/        &  /ɣɑst/       &   /ɣɛstə/     &         \\
  \relax [jɑː]        &  [ɣɑst]       &   [ɣɛstə]     & Stage 1 \\\tablevspace
  \relax  /jɑː/       &   /ɣɑst/      &    /ɣɛstə/    &         \\
  \relax [jɑː]        &  [ɣɑst]       &   [ʝɛstə]     & Stage 2 \\\tablevspace
  \relax  /ʝɑː/       &   /ɣɑst/      &    /ɣɛstə/    &         \\
  \relax [ʝɑː]        &  [ɣɑst]       &   [ʝɛstə]     & Stage 3A\\\tablevspace
  \relax  /ʝɑː/       &   /ɣɑst/      &    /ɣɛstə/    &         \\
  \relax [ʝɑː]        &  [gɑst]       &   [ʝɛstə]     & Stage 3B\\\tablevspace
  \relax  \textit{ja} & \textit{Gast} & \textit{Gäste}& \il{Standard German}StG  \\
  \relax ‘yes’        &  ‘guest’      & ‘guest-\textsc{pl}’      &         \\
  \end{tabular}
\z 

The example [ʝɑː] ‘yes’-- recall \REF{ex:8:22} -- indicates that a new underlying dorsal fricative entered the language at Stage 3 (/ʝ/). That new palatal was a phoneme because it contrasted with /ɣ/ in words like [ɣɑst] ‘guest’. [gɑst] ‘guest’ and [ʝɛstə] ‘guest-\textsc{pl}’ are representative of an alternating pair (see \ref{ex:8:24}). At Stage 2, /ɣ/ surfaced as [ʝ] before a front vowel in items like [ʝɛstə] ‘guest-\textsc{pl}’ because /ɛ/ belonged to the set of triggers for fronting. At Stage 3A, the underlying representation for those alternating pairs did not change; hence, /ɣ/ from Stage 2 was inherited as /ɣ/ at Stage 3A and Stage 3B.

The word-initial pattern described above for \ipi{Dingelstedt am Huy} is well-at\-test\-ed in LG. Two very similar \il{Eastphalian}Eph varieties are presented in \REF{ex:8:28} and \REF{ex:8:29}. The two dialects listed here exemplify \REF{ex:8:1b}, although examples in \ipi{Magdeburger Börde} with word-initial [ʝ] before front vowels appears to be limited to names \citep[17]{Roloff1902}.

\ea%28
\label{ex:8:28}\ipi{Magdeburger Börde} (\citealt{Roloff1902}; \mapref{map:7}):
\ea\label{ex:8:28a} ɡɑlə  \tab  [gɑlə]   \tab Galle    \tab  ‘bile’ \tab 22
\ex\label{ex:8:28b} ɡråm̩ \tab   [grɑm̩] \tab   graben \tab    ‘bury\textsc{{}-inf}’ \tab 18
\ex\label{ex:8:28c} jęlt  \tab  [ʝɛlt]   \tab Geld     \tab  ‘money’ \tab 21
\ex\label{ex:8:28d} juŋk  \tab  [ʝuŋk]   \tab jung     \tab  ‘young’ \tab 17
\z 
\ex%29
\label{ex:8:29}\ipi{Göddeckenrode} and \ipi{Isingerode} (\citealt{Lange1963}; \mapref{map:7}):
\ea\label{ex:8:29a} ga\k{u}s \tab [gɑʊs] \tab Gans \tab ‘goose’ \tab 227
\ex\label{ex:8:29b} glā(ə)s \tab [glɑː(ə)s] \tab Glas \tab ‘glass’ \tab 227
\ex\label{ex:8:29c} j\={ę}l \tab [ʝɛːl] \tab  gelb \tab ‘yellow’ \tab 227
\ex\label{ex:8:29d} juŋk \tab [ʝuŋk] \tab jung \tab ‘young’ \tab 208\\
    jīək \tab [ʝiːək] \tab Joch \tab ‘yoke’ \tab 208
\z 
\z

ELG varieties displaying a similar pattern include \ipi{Lauenburg} (\il{East Pomeranian}EPo; \citealt{Pirk1928}; \mapref{map:18}), Kreis \ipi{Saatzig} (\il{East Pomeranian}EPo; \citealt{Kühl1932}; \mapref{map:18}), \ipi{Neumark} (\il{Brandenburgish}Brb; \citealt{Teuchert1907a,Teuchert1907b}; \mapref{map:17}), \ipi{Letschin} (Brd; \citealt{Teuchert1930}; \mapref{map:17}), and \ipi{Neu-Golm} (\il{Brandenburgish}Brb; \citealt{Siewert1912}; \mapref{map:17}). Those places are discussed in \chapref{sec:11}.

\section{{Eastphalian} {(part} {3)}}\label{sec:8.5}

\citet{Schütze1953} describes the \il{Eastphalian}Eph dialect once spoken in the community of \ipi{Neuendorf} (\mapref{map:7}). The phonemic front and back vowels in that variety are /iː ɪ eː ɛː ɛ/ and /uː ʊ oː ɔː ɔ ɑ ə/ respectively. The dialect possesses the dorsal fricatives [x ç ɣ ʝ], of which [x ç ʝ] surface word-initially. This section concerns itself with the contrast between [x ç] in word-initial position, which is depicted in \REF{ex:8:30}. The \isi{etymological palatal} [ʝ] (/ʝ/) (<Wmc \textsuperscript{+}[j]) is included for reference. I demonstrate below that [x] and the corresponding palatal [ç] contrast before front vowels, but only the velar occurs before back vowels, as in \REF{ex:8:1c}. The changes that occurred in \ipi{Neuendorf} are shown below to exemplify pattern \REF{ex:8:6}.

\ea%30
\label{ex:8:30}\begin{forest}
          [,phantom
            [/x/ [{[x]}]]  [/ç/ [{[ç]}]]   [/ʝ/ [{[ʝ]}]]
          ]
\end{forest}
\z 

In word-initial position [x] occurs before a back vowel in (\ref{ex:8:31a}) or consonant in (\ref{ex:8:31b}) and [ç] before any front vowel in (\ref{ex:8:32}). The word-initial dorsal fricatives in all of these examples derived historically from \ili{WGmc} \textsuperscript{+}[ɣ], which is reflected as \textit{g} in the \il{Standard German}StG orthography in the third column. [x]{\textasciitilde}[ç] alternations are provided in \REF{ex:8:33}. I discuss the correct underlying representations for the \ipi{Neuendorf} data below. There is no indication in the original source that there are constraints on the nature of the back vowel after [x] or the front vowel after [ç]. The kind of consonant after [x] is restricted to coronal sonorants.

\ea%31
\label{ex:8:31}Word-initial [x] before back vowels or consonants:
\ea\label{ex:8:31a} xolt  \tab [xɔlt]    \tab  Gold   \tab ‘gold’ \tab 32\\
    xǭn   \tab [xɔːn]    \tab  gehen  \tab ‘go\textsc{{}-inf}’ \tab 10\\
    xɑ̄wət \tab  [xɑːvət] \tab  gut    \tab ‘good’   \tab 32
\ex\label{ex:8:31b} xlīk  \tab [xliːk]   \tab gleich  \tab ‘same’   \tab 15\\
    xnǭdə \tab [xnɔːdə]  \tab  Gnade  \tab ‘mercy’  \tab  22
\z
\ex%32
\label{ex:8:32}
  Word-initial [ç] before front vowels:\\
  \begin{xlist}
  \sn
  χītsiχ  \tab [çiːtsiç] \tab geizig   \tab ‘stingy’ \tab 32\\
  χistərn \tab [çɪstərn] \tab gestern  \tab  ‘yesterday’ \tab 32\\
  χēwl    \tab [çeːvl̩]  \tab  Giebel  \tab  ‘gable’ \tab 9\\
  χę̄l    \tab  [çɛːl]   \tab gelb     \tab  ‘yellow’ \tab 32
\end{xlist}
\ex%33
\label{ex:8:33}Word-initial [x]{\textasciitilde}[ç] alternations:
\ea\label{ex:8:33a}  xūl    \tab [xuːl]    \tab Gaul   \tab ‘horse’  \tab 17 \\
     χīlə   \tab [çiːlə]   \tab Gäule  \tab ‘horse-\textsc{pl}’ \tab 18
\ex\label{ex:8:33b}  xot    \tab [xɔt]     \tab Gott   \tab ‘God’    \tab 10\\
     χetərə \tab [çɛtərə]  \tab Götter \tab ‘God-\textsc{pl}’   \tab 46
\ex\label{ex:8:33c}  xɑns   \tab [xɑns]    \tab Gans   \tab ‘goose’  \tab 27\\
     χenzə  \tab [çɛnzə]   \tab Gänse  \tab ‘goose-\textsc{pl}’  \tab 27
  \z
\z

The \isi{etymological palatal} [ʝ] (/ʝ/) occurs word-initially before front or back vowels, e.g. [ʝɔː] ‘yes’.

\begin{sloppypar}
The data presented in \REF{ex:8:34b} indicate that \ipi{Neuendorf} also possesses many words in which [x] surfaces in word-initial position before a front vowel. As revealed in the \il{Standard German}StG orthography, the [x] in those examples derived historically from \ili{WGmc} \textsuperscript{+}[ɣ] followed by [r] (by \isi{r-Deletion} in \ref{ex:8:7}). The examples in \REF{ex:8:34a} illustrate that \isi{r-Deletion} also occurred between [x] and a back vowel. Observe that \isi{r-Deletion} has the function of creating opaque velar plus front vowel sequences in \REF{ex:8:34b}.\footnote{The final item in \REF{ex:8:34b} derives from \ili{OSax} \textit{grīpan}.}
\end{sloppypar}

\ea%34
\label{ex:8:34}Word-initial [x] before back vowels or front vowels:
\ea\label{ex:8:34a} xunt \tab [xʊnt] \tab Grund \tab ‘reason’ \tab 11\\
    xošn \tab [xɔʃn̩] \tab Groschen \tab ‘penny’ \tab 26\\
    xof \tab [xɔf] \tab grob \tab ‘rough’ \tab 48\\
    xōwə \tab [xoːvə] \tab grobe \tab ‘rough\textsc{{}-infl}’ \tab 48\\
    xǭs \tab [xɔːt] \tab groß \tab ‘large’ \tab 26\\
    xoin \tab [xoin] \tab grün \tab ‘green’ \tab 26\\
    xɑf \tab [xɑf] \tab Grab \tab ‘grave’ \tab 26\\
\ex\label{ex:8:34b}  xīs \tab [xiːs] \tab grau \tab ‘gray’ \tab 15\\
     xīpm \tab [xiːpm̩] \tab greifen \tab ‘grasp\textsc{{}-inf}’ \tab 15\\
     xitə \tab [xɪtə] \tab Grütze \tab ‘groat’ \tab 26\\
     xēln \tab [xeːln] \tab grölen \tab ‘bellow\textsc{{}-inf}’ \tab 28\\
     xetər \tab [xɛtər] \tab größer \tab ‘bigger’ \tab 20\\
     xēpm  \tab [xɛːpm̩] \tab Mistgabel \tab ‘pitchfork’ \tab 18
    \z
\z 

Note that \ipi{Neuendorf} possesses words with [x]{\textasciitilde}[ç] alternations in (\ref{ex:8:33}) as well as words without such an alternation, e.g. [xɔːt] ‘large’ vs. [xɛtər] ‘larger’ in (\ref{ex:8:34}).\footnote{{The [x]{\textasciitilde}[ç] alternations in \REF{ex:8:33} are nouns, but the one example of a nonalternating pair referred to here is an adjective. I do not consider the lexical category to be significant. The reason the [x] in [xɔːt] ‘large’ fails to alternate with [ç] in [xɛtər] ‘larger’ is that the [x] in the latter word was once followed by [r] and not that it is an adjective.} }

The significance of the \ipi{Neuendorf} data is that [x] and [ç] contrast in word-initial position before a front vowel; see \REF{ex:8:32} vs. \REF{ex:8:34b}. It is not difficult to find examples where [x] and [ç] contrast before the same front vowel, e.g. [çiːtsɪç] ‘stingy’ vs. [xiːs] ‘gray’.

\citet{Schütze1953} gives every indication that \isi{r-Deletion} is an exceptionless, Neo\-gram\-mar\-ian-style sound change. I contend that \isi{r-Deletion} altered underlying representations from one generation to the next. Thus, an older generation of speakers retained the [r], while the younger and clearly more innovative generation does not, e.g. [xrɪs] /xrɪs/ shifted to [xɪs] /xɪs/. The latter underlying representations are the ones present in the grammar of the informants for \citet{Schütze1953}.

In \REF{ex:8:35} I give representative examples for phonetic and underlying representations for all of the datasets presented above. In the context before a front vowel, [ç] and [x] contrast, and hence, they are phonemic (\ref{ex:8:35c} vs. \ref{ex:8:35e}). \REF{ex:8:35f} represents [x]{\textasciitilde}[ç] alternations. Velar /x/ cannot be the underlying sound in that type of alternation, otherwise velar fronting (triggered by all front vowels) would incorrectly convert the /x/ in words like \REF{ex:8:35e} into [ç]. For this reason the underlying representation of the initial sound is /ç/; see (\ref{ex:8:35f}). In the context before a back vowel or consonant in nonalternating morphemes, surface [x] is underlyingly /x/; see (\ref{ex:8:35a}, \ref{ex:8:35b}, \ref{ex:8:35d}). Note that /x/ is inherited without change from earlier /x/. See below for discussion.

\ea\label{ex:8:35}
\ea\label{ex:8:35a} \relax [xɔlt]  \tab /xɔlt/   \tab ‘gold’   \tab (=\ref{ex:8:31a})
\ex\label{ex:8:35b} \relax [xliːk] \tab  /xliːk/ \tab ‘same’   \tab (=\ref{ex:8:31b})
\ex\label{ex:8:35c} \relax [çɛːl]  \tab /çɛːl/   \tab ‘yellow’ \tab (=\ref{ex:8:32})
\ex\label{ex:8:35d} \relax [xʊnt]  \tab /xʊnt/   \tab ‘reason’ \tab (=\ref{ex:8:34a})
\ex\label{ex:8:35e} \relax [xiːs]  \tab /xiːs/   \tab ‘gray’   \tab (=\ref{ex:8:34b})
\ex\label{ex:8:35f} \relax [xɑns]  \tab /çɑns/   \tab ‘goose’  \tab (=\ref{ex:8:33c})\\
    \relax [çɛnzə] \tab  /çɛnzə/ \tab ‘goose-\textsc{pl}’  \tab (=\ref{ex:8:33c})
\z 
\z 

Significantly, \ipi{Neuendorf} does not possess any version of word-initial velar fronting, but instead a rule backing a word-initial palatal, which I state in \REF{ex:8:36}. \isi{Wd-Initial Palatal Retraction} is a \isi{neutralization} because it suspends the contrast between /x/ and /ç/ to [x]. I discuss the way in which that process might be analyzed featurally in \sectref{sec:8.6.2}.

\ea%36
\label{ex:8:36}\isi{Wd-Initial Palatal Retraction}:\\
  /ç/ → [x] / \textsubscript{wd}[ {\longrule}{\longrule} back vowel
\z 

\ipi{Neuendorf} is the only variety of German discovered in the present survey requiring a rule backing a palatal rather than one fronting a velar. Since the dialect as it was described in 1953 represents the outgrowth of an earlier one in which a velar fronted to palatal, the conclusion is that \isi{rule inversion} transpired (\citealt{Vennemann1972}, \citealt{McCarthy1991}, \citealt{Blevins2004}, \citealt{Hall2009a}). In the following, I discuss how the original rule of velar fronting inverted itself into \isi{Wd-Initial Palatal Retraction}.\footnote{\label{fn:8:8}On the basis of data from \ili{English} dialects involving intrusive-r, \citet{McCarthy1991} argues that true \isi{rule inversion} (i.e. the replacement of the original rule of \isi{r-Deletion} with \is{r-Epenthesis (English)}r-Epenthesis) never occurred. Instead, the original deletion exists side by side with the innovative rule of \isi{r-Deletion}. In contrast to those \ili{English} dialects, true \isi{rule inversion} occurred in \ipi{Neuendorf}. For discussion of McCarthy’s claim, the reader is referred to \citet{Hall2009a}.}

The emergence of the word-initial velar vs. palatal contrast as it was described in 1953 (=\ref{ex:8:6}) is illustrated with the four representative examples in \REF{ex:8:37}. \ili{WGmc} \textsuperscript{+}[ɣ] (/ɣ/) was restructured to [x] (/x/) by \isi{Wd-Initial ɣ-Fortition}, which surfaced consistently as velar at Stage 1. At Stage 2, \isi{Wd-Initial Velar Fronting-3} (in \ref{ex:8:16}) was phonologized as an allophonic process; hence, the /x/ in /xenzə/ was realized as [ç] because that sound was followed by a front vowel, but the same sound surfaced as [x] before a back vowel or consonant. When \isi{r-Deletion} restructured underlying representations at Stage 3 without /r/ as in the final two examples, [x] and [ç] contrasted in word-initial position before a front vowel.

\ea%37
\label{ex:8:37}
\begin{tabular}[t]{@{} lllll @{}}
 \relax /xɑns/        &    /xɛnz-ə/    & /xrɔːt/       &   /xrɛt-ər/     &         \\
 \relax [xɑns]        & [xɛnzə]        & [xrɔːt]       &  [xrɛtər]       & Stage 1 \\\tablevspace
 \relax /xɑns/        &    /xɛnz-ə/    & /xrɔːt/       &   /xrɛt-ər/     &         \\
 \relax [xɑns]        & [çɛnzə]        & [xrɔːt]       &  [xrɛtər]       & Stage 2 \\\tablevspace
 \relax /çɑns/        &    /çɛnz-ə/    & /xɔːt/        &  /xɛt-ər/       &         \\
 \relax [xɑns]        & [çɛnzə]        & [xɔːt]        & [xɛtər]         &  Stage 3\\\tablevspace
 \relax \textit{Gans} & \textit{Gänse} & \textit{groß} &  \textit{größer}&  \il{Standard German}StG \\
 \relax ‘goose’       & ‘goose-\textsc{pl}’        &  ‘large’      &  ‘larger’       &         \\
\end{tabular}
\z 

The contrast between [x] and [ç] at Stage 3 is significant for two reasons. First, it triggered the \isi{phonemicization} of /ç/ followed by a front vowel in every example given above. That restructuring therefore occurred in [çɛːl] ‘yellow’ in (\ref{ex:8:35c}) without a [x]-alternant, as well as in [çɛnzə] ‘goose-\textsc{pl}’, which alternates with [x] in [xɑns] ‘goose’. Since the original /x/ was restructured to /ç/ in [çɛnzə], the /x/ in the alternant with [x] before a back vowel was likewise restructured, i.e. [xɑns] /xɑns/ > [xɑns] /çɑns/. By contrast, historical /x/ in nonalternating morphemes in (\ref{ex:8:35a}, \ref{ex:8:35b}, \ref{ex:8:35d}) is inherited at Stage 3 without change as /x/. Note that /x/ is the underlying sound here even though [x] never contrasts with [ç] in word-initial position before a back vowel. The same reasoning has been applied to underlying representations in languages like German with fortis-lenis alternations. Thus, underlying representations with a lenis sound are posited for alternating morphemes, e.g. final /d/ in [hʊnt] ‘dog’ vs. [hʊndə] ‘dog-\textsc{pl}’, but underlying representations with fortis sounds are postulated in nonalternating morphemes, e.g. /t/ in [ʃtɑt] ‘city’ (\citealt{Kiparsky1982a}: 17 and subsequent work by many authors).

The second reason the contrast between [x] and [ç] is significant is that it led to \isi{rule inversion}. In all likelihood \isi{rule inversion} in \ipi{Neuendorf} was abrupt. As noted above, Schütze’s description of \ipi{Neuendorf} suggests that \isi{r-Deletion} was a regular (exceptionless) change. Since there was a large number of new r-less words like [xɛtər] ‘larger’ (from \ref{ex:8:34b}) and since there were no restrictions on the type of front stem vowel situated after the deleted rhotic, language learners were confronted a plethora of [x] vs. [ç] contrasts. Those contrasts led to the restructuring of /x/ to /ç/ in pairs of words like [çɛnzə] ‘goose-\textsc{pl}’ and [xɑns]  ‘goose’. The earlier allophonic process of \isi{Wd-Initial Velar Fronting-3} was consequently replaced with \isi{Wd-Initial Palatal Retraction}.\footnote{{In postsonorant position the four dorsal fricatives of \ipi{Neuendorf} are [x ç ɣ ʝ]. The basic pattern is that the palatals surface after coronal sonorants and the velars after back vowels; recall the \il{Eastphalian}Eph variety of \ipi{Dorste} (\sectref{sec:4.4}). However, \citet{Schütze1953} also lists several words in her grammar with opaque palatals, such as [ç] after a back vowel that was historically front, e.g. [dɑːç] ‘dough’ (=⟦dɑχ⟧). Opaque palatals like those are underlying (/ç/) and not derived; see \chapref{sec:9} for similar examples from other dialects. As I demonstrate in \chapref{sec:9}, in dialects where [x] and [ç] contrast after a back vowel velar fronting is present as a rule of \isi{neutralization} in word pairs with \isi{Umlaut} alternations (cf. \il{Standard German}StG [bɑx] ‘stream’ vs. [bɛçə] ‘stream-\textsc{pl}’). Since \ipi{Neuendorf} contrasts [x] and [ç] after a back vowel, velar fronting is present in the synchronic grammar in postsonorant position. Thus, \isi{rule inversion} occurred in \ipi{Neuendorf} only in word-initial position.}}

The word-initial pattern for \ipi{Neuendorf} is apparently unique; no other variety with contrastive [ç] and [x] in word-initial position has been discovered, nor is \isi{r-Deletion} attested in other dialects. The varieties of German spoken closest to \ipi{Neuendorf} are \ipi{Reinhausen} (\il{Eastphalian}Eph; \citealt{Jungandreas1926,Jungandreas1927}; \mapref{map:7}) in Lower Saxony (Niedersachsen) and \ipi{Leinefelde} (\il{Thuringian}Thrn; \citealt{Hentrich1905}; \mapref{map:12}) in Thuringia (Thüringen). In \ipi{Reinhausen} \ili{WGmc} \textsuperscript{+}[ɣ] is realized in word-initial position allophonically as [x] before back vowels and [ç] before front vowels or coronal consonants, although the \isi{palatal quasi-phoneme} /ç/ occurs word-initially before /ʀ/. However, [x] and [ç] do not contrast in initial position. As in other \il{Thuringian}Thrn dialects (and \il{Standard German}StG), the reflex of word-initial \ili{WGmc}  \textsuperscript{+}[ɣ] is [g] in \ipi{Leinefelde}.\il{Eastphalian|)}

\section{{Discussion}}\label{sec:8.6}

\subsection{Velar fronting as a Neogrammarian change}\label{sec:8.6.1}

Velar fronting was phonologized in word-initial position as an allophonic process in all of the \il{Eastphalian}Eph varieties discussed above, but \isi{Glide Hardening} caused its status to change to a \isi{neutralization} in both \ipi{Eilsdorf} and \ipi{Dingelstedt am Huy}. One point not discussed earlier concerns the exceptionless nature of velar fronting. Thus, \ili{WGmc} \textsuperscript{+}[ɣ] shifted to palatal in word-initial position before a front vowel in true Neogrammarian fashion, meaning that there were no deviant items with a word-initial \textsuperscript{+}[ɣ] followed by a front vowel. That allophonic processes {}-- both synchronic and diachronic {}-- are exceptionless is hardly surprising, but the exceptionless nature of word-initial velar fronting has apparently continued even after the rule morphed into a \isi{neutralization}. Examples were provided earlier for morphemes alternating between velar and palatal depending on whether or not the stem vowel showed the effects of a stem vowel mutation such as \isi{Umlaut} in (\ref{ex:8:17}) for \ipi{Eilsdorf} and in (\ref{ex:8:24}) for \ipi{Dingelstedt am Huy}. By definition, \isi{Umlaut} is irregular in the sense that it is difficult if not impossible to predict which morphemes undergo fronting in which morphological context, but the point is that if the umlauted allomorph of a stem is present, then the velar fricative preceding that fronted vowel always shifts to palatal. The exceptionless nature of neutralizations is not unattested in the languages of the world, but many linguists have observed that the shift in status from a rule relating allophones to a \isi{neutralization} often correlates with other changes, including the emergence of idiosyncratic exceptions, as well as the restriction of the rule to derived environments. One example discussed in the literature involves the progression from the originally allophonic rule which voiced (lenited) fricatives /f s θ/ to [v z ð] in \ili{OE} to the \isi{phonemicization} of /v z ð/ and then to the morphologization of the rule in \ili{ME} (\citealt{RingeEska2013}: 141--144; \citealt{Minkova2014}: 89--98). The conclusion drawn on the basis of the material discussed above (and below) is that the correlation described above does not hold in German dialects.

\subsection{Irregularities and analogy}\label{sec:8.6.2}

Both \citet{Block1910} and  \citet{Hille1939} have identified a very small number of items in their respective dialects which contain a word-initial palatal [ʝ] (<\ili{WGmc} \textsuperscript{+}[ɣ]) which is historically opaque because it stands before a back vowel. Those opaque examples can be placed into two categories. In the first category are words where the palatal can be shown to have undergone velar fronting because the back vowel was originally front. In the second category the palatal did not undergo velar fronting because the back vowel was always back.

The number of words belonging to both categories is very small. For \ipi{Eilsdorf} I have found one word belonging to the first category and four words in the second. The numbers are similar for \ipi{Dingelstedt am Huy}. With this in mind, consider the two examples in \REF{ex:8:38} from \ipi{Eilsdorf} \citep{Block1910}:

\ea\label{ex:8:38}
\ea\label{ex:8:38a}  j\k{u}ŋk \tab [ʝʊŋk] \tab ging \tab ‘go\textsc{{}-pret}’      \tab  342
\ex\label{ex:8:38b}  j\k{u}lt \tab [ʝʊlt] \tab galt \tab ‘be valid\textsc{{}-pret}’ \tab  342
\z 
\z 

The corresponding \ili{OSax} etymon \textit{gieng} ‘go\textsc{{}-pret}’ reveals that the fricative in \REF{ex:8:38a} was followed by a historical front vowel. The change from front vowel to back vowel ([ʊ]) in that word can be thought of as specific instance of \isi{Vowel Retraction} (recall \ref{ex:8:8}). From a formal point of view, the palatal in \REF{ex:8:38a} arose just as palatal quasi-phonemes (\chapref{sec:7}): The feature [coronal] of the front vowel of the stem was simultaneously linked to the preceding palatal sound. When the front stem vowel was restructured to a back vowel by \isi{Vowel Retraction}, the feature [coronal] was delinked from the vowel but remained anchored to the palatal, thereby creating the phoneme /ʝ/. That new \isi{phonemic palatal} has an opaque history because it shows that \isi{Vowel Retraction} counterbled velar fronting.

Consider now \REF{ex:8:38b}. The [ʝ] in that item was likewise a historical velar (< \ili{WGmc} \textsuperscript{+}[ɣ]), but it cannot have come about by the sound change that created the palatal in \REF{ex:8:38a} because the stem vowel in \REF{ex:8:38b} was always back (cf. \ili{OSax} \textit{gald}). The question is simple: What is the explanation for the emergence of the irregular palatal in \REF{ex:8:38b}?

The answer did not involve velar fronting in any sense of the word. There are two related reasons for why the palatal in \REF{ex:8:38b} has an explanation that lies outside of the domain of phonology and for why its emergence therefore does not fall into the scope of the present book. First, the change from velar to palatal before a back vowel only occurs in three other words in the \ipi{Eilsdorf} dialect, but that development failed to affect the [ɣ] in all other items beginning with [ɣ]; recall the examples in \REF{ex:8:13a} which are representative of a much larger class of words. Second, the change from velar to palatal in \REF{ex:8:38b} occurs in the context before a back vowel, but both the historical rule (\isi{Wd-Initial Velar Fronting-3}) and the corresponding synchronic rule apply as assimilations, i.e. before front vowels. One cannot deny that many dialects saw a true sound change transforming a velar (\ili{WGmc} \textsuperscript{+}[ɣ]) into palatal [ʝ] in word-initial position before any segment, including back vowels (\chapref{sec:14}). However, as discussed in that chapter, that nonassimilatory development was a true Neogrammarian sound change which applied in many LG and CG varieties without exception.

The most reasonable explanation for the irregular palatal in \REF{ex:8:38b} is \isi{analogy}:  The original velar in \REF{ex:8:38b} was restructured to a palatal (/ʝ/) under the influence of the [ʝ] in  morphologically-related words, e.g. [ʝɪln] ‘be valid\textsc{{}-inf}’. But \isi{analogy} is not phonology. This means that any and all analogical developments involving the change from velar to palatal -- changes that were irregular by definition -- lie outside the domain of this book because they did not involve velar fronting.

\subsection{Rule inversion}\label{sec:8.6.3}
\begin{sloppypar}
The originally allophonic process of velar fronting had a very different fate in \ipi{Neuendorf}. As in \ipi{Eilsdorf} and \ipi{Dingelstedt am Huy}, \ili{WGmc} \textsuperscript{+}[ɣ] shifted to palatal in word-initial position before a front vowel in \ipi{Neuendorf}, but when \isi{r-Deletion} restructured underlying representations, the velar vs. palatal contrast before front vowels led to the restructuring of word-initial /x/ to /ç/ in [x]{\textasciitilde}[ç] alterations. As described above, one of the consequences of that restructuring was \isi{rule inversion}; hence, \isi{Wd-Initial Velar Fronting-3} was replaced with \isi{Wd-Initial Palatal Retraction}.
\end{sloppypar}

Rule inversion has been discussed in a number of works cited earlier (\citealt{Vennemann1972}, \citealt{McCarthy1991}, \citealt{Blevins2004}, \citealt{Hall2009a}). One generalization discussed in that literature is that inverted rules are often typological oddities, two examples being \ili{English} \is{r-Epenthesis (English)}r-Epenthesis (\fnref{fn:8:8}) and \ipi{Imst} German \isi{Buccalization} (/h/→[x] /{\longrule} ]\textsubscript{wd}; \citealt{Hall2009a,Hall2010,Hall2009a,Hall2011a}). The inverted rule of \isi{Wd-Initial Palatal Retraction} in \ipi{Neuendorf} may strike the reader as a counterexample, since it appears to be a clear-cut case involving the assimilation of a front sound to a back sound in the neighborhood of back vowels. However, the featural system adopted in this book does not allow one to characterize that process as an assimilation. The reason is that palatals like /ç/ are [coronal] and [dorsal], velars like /x/ are simplex [dorsal], while back vowels are [dorsal]. The change from /ç/ to [x] in the neighborhood of a back sound therefore requires [coronal] to delete in the context of a complex [coronal, dorsal] sound, clearly a textbook case for an ad hoc change.

\begin{sloppypar}
One might conclude that the featural conundrum described above can be solved by simply replacing that presumably defective featural system with one which enables \isi{Wd-Initial Palatal Retraction} to be expressed as an assimilation. Two points suggest that a reanalysis along those lines would not be prudent. First, \REF{ex:8:36} is the only example attested in the present survey requiring that a palatal shift to velar, while all other varieties necessitate some version of velar fronting (both word-initial and in postsonorant position). Second, \isi{Wd-Initial Palatal Retraction} is the product of \isi{rule inversion}. Since inverted processes are known to be \textsc{crazy} \textsc{rules} (\citealt{BachHarms1972}), I opt to retain the featural system and postulate that palatal to velar retraction rules like the one in \REF{ex:8:36} are not assimilatory. That treatment derives support from the typological literature on Palatalizations, which is silent on whether or not there are rules attested in natural languages that must involve a palatal changing into a velar.\footnote{Since \REF{ex:8:36} does not involve a Palatalization according to any definition of the word, it is understandable that the typological literature on Palatalizations (\sectref{sec:2.3}) has not investigated that type of change. One work to my knowledge in which the change from palatal to velar is discussed from the cross-linguistic perspective is \citet[241--243]{Kümmel2007}. However, his examples involve unconditioned changes or dissimilations. Noticeably absent from his list are languages with rules changing a palatal to a velar in the neighborhood of all back vowels. Kümmel’s material is drawn from \ili{Semitic}, \ili{Indo-European}, and \ili{Uralic}, but no comparable study is known to me at present which addresses the issue (i.e. cases of assimilation of palatals to velars) with a broader source of languages. I consider this to be a potentially promising area for future research.}
\end{sloppypar}

\section{{Areal} {distribution} {of} {word-initial} {phonemic} {palatals}}\label{sec:8.7}

The survey of German dialects in this chapter indicates that phonemic palatals in word-initial position are well-attested throughout North Germany. Several of the dialects investigated in \chapref{sec:11} can be added to the list as well. Tables \ref{tab:8.1}, \ref{tab:8.2}, and \ref{tab:8.3} list varieties of German exemplifying one of the three contrast types defined in \sectref{sec:8.1}. The \il{East Pomeranian}EPo, \il{Low Prussian}LPr, and \il{High Prussian}HPr varieties listed below are indicated on \mapref{map:18}. All of the places listed in Tables \ref{tab:8.1}, \ref{tab:8.2}, and \ref{tab:8.3} are plotted on \mapref{map:14}.

\begin{table}
\caption{\label{tab:8.1}Varieties of WLG and WCG illustrating Contrast Type A}
\begin{tabularx}{.8\textwidth}{XXl}
\lsptoprule
Place & Dialect & Source\\\midrule
\,\ipi{Lathen} & NLG & \citet{Schönhoff1908}\\
\,\ipi{Homberg} & LFr & \citet{Meynen1911}\\
\,\ipi{Kalkar} & LFr & \citet{Hanenberg1915}\\
\,Ronsdorf & Rpn & \citet{Holthaus1887}\\
\,\ipi{Montzen} & Rpn & \citet{Welter1933}\\
\lspbottomrule
\end{tabularx}
\end{table}\il{Ripuarian}

\begin{table}
\caption{\label{tab:8.2}Varieties of LG and \il{High Prussian}HPr illustrating Contrast Type B}
\begin{tabularx}{.8\textwidth}{lll}
\lsptoprule
Place & Dialect & Source\\\midrule
\,\ipi{Magdeburger Börde} & \il{Eastphalian}Eph & \citet{Roloff1902}\\
\,\ipi{Eilsdorf} & \il{Eastphalian}Eph & \citet{Block1910}\\
\,\ipi{Cattenstedt} & \il{Eastphalian}Eph & \citet{Damköhler1919}\\
\,\ipi{Lesse} & \il{Eastphalian}Eph & \citet{Löfstedt1933}\\
\,\ipi{Dingelstedt am Huy} & \il{Eastphalian}Eph & \citet{Hille1939}\\
\,\ipi{Isingerode}/\ipi{Göddeckenrode} & \il{Eastphalian}Eph & \citet{Lange1963}\\
\,Kreis \ipi{Konitz} & \il{East Pomeranian}EPo & \citet{Semrau1915a,Semrau1915b}\\
\,\ipi{Lauenburg} & \il{East Pomeranian}EPo & \citet{Pirk1928}\\
\,Kreis \ipi{Bütow} & \il{East Pomeranian}EPo & \citet{Mischke1936}\\
\,Kreis \ipi{Rummelsburg} & \il{East Pomeranian}EPo & \citet{Mischke1936}\\
\,\ipi{Kamnitz} & \il{East Pomeranian}EPo & \citealt{Tita1921}\\
\,\ipi{Willuhnen} & \il{Low Prussian}LPr & \citet{Natau1937}\\
\,\ipi{Reimerswalde} & \il{High Prussian}HPr & \citet{KuckWiesinger1965}\\
\lspbottomrule
\end{tabularx}
\end{table}


\begin{table}
\caption{\label{tab:8.3}Variety of \il{Eastphalian}Eph illustrating Contrast Type C}
\begin{tabularx}{.8\textwidth}{XXl}
\lsptoprule
Place & Dialect & Source\\\midrule
\,\ipi{Neuendorf} & Eph & \citet{Schütze1953}\\
\lspbottomrule
\end{tabularx}
\end{table}

\begin{map}
% \includegraphics[width=\textwidth]{figures/VelarFrontingHall2021-img020.png}
\includegraphics[width=\textwidth]{figures/Map14_8.1.pdf}
\caption[Areal distribution of word-initial velar vs. palatal contrasts]{Areal distribution of word-initial velar vs. palatal contrasts. Circles represent a contrast between velar ([ɣ]) and palatal ([ʝ]) in word-initial position before front and back vowels. Squares represent a word-initial contrast between velar ([ɣ] or [g]) and palatal ([ʝ]) before back vowels and triangles a word-initial contrast between velar ([x]) and palatal ([ç]) before front vowels.}\label{map:14}
\end{map}

\clearpage An examination of some of the varieties of German spoken in the vicinity of the ones listed in \tabref{tab:8.1} may uncover additional examples of Contrast Type A. Since the phonemic palatals in Contrast Type B arise historically when a trigger for velar fronting is eliminated a more in-depth investigation of the regions affected by the sound changes listed in \REF{ex:8:4} may reveal significant generalizations concerning the areal distribution of word-initial phonemic palatals like the ones in \tabref{tab:8.2}. To the best of my knowledge, \ipi{Neuendorf} is the only variety of German exemplifying Contrast Type C.

\section{Conclusion}\label{sec:8.8}

The case studies discussed above are characterized by word-initial contrasts between velars and palatals. In \chapref{sec:9} I discuss the ways in which velar vs. palatal contrasts can arise in postsonorant position. There it is argued that a \isi{phonemic split} as in \REF{ex:8:4} is triggered in many varieties by \isi{Vowel Retraction}. In contrast to the dialects discussed above, opaque palatals resulting from \isi{Vowel Retraction} are not the result of a sporadic change, but instead represent general developments in postsonorant position. In \chapref{sec:10} I discuss a merger similar to the one in \REF{ex:8:5} which led to the \isi{phonemicization} of the original palatal fricative allophone.

One issue not directly related to the topic of \isi{phonemicization} concerns the set of triggers for velar fronting. In \ipi{Eilsdorf} and \ipi{Dingelstedt am Huy} the rule in question (\isi{Wd-Initial Velar Fronting-3}) is induced by the set of all front vowels; however, examples from other varieties of German discussed in this book point to a broader context for fronting, namely before front vowels or coronal consonants (e.g. \isi{Wd-Initial Velar Fronting-6} in \ipi{Elspe} and \ipi{Schieder-Schwalenberg} in \sectref{sec:7.2}). In any case, both the narrow set of triggers and the broader set of triggers involve assimilatory changes, which stand in contrast to the German varieties investigated in \chapref{sec:14}. In that chapter I demonstrate that many dialects are attested in which word-initial velars (e.g. \ili{WGmc} \textsuperscript{+}[ɣ]) regularly shifted to the corresponding palatals in word-initial position before any type of segment, i.e. front vowels, coronal consonants, and (most significantly) back vowels. That type of change is important because it represents the regular nonassimilatory fronting of velars.
