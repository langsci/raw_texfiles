\chapter{Introduction}\label{sec:1}

\begin{epigraphs}
\qitem{Man betrachte auch eine beliebige Gruppe von verwandten Mundarten; man wird sehen wie die Bedingungskreise der Lautgesetze sich von Ort zu Ort mannigfach verändern, man wird hier gleichsam die räumliche Projection zeitlicher Unterschiede erkennen.\footnotemark{}}{Hugo \citet[24]{Schuchardt1885}}

\qitem{... \textit{Ortsgrammatiken} provide enormous potential for  detailed work by historical linguists and phonologists ... Although much seminal analytic work has already been carried out, there is great need for the individual dialects and their interrelationships to be studied within broader well-grounded theoretical and typological frameworks.}{Robert \citet[82]{Murray2010}}
\end{epigraphs}
\footnotetext{“Just consider any particular group of related dialects. You will see how the conditional environments of the sound laws change from place to place. You will, as it were, perceive the spatial projection of temporal differencesˮ. Translated by \citet{VennemannWilbur1972}.}

\section{{An} {unfortunate} {gap}}\label{sec:1.1}
The distribution of German dorsal fricatives -- palatal [ç] and velar [x] -- has preoccupied linguists of diverse theoretical persuasions for over ninety years. Scholars who have discussed the patterning of those sounds include the following: \citet{Jones1929, Hermann1932, Bloomfield1933,Trubetzkoy1939, Moulton1947, Leopold1948, Jones1950,  Trim1951, Dietrich1953, Trost1958, Heike1961, Freudenberg1966,  Pilch1966, Adamus1967,  Vennemann1968, James1969, Ungeheuer1969, Bluhme1970, Wiesemann1970, Wurzel1970,  Kufner1971, ZacherGriščenko1971, Werner1972, Scholz1972, Werner1973, Issatschenko1973, Standwell1973, Philipp1974, Dressler1977, Griffin1977, Kohler1977b, Russ1978Development, Cercignani1979, Wurzel1980, Russ1982, vanLessenKloeke1982,vanLessenKloeke1982b, MeinholdStock1982, Vennemann1982,  Cercignani1983, Wurzel1983, Lenerz1985,  Benware1986, Lieber1987, Jessen1988, Ronneberger-Sibold1988, Hall1989, MacfarlandPierrehumbert1991, Hall1992, Yu1992, IversonSalmons1992, Borowsky1993, vandeWeijer1994, Wiese1996a,Merchant1996,Noske1997, Grijzenhout1998, Scheer2004, Fox2005, Halle2005, Glover2014}, \citet{Hall2020}, and \citet{Kijak2021}.\footnote{To the best of my knowledge, the earliest work examining German [ç] and [x] from the point of view of phonology was \citet{Jones1929}. Many pre-1929 linguists -- especially those operating in the Neogrammarian (Junggrammatiker) tradition -- have discussed the dorsal fricatives of German and are cited throughout this book. For some general discussion on the status of [ç] and [x] among linguists in the nineteenth and early twentieth centuries see \sectref{sec:1.5}.}\footnote{The patterning of German [ç] and [x] has also been discussed at length in textbooks written in both German and English, e.g.  \citet[9--10]{Hyman1975}, \citet[36--37; 96]{Lass1984}, \citet[75--80; 104--106]{Ternes1987}, \citet[98--101]{RamersVater1991}, \citet[308--309]{Kenstowicz1994}, \citet[7--8; 31]{CowanRakušan1998}, \citet[62--64]{Hall2000}, \citet[62--70]{Féry2001}, \citet[59--63]{Gussmann2002}, \citet[25--28]{Fagan2009}, and \citet[115--117]{O-BrienFagan2016}.}

It would be fair to say that the works cited above have concerned themselves primarily with the distribution of [ç] and [x] in the standard language of Germany, namely Standard German (\il{Standard German}StG) -- defined here as the pronunciation encoded in the pronouncing dictionaries (\citealt{Siebs1969}, \citealt{Krech1982}, \citealt{Mangold2005}) -- but that they have said very little about the occurrence of those fricatives in regional German dialects. Two notable exceptions to that trend are \citet{Herrgen1986} and especially \citet{Robinson2001}, who both stress that much light can be shed on the correct analysis of the \il{Standard German}StG facts by considering the patterning of [ç] -- the so-called \textsc{ich-Laut} (“ich-soundˮ) -- and [x]  -- the so-called \textsc{ach-Laut} (“ach-soundˮ) -- in non-standard varieties of German. 

I contend that the cross-dialectal approach advocated by linguists such as Herrgen and Robinson represents a step in the right direction but that neither of those linguists goes far enough. In fact, it will be clear in the following chapters that those works merely scratch the surface of a deceptively complicated beast by failing to consider enough case studies from geographically-diverse regional dialects. 

The topic addressed in this book has not only been neglected by phonologists, but also by dialectologists. To cite one recent example, volume 4 of the \textit{Handbücher zur Sprach- und Kommunikationswissenschaft} \citep{HerrgenSchmidt2019} provides an impressive 1200 page overview of German dialects. That survey includes all of the dialect areas depicted on \mapref{map:44}, including varieties of German spoken in North and South America, Africa, Australia, and Oceania. Given the breadth of that state-of-the-art work, it is surprising that none of the chapters discuss the distribution of [ç] and [x] in any detail.

The goal of the present study is to fill that gap. I consider over three hundred  original sources for all of the major dialect regions spoken over a period of about one hundred sixty years (1860 to 2020) throughout the German-speaking world as it existed before 1945 up to the present day. In doing so I uncover a wealth of new data (hinted at in the Murray quote given above) involving the patterning of velar and palatal sounds. It is my hope that the data and my analysis thereof will redefine the kind of research question future works will address with respect to the patterning of German dorsal fricatives.

 In the case studies presented below I demonstrate that the phonology of palatals (such as [ç]) and velars (such as [x]) can differ from one dialect to the next in subtle but also predictable ways. The synchronic differences among dialects referred to here will be argued to mirror the way in which the original rule relating those sounds progressed historically from a low-level phonetic process to a phonological rule. The latter process has subsequently undergone changes in some varieties resulting in various idiosyncrasies not discussed in previous research that only make sense when those dialects are compared with other dialects without those quirks.  

The remainder of this chapter is structured as follows. In \sectref{sec:1.2} I provide a brief overview of the patterning of dorsal fricatives in \il{Standard German}StG and summarize some of the contentious research questions that have been the object of debate in the past. \sectref{sec:1.3} explicates the title of this book as it relates to \il{Standard German}StG and to the new data from German dialects, which are outlined briefly in \sectref{sec:1.4}. The latter section also  poses a series of new research questions regarding the new patterns exemplified in German dialects. \sectref{sec:1.5} justifies the assumption made in the present work -- echoed in the extensive literature referred to earlier -- that the phonology of German need only refer to two dorsal articulations (velar and palatal) but not to finer-grained distinctions. In \sectref{sec:1.6} I provide some remarks on the data and sources thereof. Finally, \sectref{sec:1.7} gives a brief outline of the structure of the remaining chapters. 

\section{Standard German facts and summary of previous research}\label{sec:1.2}

The words listed in \REF{ex:1:1} reveal that [x] surfaces after back vowels (e.g. [uː] in \ref{ex:1:1a}) and [ç] after front vowels (e.g. [ɪ] in \ref{ex:1:1b}), the two coronal sonorant consonants ([l] and [n] in \ref{ex:1:1c}), or the dorsal rhotic, which surfaces in coda position after a short vowel as the uvular consonant [ʀ] or the vowel [ɐ]. No native word has a dorsal fricative in word-initial position. See \chapref{sec:17} for more extensive discussion on the patterning of \il{Standard German}StG dorsal fricatives.

\TabPositions{0pt, 0.25\linewidth, 0.45\linewidth}
\ea%1
    \label{ex:1:1}
\ea{}\label{ex:1:1a}  [buːx]         \tab Buch   \tab ‘book’
\ex{}\label{ex:1:1b}  [lɪçt]         \tab Licht  \tab ‘light’
\ex{}\label{ex:1:1c}  [dɔlç]         \tab Dolch  \tab ‘dagger’\\{}
                      [mœnç]         \tab Mönch  \tab ‘monk’
\ex{}\label{ex:1:1d}  [dʊʀç], [dʊɐç] \tab durch  \tab ‘through’
\z
\z

The data in \REF{ex:1:1} show that [x] and [ç] stand in complementary distribution: The former sound occurs after a back vowel and the latter one after a front vowel, a liquid, or /n/. 

The German language also possesses many instances of alternations involving [x] and [ç]. For example, if a noun in the singular has a back vowel followed by [x], and if that back vowel is fronted in the plural, then [x] is realized as [ç], e.g. [buːx] ‘book’ vs. [byːçɐ] ‘book-\textsc{pl}’.

The examples in \REF{ex:1:2} illustrate that there are morphemes displaying an alternation between [g] and [ç]. That type of morpheme is usually captured in the literature by positing an underlying lenis stop (/g/) that shifts to the corresponding fricative and surfaces as [ç]. Note that the [ç] derived from /g/ in \REF{ex:1:2a} -- like the [ç] in \REF{ex:1:1b} -- surfaces after a front vowel ([ɪ]).

\ea\label{ex:1:2}
\ea{}  [køːnɪç]  \tab König  \tab   ‘king’\label{ex:1:2a}
\ex{} [køːnɪgə]  \tab Könige \tab   ‘king-\textsc{pl}’\label{ex:1:2b}
\z
\z

\begin{sloppypar}
The generalizations described in the preceding paragraphs need to be amended in light of the additional examples \REF{ex:1:3}, which show two additional contexts for palatal [ç]. First, [ç] surfaces as the first segment in the diminutive \is{chen@\textit{-chen}}suffix -\textit{chen} even if a back vowel precedes that suffix in (\ref{ex:1:3a}). Second, [ç] occurs word-initially in loanwords in (\ref{ex:1:3b}).
\end{sloppypar}

\ea%3
    \label{ex:1:3}
\ea{}  [tɑuçən] \tab Tauchen \tab ‘rope-\textsc{dim}’  (cf. [tau] Tau ‘rope’)\label{ex:1:3a}
\ex{}  [çemiː]  \tab Chemie  \tab ‘chemistry’\label{ex:1:3b}
\z
\z

The \il{Standard German}StG data presented above -- especially those involving palatal [ç] in \is{chen@\textit{-chen}}-\textit{chen} in \REF{ex:1:3a} -- have spawned a sizable literature couched in a wide variety of theoretical frameworks, some of which was cited in \sectref{sec:1.1}. Due to the near complementary distribution of [x] and [ç], there is widespread agreement that the two fricatives are positional variants, in which case one of the sounds derives from the other.

One question discussed at length in the literature is whether or not the underlying sound -- the \textsc{target} -- is velar or palatal. Thus, if /ç/ is taken to be basic then the rule creates [x] -- the \textsc{output} -- after back vowels -- the \textsc{triggers} --, as in \REF{ex:1:4a}. Note that the underlying palatal treatment also accounts for the occurrence of [ç] in \REF{ex:1:1c} and \REF{ex:1:3b} as well as [x] in \REF{ex:1:1a}. However, that treatment needs to account for the fact that [ç] surfaces after the back vowel [ɐ] in items like [dʊɐç] ‘through’ in \REF{ex:1:1d} and in the diminutive \is{chen@\textit{-chen}}suffix -\textit{chen} in \REF{ex:1:3a}, e.g. [tɑuçən] ‘rope-\textsc{dim}’. If /x/ is taken as the target segment and [ç] is derived from that sound, then the rule apparently necessitates two disjunct contexts, as in \REF{ex:1:4b}. Although the two sets of triggers in \REF{ex:1:4b} can account technically for the data in \REF{ex:1:1}, it is not clear how that rule generates the palatal in \REF{ex:1:3}. What is more, if the rule involved is an assimilation, then \REF{ex:1:4b} is incomplete at best because it needs to provide a convincing argument for the occurrence of the palatal after the dorsal rhotic [ʀ] and its non-front variant [ɐ] in \REF{ex:1:1d}.\footnote{In my comparison of \REF{ex:1:4a} and \REF{ex:1:4b} I abstract away from how the [g]{\textasciitilde}[ç] alternations in \REF{ex:1:2} are analyzed.}

\ea\label{ex:1:4}
\ea\label{ex:1:4a}
% /ç/ → [x] /  $\left\{\mathit{back} \mathit{vowels}\right\}$ {\longrule}{\longrule}
\phonrule{/ç/}{[x]}{$\left\{\text{back vowels}\right\}$ {\longrule}}
\ex\label{ex:1:4b}
\phonrule{/x/}{[ç]}{$\left\{\begin{matrix}\text{front vowels}\\
                    \text{sonorant consonants}\end{matrix}
                    \right\}$ 
                    {\longrule}
}
\z
\z

In an important study, \citet{Robinson2001} defends an analysis in which the underlying segment in the rule relating [x] and [ç] is /x/ and not /ç/. He argues at length that it is possible -- and desirable -- to analyze the two disjunct groups of triggers in \REF{ex:1:4b} featurally in a unified way so that the change expressed is an assimilation. In addition, he sees palatal [ç] in \REF{ex:1:3b} as a nonnative phoneme /ç/, which is also his treatment of the palatal [ç] (/ç/) in the loan \is{chen@\textit{-chen}}suffix -\textit{chen} in \REF{ex:1:3a}.

Robinson makes a compelling case for deriving palatals from velars; in fact, I adopt that position and criticize the “palatal to velarˮ alternative in \REF{ex:1:4a}. A consequence of the approach with underlying velars is that it necessitates an answer to the question of how the two categories “front vowelsˮ in (\ref{ex:1:1b}) and “/n l ʀ/ˮ in (\ref{ex:1:1c}, \ref{ex:1:1d}) can be united as the set of triggers. It needs to be stressed that Robinson’s analysis of German dorsal fricatives -- his argument for underlying velars being one example -- hinges crucially on data from regional varieties of German not usually discussed in the published literature. This is a significant point because the implication is that Robinson’s claims can potentially be falsified by data from German dialects he might not have considered. 

Robinson makes two assertions I strongly dispute. First, his analysis of dorsal fricatives implies that there is a single pandialectal rule (p. 113), according to which a palatal fricative is derived from a velar. Second, Robinson opines that his rule is “completely automaticˮ (p. 19) and that it is a “low-level, \isi{phonetic rule}ˮ (p. 77).

In the present book I evaluate and reject the two claims described in the preceding paragraph. I do so by investigating the patterning of dorsal fricatives in a number of regional German dialects discussed neither by \citet{Robinson2001} nor -- to the best of my knowledge -- by any of the other linguists cited above. In the course of that discussion I uncover new data and develop a program of research involving those data, which I summarize briefly below.

\section{{Definition} {of} {velar} {fronting}}\label{sec:1.3}

As noted in the previous section I adopt the position asserting that the distribution of [x] and [ç] in \il{Standard German}StG requires an underlying velar (/x/) to be realized as a palatal ([ç]) and not the reverse. The rule relating those two sounds -- stated provisionally in \REF{ex:1:4b} -- is referred to throughout this book as \textsc{Velar} \textsc{Fronting}, which can be thought of as a subtype of \textsc{velar} \textsc{palatalization} (e.g. \citealt{Bateman2007}). See \sectref{sec:2.3} for more in-depth discussion.

I define velar fronting henceforth as the realization of any velar consonant -- and not simply [x] -- as the corresponding fronted sound. The change in question can be diachronic or synchronic. The way in which velar fronting is characterized is necessarily determined by the German dialect data summarized briefly in \sectref{sec:1.4} -- data revealing that the targets can be drawn from the sounds listed in the velar column in \tabref{tab:1.5a}. The output of velar fronting is the corresponding palatal sound, as indicated in the final column of \tabref{tab:1.5a}. The term velar fronting also refers to the realization of the fricative [x] as the alveolopalatal (\isi{sibilant}) fricative [ɕ], as in \tabref{tab:1.5b}. As indicated below, I classify both the target (velar) and the output (palatal/alveolopalatal) as dorsal.\footnote{In my description of the phonology of velar fronting I have intentionally refrained from providing phonetic detail, e.g. the exact position of the tongue or the formant structure of velars and palatals\slash alveolopalatals. See \sectref{sec:1.5}, where I conclude that a proper understanding of velar fronting in German dialects does not (and should not) require reference to fine-grained phonetic detail.}

\begin{table}
\caption{Velar fronting targets and outputs\label{tab:1:5}}
\subfigure[Palatal output\label{tab:1.5a}]{
\begin{tabular}{lcc}
\lsptoprule
& \multicolumn{2}{c}{dorsal}\\
\cmidrule(lr){2-3}
& velar  & palatal\\
& (target) & (output)\\\midrule
Stop      & [k] & [c]\\
          & [g] & [ɟ]\\
Fricative & [x] & [ç]\\
          & [ɣ] & [ʝ]\\
Affricate & [kx] & [kç]\\
Nasal     & [ŋ] & [ɲ]\\
\lspbottomrule
\end{tabular}
}
\subfigure[Alveolopalatal output\label{tab:1.5b}]{
\begin{tabular}{lcc}
\lsptoprule
 & \multicolumn{2}{c}{dorsal}\\
 \cmidrule(lr){2-3}
 & velar & alveolopalatal\\
 & (target) & (output)\\
 \midrule
Fricative &  [x] & [ɕ]\\
\lspbottomrule
\end{tabular}
}
\end{table}

Velar fronting triggers typically consist of front vowels -- a change I interpret as an assimilation; recall \REF{ex:1:1a} vs. \REF{ex:1:1b}. However, the set of triggers can also include front (coronal) sonorant consonants like [n] and [l]; recall \REF{ex:1:1c}. A surprising finding is that velar fronting is not always assimilatory because it can occur in many varieties of German in the context of any segment, front or back.\footnote{A claim I justify at length in the following chapters is that the assimilatory change from velar to palatal is never triggered by (dorsal) sounds like [ʀ] or [ɐ]; recall \REF{ex:1:1d}.}

An important finding in this book concerns the \textsc{directionality} of velar fronting. If a target for velar fronting is situated between two sonorant sounds then the trigger for velar fronting is always to the left of the target. This means that velar fronting involves assimilation from left-to-right (\textsc{progressive}) and not from right-to-left (\textsc{regressive}) assimilation.

In sum, velar fronting is defined according to the four properties listed in \REF{ex:1:6}.\footnote{The data I discuss in this book (summarized in \sectref{sec:1.4}) also involve velar fronting in word-initial position. As stated below, property \REF{ex:1:6d} only holds if the target is not word-initial.}

\ea%6
\label{ex:1:6}
\ea\label{ex:1:6a}Targets: Any velar consonant
\ex\label{ex:1:6b}Triggers: Typically (but not always) a front segment
\ex\label{ex:1:6c}Outputs: Palatal or alveolopalatal
\ex\label{ex:1:6d}Direction: Left-to-right
\z
\z

The preceding discussion should reveal the inappropriateness of the term \textsc{Dorsal} \textsc{Fricative} \textsc{Assimilation}, which is probably the most common way of referring to the rule of \il{Standard German}StG in (\ref{ex:1:4a}, \ref{ex:1:4b}) in the recent literature. First, the process in question is not always assimilatory, and second, the target segments need not be fricatives.

\section{New data and new research questions}\label{sec:1.4}

\subsection{Parameters of variation and opacity}\label{sec:1.4.1}
\subsubsection{Overview}
This book offers an in depth investigation of velar fronting (as defined above) in German dialects. That change can involve either the diachronic fronting of a historical velar or the synchronic realization of an underlying velar as fronted (palatal). 

Before introducing the new data referred to above, I clarify the object of investigation, namely “dialects of Germanˮ. In this book I refer to both High German (HG) and Low German (LG) under the same category label (“Germanˮ), although it needs to be stressed that those two groups are different enough that they should be probably considered separate, but closely related, Continental West Germanic (\ili{WGmc}) languages. My view of HG and LG as separate languages -- and not as dialects of the same language -- is also implicit in the family tree in Appendix~\ref{appendix:e}. I offer no new definition of “dialectˮ and therefore simply assume without argument that that term refers to any regionally distinctive variety of a language (in this case either HG or LG). Thus, Westphalian (\il{Westphalian}Wph) and Eastphalian (\il{Eastphalian}Eph) are two LG dialects, while Swabian (\il{Swabian}Swb) and Bavarian (Bav) are two HG dialects. On the other hand, I also employ the word “dialectˮ to refer to very specific regional varieties within any one of those larger categories. For example, there may be two towns in the \il{Swabian}Swb-speaking region of Germany separated by a mere 10 km, and yet I refer to the HG (\il{Swabian}Swb) language spoken in those two towns as two separate “dialectsˮ. The dialects discussed below are almost always defined in terms of space (regionally), but “dialectsˮ in the present context may also be distinguished in terms of socio-linguistic variables. Seen in that light, my usage of the term “dialectˮ is equivalent to the more general term “varietyˮ, and for that reason I often employ “dialectˮ and “varietyˮ interchangeably.   

 The new data investigated in the present work involve velars and palatals in two contexts: (i) after a vowel or a sonorant consonant (henceforth \textsc{postsonorant} \textsc{position}), or (ii) word-initial position. Examples exemplifying (i) are listed in the final column of \REF{ex:1:7}. The corresponding \ili{WGmc} reflexes for the modern-day velars and palatals are provided in the first column. Appendix~\ref{appendix:f} summarizes those developments into HG varieties on which \il{Standard German}StG are based. Phonetic representations for the words listed below are not indicated because those realizations differ from dialect to dialect. The phonetic representation for the vowels in these words can be inferred on the basis of the orthography. In many regional varieties velar fronting can also affect the lenis velar fricative (\ili{WGmc} \textsuperscript{+}[ɣ]), as in \REF{ex:1:7c}, and in others velar stops and the velar nasal, as in (\ref{ex:1:7d}, \ref{ex:1:7e}).

\TabPositions{0pt, .4\linewidth}
\ea%7
    \label{ex:1:7}
\ea\label{ex:1:7a}WGmc \textsuperscript{+}[k] >   [x]/[ç]   \tab Sache ‘thing’, brechen ‘break-\textsc{inf}’, \\ \tab Dolch ‘dagger’
\ex\label{ex:1:7b}WGmc \textsuperscript{+}[x] >   [x]/[ç]  \tab Nacht ‘night’, dicht ‘dense’,\\ \tab fechten ‘fence-\textsc{inf}’
\ex\label{ex:1:7c}WGmc \textsuperscript{+}[ɣ] >   [ɣ]/[ʝ]  \tab Wagen ‘car’, liegen ‘lie-\textsc{inf}’, \\ \tab folgen ‘follow-\textsc{inf}’
\ex\label{ex:1:7d}WGmc \textsuperscript{+}[kk] > [k]/[c]  \tab Rock ‘skirt’, dick ‘fat’
\ex\label{ex:1:7e}WGmc \textsuperscript{+}[ŋg] > [ŋg]/[ɲɟ] \tab Zunge ‘tongue’, Finger ‘finger’
\z
\z

Word-initial velars (=context ii) can also show the effects of fronting. I list below typical lexical items and their \ili{WGmc} reflexes for the word-initial fricatives [x ç] in (\ref{ex:1:8a}) and for [x ç] after a word-initial \isi{sibilant} in (\ref{ex:1:8b}). Some varieties also front velar stops, as in \REF{ex:1:8c}. 

\ea%8
\label{ex:1:8}
\ea\label{ex:1:8a}WGmc \textsuperscript{+}[ɣ] >   [x]/[ç] \tab Gast ‘guest’, gestern ‘yesterday’,\\ \tab Glück ‘fortune’
\ex\label{ex:1:8b}WGmc \textsuperscript{+}[sk] > [sx]/[sç] \tab Schaf ‘sheep’, schöpfen ‘ladle-\textsc{inf}’,\\ \tab schlafen ‘sleep-\textsc{inf}’
\ex\label{ex:1:8c}WGmc \textsuperscript{+}[k] >   [k]/[c] \tab Kuh ‘cow’, Kind ‘child’
\z
\z

Individual varieties of German can possess either postsonorant velar fronting in \REF{ex:1:7}, word-initial velar fronting in \REF{ex:1:8}, or both. Those fronting processes exhibit variation along the three parameters listed in (\ref{ex:1:6a}--\ref{ex:1:6c}).

\subsubsection{Targets} In many varieties the set of target sounds consists of all and only velar fricatives (both [x] and [ɣ]); hence, palatals occur in \textit{brechen} ‘break-\textsc{inf}’, \textit{dicht} ‘dense’, and \textit{liegen} ‘lie-\textsc{inf}’ and velars [x ɣ] after back vowels, e.g. \textit{Sache} ‘thing’, \textit{Wagen} ‘car’. However, in other dialects (e.g. \il{Westphalian}Wph) the target for fronting consists solely of [x] but not [ɣ]; hence, we have palatal [ç] in \textit{brechen} ‘break-\textsc{inf}’ and \textit{dicht} ‘dense’ and velar [x] in \textit{Sache} ‘thing’, but velar [ɣ] surfaces in both \textit{Wagen} ‘car’ and \textit{liegen} ‘lie-\textsc{inf}’. In another set of dialects (e.g. High Prussian (\il{High Prussian}HPr)), velar fronting affects not only [x] and [ɣ], but also velar stops and the velar nasal; hence, [c] surfaces in \textit{dick} ‘fat’, [ɲ] in \textit{Finger} ‘finger’, [k] in \textit{Rock} ‘skirt’, and [ŋ] in \textit{Zunge} ‘tongue’.

\begin{sloppypar}
A similar generalization involving targets holds for word-initial position; hence, some dialects (e.g. \il{Westphalian}Wph) have [ç] in \textit{gestern} ‘yesterday’ and \textit{schöpfen} ‘ladle-\textsc{inf}’ and [x] in \textit{Gast} ‘guest’ and \textit{Schaf} ‘sheep’, while others also have [c] in \textit{Kind} ‘child’ and [k] in \textit{Kuh} ‘cow’.
\end{sloppypar}

\subsubsection{Triggers} The set of front vocalic triggers for velar fronting exhibits variation according to the height dimension: In some systems the triggers consist only of high front vowels, in others high and mid front vowels but not the low front vowels (e.g. [æ]), and in yet others all front vowels, regardless of height. For example, in some dialects (e.g. Highest Alemannic (\il{Highest Alemannic}HstAlmc)) [ç] surfaces after the high front vowel in \textit{dicht} ‘dense’ but [x] after the mid front vowel in \textit{fechten} ‘fence-\textsc{inf}’ and after the back vowel in \textit{Nacht} ‘night’.

The fronting of velars can also be induced by a coronal sonorant consonant ([r l n]). In one commonly occurring system (e.g. \il{Westphalian}Wph), that change is triggered by all front vowels and all of those consonants, e.g. [ç] in \textit{gestern} ‘yesterday’ and \textit{Glück} ‘fortune’, but [x] in \textit{Gast} ‘guest’. However, in other varieties (also \il{Westphalian}Wph) only front vowels, but not coronal sonorant consonants trigger fronting; e.g. [ç] in \textit{gestern} ‘yesterday’ and [x] in \textit{Glück} ‘fortune’ and \textit{Gast} ‘guest’.

In many localities, velar fronting has no segmental trigger at all. That type of system is particularly common in word-initial position, e.g. palatal [ç] or [ʝ] occur in \textit{Gast} ‘guest’, \textit{gestern} ‘yesterday’, and \textit{Glück} ‘fortune’, while the corresponding velars are absent from word-initial position entirely. In this type of dialect (e.g. Ripuarian (\il{Ripuarian}Rpn)), velar fronting is therefore not an assimilation.

\subsubsection{Outputs} The segment created by the fronting of velar /x/ in \il{Standard German}StG and in most German dialects investigated below is a (nonsibilant) palatal fricative [ç], although there is also a well-attested pattern whereby [ç] is realized as a (sibilant) alveolopalatal fricative [ɕ] (e.g. \il{Ripuarian}Rpn); recall \tabref{tab:1.5b}. For example, a word like [ɪç] ‘I’ is pronounced in \il{Ripuarian}Rpn as [ɪɕ]. I refer to the change to the [ɕ] output as \textsc{alveolopalatalization}.

\subsubsection{Transparency/opacity} A major theme of this book is the ways in which velar fronting interacts with synchronic and diachronic changes creating or eliminating structures which can potentially undergo it or trigger it. The types of interaction referred to here are categorized in terms of the criterion referred to as \textsc{transparency}\slash\textsc{opacity}.

In many dialects the relationship between velars (e.g. [x]) and the corresponding palatals ([{ҫ}]) is \textsc{transparent} in the sense that velars only occur in the back vowel context and palatals only when adjacent to front sounds. In that type of system, independent processes can interact with velar fronting in two ways: (a) They can \textsc{\isi{feed}} velar fronting (by creating additional structures which the latter can undergo); or (b) they can \textsc{\isi{bleed}} velar fronting (by eliminating potential structures to which the latter could apply). For example, in one Central Bavarian (\il{Central Bavarian}CBav) dialect historical /r/ is now realized as [x] after back vowels (e.g. \textit{schwarz} [ʃwɔxts] ‘black’) and [{ҫ}] after front vowels (e.g. \textit{Herz} [hɛ{ҫts] ‘heart’)}. The change from /r/ to a dorsal fricative therefore \isi{feeds} velar fronting. In another dialect the historical front vowel (diphthong) [ei] is now realized as the back vowel (diphthong) [ɔə]. Significantly, the historical fricative after that new back vowel is realized as [x] (e.g. \textit{Zeichen} [tsɔəxə] ‘sign’); hence, the change from [ei] to [ɔə] \isi{bleeds} velar fronting. When [x] and [{ҫ] have a transparent relationship they stand in complementary distribution and are classified as} \textsc{allophones}.

The transparent relationship between velars and palatals described above does not obtain in other dialects. For example, in many varieties, both dorsal articulations occur in the context of front segments. Thus, in addition to expected sequences like [iç], there are also unexpected ones like [ix]. In other systems velars and palatals both occur in the context of back segments; hence, expected sequences such as [iç] occur in addition to unexpected ones like [ɑç]. Both types of system exhibit \textsc{opacity} (e.g. \citealt{Kiparsky1982a}, \citealt{McCarthy2009, Baković2011}); in particular, sequences like [ix] in the first set of dialects and [ɑç] in the second set are \textsc{opaque}. A sequence like [ix] illustrates \textsc{underapplication} because the fronting of velars fails to affect [x]. By contrast, a sequence like [ɑç] displays \textsc{overapplication} because the process fronting velars in the context of front vowels apparently even applies after certain back vowels.

\subsubsubsection{Underapplication} There are two types that need to be distinguished:

In one system velar fronting can be shown to be an active synchronic process creating palatals (e.g. [ç]) from velars (e.g. /x/). The opaque velar ([x]) in the front vowel context (e.g. [ix]) is derived -- both synchronically and diachronically -- from an independent segment (represented with the abstract symbol /A/). Significantly, the only instances of opaque [x] involve [x] created from /A/. For example, in some varieties of Central German (CG) there are words like \textit{Licht} [lɪçt] ‘light’ and \textit{Bach} [bɑx] ‘stream’, where [ç] is the product of velar fronting from /x/. The same varieties have opaque words like \textit{Hirsch} [hɪxʃ] ‘deer’, in which the surface [x] is the realization of /ʀ/. Words like [hɪxʃ] illustrate \isi{underapplication} and the rule creating [x] from /ʀ/ \textsc{\isi{counterfeeds}} velar fronting.

In another type of system (\il{Highest Alemannic}HstAlmc), velar fronting is active synchronically, but [x] surfaces unexpectedly in the context of front vowels in certain diphthongs. For example, [ç] (from /x/) occurs in \textit{weich} [weiç] ‘soft’ and \textit{leicht} [liːçt] ‘easy’, and [x] in \textit{nah} [nɑːx] ‘near’. However, [x] (/x/) also occurs after the diphthong [øi] (/øi/), e.g. \textit{Rauch} [røix] (/røix/) ‘smoke’. An important generalization is that [øi] was historically a back vowel ([ou]; cf \il{Standard German}StG [ʀɑux]). Opaque velars like [x] occur after segments like [øi], which are referred to below as \textsc{neutral} \textsc{vowels}. Since [øi] is synchronically /øi/ and not /ou/, systems with neutral vowels do not involve a \isi{counterfeeding order}, as described in the preceding paragraph. However, the fronting of that originally back sound [ou] to [øi] does exemplify the historical \isi{underapplication} of velar fronting.

\subsubsubsection{Overapplication} Two types are discussed below:

In one frequent pattern, palatals (e.g. [ç]) occur in the context of front vowels and certain nonfront sounds -- represented here as [Bk] -- and velars (e.g. [x]) in the context of  nonfront sounds with the exception of [Bk]. Observe that palatal ([ç]) and velar ([x]) stand in complementary distribution. All instances of the palatal ([ç]) in the context of front vowels derive -- both synchronically and diachronically -- from the corresponding velar, but the opaque palatal ([ç]) in the context of [Bk] is underlying (/ç/) and not derived. Underlying (opaque) palatals in that type of system are referred to throughout this book as \textsc{quasi-phonemes}. For example, in one North Hessian (\il{North Hessian}NHes) dialect [ç] occurs after front vowels, e.g. \textit{brechen} [brɛçə] ‘break-\textsc{inf}’, and [x] after all back vowels with the exception of [ɑː], e.g. [bux] ‘book’. After [ɑː] the palatal surfaces, e.g. \textit{schlecht} [ʃlɑːçt] ‘bad’. Significantly, the back sound adjacent to all palatal quasi-phonemes ([Bk]) was historically front, e.g. the [ɑː] in [ʃlɑːçt] was once [e], but it is now synchronically back (/Bk/).

In another type of system, velars and palatals both surface in the neighborhood of back sounds. Since they can occur in the context of the same back vowels, velars and palatals \textsc{contrast} in that context; hence, velars and palatals are both underlying sounds (e.g. /x/ and /ç/). Underlying palatals in that type of system are referred to throughout this book as \textsc{phonemic} \textsc{palatals}. In dialects where palatals and velars are both phonemic, velar fronting can still be shown to be active synchronically in order to capture regular (i.e. exceptionless) alternations between velars and palatals. For example, in several varieties of Central Hessian (\il{Central Hessian}CHes), [x] and [ç] contrast after the back vowel [ɑ], e.g. \textit{Dach} [dɑx] ‘roof’ (/dɑx/) vs. \textit{Deich} [dɑç] ‘dike’ (/dɑç/). The same dialects also have regular alternations involving [x] and [ç], e.g. \textit{Buch} [bux] ‘book’ vs. \textit{Bücher} [biçər] ‘books’. In that type of alternation, /x/ is the underlying dorsal sound and [ç] is created by velar fronting after a front vowel. Significantly, the back vowel adjacent to the opaque (underlying) palatal was historically a front sound (e.g. the [ɑ] in [dɑç] ‘dike’) was once [ei]; hence, velar fronting overapplies in words like [dɑç] ‘dike’ from the diachronic perspective.

\subsection{Interpretation of the dialect data}\label{sec:1.4.2}

In the following chapters I present case studies for specific varieties of German illustrating the range of phenomena described above. From the synchronic perspective several versions of velar fronting are posited, which can differ according to the parameters listed in (\ref{ex:1:6a}--\ref{ex:1:6c}). Synchronic velar frontings in German dialects have a historical interpretation, which I summarize briefly here:

At a very early stage (West Germanic (\ili{WGmc})) velar fronting was not present at all; hence, velars like [x] surfaced as velar ([x]) even in the neighborhood of front vowels. That earlier stage is represented by a modern WGm language (\ili{Dutch}). It is hypothesized that velars in the high front vowel context were realized in early stages of \ili{Old High German} (OHG) and \ili{Old Saxon} (OSax) as slightly more front than in the elsewhere context where they surfaced as true velars but that this type of fronting was phonetic (\textsc{gradient}) and not phonological (\textsc{categorical}). At a later stage of \ili{OHG}/\ili{OSax} the difference between velars in the high front vowel context and velars in the elsewhere case became exaggerated to the point where the former were realized (categorically) as palatal, while the latter remained velar. This is the stage at which velar fronting was \textsc{phonologized}. At that phonologized stage, velar fronting was present as a synchronic (allophonic) process, and the set of targets consisted solely of the fortis velar fricative [x] and the triggers consisted of the high front vowels like [i].

Phonologization occurred at a particular place (see below). The original rule of velar fronting then spread temporally and geographically to include a greater set of targets and/or triggers; see \citet{Bermúdez-Otero2015} for a similar conception of language change. For example, the targets could spread to include not only fortis [x] but also lenis [ɣ]. The set of triggers could likewise later grow to subsume high and mid front vowels and then all front vowels.

Variation in terms of space (regional dialects) directly reflects changes along the temporal dimension (recall the Schuchardt quote from the beginning of \sectref{sec:1.1}). That interpretation of spatial variation is applied in the present book to velar fronting. Hence, I demonstrate that the various patterns displayed by modern dialects gives important clues telling us which regions have had velar fronting longer than others.

The evidence is strong that the \isi{phonologization} of velar fronting and the subsequent expansion of triggers and targets occurred independently at more than one place (\textsc{polygenesis}). Evidence against a single point of origin (\textsc{monogenesis}) is that there are innovative velar fronting varieties surrounded by conservative varieties preserving velar sounds even in the front vowel context (\textsc{velar} \textsc{fronting} \textsc{islands}).

When velar fronting was expanding through time and space to include more and more targets and triggers, the velar ([x]) and the corresponding palatal ([ç]) stood in a transparent (allophonic) relationship. Changes affecting the velar fronting target could interfere with the original allophonic nature of velar fronting and then produce opaque forms. For example, when original front vowel triggers shifted to back vowels, \isi{overapplication} effects could set in, i.e. palatal quasi-phonemes and/or contrasts between velars and palatals in the back vowel context. Likewise when original back segments were realized as front, \isi{underapplication} might ensue, i.e. counterfeeding \isi{opacity} or neutral vowels.

\subsection{Velar fronting from the typological perspective}\label{sec:1.4.3}

Rules fronting velar consonants to palatal (or palatal-like) sounds have been intensively investigated in previous research, e.g. \citet{Bhat1978,Guion1998, Bateman2007,Bateman2011}, \citet{Kochetov2011}, and \citet{Recasens2020}. That typological literature has concerned itself with the ways in which the parameters in \REF{ex:1:6} can vary from language to language. A natural question to ask is how the data from velar fronting in German dialects fit into that typological research.

For example, a significant finding in the literature cited above is that the front vowels inducing the fronting of a velar can refer to the height dimension, where\-by high front vowels are more favorable triggers than nonhigh front vowels. As described at length below, that finding is corroborated in my survey of German dialects. Therefore, one of the goals of this book is to consider the extent to which claims made in the typological literature are correct for the velar fronting material from German dialects. Conversely, some of the findings from German dialects cannot be confirmed in the typological works cited above. For example, that literature typically asserts (or simply assumes) that the fronting of velars is always assimilatory; however, that claim cannot be correct for the German dialects alluded to earlier in which velar fronting occurred in the context of front and back vowels.  

\subsection{Research questions}\label{sec:1.4.4}

I have referred to a number of issues and problems that are dealt with in the following chapters, the most important of which are stated below. The questions posed in \REF{ex:1:9} pertain to the new data described in \sectref{sec:1.4.1} and to their interpretation in \sectref{sec:1.4.2}. The two general typological questions in \REF{ex:1:10} were described above. \REF{ex:1:11} is a very general question of interest to dialectologists. Three of the most significant questions pertaining to the patterning of [x] and [ç] in the synchronic phonology of \il{Standard German}StG discussed in \sectref{sec:1.2} are presented in \REF{ex:1:12}.

\eanoraggedright%9
    \label{ex:1:9}
\eanoraggedright\label{ex:1:9a}Targets/triggers: What do the targets and triggers for velar fronting in German dialects tell us about the various stages of the historical rule of velar fronting?
\ex\label{ex:1:9b}Opacity: How did opaque velars and opaque palatals arise historically? To what extent can that type of \isi{opacity} help determine when velar fronting was phonologized?
\ex\label{ex:1:9c}Outputs: What is the historical origin of alveolopalatalization, and how did it spread through time and space?
\z
\ex%10
    \label{ex:1:10}
\eanoraggedright  How can the rules relating velar and palatal sounds in German dialects shed light on typological work done on similar rules in other languages?
\ex  How can the typological work done on languages fronting velars be applied to velar fronting in German dialects?
\z
\ex%11
    \label{ex:1:11}
      How are varieties of German reflecting the various options listed under (\ref{ex:1:9a}--\ref{ex:1:9c}) distributed geographically?
\ex%12
    \label{ex:1:12}
\eanoraggedright\label{ex:1:12a}What is the correct underlying sound for the rule relating [x] and [ç], i.e. /x/ or /ç/?
\ex\label{ex:1:12b}How does one unite the two categories “front vowelsˮ and “n, l, rˮ given that [ç] surfaces after a back (dorsal) sound ([ʀ]/[ɐ])?
\ex\label{ex:1:12c}Why does the palatal fricative [ç] in the diminutive suffix \is{chen@\textit{-chen}}-\textit{chen} occur after a back vowel?
\z
\z

Note that question \REF{ex:1:12a} can also be posed with respect to any German dialect in which velar fronting is active synchronically. \REF{ex:1:12b} is a specific question that can be subsumed into general questions regarding triggers in (\ref{ex:1:9a}) and \isi{opacity} in (\ref{ex:1:9b}). Question \REF{ex:1:12c} can rightfully be extended to German dialects with -\textit{chen}, although, as stressed by \citet{Robinson2001}, the question is moot for LG dialects in the north of Germany (which have some variant of [-kən]) and for Upper German (UG) dialects in the south of Germany as well as Switzerland and Austria (which have some variant of [-lɑin]). In the present book I restrict my discussion of the \is{chen@\textit{-chen}}status of -\textit{chen} to \il{Standard German}StG.

\section{{Phonology} {vs.} {phonetics}}\label{sec:1.5}

The dorsal segments that form the object of investigation in this work have been referred to above in terms of two discrete place categories, namely “velarˮ and “palatalˮ; recall \tabref{tab:1.5a}. The respective phonetic symbols for those fortis and lenis dorsal fricatives are repeated in \REF{ex:1:13}:

\ea%13
    \label{ex:1:13}
    \TabPositions{0pt, .15\linewidth, .25\linewidth}
      Velar:   \tab  [x] \tab  [ɣ]\\
      Palatal: \tab  [ç] \tab [ʝ]
\z

The phonetic symbols in \REF{ex:1:13} express broad phonetic representations, and the terms “velarˮ and “palatalˮ are likewise mere names for two phonological categories that could also be labeled “back dorsalˮ and “front dorsalˮ respectively. 

From the point of view of phonetics the two-way place dichotomy in \REF{ex:1:13} is simplistic, and some phonological treatments have accordingly adopted additional place categories. Consider first the German sound transcribed broadly as “[x]ˮ. Following \citet{Kohler1990b, Kohler1990a}, \citet[210--216]{Wiese1996a} observes that the back dorsal is realized as velar ([x]) after nonlow back tense vowels ([uː oː]) and as uvular ([χ]) after low vowels ([ɑ ɑː]). After nonlow back lax vowels ([ʊ ɔ]) there is variation between [x] and [χ], but [χ] predominates. Thus, words like \textit{Dach} ‘roof’ and \textit{Buch} ‘book’ can be narrowly transcribed as [dɑχ] and [buːx] respectively. In fact, Wiese sees [χ] as a byproduct of German phonology and not simply phonetics. Hence, he posits -- in addition to velar fronting (his \is{Dorsal Fricative Assimilation!Standard German}Dorsal Fricative Assimilation) {}-- a rule he dubs “\isi{Dorsal Fricative Lowering}ˮ, which converts velar [x] to uvular [χ] after certain back vowels \citep[213]{Wiese1996a}.\footnote{To the best of my knowledge, the first reference in the literature to a velar and a uvular realization of the German ach-Laut is \citet[164]{Forchhammer1994}. It is interesting to observe that Forchhammer's discovery was ignored by many subsequent studies of German phonetics and orthoepy, which continued to maintain that the ach-Laut has only one place of articulation, e.g. \citet[50-51]{Brandstein1950}, \citet[132-133]{Bithell1952}, \citet[76--77]{vonEssen1957}, \citet[153--154]{Heffner1960}, \citet[59]{Laziczius1961}, \citet[28--32]{Moulton1962}, \citet[176]{Richter1964}, \citet[167--168; 185--199]{MartensMartens1965}, \citet[88--89]{Schubiger1977}, \citet[39--40]{Wängler1981}, \citet[76--78]{Hakkarainen1995}, C. \citet[42--48]{Hall2003}, and \citet[76--78]{Russ2010}.}

Within the palatal category, it has long been known that the exact place of [ç] differs according to the type of front vowel that precedes it. This point is clear from the palatograms presented over one hundred years ago in \citet[309--310]{Scripture1902}, who concluded that the articulation of German [ç] “… varies with the preceding vowelˮ. It is also instructive to consider the findings of \citet{Recasens2013}, whose cross-linguistic work (which includes German) identifies four separate zones within the palatal region. No approach to my knowledge has argued that there are different surface realizations of German [ç] created by a phonological rule.

I adopt the position that velar fronting is a phonological rule which relates the two discrete categories in \REF{ex:1:13}. The exact place of articulation for sounds transcribed as “[x]/[ɣ]ˮ and “[ç]/[ʝ]ˮ is a topic that cannot be discussed because the original sources I have consulted typically do not provide such fine-grained distinctions.

It is conceivable that the German dialects discussed below possess both back dorsals  ([x] and [χ]) and that the distinction between the two was simply ignored by the linguists describing those dialects. If this plausible scenario were true then my survey of German dialects provides an excellent reason for considering the rules accounting for the distinction between velar and uvular to lie in the domain of phonetics. The reason is that no German dialect is known displaying the same kind of \isi{phonologization} of [x] and [χ] as described below for [x] and [ç]; e.g. no dialect has uvular quasi-phonemes or a contrast between a velar and a uvular.\footnote{The same reasoning argues against considering the different front dorsal articulations to be phonological. As noted in \tabref{tab:1.5b}, there are German dialects in which the output of velar fronting is an alveolopalatal (\isi{sibilant}) fricative [ɕ]. That type of dialect is consistent with the “two-category onlyˮ approach endorsed here (see \chapref{sec:10}).}

The intuition behind the classification in \REF{ex:1:13} with two discrete categories is reflected in the pronouncing dictionaries (\citealt{Siebs1969,Krech1982}, \citealt{Mangold2005}) and in colloquial speech of modern-day German speakers, who refer the palatal [ç] as the ich-Laut and the velar [x] as the ach-Laut. Significantly, there is no colloquial term for any of the sounds referred to above within either of the two categories in \REF{ex:1:13}. The fact that many grammarians describing German dialects were silent on the distinction between velar vs. uvular or between different palatals suggests that those categories were either not salient enough to be perceived or that the finer-grained distinctions were simply deemed irrelevant.

There has been a very long tradition of classifying German dorsal fricatives in terms of precisely two place categories. An example from the dawn of the nineteenth century is George Henry Noehden (1770--1826), who includes in his grammar of German an extensive discussion of pronunciation. Consider what \citet[62--63]{Noehden1800} writes of the pronunciation of German \textit{ch}:\footnote{Noehden was not the first to make these observations. Several late-eighteenth century grammarians also recognized two places of articulation for German \textit{ch} (\citealt[19--20]{Jellinek1914}, \citealt[113]{Voge1978}). The  first chronologically was Abraham Gotthelf Mäzke. In two of his works (\citealt[171]{Mäzke1776} and \citealt[27]{Mäzke1780}) there are terse statements indicating that Mäzke recognized what we would refer to today as a velar and a palatal realization of \textit{ch}. See also the remarks made on the ich-Laut and ach-Laut in \citet[128--131]{Bürger1798} by the German poet Gottfried August Bürger (1747--1794). Two additional scholars of note are Carl Philipp Moritz (1756--1793) and Wolfgang von Kempelen (1734-1804). \citet[23--24]{Moritz1784} and \citet[279--285]{vonKempelen1791} are very impressive passages indicating that both authors were aware of the fact that the front and back realizations of German \textit{ch} correlate with the tongue position of the preceding vowel. Johann Christoph Adelung (1732--1806), who is often considered to be the greatest eighteenth century German grammarian, only recognized one place of articulation for the fricative realization of \textit{ch} (e.g. \citealt{Adelung1781}). The most influential work on the German language in the first part of the nineteenth century was the \textit{Deutsche Grammatik} by Jacob Grimm (1785--1863), but in that work \citet[528]{Grimm1821} does not discuss places of articulation for German \textit{ch}.}

\begin{quote}
The English language furnishes nothing, with which the sound of this character may be compared …. This sound is twofold guttural, and palatick. The guttural is entirely formed in the throat… and answers … to the Scotch \textit{ch}, in \textit{Loch} … also to the \ili{Spanish} \textit{x} in \textit{dexar}, and the \textit{j} of the same, in \textit{lejos}. The German \textit{ch} … takes place, when joined to the vowels \textit{a, o, u}, and the diphthong \textit{au}. Examples: \textit{ach, alas}! \textit{Das Dach}, the roof; \textit{noch}, yet; \textit{das Joch}, the yoke; \textit{hoch}, high; \textit{das Buch}, the book … The palatick sound arises from a strong appulse of the breath against the palate, and is assigned to \textit{ch}, when in conjunction with \textit{e, i, ä, ö, ü, äu}. Examples: \textit{der Hecht}, the pike; \textit{schlecht}, bad; \textit{das Licht}, the light ….
\end{quote}

In an era in which the difference between sounds and letters was far from obvious, it is remarkable that Noehden was not only aware of the fact that there are exactly two sounds represented by (postvocalic) \textit{ch} --  in his words “gutturalˮ and “palatickˮ -- but also that the choice of one or the other depended on the type of preceding vowel.

Noehden’s intuition that there are exactly two categories of dorsal sounds is not an isolated example from that general time frame. In fact, it is more the rule than the exception for nineteenth and early twentieth century handbooks dealing with German sound structure (phonetics and orthoepy) to recognize exactly two discrete categories among dorsal fricatives. Examples include works written in English, German, and French published on both sides of the Atlantic (e.g.
\citealt[7]{Render1804}, \citealt[166--167]{Bauer1827}, \citealt{Follen1828}: 7, \citealt[11]{Götzinger1830}, \citealt[7]{Bernays1833}, \citealt[199--200]{Götzinger1836}, \cites[69--70]{Rapp1836}[42--44]{Rapp1841}, \citealt[19]{Fosdick1838}, \citealt[28--29]{Gortzitza1841}, \citealt[2]{Wertheim1841}, \citealt[6, 48--50]{Schwabe1842}, \citealt[8]{Becker1845}, \citealt[3]{Adler1846}, \citealt[35]{Bauer1847}, \citealt[5]{Wendeborn1849},  \citealt[1]{Mannheimer1853}, \citealt[2]{Eichhorn1854}, \citealt[6]{Ahn1855}, \citealt[6]{Strauss1856}, \citealt[48]{Brücke1856}, \citealt[9--10]{Otto1864}, \citealt[12; 38--39]{Schmitt1868}, \citealt[16]{Humperdinck1868}, \citealt[23]{Worman1868}, \citealt{Rumpelt1869}: 92--93, \citealt[11]{Whitney1870}, \citealt[11]{Weisse1872}, \citealt{Sweet1877}: 134--135, \citealt[148--149]{Viëtor1884}, \citealt{Trautmann1884}: 281, \citealt{Sievers1885}: 61--62, \citealt{Hoffmann1888}: 38--39, \citealt{Schmolke1890}: 28--29, \citealt{Soames1891}: 145; 147; 162, \citealt{Grandgent1892}: 6--7, \citealt[76--77]{bremer1893}, \citealt{Wilmanns1893}: 5, \citealt{Valentine1894}: 21, \citealt[xxvi]{Siepmann1897}, \citealt[58--59]{Siebs1898}, \citealt{Hempl1898}: 121--122, \citealt[18]{Dannheisser1899}, \citealt{Viëtor1901}: 22, \citealt{Behaghel1902}: 197, \citealt{Trautmann1903}: 92--93,  \citealt{Johannson1906}: 14--16, \citealt[14, 16]{Viëtor1906}, \citealt[13]{Bacon1906}, \citealt{Schröer1907}: xi, \citealt{Sütterlin1907}: 28, \citealt{ScholleG.Smith1907}: 97--100; 105--106, \citealt{Grossmann1910}: 7--9, \citealt{Passy1912}: 87--88,  \citealt{Jespersen1913}: 48--49, 135, \citealt{Prokosch1916}: 24--26, \citealt{Paul1916}: 307--308, \citealt{Leky1917}: 74--75, \citealt{Richter1922}: 53, 63, \citealt{Curme1922}: 29, and \citealt[116--118]{Sütterlin1925}). Significantly, many of those handbooks were known to the authors of the works I cite. See the quote by Robert Murray in \sectref{sec:1.1} and the description of the kind of original sources employed in the present book in \sectref{sec:1.6}.\footnote{Explicit reference to [ç] and [x] in the dialect literature (both HG and LG) was not common until the final two decades of the nineteenth century. The earliest reference to [ç] and [x] in the works I have consulted is \textcites[124--125]{Rapp1841}[]{Rapp1851} for HG (Swb) and \citet[302]{Rapp1840},  \citet[26]{Krüger1843} for LG.}

The “front dorsalˮ vs. “back dorsalˮ approach to the sounds in \REF{ex:1:13} may well be the dominant one these days, but a bit more needs to be said about an alternative tradition -- similar to the one championed by \citet{Wiese1996a} -- which endorses a third place of articulation among dorsal sounds. The three-way place approach referred to here has its roots in late nineteenth century Ortsgrammatiken (\sectref{sec:1.6.1}). One representative of this tradition is \citet{Batz1911}, who provides a detailed description of the East Franconian (\il{East Franconian}EFr) dialect spoken in \ipi{Bamberg} (\mapref{map:4}). In the section on the pronunciation of consonants, \citet[16]{Batz1911} has an ach-Laut (articulation on the soft palate), an ich-Laut, (articulation on the hard palate) and an \textsc{öch-Laut\is{och-Laut@öch-Laut}} (articulation between the hard and soft palate). Essentially the same type of classification has been adopted more recently by a number of linguistic atlases I cite in this book (see \sectref{sec:1.6.2}). For example, the six parts of the \textit{Bayerischer Sprachatlas} each have a “palatalˮ and a “velarˮ category, as well as a place of articulation akin to Batz’s öch-Laut\is{och-Laut@öch-Laut} which lies between the two. Those atlases consequently provide a number of very detailed maps of phonetically transcribed German words which include distinct symbols for three places of articulation among dorsal fricatives.

Given that a major goal of linguistic atlases is to document fine-grained differences in pronunciation in different regions, it is hardly surprising that the two-way place distinction in \REF{ex:1:13} is usually rejected. The six parts of the \textit{Bayerischer Sprachatlas} consequently all provide a wide array of phonetic symbols and diacritics in order to give very narrow transcriptions which account for subtle distinctions among vowel qualities (e.g. multiple vowel heights defined in terms of degrees of openness), vowel quantity (long vs. short vs. half-long), and laryngeal dimensions (lenis vs. fortis vs. categories between the two). A three-way place distinction among dorsal sounds is therefore precisely what one would expect given the goals of linguistic atlases.\footnote{\label{fn:1:8}I am familiar with two atlases which even go beyond the narrow transcriptions for dorsal fricatives in the \textit{Bayerischer Sprachatlas}. The first is \textit{Atlas zur Aussprache des Schriftdeutschen in der Bundesrepublik Deutschland} (AAS), which posits five distinct places of articulation for dorsal fricatives (Volume 1: 97--99). AAS even makes the strong claim that the distribution of those five phonetic variants are defined geographically (Maps CH.1 and CH.2 in Volume 2). AAS is outdone by \textit{Atlas zur Aussprache des deutschen Gebrauchsstandards} (AADG), which has six distinct places of articulation for the back dorsal (ach-Laut) and five for the front dorsal (ich-Laut). As in AAS, AADG shows that these articulations have geographic preferences.} 

In contrast to tradition among linguistic atlases, my treatment of velar fronting does not require reference to fine-grained place distinctions. In fact, I claim that this kind of detail would obscure the synchronic and diachronic treatment I propose below. But there is a much more straightforward reason for restricting my treatment of German dialects to the two place categories in \REF{ex:1:13}: The vast majority of original sources for German dialects do even not mention a third dorsal place, let alone the multiple places proposed in AAS and AADG (\fnref{fn:1:8}). Thus, any attempt to document velar fronting in all of the dialect areas depicted on \mapref{map:44} which takes more than two places of articulations for dorsal sounds into consideration would not be realizable.\footnote{The three-way place approach and the multiple-place approach referred to above both have late-eighteenth century precursors. It is clear that Georg Fränklin represents the former position when he writes \citep[26]{Fränklin1778} that there are three types of \textit{ch}: A low one (“tief") after \textit{a, o, u}, a mid one (“mittel") after \textit{e, ä, ö}, and a front one (“vorn") after \textit{i, ü}. Jakob Hemmer (1733--1790) opines that “our \textit{ch} is of so many kinds as we have vowels.” (“das unser \textit{ch} so filerlei ist, als wir Selbstlauter haben”; \citealt[68]{Hemmer1776} ).}

\section{{Data} {and} {sources}}\label{sec:1.6}\largerpage

The German dialect data introduced below comprise etymologically native words as in (\ref{ex:1:6}) and (\ref{ex:1:7}), although occasionally older loanwords which are fully integrated into the language are included as well. Loanwords containing velars and\slash or palatals (e.g. \textit{Chemie} ‘chemistry’ in \ref{ex:1:3b}) are not considered because most of the sources do not discuss them. That point aside, the status of dorsal fricatives in word-initial position in examples like \textit{Chemie} is controversial even in \il{Standard German}StG; see Appendix~\ref{appendix:g} for discussion. There are three types of sources I draw upon, which are described in the following three subsections.\footnote{I do not discuss the well-known questionnaires developed by the late nineteenth century linguist Georg Wenker (\isi{Wenkerbogen}), although data from those questionnaires are included in some of the phonetically transcribed texts I discuss below in \sectref{sec:1.6.3}. For recent discussion of Wenker's survey see \citet{Fleischer2017}.} 

\subsection{Ortsgrammatiken}\label{sec:1.6.1}
\subsubsection{General remarks}\largerpage[2]

The data discussed and analyzed below have been drawn from a wide variety of works dealing with a geographically-diverse selection of German dialects spoken roughly over the last one hundred forty years. Some of those studies are recent dissertations and theoretical articles based on data drawn from introspection or phonetic analysis, but the bulk of the work cited below comes from descriptive grammars of German dialects. Much of this work fits into the tradition of German dialectology known as Ortsgrammatiken, which were written in German-speaking countries during or shortly after the Neogrammarian era. The reader is referred to \citet{Murray2010} for discussion. 

It is important to stress that the basic method adopted in this book has been common for several decades among specialists of German. One example from the mid-twentieth century is \citegen{Schirmunski1962} lengthy survey of German dialects, which is based on data from Ortsgrammatiken. Other noteworthy examples include \citeauthor{Wiesinger}’s (\citeyear{Wiesinger1970a, Wiesinger1970b}) tomes on German vowels and \citegen{Howell1991} monograph on \isi{Breaking} in early Gmc. More recently, \citet{Goblirsch2018} makes extensive reference to original descriptions of German dialects in his study of the history of quantity in Gmc, and \citet{CaroReina2019} draws on data from the same type of sources in his phonology of Swabian (Swb).

It is also worth emphasizing that my approach of investigating small differences in closely-related dialects has a precedent in linguistics, where there is an entire field devoted to \textsc{microvariation}. See \citet{Brandner2012} for an overview of \isi{microvariation} in syntax and \citet{Alber2014}, who extends this approach to phonology. Microvariation applies theoretical concepts of modern generative theory to dialectal and small-scale variation, and -- in doing so -- it bridges the gap between traditional studies in dialectology -- Ortsgrammatiken in the present context -- and formal theory. Seen in this light, the present book fits into a broader contemporary enterprise involving the application of formal theory to linguistic data involving small-scale variation in German dialects. 

Since many readers may not be familiar with the Ortsgrammatik tradition, I provide some background on that type of source. The typical Ortsgrammatik consists of an in-depth description of a single locality considered to be relatively homogeneous and therefore free of dialect mixture. As pointed out by \citet[80]{Murray2010}, the grammars are usually written by phonetically well-trained native speakers grounded in the Neogrammarian tradition who employ both self-analysis and fieldwork. Most of these Ortsgrammatiken have both a synchronic and a diachronic component emphasizing the phonetics and the inflectional morphology of the dialect in question. Given the general time frame of these sources it is understandable that the synchronic discussion of sound structure involves only phonetics and not phonology (e.g. the notion of the phoneme and allophones). 

Many of the dialects described in the Ortsgrammatiken referred to above -- especially those in the north of pre-1945 Germany -- are moribund due to evacuation and forced expulsion of Germans from East Prussia (Ostpreußen), \ipi{Silesia} (Schlesien) and East Pomerania (Ostpommern) after 1945. Certain dialects in Northwest Germany in regions never subject to forced expulsion are nevertheless either extinct or on the verge of extinction.\largerpage

There is more than one reason why it is essential to investigate the sound structure of German dialects spoken a century ago. First, as noted above, many varieties are simply no longer spoken; hence, older descriptions of those dialects are often the only sources we have available. Second, a number of older dialects often preserve structures that are absent in dialects spoken in the present day. The type of dialect referred to here can therefore be thought of as a missing link without which a complete understanding of velar fronting would not be possible.

An investigation of dialects spoken in the late nineteenth and early twentieth century has an added advantage. Dialects described a century ago were written at a time when the influence of the standard language on dialects might not have been as prevalent as today because the notion of a standard language had not yet established itself. Anyone conducting fieldwork on modern German dialects can attest to the fact that it is difficult if not impossible to find dialect speakers who have no knowledge at all of the standard language. Hence, velar fronting in many German varieties spoken today may not truly reflect velar fronting in that particular dialect, but instead velar fronting in \il{Standard German}StG. By contrast, dialect speakers with little or no knowledge of \il{Standard German}StG were probably more common a century ago and could therefore give an accurate picture of velar fronting in their respective dialects.\footnote{In actuality, the language situation in the late nineteenth century was more complex than what I am suggesting here. General discussion can be found in \citet[343ff.]{Wells1985}. For an assessment of the developing standard language and its influence on regional German dialects in the late nineteenth century see \citet{Ganswindt2017}.}

I only cite sources which include enough data to draw conclusions regarding the issues mentioned above (e.g. the set of triggers and targets as well as opaque segments). Hence, I eschew sources in which not enough data are presented to determine the correct distribution of velars and palatals or sources in which the data involving the distribution of velars and palatals are simply unclear. As a general rule I prefer sources which express the difference between velars and palatals with distinct phonetic symbols. In certain exceptional cases I incorporate data in which a single phonetic symbol is used to distinguish two articulations (e.g. [x] vs. [ç]), but only if that source is clear on the distribution of those sounds.

\subsubsection{Reliability of Ortsgrammatiken}\largerpage

An objection to data from older works often raised is that those sources may not be trustworthy. In fact, I see several reasons for considering the older sources cited here to be highly reliable in their descriptions of the sounds investigated below. Consider the following:

\subsubsubsection{Phonetically-trained authors}
It is my experience that many linguists in the present day are reluctant to accept data drawn from older sources even if those individuals have never even laid eyes on such works. Those skeptics apparently believe that writers in the late nineteenth and early twentieth century simply did not know enough about sound structure to give an accurate portrayal of the phonetics. That belief is mistaken, at least in the case of the German dialect literature cited below, because the descriptive works referred to here were written by linguists trained in the Neogrammarian tradition who had a thorough grounding in phonetics. Hence, all of the authors cited below were well-aware of the classification of consonants (e.g. in terms of place, manner) and vowels (e.g. in terms of height and backness etc). All of the older sources cited -- without exception -- were intimately acquainted with the distinction between “velarˮ and “palatalˮ depicted in \REF{ex:1:13} -- recall \sectref{sec:1.5} -- and consequently transcribed those articulations with distinct phonetic symbols.

\subsubsubsection{Confirmation from multiple sources}
The pattern whereby velars like [x] are fronted in the context after all front vowels is uncontroversial in \il{Standard German}StG as well as some of the modern dialects discussed below. If a source written in the year 1880, for example, tells us that there is a small community in which [x] is fronted after all front vowels and not after other sounds then it is difficult to conclude that the source should be deemed untrustworthy. A similar point holds for sources describing a pattern distinct from the one exemplified by \il{Standard German}StG. For example, if three authors write independently from one another during the same general time frame that there are places in three separate regions separated by hundreds of kilometers in which historical [x] is realized as [ç] in the context of front vowels but not in the context of coronal consonants like [n l r], then the most reasonable assumption is that the three descriptions are accurately describing the triggers for velar fronting in their respective community.

\subsubsubsection{Consistent transcriptions}
The sources cited in the present book are remarkably consistent in their transcriptions of velars and palatals. For example, many authors observe that historical [x] is fronted to [ç] after certain vowels (e.g. [i e]) and remains [x] after others (e.g. [u o ɑ]) but that etymological [ɣ] remains [ɣ] after any vowel. That descriptive claim derives support from a plethora of examples in which [ç] occurs precisely after [i e] and nowhere else, [x] exclusively after [u o ɑ], and [ɣ] after [i e u o ɑ], but what is remarkable is that there are no deviant lexical items that might suggest the author has missed those generalizations. If the source were unreliable then one might expect there to be inconsistencies and/or errors obscuring the general pattern thereby casting doubt on the competence of the author. Such inconsistencies might involve [ç] being occasionally transcribed after sounds other than [i e] or [x] after sounds other than [u o ɑ]. Likewise, in an unreliable source the velar [ɣ] might occasionally be transcribed as palatal [ʝ] after front vowels, thereby causing one to question the claim that velar [ɣ] surfaces even after front vowels. The most reasonable conclusion is that [x] is the sole target for velar fronting and that the triggers consist of all and only front vowels. The same point holds for dorsal segments with an unexpected distribution, i.e. opaque sounds. For example, many writers have observed that velar fronting of historical [ɣ] to [ʝ] occurred in word-initial position before front vowels or \isi{schwa} ([ə]) but elsewhere stays [ɣ]. In the type of grammar referred to here one might see dozens of words beginning with [ʝ] before a front vowel or \isi{schwa} and [ɣ] before other sounds, but sequences like [ɣə] are absent.

\subsubsubsection{Linguistically plausible data}\largerpage
In virtually all of the sources cited below the conditions under which velars undergo fronting correspond to natural classes in phonology. What is more, those natural classes usually support findings from typological research referred to in \sectref{sec:1.4.3}. Those natural classes are almost never explicitly identified in the respective sources as such (since the concept did not exist at the time), but they are evident from the list of segments given that undergo or induce velar fronting. For example, multiple descriptive grammars attest to the realization velars as palatals after vowels like [i ɪ e ɛ] but not after [u ʊ o ɔ ɑ] or [æ]. Instead of considering the source to be untrustworthy a more likely explanation is that authors are describing the fronting of velars after nonlow front vowels. By contrast, an unreliable source might give a list of vocalic triggers that is completely arbitrary, e.g. after [i e ɛ o] but not after [ɪ u ʊ ɔ ɑ æ]. None of the sources cited below document that kind of bizarre context for velar fronting.


\subsection{Linguistic atlases and dialect dictionaries}\label{sec:1.6.2}\largerpage[2]

In addition to Ortsgrammatiken, I draw on some data from the linguistic atlases presented in \tabref{tab:1:1}. There are a number of excellent regional atlases (Kleinraumatlanten) for German dialects, several of which are included here. See \citet{Scheuringer2011} and \citet[35--39]{NiebaumMacha2014} for surveys of linguistic atlases for German dialects.

\begin{table}[p]
\caption{\label{tab:1:1}Linguistic atlases and their abbreviations}
\begin{tabularx}{\textwidth}{Ql}
\lsptoprule
Atlas name & Abbreviation\\\midrule
Atlas zur Aussprache des deutschen Gebrauchsstandards & AADG\\
Atlas zur Aussprache des Schriftdeutschen in der Bundesrepublik Deutschland & AAS\\
Atlas der Celler Mundart & ACeM\\
Atlas zur deutschen Alltagssprache & ADA\\
Atlas linguistique et ethnographique de l’Asace & ALA\\
Atlas linguistique et ethnographique de la Lorraine germanophone & ALLG\\
Kleiner Deutscher Sprachatlas & KDSA\\
Linguistic Atlas of Texas German & LATG\\
Luxemburgischer Sprachatlas & LSA\\
Mittelrheinischer Sprachatlas & MRhSA\\
Norddeutscher Sprachatlas & NOSA\\
Schlesischer Sprachatlas & SchlSA\\
Sudetendeutscher Atlas & SDA\\
Siebenbürgisch-Deutscher Sprachatlas & SDSA\\
Sprachatlas der deutschen Schweiz & SDS\\
Sprachatlas von Bayerisch-Schwaben & SBS\\
Sprachatlas von Mittelfranken & SMF\\
Sprachatlas von Oberbayern & SOB\\
Sprachatlas von Niederbayern & SNiB\\
Sprachatlas von Nordostbayern & SNOB\\
Sprachatlas von Unterfranken & SUF\\
Sprachatlas von Oberösterreich & SAO\\
Sprachatlas von Nord Baden-Württemberg & SNBW\\
Südwestdeutscher Sprachatlas & SSA\\
Thüringischer Dialektatlas & ThürDA\\
Tirolischer Sprachatlas & TSA\\
Vorarlberger Sprachatlas & VALTS\\
Wortatlas der deutschen Umgangssprachen & WDU\\
Wortgeographie der städtischen Alltagssprache in Hessen & WSAH\\
Zimbrisch und fersentalerischer Sprachatlas & ZFSA\\
\lspbottomrule
\end{tabularx}
\end{table}

\begin{table}[p]
\caption{\label{tab:1:1.2}Dialect dictionaries and their abbreviations}
\begin{tabularx}{\textwidth}{Ql}
\lsptoprule
Dictionary name & Abbreviation\\\midrule
Aachener Wörterbuch & AaWb\\
Dortmunder Wörterbuch & DoWb\\
Dremmener Wörterbuch & DrWb\\
Hamburgisches Wörterbuch & HaWb\\
Das Kölsche Wörterbuch & KWb\\
Mittelelbisches Wörterbuch & MiElWb\\
Wörterbuch der obersächsischen und erzgebirgischen Mundarten & ObersWb\\
Neuer Kölnischer Sprachschatz & NKSS\\
Neunkirchen-Seelscheider Sprachschatz & NSSS\\
Pommersches Wörterbuch & PWb\\
Rheinisches Wörterbuch & RWb\\
Saarbrücker Wörterbuch & SbWb\\
Schleswig-Holsteinisches Wörterbuch & SchlHWb\\
Schwäbisches Wörterbuch & SchwWb\\
Südhessisches Wörterbuch & SHesWb\\
Simmentaler Wortschatz & SiWS\\
Wörterbuch der Teltower Volkssprache & TeWb\\
Wörterbuch der Tiroler Mundarten & TiWb\\
Trierer Wörterbuch & TrWb\\
Wörterbuch der Kölner Mundart & WbKM\\
Wörterbuch der Mundart von Dobschau & WbMD\\
Wörterbuch der unteren Sieg & WbUS\\
Wörterbuch der westmünsterländischen Mundart & WMlWb\\
Wörterbuch der westphälischen Mundart & WphWb\\
\lspbottomrule
\end{tabularx}
\end{table}

As suggested by the names listed in the first column, most of the atlases focus on a particular region or dialect area: ACeM for the area in North Germany around the city of \ipi{Celle}, ALA for Alsace, ALLG for German Lorraine (Deutsch-Lothringen), LATG for Texas (USA), LSA for \ipi{Luxembourg}, NOSA for North Germany,  SchlSA for \ipi{Silesia}, SDSA for Transylvania (Siebenbürgen), SNBW for the northern part of the German state of Baden-Württemberg, SAO for \ipi{Upper Austria}, SDA for the Sudetenland (Czech Republic), SDS for German-speaking Switzerland, SSA for Southwest Germany, ThürDA for Thuringia, TSA for \ipi{Tyrol}, VALTS for \ipi{Vorarlberg}, \ipi{Liechtenstein}, West \ipi{Tyrol}, and the Allgäu, WSAH for the German state of Hesse, and ZFSA for German-language islands of Northeast Italy (Cimbrian and Fersentalerisch). MRhSA concerns itself with the central and southern region of the \textsc{Rheinischer} \textsc{Fächer} (= \is{Rhenish Fan}\textsc{Rhenish Fan}). Six atlases listed above (SBS, SMF, SOB, SNiB, SNOB, SUF) are separate parts of the \textit{Bayerischer Sprachtlas}, which covers most of Bavaria (Freistaat Bayern). One of the atlases listed above (KDSA) has as its focus all German-speaking countries given pre-1914 borders. The four works listed in \tabref{tab:1:1} which do not concern themselves specifically with German dialects are AADG, AAS, ADA, and WDU. Those works investigate regional differences in the pronunciation of contemporary German (AADG), the pronunciation of the written language (AAS), and colloquial speech (ADA, WDU).

Data from linguistic atlases are important because they make it possible to look at general patterns that might not be evident in the Ortsgrammatiken. They are also very useful because they sometimes indicate places within a broad region with the kinds of quirks regarding targets and triggers for velar fronting described above.

In this book I also make some reference to dictionaries on specific dialects. I only consider those dialect dictionaries which either contain phonetic transcriptions in lexical entries or which provide a clear statement concerning pronunciation. The dialect dictionaries I cite in this book are provided in \tabref{tab:1:1.2}, which are listed alphabetically according to the abbreviations in the second column. A discussion of the importance of dictionaries as sources in dialectology can be found in \citet[40--42]{NiebaumMacha2014}.

It can be seen that some dictionaries focus on a particular city (AaWb, DoWb, DrWb, HaWb, KWb, NKSS, NSSS, SbWb, TrWb, WbKM), state (SchlHWb), former province (PWb), former county (TeWb), region (MiElWb, RWb, SiWS, TiWb, WbUS, WMlWb), or dialect area (ObersWb, SchwWb, SHesWb, WbMD, WphWb).

Like linguistic atlases, dialect dictionaries are important for identifying broad patterns representing a particular geographic region that might not be evident in Ortsgrammatiken.

\subsection{Phonetically transcribed texts}\label{sec:1.6.3}

Considerable work on German dialects consists of phonetically transcribed texts of native speakers for a dialect spoken in a particular place. These texts might be the transcription of a conversation or the recitation of a story or fairy tale. They are also often accompanied by a written commentary. The type of phonetically transcribed text referred to here can be found in the book series \textit{Lautbibliothek der deutschen Mundarten} (until 1969) and the successor book series \textit{Phonai}. Several works cited in the following chapters appeared in either one of those series. Another type of phonetically transcribed text can be found in the realm of morphology. A number of works have appeared through the years on various aspects of the morphological structure of German dialects, e.g. noun and adjective declensions, verb conjugations. That type of work can be drawn upon as evidence of velar fronting for a particular place if distinct symbols are employed for velars and palatals.

If the phonetically transcribed text is long enough then it is possible to draw conclusions on the status of velar fronting. These texts are particularly useful if neither Ortsgrammatiken nor linguistic atlases are available for a particular area.

\section{{Structure} {of} {the} {book}}\label{sec:1.7}

The remainder of this book consists of seventeen chapters, which are summarized briefly here. 

\chapref{sec:2} introduces the theoretical underpinnings adopted in my investigation of velar fronting. That chapter includes a description of features for consonants and vowels in order to state the various versions of velar fronting in a theoretically consistent fashion, an explication of \isi{opacity}, a summary of velar fronting in the context of work done on the typology of similar rules fronting velars, and a sketch of the historical model delineating the various stages of velar fronting described above. 

The core of the book (Chapters~\ref{sec:3}--\ref{sec:15}) consists of detailed datasets from original sources involving velar fronting in HG/LG varieties and my analysis thereof. Those chapters are organized for the most part structurally and not according to geography in the following way: 

In Chapters~\ref{sec:3}--\ref{sec:4} I discuss dialects in which [x] and [ç] exhibit a transparent (allophonic) relationship. The former chapter concerns itself with those varieties in which velar fronting relates the two fortis sounds [x] and [ç]. Case studies are provided for dialects spoken in South Germany, Switzerland, and Austria. \chapref{sec:4} probes a set of dialects containing the lenis velar fricative [ɣ] and/or the lenis palatal fricative [ʝ] in addition to [x] and [ç]. The dialects investigated consist of primarily moribund varieties once spoken in North Germany.  

Chapters~\ref{sec:5}--\ref{sec:9} investigate \isi{opacity}. Chapters~\ref{sec:5}--\ref{sec:6} consider \isi{underapplication} and Chapters~\ref{sec:7}--\ref{sec:9} \isi{overapplication}. In \chapref{sec:5} I discuss German dialects spoken in Central Germany in which some independent synchronic rule creating [x] \isi{counterfeeds} velar fronting, and in \chapref{sec:6} I consider neutral vowels with data drawn from two varieties of Swiss German (SwG).

\chapref{sec:7} investigates a number of varieties not restricted to a particular region which have in common that they possess palatal quasi-phonemes. Chapters~\ref{sec:8}--\ref{sec:9} concern themselves with phonemic palatals, i.e. palatals that contrast with the corresponding velars. In \chapref{sec:8} I discuss dialects spoken in North Germany in which phonemic palatals surface word-initially and in \chapref{sec:9} I discuss dialects spoken in Central Germany with phonemic palatals in postsonorant position. 

\chapref{sec:10} is devoted to an investigation of dialects spoken in Central Germany in which the post-front vowel palatal [ç] is replaced with alveolopalatal [ɕ]. It is demonstrated that [ɕ] is an allophone of /x/ in some varieties but that in others [ɕ] contrasts with [x]. I show that alveolopalatalization does not involve \isi{opacity}.

\chapref{sec:11} investigates German dialects in which the set of targets for velar fronting include velar stops and the velar nasal (recall \ref{ex:1:7d}--\ref{ex:1:7e}). Those varieties were once spoken in the northeast of pre-1945 Germany. 

\chapref{sec:12} summarizes the findings in Chapters~\ref{sec:3}--\ref{sec:11} on the extent to which triggers and targets for velar fronting can vary from place to place.

\chapref{sec:13} discusses data from a linguistic atlas for Lower Bavaria (SNiB) which document velar fronting throughout that area. An important conclusion of that chapter is that velar fronting is not uniform in Lower Bavaria. Instead, there are three versions of the rule defined according to the nature of the triggers.

\chapref{sec:14} investigates the nonassimilatory fronting of velars.

\chapref{sec:15} documents velar fronting islands and discusses the extent to which the segments inducing that process (triggers) can differ from place to place.

\chapref{sec:16} demonstrates how linguistic evidence can shed light on how velar fronting fits into the well-established stages in the history of German (Appendix~\ref{appendix:e}). In that chapter I also consider how data from modern dialects can give evidence regarding the areas where velar fronting has been active the longest. 

\chapref{sec:17} considers the status of velar fronting in \il{Standard German}StG and addresses the research questions in \REF{ex:1:12}.

\chapref{sec:18} provides a brief conclusion in which I summarize my findings and relate those findings to the research cited earlier.

This book contains supplemental information in the form of twelve appendices. Appendix~\ref{appendix:a} provides the reader with an overview of HG and LG dialects and also includes a map indicating the distribution of those dialects in German-speaking countries in pre-1914 Europe. Appendix~\ref{appendix:b} is a historical map depicting the German Empire during the time frame 1871--1918. Appendix~\ref{appendix:c} lists tables containing all varieties of German investigated (including sources) and classifies those varieties in terms of the dialects given in Appendix~\ref{appendix:a}. Appendix~\ref{appendix:d} gives a list of the triggers and targets for all versions of velar fronting posited in the present book. Appendix~\ref{appendix:e} is a family tree for Germanic languages. Appendix~\ref{appendix:f} provides some background information on the historical reflexes of modern-day dorsal fricatives in German dialects. Appendix~\ref{appendix:g} concerns itself with the status of dorsal fricatives in loanwords in \il{Standard German}StG and other varieties. Appendix~\ref{appendix:h} gives the inventory of consonants and glides in three broad dialects (LG, CG, UG). Appendix~\ref{appendix:i} provides some discussion of the status of rules fronting velar sounds in the branches of Germanic not discussed in this book, in addition to the language families spoken in the immediate vicinity of the German-speaking world, namely \ili{Slavic} and \ili{Romance}. Appendix~\ref{appendix:j} lists the names of all 221 villages, towns, and cities where data were drawn from the linguistic atlas for Lower Bavaria (SNiB). Appendix~\ref{appendix:k} and Appendix~\ref{appendix:l} list the linguistic atlases and dialect dictionaries cited throughout this book. 
