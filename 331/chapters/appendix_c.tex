\chapter{List of German dialects investigated}\label{appendix:c}

All varieties of German discussed in this book are given below in a series of tables classified into the dialects introduced in Appendix~\ref{appendix:a}. The classification is consistent with the one in \citet{WiesingerRaffin1982} and \citet{Wiesinger1987} for those works which appeared in 1985 or before.

In the first column of the tables listed below I identify for each variety the place and/or region where it is (or was) spoken, in the second column I indicate where that place or region is (or was) situated in terms of administrative divisions, and in the final column I list the original source. For each table the dialects are listed in chronological order according to the reference given in the final column. Some of those sources focus on a very specific place (e.g. a particular village), while others describe a cluster of villages, a city, or a larger region which might be coterritorial with an administrative division (e.g. a particular county). On the other hand, some of the original sources only give a vague indication of where the variety is spoken (e.g. by referring to areas between rivers or mountain ranges). Administrative divisions differ from country to country. If the dialect is spoken in Germany then the country is not indicated in the second column, but the state (Bundesland), county (Kreis/Landeskreis), and/or government district (Regierungsbezirk) are provided. The countries referred to below are abbreviated as follows: Austria (AT), Belgium (BE), Canada (CAN), the Czech Republic (CZ), Estonia (ES), France (FR), Hungary (HU), Italy (IT), Latvia (LA), Liechtenstein (LI), Luxembourg (LX), Mexico (MEX), the Netherlands (NL), Poland (PO), Romania (RO), Russia (RUS), Slovakia (SLK), Slovenia (SL), Switzerland (CH), Ukraine (UKR), and the United States of America (USA). For those countries I only occasionally include the respective administrative divisions. For all dialects once spoken in the eastern provinces of pre-1945 Germany -- East Pomeranian (EPo), Low Prussian (LPr), High Prussian (HPr), Silesian (Sil) -- the original names of the province, county and city/town are provided. For all other dialects I list the current name of the respective county. The modern German states and pre-1945 provinces are abbreviated according to the final column of the first table.

\begin{table}
\caption{Modern States (Bundesländer) of Germany and pre-1945 provinces (Provinzen) of the German Empire}
\begin{tabularx}{\textwidth}{QQl}
\lsptoprule
 State (German) & State (English) & Abbv.\\\midrule
 Baden-Württemberg & Baden-Württemberg & BWb\\
 Bayern & Bavaria & Bvr\\
 Brandenburg & Brandenburg & Brbg\\
 Bremen & Bremen & Brm\\
 Hamburg & Hamburg & Hbg\\
 Hessen & Hesse & Hss\\
 Mecklenburg-Vorpommern & Mecklenburg-Vorpommern & MVpm\\
 Niedersachsen & Lower Saxony & LSxn\\
 Nordrhein-Westfalen & North Rhine-Westphalia & NRW\\
 Rheinland-Pfalz & Rhineland-Palatinate & RnPl\\
 Saarland & Saarland & Srd\\
 Sachsen & Saxony & Sxn\\
 Sachsen-Anhalt & Saxony-Anhalt & SxAn\\
 Schleswig-Holstein & Schleswig-Holstein & SHst\\
 Thüringen & Thuringia & Thra\\\midrule
Province (German) & Province (English) & Abbv.\\\midrule
Ostpommern & East Pomerania & EPmr\\
Ostpreußen & East Prussia & EPr\\
Posen & Posen & Pos\\
Schlesien & \ipit{Silesia} & Sil\\
Westpreußen & West Prussia & WPr\\
\lspbottomrule
\end{tabularx}
\end{table}

\begin{longtable}{>{\raggedright}p{.4\textwidth}>{\raggedright}p{.25\textwidth}>{\raggedright\arraybackslash}p{.35\textwidth-4\tabcolsep}}
\caption{High(est) Alemannic}\\
\lsptoprule Place/Region & Administ. Division & Source\\\midrule\endfirsthead
\midrule Place/Region & Administ. Division & Source\\\midrule\endhead\endfoot\lspbottomrule\endlastfoot
\ipi{Kerenzen} (\ipi{Glarus} Nord) & CH; \ipi{Glarus} & \citet{Winteler1876}\\\midrule
\ipi{St. Stephan} & CH; \ipi{Bern} & \citet{Zahler1901}\\\midrule
\ipi{Hohenems} & AT; \ipi{Vorarlberg} & \citet{Seemüller1909c}\\\midrule
\ipi{Lauterach}, \ipi{Nenzing} & AT; \ipi{Vorarlberg} & \citet{SchneiderMarte1910}\\\midrule
\ipi{Urserental} (area around Realp) & CH: \ipi{Uri} & \citet{Abegg1910}\\\midrule
Kesswil & CH: Thurgau & \citet{Enderlin1910}\\\midrule
\ipi{Todtmoos-Schwarzenbach} & BWb; Kreis Waldshut & \citet{Kaiser1910}\\\midrule
\ipi{Appenzell} & CH; \ipi{Appenzell} Innerrhoden & \citet{Vetsch1910}\\\midrule
\ipi{Visperterminen} & CH; Valais & \citet{Wipf1910}\\\midrule
In and around \ipi{St. Gallen} & CH; \ipi{St. Gallen} & \citet{Hausknecht1911}\\\midrule
\ipi{Rheintal} & CH; \ipi{St. Gallen} & \citet{Berger1913}\\\midrule
\ipi{Nufenen}, \ipi{Vals};\newline \ipi{Leissigen}, \ipi{Frutigen}, \ipi{Saanen} & CH: Grisons;\newline CH: \ipi{Bern} & \citet{Gröger1914d, Gröger1914c, Gröger1914a, Gröger1914e, Gröger1914b}\\\midrule
\ipi{Entlebuch} & CH; Lucern & \citet{Schmid1915}\\\midrule
\ipi{Glarus} & CH; \ipi{Glarus} & \citet{Streiff1915}\\\midrule
\ipi{Toggenburg} & CH; \ipi{St. Gallen} & \citet{Wiget1916}\\\midrule
\ipi{Jaun} & CH; Freiburg & \citet{Stucki1917}\\\midrule
\ipi{Obersaxen} (Mundaun) & CH; Grisons & \citet{Brun1918}\\\midrule
Bündner Herrschaft\newline (\ipi{Maienfeld}, Fläsch, Malans, Jenins) & CH; Grisons & \citet{Meinherz1920}\\\midrule
Berner Seeland (area around Biel) & CH; \ipi{Bern} & \citet{Baumgartner1922}\\\midrule
\ipi{Vandans} & AT; \ipi{Vorarlberg} & \citet{Jutz1922}\\\midrule
\ipi{Zürcher Oberland} & CH; \ipi{Zürich} & \citet{Weber1923}\\\midrule
South \ipi{Vorarlberg};\newline LI & AT; \ipi{Vorarlberg};\newline LI & \citet{Jutz1925}\\\midrule
\ipi{Markgräflerland} & BWb; Freiburg & \citet{Beck1926}\\\midrule
\ipi{Sensebezirk} and the Southeast \ipi{Seebezirk} & CH; Freiburg & \citet{Henzen1927}\\\midrule
\ipi{Lötschental} & CH; Valais & \citet{Henzen19281929,Henzen1932}\\\midrule
Area around Schächental & CH; \ipi{Uri} & \citet{Clauss1929}\\\midrule
\ipi{Schanfigg} & CH; Grisons & \citet{Kessler1931}\\\midrule
\ipi{Mutten} & CH; Grisons & \citet{Hotzenköcherle1934}\\\midrule
\ipi{Schaffhausen} & CH: \ipi{Schaffhausen} & \citet{Wanner1941}\\\midrule
\ipi{Upper Valais} & CH: Valais & \citet{Rübel1950}\\\midrule
\ipi{Walensee-Seeztal} & CH: Grisons, \ipi{Glarus} & \citet{Trüb1951}\\\midrule
\ipi{Brienz} & CH: \ipi{Bern} & \citet{Schultz1951}\\\midrule
\ipi{Bern} & CH: \ipi{Bern} & \citet{Keller1961}\\\midrule
Vorarlberger \ipi{Rheintal} (\ipi{Dornbirn}, \ipi{Hohenems},\ipi{Lustenau}) & AT; \ipi{Vorarlberg} & \citet{Gabriel1963}\\\midrule
\ipi{Jestetten} & BWb & \citet{Keller1963}\\\midrule
Kreis \ipi{Feldkirch} & AT; \ipi{Vorarlberg} & \citet{BethgeBonnin1969}\\\midrule
\ipi{Bellwald} & CH; Valais & \citet{Schmidt1969}\\\midrule
Brig-Gris & CH; Valais & \citet{Werlen1977}\\\midrule
Area between Thun and Jura & CH; \ipi{Bern} & \citet{Marti1985}\\\midrule
\ipi{Bosco Gurin} & CH; Tessin & \citet{Russ2002}\\\midrule
\ipi{Zürich} & CH; \ipi{Zürich} & \citet{FleischerSchmid2006}\\\midrule
\ipi{Kleinwalsertal}, \ipi{Damülser Tal},\newline \ipi{Tal der Bregenzer Ache},\newline \ipi{Großes Walsertal}, \ipi{Laternsertal};& AT; \ipi{Vorarlberg} & VALTS\\
\ipi{Triesenberg} & LI &  \\\midrule
\ipi{Upper Valais}, Southwest \ipi{Bernese Oberland}, \ipi{St. Antönien} & CH; Valais, \ipi{Bern}, Grisons & SDS\\
\end{longtable}

\begin{longtable}{>{\raggedright}p{.4\textwidth}>{\raggedright}p{.25\textwidth}>{\raggedright\arraybackslash}p{.35\textwidth-4\tabcolsep}}
\caption{Low Alemannic}\\
\lsptoprule Place/Region & Administ. Division & Source\\\midrule\endfirsthead
\midrule Place/Region & Administ. Division & Source\\\midrule\endhead\endfoot\lspbottomrule\endlastfoot
\ipi{Münsterthal} & FR; Alsace & \citet{Mankel1886}\\\midrule
\ipi{Ottenheim} (Schwanau) & BWb; \ipi{Ortenaukreis} & \citet{Heimburger1887}\\\midrule
Basel & CH; Basel-Stadt & \citet{Heusler1888}\\\midrule
\ipi{Forbach} & BWb; Landkreis Rastatt & \citet{Heilig1897}\\\midrule
\ipi{Colmar} & FR; Alsace & \citet{Henry1900}\\\midrule
\ipi{Oberschopfheim} (Friesenheim) & BWb; \ipi{Ortenaukreis} & \citet{Schwend1900}\\\midrule
\ipi{St. Georgen} & BWb; Schwarzwald-Baar-Kreis & \citet{Ehret1911}\\\midrule
\ipi{Rheinbischofsheim} (Rheinau) & BWb; \ipi{Ortenaukreis} & \citet{Weik1913}\\\midrule
\ipi{Oberweier} (Bühl) & BWb; Landkreis Rastatt & \citet{Wasmer1915,Wasmer1916, Wasmer1916b}\\\midrule
Area between Renchtal and Schuttertal & BWb; \ipi{Ortenaukreis} & \citet{Kilian1935}\\\midrule
\ipi{Freiburg im Breisgau} & BWb & \citet{Eckerle1936}\\\midrule
\ipi{Northwest Switzerland} & CH; Basel-Stadt & \citet{Schläpfer1956}\\\midrule
Barr & FR; Alsace & \citet{Keller1961}\\\midrule
\ipi{Blaesheim} & FR; Alsace & \citet{Philipp1965}\\\midrule
\ipi{Mulhouse} & FR; Alsace & \citet{BethgeBonnin1969}\\\midrule
\ipi{Metzeral} & FR; Alsace & \citet{Zeidler1978}\\\midrule
\ipi{Mittelbaden} (large area between Baden-\ipi{Baden} and Lahr) & BWb & \citet{Schrambke1981}\\\midrule
\ipi{Breisgau} & BWb & \citet{Klausmann1985a,Klausmann1985b}\\\midrule
\ipi{Colmar} & FR; Alsace & \citet{Klausmann1985a,Klausmann1985b}\\\midrule
\ipi{Benfeld} & FR; Alsace & \citet{Rünneburger1985}\\\midrule
\ipi{Urach} (Vöhrenbach), \ipi{Titisee-Neustadt} & BWb; Schwarzwald-Baar-Kreis, Landkreis Breisgau-Hochschwarzwald & E.M. \citet{Hall1991, Hall1991b}\\\midrule
Mortzwiller, Oberhergheim, Thanvillé, Weiterswiller, Lembach & FR & ALA\\
\end{longtable}


\begin{longtable}{>{\raggedright}p{.4\textwidth}>{\raggedright}p{.25\textwidth}>{\raggedright\arraybackslash}p{.35\textwidth-4\tabcolsep}}
\caption{Swabian}\\
\lsptoprule Place/Region & Administ. Division & Source\\\midrule\endfirsthead
\midrule Place/Region & Administ. Division & Source\\\midrule\endhead\endfoot\lspbottomrule\endlastfoot
\ipi{Horb am Neckar} & BWb; Landkreis \ipi{Freudenstadt} & \citet{Kauffmann1887, Kauffmann1890}\\\midrule
\ipi{Reutlingen} & BWb; Landkreis \ipi{Reutlingen} & \citet{Wagner1889}\\\midrule
\ipi{Münsingen} & BWb; Landkreis \ipi{Reutlingen} & \citet{Bopp1890}\\\midrule
\ipi{Villingen-Schwenningen} & BWb; Schwarzwald-Baar-Kreis & \citet{Haag1898}\\\midrule
\ipi{Ries} & Bvr: Swabia & \citet{Schmidt1898}\\\midrule
\ipi{Mühlingen} & BWb; Landkreis Konstanz & \citet{Müller1911}\\\midrule
\ipi{Liggersdorf} (Hohenfels) & BWb; Landkreis Konstanz & \citet{Dreher1919}\\\midrule
\ipi{Pforzheim} & BWb; \ipi{Pforzheim} & \citet{Sexauer1927}\\\midrule
\ipi{Blaubeuren} & BWb; Alb-Donau-Kreis & \citet{Strohmaier1930}\\\midrule
Area around \ipi{Herrenberg} & BWb; Landkreis Böblingen & \citet{Zinser1933}\\\midrule
\ipi{Staudengebiet} (southwest of Augsburg) & Bvr: Swabia & \citet{Moser1936}\\\midrule
\ipi{Dreistammesecke} & Bvr: Swabia & \citet{Nübling1938}\\\midrule
Area around \ipi{Bavendorf} (Ravensburg) & BWb; Landkreis Ravensburg & \citet{Schöller1939}\\\midrule
Beuren\ip{Beuren (Allgäu)} & BWb; Landkreis Wangen & \citet{BausingerRuoff1959}\\\midrule
\ipi{Erdmannsweiler}; Neckar- und Donaugebiet & BWb; Schwarzwald-Baar-Kreis & \citet{Besch1961}\\\midrule
\ipi{Freudenstadt} & BWb; Landkreis \ipi{Freudenstadt} & \citet{Baur1967}\\\midrule
\ipi{Memmingen} & Bvr; Swabia & \citet{Hufnagl1967}\\\midrule
Kreis \ipi{Balingen} & BWb & \citet{BethgeBonnin1969}\\\midrule
\ipi{Graben} & Bvr; Landkreis Augsburg & \citet{König1970}\\\midrule
Large area between Augsburg and Donauwörth & Bvr; Landkreis Augsburg, Landkreis Donau-\ipi{Ries} & \citet{Ibrom1971}\\\midrule
\ipi{Stuttgart} & BWb; \ipi{Stuttgart} & \citet{Frey1975}\\\midrule
\ipi{Tuningen}, \ipi{Donaueschingen} & BWb; Schwarzwald-Baar-Kreis & E.M. \citet{Hall1991, Hall1991b}\\\midrule
\ipi{Ebersbach} (near Kaufbeuren) & Bvr; Swabia & SBS\\\midrule
Büßlingen (Tengen), Überlingen, Wangen & BWb; Landkreis Konstanz; Bodenseekreis; Landkreis Ravensburg & SSA\\\midrule
Gerstetten, Sontheim an der Brenz, Rudersberg & BWb; Landkreis Heidenheim;

Rems-Murr-Kreis & SNBW\\\midrule
Wangen im Allgäu (Wangen im Allgäu) & BWb; Landkreis Ravensburg & VALTS\\
\end{longtable}

\begin{longtable}{>{\raggedright}p{.4\textwidth}>{\raggedright}p{.25\textwidth}>{\raggedright\arraybackslash}p{.35\textwidth-4\tabcolsep}}
\caption{South Bavarian}\\
\lsptoprule Place/Region & Administ. Division & Source\\\midrule\endfirsthead
\midrule Place/Region & Administ. Division & Source\\\midrule\endhead\endfoot\lspbottomrule\endlastfoot
\ipi{Imst} & AT; \ipi{Tyrol} & \citet{Schatz1897}\\\midrule
\ipi{Tyrol} & AT; \ipi{Tyrol} & \citet{Schatz1903}\\\midrule
\ipi{Silltal} & AT; \ipi{Tyrol} & \citet{Egger1909}\\\midrule
\ipi{Samnaun} & CH; Grisons & \citet{Gröger1924}\\\midrule
Area around \ipi{Meran} (\ipi{Naturns}, \ipi{Passeiertal}) & IT; South \ipi{Tyrol} & \citet{Insam1936}\\\midrule
\ipi{St. Ruprecht bei Villach} & AT; Carinthia & \citet{Kurath1965}\\\midrule
\ipi{Imst} & AT; \ipi{Tyrol} & \citet{Hathaway1979}\\\midrule
\ipi{Graz}, \ipi{Innsbruck} & AT; Styria, \ipi{Tyrol} & \citet{Moosmüller1991}\\\midrule
\ipi{Garmisch-Partenkirchen} & Bvr; Upper Bavaria & \citet{Stein-Meintker2000}\\\midrule
\ipi{Laurein} & IT; South \ipi{Tyrol} & \citet{Kollmann2007}\\\midrule
\ipi{Zillertal}; Tauferer Tal, Ultental, Eisacktal & AT; \ipi{Tyrol} IT; South \ipi{Tyrol} & TSA\\\midrule
\ipi{Ötztal}; \ipi{Passeiertal} & AT; \ipi{Tyrol} IT; South \ipi{Tyrol} & VALTS\\
\end{longtable}


\begin{longtable}{>{\raggedright}p{.4\textwidth}>{\raggedright}p{.25\textwidth}>{\raggedright\arraybackslash}p{.35\textwidth-4\tabcolsep}}
\caption{Central Bavarian}\\
\lsptoprule Place/Region & Administ. Division & Source\\\midrule\endfirsthead
\midrule Place/Region & Administ. Division & Source\\\midrule\endhead\endfoot\lspbottomrule\endlastfoot
\ipi{Vienna} & AT & \citet{Gartner1900}\\\midrule
\ipi{Rot-Tal} & Bvr; Lower Bavaria & \citet{Schwäbl1903}\\\midrule
\ipi{Loosdorf} & AT; Lower Austria & \citet{Seemüller1908a}\\\midrule
\ipi{St. Georgen} an der Gusen & AT; Upper Austria (Mühlviertel) & \citet{Seemüller1909a}\\\midrule
\ipi{Pilgersham} & AT; Upper Austria (Innkreis) & \citet{Seemüller1909b}\\\midrule
\ipi{Marchfeld} & AT; Upper Austria & \citet{Pfalz1911}\\\midrule
\ipi{Neckenmarkt} & AT; Burgenland & \citet{Bíró1918}\\\midrule
Upper Austria & AT; Upper Austria & \citet{Haasbauer1924}\\\midrule
\ipi{Hausruckviertel} & AT; Upper Austria & \citet{Mindl19241925}\\\midrule
\ipi{Böhmerwald} (broad area to the northeast of Passau) & Bav, CZ & \citet{Kubitschek1926}\\\midrule
\ipi{Freutsmoos} & Bvr; Upper Bavaria & \citet{Kufner1957}\\\midrule
\ipi{Munich} & Bvr & \citet{Kufner1957}\\\midrule
Broad area ca. 80km southeast of \ipi{Munich} and 40km northwest of \ipi{Salzburg} & Bvr & \citet{Kufner1960}\\\midrule
\ipi{Linz} and \ipi{Gmünden} & AT & \citet{Keller1961}\\\midrule
Area between Isar and Inn rivers and Austrian border (Kiefersfelden, Isarwinkel) & Bvr; Upper Bavaria & \citet{Maier1965}\\\midrule
\ipi{Munich} & Bvr & \citet{BethgeBonnin1969}\\\midrule
\ipi{Großberghofen} (Erdweg) & Bvr; Upper Bavaria & \citet{Gladiator1971}\\\midrule
Large area between Augsburg and Aichach & Bvr; Swabia & \citet{Ibrom1971}\\\midrule
Area in western Hungary at the confluence of the Danube and Raab Rivers & HU & \citet{Manherz1977}\\\midrule
\ipi{Hallertau} & Bvr; Upper Bavaria, Lower Bavaria & \citet{Zehetner1978}\\\midrule
\ipi{Vienna} & AT & \citet{Moosmüller1987}\\\midrule
\ipi{Salzburg}, and \ipi{Vienna} & AT & \citet{Moosmüller1991}\\\midrule
\ipi{Ramsau am Dachstein} & AT; Styria & \citet{Noelliste2017}\\\midrule
Grafrath, Weilheim & Bvr; Upper Bavaria & SBS\\\midrule
Many place in Lower Bavaria & Bvr; Lower Bavaria & SNiB\\
\end{longtable}

\begin{longtable}{>{\raggedright}p{.4\textwidth}>{\raggedright}p{.25\textwidth}>{\raggedright\arraybackslash}p{.35\textwidth-4\tabcolsep}}
\caption{North Bavarian}\\
\lsptoprule Place/Region & Administ. Division & Source\\\midrule\endfirsthead
\midrule Place/Region & Administ. Division & Source\\\midrule\endhead\endfoot\lspbottomrule\endlastfoot
\ipi{West Bohemia} & Bvr, CZ & \citet{Gradl1895}\\\midrule
\ipi{Nürnberg} & Bvr; Central Franconia & \citet{Gebhardt1907}\\\midrule
\ipi{Egerland} & Bvr, CZ & \citet{Eichhorn1908}\\\midrule
\ipi{Eisendorf} & CZ & \citet{Seemüller1908b}\\\midrule
\ipi{Untereichenbach} (\ipi{Schwabach}) & Bvr; Central Franconia & \citet{Hain1936}\\\midrule
\ipi{Asch} (Westsudetenland) & CZ & \citet{Gütter1962a}\\\midrule
\ipi{Schönbach} (Westsudetenland) & CZ & \citet{Gütter1962b}\\\midrule
\ipi{Lauterbach} (Westsudetenland) & CZ & \citet{Gütter1963a}\\\midrule
\ipi{Graslitz} (Westsudetenland) & CZ & \citet{Gütter1963b}\\\midrule
\ipi{Bergstetten} (Laaber) & Bvr; Upper Palatinate & \citet{Dozauer1967}\\\midrule
\ipi{Rezat-Altmühl} (area to southwest of \ipi{Nürnberg}) & Bvr; Central Franconia & \citet{Schödel1967}\\\midrule
Kreis \ipi{Wunsiedel}; Kreis \ipi{Schwabach} & Bvr; Upper Franconia; Central Franconia & \citet{BethgeBonnin1969}\\\midrule
\ipi{Windischeschenbach} & Bvr; Upper Palatinate & \citet{Denz1977}\\\midrule
\ipi{Kallmünz} & Bvr; Upper Palatinate & \citet{Götz1987}\\\midrule
\ipi{Eslarn} & Bvr; Upper Palatinate & \citet{Bachmann2000}\\\midrule
Raitenbuch, Dettenheim (Weissenburg), Mörnsheim & Bvr; Central Franconia, Upper Bavaria & SBS\\\midrule
Heuberg (Hilpoltstein), Ebenried (Allersberg) & Bvr; Central Franconia & SMF\\\midrule
Zinzenzell, Herrnsaal (Kehlheim), Atting & Bvr; Lowr Bavaria & SNiB\\\midrule
Miltach & Bvr; Upper Palatinate & SNOB\\
\end{longtable}

\begin{longtable}{>{\raggedright}p{.4\textwidth}>{\raggedright}p{.25\textwidth}>{\raggedright\arraybackslash}p{.35\textwidth-4\tabcolsep}}
\caption{South Bavarian island}\\
\lsptoprule Place/Region & Administ. Division & Source\\\midrule\endfirsthead
\midrule Place/Region & Administ. Division & Source\\\midrule\endhead\endfoot\lspbottomrule\endlastfoot
\ipi{Erdmannsdorf}/\ipi{Zillertal} & Sil; Kreis \ipi{Hirschberg}/AT; \ipi{Tyrol} & \citet{Siebs1906}\\
\end{longtable}


\begin{longtable}{>{\raggedright}p{.4\textwidth}>{\raggedright}p{.25\textwidth}>{\raggedright\arraybackslash}p{.35\textwidth-4\tabcolsep}}
\caption{East Franconian}\\
\lsptoprule Place/Region & Administ. Division & Source\\\midrule\endfirsthead
\midrule Place/Region & Administ. Division & Source\\\midrule\endhead\endfoot\lspbottomrule\endlastfoot
\ipi{Schöneck} & Sxn; Vogtlandkreis & \citet{Hedrich1891}\\\midrule
\ipi{Pfersdorf} (Hildburghausen) & Thra; Landkreis Hildburghausen & \citet{HertelHertel1902}\\\midrule
\ipi{Heilbronn} & BWb & \citet{Braun1906}\\\midrule
\ipi{Wachbach} (Bad-Mergentheim) & BWb; Main-Tauber-Kreis & \citet{Dietzel1908}\\\midrule
\ipi{Vogtland} (Trieb) & Sxn; Vogtlandkreis & \citet{Gerbet1908}\\\midrule
\ipi{Klein-Allmerspann} (Gerabronn) & BWb; Landkreis Schwäbisch Hall & \citet{Blumenstock1911}\\\midrule
\ipi{Bamberg} & Bvr & \citet{Batz1911}\\\midrule
\ipi{Rot-Tal} (area to the south of Schwäbisch Hall) & BWb; Landkreis Schwäbisch Hall & \citet{Knupfer1912}\\\midrule
\ipi{Frankenland} (Königheim, Steinbach bei Wertheim, Höpfingen) & BWb; Main-Tauber-Kreis, Neckar-Odenwald-Kreis & \citet{Heilig1912}\\\midrule
\ipi{Bonnland} & Bvr; Lower Franconia & M. \citet{MSchmidt1912}\\\midrule
\ipi{Kleinschmalkalden} (Floh-Seligenthal) & Thra; Landkreis Schmalkalden-Meiningen & \citet{Dellit1913}\\\midrule
\ipi{Schmalkalden} & Thra; Landkreis Schmalkalden-Meiningen & \citet{Kaupert1914}\\\midrule
\ipi{Gaisbach} & BWb; Hohenlohekreis & \citet{Sander1916}\\\midrule
Fichtelgebirge (area between Bayreuth and Plauen) & Bvr, Sxn & \citet{Meinel1932}\\\midrule
\ipi{Schefflenz} & BWb; Neckar-Odenwaldkreis & \citet{Roedder1936}\\\midrule
\ipi{Frankenwald} & Bvr; Upper Franconia & \citet{Werner1961}\\\midrule
\ipi{Suhl} & Thra & \citet{Kober1962}\\\midrule
\ipi{Waldau} (Schleusingen) & Thra; Landkreis Hildburghausen & \citet{Bock1965}\\\midrule
\ipi{East Franconia} (area north of Bayreuth) & Bvr & \citet{Steger1968}\\\midrule
\ipi{Spessart} & Bvr & \citet{Hirsch1971}\\\midrule
\ipi{West Central Franconia} & Bvr & \citet{Diegritz1971}\\\midrule
\ipi{Obermainraum} (area between \ipi{Bamberg} and Bayreuth) & Bvr; Upper Franconia & \citet{Trukenbrod1973}\\\midrule
In and around \ipi{Heilbronn} & BWb & \citet{Jakob1985}\\\midrule
\ipi{Weingarts} (Kunreuth) & Bvr; Upper Franconia & \citet{Schnabel2000}\\
\end{longtable}

\begin{longtable}{>{\raggedright}p{.4\textwidth}>{\raggedright}p{.25\textwidth}>{\raggedright\arraybackslash}p{.35\textwidth-4\tabcolsep}}
\caption{East Hessian}\\
\lsptoprule Place/Region & Administ. Division & Source\\\midrule\endfirsthead
\midrule Place/Region & Administ. Division & Source\\\midrule\endhead\endfoot\lspbottomrule\endlastfoot
\ipi{Bad Salzungen} & Thra; Wartburgkreis & \citet{Hertel1888}\\\midrule
\ipi{Bad Hersfeld} & Hss; Landkreis Hersfeld-Rotenburg & \citet{Salzmann1888}\\\midrule
\ipi{Rhöntal} (Eichenzell, Dipperz, Margretenhaun) & Hss, Bvr & \citet{Glöckner1913}\\\midrule
\ipi{Fulda} & Hss; Landkreis \ipi{Fulda} & \citet{Noack1938} \\\midrule
Broad area in and around \ipi{Bad Hersfeld} & Hss; Landkreis Hersfeld-Rotenburg & \citet{Martin1957}\\\midrule
\ipi{Hintersteinau} & Hss; Main-Kinzig-Kreis & \citet{Müller1958a}\\\midrule
\ipi{Werra-Fuldaraum} (area in and around Hünfeld) & Hss & \citet{Weber1959}\\\midrule
\ipi{Schlitzerland} (Area around Schlitz) & Hss; Vogelbergkreis & \citet{Krafft1969}\\\midrule
\ipi{Fuldaer Land} (Kreis \ipi{Fulda}, Kreis Hünfeld) & Hss & \citet{Wegera1977}\\\midrule
\ipi{Bad Salzschlirf} & Hss; Landkreis \ipi{Fulda} & \citet{Post1985}\\\midrule
\ipi{Petersberg} (\ipi{Fulda}) & Hss; Landkreis \ipi{Fulda} & \citet{Schwarz1992}\\\midrule
Area in and around \ipi{Fulda} & Hss; Landkreis \ipi{Fulda} & \citet{Dingeldein1995}\\
\end{longtable}


\begin{longtable}{>{\raggedright}p{.4\textwidth}>{\raggedright}p{.25\textwidth}>{\raggedright\arraybackslash}p{.35\textwidth-4\tabcolsep}}
\caption{Central Hessian}\\
\lsptoprule Place/Region & Administ. Division & Source\\\midrule\endfirsthead
\midrule Place/Region & Administ. Division & Source\\\midrule\endhead\endfoot\lspbottomrule\endlastfoot

\ipi{Naunheim} (Wetzlar) & Hss; Lahn-Dill-Kreis & \citet{Leidolf1891}\\\midrule
\ipi{Großen-Buseck} bei Gießen & Hss; Landkreis Gießen & \citet{WagnerHorn1900}\\\midrule
\ipi{Atzenhain} (\ipi{Mücke}), \ipi{Grünberg} & Hss; Vogelsbergkreis, Landkreis Gießen & \citet{Knauss1906}\\\midrule
\ipi{Schlierbach} (Bad Endbach) & Hss; Landkreis Marburg-Biedenkopf & \citet{Schaefer1907}\\\midrule
\ipi{Friedberg} & Hss; Wetteraukreis & \citet{Reuss1907}\\\midrule
\ipi{Marburg} & Hss; Landkreis Marburg-Biedenkopf & \citet{Freund1910}\\\midrule
\ipi{North Pfahlgraben} (area south of Gießen) & Hss; Landkreis Limburg-Weilburg & \citet{Faber1912}\\\midrule
\ipi{Wissenbach} (Eschenburg) & Hss; Lahn-Dill-Kreis & \citet{Kroh1915}\\\midrule
\ipi{Frankfurt am Main} & Hss & \citet{Rauh1921}\\\midrule
\ipi{Selters bei Weilburg} & Hss; Landkreis Limburg-Weilburg & \citet{Schwing1921}\\\midrule
\ipi{Langenselbold} (\ipi{Hanau}) & Hss; Main-Kinzig Kreis & \citet{Siemon1922}\\\midrule
\ipi{Hanau} & Hss; Main-Kinzig-Kreis & \citet{Urff1926}\\\midrule
\ipi{Wetterfeld} (\ipi{Laubach}) & Hss; Landkreis Gießen & \citet{Schudt1927}\\\midrule
\ipi{Ebsdorf} (Ebsdorfergrund) & Hss; Landkreis Marburg-Biedenkopf & \citet{Bender1938}\\\midrule
\ipi{Weidenhausen} (Gladenbach) & Hss; Landkreis Marburg-Biedenkopf & \citet{Friebertshäuser1961}\\\midrule
In and around \ipi{Mammolshain} (Königstein im Taunus) & Hss; Hochtaunuskreis & \citet{Schnellbacher1963}\\\midrule
Area around \ipi{Marburg} & Hss; Landkreis Marburg-Biedenkopf & \citet{Spenter1964}\\\midrule
\ipi{Frankfurt am Main} & Hss & \citet{BethgeBonnin1969}\\\midrule
\ipi{Erbstadt} (Nidderau) & Hss; Main-Kinzig-Kreis & \citet{Schudt1970} \\\midrule
\ipi{Central Vogelsberg} & Hss & \citet{Hasselbach1971}\\\midrule
\ipi{Central Hesse} (area between Gieβen and \ipi{Marburg}) & Hss & \citet{Hasselberg1979}\\\midrule
\ipi{Frankfurt am Main} & Hss & \citet{Féry2017}\\
\end{longtable}

\begin{longtable}{>{\raggedright}p{.4\textwidth}>{\raggedright}p{.25\textwidth}>{\raggedright\arraybackslash}p{.35\textwidth-4\tabcolsep}}
\caption{North Hessian}\\
\lsptoprule Place/Region & Administ. Division & Source\\\midrule\endfirsthead
\midrule Place/Region & Administ. Division & Source\\\midrule\endhead\endfoot\lspbottomrule\endlastfoot
\ipi{Blankenheim} (Bebra) & Hss; Landkreis Hersfeld-Rotenburg & \citet{Dittmar1891}\\\midrule
\ipi{Loshausen-Zella} (Willingshausen) & Hss; Schwalm-Eder-Kreis & \citet{Schoof1913a,Schoof1913b,Schoof1913c}\\\midrule
\ipi{Amtshausen} (Bad Laasphe) & NRW; Kreis Siegen-Wittgenstein & \citet{Hackler1914}\\\midrule
Kreis \ipi{Alsfeld} & Hss & \citet{Heidt1922}\\\midrule
\ipi{Oberellenbach} (Alheim) & Hss; Landkreis Hersfeld-Rotenburg & \citet{Hofmann1926}\\\midrule
\ipi{Rauschenberg} & Hss; Landkreis Marburg-Biedenkopf & \citet{Bromm1936}\\\midrule
\ipi{Loshausen} (Willingshausen) & Hss; Schwalm-Eder-Kreis & \citet{Corell1936} \\\midrule
\ipi{Niederhessen} (area south of \ipi{Kassel}) & Hss & \citet{Hofmann1940}\\\midrule
\ipi{Battenberg} (Eder), \ipi{Bad Wildungen} & Hss; Landkreis Waldeck-Frankenberg & \citet{Martin1942}\\\midrule
\ipi{Kassel} & Hss & \citet{Müller1958b}\\\midrule
\ipi{Siegerland}/\ipi{Eichsfeld} & Hss; Landkreis Waldeck-Frankenberg & \citet{Möhn1962}\\\midrule
\ipi{Holzhausen am Reinhardswald} (Immenhausen) & Hss; Landkreis \ipi{Kassel} & \citet{Arend1991}\\
\end{longtable}



\begin{longtable}{>{\raggedright}p{.4\textwidth}>{\raggedright}p{.25\textwidth}>{\raggedright\arraybackslash}p{.35\textwidth-4\tabcolsep}}
\caption{Rhenish Franconian}\\
\lsptoprule Place/Region & Administ. Division & Source\\\midrule\endfirsthead
\midrule Place/Region & Administ. Division & Source\\\midrule\endhead\endfoot\lspbottomrule\endlastfoot
\ipi{Mainz} & RnPl & \citet{Reis1892}\\\midrule
\ipi{Southeast Palatinate} & RnPl & \citet{Heeger1896}\\\midrule
\ipi{Handschuhsheim} (Heidelberg) & BnWb & \citet{Lenz1900}\\\midrule
Zaisenhausen & BnWb; Landkreis Karlsruhe & \citet{Wanner1907,Wanner1908}\\\midrule
\ipi{Ober-Flörsheim} & RnPl; Landkreis Alzey-Worms & \citet{Haster1908}\\\midrule
\ipi{Beerfelden} & Hss; Odenwaldkreis & \citet{Wenz1911}\\\midrule
\ipi{Mönchzell} (Meckesheim) & BnWb; Rhein-Neckar-Kreis & \citet{Reichert1914}\\\midrule
\ipi{Warmsroth} & RnPl; Landkreis Bad Kreuznach & \citet{Martin1922}\\\midrule
\ipi{Kaulbach} & RnPl; Landkreis Kusel & \citet{Christmann1927}\\\midrule
\ipi{Ludwigshafen am Rhein} & RnPl & \citet{Krell1927}\\\midrule
\ipi{Spessart} (Ettlingen) & BnWb; Landkreis Karlsruhe & \citet{Lauinger1929}\\\midrule
Odenwald (\ipi{Zell im Mümlingtal}, \ipi{Bad König}) & Hss & \citet{Freiling1929}\\\midrule
\ipi{Heppenheim} & Hss; Kreis Bergstrasse & \citet{Seibt1930}\\\midrule
\ipi{Plankstadt} & BnWb; Rhein-Neckar-Kreis & \citet{Treiber1931}\\\midrule
\ipi{Saarbrücken} & Sld & \citet{Kuntze1932}\\\midrule
\ipi{Speyer} & RnPl & \citet{Waibel1932}\\\midrule
\ipi{Pfungstadt} & Hss; Landkreis Darmstadt-Dieburg & \citet{Grund1935}\\\midrule
Vorderpfalz (\ipi{Nußdorf}) & RnPl; Landau & \citet{Bertram1937}\\\midrule
\ipi{Eberbach} & BnWb; Rhein-Neckar-Kreis & \citet{Kilian1951}\\\midrule
\ipi{South Odenwald}/\ipi{Ried} & Hss; Odenwaldkreis & \citet{Bauer1957}\\\midrule
\ipi{Darmstadt} & Hss & \citet{Keller1961}\\\midrule
\ipi{Oftersheim} & BnWb; Rhein-Neckar-Kreis & \citet{Liébray1969}\\\midrule
\ipi{Zweibrücken} & RnPl & \citet{Castleman1975}\\\midrule
South Palatinate (\ipi{Dahn}, \ipi{Wilgartswiesen}, \ipi{Iggelbach}) & RnPl; Landkreis Südwestpfalz, Landkreis Bad Dürkheim & \citet{Karch1980}\\\midrule
\ipi{Wackernheim} (Ingelheim am Rhein), \ipi{Nackenheim}, \ipi{Alzey}, \ipi{Wallertheim}, Bechtheim & RnPl; Landkreis Mainz-Bingen, Landkreis Alzey-Worms & \citet{Karch1981}\\\midrule
\ipi{Saarbrücken} & Sld & \citet{Steitz1981}\\\midrule
\ipi{Gabsheim} & RnPl; Landkreis Alzey-Worms & \citet{Post1987}\\\midrule
\ipi{Großrosseln} & Sld & \citet{Pützer1988}\\\midrule
\ipi{Michelstadt} & Hss; Odenwaldkreis & \citet{DurrellDavies1989} \\\midrule
Langatte, Laning, Schorbach & FR & ALLG\\\midrule
Remschingen, Bretten & BnWb; Enzkreis; Landkreis Karlsruhe & SNBW\\\midrule
Schneppenbach, Wintersbach & Bvr; Lower Franconia & SUF\\
\end{longtable}


\begin{longtable}{>{\raggedright}p{.4\textwidth}>{\raggedright}p{.25\textwidth}>{\raggedright\arraybackslash}p{.35\textwidth-4\tabcolsep}}
\caption{Moselle Franconian}\\
\lsptoprule Place/Region & Administ. Division & Source\\\midrule\endfirsthead
\midrule Place/Region & Administ. Division & Source\\\midrule\endhead\endfoot\lspbottomrule\endlastfoot
\ipi{Prüm} & RnPl; Eifelkreis Bitburg-\ipi{Prüm} & \citet{Büsch1888}\\\midrule
\ipi{Birkenfeld} & RnPl; Landkreis \ipi{Birkenfeld} & \citet{Baldes1896}\\\midrule
\ipi{Merzig} & Sld; Kreis Merzig-Waden & \citet{Fuchs1903}\\\midrule
\ipi{Lubeln}; Kanton Falkenberg & FR & \citet{Tarral1903}\\\midrule
\ipi{Siegerland} (area around Siegen) & NRW: Kreis Siegen-Wittgenstein & \citet{Reuter1903}\\\midrule
\ipi{Sehlem} & RnPl; Landkreis Bernkastel-Wittlich & \citet{Ludwig1906}\\\midrule
\ipi{Kenn} & RnPl; Landkreis Trier-Saarburg & \citet{Thomé1908}\\\midrule
\ipi{Sörth} & RnPl; Landkreis Altenkirchen & \citet{Hommer1910}\\\midrule
\ipi{Vianden} & LX & \citet{Engelmann1910}\\\midrule
\ipi{Laubach} & RnPl; Landkreis Cochem-Zell & \citet{Wimmert1910}\\\midrule
Kreis \ipi{Ottweiler} (area in and around Hasborn) & Sld & \citet{Scholl1912}\\\midrule
\ipi{Saarhölzbach} (Mettlach) & Sld & \citet{Thies1912}\\\midrule
\ipi{Ihren} (Winterspelt), \ipi{Sellerich}, \ipi{Weinsheim} & RnPl; Eifelkreis Bitburg-\ipi{Prüm} & \citet{Meyers1913, Meyers1913b}\\\midrule
\ipi{Arzbach} & RnPl; Rhein-Lahn-Kreis & \citet{Bach1921}\\\midrule
\ipi{Arel} & BE & \citet{Bertrang1921}\\\midrule
Saarlouis & Sld & \citet{Lehnert1926}\\\midrule
\ipi{Echternach} & LX; \ipi{Echternach} & \citet{Palgen1931}\\\midrule
\ipi{Ittersdorf} (Wallerfangen) & Sld; Landkreis Saarlouis & \citet{Pallier1934}\\\midrule
\ipi{Nordösling} & LX; Clervaux & \citet{Bruch1952}\\\midrule
Kreis Wittlich & RbPl & \citet{BethgeBonnin1969}\\\midrule
\ipi{East Belgium} (Burg-Reuland) & BE & \citet{Hecker1972}\\\midrule
Area around Burg-Reuland & BE & \citet{CajotBeckers1979}\\\midrule
\ipi{Bell} (Mendig) & RnPl; Landkreis Mayen-Koblenz & \citet{Mattheier1987}\\\midrule
\ipi{Horath} (Hunsrück) & RnPl; Landkreis Bernkastel-Wittlich & \citet{Reuter1989}\\\midrule
Beuren\ip{Beuren (Trier)}(near Trier) & RnPl & \citet{Peetz1989}\\\midrule
Lxm & LX & \citet{Gilles1999}\\\midrule
\ipi{Montabaur} & RnPl: Westerwaldkreis & \citet{Féry2017}\\\midrule
\ipi{Lützkampen}/\ipi{Dahnen} & RnPl; Eifelkreis Bitburg-\ipi{Prüm} & MRhSA\\\midrule
Elzange & FR & ALLG\\
\end{longtable}


\begin{longtable}{>{\raggedright}p{.4\textwidth}>{\raggedright}p{.25\textwidth}>{\raggedright\arraybackslash}p{.35\textwidth-4\tabcolsep}}
\caption{Ripuarian}\\
\lsptoprule Place/Region & Administ. Division & Source\\\midrule\endfirsthead
\midrule Place/Region & Administ. Division & Source\\\midrule\endhead\endfoot\lspbottomrule\endlastfoot
Aix-la-Chapelle (\ipi{Aachen}) & NRW & \citet{Rovenhagen1860}\\\midrule
\ipi{Cologne} & NRW & \citet{Wahlenberg1877}\\\midrule
\ipi{Krefeld} & NRW & \citet{Röttsches1877}\\\midrule
\ipi{Werden} (Essen) & NRW & \citet{Koch1879}\\\midrule
\ipi{Remscheid} & NRW & \citet{Holthausen1885, Holthausen1885b}\\\midrule
Ronsdorf (Wuppertal) & NRW & \citet{Holthaus1887}\\\midrule
\ipi{Mülheim an der Ruhr} & NRW & \citet{Maurmann1889}\\\midrule
\ipi{Aachen} & NRW & \citet{Jardon1891}\\\midrule
Large area in western part of \il{Ripuarian}Rpn dialect area & NRW & \citet{Schmitz1893}\\\midrule
\ipi{Aegidienberg} (Bad Honnef) & NRW; Rhein-Sieg Kreis & \citet{Müller1900}\\\midrule
\ipi{Erftgebiet} & NRW & \citet{Münch19041970}\\\midrule
\ipi{Wermelskirchen} & NRW; Rheinisch-Bergischer Kreis & \citet{Hasenclever1905}\\\midrule
In and around \ipi{Cologne} & NRW & \citet{Müller1912}\\\midrule
\ipi{Dülken} (\ipi{Viersen}) & NRW & \citet{Frings1913}\\\midrule
Broad area in the northeastern part of the Ripuarian dialect area & NRW & \citet{Lobbes1915}\\\midrule
\ipi{Niederembt} (Elsdorf) & NRW; Rhein-Erft-Kreis & \citet{Grass1920}\\\midrule
\ipi{Düsseldorf} & NRW & \citet{Zeck1921}\\\midrule
\ipi{Schelsen} (Grevenbroich, Mönchengladbach) & NRW; Rhein-Kreis Neuss & \citet{Greferath1922}\\\midrule
Oberste Zeith (\ipi{Seelscheid}) & NRW; Rhein-Sieg-Kreis & \citet{Mackenbach1924}\\\midrule
Broad area in Oberbergischer Kreis, e.g. \ipi{Eckenhagen}, \ipi{Berghausen} & NRW; Oberbergischer Kreis & \citet{Branscheid1927}\\\midrule
Kreis \ipi{Eupen} & BE & \citet{Welter1929}\\\midrule
\ipi{Montzen} & BE & \citet{Welter1933}\\\midrule
\ipi{Schlebusch} (Leverkusen) & NRW & \citet{Bubner1935}\\\midrule
\ipi{Aachen} & NRW & \citet{Welter1938}\\\midrule
\ipi{Cologne} & NRW & \citet{Heike1964}\\\midrule
\ipi{Gleuel} (Hürth) & NRW; Rhein-Erft-Kreis & \citet{Heike1970}\\\midrule
\ipi{Moresnet} (Plombières) & BE & \citet{Jongen1972}\\\midrule
\ipi{East Belgium} (Elsenborn, Wallerode, Recht, St. Vith, Manderfeld) & BE & \citet{Hecker1972}\\\midrule
\ipi{Burscheid} & NRW; Rheinisch-Bergischen Kreis & \citet{Heinrichs1978}\\\midrule
Area around St. Vith & BE & \citet{CajotBeckers1979}\\\midrule
\ipi{Krefeld} & NRW & \citet{Bister-Broosen1989}\\\midrule
Euskirchen, Dahlem, Monschau, Zülpich, Langerwehe, Nörvenich, Jülich, Bonn, Heinsberg, Mönchengladbach, & NRW & \citet{CornelissenEtAl1989}\\\midrule
\ipi{Rimburg} & NL; Limburg & \citet{Hinskens1992}\\\midrule
\ipi{Düsseldorf}/ \ipi{Cologne} (Lower Rhine German) & NRW & \citet{Hall1993}\\\midrule
\ipi{Erp} (Erftstadt) & NRW; Rhein-Erft-Kreis & \citet{Kreymann1994}\\\midrule
\ipi{Niederbachem}, \ipi{Oberbachem} (Wachtberg) & NRW; Rhein-Sieg-Kreis & \citet{Fuss2001}\\
\end{longtable}


\begin{longtable}{>{\raggedright}p{.4\textwidth}>{\raggedright}p{.25\textwidth}>{\raggedright\arraybackslash}p{.35\textwidth-4\tabcolsep}}
\caption{Low Franconian}\\
\lsptoprule Place/Region & Administ. Division & Source\\\midrule\endfirsthead
\midrule Place/Region & Administ. Division & Source\\\midrule\endhead\endfoot\lspbottomrule\endlastfoot
Area between \ipi{Geldern} and \ipi{Viersen} & NRW & \citet{Ramisch1908}\\\midrule
\ipi{Homberg} (Duisburg) & NRW & \citet{Meynen1911}\\\midrule
\ipi{Kalkar} & NRW; Kreis \ipi{Kleve} & \citet{Hanenberg1915}\\\midrule
Kreis \ipi{Moers} & NRW; Kreis Wesel & \citet{BethgeBonnin1969}\\\midrule
\ipi{Kleve} & NRW & \citet{Stiebels2013}\\
\end{longtable}

\begin{longtable}{>{\raggedright}p{.4\textwidth}>{\raggedright}p{.25\textwidth}>{\raggedright\arraybackslash}p{.35\textwidth-4\tabcolsep}}
\caption{Thuringian}\\
\lsptoprule Place/Region & Administ. Division & Source\\\midrule\endfirsthead
\midrule Place/Region & Administ. Division & Source\\\midrule\endhead\endfoot\lspbottomrule\endlastfoot
\ipi{North Thuringia} (in and around Nordhausen) & Thra; Landkreis Nordhausen & \citet{Schultze1874}\\\midrule
\ipi{Stiege} (Oberharz) & SxAn; Landkreis Harz & \citet{Liesenberg1890}\\\midrule
\ipi{Eisenach} & Thra & \citet{Flex1893}\\\midrule
\ipi{Bad Frankenhausen} & Thra; Kyffhäuserkreis & \citet{Frank1898}\\\midrule
\ipi{Osterland} (Oberschwöditz,  between Zeitz and Naumburg) & SxAn; Burgenlandkreis & \citet{Trebs1899}\\\midrule
Mansfeld & SxAn; Landkreis Mansfeld-Südharz & \citet{Hennemann1901}\\\midrule
\ipi{Leinefelde} & Thra; Landkreis \ipi{Eichsfeld} & \citet{Hentrich1905}\\\midrule
\ipi{Altenburg} & Thra; Landkreis Altenburger Land & \citet{Daube1906}\\\midrule
\ipi{Buttelstedt} & Thra; Landkreis Weimarer Land & \citet{KürstenBremer}\\\midrule
\ipi{Southwest Thuringia} & Thra & \citet{Kürsten1910,Kürsten1911}\\\midrule
\ipi{Niddawitzhausen} (Eschwege) & Hss; Werra-Meissner-Kreis & \citet{Rasch1912}\\\midrule
Northeast Thuringia, southeast Sachsen-Anhalt & Thra, SxAn & \citet{Hankel1913}\\\midrule
\ipi{Eichsfeld} & Northwest Thra & \citet{Hentrich1920}\\\midrule
Honsteinisch (area north of \ipi{Sondershausen}) & Thra, SxAn & \citet{Rudolph192425}\\\midrule
\ipi{Sondershausen} & Thra; Kyffhäuserkreis & \citet{Schirmer1932}\\\midrule
\ipi{Gera} & Thra & \citet{Dietrich1957}\\\midrule
\ipi{Unterellen} (Gerstungen) & Thra; Wartburgkreis & \citet{Spangenberg1962}\\\midrule
East Thuringian & Thra & \citet{Spangenberg1974}\\\midrule
\ipi{Dudenrode}, \ipi{Netra} & Hss; Landkreis Witzenhausen,  Landkreis Eschwege & \citet{Guentherodt1982}\\\midrule
\ipi{Ludwigsstadt} & Bvr; Upper Franconia & \citet{Harnisch1987}\\\midrule
Thuringian dialect overview & Thra & \citet{Spangenberg1989}\\\midrule
\ipi{Barchfeld} (Barchfeld-Immelborn)~ & Thra; Wartburgkreis & \citet{Weldner1991}\\\midrule
\ipi{Itzgrund} (area between \ipi{Bamberg} and Coburg) & Bvr; Upper Franconia & \citet{Spangenberg1998}\\
\end{longtable}

\begin{longtable}{>{\raggedright}p{.4\textwidth}>{\raggedright}p{.25\textwidth}>{\raggedright\arraybackslash}p{.35\textwidth-4\tabcolsep}}
\caption{Upper Saxon}\\
\lsptoprule Place/Region & Administ. Division & Source\\\midrule\endfirsthead
\midrule Place/Region & Administ. Division & Source\\\midrule\endhead\endfoot\lspbottomrule\endlastfoot
Erzgebirge (\ipi{Annaberg-Buchholz}, \ipi{Freiberg}) & Sxn; Erzgebirgskreis, Landkreis Mittelsachsen & \citet{Goepfert1878}\\\midrule
\ipi{Leipzig} & Sxn & \citet{Albrecht18811983}\\\midrule
\ipi{Greiz} & Thra; Landkreis \ipi{Greiz} & \citet{Hertel1887}\\\midrule
\ipi{Zwickau} & Sxn; Landkreis \ipi{Zwickau} & \citet{Philipp1897}\\\midrule
\ipi{Brüx} & CZ & \citet{Hausenblas1898}\\\midrule
\ipi{Zschorlau} & Sxn; Erzgebirgskreis & \citet{Lang1906}\\\midrule
\ipi{Schokau} (Starý Šachov) & CZ & \citet{Pompé1907}\\\midrule
\ipi{Saalkreis} & SxAn & \citet{Bremer1909}\\\midrule
\ipi{Northwest Bohemia} & CZ & \citet{Hausenblas1914}\\\midrule
Large area between \ipi{Dresden} and \ipi{Chemnitz} (meiβnisch) & Sxn & \citet{Große1955}\\\midrule
\ipi{Leipzig} & Sxn & \citet{Große1957} \\\midrule
\ipi{West Lausitz} & Sxn; Landkreis Bautzen, Landkreis Sächsische-Schweiz Osterzgebirge & \citet{Protze1957}\\\midrule
\ipi{Salzfurtkapelle} (Zörbig) & SxAn; Landkreis Anhalt-Bitterfeld & \citet{Schönfeld1958}\\\midrule
Area in and around \ipi{Dresden} & Sxn & \citet{Fleischer1961}\\\midrule
\ipi{Vorerzgebirge} & Sxn & \citet{Bergmann1965}\\\midrule
Large area, especially south of \ipi{Chemnitz} and \ipi{Freiberg} & Sxn & \citet{Becker1969}\\\midrule
Kreis \ipi{Oschatz} (ca. 55km east of \ipi{Leipzig}) & Sxn & \citet{BethgeBonnin1969}\\\midrule
\ipi{Chemnitz} & Sxn & \citet{KahnWeise2013}\\
\end{longtable}


\begin{longtable}{>{\raggedright}p{.4\textwidth}>{\raggedright}p{.25\textwidth}>{\raggedright\arraybackslash}p{.35\textwidth-4\tabcolsep}}
\caption{Silesian\label{tab:appendix:c19}}\\
\lsptoprule Place/Region & Administ. Division & Source\\\midrule\endfirsthead
\midrule Place/Region & Administ. Division & Source\\\midrule\endhead\endfoot\lspbottomrule\endlastfoot
\ipi{Seifhennersdorf} & Sxn; Landkreis Görlitz & \citet{Michel1891}\\\midrule
\ipi{Sebnitz} & Sxn; Landkreis Sächsische-Schweiz Osterzgebirge & \citet{Meiche1898}\\\midrule
\ipi{Kieslingswalde} & Sil; Kreis Habelschwerdt & \citet{Pautsch1901}\\\midrule
\ipi{Lehmwasser} & Sil; Landkreis Waldenburg & \citet{Hoffmann1906}\\\midrule
Schlesische Mundart & Sil; CZ; \ipi{North Moravia}; AT & \citet{vonUnwert1908}\\\midrule
Kreis \ipi{Hirschberg} (Riesengebirge), \ipi{Alt-Waltersdorf bei Habelschwerdt} (\ipi{Grafschaft Glatz}) & Sil & \citet{Graebisch1912a,Graebisch1912b}\\\midrule
\ipi{Kunewald} & Sil; CZ & \citet{Giernoth1917}\\\midrule
Groβ-Schönau,  Seifnehhersdorf, \ipi{Sebnitz}, Markersdorf & Sxn; Landkreis Görlitz & \citet{Wenzel1919}\\\midrule
Reichenberg & CZ & \citet{Kämpf1920}\\\midrule
\ipi{East Bohemia} & CZ & \citet{Festa1925}\\\midrule
\ipi{Römerstadt}, \ipi{Sternberg} & Sil; Troppau & \citet{Rieger1935}\\\midrule
\ipi{North Moravia} (Marschendorf, Kunzendorf, Schildberg, Nieder-Ullersdorf, Rokitnitz) & CZ & \citet{Weiser1937}\\\midrule
\ipi{Bremberg} & Sil; Kreis \ipi{Jauer} & \citet{Halbsguth1938}\\\midrule
\ipi{Grafschaft Glatz} & Sil; Kreis Glatz & \citet{Blaschke1966}\\\midrule
\ipi{Kay} & Brbg; Kreis Züllichau-Schwiebus & \citet{Messow1965}\\\midrule
\ipi{Hohenelbe}, \ipi{Grulich}, \ipi{Bärn} & Sil, CZ & SchlSA\\
\end{longtable}


\begin{longtable}{>{\raggedright}p{.4\textwidth}>{\raggedright}p{.25\textwidth}>{\raggedright\arraybackslash}p{.35\textwidth-4\tabcolsep}}
\caption{North Upper Saxon-South Markish}\\
\lsptoprule Place/Region & Administ. Division & Source\\\midrule\endfirsthead
\midrule Place/Region & Administ. Division & Source\\\midrule\endhead\endfoot\lspbottomrule\endlastfoot
\ipi{Dubraucke} (Eichwege) & Brbg; Landkreis Spree-Neiβe; Döbern & \citet{Goessgen1902}\\\midrule
\ipi{Aken} (Elbe) & SxAn; Landkreis Anhalt-Bitterfeld & \citet{Bischoff1935}\\\midrule
\ipi{South Brandenburg} & Brbg; Landkreis  Elbe-Elster & \citet{Kieser1963}\\\midrule
\ipi{Friedersdorf} (Doberlug-Kirchhain) & Brbg; Landkreis  Elbe-Elster & \citet{Seibicke1967}\\\midrule
Weidenhain (Dreiheide) & Sxn; Landkreis Nordsachsen & \citet{Krug1969}\\\midrule
\ipi{Berlin} & \ipi{Berlin} & \citet{BethgeBonnin1969}\\\midrule
\ipi{Grassau} (Schönewalde) & Brbg; Landkreis  Elbe-Elster & \citet{Stellmacher1973}\\\midrule
\ipi{Wittenberg} & SxAn; Landkreis \ipi{Wittenberg} & \citet{Langner1977}\\\midrule
\ipi{Berlin} & \ipi{Berlin} & \citet{Schönfeld1986,Schönfeld2001}\\
\end{longtable}

\begin{longtable}{>{\raggedright}p{.4\textwidth}>{\raggedright}p{.25\textwidth}>{\raggedright\arraybackslash}p{.35\textwidth-4\tabcolsep}}
\caption{High Prussian}\\
\lsptoprule Place/Region & Administ. Division & Source\\\midrule\endfirsthead
\midrule Place/Region & Administ. Division & Source\\\midrule\endhead\endfoot\lspbottomrule\endlastfoot
Kreis Wormditt, Kreis Guttstadt, Kreis Heilsberg & EPr & \citet{Stuhrmann1896}\\\midrule
WPr/EPr & general description of HPr & \citet{Ziesemer1924}\\\midrule
Rollnau, Kahlau, Hagenau, Kreis Mohrungen & EPr & \citet{Kuck1927}\\\midrule
Kreis \ipi{Rosenberg} & WPr; Kreis & \citet{Kuck1933}\\\midrule
\ipi{Reimerswalde} & EPr; Kreis Heilsberg & \citet{KuckWiesinger1965}\\\midrule
Kahlau, Hagenau, Kreis Mohrungen, Kreis Heilsberg & EPr & \citet{Tessmann1969}\\
\end{longtable}


\begin{longtable}{>{\raggedright}p{.4\textwidth}>{\raggedright}p{.25\textwidth}>{\raggedright\arraybackslash}p{.35\textwidth-4\tabcolsep}}
\caption{North Low German}\\
\lsptoprule Place/Region & Administ. Division & Source\\\midrule\endfirsthead
\midrule Place/Region & Administ. Division & Source\\\midrule\endhead\endfoot\lspbottomrule\endlastfoot

\ipi{Greetsiel} (Krummhörn) & LSxn; Landkreis Aurich & \citet{Hobbing1879}\\\midrule
\ipi{Burg} (\ipi{Dithmarschen}) & SHst: \ipi{Dithmarschen} & \citet{Kohbrok1901}\\\midrule
\ipi{Oldenburg} & LSxn; \ipi{Oldenburg} & \citet{vorMohr1904}\\\midrule
\ipi{Lathen} & LSxn; Landkreis Emsland & \citet{Schönhoff1908}\\\midrule
\ipi{Badbergen} & LSxn; Landkreis Osnabrück & \citet{Vehslage1908}\\\midrule
\ipi{Bleckede} & LSxn; Landkreis Lüneburg & \citet{Rabeler1911}\\\midrule
\ipi{Finkenwärder} (Hamburg) & Hbg & \citet{Kloeke1914}\\\midrule
\ipi{Burg} (\ipi{Dithmarschen}) & SHst: \ipi{Dithmarschen} & \citet{Stammerjohann1914}\\\midrule
Stapelholm (\ipi{Bergenhusen}) & SHst; Kreis Schleswig-Flensburg & \citet{Sievers1914}\\\midrule
\ipi{Altengamme} (Hamburg) & Hbg & \citet{Larsson1917}\\\midrule
\ipi{Hollenstedt}; Jade & LSxn; Landkreis \ipi{Harburg}; LSxn; Landkreis Wesermarsch & \citet{Götze1922}\\\midrule
\ipi{Heide} (Dithmarschen) & SHst & \citet{Jörgensen1928}\\\midrule
Kreis Herzogtum \ipi{Lauenburg} & SHst & \citet{Heigener1937}\\\midrule
\ipi{Diepenau} (Samtgemeinde Uchte) & LSxn; Landkreis Nienburg & \citet{Schmeding1937}\\\midrule
\ipi{Borgstede} (Varel) & LSxn; Landkreis Friesen & \citet{Feyer1939}\\\midrule
\ipi{Baden} (Achim) & LSxn; Landkreis Verden & \citet{Feyer1941}\\\midrule
\ipi{Grambkermoor} bei Bremen & Brm & \citet{Bollmann1942}\\\midrule
\ipi{Jadebusen} & LSxn; Wilhelmshaven & \citet{Schmidt-Brockhoff1943}\\\midrule
\ipi{Hemmelsdorf}; Kreis Eutin & SHst; Kreis Ostholstein & \citet{Pühn1956}\\\midrule
\ipi{Kirchwerder} & Hbg & \citet{vonEssen1958}\\\midrule
\ipi{Harburg} & Hbg & \citet{Keller1961}\\\midrule
Kreis \ipi{Kiel} & SHst & \citet{BethgeBonnin1969}\\\midrule
\ipi{Oldenburger Ammerland} & LSxn; \ipi{Oldenburg} & \citet{Mews1971}\\\midrule
\ipi{Nordstrand} & SHst & \citet{Willkommen1999}\\\midrule
\ipi{Altenwerder} & Hbg & \citet{Höder2010}\\
\end{longtable}

\begin{longtable}{>{\raggedright}p{.4\textwidth}>{\raggedright}p{.25\textwidth}>{\raggedright\arraybackslash}p{.35\textwidth-4\tabcolsep}}
\caption{Westphalian}\\
\lsptoprule Place/Region & Administ. Division & Source\\\midrule\endfirsthead
\midrule Place/Region & Administ. Division & Source\\\midrule\endhead\endfoot\lspbottomrule\endlastfoot
\ipi{Soest} & NRW; Kreis \ipi{Soest} & \citet{Holthausen1886}\\\midrule
Kreis \ipi{Lippe} & NRW; Kreis \ipi{Lippe} & \citet{Hoffmann1887}\\\midrule
\ipi{Adorf} (\ipi{Diemelsee}) & Hss; Landkreis Waldeck-Frankenberg & \citet{Collitz1899}\\\midrule
\ipi{Schieder-Schwalenberg} & NRW; Kreis \ipi{Lippe} & \citet{Böger1906}\\\midrule
\ipi{Kirchspiel Courl} (\ipi{Dortmund}) & NRW & \citet{Beisenherz1907}\\\midrule
\ipi{Elspe} (Lennestadt) & NRW; Kreis Olpe & \citet{Arens1908}\\\midrule
\ipi{Hiddenhausen} & NRW; Kreis Herford & \citet{Schwagmeyer1908}\\\midrule
Area in and around \ipi{Paderborn} & NRW & \citet{Brand1914}\\\midrule
\ipi{Borken} & NRW; Kreis \ipi{Borken} & \citet{Herdemann19212006}\\\midrule
\ipi{Gütersloh} & NRW; Kreis \ipi{Gütersloh} & \citet{Wix1921}\\\midrule
\ipi{Behringhausen} (Castrop-Rauxel); \ipi{Schinkel} (Osnabrück) & NRW; Kreis Recklinghausen & \citet{Götz1922}\\\midrule
\ipi{Rhoden} (Diemelstadt) & Hss; Landkreis Waldeck-Frankenberg & \citet{Martin1925}\\\midrule
\ipi{Plettenberg} & NRW; Märkischer Kreis & \citet{Gregory1934}\\\midrule
Mülheim/Ruhr, \ipi{Byfang}/Ruhr, Hamm/\ipi{Lippe} & NRW & \citet{Hellberg1936}\\\midrule
\ipi{Ostbevern} & NRW; Kreis Warendorf & \citet{Holtmann1939}\\\midrule
\ipi{Southeast Sauerland} & NRW & \citet{Schulte1941}\\\midrule
\ipi{Willingen}, \ipi{Sudeck} (\ipi{Diemelsee}), \ipi{Freienhagen} (Waldeck) & Hss; Landkreis Waldeck-Frankenberg & \citet{Martin1942}\\\midrule
\ipi{Grafschaft Bentheim} & LSxn; Landkreis \ipi{Grafschaft Bentheim} & \citet{Rakers1944}\\\midrule
\ipi{Altenluenne} & LSxn; Landkreis Emsland & \citet{Borchert1955}\\\midrule
\ipi{Lüdenscheid} & NRW; Märkischer Kreis & \citet{Frebel1957}\\\midrule
\ipi{Münster} & NRW & \citet{Keller1961}\\\midrule
Kreis \ipi{Tecklenburg} & NRW & \citet{BethgeBonnin1969}\\\midrule
\ipi{Nienberge} (\ipi{Münster}) & NRW & \citet{Seymour1970}\\\midrule
\ipi{Riesenbeck} (Hörstel) & NRW; Kreis Steinfurt & \citet{Bethge1970}\\\midrule
\ipi{Reelkirchen} (Blomberg) & NRW; Kreis \ipi{Lippe} & \citet{Stellmacher1972}\\\midrule
\ipi{Laer} & NRW; Kreis Steinfurt & \citet{Niebaum1974, Niebaum1982}\\\midrule
\ipi{Müschede} (Arnsberg) & NRW; Hochsauerlandkreis & \citet{NiebaumTeepe1976}\\\midrule 
Breckerfeld, Hagen, Iserlohn & NRW & \citet{Brandes2011}\\
\end{longtable}

\begin{longtable}{>{\raggedright}p{.4\textwidth}>{\raggedright}p{.25\textwidth}>{\raggedright\arraybackslash}p{.35\textwidth-4\tabcolsep}}
\caption{Eastphalian}\\
\lsptoprule Place/Region & Administ. Division & Source\\\midrule\endfirsthead
\midrule Place/Region & Administ. Division & Source\\\midrule\endhead\endfoot\lspbottomrule\endlastfoot

\ipi{Meinersen} (Samtgemeinde \ipi{Meinersen}) & LSxn; Landkreis Gifhorn & \citet{Bierwirth1890}\\\midrule
\ipi{Börßum} (Samtgemeinde Oderwald) & LSxn; Landkreis \ipi{Wolfenbüttel} & \citet{Heibey1891}\\\midrule
\ipi{Magdeburger Börde} (Schnarsleben) & SxAn; Landkreis Börde & \citet{Roloff1902}\\\midrule
\ipi{Eilsdorf} (Huy) & SaAn; Landkreis Harz & \citet{Block1910}\\\midrule
\ipi{Cattenstedt} (Blankenburg) & SaAn; Landkreis Harz & \citet{Damköhler1919}\\\midrule
\ipi{Reinhausen} (Gleichen) & LSxn; Landkreis Göttingen & \citet{Jungandreas1926,Jungandreas1927}\\\midrule
\ipi{Ramlingen} (Burgdorf) & LSxn; Landkreis Region \ipi{Hannover} & \citet{Jarfe1929}\\\midrule
\ipi{Lesse} (Salzgitter) & LSxn; Landkreis \ipi{Wolfenbüttel} & \citet{Löfstedt1933}\\\midrule
\ipi{Dorste} (Osterode) & LSxn; Landkreis Göttingen & \citet{Dahlberg1934,Dahlberg1937}\\\midrule
\ipi{Dorste} (Osterode), \ipi{Hasede} (Hildesheim) & LSxn; Landkreis Göttingen, Landkreis Hildesheim & \citet{Mackel1939}\\\midrule
\ipi{Dingelstedt am Huy} (Huy) & SxAn; Landkreis Harz & \citet{Hille1939}\\\midrule
\ipi{Werratal} (area surrounding Witzenhausen) & Hss; Werra-Meißner-Kreis & \citet{Hassel1942}\\\midrule
Area around \ipi{Braunschweig} & LSxn & \citet{Pahl1943}\\\midrule
\ipi{Emmerstedt} (Helmstedt) & LSxn & \citet{Brugge1944}\\\midrule
\ipi{Neuendorf} (Teistungen) & Thra; Landkreis \ipi{Eichsfeld} & \citet{Schütze1953}\\\midrule
\ipi{Mascherode} (\ipi{Braunschweig}) & LSxn & \citet{BethgeFlechsig1958}\\\midrule
\ipi{Göddeckenrode}, \ipi{Isingerode} & SxAn; Landkreis Harz LSxn; Landkreis \ipi{Wolfenbüttel} & \citet{Lange1963}\\\midrule
Kreis \ipi{Hannover}, Kreis \ipi{Wolfenbüttel} & LSxn & \citet{BethgeBonnin1969}\\\midrule
\ipi{Kamschlaken} (and several other nearby towns and villages) & LSxn; Osterode am Harz, Landkreis Göttingen & \citet{Göschel1973}\\\midrule
\ipi{Celle} & LSxn & ACeM\\
\end{longtable}


\begin{longtable}{>{\raggedright}p{.4\textwidth}>{\raggedright}p{.25\textwidth}>{\raggedright\arraybackslash}p{.35\textwidth-4\tabcolsep}}
\caption{Mecklenburgish-West Pomeranian}\\
\lsptoprule Place/Region & Administ. Division & Source\\\midrule\endfirsthead
\midrule Place/Region & Administ. Division & Source\\\midrule\endhead\endfoot\lspbottomrule\endlastfoot
\ipi{Ivenack-Stavenhagen} & MVpm; Landkreis Mecklenburgische Seenplatte & \citet{Holst1907}\\\midrule
\ipi{Barth} & MVpm; Landkreis Vorpommern-Rügen &  \citet{GSchmidt1912}\\\midrule
\ipi{Wolgast} & MVpm; Landkreis Vorpommern-\ipi{Greifswald} & \citet{Warnkross1912}\\\midrule
\ipi{West Mecklenburg} & MVpm; Landkreis  Nordwestmecklenburg & \citet{Kolz1914}\\\midrule
\ipi{South Mecklenburg} & MVpm; Landkreis Ludwigslust-Parchim & \citet{Jacobs1925a,Jacobs1925b,Jacobs1926}\\\midrule
\ipi{Rehna}, \ipi{Schwerin} & MVpm & \citet{Teuchert1927}\\\midrule
Kaarβen (Amt Neuhaus) & LSxn; Landkreis Lüneburg & \citet{Dützmann1932}\\\midrule
\ipi{Ratzeburg}, \ipi{Rostock}, \ipi{Lank} (Lübtheen) & SHst, MVpm & \citet{TeuchertSchmitt1933}\\\midrule
\ipi{Stargard} (area to the north of Neustrelitz) & MVpm & \citet{Blume1933, Blume1933b,Blume1933c, Blume1933d}\\\midrule
South \ipi{Stargard} & MVpm & \citet{Teuchert1934}\\\midrule
Kreis \ipi{Wismar} & MVpm; Landkreis Nordwestmecklenburg & \citet{BethgeBonnin1969}\\\midrule
\ipi{Greifswald}, \ipi{Schwerin} & MVpm & \citet{Prowatke1973}\\\midrule
Survey of ELG (e.g. Teterow) & MVpm & \citet{Schönfeld1989}\\
\end{longtable}


\begin{longtable}{>{\raggedright}p{.4\textwidth}>{\raggedright}p{.25\textwidth}>{\raggedright\arraybackslash}p{.35\textwidth-4\tabcolsep}}
\caption{Brandenburgish}\\
\lsptoprule Place/Region & Administ. Division & Source\\\midrule\endfirsthead
\midrule Place/Region & Administ. Division & Source\\\midrule\endhead\endfoot\lspbottomrule\endlastfoot
In and around \ipi{Magdeburg} & SxAn & \citet{Krause1895}\\\midrule
Kreis \ipi{Jerichow} I (region in and around Möckern) & SxAn; Landkreis \ipi{Jerichower Land} & \citet{Krause1896}\\\midrule
\ipi{Besten} & Brbg; Landkreis Dahme-Spreewald & \citet{Siewert1907}\\\midrule
\ipi{Neumark} & PL & \citet{Teuchert1907a,Teuchert1907b}\\\midrule
\ipi{Warthe} (Uckermark) & Brbg; Landkreis Uckermark & \citet{Teuchert1907c}\\\midrule
\ipi{Prenden} (Wandlitz) & Brbg; Landkreis Barnim & \citet{Seelmann1908}\\\midrule
\ipi{Neu-Golm} (Bad Saarow) & Brbg; Landkreis Oder-Spree & \citet{Siewert1912}\\\midrule
Ostmärkische Mundart (Kreise Arnswalde, Friedeberg) & PL & \citet{Seelmann1913}\\\midrule
\ipi{Strodehne} (Havelaue) & Brbg; Landkreis Havelland & \citet{Hildebrand1913}\\\midrule
\ipi{Lüneburger Wendland} & LSxn: Landkreis Lüchnow-Dannenberg & \citet{Selmer1918}\\\midrule
\ipi{Rebenstorf} (Lübbow) & LSxn; Landkreis Lüchnow-Dannenberg & \citet{Götze1922}\\\midrule
\ipi{Letschin} & Brbg; Landkreis Märkisch-Oderland & \citet{Teuchert1930}\\\midrule
\ipi{Jerichower Land} & SxAn & \citet{Bathe1932}\\\midrule
\ipi{Kleinwusterwitz} (\ipi{Jerichow}) & SxAn & \citet{Bathe1937}\\\midrule
\ipi{Arendsee} (Altmark) & SxAn; Altmarkkreis-Salzwedel & \citet{Törnqvist1949}\\\midrule
\ipi{Hinzdorf} (Wittenberge) & Brbg; Landkreis Prignitz & \citet{Bretschneider1951}\\\midrule
\ipi{Heckelberg} & Brbg; Landkreis Märkisch-Oderland & \citet{Teuchert1964}\\\midrule
Large area in the western part of Brandenburg & Brbg & \citet{Bathe1965}\\\midrule
\ipi{Schollene} & SxAn; Landkreis Stendal & \citet{Gebhardt1965}, \citet{Schönfeld1965}\\\midrule
Survey of ELG (e.g. \ipi{Tempelfelde}) & Brbg & \citet{Schönfeld1989}\\
\end{longtable}

\begin{longtable}{>{\raggedright}p{.4\textwidth}>{\raggedright}p{.25\textwidth}>{\raggedright\arraybackslash}p{.35\textwidth-4\tabcolsep}}
\caption{Central Pomeranian}\\
\lsptoprule Place/Region & Administ. Division & Source\\\midrule\endfirsthead
\midrule Place/Region & Administ. Division & Source\\\midrule\endhead\endfoot\lspbottomrule\endlastfoot
Kreis Greifenhagen and Kreis \ipi{Königsberg} & PL & \citet{Brose1955}\\\midrule
Burg \ipi{Stargard} & MVpm; Landkreis Mecklenburgische Seenplatte & \citet{Prowatke1973}\\
\end{longtable}

\begin{longtable}{>{\raggedright}p{.4\textwidth}>{\raggedright}p{.25\textwidth}>{\raggedright\arraybackslash}p{.35\textwidth-4\tabcolsep}}
\caption{East Pomeranian}\\
\lsptoprule Place/Region & Administ. Division & Source\\\midrule\endfirsthead
\midrule Place/Region & Administ. Division & Source\\\midrule\endhead\endfoot\lspbottomrule\endlastfoot
\ipi{Putzig} (Posen) & PL & \citet{Teuchert1913}\\\midrule
Kreis \ipi{Konitz} & WPr; Kreis \ipi{Konitz} & \citet{Semrau1915a,Semrau1915b}\\\midrule
\ipi{Lauenburg} & EPmr; Kreis \ipi{Lauenburg} & \citet{Pirk1928}\\\midrule
Kreis \ipi{Schlawe} & EPmr; Kreis \ipi{Schlawe} & \citet{Mahnke1931}\\\midrule
Kreis \ipi{Saatzig} & EPmr; Kreis \ipi{Saatzig} & \citet{Kühl1932}\\\midrule
Kreis \ipi{Bütow}, Kreis \ipi{Rummelsburg} & EPmr; Kreis \ipi{Bütow}, Kreis \ipi{Rummelsburg} & \citet{Mischke1936}\\\midrule
Kreis \ipi{Lauenburg}, Kreis \ipi{Stolp} & EPmr; Kreis \ipi{Lauenburg}, Kreis \ipi{Stolp} & \citet{Stritzel1937}\\\midrule
\ipi{Kamnitz} & EPmr; Kreis Bublitz & \citet{Tita19211965}\\\midrule
\ipi{Sępóno Krajeńskie} & WPr & \citet{Darski1973}\\
\end{longtable}

\begin{longtable}{>{\raggedright}p{.4\textwidth}>{\raggedright}p{.25\textwidth}>{\raggedright\arraybackslash}p{.35\textwidth-4\tabcolsep}}
\caption{Low Prussian}\\
\lsptoprule Place/Region & Administ. Division & Source\\\midrule\endfirsthead
\midrule Place/Region & Administ. Division & Source\\\midrule\endhead\endfoot\lspbottomrule\endlastfoot
EPr & General descriptions of LPr & \citet{Gortzitza1841}, \citet{Lehmann1842}, \citet{Förstemann1850}, \citet{Fischer1896}, \citet{Kantel1900}, \citet{Betcke1924}, \citet{Ziesemer1924}, \citet{Schönfeldt1977}\\\midrule
\ipi{Alt-Thorn} & EPr & \citet{Wagner1912}\\\midrule
\ipi{Königsberg} & EPr; Kreis \ipi{Königsberg} & \citet{Mitzka1919}\\\midrule
\ipi{Danziger Nehrung} & EPr & \citet{Mitzka1922}\\\midrule
\ipi{Willuhnen} & EPr; Kreis Pillkallen & \citet{Natau1937}\\\midrule
In and around \ipi{Mandtkeim} & EPr; Kreis Fischhausen & \citet{Bink1953}\\\midrule
\ipi{Bieberstein} bei Barten & EPr & \citet{Tessmann1966}\\
\end{longtable}

\begin{longtable}{>{\raggedright}p{.4\textwidth}>{\raggedright}p{.25\textwidth}>{\raggedright\arraybackslash}p{.35\textwidth-4\tabcolsep}}
\caption{German-language islands\label{tab:c30}}\\
\lsptoprule Place/Region & Administ. Division & Source\\\midrule\endfirsthead
\midrule Place/Region & Administ. Division & Source\\\midrule\endhead\endfoot\lspbottomrule\endlastfoot


ES, LA & LG island (\ili{Baltic German}) & \citet{Sallmann1872}, \citet{Mitzka1923b, Mitzka1923}, \citet{Masing1926}, \citet{Deeters1939}\\\midrule
\ipi{Burgberg}, \ipi{Mediasch}, \ipi{Bistritz}, \ipi{Schäßburg} & MFr island (Transylvania Saxon) in RO & \citet{Scheiner1887}, \citet{Kisch1893}, \citet{Scheiner1922}, \citet{Klein1927}, \citet{Maurer1959}, \citet{Bruch1966}\\\midrule
\ipi{Hobgarten}, \ipi{Leibitz}, \ipi{Dobschau}, \ipi{Käsmark} & CG island (Zipser German) in SLK & \citet{Lumtzer1894, Lumtzer1896}, \citet{Gréb1921}, \citet{Kövi1911}, WbMD\\\midrule
\ipi{Lusern}, \ipi{Giazza}/\ipi{Dreizehn Gemeinden}, \ipi{Sieben Gemeinden} & \il{South Bavarian}SBav (Cimbrian) islands in Northeast IT & \citet{Bacher1905, Schweizer1939, Mayer1971, Kranzmayer1981, Tyroller2003}\\\midrule
\ipi{Mitterdorf}, \ipi{Suchener Tal}, Suchen, Hinterberg, Klindorf, Niedermösel, Reichenau, Rodine, Hornberg & \il{South Bavarian}SBav island (\ipi{Gottschee}) in SL & \citet{Tschinkel1908, Seemüller1909d, Wolf1982, Lipold1984}\\\midrule
\ipi{Altstadt}, \ipi{Langenlutsch}, \ipi{Rathsdorf}, Hilbetten, Michelsdorf , \ipi{Mährisch Hermersdorf}, \ipi{Vorder-Ehrnsdorf}, \ipi{Augezd}, Kornitz, Rehsdorf, \ipi{Rothmühl} & HG island (\ipi{Schönhengst}) in CZ & \citet{Seemüller1908c, Janiczek1911, Graebisch1915, Matzke1918, Sandbach1922, Appel1963, Benesch1979}\\\midrule
RUS, UKR, MEX, USA (Indiana, Missouri, Kansas, Oklahoma), CAN & LPr island (Plautdietsch) & \citet{Quiring1928}, \citet{Goerzen1952}, \citet{Lehn1957}, \citet{Mierau1964}, \citet{Moelleken1966}, \citet{Jedig1966}, \citet{Buchheit1978}, \citet{Loewen1988}, \citet{Naiditch2005}, \citet{Nieuweboer1999}, \citet{Siemens2012}, \citet{CoxTucker2013}, \citet{teVeldeVosburg2021}\\\midrule
North UKR & CHes island & \citet{SokolskajaSinder1930}\\\midrule
Jamburg (UKR) & NBav island & \citet{Schirmunski1931}\\\midrule
Sathmar & HG island in RO & \citet{Moser1937}\\\midrule
\ipi{Libinsdorf} & CG island in CZ & \citet{Weinelt1940}\\\midrule
Many states on the East Coast and Midwest & German-language island (Pennsylvania German) in USA & \citet{Frey1942, Reed1947, Buffington1954, Kelz1971}\\\midrule
\ipi{Zarz} & Bav island in SL & \citet{Lessiak1959}\\\midrule
USA (Texas) & German language island (\ili{Texas German}, \ili{Texas Alsatian}) & \citet{Gilbert1963, Gilbert1964}, \citet{Eikel1966}, \citet{Gilbert1970}, \citet{Boas2009}, \citet{Roesch2012}, LATG\\\midrule
\ipi{Iglau} & \il{North Bavarian}NBav island in CZ & \citet{Stolle1969}\\\midrule
\ipi{Milwaukee} (USA)  and \ipi{Mucsi} (HU) & Hes island in Wisconsin (USA) & \citet{Gommermann1975}\\\midrule
\ipi{Banat} & German-language island (\ili{Banat Swabian}) in RO & \citet{Barba1982, Wolf1987, Dama1991, Mileck1997}\\\midrule
\ipi{Fersental} & \il{South Bavarian}SBav island (Mòcheno) in Northeast IT & \citet{Rowley1986}\\\midrule
\ipi{Concordia} & LG island in Missouri (USA) & \citet{Ballew1997}\\\midrule
\ipi{Issime}, \ipi{Gressoney}, \ipi{Alagna}, \ipi{Rima}, \ipi{Macugnaga} & \il{South Bavarian}SBav islands in Northwest IT & SDS\\
\end{longtable}


\begin{table}
\caption{Standard languages}
\begin{tabular}{ll}
\lsptoprule
Language & Source\\\midrule
Modern Standard German (\il{Standard German}StG) & \citet{Krech1982}, \citet{Mangold2005}\\
\ili{Standard Swiss German} (StSwG) & \citet{Hove2002}, \citet{HoveHaas2009}\\
\ili{Standard Austrian German} (StAG) & \citet{MoosmüllerBrandstätter2015}\\
\lspbottomrule
\end{tabular}
\end{table}

\begin{table}
\caption{Other varieties of German}
\begin{tabularx}{\textwidth}{Ql}
\lsptoprule
Comments & Source\\\midrule
Variety of High German spoken in \ipi{Kiel} & \citet{Glove2011,Glover2014}\\
\midrule
Unspecified variety of German; data obtained by introspection & \citet{Moltmann1990}\\
\midrule
Ethnolects spoken in \ipi{Berlin} & \citet{Auer2002},\citet{Wiese2012}, \citet{JannedyWeirich2014}\\
\lspbottomrule
\end{tabularx}
\end{table}
