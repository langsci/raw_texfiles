\chapter{Summary and conclusion}\label{sec:18}

I recapitulate here the status of velar fronting as a synchronic rule (\sectref{sec:18.1}, \sectref{sec:18.2}), provide a brief synopsis of that process from the historical perspective (\sectref{sec:18.3}), and then discuss the significance of my findings (\sectref{sec:18.4}). The chapter concludes  with a series of questions I leave open for further research (\sectref{sec:18.5}).

\section{{Velar} {fronting} {viewed} {synchronically}}\label{sec:18.1}

Velar fronting differs structurally from dialect to dialect along three parameters: (a) segments undergoing the change (targets), (b) segments inducing the change (triggers), and (c) the nature of the fronted sound created (outputs). Targets consist of one or more velar sound ([k g kx x ɣ ŋ]) and triggers of some combination of coronal sonorants, i.e. front vowels or coronal sonorant consonants ([r l n]). Velar fronting can apply either in a word-initial onset or in postsonorant position.

The relationship between velars (e.g. [x]) and palatals (e.g. [ç]) is expressed with a rule  converting the former into the latter (velar fronting) and not the reverse. Both contexts for that rule (word-initial and postsonorant) have a number of different versions depending on the nature of triggers and targets. All versions of velar fronting are regular in the sense that there are no \isi{lexical exceptions}.

In the overwhelming number of dialects investigated, the front vowel triggers for velar fronting exhibit variation along the height dimension: In some varieties, the segments inducing fronting subsume only high front vowels, in others high and mid front vowels but not the low front vowels, and in yet others all front vowels, regardless of height. The fronting of velars can also be induced by a coronal sonorant consonant ([r l n]). In the most common velar fronting system -- the default pattern -- the triggers consist of all front vowels and all coronal sonorant consonants. In many areas, historical velars succumbed to velar fronting regardless of the nature of the adjacent sound; thus, velars surfaced as palatal even in the context of back vowels. It is probably not the case that \isi{nonassimilatory velar fronting} remains active synchronically.

In many varieties, the set of target sounds for velar fronting subsumes all and only velar fricatives ([x] and [ɣ]), but in other systems the target consists solely of [x] but not [ɣ]. In yet another set of dialects, velar fronting affects not only [x] and [ɣ], but also velar stops and the velar nasal (velar noncontinuants). In dialects with the velar \isi{affricate} [kx], that sound can also undergo fronting.

In the typical velar fronting system the target segments are realized as the corresponding palatals; hence, only place changes, while manner does not, i.e. [k g kx x ɣ ŋ] surface as [c ɉ kç ç ʝ ɲ] respectively. In the type of dialect referred to here, velar fronting alters a place feature only; in the formal model adopted that feature is [coronal], which spreads from a front ([coronal]) trigger to a velar ([dorsal]) target, thereby creating a complex corono-dorsal (palatal) segment. A common pattern for many varieties of CG is that the fortis fricative /x/ is realized in the front vowel context as the (\isi{sibilant}) alveolopalatal fricative [ɕ]. Velar fronting in such alveolopalatalizing dialects only alters a place feature; hence, [coronal] spreads to a [dorsal] target, and sibilancy is assigned to that complex segment by rules of \isi{phonetic implementation}.

An important theme discussed at length in the preceding chapters is the ways in which velar fronting interacts with synchronic and diachronic changes creating or eliminating structures which can potentially undergo or trigger it. In many dialects the relationship between velars (e.g. [x]) and the corresponding palatals (e.g. [{ҫ}]) is transparent because velars only occur in the back vowel context and palatals only when adjacent to front sounds. In that type of system, independent processes can either \isi{feed} or \isi{bleed} velar fronting. When velars and palatals have a transparent relationship they stand in complementary distribution and are classified as allophones.

A transparent relationship between velars and palatals does not obtain in other dialects. For example, in many varieties, both dorsal articulations occur in the context of front segments. Hence, in addition to expected sequences (e.g. [iç]), there are also unexpected ones (e.g. [ix]). In other systems velars and palatals both occur in the context of back segments; hence, expected sequences (e.g. [ɑx]) occur alongside unexpected ones (e.g. [ɑç]). Both types of system exemplify \isi{opacity}: A sequence like [ix] in the first system and [ɑç] in the second one illustrate the \isi{underapplication} and \isi{overapplication} of velar fronting respectively.

Two types of \isi{underapplication} have been identified: In one system velar fronting actively creates palatals (e.g. [ç]) from velars (e.g. /x/), and the opaque velar in the front vowel context (e.g. [x] in [ix]) is derived from an independent segment (/A/). In that dialect a sequence like [ɪx] (from /iA/) illustrates the \isi{underapplication} because the rule creating [x] from /A/ \isi{counterfeeds} velar fronting. In another type of system, velar fronting is active synchronically (e.g. /ix/ is realized as [iç] and /ɑx/ as [ɑx]), but [x] surfaces unexpectedly in the context of \isi{neutral vowels}, i.e. front vowels that are phonetically front but which behave phonologically as nonfront (e.g. /øix/ is realized as [øix]). An important generalization is that such \isi{neutral vowels} were historically back (e.g. [øi] < [ou]). Since [øi] is synchronically /øi/ and not /ou/, systems with \isi{neutral vowels} do not involve a synchronic \isi{counterfeeding} relationship between velar fronting and \isi{Vowel Fronting} ([øi] /øi/ < [ou] /ou/). However, \isi{Vowel Fronting} does exemplify the historical \isi{underapplication} of velar fronting.

Two types of \isi{overapplication} can be distinguished: In one, palatals (e.g. [ç]) occur in the context of front vowels and certain nonfront sounds ([Bk]) and velars (e.g. [x]) only in the context of  nonfront sounds with the exception of [Bk]. Observe that palatals ([ç]) and velars ([x]) stand in complementary distribution. All instances of palatals ([ç]) in the context of front vowels derive from the corresponding velars, but opaque palatals ([ç]) in the context of [Bk] are underlying (/ç/) and not derived from velars. Underlying (opaque) palatals in like those are referred to in the present book as palatal quasi-phonemes. In another type of system, velars and palatals both contrast in the neighborhood of the same back sounds. In that type of dialect velars and palatals are both underlying sounds in the context of the same back vowels (e.g. /x/ and /ç/). Underlying palatals in that type of example are referred to throughout this book as phonemic palatals. In dialects where palatals and velars are both phonemic, velar fronting is still active synchronically in order to capture regular alternations between velars and palatals because palatals but never velars surface in the front vowel context.

\section{{Additional} {properties} {of} {velar} {fronting}}\label{sec:18.2}

Velar fronting is categorical and not \isi{gradient} because it relates only two articulations -- velar (back dorsal) and palatal (front dorsal) -- and not multiple articulations, i.e. the fine-grained back dorsals and/or front dorsals observable in the phonetics. This interpretation of velar fronting  accounts for the fact that the back dorsal fricative (e.g. [x]) and the front dorsal fricative (e.g. [ç]) can be perceived by native speakers and that there are established colloquial terms for those two categories (ach-Laut and ich-Laut). By contrast, the distinction between various articulations within the back dorsal or front dorsal category lie below the threshold of consciousness of the linguistically naïve speaker and hence no colloquial terms exist to characterize them. This assessment of velar fronting is true for \il{Standard German}StG, but it also derives support from most of the descriptive studies on German dialects cited above, whose authors decided to describe the distribution of two categories (velar and palatal) and ignore finer-grained distinctions.

In those dialects where data are available, velar fronting fails to apply in connected speech as a phrasal (postlexical) rule. The trigger and target for velar fronting (in both the word-internal and postsonorant context) therefore belong to the same word. It can also be said that the trigger and target belong to the same morpheme, although the formal rules of velar fronting posited above do not need to encode that fact into their structural description because there are no words where a target (e.g. /x/) and trigger are separated by a morpheme boundary.

In the vast majority of dialects under investigation the trigger and target for velar fronting are adjacent. In some dialects the trigger and target can be separated by an intermediate sound (Q). If Q is \isi{schwa} (/ə/) then the velar after Q surfaces as palatal if the sound preceding Q is a front trigger (e.g. /iəx/→[iə̟ç] vs. /uəx/→[uəx]). It was shown that velar fronting is such cases is fed by a process creating a fronted ([coronal]) \isi{schwa} ([ə̟]). In dialects where Q is a liquid (e.g. /ilx/→[ilç] vs. /ɑlx/→[ɑlx]) it was argued that velar fronting is fed by a process merging the frontness feature of the vowel with the frontness feature of the liquid.

One way in which rules of assimilation can vary cross-linguistically is in terms of direction: If the trigger is to the right of the target then spreading is right-to-left (regressive), but if the trigger is to the left of the target then spreading is left-to-right (progressive). If a velar target is situated between two sonorants (e.g. vowels) then spreading is always progressive. That generalization is true without exception; it holds for the native words which have been the object of investigation of the present book as well as nonnative words (Appendix~\ref{appendix:g}). Significantly, this is one way velar fronting in German dialects differs from \isi{Velar Palatalization} because typological work has demonstrated that there are languages in which the latter process can be regressive and others in which it can be progressive.

\section{{Velar} {fronting} {viewed} {diachronically}}\label{sec:18.3}

At an early point in the history of Gmc -- namely \ili{WGmc} -- velar fronting was absent (Stage 1). It is hypothesized that velars ([x]) at Stage 1 were subject to some coarticulatory (phonetic) fronting in the context of front vowels, especially high front vowels like [i]. Phonologization (Stage 2) occurred when the difference between velar [x] and the slightly fronted variant (\isi{prevelar}) was exaggerated to the point where the latter was realized as palatal ([ç]), while the latter remained velar ([x]). At that point velar fronting became active as a synchronic process relating the two dorsal sounds. The target segment for velar fronting at that early stage was the fortis fricative [x] and the triggers were high front vowels like [i].

The newly phonologized rule of velar fronting diffused in terms of time and place to include a greater set of targets (Stage 2a > Stage 2n) and/or triggers (Stage 2aa > Stage 2n). Targets could expand to include not only fortis [x] but also lenis [ɣ], and then noncontinuants ([k g ŋ]). The set of triggers likewise increased to include high and mid front vowels, then all front vowels, and finally all coronal sonorants. In some regions velar fronting went one step further in applying as a nonassimilatory change in the context of front and back segments alike. Those historical stages are all preserved in dialects described in the modern era (late nineteenth century to the present). Of particular significance is Lower Bavaria, where over two hundred villages and towns represent three distinct historical stages.

A small number of dialects display a unique behavior suggesting that the historical paths described in the preceding paragraph need not be slavishly adhered to without exception. In particular, there are cases where velar fronting triggers are sensitive to \isi{tenseness} (Kreis \ipi{Rummelsburg}), roundedness (\ipi{Plettenberg}, \ipi{South Mecklenburg}, \ipi{Mitterdorf}), and \isi{stress} (\ipi{Sörth}). Although those places suggest idiosyncratic developments, it is significant that the peculiar sets of triggers comprise natural classes of sounds (e.g. front unrounded vowels, nonlow front tense vowels, high front unstressed vowels) and not arbitrary lists of segments.

The Stage 2 allophonic rule relating [x] and [ç] has undergone a change in many CG varieties whereby the palatal allophone [ç] developed into [ɕ]. Such alveolopalatalizing dialects were shown to require more than one stage. Evidence for those stages comes from modern CG dialects.

Variation in terms of space (regional dialects) directly reflects changes along the temporal dimension. That interpretation of time is applied in the present book to velar fronting. Hence, dialects with a more restricted set of triggers (e.g. only nonlow front vowels) preserve an earlier historical stage than dialects with the full set of triggers (all coronal sonorants), which represent a later stage. The same point holds for dialects with a small set of targets (e.g. /x/) vs. those with an expanded set (e.g. /x ɣ/).

The \isi{phonologization} of velar fronting occurred independently at more than one place (\isi{polygenesis}). The most conclusive evidence against a single point of origin (\isi{monogenesis}) comes from the many velar fronting islands. Whether or not \isi{monogenesis} of \isi{polygenesis} was correct for velar fronting in areas where velar fronting is the norm (i.e. most of Germany) is a question that cannot be known.

The conclusion was drawn is that the \ili{WGmc} language represented Stage 1; hence, velar fronting at that time was absent. The reason for this conclusion is that the linguistic evidence points to velar fronting in the earliest attested stages, namely \ili{OHG} and \ili{OSax}: Although velar fronting was not phonologized in a single place at a single point in time, it can be said that the change must have had at least one point of origin somewhere in an area corresponding to modern-day northwest Germany by the end of the ninth century. The reason for that time frame is that velar fronting predated the change from \isi{full vowel} velar fronting triggers like [i] to \isi{schwa} (\isi{Vowel Reduction}), which was complete by the onset of \ili{MHG}/\ili{MLG}. Velar fronting was phonologized first in postsonorant position and the extension of that process to word-initial position came later. Evidence is strong that velar fronting is much older in CG (\il{Ripuarian}Rpn/\il{Moselle Franconian}MFr) dialects of \ili{OHG} and is of a much more recent origin in LG (\il{Westphalian}Wph) varieties of \ili{OSax}.

When velar fronting was in the process of expanding through time and space to include a greater number of targets and triggers, velars ([x]) and palatals ([ç]) stood in a transparent (allophonic) relationship. Changes affecting the velar fronting target/trigger often interfered with the allophonic nature of velar fronting by producing \isi{opacity} (Stage 3). For example, rules creating new velar targets (e.g. /ʀ/ > /x/) could exhibit \isi{underapplication} if those new velars failed to undergo velar fronting. Likewise sound changes eliminating the front ([coronal]) trigger (e.g. /i/ > /ɑ/ or /r/ > /ʀ/) could lead to the historical \isi{overapplication} of velar fronting. Overapplication occurred if the original front sound (e.g. /r/) once served as a trigger for velar fronting, but the original palatal allophone remained palatal even after the front trigger has been removed, e.g. /rx/ [rç] > /rç/ [rç]. The palatal fricative [ç] in the diminutive suffix -\textit{chen} has a similar history: That [ç] was once preceded by a front vowel (cf. \ili{MHG} -\textit{ichen}), the loss of which led directly to the creation of the underlying palatal /ç/. That palatal is retained to the present day in those dialects with -\textit{chen} [çən].

The emergence of palatal quasi-phonemes or phonemic palatals like /ç/ exemplifies what is referred to in the traditional literature on historical linguistics as a \isi{phonemic split}, whereby the original trigger for a rule creating an allophone [A] from the phoneme /B/ causes the original allophone [A] to become the phoneme /A/.

Dialect-specific changes affecting the velar fronting target/trigger could interfere with the allophonic nature of velar fronting in other ways. In particular, the historically allophonic rule of velar fronting could undergo either \isi{rule loss} or \isi{rule inversion}. Rule loss is attested most clearly in the neighboring dialects of North \ipi{Luxembourg} (\ipi{Nordösling}), \ipi{East Belgium} (in and around \ipi{Burg-Reuland}), and West Central Germany (\ipi{Lützkampen} and \ipi{Dahnen}) with (alveolo)palatals (e.g. [ɕ]/[ç]) but no velars (e.g. [x]); hence, all historical velars in those places are realized as palatals. In that type of system the original rule of velar fronting was lost because earlier velars (e.g. /x/) were later restructured as phonemic palatals (e.g. /ç/). Rule inversion is attested in a particular place (\ipi{Neuendorf}) where earlier palatal allophones ([ç] from /x/ in the context of front vowels) were restructured as underlying palatals and a rule retracting those sounds to velar ([x] from /ç/ in the context of back vowels). Rule inversion was shown to be a direct consequence of a sound change eliminating one of the earlier triggers for velar fronting.

\section{{Significance} {of} {the} {findings}}\label{sec:18.4}

The conclusions described in \sectref{sec:18.1}--\sectref{sec:18.3} bear on several questions probed at length in the cross-linguistic research on phonology (diachronic and synchronic), language-specific research on German phonology, as well as typology.

The most significant contribution of the present work to linguistic scholarship is that it represents an in-depth investigation of the ways in which a single rule (velar fronting) can be phonologized in different dialects in different ways. It is my hope that the data in the Ortsgrammatiken and linguistic atlases which served as the basis for my treatment of velar fronting will inspire future linguists to conduct similar case studies on other types of changes.

The literature on historical German phonology has remained silent on the origin of the palatal allophone [ç] because earlier stages of German (and \il{Standard German}StG) spell [x] and [ç] the same way. The present book has demonstrated that it is possible to shed light on the origin of [ç] by putting aside orthography and by considering linguistic arguments.

This book sheds light on proposals made in the literature on the \isi{life cycle of a rule}, e.g. \citet{Hyman1976}, \citet{Dressler1976}, \citet{Kiparsky1995}, \citet{Bermúdez-Otero2007}, \citet{Hyman2013}, \citet{Kiparsky2015}, \citet{Bermúdez-Otero2015}, \citet{Ramsammy2015}, \citet{Sen2016}, and \citet{Turton2017}. Although the works cited here (as well as those of scholars not mentioned) endorse a variety of different models, they generally agree that a purely phonetic (\isi{gradient}) process becomes phonologized as an allophonic (categorical) rule whose effects later become opaque and then ultimately lost from the grammar entirely. That general trajectory is corroborated in the present cross-dialectal treatment of velar fronting, although there are various quirks in the German dialects investigated (referred to above) and commented on below.

The gradual increase in the number of targets/triggers when velar fronting was phonologized as an allophonic can be captured in the \isi{rule generalization} model. That theory derives support from sound changes within and outside of Gmc, e.g. \citet{DavisSalmons1999}, \citet{Bermúdez-Otero2015}. That the historical progression among triggers proceeds according to vowel height is corroborated in the present study, although some rare places suggest that the original high front vocalic trigger may have expanded along alternate dimensions (roundedness, \isi{tenseness}, orality, \isi{stress}). The tentative analysis of the way in which \isi{rule generalization} occurred in those unique communities can be corroborated in the future if parallel cases in independent languages become known.

The present treatment sheds light on how an originally transparent change can develop opaque outputs. Although the change from a transparent system to an opaque one has been observed by a number of linguists cited earlier, the types of opaque systems attested in the present book are much more fine-grained than the commonly occurring ones discussed in the literature. Consider the following examples:

One case of \isi{underapplication} \isi{opacity} comes in the form of \isi{neutral vowels}. Precedent for \isi{neutral vowels} outside of Gmc comes from \ili{Inuit} dialects spoken in Alaska described and analyzed by \citet{Dresher2009}. However, the models cited above for the \isi{life cycle of a rule} appear not to recognize that type of change. To the best of my knowledge Dresher’s work is not referred to in the literature on the \isi{life cycle of a rule}.

Overapplication as attested in German dialects was shown to be more subtle than what is typically assumed in the literature on phonemic splits in historical linguistics. The reason is that palatal allophones of velars can develop into either palatal quasi-phonemes or phonemic palatals. Palatal quasi-phonemes are not defined the same way as the vocalic quasi-phonemes proposed by \citet{Kiparsky2015}. A significant difference between the two approaches is that palatal quasi-phonemes in the present treatment always emerge as a direct consequence of the elimination of a (velar fronting) trigger and not before that trigger is lost (as per Kiparsky). What is more, only in my approach is it possible for the original velar to revert back to an underlying velar after the loss of the conditioning environment. That change was shown to be attested in several LG varieties, e.g. \ipi{Schieder-Schwalenberg}.

The case of \isi{rule loss} mentioned above demonstrates that the expulsion of velar fronting from the grammar is not necessarily preceded by a morphologized and/or lexicalized version of velar fronting, contrary to what is sometimes postulated for the \isi{life cycle of a rule} \citep{Hyman2013}.

The one case involving the change from a historical rule of velar fronting to a later rule of retraction (\isi{Wd-Initial Palatal Retraction} in \ipi{Neuendorf}) involves a true case of \isi{rule inversion} and therefore poses a challenge for the claim made in \citet{McCarthy1991} that true \isi{rule inversion} does not exist. The fact that the inverted rule of retraction is apparently unattested cross-linguistically lends yet additional support to the established claim that \isi{rule inversion} can create \is{crazy rule}crazy rules (e.g. \citealt{Vennemann1972}, \citealt{McCarthy1991}, \citealt{Blevins2004}, \citealt{Hall2009a}).

In terms of German phonology the present cross-dialectal study sheds light on how the distribution of [x] and [ç] in \il{Standard German}StG should best be analyzed. First, the two sounds are related by a rule fronting the velar to the palatal and not the reverse (contrary to many treatments proposed in the literature cited earlier, including my own). Second, the [ç] in the diminutive suffix -\textit{chen} ([-çən]) and in the post-rhotic (/ʀ/) context are underlying palatals (/ç/). That synchronic treatment (which is corroborated by the history of [ç] in those two contexts) therefore accounts for the presence of [-çən] even after stems ending in a back vowel and [ç] after the vocalized (back) rhotic ([ɐ]). The occurrence of [ç] after [ɐ]/[ʀ] is not in any way natural, contrary to the assertion made by \citet{Robinson2001}. Finally, the investigation of German dialects undertaken in the previous chapters should put to rest \citegen{Robinson2001} claim that velar fronting is a “low-level, \isi{phonetic rule}” and his implicit claim that the rule is essentially the same in all German dialects.

The present study contributes to the literature on \isi{Velar Palatalization} typology (e.g. \citealt{Neeld1973}, \citealt{Chen1973}, \citealt{Bhat1978}, and especially \citealt{Bateman2007,Bateman2011,Bateman2007}, \citealt{Kochetov2011}, and \citealt{KrämerUrek2016}). That the front vowel triggers for velar fronting vary along the height dimension derives support from that literature. This book also corroborates the finding in the cross-linguistic studies referred to above that front vowel triggers for velar fronting only rarely refer to \isi{nonheight features}. Another significant finding in the present study is that velar fronting can be triggered by front vowels and front (coronal) consonants. That finding does not appear to have support outside of German. The fricative targets for velar fronting in German dialects affect /x/ or /x ɣ/ but not /ɣ/ to the exclusion of /x/. That generalization is a corollary of similar claims made in the literature (e.g. \citealt{Guion1998}, \citealt{HallHamann2006} and \citealt{HallZygis2006}).

A typological oddity uncovered in the present study is the synchronic rule retracting an underlying palatal to velar in the back vowel context (\ipi{Neuendorf}), which represents one of the few known cases of “palatal to velar” assimilations. I am unaware of parallel examples outside of German.

\section{{Questions} {for} {future} {research} }\label{sec:18.5}

Any book of this magnitude will inevitably leave many questions open, and the present work is no exception. I describe below several general and specific topics touched on briefly in Chapters~\ref{sec:2}--\ref{sec:17} that could be pursued in future research.

A number of open questions pertain specifically to phonological models. Some of those issues are described in (\ref{ex:18:1}--\ref{ex:18:5}). A question concerning phonetics is posed in \REF{ex:18:6}.

\eanoraggedright%1
    \label{ex:18:1}
          Structure of palatals: A complex place representation for palatals was adop\-ted, according to which those segments are both [coronal] and [dorsal]. One could alternatively argue that palatals are simplex [coronal] or simplex [dorsal] segments (see \sectref{sec:2.2.2} for references). No attempt was made in this book to compare and contrast the complex representation with simplex one. Whether or not there are significant differences among the various approaches is a question that needs to be determined.
\ex%2
    \label{ex:18:2}
          Structure of alveolopalatals: It was argued (\chapref{sec:10}) that alveolopalatal sounds like /ɕ/ have a structure that is identical to the corresponding palatals (/ç/) and that the difference between the two types of articulation involves rules of \isi{phonetic implementation}. This approach is very different from the one proposed by authors who have looked at alveolopalatals in German (e.g. \citealt{Herrgen1986}, \citealt{Hall2014b}, \citealt{Féry2017}) as well as the equivalent sounds in other languages (e.g. \citealt{Rubach1984} for \ili{Polish}). It remains to be seen whether or not the \isi{phonetic implementation} approach endorsed in \chapref{sec:10} has more to offer than the ones cited above.
\ex%3
    \label{ex:18:3}
          Analysis of front vowels: A featural model was adopted in which front vowels are [coronal] and back vowels (including phonetically \isi{central vowels}) are [dorsal]. That treatment can be contrasted with approaches (e.g. \citealt{ChomskyHalle1968}, \citealt{Sagey1986}, \citealt{Kostakis2015}). No attempt has been made in this book to compare the present treatment with those alternative ones, but this endeavor could be undertaken in the future.
\ex%4
    \label{ex:18:4}
          Adjacency: In the default case, the velar fronting target is adjacent to its trigger, but several patterns involving nonadjacency are well-attested in German dialects (\sectref{sec:12.8.1}). Much research in phonology has concerned itself with the topic of \isi{adjacency} (e.g. \citealt{Odden1994}); hence, one could consider how any of the patterns involving the nonadjacency of velar fronting targets and triggers fits into this overall research program.
\ex%5
    \label{ex:18:5}
          Opacity: This is a topic that has been discussed at length in theoretical phonology. A number of models have been proposed to account for various types of \isi{opacity}, but those models have been shown to make different predictions. In particular, proponents of \isi{Optimality Theory} have put forth a number of specific proposals concerning opaque rule interaction (see \citealt{McCarthy2002} for discussion). Since the present study has dealt with a number of cases involving both synchronic and diachronic \isi{opacity} one could apply those formal models to the German data presented in this book.
\ex%6
    \label{ex:18:6}
          Non-velar fronting varieties: A number of places have been identified with velar sounds like [x] without a corresponding palatal. Little was said about that type of system, but it would be interesting to take a close look at the realization of those velars after all phonemic vowels and sonorant consonants in order to determine whether or not the degree of fronting in the coronal sonorant context in the phonetics matches the proposed steps for Stage 2 for the phonology. Is there a significant difference between non-velar fronting varieties, or do the same facts obtain in all of them?
\z 

Several open questions fit into the literature cited throughout this book on \isi{Velar Palatalization} typology. Three such issues are described here:

\eanoraggedright %7
    \label{ex:18:7}
           \isi{Palatal Retraction}: The \il{Eastphalian}Eph variety once spoken in \ipi{Neuendorf} was shown to have regular alternations between [x] and [ç] requiring a synchronic rule converting the former (/ç/) into the latter ([x]) in word-initial position before back vowels (\sectref{sec:8.5}). That rule (\isi{Wd-Initial Palatal Retraction}) was the product of \isi{rule inversion}. A question for further research concerns languages with similar rules changing a palatal into a velar in the neighborhood of back sounds. As noted earlier, no examples are presently known to me, nor are such examples discussed in the \isi{Velar Palatalization} literature. If such rules are attested were they the result of \isi{rule inversion} or did they arise in some other way?
\ex%8
    \label{ex:18:8}
           Vocalic triggers for velar fronting: The triggers for the various versions of velar fronting are defined primarily in terms of vowel height. A few varieties were discussed in which the triggers are \isi{nonheight features}, namely \isi{tenseness}, \isi{rounding}, and \isi{stress}. A recent publication (\citealt{CardosoHoneybone2022}) argues that vowel length is a factor in defining the set of triggers for velar fronting in \ili{Liverpool English}. What is the entire range of parameters defining the set of triggers for velar fronting (\isi{Velar Palatalization}) in the languages of the world?
\ex%9
    \label{ex:18:9}
          Adjacency: The dialects under investigation reveal various conditions on the type of segment that can intervene in nonadjacent velar fronting targets and triggers  (\sectref{sec:12.8.1}). Are other languages attested with similar patterns, or is German unique?
\z 

\begin{sloppypar}
The present work has left several questions unanswered concerning velar fronting in German dialects. The topic I find the most intriguing is stated here:
\end{sloppypar}

\eanoraggedright %10
    \label{ex:18:10}
           \isi{Alveolopalatalization}: This has been a change in progress primarily in CG from at least the late nineteenth century to the present day. It was proposed (\chapref{sec:10}) that there are two distinct stages, but a question for future work is whether or not this is the correct prediction for German varieties that are just starting to undergo \isi{alveolopalatalization}. Does the \isi{phonologization} of alveolopalatalization always involve those two stages, or are other stages attested?
\z 

Finally, the treatment of velar fronting begs several questions that in all likelihood have no answer. The three most intriguing questions in my view are the ones stated below. Recall that all three questions were mentioned briefly in previous chapters.

\eanoraggedright\sloppy %11
\label{ex:18:11}Actuation Problem: Why was velar fronting phonologized in certain places (e.g. Germany) but not in others (e.g. most of German-speaking Switzerland and West Tirol)?
\ex%12
\label{ex:18:12}Directionality: Why was velar fronting phonologized as a progressive spreading (and not as a regressive spreading) in all HG and LG varieties with that rule?
\ex%13
\label{ex:18:13}Uniqueness: Velar fronting in the many varieties of HG and LG is a textbook case of assimilation, which can easily be expressed with phonological units. If this is the case, then why is it that the typological literature referred to earlier has not discovered a parallel case outside of German with the unique properties associated with velar fronting (e.g. target includes at least one velar fricative, triggers include coronal consonants, left-to-right spreading)?
\z 

Since I cannot offer answers to (\ref{ex:18:11}--\ref{ex:18:13}) I simply leave them open for the inquisitive reader to ponder.
