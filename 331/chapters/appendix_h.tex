\chapter{Inventories of nonsyllabic sounds}\label{appendix:h}

The system of phonemic (contrastive) nonsyllabic sounds (consonants and glides) in the broad dialect groupings from Appendix~\ref{appendix:a} (UG, CG, LG) are discussed below. Those three groupings are indicated on \mapref{map:44}. Some discussion of consonants (and vowels) in more specific regional varieties of German can be found in \citet{Keller1961} and \citet{Russ1989}. Two important sources for LG are \citet{Sarauw1921} and \citet{Foerste1957}.

In \tabref{tab:appendix:h:1} I list the underlying (phonemic) nonsyllabic segments in typical UG dialects investigated in this book. Stops (but not affricates or fricatives) show a two-way laryngeal contrast (i.e. fortis /t/ vs. lenis /d/). The \isi{affricate} /kx/ is enclosed in parentheses because it is restricted to certain Almc varieties of SwG and to Tyr varieties of \il{South Bavarian}SBav. The one rhotic phoneme can be either coronal (/r/) or dorsal (/ʀ/), depending on dialect.\footnote{I omit from consideration those segments that only occur in nonnative words, namely the lenis postalveolar fricative /ʒ/ and the lenis postalveolar \isi{affricate} /dʒ/. The original sources cited in the present book often provide very detailed phonetic descriptions for the consonants and vowels in the respective dialects. Some of those descriptions refer to sounds not discussed in this appendix, but on closer inspection many of those segments can be analyzed as allophones of one of the sounds present in Tables~\ref{tab:appendix:h:1}--\ref{tab:appendix:h:3}. In an effort to maintain a clear focus I try not to burden the reader with unnecessary commentaries regarding sounds that might not be relevant for my analysis of velar fronting.}

\begin{table}%1
\caption{\label{tab:appendix:h:1}UG nonsyllabic segments}
\begin{tabularx}{.8\textwidth}{lXXXXXX}
\lsptoprule
stops & p b & t d &  &  & k g & \\
affricates & pf & ts & tʃ &  & (kx) & \\
fricatives & f & s & ʃ &  & x & h\\
nasals & m & n &  &  & ŋ & \\
liquids &  & l, r &  &  &  & \\
glides & w &  &  & j &  & \\
\lspbottomrule
\end{tabularx}
\end{table}

The palatal fricative [ç] is present in most UG dialects investigated in this book, although that sound is derived synchronically from /x/. Rare varieties of \il{Low Alemannic}LAlmc treat [ç] as a phoneme (/ç/); see \sectref{sec:14.3.2}. The initial sound in \il{Standard German}StG words like \textit{ja} ‘yes’ behaves phonologically in UG as a glide ([j]) and not as a fricative ([ʝ]). The glide /w/ (=/v/ in \il{Standard German}StG words like [vɑs] ‘what’ and [tsvɑi] ‘two’) is referred to in some dialect descriptions as a (lenis) bilabial fricative (=IPA [β]).

In \tabref{tab:appendix:h:2} and \tabref{tab:appendix:h:3} I present a list of the contrastive nonsyllabic segments in the CG/LG dialects under investigation. A two-way laryngeal contrast characterizes most of the stops (e.g. fortis /t/ vs. lenis /d/) and most of the fricatives (e.g. fortis  /s/ vs. lenis /z/). Affricates are absent from LG. In CG only /pf/ and /ts/ -- but never /kx/ -- are present. As in \tabref{tab:appendix:h:1}, the one rhotic consonant in Tables~\ref{tab:appendix:h:2} and \ref{tab:appendix:h:3} is either as coronal (/r/), or dorsal (/ʀ/) depending on the dialect. The postalveolar fricative /ʃ/ is absent in many conservative varieties of WLG which preserve \ili{WGmc} \textsuperscript{+}[s] as [s] (/s/) before a consonant (e.g. [s] for [ʃ] in \il{Standard German}StG \textit{Stadt} [ʃtɑt] ‘city’, \textit{schreiben} [ʃʀɑibən] ‘write-\textsc{inf}’) or after a rhotic (e.g. [s] for [ʃ] \il{Standard German}StG \textit{Kirsche} [kɪʀʃə] ‘cherry’). Other varieties of LG have phonemicized [ʃ] (/ʃ/) in those contexts. The \isi{sibilant} fricative [ʃ] (/ʃ/) in many varieties of CG is realized as alveolopalatal [ɕ]; see \chapref{sec:10}.

\begin{table}%2
\caption{\label{tab:appendix:h:2}CG nonsyllabic segments}
\begin{tabularx}{.8\textwidth}[t]{lXXXXXX@{}}
\lsptoprule
stops & p b & t d &  &  & k (g) & \\
affricates & pf & ts & tʃ &  &  & \\
fricatives & f v & s z & ʃ & ʝ & x (ɣ) & h\\
nasals & m & n &  &  & ŋ & \\
liquids &  & l, r &  &  &  & \\
\lspbottomrule
\end{tabularx}
\end{table}

\begin{table}%3
\caption{\label{tab:appendix:h:3}LG nonsyllabic segments}
\begin{tabularx}{.8\textwidth}{lXXXXXX}
\lsptoprule
stops & p b & t d &  &  & k (g) & \\
fricatives & f v & s z & (ʃ) & ʝ & x (ɣ) & h\\
nasals & m & n &  &  & ŋ & \\
liquids &  & l, r &  &  &  & \\
\lspbottomrule
\end{tabularx}
\end{table}

The two sounds [g] and [ɣ] (as well as [ʝ] and [x ç]) in Tables~\ref{tab:appendix:h:2} and~\ref{tab:appendix:h:3} are related diachronically and synchronically. In many dialects -- including \il{Standard German}StG -- there are regular alternations between [g] and [x ç], although other dialects show alternations between [g] and [ɣ ʝ x ç]. For example, in one commonly attested system, [g] surfaces as [g] word-initially and as [ɣ] or [ʝ] in a word-internal onset depending on whether or not a back vowel or a front vowel precedes. In that type of system, the dorsal fricatives derived from /g/ surface as [x] or [ç] in coda position after a back vowel and front vowel respectively. Thus, there is synchronic rule of g-Spirantization, which itself feeds velar fronting.

A number of writers have observed that the sound transcribed in Tables~\ref{tab:appendix:h:2} and \ref{tab:appendix:h:3} as [v] is realized as an obstruent ([v]) in syllable-initial position (e.g. [vɑs] ‘what’) and as a glide-like (approximant) sound in the context after a word-initial consonant; the symbol usually used for that realization is [ʋ]. Thus, the [v] in a \il{Standard German}StG word like [tsvɑi] ‘two’ is realized in that type of dialect as [tsʋɑi]; see \citet[235--242]{Wiese1996a}. An extensive discussion of similar data from \il{Westphalian}Wph can be found in \citet{Hall2014c}.

