\chapter{Theoretical background}\label{sec:2}


\section{Introduction}\label{sec:2.1}

The goal of this chapter is to introduce the formal models of phonology and phonological change I adopt in this book and to discuss the findings in the typological research as they relate to velar fronting. \sectref{sec:2.2} makes explicit several assumptions involving levels of grammar, features for consonants and vowels, and \isi{opacity}.  \sectref{sec:2.3} gives a synopsis of some of the findings in the typological work on \isi{Velar Palatalization}, and \sectref{sec:2.4} presents models of historical phonology and lays out some underlying assumptions concerning historical phonology. The diachronic model for velar fronting defended in the present book is described in detail in \sectref{sec:2.5}. 

\section{{Phonological} {models}}\label{sec:2.2}

\begin{sloppypar}
My treatment of velar fronting in German dialects presupposes a model of grammar in which phonetics and phonology are two separate components. That model is described in \sectref{sec:2.2.1}. Since velar fronting is typically assimilatory, it is essential to adopt a theoretical framework that is able to express the correct triggers and targets for that process. To achieve that end I adopt a model of features described in \sectref{sec:2.2.2} (for consonants) and \sectref{sec:2.2.3} (for vowels). \sectref{sec:2.2.4} defines the different types of rule interaction discussed in the ensuing chapters (e.g. transparent vs. opaque; \isi{underapplication} vs. \isi{overapplication}). That discussion provides the necessary background in order to understand the way in which velar fronting interacts with independent processes that create and eliminate potential targets and triggers.
\end{sloppypar}

\subsection{Levels of grammar}\label{sec:2.2.1}

I follow earlier authors in the generative tradition who posit an architecture of grammar consisting of more than one representational level (e.g. \citealt{ChomskyHalle1968} and subsequent work by many authors). I adopt the model in \tabref{tab:fromfig:representationallevels}, which is similar to the ones presupposed by other writers (e.g. \citealt{Keating1990}, \citealt{Cohn1993}, \citealt{Keating1996}, \citealt{HaleReiss2015}, \citealt{Bermúdez-Otero2015}).

As indicated in \tabref{tab:fromfig:representationallevels}, the input to the phonology is the underlying representation (enclosed in diagonal slashes: / ... /). By definition the underlying representation contains the stored forms of morphemes or sequences of morphemes in morphologically-complex words. The segments present in underlying representation (expressed throughout this book with IPA symbols) are mere abbreviations for bundles of distinctive features (\sectref{sec:2.2.2} and \sectref{sec:2.2.3}).

\begin{table}
\caption{Representational levels\label{tab:fromfig:representationallevels}}
\begin{tabular}{cl}
\lsptoprule
/ ... / & Underlying representation\\
 & \multicolumn{1}{c}{\itshape Phonology} \\\relax
[ ... ] & Phonetic representation\\
 & \multicolumn{1}{c}{\itshape Phonetics}\\
... & Speech\\
\lspbottomrule
\end{tabular}
\end{table}

Phonology (=phonological component) is the mapping of underlying representations to phonetic representations (enclosed in single square brackets: [...]). Representations in the phonology consist of the same set of units (=features) necessary to express the underlying representations. Complete phonological representations also require prosodic constituents such as syllables and feet as well as association lines connecting those units with one another and with the features.

Words in the phonetic representation consist of the same phonological units described above, e.g. features, syllables, feet. There are no units required for the phonetic representation that are not present in the underlying representation or in the phonology. The representational alphabet for the phonetic representation is therefore the same for the underlying representation and for the phonology.

The change from underlying representations into phonetic representations via the phonology takes the form of phonological rules, although the mapping described here is also consistent with an Optimality Theoretic approach with constraints instead of rules (\citealt{PrinceSmolensky2004}).

The Phonetics component in \tabref{tab:fromfig:representationallevels} is the locus of \is{phonetic rule}\textsc{phonetic rules}, which are characterized by gradient outputs. By contrast, phonological rules are categorical. According to \citet[452]{Keating1990}, “Phonetic rules can thus, for example, assign a segment only a slight amount of some property, or an amount that changes over time during the segmentˮ.

I assume two types of \is{phonetic rule}phonetic rules, namely \textsc{coarticulation} and \textsc{phonetic} \textsc{implementation}. The two terms are used in the literature in slightly different ways. For my purposes I define them as follows:

Coarticulation is the overlapping of adjacent articulations. For example, \citet{Cohn1993} demonstrates that vowels in \ili{English} are gradiently nasalized before nasal consonants, e.g. the /i/ in \textit{dean} (/din/). In that example nasal airflow on the vowel gradually increases throughout the duration of that vowel, thereby indicating that the velum lowers not at once, but instead over the course of the vowel. That type of \isi{gradient} coarticulation can be contrasted with phonological Nasalization in other languages, in which the vowel is nasal over its entire duration (categorical).

Phonetic implementation is responsible for the interpretation of low-level distinctions that play no role in the phonology. Consider the following examples involving manner and place of articulation of consonants: The one rhotic phoneme -- present in all German dialects -- surfaces initially in items like \textit{rot} ‘red’ in a number of different ways, e.g. approximant, trill. The inter- and intraspeaker variation pertaining to those manner categories is determined not in the phonology but instead in the phonetics by rules of \isi{phonetic implementation}. Hence, the phonological representation of the rhotic consists of a set of distinctive features described in \sectref{sec:2.2.2} which make no reference to categories like “approximantˮ or “trillˮ. Phonetic implementation is also necessary to express the exact place of articulation of sounds like /t/ and /d/. As demonstrated below in \sectref{sec:2.2.2} the phonology specifies that /t/ and /d/ bear the distinctive feature referring to an articulation with the front part of the tongue ([coronal]) to distinguish them from labials (/p b/) and velars (/k g/). However, the realization of /t d/ as dental or alveolar is determined by rules of \isi{phonetic implementation}.

The level of grammar referred to in \tabref{tab:fromfig:representationallevels} as “Speechˮ is intended to express the actual phonetic realization of the word in question. Seen in this light, Speech requires a conversion of the nature of objects involved because the underlying representation, the phonology and the phonetic representation utilize phonological units such as features, association lines and prosodic constituents, but the phonetic realization of those abstract representation involves the actual organs of Speech. Thus, a phonetic representation is converted into both an articulatory act involving a coordination of muscles in the jaw, throat and lungs, as well as an acoustic output consisting of sound waves.

The example discussed in \citet{HaleReiss2015} illustrating \tabref{tab:fromfig:representationallevels} is the \ili{English} word \textit{keep}, which has the underlying representation /kip/, where those three segments represent three distinct feature bundles. The /k/ in /kip/ undergoes a phonological rule of Aspiration, which produces the phonetic representation [kʰip]. The phonetic representation [kʰip] is expressed in terms of phonological units, but its actual articulatory and acoustic realization (=Speech) requires reference to factors such as the exact place of contact between the tongue body and the roof of the mouth and the length in terms of milliseconds of the release of the closure of the velar stop until the onset of voicing of the following vowel.

The phonological component in \tabref{tab:fromfig:representationallevels} is often argued to be subdivided into domains of various sizes to which rules are assigned. For example, in the model of \isi{Lexical Phonology and Morphology} (e.g. \citealt{Kiparsky1982b}, \citealt{KaisseShaw1985}, \citealt{Mohanan1986}, \citealt{HargusKaisse1993}) phonological rules can apply across words (postlexical rules) or within words (lexical rules). In \isi{Stratal Optimality Theory} (\citealt{Kiparsky2000}, \citealt{Rubach2000}, \citealt{KaisseMcMahon2011}, \citealt{Bermúdez-Otero2015}) a distinction is drawn between phrase level, word level, and stem level rules.

An example of a phrase level rule is \is{Flapping (American English)}Flapping in \ili{American English}, which is responsible for the realization of /t d/ as [ɾ] both word-internally (e.g. \textit{ci[ɾ]y}) and across words (e.g. \textit{si[ɾ] in a chair}). Stem level rules only apply within words. They show alternations triggered by certain suffixes (stem level suffixes) but not by others (word level suffixes). For example, in \ili{English} the rule of \is{Trisyllabic Laxing (English)}Trisyllabic Laxing creates a lax vowel in the antepenultimate syllable in words like \textit{n}\textbf{\textit{a}}\textit{tional} (vs. \textit{n}\textbf{\textit{a}}\textit{tion}) but not in words like \textit{n}\textbf{\textit{a}}\textit{tionhood}. \is{Trisyllabic Laxing (English)}Trisyllabic Laxing applies in \textit{national} but not in \textit{nationhood} because -\textit{al} is a stem level affix but -\textit{hood} is a word level affix. Word level rules are similar to stem level rules in the sense that they never apply across words. Within words they are triggered only by word level affixes. For example, the \ili{English} rule of \is{n-Deletion (English)}n-Deletion which eliminates /n/ after another nasal applies in \textit{damning} (from /dæmn-ɪŋ/) because -\textit{ing} is a word level affix, but not in \textit{damnation} ([dæmneiʃǝn]), which contains the stem level affix -\textit{ation}.

The distinction between the three types of domains described above does not play a role in my treatment of velar fronting. See \sectref{sec:5.5.1} and especially \sectref{sec:12.8.2} for discussion.

All of the authors cited in the present section adopt the basic premise that there is a fundamental difference between phonology and phonetics. A few of the properties characterizing those two components \citep{Bermúdez-Otero2015} are listed in \tabref{tab:fromex:2:1}.

\begin{table}
\caption{Phonetics vs. phonology\label{tab:fromex:2:1}}
\begin{tabularx}{\textwidth}{lQ}
\lsptoprule
Component & Properties\\\midrule
Phonology & discrete phonological objects (e.g. segmental features, prosodic nodes, association lines)\\
Phonetics & continuous phonetic dimensions (e.g. formant frequencies, gesture amplitudes and durations)\\
\lspbottomrule
\end{tabularx}
\end{table}

\subsection{Featural representations for consonants and glides}\label{sec:2.2.2}
\begin{sloppypar}
The most important consonants investigated below are velars because those sounds -- or a subset thereof -- serve as targets for velar fronting. Velar fricatives (/x ɣ/), velar stops (/k g/) and the velar nasal (/ŋ/) all bear the place feature [dorsal], as in (\ref{ex:2:2a}--\ref{ex:2:2c}). In contrast to velars, palatals are phonologically complex segments in the sense that they are [coronal] and [dorsal]; see \REF{ex:2:2d}.\footnote{\label{fn:2:1}The place features depicted in \REF{ex:2:2} ([coronal], [dorsal]) are present on velars and palatals in all dialects discussed below with the exception of two SwG varieties (\chapref{sec:6}). In those two dialects, [dorsal] in the representation of velars and palatals is argued to be replaced with [peripheral]; see \citet{Rice1994} for the latter feature. \REF{ex:2:2d} can be contrasted with the proposal based on cross-linguistic work that palatal fricatives are simplex coronals \citep{Hume1994} or simplex ([−back]) dorsal sounds \citep{Hall1997}. Since a comparison of those approaches with \REF{ex:2:2d} is given elsewhere (e.g. \citealt{Hall2014b}), I do not discuss this debate here.} A representation like the one in \REF{ex:2:2d} is defended by \citet{Robinson2001} and \citet{Hall2014b}. The branching structure depicted here holds for all palatal segments, regardless of manner, i.e. palatal stops (/c ɟ/), palatal fricatives (/ç ʝ/), the palatal nasal (/ɲ/), and the palatal glide (/j/). Manners of articulation (e.g. stop vs. fricative vs. nasal vs. liquid) are expressed with the major class features [±cons(onantal)], [±son(orant)] and the manner features [±cont(inuant)], [±nas(al)].\footnote{Segments playing a minimal role in the following chapters are [labial] sounds like /p b f v/ and [coronal] sounds such as /t d s z ʃ ʒ/. In many approaches to features (e.g. \citealt{Sagey1986}, \citealt{ClementsHume1995}, \citealt{Hall1997}) alveolar and postalveolar [coronal] sounds (/s z/ vs. /ʃ ʒ/) are distinguished with the feature [±anterior], while \citet{Hall1992} and \citet{Wiese1996a} argue on the basis of \il{Standard German}StG that it is the feature [±high]. In the majority of case studies discussed in Chapters~\ref{sec:2}--\ref{sec:9} the choice between [±anterior] and [±high] is not significant. The structure of postalveolar sounds like /ʃ/ is shown to be relevant to velar fronting in the context of \isi{alveolopalatalization} discussed in \chapref{sec:10}.} The place features are privative and all other features binary, although that assumption is not justified here because my analysis does not crucially depend on that approach.
\end{sloppypar}

\ea\label{ex:2:2}
\begin{xlist}
\begin{multicols}{3}\raggedcolumns
\ex \label{ex:2:2a}\attop{\begin{forest} [\avm{[+cons\\−son\\+cont]} [\avm{[dorsal]}]] \end{forest}}
\columnbreak
\ex \label{ex:2:2b}\attop{\begin{forest} [\avm{[+cons\\−son\\−cont]}  [\avm{[dorsal]}] ] \end{forest}}
\columnbreak
\ex \label{ex:2:2c}\attop{\begin{forest} [\avm{[+cons\\+son\\+nas]} [\avm{[dorsal]}] ] \end{forest}}
\columnbreak
\ex \label{ex:2:2d}\begin{forest}
     [\avm{\dots}
       [\avm{\scshape coronal}]
       [\avm{\scshape dorsal}]
       ]
  \end{forest}
\end{multicols}
\end{xlist}
\z


I henceforth adopt an abbreviatory convention whereby all features other than [labial], [coronal], and [dorsal] are listed in the topmost matrix (root node); hence, separate nodes relating to manner and/or laryngeal dimensions are not necessary. I similarly omit the place node for simplicity.

\begin{sloppypar}
I follow earlier research which draws a distinction between underlying (phonemic) glides and derived glides (e.g. \citealt{Levi2004}, \citealt{Hall2017} and literature cited therein). Underlying glides are transcribed henceforth with distinct phonetic symbols, i.e. /j/ is the underlying palatal glide and /w/ is the underlying labial glide. Of those two sounds the former is more important than the latter because it has a close synchronic and diachronic relationship with the sounds produced by velar fronting (palatal fricatives). The underlying palatal glide /j/ has the place structure depicted in \REF{ex:2:2d}. It is distinct from the homorganic vowel /i/, palatal fricatives, and the palatal nasal with the major class/manner features referred to above; thus, /j/ (and /w/) are [+consonantal, +sonorant, --nasal]. Derived glides are the nonsyllabic components of diphthongs, which are often transcribed with the subscript arch in a narrow phonetic transcription, e.g. [au̯] and [ai̯]. Those glides are (synchronically) derived from the corresponding vowels in the sense that their nonsyllabicity is a function of sonority \citep{Hall2017}. I refrain from making use of the subscript arch in diphthongs and simply transcribe diphthongs as a sequence of two distinct vowel symbols, e.g. [au] (/au/), [ai] (/ai/). Both of the components in diphthongs are [−consonantal]; see \sectref{sec:2.2.3}.
\end{sloppypar}

In many languages there is a contrast between a \isi{sibilant} fricative and a nonsibilant fricative (e.g. /s/ vs. /θ/ in \ili{English}), which is traditionally captured with the feature [±strident]. That type of contrast is absent in all of the dialects investigated in the present book; hence, [±strident] is not a distinctive feature and therefore plays no role in the phonology. The implication is that segments such as /s z ʃ ʒ/ have phonological representations consisting solely of the features described above and that the realization of those sounds as sibilants at the level of Speech is expressed with dialect specific rules of \isi{phonetic implementation} (\sectref{sec:2.2.1}). The relevance of that type of rule of \isi{phonetic implementation} for the topic of velar fronting is discussed in \chapref{sec:10}.

Most of the dialects discussed below have a laryngeal contrast among stops and fricatives (e.g. fortis /s/ vs. lenis /z/). In a subset of those varieties, that laryngeal contrast also holds for velar sounds, i.e. fortis /x/ vs. lenis /ɣ/. A distinctive laryngeal feature is required for that type of system, which I express with the descriptive cover feature [±fortis], e.g. /x/ is [+fortis] /ɣ/ is [−fortis]. It is assumed here that dialects in which /x/ is the only velar fricative (e.g. \il{Standard German}StG) do not mark that fricative with the feature [+fortis] because [±fortis] is not distinctive for dorsal fricatives.\footnote{The nature of the distinctive laryngeal feature in German phonology and its relationship to the phonetics has been the object of debate for many years. According to some approaches, [±fortis] is interpreted as [±voice], while others advocate an aspiration feature ([±spread glottis]) or a feature of length (singleton vs. geminate). See \citet{IversonSalmons1995}, \citet{Wiese1996a}, \citet{JessenRingen2002}, and \citet{BeckmanRingen2009} for various proposals. The present treatment does not require a commitment to any one of those approaches.}

Velar and palatal affricates have the same place structure depicted above in \REF{ex:2:2}. Those affricates are important because several SwG varieties have a distribution of velar [kx] and palatal [kç] that parallels that of [x] and [ç]. I adopt a representation of affricates in which those sounds bear [−continuant] and [+continuant], as in \REF{ex:2:3a}; see \citet{Sagey1986} and \citet{Lombardi1990} and more recently \citet{Hall2012}.\footnote{\citet{Sagey1986} proposes a contour segment representation for affricates, while \citet{Lombardi1990} endorses the complex segment representation. My analysis is compatible with both models.} Affricates are thereby distinct from stops and fricatives (\ref{ex:2:3}b,c). Note that the structures in \REF{ex:2:3} depart from the abbreviatory convention in \REF{ex:2:2} because [±continuant] is placed on a tier separate from the root node.


\ea\label{ex:2:3}
\begin{xlist}%
\begin{multicols}{3}\raggedcolumns%
\ex\label{ex:2:3a}\attop{\begin{forest} [\avm{[+cons\\−son]} [\avm{[−cont]}] [\avm{[+cont]}]] \end{forest}}
\columnbreak
\ex \label{ex:2:3b}\attop{\begin{forest} [\avm{[+cons\\−son]}  [\avm{[−cont]}] ] \end{forest}}
\columnbreak
\ex \label{ex:2:3c}\attop{\begin{forest} [\avm{[+cons\\−son]} [\avm{[+cont]}] ] \end{forest}}
\end{multicols}
\end{xlist}
\z

The treatment of affricates adopted here can be contrasted with the one proposed by linguists such as \citet{LaCharité1993}, \citet{Rubach1994}, \citet{Clements1999}, \citet{Kim2001} and \citet{Kehrein2002} (on the basis of \citealt{JakobsonEtAl1951}), which sees affricates as strident stops without a [+continuant] component, e.g. /t/ is [−strident] and /ts/ is [+strident]. The strident stop representation is rejected here because it can capture neither the nonstrident \isi{affricate} /kx/ nor the natural class of /x/ and /kx/.\largerpage

The place features for the coronal nasal (/n/) and coronal liquids (/l r/) are important because those sounds can function as triggers for velar fronting. All dialects investigated have the three contrastive nasals /m n ŋ/, as well as two liquids, namely /l/ and the consonantal rhotic, which can be either coronal (/r/) or dorsal (/ʀ/) depending on dialect. Representations for /n l r/ are posited in (\ref{ex:2:4a}--\ref{ex:2:4c}) below. The dorsal (phonetically uvular) rhotic (/ʀ/) is depicted in \REF{ex:2:4d}.

\protectedex{\ea\label{ex:2:4}
\begin{xlist}
\begin{multicols}{4}\raggedcolumns
 \ex \label{ex:2:4a}\attop{\begin{forest} [\avm{[+cons\\+son\\+nas]} [\avm{[coronal]}, l+=9.1mm] ] \end{forest}}
 \columnbreak
 \ex \label{ex:2:4b}\attop{\begin{forest} [\avm{[+cons\\+son\\−nas\\−cont]} [\avm{[coronal]}] ] \end{forest}}
 \columnbreak
 \ex \label{ex:2:4c}\attop{\begin{forest} [\avm{[+cons\\+son\\−nas\\+cont]} [\avm{[coronal]}] ] \end{forest}}
 \columnbreak
 \ex \label{ex:2:4d}\attop{\begin{forest} [\avm{[+cons\\+son\\−nas\\+cont]} [\avm{[dorsal]}] ] \end{forest}}
\end{multicols}
\end{xlist}
\z}

It is shown below that the place features for /r/ and /ʀ/ are phonologically relevant and that rhotics cannot be analyzed as placeless. As noted in \sectref{sec:2.2.1} the finer-grained manner distinctions among rhotics (e.g. approximant, trill) are irrelevant for the phonology.

\subsection{Featural representations for vowels}\label{sec:2.2.3}

All German dialects discussed reflect the predominant cross-linguistic pattern in the sense that they contrast front vowels and back vowels. I adopt the proposal defended in a number of works, according to which front vowels are phonologically [coronal], as in \REF{ex:2:5}; see \citet{Clements1976}, \citet{LahiriEvers1991}, \citet{Hume1994}, and \citet{ClementsHume1995}, as well as \citet{Robinson2001}, \citet{Glover2014}, and \citet{Hall2014b}, who have extended that proposal to German (including regional dialects). A distinction between C-place and V-place [coronal] (e.g. \citealt{ClementsHume1995}) is not crucial and is therefore ignored.\largerpage

\ea%5
    \label{ex:2:5}\begin{forest} [\avm{[−cons\\+son]} [\avm{[coronal]}] ] \end{forest}
\z

The advantage of analyzing front vowels as [coronal] is that those sounds can be grouped together with /n l r/ as the natural class of coronal sonorants, which form the set of triggers for velar fronting in many dialects. The left-to-right spreading is depicted in the template for velar fronting in \REF{ex:2:6a}. The features of the leftmost segment (trigger) and of the rightmost segment (target) are omitted here because they differ from dialect to dialect. The word-initial analogue of \REF{ex:2:6a} is presented in \REF{ex:2:6b}.{\interfootnotelinepenalty=10000\footnote{I refer henceforth to the rule in question in the upper case if it is a specific instantiation of either \REF{ex:2:6a} or \REF{ex:2:6b}. To distinguish the various versions in individual dialects, I also include numerical suffixes, e.g. \isi{Velar Fronting-1}, Word-Initial \isi{Velar Fronting-1} etc. By contrast, the rule is given in the lower case (velar fronting) in reference to fronting in general.}}\pagebreak

\ea%6
    \label{ex:2:6}
    \begin{multicols}{2}\raggedcolumns
\ea Velar Fronting:\label{ex:2:6a}
\begin{forest}
[,phantom
[{[ … ]} [{[\textsc{coronal}]},name=coronal]]
[{[ … ]},name=parent [{[\textsc{dorsal}]}]]
]
\draw [dashed] (coronal.north) -- (parent.south);
\end{forest}
\columnbreak
\ex Word-Initial Velar Fronting:\label{ex:2:6b}
\begin{forest}
[,phantom
[{[ … ]},name=parent [{[\textsc{dorsal}]}]]
[{[ … ]} [{[\textsc{coronal}]},name=coronal]]
]
\node[left=.5cm of parent] {\textsubscript{wd}[};
\draw[dashed] (parent.south) -- (coronal.north);
\end{forest}
\z 
\end{multicols}
\z 

Given the spreadings depicted in \REF{ex:2:6}, the output segment is the complex cor\-ono-dor\-sal segment for palatals (=\ref{ex:2:2d}). In phonological representations there is no temporal ordering involving place features. Thus, the features [coronal] and [dorsal] can appear in that linear sequence, as in (\ref{ex:2:6a}), or in the reverse, in (\ref{ex:2:6b}), but both structures represent palatals, as in (\ref{ex:2:2d}).

The features for back vowels are not crucial in my treatment because they do not serve as triggers for velar fronting. There is more than one way to analyze such segments (e.g. /u o ɑ/); I posit that they are [dorsal], although it is argued in \chapref{sec:6} that two varieties of SwG require [peripheral] instead; recall \fnref{fn:2:1}.

The German dialects investigated provide no evidence for drawing a distinction between back vowels and \isi{central vowels} (although see \sectref{sec:15.5} for the one counterexample). From the phonological perspective, phonetically \isi{central vowels} and phonetically back vowels are [dorsal], e.g. \citet{ChomskyHalle1968} and \citet{Rice2002}. The treatment of \isi{central vowels} as [dorsal] works well in languages like \il{Standard German}StG (and in the German dialects discussed below) because phonetically \isi{central vowels} and phonetically back vowels always differ in terms of at least one other feature, one of which can be interpreted as distinctive. For example, many dialects possess a low back unrounded vowel (/ɑ/) and a mid rounded back vowel /ɔ/. Those two [dorsal] can be distinguished from one another if: (a) /ɔ/ is  [−low] and /ɑ/ is [+low], or (b) /ɔ/ is [labial] and /ɑ/ lacks that feature. In dialects where /ɑ/ contrasts with a low front vowel /æ/, /æ/ is [coronal] and /ɑ/ is [dorsal]. In rare dialects there are two low nonfront unrounded vowels (/ɑ/ and /a/), but the feature distinguishing those segments is [±tense]; see \sectref{sec:11.5}. The present survey of German dialects therefore only requires [coronal] and [dorsal] but not an additional feature like [central] for capturing the frontness/backness dimension among vowels.\footnote{A number of linguists have pointed out that the approach to \isi{central vowels} described above cannot be universally true because some languages contrast a central and back vowel that agree in \isi{lip rounding} (\citealt{Parker2000}, \citealt{Rice2002}). For example, \ili{Norwegian} \citep{Kristoffersen2000} contrasts three high rounded vowels, i.e. front rounded /y/, central rounded /ʉ/, and back rounded /u/. Kristoffersen consequently argues that back and \isi{central vowels} are both [dorsal] and that they are distinguished with the feature [±back], i.e. /ʉ/ is [dorsal, –back] and /u/ is [dorsal, +back]. None of the German dialects investigated below has such contrasts.}

One phonetically \isi{central vowel} present in almost all German dialects discussed below is \isi{schwa} (/ə/). A possible featural analysis for that phonetically mid \isi{central vowel} is one in which it is a simplex [dorsal] sound, which is distinct from the mid back vowel /ɔ/ (=[dorsal, labial]). I alternatively adopt the proposal that \isi{schwa} consists of a placeless root node, as in \REF{ex:2:7}. All other vowels -- referred to below as \textsc{full} \textsc{vowels} − possess place features.

\ea%7
    \label{ex:2:7}
        /\text{ǝ}/: \avm{[−cons\\+son]}
\z

A representation for \isi{schwa} like the one in \REF{ex:2:7} is defended by \citet[133--134]{vanOostendorp2000} for \ili{Dutch}. According to that author’s first property of \isi{schwa} (p. 133), that vowel bears no phonetic features. From the point of view of phonology, van Oostendorp consequently argues that \ili{Dutch} \isi{schwa} is not marked for any of his vocalic (phonological) features ([high], [low], [lax], [coronal], [labial], [dorsal] in his featural system).

What is significant about the structure in \REF{ex:2:7} is that \isi{schwa} has no place features, in contrast to all other vowels. Representations similar to the one in \REF{ex:2:7} have been proposed in the literature on \isi{schwa} in \il{Standard German}StG (e.g. \citealt{Hall1992}: 208--212, \citealt{Wiese1996a}: 153, 159, \citealt{Trommer2021}). In contrast to the approach taken by Hall and Wiese, my treatment requires no default rule which supplies the representation in \REF{ex:2:7} with features. Thus, the structure in \REF{ex:2:7} depicts underlying \isi{schwa} (e.g. in \il{Standard German}StG /gənɑu/ for [gənɑu] ‘exactly’ or /ʃʀɑŋkə/ for [ʃʀɑŋkə] ‘barrier’), which remains placeless throughout the entire phonology. Many instances of \isi{schwa} in \il{Standard German}StG have been argued to be epenthetic, e.g. [hɪməl] ‘sky’ from /hɪml/; see \citet{Wiese1988}, \citet{Hall1992}, \citet{Noske1993}, \citet{Wiese1996a}. Epenthetic \isi{schwa} in German dialects (\sectref{sec:5.4}) -- like underlying \isi{schwa} -- has the placeless structure in \REF{ex:2:7}.

Since the system of phonemic vowels can differ considerably from dialect to dialect it is not feasible to list a single set of matrices here with distinctive features for individual vowels. Instead, the reader is referred to the beginning of each case study in which I list the phonemic vowels for the German variety under discussion. In the remainder of this section I consider the features expressing height and \isi{tenseness} necessary to capture certain commonly occurring vowel contrasts present in German dialects.

Vowel height is captured phonologically with [±high] and [±low], e.g. /i u/ are [+high] and /e o ɑ/ are [−high]. /ɑ/ can be distinguished from /o/ by the feature [±low], or by the feature [labial]/[dorsal], e.g. /ɑ/ is [+low] and /o/ is [−low] or /ɑ/ is [dorsal] and [o] is [labial]. \isi{Tenseness} ([±tense]) distinguishes vowels at any given height, e.g. /i ɪ/ are [+high], /i/ is [+tense] and /ɪ/ is     [−tense].

 Some dialects are attested with phonetically low front vowels (/æ/) in addition to mid front vowels (e.g. /e ɛ/). In that type of system, /æ/ is [+low], while /e ɛ/ are [−low] and then further distinguished with [±tense], i.e. /e/ is [+tense] and /ɛ/ is [−tense]. The majority of dialects investigated below have mid front vowels (e.g. /e ɛ/) but no phonetically low front vowel /æ/. In that type of inventory the default assumption is that /e ɛ/ are distinguished from high vowels (e.g. /i/) with the feature [±high] but that they do not bear any specification of [±low], which is not a distinctive feature. As indicated in \REF{ex:2:8a}, /i e ɛ/ are assigned a plus or minus value for [±high], and then the nonhigh vowels are distinguished by [±tense]. In some dialects with /e ɛ/ and no /æ/, /ɛ/ behaves phonologically like a low vowel (/æ/). For precedent outside of German dialects see \citet[78]{vanOostendorp2000} on \ili{Dutch} /ɛ/ and \citet[182]{Dresher2009} on /ɛ œ/ in \ili{Xibe} (Manchu, Northwest China). In the type of German dialect described here, /ɛ/ is [+low], and all other front vowels are [−low]. Nonlow vowels are further distinguished by [±low], as in \REF{ex:2:8b}.

 
\ea%8
\label{ex:2:8}
\begin{multicols}{2}\raggedcolumns
\ea\label{ex:2:8a}
\begin{tabular}[t]{rccc}
\lsptoprule
      & i & e & ɛ  \\\midrule
\relax [high] & + & − & −  \\
\relax [tense] &  & + & −  \\
\lspbottomrule
\end{tabular}\columnbreak
\ex\label{ex:2:8b}\begin{tabular}[t]{rccc}
\lsptoprule
   & i & e & ɛ\\\midrule
\relax [low] & − & − & + \\
\relax [high] & + & − &  \\
\lspbottomrule
\end{tabular}
\z
\end{multicols}
\z

Following \citet{Dresher2009} I assume that distinctive features are assigned to the phonemic inventory in a step-wise fashion. For example, given the vowels in \REF{ex:2:8a} only the [−high] vowels are assigned a value of [±tense] because [±tense] is not distinctive for [+high] vowels. [±high] is likewise assigned to the two [−low] vowels in \REF{ex:2:8b} but not to the one [+low] vowel. Note that the analysis of German vowels described here eschews default rules filling in the blanks in \REF{ex:2:8a} and \REF{ex:2:8b} with plus or minus values.

Front rounded vowels and front unrounded vowels contrast in many German dialects, e.g. \il{Standard German}StG [tiːɐ] ‘animal’ vs. [tyːɐ] ‘door’. The feature that distinguishes those two types of vowels does not play a role in most of the case studies discussed in this book. However, in certain cases a feature expressing (un)round\-ed\-ness is crucial. For those few cases I adopt one of two approaches. The first one expresses front rounded vowels as complex segments consisting of [coronal] and [peripheral]; recall that the latter feature was mentioned above as one way of expressing back vowels. The complex feature approach is unique to those SwG dialects discussed in \chapref{sec:6}. The second approach captures the distinction between front rounded vowels and their unrounded counterparts with the binary feature [±round], e.g. /y/ is [+round] and /i/ is [−round]. This treatment is the one adopted for certain LG dialects (\sectref{sec:12.6.1}) and a \isi{velar fronting island} (\sectref{sec:15.3}).

Diphthongs are represented as a sequence of two separate root nodes joined together under a single nucleus. That both parts of diphthongs have two separate root nodes has been defended by \citet{Schane1995}, \citet{Booij1995} for \ili{Dutch} as well as \citet{Wiese1996a} and \citet{Hall2002} for \il{Standard German}StG. I give representations for the diphthong /ɑi/ (e.g. \il{Standard German}StG [tsɑit] ‘time’) in \REF{ex:2:9a} and /iǝ/ in \REF{ex:2:9b}. Note that the representation in \REF{ex:2:9b} for \isi{schwa} lacks place features but that the representation does have a root node (=\ref{ex:2:7}). If both components of a diphthong are front (or back) then the place feature ([coronal]/[dorsal]) is shared by the Obligatory Contour Principle (\isi{OCP}; \citealt{Goldsmith1976}, \citealt{McCarthy1986}, \citealt{Yip1988}); see \REF{ex:2:9c} for the diphthong /ei/.\footnote{In earlier models [±high] and [±low] were argued to be under [dorsal], e.g. \citet{Sagey1986}. I follow later studies which have shown that those features are independent of [dorsal], e.g. \citet{LahiriEvers1991}.} As mentioned in \sectref{sec:2.2.2} the phonetic glide in diphthongs is not transcribed with a diacritic because its nonsyllabicity plays no role.

\ea\label{ex:2:9}%9
\begin{xlist}
\begin{multicols}{2}
\raggedcolumns
  \ex\label{ex:2:9a}\begin{forest}
  [,phantom [ \avm{[−cons\\+son\\+low]} [\avm{[dorsal]}] ]
                                        [ \avm{[−cons\\+son\\+high]} [\avm{[coronal]}] ]]
  \end{forest}
  \columnbreak
  \ex\label{ex:2:9b}
  \begin{forest}
  [,phantom [ \avm{[−cons\\+son\\+high]} [\avm{[coronal]}] ]
                                      [ \avm{[−cons\\+son]}, l-=1mm ]]
  \end{forest}
\end{multicols}
\z
\begin{xlist}
\setcounter{xnumii}{2}
  \ex\label{ex:2:9c}
  \begin{forest} for tree = {grow'=90}
      [{[\textsc{coronal}]}
        [\avm{[−cons\\+son\\−high]}]
        [\avm{[−cons\\+son\\+high]}]
      ]
  \end{forest}
  \z
\z

I do not include syllable structure (e.g. nucleus) in \REF{ex:2:9} or in any representation posited below for diphthongs in German dialects because that structure is not relevant for my treatment of velar fronting. The same point holds for phonological units capturing length (skeletal slots) or weight (moras).

Surface diphthongs in some languages have been argued to be derived from underlying monophthongs. For example, \citet[78]{vanOostendorp2000} analyzes \ili{Dutch} /ɛi/, /œy/, and /ɒu/ as the surface manifestation of underlying short high lax vowels. The default assumption I make is that diphthongs are phonemic unless evidence can be provided to the contrary.

\subsection{Opacity (part 1)}\label{sec:2.2.4}

Many languages are attested with phonetic representations that seem to contradict the phonological rules of the language in question (e.g. \citealt{Kiparsky1982a}, \citealt{McCarthy2009}, \citealt{Baković2011} and references cited therein). The phenomenon described here is referred to as \isi{opacity}, which has the formal definition \citep[79]{Kiparsky1973} in \REF{ex:2:10}.

\ea%10
    \label{ex:2:10}
          A phonological rule P of the form A ${\rightarrow}$ B / C {\longrule} D is opaque if there are surface structures with either of the following characteristics:
\ea\label{ex:2:10a} Instances of A in the environment C {\longrule} D;
\ex\label{ex:2:10b} Instances of B derived by P that occur in environments other than C {\longrule} D.
\z 
\z 

A rule is opaque if there are surface structures (phonetic representations) that look like they should have undergone it \REF{ex:2:10a} or surface structures that underwent the rule but look like they should not have \REF{ex:2:10b}. By contrast, a rule is transparent if neither of the two conditions in \REF{ex:2:10} holds.

The two types of \isi{opacity} described in \REF{ex:2:10} are referred to in the later literature as \isi{underapplication} and \isi{overapplication} respectively (e.g. \citealt{McCarthy2009}, who adopts the two terms from \citealt{Wilbur1974}). Thus, rule P underapplies in \REF{ex:2:10a} because there is a surface structure ([CAD]) in which the rule should have applied, but rule P overapplies in \REF{ex:2:10b} because is creates a structure ([B]) not specified in its structural description.

\citet{Kiparsky1973} argues that the two types of \isi{opacity} in \REF{ex:2:10} can be equated with \isi{counterfeeding} and \isi{counterbleeding} orderings respectively. Transparent orderings involve the converse orderings, namely \isi{feeding} and \isi{bleeding}.

Rule \isi{opacity} and rule \isi{transparency} and their relationship to the orderings referred to above can be illustrated with a simple example involving the two rules in \REF{ex:2:11} from a hypothetical language. The example discussed here (modified slightly) is drawn from \citet{Baković2011}.


\ea%11
    \label{ex:2:11}
\ea   Deletion:  vowel → ∅ / {\longrule} vowel
\ex   /t/-Palatalization: /t/ → [tʃ] / front vowel {\longrule}
\z 
\z

\begin{sloppypar}
Given the underlying representations /tue/ and /tio/, the respective phonetic representations are transparent if Deletion applies before /t/-Palatalization (see \ref{ex:2:12a}). In (\ref{ex:2:12a}i), Deletion \isi{feeds} /t/-Palatalization because the elimination of /u/ places the preceding /t/ before a front vowel, which is precisely the context for /t/-Palatalization. Put differently, this is a \isi{feeding order} because Deletion creates a structure to which /t/-Palatalization can apply. Deletion \isi{bleeds} /t/-Palatalization in (\ref{ex:2:12a}ii) because the elided /i/ would have triggered the Palatalization of the preceding /t/ if it had not been eliminated. This is a \isi{bleeding order} because there is a potential structure to which /t/-Palatalization could apply which is removed by Deletion. Significantly, neither output in \REF{ex:2:12a} is opaque; hence, [tʃe] and [to] show the transparent application of Deletion and /t/-Palatalization.
\end{sloppypar}

\ea%12
    \label{ex:2:12}
\ea Feeding \is{feeding order}and \isi{bleeding}:\label{ex:2:12a}\\
    \begin{tabular}{ll@{ }ll@{ }l}
                     & i. & /tue/  & ii. &  /tio/       \\
  Deletion           &    &  te    &     &    to        \\
  /t/-Palatalization &    &  tʃe   &     &  {}-{}-{}-{}-\\  
                     &    & [tʃe]  &     &  [to]        \\
     \end{tabular}
\ex \isi{Counterfeeding} and \isi{counterbleeding}:\label{ex:2:12b}\\
\begin{tabular}{ll@{ }ll@{ }l}
                    &  i.&     /tue/   & ii. &   /tio/\\
 /t/-Palatalization &    & -{}-{}-{}-  &     &   tʃio \\
 Deletion           &    &    te       &     &   tʃo  \\
                    &    &   [te]      &     &  [tʃo] \\
\end{tabular}
\z 
\z

Reversing the ordering of the two rules (=\ref{ex:2:12b}) yields \isi{opacity}. In (\ref{ex:2:12b}i) Deletion \isi{counterfeeds} /t/-Palatalization. The reason is that the elimination of /u/ causes /t/ to become adjacent to the front vowel /e/, but /t/-Palatalization cannot apply because that rule precedes Deletion. In (\ref{ex:2:12b}ii) Deletion \isi{counterbleeds} /t/-Palatalization because the deleted /i/ is a front vowel, which triggers /t/-Pal\-a\-tal\-i\-za\-tion before deleting. In (\ref{ex:2:12b}i) /t/-Palatalization underapplies because there is a surface structure in which the rule should have applied (i.e. [te]). By contrast, in (\ref{ex:2:12b}ii) /t/-Palatalization overapplies because there is an instance of a sound created by that rule that occurs in a context not specified by the rule (i.e. [tʃ] before the back vowel [o]).

Rules interacting with velar fronting can alter either the triggers or the targets for that process. That point can be illustrated with the hypothetical language in \REF{ex:2:12}. Consider first (\ref{ex:2:12a}i), where Deletion \isi{feeds} /t/-Palatalization by creating a new trigger for the latter process. Changes not depicted in \REF{ex:2:12} might \isi{feed} /t/-Palatalization by increasing the number of targets for that rule. For example, if there were a change from /d/ to [t] in word-initial position (/d/-Fortition), and if the output of that change undergoes /t/-Palatalization, then /d/-Fortition \isi{feeds} /t/-Palatalization by creating a new target; hence, /di/ surfaces as [tʃi]. In (\ref{ex:2:12b}i) Deletion \isi{counterfeeds} /t/-Palatalization by increasing the number of potential triggers. However, if /d/-Fortition does not \isi{feed} /t/-Palatalization then the latter process is counterfed by the former because it creates a new target which is immune to /t/-Palatalization; hence, /di/ surfaces as [ti].

Transparent and opaque interactions are summarized in \REF{ex:2:13}, which is taken from \citet[43]{Baković2011}, who in turn bases this classification on earlier work by Kiparsky and McCarthy.\footnote{\label{fn:2:8}\citet{Baković2011} shows that the classification in \REF{ex:2:13} is not sufficient for several reasons. For example, a \isi{counterfeeding} interaction does not always result in \isi{underapplication}, and \isi{counterbleeding} is not the only way to describe actual examples illustrating \isi{overapplication}. What is more, a \isi{counterbleeding} relationship does not always exhibit \isi{overapplication}.}

\ea%13
    \label{ex:2:13}
    \begin{forest}
     [transparent [feeding] [bleeding]]
    \end{forest}\hfill    
    \begin{forest}
     [opaque 
       [type \REF{ex:2:10a}\\
        \isi{underapplication},align=center 
          [counterfeeding]
        ] 
        [type \REF{ex:2:10b}\\
         \isi{overapplication},align=center 
           [counterbleeding]
         ]
       ]
    \end{forest}
\z 

An additional type of interaction involves the \is{mutual bleeding order}\textsc{mutual bleeding} of two rules. That refers to a situation in which a rule A \isi{bleeds} a later-ordered rule B and where rule B would also \isi{bleed} rule A if it were ordered before rule A \citep{Baković2011}. That type of interaction is illustrated in two sets of German dialects in \REF{ex:2:14} modified slightly from \citet[66]{Kiparsky1982a}. (\ref{ex:2:14}a) represents dialects in North Germany and (\ref{ex:2:14}b) from another set of dialects (e.g. StG). For discussion see \citet[Chapter 4]{Hall1992} and references cited therein.

\ea%14
\label{ex:2:14}Mutual bleeding:\\
\begin{tabular}{@{}  ll@{ }ll@{ }l  @{}}
                &  a. &  /laŋg/     & b. &    /laŋg-ə/         \\
\isi{Final Fortition} &     &  laŋk       &    & {}-{}-{}-{}-        \\
\isi{g-Deletion}      &     &  -{}-{}-{}- &    &  laŋə               \\
                &     & [laŋk]      &    & [laŋə]              \\
                &     & ‘long’      &    & ‘long-\textsc{infl}’\\\tablevspace
                &  c. & /laŋg/        & d. & /laŋg-ə/          \\
     \isi{g-Deletion} &     &  laŋ          &    &   laŋ             \\
\isi{Final Fortition} &     & {}-{}-{}-{}-  &    &   {}-{}-{}-{}-    \\
                &     & [laŋ]        &    &   [laŋə]          \\
                &     &               &    &                   \\
\end{tabular}
\z 

\isi{Final Fortition} affects all obstruents in a coda, while \isi{g-Deletion} eliminates /g/ after a nasal. In (\ref{ex:2:14}a), \isi{Final Fortition} \isi{bleeds} \isi{g-Deletion}, while the reverse ordering in (\ref{ex:2:14}b) shows that \isi{g-Deletion} \isi{bleeds} \isi{Final Fortition}. Examples (\ref{ex:2:14}b) and (\ref{ex:2:14}d) illustrate \isi{g-Deletion} in the context before a vowel.

\isi{Mutual bleeding} -- in contrast to \isi{counterbleeding} -- does not involve \isi{opacity}. In particular, neither [laŋk] in (\ref{ex:2:14}a) nor [laŋə] in (\ref{ex:2:14}b, \ref{ex:2:14}d) exhibit \isi{overapplication} or \isi{underapplication} of \isi{Final Fortition} or \isi{g-Deletion}. \isi{Mutual bleeding} therefore exemplifies a transparent interaction.

The distinction between \isi{transparency} and \isi{opacity} as they relate to velar fronting is a significant theme in the present book. I show below that the fronting of velars is transparent in some dialects and opaque in others. From the synchronic perspective, the transparent process of velar fronting is either fed or bled by an independent process, or velar fronting and another process stand in a \isi{mutually bleeding} relationship. Several dialects are attested in which velar fronting exhibits \isi{underapplication} because it is counterfed synchronically by another process (as in \ref{ex:2:12b}i). No dialect has been found in which velar fronting overapplies synchronically by being counterbled by another process (as in \ref{ex:2:12b}ii); see \chapref{sec:5} for discussion.\footnote{My main concern is \isi{opacity} as it relates to velar fronting. It will be seen in the ensuing chapters that synchronic \isi{counterbleeding} orderings are indeed required (=\ref{ex:2:12b}ii) but that neither of the rules involved is velar fronting. However, that \isi{counterbleeding} relationship does not result in \isi{overapplication} (recall \fnref{fn:2:8}).}

Opacity is defined above in synchronic terms, but it is also possible to view diachronic changes as opaque or transparent even though the sound changes are no longer active as synchronic rules. It is demonstrated in the following chapters that the historical process fronting velars has become opaque through time in many varieties and that the type of \isi{opacity} referred can involve both \isi{underapplication} as well as \isi{overapplication}; see Chapters~\ref{sec:6}--\ref{sec:9} for extensive discussion.

\section{Typology of \isi{Velar Palatalization}}\label{sec:2.3}

One of my goals is to compare the patterning of velar fronting in German dialects with rules fronting velars in other languages; recall research questions \REF{ex:2:10} in \sectref{sec:1.4.4}. Processes fronting velar sounds like /k/ and /x/ to a position towards the front of the oral cavity in the neighborhood of front vowels have been studied extensively in the literature, which traditionally refers to the change in question as \textsc{velar} \textsc{palatalization}. I retain the term \textsc{velar} \textsc{fronting}, which can be viewed as a special type of \isi{Velar Palatalization}. In the present section I clarify that assertion by examining the findings in the typological literature on \isi{Velar Palatalization}. The reader is referred to Appendix~\ref{appendix:i}, which contains some discussion of \isi{Velar Palatalization} in the branches of Germanic (Gmc) not discussed in this book (\ili{WGmc} and North Germanic (\ili{NGmc})) as well as two language families spoken adjacent to German-speaking countries, namely \ili{Romance} and \ili{Slavic}.

\subsection{Introduction}\label{sec:2.3.1}

The cross-linguistic literature on \isi{Velar Palatalization} is extensive. Many linguists consider the phonetics of \isi{Velar Palatalization} (e.g. \citealt{Guion1998}, \citealt{Recasens2020}), while others examine the phonology (e.g. the featural aspects), e.g. \citet{LahiriEvers1991}, \citet{Hume1994}, \citet{ClementsHume1995}. Considerable work focuses on \isi{Velar Palatalization} (synchronic or diachronic) in individual languages or language families. Some of that research (listed alphabetically in terms of the language) includes \ili{Albanian} (\citealt{Kolgjini2004}), \ili{Greek} (\citealt{Newton1972a}, \citealt{ManolessouPantelidis2013}), \ili{Old Chinese} (\citealt{Schuessler1996}), \ili{Polish} (\citealt{Ćavar2004,Gussmann2004}), \ili{Romance} (\citealt{Repetti2016, Schmid2016}), \ili{Slovene} \citep{Jurgec2016}, and \ili{Latvian} \citep{Urek2016}, although many other language families could be added to that list. There is also a small but growing body of research investigating the typological aspects of \isi{Velar Palatalization}, e.g. \citet{Neeld1973}, \citet{Chen1973}, \citet{Bhat1978}, and most recently \citet{Bateman2007, Bateman2011}, \citet{Kochetov2011}, \citet{KrämerUrek2016}, and \citet{Recasens2020}.\footnote{The typological literature cited above investigates \isi{Velar Palatalization} in the context of Palatalization in the broad sense of the word. For example, in many languages alveolar (coronal) sounds like /t/ are realized as postalveolar ([tʃ]) in the context of front vowels (recall the hypothetical rule in \ref{ex:2:11}b). The typological literature also considers the Palatalization of labial sounds like /p/, which is rare.}

This typological research -- in particular \textcite{Bateman2007,Bateman2011} -- has shown that Palatalization can target velar consonants and that the outputs can be quite diverse. In \tabref{tab:fromex:2:15} I present the most common targets and outputs for \is{Velar Palatalization}Velar Palatalizations as discussed in the literature. Velar fronting in German dialects has a much more restricted set of outputs, as indicated in \tabref{tab:fromex:2:16}.\footnote{The lenis velar fricative [ɣ] and the velar \isi{affricate} [kx] are not included in \tabref{tab:fromex:2:15} because the languages surveyed with \isi{Velar Palatalization} do not have those sounds. I do not consider that omission to be significant.}


\begin{table}
\begin{floatrow}
\captionsetup{margin=.05\linewidth}
\ttabbox{\begin{tabular}{lcc}
\lsptoprule
& target & output\\\midrule
Stop & [k] & [kʲ c cʲ tʃ tʃʲ ç]\\
     & [g] & [gʲ ɟ dʒ]\\
Fricative & [x] & [xʲ ç ɕ ʃ]\\
Nasal     & [ŋ] & [ŋʲ ɲ]\\
\lspbottomrule
\end{tabular}}
{\caption{Velar Palatalization targets and outputs\label{tab:fromex:2:15}}}
\ttabbox{\begin{tabular}{lcc}
\lsptoprule
& target & output\\\midrule
Stop       & {[k]} & [c]\\
           &  [g]  & [ɟ]\\
Fricative  & {[x]} & [ç ɕ]\\
           & [ɣ]   & [ʝ]\\
Affricate  & [kx]  & [kç]\\
Nasal      & [ŋ]   & [ɲ]\\
\lspbottomrule
\end{tabular}}
{\caption{\label{tab:fromex:2:16}Velar fronting targets and outputs}}
\end{floatrow}
\end{table}

There are two  significant differences between Tables~\ref{tab:fromex:2:15} and~\ref{tab:fromex:2:16}: (a) The output of velar fronting is the corresponding palatal (i.e. [k g x ɣ kx ŋ] are fronted to [c ɟ ç ʝ kç ɲ]); hence, the manner of articulation does not change. However, \isi{Velar Palatalization} often changes the manner of articulation for stop targets ([k g]), which can surface as affricates ([tʃ dʒ]);\footnote{Both \isi{Velar Palatalization} and velar fronting can involve a minor manner change in the case of the target nonsibilant fricative [x], which fronts/palatalizes to the \isi{sibilant} fricative [ɕ].} and (b) \isi{Velar Palatalization} changes either (i) the primary place of articulation (\textsc{full} \textsc{velar} \textsc{palatalization}), e.g. velar [x] is realized as palatal [ç], or (ii) adds \is{secondary palatalization}\textsc{secondary palatalization} to a primary place of articulation, e.g. velar [k] surfaces as secondarily palatalized velar [kʲ]. By contrast, velar fronting changes only the primary place of articulation.\footnote{My assertion that German dialects exhibit the restricted set of outputs in \tabref{tab:fromex:2:16} and not the broad one in \tabref{tab:fromex:2:15} is based on my scrutiny of the original sources for over three hundred varieties of German. To be clear: I do not deny that there might be dialects of German with the broad set of outputs in \tabref{tab:fromex:2:15}, e.g. [xʲ] for /x/ or [tʃ] for /x/. See \sectref{sec:11.1} for brief discussion of the realization of fronted velar stops as affricates. However, based on the preponderance of the evidence discussed in the remainder of this book, the broad set of outputs in \tabref{tab:fromex:2:15} clearly represents less preferred patterns.}

\isi{Velar Palatalization} and velar fronting differ in terms of triggers. The typological literature on \isi{Velar Palatalization} demonstrates that triggers for that process consist of front vowels (or some subset thereof) and the palatal glide [j] (if present). My own study reveals that there are two ways in which velar fronting triggers are broader than \isi{Velar Palatalization} triggers. First, velar fronting is typically induced by front vowels and coronal sonorant consonants ([r l n]). Second, velar fronting can occur in the context of one or more back vowel. In fact, in many dialects velar fronting affects velar sounds adjacent to any sound; hence, in that type of system velar fronting has no segmental trigger at all.

The restricted set of triggers for \isi{Velar Palatalization} has led many researchers to make the following assumptions:\largerpage

\ea%17
\label{ex:2:17}
\ea\label{ex:2:17a}Velar Palatalization is always assimilatory;
\ex\label{ex:2:17b}Velar Palatalization is always triggered by one or more front vowel;
\ex\label{ex:2:17c}Velar Palatalization cannot occur in the context of back vowels;
\ex\label{ex:2:17d}Velar Palatalization must have a segmental trigger;
\ex\label{ex:2:17e}Velar Palatalization is not triggered by consonants in addition to (front) vowels.
\z 
\z\is{Velar Palatalization}

Note that the two statements in (\ref{ex:2:17c}, \ref{ex:2:17d}) are corollaries of \REF{ex:2:17a}. Since Palatalization is considered to be a prototypical rule involving consonant-vowel place interactions, the trigger is said to comprise front vowels only (=\ref{ex:2:17b}), but not a set of (front) consonants and (front) vowels (=\ref{ex:2:17e}).\footnote{According to \citet{Kochetov2011}, Palatalization \textit{usually} (my emphasis) arises under the influence of an adjacent front vowel (including [j]). \citet[2]{KrämerUrek2016} make passing reference to languages in which some kind of Palatalization occurs without a front vowel trigger, although they refrain from discussing those examples. That point aside, there is certainly unanimous agreement in the literature that a system in which \isi{Velar Palatalization} is triggered by a back vowel is peculiar and possibly without precedent.}

The behavior of velar fronting in German dialects is significant because it demonstrates that none of the statements in \REF{ex:2:17} can be unconditionally true. First, many varieties are attested in which velars undergo fronting regardless of the nature of the following sound. That type of velar fronting is significant because it poses a clear challenge for (\ref{ex:2:17a}, \ref{ex:2:17b}, \ref{ex:2:17d}). Second, velar fronting in many varieties -- including \il{Standard German}StG -- consists of front vowels and coronal sonorant consonants, thereby counter-exemplifying \REF{ex:2:17e}. Third, a set of dialects is attested in which velar fronting occurs in the context of a preceding back vowel, thereby calling \REF{ex:2:17c} into question.

In \REF{ex:2:18a} I provide the definition for full Palatalization \citep{Bateman2011} and in \REF{ex:2:18b} a parallel definition of velar fronting. The term \textsc{vocoid} in \REF{ex:2:18a} is the set of vowels and glides (i.e. palatal [j]).

\eanoraggedright%18
\label{ex:2:18}Definition of full Palatalization in (\ref{ex:2:18a}) and velar fronting in (\ref{ex:2:18b}):
\eanoraggedright “A consonant changes its primary place of articulation and often its manner of articulation, while moving toward the palatal region of the vocal tract when adjacent to a high and/or front \isi{vocoid}…”. \citep[589]{Bateman2011}.\label{ex:2:18a}
\ex A velar consonant changes its primary place of articulation, while moving toward the palatal region of the vocal tract (thereby creating palatal or alveolopalatal sounds) usually when adjacent to a front vowel or coronal sonorant consonant.\label{ex:2:18b}
\z
\z 

Note that the wording of \REF{ex:2:18a} accounts for the diverse set of outputs in \tabref{tab:fromex:2:15} while simultaneously capturing the generalizations in \REF{ex:2:17}. By contrast, velar fronting is defined in such a way to admit only the restricted outputs in \tabref{tab:fromex:2:16}, but it does not imply the validity of the statements in \REF{ex:2:17}.\footnote{{In an attempt to eschew an overly wordy definition, I do not attempt to express the fact that the alveolopalatal sound referred to in \REF{ex:2:18b} is the fricative [ɕ] from the target [x]. It should go without saying that the properties of velar fronting described in the remainder of this section cannot be included in the definition presented in \REF{ex:2:18b}.}} In any case, the prose statement of velar fronting in \REF{ex:2:18b} is expressed formally as the spreading of the frontness feature [coronal], as in \REF{ex:2:6}, or as the addition of that feature in dialects where velar fronting does not function as an assimilation.

Tables~\ref{tab:fromex:2:15} and~\ref{tab:fromex:2:16} and the description of triggers listed above do not give any indication of what targets and triggers are more common or whether or not there are any exceptionless cross-linguistic generalizations which can be made. In \sectref{sec:2.3.2}--\sectref{sec:2.3.5} I discuss that type of issue.

\subsection{Targets}\label{sec:2.3.2}

According to \citet[56ff.]{Bateman2007} the most common targets for Palatalizations (in the broad sense of the word) are obstruents (as opposed to sonorants). Languages with stops as targets outnumber those with fricatives. The next best targets are nasals followed by laterals, and finally by rhotics. It is not possible to posit implications involving the preference for stops over fricatives (e.g. “If a fricative is a target for Palatalization, then a stop is also a targetˮ) because there are too many counterexamples, i.e. languages in which fricatives but not stops serve as Palatalization targets. Bateman writes that “…there is an overwhelming tendency in most languages for obstruents to palatalize most often, followed by the other manners of articulation …ˮ.

The generalization described in the preceding paragraph concerning obstruent vs. sonorant targets also holds for velar fronting, although the only sonorant target for velar fronting is [ŋ].\footnote{{No dialect of German is attested with a velar lateral (/ʟ/) which could potentially serve as a target for velar fronting. Since /ʀ/ is represented as [dorsal] (=\ref{ex:2:3}d), it is a potential target segment. No dialect of German -- or any natural language to the best of my knowledge -- is attested in which /ʀ/ undergoes fronting (=\ref{ex:2:6}).} } Only a small number of dialects exhibit the fronting of a velar nasal; however, of those dialects with that change, velar stops and velar fricatives also undergo fronting. One exceptionless generalization for velar fronting is expressed in \REF{ex:2:19}.

\eanoraggedright%19
    \label{ex:2:19}
          \textsc{{Implicational} \textsc{Universal} \textsc{for} \textsc{Velar} \textsc{Fronting} \textsc{Targets-1}}:\\
          If a velar stop (/k g/) undergoes velar fronting then the corresponding fricative (/x ɣ/) does as well.
\z
\REF{ex:2:19} suggests that the preferred target for velar fronting is a fricative (/x/) and not a stop (/k/); recall the \il{Standard German}StG data in \sectref{sec:1.2}. Dialects lending support to \REF{ex:2:19} are discussed in \chapref{sec:11}.

One generalization concerning velar fronting targets not discussed in \citet{Bateman2007,Bateman2011} relates to the distinction between lenis (e.g. /g/) vs. fortis (e.g. /k/) sounds. There is strong evidence from phonetics -- also reflected in typological work -- that fortis sounds make for better targets than lenis sounds. That generalization apparently holds for stops at all places of articulation. For example, in their typological survey of \is{Assibilation}\textsc{Assibilations} -- the change from an alveolar stop like /t/ or /d/ to an \isi{affricate} like [ts] or [dz] in the context of a front \isi{vocoid} -- \citet{HallHamann2006} show that lenis /d/ cannot assibilate unless the corresponding fortis sound (/t/) does. The phonetic reason for that observation is discussed in \citet{HallZygis2006}: In a sequence like /ti/ the friction phase (which arises after the release of a coronal stop before a high vowel) has a longer duration than the one in a sequence like /di/. In her study of \isi{Velar Palatalization}, \citet[20]{Guion1998} observes the same asymmetry involving lenis (“voicedˮ) vs. fortis (“voicelessˮ) velar targets and concludes that “[v]oiceless velars are more likely to palatalize than voiced velarsˮ. Guion observes that the cases of \isi{Velar Palatalization} discussed in \citet{Bhat1978} involve either: (a) cases of lenis and fortis targets, or (b) fortis only targets, but no cases of lenis only targets. The studies cited here suggest the implication in \REF{ex:2:20}. I state \REF{ex:2:20} with respect to velar fronting, although it is probably more general in its scope.

\eanoraggedright%20
    \label{ex:2:20}
          \textsc{{Implicational} \textsc{Universal} \textsc{for} \textsc{Velar} \textsc{Fronting} \textsc{Target-2}}:\\
          If a lenis sound undergoes velar fronting then the corresponding fortis sound does as well.
\z
\REF{ex:2:20} is exceptionless in the studies cited above (for velar stops as target segments). It remains to be seen whether or not that implication can also be confirmed for velar fricatives as targets. In any case, it is demonstrated below that German dialects obey (\ref{ex:2:19}) and (\ref{ex:2:20}) for either velar stops or velar fricatives as target segments; see \sectref{sec:4.5.1}, \sectref{sec:11.9.1}, and \sectref{sec:12.7.2} for discussion.

\subsection{Triggers}\label{sec:2.3.3}

It is undeniably the case that the unmarked context for \isi{Velar Palatalization} is the set of front vowels, especially the high front vowel /i/; see \citet[62]{Bateman2007} and \citet{Kochetov2011}. The latter author notes that Palatalization (in the general sense) is only rarely triggered by low front vowels (e.g. /æ/). In fact, there is agreement in the literature that low and mid front vowels only trigger Palatalization if high vowels trigger it as well (\citealt{Neeld1973}: 37, \citealt{Chen1973}: 177, \citealt{Bateman2007}: 64, and \citealt{Kochetov2011}). \citet[64]{Bateman2007} posits the implication in \REF{ex:2:21}, which is apparently exceptionless. The implication is also shown below to be exceptionless for velar fronting in German dialects.

\eanoraggedright%21
    \label{ex:2:21}
          \textsc{{Implicational} \textsc{Universal} \textsc{for} \textsc{Palatalization} \textsc{Triggers}}:\\
          If lower front vowels trigger Palatalization, then so will higher front vowels.
\z

The generalization expressed in \REF{ex:2:21} has been argued to be grounded in phonetics. For example, in her perception study on the realization of velars like /k/ as postalveolar affricates ([tʃ]), \citet{Guion1998} shows that the acoustic similarity between the target ([k]) and output ([tʃ]) is greater before high front vowels than before mid and low front vowels. The conclusion is that high front vowels are more favorable triggers for velar fronting than nonhigh front vowels.

\isi{Nonheight features} seldom play a role in defining the natural class of vocalic triggers for \isi{Velar Palatalization} \citep[62]{Bateman2007}. In particular, Bateman finds that features such as vowel length, \isi{rounding}, or nasality do not make a difference in a front vowel’s ability to serve as a trigger. Thus, short front vowels, long front vowels, front rounded vowels, front unrounded vowels, front nasalized vowels and front oral vowels can induce Velar Palatalization. One exception to this generalization \citep[54--55]{Bateman2007} is \ili{Fanti} (Niger-Congo, Ghana), in which /x/ palatalizes only before a front non-nasal vowel. German dialects in which velar fronting is sensitive to \isi{nonheight features} are rare but attested; see \sectref{sec:12.6} for dialects in which roundedness, \isi{tenseness}, and \isi{stress} can play a role in defining the set of front vocalic triggers. The role of nasality as a factor in defining the triggers for velar fronting is discussed briefly in \sectref{sec:15.9}.\footnote{{One feature not mentioned above is vowel length. In a recent study, \citet{CardosoHoneybone2022} show that the velar fricative derived through the lenition of /k/ in \ili{Liverpool English} surfaces as palatal ([ç]) after a high, front vocalic trigger. Significantly, that trigger must be bimoraic, i.e. a long monophthong (/iː/) or a diphthong (e.g. /aɪ/), since that change does not occur after a short monophthong (/ɪ/). From the formal point of view, their rule (\is{Dorsal Fricative Assimilation!Liverpool English}Dorsal Fricative Assimilation) spreads the frontness feature ([palatality] in their system) to the right, but spreading only occurs if the vocalic trigger is foot-final, which is precisely the case when that vowel is bimoraic. I do not discuss quantity-sensitivity as a trigger in this book because no parallel cases involving velar fronting in German dialects are known to me.}}

\subsection{Outputs }\label{sec:2.3.4}

Typological studies agree that the preferred outputs for \isi{Velar Palatalization} with stops as targets (/k g/) are \isi{sibilant} affricates (i.e. postalveolar [tʃ dʒ]). That type of output is more common than palatal nonsibilant stops (=[c ɟ]).  \citet[595]{Bateman2011} writes: “The most common full palatalization outcomes for the coronal and dorsal oral stops /t/, /d/ and /k/, /g/ are … [tʃ] and [dʒ]ˮ. \citet{Kochetov2011} likewise writes: “Overall, there is a tendency for place-changing palatalizations to result in sibilants rather than non-sibilantsˮ. The data in those sources (and in \citealt{Guion1998}) suggest that there is a similar generalization for \isi{Velar Palatalization} with fricatives as targets in the sense that \isi{sibilant} fricatives ([ʃ ʒ]) are the preferred output to nonsibilant fricatives (=[ç ʝ]). These generalizations are stated in \REF{ex:2:22}:

\eanoraggedright%22
    \label{ex:2:22}
\eanoraggedright
      If the target for \isi{Velar Palatalization} is a stop (/k g/) then the preferred output is a \isi{sibilant} \isi{affricate} ([tʃ dʒ]).
\ex   If the target for \isi{Velar Palatalization} is a fricative (e.g. /x/) then the preferred output is a \isi{sibilant} (e.g. [ʃ]) rather than a nonsibilant ([ç]).

\z 
\z

The rarity of sounds like [ç] as the output correlates with the findings in \citet[43--47]{Maddieson1984}, who concludes that the \isi{sibilant} [ʃ] is the second most common fricative (behind [s]), while the nonsibilant [ç] was the second least common (before pharyngeal [ħ]).

The data from German dialects discussed below reveal that neither of the statements in \REF{ex:2:22} can be confirmed: For both velar stops and velar fricatives the preferred output is a nonsibilant (e.g. \il{Standard German}StG). However, there are some areas to be investigated below (\chapref{sec:10}) in which the target fricative /x/ surfaces as the \isi{sibilant} alveolopalatal fricative [ɕ].

\subsection{Directionality and adjacency}\label{sec:2.3.5}\is{Directionality|(}\is{adjacency|(}

An additional parameter discussed in the typological literature on Palatalization is \isi{directionality}. If the target (e.g. /k/) is situated to the left of the trigger (e.g. /i/) then Palatalization occurs from right-to-left (regressive), but if the target is to the right of the trigger then Palatalization occurs from left-to-right (progressive). The literature is in agreement that both options are well-attested, but that regressive assimilation is the preferred option. I refer to that generalization as the \isi{Directionality Parameter for Palatalization}, which I state in \REF{ex:2:23}; see \citet[75--77]{Bateman2007}. A final parameter is whether or not the target and trigger can be separated by an intervening sound  (\textsc{adjacency}). The literature is in agreement that in the overwhelming number of cases the trigger and target for Palatalization must be adjacent; see \citet[75--77]{Bateman2007} and \citet{Kochetov2011}. I state that generalization in \REF{ex:2:24}.

\eanoraggedright%23
    \label{ex:2:23}
          Directionality Parameter for Palatalization:\is{Directionality Parameter for Palatalization} The preferred direction for Palatalization is right-to-left (regressive); hence, the trigger follows the target. Progressive Palatalization is also possible, although it is less preferred.
\ex%24
    \label{ex:2:24}
          \isi{Adjacency Parameter for Palatalization}: The trigger and target for Palatalization are preferably adjacent.
\z

The Directionality Parameter in (\ref{ex:2:23}) is counterexemplified by velar fronting, which applies progressively, e.g. \il{Standard German}StG [kuːxən] ‘cake’ vs. [kʏçə] ‘kitchen’. Surprisingly, German dialects do not exhibit variation with respect to the \isi{directionality} of velar fronting. I state that exceptionless generalization in \REF{ex:2:25}. Finally, the data discussed below from German dialects support \REF{ex:2:24}, which I restate in \REF{ex:2:26} in terms of velar fronting:

\eanoraggedright%25
    \label{ex:2:25}
          Directionality of Velar Fronting: If a target for velar fronting is situated after a sonorant and before a vowel then the trigger is always the sonorant to the immediate left of that velar sound.
\ex%26
    \label{ex:2:26}
          \isi{Adjacency Parameter for Velar Fronting}: The trigger and target for Palatalization are preferably adjacent.
\z

The relevance of \isi{adjacency} is discussed in the dialects investigated in \chapref{sec:6} and \chapref{sec:11}; see also the discussion in \sectref{sec:12.8.1}. \isi{Directionality} is discussed in \sectref{sec:6.5.2} and \sectref{sec:16.5}.\is{Directionality|)}\is{adjacency|)}

\section{{Historical} {phonology}}\label{sec:2.4}

I adopt historical models that account for the changes involving trigger and target segments for velar fronting (\sectref{sec:2.4.1}) as well as historically opaque velars and palatals (\sectref{sec:2.4.2}, \sectref{sec:2.4.3}). Structural and nonstructural causes of velar fronting are discussed in \sectref{sec:2.4.4}.

\subsection{Rule generalization}\label{sec:2.4.1}
\begin{sloppypar}
Sound change often begins with a highly restricted environment in which phonetic conditions are particularly favorable and then progressively spreads through time and space to include more general triggers. The name for type of development is \textsc{rule} \textsc{generalization} (\citealt{Vennemann1978}, \citealt{Bermúdez-Otero2015}, \citealt{Hinskens2021}).\footnote{The phenomenon is also frequently referred to by alternate names, e.g. \textsc{phonetic} \textsc{analogy} (e.g. \citealt{Benware1996} from \citealt{Schuchardt1885}).}
\end{sloppypar}

Rule generalization can be illustrated with the material discussed \citet{Benware1996} and more recently \citet{Ramsammy2015}, which involves the change from [s] (/s/) to [ʃ] (/ʃ/) in word-initial position before a sonorant consonant in German dialects (\isi{s-Palatalization}). On the basis of orthographic evidence in manuscripts written between 1300 and 1550, Benware shows that \isi{s-Palatalization} occurred first before /l/, next before /l n/, then before /l n m/, and finally before /l n m w/. Those four historical stages are illustrated in \REF{ex:2:27}. Note that the [ʃ] (/ʃ/) realization is reflected in \il{Standard German}StG orthography as \textit{sch}.

\ea\label{ex:2:27}
\TabPositions{0pt, .425\linewidth}
\ea\label{ex:2:27a}/s/ > /ʃ/ /    \textsubscript{wd}[  {\longrule}{\longrule} /l/       \tab MHG \textit{sleht} > \il{Standard German}StG \textit{schlecht}  ‘bad’
\ex\label{ex:2:27b}/s/ > /ʃ/ /   \textsubscript{wd}[  {\longrule}{\longrule} /l n/     \tab MHG \textit{snel} > \il{Standard German}StG \textit{schnell}  ‘fast’
\ex\label{ex:2:27c}/s/ > /ʃ/ /   \textsubscript{wd}[  {\longrule}{\longrule} /l n m/   \tab MHG \textit{smal} > \il{Standard German}StG \textit{schmal}  ‘narrow’
\ex\label{ex:2:27d}/s/ > /ʃ/ /   \textsubscript{wd}[  {\longrule}{\longrule} /l n m w/ \tab MHG \textit{swarz} > \il{Standard German}StG \textit{schwarz}  ‘black
\z
\z 

The four contexts in \REF{ex:2:27} reflect the progressive historical stages of \is{s-Palatalization}s-Pal\-a\-tal\-i\-za\-tion, as illustrated in \REF{ex:2:28}:

\ea%28
\label{ex:2:28}Increase in triggers for \isi{s-Palatalization}:\\
\begin{tabular}{@{}lcccc@{}}
  Stage A &  yes    &     no   &      no    &     no     \\
  Stage B &  yes    &     yes  &      no    &     no     \\
  Stage C &  yes    &     yes  &     yes    &     no     \\
  Stage D &  yes    &     yes  &     yes    &     yes    \\
          &    ↑    &      ↑   &      ↑     &      ↑     \\
          &  \{l\}  &  \{l n\} &  \{l n m\} & \{l n m w\}\\
\end{tabular}
\z

Rule generalization as proposed in the literature cited above is defined in terms of triggers, but other linguists have made similar claims concerning targets. In particular, the proposal has been made that sound change can involve a gradual extension in the number of segments undergoing the change.\footnote{An earlier proponent of that approach is defended by \citet[58--63]{King1969}, who discusses various changes in the history of German involving the extension of target segments (= \is{rule simplification}\textsc{rule simplification} in his terminology). For example, King demonstrates that the historical rule of German devoicing obstruents in final position (\isi{Final Fortition} in \ref{ex:2:14}) was preceded by a stage in which only fricatives but not stops devoiced.} For example, \citet{DavisSalmons1999} argue that there was a gradual increase in the number of target segments that underwent the historical change from /p t k/ to the corresponding affricates or fricatives in German dialects (\isi{High German Consonant Shift}). See \citet[82--95]{Braune2004} for a summary of the facts and a survey of the literature of that sound change. The generalization -- according to \citet{DavisSalmons1999} -- is that /t/ was affected first, followed by /p/, and then /k/. The gradual increase in target segments for affrication in word-initial position is depicted in \REF{ex:2:29}. \citet{DavisSalmons1999} argue that the place asymmetry illustrated here is a consequence of phonological \isi{markedness} and complexity of representation; hence, /t/ was affected first because it was the least marked (and has the least complex phonological representation), and /p/ was affected more than /k/ for the same reasons.

\ea%29
\label{ex:2:29}Increase in number of targets (\isi{High German Consonant Shift}):
\begin{tabular}{@{}lccc@{}}
  Stage A &  yes   &    no   &     no   \\
  Stage B &  yes   &    yes  &     no   \\
  Stage C &  yes   &    yes  &     yes  \\
          &   ↑    &     ↑   &      ↑   \\
          & \{t\}  & \{p t\} & \{p t k\}\\
\end{tabular}
\z 

Sound change begins in a \textsc{focal} \textsc{area} \citep[440]{Hock1986} and then spreads both temporally and geographically from that point of origin. Spreading typically involves triggers and/or targets, which gradually expand in the \isi{focal area} to include more and more segments. The original change in the \isi{focal area} also spreads geographically in the sense that outlying areas adopt it. Significantly, the change is active the longest in the \isi{focal area}, and it is there where it reaches its most general form in terms of the number of triggers/targets. However, in some of the outmost areas the change never progresses to the more general contexts in the \isi{focal area}. The important point is that the \isi{focal area} is the place where that process has the most general the set of triggers/targets.\footnote{This interpretation of the spread of a rule from a \isi{focal area} has been endorsed by a number of linguists. One of the first was \citet[61f.]{Schuchardt1885}. See also \citet[345]{RobinsonCoetsem1973} and \citet[393--394]{Kiparsky1988}.}

For \REF{ex:2:29}, dialects with the largest set of targets (/p t k/) reflect those areas where the change began (Switzerland), while those with the fewest targets (/t/) indicate regions where the change was most recent (parts of Central Germany). This means that the change was phonologized (in Switzerland) by affecting only /t/, and then /p/ and /k/ were eventually added to the set of targets in that order. While \isi{rule generalization} was transpiring temporally in Switzerland, it also spread geographically (to the north).\footnote{\citet{DavisSalmons1999} also argue that the change from /p t k/ to the corresponding fricatives in word-internal and word-final position (\isi{High German Consonant Shift}) exhibited a gradual increase in the number of triggers: In the first stage that change was induced by preceding short vowels and in the next stage the triggers were extended to include long vowels. In the final stage consonants served as triggers.}

An examination of the material from German dialects reveals that the set of triggers and targets for velar fronting exhibits variation. I argue that that variation in terms of space (dialects) is a reflection of temporal change; hence, the set of triggers and targets initially consisted of a small number of segments, and language change involved the gradual extension of both trigger and target segments. I employ the term “rule generalizationˮ to describe both the increase in triggers and targets.\footnote{Hypothetically one might argue that change involves not an extension of targets and triggers, but instead the opposite. On that view, \isi{s-Palatalization}, for example, applied first in the general environment in \REF{ex:2:27d} and then worked its way to the least specific environment in \REF{ex:2:27a}. I reject that interpretation of targets and triggers in the material presented in the ensuing chapters. An advantage of the present approach is that an extension of velar fronting triggers from specific to general can be shown to be phonetically grounded; recall the discussion of the implications in \REF{ex:2:20} and \REF{ex:2:21} in \sectref{sec:2.3}. By contrast, the change from general triggers to specific ones would be phonetically arbitrary.}

The way in which \isi{rule generalization} works in terms of time and space is depicted abstractly in \figref{fig:2.2}. The three stages referred to here can be thought of in terms of \isi{rule generalization}: The white squares illustrate the rule as it is first phonologized with a narrow set of targets and/or triggers (X). The gray squares show the same rule with an expanded set of targets and/or triggers (X, Y), and the black squares represent the same rule with targets and/or triggers that are further expanded (X, Y, Z).


\begin{figure}
  \begin{tikzpicture}
  \matrix (matrix) [nodes={minimum width=1cm, minimum height=1cm}]
    {
      \node{A}; &[20mm] & \node{B}; & \\
      \node[draw, fill=black] {}; & \node[draw, fill=black] {}; & \node[overlay, draw, fill=black!50, yshift=-5mm] {}; & \node[draw] {};\\
      \node[draw, fill=black!50] {}; & \node[draw, fill=black!50] {}; &  \node[overlay, draw, yshift=-5mm] {};& \\
      \node[draw] {}; & \node[draw] {}; & & \\[7mm]
      \node (P1a) {P\textsubscript{1}}; & \node (P1b) {P\textsubscript{1}}; & \node (P2) {P\textsubscript{2}}; & \node (P3) {P\textsubscript{3}};\\
    };
      \draw[-{Triangle[]}] (P1a.north) -- ++(0,5mm);
      \draw[-{Triangle[]}] (P1b.north) -- ++(0,5mm);
      \draw[-{Triangle[]}] (P2.north)  -- ++(0,5mm);
      \draw[-{Triangle[]}] (P3.north)  -- ++(0,5mm);
      \draw[-{Triangle[]}] (matrix.south west) -- (matrix.south east) node [midway, below] {space};
      \draw[-{Triangle[]}] (matrix.south west) -- (matrix.north west) node [midway, sloped, above] {time};
  \matrix (legend) [right=1cm of matrix, nodes={minimum width=4mm, minimum height=4mm}] %%5mm has weird effects on vertical spacing of the boxes
    {
       & \node{Stage}; & \node{Targets/triggers};\\
      \node[draw] {}; & \node {A}; & \node{X};\\
      \node[draw, fill=black!50] {}; & \node {B}; & \node {X, Y};\\
      \node[draw, fill=black] {}; & \node {C}; & \node{X, Y, Z};\\
    }; 
  \end{tikzpicture}
\caption{Rule generalization in time and space\label{fig:2.2}}
\end{figure}

Consider first column A, which illustrates how a rule (R) spreads temporally: R is phonologized in a particular place (P\textsubscript{1}) for a certain set of targets and/or triggers which are defined as Stage A (white square). At some point in the future (gray square) R generalizes in P\textsubscript{1} to include more target segments and/or more triggers (Stage B), and then at a later point R is generalized in P\textsubscript{1} further (black square) to attain Stage C.

Now consider column B, which depicts R in time (vertically) and in space (horizontally). As in column A, R is phonologized in column B in a particular place (P\textsubscript{1}) for targets and/or triggers defined as Stage A (white square), and at a later point in time R generalizes its targets and/or triggers to attain Stage B in P\textsubscript{1} (gray square). Sometime after R has been active in P\textsubscript{1} at Stage 1 R also spreads geographically by phonologizing in a neighboring place (P\textsubscript{2}; white square). When R is phonologized in P\textsubscript{2} its targets and/or triggers are defined narrowly as Stage A (white square). At the top of column B it can be seen that R generalizes further in P\textsubscript{1} to attain Stage C (black square) and that R also spreads temporally in P\textsubscript{2} by attaining the targets and/or triggers representing Stage B (gray square). At around the same time, R phonologizes with the narrow set of targets/triggers (white square) in a third place (P\textsubscript{3}).

The examples presented above and throughout this book should make it clear that \isi{rule generalization} is well-attested in Germanic. However, the literature is also clear that the same phenomenon can be found in other language families. Consider the history of \ili{Romance}. An early study documenting \isi{rule generalization} in that language family is \citet{Foley1975}, who writes (p. 47): “Many rules which apply in only a restricted environment in \ili{Latin} apply with less restriction in Romanceˮ. For example, Foley observes that the rule of Nasalization in \ili{Latin} (/VN/→[\~{V}]) applied only before continuants but that the same process in \ili{French} is triggered by a following continuant or stop. His remaining five examples involve \isi{Syncope}, Vowel Shortening, Vocalization, and Vowel Lowering which all applied in \ili{Latin} in specific contexts and were then generalized to applying in a broader set of contexts in modern \ili{Romance} languages.

A more recent case study involving \isi{rule generalization} in the history of \ili{Romance} is discussed by \citet{Ramsammy2015}. His example concerns the context for \is{Velarization (Spanish)}Velarization -- the \isi{neutralization} of place contrasts to [ŋ] in word-final position -- in modern varieties of \ili{Spanish}. In some dialects (e.g. \ili{Peninsular Spanish}, \ili{Cuban Spanish}), \is{Velarization (Spanish)}Velarization only applies prepausally and prevocalically. However, in other dialects (Caracas dialect of \ili{Venezuelan Spanish}) \is{Velarization (Spanish)}Velarization also occurs before a consonant. Ramsammy’s data reveal that within the latter variety there are three distinct patterns which depend on the place of articulation of the consonant following the nasal. To simplify, the three synchronic patterns support a diachronic trajectory with \isi{rule generalization}: The first stage is the avoidance of a [dorsal] consonant, the second the avoidance of a [dorsal] or a [labial] consonant, and the third stage the avoidance of a [dorsal], [labial], or [coronal] consonant.

\subsection{Opacity (part 2): Neutral vowels}\label{sec:2.4.2}\is{Neutral vowels|(}

\isi{Neutral vowels} are defined as phonetically front (coronal) vowels that do not behave phonologically as coronal. The term “neutral” is taken from the literature on \isi{Vowel Harmony} (e.g. \citealt{vanderHulstWeijer1995}), although my usage of the term is not exactly the same as the usage of the term in that literature. For clarity, front vowels that behave phonologically as coronal are referred to below as \isi{nonneutral vowels}.

Front \isi{nonneutral vowels} are represented with the feature [coronal], as in \REF{ex:2:5}. That structure is repeated in \REF{ex:2:30a} with the addition of [αF], which is intended to indicate the presence of other distinctive features (e.g. [±high], [±low]). The structure in \REF{ex:2:30a} contrasts with the one in \REF{ex:2:30b} for \isi{neutral vowels}. It can be seen that \REF{ex:2:30b} is a vowel marked for major class features and other nonplace distinctive features ([αF]) but not for place features. Vowels with that representation cannot behave phonologically like front vowels because they lack the feature [coronal]. Since back vowels bear at least one place feature (e.g. [dorsal]), the structure in \REF{ex:2:30b} cannot be interpreted as a phonologically back vowel. \REF{ex:2:30b} is also distinct from \isi{schwa} (/ə/), which is only marked for the two major class features (recall \ref{ex:2:7}).

%
\ea \label{ex:2:30}
\begin{multicols}{2} 
\ea
\label{ex:2:30a} Front \isi{nonneutral vowel}\\
  \begin{forest}
   [\avm{[−cons\\+son\\αF]} [{[\textsc{coronal}]}]]
  \end{forest}

\ex
\label{ex:2:30b} Neutral vowel\smallskip\\
  \avm{[−cons\\+son\\αF]}
\z
\end{multicols}
\z

An example of a non-Gmc language with a \isi{neutral vowel} that contrasts with a \isi{nonneutral vowel} comes from \ili{Inuit} dialects spoken in Alaska described by \citet[166--167]{Dresher2009}, although Dresher does not use the terms “neutral vowelˮ or “\isi{nonneutral vowel}ˮ. Dresher draws a distinction between two kinds of /i/, which he refers to as “strong iˮ (<\textsuperscript{+}/i/) and “weak iˮ (<\textsuperscript{+}/ə/). In \ili{North Alaskan Inupiaq}, strong /i/ triggers the Palatalization of alveolar consonants, but weak /i/ does not. The contrast between these two /i/ sounds is illustrated in \REF{ex:2:31}. The suffixes in \REF{ex:2:31a} have an initial alveolar consonant (/l n t/) following a stem ending in the vowel /u/. The suffixes in \REF{ex:2:31b} show the effects of a rule (Palatalization) changing a suffix-initial consonant following strong /i/. Note that Palatalization involves the change from /l n t/ to [ʎ ɲ s]. The examples in \REF{ex:2:31b} can be contrasted with the ones in \REF{ex:2:31c}, which illustrate that Palatalization does not occur after the weak /i/. Hence, /l n t/ surface after weak /i/ as [l n t] without change. Weak /i/ can therefore be thought of as opaque because Palatalization underapplies after that sound.

\begin{exe}
\ex%31
\label{ex:2:31}
\TabPositions{0pt, .125\linewidth, .3\linewidth, .45\linewidth, .6\linewidth, .75\linewidth, .9\linewidth}
\begin{xlist}
\exi{}  {Stem} \tab  {Gloss} \tab  {‘and a N’} \tab  {‘N plural’} \tab  {‘like a N’}
\ex \label{ex:2:31a} iɡlu \tab ‘house’ \tab iɡlulu \tab iɡlunik \tab iɡlutun
\ex \label{ex:2:31b} iki  \tab ‘wound’ \tab ikiʎu  \tab ikiɲik  \tab ikisun
\ex \label{ex:2:31c} ini  \tab ‘place’ \tab inilu  \tab ininik  \tab initun
\end{xlist}
\end{exe}

There are additional arguments supporting the distinction between the two kinds of /i/ vowels. First, only the weak /i/ changes to [ɑ] before another vowel, but strong /i/ does not. Second, only weak /i/ alternates with [u] and with zero (i.e. it syncopates).

In present terms, strong /i/ in \REF{ex:2:31b} has the nonneutral representation in \REF{ex:2:30a}, and weak /i/ is a \isi{neutral vowel} with the representation in \REF{ex:2:30b}. This analysis is essentially the same as the one proposed by \citet[166]{Dresher2009}, who analyzes strong /i/ as [coronal] and weak /i/ as “ … not coronalˮ. Since Palatalization only applies after a [coronal] vowel, only the stem-final /i/ in \REF{ex:2:31b} will trigger the change, but not the stem-final /i/ in \REF{ex:2:31c}. The discussion of neutral and \isi{nonneutral vowels} is summarized in \tabref{tab:2:32}.

\begin{table}%32
\caption{\label{tab:2:32}Two properties of neutral and \isi{nonneutral vowels}}
\begin{tabular}{lcc}
\lsptoprule
 & [coronal] present & Historically back\\\midrule
Neutral vowel (=\ref{ex:2:30b}) & no & yes\\
Non\isi{neutral vowel} (=\ref{ex:2:30a}) & yes & no\\
\lspbottomrule
\end{tabular}
\end{table} 

Two \isi{velar fronting islands} of Switzerland (\il{Highest Alemannic}HstAlmc) are discussed in \chapref{sec:6} with neutral vowels exhibiting the two properties in \tabref{tab:2:32}. As noted briefly in \sectref{sec:1.4.1}, in one of those dialects, /x/ undergoes velar fronting after /ei/ but not after /øi/. I argue that the /i/ in the former diphthong has the nonneutral representation in \REF{ex:2:30a} but the /i/ in opaque diphthongs like /øi/ has the neutral structure in \REF{ex:2:30b}. Since /øi/ derived historically from the back vowel /ou/, examples in which /x/ surfaces without change as velar after /øi/ exhibit \isi{underapplication} in the sense that velar fronting is counterfed historically by the rule that restructured /ou/ to /øi/.\is{Neutral vowels|)}

\subsection{Opacity (part 3): Quasi-phonemicization and phonemicization}\label{sec:2.4.3}\is{phonemicization|(}
\begin{sloppypar}
A number of scholars in the traditional literature on historical linguistics have observed that the elimination of the trigger for a rule creating an allophone [A] from the phoneme /B/ can cause the original allophone [A] to become the phoneme /A/, which then contrasts with /B/, e.g. \citet{Hoenigswald1960}, \citet{Hock1986}. That type of change (\textsc{phonemic} \textsc{split}) is depicted in \REF{ex:2:33}:
\end{sloppypar}

\ea%33
    \label{ex:2:33}
     Phonemic split\\
    \begin{tikzpicture}
      \matrix (matrix) [matrix of nodes, nodes in empty cells, ampersand replacement=\&]
        {
          /\textsc{B}/    \&                 \& > \&  /\textsc{B}/    \& /\textsc{A/}\\[5mm]
          {[\textsc{B}]}  \&  {[\textsc{A}]} \&   \& {[\textsc{B}]}   \& {[\textsc{A}]}\\
        };
       \draw (matrix-1-1.south) -- (matrix-2-1);
       \draw (matrix-1-1.south) -- (matrix-2-2);
       \draw (matrix-1-4.south) -- (matrix-2-4);
       \draw (matrix-1-5.south) -- (matrix-2-5);
    \end{tikzpicture}
\z 

One of the most celebrated examples of \REF{ex:2:33} is the historical rule of \textsc{\isi{i-Umlaut}} in German, a process that fronted back vowels before high front vocoids ([i] or [j]) in the following syllable (e.g. \citealt{Twaddell1938}, \citealt{Penzl1949}, \citealt{Becker1967}, \citealt{King1969}, \citealt{Buccini1992}, \citealt{DavisIverson1995}, \citealt{DavisSalmons1999}, \citealt{Fulk2018}). At an earlier stage (\ili{OHG}), back vowels had front vowel allophones before sounds like [i] (/i/), but at a later stage (\ili{MHG}) that front \isi{vocoid} trigger was reduced to \isi{schwa} ([ə] /ə/) by a change I call \isi{Vowel Reduction}. The latter change triggered the \isi{phonemicization} of the original front vowel allophones, e.g. OHG [hyːti] ‘skin-\textsc{pl}’ from /huːt-i/ (cf. OHG [huːt] /huːt/ ‘skin-\textsc{sg}’) > MHG [hyːtə] from /hyːt-ə/. Significantly, the new front vowel phonemes contrasted with the corresponding back vowels before \isi{schwa}, cf. MHG [kruːtə] ‘herb-\textsc{dat}.\textsc{sg}’ from /kruːt-ə/. This example illustrates the historical \isi{overapplication} of \isi{i-Umlaut} in examples like [hyːtə], which is counterbled by \isi{Vowel Reduction} (recall the synchronic example in \ref{ex:2:12b}ii).

 In this book I show how \REF{ex:2:33} can be applied to the historical fronting of velars. The relevant stages are depicted in \REF{ex:2:34}, where \textsc{Ve} and \textsc{Pa} represent velar and palatal respectively. It is assumed here that Stage 2 -- characterized by the presence of velar and palatal allophones -- was preceded by a stage without the palatal allophone (Stage 1), although assumption is not crucial for the discussion below.

\ea%34
    \label{ex:2:34}
    Three stages for velar fronting\\
    \begin{tikzpicture}
      \matrix (matrix) [matrix of nodes, nodes in empty cells, ampersand replacement=\&]
        {
             Stage 1      \&     \&              Stage 2           \&   \&          \&     Stage 3\\
           /\textsc{Ve}/  \&  >  \& /\textsc{Ve}/ \&               \& > \& /\textsc{Ve}/ \& /\textsc{Pa}/\\[5mm]
           {[\textsc{Ve}]}  \&     \& {[\textsc{Ve}]} \& {[\textsc{Pa}]} \&   \& {[\textsc{Ve}]} \& {[\textsc{Pa}]}\\
        };
       \draw (matrix-2-1.south) -- (matrix-3-1);
       \draw (matrix-2-3.south) -- (matrix-3-3);
       \draw (matrix-2-3.south) -- (matrix-3-4);
       \draw (matrix-2-6.south) -- (matrix-3-6);
       \draw (matrix-2-7.south) -- (matrix-3-7);
    \end{tikzpicture}
\z

The development in \REF{ex:2:34} needs to distinguish two types of palatals at Stage 3: In one type of system there was a \isi{phonemic split}, as in (\ref{ex:2:33}), involving the creation of /\textsc{Pa}/ from the Stage 2 palatal allophone [\textsc{Pa}]. That split led to a contrast between the velar and the corresponding palatal (e.g. /x/ vs. /ç/); hence, the velar and palatal occurred in the context of the same vowel. For example, in the \il{Central Hessian}CHes dialects discussed in \chapref{sec:9}, [x] contrasts with [ç] after [ɑ] in [dɑx] (/dɑx/) ‘roof’ vs. [dɑç] (/dɑç/) ‘dike’. The /ç/ in that type of example is referred to in the following chapters as a \textsc{phonemic} \textsc{palatal,} and the change leading to that segment is called \textsc{Phonemicization}.

Phonemic palatals (e.g. /ç/) have an opaque history because they can stand next to a back sound that was originally front. For example, in the \il{Central Hessian}CHes varieties referred to in the preceding paragraph, the [ç] surfacing after [ɑ] (e.g. in [dɑç] /dɑç/ ‘dike’)  derived historically from the front vowel [iː], e.g. [dɑç] /dɑç/ <  [diːç] /diːx/. Since the front trigger for the original rule of velar fronting (/iː/) is no longer present, the \isi{phonemicization} of palatals involves the historical \isi{overapplication} of velar fronting, which is counterbled by the historical change eliminating the front vowel trigger, namely /iː/ > /ɑ/;  recall (\ref{ex:2:12b}ii).

The change affecting the palatal allophone at Stage 2 in \REF{ex:2:34} can also lead to a different type of system, namely one with a \textsc{palatal} \textsc{quasi-phoneme} (depicted in \ref{ex:2:34} as /\textsc{Pa}/ at Stage 3).\footnote{{Quasi-phonemes play a prominent role in the treatment of German \isi{i-Umlaut} proposed by \citet{Kiparsky2015}, although his definition of quasi-phonemes is not quite the same as the one adopted in the present book; see \sectref{sec:7.4.4} for discussion.}} Palatal quasi-phonemes were described briefly in \sectref{sec:1.4.1}: In that type of system palatals (e.g. [ç]) occur in the context of front vowels and in the context of some back vowels that were historically front (referred to here as [Bk]), but velars (e.g. [x]) surface in the context of all back sounds with the exception of [Bk]. Palatal ([ç]) and velar ([x]) do not contrast because they stand in complementary distribution. All instances of palatals in the context of [Bk] are quasi-phonemes (/ç/).

Palatal quasi-phonemes always have an opaque history because they are situated next to a sound ([Bk]) that was once front. For example, in a \il{North Hessian}NHes dialect discussed in \chapref{sec:7}, [ç] occurs after front vowels (e.g. [liçt] /lixt/ ‘light’) or after [ɑː] (e.g. [ʃlɑːçt] ‘bad’) and [x] after back vowels other than [ɑː] (e.g. [bux] /bux/ ‘book’). Significantly, [ɑː] (/ɑː/) derived from earlier [ɛ] (/ɛ/). The vocalic change (/ɛ/ > /ɑː/) counterbled the historical process of velar fronting; recall (\ref{ex:2:12b}ii).

From the synchronic perspective there are two palatal categories: (a) \textsc{underlying} \textsc{palatals} (e.g. /ç/) and (b) \textsc{derived} \textsc{palatals} (e.g. [ç] from /x/). Two examples illustrating (a) were discussed above, namely phonemic palatals and palatal quasi-phonemes, although a third variant is discussed below. One example exemplifying (b) was referred to above, namely allophonic palatals (depicted at Stage 2 in \ref{ex:2:33}). However, there are two other kinds of \isi{derived palatals} that I discuss. Consider first systems with phonemic palatals. In the \il{Central Hessian}CHes varieties referred to above (from \chapref{sec:9}), [x] (/x/) and [ç] (/ç/) contrast after back vowels like [ɑ] in words like [dɑx] (/dɑx/) ‘roof’ vs. [dɑç] (/dɑç/) ‘dike’, but after front vowels only [ç] occurs. Since [ç] in the front vowel context is predictably palatal (because it does not contrast with [x] in that environment), it is derived from the velar /x/, e.g. [ʃlɛçt] ‘bad’ is /ʃlɛxt/. In that type of example, the [ç] in [ʃlɛçt] ‘bad’ is derived synchronically by velar fronting, which functions not as an allophonic rule, but instead as a \textsc{neutralization}. In systems with palatal quasi-phonemes there are likewise many examples exhibiting category (b) which are not allophonic palatals. In the \il{North Hessian}NHes dialect with palatal quasi-phonemes described above (from \chapref{sec:7}), palatals occurring in the front vowel context are derived by velar fronting (e.g. the word [liçt] ‘light’ is underlyingly /lixt/). However, velar fronting is not an allophonic operation because the same system has underlying palatals (/ç/) after [Bk] segments (e.g. in [ʃlɑːçt] ‘bad’), nor is velar fronting a \isi{neutralization} because there is no velar vs. palatal contrast. Instead, velar fronting in the type of system just described functions as a \textsc{quasi-neutralization}.

The overwhelming number of palatals investigated here were etymological velars. First, there are those palatals described above that continue to be derived from velars in the synchronic phonology (\textsc{derived} \textsc{palatals}). Second, there are underlying palatals (palatal quasi-phonemes, or phonemic palatals). But a third type of underlying palatal needs to be distinguished as well, namely the lenis fricative [ʝ] in words like [ʝɑː] ‘yes’, which is referred to below as the \textsc{etymological} \textsc{palatal}. That segment is different from all of the other types of palatals discussed above because of its unique history: The \isi{etymological palatal} derived from the homorganic (palatal) glide [j] (/j/) by a change referred to below as \isi{Glide Hardening}, e.g. [ʝɑː] /ʝɑː/ ‘yes’ < [jɑː] /jɑː/. The \isi{etymological palatal} [ʝ] can occur in the context of a back vowel; however, that type of [ʝ] does not reflect \isi{opacity} (\isi{overapplication}) because it never derived from a velar.

In \tabref{tab:2:35} I summarize the four kinds of palatals discussed above. Any type of palatal can belong to the first three categories (a-c), but category (d) is always either the lenis palatal fricative /ʝ/ or the fortis palatal fricative /ç/ if that sound derived historically from /ʝ/, e.g. [çɑː] /çɑː/ ‘yes’ < [ʝɑː] /ʝɑː/ < [jɑː] /jɑː/.

\begin{table}%35
\caption{\label{tab:2:35}Four types of palatals}
\begin{tabular}{l@{ }lcc}
\lsptoprule
   &         & Contrasts with  & Opaque history\\
   &         &  velar          & (counterbleeding)\\\midrule
a. & \isi{Derived palatal} (e.g. [ç] from /x/) & no     & no\\
b. & Phonemic palatal (e.g. /ç/)         & yes    & yes\\
c. & Palatal quasi-phoneme (e.g. /ç/)    & no     & yes\\
d. & Etymological palatal (e.g. /ʝ/)     & yes/no & no\\
\lspbottomrule
\end{tabular}
\end{table}

\isi{Derived palatals} (=\tabref{tab:2:35}a) do not contrast with the corresponding velars in the context for fronting. This is clearly the case for allophonic palatals because those palatals stand in complementary distribution with the corresponding velars, but it is also true for palatals derived by neutralizations or quasi-neutralizations. In the latter type of system there is no contrast at all between velar and palatal, which stand in complementary distribution. In the case of neutralizations there is a contrast between velar and palatal, although that contrast is virtually always in the context of one or more back vowel, but in the context of front vowels, only palatals surface. \isi{Derived palatals} have a distribution that is transparent because they are the modern reflexes of earlier velars that fronted in the front vowel context. That historical rule of velar fronting is therefore still present as a synchronic process.

Phonemic palatals (=\tabref{tab:2:35}b) always contrast with the corresponding velars in at least one context. Phonemic palatals have an opaque history. This point can be illustrated in dialects like the one described above (from \chapref{sec:9}), in which those sounds arose due to the elimination of the front vowel trigger for the original palatal allophone, e.g. [dɑç] (/dɑç/) ‘dike’ < [diːç] (/diːx/). In that type of example, the historical rule of velar fronting overapplied because it is counterbled by a vocalic change (/iː/ > /ɑ/).

Palatal quasi-phonemes (=\tabref{tab:2:35}c) never contrast with the corresponding velars. Like phonemic palatals, they have an opaque history which always involves \isi{overapplication}. Consider once again the dialect described above (from \chapref{sec:7}) in which [ç] surfaces after front vowels [liçt] /lixt/ ‘light’) or after [ɑː] (e.g. [ʃlɑːçt] ‘bad’) and [x] after back vowels other than [ɑː] (e.g. [bux] /bux/ ‘book’). In that example the historical rule of velar fronting overapplied because it is counterbled by a vocalic change (/ɛ/ > /ɑː/).

The \isi{etymological palatal} (=\tabref{tab:2:35}d) does not have an opaque history because it was never a velar. In some dialects that palatal does not contrast with the corresponding velar because that historical velar is no longer present. For example, in a dialect discussed in \chapref{sec:4}, the original glide [j] (/j/) underwent \isi{Glide Hardening} in [ʝʊŋə] (/ʝʊŋə/) ‘boy’, but the original [ɣ] (/ɣ/) is now fortis [x] (/x/) in [xʊnst] (/xʊnst/) ‘favor’; thus, the dialect does not contrast /ʝ/ and /ɣ/ because the latter sound does not occur word-initially. However, in other varieties the \isi{etymological palatal} can contrast with the corresponding velar. This is the case in which a velar ([ɣ]) and the \isi{etymological palatal} ([ʝ]) surface in the context of the same back vowels, e.g. in the pair [ɣɑt] (/ɣɑt/) ‘hole’ vs. [ʝɑː] (/ʝɑː/) ‘yes’ present in a dialect discussed in \chapref{sec:8}. Since the \isi{etymological palatal} contrasts with the corresponding velar in that type of system, the former sound is also a \isi{phonemic palatal}.\is{phonemicization|)}

\subsection{Polycausality in language change }\label{sec:2.4.4}

There is little question that velar fronting has a structural (phonological) motivation, a point that is hardly controversial when one examines the literature on this topic cited earlier. The structural reason for the fronting of velars is clear when one considers the targets and triggers for the process: A back sound (velar) is realized as palatal in the neighborhood of a front segment. Any assimilatory process like this one has a structural cause, which is captured in the present framework with the spreading of the feature [coronal]; recall \REF{ex:2:6}.

This point aside, it has to be acknowledged that there may be nonstructural (social) factors that also contribute to the fronting of velars in the neighborhood of front sounds. For example, in a set of dialects discussed in \chapref{sec:11} once spoken in the eastern part of pre-1945 Germany some of the palatal sounds created by velar fronting (e.g. the palatal stop [c]) also occurred in (non-Germanic) loanwords acquired from (\ili{Slavic}) languages spoken in the direct vicinity of the German dialects with the palatal stops in question. In that chapter I conclude that contact with non-Germanic languages in the form of \ili{Slavic} loanwords with palatal sounds probably went hand-in-hand with velar fronting. This scenario suggests that the proper explanation of velar fronting for those varieties of German needs to take social factors (language contact) into account, in addition to structural (phonological) ones. The term I use to describe this state of affairs is \textsc{polycausality}.

The connection between social and the structural factors with respect to velar fronting is mostly unexplored, although some reference to social factors can be found the literature I refer to below. For example, in one study I cite in \sectref{sec:12.3.2} a connection is fleetingly mentioned between the choice of triggers for postsonorant velar fronting and religious affiliation (Catholics vs. Protestants). By contrast, one issue that has received considerable attention in recent years is \isi{alveolopalatalization} as a marker for various \is{ethnolect}ethnolects (\sectref{sec:10.6.1}).

It is not difficult to find parallel cases involving \isi{polycausality} in the literature on language change. One well-known example that comes to mind is the \isi{phonemicization} of lenis (voiced) fricatives in the history of \ili{English} ([v z ð]); see \sectref{sec:8.6.1} and \sectref{sec:11.9.2}. The literature on this topic is in agreement that that the structural (phonological) reason for the change from fortis to lenis was reinforced by the occurrence of \ili{French} loanwords with those sounds.

Aside from the dialects referred to above the sources cited in this book do not present evidence that social factors play a role in velar fronting. For this reason, \isi{polycausality} is a topic that only plays a minor role in this book.

\section{The historical model}\label{sec:2.5}

\subsection{Introduction}\label{sec:2.5.Introduction}

The case studies presented in the ensuing chapters reveal that velar fronting can differ synchronically from dialect to dialect in terms of  targets and triggers and in terms of the presence or absence of \isi{opacity}. I argue that those synchronic areal differences reflect the ways in which the originally phonetically-induced fronting of velars was phonologized and then gradually became embedded into the grammar of individual dialects. In particular, the German dialects support a model in which velar fronting exhibits the \textsc{life} \textsc{cycle} depicted in \figref{fig:2.3}. The claim that rules can have a life cycle is well-established, although the individual models differ from author to author, e.g. \citet{BaudouindeCourtenay1895}, \citet{Hyman1976}, \citet{Dressler1976}, \citet{Kiparsky1995}, \citet{Bermúdez-Otero2007}, \citet{Roberts2012}, \citet{Hyman2013}, \citet{Kiparsky2015}, \citet{Bermúdez-Otero2015}, \citet{Ramsammy2015}, \citet{Sen2016}, \citet{Turton2017}, and \citet{Hinskens2021}.

\begin{figure}
\begin{tabular}{cl}
       Stage 3: & Opaque (velar fronting under- or overapplies)\\
             ↑  & \\
       $\left\{\begin{tabular}{@{}c@{}}
                Stage 2n(n)\\
                    ↑      \\
                Stage 2a(a)\\
                \end{tabular}\right\}$  & Allophonic (velar fronting is transparent)\\
             ↑  & \\
        Stage 1: & Phonetic (\isi{coarticulation})\\
\end{tabular}
\caption{The life cycle of velar fronting\label{fig:2.3}}
\end{figure}

Phonologization (not depicted above) refers to the change from a \isi{gradient} phonetic process (Stage 1) to a categorical phonological rule producing palatal allophones (Stage 2). I describe the subscripts for Stage 2 indicated in \figref{fig:2.3} in \sectref{sec:2.5.2}.

The focus of the present work falls squarely on Stage 2 and Stage 3. I make first some brief remarks on Stage 1.

\subsection{Stage 1}

A number of linguists have observed that it is common for a velar like /k/ to be articulated in a slightly more forward position along the palate in the neighborhood of front vowels than in the neighborhood of back vowels. An early study in which this issue is mentioned is \citet[52]{Sapir1921}, who compared the realization of \ili{English} /k/ in \textit{keep} and \textit{cool}. The coarticulatory fronting of velars like /k/ in the context of front vowels (especially /i/) represents Stage 1. That the fronting described here is \isi{gradient} is the conclusion drawn by \citet{KeatingLahiri1993}, who consider articulatory and acoustic data involving the realization of velars in the neighborhood of various vowels in \ili{English} and other languages.

I claim that there was an earlier point in the history of Germanic when velar fronting was not present (Stage 1). At that time, velar sounds like /x/ succumbed to coarticulatory fronting in the context of one or more front vowel, which probably served as the impetus for Stage 2. I refer henceforth to this phonetically fronted velar as a \textsc{\isi{prevelar}} and represent it in a narrow phonetic transcription with the IPA diacritic for an advanced articulation, e.g. [k̟] for a \isi{prevelar} stop and [x̟] for a \isi{prevelar} fricative; see \sectref{sec:12.9.1} for some discussion of \isi{prevelars}.

It is not possible to provide evidence for coarticulatory velar fronting (\isi{prevelars}) in the broad spectrum of German dialects I investigate because the sources for those dialects do not provide that information.

\subsection{Stage 2}\label{sec:2.5.2}

The difference between phonetically fronted velars (\isi{prevelars}) in the context of front vowels and plain velars in the context of back vowels (Stage 1) is eventually exaggerated to the point where speakers perceive of the two consonants as different sounds. At that point (Stage 2) those two sounds are realized as palatal (e.g. ich-Laut) and velar (e.g. ach{}-Laut).

Since velars and palatals do not contrast at Stage 2, those segments stand in complementary distribution; thus, [ç] and [x] are allophones of the phoneme /x/, whose realization as palatal is expressed formally with a specific version of velar fronting (recall \ref{ex:2:6}). Hence, \isi{phonologization} (Stage 2) can be thought of as \is{rule addition}\textsc{rule addition} (\citealt{King1969, RingeEska2013}) in the sense that velar fronting is present in the Phonology component (\tabref{tab:fromfig:representationallevels}) but was absent in the Phonology component at Stage 1. Once in the grammar at Stage 2 that synchronic process remains active until it is modified in light of the various changes involving triggers and targets discussed below.

Stage 2 consists of a series of stages expressed with the subscripts in \figref{fig:2.3}, i.e. (a) and (n). These incremental steps are intended to reflect the \isi{rule generalization} model described above (\figref{fig:2.2}): The newly phonologized rule of velar fronting gradually incorporates a greater number of targets and/or triggers and when it does so, it enters into the immediately following stage. \figref{fig:2.4} illustrates how the set of triggers for postsonorant velar fronting grows from Stage 2a (high front vowels, represented here as /i/) to Stage 2b (nonlow front vowels, represented by /i/ and /e/). The vowel /ɑ/ represents all back vowels. Stage 2n represents the point with a broader set of triggers than the nonlow front vowels (see \chapref{sec:12}). Note that the output of velar fronting ([ç]) throughout Stage 2 is the \isi{derived palatal} from \tabref{tab:2:35}(a).

\begin{figure}
  Stage 2n:    /ix/ [iç], /ex/ [eç] …. /ɑx/ [ɑx]\\
       ↑\\
  Stage 2b:    /ix/ [iç], /ex/ [eç] …. /ɑx/ [ɑx]\\
       ↑\\
  Stage 2a:    /ix/ [iç], /ex/ [ex] …. /ɑx/ [ɑx]\\
                  ↑\\
  Stage 1:  /ix/ [ix], /ex/ [ex] …. /ɑx/ [ɑx]
\caption{\label{fig:2.4}Stage 1 and Stage 2 (for triggers) in the life cycle of velar fronting}
\end{figure}

Synchronically the rule of velar fronting at Stage 2a spreads the frontness feature ([coronal]) from a high front vowel ([−consonantal, +high, coronal]) to a velar target (/x/). At Stage 2b that rule is broadened, so that [coronal] spreads from a nonlow front vowel ([−consonantal,     −low, coronal]).

\figref{fig:2.5} shows that the initial target for postsonorant velar fronting (Stage 2aa) is fortis /x/ and that at a later point (Stage 2bb) the lenis counterpart /ɣ/ is incorporated as a target segment as well. Stage 2nn refers to the point when additional velar consonants serve as triggers (e.g. /k/). Dialects with the broadest set of targets are discussed in detail in \chapref{sec:11}.

\begin{figure}
  Stage 2nn:    /ix/ [iç], /iɣ/ [iʝ] ….\\
       ↑\\
  Stage 2bb:    /ix/ [iç], /iɣ/ [iʝ] ….\\
       ↑\\
  Stage 2aa:    /ix/ [iç], /iɣ/ [iɣ] ….\\
                  ↑\\
  Stage 1:    /ix/ [ix], /iɣ/ [iɣ] ….
\caption{\label{fig:2.5}Stage 1 and Stage 2 (for targets) in the life cycle of velar fronting}
\end{figure}

From the synchronic perspective the rule of velar fronting at Stage 2aa had a target defined as the features for /x/ ([+consonantal, −sonorant, +continuant, +fortis, dorsal]), but that set of features is modified for the synchronic rule at Stage 2bb ([+consonantal, −sonorant, +continuant, dorsal]).

The changes from a narrow to broad set of triggers (\figref{fig:2.4}) and targets (\figref{fig:2.5}) need not match up. Put differently, velar fronting is phonologized at Stage 2a for triggers and Stage 2aa for targets, but some dialects extend the set of triggers at a faster rate than the set of targets. This accounts for the fact that many varieties of HG/LG are attested with the narrowest set of targets (/x/) but with the broadest set of triggers (coronal sonorants); see \chapref{sec:12}.

Data documenting the gradual spread from specific to general targets/triggers (Figures~\ref{fig:2.4} and \ref{fig:2.5}) along the time dimension alone are not known to me. That type of evidence would consist of a description of velar fronting in a specific place at a specific time as well as a parallel set of data from the same place but at a later point in time. Since I am not aware of such longitudinal studies I focus instead on the place dimension. That type of evidence consists of a comparison of the description of velar fronting in one place with data in a neighboring place where both dialects were spoken at roughly the same general time frame. By comparing dialects spoken in different places at approximately the same time it is possible to draw conclusions on how velar fronting progressed along the temporal dimension.

As described in \sectref{sec:2.4.1} the gradual spread of triggers and targets as depicted in Figures~\ref{fig:2.4} and~\ref{fig:2.5} occurred in time and space. In that section I discussed how temporal and spatial spreading go hand in hand (\figref{fig:2.2}). The extension of targets and triggers in space is not expressed in \figref{fig:2.3}, which only indicates the time dimension.

\figref{fig:2.3} does not mean to suggest that the originally \isi{gradient} fronting of velars (Stage 1) is simply replaced by the categorical rule of velar fronting (Stage 2). Instead, it is conceivable that Stage 2 has both a phonological fronting of velars which was the outgrowth of an earlier coarticulatory fronting still present in the same dialect. For example, it might be the case that /x/ once underwent coarticulatory fronting (Stage 1). Later on velar fronting was phonologized with /x/ as the sole target segment (Stage 2), but for those same speakers other velar sounds (e.g. /k/) continued to undergo \isi{gradient} fronting (Stage 1). Alternatively, the \isi{gradient} fronting of /x/ after front vowels (Stage 1) might have been phonologized as the corresponding palatal after front vowels (Stage 2), but word-initially /x/ is still at Stage 1. See \citet{Turton2017}, who shows /l/ can show both \isi{gradient} velarization and categorical velarization in varieties of \ili{English}. Since the descriptions for German dialects on which my analysis is based do not have information on coarticulatory fronting it is not possible to draw conclusions concerning the scenario described above.

A contentious issue in historical linguistics is the locus of sound change. This topic is discussed in several chapters in \citet{HoneyboneSalmons2015a} and summarized in \citet[8--9]{HoneyboneSalmons2015b}. One aspect of this debate involves the relationship between historical change and language \isi{acquisition}. Some linguists (e.g. \citealt{HaleReiss2015}) contend that all change is intergenerational. This means that children derive a grammar which is different from adults. According to this point of view, all language change occurs in \isi{acquisition}. That approach can be contrasted with the one adopted by other scholars; for example, it has alternatively been argued that some (but not all) change occurs in \isi{acquisition}, or that some change can occur within the lifespan of adults.

A related question is the extent to which sound change is driven by the listener, a topic discussed at length in the works of John Ohala (e.g. \citealt{Ohala1981} and numerous subsequent works). The same type of approach is adopted in various theoretical frameworks by many other linguists, including -- but not limited to -- \citet{Holt1997}, \citet{HumeJohnson2001}, and \citet{Hamann2009}. The role of the listener in sound change is also a central claim in the Evolutionary Phonology framework \citep{Blevins2004}. Although the authors cited here do not endorse exactly the same model, they agree that sound change can occur when a listener misperceives sounds uttered by a speaker.

In my treatment of velar fronting I assume a model whereby change is intergenerational and listener-driven. In particular, it involves the interaction between the speaker and the listener in \isi{acquisition}. The way in which original velars like /x/ are misperceived as palatals is described briefly below. This treatment follows closely the source of sound change \citet[32--34]{Blevins2004} refers to as \textsc{change}.

My approach can be applied to \figref{fig:2.4} in the following manner: Stage 1 represents an adult speaker (P\textsubscript{1}) and Stage 2 the child acquiring the language (P\textsubscript{2}). P\textsubscript{1} utters a word with the vowel [i] followed by the dorsal fricative [x] -- realized in Speech (\tabref{tab:fromfig:representationallevels}) as \isi{prevelar} ([x̟]), --  but P\textsubscript{2} hears the palatal fricative [ç] and therefore pronounces the word as [iç]. The change from Stage 1 to Stage 2a therefore involves a subtle pronunciation change due to P\textsubscript{2}’s misperception of a sound uttered by P\textsubscript{1}.

What is more significant than the change in pronunciation from Stage 1 (=P\textsubscript{1}) to Stage 2a (=P\textsubscript{2}) is P\textsubscript{2}’s interpretation of the new sound [ç] as a phonological unit -- the ich-Laut -- and not a phonetic one. This means that P\textsubscript{2} classifies [ç] as a palatal fricative with a unique featural representation  (=\ref{ex:2:2}d). Since that sound has a distribution restricted to the context after [i], P\textsubscript{2} analyzes [ç] as an allophone of the corresponding velar ([x]), which never occurs in that context. This is accomplished by acquiring the phonological rule of velar fronting with the narrowest set of triggers (only before /i/).

The addition of triggers and targets to the newly acquired rule of velar fronting follow the same approach described above. For example, in \figref{fig:2.5}, Stage 2aa represents the grammar of an adult speaker (P\textsubscript{1}) who has phonologized velar fronting for the target /x/ and the trigger /i/. When P\textsubscript{1} utters words with [i] followed by the (\isi{prevelar}) lenis fricative /ɣ/, that sound is misperceived by the acquirer at Stage 2bb (P\textsubscript{2}), who hears the palatal fricative [ʝ] and then treats it as a phonological unit on par with [ç].

In the remainder of this book I make extensive references to the various stages referred to above -- stages which are made more explicit in Chapters~\ref{sec:12} and~\ref{sec:13}. It needs to be stressed that terms like “Stage 1”, “Stage 2a” etc. in Figures~\ref{fig:2.4} and \ref{fig:2.5} are simply a different way of saying that sound change occurs in \isi{acquisition} between speakers and listeners.

When velar fronting is phonologized at Stage 2a\slash Stage 2aa it enters the grammar as a regular rule that has no exceptions; thus, there is no evidence for \textsc{lexical diffusion} (\citealt{ChenWang1975,Kiparsky1995,Phillips2006}).
Evidence for my claim that velar fronting is regular is that the data provided in the original sources give no indication at all that velar fronting is (or that it ever was) irregular.  Hence, velar fronting is a classic example of a \isi{Neogrammarian change}.
From the diachronic perspective, velar fronting is phonologized (acquired) at Stage 2 by the younger generation as a regular sound change.
My assumption that sound change is regular and exceptionless holds not only for velar fronting, but also for the changes that interact with velar fronting which I discuss in ensuing chapters.\footnote{Lexical
 exceptions to velar fronting are not attested in any dialect of German. For discussion see \sectref{sec:2.5.2}, \sectref{sec:12.8.3}, and \sectref{sec:13.5.3}.
}



A major topic discussed in Chapters~\ref{sec:3}--\ref{sec:10} is the way in which velar fronting interacts with synchronic and diachronic processes increasing or decreasing the number of potential targets and/or triggers for velar fronting. \tabref{tab:2.wxyz} lists the four logical possibilities (second column), which are referred to below with the abstract designations in the first column.

\begin{table}
\caption{Rules increasing/decreasing potential targets and/or triggers for velar fronting\label{tab:2.1}
\label{tab:2.wxyz}}
\small
\begin{tabularx}{\textwidth}{lQQ}
\lsptoprule
Abstract rule & Definition & Example\\
\midrule
Rule W & Increases the number of potential velar targets & A non-velar segment is realized as velar in the context for velar fronting\\
Rule X & Increases the number of potential front segment triggers & A back sound is realized as front in the context for velar fronting\\
Rule Y & Decreases the number of potential velar targets & A velar target (e.g. /x/) deletes or converts to another sound in the context for velar fronting\\
Rule Z & Decreases the number of potential front segment triggers & A front trigger (e.g. /i/) deletes or converts to a back sound in the context for velar fronting\\
\lspbottomrule
\end{tabularx}
\end{table}

In Chapters~\ref{sec:3}--\ref{sec:10} I discuss a number of specific examples of synchronic and diachronic processes corresponding to Rule W, X, Y, and/or Z, as defined above. I describe below the most common patterns.

When velar fronting is phonologized (Stage 2) it always interacts transparently with Rules W-Z (if present); hence, velar fronting is added at the end of the grammar and is either fed or bled by Rule W-Z. This means that the palatals produced by the synchronic rule of velar fronting for Stage 2 speakers only occur in the neighborhood of front sounds, and velars in the neighborhood of back sounds. In Figures~\ref{fig:2.6} and~\ref{fig:2.7} I show the transparent interaction of synchronic rules for four hypothetical dialects; specific examples exemplifying those four systems are discussed in Chapters~\ref{sec:3} and~\ref{sec:4}. In Figures~\ref{fig:2.6} and~\ref{fig:2.7} /i e/ and /u o ɑ/ represent front and back vowels respectively.  “/A/” is a cover symbol for any segment other than /x/, and Vel Fr = (postsonorant) velar fronting.\largerpage[-1]

\begin{figure}
\begin{tabular}{llll llll}
         &  \multicolumn{3}{c}{Dialect A}     &         & \multicolumn{3}{c}{Dialect B}\\\cmidrule(lr){2-4}\cmidrule(lr){6-8}
         & /ɑx/        &    /iA/  &   /ix/    &         & /ɑx/       & /ix/      & /ux/\\
  Rule W & {}-{}-{}-   &    ix    & {}-{}-{}- &  Rule X & {}-{}-{}-  & {}-{}-{}- & ix  \\
  Vel Fr & {}-{}-{}-   &    iç    &  iç       &  Vel Fr & {}-{}-{}-  & iç        & iç  \\
         & [ɑx]        &    [iç]  &   [iç]    &         & [ɑx]       & [iç]      & [iç]\\
\end{tabular}
\caption{\label{fig:2.6}Velar fronting fed by Rule W/Rule X in the synchronic phonology}
\end{figure}

Dialect A has  a process converting /A/ into the fortis velar fricative. Since that synchronically derived fricative undergoes velar fronting, the latter process is fed by /A/→[x]. /A/→[x] illustrates Rule W (\tabref{tab:2.wxyz}) because it increases the number of target segments for that process. Dialect B possesses a synchronic rule creating a front vowel from a back vowel. That process exemplifies Rule X (\tabref{tab:2.wxyz}) because it increases the number of triggers for velar fronting. Rule X \isi{feeds} velar fronting because it creates a structure ([i]) to which velar fronting can apply.\footnote{\label{fn:2:25}As indicated in \figref{fig:2.6} the segment undergoing Rule W in Dialect A is present in the underlying representation (/iA/). It some of the case studies discussed below the target for Rule W (i.e. /A/) can itself be synchronically derived from another segment. The same point holds for the back vowel in Dialect B which undergoes Rule X. That back vowel (/u/ in \figref{fig:2.6}) can be present in the underlying representation, or it can alternatively be derived from an independent rule.} \figref{fig:2.7} illustrates a synchronic \isi{bleeding} relationship.

\begin{figure}
\begin{tabular}{*8{l}}
         &  \multicolumn{3}{c}{Dialect C}      &         & \multicolumn{3}{c}{Dialect D} \\\cmidrule(lr){2-4}\cmidrule(lr){6-8}
         &  /ɑx/     &  /ix/       &  /ix-ə/   &         & /ɑx/      & /ix/      & /ex/      \\
  Rule Y & {}-{}-{}- &   {}-{}-{}- &  iAə      & Rule Z  & {}-{}-{}- & {}-{}-{}- & ox        \\
  Vel Fr & {}-{}-{}- &  iç         & {}-{}-{}- & Vel Fr  & {}-{}-{}- &  iç       &  {}-{}-{}-\\
         & [ɑx]      & [iç]        & [iAə]     &         &  [ɑx]     & [iç]      & [ox]      \\
\end{tabular}
\caption{\label{fig:2.7}Velar fronting bled by Rule Y/Rule Z in the synchronic phonology}
\end{figure}

Dialect C has a synchronic process for the example /ix-ə/, which  converts a velar fronting target (/x/) into another sound ([A]) in the context between vowels. The rule /x/→[A] exemplifies Rule Y (\tabref{tab:2.wxyz}) because it decreases the number of target segments for velar fronting. Rule Y in Dialect C therefore \isi{bleeds} velar fronting. In Dialect D there is a rule converting certain triggers for velar fronting (e.g. /e/) into back sounds. That rule (/e/→[o]) exemplifies Rule Z (\tabref{tab:2.wxyz}) because it decreases the number of potential front triggers for velar fronting. In that example Rule Z \isi{bleeds} velar fronting.\largerpage[-1]

The four systems depicted in \figref{fig:2.6} and \figref{fig:2.7} are also attested when Rules W-Z are diachronic processes that restructure underlying representations. \figref{fig:2.8} depicts the two most common diachronic patterns, which are referred to here as Dialect E and Dialect F.

\begin{figure}
\begin{minipage}[b]{.5\linewidth}
  \centering
  \tabcolsep=3pt
  \begin{tabular}{ *{7}{c} }
  \multicolumn{7}{c}{Dialect E}\\\cmidrule(lr){1-7}
   /ɑx/  & /ix/ &  /oux/ & > & /ɑx/  & /ix/ &  /eix/\\\relax
   [ɑx]  & [iç] & [oux]  &   & [ɑx]  & [iç] &  [eiç]\\   
  \end{tabular}
\end{minipage}%
\begin{minipage}[b]{.5\linewidth}
    \centering
    \tabcolsep=3pt
     \begin{tabular}{ *{7}{c} }                               
         \multicolumn{7}{c}{Dialect F}\\\cmidrule(lr){1-7}
         /ɑx/ &  /ix/&  /eix/ &   > & /ɑx/  & /ix/  & /oux/\\\relax
         [ɑx] & [iç] & [eiç]  &     & [ɑx]  & [iç]  & [oux]\\
     \end{tabular}
\end{minipage}
\caption{\label{fig:2.8}Velar fronting fed/bled by Rule X/Rule Z in the diachronic phonology}
\end{figure}

It can be seen in the three examples to the left of the wedge in Dialect E that velar fronting is active synchronically and that the outputs are transparent. Sound change occurs in Dialect E, namely /ou/ > /ei/. That change involves a restructuring of the underlying representation because there are no alternations between those two diphthongs [ou] and [ei] that would motivate treating it as a synchronic rule. The phonetic representations to the right of the wedge reveal that velar fronting is still active as a Stage 2 rule. The significance of this example is that the sound change /ou/ > /ei/ \isi{feeds} velar fronting in the final example. The diachronic change /ou/ > /ei/ exemplifies Rule X (\tabref{tab:2.wxyz}) because it increases the number of triggers for velar fronting. Dialect F illustrates the opposite diachronic change, namely /ei/ > /ou/. The three examples to the left of the wedge show that velar fronting is active synchronically before the restructuring of /ei/. After the restructuring (to the right of the wedge) velar fronting is still active synchronically, but there is one less trigger because /ei/ was eliminated by the sound change /ei/ > /ou/. Since that sound change decreases the number of triggers it exemplifies Rule Z (\tabref{tab:2.wxyz}).

\subsection{Stage 3}
This is a cover term referring to the point when: (a) some velars surface unexpectedly as velars in the context of velar fronting (\isi{underapplication}); or (b) some palatals deriving historically from velars occur unexpectedly in the back vowel context (\isi{overapplication}).

I consider \isi{underapplication} and \isi{overapplication} in order. There are two types of \isi{underapplication} (Stage 3aa and Stage 3ab), which are described here:

\subsubsection{Stage 3aa}
% % % (aa): 
An independent process (Rule W) creates new segments which can potentially undergo velar fronting. Since those new velars fail to undergo velar fronting, the latter process is counterfed by Rule W. In the case studies exemplifying Stage 3aa discussed in \chapref{sec:5} both velar fronting and Rule W are active synchronically. Stage 3aa is depicted in \figref{fig:2.9}.

\begin{figure}
\begin{tabular}{*4{l}}
             & \multicolumn{3}{c}{Dialect G}\\\cmidrule(lr){2-4}
             &   /iA/      & /ix/      & /ɑx/     \\
    Vel Fr   &   {}-{}-{}- & iç        & {}-{}-{}-\\
    Rule W   &    ix       & {}-{}-{}- & {}-{}-{}-\\
             &   [ix]      & [iç]      & [ɑx]     \\
\end{tabular}
\caption{Rule W \isi{counterfeeds} velar fronting in the synchronic phonology\label{fig:2.9}}
\end{figure}

The examples /ix/ and /ɑx/ illustrate that velar fronting is active synchronically, since /x/ is realized as palatal after front vowels like /i/ and as velar after back vowels like /ɑ/. By contrast, the realization of the underlying representation /iA/ as [ix] exemplifies \isi{underapplication} Stage 3aa because Rule W (/A/→[x]) \isi{counterfeeds} velar fronting. Note the difference between Dialect G in \figref{fig:2.9} and Dialect A depicted in \figref{fig:2.6}. In \chapref{sec:5} I discuss opaque systems like Dialect G in \figref{fig:2.9} and show that they developed out of the transparent ones like Dialect A in \figref{fig:2.6}.\footnote{In \figref{fig:2.9} it can be seen that the target segment for Rule W (i.e. /A/) is present in the underlying representation. In one set of dialects discussed in \chapref{sec:5} it is shown that the target for Rule W can itself be synchronically derived from another sound. Recall \fnref{fn:2:25}.}

\subsubsection{Stage 3ab}
% % % (ab): 
A historical process (Rule X) creates new front vowels which can potentially serve as triggers for velar fronting. Since those new front vowels fail to induce velar fronting, the latter process is counterfed historically by Rule X. In the case studies discussed in \chapref{sec:6} illustrating Stage 3ab, Rule X is no longer active synchronically. Instead, it has the effect of restructuring underlying representations for a younger generation of speakers to the ones depicted in \REF{ex:2:30b} for \isi{neutral vowels}. \figref{fig:2.10} illustrates Stage 3ab.

\begin{figure}
\tabcolsep=3pt
\begin{tabular}{*{7}{c}}
                   \multicolumn{7}{c}{Dialect H}\\\cmidrule(lr){1-7}
 \multicolumn{3}{c}{Stage 2}   & &  \multicolumn{3}{c}{Stage 3}\\\cmidrule(lr){1-3}\cmidrule(lr){5-7}
      /oux/  &   /ix/  &   /eix/ &   >   &  /øix/  &  /ix/   &  /eix/\\\relax
      [oux]  &  [iç]   & [eiç]   &       &  [øix]  &  [iç]   &  [eiç]\\
\end{tabular}
\caption{\label{fig:2.10}Rule X \isi{counterfeeds} velar fronting in the diachronic phonology}
\end{figure}

In this example there is a historical process (/ou/ > /øi/) that creates new front vowels that are potential triggers for velar fronting. Since those new front vowels do not \isi{feed} velar fronting, the latter is counterfed by the change from /ou/ to /øi/ (Rule X).

There are two very similar types of \isi{overapplication} (=Stage 3ba and Stage 3bb), which are described in order.

\subsubsection{Stage 3ba}
% % (ba): 
\begin{sloppypar}
A historical process (Rule Z) eliminates triggers for velar fronting, but that change fails to \isi{bleed} velar fronting. An example of Rule Z is the change from any unstressed vowel (including crucially front vowels) to \isi{schwa} (/ə/) in an unstressed syllable. This change is illustrated in \figref{fig:2.11}, which is well-attested in the varieties discussed in \chapref{sec:7}. At Stage 2 velar fronting is active in word-initial position. When /i/ changes to /ə/ the palatal remains even though \isi{schwa} would be expected to be preceded by [x]. Ellipsis (“…ˮ) in the first example at Stage 2 and Stage 3 means that there is a part of the word containing a stressed vowel.
\end{sloppypar}

\begin{figure}
\tabcolsep=3pt
\begin{tabular}{*{7}{c}}
\multicolumn{7}{c}{Dialect I}\\\cmidrule(lr){1-7}
\multicolumn{3}{c}{Stage 2} &   & \multicolumn{3}{c}{Stage 3}\\\cmidrule(lr){1-3}\cmidrule(lr){5-7}
      /xi.../ & /xe/ & /xɑ/ & > & /çə.../ & /xe/ & /xɑ/\\\relax
      [çi...] & [çe] & [xɑ] &   & [çə...] & [çe] & [xɑ]
\end{tabular}
\caption{\label{fig:2.11}Opacity (overapplication) in the creation of palatal quasi-phonemes}
\end{figure}

When unstressed front vowels like /i/ are restructured to /ə/ for the next generation those speakers also reanalyze the palatal allophone [ç] from Stage 2 as an underlying palatal (/ç/) because the trigger for velar fronting has been eliminated. The underlying palatal /ç/ at Stage 3 is a \isi{palatal quasi-phoneme} (=\tabref{tab:2:35}c) because there is no contrast between velars and palatals in the context before \isi{schwa}, where only [ə] occurs.

\subsubsection{Stage 3bb}
% % (bb) 
In this type of system there is a historical process (Rule Z) which eliminates triggers for velar fronting, but that change does not \isi{bleed} velar fronting. An example of Rule Z attested in the dialects discussed in \chapref{sec:9} is the replacement of a diphthong ending in a front vowel with a back monophthong (/ɑi/ > /ɑ/); see \figref{fig:2.12}.

\begin{figure}
\tabcolsep=3pt
 \begin{tabular}{*{7}{c}}
       \multicolumn{7}{c}{Dialect J}\\\cmidrule(lr){1-7}
       \multicolumn{3}{c}{Stage 2} & & \multicolumn{3}{c}{Stage 3}\\\cmidrule(lr){1-3}\cmidrule(lr){5-7}
        /ɑx/ & /ix/ &  /ɑix/ & > & /ɑx/ & /ix/ & /ɑç/\\\relax
        [ɑx] & [iç] &  [ɑiç] &   & [ɑx] & [iç] & [ɑç]\\
  \end{tabular}
\caption{\label{fig:2.12}Opacity (overapplication) in the creation of phonemic palatals}
\end{figure}

When /ɑi/ is restructured to /ɑ/ at Stage 3 speakers have no alternative but to reanalyze the palatal allophone [ç] from Stage 2 as an underlying palatal (/ç/) because the trigger for velar fronting is no longer present. The underlying palatal at Stage 3 is a \isi{phonemic palatal} (= \tabref{tab:2:35}b) because it contrasts with [x] after the vowel [ɑ]. Concrete examples of German dialects exemplifying Dialect J are discussed in \chapref{sec:9}.

The two \isi{overapplication} outcomes (Stage ba and Stage bb) do not imply that velar fronting is lost at Stage 3. This point can be illustrated by considering Stage 3 for Dialect J in \figref{fig:2.12}. [ç] is clearly an underlying palatal (/ç/) in the context after back vowels like /ɑ/; however, [ç] can still be synchronically derived from /x/ in the context of front vowels (e.g. in [iç] from /ix/) because only [ç] but not [x] occurs in that context. In Dialect J the transition from Stage 2 to Stage 3 therefore entails two changes for velar fronting. First, the original palatal allophone for the older generation is reanalyzed as an underlying palatal for the younger generation. The change from a \isi{derived palatal} (=\tabref{tab:2:35}a) at Stage 2 to an underlying palatal (=\tabref{tab:2:35}b or \tabref{tab:2:35}c) comes about because the original trigger for velar fronting is lost. Second, velar fronting undergoes the change from an allophonic process (Stage 2) to a \isi{neutralization} (Stage 3).

The four opaque systems described above (Stage 3aa, Stage 3ab, Stage 3ba, Stage 3bb) are not mutually exclusive. A single dialect can therefore have more than one opaque sound. For example, several varieties of German are attested with palatal quasi-phonemes (=Stage 3ba) and phonemic palatals (=Stage 3bb). Likewise one of the varieties of German with a \isi{counterfeeding order} (=Stage 3aa) also has a \isi{palatal quasi-phoneme} (=Stage 3ba).

\subsection{Further remarks on the historical model}
What is not expressed in \figref{fig:2.3} is the \isi{phonemicization} of palatal fricatives that were not the product of earlier velars. The type of segment referred to here is the \isi{etymological palatal} (/ʝ/) from \tabref{tab:2:35}(d). Since the palatal fricative (/ʝ/) in question derives from an earlier palatal glide (i.e. [j] (/j/), the change from the latter to the former by \isi{Glide Hardening} does not involve a change that \isi{counterbleeds} or \isi{counterfeeds} velar fronting. The change /j/ > /ʝ/ is not a part of \figref{fig:2.3} because there is no \isi{transparency} and/or \isi{opacity}. However, it is shown in \chapref{sec:8} that \isi{Glide Hardening} often results in a phonemic contrast between palatals (/ʝ/ from earlier /j/) and velars (inherited /ɣ/).

One of the parameters mentioned earlier (output of velar fronting) is not indicated in \figref{fig:2.3}. Recall that there are two different outcomes for a /x/ target: nonsibilant palatal [ç] or \isi{sibilant} alveolopalatal [ɕ]. \isi{Alveolopalatalization} requires two modifications to the Stage 2 system with the allophones [x] and [ç]. First, [ç] is realized for innovative speakers as the new allophone [ɕ] which is phonetically and phonologically distinct from the inherited postalveolar [ʃ] (/ʃ/). Second, [ɕ] and [ʃ] merge for the next generation to [ɕ], which is phonemic (/ɕ/) because it contrasts with [x] (/x/) in the context after a back vowel. That merger does not exhibit \isi{opacity} because the new phoneme /ɕ/ in the context after a back vowel did not derive historically from a velar (but instead from the coronal [ʃ]).

It is argued in \chapref{sec:10} that \isi{alveolopalatalization} ([ç ʃ] > [ɕ]) is not expressed in terms of phonological rules; hence the realization of /x/ as [ɕ] is captured formally with the same rule of velar fronting (=\ref{ex:2:6a}) that produces [ç] from /x/. That the output of velar fronting is realized first as a nonsibilant and then only later as a \isi{sibilant} is expressed not in the phonology, but instead with rules of \isi{phonetic implementation}.

In the model in \figref{fig:2.3} change only occurs from bottom to top, where Stage 1 develops into Stage 2 and Stage 2 into Stage 3, but never in the opposite direction. However, the evidence discussed below indicates that the rule at Stage 3 must not necessarily have passed through each of the individual steps at Stage 2. For example, in one HG dialect with \isi{neutral vowels} the set of triggers for velar fronting consists only of high front vowels (\chapref{sec:6}). This suggests that the opaque effects (i.e. the creation of \isi{neutral vowels}) occurred at a very early point at Stage 2 (i.e. Stage 2a), before the set of triggers for velar fronting could expand to include all front vowels. Likewise in one LG dialect, velar fronting only applies after nonlow front vowels (Stage 2b), but that same rule of velar fronting is counterfed by another rule, as in Dialect C from \figref{fig:2.7}.

On the basis of the synchronic material from German dialects it can be deduced that there could not have been a single \isi{focal area} for velar fronting. Instead, the evidence points to several different points of origin; see \sectref{sec:12.5.2} and especially \sectref{sec:16.4}. This means that the historical model in \figref{fig:2.3} (including \isi{rule generalization} in \figref{fig:2.2}) occurred independently at various places in the German-speaking world. Polygenesis derives additional support from alveolopalatalizing dialects, since it can be shown that \isi{alveolopalatalization} occurred in places surrounded by non-alveolopalatalizing dialects (\sectref{sec:10.6.1}).

In Chapters~\ref{sec:5}--\ref{sec:15} I discuss the synchronic and diachronic behavior of fronted palatals in a wide selection of HG and LG dialects, although I do not show how the historical process of velar fronting fits into the established stages in the history of German (Appendix~\ref{appendix:e}). Linguistic evidence is adduced in later chapters that velar fronting must have been phonologized at a very early stage, namely either in \ili{OHG} (750--1050) or \ili{OSax} (800--1150), although \isi{phonologization} in some dialects may have postdated that time frame. The dating of velar fronting is discussed in \chapref{sec:16}.
