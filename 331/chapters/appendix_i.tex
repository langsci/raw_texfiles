\chapter{Velar fronting parallels in a selection of Indo-European languages}\label{appendix:i}

The typological literature cited throughout this book stresses that the fronting of velar sounds in the neighborhood of front vocoids like [i] and [j] is a phonetically plausible development that is well-attested in the languages of the world. The purpose of this appendix is to briefly assess the (in)stability of velars in the neighborhood of front vocoids in a small set of \ili{Indo-European} languages. In particular, I focus on those Gmc languages (\ili{WGmc}/\ili{NGmc}) not discussed in this book, as well as the two major language families spoken in the immediate vicinity of German-speaking countries, namely \ili{Slavic} and \ili{Romance}. The name for the fronting of velars in the literature cited below differs from author to author; for the sake of consistency, I refer to it as \isi{Velar Palatalization}, which is also the term typically adopted in the typological literature (\sectref{sec:2.3}). In the following paragraphs I consider the status of \isi{Velar Palatalization} from the diachronic perspective, but I also assess its role as a synchronic process in modern languages.

The purpose of this appendix is not to present data illustrating \isi{Velar Palatalization} in a representative selection of phonological contexts for each language. Instead, I summarize the basic facts as they are presented in the works cited and give a few selected examples for illustration. With the exception of my discussion of \ili{North Frisian}, I restrict my discussion of the standard languages and make no attempt to assess the status of the palatalization/fronting of velars in regional dialects.

In order to facilitate a comparison between velar fronting in German dialects and \isi{Velar Palatalization} in the languages spoken (or once spoken) in north-cen\-tral Europe it is important to consider \isi{Velar Palatalization} in terms of the same parameters for velar fronting. Those parameters are: (a) the nature of the target velar consonant, (b) the nature of the trigger, (c) the nature of the output, (d) \isi{directionality} (right-to-left or left-to-right), and (e) the position of the target consonant in the word (word-initial, word-medial, word-final).

I turn now to the individual language families:

\section{Germanic}
The fronting of a velar in the neighborhood of front vocoids is not well-attested as a synchronic rule in modern Gmc languages \citep{Hall2020}, although that type of historical change has occurred. I consider \ili{NGmc} and \ili{WGmc} in that order:\footnote{{I do not discuss the philological evidence purported to document \isi{Velar Palatalization} in earlier stages of Gmc (e.g. \citealt{VanderHoek2010} on \ili{OHG} and \ili{OLF}) because that evidence is simply too sparse and speculative to draw conclusions concerning the status of the parameters listed above.}}

\subsection{\ili{North Germanic}}
In an early stage (ca. thirteenth century) velar stops (/k g/) were fronted before front vocoids (\citealt{Haugen1976,Haugen1982}). The change was regular in word-initial position, but in word-medial position it was not as widespread. The output sounds of \isi{Velar Palatalization} when it was phonologized were probably the corresponding palatal stops ([c ɟ]), which were later realized differently depending on the language. In particular, earlier [c ɟ] are retained as palatals in \ili{Icelandic} ([cʰ c]), but in \ili{Norwegian} they are realized as [ç j] and in \ili{Swedish} as [ɕ j], cf. the initial segment in the verb ‘give-\textsc{inf}’: \ili{Icelandic} \textit{gefa} [cɛːva], \ili{Norwegian} \textit{gi} [jiː], \ili{Swedish} \textit{ge} [jeː]. The palatal sounds in those cognates derive from velar [g] in ON \textit{gefa}.

In modern Scandinavian languages there are vestiges of the historical process of \isi{Velar Palatalization} in the form of morphophonemic alternations; see \citet[112]{Kristoffersen2000} for \ili{Norwegian}, \citet[101--103]{arnason2011} for \ili{Icelandic}, and \citet[109]{Riad2014} for \ili{Swedish}. Although \isi{Velar Palatalization} was once an allophonic process (e.g. [k] and [c] were positional variants), the modern reflexes of the palatals created by that historical process (or the sounds they later developed into) now contrast with velars; hence, any synchronic process mirroring \isi{Velar Palatalization} is a rule of \isi{neutralization}. For example, [k] and [c] contrast in \ili{Icelandic}, e.g. [cœːr̥] ‘done’ vs. [kœːrouhtʏr̥] ‘impure, feculent’; alternating examples include [kʰɔːma] ‘come-\textsc{inf}’ vs. [cʰɛːmʏr̥] ‘come-\textsc{3}\textsc{sg}’. Recall from \sectref{sec:6.5.1} and \sectref{sec:7.4.1} that \citet{Anderson1981} and \citet{Calabrese2005} both capture similar velar vs. palatal alternations in \ili{Icelandic} with synchronic rules mirroring the historical process of \isi{Velar Palatalization}.

\citet[108, Footnote 27]{Riad2014} observes that \isi{Velar Palatalization} also affected the historical lenis velar (\ili{PGmc} \textsuperscript{+}[ɣ]) in the context after liquids (/l r/) in \ili{Swedish}. That change can be observed in \ili{Swedish} words like [bærːj] ‘mountain’, where palatal [j] corresponds to /g/ in \il{Standard German}StG, cf. the cognate [bɛʀk] /bɛʀg/).

\subsection{\ili{West Germanic}}

\subsubsection{English}
\begin{sloppypar}
As discussed at length in the scholarly literature, \isi{Velar Palatalization} regularly applied in the context of front segments in \ili{OE}; \citet[252--270]{Hogg2011} and \citet[84--88]{Minkova2014} offer two recent treatments of this topic.
\end{sloppypar}

\citet[252--270]{Hogg2011} presents a very detailed discussion of \isi{Velar Palatalization} in \ili{OE}. Although the generalization is simple -- velar consonants are fronted in the context of front segments -- there are a number of restrictions regarding the target velar, the front \isi{vocoid} trigger, and the position of the target and trigger within the word. (\citealt{Hogg2011}: 253--254 opines that the complex set of conditions can be simplified by taking syllable structure into consideration). The conditions referred to are as follows: In initial position any velar consonant underwent \isi{Velar Palatalization} before a front vowel, e.g. \textit{\textsuperscript{+}}\textit{ɣellɑn > yell; \textsuperscript{+}}\textit{kīdɑn > chide}. In word-final position all velar consonants were palatalized after (short or long) /i/, e.g. \textit{\textsuperscript{+}}\textit{dīk > ditch}, but after nonhigh front vowels only velar fricatives served as targets, e.g. \textit{\textsuperscript{+}}\textit{dæɣ > day}. In word-medial position a velar consonant was always palatalized before /i/ or /j/. Velar fricatives underwent the same change in medial position after any front vowel provided that a back vowel did not directly follow, e.g. \textit{\textsuperscript{+}}\textit{reɣn > rain}.

In its earliest stage \isi{Velar Palatalization} created palatal allophones (e.g. [c ç] from /k x/), but the pronunciation of those palatal sounds was modified by later changes. For example, palatal stops like [c] is now realized as the postalveolar \isi{affricate} ([tʃ]), as indicated in the modern \ili{English} examples listed above.

Modern \ili{English} has many alternations involving a velar stop ([k g]) and a coronal fricative or affricate ([s ʃ dʒ]), e.g. \textit{electri[k]{\textasciitilde}electri[s]ity, logi[k]{\textasciitilde}logi[ʃ]ian, analo[g]ous{\textasciitilde}analo[dʒ]y}. Those alternating forms have been argued to involve the fronting an underlying velar (/k g/) in the context of a following front vocoid by rules of \is{Velar Softening (English)}Velar Softening and Palatalization \citep{ChomskyHalle1968, Borowsky1990, Halle2005}.

\subsubsection{Frisian}
\ili{WGmc} \textsuperscript{+}/k/ and \textsuperscript{+}/ɣ/ underwent \isi{Velar Palatalization} in initial position before front segments in \ili{OFr} (\citealt{Laker2007, Bremmer2009}). According to the latter author \citep[30--31]{Bremmer2009}, /k/ was realized as the \isi{affricate} [ts] and /ɣ/ as a continuant.\footnote{{Bremmer’s symbol for [ɣ] is ⟦g⟧, and his symbol for the corresponding continuant is ⟦j⟧, the latter of which was realized orthographically in OFr as} \textrm{\textit{i}}\textrm{. I interpret Bremmer’s ⟦j⟧ as the corresponding palatal fricative /ʝ/ [ʝ]. Bremmer assumes that the change from} \textrm{\textsuperscript{+}}\textrm{/k/ to [ts] included more than one intermediate stage, namely} \textrm{\textsuperscript{+}}\textrm{/k/ > /k}\textrm{\textsuperscript{j}}\textrm{/ > /t}\textrm{\textsuperscript{j}}\textrm{/ > /ts/.}} Examples include \textsuperscript{+}\textit{kerkɑ}{}- > \textit{tserl} ‘man’ (cf. \il{Standard German}StG [kɛʀl] ‘fellow’) and \textsuperscript{+}\textit{geldɑ}{}- > \textit{ield} ‘money’ (cf. \il{Standard German}StG [gɛlt]). In word-medial position, \textsuperscript{+}/k/ likewise underwent the same changes to [ts] before \textsuperscript{+}/i/ or \textsuperscript{+}/j/, e.g. \textsuperscript{+}\textit{dīkjɑn > dītsɑ} ‘build-\textsc{inf} dike-\textsc{pl}’ (cf. English \textit{dike}). In medial position the geminate stop \textsuperscript{+}/gg/ and the nasal-stop cluster \textsuperscript{+}/ng/ (\textsuperscript{+}[ŋg]) fronted before \textsuperscript{+}/i/ or \textsuperscript{+}/j/. \textsuperscript{+}/gg/ was realized as the lenis \isi{affricate} [dz], \textsuperscript{+}\textit{saggjɑn}{}- > \textit{sedzɑ} ‘say-\textsc{inf}’ (cf. \il{Standard German}StG [zɑːgən]), and \textsuperscript{+}/ng/ (\textsuperscript{+}[ŋg]) as [ndz], e.g. \textsuperscript{+}\textit{langi}{}- > \textit{lendze} ‘length’ (cf. \il{Standard German}StG [lɛŋə]). In final position, \textsuperscript{+}/ɣ/ was realized as [ʝ] in the context after /e/, e.g. \textsuperscript{+}\textit{wega}{}- > \textit{wei} ‘way’ (cf. \il{Standard German}StG [veːk] /veːg/). Additional complications include the etymological source of the palatalization triggers and the retention of \textsuperscript{+}/k/ in \textsuperscript{+}/sk/ clusters.

Modern Frisian consists of three separate branches \citep{Walker1989}: West Frisian (spoken in the Dutch province of Friesland), North Frisian (spoken in the county of Nordfriesland in the German state of Schleswig-Holstein), and Saterland Frisian (spoken in the district of Cloppenburg in the German state of Lower Saxony). The location of all three Frisian languages is indicated on \mapref{map:44}. Given that \ili{North Frisian} and \ili{Saterland Frisian} are coterritorial with a velar fronting language (LG), one might suspect that those Frisian languages also have some version of velar fronting.\footnote{\ili{West Frisian} velars (e.g. [x]) are stable in the context before or after front vowels \citep{Sipma1913, CohenEbelingeringafokkemaholk1959, Hoekstra2001}.} This appears to be the case for \ili{North Frisian}, although some sources simply make passing reference to velar fronting without providing the necessary details. For example, \citet[25]{Bauer1925} writes that the Moringer dialect has the (fortis) velar and palatal fricatives and that those sounds have a distribution as in \il{Standard German}StG. \citet[43]{Brandt1913} makes a similar statement for the Goeharden dialect. \citet[44--45]{Jensen1925} likewise asserts that the velar and palatal fricatives in Wiedingharde are distributed according to the frontness of the preceding vowel. Unfortunately, Bauer, Brandt, and Jensen transcribe velars and palatals with the same phonetic symbol; hence, it is not possible to determine the parameters for velar fronting in the dialects they describe. \citet[20]{Tedsen1906} observes that the \ili{North Frisian} dialect spoken on the island of Föhr has a fortis palatal and a fortis velar fricative which are transcribed with two distinct symbols, i.e. ⟦χ⟧ (=[ç]) and ⟦x⟧ (=[x]). The dialect also has the lenis velar fricative [ɣ] (=⟦ʒ⟧), which can occur after any type of vowel. On the basis of the data from \citet{Tedsen1906} it can be concluded that velar fronting only affects the fortis fricative /x/, which has the allophone [ç] after high front vowels ([i y]) and [x] after back vowels, e.g. ⟦ɡiχl̥⟧ ‘violin’ (cf. \il{Standard German}StG \textit{Geige}), ⟦ryχ⟧ ‘rough’ (cf. \il{Standard German}StG \textit{rauh}) vs. ⟦lɑxt⟧ ‘easy’ (cf. \il{Standard German}StG \textit{leicht}). Since no examples were found in that source for either [ç] or [x] in the context after nonhigh front vowels or consonants it is not possible to know for sure whether or not the set of triggers consists only of high front vowels. \citet[176]{Siebs1909} states that the \ili{North Frisian} variety of Helgoland has an ich-Laut and an ach-Laut. Since the dictionary in that work gives lexical entries phonetically with separate symbols for velars and palatals it is easy to see that [ç] surfaces after any front vowel and [x] after any back vowel. (No examples were found in \citealt{Siebs1909} for the context after a consonant).  

According to \citet[67]{Sjölin1969}, \citet[65]{Fort1980}, and \citet[412]{Fort2001} \ili{Saterland Frisian} has both [x] and [ɣ], but there are no corresponding palatals. In his phonetic study of \ili{Saterland Frisian}, \citet{Peters2017} writes that /x/ is usually realized as a velar fricative, but that some speakers have a palatal variant after front vowels.

\subsubsection{Afrikaans}
According to \citet[80]{CombrinkStadler1987}, the velar stop /k/ (= orthographic \textit{k}) and the velar fricative /x/ (= orthographic \textit{g}) surface as the corresponding palatals ([c] and [ç]) in word-initial position before a front vowel, e.g. the initial segment in \textit{gieter} ‘watering’ (cf. \il{Standard German}StG [giːsən] ‘water-\textsc{inf}’) and \textit{geld} ‘money’ (cf. \il{Standard German}StG [gɛlt] ‘money’) is [ç], and the \textit{k} in \textit{kies} ‘choose-\textsc{inf}’ (cf. \il{Standard German}StG [kiːzə] ‘choose-\textsc{inf}’) is [c]. The rule of \isi{Velar Palatalization} (“Palatalisasie”) posited by \citet[80]{CombrinkStadler1987} is triggered by a front vowel but not by a consonant. Since [ç] and [c] are not contrastive sounds of \ili{Afrikaans}, \isi{Velar Palatalization} is an allophonic process. The generalizations concerning the distribution of the velars [k x] and the corresponding palatals are also clear from earlier sources for \ili{Afrikaans} \citep{Wilson1964, DeVilliers1969}.\footnote{Data and references for word-initial velar fronting in Afrikanns can be found under “Palatalisation” in the online grammar of Afrikaans in Taalportaal (\url{https://taalportaal.org}). According to that source, word-initial velar fronting is only triggered by a “high vowel, (especially the high front [i] vowel)". Taalportaal also notes that /ɦ/ undergoes fronting to [ʝ] before a high front vowel, e.g. [ʝiərs] (/ɦers/) 'reign-\textsc{inf}'.}


\section{Slavic}
\isi{Velar Palatalization} occurred more than once in the history of \ili{Slavic} \citep{Carlton1990}. Those changes are usually referred to in the literature as \is{First Velar Palatalization (Slavic)}First Velar Palatalization and \is{Second Velar Palatalization (Slavic)}Second Velar Palatalization. Both had in common that they affected velar stops and fricatives in the context of a following front \isi{vocoid}, but -- as shown below -- they created a different set of outputs. Those historical changes have left their trace in modern \ili{Slavic} languages in the form of alternations involving velars and coronals (see \citealt{Rubach2011} for a survey). For example, the targets for the \is{First Velar Palatalization (Slavic)}First Velar Palatalization in \ili{Kashubian} (\ili{West Slavic}, \mapref{map:44}) are /k g x/, the outputs are [tʃʲ dʒʲ  ʃʲ], and the triggers are front vowels (/i ɛ/) which follow the targets, cf. \textit{kale[k]-a} ‘invalid’ vs. \textit{kale[tʃʲ]-i} ‘invalid-\textsc{nom}.\textsc{pl}’, \textit{dro[g]-a} ‘road' vs. \textit{dro[dʒʲ]-i} ‘road-\textsc{nom.pl}’, \textit{mu[x]-a} ‘fly' vs. \textit{mu[ʃʲ]-i} ‘fly-\textsc{nom.pl}’. By contrast, the \is{Second Velar Palatalization (Slavic)}Second Velar Palatalization creates dental sibilants, but the context is morphologically conditioned. For example, in \ili{Ukrainian} (\ili{East Slavic}) the targets are /k ɣ x/, which surface as [tsʲ zʲ sʲ] before an /i/, but only in the dative or locative singular, e.g. \textit{ru[k]-a} ‘hand’ vs. \textit{ru[tsʲ]-i} ‘hand-\textsc{dat/loc}.\textsc{sg}’, \textit{mu[x]-a} ‘fly' vs. \textit{mu[sʲ]-i} ‘fly-\textsc{dat/loc.sg}'.

\section{Romance} The palatalization of velars was an important sound change that applied more than once in the history of Romance languages (\citealt{Buckley2009} and references cited therein). The \is{First Palatalization (Romance)}First Palatalization occurred in \ili{Proto-Romance} (third century), at which point /k/ and /g/ served as targets in the context before front vowels (/i e ɛ/). The eventual outputs in \ili{Old French} for those two target segments were the coronal affricates [ts dʒ], which later shifted to [s z] in modern French. For example, the [ts] and [dʒ] in \ili{Old French} /tsɛnt/ ‘hundred’ and /ardʒɛnt/ ‘silver, money’ were originally [k] and [g], but they are now realized as [s] and [ʒ], i.e. French [sã], [aʀʒã]. The \is{Second Palatalization (Romance)}Second Palatalization occurred in Gallo-Romance, two centuries after the First Palatalization. The velar target sounds for the \is{Second Palatalization (Romance)}Second Palatalization were /k g/, which became /tʃ dʒ/ in \ili{Old French}. Since the \is{First Palatalization (Romance)}First Palatalization had eliminated most sequences of /k g/ plus front vowel there were very few native words with those sequences when  the \is{Second Palatalization (Romance)}Second Palatalization was active; however, some loanwords demonstrate that front vowels served as triggers for the \is{Second Palatalization (Romance)}Second Palatalization, and some native items show that the glide /j/ could also induce fronting of a preceding velar, e.g. the initial segment in \ili{Old French} /tʃjær/ ‘dear’ was originally /k/. However, the vowel that most commonly served as the trigger for the \is{Second Palatalization (Romance)}Second Palatalization is usually transcribed as ⟦a⟧, e.g. \ili{Old French} /tʃamp/ ‘field’, /dʒambə/ ‘leg’, where the initial segments derived historically from /k/ and /g/ respectively. \citet{Buckley2009} argues that ⟦a⟧ represented the low front vowel [æ] when the  \is{Second Palatalization (Romance)}Second Palatalization was active, in which case the sounds that served as triggers for that change were all and only front vocoids.

Among the modern \ili{Romance} languages, \ili{Italian} has been argued to have a synchronic rule of \isi{Velar Palatalization} which is an outgrowth of the same process in \ili{Latin} \citep{Krämer2009}. According to that source, \isi{Velar Palatalization} is both phonologically and morphologically conditioned. For example, a velar stop (/k/) is realized as [tʃ] in the context before /i/ in noun plurals, e.g. [a'miːko] ‘friend’ {\textasciitilde} [a'miːtʃi] ‘friend-\textsc{pl}’ {\textasciitilde} [a'miːke] ‘friend-\textsc{fem}.\textsc{pl}’. \isi{Velar Palatalization} similarly accounts for the alternation between [g] and [dʒ] in second conjugation nouns, but the same process fails to apply in first conjugation nouns.

\section{Conclusion}
It was mentioned above that the historical processes of \isi{Velar Palatalization} -- like the historical process of velar fronting in German -- underwent more than one stage. Those stages can be defined according to the nature of the output (e.g. [ki] > [ci] > [tʃi] for \ili{English}), but they can also be interpreted in terms of the life cycle proposed by \citet{Hyman2013} from \sectref{sec:14.6.3}. For example, in most of the languages discussed in this appendix \isi{Velar Palatalization} in its initial stage created fronted allophones (e.g.. [c], [ɉ], [ç], [ʝ]) which later became phonemicized. Depending on the language, the original allophonic process of \isi{Velar Palatalization} might have later become morphologized (e.g. in \ili{Ukrainian}) and ultimately lost (in the case of \ili{English}).

Although there are clear parallels between \isi{Velar Palatalization} and velar fronting in the languages/language families discussed in this appendix, it is important stress that there are four significant differences:

\begin{description}
\item[Targets:] The target segments for the languages with \isi{Velar Palatalization} all include velar stops. By contrast, velar fronting in German dialects always affects at least one velar fricative, but in the unmarked case, velar stops are unaffected. Those German dialects in which velar fronting affects one or more velar stop are not common and are restricted geographically to the areas described in \chapref{sec:11}.

\item[Triggers:] It has been stressed throughout this book that the triggers for velar fronting in the unmarked HG/LG dialects include not only front vowels but also coronal consonants, i.e. /l r n/. By contrast, the unmarked triggers for \isi{Velar Palatalization} in the languages discussed above do not include consonants. The one counterexample to this generalization is \ili{Swedish}, where /l r/ served as triggers for a following velar.

\item[Outputs:] If the input segment for \isi{Velar Palatalization} is a stop, then the output is typically a coronal \isi{affricate}, e.g. /k/ is realized as [tʃ] (or in some languages as [ts]). In those marked German dialects in which a velar stop serve as targets for velar fronting, the output is a palatal stop, e.g. /k/ is realized as [c]. By contrast, no variety of German has been found in the present survey which creates an affricate (e.g. [tʃ]) from an underlying stop (e.g. /k/).

\item[Directionality:] If \isi{Velar Palatalization} applies in word-medial position then the trigger is to the right of the target; hence, \isi{Velar Palatalization} applies regressively (from right-to-left). The two examples discussed above involving left-to-right palatalization (\ili{Swedish}, \ili{OE}) also had spreading in the opposite direction. By contrast, in word-medial position velar fronting applies from left-to-right in every dialect of HG and LG without exception.
\end{description}

The conclusion is that velar fronting must be seen as a phenomenon distinct from \isi{Velar Palatalization}.
