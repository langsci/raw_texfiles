\chapter{Allophony (Part 1)}\label{sec:3}


\section{Introduction}\label{sec:3.1}

The present chapter investigates German dialects in which a velar (e.g. [x]) and the corresponding palatal ([ç]) stand in complementary distribution and therefore never contrast. The relationship between velar and palatal is an allophonic one, which is captured by deriving the latter synchronically from the former by specific versions of velar fronting. There are no palatal fricatives in the neighborhood of back segments, nor are there velar fricatives in the context of front segments; hence, palatals like [ç] only surface in the context specified in the structural description of velar fronting and the velar only in the elsewhere case; see Stage 2 in \figref{fig:2.3}. Velar and palatal fricatives in the dialects discussed below have a transparent distribution; hence, velar fronting is fed or bled by processes altering the number of potential targets or triggers for velar fronting.

Dialects with an allophonic distribution of [x] and [ç] are important to discuss in detail because velars and palatals pattern differently in other varieties of German. For example, many dialects are attested with palatal quasi-phonemes (\chapref{sec:7}) or phonemic palatals (Chapters~\ref{sec:8}--\ref{sec:10}). The material investigated below can therefore serve as a basis of comparison for the data discussed in later chapters.

In this chapter and in \chapref{sec:4} I show that velar fronting applies synchronically at the end of the grammar in several diverse dialects. It is assumed here that the synchronic relationship involving rules \isi{feeding} or \isi{bleeding} velar fronting (Rules W-Z from \tabref{tab:2.wxyz}) mirrors the diachronic relationship. Thus, if Rules W-Z \isi{feed} or \isi{bleed} velar fronting synchronically, then Rules W-Z were present in that dialect before velar fronting was phonologized, and the synchronic state therefore implies that velar fronting was phonologized at the end of the grammar. Independent linguistic (or philological) evidence confirming that velar fronting was phonologized later than the specific processes corresponding to Rules W-Z in the case studies discussed in this chapter and in \chapref{sec:4} is sparse. See \sectref{sec:5.5.2} for some discussion.

Although varieties of German displaying allophony between velar and palatal are not restricted to one particular area, it is nevertheless possible to state at the outset that such systems are particularly common in Almc and Bav. This chapter therefore focuses on several specific varieties belonging to those two broad dialects, namely \il{Swabian}Swb in \sectref{sec:3.2}, \il{High Alemannic}HAlmc in \sectref{sec:3.3}, \sectref{sec:3.4}, and \il{Central Bavarian}CBav in \sectref{sec:3.5}, \sectref{sec:3.6}. I make some concluding remarks in \sectref{sec:3.7}.

\section{Swabian}\label{sec:3.2}\il{Swabian|(}

I focus on the description of a \il{Swabian}Swb dialect spoken in a specific region, although as I note below the same rule of velar fronting in that variety can be found not only in other UG-speaking communities, but more generally in many places in the German-speaking dialect continuum. The distribution of [x] and [ç] in this one corner of southwest Germany can therefore be thought of as the default pattern.

\citet{Besch1961} provides a detailed overview of the sounds in a variety of \il{Swabian}Swb spoken in a broad area between the Neckar and Danube Rivers (“Neckar- und Donaugebietˮ) in the German state of Baden-Württemberg (\mapref{map:1}).\footnote{{The sources referred to under \mapref{map:1} and under all other locator maps are indicated with the corresponding number as circles (representing the absence of postsonorant velar fronting) or squares (representing the presence of some version of postsonorant velar fronting). The phonetic symbols in the legend for velars ([x]), palatals ([ç]) and triggers ([i ɑ]) are not intended to express the different types of triggers (e.g. nonlow front vowels vs. all front vowels), targets (e.g. /x/ vs. /x ɣ/), and/or outputs (e.g. [ʝ] vs. [ç] vs. [ɕ]).}} The author focuses on the town of \ipi{Erdmannsweiler}, although he also considers other communities in the same region. For simplicity I refer to the dialect described by Besch as \ipi{Erdmannsweiler}.

\begin{map}
% \includegraphics[width=\textwidth]{figures/VelarFrontingHall2021-img001.png}
\includegraphics[width=\textwidth]{figures/Map1_3.1.pdf}
\caption[Low Alemannic and Swabian]{Low Alemannic (LAmc) and Swabian (\il{Swabian}Swb). Squares indicate postsonorant velar fronting and circles the absence of velar fronting. 1=\citet{Mankel1886}, 2=\citet{Heimburger1887}, 3=\citet{Heusler1888}, 4=\citet{Heilig1897} 5=\citet{Henry1900}, 6=\citet{Schwend1900}, 7=\citet{Ehret1911}, 8=\citet{Weik1913}, 9=\citet{Wasmer1915, Wasmer1916, Wasmer1916b}, 10=\citet{Kilian1935}, 11=\citet{Eckerle1936}, 12=\citet{Schläpfer1956}, 13=\citet{Keller1961} (Barr), 14=\citet{Philipp1965}, 15=\citet{BethgeBonnin1969}, 16=\citet{Zeidler1978}, 17=\citet{Schrambke1981}, 18=\citet{Klausmann1985a, Klausmann1985b}, 19=\citet{Rünneburger1985}, 20=\citet{PhilippBothorel-Witz1989}, 21=E.M.\citet{Hall1991, Hall1991b} (\ipi{Urach}), 22=E.M.\citet[]{Hall1991} (\ipi{Titisee-Neustadt}), 23=ALA (Mortzwiller), 24=ALA (Oberhergheim), 25=ALA (Thanvillé), 26=ALA (Weiterswiller), 27=ALA (Lembach), 28=\citet{Kauffmann1887, Kauffmann1890}, 29=\citet{Wagner1889}, 30=\citet{Bopp1890}, 31=\citet{Haag1898}, 32=\citet{Schmidt1898}, 33=\citet{Müller1911}, 34=\citet{Dreher1919}, 35=\citet{Sexauer1927}, 36=\citet{Strohmaier1930}, 37=\citet{Zinser1933}, 38=\citet{Moser1936}, 39=\citet{Nübling1938}, 40=\citet{Schöller1939}, 41=\citet{BausingerRuoff1959}, 42=\citet{Besch1961}, 43=\citet{Baur1967}, 44=\citet{Hufnagl1967}, 45=\citet{BethgeBonnin1969}, 46=\citet{König1970}, 47=\citet{Ibrom1971}, 48=\citet{Frey1975}, 49=E.M.\citet{Hall1991, Hall1991b} (\ipi{Tuningen}), 50=E.M.\citet{Hall1991, Hall1991b} (\ipi{Donaueschingen}), 51=SBS (\ipi{Ebersbach}), 52=SSA (Überlingen), 53=SSA (Büßlingen), 54=SSA (Wangen), 55=VALTS (Niederwangen), 56=SNBW (Gerstetten), 57=SNWB (Sontheim an der Brenz), 58=SNBW (Rudersberg).}
\label{fig:3.1}\label{map:1}
\end{map}

In all of the case studies presented in this book it is essential to know the phonemic vowels. This is especially true of the front vowels, since those segments  serve as potential triggers for the assimilation of an adjacent velar to palatal. For this reason, I attempt here and in all subsequent case studies to give a representative example for the realization of dorsal fricatives in the context of every phonemic vowel.

The phonemic monophthongs of \ipi{Erdmannsweiler} consist of the front vowels /iː i eː e æː æ/ and the back vowels /uː u oː o ɔː ɔ ɑː ɑ ə/. The two vowels /æː æ/ are transcribed in the original source as ⟦ǟ ä⟧\footnote{{Throughout this book I enclose phonetic transcriptions as they appear in all original sources within double square brackets ⟦ ... ⟧. My own transcriptions (with a consistent set of phonetic symbols) are given in single square brackets [ ... ]. The latter transcription is important because the original sources cited in this book employ a wide variety of symbols, some of which are not obvious to linguists in the present day.} }. I interpret those two vowels as low ([æː æ]) and not mid ([ɛː ɛ]) because they are characterized by a degree of openness (“weit offen”) greater than the degree of openness for vowels like [ɛ] (=⟦ę⟧), which Besch describes as half open (“halb offen”). Besch also includes nasalized monophthongs which I ignore because they do not occur in the context of dorsal fricatives. The dialect also has a number of phonemic diphthongs whose second component can be front (/ei ai/) or back (\isi{schwa}), i.e. /iːə iə æːə æə uə/.

[x] and [ç] are the only two dorsal fricatives; those two sounds are only attested in postsonorant position but never word-initially.\footnote{In contrast to many varieties of CG/LG discussed in later chapters, velar [ɣ] and palatal [ʝ] are absent from UG dialects like \ipi{Erdmannsweiler}; Appendix~\ref{appendix:f} provides historical background accounting for those gaps. Alternations involving [g] and [ç] as in \il{Standard German}StG (\sectref{sec:1.2}) are similarly absent in the varieties of German discussed below. Appendix~\ref{appendix:h} provides a consonant inventory for typical UG dialects like \ipi{Erdmannsweiler}.} As depicted in \REF{ex:3:1}, [x] and [ç] stand in an allophonic relationship. I assume without argument that the underlying sound is velar (/x/) from which the palatal ([ç]) is synchronically derived. The arguments against a rule deriving [x] from an underlying palatal /ç/ -- in \ipi{Erdmannsweiler} and in the velar fronting varieties of German dialects addressed below -- cannot be evaluated until all German dialects have been discussed (\sectref{sec:17.3.3}).

\ea%1
\label{ex:3:1}
      \begin{forest}
      [/x/  [{[x]}]  [{[ç]}]]
      \end{forest}
\z 

I consider first the distribution of [x] and [ç] from the synchronic perspective and then I examine the facts diachronically.\footnote{\label{fn:3:4}One could argue that [h] is an allophone of /x/ as well, since [h] never contrasts with [x]/[ç]. ([h] surfaces only word-initially before a vowel). I do not discuss the treatment of [h] as an allophone of /x/ for \ipi{Erdmannsweiler}, although I return to this point in a related dialect (\sectref{sec:3.3}) which has alternations between [h] and [x]/[ç].}

The data presented below illustrate that [x] surfaces after a back vowel in (\ref{ex:3:2a}) and [ç] (=⟦X⟧ for Besch) after a front vowel in (\ref{ex:3:2b}) or a coronal sonorant consonant in (\ref{ex:3:2c}). There are no dorsal fricatives after consonants other than [l] or [r], (e.g. [n]); hence, liquids are the only coronal sonorant consonants after which [ç] surfaces. Note that [r] fails to vocalize to [ɐ] as in other varieties (e.g. \il{Standard German}StG in \sectref{sec:1.2} and \sectref{sec:17.2}). The historical source for the dorsal fricatives in \REF{ex:3:2} and in the other UG dialects discussed in this chapter is \ili{WGmc} \textsuperscript{+}[k] or \textsuperscript{+}[x]; see Appendix~\ref{appendix:f}.\footnote{{In \REF{ex:3:2} and in all subsequent data sets, I present the transcription in the original source in the first column, my interpretation of that transcription with IPA symbols in the second column. In the third column I give the \il{Standard German}StG orthography for reference, in the fourth column the English gloss, and in the final column the page number in the original source.}}

\ea Postsonorant [x] and [ç] (from /x/):\label{ex:3:2}

\ea\begin{tabular}[t]{@{}p{1.2cm}p{1.5cm}p{2.2cm}p{3.2cm}r}
    sūxd   & [suːxt]   & Sucht    & ‘addiction’          & 30\\
    dsuxd  & [tsuxt]   & Zucht    & ‘breeding’           & 29\\
    hōx    & [hoːx]    & hoch     & ‘high’               & 38\\
    nǭxbər & [nɔːxbər] & Nachbar  & ‘neighbor’           & 32\\
    nāxd   & [nɑːxt]   & Nacht    & ‘night’              & 18\\
    laxə   & [lɑxə]    & lachen   & ‘laugh-\textsc{inf}’ & 18\\
    liəxd  & [liəxt]   & Licht    & ‘light’              & 45\\
    fīəxdə & [fiːəxtə] & fürchten & ‘fear-\textsc{inf}’  & 31\\
    buəx   & [buəx]    & Buch     & ‘book’               & 47\\
    dsǫəxə & [tsɔəxə]  & Zeichen  & ‘sign’               & 43
    \end{tabular}\label{ex:3:2a}

\ex\begin{tabular}[t]{@{}p{1.2cm}p{1.5cm}p{2.2cm}p{3.2cm}r}
    filīXd  & [filiːçt] & vielleicht    & ‘maybe’               & 36\\
    frēliX  & [freːliç] & fröhlich      & ‘happy’               & 38\\
    nēXd    & [neːçt]   & gestern abend & ‘yesterday evening’   & 21\\
    knǟXd   & [knæːçt]  & Knecht        & ‘vassal’              & 24\\
    häXlə   & [hæçlə]   & hecheln       & ‘heckle-\textsc{inf}’ & 21\\
    blaiX   & [blɑiç]   & bleich        & ‘pale’                & 43\\
    reiX    & [reiç]    & reich         & ‘rich’                & 37\\
    \end{tabular}\label{ex:3:2b}

\ex \begin{tabular}[t]{@{}p{1.2cm}p{1.5cm}p{2.2cm}p{3.2cm}r}
    khalX & [kʰɑlç] & Kalk   & ‘lime’   & 18\\
    kherX & [kʰerç] & Kirche & ‘church’ & 17\\
    \end{tabular}\label{ex:3:2c}
\z 
\z 

As noted in \chapref{sec:1}, the focus in this book is on the patterning of velars and palatals in native words, although I include occasional older borrowings which I consider to be assimilated loan words. For example, [kʰɑlç] ‘lime’ in (\ref{ex:3:2c}) was borrowed many centuries ago from Latin \textit{calx}.

As indicated in the heading for \REF{ex:3:2}, I analyze the underlying dorsal sound as /x/, which surfaces as palatal in (\ref{ex:3:2b}, \ref{ex:3:2c}) by \REF{ex:3:3}. The [ç] in (\ref{ex:3:2b}, \ref{ex:3:2c}) therefore exemplifies the \isi{derived palatal} category described in \sectref{sec:2.4.3}. I analyze front vowels and liquids (/l r/) as [+sonorant, coronal]. Given that analysis, underlying /x/ fronts to palatal [ç] after a front (i.e. coronal) vowel in (\ref{ex:3:2b}) or after a front (i.e. coronal) sonorant consonant in (\ref{ex:3:2c}) and otherwise (i.e. after a back vowel) surfaces without change as [x] in (\ref{ex:3:2a}).

\ea%3
    \label{ex:3:3}
    \isi{Velar Fronting-1}:\\
    \begin{forest}
    [,phantom
      [\avm{[+son]} [\avm{[coronal]},name=coronal,tier=word]]
      [\avm{[−son\\+cont]},name=parent [\avm{[dorsal]},tier=word]]
    ]
    \draw [dashed] (parent.south) -- (coronal.north);
    \end{forest}      
\z 

\REF{ex:3:3} spreads [coronal] from a [+sonorant] sound to a [dorsal] fricative, thereby creating the complex corono-dorsal segment [ç]. It is not necessary to specify that the target segment be marked for a laryngeal feature ([{}--fortis]) because there are no [+fortis] dorsal fricatives that could potentially undergo the rule. The target is specified as a dorsal fricative ([{}--sonorant, +continuant, dorsal]) and not as a dorsal obstruent ([{}--sonorant, dorsal]) or dorsal consonant ([+consonantal, dorsal]) because there is no indication in the original source that other velar sounds ([k g ŋ]) show a fronted variant after coronal sonorants. Unless specific evidence is provided to the contrary, I assume in all following case studies that velar fricatives (and not velar stops or the velar nasal) undergo fronting. In \chapref{sec:11} I discuss varieties of German in which all velar consonants undergo fronting to the corresponding palatals.\footnote{{The velar \isi{affricate} [kx] is a sound attested in many (but not all) varieties of Almc, and in a few of those varieties [kx] has a palatal allophone [kç] (e.g. \sectref{sec:3.4}). The default assumption I adopt below (reflected in my description of \ipi{Erdmannsweiler}) is that [kx] is absent unless I explicitly state that it is present.} }

The data in \REF{ex:3:2} reveal a few gaps. For example, no words were found in the original source with a dorsal fricative preceded by the front vowel [e] or the back vowels [o ɔ ə]. Unless evidence can be adduced to the contrary, I assume that [e] patterns phonologically with the other front vowels and [o ɔ ə] with the other back vowels. Hence, the expectation is that [ç] would surface after [e] and [x] after [o ɔ ə].

The front vowel triggers for \isi{Velar Fronting-1} also include segments that alternate with back vowels. The most well-known front-back alternations are the ones referred to in the traditional literature as \textsc{Umlaut}. For example, in \il{Standard German}StG, many singular nouns with a back stem vowel surface with the corresponding front vowel in the plural, e.g. \textit{Stuhl} [ʃtuːl] ‘chair’ vs. \textit{Stühle} [ʃtyːlə] ‘chair-\textsc{pl}’. A representative example of such front vowel vs. back vowel alternations in \ipi{Erdmannsweiler} is presented in \REF{ex:3:4}. Example (\ref{ex:3:4b}) illustrates that the front vowel [ei] triggers the change from /x/ as [ç].\footnote{{From the historical perspective the fronted stem vowel in examples like the one in (\ref{ex:3:4b}) was an etymological back vowel, cf. OHG} {\textit{rouh}}{. The fronting of back vowels was either a consequence of \isi{i-Umlaut} (\sectref{sec:2.4.3}), which was triggered by a once overt front vowel suffix [i], or by \isi{analogy}. The distinction between the two (sound change vs. \isi{analogy}) is not important for present purposes and is therefore not discussed below.}}

\TabPositions{0pt, .15\linewidth, .3\linewidth, .45\linewidth, .66\linewidth}
\ea\label{ex:3:4}
\ea\label{ex:3:4a}roux  \tab [roux]  \tab Rauch    \tab ‘smoke’ \tab  44
\ex\label{ex:3:4b}reiXə \tab [reiçə] \tab räuchern \tab ‘smoke-\textsc{inf}’ \tab 44
\z 
\z

Although the literature on \isi{Umlaut} in the synchronic grammar of \il{Standard German}StG is vast and the proposals are quite diverse (e.g. \citealt{Zwicky1967}, \citealt{Vennemann1968}, \citealt{Wurzel1970}, \citealt{BachKing1970}, \citealt{Lieber1980}, \citealt{vanLessenKloeke1982}, \citealt{Strauss1982}, \citealt{Janda1987}, \citealt{Lieber1987}, \citealt{Lodge1989}, \citealt{Klein1995}, \citealt{Wiese1996b, Wiese1996a}, \citealt{Trommer2021}), it is nevertheless possible to identify two contrastive approaches. According to one, the stem vowel in alternating pairs is underlyingly back and the front vowel alternant is derived from that back vowel if \isi{Umlaut} is analyzed as a synchronic rule. Underlying representations for the examples in \REF{ex:3:4} according to that approach are provided in (\ref{ex:3:5a}). The form of the synchronic rule of \isi{Umlaut} presupposed in (\ref{ex:3:5a}) assumes that the underlying representation for the plural form is equipped with a floating frontness feature (see the literature cited above). According to the second approach, the stem alternants are lexically listed (suppletive) allomorphs, in which case \isi{Umlaut} does not have the status of a synchronic rule. Underlying representations for the examples in \REF{ex:3:4} according to that approach are given in (\ref{ex:3:5b}). The latter treatment derives support from the fact that alternations like the ones in \REF{ex:3:4} -- regardless of dialect -- are irregular because they are triggered by certain morphological categories but not by others. Thus, according to (\ref{ex:3:5b}), \isi{Umlaut} has been morphologized. The first approach (=\ref{ex:3:5a}) is defended by \citet{Wiese1996b} and \citet{Trommer2021} and the second (=\ref{ex:3:5b}) by \citet{Booij2010}.

\ea%5
    \label{ex:3:5}
\ea\label{ex:3:5a}/roux/\textsubscript{,} /roux-ə/
\ex\label{ex:3:5b}/roux/\textsubscript{,} /reix-ə/
\z 
\z 

Both treatments in \REF{ex:3:5} are consistent with the data from \ipi{Erdmannsweiler}, as well as similar data from other German dialects. In the present book I adopt (\ref{ex:3:5b}), although the analyses I discuss are also compatible with (\ref{ex:3:5a}).

In the dialect of \ipi{Erdmannsweiler} as it was described in 1961, /x/ is realized as [ç] not only after historically front vowels (=\ref{ex:3:6a}), but also after etymological back vowels that underwent the historical fronting, e.g. \isi{i-Umlaut} or \isi{analogy} (=\ref{ex:3:6b}). The surface velar [x] occurs after etymological back vowels (=\ref{ex:3:6c}) and after back vowels that were originally front (=\ref{ex:3:6d}). The reconstructed examples in the second column are my own. They are intended to represent the point before velar fronting was phonologized. The etymological information in \REF{ex:3:6} has been drawn from \citet{Seebold2011}, which is my source for etymologies in all subsequent datasets unless otherwise noted.

\ea\label{ex:3:6}%6
\TabPositions{1.5cm, 1.8cm, 2cm, 3.5cm, 5.5cm, 8.25cm}
\ea\label{ex:3:6a}\relax  [reiç]   \tab <  \tab \textsuperscript{+}[riːx]   \tab ‘rich’               \tab cf. MHG \textit{rīch(e)} \tab (from \ref{ex:3:2b})
\ex\label{ex:3:6b}\relax  [reiçə]  \tab <  \tab \textsuperscript{+}[rouxə]  \tab ‘smoke-\textsc{inf}’ \tab cf. MHG \textit{rouch}   \tab (from \ref{ex:3:4b})
\ex\label{ex:3:6c}\relax  [hoːx]   \tab <  \tab \textsuperscript{+}[hoːx]   \tab ‘high’               \tab cf. MHG \textit{hōch}    \tab (from \ref{ex:3:2a})
\ex\label{ex:3:6d}\relax  [tsɔəxə] \tab <  \tab \textsuperscript{+}[tseixə] \tab ‘sign’               \tab cf. MHG \textit{zeichen} \tab (from \ref{ex:3:2a})
\z 
\z 

The pan-\il{Swabian}Swb vocalic development depicted in (\ref{ex:3:6d}) involves the change from a front sound to a back sound; see \citet[42--43]{Besch1961}. It is a specific instance of a historical shift I call \textsc{Vowel} \textsc{Retraction}, which can be defined as any change from a front vowel to a back vowel, although the particular vowels that undergo it differ from dialect to dialect. \isi{Vowel Retraction} therefore decreases the number of potential triggers for velar fronting (=Rule Z from \tabref{tab:2.wxyz}); recall Dialect F from \figref{fig:2.8}. The general change is stated in (\ref{ex:3:7a}). The vocalic change depicted in (\ref{ex:3:6b}) is a specific example of what is referred to throughout this book as \textsc{Vowel} \textsc{Fronting}, which has the general form in (\ref{ex:3:7b}). \isi{Vowel Fronting} increases the number of potential triggers for velar fronting (=Rule X from \tabref{tab:2.wxyz}; recall Dialect E from \figref{fig:2.8}). The specific examples illustrating \isi{Vowel Retraction} and \isi{Vowel Fronting} discussed below could alter underlying representations, although some examples are still active synchronically (e.g. the analysis of \isi{Umlaut} in 5a as the modern reflex of the historical fronting in \ref{ex:3:6b}).

\ea\label{ex:3:7}%7
\begin{multicols}{2}\raggedcolumns
\ea\label{ex:3:7a}Vowel Retraction:\smallskip\\       
     \{ front vowel \} > \{ back vowel \}
\columnbreak
\ex\label{ex:3:7b}Vowel Fronting:\smallskip\\
    \{ back vowel \} > \{ front vowel \}
\z 
\end{multicols}
\z 


As I show in this book, dorsal fricatives behave differently in German dialects when they are in the context of a vowel that has undergone either \isi{Vowel Retraction} or \isi{Vowel Fronting}. In \ipi{Erdmannsweiler} and in many other dialects discussed below a dorsal fricative to the right of a new back vowel (=\ref{ex:3:7a}) surfaces transparently as velar, but in other dialects the dorsal fricative in the context of a new back vowel surfaces instead as an opaque palatal. A dorsal fricative adjacent to a new front vowel (=\ref{ex:3:7b}) in \ipi{Erdmannsweiler} is realized transparently as palatal, but in other German dialects that sound is an opaque velar.

\ipi{Erdmannsweiler} as it was described in 1961 was an outgrowth of an earlier stage in which /x/ was realized as [x], regardless of the nature of the preceding sound. I refer henceforth to that earlier point as Stage 1 (\figref{fig:2.3}) and postulate that the surface [x] showed the effects of coarticulatory velar fronting (to \isi{prevelar}) at the level of Speech. Such non-fronting Stage 1 dialects are attested in the present day, e.g. in \il{Low Alemannic}LAlmc, Halmc, and \il{South Bavarian}SBav (\sectref{sec:3.3}, \sectref{sec:12.3.1}, \sectref{sec:12.3.2}). The \isi{phonologization} of \isi{Velar Fronting-1} (=Stage 2 from \sectref{sec:2.5}) is shown in \REF{ex:3:8} with three representative examples from \REF{ex:3:6}. As a point of reference I give the \il{Standard German}StG forms in the bottom row.  As described in \sectref{sec:2.5} the change from one stage to the next was intergenerational, involving the interaction between the speaker and the listener in \isi{acquisition}.\pagebreak

\ea%8
    \label{ex:3:8}
    \columnsep=-1.25cm
    \begin{multicols}{2}\raggedcolumns
\ea \label{ex:3:8a}\begin{tabular}[t]{@{}lll@{}} 
    /riːx/ & /hoːx/  & /tseixə/ \strut\\\relax
    [riːx] & [hoːx]  &  [tseixə]\strut\\\tablevspace
           &         &          \strut\\
           &         &          \strut\\    \tablevspace
    /reix/ & /hoːx/  & /tsɔəxə/ \strut\\\relax
    [reiç] &  [hoːx] &  [tsɔəxə]\strut\\\tablevspace
    \textit{reich} & \textit{hoch} &  \textit{Zeichen}\\
    ‘rich’         &  ‘high’       &   ‘sign’         \\
    \end{tabular}
\columnbreak
\ex \label{ex:3:8b}\begin{tabular}[t]{@{}llll@{}}
    /riːx/        &  /hoːx/        & /tseixə/         & Stage 1\\\relax
    [riːx]         & [hoːx]         & [tseixə]         &        \\\tablevspace
    /reix/         & /hoːx/         & /tseixə/         & Stage 2\\\relax
    [reiç]         &  [hoːx]        & [tseiçə]         &        \\\tablevspace
    /reix/         & /hoːx/         & /tsɔəxə/         & Stage 2\\\relax
    [reiç]         &  [hoːx]        & [tsɔəxə]         &        \\\tablevspace
    \textit{reich} & \textit{hoch}  & \textit{Zeichen} & \il{Standard German}StG \\
     ‘rich’        & ‘high’         & ‘sign’           &        \\
    \end{tabular}
\z 
\end{multicols}
\z 

Two possible chronologies involving \isi{Velar Fronting-1} and \isi{Vowel Retraction} (/ei/ > /ɔə/ in \textit{Zeichen}) are depicted in (\ref{ex:3:8a}) and (\ref{ex:3:8b}). According to (\ref{ex:3:8a}), \isi{Velar Fronting-1} was phonologized at the same time -- or perhaps even after -- \isi{Vowel Retraction} restructured underlying representations. As depicted in (\ref{ex:3:8b}) it is also conceivable \isi{Velar Fronting-1} was active before \isi{Vowel Retraction}. According to that scenario, there was a stage in which \isi{Velar Fronting-1} created palatal [ç] in words like \textit{Zeichen} before the stem vowel was restructured to /ɔə/. In later chapters I demonstrate that the chronology in (\ref{ex:3:8b}) must have been correct for the other German dialects because those dialect-specific changes from front vowel to back vowel led to \isi{opacity} via the \isi{phonemicization} of the palatal allophone [ç] (/x/) to /ç/. However, one cannot know for certain whether or not the chronology in (\ref{ex:3:8b}) or (\ref{ex:3:8a}) is correct for \ipi{Erdmannsweiler} because \isi{Vowel Retraction} led to transparent outputs according to either scenario. Note that in both (\ref{ex:3:8a}) and (\ref{ex:3:8b}) the historical process of \isi{Vowel Retraction} \isi{bleeds} \isi{Velar Fronting-1} in examples like [tsɔəxə].

The descriptive literature on many of the dialects spoken in Baden-Würt\-tem\-berg -- both \il{Swabian}Swb and \il{Low Alemannic}LAlmc -- published from the late nineteenth century up to the present suggests that the transparent distribution of [x] and [ç] is precisely the same as it is in \ipi{Erdmannsweiler}. For example, one can observe that [ç] only surfaces after a coronal sonorant and [x] only after a back vowel in the varieties spoken in \ipi{Horb am Neckar} \citep{Kauffmann1887, Kauffmann1890}, \ipi{Forbach} \citep{Heilig1897}, \ipi{Pforzheim} \citep{Sexauer1927}, \ipi{Freudenstadt} \citep{Baur1967}, \ipi{Stuttgart} \citep{Frey1975}, and the broad area around \ipi{Villingen-Schwenningen}
(E. M. \citealt[55--56]{Hall1991}). All of those places -- as well as a number of other ones in the same area -- are indicated on \mapref{map:1}.

However, the default pattern exemplified in the communities listed in the previous paragraph stands in contrast with the \il{Swabian}Swb and \il{Low Alemannic}LAlmc varieties discussed in \sectref{sec:14.3.2}. In that section I demonstrate that several German dialects in Baden-Württemberg have been described in which [ç] occurs not only after coronal sonorants, but also after one or more back vowel, e.g. \ipi{Blaubeuren} \citep{Strohmaier1930}, \ipi{Mühlingen} \citep{Müller1911}, and \ipi{Liggersdorf} \citep{Dreher1919}. There are also a few isolated pockets within the \il{Low Alemannic}LAlmc/\il{Swabian}Swb dialect region in which only a subset of coronal sonorants triggers velar fronting (\sectref{sec:12.3}). The lesson to be learned from these surprising revelations is that one cannot assume a priori that the default pattern for any given dialect region is correct until one has examined the entire range of facts.\il{Swabian|)}

\section{{High} {Alemannic} {(part} {1)}}\label{sec:3.3}\il{High Alemannic|(}

SwG is typically characterized by the presence of a dorsal fricative surfacing invariantly as back even in a front vowel context; hence, in the unmarked case, speakers of SwG have [x] but no [ç]. Descriptions of H(st)Almc dialects of Switzerland with [x] \textit{sans} [ç] therefore represent the norm for that region. For example, in \citegen{Keller1961} description of \ipi{Bern} German (\mapref{map:2}) he writes (p. 51): “[x] is a velar fricative articulated rather far back. The place of articulation is not influenced by the surrounding sounds”. An early twentieth century description of the \il{High Alemannic}HAlmc dialect spoken in the canton of \ipi{Glarus} (\mapref{map:2}) is essentially the same \citep{Streiff1915}. \citet[12]{Streiff1915} writes that [x] is articulated on the soft palate (“am weichen Gaumenˮ) and that the dialects she describes do not have a palatal articulation of that sound. (“Einen palatalen x-Laut kennen unsere Mundarten nicht …”), e.g. [tseːxə] ‘toe’. Additional varieties of H(st)Almc with velars (/x/) without palatal allophones are indicated on \mapref{map:2}.

A few H(st)Almc dialects have been described which possess both velar and palatal fricatives, and a subset of those dialects displays a parallel distribution of velar and palatal affricates. In the present section and in the following one I discuss two varieties in which velars and palatals stand in an allophonic relationship. The distribution of the velar and palatal sounds discussed below can be contrasted with the very different patterning one finds in the two \il{Highest Alemannic}HstAlmc dialects discussed in \chapref{sec:6}, which possess neutral vowels.

\citet{Meinherz1920} offers a detailed account of the \il{High Alemannic}HAlmc dialect spoken in the northernmost part (Region Landquart) of the canton of Grisons (Graubünden) in \ipi{East Switzerland}. The region is known historically as the Bündner Herrschaft; see also \sectref{sec:15.11} and \mapref{map:41}.

\citet[20]{Meinherz1920} draws a distinction between the dialect he calls H\textsubscript{1}, which is spoken in the municipalities (Gemeinden) of \ipi{Maienfeld}, Fläsch, and Malans, and the dialect referred to as J, which is spoken only in Jenins. I concentrate below on H\textsubscript{1} because this variety has velar fronting. The dialect is referred to henceforth as \ipi{Maienfeld}.

\begin{map}
% \includegraphics[width=\textwidth]{figures/VelarFrontingHall2021-img002.png}
\includegraphics[width=\textwidth]{figures/Map2_3.2.pdf}
  \caption[High Alemannic  and Highest Alemannic]{High Alemannic (\il{High Alemannic}HAlmc) and Highest Alemannic (\il{Highest Alemannic}HstAlmc). White squares indicate postsonorant velar fronting and circles the absence of velar fronting. 1=\citet{Winteler1876}, 2=\citet{Abegg1910}, 3=\citet{Enderlin1910}, 4=\citet{Kaiser1910}, 5=\citet{Vetsch1910}, 6=\citet{Wipf1910}, 7=\citet{Hausknecht1911},  8=\citet{Berger1913}, 9=\citet{Gröger1914a}, 10=\citet{Gröger1914b}, 11=\citet{Gröger1914c}, 12=\citet{Gröger1914d}, 13=\citet{Gröger1914e}, 14=\citet{Schmid1915}, 15=\citet{Streiff1915}, 16=\citet{Wiget1916}, 17=\citet{Stucki1917}, 18=\citet{Brun1918}, 19=\citet{Meinherz1920}, 20=\citet{Baumgartner1922}, 21=\citet{Jutz1922}, 22=\citet{Weber1923}, 23=\citet{Jutz1925}, 24=\citet{Beck1926}, 25=\citet{Henzen1927} (\ipi{Sensebezirk}), 26=\citet{Henzen1927} (Obersimmental), 27=\citet{Henzen1928, Henzen1932}, 28=\citet{Clauss1929}, 29=\citet{Kessler1931}, 30=\citet{Hotzenköcherle1934}, 31=\citet{Wanner1941}, 32=\citet{Schultz1951}, 33=\citet{Keller1961}, 34=\citet{Keller1963}, 35=\citet{Schmid1969}, 36=\citet{BethgeBonnin1969}, 37=\citet{Werlen1977}, 38=\citet{Marti1985}, 39=\citet{Russ2002}, 40=\citet{FleischerSchmid2006}.}
  \label{fig:3.2}\label{map:2}
\end{map}

The phonemic monophthongs of \ipi{Maienfeld} \citep[22]{Meinherz1920} consist of the front vowels /iː i ɪː ɪ eː e ɛː ɛ/ and the back vowels /uː u ʊː ʊ oː o ɔː ɔ ɑː ɑ ə/. As in many other varieties of SwG, length and \isi{tenseness} can be combined in the mid and high vowels to yield a system with a large number of monophthongs. Meinherz also includes several nasalized monophthongs which I ignore because they do not occur in the data I investigate below with dorsal fricatives. \citet[22]{Meinherz1920} lists fifteen diphthongs, but the material I consider below only contains /æi/ and /uə/.

\ipi{Maienfeld} differs from\textsubscript{} the variety spoken in the municipality of Jenins (see above) in terms of the realization of /x/; see \citet[26]{Meinherz1920}. Jenins exhibits the default (Stage 1) pattern for SwG in the sense that that dorsal fricative is consistently realized as [x], regardless of what segment precedes or follows. By contrast, \ipi{Maienfeld} has a palatal realization of /x/, which occurs only after a coronal sonorant (see below). The allophonic relationship between [x] and [ç] is depicted in \REF{ex:3:1}. The two surface fricatives [x] and [ç] are only attested in postsonorant position but never word-initially. Words in the Jenins dialect with word-initial [x] (<\ili{WGmc} \textsuperscript{+}[k]) are realized in \ipi{Maienfeld} as [k\textsuperscript{h}], e.g. \textit{Käfer} [k\textsuperscript{h}ɛːfər] ‘bug’ vs. Jenins [xɛfər]; \citet[134]{Meinherz1920}.

Although the vowels of \ipi{Maienfeld} differ from those of \ipi{Erdmannsweiler} the generalization concerning the distribution of [x] and [ç] is the same in both dialects: [x] surfaces after a back vowel in (\ref{ex:3:9a}) and [ç] after a front vowel in (\ref{ex:3:9b}) or a coronal sonorant consonant in (\ref{ex:3:9c}); see \citet[135]{Meinherz1920}. \citet[27]{Meinherz1920} is clear that [r] is a coronal (apical) trill and not a uvular (i.e. dorsal) sound (“r ist stark gerolltes Zungen-r”). The two dorsal fricatives [x]/[ç] as in \REF{ex:3:9} derive from etymological velars (\ili{WGmc} \textsuperscript{+}[k x]).

There are no dorsal fricatives after consonants other than [l] or [r], (e.g. [n]); hence, liquids are the only coronal sonorant consonants after which [ç] surfaces. No examples were found in the original source with a dorsal fricative preceded by the front vowels [œ œː] or the back vowel [ɑː]. I treat these gaps as accidental.

\ea Postsonorant [x] and [ç] (from /x/):\label{ex:3:9}
\ea \label{ex:3:9a}\begin{tabular}[t]{@{}p{2cm}p{2cm}p{2cm}p{2cm}>{\raggedleft\arraybackslash}p{8mm}@{}}
brūx  & [bruːx]  & Brauch  & ‘custom’   & 135\\
brux  & [brux]   & Bruch   & ‘fracture’ & 135\\
hōx   & [hoːx]   & hoch    & ‘high’     & 144\\
šprɔx & [ʃprɔːx] & Sprache & ‘language’ & 135\\
lɔx   & [lɔx]    & Loch    & ‘hole’     & 135\\
bɑx   & [bɑx]    & Bach    & ‘stream’   & 135\\
buəx  & [buəx]   & Buch    & ‘book’     & 135\\
\end{tabular}
\ex\label{ex:3:9b}\begin{tabular}[t]{@{}p{2cm}p{2cm}p{2cm}p{2cm}>{\raggedleft\arraybackslash}p{8mm}@{}}
    rīχ   & [riːç]  & reich & ‘rich’               & 135\\
    štiχ  & [ʃtiç]  & Stich & ‘sting’              & 135\\
    šǖχ   & [ʃyːç]  & scheu & ‘timid’              & 144\\
    tsyχt & [tsyçt] & zieht & ‘move-\textsc{3sg}’ & 143\\
    tsɛχ  & [tsɛːç] & zäh   & ‘tough’              & 144\\
    frɛχ  & [frɛç]  & frech & ‘impudent’           & 135\\
    hȫχs  & [høːçs] & hohes & ‘high-\textsc{infl}’ & 143\\
    wæiχ  & [wæiç]  & weich & ‘soft’               & 135\\
    \end{tabular}
\ex\label{ex:3:9c} \begin{tabular}[t]{@{}p{2cm}p{2cm}p{2cm}p{2cm}>{\raggedleft\arraybackslash}p{8mm}@{}}
    milχ  & [milç]  & Milch   & ‘milk’ & 137\\
    khɑlχ & [kʰɑlç] & Kalk    & ‘lime’ & 137\\
    wɛrχ  & [wɛrç]  & Werk    & ‘work’ & 137\\
    štɔrχ & [ʃtɔrç] & Storch  & ‘stork’& 137\\   
    \end{tabular}
\z 
\z 

The complementary distribution of [x] and [ç] is captured by positing an underlying /x/ which surfaces as [ç] after a coronal sonorant by \isi{Velar Fronting-1} (recall \ref{ex:3:3}). As in \ipi{Erdmannsweiler} (recall \ref{ex:3:4}), the front vowel triggers for that process in \ipi{Maienfeld} also include alternating examples involving \isi{Umlaut} like [høːçs] ‘high-\textsc{infl}’ (cf. [hoːx] ‘high’).\footnote{{\citet{Meinherz1920} is one of the rare examples of a descriptive grammar which states explicitly that velar stops do not undergo fronting. \citet[25]{Meinherz1920} writes: “Zwischen} {\textit{ɡ}} {in} {\textit{ɡi, iɡ}} {und} {\textit{ɡɑ, ɑɡ}} {sowie zwischen} {\textit{k}} {in} {\textit{ki, ik}} {und} {\textit{kɑ, ɑk}} {konnte ich keinen merklichen Unterschied hören”. (“I could not hear a noticeable difference between} {\textit{ɡ}} {in} {\textit{ɡi, iɡ}} {and} {\textit{ɡɑ, ɑɡ}} {as well as between} {\textit{k}} {in} {\textit{ki, ik}} {and} {\textit{kɑ, ɑk}}{”.)} }

One difference between \ipi{Erdmannsweiler} and \ipi{Maienfeld} is that /x/ in all of the examples presented in \REF{ex:3:9} is in coda position; cf. \REF{ex:3:2} and \REF{ex:3:4} with several words in which [ç]/[x] are situated between vowels and are hence in the onset. There is no reason to specify that \isi{Velar Fronting-1} for \ipi{Maienfeld} only affects a coda sound because there is no /x/ in a word-initial onset or a word-internal onset (e.g. intervocalic position) which could potentially undergo the rule. As noted above, dorsal fricatives do not occur in word-initial onsets. The reason there are no word-internal onsets with a dorsal fricative is indicated in \REF{ex:3:10}. \citet[26]{Meinherz1920} shows that \ipi{Maienfeld} has debuccalized \ili{WGmc} \textsuperscript{+}[x] to [h] in the context between vowels (in the first example \ref{ex:3:10a} and \ref{ex:3:10b}) or between a liquid and vowel (in the first example in \ref{ex:3:10c}). I interpret those two contexts as onset position; in \REF{ex:3:10} and elsewhere the dot in the phonetic transcriptions indicates the syllable boundary. By contrast, \ili{WGmc} \textsuperscript{+}[x] is retained as a fricative ([x] or [ç]) in coda position, as in the second and third example in (\ref{ex:3:10a}) and (\ref{ex:3:10b}) and in the second example in (\ref{ex:3:10c}). The consequence of the debuccalization of \ili{WGmc} \textsuperscript{+}[x] to [h] in intervocalic position is that there are now synchronic alternations between [h] and [x]/[ç].


\ea {[x]}{\textasciitilde}[ç]{\textasciitilde}[h] alternations:\label{ex:3:10}
\ea \label{ex:3:10a}\begin{tabular}[t]{@{}p{2cm}p{2cm}p{2cm}p{2cm}>{\raggedleft\arraybackslash}p{8mm}@{}}
    mɑhə  & [mɑ.hə] & mache  & ‘do-\textsc{1sg}’   & 136\\
    mɑxšt & [mɑxʃt] & machst & ‘do-\textsc{2sg}’   & 136\\
    mɑxt  & [mɑxt]  & macht  & ‘do-\textsc{3sg}’ & 136\\
    \end{tabular}
\ex \label{ex:3:10b}\begin{tabular}[t]{@{}p{2cm}p{2cm}p{2cm}p{2cm}>{\raggedleft\arraybackslash}p{8mm}@{}}
    štrīhə  & [ʃtriː.hə] & streiche  & ‘paint-\textsc{1sg}’   & 136\\
    štrīχšt & [ʃtriːçʃt] & streichst & ‘paint\textsc{{}-2sg}’ & 136\\
    štrīχt  & [ʃtriːçt]  & streicht  & ‘paint-\textsc{3sg}’  & 136\\
    \end{tabular}
\ex \label{ex:3:10c}\begin{tabular}[t]{@{}p{2cm}p{2cm}p{2cm}p{2cm}>{\raggedleft\arraybackslash}p{8mm}@{}}
    štɔrhə & [ʃtɔr.hə] & Haus zum Storchen & ‘(name)’ & 137\\
    štɔrχ  & [ʃtɔrç]   & Storch            & ‘stork’  & 137\\
    \end{tabular}
\z 
\z 

From the synchronic perspective, /x/ is the underlier and the historical process of \isi{Debuccalization} (Debucc) remains active as a synchronic rule:

\ea%11
    \label{ex:3:11}
    \begin{tabular}[t]{@{}lll@{}}
         &  /ʃtriːx-t/     & /ʃtriːx-ə/\\
Debucc   & {}-{}-{}-{}-{}-{}- & ʃtriː.hə\\
Vel Fr-1 & ʃtriːçt         & -{}-{}-{}-{}-{}-\\
         & [ʃtriːçt]       & [ʃtriː.hə]\\
         & ‘paint-\textsc{3sg}’ & ‘paint-\textsc{1sg}’
    \end{tabular}
\z 

In \REF{ex:3:11} \isi{Debuccalization} (/x/→[h] / \textsubscript{${\sigma}$}[ {\longrule}V) \isi{bleeds} \isi{Velar Fronting-1} (Vel Fr-1) in example [ʃtriː.hə]. Note that the treatment in \REF{ex:3:11} is consistent with treating [h] as an allophone of /x/ (\fnref{fn:3:4}).\footnote{{The reader is referred to Hall’s (\citeyear{Hall2009a,Hall2010,Hall2011a}) treatment of [h]{\textasciitilde}[x] alternations akin to the ones in \REF{ex:3:10} in the related \il{South Bavarian}SBav dialect spoken in \ipi{Imst} (\citealt{Schatz1897}; \mapref{map:3}). \ipi{Imst} differs from \ipi{Maienfeld} because [h] and [x] contrast in word-medial position and alternations like the ones in \REF{ex:3:10} must be accounted for with a rule converting /h/ to [x] in coda position (\isi{Buccalization}). In the analysis for \ipi{Imst} described here,} {\textsc{rule} \textsc{inversion}} {has occurred (\citealt{Vennemann1972}, \citealt{Hall2009a} and \sectref{sec:8.6.3}) because \isi{Debuccalization} has been reanalyzed as a rule of \isi{Buccalization} with /h/ as the target.}} The \isi{bleeding} relationship in \REF{ex:3:11} is a specific instantiation of Dialect C from \figref{fig:2.7}.

\ipi{Maienfeld} displays the default pattern described earlier for \ipi{Erdmannsweiler}: /x/ undergoes fronting after any coronal sonorant. From the diachronic perspective, any front segment serves as a trigger for \isi{Velar Fronting-1}, regardless of historical source. In contrast to \ipi{Erdmannsweiler}, there were apparently no instances of \isi{Vowel Retraction} in \ipi{Maienfeld} (recall the change from /ei/ to /ɔə/ in \ref{ex:3:6d}), although many examples illustrate \isi{Vowel Fronting} (=\isi{i-Umlaut}), e.g. the [øː] in [høːçs] ‘high-\textsc{infl}’ which is etymologically [oː]; cf. OHG \textit{hōh}. Thus, the front segments that trigger \isi{Velar Fronting-1} were either historically front or they were historically back and underwent \isi{Vowel Fronting} (=\isi{i-Umlaut}).

\section{{High} {Alemannic} {(part} {2)} }\label{sec:3.4}

\citet{Berger1913} describes a variety of \il{High Alemannic}HAlmc spoken in \ipi{Rheintal} in Northeast Switzerland in the canton of \ipi{St. Gallen} (\mapref{map:2}). \ipi{Rheintal} is a large area indicated in greater detail on \mapref{map:41}, which depicts velar fronting areas in \ipi{East Switzerland}, \ipi{Liechtenstein}, and \ipi{Vorarlberg}. It is clear from \citet{Berger1913} that velar fronting is active in \ipi{Rheintal}, but it is also evident that the facts involving velars and palatals in \ipi{Rheintal} differ in various ways from the distribution of [x] and [ç] in \ipi{Maienfeld}.

In addition to [x] and [ç], \ipi{Rheintal} also possesses the corresponding affricates, i.e. velar [kx] and palatal [kç]. Velars ([x], [kx]) and the corresponding palatals ([ç], [kç]) stand in an allophonic relationship: In word-initial position, the two dorsal affricates are positional variants (see \ref{ex:3:12a}). In postsonorant position (see \ref{ex:3:12b}) two dorsal fricatives and the two dorsal affricates are likewise allophones. The dorsal fricatives in (\ref{ex:3:12b}) are shown below to have prosodically-determined fortis geminate counterparts ([xx] and [çç]), which exhibit the same distribution as the corresponding lenis sounds.

\ea\label{ex:3:12}%12
\begin{multicols}{3}
\ea\label{ex:3:12a}\begin{forest} [/kx/  [{[kx]}]    [{[kç]}]] \end{forest}
\ex\label{ex:3:12b}\begin{forest} [/x/   [{[x]}]   [{[ç]}]] \end{forest}
\label{ex:3:12c}\begin{forest} [/kx/   [{[kx]}]     [{[kç]}]] \end{forest}
\z 
\end{multicols}
\z 

The phonemic monophthongs of \ipi{Rheintal} consist of the front vowels /iː i yː y ɪː ɪ eː e øː ø ɛː ɛ œː œ/ and the back vowels /uː u oː o ɔː ɔ ɑː ɑ ə/. Diphthongs occurring adjacent to a dorsal fricative are /iːə yːə eə ɛːə ɛə œːə uə ɔːə/. Note that all of those diphthongs end in \isi{schwa}.

The patterning of the fricatives and affricates in \REF{ex:3:12} requires that the mid front lax series of vowels (/ɛː ɛ œː œ/) be analyzed as phonologically [+low], as in \tabref{ex:3:13}; I make the additional assumption that the corresponding back vowels (/ɔː ɔ/) are likewise [+low]. [+low] front vowels include the monophthongs /ɛː ɛ œː œ/ as well as the /ɛː ɛ œː œ/ component of diphthongs. The analysis of vowels in \tabref{ex:3:13} is analogous to the treatment of /i e ɛ/ described in (\ref{ex:3:7b}) of \sectref{sec:2.2.3}.

\begin{table}
\caption{\label{ex:3:13} Distinctive features for vowels (Rheintal)}
\begin{tabular}{l cc cc cc cc cc c}
\lsptoprule
  & iː  i & ɪː  ɪ & eː e & ɛː ɛ & yː y & øː ø & œː œ & uː u & oː o & ɔː ɔ & ɑː ɑ\\\midrule
\relax [coronal] & \langscicheckmark{} & \langscicheckmark{} & \langscicheckmark{} & \langscicheckmark{} & \langscicheckmark{} & \langscicheckmark{} & \langscicheckmark{} &  &  &  & \\
\relax [dorsal] &  &  &  &  &  &  &  & \langscicheckmark{} & \langscicheckmark{} & \langscicheckmark{} & \langscicheckmark{}\\
\relax [labial] &  &  &  &  & \langscicheckmark{} & \langscicheckmark{} & \langscicheckmark{} & \langscicheckmark{} & \langscicheckmark{} & \langscicheckmark{} & \\
\relax [low] & − & − & − & + & − & − & + & − & − & + & \\
\relax [high] & + & + & − &  & + & − &  & + & − &  & \\
\relax [tense] & + & − &  &  &  &  &  &  &  &  & \\
\lspbottomrule
\end{tabular}
\end{table}

It can be seen that front vowels are [coronal], and back vowels are [dorsal]. All rounded vowels are [labial], while unrounded vowels are unmarked for that feature. For front unrounded vowels, front rounded vowels, as well as back vowels, either [+low] or [{}--low] is assigned. Among all vowels bearing specification for [--low], [+high] is assigned to the high vowels, while mid vowels receive [--high]. The feature [±tense] distinguishes /iː i/ from /ɪː ɪ/. I omit \isi{schwa} from \tabref{ex:3:13}, which is placeless.\footnote{{\citet[7]{Berger1913} also lists among the monophthongs the phonetically low front vowels [æ] (=⟦æ⟧) and [æː] (=⟦\={æ}⟧), but it is clear from the discussion in that source that [æ] and [æː] occur in some communities in place of the two vowels [ɛ] and [ɛː].}}

Data illustrating the complementary distribution of [kx] and [kç] in word-initial position (=\ref{ex:3:12a}) are presented in \REF{ex:3:14}. The examples show that the velar occurs before a back vowel in (\ref{ex:3:14a}) or a [+low] front vowel in (\ref{ex:3:14b}). The palatal surfaces before a [{}--low] front vowel in (\ref{ex:3:14c}), or a coronal sonorant consonant ([r l n]), in (\ref{ex:3:14d}). \citet[11]{Berger1913} describes the rhotic [r] as a tongue-tip trill (“Zungenspitzen-r”). There are no restrictions concerning the type of vowel that can follow the sonorant consonant in (\ref{ex:3:14d}).\footnote{\label{fn:3:11}Dorsal fricatives do not occur in word-initial position in the communities whose phonology is described below, although other places in the same region have dorsal fricatives instead of affricates in \REF{ex:3:14}. Among speakers with word-initial dorsal fricatives, their distribution mirrors that of [kx] and [kç].} The historical source for the word-initial affricates was \ili{WGmc} \textsuperscript{+}[k], which is preserved as [k] in other dialects, cf. the \il{Standard German}StG orthography in the third column.\largerpage

\ea Word-initial [kx] and [kç] (from /kx/):\label{ex:3:14}
    
\ea \begin{tabular}[t]{@{}p{2cm}p{2cm}p{2cm}p{2cm}>{\raggedleft\arraybackslash}p{8mm}@{}}
    kxūšt   & [kxuːʃt]  & Kunst      & ‘art’               & 134\\
    kxuttɩɡ & [kxuttɪg] & wählerisch & ‘choosy’            &  44\\
    kxopf   & [kxopf]   & Kopf       & ‘head’              &  42\\
    kxɔ̄     & [kxɔː]    & kommen     & ‘come-\textsc{inf}’ & 134\\
    kxɔrn   & [kxɔrn]   & Korn       & ‘grain’             &  42\\
    kxɑts   & [kxɑts]   & Katze      & ‘cat’               & 134\\
    \end{tabular}\label{ex:3:14a}
\ex \begin{tabular}[t]{@{}p{2cm}p{2cm}p{2cm}p{2cm}>{\raggedleft\arraybackslash}p{8mm}@{}}
    kxɛəfər  & [kxɛəfər]  & Käfer   &  ‘bug’    & 33\\
    kxɛ̄ər    & [kxɛːər]   & Keller  &  ‘cellar’ & 34\\
    kxɛ̄ənnə  & [kxɛːənnə] & Kern    &  ‘core’   & 34\\
    \end{tabular}\label{ex:3:14b}
\ex \begin{tabular}[t]{@{}p{2cm}p{2cm}p{2cm}p{2cm}>{\raggedleft\arraybackslash}p{8mm}@{}}
    kχittɩl             & [kçittɪl]  &  Kittel  &  ‘smock’                  & 134\\
    kχündə              & [kçyndə]   &  künden  &  \mbox{‘proclaim-\textsc{inf}’}  &  46\\
    kχȫərə        & [kçøːərə]  &  gehören &  \mbox{‘belong to-\textsc{inf}’} &  17\\
    kχeərχχə  & [kçeərççə] &  Kirche  &  ‘church’                 & 136\\
    \end{tabular}\label{ex:3:14c}
\ex \begin{tabular}[t]{@{}p{2cm}p{2cm}p{2cm}p{2cm}>{\raggedleft\arraybackslash}p{8mm}@{}}
    kχlɛəbə & [kçlɛəbə] & kleben    & ‘stick-\textsc{inf}’ & 134\\
    kχrott  & [kçrott]  & Kröte     & ‘toad’               & 134\\
    kχnoblə & [kçnoblə] & Knoblauch & ‘garlic’             & 136\\
    \end{tabular}\label{ex:3:14d}
\z 
\z 

Additional evidence that only [{}--low] front vowels are preceded by the palatal \isi{affricate} can be observed in \REF{ex:3:15}. The word-initial \isi{affricate} in the first item in (\ref{ex:3:15a}) is predictably velar because the following vowel is back. When that vowel shows the effects of \isi{Umlaut} in the second word in (\ref{ex:3:15a}) the \isi{affricate} remains velar because the front vowel ([œ]) is [+low]. That nonalternating [kx] can be contrasted with the [kx] that alternates with [kç] before a front [{}--low] vowel, as in (\ref{ex:3:15b}).

\ea%15
\label{ex:3:15}
\ea  \begin{tabular}[t]{@{}p{2cm}p{2cm}p{2cm}p{2cm}>{\raggedleft\arraybackslash}p{8mm}@{}}
     kxɔrəb & [kxɔrəb] & Korb  &‘basket’   & 108\\
     kxɔrbə & [kxœrbə] & Körbe & ‘basket-\textsc{pl}’ &  75\\
     \end{tabular}\label{ex:3:15a}
\ex  \begin{tabular}[t]{@{}p{2cm}p{2cm}p{2cm}p{2cm}>{\raggedleft\arraybackslash}p{8mm}@{}}
     kxuɡələ & [kxugələ] & Kugel        & ‘ball’              & 44\\
     kχüɡəli & [kçygəli] & kleine Kugel & ‘ball-\textsc{dim}’ & 65\\
     \end{tabular}\label{ex:3:15b}
\z 
\z 

I account for the distribution of word-initial velar and palatal affricates by positing that the underlying sound in \REF{ex:3:14} and \REF{ex:3:15} is velar /kx/, which surfaces as palatal by either (\ref{ex:3:16a}) or (\ref{ex:3:16b}). In the elsewhere case, /kx/ is realized without change as [kx]. (\ref{ex:3:16a}) converts a velar to the corresponding palatal in word-initial position before a [--low] front vowel, while (\ref{ex:3:16b}) creates a palatal before a coronal sonorant consonant. The two operations cannot be collapsed into a single one because [±low] is not a distinctive feature for consonants. The target of both (\ref{ex:3:16a}) and (\ref{ex:3:16b}) is a dorsal [--sonorant, +continuant] sound, which is either /kx/ or /x/; recall the representations in \REF{ex:3:2} of \sectref{sec:2.2.2}. This is the correct prediction because the fricatives [x] and [ç] for many speakers have a distribution that parallels the patterning of the corresponding affricates (see \fnref{fn:3:11}).

\ea%16
    \label{ex:3:16}
    \begin{multicols}{2}\raggedcolumns
\ea \isi{Wd-Initial Velar Fronting-1}:\\\label{ex:3:16a}
\begin{forest}
 [,phantom
   [  \avm{[−son\\+cont]}, name=parent [\avm{[dorsal]}, tier=word] ]
   [  \avm{[−low]}   [\avm{[coronal]}, name=coronal, tier=word]]  
 ]
 \node[left=1mm of parent.west] {\textsubscript{wd} [};
 \draw [dashed] (parent.south) -- (coronal.north);
\end{forest}\columnbreak
\ex  \isi{Wd-Initial Velar Fronting-2}:\\\label{ex:3:16b}
\begin{forest}
 [,phantom
   [  \avm{[−son\\+cont]}, name=parent [\avm{[dorsal]}, tier=word] ]
   [  \avm{[+cons\\+son]},  [\avm{[coronal]},name=coronal, tier=word]]
 ]
 \node[left=1mm of parent.west] {\textsubscript{wd} [};
 \draw [dashed] (parent.south) -- (coronal.north);
\end{forest}
\z 
\end{multicols}
\z 

According to both (\ref{ex:3:16a}) and (\ref{ex:3:16b}) the feature [coronal] spreads leftward onto a [--sonorant, +continuant, dorsal] segment (i.e. /kx/ or /x/). Dorsal stops (/g k/) cannot undergo the change because all stops are [{}--sonorant, --continuant].

Velar and palatal fricatives stand in an allophonic relationship in postsonorant position (=\ref{ex:3:12b}), as illustrated in \REF{ex:3:17}: Velars surface after a back vowel in (\ref{ex:3:17a}) or a [+low] front vowel in (\ref{ex:3:17b}) and palatals after a front [--low] vowel in (\ref{ex:3:17c}) or liquid in (\ref{ex:3:17d}).\footnote{{Dorsal fricatives surface either as lenis ([x]/[ç]) or fortis ([xx]/[çç]) depending on the length of the preceding vowel. In the analysis I present below I ignore the fortis vs. lenis distinction.}}  No examples were found in the original source with a dorsal fricative after [l] {}-- a gap I consider to be accidental. Due to an added complication, I delay discussion of velars and palatals after diphthongs to the end of this section. The diachronic source for the dorsal fricatives in \REF{ex:3:17} is \ili{WGmc} \textsuperscript{+}[x k].

\ea Postvocalic dorsal fricatives (from /x/):\label{ex:3:17}
\ea \begin{tabular}[t]{@{}p{2cm}p{2cm}p{2cm}p{2cm}>{\raggedleft\arraybackslash}p{8mm}@{}}
    šlūx  & [ʃluːx] & Schlauch & ‘hose’     &   13\\
    bruxx & [bruxx] & Bruch    & ‘fracture’ &  135\\
    rōx   & [roːx]  & Rauch    & ‘smoke’    &   13\\
    dox   & [dox]   & doch     & ‘however’  &  140\\
%     \end{tabular}\label{ex:3:17a}
% \ex \begin{tabular}[t]{@{}p{2cm}p{2cm}p{2cm}p{2cm}>{\raggedleft\arraybackslash}p{8mm}@{}}
    šprɔ̄x   & [ʃprɔːx]  & Sprache    & ‘language’  & 135\\
    lɔxx    & [lɔxx]    & Loch       & ‘hole’      & 135\\
    tɑxx    & [tɑxx]    & Dach       & ‘roof’      & 135\\
    feələxt & [feələxt] & vielleicht & ‘maybe’     &  38\\
    \end{tabular}\label{ex:3:17a}
\ex \begin{tabular}[t]{@{}p{2cm}p{2cm}p{2cm}p{2cm}>{\raggedleft\arraybackslash}p{8mm}@{}}
     nɛ̄x   & [nɛːx] & nahe          & ‘near’         & 140\\
     nɛxt  & [nɛxt] & gestern abend & ‘last evening’ & 140\\
     \end{tabular}\label{ex:3:17b}
\ex \begin{tabular}[t]{@{}p{2cm}p{2cm}p{2cm}p{2cm}>{\raggedleft\arraybackslash}p{8mm}@{}}
     rīχχ   & [riːçç]  & reich    & ‘rich’              & 135\\
     štiχχ  & [ʃtiçç]  & Stich    & ‘sting’             & 38 \\
     šǖχ              & [ʃyːç]   & scheu    & ‘timid’             & 140\\
     flüχšt           & [flyçʃt] & fliehst  & ‘flee-\textsc{2sg}’ &  58\\
     löχχli & [løççli] & Löchlein & ‘hole-\textsc{dim}’ &  43\\
    \end{tabular}\label{ex:3:17c}
\ex \begin{tabular}[t]{@{}p{2cm}p{2cm}p{2cm}p{2cm}>{\raggedleft\arraybackslash}p{8mm}@{}}
     mɑ̄rχχ    & [mɑːrçç]   & (Grenz)mark & ‘borderland’ & 136\\
     kχeərχχə & [kçeərççə] & Kirche      & ‘church’      & 136\\
     \end{tabular}\label{ex:3:17d}
\z 
\z

The generalizations concerning the distribution of velars and palatals after vowels are clear from the original source \citep[113]{Berger1913}.


Examples like [lɔxx] ‘hole’ vs. [løççli] ‘hole-\textsc{dim}’ in \REF{ex:3:17} display velar vs. palatal alternations triggered by Umlaut-induced stem alternations (cf. \ref{ex:3:15b}). Velar vs. palatal pairs like [lɔxx] vs. [løççli] can be contrasted with  the nonalternating velar [x] in \REF{ex:3:18}. Note that the dorsal fricative in the first example in both word pairs is velar because it follows a back vowel. The /x/ following the front alternant surfaces without change as [x] because the front vowel is [+low] (cf. \ref{ex:3:15a}).\footnote{Berger 
    notes that the front counterpart of [ɑ] in \isi{Umlaut} alternations like the one in (\ref{ex:3:18b}) can be [e] for some words. He documents some doublets, i.e. words whose fronted vowel is [ɛ] or [e]. One such example is the morpheme \textit{Nacht} in (\ref{ex:3:18b}). 
    Significantly, the pronunciation with [e] requires the dorsal fricative to surface as palatal, i.e. [neçt] (=⟦neχt⟧). The palatal realization of /x/ confirms the analysis of /e/ as a front [--low] vowel (recall \tabref{ex:3:13}).
}

\ea \label{ex:3:18}
\ea \label{ex:3:18a}\begin{tabular}[t]{@{}p{2cm}p{2cm}p{2cm}p{2cm}>{\raggedleft\arraybackslash}p{8mm}@{}}
     špr\={ɔ}x   & [ʃprɔːxǝ]  & Sprache & ‘language’          & 49\\
     špr\={ɔ̈̄}xli & [ʃprœːxli] & reden   & ‘talk-\textsc{dim}’ & 49\\
    \end{tabular}
\ex \label{ex:3:18b}\begin{tabular}[t]{@{}p{2cm}p{2cm}p{2cm}p{2cm}>{\raggedleft\arraybackslash}p{8mm}@{}}
      nɑxt & [nɑxt] & Nacht  & ‘night’  & 125\\
      nɛxt & [nɛxt] & Nächte & ‘night-\textsc{pl}’ &  31\\
    \end{tabular}
\z 
\z 

The distribution of postvocalic dorsal affricates mirrors the distribution of the equivalent fricatives (=\ref{ex:3:12b}). Thus, the velar surfaces after a back vowel in (\ref{ex:3:19a}) or a front [+low] vowel in (\ref{ex:3:19b}) and the palatal after a [--low] front vowel in (\ref{ex:3:19c}) or liquid in (\ref{ex:3:19d}). No examples were found in the original source with a dorsal \isi{affricate} after [r].\footnote{{Dorsal affricates also occur after a nasal, but it is not clear from the original source whether or not the nasal in question is velar ([ŋ]) or palatal ([ɲ]). For this reason I refrain from discussing these examples. See \citet[137]{Berger1913} for discussion.}}



\ea Postvocalic [kx] and [kç] (from /kx/):\label{ex:3:19} 
\ea \label{ex:3:19a}\begin{tabular}[t]{@{}p{2cm}p{2cm}p{2cm}p{2cm}>{\raggedleft\arraybackslash}p{8mm}@{}}
    trukxə & [trukxə] & drücken  & ‘press-\textsc{inf}’ & 137\\
    rokx   & [rokx]   & Rock     & ‘skirt’              & 137\\
    sɑkx   & [sɑkx]   & Sack     & ‘sack’               & 137\\
    \end{tabular}
\ex \label{ex:3:19b}\begin{tabular}[t]{@{}p{2cm}p{2cm}p{2cm}p{2cm}>{\raggedleft\arraybackslash}p{8mm}@{}}
    ɛkxər & [ɛkxə] & Äcker & ‘field-\textsc{pl}’ & 31
    \end{tabular}
\ex \label{ex:3:19c}\begin{tabular}[t]{@{}p{2cm}p{2cm}p{2cm}p{2cm}>{\raggedleft\arraybackslash}p{8mm}@{}}
    štrikχ        & [ʃtrikç]  & Strick    & ‘cord’                 & 137\\
    ɡlükχ         & [glykç]   & Glück     & ‘fortune’              &  46\\
    trükχnə       & [trykçnə] & trocknen  & ‘dry-\textsc{inf}’     &  137\\
    rökχli        & [røkçli]  & Röcklein  & ‘skirt-\textsc{dim}’   &   43\\
    brökχə  & [brøkçə]  & Brocken   & ‘chunk’                &   62\\
    štrekχə & [ʃtrekçə] & strecken  & ‘stretch-\textsc{inf}’ &  137\\
    \end{tabular}
\ex \label{ex:3:19d}\begin{tabular}[t]{@{}p{2cm}p{2cm}p{2cm}p{2cm}>{\raggedleft\arraybackslash}p{8mm}@{}}
    wolkχə       & [wolkçə]   & Wolke       & ‘cloud’       &  136\\
    milkχ              & [milkç]    & Milch       & ‘milk’        &  137\\
    \end{tabular}
\z 
\z 

As in word-initial position, \ipi{Rheintal} requires two distinct rules to capture the distribution of dorsal fricatives and affricates in postsonorant position: One applies after a [{}--low] front vowel (=\ref{ex:3:20a}) and the other after a coronal sonorant consonant (=\ref{ex:3:20b}). In the elsewhere case (after [+low] front vowels or back vowels) /x/ and /kx/ surface without change.

\ea%20
\label{ex:3:20}
\begin{multicols}{2}\raggedcolumns
\ea \label{ex:3:20a}\isi{Velar Fronting-2}:\\
    \begin{forest}
    [,phantom
      [ \avm{[−low]} [\avm{[coronal]}, tier=word, name=coronal] ]
      [ \avm{[−son\\+cont]},name=parent [\avm{[dorsal]},tier=word]]
    ]
    \draw [dashed] (parent.south) -- (coronal.north);
    \end{forest}\columnbreak
\ex \isi{Velar Fronting-3}:\\\label{ex:3:20b}
    \begin{forest}
    [,phantom
        [\avm{[+cons\\+son]} [\avm{[coronal]},tier=word,name=coronal]]
        [\avm{[−son\\+cont]},name=parent [\avm{[dorsal]},tier=word]]
    ]
    \draw [dashed] (parent.south) -- (coronal.north);
    \end{forest}
\z 
\end{multicols}
\z 

As in \REF{ex:3:16}, \isi{Velar Fronting-2} and \isi{Velar Fronting-3} cannot be collapsed into the same rule because [--low] is distinctive for vowels but not for consonants.

The data presented above illustrate that palatal fricatives and affricates surface in the neighborhood of nonlow front vowels that were originally front as well as nonlow front vowels that were originally back, e.g. the [ø] in [løççli] ‘hole-\textsc{dim}’ from (\ref{ex:3:17c}), which was originally [o] (in \ili{OHG}). Thus, \isi{i-Umlaut} (as an instance of \isi{Vowel Fronting}) fed \isi{Velar Fronting-2}.

The final set of examples (=\ref{ex:3:21}) show the distribution of dorsal fricatives after diphthongs. The generalization is that the palatal fricative occurs after a schwa-final diphthong only if the first part of that diphthong is a [--low] front vowel, as in (\ref{ex:3:21c}). If the first member of a schwa-final diphthong is back (\ref{ex:3:21a}), or [+low] and front (\ref{ex:3:21b}), then the dorsal fricative surfaces as velar. There do not appear to be examples in the original source in which a dorsal \isi{affricate} follows a diphthong whose first member is [--low] and front, but the expectation is that the dorsal affricates would behave like the dorsal fricatives.

\ea 
\label{ex:3:21}Dorsal fricatives (from /x/) after diphthongs:
\ea \label{ex:3:21a}\begin{tabular}[t]{@{}p{2cm}p{2cm}p{2cm}p{2cm}>{\raggedleft\arraybackslash}p{8mm}@{}}
    buəx    & [buəx]    & Buch    & ‘book’               & 135\\
    fluəxxə & [fluəxxə] & fluchen & ‘curse-\textsc{inf}’ & 135\\
    ɡlɔ̄əx   & [glɔːəx]  & Gelenk  & ‘joint’              &  54\\
    \end{tabular}
\ex \label{ex:3:21b}\begin{tabular}[t]{@{}p{2cm}p{2cm}p{2cm}p{2cm}>{\raggedleft\arraybackslash}p{8mm}@{}}
    štr\={ɔ̈̄}əx &  [ʃtrœːəx]&  Streich& ‘prank’   &  55\\
    frɛəxx  & [frɛəxx]  & frech   &‘impudent’ &135\\
    \end{tabular}
\ex \label{ex:3:21c}\begin{tabular}[t]{@{}p{2cm}p{2cm}p{2cm}p{2cm}>{\raggedleft\arraybackslash}p{8mm}@{}}
    līəχt            &  [liːəçt] & Licht  & ‘light’  & 140\\
    fǖəχt            & [fyːəçt]  & feucht & ‘damp’   & 75 \\
    seəχχə & [seəççə]  & Sichel & ‘sickle’ & 135\\
    \end{tabular}
\z 
\z 

\begin{sloppypar}
The generalizations described above are also visible in word pairs with Umlaut-induced stem vowel alternations, e.g. [psuːəx] ‘visit’ (=⟦psūəx⟧) with [x] after a schwa-final diphthong preceded by a back vowel vs. [psyːəçç] ‘visit-\textsc{pl}’ (=⟦psǖəχχ⟧) with [çç] after a schwa-final diphthong preceded by a [--low] front vowel.
\end{sloppypar}

I argue that the vowel transcribed in \REF{ex:3:21} as \isi{schwa} ([ə]) is phonologically front ([coronal]) in the context after a front vowel but that it remains placeless (recall \sectref{sec:2.2.3}) in the elsewhere case, e.g. after a back vowel. The change from /ə/ to a coronal vowel is accomplished with \REF{ex:3:22}. A slightly different version of the same process is posited below for a different set of dialects (\sectref{sec:5.4}). \isi{Schwa Fronting-1} is also discussed in \sectref{sec:13.5.2} and a similar epenthetic process (\isi{Schwa Fronting-2}) in \sectref{sec:5.4} and \sectref{sec:15.3}. For general discussion see \sectref{sec:12.8.1}.\footnote{{As stated in \REF{ex:3:22}, \isi{Schwa Fronting-1} spreads [coronal] from any front vowel, including [+low] front vowels in words like [frɛəxx] ‘impudent’ from (\ref{ex:3:21b}). The reason the dorsal fricative surfaces as velar in that type of word is that /ɛ/ is [--low]. Alternatively, one could restrict the set of triggers of \isi{Schwa Fronting-1} to nonlow front vowels. Since it cannot be determined which of the two options is correct I simply leave this question open.} }

\ea%22
      \isi{Schwa Fronting-1}:\\\label{ex:3:22}
      \begin{forest}
      [,phantom
        [\avm{[−cons]} [\avm{[coronal]},name=coronal]]
        [\avm{[−cons\\+son]},name=parent [,phantom]]
      ]
      \draw [dashed] (parent.south) -- (coronal.north);
      \end{forest}
\z 

\isi{Schwa Fronting-1} makes sense from the point of view of phonetics because \isi{schwa} is usually seen as a targetless vowel whose production does not involve an active articulatory gesture (e.g. \citealt{Barry1995} for German \isi{schwa}). For that reason, \isi{schwa} is therefore highly susceptible to coarticulatory influences from neighboring segments, as expressed in \REF{ex:3:22}.

\REF{ex:3:22} is a specific instantiation of \isi{Vowel Fronting} (=\ref{ex:3:7b}). The data in \REF{ex:3:21} require that \isi{Schwa Fronting-1} \isi{feeds} \isi{Velar Fronting-2}, which is precisely what one would expect in a dialect like \ipi{Rheintal} with a transparent distribution of velars and palatals. For example, in the word [liːəçt] ‘light’ (/liːəxt/ from \ref{ex:3:21c}), the feature [coronal] spreads from /iː/ to \isi{schwa}, at which point the derived front vowel spreads the inherited [coronal] feature to /x/, thereby creating the palatal fricative [ç]. For \isi{transparency} I transcribe the fronted realization of \isi{schwa} with a diacritic: /liːəxt/→{\textbar}liːə̟xt{\textbar}→[liːə̟çt]. (Here and below I enclose sounds at an intermediate synchronic stage in vertical lines, e.g. {\textbar}x{\textbar}). The \isi{feeding} relationship described here is a specific instantiation of Dialect B from \figref{fig:2.6}.\il{High Alemannic|)}

\section{{Central} {Bavarian} {(part} {1)}}\label{sec:3.5}\il{Central Bavarian|(}

The occurrence of [x] and [ç] as positional variants (allophones) after any coronal sonorant is a feature of \il{Central Bavarian}CBav (as well as the related UG dialects \il{North Bavarian}NBav and \il{East Franconian}EFr). By contrast, \il{South Bavarian}SBav often preserves [x] even in the context of front vowels. An example of a non-velar fronting (Stage 1) \il{South Bavarian}SBav place is \ipi{Imst} (\citealt{Schatz1897}; \mapref{map:3}). According to Schatz’s phonetic description (p. 9), the ach-Laut is articulated on the back part of the soft palate (“am hinteren weichen Gaumen”), regardless of what kind of sound precedes, e.g. [tsøx] ‘tick’. \citet[85]{Hathaway1979} investigated the same dialect over seventy years later and did not detect a change.

\citet{Noelliste2017} describes the realization of dorsal fricatives for speakers in the Austrian town of \ipi{Ramsau am Dachstein} in the state of Styria (Steiermark;  \mapref{map:3}). The dialect is discussed below as a representative example of a velar fronting variety of \il{Central Bavarian}CBav. \mapref{map:4} (\il{North Bavarian}NBav and \il{East Franconian}EFr) is given here for reference, even though the varieties depicted on the that map are not discussed until later chapters.

The phonemic monophthongs of \ipi{Ramsau am Dachstein} consist of the front vowels /i ɪ e ɛ/ and the back vowels /u ʊ o ɔ ɑː ɑ ə/. The phonemic diphthongs are /ɑi ɔi ɑu/, although I do not discuss /ɔi/ because of its rarity. The dialect also has diphthongs that Noelliste considers to be synchronically derived from monophthongs, e.g. [eə] (from /e/). The only dorsal fricatives are [x] and [ç], which stand in an allophonic relationship in postsonorant position as in \REF{ex:3:1}. Neither sound occurs word-initially.

The following data illustrate that [x] (<\ili{WGmc} \textsc{\textsuperscript{+}}[x k]) surfaces after back vowels (=\ref{ex:3:23a}) and [ç] after front vowels (=\ref{ex:3:23b}). As noted above, \citet{Noelliste2017} demonstrates that there is an optional rule (\isi{Diphthongization}) converting tense monophthongs (front or back) to diphthongs ending in a back vowel, e.g. /e/ can be realized as [e] or [eə] and /o/ as [o] or [oʊ]. Example (\ref{ex:3:23c}) is important because it shows that [{ҫ] surfaces as expected after [e] but that [x] occurs after the derived diphthong [e{ə}]. The realization of the dorsal fricative as [x] after [e{ə}] is expected because the second component of the diphthong ([{ə}]) is back (cf. the example [n{ə}x] ‘after’ from \ref{ex:3:23a}). Examples like [se{ə}xi] indicate that \isi{Schwa Fronting-1} (=\ref{ex:3:22}) is not active. No examples are present in Noelliste’s corpus for dorsal fricatives after [}ɔ ɛ], which she considers to be accidental gaps.\pagebreak

\begin{map}[hp]
% \includegraphics[width=\textwidth]{figures/VelarFrontingHall2021-img003.png}
\includegraphics[width=\textwidth]{figures/Map3_3.3.pdf}
 \caption[South Bavarian and Central Bavarian]{South Bavarian (\il{South Bavarian}SBav) and Central Bavarian (\il{Central Bavarian}CBav). White squares indicate assimilatory postsonorant velar fronting, shaded squares nonassimilatory postsonorant velar fronting and circles the absence of postsonorant velar fronting. 44--53 are German-language islands. 1=\citet{Schatz1897}, 2=\citet{Schatz1903}, 3=\citet{Egger1909}, 4=\citet{Gröger1924}, 5=\citet{Insam1936} (\ipi{Naturns}), 6=\citet{Insam1936} (Passeier), 7=\citet{Kurath1965}, 8=\citet{Hathaway1979}, 9=\citet{Moosmüller1991}, 10=\citet{Moosmüller1991}, 11=\citet{Stein-Meintker2000}, 12=\citet{Kollmann2007}, 13=VALTS (Steeg), 14=VALTS (\ipi{Ötztal}), 15=\citet{Gartner1900}, 16=\citet{Schwäbl1903}, 17=\citet{Seemüller1908a}, 18=\citet{Seemüller1909a}, 19=\citet{Seemüller1909b}, 20=\citet{Pfalz1911}, 21=\citet{Bíró1918}, 22=\citet{Haasbauer1924}, 23=\citet{Mindl19241925}, 24=\citet{Kubitschek1926}, 25=\citet{Kufner1957}, 26=\citet{Kufner1960}, 27=\citet{Kufner1961}, 28=\citet{Keller1961} (\ipi{Linz}), 29=\citet{Keller1961}, (\ipi{Gmünden}), 30=\citet{Maier1965} (Kiefersfelden), 31=\citet{Maier1965} (Isarwinkel), 32=\citet{BethgeBonnin1969}, 33=\citet{Ibrom1971}, 34=\citet{Gladiator1971}, 35=\citet{Manherz1977}, 36=\citet{Zehetner1978}, 37=\citet{Moosmüller1987}, 38=\citet{Moosmüller1991}, 39=\citet{Noelliste2017}, 40=SBS (Grafrath), 41= SBS (Weilheim), 42=SNiB (Heining), 43=SNiB (Dorfbach), 44=\citet{Tschinkel1908}, 45=\citet{Bacher1905}, 46=\citet{Schweizer1939}, 47=\citet{Lessiak1959}, 48=\citet{Stolle1969}, 49=\citet{Mayer1971}, 50=\citet{Kranzmayer1981}, 51=\citet{Wolf1982}, 52=\citet{Lipold1984}, 53=\citet{Rowley1986}.}
 \label{fig:3.3}\label{map:3}
\end{map}

\begin{map}[hp]
%%please move the includegraphics inside the {figure} environment
% \includegraphics[width=\textwidth]{figures/VelarFrontingHall2021-img004.png}
\includegraphics[width=\textwidth]{figures/Map4_3.4.pdf}
\caption[East Franconian and North Bavarian]{East Franconian (\il{East Franconian}EFr) and North Bavarian (\il{North Bavarian}NBav). Squares indicate postsonorant velar fronting and circles the absence of velar fronting. 1=\citet{Hedrich1891}, 2=\citet{HertelHertel1902}, 3=\citet{Braun1906}, 4=\citet{Dietzel1908}, 5=\citet{Gerbet1908}, 6=\citet{Blumenstock1911}, 7=\citet{Batz1911} 8=\citet{Knupfer1912}, 9=\citet{MSchmidt1912}, 10=\citet{Heilig1912}, 11=\citet{Dellit1913}, 12=\citet{Kaupert1914}, 13=\citet{Sander1916}, 14=\citet{Meinel1932}, 15=\citet{Roedder1936}, 16=\citet{Werner1961}, 17=\citet{Kober1962}, 18=\citet{Bock1965}, 19=\citet{Steger1968}, 20=\citet{Hirsch1971}, 21=\citet{Diegritz1971}, 22=\citet{Trukenbrod1973}, 23=\citet{Jakob1985}, 24=\citet{Schnabel2000}, 25=\citet{Gradl1895}, 26=\citet{Gebhardt1907}, 27=\citet{Eichhorn1908}, 28=\citet{Seemüller1908b}, 29=\citet{Hain1936}, 30=\citet{Gütter1962a}, 31=\citet{Gütter1962b}, 32=\citet{Gütter1963a}, 33=\citet{Gütter1963b}, 34=\citet{Dozauer1967}, 35=\citet{Schödel1967}, 36=\citet{BethgeBonnin1969} (Kreis \ipi{Wunsiedel}), 37=\citet{BethgeBonnin1969} (Kreis \ipi{Schwabach}), 38=\citet{Denz1977}, 39=\citet{Götz1987}, 40=\citet{Bachmann2000}, 41=SBS (Raitenbuch), 42=SBS (Dettenheim), 43=SBS (Mörnsheim), 44=SNiB (Zinzenzell), 45=SNiB (Herrnsaal), 46=SNiB (Atting), 47=SMF (Heuberg), 48=SMF (Ebenried), 49=SNOB (Miltach).}
\label{fig:3.4}\label{map:4}
\end{map}\clearpage

\ea Postvocalic [x] and [ç] (from /x/):\label{ex:3:23}%23
\ea \label{ex:3:23a}\begin{tabular}[t]{@{}p{2cm}p{2cm}p{2cm}p{2cm}>{\raggedleft\arraybackslash}p{8mm}@{}}
\relax[ksuxt]   & gesucht    &\mbox{‘search-\textsc{{}part}’}\\
\relax[ɔfʊx]    & einfach    &‘simple’\\
\relax[voxŋ̩]   &  Woche     & ‘week’\\
\relax[nəx]     & nach       &‘after’\\
\relax[sɑxɛ]    & Sache      &‘thing’\\
\relax[ə hɑuxl̩]&  ein Hauchl& ‘a hint’\\
\end{tabular}
\ex \label{ex:3:23b}\begin{tabular}[t]{@{}p{2cm}p{2cm}p{2cm}p{2cm}>{\raggedleft\arraybackslash}p{8mm}@{}}
\relax [si{ҫ{ɐ}]} & sicher  & ‘certainly’\\
\relax [pflɪ{ҫt]} & Pflicht &  ‘duty’\\
\relax [ʀeç]      & Reh     & ‘deer’\\
\relax [ʀɑi{ҫ]}   &  Reich  & ‘empire’\\
\end{tabular}
\ex \label{ex:3:23c}\begin{tabular}[t]{@{}p{2cm}p{2cm}p{2cm}p{2cm}>{\raggedleft\arraybackslash}p{8mm}@{}}
\relax [seҫi], [se{ə}xi] & sehe ich & ‘I see-\textsc{1sg}’
\end{tabular}
\z
\z 

Examples like the ones in \REF{ex:3:23} are captured by analyzing [x] and [{ҫ] as underlyingly /x/, which surfaces as [ç] after a coronal sonorant by \isi{Velar Fronting-1} (=\ref{ex:3:3}). I discuss below why the trigger for fronting is the set of coronal sonorants and not the set of front vowels.}

Optional forms as in (\ref{ex:3:23c}) show that \isi{Velar Fronting-1} is fully transparent. The realization [seəxi] illustrates that \isi{Diphthongization} (Diphth) preempts \isi{Velar Fronting-1}; see (\ref{ex:3:24a}). That this is a \isi{bleeding} relationship is shown in (\ref{ex:3:24b}) where \isi{Diphthongization} incorrectly \isi{counterbleeds} \isi{Velar Fronting-1} in /sex-i/. The example /ʀɑix/ shows that \isi{Diphthongization} and \isi{Velar Fronting-1} do not interact.

\ea%24
    \label{ex:3:24}
    \begin{multicols}{2}\raggedcolumns
\ea \label{ex:3:24a} \begin{tabular}[t]{@{}lll@{}}
              &  /sex-i/         &  /ʀɑix/       \\    
     Diphth   &   seəxi          &  -{}-{}-{}-{}-\\     
     Vel Fr-1 &  {}-{}-{}-{}-{}- &  ʀɑi{ҫ}       \\
              & [seəxi]          & [ʀɑi{ҫ]}      \\    
              & ‘I see-\textsc{1sg}’          & ‘empire’      \\
    \end{tabular}\columnbreak
\ex \label{ex:3:24b} \begin{tabular}[t]{@{}lll@{}}
            &  /sex-i/ &  /ʀɑix/          \\
   Vel Fr-1 &  seç-i   & ʀɑi{ҫ}           \\
    Diphth  &  seəç-i  &  -{}-{}-{}-{}-{}-\\
            & *[seəçi] &  [ʀɑi{ҫ]}        \\
    \end{tabular}
\z \end{multicols}
\z 

The \isi{bleeding} relationship in (\ref{ex:3:24a}) is a specific example of Dialect D from \figref{fig:2.7}.

\begin{sloppypar}
An important difference between \ipi{Ramsau am Dachstein} and \ipi{Maienfeld}\slash\ipi{Rheintal} (\sectref{sec:3.3}, \sectref{sec:3.4}) is that the one rhotic consonant is coronal ([r]) in \ipi{Maienfeld}\slash\ipi{Rheintal}, but uvular ([ʀ]) in \ipi{Ramsau am Dachstein}, as in the final two examples in (\ref{ex:3:23b}). I follow Noelliste in analyzing /ʀ/ as phonologically [dorsal]. As in \il{Standard German}StG (and many other regional varieties discussed below), /ʀ/ has the back vowel allophone [ɐ] -- the vocalized-r -- in coda position. The data in (\ref{ex:3:25a}) illustrate [ʀ]{\textasciitilde}[ɐ] alternations in which the consonantal sound occurs in the onset and the vocalized sound in the coda. The data in (\ref{ex:3:25b}) are significant because the dorsal fricative /x/ surfaces without change as [x] after the vocalized-r. Noelliste demonstrates that the dialect also vocalizes /l/ to the front vowel [ɪ] in coda position. A representative example (from underlying /mɔlx/) is provided in (\ref{ex:3:25}c). Note that the sound following the derived front vowel ([ɪ]) is palatal [ҫ] as expected because the sound to its immediate left is [coronal].
\end{sloppypar}

\ea%25
    \label{ex:3:25}
\ea \label{ex:3:25a}\begin{tabular}[t]{@{}p{2cm}p{2cm}p{2cm}p{2cm}>{\raggedleft\arraybackslash}p{8mm}@{}}
  \relax [meɐ]      & Meer    & ‘ocean’ \\
  \relax [me.ʀə]    & Meere   & ‘ocean-\textsc{pl}’\\
  \relax [pə.piɐ]   & Papier  & ‘paper’ \\
  \relax [pə.pi.ʀə] & Papiere & ‘paper-\textsc{pl}’\\
  \end{tabular}
\ex \label{ex:3:25b}\begin{tabular}[t]{@{}p{2cm}p{2cm}p{2cm}p{2cm}>{\raggedleft\arraybackslash}p{8mm}@{}}
 \relax [ʃtɔɐx] & Storch & ‘stork’  \\
 \relax [kiɐxŋ̩]&  Kirche&  ‘church’\\
  \end{tabular}
\ex  \begin{tabular}[t]{@{}p{2cm}p{2cm}p{2cm}p{2cm}>{\raggedleft\arraybackslash}p{8mm}@{}}
  [mɔɪҫ] & Molch & ‘salamander’
  \end{tabular}
\z 
\z 

The significant point is that [x] -- and not [{ҫ}] -- surfaces after [ɐ]. In fact, this is precisely what one would expect given the transparent distribution of [x] and [{ҫ}] because both [x] and [ɐ] are back ([dorsal]) sounds. The realization of /x/ as [x] after the vocalized-r is not simply true for \ipi{Ramsau am Dachstein}. Instead, it is a general characteristic of \ili{StAG}, a point is stressed by \citet{hildenbrandt2013} and \citet{MoosmüllerBrandstätter2015}. The data in those sources point to the occurrence of [ҫ] after front vowels and [x] after all back vowels, including the vocalized-r, e.g. [kiɐxɛ] ‘church’.  See also \citet[12]{Capell1979}, who notes that the occurrence of [x] after [ɐ] is a general pattern for Bav dialects. 

The articulation [ɐx] in a word like [ʃtɔɐx] ‘stork’ can be contrasted with other varieties of German, which have opaque palatal [{ҫ}] in that context, e.g. [ʃtɔɐ{ҫ] ‘stork’ in} \il{Standard German}StG (\sectref{sec:1.2} and \sectref{sec:17.3.1}).

The realization of /l ʀ/ as [ɪ ɐ] in \REF{ex:3:25} is accomplished with \REF{ex:3:26}. Following Noelliste, the underlying sonorant consonants consist of [+nasal] sounds (/m n ŋ/) and [--nasal]  sounds (/l ʀ/), while place features distinguish the individual members of those two groups, e.g. /l/ is [coronal] and /ʀ/ is [dorsal]. \isi{Liquid Vocalization} changes the [dorsal] rhotic /ʀ/ into the [--consonantal] sound [ɐ], but [ɐ] retains [dorsal] since the only feature that changes in \REF{ex:3:26} is [consonantal].\footnote{The vocalization of liquids in Bav (and in other German dialects) has been well-documented in the descriptive and theoretical literature. In addition to \citet{Noelliste2017} the reader is referred to \citet[107ff.]{Schmeller1821}, \citet{Selmer1933},  \citet[119ff.]{Kranzmayer1956}, \citet{Rein1974}, \citet{Haas1983}, \citet{Merkle1984}, \citet{Glover2014}, and \citet{Noelliste2019}, not to mention the linguistic atlases for Bavaria and \ipi{Upper Austria} (\sectref{sec:1.6.2}). Map 60 in WDU (Volume 4) depicts the vocalization of /l/ to a front vowel throughout most of Austria and Bavaria.}

\ea%26
Liquid Vocalization:\label{ex:3:26}\smallskip\\
\avm{[+cons\\+son\\−nas]} → [−cons] / \_\_\_ C\textsubscript{0} ]\textsubscript{${\sigma}$}
\z 

\isi{Liquid Vocalization} (Liq Voc) and \isi{Velar Fronting-1} do not interact, as illustrated in \REF{ex:3:27}. Thus, [x] surfaces after [ɐ] given either ordering relationship. Observe that /x/ surfaces as [{ҫ}] after /l/ in /mɔlx/ because the set of triggers for \isi{Velar Fronting-1} consists of all coronal sonorants.

\ea%27
\label{ex:3:27}
\tabcolsep=3pt
\begin{multicols}{2}\raggedcolumns
\ea \begin{tabular}[t]{@{}lll@{}}
          & /ʃtɔʀx/         & /mɔlx/ \\ 
Liq Voc   & ʃtɔɐx           &  mɔɪx  \\              
Vel Fr-1  & {}-{}-{}-{}-{}- &  mɔɪ{ҫ}\\ 
          &  [ʃtɔɐx]        & [mɔɪҫ] \\              
          &  ‘stork’        & ‘salamander’
      \end{tabular}
\ex \begin{tabular}[t]{@{}lll@{}}      
         & /ʃtɔʀx/         & /mɔlx/  \\
Vel Fr-1 & {}-{}-{}-{}-{}- & mɔl{ҫ}  \\
 Liq Voc &  ʃtɔɐx          & mɔɪ{ҫ}  \\
         & [ʃtɔɐx]         & [mɔɪ{ҫ]}\\
         &    ~             &  ~  \\
    \end{tabular}
\z
\end{multicols}
\z 

There is a reason why the correct version of velar fronting for \ipi{Ramsau am Dachstein} specifies that the trigger is the set of coronal sonorants ([+sonorant, coronal]) and not the set of front vowels ([--consonantal, coronal]): A few words are attested in which coda /l/ unexpectedly surfaces as [l], but the following /x/ is realized with the palatal allophone, e.g. [val{ҫ] ‘goatgrass’. The}  pronunciation with [{ҫ] in that type of word follows directly if the set of triggers for \isi{Velar Fronting-1} is [+sonorant, coronal].}

In sum, [x] and [{ҫ] have a transparent distribution on the surface: [ҫ] occurs only after coronal sonorants and [x] after back vowels. The two sounds never contrast.}

Modern-day velar and palatal fricatives in \ipi{Ramsau am Dachstein} surface after back vowels and front vowels respectively regardless of the etymological source of those vowels. A palatal after a historical front vowel is provided in (\ref{ex:3:28a}) and a velar after a historical back vowel in (\ref{ex:3:28b}). The example in (\ref{ex:3:28c}) shows that [x] follows an etymological front vowel which now surfaces as a diphthong ending in \isi{schwa} ([eə]). That vocalic change (\isi{Diphthongization}) is a specific example of \isi{Vowel Retraction} (=\ref{ex:3:7a}) because the component of the diphthong adjacent to the dorsal fricative in words like the one in (\ref{ex:3:28c}) is back. Example (\ref{ex:3:28d}) illustrates the change from the modern-day allophone [ɐ] (/ʀ/) from the earlier [coronal] rhotic [r] (/r/). The change from coronal to dorsal was accomplished by the process I refer to below as \isi{r-Retraction} in \REF{ex:3:29}. The reconstructions in the second column below are my own.\pagebreak

\ea%28
\TabPositions{1.5cm, 1.8cm, 2cm, 3.5cm, 5cm, 8cm}
\label{ex:3:28}
\ea\label{ex:3:28a} \relax [ʀɑiҫ]   \tab  <  \tab \textsuperscript{+}[riːx]  \tab ‘empire’ \tab  cf. MHG \textit{rīch(e)}   \tab (from \ref{ex:3:23b})
\ex\label{ex:3:28b} \relax [nəx]    \tab  <  \tab \textsuperscript{+}[nɑːx]  \tab ‘after’  \tab  cf. MHG \textit{nāch}      \tab (from \ref{ex:3:23a})
\ex\label{ex:3:28c} \relax [seəxi]  \tab  <  \tab \textsuperscript{+}[sexi]  \tab ‘I see’  \tab  cf. MHG \textit{sehe ich}  \tab (from \ref{ex:3:23c})
\ex\label{ex:3:28d} \relax [ʃtɔɐx]  \tab  <  \tab \textsuperscript{+}[ʃtɔrx] \tab ‘stork’  \tab  cf. MHG \textit{storch(e)} \tab (from \ref{ex:3:25b})
\z 
\ex%29
    \isi{r-Retraction}:\label{ex:3:29}\\
    /r/ >  /ʀ/
\z 

From the formal perspective \isi{r-Retraction} deleted [coronal] and added [dorsal]. That change is assumed to have involved a restructuring of the underlying representation because it was obligatory and did not create alternations.\footnote{The treatment described here is also compatible with one in which \isi{r-Retraction} is active as a synchronic rule. I leave this possibility open. The change from a coronal (apical) /r/ to the dorsal (uvular) /ʀ/ has been discussed at length in the literature on German dialects. One such study is \citet[29]{Wiese2003}, who observes that the change is a very recent one in the speech of post-war actors. \citet{Ehlers2021} is an in-depth study of the abrupt shift from /r/ to /ʀ/ in the mid-twentieth century among LG speakers in Mecklenburg. A similar study for NLGm is \citet[32-33]{Wilcken2013}.}

Before I continue my discussion of velar fronting in \ipi{Ramsau am Dachstein} from the diachronic perspective I provide some background on the phonology of rhotics necessary to better understand the function of \isi{r-Retraction}, since that sound change plays an important role in a number of case studies investigated below. A number of proposals have been made concerning the nature of the rhotic consonant phoneme in the history of Gmc; a few of those studies penned in the modern era include \citet{Runge1973}, \citet{Howell1991}, \citet{KingBeach1998}, \citet{Denton2003}, and \citet{Kostakis2015}. Some of that earlier research has proposed that the phonetic variation involving the manner and place of articulation for the rhotic consonant in modern German (\citealt{Kohler1977}, \citealt{Hall1993}) was already present in early Gmc and that the different realizations of the early Gmc rhotic can shed light on sound changes that were triggered by it.

A significant generalization that is sometimes missed in that earlier discussion is that there  never was a single variety of Gmc with a rhotic displaying either a manner contrast (trill, approximant, flap) or place contrast (alveolar, velar, uvular). Since the present discussion concerns itself with the place dimension, I state the following generalization which is true for German dialects without exception: There are German dialects with a [coronal] rhotic (/r/) and those with a [dorsal] rhotic (/ʀ/), but no variety of German contrasts the two sounds. In the present framework I therefore posit that the one rhotic phoneme can differ from dialect to dialect in terms of its distinctive features; recall the two structures posited in (\ref{ex:2:4c}, \ref{ex:2:4d}) in \sectref{sec:2.2.2}. Thus, there are dialects like \ipi{Erdmannsweiler}, \ipi{Maienfeld}, \ipi{Rheintal}, and \ipi{Upper Austria} (see \sectref{sec:3.6}) where /r/ is phonologically [coronal], as well as ones like \ipi{Ramsau am Dachstein} where /ʀ/ is phonologically [dorsal].

\isi{r-Retraction} in \REF{ex:3:29} can be viewed as a sound change that has the function of changing the [coronal] rhotic phoneme into a [dorsal] rhotic phoneme. The example given earlier in (\ref{ex:3:28d}) means that \ipi{Ramsau am Dachstein} was once a dialect with /r/ and that \isi{r-Retraction} restructured it to /ʀ/. Evidence for my claim is that the rhotic phoneme in the broader region (Austria) is primarily [coronal] (/r/) and that areas that once had /r/ now have /ʀ/. See in particular \citet[121]{Kranzmayer1956}, who observes that Bav dialects with the [dorsal] rhotic -- Zäpfchen-r (“uvular-rˮ) in Kranzmayer’s terms -- are gradually spreading throughout several regions in Austria where the [coronal] rhotic -- Kranzmayer’s Zungen-r (“tongue [tip]-rˮ) -- was once predominant.\footnote{{It is often assumed in the traditional literature that the original language (\ili{PGmc}) had /r/ and that modern German dialects with /ʀ/ were therefore all innovative. The treatment of \ipi{Ramsau am Dachstein} described above is consistent with that approach. Alternatively, one could argue that even in earlier stages of Gmc (e.g. \ili{OHG}, \ili{MHG}) dialects with /r/ and dialects with /ʀ/ coexisted side by side; note that the latter approach is more in line with the findings of \citet{Howell1991} than the former one. In this book I do not discuss cases where a [dorsal] rhotic is preserved from an earlier system with a [dorsal] rhotic, although I do not deny that that type of system could be attested. It is important to stress that the [dorsal] rhotic phoneme in many of the case studies discussed below (\chapref{sec:7}) must have been [coronal] at an earlier stage.}}

\isi{Velar Fronting-1} in modern-day \ipi{Ramsau am Dachstein} (Stage 2) arose out of Stage 1, where /x/ surfaced invariably as [x]. The two stages referred to here are depicted in \REF{ex:3:30} for three of the items presented above:\il{Standard German}

\ea%30
    \label{ex:3:30}
\begin{tabular}[t]{@{}p{2cm}p{2cm}p{2cm}p{2cm}>{\raggedleft\arraybackslash}p{8mm}@{}}
  /pflɪxt/   & /sɑxɛ/  &  /stɔrx/  &          \\\relax
  [pflɪxt]   & [sɑxɛ]  &  [stɔrx]  &   Stage 1\\\tablevspace
  /pflɪxt/   &  /sɑxɛ/ &  /ʃtɔʀx/  &          \\\relax
  [pflɪ{ҫt]} &  [sɑxɛ] &  [ʃtɔɐx]  &  Stage 2\\\tablevspace
   \textit{Pflicht} & \textit{Sache}   &    \textit{Storch} & StG\\
  ‘duty’            &   ‘thing’        &     ‘stork’        &       \\
\end{tabular}
\z 

As noted above, the realization of /x/ as [x] after both back and front vowels at Stage 1 attested in \il{South Bavarian}SBav varieties, e.g. \ipi{Imst} (\citealt{Schatz1897}, \citealt{Hathaway1979}; \mapref{map:3}). The coronal articulation of the rhotic ([r]) is the realization of /r/ in the \il{Swabian}Swb and \il{High Alemannic}HAlmc varieties discussed above (\mapref{map:1} and \mapref{map:2}). [r] (/r/) is also the realization among speakers of \il{South Bavarian}SBav spoken in the \ipi{Oberinntal} (Upper Inn Valley) to the West of \ipi{Innsbruck} (as observed by \citealt{Schatz1897}: 6, 11). The same point holds for the \il{South Bavarian}SBav variety in \ipi{Samnaun} in the far eastern part of Switzerland (\citealt{Gröger1924}: 126; \mapref{map:3}).

At some point at Stage 1, /x/ was slightly fronted in the context after a front vowel (i.e. it was \isi{prevelar}), and at Stage 2 that coarticulatory process of velar fronting was phonologized as \isi{Velar Fronting-1}, which now applies categorically after front vowels and coronal consonants (=[+sonorant, coronal]).

The pattern of velar fronting in \ipi{Ramsau am Dachstein} exemplifies the default case discussed earlier for \ipi{Erdmannsweiler} (\sectref{sec:3.2}) in the sense that /x/ surfaces as [{ҫ}] after any coronal sonorant. That same default pattern is the one attested in other varieties of \il{Central Bavarian}CBav spoken in both Austria and Germany (Bavaria). One Austrian variety documenting the presence of the default pattern approximately one century ago is the phonetic study of consonants and vowels in \ipi{Marchfeld} (\citealt{Pfalz1911}; \mapref{map:3}). The material in the latter source reveals that [{ҫ}] only surfaces after front vowels and [x] after back vowels. Since liquids are vocalized (as in \ref{ex:3:25b}, \ref{ex:3:25}c), there do not appear to be examples in \citet{Pfalz1911} where dorsal fricatives surface after consonants. As in \ipi{Ramsau am Dachstein}, [x] surfaces after the vocalized-r ([ɐ]).

The same default pattern involving [x] and [{ҫ}] has been observed for well over a century in descriptions of \il{Central Bavarian}CBav dialects spoken in Bavaria. One older source stating that the palatal only occurs after a front vowel and the velar after a back vowel is \citet[46]{Schwäbl1903} for the \ipi{Rot-Tal} region (\mapref{map:3}). \textcites[178--179]{Kufner1957}[12--13]{Kufner1960} makes the same observation concerning the realization of [x] and [ҫ] in the same region (\mapref{map:3}). The status of velar fronting in Bav with particular reference to Lower Bavaria is the topic of \chapref{sec:13}.

\section{{Central} {Bavarian} {(part} {2)}}\label{sec:3.6}

\citet{Haasbauer1924} provides a historical description of the consonants and vowels of a broad \il{Central Bavarian}CBav-speaking region in the Austrian state of \ipi{Upper Austria} (Oberösterreich; \mapref{map:3}).

The phonemic monophthongs and diphthongs differ slightly from community to community. The data presented below have dorsal fricatives in the neighborhood of front vowels (/i ɪ e ɛ/), back vowels (/u o ɔ ɑː ɑ ə/), diphthongs ending in a front vowel (/ɔɪ æɛ/) or diphthongs ending in a back vowel (/uɒ/). The only dorsal fricatives are [x] and [ç], which stand in an allophonic relationship in postsonorant position as in \REF{ex:3:1}. Neither sound occurs word-initially.

[x] surfaces after back vowels (=\ref{ex:3:31a}) and [ç] after front vowels (=\ref{ex:3:31b}) or the vocalized /l/ (=\ref{ex:3:31c}); see \citet[100]{Haasbauer1924} for discussion of the phonetics of [x] and [ç]. No data were found in the original source in which /r/ occurs before /x/. The diachronic source for [x]/[ç] is \ili{WGmc} \textsc{\textsuperscript{+}}[k x].

\ea%31
 \relax [x] and [ç] (from /x/):\label{ex:3:31}
\ea\label{ex:3:31a}\begin{tabular}[t]{@{}p{2cm}p{2cm}p{2cm}p{2cm}>{\raggedleft\arraybackslash}p{8mm}@{}}
khuxü̜ &  [kʰuxʏ]& Küche     & ‘kitchen’&   92\\
woxɒ   & [woxɒ]  &Woche     & ‘week’    &   91\\
bǫxɒ   & [bɔxɒ]   & backen  & ‘bake-\textsc{inf}’  &  107\\
suɒxɒ  & [suɒxɒ] & suchen  & ‘search-\textsc{inf}’&   107\\
ɡɑ̄x    & [gɑːx]   & jäh     & ‘abruptly’&  92\\
bɑxd   & [bɑxt]   & Gebäck  & ‘pastry’  & 107\\
\end{tabular}
\ex\label{ex:3:31b}\begin{tabular}[t]{@{}p{2cm}p{2cm}p{2cm}p{2cm}>{\raggedleft\arraybackslash}p{8mm}@{}}
siχɒ       & [siçɒ]       & sicher   & ‘certainly’ &107\\
seχtɒ      & [seçtɒ]      & Gefäß    & ‘container’ &89 \\
šlęχd      & [ʃlɛ{ҫt]}    & schlecht & ‘bad’       &107\\
ǫ\k{i}χŋ   & [ɔɪ{ҫŋ{̍}]}  &  Eiche   &  ‘oak tree’ & 97\\
fæęχtn     & [fæɛ{ҫtn{̩}]}& Fichte   &  ‘spruce’   & 95\\
\end{tabular}
\ex\label{ex:3:31c}\begin{tabular}[t]{@{}p{2cm}p{2cm}p{2cm}p{2cm}>{\raggedleft\arraybackslash}p{8mm}@{}}
mu\k{i}χ & [muɪҫ] & Milch  & ‘milk’ & 90
\end{tabular}
\z
\z 

The data in \REF{ex:3:31} are captured by analyzing [x] and [{ҫ] as underlyingly /x/, which surfaces as [ç] after a coronal sonorant by \isi{Velar Fronting-1} (=\ref{ex:3:3}).}

The importance of the patterning of dorsal fricatives in \ipi{Upper Austria} lies in the realization of /r/. In most areas in \ipi{Upper Austria}, that sound is coronal [r] in word-initial position (\ref{ex:3:32a}) or in a word-internal onset (\ref{ex:3:32b}). Haasbauer describes the sound as an untrilled dental-r (“ungerolltes Zungen-r”), although he also notes that some areas have a dorsal (uvular) articulation (“Zäpfchen-r”; \citealt{Haasbauer1924}: 100). The most significant examples are the ones in \REF{ex:3:33}, which illustrate the realization of /r/ in coda position after a vowel and before a fortis obstruent. The generalization is that /r/ surfaces as [x] after a back vowel in (\ref{ex:3:33}a) or as [{ҫ}] after a front vowel in (\ref{ex:3:33}b). The data in \REF{ex:3:33} are typical of the \ipi{Hausruckviertel}, although similar examples obtain elsewhere, e.g. in the region around Ebensee and in the northwest of the Salzkammergut.\footnote{{A number of studies have documented the realization of the rhotic phoneme as a fortis dorsal fricative before fortis sounds like [t] in varieties of Bav. See, for example, \citet[77--78]{Schönberger1934}, \citet[203--207]{Roitinger1954}, \citet[124--127]{Kranzmayer1956}, and \citet[298--299]{Zehetner1978}. In contrast to Haasbauer, the aforementioned authors employ a single symbol representing a fortis dorsal fricative (e.g. ⟦x⟧ or ⟦χ⟧) without saying explicitly whether or not that sound can be realized as a palatal. The linguistic atlas for \ipi{Upper Austria} (SAO) indicates certain parts of \ipi{Upper Austria} (e.g. to the north and west of Wels) where etymological /r/ surfaces as a fricative after a back vowel. For example, it is stated in the commentary for Map I 64 for the word} {\textit{schwarz}} {‘black’ that the fricative realization for /r/ is velar. /r/ is likewise realized as a fricative after a front vowel in the word} {\textit{Herz}} {‘heart’ on Map I 98, but that map does not distinguish velar from palatal place of articulation. The same drawback holds for Maps 69 (for} {\textit{Wort} }{‘word’) and Map 70 (for} {\textit{Herzen}} {‘heart-}{\textsc{dat}.\textsc{sg}}{) in the} {\textit{Kleiner Deutscher Sprachatlas} }{(KDSA). According to those maps, there is a region in Austria between \ipi{Innsbruck}, \ipi{Salzburg}, and \ipi{Linz}, as well as parts of Bavaria to the south and west of Munich, where the /rt/ and /rts/ sequences in those two words are realized as ⟦cht⟧ and ⟦chz⟧ respectively. There can be little doubt that ⟦ch⟧ on the KSDA maps represents a (fortis) dorsal fricative, but it is not possible -- given the broad transcription -- to conclude that it is [x] or [ç]}}

\ea%32
    \label{ex:3:32} Postrhotic [x] and [ç] (from /r/):\\
\ea\label{ex:3:32a} \begin{tabular}[t]{@{}p{2cm}p{2cm}p{2cm}p{2cm}>{\raggedleft\arraybackslash}p{8mm}@{}}
ruɒ   &   [ruɒ] &  Ruhe & ‘quiet’ &    105\\
\end{tabular}
\ex\label{ex:3:32b} \begin{tabular}[t]{@{}p{2cm}p{2cm}p{2cm}p{2cm}>{\raggedleft\arraybackslash}p{8mm}@{}}
mǫriŋ &   [mɔ.riŋ] & morgen & ‘tomorrow’ & 105\\
\end{tabular}
\z 
\z

\ea%33
    \label{ex:3:33}Postrhotic [x] and [ç] (from /r/):
\ea \begin{tabular}[t]{@{}p{2cm}p{2cm}p{2cm}p{2cm}>{\raggedleft\arraybackslash}p{8mm}@{}}
ɡfiɒxd & [gfiɒxt] &  geführt &  ‘lead-\textsc{part}’ &  105\\
šwǫxds & [ʃwɔxts] & schwarz  & ‘black’              &  105\\
\end{tabular}
\ex \begin{tabular}[t]{@{}p{2cm}p{2cm}p{2cm}p{2cm}>{\raggedleft\arraybackslash}p{8mm}@{}}
meχkɒ   & [me{ҫk{ɒ}]} & merken   & ‘notice-\textsc{inf}’     & 105\\
gšbeχt  & [gʃpe{ҫt]}  & gesperrt & \mbox{‘block\textsc{{}-part}’} &  90\\
hęχds   & [hɛ{ҫts]}   & Herz     & ‘heart’                   & 105\\
\end{tabular}
\z
\z 

The dorsal fricatives [x {ҫ] in \REF{ex:3:33} derived historically from a rhotic (\ili{WGmc}} \textsuperscript{+}[r]). My assumption is that these and similar words retain /r/ in the underlying representation in the synchronic phonology, e.g. /ʃwɔrts/ for [ʃwɔxts] and /hɛrts/ for [hɛ{ҫts]. First, speakers with the pronunciation in \REF{ex:3:33} are certainly aware of the fact that these items surface with [r] in neighboring areas of \ipi{Upper Austria}, e.g. [me{ː}rk{ɒ}] ‘notice-}\textsc{inf}’ for [me{ҫk{ɒ}] in (\ref{ex:3:33}b). Second, [r] presumably surfaces as an alternant in a word-internal onset for the items listed in \REF{ex:3:33} in which the fortis consonant following the dorsal fricative is an inflectional suffix, e.g. the /r/ in [g{ʃ}peҫt] ‘block}\textsc{{}-part}’ (from /g-ʃper-t/), cf. \il{Standard German}StG [ʃpɛʀən] ‘block\textsc{{}-inf}’ (from /ʃpɛʀ-ən/). Unfortunately, Haasbauer does not provide alternating examples.

The /r/ in \REF{ex:3:33} undergoes a change to a fortis velar fricative {\textbar}x{\textbar}, which in turn \isi{feeds} \isi{Velar Fronting-1}. It is possible to account for the change from /r/ to {\textbar}x{\textbar} in a single step; I posit two separate changes in (\ref{ex:3:34}) on the basis of my treatment of a CG (\il{Ripuarian}Rpn) dialect in which two similar rules are synchronically motivated (\sectref{sec:5.3.1}). \isi{Desonorization-1} converts /r/ into a dorsal obstruent ({\textbar}ʁ{\textbar}), while \isi{Laryngeal Assimilation-1} ensures that obstruents (including derived {\textbar}ʁ{\textbar}) shift to a fortis sound (e.g. {\textbar}x{\textbar}) before a fortis obstruent. The two changes described here are illustrated representationally in \REF{ex:3:35}. I assume that the feature [--nasal] is not present in {\textbar}ʁ{\textbar} or [x], although this point is not crucial. \isi{Laryngeal Assimilation-1} is an example of a change that increases the number of target segments for \isi{Velar Fronting-1} (=Rule W from \tabref{tab:2.wxyz}).

\ea%34
    \label{ex:3:34}
    \ea \isi{Desonorization-1}:\smallskip\\
        \avm{[+cons\\+son\\−nasal\\+cont\\coronal]}  → \avm{[−son\\dorsal]} / \_\_\_ \avm{[−son\\+fortis]}
    \ex \isi{Laryngeal Assimilation-1}:\smallskip\\
        \avm{[−son]} → \avm{[+fortis]} / \_\_\_ \avm{[−son\\+fortis]}
    \z
\ex%35
    \label{ex:3:35}
    \begin{tabular}{@{} ccccc @{}}
      /r/   &  →      &    {\textbar}ʁ{\textbar} &  →   &   {\textbar}x{\textbar}\\
            & Deson-1 &                          & Lar Assim-1 & \\
      \begin{forest}
      [\avm{[+cons\\+son\\−nasal\\+cont]} [\avm{[coronal]}]]
      \end{forest} & & 
      \begin{forest}
      [\avm{[+cons\\−son\\+cont]} [\avm{[dorsal]}]]
      \end{forest} & & 
      \begin{forest}
      [\avm{[+cons\\−son\\+cont\\+fortis]} [\avm{[dorsal]}]]
      \end{forest}
    \end{tabular}
\z

As a representative example, consider the word [hɛҫts] (from \ref{ex:3:33}b) in (\ref{ex:3:36a}): \isi{Desonorization-1} (Deson-1) creates {\textbar}{ʁ}{\textbar}, which undergoes \isi{Laryngeal Assimilation-1} (Lar Assim-1), thereby resulting in {\textbar}x{\textbar}. That derived fricative surfaces as palatal by \isi{Velar Fronting-1}. The word [{ʃ}l{ɛ}ҫt] ‘bad’ (from \ref{ex:3:31b}) in (\ref{ex:3:36a}) represents an example with /x/ after a front vowel for comparison. \isi{Laryngeal Assimilation-1} cannot \isi{counterfeed} \isi{Velar Fronting-1}, as shown in (\ref{ex:3:36b}).

\ea%36
\label{ex:3:36}
\ea\label{ex:3:36a} 
\begin{tabular}[t]{@{}lll@{}}    
            &  /hɛrts/  & /ʃlɛxt/       \\  
Deson-1     &  hɛʁts    &  -{}-{}-{}-{}-\\  
Lar Assim-1 &  hɛxts    &  -{}-{}-{}-{}-\\  
Vel Fr-1    &  hɛ{ҫts}  &  ʃlɛ{ҫt}      \\  
            & [hɛ{ҫts]} &  [ʃlɛ{ҫt]}    \\  
            & ‘heart’   &   ‘bad’       \\
\end{tabular}
\ex\label{ex:3:36b}\begin{tabular}[t]{@{}lll@{}}
             & /hɛrts/         &  /ʃlɛxt/      \\
 Deson-1     & hɛʁts           & -{}-{}-{}-{}- \\
 Vel Fr-1    &  -{}-{}-{}-{}-  & ʃlɛ{ҫt}       \\
 Lar Assim-1 &  hɛxts          &  -{}-{}-{}-{}-\\
             & *[hɛxts]        &    [ʃlɛ{ҫt]}  \\
\end{tabular}
\z 
\z 

In sum, (\ref{ex:3:36a}) demonstrates that the surface distribution of [x] and [{ҫ] is transparent and not opaque. The \isi{feeding} relationship between Laryngeal As\-sim\-i\-la\-tion-1 and \isi{Velar Fronting-1} in (\ref{ex:3:36a}) is similar to the \isi{feeding} relationship depicted for Dialect A in \figref{fig:2.6}.}

The importance of the Upper Austrian data in \REF{ex:3:33} is made clear in \sectref{sec:5.3}, where I show that \isi{underapplication} \isi{opacity} as in (\ref{ex:3:36b}) is correct in other dialects.\il{Central Bavarian|)}


\section{Conclusion}\label{sec:3.7}

In three of the UG varieties discussed above (\ipi{Erdmannsweiler}, \ipi{Maienfeld}, \ipi{Ramsau am Dachstein}) a default pattern was established, whereby a single velar target segment (/x/) is realized as the corresponding palatal ([ç]) after a set of triggers defined as the class of all coronal sonorants. Elsewhere -- that is, after a back vowel -- /x/ is realized without change as [x]. However, that default pattern is not what one encounters in the data from \ipi{Rheintal}. First, the set of triggers is narrower than the one for the default pattern in the sense that it only consists of [--low] front vowels or a coronal sonorant consonant. Second, the set of targets is broader than in the default pattern because it includes not only the fricative /x/ but also another velar sound, namely the \isi{affricate} /kx/. The conclusion is that one cannot know for certain whether or not the default pattern holds for any given velar fronting variety. It is therefore essential for any cross-dialectal study to determine for any given variety both (a) the set of velar sounds that undergo fronting (targets), and (b) the set of sounds that induce fronting (triggers).

\ipi{Rheintal} is also significant because it exemplifies an allophonic distribution of velar and palatal in word-initial position. A cross-dialectal analysis like the present one therefore needs to consider the patterning of velar and palatal sounds in word-initial position (if present) and to determine the set of targets and triggers for that fronting process.

A final point worth emphasizing is that the occurrence of palatals ([ç]) in the neighborhood of front sounds and velars ([x]) in the neighborhood of back sounds holds regardless of the historical source of the sounds that induce fronting. In all of the dialects discussed above, palatals like [ç] occur not only after front segments that were historically front, but also after front sounds that were historically back. The processes fronting sounds like /x/ to [ç] were therefore transparent because they were fed by historical changes of \isi{Vowel Fronting}, e.g. \isi{i-Umlaut}. Likewise, velars such as [x] surface after back segments that were historically back (e.g. [u o ɑ]), but also after back segments that were historically front (e.g. [ɐ] /ʀ/ from earlier [r] /r/). Sound changes creating back sounds from front sounds such as \isi{r-Retraction} and \isi{Vowel Retraction} therefore bled the fronting processes that created palatals ([ç]) from velars ([x]). Finally, the sounds undergoing velar fronting (targets) include not only underlying velar sounds (e.g. /x/) but also new velars created by other changes (\isi{Desonorization-1}, \isi{Laryngeal Assimilation-1}), e.g. [x] from /r/.
