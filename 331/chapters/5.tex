\chapter{Underapplication opacity}\label{sec:5}

\section{Introduction}\label{sec:5.1}

In the type of system referred to here, velar fronting is an active synchronic process creating palatal [ç] from velar /x/, but that system also includes many instances of velar {\textbar}x{\textbar} deriving synchronically from a different sound (/A/). The rule creating {\textbar}x{\textbar} from /A/ (=Rule W in \tabref{tab:2.wxyz}) \isi{counterfeeds} velar fronting. Hence, on the surface there are front vowel plus palatal sequences like [iç] deriving from /ix/ via velar fronting, as well as front vowel plus velar sequences like [ix], which originate from /iA/ via Rule W. Velar fronting underapplies because the {\textbar}x{\textbar} produced by Rule W potentially \isi{feeds} velar fronting, but in actuality, it does not. Examples like [ix] from /iA/ via Rule W exemplify \isi{underapplication} \isi{opacity}.

From the diachronic perspective, it is argued that velar fronting was phonologized at the end of the grammar (recall Chapters~\ref{sec:3} and~\ref{sec:4}), at which point it applied transparently because it was fed by processes already active in the grammar which created derived velars ({\textbar}x{\textbar} from Rule W). Underapplication \isi{opacity} was the result of velar fronting moving up in the derivation so that it was then counterfed by Rule W.

In \sectref{sec:5.2} I discuss a \il{Westphalian}Wph system in which the rule \isi{counterfeeding} velar fronting is Final Fortition. In \sectref{sec:5.3} I consider \il{Ripuarian}Rpn and \il{South Bavarian}SBav varieties in which the opaque velar fricative derives synchronically (and diachronically) from the rhotic phoneme (/ʀ/). In \sectref{sec:5.4} I discuss an apparent example of a rule \isi{counterbleeding} velar fronting in the synchronic grammar. I argue that there is a plausible alternative treatment in which velar fronting is transparent and conclude that the only cases in which velar fronting is opaque in the synchronic grammar involve \isi{underapplication} in the form of \isi{counterfeeding} orders. \sectref{sec:5.5} provides some discussion of two issues, namely the \textsc{domain} \textsc{narrowing} approach to language change endorsed by \citet{Bermúdez-Otero2007,Bermúdez-Otero2015} and \citet{Ramsammy2015} and linguistic/philological evidence for the historical stages presupposed in this chapter and in Chapters~\ref{sec:3} and~\ref{sec:4}. In \sectref{sec:5.6} I conclude.

\section{{Westphalian}}\label{sec:5.2}\il{Westphalian|(}

The data discussed below have been drawn from the \il{Westphalian}Wph dialect of \ipi{Rhoden}, a district of Diemelstadt, in the German state of Hesse (\citealt{Martin1925}; \mapref{map:6}).

\ipi{Rhoden} has front vowels (/i ɪ y ʏ eː ɛː ɛ øː œ æ/), back vowels (/uː u ʊ oː ɔː ɔ ɑː ɑ ə/), diphthongs ending in a front vowel (/ei ɛi øy ɑi iɛ yœ/), and diphthongs ending in a back vowel (/ou ɑu uɔ/). Martin’s symbol ⟦ɑ̇⟧ is transcribed here and below as [æ] because it is low (“niedrigˮ) and front (“[p]alatalˮ). The author also notes (p.12) that ⟦ɑ̇⟧ is pronounced like the vowel in the English word \textit{fat} (“wird gesprochen wie das \textit{a} in englisch \textit{fat}ˮ). \ipi{Rhoden} possesses the four dorsal fricatives [x ç ɣ ʝ]. The relationship between those sounds is depicted in \REF{ex:5:1} for the environment after a sonorant, which is the context I focus on below.\footnote{{The word-initial system for \ipi{Rhoden} consists of the \isi{etymological palatal} [ʝ] (/ʝ/), [g] (/g/), and [x] (/x/) in [ʃx] (/ʃx/) clusters. The [x] in [ʃx] (<\ili{WGmc}} \textrm{\textsuperscript{+}}\textrm{[sk]) surfaces even before front vowels, e.g. [ʃxip] ‘ship’. Since [g] (< \ili{WGmc}} \textrm{\textsuperscript{+}}\textrm{[ɣ]) only occurs in word-initial position \citep[51--53]{Martin1925} I treat it as a word-initial allophone of /ɣ/, as in \ipi{Dingelstedt am Huy} (\sectref{sec:8.4}). \citet[14]{Martin1925} writes that velar stops (his ⟦k⟧ and ⟦g⟧) and the velar nasal (his ⟦ŋ⟧) also have a palatal realization, although he does not describe the context, nor does he transcribe the palatals in question with separate symbols. As I point out in \sectref{sec:11.2}, the claim that there are fronted variants of [k g ŋ] is not uncommon in descriptive work during the period in which Bernhard Martin penned his grammar of \ipi{Rhoden}.} } The \ipi{Rhoden} system in \REF{ex:5:1} is strikingly similar to the one in the related \il{Westphalian}Wph variety of \ipi{Soest} (\sectref{sec:4.3}), although the crucial difference between the two is that only \ipi{Rhoden} is characterized by \isi{counterfeeding} \isi{opacity}.

\ea%1
    \label{ex:5:1}
    \begin{forest}
      [,phantom
            [/x/,calign=first [{[x]}]   [{[ç]}]]          
            [/ʝ/ [{[ʝ]}]]
      ]
    \end{forest}
\z 

The patterning of [ç] in \REF{ex:5:1} requires that vowels be marked for the distinctive feature [±low]. As indicated in \tabref{tab:5:2}, that feature is assigned to all vowels. Those vowels marked [−low] receive the feature [±high], and if two vowels share that height feature, then they are distinguished with [±tense]. All phonemic vowels are listed here with the exception of \isi{schwa}, which is placeless. The features for vowels in \tabref{tab:5:2} also hold for the individual components of the diphthongs. Most significantly, the second part of /ei ɛi øy ɑi iɛ yœ/ is [coronal] and [−low]. Four vowel pairs in \tabref{tab:5:2} are listed together under the same column (/ɛː ɛ/, /uː u/, /ɔː ɔ/, /ɑː ɑ/). The two vowels in each of those pairs differ in terms of length units, which are not given here.

\begin{table}
\caption{Distinctive features for vowels (Rhoden)\label{tab:5:2}}
\fittable{\begin{tabular}{lcccccccccccccc}
\lsptoprule
         & i & ɪ & eː & ɛː ɛ & æ & y & ʏ & øː & œ & uː u & ʊ & oː & ɔː ɔ & ɑː ɑ\\\midrule
\relax [coronal] & \ding{51} & \ding{51} & \ding{51} & \ding{51} & \ding{51} & \ding{51} & \ding{51} & \ding{51} & \ding{51} &  &  &  &  & \\
\relax [dorsal] &  &  &  &  &  &  &  &  &  & \ding{51} & \ding{51} & \ding{51} & \ding{51} & \ding{51}\\
\relax [labial] &  &  &  &  &  & \ding{51} & \ding{51} & \ding{51} & \ding{51} & \ding{51} & \ding{51} & \ding{51} & \ding{51} & \\
\relax [low] & − & − & − & − & + & − & − & − & − & − & − & − & − & +\\
\relax [high] & + & + & − & − &  & + & + & − & − & + & + & − & − & \\
\relax [tense] & + & − & + & − &  & + & − & + & − & + & − & + & − & \\
\lspbottomrule
\end{tabular}}
\end{table}

The data in \REF{ex:5:2} and \REF{ex:5:3} reveal that [x] occurs after a back vowel in (\ref{ex:5:2a}--\ref{ex:5:2g}) or the [+low] front vowel [æ] in (\ref{ex:5:2h}) and that [ç] surfaces after a [−low] front vowel in (\ref{ex:5:3}). There are no examples in which /x/ occurs after a consonant. The [x ç] in these examples derive historically from \ili{WGmc} \textsuperscript{+}[x].

\TabPositions{.2\textwidth, .4\textwidth, .58\textwidth, .82\textwidth}
\ea%3
\label{ex:5:2}
Postvocalic [x] from /x/:
\ea\label{ex:5:2a} jūxən    \tab [ʝuːxən] \tab jauchzen \tab ‘cheer-\textsc{inf}’ \tab 35 \\
\ex\label{ex:5:2b} jux      \tab [ʝux]    \tab euch     \tab ‘you-\textsc{acc/dat} \textsc{pl}’ \tab  229\\
\ex\label{ex:5:2c} b\k{u}xt \tab [bʊxt]   \tab Bucht    \tab ‘bay’      \tab 201 \\
\ex\label{ex:5:2d} dǫxtər   \tab [dɔxtər] \tab Tochter  \tab ‘daughter’ \tab  63 \\
\ex\label{ex:5:2e} nɑxt     \tab [nɑxt]   \tab Nacht    \tab ‘night’    \tab  63 \\
\ex\label{ex:5:2f} hɑux     \tab [hɑux]   \tab hoch     \tab ‘high’     \tab  223\\
\ex\label{ex:5:2g} duǫxt    \tab [duɔx]   \tab doch     \tab ‘however’  \tab  206\\
\ex\label{ex:5:2h} šlɑ̇xt   \tab [ʃlæxt]  \tab schlecht \tab ‘bad’      \tab 63\\
    hɑ̇x                     \tab [hæx]    \tab Hauch    \tab ‘breath’  \tab 221
\z
\ex%4
    \label{ex:5:3}
    Postvocalic [ç] from /x/:
\ea\label{ex:5:3a} l\k{i}χtə      \tab [lɪçtə]   \tab leicht     \tab ‘light’   \tab 63\\
    \k{i}nz\k{i}χt \tab [ɪnzɪçt]  \tab Einsicht   \tab ‘insight’ \tab 228
\ex\label{ex:5:3b} ly˛χtən        \tab [lʏçtən]  \tab leuchten   \tab ‘glow-\textsc{inf}’ \tab  63\\
\ex\label{ex:5:3c} lęχt           \tab [lɛçt]    \tab  Licht     \tab ‘light’   \tab 63\\
    tręχtər        \tab [trɛçtər] \tab  Trichter  \tab  ‘funnel’ \tab 20\\
\ex\label{ex:5:3d} fərnø̜χtərən    \tab [fərnœçtərən] \tab  ernüchtern \tab ‘sober-\textsc{inf}’\tab 211\\
   \z
\z 


The data in \REF{ex:5:3} indicate that [ç] surfaces in coda position after any one of the four [−low] short lax vowels [ɪ ʏ ɛ œ], provided that [t] follows; recall from \sectref{sec:4.3} that this is a pattern common among \il{Westphalian}Wph dialects.

Since \ipi{Rhoden} lies in the vicinity of the HG dialect continuum (\il{North Hessian}NHes), it is not surprising that the dialect has adopted copious HG loanwords, as in \REF{ex:5:4}. These data illustrate that velar [x] surfaces after any back vowel in (\ref{ex:5:4a}) or after the [+low] front vowel [æ] in (\ref{ex:5:4c}). By contrast, palatal [ç] occurs after any [−low] front vowel in (\ref{ex:5:4b}).

\ea%4
    \label{ex:5:4}
    [x]/[ç] (from /x/) after vowels in loanwords:
\ea\label{ex:5:4a} lǫx   \tab [lɔx]    \tab Loch   \tab  ‘hole’  \tab 49\\
    drɑxə \tab  [drɑxə] \tab Drache \tab ‘dragon’ \tab  49
\ex\label{ex:5:4b} ry˛χən \tab  [rʏçən] \tab riechen \tab ‘smell-\textsc{inf}’ \tab 124\\
    špr\k{i}χələn  \tab [ʃprɪçələn] \tab hochdeutsch sprechen \tab\tab\tab\tab ‘speak\textsc{{}-inf} High German’ \tab 270\\
    leiχə      \tab [leiçə]     \tab Leiche               \tab ‘body’                             \tab  49\\
    keiχən     \tab [keiçən]    \tab keuchen              \tab ‘gasp-\textsc{inf}’                \tab 231\\
    zęiχən     \tab [zɛiçən]    \tab seichen              \tab ‘piss-\textsc{inf}’                \tab 260\\
    šmęiχəln   \tab [ʃmɛiçəln]  \tab schmeicheln     \tab       ‘flatter-\textsc{inf}’             \tab 266\\
    ɡnøyχələn  \tab [gnøyçələn] \tab lächeln              \tab ‘smile\textsc{{}-inf}’             \tab 220
\ex\label{ex:5:4c} frɑ̇x             \tab [fræx]      \tab frech                \tab ‘impudent’                         \tab  49
\z
\z 

The items listed in \REF{ex:5:4b} with diphthongs ([ei ɛi øy]) are important because they show that [ç] occurs after any [−low] front vowel and not simply after the four [−low] short lax vowels [ɪ ʏ ɛ œ] in the native examples in \REF{ex:5:3}.

  Velar [ɣ] (=⟦γ⟧) surfaces in a word-internal onset after a sonorant; recall \ipi{Soest} (\sectref{sec:4.3}). The following examples exemplify the occurrence of [ɣ] after a back vowel in (\ref{ex:5:5a}), the [+low] front vowel in (\ref{ex:5:5b}), a [−low] front vowel in (\ref{ex:5:5c}), or a liquid in (\ref{ex:5:5d}).

\ea%5
    \label{ex:5:5}
    Postsonorant [ɣ] (from /ɣ/):
\ea\label{ex:5:5a}
frǭɣən \tab  [frɔː.ɣən] \tab   fragen \tab ‘ask-\textsc{inf}’  \tab  52\\
rǫɣən  \tab  [rɔ.ɣən]   \tab Roggen   \tab ‘rye’               \tab  52\\
mɑ̄ɣən  \tab [mɑː.ɣən]   \tab Magen    \tab ‘stomach’           \tab  52\\
zouɣən \tab  [zou.ɣən]  \tab  saugen  \tab ‘suck-\textsc{inf}’ \tab  34
\ex\label{ex:5:5b}   zɑ̇ɣən      \tab [zæ.ɣən]    \tab  säen      \tab ‘sow-\textsc{inf}’ \tab 260 \\
\ex\label{ex:5:5c}  iɣəl        \tab [i.ɣəl]     \tab  Igel      \tab ‘hedgehog’         \tab 52  \\
     dr\k{i}ɣərt \tab [drɪ.ɣərt]  \tab  ¾ Morgen  \tab  ‘ca. 1 acre’      \tab 52  \\
     bry˛ɣə      \tab [brʏ.ɣə]    \tab  Brücke    \tab  ‘bridge’          \tab 52  \\
     węɣə        \tab [wɛ.ɣə]     \tab  Weck      \tab  ‘bread roll’      \tab  52 \\
     šn\={ę}ɣəl  \tab [ʃnɛː.ɣəl]  \tab  Schnecke  \tab  ‘snail’           \tab  52 \\
     mø̜ɣə       \tab  [mœ.ɣə]    \tab   Mühe     \tab    ‘trouble’       \tab   54\\
     zɑiɣən      \tab [zɑi.ɣən]   \tab säugen     \tab   ‘lactate-\textsc{inf}’ \tab 52  
\ex\label{ex:5:5d} fɑ̇lɣə        \tab [fæl.ɣə]    \tab  Felge     \tab   ‘wheel rim’            \tab   52 \\
    zuǫrɣə       \tab [zuɔr.ɣə]   \tab  Sorge     \tab   ‘sorrow’               \tab  52  
   \z
\z 


The items listed in \REF{ex:5:5c} show that /ɣ/ is not a target for velar fronting (see below). \citet[52]{Martin1925} is clear on this point when he writes that [ɣ] surfaces in word-internal position even after a front vowel (“auch nach palatalem vocalˮ).

The data in (\ref{ex:5:3}--\ref{ex:5:4}) show that [ç] occurs after a [−low] front vowel and velar [x] in the elsewhere case. [x]/[ç] derive from /x/ by (\ref{ex:5:6}). Note that \isi{Velar Fronting-5} is distinct from \isi{Velar Fronting-2} (\sectref{sec:3.4}) because only the former requires that the target be specified for a laryngeal feature ([+fortis]).

\ea%6
\label{ex:5:6}\isi{Velar Fronting-5}:\\
\begin{forest}
[,phantom
  [\avm{[−low]} [\avm{[coronal]},tier=word,name=target]]
  [\avm{[−son\\+cont\\+fortis]},name=parent [\avm{[dorsal]},tier=word]]
]
\draw [dashed] (parent.south) -- (target.north);
\end{forest}
\z 

The set of triggers for \isi{Velar Fronting-5} consists of all [--low, coronal] segments. Since [±low] is distinctive only for vowels and not for consonants there is no need to specify the leftmost segment of that rule (the target) as [--consonantal]. The \isi{loanword} data in \REF{ex:5:4} are significant because they show that fronting is triggered by any [−low] front vowel and not simply by the four [−low] short lax vowels present in the native words in \REF{ex:5:3}.

As in \ipi{Soest}, many morphemes in \ipi{Rhoden} exhibit [x]{\textasciitilde}[ɣ] alternations, where both fricatives derive historically from \ili{WGmc}  \textsuperscript{+}[ɣ]. The word pairs in \REF{ex:5:7} illustrate that [x] is in the coda and [ɣ] in a word-internal onset. [x] surfaces after a back vowel in (\ref{ex:5:7a}--\ref{ex:5:7e}) or a liquid in (\ref{ex:5:7f}). By contrast, [ɣ] can surface after any type of sound, i.e. front vowel, back vowel or liquid. As indicated below, the sound underlying [x]{\textasciitilde}[ɣ] alternations as in \REF{ex:5:7} is /ɣ/.

\ea%7
\label{ex:5:7}
   [x]{\textasciitilde}[ɣ] alternations (from /ɣ/): 
\ea\label{ex:5:7a} plōx       \tab  [ploːx]    \tab Pflug   \tab ‘plow’              \tab 254\\
    pl\={ø}ɣən \tab  [pløː.ɣən] \tab pflügen \tab ‘plow-\textsc{inf}’ \tab 254
\ex\label{ex:5:7b} 
dɑx   \tab  [dɑx]    \tab Tag  \tab ‘day’  \tab 52\\
dɑɣən \tab  [dɑ.ɣən] \tab Tage \tab ‘day-\textsc{pl}’ \tab  202
\ex\label{ex:5:7c}
kynəx  \tab [ky.nəx]   \tab König  \tab ‘king’   \tab 86\\
kynəɣə \tab [ky.nə.ɣə] \tab Könige \tab  ‘king-\textsc{pl}’ \tab 86
\ex\label{ex:5:7d}
truǫx   \tab [truɔx]   \tab Trog  \tab ‘trough’   \tab 86\\
truǫɣə  \tab [truɔ.ɣə] \tab Tröge \tab  ‘trough-\textsc{pl}’ \tab  86
\ex\label{ex:5:7e}
flɑux   \tab [flɑux]    \tab flog    \tab ‘fly\textsc{{}-pret}’ \tab 74\\
flēɣən  \tab [fleː.ɣən] \tab fliegen \tab ‘fly-\textsc{inf}’     \tab 213\\
flyø̜ɣən\tab  [flyœ.ɣən]\tab  flögen \tab  ‘fly-\textsc{subj}’   \tab 26
\ex\label{ex:5:7f}
bɑlx   \tab [bɑlx]   \tab Balg  \tab ‘brat’  \tab 86\\
bɑ̇lɣə  \tab [bæl.ɣə] \tab Bälge \tab ‘brat-\textsc{pl}’ \tab  86
 \z
\z 

Similar fortis vs. lenis alternations involve other fricative (and stop) pairs, e.g. [glɑs] ‘glass’ vs. [glɛː.zə.rə] ‘glass-\textsc{pl}’.

Fortis-lenis alternations like the ones in \REF{ex:5:7} are captured with underlying lenis sound (/ɣ/), which surface as fortis in coda position by \isi{Final Fortition} (\sectref{sec:4.2}) in \REF{ex:5:8}. Nonalternating fortis sounds are underlyingly fortis, e.g. /x/ in [lɑ.xən] ‘laugh-\textsc{inf}’.

\ea%8
    \label{ex:5:8}
          \isi{Final Fortition}:

[--sonorant] → [+fortis] / {\longrule} C\textsubscript{0} ]\textsubscript{${\sigma}$}

\z 

The words in \REF{ex:5:9} are like the ones in \REF{ex:5:7} in the sense that they exhibit [x]{\textasciitilde}[ɣ] alternations derived from /ɣ/. In contrast to the items presented in \REF{ex:5:7}, the segment preceding the [x] in \REF{ex:5:9} is a [−low] front vowel. The significance of the data in \REF{ex:5:9} is that the {\textbar}x{\textbar} created by \isi{Final Fortition} does not undergo \isi{Velar Fronting-5}. Hence, the {\textbar}x{\textbar} produced by \isi{Velar Fronting-5} is opaque and not transparent.


\TabPositions{.15\textwidth, .3\textwidth, .5\textwidth, .8\textwidth}
\ea%9
    \label{ex:5:9}
           [ɣ]{\textasciitilde}[x] alternations (from /ɣ/):
\ea\label{ex:5:9a} 
z\={ę}ɣən \tab  [zɛː.ɣən] \tab sagen  \tab ‘say-\textsc{inf}’             \tab 82\\
z\={ę}x   \tab  [zɛːx]    \tab sag    \tab ‘say-\textsc{imp}.\textsc{sg}’ \tab 82\\
zięxtə    \tab [ziɛx.tə]  \tab sagte  \tab ‘say-\textsc{pret}’           \tab 82\\
əzięxt    \tab [əziɛxt]   \tab gesagt \tab  ‘say-\textsc{part}’          \tab 82\\

\ex \label{ex:5:9b}
w\={ę}x   \tab [βɛːx]   \tab Weg  \tab ‘path’   \tab 85\\
w\={ę}ɣə  \tab [βɛː.ɣə] \tab Wege \tab  ‘path-\textsc{pl}’ \tab 85\\

\ex \label{ex:5:9c}
ęɣən  \tab [ɛ.ɣən]  \tab eggen   \tab ‘harrow-\textsc{inf}’     \tab 80 \\
ięxtə \tab [iɛx.tə] \tab  eggte  \tab  ‘harrow-\textsc{pret}’ \tab 80 \\

\ex\label{ex:5:9d}
l\={ę}ɣən     \tab  [lɛː.ɣən]  \tab legen     \tab ‘place-\textsc{inf}’ \tab  240 \\
\k{u}ngǝlięxt \tab [ʊngǝliɛxt] \tab  ungelegt \tab ‘unplaced’           \tab  279 \\

\ex\label{ex:5:9e}
kreiɣən \tab [krei.ɣən] \tab  kriegen \tab ‘wage war\textsc{{}-inf}’ \tab 236\\
kreix   \tab [kreix]    \tab  Krieg   \tab ‘war’                     \tab 236\\

\ex\label{ex:5:9f}
dɑ̄ɣən      \tab  [dɑː.ɣən] \tab tagen   \tab ‘hold a meeting-\textsc{inf}’ \tab 202\\
d\={ę}xlǝk \tab [dɛːx.lǝk] \tab täglich \tab ‘daily’ \tab  203


\ex\label{ex:5:9g}
myɣən  \tab [my.ɣən]  \tab mögen    \tab ‘like-\textsc{inf}’ \tab  246\\
myxlək \tab [myx.lək] \tab  möglich \tab ‘possible’          \tab  246\\

\ex\label{ex:5:9h}
tøyɣə       \tab [tøy.ɣə]   \tab Zeuge   \tab ‘witness’     \tab 277\\
tøyxn\k{i}s \tab [tøyx.nɪs] \tab Zeugnis \tab ‘testimonial’ \tab 277\\
      \z
\z 

Martin does not provide examples for [x] (from /ɣ/) after any of the four [−low] short  front lax vowels ([ɪ ʏ ɛ œ]) present in the (native) items in \REF{ex:5:3}, but I consider this to be an accident. In particular, Martin’s description of the inflectional morphology (pp. 72--95)  implies that there should be examples in which [x] (from /ɣ/) also occurs after [ɪ ʏ ɛ œ]. For example, the [ɣ] in a word-internal onset surfaces in coda position as [x] in the imperative singular of weak verbs, e.g. [zɛːx] ‘say-\textsc{imp}.\textsc{sg}’ in (\ref{ex:5:9a}). \ipi{Rhoden} has several weak verbs with vowels like [ɪ ʏ ɛ œ] followed by /ɣ/, e.g. [ʃpɪɣən] ‘spit-\textsc{inf}’ (=⟦šp\k{i}ɣən⟧), which presumably surface with [x] after the same stem vowel in the imperative singular, but these examples are not provided in the original source.

It is interesting to consider the passages in \citet{Martin1925} describing the data presented above because  he indicates not only that he is aware of the regular patterning of [x] and [ç] but also the aberrant instances of [x]. For example, \citet[63]{Martin1925} states with respect to \REF{ex:5:3} that \ili{WGmc} \textsuperscript{+}[x] (before [t]) is realized as [ç] after a front vowel (“nach palatalen [Vocalen]ˮ). However, \citet[14]{Martin1925} also notes in the introduction that quite often one hears [x] after a front vowel (“ ...hört man sehr oft \textit{x} ... nach palatalen Vocalenˮ). It is especially significant to observe that \citet[52]{Martin1925} recognizes that the modern reflex of historical \textsuperscript{+}[ɣ] in \REF{ex:5:10} is a voiceless velar fricative in coda position \textit{even if the preceding vowel is front} (my emphasis). He writes: “Im Auslaut wird wg. \textit{g} (=\textsuperscript{+}[ɣ]) zum stimmlosen velaren Spiranten (auch bei vorausgehenden palatalen Vocalen)...ˮ

The examples given above show that it is difficult to find examples in which [x] and [ç] occur after precisely the same vowel. As noted earlier, for historical reasons [ç] (from /x/) is only attested after the four [−low] front  vowels [ɪ ʏ ɛ œ] and before [t]. By contrast, the bulk of the native examples with [x] (from /ɣ/) show that fricative occurring after vowels other than [ɪ ʏ ɛ œ]. What is more, it is clear from \REF{ex:5:3} that [ç] (from /x/) surfaces after any [−low] front vowel and that the opaque [x] derives from /ɣ/.\largerpage

Opaque examples like the ones in \REF{ex:5:9} are accommodated by ensuring that only an underlying /x/ but not the {\textbar}x{\textbar} derived from /ɣ/ by \isi{Final Fortition} (Fnl For) undergoes \isi{Velar Fronting-5} (Vel Fr-5). This is captured in \REF{ex:5:10a}, which illustrates \isi{underapplication} (\isi{counterfeeding}) \isi{opacity}.  The two examples are drawn from \REF{ex:5:9a} and \REF{ex:5:3c}.\pagebreak

\ea%11
    \label{ex:5:10}
    \begin{multicols}{2}\raggedcolumns
\ea \label{ex:5:10a}
\begin{tabular}[t]{@{} lll @{}} 
 & /zeːɣ/ & /lɛxt/\\
Vel Fr-5 & {}-{}-{}-{}-{}-{}-{}- &   lɛçt  \\  
Fnl For & zɛːx & {}-{}-{}-{}-{}-{}-      \\
        & [zɛːx] & [lɛçt]\\
& ‘say-\textsc{imp}.\textsc{sg}’  & ‘light’\\
\end{tabular}
\columnbreak
\ex\label{ex:5:10b}
\begin{tabular}[t]{@{} lll @{}} 
  &/zɛːɣ/ & /lɛxt/\\
Fnl For   & zɛːx     & {}-{}-{}-{}-{}-{}-\\
Vel Fr-5  &  zɛːç    &  lɛçt             \\
          &  *[zɛːç] &   [lɛçt]          \\
\end{tabular}
\z 
\end{multicols}
\z 

If the output of \isi{Final Fortition} were to feed \isi{Velar Fronting-5} in (\ref{ex:5:10b}), then the derived fortis dorsal fricative {\textbar}x{\textbar} in words like /zɛːɣ/ ‘say-\textsc{imp}.\textsc{sg}’ would incorrectly surface as the palatal [ç]. The distinction between an underlying /x/ and a derived {\textbar}x{\textbar} is correctly captured in \REF{ex:5:10a} if \isi{Final Fortition} \isi{counterfeeds} \isi{Velar Fronting-5}. This means that the underlying /x/ in words like /lɛxt/ ‘light’ shifts to a palatal {\textbar}ç{\textbar} before the underlying /ɣ/ in words like /zeːɣ/ ‘say-\textsc{imp}.\textsc{sg}’ becomes a fortis velar [x] by \isi{Final Fortition}. The opaque system in \REF{ex:5:10a} is a specific example of the hypothetical Dialect G from \figref{fig:2.9}.

The development of the three typical words from (\ref{ex:5:3c}) and (\ref{ex:5:9a}) is depicted in \REF{ex:5:11} for the three historical stages referred to in previous chapters. For each stage the underlying representation and the phonetic representation are provided.

\ea%11
    \label{ex:5:11}
    \begin{tabular}[t]{@{} llll @{}}
  /zɛːɣ-ən/           &   /zɛːɣ/             &   /lɛxt/                 \\
  \relax [zɛː.ɣən]           &   [zɛːx]             &    [lɛxt] & Stage 1      \\\tablevspace
  /zɛːɣ-ən/           &   /zɛːɣ/             &   /lɛxt/                 \\
  \relax [zɛː.ɣən]           &   [zɛːç]             &    [lɛçt] & Stage 2      \\\tablevspace
  /zɛːɣ-ən/           &   /zɛːɣ/             &   /lɛxt/                 \\
  \relax [zɛː.ɣən]           &   [zɛːx]             &    [lɛçt] & Stage 3      \\\tablevspace
  \textit{sagen}      &   \textit{sag}       &   \textit{Licht} & \il{Standard German}StG\\
  ‘say-\textsc{inf}’  &  ‘say-\textsc{imp}.\textsc{sg}’  &      ‘light’ \\
  \end{tabular}
\z 

It is assumed above that \isi{Final Fortition} was already present in the grammar at Stage 1; see \sectref{sec:5.5.2} for discussion. Stage 2 depicts the point where \isi{Velar Fronting-5} was phonologized at the end of the grammar. Examples like [zɛːç] ‘say-\textsc{imp}.\textsc{sg}’ and [lɛçt] ‘light’ indicate that the rule was transparent because it was fed by \isi{Final Fortition}. The dialect of \ipi{Rhoden} as it was described in 1925 by Bernhard Martin is represented by Stage 3. Stage 2 in \REF{ex:5:11} is attested in \ipi{Soest} (\sectref{sec:4.3}), which is repeated in \REF{ex:5:12}. Recall that this transparent system is a specific example of the hypothetical Dialect A from \figref{fig:2.6}.

\ea%12
    \label{ex:5:12}
    \begin{tabular}[t]{@{} lll @{}}
         & /stɪɣ-st/             &     /fʀœxt-n̩/        \\
Fnl For  & stɪx-st               &    -{}-{}-{}-{}-{}-   \\
Vel Fr-4 &   stɪçst              &     fʀœçtn̩           \\
         & [stɪçst]              &    [fʀœçtn̩]          \\
         & ‘climb-\textsc{2sg}’  &    ‘fear-\textsc{inf}’\\
    \end{tabular}
\z 

Stage 3 in \REF{ex:5:11} therefore involved the change from a transparent relationship between \isi{Final Fortition} and Vel Fr-4 in \REF{ex:5:12} to the opaque relationship in \REF{ex:5:10a}.

In \sectref{sec:2.5} I described the historical model adopted in this book, which sees change from one stage to the next as one involving a speaker pronouncing words which are then misperceived by listeners in \isi{acquisition}. It is this misparsing of sounds uttered by adults that leads to the \isi{acquisition} of the rule of velar fronting.

The change from transparent Stage 2 to opaque Stage 3 in \REF{ex:5:11} does not involve misperception. The interesting example is the shift from Stage 2 [zɛːç] ‘say-\textsc{imp}.\textsc{sg}’ to Stage 3 [zɛːx]. If a speaker (P\textsubscript{1}) utters [zɛːç] (from /zeːɣ/) at Stage 2, then the listener (P\textsubscript{2}) correctly hears [zɛːç] and hence the question is why P\textsubscript{2} would opt for the Stage 3 opaque realization ([zɛːx]) rather than the Stage 2 transparent one ([zɛːç]). The answer is that P\textsubscript{2} has adopted a feature specific \is{paradigm uniformity}\textsc{paradigm uniformity} constraint (e.g. \citealt{DowningRaffelsiefen2005}): P\textsubscript{2} posits that the place of articulation of consonants in the cells of paradigms (verb conjugations and noun/adjective declensions) must remain the same. Given that requirement, the Stage 2 alternation between velar ([ɣ]) and palatal ([ç]) in examples like [zɛː.ɣən] vs. [zɛːç] is levelled to velar at Stage 3, namely [zɛː.ɣən] vs. [zɛːx]. Since P\textsubscript{2} is already aware of fortis vs. lenis alternations like the one in \REF{ex:5:7}, (s)he has internalized \isi{Final Fortition} and knows that the fortis vs. lenis alternation in pairs of words like [zɛː.ɣən] vs. [zɛːx] requires /ɣ/ and not /x/. For these reasons, P\textsubscript{2} posits a Stage 3 grammar in which \isi{Final Fortition} \isi{counterfeeds} velar fronting.

The description of changes involving dorsal fricatives in \citegen{Martin1925} grammar can be confirmed in the 97-page appendix of that work, which consists of a list in alphabetical order (in phonetic transcription) of all of the words cited in the grammar. An examination of that list also includes a small number of items within which [x] (from /x/) unexpectedly occurs after a front vowel; see \REF{ex:5:13}.

\TabPositions{0pt, .20\textwidth, .4333\textwidth, .58\textwidth, .725\textwidth}
\ea\label{ex:5:13}
\ea\label{ex:5:13a} ɡəšx\k{i}xtə     \tab [gəʃxɪxtə]           \tab Geschichte \tab ‘history’ \tab 188, 218
\ex\label{ex:5:13b} fy˛xtə           \tab [fʏxtə]              \tab Feuchte    \tab ‘humidity’\tab 36, 216
\ex\label{ex:5:13c} nø̜xtərən        \tab  [nœxtərən]           \tab  nüchtern  \tab  ‘sober’  \tab 34, 248
\ex\label{ex:5:13d} lęxt, lęχt       \tab [lɛxt], [lɛçt]       \tab Licht      \tab ‘light’   \tab 40, 63, 87, \\ \tabto{.725\textwidth} 240
\ex\label{ex:5:13e} tręxtər, tręχtər \tab [trɛxtər], [trɛçtər] \tab Trichter   \tab ‘funnel’  \tab 20, 277
\z 
\z 

I refer to the words in \REF{ex:5:13} with [x] as irregularities to the otherwise regular process fronting /x/ after [−low] front vowels (\isi{Velar Fronting-5}). For reasons that will become clear in \sectref{sec:12.8.3} the words in \REF{ex:5:13} with the pronunciation [x] do not exemplify \isi{lexical exceptions} as that term is usually employed in the literature.

Consider first the three items in (\ref{ex:5:13a}--\ref{ex:5:13c}). \citet[218]{Martin1925} transcribes the word \textit{Geschichte} ‘history’ in \REF{ex:5:13a} with his symbols for [ɪx] in both the appendix (p. 218) and in his transcription of an informant’s recitation of a fairy tale (p. 188). One might argue that the post-[ɪ] dorsal fricative in [gəʃxɪxtə] is [x] and not [ç] because the vowel [ɪ] is preceded by [x]. This cannot be the correct interpretation because there are other words in which [ç] surfaces as expected after [ɪ] even though [x] precedes the vowel, e.g. [ʃxɪçt] ‘shift’ \citep[262]{Martin1925}. Note too that the second [x] in [gəʃxɪxtə] occurs after the vowel [ɪ] but the words [lɪçtə] ‘light’ and [ɪnzɪçt] ‘insight’ from \REF{ex:5:3} show the regular pattern whereby /x/ surfaces as [ç] after that vowel. The same point holds for [fʏxtə] ‘humidity’ in \REF{ex:5:13b} in which the [x] contrasts with the [ç] in [lʏçtən] ‘glow-\textsc{inf}’ from \REF{ex:5:3}. In [nœxtərən] ‘sober’ in \REF{ex:5:13c} the velar [x] similarly surfaces unexpectedly after the nonlow front vowel [œ], but in [fərnœçtərən] ‘sober-\textsc{inf}’ from \REF{ex:5:3}, [ç] occurs as expected after [œ].

The items in (\ref{ex:5:13d}, \ref{ex:5:13e}) differ from the ones in (\ref{ex:5:13a}--\ref{ex:5:13c}) because they exhibit both the expected pronunciation with [ç] as well as the unexpected pronunciation with [x]. The pronunciation with [x] for \REF{ex:5:13d} occurs only once (p. 240) and the realization with a palatal three times (p. 40, 63, 87). The pronunciation with [x] for \REF{ex:5:13e} is attested once (p. 277) and the expected realization with a palatal once (p. 20).

One might argue that Martin’s ⟦x⟧ in \REF{ex:5:13} is merely a transcriptional error, but I consider that interpretation to be dubious. First, as indicated in the page numbers listed in the final column of \REF{ex:5:13}, several of the irregular words are transcribed with ⟦x⟧ at more than one point in Martin’s grammar. For example, if the word \textit{Geschichte} were incorrectly transcribed with ⟦x⟧ after the vowel ⟦\k{i}⟧ on p. 218, what are the chances that Martin would make precisely the same mistake in the same word on p. 188? Note too that two other words are given on p. 218 in which Martin’s ⟦χ⟧ (=[ç]) surfaces after ⟦\k{i}⟧, namely ⟦ɡəz\k{i}χtə⟧ ‘face’ (=[gəzɪçtə]) and ⟦ɡər\k{i}χt⟧ ‘dish’ (=[gərɪçt]). Second, the aberrant items in \REF{ex:5:13} always involve ⟦x⟧ after a [−low] front vowel, but never ⟦χ⟧ after a [+low] front vowel or back vowel. That generalization correlates with the author’s observation commented on earlier that velar [x] is often heard in the front vowel context. If the ⟦x⟧ in \REF{ex:5:13} were simply a transcriptional error, then one would expect the author to also incorrectly transcribe a palatal ⟦χ⟧ after the vowels [uː u ʊ oː ɔː ɔ ɑː ɑ ə] or [æ], but no such examples are present in \citet{Martin1925}. Third, several commentators have observed that [x] can surface in the neighborhood of front vowels in ELG even when velar fronting can be shown to be active (see \sectref{sec:12.8.3} for discussion). Hence, the unexpected items in \REF{ex:5:13} are representative of LG in general.

I claim that there is a connection between the items in \REF{ex:5:13} and the opaque [x] in \REF{ex:5:9}, although further study is necessary to determine the nature of that connection. According to one scenario (Analysis A), when \isi{Velar Fronting-5} was operative at Stage 2 in \REF{ex:5:11}, it was not only transparent (because it was fed by \isi{Final Fortition}), but also regular because there were no items like the ones in (\ref{ex:5:13}). At some point still at Stage 2 irregularities emerged, e.g. the earlier realization [gəʃxɪçtə] ‘history’ was replaced with the irregular [gəʃxɪçtə] in \REF{ex:5:13a}, and then eventually more aberrant items arose. The presence of those words eventually signalled to the listener that sequences such as [ɪx] are acceptable, which then served as a catalyst for the shift from \isi{Velar Fronting-5} as a rule applying at the end of the grammar at Stage 2 to an opaque rule counterfed by \isi{Final Fortition} at Stage 3. According to a second alternative (Analysis B), it was the other way around: At Stage 2 there were no irregularities at all, and then \isi{Velar Fronting-5} moved up so that it was counterfed by \isi{Final Fortition}. According to Analysis B it was the presence of opaque examples like [zɛːx] say-\textsc{imp}.\textsc{sg}’ (from /zɛːɣ/) that signaled to the listener that [x] is acceptable after a nonlow front vowel, which then served as a catalyst for the emergence of the items in \REF{ex:5:13}.

At this point one cannot know for certain which of the two scenarios is the more likely. On the one hand, there are LG varieties referred to earlier (discussed in \sectref{sec:12.8.3}) with irregularities like the ones in \REF{ex:5:13} but no opaque forms, which would argue against Analysis B. However, there are also dialects with \isi{opacity} but without irregularities (\sectref{sec:5.3.1}), which would pose a problem for Analysis A.

To summarize, the \ipi{Rhoden} system involving \isi{underapplication} \isi{opacity} (recall \ref{ex:5:10}) is unique. Although several other varieties of German are described in \sectref{sec:5.3} where a rule creating a fortis dorsal fricative {\textbar}x{\textbar} \isi{counterfeeds} velar fronting, the derived {\textbar}x{\textbar} in those dialects derives synchronically (and diachronically) from the rhotic phoneme. By contrast, \ipi{Rhoden} is the only dialect discovered in the present survey where \isi{Final Fortition} creates {\textbar}x{\textbar} from /ɣ/, which in turn \isi{counterfeeds} velar fronting.\footnote{{There is scant evidence from a brief description of the variety of LFr of \ipi{Homberg} (\citealt{Meynen1911}; \mapref{map:8}) suggesting that there is a similar pattern attested elsewhere. \ipi{Homberg} has a version of velar fronting in which the target is /x/ and the trigger is any preceding front vowel. Meynen gives a very small number of words ending in a front vowel plus [x], but in those examples the [x] derives from [ɣ] which was followed historically by \isi{schwa}, e.g. ⟦zæ}\textrm{\textsuperscript{1}}\textrm{x⟧ ‘saw’ (cf. \il{Standard German}StG} \textrm{\textit{Säge}}\textrm{). That word can be compared to one in which [x] derives from /x/, where there was no following \isi{schwa}, e.g. ⟦ræ}\textrm{\textsuperscript{1}}\textrm{χ⟧ ‘right’ (cf. \il{Standard German}StG} \textrm{\textit{recht}}\textrm{). Since \citet{Meynen1911} does not provide enough data to draw the correct conclusions I do not discuss this example further.}}\il{Westphalian|)}

\section{Opaque realization of /r/ and velar fronting}\label{sec:5.3}

German velar fronting dialects are attested in which the rhotic consonant (/ʀ/) is realized as a fortis velar fricative ([x]) in the context after front vowels. The most detailed treatment of that opaque rhotic reported in the literature to my knowledge is a \il{Ripuarian}Rpn variety described in \sectref{sec:5.3.1}. \sectref{sec:5.3.2} investigates a strikingly similar pattern for a community of early twentieth century \il{South Bavarian}SBav speakers in \ipi{Silesia}. \sectref{sec:5.3.3} discusses the areal distribution of \isi{counterfeeding} \isi{opacity} in German dialects

\subsection{Ripuarian}\label{sec:5.3.1}\il{Ripuarian|(}

The data given below are drawn from \citet{Hall1993}, who describes and analyzes the speech of several informants living in the general vicinity between \ipi{Düsseldorf} and \ipi{Cologne} (Köln); see \mapref{map:8}. Following the original source, I refer to this \il{Ripuarian}Rpn variety as Lower Rhine German (LRG).

\begin{map}
% \includegraphics[width=.85\textwidth]{figures/VelarFrontingHall2021-img010.png}
 \centering\includegraphics[width=.85\textwidth]{figures/Map8_5.1.pdf}
 \caption[Ripuarian and Low Franconian]{Ripuarian (\il{Ripuarian}Rpn) and Low Franconian (LFr). Squares indicate postsonorant velar fronting and circles the absence of postsonorant velar fronting. 1=\citet{Rovenhagen1860}, 2=\citet{Wahlenberg1877},  3=\citet{Röttsches1877}, 4=\citet{Koch1879}, 5=\citet{Holthausen1885}, 6=\citet{Holthaus1887}, 7=\citet{Maurmann1889},  8=\citet{Jardon1891}, 9=\citet{Schmitz1893},  10=\citet{Müller1900}, 11=\citet{Münch1904}, 12=\citet{Hasenclever1905}, 13=\citet{Müller1912}, 14=\citet{Frings1913}, 15=\citet{Lobbes1915}, 16=\citet{Grass1920}, 17=\citet{Zeck1921},  18=\citet{Greferath1922}, 19=\citet{Mackenbach1924}, 20=\citet{Branscheid1927} (\ipi{Eckenhagen}), 21=\citet{Branscheid1927} (\ipi{Berghausen}), 22=\citet{Welter1929}, 23=\citet{Welter1933}, 24=\citet{Bubner1935}, 25=\citet{Welter1938}, 26=\citet{Heike1964}, 27=\citet{Heike1970}, 28=\citet{Jongen1972}, 29=\citet{Hecker1972}, 30=\citet{Heinrichs1978}, 31=\citet{CajotBeckers1979}, 32=\citet{Bister-Broosen1989}, 33--42=\citet{CornelissenEtAl1989} (Euskirchen (33), Dahlem (34), Monschau (35), Zülpich (36), Langerwehe (37), Nörvenich (38), Jülich (39), Bonn (40), Mönchengladbach (41), Heinsberg (42)), 43=\citet{Hinskens1992}, 44=\citet{Hall1993}, 45=\citet{Kreymann1994}, 46=\citet{Fuss2001}, 47=\citet{Ramisch1908}, 48=\citet{Meynen1911}, 49=\citet{Hanenberg1915}, 50=\citet{BethgeBonnin1969}, 51=\citet{Stiebels2013}.}
  \label{fig:5.1}\label{map:8}
\end{map}

LRG has the same vocalic sounds as \il{Standard German}StG (\sectref{sec:17.2}), namely front vowels /iː ɪ yː ʏ eː ɛː ɛ øː œ/, back vowels /uː u ʊ oː ɔ ɑː ɑ ə/), and diphthongs ending in a front vowel (/ɑi ɔy/) or back vowel (/ɑu/). The relationship between the two surface dorsal fricatives ([x]/[ç]), which only occur in postsonorant position, is depicted in \REF{ex:5:15}.\footnote{{\citet{Hall1993} transcribes [x] narrowly as uvular (⟦χ⟧). The \isi{etymological palatal} [ʝ] surfaces word-initially before any vowel. LRG has no [ɣ] because \isi{g-Formation-1} (\sectref{sec:4.2}) restructured \ili{WGmc}} \textrm{\textsuperscript{+}}\textrm{[ɣ] (/ɣ/) to [g] (/g/). As in \il{Standard German}StG, LRG [g] alternates with [ç] after the front vowel [ɪ], e.g. [køːnɪç] ‘king’ vs. [køːnɪgə] ‘king-\textsc{pl}’ (\sectref{sec:1.2}). I ignore data like these because they are peripheral; see \sectref{sec:17.2} for discussion.}}\largerpage[-2]

\ea%15
    \label{ex:5:15}
     \begin{forest}
      [/x/
          [{[x]}]    [{[ç]}]
      ]
  \end{forest}
\z 

[x] occurs after a back vowel in (\ref{ex:5:16a}) and [ç] after a front vowel in (\ref{ex:5:16b}) or a sonorant consonant in (\ref{ex:5:16c}).\footnote{{In contrast to some of the dialects discussed earlier (e.g. \ipi{Ramsau am Dachstein} in \sectref{sec:3.5} and \ipi{Soest} in \sectref{sec:4.3}), palatal [ç] surfaces unexpectedly in LRG after the back vowel [ɐ] (from /ʀ/), e.g. [dʊɐç] ‘through’. This is an example of a \isi{palatal quasi-phoneme}, which is discussed in detail in \chapref{sec:7} for several other regional varieties and in \sectref{sec:17.2} for \il{Standard German}StG.}}

\ea%16
    \label{ex:5:16}
          [x] and [ç] (from /x/):
\ea\label{ex:5:16a}
\begin{tabularx}{.8\textwidth}[t]{@{}XXX@{}}
\relax
[tuːx] & Tuch  &  ‘towel’ \\
\relax [bʊxt] & Bucht &    ‘bay’ \\
\relax [hoːx] & hoch  &   ‘high’ \\
\relax [kɔx]  & Koch  &   ‘cook’ \\
\relax [bɑx]  & Bach  &  ‘stream’\\
\relax [nɑːx] & nach  &   ‘after’\\
\relax [bɑux] & Bauch & ‘stomach’\\
\end{tabularx}

\ex\label{ex:5:16b} 
\begin{tabularx}{.8\textwidth}[t]{@{}XXX@{}}
\relax
[ziːç]   & siech   & ‘ailing’ \\
\relax [lɪçt]   & Licht   & ‘light’  \\
\relax [gəʀʏçt] & Gerücht &   ‘rumor’\\
\relax [ʀɛçt]   &  recht  &  ‘right’ \\
\relax [ʀɑiç]   &  Reich  &  ‘empire’\\
\relax [ɔyç]    &  euch   & ‘you-\textsc{acc/dat}.\textsc{pl}’\\
\end{tabularx}

\ex\label{ex:5:16c}
\begin{tabularx}{.8\textwidth}[t]{@{}XXX@{}}
\relax
[mœnç] & Mönch  &  ‘monk’  \\
\relax [dɔlç] & Dolch  &  ‘dagger’\\
\end{tabularx}
\z
\z 

\isi{Umlaut} alternations predictably trigger the occurrence of [x] or [ç], e.g. [buːx] ‘book’ vs. [byːçɐ]  ‘book-\textsc{pl}’.

The complementary distribution of [x] and [ç] is expressed by analyzing [ç] as a positional variant of /x/. The rule capturing the data in \REF{ex:5:16} is \isi{Velar Fronting-1}, which is reproduced in \REF{ex:5:17}.

\ea%17
    \label{ex:5:17}
          \isi{Velar Fronting-1}:\\
  \begin{forest}
  [,phantom
     [\avm{[+son]} [\avm{[coronal]},name=target,tier=word]]   
     [\avm{[−son\\+cont]},name=parent [\avm{[dorsal]},tier=word]]
  ]
     \draw [dashed] (parent.south) -- (target.north);
  \end{forest}
\z 


Since the opaque [x] discussed below derives from /ʀ/ it is essential that the phonological patterning and phonetic realization of that liquid be addressed.

As is \il{Standard German}StG, the one underlying rhotic (/ʀ/) patterns in LRG as a [+sonorant] sound, although it can optionally surface as an obstruent ([ʁ]). The disconnect between the phonological patterning and the phonetic realization is discussed at length in \citet{Hall1993}. The claim defended in that work -- also adopted here -- is that the realization of /ʀ/ as an obstruent is expressed as an optional synchronic process that has become phonologized in LRG.

\begin{sloppypar}
/ʀ/ is phonologically a sonorant because it patterns together with other sonorants in terms of syllabification. German syllables obey the \textsc{Sonority} \textsc{Sequencing} \textsc{Generalization} (e.g. \citealt{Clements1990}, \citealt{Parker2011}) in the sense that syllable-initial clusters exhibit a sonority rise (from left-to-right) and syllable-final clusters a sonority fall (from left-to-right). The \isi{Sonority Hierarchy} for German (\sectref{sec:4.5.2}) makes crucial reference to /ʀ/.
\end{sloppypar}


The distinction between /ʀ/, /l/ and the nasals derives motivation from the fact that word-final /ʀ/+/l/, /l/\,+\,nasal and /ʀ/\,+\,nasal are all parsed as coda clusters in (\ref{ex:5:18a}), while /l/\,+\,/ʀ/, nasal\,+\,/l/, and nasal\,+\,/ʀ/ in the same context are heterosyllabified in (\ref{ex:5:18b}).\largerpage

\ea%18
\TabPositions{.2\textwidth, .4\textwidth, .6\textwidth}
    \label{ex:5:18}
\ea\label{ex:5:18a} 
\relax [kɛʀl]  \tab Kerl \tab ‘fellow’\\
     \relax [fɪlm]  \tab Film \tab ‘film’\\
     \relax [ɑʀm]   \tab Arm  \tab ‘arm’ \\
     \relax [tsɔʀn] \tab Zorn \tab ‘anger’  
\ex \label{ex:5:18b} 
\relax [kɛlɐ]  \tab  Keller \tab ‘cellar’ \\
     \relax [tʊnl̩]\tab  Tunnel \tab  ‘tunnel’\\
     \relax [hɪml̩] \tab  Himmel\tab ‘sky’ 
\z
\z 
\TabPositions{.275\textwidth, .45\textwidth, .66\textwidth}

The \isi{Sonority Hierarchy} supports the analysis of /ʀ/ as a [+sonorant] sound. Were /ʀ/ analyzed as [--sonorant], then the generalization would be lost that the entire natural class of [+sonorant] sounds is more sonorous than the class of [--sonorant] sounds.

A number of studies have shown that one of the realizations of German /ʀ/ is a lenis uvular fricative ([ʁ]), e.g. \citet{Ulbrich1972}, \citet[169]{Kohler1977}. In particular, the amount of constriction in the vocal tract for the consonantal rhotic can increase to the point where friction occurs. According to \citet{Ulbrich1972}, the most common realization of the consonantal (nonvocalized) rhotic is the fricative [ʁ], although a non-fricative sound ([ʀ]) is also common.\footnote{{The non-fricative articulation referred to here is either a trill or an approximant. The distinction between those two realizations is not important in the following discussion because both trills and approximants are [+sonorant] from the point of view of phonology.}} Data displaying the variation between the sonorant [ʀ] and the obstruent [ʁ] are presented for word-initial position in (\ref{ex:5:19a}), between vowels in (\ref{ex:5:19b}), the second member of an onset cluster in (\ref{ex:5:19c}), and coda position after a short vowel in (\ref{ex:5:19d}).

\TabPositions{.275\textwidth, .45\textwidth, .66\textwidth}
\ea%19
\label{ex:5:19}
\ea\label{ex:5:19a} \relax [ʀɑːzən], [ʁɑːzən] \tab Rasen    \tab ‘lawn’
\ex\label{ex:5:19b} \relax [mʏʀɪʃ], [mʏʁɪʃ]   \tab mürrisch \tab ‘sullen’
\ex\label{ex:5:19c} \relax [dʀɑŋ], [dʁɑŋ]     \tab Drang    \tab ‘impulse’
\ex\label{ex:5:19d} \relax [hɛʀ], [hɛʁ]       \tab Herr     \tab ‘gentleman’
\z 
\z 

The data discussed above require the operation in \REF{ex:5:20}, which converts the sonorant /ʀ/ into the corresponding obstruent. \isi{Desonorization-2} differs minimally from \isi{Desonorization-1} (\sectref{sec:3.6}), which is not context-free. Desonorization-2 is optional in order to account for both realizations in \REF{ex:5:19}.


\ea%20
    \label{ex:5:20}
          \isi{Desonorization-2}:\smallskip\\
          \avm{[+cons\\+son\\−nasal\\dorsal] →
               [−son\\+cont]}
\z 

One might assume that the variation in \REF{ex:5:19} is purely phonetic and not phonological. This might be the case in some German dialects, but it will be argued below that Desonorization{}-2 was phonologized in LRG because the derived sound it creates ({\textbar}ʁ{\textbar}) forms the input to the rule creating the opaque [x], which itself is nondistinct from underlying /x/. Since the assimilatory operation posited below creating opaque [x] is a phonological rule, the implication is that Desonorization{}-2 cannot be a rule of \isi{phonetic implementation}; recall the relationship between Phonology and Phonetics in \figref{tab:fromfig:representationallevels}.



/ʀ/ vocalizes to [ɐ] in coda position. Alternations between [ɐ] and [ʀ]/[ʁ] are presented in \REF{ex:5:21}. The change from /ʀ/ to [ɐ] by \isi{r-Vocalization} (\sectref{sec:4.3}) in \REF{ex:5:22} is obligatory after a long vowel in (\ref{ex:5:21a}) and optional after a short vowel in (\ref{ex:5:21c}). I offer no explanation for the condition on optionality.\footnote{{The vocalized-r is transcribed in \citet{Hall1993} as ⟦ʌ⟧, which I render for the sake of consistency with other varieties of German as [ɐ].}}



\ea%21
\label{ex:5:21}
\ea\label{ex:5:21a} \relax [tiːɐ]              \tab Tier   \tab ‘animal’
\ex\label{ex:5:21b} \relax [tiː.ʀə], [tiː.ʁə]  \tab Tiere  \tab ‘animal-\textsc{pl}’
\ex\label{ex:5:21c} \relax [hɛɐ], [hɛʀ], [hɛʁ] \tab Herr   \tab ‘gentleman’
\ex\label{ex:5:21d} \relax [hɛ.ʀən]            \tab Herren \tab ‘gentleman-\textsc{pl}’
\z 
\hspace{1cm}
\ex%22
    \label{ex:5:22}
    \isi{r-Vocalization}:\smallskip\\
    \avm{[+cons\\+son\\−nasal\\dorsal] → [−cons]} / \_\_\_\_ C\textsubscript{0} ]\textsubscript{${\sigma}$}\smallskip\\
Condition: Optional after a short vowel
\z 


The data in \REF{ex:5:16} suggest that the distribution of [x] and [ç] is fully transparent. That this is not the case, is illustrated in the additional data below. Those examples \citep[92--93]{Hall1993} reveal that there are two optional realizations of an underlying /ʀ/ in the context after a short vowel and before a fortis coronal obstruent. In \REF{ex:5:23} the coronal referred to here is word-final, and in \REF{ex:5:24} it is in the onset. The first column in both data sets shows that underlying /ʀ/ -- indicated in the orthography as \textit{r} -- is realized either as the vowel [ɐ] or as the dorsal fricative [x]. The [x] in the examples listed below can occur after any short vowel, regardless of whether or not it is back in (\ref{ex:5:23a}, \ref{ex:5:24a}) or front in (\ref{ex:5:23b}, \ref{ex:5:24b}). Surface [x] in words like the ones in \REF{ex:5:23} and \REF{ex:5:24} must derive synchronically from /ʀ/ because that is the only source for the [ɐ] allophone present in the other optional variant.\footnote{{In the context after a short vowel and before anything other than a coronal obstruent, /ʀ/ in the coda surfaces either as [ɐ] or as the lenis dorsal (uvular) fricative [ʁ], but not as [x], e.g. [mɑɐkt], [mɑʁkt] ‘market’ (*[mɑxkt]). An /ʀ/ in coda position after a long vowel surfaces obligatorily as [ɐ], e.g. [leːɐt] (*[leːxt], *[leːʁt]) ‘teach-}\textrm{\textsc{3sg}}\textrm{’ (from /leːʀ-st/; cf. [leːʀən] ‘teach-}\textrm{\textsc{inf}}\textrm{’).}}


\ea%23
    \label{ex:5:23}
          [x] (from /ʀ/):
\ea\label{ex:5:23a}
\begin{tabularx}{.8\textwidth}[t]{@{}p{4cm}XX@{}}
\relax [kʊɐs], [kʊxs] & Kurs & ‘course’\\
\relax [vɔɐt], [vɔxt] & Wort & ‘word’  \\
\relax [mɑɐs], [mɑxs] & Mars & ‘Mars’  \\
\end{tabularx}
\ex\label{ex:5:23b}
\begin{tabularx}{.8\textwidth}[t]{@{}p{4cm}XX@{}}
\relax [hɪɐʃ], [hɪxʃ]       & Hirsch  & ‘deer’ \\
\relax [vɪɐt], [vɪxt]       & Wirt    & ‘host’ \\
\relax [gəvʏɐts], [gəvʏxts] & Gewürz  & ‘spice’\\
\relax [fɛɐs], [fɛxs]       & Vers    & ‘verse’\\
\end{tabularx}
\z 

\ex%24
\label{ex:5:24}\relax[x] (from /ʀ/):
\ea\label{ex:5:24a}
\begin{tabularx}{.8\textwidth}[t]{@{}p{4cm}XX@{}}
\relax [ʊɐ.tail], [ʊx.tail]   & Urteil   & ‘judgement’            \\
\relax [vʊɐ.tsəl], [vʊx.tsəl] & Wurzel   & ‘root’                 \\
\relax [fɔɐ.ʃən], [fɔx.ʃən]   & forschen & ‘research-\textsc{inf}’\\
\relax [vɑɐ.tən], [vɑx.tən]   & warten   & ‘wait-\textsc{inf}’    \\
\end{tabularx}
\ex\label{ex:5:24b}
\begin{tabularx}{.8\textwidth}[t]{@{}p{4cm}XX@{}}
\relax [fɛɐ.tɪç], [fɛx.tɪç] & fertig  & ‘ready’          \\
\relax [hɪɐ.ʃə], [hɪx.ʃə]   & Hirsche &  ‘deer-\textsc{pl}’\\
\relax [kʏɐ.tsɐ], [kʏx.tsɐ] & kürzer  & ‘shorter’        \\
\end{tabularx}
\z
\z 

The most significant examples presented above are the ones in \REF{ex:5:23b} and \REF{ex:5:24b}, which show that velar [x] can occur after a front vowel. Regardless of how one analyzes the data, it is undeniably the case that LRG is a dialect in which [x] and [ç] contrast on the surface after front vowels (represented by [ɪ ʏ ɛ] below). This contrast is illustrated with several of the examples given earlier, which I repeat in \REF{ex:5:25}: In \REF{ex:5:25a} velar [x] (from /ʀ/) surfaces after [ɪ ɛ ʏ] and before a fortis coronal obstruent, and in \REF{ex:5:25b} palatal [ç] (from /x/) surfaces in the same context. See also \citet[67]{Wiesemann1970}, who discusses a nearly identical set of data in a variety of German she calls “Northern Standard German”. Wiesemann correctly observes that [x] (from /x/) and [ç] (from /ʀ/) contrast in the contrast after a front vowel.

\ea%25
\label{ex:5:25}Surface contrasts between [x] and [ç] after /ɪ ɛ ʏ/:
\ea\label{ex:5:25a}
\begin{tabularx}{.8\textwidth}[t]{@{}XXX@{}}
\relax [vɪxt]    & Wirt   & ‘host’ \\
\relax [gəvʏxts] & Gewürz & ‘spice’\\
\relax [fɛxs]    & Vers   & ‘verse’\\
\end{tabularx}
\ex\label{ex:5:25b} 
\begin{tabularx}{.8\textwidth}[t]{@{}XXX@{}}
\relax [lɪçt]   & Licht   & ‘light’\\
\relax [gəʀʏçt] & Gerücht & ‘rumor’\\
\relax [ʀɛçt]   & recht   & ‘right’\\
\end{tabularx}
\z
\z 

On the basis of the surface contrasts in \REF{ex:5:25} one might be inclined to analyze both the [ç] in words like [lɪçt] ‘light’ and the [x] in words like [vɪxt] ‘host’ as phonemic, i.e. /lɪçt/ vs. /vɪxt/, and to deny that [ç] is an allophone of /x/. I reject that treatment because it fails to recognize that the fully transparent [ç] in [lɪçt] has a different synchronic source than the opaque [x] in [vɪxt]. In particular, the [ç] in the former type of example is the surface realization of underlying /x/ produced by \isi{Velar Fronting-1}, whereas the opaque [x] in words like [vɪxt] is a sound that derives from the rhotic phoneme /ʀ/. Seen in this light, the examples in \REF{ex:5:23} and \REF{ex:5:24} show that \isi{Velar Fronting-1} is active but that it is opaque in examples like the ones in \REF{ex:5:23b} and \REF{ex:5:24b}.

LRG involves an interaction between \isi{Desonorization-2} in \REF{ex:5:20} and the process of laryngeal assimilation accounting for fortis vs. lenis alternations in examples like [leːst] ‘read-\textsc{2pl}’ with fortis [s] before fortis [t] vs. [leːzən] ‘read-\textsc{inf}’ with lenis [z] between vowels. The assimilation rule referred to above is stated linearly in \REF{ex:5:26}, which differs only minimally from the eponymous assimilatory process posited  in \sectref{sec:3.6} for \ipi{Upper Austria} because the trigger for \REF{ex:5:26} is specified as a fortis coronal obstruent. It is argued in the original source to be required in addition to \isi{Final Fortition}, which ensures that an obstruent is  [+fortis] at the right edge of a syllable.

\ea%26
  \label{ex:5:26}
  \isi{Laryngeal Assimilation-2} (Lar Assim-2):\smallskip\\
  \avm{[−son] → [+fortis]} / \_\_\_\_ \avm{[−son\\+fortis\\coronal]}
\z 

In \REF{ex:5:27} I show how \isi{Desonorization-2} (Deson-2) \isi{feeds} \isi{Laryngeal Assimilation-2} (Lar-Assim-2): The former creates {\textbar}ʁ{\textbar} and the latter {\textbar}x{\textbar}. Given the approach presupposed here, derived {\textbar}x{\textbar} in \REF{ex:5:27} has the same features as underlying /x/. As in \ipi{Upper Austria} (\sectref{sec:3.6}), the assumption is that [--nasal] is not present in {\textbar}ʁ{\textbar} or {\textbar}x{\textbar}.

\ea%27
    \label{ex:5:27}
    \begin{tabular}[t]{@{} *{5}{c} @{}}
     /ʀ/ &  → & {\textbar}ʁ{\textbar} & → &                {\textbar}x{\textbar}\\
         &  Deson-2 & & Lar Assim-2 \\
     \begin{forest}
      [ \avm{[+cons\\+son\\−nasal]} [\avm{[dorsal]}, l+=7.5mm ]]
     \end{forest} & & \begin{forest}
      [ \avm{[+cons\\−son\\+cont]} [\avm{[dorsal]}, l+=7.5mm ]]
     \end{forest} & & \begin{forest}
      [ \avm{[+cons\\−son\\+cont\\+fortis]} [\avm{[dorsal]}]]
     \end{forest}\\
     \end{tabular}
\z 

The opaque LRG examples presented above can be modelled in a rule-based approach  consistent with the one proposed in \citet{Hall1993}. That treatment is illustrated in \REF{ex:5:28a}: \isi{Desonorization-2} (Deson-2) \isi{feeds} \isi{Laryngeal Assimilation-2} (Lar Assim-2), which itself \isi{counterfeeds} \isi{Velar Fronting-1} (Vel Fr-1). This can be seen in the word [vɪxt] ‘host’ in \REF{ex:5:28a}, which is intended to be representative of the data in \REF{ex:5:23b} and \REF{ex:5:24b}. Significantly, \isi{Velar Fronting-1} applies at a point where the rhotic in that word has not yet been converted to [x] by \isi{Laryngeal Assimilation-2}. If the latter were to \isi{feed} \isi{Velar Fronting-1}, then the [x] in words like [vɪxt] would incorrectly shift to the palatal [ç] after a front vowel, as in \REF{ex:5:28b}.  Note that the opaque system in \REF{ex:5:28a} is a specific example of the hypothetical Dialect G from \figref{fig:2.9}.

\ea%28
    \label{ex:5:28}
    \begin{multicols}{2}\raggedcolumns
\ea\label{ex:5:28a}
\begin{tabular}[t]{@{}lll@{}}
             & /vɪʀt/&   /lɪxt/\\
Vel Fr-1    &  ---   & lɪçt   \\
Deson-2     &  vɪʁt  & ---    \\
Lar Assim-2 &   vɪxt & ---    \\
            & [vɪxt] & [lɪçt] \\
            & ‘host’ & ‘light’\\
\end{tabular}\columnbreak
\ex\label{ex:5:28b}\begin{tabular}[t]{@{}lll@{}} 
     & /vɪʀt/&  /lɪxt/\\
Deson-2     &  vɪʁt &  --- \\
Lar Assim-2 &  vɪxt &  --- \\
Vel Fr-1    &  vɪçt &  lɪçt\\
            & *[vɪçt] & [lɪçt]\\
\end{tabular}
\z \end{multicols}
\z 

Recall from \sectref{sec:3.6} that the transparent relationship between the assimilation of laryngeal features (\isi{Laryngeal Assimilation-1}) and \isi{Velar Fronting-1} as depicted above in \REF{ex:5:28b} is correct for \ipi{Upper Austria}, which corresponds to the hypothetical Dialect A from \figref{fig:2.6}.

Given the historical model introduced in \sectref{sec:2.5}, the modern-day system for LRG in \REF{ex:5:28a} represents \isi{opacity} at Stage 3, while the transparent realization in \REF{ex:5:28b} exemplifies Stage 2. Stage 1 (not depicted above) is a system with \isi{Desonorization-2} and \isi{Laryngeal Assimilation-2}, but without \isi{Velar Fronting-1}. That type of dialect is therefore one where /ʀ/ surfaces as [x] before a fortis coronal obstruent (e.g. /vɪʀt/→[vɪxt]) but where /x/ is realized as [x] even after a front vowel (e.g. /lɪxt/→[lɪxt]). Although none of the sources cited in the present survey of German dialects explicitly describe such a dialect, the research referred to in \sectref{sec:3.6} and \sectref{sec:5.3.3} suggests that there are such systems among desonorizing \il{South Bavarian}SBav localities (e.g. \citealt{Roitinger1954}: 203--207, \citealt{Kranzmayer1956}: 124--127).\il{Ripuarian|)}

\subsection{South Bavarian}\label{sec:5.3.2}

The dialect described below is a variety of \il{South Bavarian}SBav originally spoken in the Ziller Valley (\ipi{Zillertal}), ca. 40 km to the east of \ipi{Innsbruck} in the Austria state of \ipi{Tyrol} (\mapref{map:3} and \mapref{map:41}). In the year 1837 a number of those speakers -- known as the Zillertaler Protestants (“Zillertaler Inklinantenˮ) -- emigrated to Prussia for religious reasons. Those emigrants settled in and around what was then known as \ipi{Erdmannsdorf} about 20km to the northwest of \ipi{Hirschberg} in the former province of \ipi{Silesia} (\citealt{Siebs1906}; \mapref{map:9}).

\begin{map}
% \includegraphics[width=\textwidth]{figures/VelarFrontingHall2021-img011.png}
\includegraphics[width=\textwidth]{figures/Map9_5.2.pdf}
 \caption[Silesian]{Silesian (\il{Silesian}Sln). 20 is a variety of South Bavarian; 21 and 22 are German-language islands. Squares indicate postsonorant velar fronting. 1=\citet{Michel1891}, 2=\citet{Meiche1898}, 3=\citet{Pautsch1901}, 4=\citet{Hoffmann1906}, 5=\citet{vonUnwert1908}, 6=\citet{Graebisch1912a} (Kreis \ipi{Hirschberg}), 7=\citet{Graebisch1912b} (Alt-Waltersdorf), 8=\citet{Wenzel1919}, 9=\citet{Giernoth1917}, 10=\citet{Kämpf1920}, 11=\citet{Festa1925}, 12=\citet{Rieger1935}, 13=\citet{Weiser1937}, 14=\citet{Halbsguth1938}, 15=\citet{Blaschke1966}, 16=\citet{Messow1965}, 17=SchlSA (\ipi{Hohenelbe}), 18=SchlSA (\ipi{Grulich}), 19=SchlSA (\ipi{Bärn}), 20=\citet{Siebs1906}, 21=\citet{Janiczek1911}, 22=\citet{Benesch1969}.\label{fig:Map9}}\label{map:9}
\end{map}

The source for the \ipi{Erdmannsdorf} dialect is \citet{Siebs1906}, who is known as the primary author of one of the most influential pronouncing dictionary of \il{Standard German}StG (\citealt{Siebs1898, Siebs1969}). In contrast to other sources consulted in this book, \citet{Siebs1906} is quite short (24 pages), and therefore the datasets discussed below exhibit several gaps. Nevertheless, the most significant generalizations (regarding /ʀ/ and velar fronting) are quite clear from the discussion in the original source.

\ipi{Erdmannsdorf} has front vowels (/iː ɪ yː ʏ eː ɛ øː œ/) and back vowels (/ʊ oː ɔ ɑː ɑ ə/) as well as a number of diphthongs, although the data with dorsal fricatives appear primarily after monophthongs. The two dorsal fricatives [x ç] (<\ili{WGmc} \textsuperscript{+}[k] or \textsuperscript{+}[x]) stand in an allophonic relationship in postsonorant position, as in \REF{ex:5:29}.\footnote{{\citet[110]{Siebs1906} is clear that his ⟦x⟧ and ⟦χ⟧ correspond to [x] and [ç] respectively, although he notes that ⟦χ⟧ is articulated in a slightly more retracted position (“weiter hintenˮ) than [ç] in the standard language (“[B]ühnendeutschˮ). The fine-grained difference between ⟦χ⟧ and [ç] is a matter of phonetics and is therefore ignored below. \ipi{Erdmannsdorf} also possesses a dorsal \isi{affricate} (⟦kx⟧), although \citet[125]{Siebs1906} does not discuss whether or not that sound has a palatal allophone in the neighborhood of front vowels, as in \ipi{Rheintal} (\sectref{sec:3.4}). The \isi{etymological palatal} [ʝ] is restricted in its distribution to word-initial position, and there is no [ɣ].}}\largerpage[-2]

\ea%29
    \label{ex:5:29}
          [x] and [ç] (from /x/):
\ea\label{ex:5:29a}
\begin{tabularx}{.8\textwidth}[t]{@{}XXXXl@{}}
moxn̥ &  [mɔxn̩] &   machen &   ‘do-\textsc{inf}’ & 125\\
āx    & [ɑːx]    & auch     & ‘also’ &  125\\
hôax  & [hoːɑx]  & hoch     & ‘high’ &  114\\
\end{tabularx}



\ex\label{ex:5:29b}
\begin{tabularx}{.8\textwidth}[t]{@{}XXXXl@{}}
îχ        & [iːç]     &   ich  & ‘I’ &  125\\
küχn̥     &  [kʏçn̩]  &     Kuchen & ‘cake’ & 115\\
töχtr̥    &  [tœçtʀ̩] &     Töchter &  \mbox{‘daughter-\textsc{pl}’} & 122\\
k\^{ö}χ   & [køːç]    &   Brei &  ‘porridge’ & 125\\
kšleχt    & [kʃlɛçt]  &   schlecht & ‘bad’ & 125\\
\end{tabularx}

\ex
\begin{tabularx}{.8\textwidth}[t]{@{}XXXXl@{}}
milχ & [mɪlç] & Milch & ‘milk’  & 125
\end{tabularx}
\z 
\z

As indicated above, [x] occurs after a back vowel in (\ref{ex:5:29a}) and [ç] after a coronal sonorant in (\ref{ex:5:29b}, \ref{ex:5:29}c).\footnote{{As in \ipi{Maienfeld} (\sectref{sec:3.3}), there are also [x]{\textasciitilde}[ç]{\textasciitilde}[h] alternations (from /h/) in which [x] and [ç] have a transparent distribution. I ignore these data below. \ipi{Erdmannsdorf} has a \isi{palatal quasi-phoneme} after a rhotic (cf. LRG).} } There are various gaps (e.g. no dorsal fricatives after [ɪ yː ʊ]), which I interpret as accidental in light of the brevity of the source.\largerpage[2]

Words with velar [x] surfacing after a front vowel are common. The generalization is that the [x] in those examples has a different synchronic (and diachronic) source than the transparent [x ç] in \REF{ex:5:29}, namely /ʀ/.

At several points in his article, \citet{Siebs1906} discusses the realization of /ʀ/ in coda position (“Auslautˮ). In general, /ʀ/ either deletes or is vocalized in that context (\citealt{Siebs1906}: 119; 123). However, the author adds that the realization of the rhotic is [x] word-internally before a consonant or in final position. (“Neuhochd. r … im Inlaute vor Konsonanten und im Auslaute erscheint … als x (ch) …ˮ). In the context before a consonant, /ʀ/ is pronounced as [x] before [s] or [st]. (“rs erscheint als x, rst als xtˮ). Recall from \sectref{sec:3.6} that the historical change from the rhotic phoneme to a fortis dorsal fricative before sounds like [t] is well-documented in a number of varieties of Bav.

The data presented in \REF{ex:5:30} illustrate the realizations of /ʀ/ in coda position, as described in the preceding paragraph. The sound in question is reflected as \textit{r} in the \il{Standard German}StG orthography in the third column. An item showing the vocalization of /ʀ/ (=⟦a⟧) is presented in \REF{ex:5:30a}. Examples in which /ʀ/ surfaces as [x] after a back vowel and before a fortis coronal obstruent ([t] or [ts]) can be seen in \REF{ex:5:30b}. \ipi{Erdmannsdorf} /ʀ/ also surfaces as [x] in coda position even if a consonant does not follow, as in \REF{ex:5:30c}. The most significant examples are ones in which /ʀ/ surfaces as velar [x] after a front vowel, as in \REF{ex:5:30d}. The post-front vowel [x] in some examples has an alternate with [ʀ], as in \REF{ex:5:30e}.\footnote{{A peripheral point concerns the realization of w as in several items listed in \REF{ex:5:30}. \citet[109]{Siebs1906} observes that the sound in question is articulated with hardly any noticeable frication and appears to be pronounced in onset position as [b]. \REF{ex:5:30d} demonstrates that -}\textrm{\textit{ig}} \textrm{surfaces as [iːç] and not as the expected [iːk], which is the reflex of that suffix in UG. \citet[124]{Siebs1906} notes that his informants pronounced the} \textrm{\textit{g}} \textrm{in -}\textrm{\textit{ig}} \textrm{as [k] or as [ç]. Given the examples in \REF{ex:5:30c}, it is not clear why the rhotic in [fiːʀ] in \REF{ex:5:30e} fails to surface as [x].} }

\TabPositions{.1\textwidth, .266\textwidth, .4\textwidth, .66\textwidth, .75\textwidth}
% \NumTabs{5}
\ea%30
\label{ex:5:30}
Realizations of coda /ʀ/:
\ea\label{ex:5:30a} 
vôa  \tab   [foːɐ] \tab vor \tab ‘before’ \tab 119\\
\ex\label{ex:5:30b}
wuxt   \tab [bʊxt]    \tab wurde   \tab ‘become-\textsc{pret}’ \tab 120\\
wôxt   \tab [boːxt]   \tab Wort    \tab ‘word’                 \tab 114\\
šwôxts \tab [ʃboːxt]  \tab schwarz \tab ‘black’                \tab 120\\
hêaxt  \tab [heːɑxt]  \tab hört   \tab ‘hear-\textsc{3sg}’    \tab 114\\
hêaxts \tab [heːɑxts] \tab Herz   \tab ‘heart’  \tab 123\\
êaxt   \tab [eːɑxt]   \tab erst   \tab ‘only’   \tab 123\\
bîəxtə \tab [biːəxtə] \tab Bürste \tab ‘brush’  \tab 123\\

\ex\label{ex:5:30c} 
fôəx     \tab [foːəx] \tab vor’s  \tab ‘before it’        \tab 123\\
iəx     \tab [iəx]   \tab ihr    \tab  ‘you-\textsc{pl}’ \tab 120\\
w\^{ü}əx \tab [byːəx] \tab Wurst  \tab ‘sausage’          \tab 115\\

\ex\label{ex:5:30d}
fextîχ  \tab  [fɛxtiːç] \tab fertig \tab ‘ready’             \tab 117\\
fêxt    \tab  [feːxt]   \tab fährt  \tab ‘go-\textsc{3sg}’ \tab 112\\

\ex\label{ex:5:30e}
fîr    \tab   [fiːʀ]   \tab vier   \tab ‘four’   \tab 113 \\
fîxtə  \tab   [fiːxtə] \tab vierte \tab ‘fourth’ \tab  113\\
\z
\z 

The critical reader may call into question that the [x] in \REF{ex:5:30} derives from /ʀ/. Recall from \sectref{sec:5.3.1} that the /ʀ/ in similar data from LRG is justified on the basis of the optional pronunciation with [ɐ], whose only synchronic source is /ʀ/. Siebs does not say explicitly that the same kind of free variation is possible for his speakers, but he does provide examples like the ones in \REF{ex:5:30e} that justify /ʀ/.

The most important examples are the ones in \REF{ex:5:30}, which reveal the occurrence of velar [x] after a front vowel. \ipi{Erdmannsdorf} [x] and [ç] contrast on the surface after front vowels. That contrast is illustrated with several of the examples presented above, which I repeat in \REF{ex:5:31}: In \REF{ex:5:31a} velar [x] (from /ʀ/) surfaces in the coda after [iː ɛ], and in \REF{ex:5:31b} palatal [ç] (from /x/) surfaces in the same context.

\ea%31
\label{ex:5:31}Contrasts between [x] and [ç] after /iː/ and /ɛ/:

\ea\label{ex:5:31a} fîxtə  \tab [fiːxtə]  \tab vierte   \tab ‘fourth’\\
    fextîχ \tab [fɛxtiːç] \tab fertig   \tab ‘ready’
\ex\label{ex:5:31b} îχ     \tab [iːç]     \tab ich      \tab ‘I’\\
    kšleχt \tab [kʃlɛçt]  \tab schlecht \tab ‘bad’
\z
\z 

Opaque examples in which [x] surfaces after a front vowel can be accounted for if \isi{Final Fortition} \isi{counterfeeds} \isi{Velar Fronting-1}, cf. \ipi{Rhoden} (\sectref{sec:5.2}). \isi{Counterfeeding} \isi{opacity} is evident in the word [fiːxtə] ‘fourth’ in \REF{ex:5:32a}, which is intended to be representative of all of opaque examples. Significantly, \isi{Velar Fronting-1} (Vel Fr-1) applies at a point where the rhotic in that word has not yet been converted to [x] by \isi{Final Fortition} (Fnl For). If the latter were to \isi{feed} \isi{Velar Fronting-1}, then the [x] in words like [fiːxtə] would incorrectly shift to the palatal [ç] after a front vowel, as illustrated in \REF{ex:5:32b}. Note that the correct output in \REF{ex:5:32a} is obtained if /ʀ/ undergoes \isi{Desonorization-1} (Deson-1), thereby \isi{feeding} \isi{Final Fortition}.

\ea
 \label{ex:5:32} 
 \begin{multicols}{2}\raggedcolumns
 \ea\label{ex:5:32a} \begin{tabular}[t]{@{}lll@{}}
          & /fiːʀ-tə/        & /iːx/          \\
Vel Fr-1  &  ---             & iːç            \\
Deson-1   & fiːʁtə           & ---\\ 
Fnl For   & fiːxtə           & ---\\ 
          & [fiːxtə]         & [iːç]          \\
          & ‘fourth’         & ‘I’            \\
    \end{tabular}
\ex\label{ex:5:32b} \begin{tabular}[t]{@{}lll@{}}
           & /fiːʀ-tə/     &      /iːx/     \\
  Deson-1  & fiːʁtə        & ---\\
  Fnl For  & fiːxtə        & ---\\
  Vel Fr-1 &  fiːçtə       &   iːç          \\
           &    *[fiːçtə]  &        [iːç]   \\
    \end{tabular}
   \z
   \end{multicols}
\z 

(\ref{ex:5:32a}) exhibits the \isi{underapplication} of \isi{Velar Fronting-1}: The fortis velar fricative in the phonetic representation potentially forms the input to \isi{Velar Fronting-1} because it stands after a front vowel. The opaque system in \REF{ex:5:32a} exemplifies the hypothetical Dialect G from \figref{fig:2.9}.

The historical progression from \isi{transparency} to \isi{opacity} is essentially the same as the one proposed earlier for LRG. Thus, the opaque \ipi{Erdmannsdorf} system in \REF{ex:5:32a} represents Stage 3, while the transparent system in \REF{ex:5:32b} illustrates \ipi{Upper Austria} (\sectref{sec:3.6}). Not depicted above is Stage 1, where \isi{Velar Fronting-1} was absent.

The opaque system in \REF{ex:5:32a} for \ipi{Erdmannsdorf} could not have arisen under the influence of \il{Silesian}Sln dialects spoken in the general vicinity. An examination of the sources for \il{Silesian}Sln reveals that there is no evidence for contact-induced change. This point is clear from the maps in the linguistic atlas for \ipi{Silesia} (SchlSA) for words with /r/ in the coda (e.g. Map 6 for \textit{Kirche} ‘church’ and Map 38 for \textit{Tür} ‘door’). The rhotic in those words is most commonly realized on those maps as the coronal consonant [r], or it the vocalized-r; however, no variant with [x] (or [ç]) is attested. The same conclusion is drawn by all of the sources consulted for \il{Silesian}Sln. The following three examples are significant because they all reveal that [x] and [ç] have a transparent distribution.\largerpage

\citet{Halbsguth1938} describes the \il{Silesian}Sln dialect once spoken in \ipi{Bremberg} (\mapref{map:9}). He writes that the rhotic surfaces as an untrilled tongue-tip-r (“ein ungerolltes Zungenspitzen-r") in word- and syllable-initial position \citep[29--30]{Halbsguth1938}. In the coda, the sound is vocalized after a long vowel, but it is retained as [r] after a short vowel in coda position before velars and labials. In coda position after a short vowel, [r] tends to delete if the following consonant is a coronal. No mention is made of the realization of data like the ones in (\ref{ex:5:30b}--\ref{ex:5:30e}). In \ipi{Bremberg}, the palatals [ç ʝ] and velars [x ɣ] are the realization of the corresponding velars /x ɣ/ after a front vowel and back vowel respectively, but there does not appear to be an opaque [x] or an opaque [ɣ].

\citet{Pautsch1901} provides a historical grammar of the \il{Silesian}Sln variety once spoken in \ipi{Kieslingswalde} (\mapref{map:9}). On the basis of his description of the phonetics of consonants (p. 12), the one rhotic is a coronal tongue-tip sound (“Zungenspitzen-rˮ) which vocalizes in coda position. The dialect has the four dorsal fricatives [x ɣ ç ʝ], but those sounds have a transparent distribution (i.e. palatals after coronal sonorants and velars after back vowels). There do not appear to be cases involving an opaque [x] or [ɣ].

\Textcite{vonUnwert1908} is a descriptive grammar of the \il{Silesian}Sln dialect as it was spoken throughout the Prussian province of \ipi{Silesia} (\mapref{map:9}) and the neighboring areas of the Austro-Hungarian Empire (modern-day Czech Republic). According to that source (pp. 33--34), [r] is a coronal sound articulated on the tongue-tip (“Zungenspitzeˮ) in onset position. In coda position that sound tends to either delete or vocalize before coronal consonants, but no mention is made of a [x] realization, as in (\ref{ex:5:30b}--\ref{ex:5:30e}). The dialect described by von Unwert has the velar fricative [x] and the two palatal fricatives [ç] and [ʝ], but those sounds all have a transparent distribution (\citealt{vonUnwert1908}: 52--54): [ç] and [ʝ] surface after a front vowel or coronal consonant ([l] or [r]) and [x] after a back vowel. There is no evidence for an opaque [x].

\subsection{Areal distribution of opacity resulting from desonorization}\label{sec:5.3.3}
\largerpage
The two case studies discussed above have in common that /ʀ/ surfaces as the velar fricative [x] even in the context after front vowels. It is interesting to observe that the same set of facts obtain in two places (in and around \ipi{Düsseldorf} and \ipi{Erdmannsdorf}) separated by several hundred kilometers.

The realization of the consonantal rhotic ([r]/[ʀ]) as the velar fricative [x] has been discussed at length in the literature on German dialectology and phonetics. A recent assessment of the state of that research can be found in NOSA: 309--321. According to that source the change [r]/[ʀ] > [x] is most prevalent in the \il{Ripuarian}Rpn and \il{Moselle Franconian}MFr dialect areas, but it is also attested throughout various places in North Germany. NOSA also concludes that the change typically occurs in the context after a short vowel and before fortis (voiceless) coronal obstruents, or some subset thereof (e.g. [t]). The data from North Germany discussed in NOSA reveal that the change is most common in the context after short back vowels, although the percentages listed (p. 319) make it clear that [x] can also occur in the context after a short front vowel. Since the areas in North Germany discussed in NOSA have velar fronting, sequences like [ɪx] (from /ɪʀ/) are therefore opaque.

In \tabref{tab:5:5.2} I cite some of the sources known to me (other than LRG and Erdmannsdorf) which have documented the change [r]/[ʀ] > [x] in German dialects. These sources have in common that they either state explicitly that the change occurs in the context after a short vowel and before a coronal obstruent (or some subset thereof), or that context is implied by the examples they give. Since velar fronting is prevalent in all of the areas listed below any realization of [r]/[ʀ] as [x] after a front vowel implies an opaque system like the ones discussed in \sectref{sec:5.3.1} and \sectref{sec:5.3.2}. Some of the sources give such opaque examples, while others only cite data in the context after a short back vowel.\footnote{{One of the works listed in \tabref{tab:5:5.2} \citep[171]{Niekerken1963} observes that the change from rhotic to [x] occurs after the vowels [ɑ] and [o] and before [t], e.g. ⟦ɡɑxtnͅ⟧ ‘garden’. By contrast, in the context after front vowels or [u], an epenthetic (back) vowel is inserted before the [x], e.g. ⟦v\k{i}}\textrm{\textsuperscript{ǫ}}\textrm{xt⟧ ‘host’ (cf. \il{Standard German}StG [vɪʀt], LRG [vɪxt]). Epenthesis appears to be a strategy speakers adopt to avoid an opaque output. To the best of my knowledge no other German dialect is attested which involves a repair to avoid \isi{opacity}.}}

\begin{table}
  \caption{\label{tab:5:5.2}Desonorization ([r]/[ʀ] >[x])in German dialects}
\begin{tabular}{ll}
\lsptoprule
Source & Area\\\midrule
\citet[102]{Runschke1938} & \ipi{Berlin}\\
\citet[248]{Meyer-Eppler1959} & \il{Ripuarian}Rpn/West Germany\\
\citet[171--173]{Niekerken1963} & Area south of \ipi{Hannover}\\
\citet[67]{Wiesemann1970} & North Germany\\
\citet[170]{Kohler1977} & Rhineland\\
\citet[157--158]{Wängler1983} & North Germany\\
\citet[145--149]{Macha1991} & Siegburg (\il{Ripuarian}Rpn)\\
\citet[73--77]{Kreymann1994} & \ipi{Erp} (Erftstadt) (\il{Ripuarian}Rpn)\\
\citet[213]{Lauf1996} & \il{Eastphalian}Eph dialect area\\
\citet[298--300]{Cornelissen2002} & Rheinland (\il{Ripuarian}Rpn)\\
\citet[108ff.]{Elmentaler2012} & \ipi{Hannover}\\
\citet[98; 172f.]{Möller2013} & Bonn (\il{Ripuarian}Rpn)\\
\lspbottomrule
\end{tabular}
\end{table}

The occurrence of desonorization throughout the Rhineland (\il{Ripuarian}Rpn/\il{Moselle Franconian}MFr) is also documented spatially in several maps. One example already cited in \tabref{tab:5:5.2} can be found in \citet{Cornelissen2002}. Two linguistic atlases with similar maps are AAS (for \textit{Garten} ‘garden’ in Volume 2 : 197) and ADA (for \textit{Karte} ‘map’ and \textit{Sport} ‘sports’).

Recall from \sectref{sec:5.3.2} that the context for desonorization in \ipi{Erdmannsdorf} is not the same as the one attested in the places listed above. In particular, desonorization also occurs in the context (A) after a long vowel and before a fortis coronal obstruent or (B) in word-final position. The following three studies document either (A) or (B) in German dialects. Since velar fronting is active in all of these places the occurrence of derived [x] after a front vowel implies \isi{opacity}.

In his study of the CG varieties spoken in Manderfeld and Wallerode (to the north(east) of St. Vith on \mapref{map:8}), \citet[67--68]{Hecker1972} writes:

\begin{quote}
Im Auslaut kann /r/ als [x] realisiert werden, zum Beispiel /taːrt/ [taːxt] ‘Butterbrot’ ... Ein auf /r/ zurückgehendes [x] kann auch nach palatalem Silbenkern vorkommen.\smallskip

“In the coda /r/ can be realized as [x], for example /taːrt/ [taːxt] ‘bread and butter’ ... A [x] deriving from /r/ can also occur after a front vowelˮ.
\end{quote}

The example shows that desonorization occurs in context (A). No examples are provided for the opaque sequences alluded to in this quote.

A more explicit statement concerning \isi{opacity} can be found in \citet[97]{Freund1910}, who makes the following observation concerning the realization of /ʀ/ as a dorsal fricative in the variety of \il{Central Hessian}CHes spoken in \ipi{Marburg} (\mapref{map:11}). The examples discussed under cases (2) and (3) in this quote differ from the data in other case studies in this section because they show that the contrast between /ʀ/ and /ɣ/ (=⟦γ⟧) is neutralized to [ɣ] in onset position and to [x] in the coda.

\begin{quote}
“There is no difference in the pronunciation of
(1) \textit{Wacht} and \textit{ward} [ʋaxt], \textit{mocht} and \textit{Mord} [moxt], \textit{Wucht} and \textit{wurd’} [ʋuxt] (fortis x);
(2) \textit{behagt} and \textit{behaart} [bəhaːxt (lenis x); and
(3) \textit{Wagen} and \textit{waren} [vaːγn], \textit{behagen} and \textit{beharren} [bəhaːγn], \textit{saugen} and \textit{sauren} [saoγn] ...ˮ{.}
\end{quote}

The important point regarding Freund’s treatment of \ipi{Marburg} /ʀ/ is that his data show the same kind of \isi{underapplication} \isi{opacity} as in LRG and \ipi{Erdmannsdorf}. Examples from \citet[97]{Freund1910} with a derived [x] after a front vowel include [hexʃn] ‘rule-\textsc{inf}’ (cf. \il{Standard German}StG [hɛʀʃən]), [fexs] ‘verse’ (cf. \il{Standard German}StG [fɛʀs]), and [kixçə] ‘church’ (cf. \il{Standard German}StG [kɪʀçə]).\largerpage

The final example is \citet{Müller1958b}, which contains a brief description of the consonants and vowels in \ipi{Kassel} (\il{North Hessian}NHes), including a phonetically transcribed text from a native speaker (\mapref{map:11}). It is clear from the transcriptions that the data are essentially the same as the ones described above for \ipi{Marburg}. Thus, /ʀ/ is realized as [x] after both back and front vowels, e.g. \textit{erst} ‘only’ [ɛxst] (cf. \il{Standard German}StG [ɛʀst]) and \textit{mehr} ‘more’ [meːx] (cf. \il{Standard German}StG [meːɐ] from /meːʀ/). The second example illustrates context (B). Significantly, the two examples cited here demonstrate \isi{opacity}.

\section{{An} {apparent} {case} {of} {overapplication} {opacity}}\label{sec:5.4}

The dialects discussed in this chapter have in common that they involve \isi{underapplication}. In particular, they all possess a velar fricative derived from some other sound (by Rule W from \tabref{tab:2.wxyz}), but that derived velar fricative ({\textbar}x{\textbar}) fails to serve as a target segment for velar fronting. Rule W therefore \isi{counterfeeds} velar fronting synchronically.

Although the \isi{underapplication} of velar fronting in the synchronic phonology is well-attested, the \isi{overapplication} of that same process is not. There is one potential example known to me of a synchronic process \isi{counterbleeding} velar fronting thereby resulting in \isi{overapplication} \isi{opacity}. Although that type of example is very well-attested in German dialects, I demonstrate that there is a plausible alternative which does not require velar fronting to be counterbled by another process in the synchronic grammar.

\citet{Hommer1910} describes a \il{Moselle Franconian}MFr dialect spoken in the northern part of Westerwald in the German state of Rhineland-Palatinate (Rheinland-Pfalz). Hommer’s grammar focuses on the community of \ipi{Sörth} (\mapref{map:10}).

\begin{map}

% \includegraphics[width=\textwidth]{figures/VelarFrontingHall2021-img012.png}
\includegraphics[width=\textwidth]{figures/Map10_5.3.pdf}
  \caption[Moselle Franconian and Rhenish Franconian]{Moselle Franconian (\il{Moselle Franconian}MFr) and Rhenish Franconian (\il{Rhenish Franconian}RFr). White squares indicate assimilatory postsonorant velar fronting, and the shaded squares represent nonassimilatory postsonorant velar fronting. 1=\citet{Büsch1888}, 2=\citet{Baldes1896}, 3=\citet{Fuchs1903}, 4=\citet{Tarral1903}, 5=\citet{Reuter1903}, 6=\citet{Ludwig1906}, 7=\citet{Thomé1908}, 8=\citet{Hommer1910}, 9=\citet{Engelmann1910}, 10=\citet{Wimmert1910}, 11=\citet{Thies1912}, 12=\citet{Scholl1912},  13=\citet{Meyers1913, Meyers1913b}, 14=\citet{Bach1921}, 15=\citet{Bertrang1921}, 16=\citet{Lehnert1926}, 17=\citet{Palgen1931}, 18=\citet{Pallier1934}, 19=\citet{Bruch1952}, 20= \citet{BethgeBonnin1969}, 21=\citet{Hecker1972}, 22=\citet{CajotBeckers1979},  23=\citet{Mattheier1987}, 24=\citet{Reuter1989},  25=\citet{Peetz1989}, 26=\citet{Gilles1999}, 27=\citet{Féry2017}, 28=MRhSA (\ipi{Dahnen}), 29=MRhSA (\ipi{Lützkampen}), 30=ALLG (Elzange), 31=\citet{Reis1892}, 32=\citet{Heeger1896}, 33= \citet{Lenz1900}, 34= \citet{Wanner1907, Wanner1908}, 35=\citet{Haster1908}, 36=\citet{Wenz1911}, 37=\citet{Reichert1914}, 38=\citet{Martin1922}, 39=\citet{Christmann1927}, 40= \citet{Krell1927}, 41=\citet{Lauinger1929}, 42=\citet{Freiling1929}, 43=\citet{Seibt1930}, 44=\citet{Treiber1931}, 45=\citet{Kuntze1932}, 46=\citet{Waibel1932}, 47=\citet{Grund1935}, 48=\citet{Bertram1937}, 49=\citet{Kilian1951}, 50=\citet{Bauer1957}, 51=\citet{Keller1961} (\ipi{Darmstadt}), 52=\citet{Liébray1969}, 53=\citet{Castleman1975}, 54=\citet{Karch1980}, 55=\citet{Karch1981}, 56=\citet{Steitz1981}, 57=\citet{Post1987}, 58=\citet{Pützer1988}, 59=\citet{DurrellDavies1989}, 60=ALLG (Langatte), 61=ALLG (Laning) 62=ALLG (Schorbach), 63=SNBW (Remschingen), 64=SNBW (Bretten), 65=SUF (Schneppenbach), 66=SUF (Wintersbach).}\label{map:10}
\end{map}

It is clear from the material in \citet{Hommer1910} that \ipi{Sörth} possesses the two underlying velar fricatives /x ɣ/, which have palatal allophones [ç ʝ] after any front vowel in (\ref{ex:5:33c}, \ref{ex:5:33d}) or (coronal) liquid in (\ref{ex:5:33e}). After any full back vowel /x ɣ/ surface as velars in (\ref{ex:5:33a}, \ref{ex:5:33b}).

\ea%33
  \TabPositions{.1\textwidth, .266\textwidth, .4\textwidth, .6\textwidth, .7\textwidth}
\label{ex:5:33}Postsonorant [x ɣ] and [ç ʝ] (from /x ɣ/):
\ea\label{ex:5:33a}   šdrux     \tab [ʃtrux]    \tab Strauch  \tab ‘shrub’             \tab 22\\
      kōxən     \tab [koːxǝn]   \tab Kuchen   \tab ‘cake’              \tab 22\\
      kǫxən     \tab [kɔxǝn]    \tab kochen   \tab ‘cook-\textsc{inf}’ \tab 22\\
      mɑ̄xən     \tab [mɑːxǝn]   \tab machen   \tab ‘do-\textsc{inf}’   \tab 22\\
\ex\label{ex:5:33b}   fuɣəl     \tab [fuɣǝl]    \tab Vogel    \tab ‘bird’              \tab 25\\
      ōɣən      \tab [oːɣǝn]    \tab Augen    \tab ‘eye-\textsc{pl}’              \tab 25\\
      frǭɣən    \tab [frɔːɣǝn]  \tab fragen   \tab ‘ask-\textsc{inf}’  \tab 25\\
      ɡrɑ̄ɣən    \tab [grɑːɣǝn]  \tab Kragen   \tab ‘collar’            \tab 25\\
\ex\label{ex:5:33c}   sīχ       \tab [siːç]     \tab siech    \tab ‘ailing’            \tab 22\\
      liχ       \tab [lɪç]      \tab Leiche   \tab ‘body’              \tab 22\\
      kyχ       \tab [kyç]      \tab Küche    \tab ‘kitchen’           \tab 13\\
      šlēχt     \tab [ʃleːç]    \tab schlecht \tab ‘bad’               \tab 10\\
      eχ        \tab [eç]       \tab ich      \tab ‘I’                 \tab 22\\
      blęχ      \tab [blɛç]     \tab Blech    \tab ‘tin’               \tab 22\\
      s\={ø}χən \tab [søːçǝn]   \tab suchen   \tab ‘search-\textsc{inf}’ \tab 10\\
\ex\label{ex:5:33d}   ijəl      \tab [ɪʝǝl]     \tab Igel     \tab ‘hedgehog’            \tab 24\\
      flyjəl    \tab [flyʝǝl]   \tab Flügel   \tab ‘wing’                \tab 24\\
      fējən     \tab [feːʝǝn]   \tab fegen    \tab ‘sweep-\textsc{inf}’  \tab 24\\
\ex\label{ex:5:33e}   foljən    \tab [fɔlʝǝn]   \tab folgen   \tab ‘follow-\textsc{inf}’ \tab 24\\
      bɑrjən    \tab [bɑrʝǝn]   \tab borgen   \tab ‘borrow-\textsc{inf}’ \tab 24\\
  \z
\z 

The distribution of postsonorant dorsal fricatives in \REF{ex:5:33} is captured with \isi{Velar Fronting-1}.

The crucial set of examples involves the occurrence of palatal [ç] after \isi{schwa} ([ǝ]) in word-final (coda) position, as in \REF{ex:5:34}. The corresponding velar ([x]) is not attested after that vowel. The consonant preceding [ǝ] is a coronal liquid, namely [r] in (\ref{ex:5:34a}) or [l] in (\ref{ex:5:34b}), but there are no words ending in [ǝç] preceded by anything other than [l] or [r]. The sound underlying [ç] is either /x/ (e.g. in the first word in \ref{ex:5:34a}) or /ɣ/ in the remaining words, e.g. [bɑlǝç] in \REF{ex:5:34b}, in which [ç] alternates with [ʝ] in [bɑlʝǝn] ‘scrap-\textsc{inf}’ \citep[5]{Hommer1910}. Alternations involving [ʝ] and [ç] are captured with \isi{Final Fortition} in (\ref{ex:5:8}).

\ea\label{ex:5:34}
	\TabPositions{.1\textwidth, .266\textwidth, .4\textwidth, .6\textwidth, .7\textwidth}
\ea\label{ex:5:34a}  kęrəχ  \tab [kɛrǝç] \tab Kirche \tab ‘church’ \tab 22\\
     sɑrəχ  \tab [sɑrǝç] \tab Sorge  \tab ‘sorrow’ \tab 11\\
     węrəχ  \tab [vɛrǝç] \tab Werg   \tab ‘oakum’  \tab 22\\
\ex\label{ex:5:34b}  bɑləχ  \tab [bɑlǝç] \tab Balg   \tab ‘brat’   \tab 24\\
\z 
\z 

The sequence [ǝç] appears to involve \isi{overapplication} because palatal [ç] derives from velar /x/ after a vowel (\isi{schwa}) that is not front.

The \isi{schwa} in \REF{ex:5:34} is a synchronically epenthetic vowel. Hommer himself sees \isi{schwa} in examples like the ones in \REF{ex:5:34} as the product of epenthesis (Svarabhakkti). The data in the original source indicate that \isi{schwa} is epenthesized after a coronal liquid and before a coda labial or velar. Two examples with an epenthetic \isi{schwa} after a liquid and before a coda labial are presented in \REF{ex:5:35a}. Many words given in the original source are transcribed with and without \isi{schwa}, as in \REF{ex:5:35b}, indicating that the epenthesis process is optional. The examples in \REF{ex:5:35c} and \REF{ex:5:35d} show that the \isi{schwa} after a liquid and before the dorsal fricative [ç] behaves precisely like the epenthetic \isi{schwa} in (\ref{ex:5:34a}, \ref{ex:5:34b}).{\interfootnotelinepenalty=10000\footnote{Although Hommer does not say so explicitly, the pattern of epenthesis described above only holds between a liquid and a noncoronal; hence, a word like \textit{alt} ‘old’ \citep[23]{Hommer1910} is pronounced [ɑlt] and not [ɑlǝt]. Hommer provides some examples in which a \isi{schwa} appears between a nasal and a noncoronal (e.g. [hɑnǝf] ‘hemp’) and between a liquid and a coronal nasal (e.g. [gɑrǝn] ‘yarn’). These complications do not bear on the present analysis.}}\largerpage

\ea\label{ex:5:35}
\TabPositions{.2\textwidth, .45\textwidth, .55\textwidth, .75\textwidth}
\ea\label{ex:5:35a}  hɑləf        \tab [hɑlǝf]         \tab halb                  \tab ‘half’     \tab 26    \\
     kɑrəf        \tab [kɑrǝf]         \tab Korb                  \tab ‘basket’   \tab 26    \\
\ex\label{ex:5:35b}  hɑlm, hɑlǝm  \tab [hɑlm], [hɑlǝm] \tab Halm                  \tab ‘blade’    \tab  5    \\
     kɑlk, kɑlǝk  \tab [kɑlk], [kɑlǝk] \tab Kalk                  \tab ‘lime’     \tab  5    \\
     šɑrf, šɑrəf  \tab [ʃɑrf], [ʃɑrǝf] \tab scharf                \tab ‘sharp’    \tab  5    \\
\ex\label{ex:5:35c}  bęrχ, bęrəχ  \tab [bɛrç], [bɛrǝç] \tab Berg                  \tab ‘mountain’ \tab  8, 24\\
     bęrjən       \tab [bɛrʝǝn]        \tab in der Grube arbeiten \\
                  \tab                 \tab ‘work-\textsc{inf} in the pit’ \tab 8    \\
\ex\label{ex:5:35d}  bɑlχ, bɑləχ  \tab [bɑlç], [bɑlǝç] \tab Balg                  \tab ‘brat’                         \tab 5, 24\\
     bɑljən       \tab [bɑlʝǝn]        \tab balgen                \tab ‘scrap-\textsc{inf}’           \tab 5    \\
\z 
\z 

Examples like the ones in (\ref{ex:5:35c}, \ref{ex:5:35d}) indicate that there is no contrast between word-final sequences like [lǝç] and word-final [lç]. This suggests that the words in question have no \isi{schwa} in the underlying representation and that it is inserted by an optional rule. The way in which \isi{Schwa Epenthesis} is analyzed is not important for present purposes; I simply state the process in its prose form in \REF{ex:5:36} for \isi{transparency}:\largerpage[1.5]

\ea%36
\label{ex:5:36}
  \isi{Schwa Epenthesis}: Insert [ǝ] between a liquid and a labial or dorsal coda consonant.
\z 

The epenthesis of \isi{schwa} between a liquid and a labial or dorsal consonant is not restricted to \ipi{Sörth}, nor is it a defining property of \il{Moselle Franconian}MFr in general. As noted by \citet[401]{Schirmunski1962}, \isi{Schwa Epenthesis} between a liquid and labial/velar can be observed to a certain degree in all HG dialects. (“Die Erscheinung hat alle Hochdeutschen Dialekte erfasst ...ˮ) I do not attempt to provide a survey of specific varieties of German dialects with \REF{ex:5:36}; however, I provide a selection of ten WCG and UG varieties in \tabref{tab:5.1} in which the sources state explicitly that epenthesis (Svarabhakkti) is present in examples like the ones in \REF{ex:5:34} and \REF{ex:5:35}. For a discussion of the presence of \REF{ex:5:36} in German dialects the reader is referred to \citet{Auer1997}.\footnote{{Auer argues that the sound transcribed as \isi{schwa} ([ǝ]) in \REF{ex:5:35} is not the product of epenthesis. One of his reasons for questioning a traditional phonological rule of insertion is that \isi{schwa} in data like the ones in \REF{ex:5:35} can be seen as a consequence of the mistiming of articulatory gestures; see \citegen{BrowmanGoldstein1992} framework of \isi{Articulatory Phonology}. The purpose of this section is not to defend a traditional rule of epenthesis, but instead to discuss the extent to which the data in \REF{ex:5:35} illustrate the \isi{opacity} of velar fronting. Seen in this light, the treatment for \ipi{Sörth} I suggest below can be modelled in a number of frameworks, including the ones Auer endorses.}}

\begin{table}
\caption{\label{tab:5.1}Selection of UG and WCG varieties attested with Schwa Epenthesis (=\ref{ex:5:36})}
\begin{tabular}{lll}
\lsptoprule
Place/Region & Dialect & Source\\\midrule
\ipi{Wachbach} & \il{East Franconian}EFr & \citet{Dietzel1908}\\
\ipi{Suhl} & \il{East Franconian}EFr & \citet{Kober1962}\\
\ipi{Oberschopfheim} & \il{Low Alemannic}LAlmc & \citet{Schwend1900}\\
\ipi{Rheinbischofsheim} & \il{Low Alemannic}LAlmc & \citet{Weik1913}\\
\ipi{Sehlem} & \il{Moselle Franconian}MFr & \citet{Ludwig1906}\\
\ipi{Arel} & \il{Moselle Franconian}MFr & \citet{Bertrang1921}\\
Zaisenhausen & \il{Rhenish Franconian}RFr & \citet{Wanner1907, Wanner1908}\\
\ipi{Saarbrücken} & \il{Rhenish Franconian}RFr & \citet{Kuntze1932}\\
\ipi{Erftgebiet} & \il{Ripuarian}Rpn & \citet{Münch1904}\\
\ipi{Schelsen} & \il{Ripuarian}Rpn & \citet{Greferath1922}\\
\lspbottomrule
\end{tabular}
\end{table}

The pattern of epenthesis in \ipi{Sörth} and in the dialects in \tabref{tab:5.1} is also essentially the same as in \ili{Dutch} (\citealt{Trommelen1984}: 77--79, \citealt{Booij1995}: 127--128, \citealt{Grijzenhout1998}: 39--42, \citealt{vanOostendorp2000}). \ili{Dutch} words showing the (optional) epenthesis of \isi{schwa} between a liquid and a final noncoronal (from \citealt{Booij1995}) include \textit{arm} ‘arm’ and \textit{elf} ‘eleven’, which can surface as [ɑrǝm] and [ɛlǝf] respectively. By contrast, there is no \isi{schwa} between a liquid and a coronal obstruent, e.g. \textit{halt} [hɑlt] ‘stop-\textsc{imp} \textsc{sg}’.

One approach to the data in \REF{ex:5:34} -- which I reject -- requires \isi{Schwa Epenthesis} to \isi{counterbleed} \isi{Velar Fronting-1} (Vel Fr-1), as in \REF{ex:5:37a}. This is a \isi{counterbleeding} relationship because the reverse ordering in (\ref{ex:5:37b}) requires \isi{Schwa Epenthesis} to \isi{bleed} \isi{Velar Fronting-1}.

\ea%37
    \label{ex:5:37}
      Counterbleeding \is{counterbleeding order}order in Moselle Franconian (rejected):
  \begin{multicols}{2}
    \ea\label{ex:5:37a}
    \begin{tabular}[t]{@{}ll@{}}
                 & /kɛrx/  \\ 
Vel Fr-1         &  kɛrç   \\ 
\isi{Schwa Epenthesis} &  kɛrǝç  \\ 
                 &  [kɛrǝç]\\ 
                 & ‘church’\\
     \end{tabular}
    \ex\label{ex:5:37b}\begin{tabular}[t]{@{}ll@{}}
                  & /kɛrx/  \\
 \isi{Schwa Epenthesis} & kɛrǝx   \\
 Vel Fr-1         & ---     \\
                  & *[kɛrǝx]\\
        \end{tabular}
    \z
  \end{multicols}
\z 

(\ref{ex:5:37a}) implies that a phonetic representation like [kɛrǝç] is opaque on the surface because it involves the \isi{overapplication} of \isi{Velar Fronting-1}. The aforementioned process overapplies in \REF{ex:5:37a} because it can only create a palatal after a front ([coronal]) vowel; since \isi{schwa} is not a [coronal] vowel, a surface form like [kɛrǝç] shows that \isi{Velar Fronting-1} also appears to apply in a context not specified in the structural description of the rule.

There is a plausible alternative analysis for \ipi{Sörth} that eschews \isi{opacity}. I argue that the epenthetic \isi{schwa} is a surface front vowel because it occurs after a front consonant. That derived front vowel is created by \isi{Schwa Fronting-2} in \REF{ex:5:38}, which spreads the feature [coronal] rightward from a liquid to \isi{schwa}. Recall from \sectref{sec:3.4} that an eponymous process was posited for \ipi{Rheintal} to account for the realization of dorsal fricatives and affricates as palatal in the context after diphthongs ending in \isi{schwa} if that \isi{schwa} is preceded by a front vowel. The difference between \isi{Schwa Fronting-1} and \isi{Schwa Fronting-2} is the set of triggers: For the former it is front vowels and for the latter it is liquids. For some discussion of both processes of \isi{schwa} fronting the reader is referred to \sectref{sec:12.8.1}.

\ea 
 \label{ex:5:38}\isi{Schwa Fronting-2}:\\
 \begin{forest}
 [,phantom
   [\avm{[+son\\−nas]} [\avm{[coronal]},name=target]]
   [\avm{[−cons\\+son]},name=parent]
 ]
 \draw [dashed] (parent.south) -- (target.north);
 \end{forest}
\z 

The target for \isi{Schwa Fronting-2} is a placeless front vowel (=/ǝ/; recall \sectref{sec:2.2.3}). When that assimilation applies a derived front vowel is created which bears the three features [--consonantal, +sonorant, coronal] but no others. That synchronically derived feature complex is distinct from the features characterizing all underlying front vowels, which bear specification for either height features, the \isi{tenseness} feature, or both.\footnote{{The target must be a placeless vowel. If a full back vowel occurs after a liquid and before /x/, then Schwa Fronting{}-2 fails to apply; cf. [ʃtrʊx] ‘shrub’ from \REF{ex:5:33a}, which surfaces with [x] and not [ç]. The [coronal] feature cannot progressively assimilate from /r/ to /ʊ/ in that type of example because /ʊ/ is a full back vowel, which by definition bears a place feature (e.g. [dorsal]).}}

  \isi{Schwa Epenthesis} \isi{feeds} \isi{Schwa Fronting-2}, thereby creating the derived front vowel, which in turn \isi{feeds} \isi{Velar Fronting-1} (Vel Fr-1). The transparent system depicted in \REF{ex:5:39} is a specific example of the hypothetical Dialect B from \figref{fig:2.6}.

\ea%39
    \label{ex:5:39}
            \begin{tabular}[t]{ll}
                  & /kɛrx/  \\
\isi{Schwa Epenthesis}  & kɛrǝx   \\
\isi{Schwa Fronting-2}  & kɛrə̟x  \\
Vel Fr-1          & kɛrə̟ç  \\
                  & [kɛrə̟ç]\\
                  & ‘church’\\
              \end{tabular}
\z 

\begin{sloppypar}
The advantage of the present treatment is that it is fully transparent and therefore does not rely on an otherwise unattested type of (synchronic) \isi{opacity}, namely the \isi{overapplication} of velar fronting.
\end{sloppypar}

One conceivable objection to the transparent treatment in \REF{ex:5:39} is that the \isi{schwa} is not transcribed in \citet{Hommer1910} as a fronted vowel. However, an examination of the other sources for dialects with \isi{Schwa Epenthesis} in \tabref{tab:5.1} reveals that the epenthetic \isi{schwa} in the context between a liquid and a palatal fricative is typically transcribed with a distinct front vowel symbol from the \isi{schwa} in other contexts. As a representative example, consider the \il{East Franconian}EFr dialect spoken in \ipi{Suhl} (\citealt{Kober1962}; \mapref{map:4}). In that dialect the only (synchronic) target for velar fronting is /x/, which surfaces as [ç] after any front vowel and [x] after any back vowel. As in a number of case studies cited earlier, Kober’s symbols for the fortis palatal and velar fricatives are ⟦χ⟧ and ⟦x⟧ respectively, as in (\ref{ex:5:40a}, \ref{ex:5:40b}). It is clear from the material presented in the original source that there is an epenthetic \isi{schwa} between a liquid and /x/, as in \REF{ex:5:40c}. The additional data in \REF{ex:5:40d} illustrate that -- in contrast to \ipi{Sörth} -- epenthesis is not triggered by a labial or velar. The important point is that the epenthetic \isi{schwa} is transcribed in \citet{Kober1962} as ⟦ɩ̣⟧, which itself is not a phonemic vowel, but it differs minimally from the author’s (phonemic) high front unrounded lax vowel ⟦ɩ⟧, e.g. ⟦dɩχd⟧ ‘tight’. The item listed in \REF{ex:5:40a} is significant because it indicates Kober has the symbol for \isi{schwa} (⟦ǝ⟧), which is present if not inserted between a liquid and velar. The IPA transcriptions in \REF{ex:5:40} are the ones I assume to be correct. Kober’s ⟦ɩ̣⟧ is my [ə̟].

\ea\label{ex:5:40}
\ea\label{ex:5:40a}     wīχǝ   \tab [viːçǝ]  \tab   Wiege  \tab ‘cradle’   \tab 84  \\
\ex\label{ex:5:40b}     bōx    \tab [boːx]   \tab   Bogen  \tab ‘bow’      \tab 84  \\
\ex\label{ex:5:40c}     sǫrɩ̣χ \tab  [sɔrə̟ç]\tab    Sorge \tab   ‘sorrow’ \tab   70\\
                        folɩ̣χ \tab  [folə̟ç]\tab    folgen\tab   ‘follow-\textsc{inf}’ \tab 84  \\
\ex\label{ex:5:40d}     bɑlɡ   \tab [bɑlg]   \tab  Balken  \tab ‘beam’ \tab 87  \\
                        wɑrm   \tab [vɑrm]   \tab  warm    \tab ‘warm’ \tab 70  \\
\z 
\z 

Although \citet{Hommer1910} fails to provide a separate phonetic symbol for a fronted \isi{schwa}, I contend that his transcription was broad and was therefore not intended to capture a fine-grained distinction between two types of \isi{schwa}. All of the sources listed in \tabref{tab:5.1} with the exception of \citet{Münch19041970} and \citet{Greferath1922} transcribe the epenthetic \isi{schwa} differently than the underlying \isi{schwa}.\footnote{On the basis of some of the works cited in \tabref{tab:5.1} it appears that the fronted \isi{schwa} occurs between a liquid and a palatal (e.g. [ç]) or velar (e.g. [k]). This suggests that the target segment in \isi{Schwa Fronting-2} must be followed by a [dorsal] consonant. Since my treatment is not affected by this modification I do not discuss this matter further.}

Since velar fronting (regardless of dialect) is a phonological process and not a phonetic one the implication in \REF{ex:5:39} is that \isi{Schwa Fronting-2} is also phonological and not phonetic. If this is correct, then one would expect there to be dialects with some version of velar fronting and some version of \isi{schwa} epenthesis but without \isi{Schwa Fronting-2}. In that type of dialect palatals would surface after a front vowel (e.g. [siːç] ‘ailing’ in \ref{ex:5:33c}), velars after a full back vowel (e.g. [ʃtrux] ‘shrub’ in \ref{ex:5:33a}) and after the epenthetic \isi{schwa} (e.g. [kɛrǝx] ‘church’). Thus, \isi{Schwa Epenthesis} would \isi{bleed} velar fronting, as in \REF{ex:5:37b}. That type of dialect is extremely rare; in fact, the present survey has only uncovered one, namely the town of \ipi{Langenlutsch} in the former German-language island of \ipi{Schönhengst} in the Czech Republic (\citealt{Janiczek1911}; \mapref{map:9}). That dialect is discussed in \sectref{sec:15.3}.

The reason dialects like the one described by \citet{Janiczek1911} are so rare can be attributed to the geographic spread of velar fronting. My survey reveals that -- with only a small number of exceptions -- some version of postsonorant velar fronting is active in virtually all present-day German dialects (see \chapref{sec:12}). Given the extent to which velar fronting predominates geographically it is difficult -- although not impossible -- to find a non-velar fronting variety like the one documented by \citet{Janiczek1911}.

\section{{Discussion}}\label{sec:5.5}

\subsection{Rule reordering and domain narrowing}\label{sec:5.5.1}\is{Rule reordering|(}

The diachronic treatment of \isi{underapplication} \isi{opacity} proposed in this chapter is viewed below in the context of \isi{domain narrowing} (e.g. \citealt{Bermúdez-Otero2007,Bermúdez-Otero2015,Bermúdez-Otero2007}, \citealt{Ramsammy2015}). According to that theory, rules are phonologized at the end of the grammar and then gradually work their way up into smaller domains. As described in \sectref{sec:2.2.1}, many phonologists argue that the phonological component is subdivided into domains of various sizes to which rules are assigned. For example, in \isi{Stratal Optimality Theory} a distinction is drawn between phrase level, word level, and stem level rules. Examples from \ili{American English}  for those three domains were discussed in that earlier section, i.e. \is{Trisyllabic Laxing (English)}Trisyllabic Laxing (stem level), \is{n-Deletion (English)}n-Deletion (word level), and \is{Flapping (American English)}Flapping (phrase level). 

Domain narrowing postulates that rules work their way up (diachronically) from the lowest level (i.e. the largest domain) to higher levels (i.e. narrow domains). In particular, it is argued in the literature cited above that \is{phonetic rule}phonetic rules become categorical (i.e. phonological), at which point they are phrase level rules, and then they gradually become word level rules and finally stem level rules. A striking example \citep{Bermúdez-Otero2015} supporting \isi{domain narrowing} is the progression of postnasal \isi{g-Deletion} from phrase final position to word-final position and finally to stem-final position at various stages in the history of \ili{English} (see \tabref{tab:fromex:41}). The way in which the /ng/ sequence in bold (fifth column) is realized at the various stages is illustrated here. For recent discussion on the realization of /ng/ the reader is referred to \citet{Bailey2021}.

\begin{table}
\caption{Domain narrowing in the history of English g-Deletion (adapted from \citealt{Bermúdez-Otero2015}: 384)\label{tab:fromex:41}}
\begin{tabular}{ccccll}
\lsptoprule
 Stage 0 & Stage 1 & Stage 2 & Stage 3 & /ng/ &  \\\midrule
 \relax [ŋg] & [ŋg] & [ŋg] & [ŋg] & elo\textbf{ng}ate &  \textsc{{}-{}-{}-{}-{}-}\\
 \relax [ŋg] & [ŋg] & [ŋg] & [ŋ] & prolo\textbf{ng}{}-er & \textsc{stem} \textsc{level} \\
 \relax [ŋg] & [ŋg] & [ŋ] & [ŋ] & prolo\textbf{ng} it & \textsc{word} \textsc{level}\\
 \relax [ŋg] & [ŋ] & [ŋ] & [ŋ] & prolo\textbf{ng} ⟧ & \textsc{phrase} \textsc{level}\\
\lspbottomrule
\end{tabular}
\end{table}

Each of the four stages depicted in \tabref{tab:fromex:41} is shown in \citet{Bermúdez-Otero2015} to be attested. \isi{g-Deletion} was absent at Stage 0. At Stage 1 it applied at the end of a phrase (⟧), at Stage 2 at the end of a word, and at Stage 3 at the end of a stem.

I consider and reject casting my treatment of \ipi{Rhoden} from \sectref{sec:5.2} in the \isi{domain narrowing} approach. The same conclusions holds for the other case studies discussed in this chapter (in \sectref{sec:5.3}).

Recall that my treatment presupposes \isi{Final Fortition} was present in the grammar before \isi{Velar Fronting-5}, which was then added at the end of the grammar to produce the transparent Stage 2 \ipi{Soest} system. At Stage 3, those two rules are \isi{reordered}, thereby producing the opaque \ipi{Rhoden} system. That treatment is given in a simplified form in \tabref{tab:fromex:42}. I omit the numerical suffix on velar fronting for greater \isi{transparency}.

\begin{table}
\caption{\label{tab:fromex:42}Monostratal approach}
\begin{tabular}{ccc}
\lsptoprule
Stage 1 & Stage 2 & Stage 3 \\\midrule
\isi{Final Fortition} & \isi{Final Fortition} & Velar Fronting\\
                & Velar Fronting &  \isi{Final Fortition}\\
\lspbottomrule
\end{tabular}
\end{table}

The monostratal treatment in \tabref{tab:fromex:42} does not refer to the distinction between phrase level, word level, and stem level, as in \tabref{tab:fromex:41}. One way of applying that approach to \tabref{tab:fromex:42} would be to modify it by taking the three levels into consideration (\tabref{tab:fromex:43}):

\begin{table}%43
\caption{\label{tab:fromex:43}Domain narrowing}
\fittable{\begin{tabular}{ccccc}
\lsptoprule
Stage 1 & Stage 2i & Stage 2ii & Stage 3 & \\\midrule
        &          &           & {Velar Fronting} & \textsc{Stem level}\\
        &          &           &                  &  \\
{Final Fortition} & {Final Fortition} & {Final Fortition} &  {Final} Fortition & \textsc{Word level}\\
        &                   & {Velar} Fronting & & \\
        & {Velar} Fronting & & & \textsc{Phrase level}\\
\lspbottomrule
\end{tabular}}
\end{table}


At Stage 1, \isi{Final Fortition} is present (and is assumed to be at the word level), but velar fronting is absent. The latter process is added at the end of the grammar at Stage 2i at the phrase level and then works its way up to the word level at Stage 2ii. At Stage 3 (\ipi{Rhoden}) the domain of velar fronting narrows even further to the stem level. Stages 2i and 2ii correspond to two stages of \ipi{Soest} (not distinguished above).

In \sectref{sec:12.8.2} it is shown that velar fronting is a word level rule in all dialects for which data are available. There is therefore no evidence that there is some variety of German in which velar fronting is (or ever was) a phrase level rule. Given that conclusion I reject Stage 2i in \tabref{tab:fromex:43}. It is possible in theory to adopt an approach with a word level and a stem level as in \tabref{tab:fromex:43}, but it is also possible to collapse the two into a single level (word level), as in \tabref{tab:fromex:42}. Occam’s Razor points to \tabref{tab:fromex:42} as the simpler of the two treatments, and it is therefore the one I adopt. An additional argument pointing to the monostratal approach in \tabref{tab:fromex:42} is that, to the best of my knowledge, there is no evidence from German dialects for a distinction between stem level affixes and word level affixes, as proposed for \ili{English} (\sectref{sec:2.2.1}).\is{Rule reordering|)}

\subsection{Linguistic and philological evidence for historical stages}\label{sec:5.5.2}

From the synchronic perspective velar fronting has been shown in Chapters~\ref{sec:3}--\ref{sec:4} to be either fed or bled by another rule. The synchronic relationships (“Rules W/X \isi{feed} velar frontingˮ or “Rules Y,Z \isi{bleed} velar frontingˮ) are assumed to directly reflect history in the sense that Rules W-Z preceded velar fronting temporally. The same point holds for the relationship between the epenthesis of \isi{schwa} and velar fronting discussed in \sectref{sec:5.4}. In \sectref{sec:5.2} and \sectref{sec:5.3} I considered dialects where velar fronting is counterfed by another process (Rule W) and postulated that the synchronic ordering was originally the reverse, i.e. “Rule W \isi{feeds} velar frontingˮ > “velar fronting is counterfed by Rule Wˮ.

One question not discussed above is whether or not the historical relationships between velar fronting and the specific changes referred to as Rule W-Z can actually be confirmed with linguistic and/or philological evidence. Unfortunately that evidence is often (but not always) lacking. I discuss briefly the case studies referred to above.

Linguistic argumentation discussed in later chapters can be adduced that velar fronting must have been phonologized very early, namely in \ili{OHG} (500--1050) for HG and \ili{OSax} (800--1150) for LG, although it is also possible that \isi{phonologization} occurred in some places at a later time. This topic is discussed in \chapref{sec:16}. The important point is that if Rules W-Z were present in the grammar before velar fronting, then the former changes must have been very early ones.

My claim that the epenthesis of \isi{schwa} precedes velar fronting diachronically (\sectref{sec:5.4}) derives strong independent support. \citet[71--73]{Braune2004} discusses orthographic evidence for \isi{Schwa Epenthesis} (“Sprossvokaleˮ) at length, concluding that vowels were often epenthesized between a liquid and \textit{h} in \ili{OHG}, e.g. OHG \textit{duruh} (< \ili{WGmc} \textsuperscript{+}[θurx]) ‘through’. That type of epenthesis was especially prevalent in Franconian varieties of \ili{OHG}, which was the immediate precursor of \il{Moselle Franconian}MFr varieties like \ipi{Sörth} discussed above.

Two of the case studies discussed earlier involved SwG varieties. In \ipi{Maienfeld} (\sectref{sec:3.3}) velar fronting is bled by a rule of \isi{Debuccalization}, and in \ipi{Rheintal} (\sectref{sec:3.4}) it is fed by a rule fronting \isi{schwa} (\isi{Schwa Fronting-1}). However, there is no evidence available for those two places concerning the chronology of velar fronting or the processes debuccalizing /x/ or fronting \isi{schwa}. There is no reason to assume that velar fronting must have been active in \ili{OHG} in either of those two dialects because they each phonologized that process independently from one another and independently from all other German dialects.

A greater challenge is to confirm the relationship between \isi{Final Fortition} and velar fronting presupposed for LG in \REF{ex:5:11} and \REF{ex:5:12}. Orthographic evidence suggests that some version of \isi{Final Fortition} was probably already present in \ili{OSax}. \citet[78]{Holthausen1900} observes that the lenis labial fricative in that language (traditionally transcribed as [ƀ]) was realized as fortis [f] in word-final position and before fortis obstruents. \citet[81]{Holthausen1900} also assumes that [ɣ] was realized as fortis [x] in final position, although the orthographic evidence he cites is only sporadic. The philological evidence is admittedly thin; however, it is conceivable that a specific version of \isi{Final Fortition} with fricatives as targets was present before velar fronting was phonologized for the \il{Westphalian}Wph dialects in question (\ipi{Soest}, \ipi{Rhoden}). See \citet[1759]{Foerste1957} and \citet[23--27]{Woods1975} for some discussion of the status of \isi{Final Fortition} in \ili{OSax}.

A difficult claim to confirm is that the rules creating [x] from /ʀ/ were already active and applied transparently before the opaque stage arose (recall \ref{ex:5:28} and \ref{ex:5:32}). To the best of my knowledge, no linguistic or philological evidence is available which might (dis)confirm that treatment. It was assumed above that the {\textbar}ʁ{\textbar} created from /ʀ/ surfaced as [x] in coda position by either \isi{Final Fortition} or \isi{Laryngeal Assimilation-2}. According to \citet[131--133]{Paul2007}, orthographic evidence from \ili{OHG} and \ili{MHG} suggests that there was considerable variation concerning when and where lenis fricatives were realized as fortis. For more extensive discussion on dating \isi{Final Fortition} the reader is referred to \citet{Mihm2004}. In any case, no orthographic evidence from \ili{OHG} or \ili{MHG} suggests that \textit{r} in those earlier stages had a [x] realization in coda position.

\section{Conclusion}\label{sec:5.6}

The opaque examples discussed in this chapter can all be captured procedurally in terms of the interaction of one rule creating palatal [ç] from velar /x/ (velar fronting) and another one deriving {\textbar}x{\textbar} from an independent segment (Rule W). Since the velar derived by Rule W does not \isi{feed} velar fronting, the former \isi{counterfeeds} the latter; hence, opaque sequences like [ix] involve \isi{underapplication} \isi{opacity} in the synchronic grammar.

In the following chapter I discuss two dialects in which velar fronting creates palatals like [ç] from the corresponding velars (/x/), but those dialects also possess regular instances of [x] in the context of front vowels that also derive from /x/. The dialects in question therefore have sequences like [iç] (from /ix/) and ones like [ix] (from /ix/). The unexpected (opaque) velar referred to here is therefore not the consequence of a \isi{counterfeeding order} in the synchronic phonology. Instead, I demonstrate that the opaque velar [x] in sequences like [ix] are the consequence of a unique representation for the preceding front vowels.
