\chapter{Classification of High and Low German dialects}\label{appendix:a}

The classification of German dialects has been discussed at length in the literature on dialectology from the early nineteenth century up to the present day (e.g. \citealt{Schmeller1821, Götzinger1836, Wenker1877, Behaghel1911, Reise1912, Lenhardt1916, Weise1919, Sütterlin1924, Mitzka1943, PriebschCollinson1958, Martin1959, Schirmunski1962, König1978, Noble1983, Wolf1983, Schönfeld1983, Wiesinger1983a, Lameli2013, NiebaumMacha2014, HerrgenSchmidt2019}). There is consensus that dialects can be organized into two large categories, namely High German (HG) and Low German (LG). There is also agreement that the former can be split into two groups as well: Central German (CG) and Upper German (UG). The overall classification can therefore be depicted as in Figure \ref{ex:appendix:a:1}: 

\begin{figure}%1
    \caption{High German vs. Low German\label{ex:appendix:a:1}}
    \begin{forest} for tree={grow'=0}
    [,phantom
    [Low German (Niederdeutsch)]
    [High German (Hochdeutsch)
       [Central German (Mitteldeutsch)]
       [Upper German (Oberdeutsch)]
    ]
    ]
    \end{forest}
\end{figure}

The three broad groupings depicted above (LG, CG, UG) can be further subdivided. Thus,  CG and LG can be seen as consisting of a western and an eastern half, i.e. West Central German (WCG), East Central German (ECG), West Low German (WLG), and East Low German (ELG). UG can likewise be broken down into three groups: Alemannic (Almc), Bavarian (Bav) and East Franconian (EFr). The dialect groups just described (WCG, ECG, WLG, ELG, Almc, Bav, EFr) can be further decomposed into more fine-grained categories, although the proposals in the literature differ slightly from author to author. See Figures~\ref{ex:appendix:a:2} and~\ref{ex:appendix:a:3} for the expanded list of the LG and HG dialects that I will be adopting and making reference to throughout this book. The names for the specific categories within LG and HG are the one from \citet{Wiesinger1983a}, although he eschews the two broad groupings WLG and ELG. The dialects listed in Figures~\ref{ex:appendix:a:2} and~\ref{ex:appendix:a:3} are indicated below on \mapref{map:44}.

\begin{figure}[hp]%2
    \caption{Branches of Low German\label{ex:appendix:a:2}}
    \begin{forest} for tree={grow'=0, folder, l sep=12pt, s sep=0pt}
    [Low German (Niederdeutsch):
    [West Low German (Westniederdeutsch)
      [North Low German (Nordniederdeutsch)]
      [Westphalian (Westfälisch)]
      [Eastphalian (Ostfälisch)]
    ]
    [East Low German (Ostniederdeutsch)
      [Brandenburgish (Brandenburgisch)]
      [Mecklenburgish-West Pomeranian (Meklenburgisch-Vorpommersch)]
      [Central Pomeranian (Mittelpommersch)]
      [East Pomeranian (Ostpommersch)]
      [Low Prussian (Niederpreußisch)]
    ]
]
\end{forest}
\end{figure}

\begin{figure}[hp]
    \footnotesize
    \caption{Branches of High German\label{ex:appendix:a:3}}
    \begin{forest} for tree={grow'=0, folder, l sep=12pt, s sep=0pt}
    [High German (Hochdeutsch):
     [Central German (Mitteldeutsch):
         [West Central German (Westmitteldeutsch)
         [Low Franconian (Niederfränkisch)]
        [Rhenish Franconian (Rheinfränkisch)]
        [Central Franconian (Mittelfränkisch)
          [Moselle Franconian (Moselfränkisch)]
          [Ripuarian (Ripuarisch)]
        ]
        [Central Hessian (Zentralhessisch)]
        [North Hessian (Nordhessisch)]
        [East Hessian (Osthessisch)]]
      [East Central German (Ostmitteldeutsch)
        [Thuringian (Thüringisch)]
        [Upper Saxon (Obersächsisch)]
        [North Upper Saxon-South Markish (Nordobersächsisch-Südmärkisch)]
        [Silesian (Schlesisch)]
        [High Prussian (Hochpreußisch)]]
       ]
      [Upper German (Oberdeutsch):
        [Alemannic (Alemannisch)
          [High Alemannic (Hochalemannisch)]
          [Highest Alemannic (Höchstalemannisch)]
          [Low Alemannic (Niederalemannisch)]
          [Swabian (Schwäbisch)]
         ]
        [Bavarian (Bairisch)
          [North Bavarian (Nordbairisch)]
          [Central Bavarian (Mittelbairisch)]
          [South Bavarian (Südbairisch)]
         ]
        [East Franconian (Ostfränkisch)]
      ]
   ]
   \end{forest}
\end{figure}
\clearpage
\begin{map}
% \includegraphics[width=\textwidth]{figures/VelarFrontingHall2021-img050.png}
\includegraphics[width=\textwidth]{figures/Map44_A1.pdf}
  \caption[Dialects of High German and Low German]{Dialects of High German and Low German. Dialect boundaries from \citet{Wiesinger1983a}.
           \label{fig:a.4}}\label{map:44}
\end{map}
