\chapter{List of dialect dictionaries}\label{appendix:k}

\begin{description}[font=\normalfont]\sloppy
\item[AaWb:]  \textit{Aachener Sprachschatz. Wörterbuch der Aachener Mundart}. \textit{Beiträge zur Kultur- und Wirtschafts-Geschichte Aachens und seiner Umgebung}. Hermanns, Will. 1970. Aachen: J.A. Mayer Verlag.
\item[DoWb:]  \textit{Dortmunder Wörterbuch}. Schleef, Wilhelm. 1967. Cologne: Böhlau Verlag.
\item[DrWb:]  \textit{Mundart im Heinsberger Land. Dremmener Wörterbuch}. Gillessen, Leo. 1999. Cologne: Rheinland-Verlag.
\item[HaWb:]  \textit{Hamburgisches Wörterbuch}. Kuhn, Hans \& Ulrich Pretzel (eds.), 1956--2006. 5 volumes. Neumünster: Karl Wachholtz.
\item[KWb:]  \textit{Das Kölsche Wörterbuch. Kölsche Wörter von A-Z.} Bhatt, Christa \& Alice Herrwegen. 2005. Cologne: Verlag J. P. Bachem.
\item[MiElWb:]  \textit{Mittelelbisches Wörterbuch}. Kettmann, Gerhard (ed.), 2002--2008. 2 volumes. Berlin: Akademie Verlag.
\item[NKSS:]  \textit{Neuer Kölnischer Sprachschatz}. 1956. Wrede, Adam. 3 volumes. Cologne: Greven.
\item[NSSS:]  \textit{Neunkirchen-Seelscheider Sprachschatz}. 2013. Zweite Auflage. Lammert, Leo \& Paul Schmidt. Neunkirchen-\ipi{Seelscheid}: Heimat und Geschichtsverein Neunkirchen-Seelscheid e.V.
\item[ObersWb:]  \textit{Wörterbuch der obersächsischen und erzgebirgischen Mundarten.} Müller-Fraureuth, Karl. 1914. 2 volumes. Dresden: Wilhelm Baensch.
\item[PWb:]  \textit{Pommersches Wörterbuch.} Herrmann-Winter, Renate \& Matthias Vollmer. 2007. Berlin: Akademie Verlag.
\item[RWb:]  \textit{Rheinisches Wörterbuch.} Müller, Josef (ed.), 1928--1971. 9 volumes. Bonn: Fritz Klopp Verlag.
\item[SbWb:] \textit{Saarbrücker Wörterbuch.} Braun, Edith \& Max Mangold. 1984. Saarbrücken: Saabrücker Druckerei und Verlag.
\item[SchlHWb:]  \textit{Schleswig-Holsteinisches Wörterbuch}. (Volksausgabe). Mensing, Otto. 1927--1935, 5 volumes. Neumünster: Karl Wachholtz.
\item[SchwWb:]  \textit{Schwäbisches Wörterbuch.} Auf Grund der von Adelbert v. Keller begonnenen Sammlungen und mit Unterstützung des württembergischen Staates. Bearbeitet von Fischer, Hermann. 1904–1936. 6 Volumes. Tübingen: H. Laupp’schen Buchhandlung.
\item[SHesWb:] \textit{Südhessisches Wörterbuch.} Begründet von Friedrich Maurer nach den Vorarbeiten von Friedrich Mauer, Friedrich Stroh und Rudolf Mulch. Bearbeitet von Rudolf Mulch. 1965--2010. 6 volumes. Marburg: N.G. Elwert.
\item[SiWS:] \textit{Simmentaler Wortschatz. Wörterbuch der Mundart des Simmentals (Berner Oberland). Mit einer grammatischen Einleitung und mit Registern.} Armin Bratschi und Rudolf Trüb unter Mitarbeit von Lily Trüb sowie Maria Bratschi und Ernst Max Perren. Zeichnungen von Rolf Oberhänsli. Thun: Ott Verlag.
\item[TeWb:]  \textit{Wörterbuch der Teltower Volkssprache}. (\textit{Telschet Wöderbuek}). Lademann, Willy. 1956. Berlin: Akademie-Verlag.
\item[TiWb:] \textit{Wörterbuch der Tiroler Mundarten.} Schatz, Josef. 1955. 2 volumes. Innsbruck: Universitätsverlag Wagner.
\item[TrWb:] \textit{Trierer Wörterbuch.} \textit{Mit Sprachgesetzen derselben und Sprachproben in Prosa und Poesie.} Christa, Peter. 1927/1969. Wiesbaden: Dr. Martin Sandig.
\item[WbKM:]  \textit{Wörterbuch der Kölner Mundart.} Hönig, Fritz. 1952. Cologne: Verlag J. P. Bachem.
\item[WbMD:]  \textit{Wörterbuch der Mundart von Dobschau.} Lux, Julius. 1961. Marburg: N.G. Elwert.
\item[WbUS:]  \textit{Wörterbuch der unteren Sieg.} Fischer, Helmut. 1985. Cologne: Rheinland Verlag.
\item[WMlWb:] \textit{Wörterbuch der westmünsterländischen Mundart}. Piirainen, Elisabeth \& Wilhelm Elling. 1992. Vreden: Heimatverein Vreden
\item[WphWb:] \textit{Wörterbuch der westphälischen Mundart}. Woeste, Friedrich. 1882. Norden: Heinrich Soltau.
\end{description}

