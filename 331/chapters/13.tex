\chapter{Velar fronting in Lower Bavaria}\label{sec:13}\il{Central Bavarian|(}

\section{Introduction}\label{sec:13.1}

The triggers for the synchronic rule of velar fronting were shown in \chapref{sec:12} to mirror a series of historical stages, as in \tabref{tab:13.1}. Recall that HFV=high front vowel, MFV=mid front vowel, LFV=low front vowel, and CC=coronal sonorant consonant (/r l n/). The final column gives phonetic transcriptions for sequences corresponding to the various triggers for postsonorant velar fronting.

\begin{table}
\caption{Six velar fronting stages\label{tab:13.1}}
\begin{tabular}{ll ccccccc}
\lsptoprule
Stage & Triggers        & \multicolumn{7}{l}{Example (postsonorant context)}\\
\midrule
1   & ---               & [ix]& [ɪx]& [ex]& [ɛx]& [rx]& [æx]& [ɑx]\\
2a  & HFV               & [iç]& [ɪç]& [ex]& [ɛx]& [rx]& [æx]& [ɑx]\\
2b  & HFV, MFV          & [iç]& [ɪç]& [eç]& [ɛç]& [rx]& [æx]& [ɑx]\\
2c  & HFV, MFV, CC      & [iç]& [ɪç]& [eç]& [ɛç]& [rç]& [æx]& [ɑx]\\
2c' & HFV, MFV, LFV     & [iç]& [ɪç]& [eç]& [ɛç]& [rx]& [æç]& [ɑx]\\
2d  & HFV, MFV, LFV, CC & [iç]& [ɪç]& [eç]& [ɛç]& [rç]& [æç]& [ɑx]\\
\lspbottomrule
\end{tabular}
\end{table}

Stage 1 represents the point where velar fronting was absent; hence, velar sounds like /x/ surface as [x] regardless of the nature of the preceding sound. At Stage 2a only high front vowels induced velar fronting, and each subsequent stage incorporates more segments into the set of velar fronting triggers. Recall that Stage 2c and Stage 2c{}' are coterminous. Thus, Stage 2b has the choice of adding coronal sonorant consonants to the set of triggers at a later stage (Stage 2c) or low front vowels (Stage 2c{}').

Of the five velar fronting stages listed above, Stage 2a is the least inclusive because it reflects the narrowest natural class (high front vowels), while Stage 2d is the most inclusive because it captures the broadest natural class (coronal sonorants). Recall that Stage 2d represents the default pattern for German dialects. By contrast, Stage 2a was shown to be extremely rare, being only attested in two places, namely \ipi{Visperterminen} (\il{Highest Alemannic}HstAlmc) in Switzerland (\ipi{Upper Valais}) and \ipi{Plettenberg} (\il{Westphalian}Wph) in Germany (North Rhine-Westphalia).

What is lacking is a study of velar fronting in a particular area which documents cities, towns, and villages representing more than one of the postulated historical stages. The purpose of the present chapter is to eliminate this gap by discussing the synchronic state of velar fronting in Lower Bavaria (Niederbayern). The case study undertaken below demonstrates that velar fronting is the norm throughout Lower Bavaria, but a closer scrutiny of the data reveals that those velar fronting places exemplify four of the historical stages in \tabref{tab:13.1}, namely Stage 1, Stage 2a, Stage 2b, and Stage 2c'. A surprising -- but welcome -- result of the present investigation is that the rarest one of all, namely Stage 2a, is by far the most common one throughout Lower Bavaria.

The data are drawn exclusively from Part 5 of the six part \textit{Bayerischer Sprachatlas}, namely the \textit{Sprachatlas von Niederbayern} (SNiB). SNiB consists of seven volumes containing maps and narrow phonetic representations for the examples depicted on those maps. The reason this atlas is particularly suitable for this investigation is that Volume 3 on vowels, Volume 4 on consonants, and Volume 7 on the morphology of nouns contain phonetic representations for all of the two hundred twenty-one places in Lower Bavaria which were the sources for data (Belegorte).

Drawing data from a single source for a large number of places in a single area is advantageous because the result can be thought of as a snapshot of a particular area at a particular point in time. This snapshot is important because it gives clues as to how the original rule was originally phonologized and spread temporally and spatially.

The remainder of this chapter is organized as follows: \sectref{sec:13.2} provides some important background information, and \sectref{sec:13.3} gives data from SNiB representing Stage 1, Stage 2a, Stage 2b, and Stage 2c'. \sectref{sec:13.4} discusses the areal distribution of those four stages in Lower Bavaria and provides a map illustrating that distribution. In \sectref{sec:13.5} I discuss several issues that arise in the course of this chapter. \sectref{sec:13.6} concludes.

\section{Background}\label{sec:13.2}

Before presenting the data from SNiB it is essential that some information be provided on the geography of Lower Bavaria (\sectref{sec:13.2.1}), the state of velar fronting in Bavarian phonology on the basis of the descriptions of that dialect area cited in previous chapters (\sectref{sec:13.2.2}), and the SNiB transcription system (\sectref{sec:13.2.3}).

\subsection{Geography of Lower Bavaria}\label{sec:13.2.1}

Bavaria is divided into seven large administrative divisions called government districts (Regierungsbezirke), one of which is Lower Bavaria (\mapref{map:26}). It is bounded by the government districts of Upper Bavaria (Oberbayern) to the south and west and Upper Palatinate (Oberpfalz) to the north. To the northeast is the Czech Republic (South Bohemia), and to the southeast is Austria (\ipi{Upper Austria}). The numbers depicted on \mapref{map:26} represent the cites, towns, and villages which constitutes the Belegorte for SNiB. The names of those places are provided in Appendix~\ref{appendix:j}.


\begin{map}
% \includegraphics[width=\textwidth]{figures/VelarFrontingHall2021-img032.png}
\includegraphics[width=\textwidth]{figures/Map26_13.1.pdf}
\caption{Lower Bavaria}\label{map:26}
\end{map}

In contrast to Upper Bavaria and Upper Palatinate, there are no large urban centers in Lower Bavaria.  The largest three cities (with the approximate population in parentheses) are the capital Landshut (ca. 73,000), Passau (ca. 52,000), and Straubing (ca. 47,000).

Almost all of Lower Bavaria is situated within the \il{Central Bavarian}CBav dialect area (\mapref{map:3}), although some of the places in the north are classified as \il{North Bavarian}NBav (\mapref{map:4}). Four places in Lower Bavaria are depicted on those two maps, namely Heining (=[58]) on \mapref{map:3} and Zinzenzell (=[2]), Herrnsaal (=[38]), and Atting (=[40]) on \mapref{map:3}.

\subsection{Velar fronting in Bavarian phonology}\label{sec:13.2.2}

Most of the studies cited in this book on \il{Central Bavarian}CBav and \il{North Bavarian}NBav have shown that [x] and [ç] stand in an allophonic relationship which is expressed by some version of velar fronting. In this section I discuss briefly the segments that induce that process (triggers) as well as the sounds that undergo it (targets) with particular reference to Bav. I refer to the triggers as “potential” velar fronting triggers because the data discussed in this chapter reveal that those triggers are not the same in every place in Lower Bavaria.

Potential velar fronting triggers consist of some subset of the coronal sonorants -- coronal sonorant consonants and front vocoids (vowels and glides). Consider first the vocoids. In a number of case studies discussed earlier it has been demonstrated that the system of vowels -- both monophthongs and diphthongs -- can differ from place to place, even within the same dialect area. The same is true for Bav; hence, it needs to be stressed that the inventory of monophthongs I posit in \tabref{tab:fromex:13:1} might not be the same in all Bav varieties. The system in \tabref{tab:fromex:13:1} is very similar to the ones posited by other authors of Bav dialects, e.g. \citet[207]{Keller1963} for Upper Austrian (\il{Central Bavarian}CBav), \citet[422]{Rowley1989} for \il{North Bavarian}NBav, \citet[485--486]{Wiesinger1989} for the variety of \il{Central Bavarian}CBav spoken in and around \ipi{Munich} (München), \citet[17]{Bachmann2000} for the \il{North Bavarian}NBav variety of \ipi{Eslarn} (Upper Palatinate), and \citet{Bolter2021} for Austrian German varieties spoken in Styria. There is general agreement that front rounded vowels (e.g. /y ø/) are absent from the set of contrastive vowels \citep[452]{Wiesinger1989}.\footnote{The differences between the vowels in \tabref{tab:fromex:13:1} and the ones posited in the aforementioned studies is immaterial. For example, Keller assumes that front rounded vowels are phonemic in some varieties in \ipi{Upper Austria}, Keller and Bachmann also have nasalized vowels, Bachmann posits \isi{schwa} (/ə/), and Keller, Rowley and Bolter have /a/ instead of /æ/.}

\begin{table}
\caption{\label{tab:fromex:13:1}Distinctive vowels of Bavarian}
\begin{tabular}{lcc}
\lsptoprule
high & i & u \\
mid & e & o\\
    & ɛ & ɔ\\
low & æ & ɑ\\
\lspbottomrule
\end{tabular}
\end{table}

The important takeaway from \tabref{tab:fromex:13:1} is that there are four segments that are potential velar fronting triggers, namely /i e ɛ æ/. Those four segments are distinguished along the height dimension; hence, /i/ is high, /e ɛ/ are mid, and /æ/ is low. In this book I have not committed myself to a particular phonetic property distinguishing the two mid front vowels, /e/ and /ɛ/; in previous chapters I simply assume the cover feature [±tense], e.g. /e/ is [+tense] and /ɛ/ is [--tense]. It will be useful in the remainder of this chapter to think of those two vowels in terms of degrees of openness in the IPA tradition. Hence, /e/ is more close than /ɛ/; or -- put differently -- /e/ occupies a higher level than /ɛ/, which is implicit in the way those two vowels are displayed in \tabref{tab:fromex:13:1}. Some authors similarly posit two (phonemic) levels of high front vowels for Bav, e.g. \citet{Noelliste2017} has both /i/ and /ɪ/ for \ipi{Ramsau am Dachstein} in Styria (\il{Central Bavarian}CBav).

In addition to the monophthongs in \tabref{tab:fromex:13:1} Bav has a number of phonemic diphthongs. Six of the diphthongs \citet{Rowley1989} lists for \il{North Bavarian}NBav are /ei/, /oi/, /ai/, /ou/, /ɔu/, /ɑu/. Of those diphthongs only the former three can potentially induce velar fronting because they end in a front \isi{vocoid}. All diphthongs in Bav are falling in the sense that the second component and not the first is nonsyllabic, i.e. a glide (e.g. [ei̯] in a narrow transcription).

The set of potential velar fronting triggers also consists of the three sonorant consonants /r l n/. In Bav those sounds are either dental or alveolar and hence phonologically [coronal]. See for example the consonant phonemes in \citet[423]{Rowley1989} as well as the other studies cited above. Both \citet{Rowley1989} and \citet{Wiesinger1989} note that the two liquids /l r/ are typically vocalized in coda position (recall \sectref{sec:3.5}). Since the vocalized /l/ surfaces as a front \isi{vocoid}, it is a potential velar fronting trigger, e.g. the pronunciation [zɔidz] (from (/zɔldz/) for \textit{Salz} ‘salt’ (\citealt{Wiesinger1989}: 459, 486).

The works cited in this book on Bav dialects agree that [x] and [ç] are the only two dorsal fricatives and that those two sounds stand in complementary distribution in postsonorant position, e.g. \citet[71]{Kranzmayer1956}, \citet[12--13]{Kufner1960} for Upper Bavaria; \citet[81–83; 103–104]{Dozauer1967} for \ipi{Bergstetten} (Upper Palatinate); \citet[43]{Bachmann2000} for \ipi{Eslarn} (Upper Palatinate), \citet{Noelliste2017} for \ipi{Ramsau am Dachstein} (Styria). The relationship between the ich-Laut and the ach-Laut in Bav phonology presupposed by those authors is depicted in \REF{ex:13:2}:

\ea%2
\label{ex:13:2}
\begin{forest}
[/x/  [{[x]}]    [{[ç]}]]
\end{forest}
\z 

Some of the authors referred to above observe that velar and palatal allophones also occur as geminates (e.g. \citealt{Bachmann2000} for \ipi{Eslarn}, \citealt{Dozauer1967} for \ipi{Bergstetten}). In some Bav varieties with the \isi{affricate} /kx/ that sound can likewise occur as the equivalent palatal ([kç]) in the context of front sounds (\chapref{sec:15}). As is typical for UG, the diminutive suffix \is{chen@\textit{-chen}}-\textit{chen} is absent in Bav; hence, there are no cases of [x] and [ç] occurring after back vowels, as in \il{Standard German}StG [tauxən] ‘dive-\textsc{inf}’ vs. [tauçən] ‘rope-\textsc{dim}’.

The literature cited throughout this book on Bav indicates that velar fronting in that dialect region is characterized by the broadest set of triggers (=coronal sonorants). This assessment is made explicit in \tabref{tab:12.6}, which lists sources for thirty-eight varieties of Bav and classifies them in terms of the targets and triggers for postsonorant velar fronting. Only three of those thirty-eight sources show that a subset of the coronal sonorants induce velar fronting, namely (\il{North Bavarian}NBav) \ipi{Eisendorf} \citep{Seemüller1908b}, and (\il{Central Bavarian}CBav) Isarwinkel \citep{Maier1965}, where velar fronting fails to be induced by a coronal consonant (/r/), and (\il{Central Bavarian}CBav) \ipi{Großberghofen} \citep{Gladiator1971}, where only nonlow front vowels trigger the change. In the remainder of this chapter I demonstrate that those three examples are more the rule than the exception in Lower Bavaria.

\subsection{SNiB transcription system}\label{sec:13.2.3}

SNiB follows the tradition adopted in some linguistic atlases of providing extremely narrow phonetic transcriptions which express very subtle articulations that are usually ignored in the descriptive grammars cited throughout this book, including the ones mentioned in the previous section. It is not always clear how the symbols and diacritics in SNiB match up with the ones adopted in the previous section, nor is it evident what segments are phonemic. For these reasons I present data from SNiB throughout this chapter using that source’s transcription system and make no attempt to translate those transcriptions into the symbols employed in the first twelve chapters of this book. It is therefore imperative to clarify the SNiB system of symbols and diacritics for velar fronting triggers and targets.

Consider first vowels. SNiB adopts the five basic phonetic symbols ⟦i e u o ɑ⟧, which represent the cross-linguistically common five vowel system. There are no front rounded vowels in the material discussed below, although SNiB also includes the symbols ⟦ü ö⟧.

The vowel symbols ⟦i e u o ɑ⟧ are enhanced with diacritics which capture the degree of openness. \tabref{tab:13.2} indicates that there are five such levels. I refer to the vowels in the first and second columns below as \textsc{i-vowels} and \textsc{e-vowels} throughout the remainder of this chapter. I only consider i-vowels and e-vowels because those are the front vowels which are potential velar fronting triggers. The back vowels ⟦u o ɑ⟧ can likewise be referred to as \textsc{u-vowels}, \textsc{o-vowels}, and \textsc{ɑ{}-vowels}.

\begin{table}
\caption{SNiB symbols for front vowels\label{tab:13.2}}
\begin{tabular}{lll}
\lsptoprule
i-vowels & e-vowels & Description\\\midrule
⟦i̤⟧       &  ⟦e̤⟧   &  very close\\
⟦ị⟧       &  ⟦ẹ⟧   &   close    \\
⟦i⟧        & ⟦e⟧     &  neutral   \\
⟦\k{i}⟧    & ⟦ę⟧     &  open      \\
⟦i͈⟧       &  ⟦e͈⟧   &     very open\\
\lspbottomrule
\end{tabular}
\end{table}

It should be clear that the five levels for i-vowels and e-vowels are considerably more fine-grained that the one in \tabref{tab:fromex:13:1}. What is not expressed in \tabref{tab:13.2} is that the transcriptions are even more narrow than what the diacritics suggest because SNiB occasionally encloses the diacritics expressing the five degrees of openness in parentheses. I do not include the parenthesis in the phonetic transcriptions given below.\footnote{{Some works cited earlier for Bav depart from \tabref{tab:fromex:13:1} by adopting more than two levels for i-vowels and/or e-vowels, but the maximum number of levels in those studies is three, e.g. \citet[X]{Kranzmayer1956} has three levels for e-vowels and \citet[1]{Wiesinger1970a} has three levels for i-vowels and e-vowels. Neither of those studies has demonstrated that the vowels posited are actually contrastive (phonemic).}}

Two additional vocalic sounds that play an important role below are ⟦ə⟧ and ⟦α⟧. It is clear from the way in which the symbols ⟦ə⟧ and ⟦α⟧ are employed in transcriptions that they represent two variant pronunciations of the sound referred to throughout this book as the vocalized-r, cf. \il{Standard German}StG [ɐ] in [uːɐ] ‘clock’ and [fɑːtɐ] ‘father’.

The vocalized pronunciation of coda /l/ is transcribed in SNiB as one of the i-vowels or e-vowels depending on the place in Lower Bavaria.

In addition to front vowels, coronal sonorant consonants (/r l n/) are potential velar fronting triggers. SNiB has a number of symbols for those three sounds, but the most important ones are ⟦r l n⟧. The pronunciation guide for all three volumes consulted does not discuss the place of articulation for those three sounds, although it is reasonable to assume on the basis of what is known about the phonetics and phonology of consonants in other varieties of Bav (see \sectref{sec:13.2.3}) that ⟦l n⟧ are denti-alveolar and that they are therefore phonologically marked for the frontness feature ([coronal]). SNiB notes that ⟦r⟧ is an apical trilled sound (“Zungenspitzen-r (gerollt)ˮ), which implies that ⟦r⟧ is phonologically [coronal]. SNiB also includes a symbol for an apical rhotic with a different manner of articulation (⟦ɹ⟧), which is referred to as “rubbedˮ (“geriebenˮ), as well as two symbols for uvular rhotics (⟦ʀ ʁ⟧). None of the data discussed below include the rhotics transcribed as ⟦ɹ ʀ ʁ⟧.

SNiB follows the tradition described in \sectref{sec:1.5} of assigning three distinct places of articulation to dorsal fricatives. Since SNiB also postulates fortis and lenis obstruents, there are consequently six separate categories with unique symbols, which are given in \tabref{tab:13.3}:

\begin{table}
\caption{SNiB symbols for dorsal fricatives\label{tab:13.3}}
\begin{tabular}{lll}
\lsptoprule
SNiB term & lenis & fortis\\\midrule
Palatal (ich-Laut) & \ExtraChars{ꭓ} & \ExtraChars{ꭔ}\\
Between ich and ach & \ExtraChars{ꭗ} & \ExtraChars{ꭘ}\\
Velar (ach-Laut) & \ExtraChars{x} & \ExtraChars{ꭖ}\\
\lspbottomrule
\end{tabular}
\todo[inline]{glyphs}
\end{table}

It has been made abundantly clear throughout this book that I follow the alternative tradition which posits two places of articulation for dorsal fricatives, namely front dorsals (palatals) and back dorsals (velars). This is also the position adopted by all of the authors cited in \sectref{sec:13.2.2}. As indicated in \tabref{tab:13.4}, I treat SNiB palatal sounds as front dorsals (palatals), but I collapse the SNiB velar sounds and the sounds belonging to the intermediate category as back dorsals. See also \chapref{sec:15} for a similar interpretation of the three-way place distinction for dorsal fricatives presupposed in two other linguistic atlases, namely SDS and VALTS. The fortis and lenis articulations in \tabref{tab:13.3} are grouped together in \tabref{tab:13.4} because none of the data considered below suggest that the front and back dorsals belonging to those two categories behave differently.

\begin{table}
\caption{SNiB symbols for dorsal fricatives and their probable interpretation\label{tab:13.4}}
\begin{tabular}{lll}
\lsptoprule
SNiB symbols & Phonological features & Probable phonetic \\ & & realization\\\midrule
Palatal ⟦\ExtraChars{ꭓ}⟧/⟦\ExtraChars{ꭔ}⟧ & [coronal, dorsal] (front dorsal) & Palatal [ç]\\
Intermediate ⟦\ExtraChars{ꭗ}⟧/⟦\ExtraChars{ꭘ}⟧ & [dorsal] (back dorsal) & Velar [x]\\
Velar ⟦\ExtraChars{x}⟧/⟦\ExtraChars{ꭖ}⟧ & [dorsal] (back dorsal) & Uvular [\ExtraChars{χ}]\\
\lspbottomrule
\end{tabular}
\todo[inline]{glyphs}
\end{table}

In the second column of \tabref{tab:13.4} I provide the place features for the three SNiB places of articulation, and in the final column I give what I consider to be the most likely phonetic realization of the corresponding sounds. Thus, the palatal place of articulation (front dorsal) is pronounced as [ç], but the simplex [dorsal] articulation can either be realized as velar or as uvular.

Recall from \sectref{sec:12.9.2} and \tabref{tab:12.37} that the distinction between velar and uvular is not relevant for the phonology of German dialects. Hence, the two simplex [dorsal] articulations from SNiB are realized as velar or uvular by \is{phonetic rule}phonetic rules.

The clearest argument for grouping SNiB’s intermediate category (⟦\ExtraChars{ꭗ}⟧/⟦\ExtraChars{ꭘ}⟧) together with the velar category (⟦x⟧/⟦\ExtraChars{ꭖ}⟧) is that those two sets of sounds behave phonologically the same way, namely as back dorsals. This point can be made clear by considering the phonetic transcriptions in SNiB for the dorsal fricative in the context after a back vowel. A typical example is the word \textit{Joch}  ‘yoke’ from Map 94, Volume 4. The data accompanying that map reveal that there are sixty-nine places in Lower Bavaria in which the dorsal fricative is transcribed after an o-vowel with the symbols for the intermediate category (=⟦\ExtraChars{ꭗ} \ExtraChars{ꭘ}⟧), thirty-eight places where that fricative is transcribed with a symbol for the velar category (=⟦\ExtraChars{x ꭖ}⟧), and five places with transcriptions with symbols for the palatal category (=⟦\ExtraChars{ꭓ ꭔ}⟧). Although palatals can regularly occur after all or some back vowels in the areas discussed in \chapref{sec:14} and \chapref{sec:15}, none of those places are in Lower Bavaria. This suggests that the five places with a palatal after the back vowel in \textit{Joch} are simply anomalies; this assessment derives further support from the fact that dorsal fricatives in those five places are not realized as palatal after back vowels in other words. What is more significant is that the velar category (⟦\ExtraChars{x ꭖ}⟧) and the intermediate category (⟦\ExtraChars{ꭗ ꭘ}⟧) both predominate in \textit{Joch} (i.e. after an o-vowel), which is not surprising given that back vowels (such as o-vowels) are the prototypical context for the ach-Laut in \il{Standard German}StG and in all of the velar fronting varieties discussed in previous chapters. Similar statistics can be obtained from the data accompanying SNiB maps for other words which contain a back vowel followed by a dorsal fricative.

As noted earlier, I provide all data below in the original transcriptions from SNiB. I make extensive reference to the five levels of i-vowels and e-vowels and show how those sounds can be understood given the featural system adopted in previous chapters. I refer to dorsal fricatives in terms of two place categories, namely front dorsals (palatals) and back dorsals (velars).

\section{{Data} {representing} {four} {historical} {stages}}\label{sec:13.3}

The distribution of front dorsals and back dorsals in the data accompanying the SNiB maps reveal that the places indicated on \mapref{map:26} exhibit one of the four stages from \tabref{tab:13.1}, namely Stage 1 (\sectref{sec:13.3.1}), Stage 2 (\sectref{sec:13.3.2}), Stage 2b (\sectref{sec:13.3.3}), or Stage 2c{}' (\sectref{sec:13.3.4}).

\subsection{Stage 1}\label{sec:13.3.1}
\begin{sloppypar}
As is demonstrated below, velar fronting is the norm throughout Lower Bavaria. Nevertheless, conservative enclaves are attested without velar fronting. This statement is based on the maps in SNiB for the modern reflexes of \ili{MHG} /x/ in the context after a vowel. Nonvelar fronting places are characterized by realizations of that sound with back dorsals (⟦\ExtraChars{ꭗ ꭘ x ꭖ}⟧) after front and back vowels alike. Needless to say, front dorsals (⟦\ExtraChars{ꭓ ꭔ}⟧) are absent in Stage 1 varieties.
\end{sloppypar}

In \REF{ex:13:3} I present data from one particular place representing Stage 1, namely Rinchnach (=[33]). In the first column I give the phonetic transcriptions from SNiB, and in the final column I list the volume and map number of the corresponding map and list of data. In many of the phonetic transcriptions from SNiB the original fricative (/x/) is now realized either as [h] or as a manner of articulation somewhere between that of a fricative and that of an approximant. I only consider data in which \ili{MHG} /x/ is realized as a dorsal fricative and therefore do not take reduced variants into consideration. Some of the examples in \REF{ex:13:3} and below are the transcriptions for words embedded in a longer phrase, which I do not include. The data are arranged according to the degree of openness of the vowel preceding /x/, namely i-vowels in (\ref{ex:13:3a}--\ref{ex:13:3c}), e-vowels in (\ref{ex:13:3d}--\ref{ex:13:3f}), and back vowels in (\ref{ex:13:3g}--\ref{ex:13:3m}). The dataset in \REF{ex:13:3} shows that palatals (⟦\ExtraChars{ꭓ ꭔ}⟧) are absent entirely and that \ili{MHG} /x/ surfaces as a back dorsal [x] (=⟦\ExtraChars{ꭗ} x⟧) regardless of the nature of the preceding vowel.\footnote{{The SNiB transcription system incorporates a number of diacritics which express various articulations not directly relevant to velar fronting. Examples include \isi{lip rounding} (e.g. ⟦ë⟧), \isi{nasalization} (e.g. ⟦ẽ⟧), length (e.g. ⟦ē⟧), half-length (e.g. ⟦ê⟧), unexpected conspicuous shortening (e.g. ⟦ĕ⟧), aspiration (e.g. ⟦th⟧/⟦t}\textrm{ʰ}\textrm{⟧), strengthened lenis sounds (e.g. ⟦b̩⟧), and weakened fortis sounds (e.g. ⟦t͓⟧).}}


\todo{glyphs} 

\TabPositions{.15\textwidth, .45\textwidth, .66\textwidth, .8\textwidth}
\ea%3
\label{ex:13:3}Stage 1 for Rinchnach (=[33])
\ea\label{ex:13:3a} š̩d̩ị̄\ExtraChars{ꭗ} \tab Stiche \tab  ‘sting-\textsc{pl}’ \tab 7: 139
\ex\label{ex:13:3b} šdĭ\ExtraChars{ꭗ} \tab Stich \tab ‘sting’ \tab 3: 4\\
    we͈i\ExtraChars{ꭗ}α \tab weihen \tab ‘sanctify-\textsc{inf}’ \tab 4: 122
\ex\label{ex:13:3c} id̩lɑ̣\k{i}\ExtraChars{ꭗ} \tab (in die) Leich \tab ‘(to the) burial’ \tab 4: 28\\
    khọ̄\k{i}\ExtraChars{ꭗ} \tab Kalk \tab  ‘lime’ \tab 4: 80\\
    bɑ̣\k{i}\ExtraChars{ꭗ} \tab Bäuche \tab ‘stomach-\textsc{pl}' \tab 7: 82
\ex\label{ex:13:3d} b̩ẹ\ExtraChars{ꭗ} \tab Pech \tab ‘misfortune’ \tab 4: 39\\
    ǫẹ\ExtraChars{ꭗ}α \tab Eiche \tab ‘oak tree’ \tab 7: 57\\
    lẹ̆\ExtraChars{ꭗ}α \tab Löcher \tab ‘hole-\textsc{pl}’ \tab 7: 105
\ex\label{ex:13:3e} re\ExtraChars{ꭗ}α \tab Rechen \tab ‘rake’ \tab 3: 30
\ex\label{ex:13:3f} we͈\ExtraChars{ꭗ}α \tab weihen \tab ‘sanctify-\textsc{inf}’ \tab 4: 122
\ex\label{ex:13:3g} rɑ̣o̤xα \tab rauchen \tab ‘smoke-\textsc{inf}’ \tab 4: 127
\ex\label{ex:13:3h} jọ̆x \tab Joch \tab ‘yoke’ \tab  4: 94
\ex\label{ex:13:3i} dox \tab Dach \tab ‘roof’ \tab 4: 131\\
    moxα \tab machen \tab ‘do-\textsc{inf}’ \tab 3: 78\\
    woxαn \tab Wochen \tab ‘week-\textsc{pl}’ \tab 7: 61\\
    brɑ̣oxα \tab brauchen \tab ‘need-\textsc{inf}’ \tab 4: 128
\ex\label{ex:13:3j} nǫxt͓ʰ \tab Nacht \tab ‘night’ \tab 7: 75
\ex\label{ex:13:3k} dɑ̣xα \tab Dächer \tab ‘roof-\textsc{pl}’ \tab 7: 113\\
    nɑ̣xd͈ʰ \tab Nächte \tab  ‘night-\textsc{pl}’ \tab 7: 76
\ex\label{ex:13:3l} buαxα \tab Buche \tab ‘beech tree’ \tab 3: 130\\
    duαx \tab durch \tab ‘through’ \tab 3: 29
\ex\label{ex:13:3m} s̩ɡ̩v̩iə\ExtraChars{ꭗ}αd \tab das Vieh (Gefiechert) \tab ‘the cattle’ \tab 3: 5
\z
\z 

The items \textit{Kalk} in \REF{ex:13:3c} and \textit{durch} (\ref{ex:13:3l}) illustrate \isi{Liquid Vocalization}, which was already shown to be active in Austrian varieties of CG in \sectref{sec:3.5}. The generalization is that /l/ vocalizes to a front vowel and /r/ to a back vowel.\footnote{{The presence of the underlying /r/ and /l/ in these items can be inferred from German orthography, as indicated in the second column of \REF{ex:13:3}. One could take the alternative position that the ⟦\k{i}⟧ in} \textrm{\textit{Kalk}} \textrm{and the ⟦α⟧ in} \textrm{\textit{durch}} \textrm{are present in the underlying representation, in which case \isi{Liquid Vocalization} is not a synchronic rule, although it was uncontroversially active diachronically. In the remainder of this section I assume that \isi{Liquid Vocalization} operates synchronically, although my treatment of velar fronting is also compatible with the alternative approach.}}

Stage 1 also includes places with the general pattern as in \REF{ex:13:3} but with one word with an unexpected palatal after an i-vowel. One example is Pocking-Hartkirchen (=[205]), which has data comparable to the ones in \REF{ex:13:3} with the symbols for [x] (=⟦\ExtraChars{ꭗ} x⟧) after all vowels, e.g. ⟦nǫxd⟧ ‘night’, ⟦v̩ɑ̣e̤\ExtraChars{ꭗ}n̥⟧ ‘spruce’ (4: 125), ⟦khọ\k{i}\ExtraChars{ꭗ}⟧ ‘lime’, ⟦vī\ExtraChars{ꭗ}⟧ ‘cattle’ with the exception of the word ⟦dei\ExtraChars{ꭓ}dα⟧ ‘daughters’ (=7: 123) with a palatal after the i-vowel ⟦i⟧.

\subsection{Stage 2a}\label{sec:13.3.2}

In the SNiB transcription system Stage 2a dialects can be identified if ⟦\ExtraChars{ꭓ ꭔ}⟧ (=[ç]) occur after i-vowels and ⟦\ExtraChars{ꭗ ꭘ x ꭖ}⟧ (=[x]) after e-vowels and back vowels.  

Stage 2a is attested in Wurmsham (=[207]), which possesses the four front vowels ⟦i e̤ ẹ e⟧. The dataset in \REF{ex:13:4} shows that ⟦\ExtraChars{ꭓ}⟧ (=[ç]) surfaces after ⟦i⟧ in (\ref{ex:13:4a}) and ⟦\ExtraChars{ꭗ} x⟧ (=[x]) after ⟦e̤⟧ in (\ref{ex:13:4b}), ⟦ẹ⟧ in (\ref{ex:13:4c}), ⟦e⟧ in (\ref{ex:13:4d}), and back vowels in (\ref{ex:13:4e}--\ref{ex:13:4h}).

\ea%4
\label{ex:13:4}Stage 2a for Wurmsham (=[207])
\ea\label{ex:13:4a} šdi\ExtraChars{ꭓ} \tab Stich \tab ‘sting’ \tab 3: 4\\
    šdi\ExtraChars{ꭓ} \tab Stiche \tab ‘sting-\textsc{pl}’ \tab 7: 139\\
    vī\ExtraChars{ꭓ} \tab Vieh \tab ‘cattle’ \tab 3: 5\\
    khọi\ExtraChars{ꭓ} \tab Kalk \tab ‘lime’ \tab 4: 80\\
\ex\label{ex:13:4b} ɑ̣d̩lɑ̣e̤\ExtraChars{ꭗ}ɡẹ̄ \tab (auf die) Leich \tab ‘(to the) burial’ \tab 4: 28\\
    v̩ɑ̣e̤\ExtraChars{ꭗ}n̥ \tab Fichte \tab ‘spruce’ \tab 4: 125
\ex\label{ex:13:4c} b͈ẹ\ExtraChars{ꭗ} \tab Pech \tab ‘misfortune’ \tab 4: 39\\
    bɑ̣e̤\ExtraChars{ꭗ} \tab Bäuche \tab ‘stomach-\textsc{pl}’ \tab 7: 82
\ex\label{ex:13:4d} d̩ë\ExtraChars{ꭗ}dα \tab Töchter \tab ‘daughter-\textsc{pl}’ \tab 7: 123\\
    ɡŋe\ExtraChars{ꭗ}th \tab Knecht \tab ‘vassal’ \tab 4: 124
\ex\label{ex:13:4e} lọ\ExtraChars{ꭗ} \tab Loch \tab ‘hole’ \tab 7: 104\\
    d̩ọ\ExtraChars{ꭗ}dα \tab Tochter \tab ‘daughter’ \tab 7: 122\\
    bɑọ\ExtraChars{ꭗ} \tab Bauch \tab ‘stomach’ \tab 7: 81
\ex\label{ex:13:4f} dǫx \tab Dach \tab ‘roof’ \tab 7: 112\\
    nǫxth \tab Nacht \tab ‘night’ \tab 7: 75
\ex\label{ex:13:4g} nɑ̣x̩th \tab Nächte \tab ‘night-\textsc{pl}'\tab 7: 76\\
    dɑ̣\ExtraChars{ꭗ}α \tab Dächer \tab ‘roof-\textsc{pl}’ \tab 7: 113
\ex\label{ex:13:4h} ọαx̩ \tab Eiche \tab ‘oak tree’ \tab 7: 57\\
    bu̩αx̩ \tab Buche \tab ‘beech tree’ \tab 3: 130\\
    dûαx \tab durch \tab ‘through’ \tab 3: 29
    \z
\z 

The word ⟦khọi\ExtraChars{ꭓ}⟧ in \REF{ex:13:4a} reveals that the vowel produced by \isi{Liquid Vocalization} for a target /l/ is a high front vowel that serves as a trigger for velar fronting. Thus, the i-vowels that induce velar fronting include not only phonemic i-vowels, but also synchronically-derived i-vowels. As shown below in \sectref{sec:13.5.2}, the \isi{feeding order} between \isi{Liquid Vocalization} (for /l/) and velar fronting is not only true for Wurmsham (=[207]), but it represents the unmarked pattern for velar fronting in Lower Bavaria.

The five i-vowels ⟦i̤ ị i i͈ \k{i}⟧ are phonologically [+high] throughout Lower Bavaria. In that type of Stage 2a system, velars ([x]) and palatals ([ç]) are allophones of underlying velars (/x/), and velar fronting is restricted to the context after [+high] vowels. The rule of \isi{Velar Fronting-6} -- posited earlier in \sectref{sec:6.2.2} for \ipi{Visperterminen} -- is way of expressing formally the restricted set of triggers characterized by Stage 2a. Recall that back vowels are analyzed in \ipi{Visperterminen} as [peripheral], which I replace with [dorsal] in \REF{ex:13:5}.

\ea%5
\label{ex:13:5}\isi{Velar Fronting-6}:\\
\begin{forest} for tree = {fit=band}
  [,phantom
    [\avm{[+high]} [\avm{[coronal]},name=target,tier=word]]
    [\avm{[−son\\+cont]}, name=parent [\avm{[dorsal]},tier=word]]
  ]
  \draw [dashed] (parent.south) -- (target.north);
  \end{forest}
\z 

Most of the Stage 2a varieties identified below display a very regular system like the one in \REF{ex:13:4}. However, several Stage 2a places display one of two types of irregularity. In the first, [ç] regularly occurs after i-vowels and elsewhere [x], but there is a very small number of words (one or two) with an unexpected [x] after an i-vowel. In the second, [ç] and [x] have the expected distribution for a Stage 2a system, but there is an unexpected instance of [ç] after a non-i-vowel.

The first type of dialect is represented by Zinzenzell (=[2]). The maps in SNiB reveal that this is a clear case of Stage 2a with palatals after two levels of i-vowels (⟦i \k{i}⟧), and elsewhere velars, e.g. ⟦š̩d͈î\ExtraChars{ꭓ}⟧ ‘sting’, ⟦v̩e\k{i}\ExtraChars{ꭓ}α⟧ ‘cattle-\textsc{pl}’, ⟦khô\k{i}\ExtraChars{ꭓ}⟧ ‘lime’, ⟦wɑ̣\k{i}\ExtraChars{ꭓ}α⟧ ‘sanctify-\textsc{inf}’, ⟦bɑ̣\k{i}\ExtraChars{ꭓ}⟧ ‘stomach-\textsc{pl}’, ⟦dəs͈\k{i}\ExtraChars{ꭓ}⟧ ‘(the) colter’ (4: 132) vs. ⟦blẹ\ExtraChars{ꭗ}⟧ ‘tin’ (4: 130). The important point is that this Stage 2a system also possesses the aberrant item ⟦i͈nd̩lɑ̣\^{\k{i}}\ExtraChars{ꭗ}d͈⟧ ‘(to the) burial’ with ⟦\ExtraChars{ꭗ}⟧ (=[x]) after an i-vowel.

Neukirchen am Inn (=[178]) exemplifies the second type of Stage 2a system. The maps in SNiB indicate a clear Stage 2a pattern in which [ç] surfaces after i-vowels (⟦i \k{i}⟧) and [x] after e-vowels (⟦ẹ ę⟧) and back vowels, e.g. ⟦š̩di\ExtraChars{ꭓ}⟧ ‘sting’, ⟦vɑ̣i\ExtraChars{ꭓ}n⟧ ‘spruce’, ⟦v\^{\k{i}}\ExtraChars{ꭓ}α⟧ ‘cattle-\textsc{pl}’, ⟦ɑvd̩lɑ̣\^{\k{i}}\ExtraChars{ꭓ}⟧ ‘(to the) burial’, ⟦kho\k{i}\ExtraChars{ꭓ}⟧ ‘lime’ vs. ⟦b͈lẹ\ExtraChars{ꭗ}⟧ ‘tin’, ⟦rę\ExtraChars{ꭗ}α⟧ ‘rake’, and ⟦d̩ɑ̣xə⟧ ‘roof-\textsc{pl}’ The surprising item is the word ⟦d̩ę\ExtraChars{ꭓ}d͈α⟧ ‘daughter-\textsc{pl}’ with an unexpected palatal after an e-vowel.

Sandbach (=[157]) is similar to Neukirchen am Inn ([178]) with the one difference being that the unexpected palatal occurs after a back vowel, i.e. ⟦jo\ExtraChars{ꭓ}⟧ ‘yoke’. The occurrence of a palatal in the context after a back vowel is well-attested in various places outside of Lower Bavaria discussed in \chapref{sec:14}, but this is not a common feature in Lower Bavaria. Thus, phonetic transcriptions like ⟦jo\ExtraChars{ꭓ}⟧ are sporadic and therefore do not reflect a significant pattern; recall the discussion in \sectref{sec:13.2.3}.

\subsection{Stage 2b }\label{sec:13.3.3}

Stage 2b dialects are defined as places where [ç] occurs after i-vowels and [x] after back vowels. Within the class of e-vowels there is a threshold below which [x] occurs. The e-vowels after which [ç] surfaces are phonologically mid, while the e-vowels after which [x] occurs are phonologically low. The exact cut-off point between mid e-vowels and low e-vowels can differ from place to place.

I illustrate Stage 2b with three different places. The first is Voglarn (=[201]), which has one i-vowel ⟦i⟧ and four e-vowels ⟦e̤ ẹ e ę⟧. In that town, [ç] surfaces after the nonlow front vowels ⟦i e̤ ẹ e⟧ (=\ref{ex:13:6a}--\ref{ex:13:6d}) and [x] after the low front vowel ⟦ę⟧ \REF{ex:13:6e} and after back vowels (=\ref{ex:13:6f}--\ref{ex:13:6j}).

\ea%6
\label{ex:13:6}Data for Stage 2b in Voglarn (=[201])
\ea\label{ex:13:6a} šdī\ExtraChars{ꭓ} \tab Stich \tab ‘sting’ \tab 3: 4\\
    šdï\ExtraChars{ꭓ} \tab Stiche \tab ‘sting-\textsc{pl}’ \tab 7: 139\\
    v\"{î}\ExtraChars{ꭓ} \tab Vieh \tab ‘cattle’ \tab 3: 5
\ex\label{ex:13:6b} ɑ̣d̥lɑ̣e̤\ExtraChars{ꭓ} \tab (auf die) Leich \tab ‘(to the) burial’ \tab 4: 28\\
    bɑ̣e̤\ExtraChars{ꭓ} \tab Bäuche \tab ‘stomach-\textsc{pl}’ \tab 7: 82
\ex\label{ex:13:6c} v̩ɑ̣ẹ\ExtraChars{ꭓ}d͐n̥ \tab Fichte \tab ‘spruce’ \tab 4: 125\\
    khǫẹ\ExtraChars{ꭓ}̩ \tab Kalk \tab  ‘lime’ \tab 4: 80\\
    b̩ẹ\ExtraChars{ꭓ} \tab Pech \tab ‘misfortune’ \tab 4: 39\\
\ex\label{ex:13:6d} ble\ExtraChars{ꭓ} \tab Blech \tab ‘tin’ \tab 4: 130
\ex\label{ex:13:6e} ɡŋę\ExtraChars{ꭗ}t͓ʰ \tab Knecht \tab ‘vassal’ \tab 4: 124\\
    d\"{ę}\ExtraChars{ꭗ}dα \tab Töchter \tab ‘daughter-\textsc{pl}’ \tab 7: 123
\ex\label{ex:13:6f} jọ\ExtraChars{ꭗ} \tab Joch \tab ‘yoke’ \tab 4: 94\\
    lọ\ExtraChars{ꭗ} \tab Loch \tab ‘hole’ \tab 7: 104
\ex\label{ex:13:6g} do\ExtraChars{ꭗ}dα \tab Tochter \tab ‘daughter’ \tab 7: 122
\ex\label{ex:13:6h} dǫx̩ \tab Dach \tab ‘hole’ \tab 7: 112\\
    bǫx \tab Bach \tab ‘stream’ \tab 4: 33\\
    nǫ\ExtraChars{ꭗ}th \tab Nacht \tab ‘night’ \tab 7: 75
\ex\label{ex:13:6i} nɑ̣\ExtraChars{ꭗ}t͐n̥ \tab Nächte \tab ‘night-\textsc{pl}' \tab 7: 76
\ex\label{ex:13:6j} oα\ExtraChars{ꭗ}̩ \tab Eiche \tab ‘oak tree’ \tab 7: 57
\z
\z 

The featural system for the vowels of Voglarn is posited in \tabref{tab:fromex:13:7}. Recall from previous chapters that other dialects are attested in which certain e-vowels -- typically [ɛ] in the symbols given in \tabref{tab:fromex:13:1} -- are phonologically low vowels, while other e-vowels (i.e. [e]) are phonologically mid. The nature of the features distinguishing two or more vowels in the same column is not important and is therefore not discussed.

\begin{table}%7
\caption{Distinctive features for vowels (Voglarn)\label{tab:fromex:13:7}}
\begin{tabular}{lccccc}
\lsptoprule
                 & ⟦i⟧ & ⟦e̤ ẹ e⟧ & ⟦ę⟧ & ⟦ọ o ǫ⟧ & ⟦ɑ̣ ɑ⟧ \\\midrule
\relax [coronal] & \ding{51} & \ding{51} & \ding{51} &  & \\
\relax [dorsal] &  &  &  & \ding{51} & \ding{51}\\
\relax [low] & − & − & + & − & +\\
\relax [high] & + & − &  &  & \\
\lspbottomrule
\end{tabular}
\end{table}


Given the features in \tabref{tab:fromex:13:7} the rule for all Stage 2b dialects (=\ref{ex:13:8}) spreads the frontness feature ([coronal]) to the right from any [coronal, {}--low] vowel onto a dorsal fricative (/x/), thereby deriving [ç]. Recall that \isi{Velar Fronting-2} was shown to be active synchronically in a number of German dialects outside of Lower Bavaria, e.g. in \ipi{Rheintal} (Switzerland) in \sectref{sec:3.4}.

\ea%8
\label{ex:13:8}\isi{Velar Fronting-2}\\
\begin{forest}
[,phantom
  [\avm{[−low]} [\avm{[coronal]},name=target,tier=word]]
  [\avm{[−son\\+cont]},name=source [\avm{[dorsal]},tier=word]]
]
\draw [dashed] (target.north) -- (source.south);
\end{forest}
\z 

The second Stage 2b system is Reicheneibach (=[185]), which possesses one i-vowel ⟦i⟧ and three e-vowels ⟦ẹ e ę⟧. In that town, [ç] surfaces after the nonlow front vowels ⟦i ẹ⟧ (=\ref{ex:13:9a}, \ref{ex:13:9b}) and [x] after the low front vowels ⟦e ę⟧ (=\ref{ex:13:9c}, \ref{ex:13:9d}) and after back vowels (=\ref{ex:13:9e}--\ref{ex:13:9i}).

\ea%9
\label{ex:13:9}Data for Stage 2b in Reicheneibach (=[185])
\ea\label{ex:13:9a} šdî\ExtraChars{ꭓ} \tab Stich  \tab  ‘sting’ \tab 3: 4\\
    šd̩ï\ExtraChars{ꭓ} \tab Stiche  \tab  ‘sting-\textsc{pl}’ \tab 7: 139\\
    v̩ī\ExtraChars{ꭓ} \tab Vieh  \tab  ‘cattle’ \tab 3: 5
\ex\label{ex:13:9b} ɑ̣d̤lɑ̣ẹ\ExtraChars{ꭓ} \tab (in die) Leich \tab ‘(to the) burial’ \tab 4: 28\\
    v̩ɑ̣ẹ\ExtraChars{ꭓ}t͐n̥ \tab Fichte  \tab  ‘spruce’ \tab 4: 125\\
    bɑ̣ẹ\ExtraChars{ꭓ} \tab Bäuche  \tab  ‘stomach-\textsc{pl}’ \tab 7: 82\\
    khoẹ\ExtraChars{ꭓ} \tab Kalk  \tab   ‘lime’ \tab 4: 80\\
    b̩ẹ\ExtraChars{ꭓ} \tab Pech  \tab  ‘misfortune’ \tab 4: 39
\ex\label{ex:13:9c} b̩le\ExtraChars{ꭗ} \tab Blech  \tab  ‘tin’ \tab 4: 130
\ex\label{ex:13:9d} ɡnę\ExtraChars{ꭗ}tʰ \tab Knecht  \tab  ‘vassal’ \tab 4: 124\\
    t͓\"{ę}\ExtraChars{ꭗ}d̩α \tab Töchter  \tab  ‘daughter-\textsc{pl}’ \tab 7: 123
\ex\label{ex:13:9e} lọ\ExtraChars{ꭗ} \tab Loch  \tab  ‘hole’ \tab 7: 104\\
    d̩ọxd̩α \tab Tochter  \tab  ‘daughter’ \tab 7: 122
\ex\label{ex:13:9f} dox \tab Dach  \tab  ‘roof’ \tab 7: 112\\
    jo\ExtraChars{ꭗ} \tab Joch  \tab  ‘yoke’ \tab 4: 94\\
    bɑ̜ọx \tab Bauch  \tab  ‘stomach’ \tab 7: 81
\ex\label{ex:13:9g} nǫ\ExtraChars{ꭗ}t͓h \tab Nacht  \tab  ‘night’ \tab 7: 75
\ex\label{ex:13:9h} nɑ̣\ExtraChars{ꭗ}t͓h \tab Nächte  \tab  ‘night-\textsc{pl}’ \tab 7: 76
\ex\label{ex:13:9i} dûαx \tab durch  \tab  ‘through’ \tab 3: 29
\z
\z 

The Reicheneibach system has the featural specifications in \tabref{tab:fromex:13:10}. The crucial difference between \tabref{tab:fromex:13:10} and \tabref{tab:fromex:13:7} is that ⟦ẹ e⟧ are both mid vowels in \tabref{tab:fromex:13:7}, but in \tabref{tab:fromex:13:10} only ⟦ẹ⟧ is mid, while ⟦e⟧ is low.

\begin{table}%10
\caption{Distinctive features for vowels (Reicheneibach)\label{tab:fromex:13:10}}
\begin{tabular}{lccccc}
\lsptoprule
         & ⟦i⟧ & ⟦ẹ⟧ & ⟦e ę⟧ & ⟦ọ o ǫ⟧ & ⟦ɑ̣⟧ \\\midrule
\relax [coronal] & \ding{51} & \ding{51} & \ding{51} &  & \\
\relax [dorsal] &  &  &  & \ding{51} & \ding{51}\\
\relax [low] & − & − & + & − & +\\
\relax [high] & + & − &  &  & \\
\lspbottomrule
\end{tabular}
\end{table}

The third Stage 2b system is Martinshaun (=[125]), which has the two i-vowels ⟦i \k{i}⟧ and the four e-vowels ⟦e̤ ẹ e ę⟧. In that dialect, [ç] occurs after the nonlow front vowels ⟦i \k{i} e̤⟧ (=\ref{ex:13:11a}--\ref{ex:13:11c}), while [x] surfaces after the low front vowels ⟦ẹ e ę⟧ (=\ref{ex:13:11d}--\ref{ex:13:11f}) and back vowels (=\ref{ex:13:11g}--\ref{ex:13:11l}).

\ea%11
\label{ex:13:11}Data for Stage 2b in Martinshaun (=[125])
\ea\label{ex:13:11a} šdî\ExtraChars{ꭓ} \tab Stich \tab ‘sting’ \tab 3:\\
    v̩ī\ExtraChars{ꭓ} \tab Vieh \tab ‘cattle’ \tab 3: 5
\ex\label{ex:13:11b} šd\k{i}\ExtraChars{ꭓ}̩ \tab Stich \tab ‘sting’ \tab 3: 4\\
    m\k{u}\k{i}\ExtraChars{ꭓ}̩ \tab Milch \tab ‘milk’ \tab 3: 10\\
    khǫ\k{i}\ExtraChars{ꭓ} \tab Kalk \tab  ‘lime’ \tab 4: 80\\
    šd\k{i}\ExtraChars{ꭓ}̩ \tab Stiche \tab ‘sting-\textsc{pl}’ \tab 7: 139
\ex\label{ex:13:11c} î̩dlɑ̣e̤\ExtraChars{ꭓ} \tab (in die) Leich \tab ‘(to the) burial’ \tab 4: 28\\
    vɑ̣e̤\ExtraChars{ꭓ}tn \tab Fichte \tab ‘spruce’ \tab 4: 124\\
    b̩ɑ̣e̤\ExtraChars{ꭓ} \tab Bäuche \tab ‘stomach-\textsc{pl}' \tab 7: 82
\ex\label{ex:13:11d} bẹ\ExtraChars{ꭗ} \tab Pech \tab ‘misfortune’ \tab 4: 39\\
    blẹ\ExtraChars{ꭗ} \tab Blech \tab ‘tin’ \tab 4: 130
\ex\label{ex:13:11e} ɡŋex̩d̩ \tab Knecht \tab ‘vassal’ \tab 4: 124
\ex\label{ex:13:11f} r\={ę}xα \tab Rechen \tab ‘rake’ \tab 3: 30\\
    dęxdα \tab Töchter \tab ‘daughter-\textsc{pl}’ \tab 7: 123
\ex\label{ex:13:11g} jọ\ExtraChars{ꭗ} \tab Joch \tab ‘yoke’ \tab 4: 94\\
    b̩ɑọx \tab Bauch \tab ‘stomach’ \tab 4: 129
\ex\label{ex:13:11h} lōx \tab Loch \tab ‘hole’ \tab 7: 104
\ex\label{ex:13:11i} dǫx̩dα \tab Tochter \tab ‘daughter’ \tab 7: 122\\
    bǫx \tab Bach \tab ‘stream’ \tab 4: 33
\ex\label{ex:13:11j} nɑ̣x̩d \tab Nacht \tab ‘night’ \tab 7: 75\\
    nɑ̣x̩d \tab Nächte \tab ‘night-\textsc{pl}’ \tab 7: 76
\ex\label{ex:13:11k} dɑ̩x \tab Dach \tab ‘roof’ \tab 7: 112
\ex\label{ex:13:11l} dûαx \tab durch  \tab  ‘through’ \tab 3: 29
\z
\z 

The featural system for the vowels in Martinshaun ([125]) is given in \tabref{tab:fromex:13:12}. These features differ crucially from the ones in \tabref{tab:fromex:13:7} and \tabref{tab:fromex:13:10} in terms of the cut-off point between mid front and low front vowels. Thus, \tabref{tab:fromex:13:7} treats the two e-vowels ⟦ẹ e⟧ as mid ([{}--low, --high]), \tabref{tab:fromex:13:10} analyzes ⟦ẹ⟧ as mid      ([--low, --high]) and ⟦e⟧ as low ([+low]), and \tabref{tab:fromex:13:12} analyzes both of those e-vowels as low ([+low]).

\begin{table}%12
\caption{Distinctive features for vowels (Martinshaun)\label{tab:fromex:13:12}}
\begin{tabular}{lccccc}
\lsptoprule
                 & ⟦i \k{i}⟧ & ⟦e̤⟧ & ⟦ẹ e ę⟧ & ⟦ọ o ǫ⟧ & ⟦ɑ̣ ɑ̩⟧ \\\midrule
\relax [coronal] & \ding{51} & \ding{51} & \ding{51} &  & \\
\relax [dorsal]  &  &  &  & \ding{51} & \ding{51}\\
\relax [low] & − & − & + & − & +\\
\relax [high] & + & − &  &  & \\
\lspbottomrule
\end{tabular}
\end{table}

Although the featural systems proposed for Voglarn ([201]), Reicheneibach ([185]), and Martinshaun ([125]) are not the same, all of those possess precisely the same version of velar fronting, stated in \REF{ex:13:8}.

A number of other places in Lower Bavaria have Stage 2b as defined above, but those systems also possess irregularities where [x] surfaces after the e-vowels that are expected to always be the context for [ç]. Consider Malgersdorf (=[170]), where [ç] surfaces after i-vowels and [x] after ⟦e⟧, e.g. ⟦štī\ExtraChars{ꭓ}⟧ ‘sting’, ⟦svī\"{}\ExtraChars{ꭓ}⟧ ‘(the) cattle’ vs. ⟦k͓ʰǫe\ExtraChars{ꭗ}⟧ ‘lime’, ⟦de\ExtraChars{ꭗ}d̩α⟧ ‘daughter-\textsc{pl}’. No example is attested in this place with a dorsal fricative after e-vowel lower than ⟦e⟧, i.e. ⟦ę⟧ or ⟦e͈⟧. After the e-vowel one level above ⟦e⟧ palatals occur in examples like ⟦v̩ɑ̤ẹ\ExtraChars{ꭓ}tn̥⟧ ‘spruce’⟦bɑ̣ẹ\ExtraChars{ꭓ}⟧ ‘stomach-\textsc{pl}', but Malgersdorf also has the two irregular forms ⟦dlɑ̣ẹ\ExtraChars{ꭗ}⟧ ‘(to the) burial’ and ⟦p͓ẹ\ExtraChars{ꭗ}⟧ ‘misfortune’.

\subsection{Stage 2c{}'}\label{sec:13.3.4}

Stage 2c{}' is defined as any dialect in which front dorsals ([ç]) occur after all i-vowels and after all e-vowels, while back dorsals ([x]) surface after back vowels.

Stage 2c{}' is exemplified by Herrnsaal (=[38]), which possesses two i-vowels (⟦i \k{i}⟧) and three e-vowels (⟦ẹ e ę⟧). The dataset in \REF{ex:13:13} shows that palatals occur after every one of those front vowels in (\ref{ex:13:13a}--\ref{ex:13:13e}), while velars (=⟦\ExtraChars{ꭗ ꭘ x ꭖ}⟧) surface after back vowels (=\ref{ex:13:13f}--\ref{ex:13:13j}).

\ea%13
\label{ex:13:13}Stage 2c{}' for Herrnsaal (=[38]):
\ea\label{ex:13:13a} h\^{ę}i\ExtraChars{ꭓ}α \tab höher \tab ‘higher’ \tab 4: 123\\
    m\^{ę}i\ExtraChars{ꭓ} \tab Milch \tab ‘milk’ \tab 3: 10\\
    ɑ̣ov̩d̩ɑ̣l\^{ę}i\ExtraChars{ꭓ} \tab (auf die) Leich \tab ‘(to the) burial ’ \tab 4: 28
\ex\label{ex:13:13b} b͈v̩\={\k{i}}\ExtraChars{ꭓ}α \tab Vieh \tab ‘cattle’ \tab 3: 5
\ex\label{ex:13:13c} lẹ\ExtraChars{ꭓ}α \tab Löcher \tab ‘hole-\textsc{pl}’ \tab 7: 105\\
    blẹ\ExtraChars{ꭓ} \tab Blech \tab ‘tin’ \tab 4: 130\\
    b̤ʰẹ\ExtraChars{ꭓ} \tab Pech \tab ‘misfortune’ \tab 4: 39
\ex\label{ex:13:13d} v̩ɑ̣e\ExtraChars{ꭓ}t͐n̥ \tab Fichte \tab ‘spruce’ \tab 4: 125\\
    bɑ̣e\ExtraChars{ꭓ} \tab Bäuche \tab ‘stomach-\textsc{pl}' \tab 7: 82\\
    lɑ̣e\ExtraChars{ꭓ}α \tab leihen \tab ‘lend-\textsc{inf}’ \tab 4: 120\\
    ọe\ExtraChars{ꭓ}α \tab Eiche \tab ‘oak tree’ \tab 7: 57\\
    ọe\ExtraChars{ꭓ}αn \tab Eichen \tab ‘oak tree-\textsc{pl}’ \tab 7: 58
\ex\label{ex:13:13e} ɡnę\ExtraChars{ꭓ}t͓ \tab Knecht \tab ‘vassal’ \tab 4: 124\\
    rę\ExtraChars{ꭓ}α \tab Rechen \tab ‘rake’ \tab 3: 5
\ex\label{ex:13:13f} khu̩\ExtraChars{ꭗ}l̥ \tab Küche \tab ‘kitchen’ \tab 4: 165
\ex\label{ex:13:13g} rɑo\ExtraChars{ꭗ}α \tab rauchen \tab ‘smoke-\textsc{inf}’ \tab 4: 127\\
    bɑ̣ox \tab Bauch \tab ‘stomach’ \tab 7: 81\\
    b̩rɑ̣oxα \tab brauchen \tab ‘need-\textsc{inf}’ \tab 4: 128
\ex\label{ex:13:13h} lǫx \tab Loch \tab ‘hole’ \tab 7: 104\\
    wǫxα \tab Wochen \tab ‘week-\textsc{pl}’ \tab 7: 61\\
    b̩ǫx \tab Bach \tab ‘stream’ \tab 4: 33\\
    d̩ǫx \tab Dach \tab ‘roof’ \tab 4: 131
\ex\label{ex:13:13i} nɑ̣xt͓ \tab Nacht \tab ‘night’ \tab 7: 75\\
    nɑ̣xt͓ \tab Nächte \tab ‘night-\textsc{pl}’ \tab 7: 76\\
    d̩ɑ̣xα \tab Dächer \tab ‘roof-\textsc{pl}’ \tab 7: 113
\ex\label{ex:13:13j} mɑ̩xα \tab machen \tab ‘do-\textsc{inf}’ \tab 3: 78
\z
\z 

The Stage 2{}' rule of velar fronting needs to capture the fact that all and only front vowels -- the conjunction of i-vowels and e-vowels -- serve as triggers. This can be accomplished by positing that the trigger is a front ([coronal]) vowel \mbox{([--consonantal])}, as expressed in \REF{ex:13:14}:

\ea%14
\label{ex:13:14}\isi{Velar Fronting-13}\\
\begin{forest}
[,phantom
  [\avm{[−cons]} [\avm{[coronal]},name=target,tier=word]]
  [\avm{[−son\\+cont]},name=source [\avm{[dorsal]},tier=word]]
]
\draw [dashed] (source.south) -- (target.north);
\end{forest}
\z 

\isi{Velar Fronting-13} spreads the frontness feature from any front vowel ([--con\-so\-nan\-tal, coronal]) to a /x/ target. The featural system for Stage 2{}' dialects does not crucially require a particular analysis of e-vowels; hence, any of the matrices posited above work.

One might argue that the correct triggers for Herrnsaal (and presumably for any Stage 2c{}' dialect) is the class of front sonorants ([+sonorant, coronal]). Although that broader version of velar fronting works technically for the data in \REF{ex:13:13} there is good reason for questioning it. See \sectref{sec:13.5.2} for discussion.

The Stage 2c{}' system for Herrnsaal is very regular in the sense that back dorsals (⟦\ExtraChars{ꭗ ꭘ x ꭖ}⟧) are absent after both i-vowels and e-vowels. In other Stage 2c{}' systems it is possible to find an occasional example of a back dorsal in the front vowel context. This is precisely the case in Ruppertskirchen (=[152]), e.g. words like ⟦ed͈͈lɑ̣\k{i}\ExtraChars{ꭓ}⟧ ‘(to the) burial’, ⟦š̩d͈\k{i}\ExtraChars{ꭓ}⟧ ‘sting’, ⟦blẹ\ExtraChars{ꭓ}⟧ ‘tin’, ⟦be\ExtraChars{ꭓ}⟧ ‘misfortune’, ⟦ɡnę\ExtraChars{ꭓ}d͈ʰ⟧ ‘vassal’ display the regular pattern as in \REF{ex:13:13}, but there is the irregular form ⟦v\k{i}x⟧ ‘cattle’.

The regular Stage 2c{}' system in for Herrnsaal ([38]) also has an irregularity, namely there is one word with a palatal after the back vowel [u], i.e. ⟦bǫu\ExtraChars{ꭓ}α⟧ ‘beech tree’. Recall from \sectref{sec:13.3.2} that one occasionally finds velar fronting varieties with a palatal in the context after a back vowel but -- because of their rarity -- that no significance can be attributed to this type of anomaly.

\section{{Areal} {distribution} {of} {velar} {fronting} {stages} {in} {Lower} {Bavaria}}\label{sec:13.4}

Dialects representing the four stages described in the preceding section do not have an equal areal distribution in Lower Bavaria. As indicated in \tabref{tab:13.5}, Stage 2a represents by far the most common one. While Stage 2b is attested in a sizable number of places, Stage 2c{}' is extremely rare. It is interesting to observe that the most common stage in the dialects discussed in Chapters \ref{sec:3}--\ref{sec:12} -- Stage 2d from \tabref{tab:13.1} -- is not attested at all in Lower Bavaria.\footnote{{\tabref{tab:13.5} only lists two hundred thirteen places and not the two hundred twenty-one places depicted on \mapref{map:26}. The eight places which have not been taken into consideration ae Zell (=[5]), Teichnach (=[17]), Bischofsmais (=[47]), Ringelai (=[72]), Rottenmann (=[86]), Altreichenau (=[97]), Eging am See (=[114]), and Wegscheid (=[162]). The reason I ignore those places is that there are too few phonetic transcriptions with dorsal fricatives on the SNiB maps to know for certain whether or not velar fronting is active. For example, Wegscheid (=[162]) has several words with back dorsals after e-vowels, but no words are given for that place with dorsal fricatives after i-vowels. Wegscheid (=[162]) could therefore either represent velar fronting (Stage 2a) or non-velar fronting (Stage 1).}}

\begin{table}
\caption{\label{tab:13.5}Velar fronting and non-velar fronting places in Lower Bavaria. n=number of places for the corresponding stage.}
\begin{tabularx}{\textwidth}{lrQ}
\lsptoprule
Stage & $n$ & Places in Lower Bavaria\\\midrule
 1    & 16 & 9, 10, 33, 42, 45, 46, 52, 64, 119, 135, 161, 205, 211, 215, 220, 221\\
 2a   & 112 & 2, 3, 6, 7, 11, 13, 14, 16, 18, 19, 20, 21, 24, 26, 28, 29, 30, 31, 32, 34, 35, 40, 41, 43, 44, 48, 49, 50, 51, 53, 54, 57, 58, 60, 61, 62, 63, 65, 66, 67, 68, 69, 70, 71, 73, 74, 75, 76, 77, 79, 81, 82, 85, 88, 89, 90, 91, 92, 93, 94, 95, 98, 99, 100, 103, 104, 105, 106, 110, 112, 115, 116, 117, 118, 120, 129, 136, 138, 139, 140, 141, 144, 145, 146, 147, 148, 149, 150, 151, 153, 155, 156, 157, 158, 159, 160, 164, 176, 177, 178, 183, 190, 191, 192, 193, 196, 200, 203, 204, 207, 212, 213\\
 2b & 50 &  8, 22, 25, 36, 59, 80, 96, 109, 122, 123, 124, 125, 126, 128, 130, 131, 132, 134, 142, 163, 166, 167, 168, 169, 170, 171, 172, 173, 174, 179, 180, 182, 184, 185, 186, 187, 189, 194, 195, 198, 199, 201, 202, 206, 208, 209, 210, 217, 218, 219\\
2c' & 5 & 23, 37, 38, 152, 175\\
 2 & 30 & 1, 4, 12, 15, 27, 39, 55, 56, 78, 83, 84, 87, 101, 102, 107, 108, 111, 113, 121, 127, 133, 137, 143, 154, 165, 181, 188, 197, 214, 216\\
\lspbottomrule
\end{tabularx}
\end{table}

Thirty places in Lower Bavaria are categorized as Stage 2. That type of system has palatals after i-vowels, but there are two reasons why those thirty places cannot be unambiguously placed into any of the three velar fronting stages (2a, 2b, 2c{}'): (a) There are not enough examples to determine which of the three velar fronting stages is correct, or (b) there is too much fluctuation between front dorsals and back dorsals after e-vowels to distinguish between Stage 2b and 2c{}'. Haidlfing (=[108]) illustrates (a). That place has front dorsals in ⟦v̩\={\k{i}}\ExtraChars{ꭓ}ɑ⟧ ‘cattle’ and ⟦vɑ̣e̤\ExtraChars{ꭓ}tn̥⟧ ‘spruce’ and back dorsals after back vowels, but since there are no examples with dorsal fricatives after other e-vowels it is not possible to know whether or not Haidlfing represents Stage 2a, Stage 2b, or Stage 2c{}'. Peising (=[39]) illustrates (b). In that place palatals occur after i-vowels and after the fourth level of e-vowel (⟦ę⟧) and velars after back vowels. Those generalizations suggest that Peising represents Stage 2c'. The problem is that after e-vowels higher than ⟦ę⟧ velars occur in some words and palatals in others, e.g. some items have a front dorsal after ⟦ẹ⟧/⟦e⟧ while other ones have a back dorsal. It could be that the words with a back dorsal after e-vowels higher than ⟦ę⟧ are irregular, in which case Peising would represent Stage 2c'. On the other hand, it could be that Peising is a Stage 2b dialect and the two items with a palatal after ⟦ę⟧ are additional anomalies.

All of the places listed in \tabref{tab:13.5} are indicated on \mapref{map:27}. As noted above, there are five places representing the broadest set of velar fronting triggers (Stage 2c{}'), namely Baiersdorf (=[23]), Kelheim (=[37]), Herrnsaal (=[38]), Ruppertskirchen (=[152]), and Sachsenham (=[175]). The first three are in close proximity in the northwest, while the latter two are about 30km apart in the south central region. Most places representing Stage 2b can be found in the area south of the Danube River. Stage 2a places are most visible in the northeast, although they are also attested in the south and west. Non-velar fronting places are found in two areas: (a) In the south along the Inn River and (b) in the area between the Danube River and the border with Upper Palatinate and the Czech Republic. Outside of those two areas Stage 1 varieties are not attested in Lower Bavaria.

\begin{map}
% % % \includegraphics[width=\textwidth]{figures/VelarFrontingHall2021-img033.png}
\includegraphics[width=\textwidth]{figures/Map27_13.2.pdf}
\caption[Areal distribution of velar fronting stages in Lower Bavaria]{Areal distribution of velar fronting stages in Lower Bavaria. Circles indicate the absence of (postsonorant) velar fronting. Black squares show velar fronting after high front vowels, blue squares after nonlow front vowels, and red squares after high front vowels, mid front vowels, and low front vowels.}\label{map:27}
\end{map}

In the remainder of this section I interpret the places on \mapref{map:27} representing Stages 2a, 2b, and 2c{}' historically in the \isi{rule generalization} approach (\sectref{sec:2.4.1}). In the course of that discussion I refer only to the markers in Lower Bavaria, although a complete treatment would also have to take neighboring places in Upper Bavaria, Upper Palatinate, and \ipi{Upper Austria} into consideration. Since the data from those places are lacking I do not discuss them.

Stage 2c' has the broadest set of triggers; hence, the five places Baiersdorf (=[23]), Kelheim (=[37]), Herrnsaal (=[38]), Ruppertskirchen (=[152]), and Sachsenham (=[175]) can be thought of as focal areas for velar fronting. Baiersdorf (=[23]), Kelheim (=[37]), Herrnsaal (=[38]) are in close proximity; hence, I see them as a single \isi{focal area}, which I refer to as F\textsubscript{1}. Ruppertskirchen (=[152]) and Sachsenham (=[175]) could represent two separate focal areas or possibly a single one. I assume the latter for simplicity, which I call F\textsubscript{2}. The focal areas are those places where velar fronting was originally phonologized. I refer to the point in time when \isi{phonologization} occurred in F\textsubscript{1} and F\textsubscript{2} henceforth as T\textsubscript{1}. Phonologization began in F\textsubscript{1} and F\textsubscript{2} at T\textsubscript{1} with the narrowest set of triggers; hence, F\textsubscript{1} and F\textsubscript{2} were Stage 2a at T\textsubscript{1}. By contrast, all other places in Lower Bavaria -- that is, the blue and black markers depicted on \mapref{map:27} -- had no velar fronting (Stage 1) at T\textsubscript{1}.

Velar fronting then spread both temporally and spatially; recall that the two types of spreading are represented on \figref{fig:2.2}. Temporal spreading means that the places I call F\textsubscript{1} and F\textsubscript{2} which represented Stage 2a at T\textsubscript{1} added mid front vowels to the set of velar fronting triggers (at time T\textsubscript{2}), thereby becoming Stage 2b. Later on (at time T\textsubscript{3}), low front vowels were added to the set of velar fronting triggers in F\textsubscript{1} and F\textsubscript{2}, which is precisely the state of affairs represented by the red markers on \mapref{map:27}.

At a point in time after F\textsubscript{1} and F\textsubscript{2} had phonologized velar fronting (T\textsubscript{2}) the rule started to spread spatially. This means that communities near F\textsubscript{1} and F\textsubscript{2} phonologized the rule with the high front vowels as triggers (Stage 2a); these are the black markers depicted in the present day (T\textsubscript{3}) on \mapref{map:26}. Some of the Stage 2a places at T\textsubscript{2} eventually added mid front vowels to the set of velar fronting triggers and thereby became Stage 2b places; these are the blue markers at T\textsubscript{3} on \mapref{map:27}.

Why is Stage 2a so well-attested in Lower Bavaria but so rare elsewhere? Before addressing this question it is important to bear in mind is that the areas representing Stage 2a probably include regions outside of Lower Bavaria. For example, the linguistic atlas for Upper Bavaria (SOB) provides some evidence that the most common velar fronting variety in Lower Bavaria (Stage 2a) is also the norm in Upper Bavaria. Map 36 in Volume 2 for the word \textit{Vieh} ‘cattle’ shows the symbol for a palatal fricative in the context after a high front vowel throughout most of Upper Bavaria. By contrast, Map 2 in Volume 2 for \textit{Blech} ‘tin’ illustrates that the mid front vowel is followed by symbols for the back dorsal. (SOB has a three-way place distinction for dorsal fricatives as in \tabref{tab:13.3}). If these examples are representative, then they suggest that Stage 2a is even more widespread than suggested in this chapter.

The prevalence of Stage 2a throughout Lower Bavaria (and probably Upper Bavaria) and its rarity elsewhere make sense when one considers when velar fronting might have been phonologized. As stressed throughout this book, velar fronting must have had more than one point of origin (\isi{focal area}). Polygenesis is the only sensible explanation for the existence of velar fronting islands, which by definition phonologized velar fronting independently (\chapref{sec:15}).  In \chapref{sec:16} I argue on the basis of linguistic evidence that velar fronting must have been phonologized in WCG and WLG as early as \ili{OHG}/\ili{OSax}. Given the extreme age of velar fronting in LG and CG it makes sense that Stage 2a varieties would be rare in those areas because the original rule would have had many centuries to diffuse itself spatially and temporally. This meant that there was ample time to acquire the full set of triggers characterized by Stage 2d.

The reason Stage 2a is so common throughout Lower Bavaria and probably also Upper Bavaria is that velar fronting must have been phonologized in Southeast Germany much more recently than in CG and LG areas. Although it is not possible to give a precise century for the \isi{phonologization} of velar fronting in Bavaria, it must have been recent because of the prevalence of places which still represent Stage 2a.

\section{{Discussion}}\label{sec:13.5}

This section addresses three issues. First, a number of velar fronting dialects listed in \tabref{tab:13.5} only apply the rule after an i-vowel that is a monophthong but not after an i-vowel that is the second component of a diphthong (\sectref{sec:13.5.1}). Second, nothing has been said in this chapter about the status of consonants (e.g. /l/ and /r/) that serve as triggers for velar fronting in areas outside of Lower Bavaria. In \sectref{sec:13.5.2} I assess whether or not there is evidence from SNiB that bears on this question. Third, reference was made above to irregular forms (recall \sectref{sec:12.8.3} on LG).  In \sectref{sec:13.5.3} I address the nature of irregularities with respect to velar fronting in Lower Bavaria.

\subsection{Velar fronting in monophthongs and diphthongs}\label{sec:13.5.1}

The velar fronting places listed in \tabref{tab:13.5} have in common that the rule is always triggered by i-vowels. As indicated in the datasets presented in \sectref{sec:13.3} those i-vowels can be either monophthongs or the second component of a diphthong. A typical example is Stage 2a Zinzenzell (=[2]). In the data presented above for that place, velar fronting applies after i-vowels (⟦i \k{i}⟧) in monophthongs (e.g. ⟦š̩d͈î\ExtraChars{ꭓ}⟧ ‘sting’, ⟦dəs͈\k{i}\ExtraChars{ꭓ}⟧ ‘(the) colter’) and diphthongs (e.g. ⟦v̩e\k{i}\ExtraChars{ꭓ}α⟧ ‘cattle-\textsc{pl}’, ⟦khô\k{i}\ExtraChars{ꭓ}⟧ ‘lime’, ⟦wɑ̣\k{i}\ExtraChars{ꭓ}α⟧ ‘sanctify-\textsc{inf}’, ⟦bɑ̣\k{i}\ExtraChars{ꭓ}⟧ ‘stomach-\textsc{pl}’). That speakers in Zinzenzell do not draw a distinction between i-vowels in monophthongs and i-vowels in diphthongs makes sense if speakers treat the i-vowel in diphthongs phonologically the same way as the i-vowel in monophthongs. This is illustrated in \REF{ex:13:15}:

\ea%15
\label{ex:13:15}
\ea\label{ex:13:15a}\begin{tabular}[t]{@{}cc@{}}
       /i        &       x/\\
       \avm{[−cons\\+high\\coronal]} & \avm{[+cons\\−son\\+cont\\dorsal]}
     \end{tabular}
\ex\label{ex:13:15b}\begin{tabular}[t]{@{}ccc@{}}
       /ɑ        &     i      &         x/\\
       \avm{[−cons\\+low]} & \avm{[−cons\\+high\\coronal]} & \avm{[+cons\\−son\\+cont\\dorsal]}
     \end{tabular}
\z 
\z 

Since the i-vowel in \REF{ex:13:15a} as well as the i-vowel in \REF{ex:13:15b} are [+high] they both trigger velar fronting.

The pattern represented by Zinzenzell can be contrasted with another one. In particular, a number of velar fronting varieties included among the ones listed in \tabref{tab:13.5} are places where only i-vowels in monophthongs trigger velar fronting but not the i-vowels in diphthongs. A typical example is Schöllnach (=[90]). As indicated below, front dorsals occur after the i-vowels in monophthongs in \REF{ex:13:16a} but back dorsals are found after the i-vowels in diphthongs in \REF{ex:13:16b}. Note that velar fronting is not sensitive to vowel length because palatals occur in \REF{ex:13:16a} after vowels that are short, long, or extra short.

\ea%16
\label{ex:13:16}Data for Stage 2a Schöllnach (=[90])
\ea\label{ex:13:16a} š̩d̩î\ExtraChars{ꭓ} \tab Stich  \tab  ‘sting’ \tab 3: 4\\
    š̩d̩î\ExtraChars{ꭓ} \tab Stiche  \tab  ‘sting-\textsc{pl}’ \tab 7: 139\\
    vi\ExtraChars{ꭓ}d̩͐n̥ \tab Fichte  \tab  ‘spruce’ \tab 4: 125\\
    vī\ExtraChars{ꭓ} \tab Vieh  \tab  ‘cattle’ \tab 3: 5\\
    vī\ExtraChars{ꭓ} \tab Vieh  \tab  ‘cattle-\textsc{pl}’ \tab 3: 5\\
\ex\label{ex:13:16b} ẹd̩lɑ̣i\ExtraChars{ꭗ} \tab (in die) Leich \tab ‘(to the) burial’ \tab 4: 28\\
    kǫi\ExtraChars{ꭗ} \tab Kalk  \tab  ‘lime’ \tab 4: 80\\
    wɑ̣i\ExtraChars{ꭗ}α \tab weihen  \tab  ‘sanctify-\textsc{inf}’ \tab 4: 122\\
\z
\z 

Places in Lower Bavaria which display a pattern akin to the one in \REF{ex:13:16} are centered primarily in the north-central region bounded by the Danube (Donau), the border to the government district of Upper Palatinate, and the Czech Republic, namely  Rattiszell (=[13]), Brandten (=[18]), Rabenstein (=[19]), Zwiesel (=[20]), Lindberg (=[21]), Perasdorf (=[28]), Achslach (=[29]), Zachenberg (=[30]), Schwarzach (=[44]), Kirchberg im Wald (=[48]). The other five places are isolates situated along the Isar River (Aholming=[110], Mamming=[129]), the Ilz River (Büchlberg=[138]), the Inn River (Malching=[213]), and ca. 20km south of Regensburg (Aholming=[58]).

From the formal perspective, speakers from Schöllnach ([90]) treat i-vowels in monophthongs as phonologically [+high], but i-vowels in diphthongs as unmarked for that feature. The two types of i-vowels are depicted in \REF{ex:13:17}.

\ea%17
\label{ex:13:17}
\ea\label{ex:13:17a}\begin{tabular}[t]{@{}cc@{}}
    /i & x/\\
    \avm{[−cons\\+high\\coronal]} & \avm{[+cons\\−son\\+cont\\dorsal]}\\
    \end{tabular}
\ex\label{ex:13:17b}\begin{tabular}[t]{@{}ccc@{}}
    /ɑ & i & x/\\
    \avm{[−cons\\+low]} & \avm{[−cons\\coronal]} & \avm{[+cons\\−son\\+cont\\dorsal]}\\
    \end{tabular}
\z 
\z 

Given that treatment, velar fronting correctly produces a palatal from /x/ in \REF{ex:13:17a} but not in \REF{ex:13:17b}.

The reason why i-vowels in monophthongs are treated differently from the i-vowels in diphthongs is that speakers of these dialects are cognizant of the fact that the feature [high] is not distinctive for the second component of diphthongs. In the system of monophthongs for every German dialect without exception an i-vowel like /i/ must be [+high] to distinguish it from /e/, which is [--high]. But in diphthongs like /ɑi/ the /i/ is not distinctively [+high] because /ɑi/ does not contrast with /ɑe/, which is non-occurring in this variety. The approach described here therefore makes the prediction that the pattern displayed in \REF{ex:13:17} could not obtain in a dialect with both /ɑi/ and /ɑe/.\footnote{The featural approach for diphthongs in \REF{ex:13:17b} is consistent with the contrastive hierarchy of \citet{Dresher2009}, which has been presupposed throughout this book. A complete analysis of Schöllnach (=[90]) and the other places mentioned above would need to take all of the contrastive diphthongs into consideration. That type of analysis is not possible because SNiB does not provide enough data to know for certain which diphthongs are contrastive in which place in Lower Bavaria.}\largerpage

Independent evidence for the treatment proposed in \REF{ex:13:17b} comes from \il{Standard German}StG. According to one pronouncing dictionary \citep{Mangold2005} the second component of the three native diphthongs of \il{Standard German}StG is transcribed with the phonetic symbol for high vowels, i.e. /ai/, /au/, /ɔy/ (together with the bottom ligature). By contrast, another pronouncing dictionary \citep{Krech1982} transcribes the second component of the same diphthongs with the phonetic symbols for mid vowels, i.e. /ae/, /ao/, /ɔø/ (together with the bottom ligature). The reason those two sources differ in their transcriptions is precisely because vowel height is not distinctive -- or, to be colloquial, it does not matter -- for the second component of diphthongs. By contrast, \citet{Mangold2005} and \citet{Krech1982} both transcribe high monophthongs like /iː/ (but never as /eː/) because vowel height is distinctive for monophthongs, cf. [ziː] ‘they’ vs. [zeː] ‘lake’.\footnote{{The approach to distinctive features in diphthongs described here is defended in \citet{Noelliste2017} for \ipi{Ramsau am Dachstein}. See also the treatment of diphthongs like /ɑi/ in \il{Standard German}StG in \sectref{sec:16.2}.}}

\subsection{The status of velar fronting after consonants}\label{sec:13.5.2}

Recall that the default set of triggers for velar fronting in German dialects consists of all front vowels and coronal sonorant consonants (/r l n/), which is precisely the pattern attested in \il{Standard German}StG (\chapref{sec:17}). The datasets from SNiB in \sectref{sec:13.3} all have in common that the segments inducing velar fronting consist solely of front vowels. The question I explore below is whether or not there is material available from SNiB that can shed light on whether or not consonants might also be velar fronting triggers, as in Stage 2c and Stage 2d in \tabref{tab:13.1}.

It is difficult to test whether or not there are Stage 2c/2d places in Lower Bavaria for two reasons. First, none of the maps in SNiB has a word with /n/ followed by /x/, e.g. \textit{manchmal} ‘sometimes’ (cf. \il{Standard German}StG [mɑnçmɑl]). Second, in those maps in SNiB with liquids followed by /x/ the liquid stands in coda position and therefore undergoes \isi{Liquid Vocalization}, at least in the unmarked case.

SNiB provides extensive discussion of the places in Lower Bavaria where coda liquids do and do not vocalize, e.g. Map 140 in Volume 4. What is important for present purposes are words with a coda liquid followed by a dorsal fricative. A careful examination of the data accompanying those SNiB maps reveals that there are a few places in Lower Bavaria where coda liquids surface but do not undergo \isi{Liquid Vocalization}. In those places the dorsal fricative following [l] and [r] is realized as either a front dorsal (=⟦\ExtraChars{ꭓ ꭔ}⟧) or a back dorsal (=⟦\ExtraChars{ꭗ ꭘ x ꭖ}⟧). In \tabref{tab:13.6} I list the places in Lower Bavaria with the four attestations [lx], [lç], [rx], [rç].\largerpage

\begin{table}
\caption{\label{tab:13.6}Realization of /x/ as [x] or [ç] after [l]/[r]}
\begin{tabularx}{\textwidth}{lQll}
\lsptoprule
Pattern & Places in Lower Bavaria & Example & Map\\\midrule
/lx/→[lç] & 79 & ⟦muïl\ExtraChars{ꭓ}⟧ ‘milk’ & 3: 10\\
/lx/→[lx] & 24, 55, 78 & ⟦mïl̥x⟧ ‘milk’ & 3: 10\\
/rx/→[rç] & 37, 179 & ⟦v̩u̩r\ExtraChars{ꭓ}n̥⟧ ‘furrow-\textsc{pl}' & 3: 24 \\
/rx/→[rx] & 56, 76, 77, 78, 101, 102, 121, 124, 140, 142, 143, 145, 146, 147, 148, 159, 163, 165, 166, 167, 180, 
181, 182, 196, 197, 206, 207, 210, 219 & ⟦šnɑ̩rxα⟧ ‘snore-\textsc{inf}’ & 3: 86 \\
\lspbottomrule
\end{tabularx}
\end{table}

Consider the first two rows of \tabref{tab:13.6}.  The predominance of \isi{Liquid Vocalization} with /l/ as the target segment is reflected in the fact that only four places have realizations of words like \textit{Milch} ‘milk’ (or \textit{Kalk} ‘lime’) with a consonantal [l] followed by a dorsal fricative. Three of those places have a back dorsal ([x]) following that [l], while only one place has a palatal.

The maps in SNiB with [r] followed by a dorsal fricative point to a similar conclusion. The first observation is that there are only two words (\textit{schnarchen} ‘snore-\textsc{inf}’ and \textit{Furchen} ‘furrow-\textsc{pl}’) in which /r/ is realized as [r] followed by a dorsal fricative. The second observation is that the occurrence of a palatal after [r] is restricted to two places, but the pronunciation [rx] is robustly attested in twenty-nine places.

The areal distribution of the four types of liquid plus dorsal fricative sequence in \tabref{tab:13.6} is depicted on \mapref{map:28}.

\begin{map}
% % % \includegraphics[width=\textwidth]{figures/VelarFrontingHall2021-img034.png}
\includegraphics[width=\textwidth]{figures/Map28_13.3.pdf}
\caption[Areal distribution of {[rx]}, {[lx]}, {[rç]}, {[lç]} sequences in Lower Bavaria]{Areal distribution of [rx], [lx], [rç], [lç] sequences in Lower Bavaria. All of the places depicted above have some version of velar fronting after front vowels.}\label{map:28}
\end{map}

\mapref{map:28} shows that places with [rx] are clustered in the southwest with a small number of isolates to the east. The few places with [lx] are attested in the northwest. It is interesting to observe that there are more places in Lower Bavaria with [rx] than on \mapref{map:21}, which depicts those sequences throughout German-speaking countries.

The most significant generalization from \tabref{tab:13.6} is that if speakers do not vocalize /r/ in the context before /x/ then the default pattern is for the latter segment to be realized as a back dorsal ([x]) and not as a front dorsal ([ç]). This means that the set of triggers for velar fronting in those areas with [rx] must not include coronal sonorant consonants (specifically /r/). Those places include Stage 2a, Stage 2b, and Stage 2. The absence of places representing Stage 2c' from this list can be attributed to the rarity of Stage 2c'.

The occurrence of [l]/[r] followed by a palatal in \tabref{tab:13.6} suggest that the version of velar fronting presupposed for those places may in fact be broader than what was assumed. The case of Kelheim (=[37]) -- classified in \tabref{tab:13.5} as Stage 2c' -- is an interesting one. Assuming that the occurrence of [rç] in Kelheim is a regularity and not one of the irregular articulations referred above and assuming that [ç] also surfaces for those speakers after [l] and [n], then Kelheim represents the broadest set of triggers, namely Stage 2d. The case of Stage 2a Herrngiersdorf (=[79]) is intriguing as well. If the occurrence of [ç] after [l] (and [r], [n]) is regular in that place then the set of velar fronting triggers would consist of high front vowels or coronal sonorant consonants. Although that historical stage is absent from \tabref{tab:13.6}, it was discussed in \sectref{sec:12.7.1}. That section investigated the status of several unattested velar fronting Trigger Types, including one consisting of high front vowels and coronal sonorant consonants (Trigger Type W in \tabref{tab:12.32}). Significantly, it was argued in that section that nothing in the present analysis speaks against that type of conjunction; hence, Herrngiersdorf potentially fills an accidental gap.\footnote{{The third place listed in \tabref{tab:13.6} with a liquid followed by a palatal is Haunwang (=[179]). Since that place is classified in \tabref{tab:13.6} as the indeterminate Stage 2 one should not even attempt to speculate on what the realization of /rx/ as [rç] means for the reinterpretation of Haunwang’s historical stage.}}

A final issue to consider is the realization of dorsal fricatives after vocalized liquids. The datasets from \sectref{sec:13.3} for words containing coda /l/ and coda /r/ reveal the predominant pattern discussed above: \isi{Liquid Vocalization} produces a front vowel from /l/, and those derived front vowels count as velar fronting triggers. This pattern is reflected in a pronunciation such as [kɑiç] for \textit{Kalk}. By contrast, if /r/ undergoes \isi{Liquid Vocalization} then the following /x/ is realized as [x] because the vocalized-r is not a front sound. This generalization can be seen in a pronunciation such as [duɑx] for \textit{durch}. Put differently, realizations like [kɑiç] and [duɑx] illustrate transparent outputs without any trace of \isi{opacity}. In the former word \isi{Liquid Vocalization} \isi{feeds} velar fronting and in the latter velar fronting and \isi{Liquid Vocalization} do not interact, i.e. they are unordered in a rule-based model.

The data accompanying the SNiB maps for words with a vocalized-r followed by dorsal fricatives reveal two patterns: One with a back dorsal (=⟦\ExtraChars{ꭗ ꭘ x ꭖ}⟧) and the other with a front dorsal (=⟦\ExtraChars{ꭓ ꭔ}⟧). Representative examples are given in (\ref{ex:13:18a}--\ref{ex:13:18d}) for the word \textit{Kirche} ‘church’ from the data accompanying Map 13 in Volume 3. The back dorsal realizations in (\ref{ex:13:18a}, \ref{ex:13:18b}) can be compared with the front dorsal pronunciation in (\ref{ex:13:18c}, \ref{ex:13:18d}). As noted earlier in \sectref{sec:13.2.3}, the narrow phonetic transcriptions in SNiB express the vocalized-r either as ⟦ə⟧ or ⟦α⟧. All four of the places indicated below possess some version of velar fronting.

\ea%18
\label{ex:13:18}
\ea\label{ex:13:18a}k͓h\k{i}α\ExtraChars{ꭗ}α \tab Hofendorf (=[102])
\ex\label{ex:13:18b}khīə\ExtraChars{ꭗ}α     \tab Schwarzach ([44])
\ex\label{ex:13:18c}k͓hîα\ExtraChars{ꭓ}α     \tab  Kelheim (=[37])
\ex\label{ex:13:18d}kʰiə\ExtraChars{ꭓ}α     \tab Schöllnach (=[90])
\z 
\z 

Of the realizations in \REF{ex:13:18} the ones with the back dorsals in (\ref{ex:13:18a}, \ref{ex:13:18b}) are much more common than the ones with front dorsals in (\ref{ex:13:18c}, \ref{ex:13:18d}). For example, the data for Map 13 in Volume 3 indicate sixty-nine places with ⟦α⟧ realizations followed by a back dorsal as in \REF{ex:13:18a}, eighteen realizations with \isi{schwa} (⟦ə⟧) followed by a back dorsal as in \REF{ex:13:18b} but only ten with palatal realizations after \isi{schwa} as in \REF{ex:13:18d} and 7 with the palatal pronunciation after ⟦α⟧ as in \REF{ex:13:18c}.

Although not particularly common, the palatals in (\ref{ex:13:18c}, \ref{ex:13:18d}) nevertheless deserve comment.  Consider first the pronunciation in \REF{ex:13:18d} in which the vocalized-r surfaces as \isi{schwa}. I argue that the palatal in that context is synchronically derived from the i-vowel preceding \isi{schwa}. The treatment endorsed here derives support from other varieties of German discussed in previous chapters. For example, in \sectref{sec:3.4} it was demonstrated that dorsal fricatives surface in \ipi{Rheintal} as palatal after a nonlow front vowel even if a \isi{schwa} intervenes between that sound and the target. Thus, /iəx/ surfaces as [iəç], but /x/ after a \isi{schwa} preceded by either a low front vowel or a back vowel surfaces as velar, i.e. [uəx].

To account for \REF{ex:13:18d}, I argue that \isi{schwa} inherits the frontness feature ([coronal]) from a preceding front vowel, which then spreads to /x/ and creates [ç]. The rule referred to here (from \sectref{sec:3.4}) is stated in \REF{ex:13:19}:

\ea%19
\label{ex:13:19}\isi{Schwa Fronting-1}:\\
\begin{forest}
[,phantom
  [\avm{[−cons]} [\avm{[coronal]},name=target]]
  [\avm{[−cons\\+son]},name=source]
]
\draw [dashed] (source.south) -- (target.north);
\end{forest}
\z 

According to the treatment proposed here \isi{r-Vocalization} \isi{feeds} \is{Schwa Fronting-1}Schwa Front\-ing-1, which in turn \isi{feeds} velar fronting. This can be expressed with a diacritic indicating frontness: /irx/→{\textbar}iəx{\textbar}→{\textbar}iə̟x{\textbar}→[iə̟çt]. Since back vowels like /u/ do not bear the frontness feature they do not undergo \isi{Schwa Fronting-1}; the /x/ in an underlying representation like /urx/ surfaces without change: /urx/ → {\textbar}uəx{\textbar} → [uəx].\footnote{{As stated in \REF{ex:13:19} the frontness feature spreads from any front vowel. It was noted earlier in \sectref{sec:3.4} that it may be necessary to restrict those segments further, e.g. only nonlow front vowels or high front vowels.} }

The analysis proposed here derives support from additional maps in SNiB which depict the modern realization of historical /x/ after \isi{schwa}. Of particular relevance are words containing a back vowel plus the \isi{schwa} realization of /r/ followed by /x/, e.g. in Map 24 in Volume 3 for \textit{Furchen} ‘furrow-\textsc{pl}’. The data accompanying that map reveal no realization at all with a palatal preceded by [u] plus \isi{schwa} ([uəç]), while three places are attested with the back dorsal ([uəx]). (The most common realization for \textit{Furchen} contains the sequence [uɑx]).

The treatment proposed above presupposes that the Lower Bavarian rule of \isi{r-Vocalization} differs from the same process in other dialects (and in \il{Standard German}StG) in the sense that the output is a placeless segment (\isi{schwa}). It was assumed in preceding chapters that \isi{r-Vocalization} for an input [dorsal] rhotic (i.e. /ʀ/) simply changes [+consonantal] into [--consonantal]; thus, the [dorsal] input /ʀ/ is [dorsal] in the output ([ɐ]). If the input is [coronal] (i.e. /r/) then \isi{r-Vocalization} creates a [dorsal] output (i.e. [ɐ]); hence, \isi{r-Vocalization} changes [+consonantal, coronal] into [--consonantal, dorsal]. In dialects like the one in Lower Bavaria with a more general process of \isi{Liquid Vocalization}, a target /l/ surfaces as [--consonantal, coronal].\footnote{{To express the fact that /l/ surfaces as an i-vowel in many places in Lower Bavaria, \isi{Liquid Vocalization} must ensure that /l/ surfaces as [+high]. I do not pursue this analysis further because it is not directly relevant to the question discussed above.}}

Since \isi{Liquid Vocalization} in Lower Bavaria can produce \isi{schwa} ([ə]) for a target /r/, that rule changes [+consonantal] into [--consonantal] and deletes all place features. Thus, if /r/ is [coronal], that feature is deleted in coda position, thereby producing placeless \isi{schwa}. This is an important assumption because only placeless sounds constitute the input to \isi{Schwa Fronting-1}.

Consider now the occurrence of the palatal after the ⟦α⟧ realization of /r/, as in \REF{ex:13:18c}. As noted above, of the four patterns in \REF{ex:13:18}, the one with ⟦α⟧ in \REF{ex:13:18c} is the least common. The same generalization obtains when one considers other maps. For example, Map 3: 24 for \textit{Furchen} ‘furrow-\textsc{pl}’ has no attestations at all of a palatal followed by the ⟦α⟧ realization of the vocalized-r.

I argue that the sound transcribed as ⟦α⟧ (for the coda realization of /r/) is simply a low-level variant of ⟦ə⟧. This means that the Lower Bavarian version of \isi{Liquid Vocalization} changes a target /r/ sound into a [--consonantal] sound that lacks place features. That treatment implies that the palatal in \REF{ex:13:18c} is an underlying velar (/x/), which inherits the frontness feature ([coronal]) from the preceding vowel ⟦α⟧, which in turn receives that frontness feature from the preceding i-vowel by \isi{Schwa Fronting-1}. The difference between the two low-level variants of the vocalized-r -- ⟦α⟧ and ⟦ə⟧ -- lies outside the domain of phonology and can therefore only be understood by taking phonetics into consideration.\footnote{{In \il{Standard German}StG and in many of the dialects discussed in earlier chapters palatals occur in the context after the vocalized-r, e.g. [fʏɐçtən] ‘fear-}\textrm{\textsc{inf}}\textrm{’; see \chapref{sec:17}. In \chapref{sec:7} it was shown that the palatals in this context are underlying palatals in the synchronic phonology (/ç/) because they cannot be derived by any version of velar fronting. Pronunciations like [fʏɐçtən] are historically opaque because the trigger for the once allophonic rule of velar fronting (i.e. /r/) is no longer present in the phonetic representation. One cannot argue that the palatal in \REF{ex:13:18c} is an underlying palatal as in \il{Standard German}StG because this would imply that the /r/ was once a trigger for velar fronting.}}

\subsection{Irregularities}\label{sec:13.5.3}

Reference was made throughout this chapter to irregular forms. For example, in a Stage 2a system an irregularity would be either (A) an unexpected [x] after an i-vowel, or (B) an unexpected [ç] after anything other than an i-vowel. Both cases need to be assessed, especially in light of the claim I have made throughout this book that the historical rule of velar fronting was a classic \isi{Neogrammarian change} that regularly affected every target velar and that the synchronic reflex of velar fronting operates as an exceptionless rule.

In \sectref{sec:12.8.3} I discussed the status of irregularities like the ones in (A) and (B) for a different set of dialects (LG) in a very different area (North Germany). One takeaway from that section is that irregularities do not fit the textbook example of \isi{lexical exceptions}. Recall that the \ili{English} word \textit{obesity} is a true \isi{lexical exception} to the rule of \is{Trisyllabic Laxing (English)}Trisyllabic Laxing, which applies in words like \textit{sincerity}. The reason the LG irregularities are very different from an \ili{English} word like \textit{obesity} is that LG speakers fluctuate between the irregular (unexpected) pronunciation and the regular (expected) pronunciation of the same word. It is possible to draw this conclusion from observations of several different linguists describing those LG dialects as well as from phonetically transcribed texts from a single speaker. In \sectref{sec:12.8.3} I conjectured that the irregular forms in LG are tokens from neighboring dialects that are adopted by speakers having contact with speakers of those other dialects.

I claim that the same explanation holds for the irregularities present in Lower Bavaria belonging to both category (A) and category (B). The data from Lower Bavaria discussed in this section have been drawn from a linguistic atlas; hence, it is difficult to observe the kind of fluctuation referred to in the preceding paragraph between the irregular and the regular pronunciation of any given word. However, it is important to stress that not a single variety of German has been discovered in this book with true \isi{lexical exceptions} to velar fronting. Given this finding it would be surprising to find true \isi{lexical exceptions} to velar fronting in the material discussed in the present chapter.

\section{{Conclusion}}\label{sec:13.6}

The aim of this chapter has been to assess the state of velar fronting in Lower Bavaria on the basis of data drawn from a linguistic atlas (SNiB). It has been shown that over 200 places in Lower Bavaria reflect three of the historical stages for velar fronting defined according to Trigger Types which were posited in \chapref{sec:12}: Velar fronting after high front vowels (Stage 2a), after nonlow front vowels (Stage 2b), and after all front vowels (Stage 2c'). The data discussed above demonstrate that Stage 2a places constitute the majority pattern while Stage 2c' places reflect the rarest one. The areal distribution of towns belonging to the three velar fronting stages was interpreted historically on the basis of the \isi{rule generalization} model.

The discussion of targets and triggers for velar fronting is continued in \chapref{sec:14}: In that chapter I discuss a number of dialects where the triggers for velar fronting include not only front (coronal) segments, but also back sounds, such as back vowels.\il{Central Bavarian|)}
