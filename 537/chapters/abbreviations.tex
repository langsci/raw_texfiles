\addchap{\lsAbbreviationsTitle}\label{abbreviations}
% \addchap{Abbreviations and symbols}
%The category labels for abbreviations follow the Leipzig Glossing Rules.\footnote{\url{http://www.eva.mpg.de/lingua/resources/glossing-rules.php}} Additional category labels are listed below.
%\vspace{.5cm}

\begin{tabularx}{.45\textwidth}{lQ}
    \textsc{1} & first person\\
    \textsc{2} & second person\\
	\textsc{3} & third person\\
    \textsc{adju} & adjunct\\
    \textsc{aux} & auxiliary\\
    \textsc{bkg} & backgrounding\\
    \textsc{caus} & causative\\
    \textsc{ch} & change of state\\
    \textsc{cntrfact} & contrafactual\\
    \textsc{comit} & comitative\\
    \textsc{compl} & copular complement\\
    \textsc{cont} & continuative\\
    \textsc{cq} & content question\\
    \textsc{decl} & declarative\\
    \textsc{dem} & demonstrative\\
    \textsc{dirq} & direct quote\\
    \textsc{dist} & distal\\
    \textsc{distr} & distributive\\
    \textsc{dup} & reduplication\\
    \textsc{e} & eyewitness\\
    \textsc{ex} & exclusive\\
    \textsc{f} & feminine agreement\\
    \textsc{(f)} & feminine inherent gender\\
    \textsc{fp} & far past\\
    \textsc{fpl} & feminine and plural\\
    \textsc{fut} & future\\
    \textsc{hab} & habitual\\
    \textsc{hypoth} & hypothetical\\
    \textsc{immed} & immediate\\
    \textsc{imp} & imperative\\
    \textsc{in} & inclusive\\
    \textsc{int} & intentive\\
\end{tabularx}
\begin{tabularx}{.45\textwidth}{lQ}
    \textsc{ip} & immediate past\\
    \textsc{irr} & irrealis\\
    \textsc{list} & list construction\\
    \textsc{loc} & locative\\
    \textsc{m} & masculine agreement\\
    \textsc{(m)} & masculine inherent gender\\
    \textsc{mc} & main clause\\
    \textsc{med} & medial clause\\
    \textsc{mood} & mood\\
    \textsc{n} & non-eyewitness\\
    \textsc{neg} & negative\\
    \textsc{nfin} & non-finite\\
    \textsc{nom} & nominative\\
    \textsc{npq} & noun phrase question\\
    \textsc{o} & object\\
    \textsc{oc} & O-construction marker\\
    \textsc{pfut} & past in future\\
    \textsc{pfc} & postposed finite clause\\
    \textsc{pl} & plural\\
    \textsc{poss} & possessor/possessor marker\\
    \textsc{recip} & reciprocal\\
    \textsc{refl} & reflexive\\
    \textsc{rel} & relative clause\\
    \textsc{rep} & reported\\
    \textsc{rp} & recent past\\
    \textsc{s} & subject\\
    \textsc{sec} & secondary verb\\
    \textsc{sg} & singular\\
    sp & species\\
    \textsc{super} & superlative\\
    \textsc{voc} & vocative\\
    &\\
\end{tabularx}
