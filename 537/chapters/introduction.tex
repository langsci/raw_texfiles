\section{The Jarawaras}\label{sec:jarawaras}

The Jarawaras are a small indigenous people (pop. about 240) inhabiting three main villages in the Terra Indígena Jarawara/Jamamadi/Kanamanti in the municipalities of Lábrea and Tapauá, in Amazonas state, Brazil.\footnote{This volume has benefitted from comments by Paul Unger, Janet Allen, and two anonymous reviewers; and Christian Döhler's generous help as editor is reflected throughout the volume.}\footnote{For a map of the reserve, see \url{https://terrasindigenas.org.br/pt-br/terras-indigenas/3946}.
} The Jarawaras share this reserve with the Jamamadis, a closely related group. Jarawara and Jamamadi have been classified as dialects of a single language, along with Banawá, which is spoken in an adjacent reserve. While the three groups accept these names for the dialects they speak, there is no generally accepted name for the language. \citet{Dixon:2004xr} has suggested the name Madi, and this is the name used in the Glottolog. Madi is one of a handful of languages that compose the Arawá language family, the other living languages being Kulina, Deni, Paumari, and Suruwahá, (\cite{Dienst:2014rt})\footnote{Dienst also lists Western Jamamadi, which is most closely related to Kulina and Deni (what I have referred to as Jamamadi he calls Eastern Jamamadi). No linguistic studies of Western Jamamadi have been published, and I do not know whether the language is still used as a regular means of communication in any community.} and probably Hi-Merimã (\cite{Shiratori:2021uy}), for which no linguistic studies have been done; it is spoken by a very small group that has no regular contact with outsiders.

My first contact with the Jarawaras was in 1984, and starting in 1987 my wife and I were assigned to SIL's Jarawara project. At that time Siko and Yowao, whose stories are the subject of this volume, were among the oldest living Jarawaras, and both were living in Casa Nova village where we also lived. Both men continued to live in Casa Nova village until their respective deaths in 1994 and 2002. Siko and Yowao were first cousins, since their fathers were brothers. Just about everyone in Casa Nova village is a descendant of Yowao or married to a descendant, but Siko does not have any descendants in Casa Nova village, although he does have descendants in other Jarawara villages.

Jarawara is a threatened language, mainly because of the small population. But the use of Jarawara by the people is not decreasing. Although all Jarawaras also speak Portuguese, they use their language exclusively among themselves. Furthermore, they are not ashamed to use it in the presence of non-indigenous Brazilians.\footnote{Throughout this volume I use the shortened term ``Brazilian(s)'' to refer to the non-indigenous members of the national Brazilian society. The Jarawaras call them \textit{yaras} in Jarawara or \textit{brancos} (`whites/white people') in Portuguese.}

The area inhabited by Arawan-speaking ethnic groups is lowland southwestern Amazonia, specifically the basins of the Purus and Juruá Rivers, southern tributaries of the Amazon. The Jarawaras live in the area of the middle Purus River, near the main river on the western side. One of the Jarawara villages is opposite the town of Lábrea, but Casa Nova village is further upstream. The seasonal variation is quite marked in this region, because of the seasonal flooding in the rainy season from approximately December to May, and this affects travel a lot. In the rainy season travel is much easier, because canoes and motor boats can be used more easily. In the dry season travellers must walk much further on trails. There are no roads in the area inhabited by the Jarawaras. In fact, there are no roads on the western side of the Purus in the whole area. The Trans-Amazon Highway ends at Lábrea, on the eastern side of the Purus.

\section{Organization of the stories}\label{sec:organization}

The Jarawaras love to tell and listen to stories, and this volume consists of a selection of the stories of two of their elderly men, Siko and Yowao, both of whom are now deceased. All of the stories, along with many others, have previously been published online as (\cite{Vogel2022sa}) for Siko's stories and (\cite{Vogel2023yo}) for Yowao's stories. The selection in this volume is being published in the hopes that it will stimulate interest in this rich heritage of the Jarawaras.

I have organized the stories roughly by the time period in which the events occurred. Jarawara history can be divided into three periods. In the most remote period, they did not yet have contact with Brazilians, and they had battles with neighboring tribes. They did not know horticulture, but relied on hunting and gathering. They hunted with poisoned arrows and blowguns. They lived in longhouses. They wore breechcloths and pierced their bodies and used many feathered decorations.

In an intermediate period, the Jarawaras began to practice horticulture, planting manioc and other crops. They did not yet have the large iron pans that are used nowadays to make \textit{farinha} from manioc. They began to have contact, sometimes violent, with Brazilians. They began to adopt many of the ways of the Brazilians: houses made of split palm, rifles, clothing, and metal tools. They began to use hammocks bought from the Brazilians rather than their traditional hammocks made from native cotton or inner bark. It seems that during this period the Jarawaras began to depend more on fishing than on hunting. Siko and Yowao's parents were alive during part of this period.

Siko and Yowao were probably born in the 1920s. In their lifetimes, they saw the making of \textit{farinha} adopted, and rifles substituted by shotguns. Hunting with bow and arrow and blowguns ceased to be practiced. Whereas probably all men had formerly been trained as shamans, and both Siko and Yowao were shamans, the training of new shamans ceased some time before I began to have contact with the Jarawaras in the 1980s.

For more information on the culture and history of contact of the Jarawaras and the closely related Jamamadis and Banawás, the reader is directed to \citet{Maizza:2012qe} for the Jarawaras, \citet{Shiratori:2018kn} for the Jamamadis, and \citet{Aparicio:2019rm} for the Banawás.

Jarawaras can have several names, which may be used by different people, and these can change over time as well. As a minimum, Jarawaras typically have both a Jarawara name and a Portuguese name. The name Siko is from Portuguese Chico. Siko's Jarawara name was Bai Abono, which means `spirit of the thunder/sun', but people in the village always called him Siko, so that is what I have used here. Similarly, Yowao is from Portuguese João. Yowao's Jarawara name was Kana Abono, which means `spirit of sugar cane', but other Jarawaras typically called him Yowao.

In selecting the stories, I have tried to include stories that illustrate common themes. Shamanism and spirits are common themes. Closely related to these are stories about monsters, and those about people who turn into animals. In the Jarawara worldview, people came into existence first, and then the various species of animals appeared. Another common theme is jaguars, which is understandable since one of the things the Jarawaras fear the most is meeting up with one in the forest. The stories also illustrate some aspects of Jarawara social organization, such as polygyny consisting of a man marrying two sisters. Both Siko and Yowao tell stories about a character called Bahi (`Sun'), and these stories explain the origin of the only traditional ritual which is still practiced today, the coming out ritual for pubescent girls.

Although Siko and Yowao told versions of some of the same stories, their repertories were mostly different. Furthermore, each had his own distinctive style. Siko tended more often than Yowao to tell humorous stories, and his use of sound effects was especially effective. Yowao's stories tended much more to focus on spirits and shamanism. Apparently Siko got a good number of his stories from his wife Amoro, who was from a closely related group, the Wa Yafis, that no longer exists.

\tabref{tab:texts} provides an overview of the texts included in this collection.

\begin{xltabular}{.8\textwidth}{lXr}
    \caption{The texts in this collection.} \label{tab:texts}\\
    \lsptoprule
        \textsc{text} & \textsc{title} & \textsc{words}\\
    \midrule
    \endfirsthead
    \multicolumn{3}{c}%
    {\tablename\ \thetable{} -- continued from previous page}\\
    \lsptoprule
        \textsc{text} & \textsc{title} & \textsc{words}\\
	\midrule
    \endhead
    \hline \multicolumn{3}{r}{{Continued on next page}}\\
    \endfoot   
    \endlastfoot

		\ref{text:01}&\textit{Bahi} (Yowao)&531\\
        \ref{text:02}&\textit{Bahi} (Siko)&2,020\\
        \ref{text:03}&\textit{The first corn}&991\\
        \ref{text:04}&\textit{The old man who went up in the sky}&663\\
        \ref{text:05}&\textit{The hungry old man}&986\\    
        \ref{text:06}&\textit{The man who got away from the Kose Mati}&1,121\\    
        \ref{text:07}&\textit{The caimans killed the man}&1,723\\    
        \ref{text:08}&\textit{Siraba}&1,070\\    
        \ref{text:09}&\textit{The jaguars ate the girls}&1,315\\    
        \ref{text:10}&\textit{The toto spirit}&1,515\\    
        \ref{text:11}&\textit{Casting spells on each other}&817\\    
        \ref{text:12}&\textit{Maiko}&1,478\\    
        \ref{text:13}&\textit{The Yimas}&1,100\\    
        \ref{text:14}&\textit{Mayawari}&428\\    
        \ref{text:15}&\textit{Kamabira}&430\\    
        \ref{text:16}&\textit{The poison root song}&1,013\\    
        \ref{text:17}&\textit{My father fought a cougar}&396\\    
        \ref{text:18}&\textit{My father and his familiar spirit}&395\\    
        \ref{text:19}&\textit{A pair of monsters}&567\\    
        \ref{text:20}&\textit{White-lipped peccaries and a jaguar}&457\\    
    \midrule    
        \textbf{total}&&\textbf{19,016}\\
    \lspbottomrule
\end{xltabular}

\section{The interlinear presentation}\label{sec:interlinearpres}

Before the interlinear presentation for each story, I have provided an introductory note, which includes a short summary of the plot of the story. This is followed by a paragraph translation, which is less literal than the sentence-by-sentence transation included in the interlinear presentation. The paragraph translation eliminates some repetition, and it conforms more to the conventions of English discourse, for example with respect to choice of pronoun or proper name for person reference.

The texts are not edited, except that comments by members of the audience are usually excluded. Also, occasionally the storytellers gave explanations in Portuguese, and these are not interlinearized. False starts and other kinds of errors are included; I have noted some, but not by any means all, of these in footnotes.

There is a temptation for the researcher to exclude or change parts of a text that they don't understand or that don't fit their ideas about the grammar of the language, so in not editing the texts I have sought to make them an authentic representation of how the Jarawaras talk, within the restrictions of the genre of storytelling. This is also why I have made the unedited recordings available, at the websites mentioned above.

The first tier of the interlinear presentation, the text tier, is orthographic. The orthography of Jarawara is mostly phonemic, with the major exception that long vowels are unrepresented;\footnote{For more information on the Jarawara orthography and pronunciation, see the introduction to my Jarawara-English dictionary (\cite{Vogel:2016yw}).} but long vowels are represented in the morpheme tier. Each numbered line represents a sentence in Jarawara, which in many cases does not correspond to a single sentence in the English translation. For information on how a Jarawara sentence may be defined, see \citet{Vogel:2022du}.

The second tier represents underlying forms of morphemes, and there is further discussion of these below. In this tier, long vowels are represented as double vowels. The third tier is glosses, including grammatical information, and the fourth tier is a free translation that is more literal than the paragraph translation. A key to the abbreviations is included on page \pageref{abbreviations} above.

Jarawara is rich in morphophonological processes, and they may be divided into two types: those that are automatic, and those that express grammatical meaning. These two types are handled differently in the interlinear presentation. Examples of the first type are the rules involving the first syllable of tense-modal suffixes, which has the form \textit{-hV}. For the intentive suffix, for example, the form of this syllable is \textit{-ha} for the feminine form \textit{-}\textit{\textbf{ha}}\textit{bone} and \textit{-hi} for the masculine form \textit{-}\textit{\textbf{hi}}\textit{bona}. In a context such as that in (\ref{ex:intro1}), there is no morphophonological change.

\ea\label{ex:intro1}
    tafahabone\\
    \gll tafa-habone\\
     eat-\textsc{int.f}\\
    \glt `She will eat.'
\z

However, there is a change involved when the form is masculine, as in (\ref{ex:intro2}), to agree with the masculine subject. The \textit{-hV} syllable causes \textit{tafa} to be raised to \textit{taf\textbf{e}}.

\ea\label{ex:intro2}
    tafehibona\\
    \gll tafa-hibona\\
    eat-\textsc{int.m}\\
    \glt `He will eat.'
\z

In a case such as this, I have chosen to maintain the underlying form \textit{taf\textbf{a}} in the morpheme tier because this is an automatic change; but I have shown the form that is actually pronounced in the text tier.

Another change involving the \textit{-hV} syllable occurs when the preceding number of moras in the word is odd: in this case, the \textit{-hV} syllable is elided, as in (\ref{ex:intro3}).

\ea\label{ex:intro3}
    otafabone\\
    \gll o-tafa-habone\\
    1\textsc{sg.s}-eat-\textsc{int.f}\\
    \glt `I will eat.'
\z

Again, I maintain the underlying form with the \textit{-hV} syllable in the morpheme tier, but show the pronounced form in the text tier.

There are, however, other morphophonological changes that are not automatic, but instead express grammatical information. For example, a particular kind of nonfinite verb form is expressed phonologically by a change in the final vowel of a verb stem from \textit{a} to \textit{i}. As is very common for verbs in Jarawara, the verb \textit{tafa} `eat' ends with \textit{a}, as we have seen above. The nonfinite form of this verb is \textit{taf\textbf{i}}, and this is the form used on both the text tier and the morpheme tier, as in (\ref{ex:intro4}).

\ea\label{ex:intro4}
    tafi\\
    \gll tafi\\
    eat.\textsc{nfin}\\
    \glt `her/his eating'
\z

Similarly, the final \textit{a} of a verb stem, when it is the last vowel of the phonological word (and the verb has covert tense, cf. \citet{Vogel:2009ug,Vogel:2022du}), has an inflection that consists in raising to \textit{e} for masculine, and no change for feminine, as in (\ref{ex:intro5}) and (\ref{ex:intro6}).

\ea\label{ex:intro5}
    tafe\\
    \gll tafe\\
    eat.\textsc{m}\\
    \glt `He ate.'
\z

\ea\label{ex:intro6}
    tafa\\
    \gll tafa\\
    eat.\textsc{f}\\
    \glt `She ate.'
\z

 When the last vowel of the verb stem is not \textit{a}, then \textit{ha} is added for the feminine (\ref{ex:intro7}), and (optionally) \textit{hi} for the masculine (\ref{ex:intro8}).

\ea\label{ex:intro7}
    yaboha\\
    \gll yaboha\\
    be\_long.\textsc{f}\\
    \glt `It was a long time.'
\z

\ea\label{ex:intro8}
    yabohi\\
    \gll yabohi\\
    be\_long.\textsc{m}\\
    \glt `It was a long time.'
\z

Readers who compare these texts to the versions I have published previously will notice many differences in the interlinear presentation. I have made a good number of updates, to be sure, but the changes that are systematic do not represent changes in my analysis of Jarawara, but reflect my efforts to work within the constraints of the Leipzig Glossing Rules for this volume, and to otherwise simplify the presentation.

In particular, there are many contexts in Jarawara in which the auxiliary \textit{na} is dropped, and some in which the auxiliary \textit{ha} is dropped as well. In previous publications I have shown the deleted auxiliaries in the morpheme tier, even though they do not appear on the surface. In this volume, in order to make the interlinear presentation less abstract, I have only included the auxiliaries if they are actually pronounced. As a result, some words look like they consist entirely of prefixes and suffixes, since the auxiliary that the prefixes and suffixes attach to is absent. For example, in (\ref{ex:intro9}), the word \textit{tibana} consists of a prefix followed by a suffix.

\ea\label{ex:intro9}
    amo tibana?\\
    \gll amo ti-bana\\
    sleep 2\textsc{sg.s}-\textsc{fut}\\
    \glt `Are you going to sleep?'
\z

When we compare this sentence with the equivalent with a third person subject (\ref{ex:intro10}), it becomes evident that the auxiliary has been deleted in (\ref{ex:intro9}). We could, if we wanted to, have \textit{ti-na-bana} in the morpheme tier in (\ref{ex:intro9}).

\ea\label{ex:intro10}
    tikoto amo nibana?\\
    \gll tikoto amo ni-bana\\
    2\textsc{sg}.\textsc{poss}.daughter(\textsc{f}) sleep \textsc{aux}.\textsc{f}-\textsc{fut}\\
    \glt `Is your daughter going to sleep?'
\z

Similarly, some words now consist entirely of a suffix, and the suffix looks like a root. This is sometimes the case with the negative suffix \textit{-ra}. The negative suffix \textit{-ra} occurs as a normal suffix with an inflected root such as \textit{nafi} `be big' (\ref{ex:intro11}).
\ea\label{ex:intro11}
    nafira\\
    \gll nafi-ra\\
    be\_big-\textsc{neg}.\textsc{f}\\
    \glt `She is big.'
\z

But with a non-inflecting verb, the \textit{-ra} occurs with the auxiliary, and at the same time causes the auxiliary to be deleted in many contexts, so it looks like \textit{-ra} is a separate word, as in (\ref{ex:intro12}).

\ea\label{ex:intro12}
    ohi ra\\
    \gll ohi ra\\
    cry \textsc{neg}.\textsc{f}\\
    \glt `She did not cry.'
\z

But when \textit{-ra} occurs with another suffix such as \textit{-ma} `back', the auxiliary makes its presence known, and we see that \textit{-ra} is a suffix, as in (\ref{ex:intro13}) whether the verb is inflecting or non-inflecting.

\ea\label{ex:intro13}
    ohi namara\\
    \gll ohi na-ma-ra\\
    cry \textsc{aux}-back-\textsc{neg}.\textsc{f}\\
    \glt `She didn't cry anymore.'
\z

Thus it should be kept in mind that phonetically deleted auxiliaries are not represented in the interlinear presentation in this volume, even though they can be analyzed as being present underlyingly.

The prefix \textit{to-} likewise can be analyzed as being present underlyingly in contexts in which it is not pronounced. This prefix can be said to have two meanings, one being `away' (i.e. movement away from the speaker) and the other being `change of state'. Both uses are derivational, so that, for example, \textit{kita} `be strong' and \textit{tokita} `become strong', while related derivationally, are different lexical items.

Any verb that has the derivational suffix -\textit{witi}/\textit{-wite}\footnote{A number of suffixes and roots in Jarawara have a morphophoneme which is alternately realized as \textit{i} or \textit{e}, depending on whether the number of preceding moras in the phonological word is even or odd, respectively (\cite[40f.]{Dixon:2004xr}).} `out' also must have the prefix \textit{to-} `away', as the meaning requires it. However, whenever there is a person prefix \textit{o-} 1\textsc{sg}, \textit{ti-} 2\textsc{sg}, or the O-construction marker \textit{hi-}, these take the place of \textit{to-}, and \textit{to-} does not occur (this is a general rule). This can be seen in a comparison of (\ref{ex:intro14}) and (\ref{ex:intro15}).

\ea\label{ex:intro14}
    yama owawite\\
    \gll yama o-awa-wite\\
    thing(\textsc{f}) 1\textsc{sg}.\textsc{s}-see-out\\
    \glt `I went to look at it.'
\z

\ea\label{ex:intro15}
    yama towawite\\
    \gll yama to-awa-wite\\
    thing(\textsc{f}) away-see-out\\
    \glt `S/he went to look at it.'
\z

But the form of the verb as a lexical item is \textit{towawite}, and it would not be inapropriate to include \textit{to-} in the morpheme tier in (\ref{ex:intro14}), even though it is not pronounced. In this volume, though, I have chosen not to do this, for the reasons given above.

\section{Preparation of the texts}\label{sec:preparation}

I began recording Jarawara texts in 1987, and I continued to make recordings until 2023. All the recordings were made in Casa Nova village. The last recording of a text by Yowao was in 1997, five years before his death. As is evident in the collections I have published online at \href{https://www.silbrazil.org/resources/jarawara_interlinear_texts_vol_1}{https://www.silbrazil.org/resources/jarawara\_interlinear\_texts\_vol\_1} and \href{https://www.silbrazil.org/resources/jarawara_interlinear_texts_vol_2}{https://www.silbrazil.org/resources/jarawara\_interlinear\_texts\_vol\_2}, I recorded texts by both men and women of various age groups. Although most of the authors were from Casa Nova village, there is a sprinkling of texts by authors from the other Jarawara villages. The earlier texts were recorded on cassette tape, and the later texts were recorded digitally. MP3 files of all the interlinearized texts are available for listening or downloading at these two websites, and also at the websites for Siko's and Yowao's stories mentioned previously.

Starting in 2006, I realized the need for getting written permission from the Jarawara authors to publish their texts. By that time Siko and Yowao were both dead, so their representatives signed the permission forms. The permission forms give me permission to use the texts as long as it is not to make money.

When SIL began the Jarawara project in 1987, no Jarawaras were literate in either their language or in Portuguese. Based on a study of the phonology, I proposed an orthography, based almost entirely on the the orthography of Jamamadi, which was already available.\footnote{The only significant phonological difference between Jarawara and Jamamadi for the purpose of orthography development is that Jamamadi has both /t/ and /d/, while Jarawara has no phonemic voicing contrast at this point of articulation, so we used only <t>.} Using this orthography, the team from Youth With a Mission who were working in Água Branca village developed a primer. My wife Lucilia taught three young men in Casa Nova village using this primer, and she also taught them to teach others, following the Laubach method that the Youth With a Mission team were employing. In this way literacy spread in the two villages, and eventually to the third Jarawara village, Saubinha. 

I taught literate Jarawaras to transcribe taped stories, and I paid them to do this. This was a great help to me, as they caught many parts of the recordings that I could not. Then I input the transcriptions in the interlinearizing program (for many years Toolbox, and more recently FieldWorks Language Explorer), and I interlinearized the parts that I was able to. But a lot of the interlinearizing and translating could only be done after I went over a given recording with a Jarawara speaker. I found that some speakers had much more sophisticated knowledge of the language than others, and so almost all of the texts were gone over with just two men, Okomobi and Bibiri, a son and grandson of Yowao, respectively. For some texts it was necessary to go over the recording with both of them. And even after this, it was necessary to come back to one or the other of them with unresolved questions about particular passages. The language of traditional Jarawara stories is much more difficult than, for example, narratives of personal experiences told by middle-aged people, because of archaicisms in the vocabulary and the morphology.