The story \textit{Bahi} was told by Yowao in 1991. The corresponding audio file can be downloaded at: \url{https://www.silbrazil.org/sites/brazil/files/yowao-bahi1.mp3}.

This text is intended to explain the initiation of pubescent girls, which is still celebrated today. It has the following components: When the girl has her first menstruation, she goes into a seclusion hut, which is totally dark inside. She stays there by herself for several months, only going out to bathe with female relatives, and to go to the garden with them. When she goes out, she wears a covering over her head so that no men or boys see her face, and so she doesn't see them. It is said that if she sees them, they will get sick. At her coming out feast, there are several days of feasting, culminating with bringing the girl out and placing her lying face down on two logs. The men whip her bare back and legs with switches. She is taken away, and she hardly eats anything for several days, while her wounds heal. The women go after the men with faggots, and burn them in the stomach, to pay them back for harming the girl. Since the men are drunk, they can't get away and they do get burned.

The main character of the text is Bahi, which is the same as the word for `sun/thunder'. There is in fact a connection between the character Bahi and the sun: Bahi menstruates, and the red color of blood is said to be connected with the color of the sun at sunrise. There are various connections between the events in the story and aspects of the girls' initiation. Interestingly, Bahi is a man. He is teaching the women how they will be.