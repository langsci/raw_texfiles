\section{Word classes}\label{sec:wordclasses}

Clearly verbs constitute the most important word class of Jarawara; in fact, it is common for sentences to consist solely of a verb. This is because subject NPs are generally not required; NPs referencing subjects, when present, do not have the status of configurational arguments. The pronominals referencing both subject and object are almost always indexes within the verb, rather than NPs. 

For phonological reasons the verb is often represented as two or more separate orthographic words. Not only can person indexes constitute separate words, but other parts of the verb, including auxiliaries and some suffix-like elements, can also constitute separate words.

These points are illustrated in the following example (\ref{ex:intro16}), which comes from Siko's text, ``My father fought a cougar'' (\textref{text:17}).

\ea\label{ex:intro16}
    Owa ai tina ama tini.\\
    \gll owa ahi ti-na ama ti-ni\\
    1\textsc{sg}.\textsc{o} mess\_with 2\textsc{sg}.\textsc{s}-\textsc{aux}.\textsc{f} \textsc{sec} \textsc{2sg}.\textsc{s}-\textsc{bkg}.\textsc{f}\\
    \glt `You are messing with me.' [Q.\ref{ex:17-a21}]
\z

The lexical verb in this sentence is \textit{ahi na} `mess with', consisting of the root \textit{ahi} and its associated auxiliary \textit{na}. But all the words in the sentence are part of the verb, considered as a constituent. The pronominal \textit{owa} is a person index, not an NP. The pronominal \textit{ti-} is also a person index. The word \textit{ama} is what \citet{Dixon:2004xr} called a secondary verb, but it is not a separate verb. It has the status of a functional suffix.

The majority of verbs in Jarawara are associated with the auxiliary \textit{na} in this manner: if there are prefixes or suffixes, they are attached to the auxiliary rather than to the root.\footnote{Such verb roots can, however, be reduplicated, and there is a single suffix, i.e. distributive \textit{-ri}, which attaches to the root rather than to the auxiliary.} A sizeable minority of verbs are not associated in this way with an auxiliary, including for example \textit{tafa} `eat' and \textit{yabo} `be far/long' in (\ref{ex:intro1})-(\ref{ex:intro8}) above. And there are a handful of verbs that are associated with the auxiliary \textit{ha} rather than \textit{na}.

Nouns have inherent gender, either feminine or masculine. There is also a fairly large class of inalienably possessed nouns, which do not have inherent gender. Instead, they get their gender from their possessor, and this is often marked on the inalienably possessed noun. 

There is gender agreement at various points within the verb, mostly with the subject but often also with the object. Gender agreement in the verb, when it is with a pronominal, is always feminine. In (\ref{ex:intro16}), for example, there is feminine agreement with the second person singular subject, even though in the context of the story, this pronominal referred to a cougar, which has masculine inherent gender.

Other smaller word classes include adjectives, demonstratives, pronouns, question words, conjunctions, adverbs, postpositions, interjections, and sound words (i.e. ideophones). 

\section{The structure of a clause}\label{sec:clausestructure}

All types of clauses in Jarawara are verb-final. For intransitive clauses, only the verb is required. For copular clauses, the complement is required, followed by the verb. For transitive clauses, the structure depends on which transitive construction is used, the A-construction or the O-construction (\cite[chap.16]{Dixon:2004xr}). The A-construction is similar to intransitives in that the subject is the topic of the clause. In A-construction clauses, if the object is a third person, an NP referencing it is generally required. In O-construction clauses, both the subject and the object are topical, but the object is the main topic. Neither the subject nor the object is required to be referenced by an NP. If there are NPs referencing both the subject and the object, the subject NP usually occurs before the object NP in A-construction clauses, and usually after the object in O-construction clauses; but these are only tendencies.

The normal position of adjuncts is at the beginning of the clause.

There are various kinds of elements that can occur after the verb in a clause, all of which can be classed as parenthetical. These include: NPs, a small class of words such as \textit{tasa} `again' and \textit{bisa} `also', and adjuncts. There is a kind of finite clause that also occurs in this position, which likewise contributes parenthetical information.

\section{The Jarawara sentence}\label{sec:sentence}

A sentence consists minimally of a main clause, which normally can be distinguished from other clause types by its characteristic verbal morphology, especially mood morphemes. There are often one or more finite clauses that are sentence constituents that precede the main clause, and I have characterized these as medial clauses, proposing (\cite{Vogel:2022du}) that this constitutes clause chaining. Medial clauses typically convey events and states which refer to a time before the time of the main clause, or are concurrent with it.

Another kind of sentence consists of a juxtaposed clause followed by a main clause, which typically conveys the idea of frustrated expectation or intention usually translated by `but'.

There is also a list construction, which consists of two or more non-finite clauses with a special morphology and prosody, followed by a main clause that has as its verb the auxiliar \textit{na}.

Jarawara has complement clauses that are constituents not of sentences but of clauses, functioning as subjects, objects, and adjuncts. I have characterized these as non-finite (\cite{Vogel:2015az}), and in the interlinear presentation the verbs of these clauses are labelled \textsc{nfin}. There are also what \citet{Dixon:2004xr} called nominalized clauses. These are frequently the main constituent of adjuncts, and they can also occupy the subject position in a copular construction.

Direct quotes and indirect quotes likewise can be characterized as constituents of clauses. They can be seen as occupying the object position in the clause, with the verb \textit{ati na} `say' as the matrix verb.

Finally, finite clauses often function as relative clauses. Since in Jarawara these are internally headed, they can be characterized as NPs, rather than as modifiers of NPs.

In a presentation of interlinearized texts, while it is possible to show morphology in fair detail, it is not practical to show the complexities of the syntax, which in the case of Jarawara is quite complex indeed. Readers who wish to obtain more detailed information about Jarawara syntax are referred to Dixon's (\citeyear{Dixon:2004xr}) grammar and to my (\cite{Vogel:2022du}) monograph.

\section{Syntactic analysis of two short passages}

While it is not possible in the present volume to describe Jarawara syntax in detail, I think it is worthwhile dedicating a few pages to the syntactic analysis of a couple short passages from the texts. These selections should give the reader a good idea of how Jarawara sentences are organized.

The first selection is a passage from Siko's story, \textit{My father fought a cougar} (\textref{text:17}, lines \ref{ex:17-a26}-\ref{ex:17-a31}).\footnote{The following examples contain a source code in [square brackets] which takes the reader to the respective example in the text.} In the context, Siko's father had resisted the attacking cougar, and when he banged the cougar's head against the trunk of a miriti palm, the cougar had stopped attacking him and had run away. Siko's father had run to get his rifle, which he had left at the edge of the swamp.

\ea\label{ex:intro17}
    \textup{[}Faya okobi kame\textup{]\textsubscript{\textsc{med}}} \textup{[}bara ibewame\textup{]\textsubscript{\textsc{med}}} \textup{[}yama ki tonikimematamonaka\textup{.]\textsubscript{\textsc{mc}}}\\
    \gll faya okobi ka-me bara ibe-waha-me yama kii to-na-kima-himata-mona-ka\\
    so my\_father(\textsc{m}) come-back.\textsc{m} bullet(\textsc{f}) put\_inside-change-back.\textsc{m} thing(\textsc{f}) look\_at \textsc{ch}-\textsc{aux}-two-\textsc{fp}.\textsc{n}.\textsc{m}-\textsc{rep}.\textsc{m}-\textsc{decl}.\textsc{m}\\
    \glt `My father came back. He put a bullet in the rifle, and looked around.' \exsource{Q.\ref{ex:17-a26}}
\z

This sentence consists of a main clause, preceded by two medial clauses (\ref{ex:intro17}). The main clause is marked as such by the tense-modals \textit{-himata} and \textit{-mona}, and the mood morpheme \textit{-ka}. It is typical for medial clauses to precede the main clause in this way, expressing events or states that immediately precede or are concurrent with the event of the main clause.
	
Within the first medial clause, the first word \textit{faya} is a conjunction that can be analyzed as an adjunct. The clause is intransitive. The word \textit{okobi} refers to the subject, but it should not be considered a configurational subject phrase, because no subject NP is required for intransitives, when the subject is third person.
	
The second medial clause is transitive. The third person masculine subject is null, and \textit{bara} is the O. It is an A-construction clause, as shown by the gender agreement with the subject. Since it is an A-construction, the O NP \textit{bara} is required.

The main clause is likewise an A-construction, with the same syntax as the preceding medial clause.

\ea\label{ex:intro18}
    \textup{[}Watarematamonaka\textup{.]\textsubscript{\textsc{mc}}}\\
    \gll wata-ra-himata-mona-ka\\
    exist-\textsc{neg}-\textsc{fp}.\textsc{n}.\textsc{m}-\textsc{rep}.\textsc{m}-\textsc{decl}.\textsc{m}\\
    \glt `It wasn't there.' \exsource{Q.\ref{ex:17-a27}}
\z

This is a main clause, and is intransitive. The sentence consists of a single word, the verb. The subject is null.

\ea\label{ex:intro19}
    \textup{[}Tokomematamonaka\textup{]\textsubscript{\textsc{mc}}} \textup{[}tati karimari ahi \textup{[}yifo ya.\textup{]\textsubscript{\textsc{adju}}]\textsubscript{\textsc{pfc}}}
    \gll to-ka-ma-himata-mona-ka tati karima-haari ahi yifo ya\\
    away-go-back-\textsc{fp}.\textsc{n}.\textsc{m}-\textsc{rep}.\textsc{m}-\textsc{decl}.\textsc{m} head hit-\textsc{ip}.\textsc{e}.\textsc{m} there miriti(\textsc{m}) \textsc{adju}\\
    \glt `After hitting its head on the miriti palm, it went away.' \exsource{Q.\ref{ex:17-a28}}
\z

The first clause is an intransitive main clause, followed by a postposed finite clause, which contains parenthetical information. The postposed clause is intransitive, with the subject NP consisting of the inalienably possessed noun \textit{tati}, the third person masculine possessor of which is null. As I discuss in \citet[129]{Vogel:2022du}, unlike other subject NPs, a subject NP consisting of an inalienably possessed noun cannot normally be deleted. 
	
The verb in the postposed clause contains the immediate past eyewitness tense morpheme. Whereas the tense of the main clause is far past non-eyewitness because it relates to the perspective of the narrator (i.e. Siko), the postposed clause is immediate past eyewitness because it relates to the perspective of one of the participants (i.e. Siko's father). How tenses are interpreted is discussed in \citet[87f]{Vogel:2009ug}, and also the contrasts in how tenses are realized morphologically in main clauses and subordinate clauses, respectively (\citeyear[63]{Vogel:2009ug}).
	
The postposed clause contains a postposed adjunct. The word \textit{ahi} could also be analyzed as another postposed adjunct within the postposed clause.

\ea\label{ex:intro-20}
    \textup{[}Faya tati karime\textup{]\textsubscript{\textsc{med}}} \textup{[}kamarematamonane\textup{.]\textsubscript{\textsc{mc}}}\\
    \gll faya tati   karime ka-ma-ra-himata-mona-ne\\
    so head hit.\textsc{m} come-back-\textsc{neg}-\textsc{fp}.\textsc{n}.\textsc{m}-\textsc{rep}.\textsc{m}-\textsc{bkg}.\textsc{m}\\
    \glt `After hitting its head on the miriti palm, it didn't come back.' \exsource{Q.\ref{ex:17-a29}}
\z

The first clause is a medial clause, followed by the main clause. This shows one of the common uses of medial clauses, which is for what \citet[273]{Thompson:2007bv} call ``tail-head linkage''. Information from a preceding sentence is repeated at the beginning of the next sentence, forming a bridge between the two sentences (\cite[89f.]{Vogel:2022du}). The information is not always from the immediately preceding sentence, but it usually is, as in this case.

\ea\label{ex:intro-21}
    \textup{[}Towakematamona\textup{.]\textsubscript{\textsc{mc}}}\\
    \gll to-ka-ka-himata-mona\\
    away-\textsc{comit}-go-\textsc{fp}.\textsc{n}.\textsc{m}-\textsc{rep}.\textsc{m}\\
    \glt `It went away, hurt.' \exsource{Q.\ref{ex:17-a30}}
\z

The sentence consists of a main clause alone, which in turn consists solely of a verb. This verb form shows one of the meanings of the comitative prefix \textit{ka}-, which is `hurt/sick/high'. This is one of the meanings of \textit{ka}- which has no syntactic effect.

\ea\label{ex:intro-22}
    \textup{[[[}Fawafawa raba re\textup{]\textsubscript{\textsc{med}}} \textup{[}kamamatare\textup{]\textsubscript{\textsc{mc}}} \textup{[}hamahari\textup{]\textsubscript{\textsc{pfc}}]\textsubscript{\textsc{dirq}}} okobi ati nematamonaka\textup{.]\textsubscript{\textsc{mc}}}\\
    \gll fawa-fawa raba re ka-ma-mata-re hama-haari okobi ati na-himata-mona-ka\\
    \textsc{dup}-disappear a\_bit \textsc{neg}.\textsc{m} come-back-short\_time-\textsc{neg}.\textsc{m} be\_fierce-\textsc{ip}.\textsc{e}.\textsc{m} my\_father(\textsc{m}) say \textsc{aux}-\textsc{fp}.\textsc{n}.\textsc{m}-\textsc{rep}.\textsc{m}-\textsc{decl}.\textsc{m}\\
    \glt `{``}It disappeared. The fierce animal didn't come back,'' my father said.' \exsource{Q.\ref{ex:17-a31}}
\z

The sentence consists of a direct quote followed by the quote margin, which is marked as the main clause. The direct quote can be considered the O of the quote margin verb.
	
Usually the main clause of a direct quote will be morphologically marked as a main clause as well, but in this case the main clause of the quote (i.e., the second clause) is morphologically marked as if it were a medial clause. As discussed in \citet[52f.]{Vogel:2022du}, main clauses that are marked like medial clauses are not infrequent. But in this particular case it is clear that the second clause is a main clause, because it is followed by a postposed finite clause.
	
The second selection is from Yowao's story, \textit{Siraba} (\textref{text:08}, lines \ref{ex:08-a27}-\ref{ex:08-a33}). In the context, the teenage Siraba and her cousin have encountered a spirit in the forest, and they managed to run away from it. As they ran, they encountered Siraba's father on the trail, as he was weaving a basket.

\ea\label{ex:intro-23}
    \textup{[}Isiri kowe\textup{]\textsubscript{\textsc{med}}} \textup{[[[}ite\textup{]\textsubscript{\textsc{rel}}} ni ya\textup{]\textsubscript{\textsc{adju}}} bite kana ni tofotomemetemoneni\textup{.]\textsubscript{\textsc{mc}}}\\
    \gll isiri kowe ite ni ya bite kana ni to-foto-ma-hemete-mone-ni\\
    basket(\textsc{f}) weave.\textsc{m} sit.\textsc{m} to \textsc{adju} 3\textsc{sg}.\textsc{poss}.daughter(\textsc{f}) run \textsc{aux}.\textsc{nfin} away-appear-back-\textsc{fp}.\textsc{n}.\textsc{f}-\textsc{rep}.\textsc{f}-\textsc{bkg}.\textsc{f}\\
    \glt `As he sat there weaving the basket, his daughter appeared, running.' \exsource{H.\ref{ex:08-a27}}
\z

The main clause is preceded by a medial clause, which is a transitive A-con\-struction. At the beginning of the main clause there is an adjunct which contains a relative clause. We know that \textit{ite} is a relative clause, because there has to be an NP with \textit{ni ya}; that is, \textit{ni ya} cannot occur with a null category (\cite[143f.]{Vogel:2022du}).

The string \textit{isiri kowe ite} could also be analyzed as a complex relative clause, in which the subject of the relative clause \textit{ite} is another relative clause \textit{isiri kowe}. In any case, the NP which occurs with \textit{ni ya} is a relative clause.

The subject of the main clause verb, which is intransitive, is the complement clause \textit{bite kana ni}.

\ea\label{ex:intro-24}
    \textup{[[}Abi afa ama tini?\textup{]\textsubscript{\textsc{dirq}}} ati nemetemonehe\textup{.]\textsubscript{\textsc{mc}}}\\
    \gll abi afa ama ti-ni ati na-hemete-mone-he\\
    father(\textsc{m}) this.\textsc{f} be 2\textsc{sg}.\textsc{s}-\textsc{bkg}.\textsc{f} say \textsc{aux}-\textsc{fp}.\textsc{n}.\textsc{f}-\textsc{rep}.\textsc{f}-\textsc{dup}\\
    \glt `{``}Father, is it you?'' she said.' \exsource{H.\ref{ex:08-a28}}
\z

This is the most common greeting which Jarawaras use. It seems to be, at least historically, connected with the belief that spirits can appear as human beings, as in fact one does in this story. According to this view, Siraba is making sure it really is her father and not a spirit.

\ea\label{ex:intro-25}
    \textup{[[}Afa ama okene\textup{]\textsubscript{\textsc{dirq}}} ati ne\textup{]\textsubscript{\textsc{med}}} \textup{[[[}Abi inamati otara kiyo\textup{]\textsubscript{\textsc{med}}} \textup{[}kamaki ahi\textup{]\textsubscript{\textsc{mc}}]\textsubscript{\textsc{dirq}}} ati nemetemoneni\textup{.]\textsubscript{\textsc{mc}}}\\
    \gll afa ama o-ke-ne ati ne abi inamati otara  kiyo ka-maki ahi ati na-hemete-mone-ni\\
    this.\textsc{f} be 1\textsc{sg}.\textsc{s}-\textsc{decl}.\textsc{f}-\textsc{irr}.\textsc{f} say \textsc{aux}.\textsc{m} father(\textsc{m}) spirit(\textsc{m}) 1\textsc{ex}.\textsc{o} chase.\textsc{m} come-following.\textsc{nom} there say \textsc{aux}-\textsc{fp}.\textsc{n}.\textsc{f}-\textsc{rep}.\textsc{f}-\textsc{bkg}.\textsc{f}\\
    \glt `{``}It is me,'' he said. ``Father, there is a spirit chasing us, and he is coming this way,'' she said.' \exsource{H.\ref{ex:08-a29}}
\z

This rather complex sentence consists of a main clause preceded by a medial clause, both of which are direct quotes. The medial clause consists of the quote of the father with its quote margin. The main clause is the daughter's quote with its margin. 
	
The direct quote within the main clause, in turn, consists of a main clause preceded by a medial clause. This medial clause is a transitive A-construction, as seen by the gender agreement with the subject. Alternatively, this clause could be analyzed as a relative clause which is the subject of the main clause.

The main clause is not finite, but is a nominalized clause. The perceptive reader might ask, Why isn't \textit{kamaki} analyzed as a finite clause? The form indeed is a finite masculine form in many contexts. But when I asked what the word \textit{kamaki} would be if the subject were feminine instead of masculine, the speaker said it would still be \textit{kamaki}. That would not be true if \textit{kamaki} were masculine, because the feminine form would be \textit{kamakia} in that case. But if \textit{kamaki} is a nominalized form, it would be \textit{kamaki} whether the subject is masculine or feminine; therefore, it is a nominalized clause. \citet[485f.]{Dixon:2004xr} describes nominalized clauses that function as main clauses.

\ea\label{ex:intro-26}
    \textup{[[}Inamati toe awane\textup{]\textsubscript{\textsc{dirq}}} ati nematamona\textup{.]\textsubscript{\textsc{mc}}}\\
    \gll inamati to-he awa-ne ati na-himata-mona\\
    spirit(\textsc{m}) \textsc{ch}-be.\textsc{m} seem-\textsc{bkg}.\textsc{m} say \textsc{aux}-\textsc{fp}.\textsc{n}.\textsc{m}-\textsc{rep}.\textsc{m}\\
    \glt `{``}I guess it is a spirit,'' he said.' \exsource{H.\ref{ex:08-a30}}
\z

\ea\label{ex:intro-27}
    \textup{[}Inamati fotomaki kerewe rematamonaha\textup{.]\textsubscript{\textsc{mc}}}\\
    \gll inamati foto-maki kerewe ra-himata-mona-ha\\
    spirit(\textsc{m}) appear-following.\textsc{nfin} be\_slow \textsc{neg}-\textsc{fp}.\textsc{n}.\textsc{m}-\textsc{rep}.\textsc{m}-\textsc{dup}\\
    \glt `The spirit didn't take long to appear.' \exsource{H.\ref{ex:08-a31}}
\z

The sentence consists of a main clause that is intransitive. The subject, \textit{inamati fotomaki}, is an intransitive complement clause.

\ea\label{ex:intro-28}
    \textup{[}Inamati fotomaki kerewe rematamonaka\textup{.]\textsubscript{\textsc{mc}}}\\
    \gll inamati foto-maki kerewe ra-himata-mona-ka\\
    spirit(\textsc{m}) appear-following.\textsc{nfin} be\_slow \textsc{neg}-\textsc{fp}.\textsc{n}.\textsc{m}-\textsc{rep}.\textsc{m}-\textsc{decl}.\textsc{m}\\
    \glt `The spirit didn't take long to appear.' \exsource{H.\ref{ex:08-a32}}
\z

This is identical with the preceding sentence, except that it has the declarative morpheme.

\ea\label{ex:intro-29}
    \textup{[}Kana ni kamakimatamona ahi\textup{]\textsubscript{\textsc{mc}}} \textup{[}fe bata torebanoho\textup{.]\textsubscript{\textsc{pfc}}}\\
    \gll kana ni ka-maki-himata-mona ahi fee bata to-re-hiba-no-ho\\
    run \textsc{aux}.\textsc{nfin} come-following-\textsc{fp}.\textsc{n}.\textsc{m}-\textsc{rep}.\textsc{m} then 3\textsc{sg}.\textsc{o} surprise \textsc{ch}-raised\_surface-\textsc{fut}.\textsc{m}-\textsc{ip}.\textsc{n}.\textsc{m}-\textsc{dup}\\
    \glt `He came running, and he [Siraba's father] surprised him.' \exsource{H.\ref{ex:08-a33}}
\z

The main clause has as its subject the intransitive complement clause \textit{kana ni}. The main clause is followed by a postposed finite clause. The verb of the postposed clause has the interesting combination of the future morpheme with the immediate past morpheme. As \citet[212]{Dixon:2004xr} points out, this combination is used in a subordinate clause when the event of the subordinate clause is future with respect to the time of the main clause, but past with respect to the time of the narrator.

The postposed clause is an A-construction, as seen by the fact that there is no O-construction marker \textit{hi-} even though both the subject and the object are third person. In this context, there must be an overt NP referencing the O, and this is what \textit{fee} is. Pronominals in Jarawara are almost always indexes rather than NPs, but \textit{fee} is an exception.

The verbal suffix \textit{-re}/\textit{-ri} almost always means `raised surface' or related `edge', but there are a few contexts such as the verb \textit{bata tore} `surprise' in which it is difficult to see how the meaning could be anything like `raised surface'. But it definitely is the same suffix.

These examples should be sufficient to show the reader some of the complexities of Jarawara syntax, along with some of the possibilities for analysis.