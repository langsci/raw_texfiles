\documentclass[output=paper]{langsci/langscibook} 
\ChapterDOI{10.5281/zenodo.1441333}
\title{Foreword} 
\author{Pilar Prieto\affiliation{ICREA-Universitat Pompeu Fabra}}
% \chapterDOI{} %will be filled in at production

\abstract{\noabstract}

\maketitle\rohead{Foreword}
\begin{document}\label{chap:prefaceEd}
In the last few decades, language researchers have highlighted the pivotal role of \isi{prosody} in language production and language comprehension, showing the tight links between \isi{prosody} and other language components such as syntax and \isi{pragmatics}. First and foremost, \isi{prosody} in spoken language reflects the ``organizational structure of speech'' \citep{Beckman.1996}. Speakers use it to separate speech into chunks of information, or \isi{prosodic} constituents, thus helping listeners to parse discourse into meaningful syntactic units and sending signals about when to take turns in conversational exchanges. Secondly, \isi{prosody} plays a key role in pragmatic communication. Prosodic and \isi{intonational pattern}s express a broad variety of communicative meanings, ranging from \isi{speech act} information (assertion, question, request, etc.) and information status (given vs. new information, broad focus vs. \is{focus!narrow}narrow focus, contrast) to knowledge state (or epistemic position of the speaker with respect to the information exchange), affective state, and \isi{politeness} (\citealt{Gussenhoven2004,Ladd2008,Nespor2007}; see \citealt{Prieto.2015} for a review). 

Speech \isi{prosody} nowadays constitutes an active interdisciplinary research area which has drawn insights from different disciplines (like semantics, \isi{pragmatics}, syntax, language typology, and language processing) and a variety of methodologies, including psycholinguistic and computational modeling. Given this broad spectrum, carrying out research in \isi{prosody} now requires a high level of interdisciplinary awareness. It is for this reason that we welcome the initiative taken by three young but highly accomplished researchers, Ingo Feldhausen, Jan Fliessbach, and Maria del Mar Vanrell to compile a book about current research methods in \isi{prosody} from a \ili{Romance} perspective. The immediate aim is to offer in one volume a representative set of \isi{prosodic} investigations on \ili{Romance} languages which use diverse methods and data sources. However, taken as a whole, the interdisciplinary and critical perspective collectively represented here also reflects the methodological challenges currently facing the field of \isi{prosody}. As we will see below, those challenges include the need to develop more ecologically valid research methods for data elicitation, the use of triangulation methods for analyzing and interpreting quantitative findings, the complementary \isi{phonetic} and phonological analyses, and, above all, the integration of experimental and computational methods into \isi{prosodic} studies. 

\textit{Methods in \isi{prosody}: A \ili{Romance} language perspective} is made up of seven chapters, which are grouped to form the three parts of the book, each one centered around a particular topic. The first part focuses on the need to devote more research to the automatic \isi{prosodic analysis} of large-speech corpora, including different speech styles such as \isi{spontaneous speech} and dialogues. The second part highlights the importance of taking into account the various complementary levels of \isi{prosodic analysis}, such as multimodal analysis, \isi{phonetic} and acoustically-based labeling systems of \isi{intonation}, \isi{prosodic prominence}, and \isi{prosodic phrasing}, as well as perception-based analyses of \isi{prosody}. The third and final part of the book deals with data elicitation methods and points to the need for more refined elicitation methods to incorporate more ecologically-valid data and triangulation methods, as well as perceptual validation methods. In the short reviews that follow, I will try to highlight the particular issue that each chapter raises but also note the special insights that respective authors offer to the field as a whole.

Under the subheading \textit{Large corpora and spontaneous speech}, the first part of the book (Chapters~\ref{ch:1} and \ref{ch:2}) deals with the still undervalued application of automatic \isi{prosodic} annotation tools to large oral databases, as well as the analysis of \isi{spontaneous speech} for the study of \isi{prosody}. As is well known, the various syntactic and semantico-pragmatic functions of \isi{prosody} are manifested through the acoustic realization of \isi{prosody} by means of \isi{prosodic} phrasal grouping (via phrasal \isi{intonation} markers), intonational \isi{prominence}, and intonational modulations. Recent technological developments have greatly facilitated data collection, leading to the creation of freely accessible, large-scale audio and video corpora for various languages, such as \textit{Glissando} for \ili{Spanish} and \ili{Catalan}, which constitute a potential goldmine of information on \isi{prosodic} production. Similarly, acoustic\slash \isi{phonetic} tools such as \isi{Praat} (see \citealt{Boersma.praat}) have had a profound impact on our ability to measure and analyze \isi{prosodic} data. 


In \textbf{\chapref{ch:1}}, entitled ``Using large corpora and computational tools to describe \isi{prosody}: An exciting challenge for the future with some (important) pending problems to solve”, J. M. Garrido describes a set of tools that can take audio speech data and automatically output full orthographic and \isi{prosodic} transcriptions of the audio content and then segment and align them at phoneme, \isi{syllable}, word, and \isi{intonational phrase} levels. The author explains a set of tools that range from automatic \isi{orthographic transcription} of oral corpora, as well as tools that perform \isi{automatic transcription} and word segmentation, as well as \isi{prosodic} segmentation and \isi{prosodic} transcription. Though many of the tools have been specifically developed for \ili{Romance} languages (\ili{Catalan}, \ili{French}, \ili{Portuguese}, and \ili{Spanish} in particular), some of them have been extended to other languages. Garrido also reviews the results of pitch analysis experiments performed on large corpora. 

\textbf{\chapref{ch:2}} shows how spontaneous conversation can be used to uncover \isi{intonational pattern}s reflecting topic and focus functions. In ``The \isi{intonation} of pronominal subjects in Porteño \ili{Spanish}: an analysis of \isi{spontaneous speech}”, A. Pešková examines the intonational realizations of pronominal subjects in Buenos Aires \ili{Spanish} using a corpus of spontaneous conversational speech and shows that while intonational differences characterize the distinction between focused and topicalized pronominal subjects, this is not the case for the distinction between different types of topics. The analysis presented nicely combines a phonological analysis of the data using the autosegmental Sp\_ToBI \isi{prosodic} labeling methodology with an acoustic-\isi{phonetic} analysis of the target pronouns. The author uses this twofold strategy to argue that both \isi{spontaneous speech} and experimental laboratory database techniques are indispensable for the study of linguistic \isi{prosody}.

Under the heading \textit{Approaches to \isi{prosodic} analysis}, the second part of the book (Chapters~\ref{ch:3}--\ref{ch:5}) covers important issues including the importance of recognizing the multimodal – that is, verbal but also gestural – nature of communication, and the desirability of looking at both perception and production in the analysis of \isi{intonation} and \isi{prosodic prominence}. 

Research in the last few decades has highlighted the importance of \isi{visual information} in linguistic communication, but more work needs to be carried out within the domain of what is now known as \textit{visual prosody}. \textbf{\chapref{ch:3}}, entitled ``Multimodal analyses of \isi{audio-visual information}: Some methods and issues in prosody research”, represents a good step in this direction. The author, B. Gili Fivela, nicely reviews the methods which have been used to perform multimodal analyses of audio-visual speech materials, focusing especially on linguistic distinctions conveyed by \isi{prosody} (e.g., \isi{prosodic} focus, \isi{sentence modality}). The paper discusses a set of methods used to analyze articulatory kinematic data and speech-accompanying gestures (like head movements and facial expressions) across different sentence types, using examples from the literature mainly on \ili{Italian} and other \ili{Romance} languages. A good assessment of the pros and cons of articulatory and visual analysis methods of speech data is presented. The author highlights the fact that multimodal analysis of \isi{audio-visual information} has helped researchers to characterize various aspects of linguistic \isi{prosody} and that it is a necessary tool to provide a comprehensive analysis of \isi{prosody} in communication. 

An analysis of \isi{prosodic prominence} can reveal important information about under-described languages. In \textbf{\chapref{ch:4}}, entitled ``The Realizational Coefficient: Devising a method for empirically determining prominent positions in \il{Quechua!Conchucos}Conchucos Quechua”, T. Buchholz and U. Reich reveal how they went about describing \isi{prosodic prominence} in this Central \ili{Quechua} dialect using a methodology based on acoustic measurements of duration, pitch, and intensity. From these acoustic patterns, they obtained an overall realizational value which they label the ``Realizational Coefficient” by calculating the ratio of \isi{syllable duration}, mean F0, \isi{pitch range}, and intensity of one \isi{syllable} with respect to its adjacent syllables. This calculation expresses a measure of the relative \isi{realizational strength} of one \isi{syllable} over others, which can be helpful in describing \isi{prominence} patterns in languages that have yet to be fully analyzed.

Perceptual measures can be crucial in identifying contrastive patterns in intonational phonology. \textbf{\chapref{ch:5}}, entitled ``On the role of \isi{prosody} in disambiguating wh-exclamatives and wh-interrogatives in Cosenza \ili{Italian}”, O. Kellert, D. Panizza, and C. Petrone investigate the role of \isi{prenuclear} and nuclear \isi{prosodic} features in the perceptual identification of these structures in this \ili{Romance} variety. A two-alternative \isi{forced-choice identification task} together with reaction time measures were employed to test the listeners’ ability to distinguish between the two types of sentences. While the results support the hypothesis that the most important \isi{prosodic} cues for \isi{sentence-type disambiguation} are located at the end of the utterance, the fact that duration patterns in initial and mid-sentence positions regions significantly predicted reaction times strongly suggests that \isi{prenuclear} regions are actively exploited by listeners. The chapter also discusses why online measures like reaction times should be preferred to offline measures like gating responses. Importantly, the combination of identification tasks together with reaction times allows for an assessment of not only accuracy in \isi{prosodic} disambiguating but also the time location of the processing difficulties. 

The third part of the book includes two chapters (\ref{ch:6} and \ref{ch:7}) which deal with \textbf{elicitation methods} that can be used to collect speech data. A variety of such elicitation methods have been used in the field of \isi{prosody}, with some of them like the Discourse Completion Task proving particularly useful. Although the relative advantages and disadvantages of these elicitation methods have received some attention in the literature, a systematic critical assessment of their relative efficacy and ecological validity is thus far lacking. The two articles here constitute a first step in this direction.

One of the goals of intonational phonology is to be able to identify the distinctive pitch patterns in a given language in relation to systematic pragmatic differences like \isi{speech act} differences, focus categories, etc. In \textbf{\chapref{ch:6}}, entitled “The Discourse Completion Task in \ili{Romance} prosody research: Status quo and outlook”, M. M. Vanrell, I. Feldhausen, and L. Astruc superbly describe and critically assess the strengths and weaknesses of the Discourse Completion Task elicitation methodology, which has been extensively applied in research on \ili{Romance} \isi{prosody} in the last two decades. Their overall assessment of the method as a data collection instrument is positive. Among other things, they point to a set of important strengths like time-efficiency, the ease with which pragmatic and contextual factors can be controlled for, and the feasibility of using the task with illiterate or elderly participants. Among its weaknesses, they point out factors such as the dependency of the results on the initial set of discourses and also on the importance of contextual information. To address these weaknesses, the authors propose a set of modifications to the method centered around carefully crafting the context scenarios for each of the situations in order to better elicit specific speech acts and foster participant engagement. These reflections point to not only the practical need to refine this popular tool but also the need for ongoing research on data elicitation methods. 

Continuing with the quest for distinctive pitch patterns, in \textbf{\chapref{ch:7}}, entitled ``Describing the \isi{intonation} of speech acts in \ili{Brazilian Portuguese}: methodological aspects”, J. Moraes and A. Rilliard assess the results of applying to a set of \ili{Portuguese} data a production/perceptual methodology initially proposed by the Dutch School of \isi{prosody}. The paper describes how systematic modifications of pitch contours using \isi{resynthesis} techniques influence how \ili{Brazilian Portuguese} listeners interpret seven speech acts. The authors also look into the well-known phenomenon of inter-speaker variability in terms of interpreting \isi{prosody} and attempt to define what is universally acceptable and unacceptable across speakers in terms of various \isi{prosodic} parameters. Perceptual validation of these data show on the one hand the greater importance of pitch in comparison to duration or intensity patterns in conveying \isi{prosodic} distinctions in \ili{Portuguese} and on the other the importance of pitch-scaling patterns, specifically the need for three pitch levels (instead of two) for the intonational phonology of speech acts in this language. 

\newpage 

Taken as a whole, this volume will be of interest to those scholars and students of \isi{prosody} and linguistics interested in broadening their knowledge about current empirical methods. It also brings us a step forward in our assessment of the variety of methods currently in use for \isi{prosodic analysis}. One inescapable conclusion to be drawn from all this work is that \isi{prosodic analysis} is closely intertwined with many other systems of language, including pragmatic knowledge, and that mastery of a variety of complementary methods is of vital importance for \isi{prosody} researchers. Though the multidisciplinary approach reflected in this volume has already yielded a significant body of essential information regarding the use and assessment of a variety of methods in the field of \isi{prosody} there is still a need for an overarching theory that can not only encompass and explain perception and production patterns — which have traditionally been studied separately — but also take into account the complex relationships between \isi{prosodic} abilities and other linguistic, communicative, and cognitive skills. For example, though sometimes neglected, \isi{prosody} is a robust cue for the conveyance of essential pragmatic information in communication exchanges. As we have noted above, given the range of fields involved in such an endeavor, this goal calls for a high level of interdisciplinary awareness. 

There are also methodological challenges ahead, including the need to find more ecologically valid research methods that can combine experimental and computational methods in future studies (see \citealt{Prieto.2012} for a review). To illustrate this, for both perception and comprehension, behavioral data should be complemented by ERP and fMRI studies for a fuller picture of how the human brain produces and processes \isi{prosodic} features. Recent technological developments will greatly facilitate this kind of endeavor and will have a profound impact on our ability to measure and analyze \isi{prosodic} data. This combination of high quality recorded corpora and tools that automatically code acoustic cues has proved invaluable to research and must be further exploited, for it has huge potential to yield important results. This volume can therefore be read as both a snapshot of the current state-of-the-art in \isi{prosodic analysis} but also a signpost for future directions in \isi{prosodic} research.


{\sloppy
\printbibliography[heading=subbibliography,notkeyword=this]
}
\end{document}
