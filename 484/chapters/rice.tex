\documentclass[output=paper]{langscibook}
\ChapterDOI{10.5281/zenodo.15697595}
% \let\eachwordone=\textit
\author{Alexander Rice\orcid{}\affiliation{University of Alberta}}
\title{\textit{Mashti}: \uppercase{A} multipurpose filler in \uppercase{N}orthern \uppercase{P}astaza \uppercase{K}ichwa}
\abstract{This chapter presents a multimodal analysis of the filler word \textit{mashti} and its functions in Northern Pastaza Kichwa, a Quechuan language spoken in the Amazonian lowlands of Ecuador and Peru. In discourse, \textit{mashti} occurs less frequently than the language’s other fillers, but it has some unique properties that set it apart from the others. Unlike the language’s discourse interjection and demonstrative-derived fillers, \textit{mashti} derives from a lexicalized “whatchamacallit” type construction consisting of an interrogative \textit{ima} ‘what’ and the noun \textit{shuti} ‘name’. In discourse, \textit{mashti} can serve three functions: as a non-referential and distributionally unrestricted hesitative filler, as a nominal and verbal placeholder filler (which can fully mirror the morphology of the target word), and as a relexicalized pro-verb that reiterates a previously uttered action or event. These functions are corroborated not only by their morphosyntactic properties, but also by their relative co-occurrence with bodily correlates of disfluency and hesitation, such as prosodic lengthening, and the deactivation/stalling of manual gestures. Thus, this chapter contributes to and expands the typological research on hesitative and placeholder words by describing a multifunctional filler through a novel approach that takes into account the multimodal and prosodic signals associated with its different discourse functions.

\keywords{Quechuan languages, hesitator, placeholder, multimodality, gestures, prosody}
}

\IfFileExists{../localcommands.tex}{
  \addbibresource{../localbibliography.bib}
  % add all extra packages you need to load to this file

\usepackage{tabularx,multicol}
\usepackage{url}
\urlstyle{same}

\usepackage{listings}
\lstset{basicstyle=\ttfamily,tabsize=2,breaklines=true}

\usepackage{langsci-basic}
\usepackage{langsci-optional}
\usepackage{langsci-lgr}
\usepackage{langsci-osl}
% \usepackage{./langsci/styles/langsci-lgr}
% \usepackage{./langsci/styles/langsci-osl}
% \usepackage{langsci-gb4e}

\usepackage{tikz}
\usetikzlibrary{patterns,calc}
\pgfdeclarepatternformonly{south east lines}{\pgfqpoint{-0pt}{-0pt}}{\pgfqpoint{3pt}{3pt}}{\pgfqpoint{3pt}{3pt}}{
    \pgfsetlinewidth{0.6pt}
    \pgfpathmoveto{\pgfqpoint{0pt}{3pt}}
    \pgfpathlineto{\pgfqpoint{3pt}{0pt}}
    \pgfpathmoveto{\pgfqpoint{.2pt}{-.2pt}}
    \pgfpathlineto{\pgfqpoint{-.2pt}{.2pt}}
    \pgfpathmoveto{\pgfqpoint{3.2pt}{2.8pt}}
    \pgfpathlineto{\pgfqpoint{2.8pt}{3.2pt}}
    \pgfusepath{stroke}}
    
\usepackage{stmaryrd}
\usepackage{wasysym}
\usepackage{multirow}
\usepackage{caption}
\usepackage{subcaption}
\usepackage{mathrsfs}
\usepackage{qtree}

\usepackage{linguex}


  %pminos do not split footnotes
% \interfootnotelinepenalty=10000 %Footnote in Laporte chapters has to be split SN


%\DeclareIndexNameFormat{default}{%
%\nameparts{#1}%
%\usebibmacro{index:name}%
%{\index[names]}%
%{\namepartfamily}%
%{\namepartgiveni}%
% {}% L1
% {}% L2
%{\namepartprefix}% generates spurious space L3
%{\namepartsuffix}% generates spurious space L4
%}

%  {\DeclareIndexNameFormat{default}{%
%     \usebibmacro{index:name}{\index[names]}{#1}{#3}{#5}{#7}}}

%\DeclareIndexNameFormat{default}{%
%  \usebibmacro{index:name}{\sindex[nom]}{#1}{#3}{#5}{#7}}

%\DeclareIndexNameFormat{default}{%
%  \usebibmacro{index:name}{\sindex[person]}{#1}{#3}{#5}{#7}}
%\DeclareIndexNameFormat{default}{%
%\nameparts{#1} \usebibmacro{index:name}{\sindex[person]]}{\namepartfamily}{‌​\namepartgiven}{\nam‌​epartprefix}{\namepa‌​rtsuffix}}

%\newcommand{\smiley}{:)}

%\renewbibmacro*{index:name}[5]{%
%\usebibmacro{index:entry}{#1}%
%{\iffieldundef{usera}{}{\thefield{usera}\actualoperator}\mkbibindexname{#2}{#3}{#4}{#5}}}

% \newcommand{\noop}[1]{}

%remove for final
%\overfullrule=1mm

\newcommand{\tobi}[2]}}
\renewcommand{\S}[1]{\tobi{#1}{\textsc{*}}}

% this volume references
% puts: [this volume]
% already defined: \citetv
%\newcommand{\citepv}[1]{(\citeauthor{#1} \citeyear*{#1} [this volume])}
\newcommand{\citealtv}[1]{\citeauthor{#1} \citeyear*{#1} [this volume]}

%parentheses around example number
\newcommand{\pref}[1]{(\ref{#1})}

% in-text examples

\newcommand{\lnex}[1]{\textit{#1}} %target lang word
\newcommand{\lnlit}[1]{(lit.: `#1')} %literal reading
\newcommand{\lnlat}[1]{(#1)} % latinization
\newcommand{\lntrans}[1]{`#1'} %translation
\newcommand{\lnexl}[2]%
{\lnex{#1}{} \lnlat{#2}} % ex with latinization
\newcommand{\lnexlat}[3]{\lnex{#1}{} \lnlat{#2}{} \lntrans{#3}} % ex with latinization and tranl.

%ch01
\newcommand{\co}[1]{\mbox{\textbf{#1}}}

%ch09

\newcommand{\cyrbulg}[1]{\begin{otherlanguage*}{bulgarian}#1\end{otherlanguage*}}


%ch10
\newcommand{\nlp}{{\small NLP}}
\newcommand{\mwe}{{\small MWE}}
\newcommand{\rae}{{\small RAE}}
\newcommand{\lvc}{{\small LVC}}
\newcommand{\pos}{{\small P}o{\small S}}
%\newcommand{\todo}[1]{ \textcolor{red}{#1} }

%\renewcommand{\labelenumi}{\theenumi}
%\ainamefmt{{vv}{ll}{, ff}{, jj}} % fullname

\newcommand{\biberror}[1]{{\color{red}#1}}

\newcommand{\osenovaitem}{--~} 
  %% hyphenation points for line breaks
%% Normally, automatic hyphenation in LaTeX is very good
%% If a word is mis-hyphenated, add it to this file
%%
%% add information to TeX file before \begin{document} with:
%% %% hyphenation points for line breaks
%% Normally, automatic hyphenation in LaTeX is very good
%% If a word is mis-hyphenated, add it to this file
%%
%% add information to TeX file before \begin{document} with:
%% %% hyphenation points for line breaks
%% Normally, automatic hyphenation in LaTeX is very good
%% If a word is mis-hyphenated, add it to this file
%%
%% add information to TeX file before \begin{document} with:
%% \include{localhyphenation}
\hyphenation{
    Beck-man
    Ngu-yen
    back-chan-nel
    back-chan-nels
    mo-not-o-nous
    ste-reo-typ-i-cal
}

\hyphenation{
    Beck-man
    Ngu-yen
    back-chan-nel
    back-chan-nels
    mo-not-o-nous
    ste-reo-typ-i-cal
}

\hyphenation{
    Beck-man
    Ngu-yen
    back-chan-nel
    back-chan-nels
    mo-not-o-nous
    ste-reo-typ-i-cal
}
 
  \togglepaper[1]%%chapternumber
}{}

\begin{document}
\maketitle 
\graphicspath{{figures/rice}}

\section{Introduction}
\label{sec:rice:1}

Typological descriptions of minority languages have historically relied on “clean” language data and therefore prioritized samples of monologic language composed of tidy and fluently produced utterances. More recently, studies in language documentation and cognitive linguistics have come to recognize the primacy of natural and contextualized language data and of language function. Research in this vein thus places a premium on language data coming from people immersed in real-life situations regardless of how “messy” or “non-standard” such data may be given their importance to the understanding of language as an emergent phenomenon rooted in social interaction.

One of the hallmarks of naturalistic speech is disfluency, or problems that arise in the planning and execution of speech \citep[452]{Lickley2015}.  To manage disfluencies, speakers make use of a range of discourse strategies which include hesitations, pauses, false starts, repairs, and prosodic lengthening. One such strategy is the use of \textsc{filler} words, which speakers use to buy time while experiencing difficulty remembering a certain word or content to express \citep{Fox2010}.

In this chapter, I describe the uses of the filler word \textit{mashti} in Northern Pastaza Kichwa, an indigenous language of South America spoken in the Ecuadorian-Peruvian Amazon. I show that speakers of Northern Pastaza Kichwa use the filler \textit{mashti} as a \textsc{hesitator} (\textsc{hes}) or as a lexical \textsc{placeholder} (\textsc{ph}). Furthermore, \textit{mashti} can be used as \textsc{resumptive} \textsc{pro-verb} (\textsc{prov}). These functions of \textit{mashti} are largely determined by the way it relates to the surrounding discourse. Crucially, this analysis is also multimodal and takes into account phenomena in the vocal-aural and visuo-spatial modalities. Specifically, I show that prosodic lengthening and gestural disfluency may be correlated with the use of \textit{mashti} as a filler. Prosodic lengthening, in particular, is crucial in distinguishing between the hesitator and other uses of \textit{mashti} when the degree of morphosyntactic integration is not obvious. Finally, the lexical origins and synchronic distribution of \textit{mashti} functions suggest a multipronged path of development.

In the remaining sections of this chapter, I present the language and the nature of the data used for this study (\sectref{sec:rice:2.1})-(\sectref{sec:rice:2.2}) and offer primers on fillers (\sectref{sec:rice:2.3}) and multimodal analysis (\sectref{sec:rice:2.4}). Subsequently, I describe the functions of \textit{mashti} with contextualized examples (\sectref{sec:rice:3}), as well as the co-occurring multimodal disfluency phenomena (\sectref{sec:rice:4}). In \sectref{sec:rice:5}, I discuss the origin and possible pathways of the development of \textit{mashti} and conclude with a brief overview of the findings of this chapter in \sectref{sec:rice:6}.

\section{Preliminaries} 
\label{sec:rice:2}
The objective of this chapter is to explore how the word \textit{mashti} in Northern Pastaza Kichwa is used as a filler and pro-verb in discourse. In service of this objective, this section sets up the background information and essential concepts. In \sectref{sec:rice:2.1} I provide a brief introduction to Northern Pastaza Kichwa and the context its speakers inhabit. In \sectref{sec:rice:2.2} I sketch the outlines of the corpus used for this chapter as well as the nature of the data. In \sectref{sec:rice:2.3} I outline the concepts of filler words and the criteria used to distinguish between hesitative fillers and placeholder fillers. Finally, in \sectref{sec:rice:2.4}, I present the basic tenets of multimodal analysis and illustrate how I represent multimodal elements in the linguistic examples used in this chapter. 

\subsection{Northern Pastaza Kichwa and its speakers} 
\label{sec:rice:2.1}
Northern Pastaza Kichwa is a member of the fairly large and diverse Quechuan language family, which is native to the Andean mountain range of South America. Northern Pastaza Kichwa belongs to the northern branch of the Quechuan languages, Quechua II-B \citep{Torero} spoken in the northern Peruvian Amazon, Ecuador and southern Colombia\footnote{This branch of Quechua is also sometimes referred to as “Late Northern Quechua” \citep{Muysken2021}. This branch is listed on Glottolog as “Colombia-Ecuador Quechua”: \url{https://glottolog.org/resource/languoid/id/colo1257}.}  and is also known as Bobonaza Quichua (\citealt{Orr1965}) and Pastaza Quichua \citep{Orr1991}, since most speakers live along the Bobonaza river in the Pastaza province of Ecuador (\citealt{Nathan2014}: 182).

In Ecuador, the varieties of Quechua are referred to as “Kichwa/Quichua” and, traditionally, all varieties were considered to be dialects of a single Kichwa/Quichua language. However, recent research has moved towards positing at least two languages: Ecuadorian Highland Kichwa, which is spoken in the Ecuadorian Andes, and Amazonian Kichwa,\footnote{Amazonian Kichwa is sometimes referred to as “Ecuadorian Lowland Quechua” in some descriptive literature.} which is spoken primarily in the Ecuadorian Amazon with some extensions into the Peruvian Amazon (\citealt{Grzech2019}; \citealt{Muysken2019a}). Under this view, Northern Pastaza Kichwa is thus a dialect or variety of Amazonian Kichwa.

Estimates of the number of speakers of Northern Pastaza Kichwa range between 10,000 and 30,000 and the vitality status of Northern Pastaza Kichwa has been rated as “threatened” and “definitely endangered” (\citealt{Nathan2014}: 186). Despite the number of speakers, Northern Pastaza Kichwa and the other Indigenous languages of Ecuador face the challenge of stagnating intergenerational transmission. It has become increasingly difficult to find fluent speakers under the age of 30 and young adults and youth appear to feel more confident speaking Spanish (\citealt{Grzech2019}: 131–132; \citealt{Nathan2014}: 186). 

In terms of typological characteristics, Northern Pastaza Kichwa looks like other Quechuan languages: small vowel inventory, agglutinative word structure, exclusively suffixing, and heavy use of verbs and nominalizations. That said, Northern Pastaza Kichwa and the other varieties in the northern branch of Quechua are somewhat unique in that they are morphosyntactically “simplified” compared to the rest of the Quechuan family. Muysken has argued extensively that the Ecuadorian-Colombian Quechua varieties represent “pidginized” or “creolized” forms of Quechua that arose from the demographic collapse and reshuffling of Indigenous populations in Ecuador following contact with European colonizers in the sixteenth century. Speakers of non-Quechuan languages would have then shifted to one or more varieties of Quechua as the available lingua franca after their communities were destroyed by disease and colonial violence, thus resulting in the phonological and morphosyntactic simplification of the Quechuan varieties spoken there (\citealt{Muysken2000,Muysken2009,Muysken2019a,Muysken2021}).

“Kichwa/Quichua” is an exonym that is used most frequently in academic literature and socio-political discourse in Ecuador. Northern Pastaza Kichwa speakers refer to their language as \textit{runa shimi} ‘human language’ and themselves as \textit{runa} ‘human’. There is no single ethnic identity associated with Northern Pastaza Kichwa. Some speakers identify themselves as belonging to one of the officially recognized Indigenous nationalities of Ecuador: Andwa, Sápara, or Kichwa Amazónico. Other speakers eschew the national identities and instead self-identify on the basis of their community of origin. Northern Pastaza Kichwa is also the second or third language of speakers of Wao Tedeo and the Chicham languages,\footnote{Wao Tedeo [ISO 639-3: auc] is an isolate also known as Wao Terero, Wao, and Huaorani. The main Chicham language spoken in this area of Ecuador is Achuar-Shiwiar [ISO 639-3: acu].} and intermarriage between the language groups is common (\citealt{Reeve2022}: 5–6). Traditionally, Northern Pastaza Kichwa speakers practiced subsistence agriculture, hunting, fishing, and foraging in the rainforest as well as riverine trade with extended kinship networks. Today, this lifestyle is becoming increasingly non-viable due to political, economic, and environmental pressures. Many families are forced to abandon their lands and communities and move to cities and towns to participate in wage-labour employment. The resulting erosion of traditional material culture and lifestyle further accelerates language shift to Spanish. 

\subsection{Corpus and participants}
\label{sec:rice:2.2}

The data used for this chapter are drawn from video recordings collected between 2011 and 2022 by multiple researchers. The majority of the recordings in the corpus were collected by Tod Swanson and Janis Nuckolls. Some of the videos are publicly accessible on \citegen{Swansona}  and \citegen{Nuckollsc} YouTube channels. Other recordings come from my own fieldwork, and Lisa Warren Carney. Each recording has a unique file name which combines the collecting researcher’s initials, the language’s ISO code, and a three-digit numeric ID. Recordings beginning with “tds\_qvz” come from Tod Swanson, “jbn\_qvz” from Janis Nuckolls, “lwc\_qvz” from Lisa Warren Carney. My own recordings are not marked with my initials and read simply as “qvz\_001”, “qvz\_002”, etc. I manage the corpus with SayMore (\citeyear{SIL2021}) and use FLEx (\citeyear{FLEX}) and ELAN (\citeyear{ELAN}) for annotation.

The corpus is comprised of approximately six  hours of video recordings and 30,710 words. All of the videos have been transcribed and translated into Spanish and/or English. The transcriptions and translations were completed by native speakers with some assistance from researchers. The genres represented in the corpus are principally monologues and conversation, with some song, and procedural description. The recordings come from seven consultants, all adults, one male and six female, the youngest of whom at the time of recording were in their mid 30’s and the oldest in their 80’s. 

The corpus is not balanced in terms of speaker gender since most of the participants recorded are older women and all but two of the speakers are related to each other. Some speakers are also better represented than others in terms of recordings. The corpus is thus an example of convenience sampling, in which there was no conscious selection of recordings; the recordings that are available and ultimately used are a matter of coincidence rather than design \citep[64]{Seifart2008}.\footnote{Convenience sampling is often the case when working with Indigenous/minority language data. Language communities engaged in a documentation project may wish to prioritize collecting recordings of elders or specific genres of language use such as traditional narratives and discourse associated with traditional material culture.}  A summary of the recordings used in the corpus is given in Appendix~\ref{appendix:rice:I}. 

\subsection{Fillers: Hesitatives and placeholders}
\label{sec:rice:2.3}

Fillers are non-silent linguistic devices used to buy time while experiencing a disfluency event and retrieving a sought after lexical item \citep{Fox2010}. The two kinds of fillers relevant to this chapter are hesitatives and placeholders. The two categories of filler are distinguished by their degree of syntactic integration with the surrounding utterance.

Also known as “filled pauses” \citep{Lickley2015}, hesitatives are used by speakers to signal to the interlocutor that they are experiencing a disfluency. Hesitatives are used to announce that the speaker is searching for a word or is wanting to keep or cede the floor (\citealt{Clark2002}). Hesitatives can be as simple as an unspecific speech sound, such a drawn-out vowel or nasal sound, and more conventionalized but not necessarily lexical forms such the English hesitatives \textit{uh} and \textit{um}. Lexical forms and even phrasal items can also be deployed as hesitatives, such as the English \textit{well}, \textit{like}, and \textit{y’know} \citep[1]{Fox2010}.

Placeholders can also be used when a speaker experiences a disfluency while searching for a particular lexical item. However, unlike hesitatives, placeholders exhibit more morphosyntactic integration with the adjacent/surrounding utterance (\citealt{HayashiYoon2006}: 490). A placeholder thus, “holds the place” of the targeted lexical item that the speaker experiences difficulty accessing. The posterchild of (hesitative) placeholders in English is \textit{whatchamacallit} which derives from something like ‘what you may call it’. An English speaker may use \textit{whatchamacallit} during a disfluency event to hold the place of a lexical item the speaker is having difficulty acquiring.

Not all placeholders are used in contexts of hesitation or disfluency. Even if the target word or phrase is accessible to the speaker’s short term memory, placeholders can be deployed euphemistically to avoid uttering taboo words or topics \citep{Enfield2003}, or to purposefully obfuscate referents or propositions from certain interlocutors \citep{AmiridzeGeorgian2010}. Placeholders can also be deployed when the target~-- even if accessible to the speaker~-- simply is not important to mention and, thus, results in purposefully vague or imprecise expressions (\citealt{PalaciosMartinez2015}; \citealt{Podlesskaya2010}: 26; \citealt{Seraku2022}).

The majority of placeholder uses of \textit{mashti} in this corpus are used in contexts of disfluency ($N=17$), while a handful of the tokens are used for in the vague\slash imprecise referential manner ($N=5$) for which the target is never uttered. Both types of placeholders are discussed in \sectref{sec:rice:3.2}, but generally this chapter is given to the discussion of \textit{mashti} placeholders for which there is an identifiable target.

In discourse, placeholders uttered in contexts of disfluency are typically followed by utterances containing the target word when the speaker resolves the disfluency (\citealt{Keevallik2010}: 310; \citealt{Vallejos-Yopán2023}: 12). In Northern Pastaza Kichwa, placeholders that target a delayed constituent tend to carry the same morphology and syntactic position as the target item. \citet[18]{Podlesskaya2010} refers to this phenomenon as “mirroring” the “grammatical shaping” of the target and notes that placeholders can fully or partially mirror the target.

Cross linguistically, common lexical sources of fillers are demonstratives, pronouns, semantically bleached nouns, and lexicalized constructions featuring interrogatives (\citealt{Podlesskaya2010}: 12–13). In Northern Pastaza Kichwa, demonstrative pronouns are often deployed as fillers; however, other lexical items like discourse particles and conjunctives borrowed from Spanish can also be utilised as fillers. \tabref{tab:rice:1} shows the lexical items commonly used as fillers in the language.

\begin{table}
\begin{tabular}{ll}
\lsptoprule
Word & Gloss\\
\midrule
\textit{chay} & mirative distal demonstrative pronoun\\
\textit{chi} & distal demonstrative pronoun\\
\textit{este} & filler (borrowing from Spanish)\\
\textit{ima} & interrogative pronoun ‘what’\\
\textit{mashti} & hesitative; placeholder; resumptive pro-verb\\
\textit{kay} & proximal demonstrative pronoun\\
\textit{ña} & temporal discourse particle ‘now, then, already’\\
\textit{osea} & conjunctive (borrowing from Spanish)\\
\textit{y} & conjunctive (borrowing from Spanish)\\
\textit{ya} & now, already (borrowing from Spanish)\\
\lspbottomrule
\end{tabular}
\caption{\label{tab:rice:1} Lexical items used as hesitative fillers in Northern Pastaza Kichwa}
\end{table}

Most of the lexical items in \tabref{tab:rice:1} are native Quechuan roots and can be used as hesitatives. The demonstrative pronouns are among the most frequently occurring words in the corpus, and I have not yet determined how often they are used as hesitatives nor if any of them can be used as placeholders. Northern Pastaza Kichwa speakers also make use of non-lexical hesitatives. The example in \REF{ex:rice:1} shows the use of the lengthened /a/ vowel as a hesitative.

\ea%1
    \label{ex:rice:1}
    \gll  \textbf{aaa} {chi} {runduma-ta}  {chi-ta} {mana} {shuti-ta} {yacha-ni=chu}\\
    \textbf{\textsc{hes}} \textsc{dist.dem} {sedge-\textsc{obj}} \textsc{dist.dem-obj} \textsc{neg} {name-\textsc{obj}} {know-1=\textsc{neg}}\\
    \glt ‘um… that sedge, that one, I don’t know its name’ [qvz014:22]
\z

The other fillers given in \tabref{tab:rice:1} are borrowings from Spanish and have more varied lexical functions when not deployed as hesitatives. The borrowing of the Spanish demonstrative \textit{este} is used only as a hesitative in Northern Pastaza Kichwa discourse. The use of \textit{este} as a filler is quite common in Latin American Spanish (\citealt{Kany1969}; \citealt{Vallejos-Yopán2023}; \citealt{Zorraquino1999}).

Compared to the other fillers in Northern Pastaza Kichwa, \textit{mashti} has some unique properties: it likely derives from a lexical phrase (cf. \sectref{sec:rice:5}) instead of a demonstrative, it can be deployed both as hesitative and a placeholder and can be used as a resumptive pro-verb (cf. \sectref{sec:rice:3.1}). Because of these unique properties, I chose to focus on \textit{mashti} for this chapter. I identified 157 \textit{mashti} tokens in the corpus and annotated each for speaker, function (hesitative, placeholder, or pro-verb), word class (if placeholder), target (if placeholder or proverb), direction to target (before or after), and the disfluency correlates described in \sectref{sec:rice:4}. This annotation data can be found in the mashti\_dat.csv file, which is available in an OSF drive (A. \citealt{Rice2023}).\footnote{\url{https://osf.io/g84mv/}.} Likewise, the video, audio, and ELAN files for the recordings that the tokens were drawn from are also available in the OSF drive. The counts presented in the tables in this chapter were generated by an R script (\citeyear{Team2013}) and plots were generated using the ggplot2 package for R \citep{Wickham2016}. The script is available in the aforementioned drive. 

\subsection{The multimodal expression of disfluency events}
\label{sec:rice:2.4}

In this chapter, I take a multimodal approach to the use of fillers in that I consider disfluency phenomena in two modalities: the vocal-aural modality and the visuo-spatial modality.\footnote{Cf. \citet{Stivers2005} for a detailed primer on multimodality and the distinction between the vocal-aural and visuo-spatial modalities.} The vocal-aural modality concerns the production, reception, and interpretation of speech as an acoustic signal. The visuo-spatial modality refers to the use of bodily articulators beyond the vocal tract that contribute to meaning in language (posture, facial expression, hand gestures, direction of gaze, etc.). \tabref{tab:rice:2} shows the disfluency correlates that I investigate in this chapter and their corresponding modalities.

\begin{table}
\begin{tabular}{ll}
\lsptoprule
Modality & Disfluency correlate\\
\midrule
Vocal-aural modality & false starts\\
& pauses\\
& prosodic lengthening\\
Visuo-spatial modality & gaze aversion\\
% \cmidrule{-~} 
& excessive blinking\\
% \cmidrule{-~} 
& manual gesture\\
% \cmidrule{-~}
& gestural disfluency\\
\lspbottomrule
\end{tabular}
\caption{\label{tab:rice:2} Disfluency correlates by modality}
\end{table}

Many of the examples in this chapter are presented in a format to help convey the visually perceived bodily elements that occur with the prosodic and lexical material. In the examples in this chapter, I make use of special conventions to indicate different types of disfluency correlates. These conventions are given in \tabref{tab:rice:3}. 

\begin{table}
\begin{tabularx}{\textwidth}{XX}
\lsptoprule
Disfluency correlate & Convention\\
\midrule
false start & \textit{word}{}-\\
pause & (pause—XXXms)\\
prosodic lengthening & \textit{worrrd}(XXXms)\\
duration and direction of gaze & =>(right), <=(left), =˄(up), =˅(down)\\
excessive blinking & \^{}\\
duration of manual gesture & ***\\
duration of gesture hold & (hold—XXXms)\\
{\small number of repetitions in stalling gestures} & (X repetitions)\\
\lspbottomrule
\end{tabularx}
\caption{\label{tab:rice:3} Disfluency conventions used in the examples in this chapter}
\end{table}

False starts are indicated by “-” and are attached to the word that was started in error in the transcription line. The word may be fully or partially produced. False starts are represented as [false.start] in the gloss line. Pause indicators are placed in parentheses in the transcription line and are followed by an em dash with the length of the pause in milliseconds. Words that are prosodically lengthened are represented in the transcription line by repetition of the letter that corresponds to the lengthened segment, followed by a set of parentheses containing the duration of the lengthened segment in milliseconds.

The conventions used to represent disfluency correlates in the visuo-spatial channel are placed above the transcription line. The use of “=” indicates an abrupt shift in gaze and repeated instances of “=” are used to indicate the approximate duration of the new gaze direction relative to the utterance in the transcription line. The use of the caret symbols (>, <, ˄, ˅) are appended to each gaze annotation to indicate the direction of gaze (from the point of view of the camera). Excessive blinking events are indicated by repeated circumflex symbols (\^{}) and each circumflex represents one blink. The use of repeated asterisks (*) shows the duration of a manual gesture.\footnote{The convention of using “***” to indicate the relative duration of a gesture is an adaptation of Kendon's (\citeyear{Kendon2004d}) convention of using “***” above the transcription to show the duration of a gestural stroke.} Indicators for gestural holds are placed in parentheses following the asterisk strings for manual gestures. For stalling gestures, the number of repetitions is given in parentheses following the gesture indicator.

Many of the examples in this chapter represent longer stretches of discourse and, accordingly, examples are divided into intonational units using lowercase letters (a, b, c, etc.). The examples that show disfluency correlates in the visuo-spatial channel are accompanied by a figure that shows one or more still images from the video recording that the \textit{mashti} token was drawn from. The figure caption contains the recording name and the start and stop time stamps. If more than one still is present in the figure, each still is indexed with a lowercase roman numeral (i, ii, iii, etc.). In the associated example, the index is placed above the transcription lines to show when the still occurs relative to the production of the utterance. 

To illustrate the use of some of these conventions, consider \figref{fig:rice:1} and the associated stretch of discourse in example \REF{ex:rice:2}. In \REF{ex:rice:2a}, the speaker is narrating a story. In \REF{ex:rice:2b}, she begins a new intonational unit and the index (i) above the transcription refers to still (i) of \figref{fig:rice:1}, wherein the speaker is shown to be looking down at her hands while she is narrating. In \REF{ex:rice:2c}, the speaker begins another intonational unit but experiences a disfluency. The discourse particle \textit{ña} (in bold) is used as a hesitative; this is followed by a pause of 1456ms, an abrupt change in gaze to the right (the speaker’s left), and a period of excessive blinking towards the end of the pause. The index (ii) above the transcription line of \REF{ex:rice:2c} refers to still (ii) of \figref{fig:rice:1}, which captures the moment the speaker shifts her gaze to the right (as is indicated by the red arrow).

In \REF{ex:rice:2d}, the use of “<===” shows that speaker shifts her gaze to the left (her right). She also swings her right arm out to her right, extends her pointer finger, and brings it into contact with the table she is sitting at. The use of the asterisks (***) show the approximate duration of this manual gesture. The index (iii) in \REF{ex:rice:2d}, points to (iii) of \figref{fig:rice:1}, wherein the same actions are indicated with the red arrows. After starting the gesture and gaze shift, the speaker uses the distal demonstrative pronoun \textit{chi} as a hesitative. The production of \textit{chi} is prosodically lengthened; the extra \{i\} characters show that the final /i/ vowel of \textit{chi} was drawn out, and the duration of the lengthened /i/ is given in the following parentheses (311ms). This is followed by a pause of 639ms. In \REF{ex:rice:2e}, the disfluency is resolved, and the speaker continues the story, although her gaze remains directed towards the left (not shown in \figref{fig:rice:1}).

  
\begin{figure}
\includegraphics[height=\textheight]{fig1.jpg}
\caption{\label{fig:rice:1}Gaze shift and gesture in a disfluency event (qvz026, 7:08--7:18), photo © 2019 Alexander Rice.}
\end{figure}



\ea%2
    \label{ex:rice:2}
\ea \label{ex:rice:2a}
\gll {peruwano-y} {maska-nga} {ri-sha} {tupa-nau-ra} {chi} {libru-ta}\\
         {Peru-\textsc{loc}} {search-\textsc{fut}} {go-\textsc{ss}} {find-\textsc{3pl-pst}} {\textsc{dist.dem}} {book-\textsc{obj}}\\
    \glt ‘they had gone searching in Peru [and] they found the book’
\medskip

\ex \label{ex:rice:2b}
\glll {\normalfont{(i)}} \\
{chi-wi}\\
\textsc{dist.dem-loc}\\
\glt {‘there’} \\
\medskip

\ex \label{ex:rice:2c}
\glllll {} {\normalfont{(ii)}}\\
{} {\hspace{2cm}\textasciicircum\textasciicircum\textasciicircum} \\
{} {============>} \\
{\textit{\textbf{ña}}}  {(pause—1456ms)}\\
{\textbf{\textsc{dsc.prt}}} {}\\
\glt  \textbf{‘then…’}\\
\medskip

\ex \label{ex:rice:2d}
\glllll {\normalfont{(iii)}}\\
{********************}\\
{<===========================}\\
{\hspace{1cm}{\textbf{\textit{chiii}}\textbf{(311ms)}}{\hspace{1cm}(pause—639ms)}}\\
{\hspace{1cm}{\textbf{\textsc{dist.dem}}}}\\
\glt {\hspace{1cm}{‘\textbf{thaaat…}’}}\\
\medskip

\ex  \label{ex:rice:2e}\glll 
{<==================} \\
{\textit{saki-ri-k}} {\textit{runa}}\\
{leave-\textsc{refl-sbj.nmlz}} {\hspace{0.05cm}human}\\
\glt  ‘person who had stayed behind’ [qvz026:140--144]
\z
\z

\figref{fig:rice:1} and \REF{ex:rice:2} provide an example of what disfluency events look like in Northern Pastaza Kichwa in both the vocal-aural and visuo-spatial modalities. The speaker used two lexical items that are commonly deployed as fillers in the language: the demonstrative pronoun \textit{chi} and the discourse particle \textit{ña} (cf. \tabref{tab:rice:1} in \sectref{sec:rice:2.3}). The deployment of these lexical items as hesitative fillers is made obvious by their association with correlates of disfluency. In \sectref{sec:rice:4} I discuss these correlates of disfluency in greater detail using multimodal examples like the one shown here.

\section{Filler and pro-verb functions of \textit{mashti} in Northern Pastaza Kichwa}
\label{sec:rice:3}

In this section, I give an overview of the three different discourse functions of \textit{mashti} as represented in the corpus. Of the 157 total tokens in the corpus, the majority (75\%) are used as hesitative fillers. \tabref{tab:rice:4} gives the total counts of \textit{mashti} tokens by type.



\begin{table}
\begin{tabular}{lr}
\lsptoprule
\textit{mashti} type & Count\\
\midrule
hesitative filler & 119\\
placeholder filler & 22\\
pro-verb & 16\\
\lspbottomrule
\end{tabular}
\caption{\label{tab:rice:4} \textit{Mashti} tokens by type in corpus}
\end{table}

\figref{fig:rice:2} shows the counts of \textit{mashti} tokens by type and speaker. The speakers in \figref{fig:rice:2} are ordered by age at time of recording, from mid 30s (SV) to 80s (LC). Hesitative filler uses of \textit{mashti} (H) appear to be the most frequent use of \textit{mashti} for most speakers. However, as discussed in \sectref{sec:rice:2.2}, the corpus is not balanced in terms of recordings per speaker. For instance, speaker BD is overrepresented in this corpus compared to the others. Nevertheless, most speakers appear to use the hesitative filler \textit{mashti} more than the placeholder filler or pro-verb uses of \textit{mashti}.

  
\begin{figure}
\includegraphics[width=\textwidth]{fig2.png}
\caption{\label{fig:rice:2} Counts of \textit{mashti} type by speaker}
\end{figure}


I define the different types of \textit{mashti} by functional and referential criteria which tend to be reflected in the forms of \textit{mashti}. Filler instances of \textit{mashti} are distinct from pro-verb uses of \textit{mashti} in that fillers are used by the speaker to signal that they are experiencing disfluency. Pro-verb instances of \textit{mashti}, by contrast, do not signal hesitation and their principal function is to point to previously uttered propositions and events. The two types of filler \textit{mashti}~-- hesitatives and placeholders~-- are distinguished by whether or not they are syntactically integrated into the surrounding discourse. Hesitatives are not integrated, while placeholders are and, consequently, do not have additional morphology while placeholder uses of \textit{mashti} do. 

These functional and distributional generalizations are useful for categorizing most of the \textit{mashti} tokens in the corpus. However, there are some tokens that would seem to straddle the categories as set up here. In particular, there are two “undetermined” instances of \textit{mashti} in the corpus. I discuss these two instances in \sectref{sec:rice:3.1} along with the hesitative uses of \textit{mashti}. Placeholder and pro-verb uses of mashti are discussed in \sectref{sec:rice:3.2} and \sectref{sec:rice:3.3}, respectively. In \sectref{sec:rice:3.4} I consider the variation in the base form of \textit{mashti}, also produced as \textit{imashti}

\subsection{Hesitative filler \textit{mashti}}
\label{sec:rice:3.1}
As a hesitative, \textit{mashti} signals that the speaker is experiencing disfluency and serves to buy time while the speaker decides what to say next. The hesitative is not syntactically integrated into the surrounding utterance (\citealt{Hayashi2010}: 43). The example in \REF{ex:rice:3} comes from an elicitation session I had with a consultant, wherein I showed her a cartoon animation and asked her to recount the sequence of events. In \REF{ex:rice:3a}, she uses \textit{imashti} as a hesitative to buy time while she organizes her thoughts and prepares to recount the events of the cartoon animation. 

\ea \label{ex:rice:3}
\ea \label{ex:rice:3a}
\gll {kay} {pelikula-y} {riku-shka-nchi} {\textbf{imashtiii}} {\normalfont{(pause—1690ms)}}\\
{\textsc{prox.dem}}  {movie-\textsc{loc}}  {see-\textsc{perf-1pl}}   {\textbf{\textsc{hes}}\textbf{(475ms)}}\\
\glt ‘in this movie we have seen \textbf{ummm…}’
\medskip

\ex \label{ex:rice:3b} 
   \gll {misi-wan}  {alku-wan} {\normalfont{(pause—1536ms)}}\\
    {cat-\textsc{inst}}  {dog-\textsc{inst}} {}\\
    \glt ‘a cat with a dog’
\medskip

\ex \label{ex:rice:3c}
\gll {chiii} {\normalfont{(pause—1372ms)}}\\
{\textsc{dist.dem(804}ms)} {}\\
\glt ‘thaaat....’
\medskip

\ex \label{ex:rice:3d}
\gll {misi}  {allku-ta}  {wata-sha}\\
{cat}  {dog-\textsc{obj}}  {tie.up-\textsc{ss}}\\
\glt ‘the cat [was] tying up the dog’ [qvz010:22--25]
\z
\z

The \textit{imashti} in \REF{ex:rice:3a} does not occupy any syntactic slot in the utterance due to the distributional and morphological tendencies of lexical items in Quechuan languages. Like all other Quechuan languages, constituent order in Northern Pastaza Kichwa is flexible but tends towards SOV and modifiers like adjectives and adverbs precede their respective heads. Placed after the verb \textit{rikushkanchi} ‘we have seen’, \textit{imashti} is not acting as the subject, object, or modifier of any of the other constituents. Even if assuming non-standard word order, we can rule out the use of \textit{imashti} in \REF{ex:rice:3a} as a placeholder targeting an object on a morphological basis because objects in Quechuan languages~-- whether direct or indirect~-- are obligatorily marked with case suffixes.

There is also a prosodic case to be made for the hesitative classification of \textit{imashti} in \REF{ex:rice:3a}. It appears on the boundary of the intonational unit, and the production of the word itself is lengthened. Specifically, the final vowel is prolonged for 475ms. This is followed by a 1690ms pause before the next intonational unit.\footnote{Note also the disfluency episode in (3a-b) is replicated in (3c-d). In \REF{ex:rice:3c} the distal demonstrative \textit{chi} is used a hesitative and follows the same pattern as the \textit{imashti} of \REF{ex:rice:3a}. It is prosodically lengthened and followed by a pause before the target is acquired in \REF{ex:rice:3d}.} Prosodic differences between the different uses of \textit{mashti} are discussed in more detail in \sectref{sec:rice:4.1}, but prosodic lengthening is likely a key indicator of hesitative fillers compared to placeholder fillers. 

A hesitative \textit{mashti} can be more prosodically integrated into a phrase without being syntactically integrated. In \REF{ex:rice:4}, \textit{mashti} does not occur at either boundary of the intonational unit, it is not prosodically lengthened, and there are no associated pauses. Nevertheless, it is not syntactically integrated into the utterance. The \textit{mashti} “interrupts” the syntactic relationship between the proximal demonstrative \textit{kay} that points to the noun \textit{muyuwallata} ‘little fruit’. If this \textit{mashti} were a placeholder targeting the fruit, it would bear at least some of the same suffixes that \textit{muyu} ‘fruit’ bears. Thus, the \textit{mashti} in \REF{ex:rice:4} does not factor into the utterance’s syntax.

\ea%4
    \label{ex:rice:4}
    \gll {kay} {\textbf{mashti}} {muyu-wa=lla-ta}  {apa-sha}  {shamu-ra-ni} \\
    {\textsc{prox.dem}}  {\textbf{\textsc{hes}}}    {fruit-\textsc{dim=lim-obj}}  {take-\textsc{ss}}    {come-\textsc{pst}{}-1sg} \\
    
    \medskip
    
\gll {ni-shka-${\emptyset}$=lla}\\
	{say-\textsc{perf-3sg=lim}}\\
    \glt ‘“I came taking this, \textbf{um}, just this little fruit” he had just said’ [tds\_qvz036:89]
    \z

The presence or absence of morphology on a given instance of \textit{mashti} is, thus, a rule of thumb that can be used to distinguish the hesitative \textit{mashti} from the placeholder or proverb uses of \textit{mashti} if morphology is expected on the target. However, there are instances in which the degree of syntactic integration of \textit{mashti} is not clear. In such cases, additional contextual criteria may be more useful.

The following example shows one of the two \textit{mashti} tokens in the corpus which I defined as hesitative despite having additional morphology. The example in \REF{ex:rice:5} and the associated \figref{fig:rice:3} come from a recorded conversation I had a with speaker, wherein she taught me about various medicinal plants. In \REF{ex:rice:5a}, the speaker describes how the roots of a type of wild ginger can be processed to create a kind of shampoo. She ends the intonational unit in \REF{ex:rice:5a} with \textit{imashtiga}, which bears the topic enclitic \textit{=ga}. Apart from occurring at the boundary of the intonational unit, there are no other indicators of prosodic separation, such as lengthening or pauses. It is also not clear to which extent \textit{imashti} is syntactically integrated. The inclusion of the \textit{=ga} topic enclitic may indicate that \textit{imashti} is actually a placeholder that targets \textit{kaygunaga} ‘these ones’ in \REF{ex:rice:5b}, which also carries the topic enclitic. If so, this would be a case of partial mirroring instead of full mirroring (cf. \sectref{sec:rice:3.2}), because \textit{imashti} does not carry the plural suffix \textit{{}-guna} that the target demonstrative pronoun has.    

  
\begin{figure}
\includegraphics[width=\textwidth]{fig3.jpg}
\caption{\label{fig:rice:3} A speaker hesitates then points to a plant (qvz015, 9:38--9:44), photo © 2019 Alexander Rice.}
 \end{figure}

\ea%5
\label{ex:rice:5}
\ea \label{ex:rice:5a}
    \glll {} {} {} {} {} {\normalfont{(i)}}\\
    {\textit{chi-guna}}  {{\textit{m=a-u-n}}}    {\textit{sapi}}   {\textit{pay}}   {\textit{sapi-wan}} {\textbf{\textit{imashti=ga}}}\\
	{\textsc{dist.dem-pl}}  {\textsc{ev.s=cop-prog-3}}  {root}  {3}  {root-\textsc{inst}}  {\textbf{\textsc{?=top}}}\\
    \glt ‘they’re those ones, and [you mash] its root, [you prepare the shampoo] with its root, \textbf{um}’
\medskip

\ex   \label{ex:rice:5b}
\gllll {\normalfont{{\hspace{1cm}(ii)}}} \\
{********************************}\\
{\textit{kay-guna=ga {\hspace{1cm}} panga {\hspace{0.5cm}} panga}}\\
{\textsc{prox.dem-pl=top {\hspace{0.2cm}} {\normalfont{leaf}} {\hspace{1cm}}{\normalfont{leaf}}}}\\
\glt ‘these ones, [that] leaf, [that] leaf’
\medskip

\ex \label{ex:rice:5c}  
\gll {chi=was} {panga}  {m=a-naun}\\
\textsc{dist.dem=mir}    {leaf}  {\textsc{ev.s=cop-3pl}}\\
\glt   ‘[and] that leaf, those are the ones’ [qvz015:133--135]
\z
\z

The use of \textit{imashti} in \REF{ex:rice:5a} could also be interpreted as that of a hesitative filler. Evidence for this argument comes from the speaker’s gesture. In \REF{ex:rice:5a} and (i) of \figref{fig:rice:3}, the speaker has their hands together after performing a gesture mimicking the mashing movement to make the shampoo from the ginger roots. Her hands remain in this position while she utters \textit{imashti} in \REF{ex:rice:5a}. In \REF{ex:rice:5b} and (ii) of \figref{fig:rice:3}, she points to another cluster of ginger plants while uttering \textit{kaygunaga} ‘these ones’. Had she performed the gesture concurrently with, or closer to, her production of \textit{imashti} in \REF{ex:rice:5a}, that would have made a stronger case for this \textit{imashti} being a placeholder, because the gesture would clearly indicate the target of the placeholder. I lean towards interpreting the use of \textit{imashti} in \REF{ex:rice:5a} as a hesitative filler and represented it as “um” in the translation line.

The other use of \textit{mashti} that I determined to be a hesitative filler despite the presence of additional morphology is given in \REF{ex:rice:6}, which comes from the preamble of a traditional narrative. The speaker is talking about how her grandparents would tell stories to her and her siblings when she was a child. In \REF{ex:rice:6}, the speaker uses a \textit{mashti} that is prosodically integrated into the utterance, but there is no obvious syntactic relation it might have to other constituents.

\ea%6
\label{ex:rice:6}
\gll {chi-manda}  {apamama} {\textbf{mashti=shi}} {chasna} {chi-guna-ta}\\
{\textsc{dist.dem-abl}} {grandmother} {\textbf{?=\textsc{ev.o}}} {like.that}  {\textsc{dist.dem-pl-obj}}\\
\medskip

\gll {kwinta-k} {a-shka-una} {apamama}\\
{tell-\textsc{sbj.nmlz}}  {\textsc{cop-perf-3pl}}  {grandmother}\\
\glt ‘from there, grandmother [and grandfather] \textbf{um}, like that would tell those ones [stories]’ [tds\_qvz044:33]
\z

Curiously, this instance of \textit{mashti} also carries the evidential “other perspective” enclitic \textit{=shi}, which is used to mark information as coming from a perspective other than the speaker \citep{Nuck2018}.\footnote{Evidentials in Quechuan languages are often used for non-evidential functions (\citealt{Bendezu-Araujo2023}; \citealt{Grzech2016}; \citealt{Nuckolls2014a}).} There are no other constituents with the same enclitic in the utterance or the following intonational unit~-- which is not shown in \REF{ex:rice:6}~-- making this \textit{mashti} unlikely to be a placeholder. At the same time, it is not obvious what the addition of an evidential enclitic would contribute to a hesitative filler. The native speaker who translated this narrative represented this \textit{mashti} in her Spanish translation as \textit{este}, which is a Spanish demonstrative that is commonly deployed as a filler in Latin American and Amazonian Spanish \citep{Vallejos-Yopán2023}. Beyond that, the translator did not know what to make of \textit{mashtishi} in \REF{ex:rice:6}. It is possible that the inclusion of \textit{=shi} is a speech error.

\subsection{Placeholder \textit{mashti}}
\label{sec:rice:3.2}
Placeholder fillers are used in disfluency events as a substitute for a specific lexical item (the target), which is later produced when the disfluency is resolved. Unlike hesitative fillers, placeholder fillers are syntactically integrated into their surrounding utterances (\citealt{HayashiYoon2006}: 490). \tabref{tab:rice:5} shows the counts of the different targets of the placeholder uses of \textit{mashti}. The five independent placeholders that do not have a target are discussed below.

\begin{table}
\begin{tabular}{lr}
\lsptoprule
Target type & Count\\
\midrule
adposition & 1\\
converb & 3\\
deverbal noun & 1\\
none (independent) & 5\\
noun & 11\\
finite verb & 1\\
\lspbottomrule
\end{tabular}
\caption{\label{tab:rice:5} Target types of placeholder \textit{mashti} (\textit{N}=22)}
\end{table}

The placeholder filler use of \textit{mashti} is exemplified in \REF{ex:rice:7}, wherein the consultant is telling an “Amazonian” version of the biblical Noah’s Ark story. The characters in the story are building the ark by splitting the trunk of a \textit{tarapoto} tree. The consultant experiences a disfluency in recalling the name of a tree and uses \textit{mashti} in \REF{ex:rice:7a} as a substitute before acquiring and producing the name of the tree, which~-- as the target~-- is underlined in \REF{ex:rice:7b}.

\ea%7
\label{ex:rice:7}
\ea \label{ex:rice:7a}
\gll {kay} {\textbf{mashti-taaa}}    {{\normalfont{(pause—664ms)}}}\\
{\textsc{prox.dem}}  {{\textbf{\textsc{ph-obj}}}\textbf{(122ms)}} {} \\
\glt ‘this \textbf{whatchamacallit}’
\medskip

\ex \label{ex:rice:7b} \gll {\uline{taraputu-ta}}  {wakta-sha}\\
{tarapoto-\textsc{obj}}  {hit-\textsc{ss}}\\
\glt ‘\uline{[this] tarapoto [tree]}, [they were] hitting [it]’ [tds\_qvz044:59--60]
\z
\z

This use of \textit{mashti} is not prosodically integrated into the utterance, as shown by the prosodic lengthening and the introduction of a pause that splits the verbal construction between two intonational units. However, this \textit{mashti} is syntactically integrated into the utterance. The target of the \textit{mashti} placeholder~-- the \textit{tarapoto} tree,~-- is the object complement of the \textit{wakta-} ‘hit’ verb in \REF{ex:rice:5b} and thus carries the object suffix \textit{{}-ta}. The placeholder \textit{mashti} in \REF{ex:rice:5a} is likewise suffixed with the object suffix \textit{\nobreakdash-ta}, and the fact that \textit{mashti} is suffixed with the same case marker is an indicator of the syntactic integration of \textit{mashti} in the utterance as the object complement of \textit{waktasha} ‘hitting’. 

The phenomenon of a placeholder matching the grammatical shaping of the target word has been referred to as “mirroring” \citep[18]{Podlesskaya2010}. A placeholder filler can partially or fully mirror the eventually produced target word. The following example in \REF{ex:rice:8} shows a placeholder \textit{mashti} that fully mirrors the target lexical item. The \textit{mashti} in \REF{ex:rice:8b} includes not only the case suffix, but also the evidential enclitic \textit{=shi}.

\ea%8
\label{ex:rice:8}
\ea  \label{ex:rice:8a}  \gll {chiga}    {riku-kpi=ga}\\
{\textsc{dsc.prt}}  {see-\textsc{ds=top}}\\
\glt ‘and then seeing’
\medskip

\ex \label{ex:rice:8b}  \gll {\textbf{imashti-ta=shiii}}\\
{\textbf{\textsc{ph-obj=ev.o}}\textsc{(148}ms)}\\
\glt ‘{\textbf{an apparent whatchamacallit}}’
\medskip

\ex  \label{ex:rice:8c} \gll {ima} {\uline{yutu-ta=shi}} {intiru-ta} {yanu-shka-ta}\\
{what[false start]}  {{partridge-\textsc{obj=ev.o}}}  {whole-\textsc{obj}} {boil-\textsc{perf.nmlz-obj}}\\
\glt ‘\uline{a- an apparent partridge}, a whole one, a boiled one’ [qvz007:241--243]
\z
\z

The following example in \REF{ex:rice:9} shows instances of full and partial mirroring of \textit{mashti} placeholders. The first set consists of \textit{imashti} with the instrumental suffix \textit{{}-wan}: it fully mirrors the target noun \textit{sapi} ‘root’ that also bears the same instrumental suffix. The second \textit{mashti} placeholder \textit{mashtigunaga} partially matches the target nominal by bearing the same plural suffix but has a topic enclitic rather than the emphatic enclitic that is found on the target noun.       

\ea%9
\label{ex:rice:9}
\gll {mashti}  {pay-ba} {\textbf{imashti-wan}} {pay-ba}  {\uline{sapi-wan}} {\textbf{mashti-guna=ga}} \\
{\textsc{hes}} {3-\textsc{gen}} {\textbf{\textsc{ph-inst}}} {3-\textsc{gen}}  {root-\textsc{inst}} {\textbf{\textsc{ph-pl=top}}} \\
\medskip

\gll {\uline{runduma-guna=ya}}\\
{sedge-\textsc{pl=emp}}\\
\glt  ‘um, \textbf{with} \textbf{its} \textbf{whatchamacallit}, \uline{with its root} [of the] \textbf{whatchamacallits}, [of the] \uline{sedges}!’ [qvz015:131]
\z

\largerpage
The examples thus far show the placeholder \textit{mashti} targeting object nouns. The placeholder can also target other material like converbs and deverbal nouns as shown in \REF{ex:rice:10}. In \REF{ex:rice:10b}, the \textit{mashti} placeholder bears the same-subject suffix \textit{{}-sha,} which is suffixed to verbal roots that are used as converbs that head adverbial clauses. The \textit{{}-sha} same-subject suffix indicates the subject of the adverbial clause is the same as that of the following clause. The placeholder \textit{mashtisha} in \REF{ex:rice:10b} mirrors its target \textit{lluchusha} ‘peeling’ in \REF{ex:rice:10c}. Note also that the entire adverbial clause that stretches between \REF{ex:rice:10a} and \REF{ex:rice:10b} \textit{kayba karata mashtisha} ‘whatchamacalliting this one’s bark’ is reproduced with the target lexical item in \REF{ex:rice:10c} \textit{kayba karata lluchusha} ‘peeling this one’s bark’. Not only was the target word eventually produced, but the entire clause is repeated with the correct target word.\footnote{\citet[23]{Podlesskaya2010} refers to this phenomenon as “recycling”.} In \REF{ex:rice:10d} \textit{mashti} is used as a placeholder with the infinitive nominalizer \textit{{}-na} to hold the place for the deverbal noun \textit{raspana} ‘to grate’. 

\ea%10
\label{ex:rice:10}
\ea \label{ex:rice:10a} \gll {chiga} {chay-ta} {kay} \\
{\textsc{dsc.prt}}  {{\textsc{dist.dem-obj}}[false start]} {\textsc{prox.dem}[false start]}\\
\medskip

\gll {kay-ba} {kara-ta}\\
{\textsc{prox.dem-gen}}  {bark-\textsc{obj}}\\
\glt ‘and so, that one- this- this one’s bark’
\medskip

\ex \label{ex:rice:10b} \gll {\textbf{mashti-shaaa}} {\normalfont{(pause—997ms)}}\\
{\textbf{\textsc{ph-ss}}\textsc{(190}ms)} {}\\
\glt ‘\textbf{whatchamacalliting}’
\medskip

\ex  \label{ex:rice:10c}\gll {eee} {kay-ba} {kara-ta} {\uline{lluchu-sha}} \\
{\textsc{hes}}  {\textsc{prox.dem-gen}}  {bark-\textsc{obj}}  {peel-\textsc{ss}}\\
\medskip

\gll {apa-sha=mi} {\normalfont{(pause—932ms)}}\\
{take-\textsc{ss=ev.s}} {} \\
\glt  ‘um… this one’s bark, \uline{peeling} [it] and taking [it]
\medskip

\ex \label{ex:rice:10d}\gll {eee} {\textbf{mashti-naaa}} {a-n}  {\uline{raspa-na}}\\
{\textsc{hes}}  {{\textbf{\textsc{ph-inf.nmlz}}} \textsc{(118}ms)}    {\textsc{cop-3sg}}  {grate-\textsc{inf.nmlz}}\\
\glt  ‘um… \textbf{to} \textbf{whatchamallit}, \uline{to grate [it]}’ [tds\_qvz033:133--136]
\z\z

The example in \REF{ex:rice:10} also shows several overt signifiers of disfluency. Both uses of the \textit{mashti} placeholders are prosodically lengthened and occur near pauses. In \REF{ex:rice:10a}, there are two false starts, and non-lexical hesitative fillers are used in \REF{ex:rice:10c} and \REF{ex:rice:10d}. 

The previous examples show instances of \textit{mashti} being deployed as a placeholder filler, or a placeholder used in contexts of hesitation. However, not all placeholders are used in contexts of hesitation. Placeholders can be deployed for a variety of discourse functions \citep{Enfield2003}. Of the 22 placeholder uses of \textit{mashti} in the corpus, 18 are deployed as fillers in contexts of hesitation that have specific targets that are later produced in the same or a following utterance. The other four are used as “vague” or “imprecise” expressions that have no expressed target in discourse. These placeholders are used when the targeted entity or proposition is known to the speaker and interlocutor in the context of discourse and specific reference to that entity or proposition is not deemed necessary by the speaker \citep[26]{Podlesskaya2010}. I refer to these as “independent” placeholders. In \REF{ex:rice:11}, an independent placeholder is used during a speaker’s account of an experience from her life before she was married. The speaker recalls that her grandmother had accused the speaker and her cousins as being lazy and improper for not yet having husbands. The grandmother’s discourse is represented by the speaker in \REF{ex:rice:11a}-\REF{ex:rice:11c}.

\ea%11
\label{ex:rice:11}
\ea \label{ex:rice:11a}

\gll {warmi-wa-guna} {kasna} {kausa-nga} {ra-u-ngichi} {ni-sha=s}\\
{woman\textsc{{}-dim-pl}}  {like.this}  {live-\textsc{fut}}  {do-\textsc{prog-3pl}}  {say-\textsc{ss=conj}}\\
\glt ‘and [she’s] saying: “[as] young women you will live like this”'
\medskip

\ex \label{ex:rice:11b}
\gll {kungay=lla} \\
{peace=\textsc{lim}}\\
\glt `just peacefully'
\medskip

\ex \label{ex:rice:11c}
\gll {warmi} {m=a-ngichi} {yacha-na} {m=a-u-ngichi}\\
{woman} {\textsc{ev.s=cop-2pl}} {know-\textsc{inf.nmlz}} {\textsc{ev.s=cop-prog-2pl}}\\
\medskip

\gll {imasna} {\textbf{mashti-na-ta=s}} {ni-sha=s}\\
{how.many} {\textbf{\textsc{ph-inf.nmlz-obj=conj}}} {say-\textsc{ss=conj}}\\
\glt ‘she’s saying: “you all are women, you all need to know [how to do] \textbf{all} \textbf{kinds} \textbf{of} \textbf{stuff}”’
\medskip

\ex \label{ex:rice:11d}
\gll {ña} {chasna} {a-shka-y-bi} {ñuka}  {mama-ma}\\
\textsc{dsc.prt} {like.that}  {\textsc{cop-perf.nmlz-loc-loc}}  {1}  {mother-\textsc{all}}\\
\medskip

\gll {kuti=lla-ta} {bulltiya-ni}\\
{again=\textsc{lim-adv}}   {return-\textsc{1sg}}\\
\glt ‘so after it had been like that, I return again to my mother’ [qvz042:145--148]

\z
\z

In \REF{ex:rice:11c}, a placeholder \textit{mashti} occurs as head of the object complement of \textit{yachana maungichi} ‘you all need to know’.\footnote{In this example, the object follows the verb, which is a non-standard but acceptable ordering of constituents in Northern Pastaza Kichwa.}  The object marker indicates that this use of \textit{mashti} is syntactically integrated into the utterance. However, there is no target for this \textit{mashti}, neither before nor after \REF{ex:rice:11c}. Explicit reference to the target of this \textit{mashti} is not needed, because the context makes it clear that \textit{mashtinatas} refers to domestic chores that are often expected of women in Amazonian Ecuador.

In most cases, placeholder uses of \textit{mashti} are easy to detect because of the presence of morphology indicating syntactic integration and a following target in the case of placeholder fillers. However, placeholder fillers that target non-derived nouns in subject positions are more difficult to detect because there is no case marking or any kind of obligatory morphology that occur on subject nouns in Northern Pastaza Kichwa and in all other Quechuan languages (\citealt{Adelaar2004}: 213). Because of this, distinguishing between a hesitative filler \textit{mashti} and a placeholder filler \textit{mashti} that targets a subject noun can be difficult, as is shown in \REF{ex:rice:12}.

\ea%12
\label{ex:rice:12}
\ea \label{ex:rice:12a} 
\gll {aysa-shka} {m=a-ra-${\emptyset}$} {yaya}  {shuk}  {saltakama}\\
{pull-\textsc{perf.nmlz}} {\textsc{ev.s=cop-pst-3sg}} {father} {one}   {armoured.catfish}\\
\glt ‘one armoured catfish was pulled [caught] by my father’
\medskip

\ex \label{ex:rice:12b}
\gll {ishkay} \textbf{mashtiii} \uline{muta}\\
two   \textbf{\textsc{hes/ph}}\textsc{(74}ms)   muta.catfish\\
\glt ‘[and then] two, \textbf{um/whatchamacallits}, \textit{\uline{muta}} catfish’ [jbn\_qvz002:297--295]

\z
\z

In \REF{ex:rice:12a} the speaker lists the fish her father caught one day. In \REF{ex:rice:12b} she experiences disfluency and forgets the name of species of the two catfish that her father had caught. After the numeric determiner \textit{ishkay} ‘two’ she uses \textit{mashti}. This is followed by the target \textit{muta}~-- a type of catfish~-- which does not carry any additional morphology and, thus, this \textit{mashti} could be a placeholder that targets, and fully mirrors, \textit{muta}. The numeric determiner could then be modifying \textit{mashti} to produce an utterance akin to; “two whatchamacallitsss, \textit{muta} catfish” On the other hand, this \textit{mashti} may be a hesitative filler that does not participate in the utterance’s syntax. Prosodically, the \textit{mashti} in \REF{ex:rice:12b} is lengthened, which would appear to be more associated with hesitative fillers compared to placeholders, but 74ms only barely passes the threshold I establish for lengthened segments in this phonological context (cf. \sectref{sec:rice:4.1.3}).

My rule of thumb in which a \textit{mashti} lacking additional morphology is likely a hesitative filler may thus overcount hesitative filler instances of \textit{mashti} at the expense of placeholder filler uses of \textit{mashti}. This is an imperfect solution, and this analysis could be strengthened in future work by a more detailed examination of prosodic features of the fillers in this language and the extent to which placeholder fillers recycle their targets.

\subsection{Pro-verb uses of \textit{mashti}}
\label{sec:rice:3.3}
\citet[500]{HayashiYoon2006} observe that (demonstrative) placeholder fillers are necessarily referential in that they refer to a “yet-to-be-specified lexical item” and are frequently replaced by the targeted lexical item later in discourse. This means that placeholder fillers are forward-looking.\footnote{\citet[500]{HayashiYoon2006} note that while the forward looking nature of demonstrative placeholders makes them functionally similar to cataphora, the motivation is different. Placeholder fillers are motivated by hesitation and lexical retrieval, while cataphora is a deliberate discourse planning device.}  Some uses of \textit{mashti}, however, are not forward-looking, but instead look “backwards” at a previously uttered target. These uses of \textit{mashti} are integrated into the syntax of the surrounding utterance~-- as evidenced by the presence of verbal morphology~-- and serve anaphorically to reiterate a previously uttered action or event. Thus, I label this function of \textit{mashti} as a “resumptive pro-verb”. Pro-forms are semantically light forms that act as substitutes for more semantically substantive and discourse accessible elements (\citealt{Schachter2007}: 31). Thus, as a pro-verb, \textit{mashti} reiterates or summarizes a previously uttered predicate. \tabref{tab:rice:6} shows the anaphoric targets of the pro-verb uses of \textit{mashti} in the corpus. The majority target finite verbs and converbs.


\begin{table}
\begin{tabular}{lr}
\lsptoprule
\textit{Target type} & \textit{Count}\\
\midrule
deverbal noun & 1\\
clause & 3\\
converb & 5\\
finite verb & 7\\
\lspbottomrule
\end{tabular}
\caption{\label{tab:rice:6}Target types of placeholder \textit{mashti} (\textit{N}=16)}
\end{table}

A pro-verb \textit{mashti} targeting a converb is shown in \REF{ex:rice:13}, wherein the use of \textit{imashtisha} in \REF{ex:rice:13b} refers to the previously uttered adverbial clause headed by \textit{nanachisha} ‘hurting’ in \REF{ex:rice:13a}. Note that \textit{imashtisha} also mirrors \textit{nanachisha} by having the same-subject \textit{{}-sha} suffix.

\ea%13
    \label{ex:rice:13}
\ea \label{ex:rice:13a}
\gll {o} {sino=ga} {kay} {ishpa} {puru-ta} \uline{nana-chi-sha}\\
{or}  {otherwise=\textsc{top}} {\textsc{dist.dem}} {urine} {container-\textsc{obj}} {hurt-\textsc{caus-ss}}\\
\glt  ‘or rather, when the bladder is \uline{hurting}’
\medskip

\ex \label{ex:rice:13b}
\gll {kanser} {ni-k} {a-naun} {chi} {chasna} {\textbf{imashti-sha}} \\
{cancer} {say-\textsc{sbj.nmlz}} {\textsc{cop-3pl}}  {\textsc{dist.dem}}  {like.that}  {\textbf{\textsc{prov-ss}}}\\
\medskip

\gll \textit{wañu-shka-y} \\
{die-\textsc{perf.nmlz-loc}}\\
\glt ‘they call it “cancer” when, \textbf{happening[hurting]} like that, [one] is dying’ [qvz015:82]
\z
\z

The pro-verb \textit{mashti} can still point to a previous event without fully mirroring the verb. Consider the use of \textit{mashti} in \REF{ex:rice:14b} which does not mirror a specific lexical item in \REF{ex:rice:14a}. The \textit{mashti} in \REF{ex:rice:14b} is suffixed by the perfective \textit{{}-shka}, but there is no \textit{{}-shka} marked material in \REF{ex:rice:14a}. The pro-verb \textit{mashti} is referring to the overall predication of \REF{ex:rice:14a} as a completed action.

\ea%14
\label{ex:rice:14}
\ea \label{ex:rice:14a}
\gll {washa-manda} {bultiya-ri-sha} {\uline{kuti}} {\uline{pay-ta}} {\uline{randi}} \\
{after-\textsc{abl}}  {turn.around-\textsc{refl-ss}}  {again}  {3\textsc{{}-obj}}  {\textsc{coord.con}} \\
\medskip

\gll {\uline{yaku-ma}} {\uline{aysa-n}}  {@@@@@}\\
 {river-\textsc{all}} {pull{\textsc{-3sg}}}  {[laughter]}\\
\glt ‘after that, he’s turning around, but again he pulls him to the river, hahahaha’
\medskip

\ex \label{ex:rice:14b}
\gll {chasna} {\textbf{mashti-shka-${\emptyset}$}} {washa}\\
{like.that}  {{\textbf{\textsc{prov-perf}}}{\textbf{{}{\textsc{-3sg}}}}}    {after}\\
\glt ‘and after [he] had done [it] like that’ [qvz010:47--48]
\z
\z

Both instances of the pro-verb \textit{mashti} in \REF{ex:rice:13} and \REF{ex:rice:14} are modified by the manner adverb \textit{chasna} ‘like that’. Half of the 16 pro-verb \textit{mashti} tokens are modified by \textit{chasna} or another manner adverb \textit{kasna} ‘like this’. In such constructions, pro-verb uses of \textit{mashti} resemble manner demonstrative verbs, which~-- among other functions~-- make anaphoric reference to previously uttered discourse units while describing the way in which an event occurred \citep{Guérin2015}. This is congruent with the use of \textit{imashti} as a pro-verb with manner adverbs as demonstrated in \REF{ex:rice:13} and \REF{ex:rice:14}. However, without the adverbs, pro-verb \textit{mashti} simply makes reference to a previous event without specifying manner. Additionally, manner demonstrative verbs typically derive from demonstratives \citep[146]{Guérin2015}, whereas pro-verb \textit{mashti} derives from a noun phrase (cf. \sectref{sec:rice:5}).

The discourse distance between the utterance and the referring pro-verb \textit{mashti} can be quite large, as shown in \REF{ex:rice:15}. In (15a-b), the speaker recounts how she became friends with a female forest spirit and her husband. She is then about to continue the story in \REF{ex:rice:15e} when her interlocutor (TDS) interrupts and asks a question for clarification, which prompts a response in (15d-f) (transcription and interlinear gloss lines are omitted for brevity). Then in \REF{ex:rice:15g}, the speaker uses \textit{mashtishka} to refer to the events of the utterance in (15a-b) and thus gets the narrative “back on track”.

\ea%15
\label{ex:rice:15}
\ea \label{ex:rice:15a}
\gll {kunguri-sha} {kipiri-naku-sha}  {ña}\\
{kneel-\textsc{ss}} {hug-\textsc{rcpr-ss}} {then}\\
\glt    ‘kneeling, and hugging each other, then’
\medskip
    
\ex \label{ex:rice:15b}
\gll {\uline{amiga}} {\uline{tuku-ra-nchi}} {\uline{chi}} {\uline{pay}} {\uline{kari-wan=bas}}\\
{female.friend} {become-\textsc{pst-1pl}} {\textsc{dist.dem}}  {3}  {husband\textsc{{}-com=conj}}\\
\glt  ‘\uline{we became friends, and with her husband too}’
\medskip

\ex \label{ex:rice:15c}
\gll {chi} {washa} {ni-ra-${\emptyset}$}\\
{\textsc{dist.dem}} {after} {\textsc{say-pst-3sg}}\\
\glt  ‘after that she said:’
\medskip

TDS: with her husband?
\medskip

\ex \label{ex:rice:15d} ‘yes, with her husband’
\medskip

\ex  \label{ex:rice:15e}
‘the [spirit] woman and her husband’
\medskip

\ex  \label{ex:rice:15f}
‘her hair was down to here, the same as [my hair] down to here, she had long, full hair’
\medskip

\ex \label{ex:rice:15g}
\gll {chi-} {chi} {\textbf{mashti-shka-${\emptyset}$}} {washa}\\
{\textsc{dist.dem}[false start]} {\textsc{dist.dem}} {\textbf{\textsc{prov-perf-3sg}}} {after}\\
\glt  ‘that-, after that \textbf{had} \textbf{happened}’
\medskip

\ex \label{ex:rice:15h}
\gll {chi-manda} {apa-wa-ra-${\emptyset}$}\\
{\textsc{dist.dem-abl}} {take-\textsc{1.obj-pst-3sg}}\\
\glt  ‘she took me’ [tds\_qvz045:149--155]
\z
\z

The anaphoric nature of pro-verb \textit{mashti} means that it cannot be a placeholder filler used in contexts of hesitation and disfluency, if the primary functional criterion of such fillers is the hesitating/delaying function in the service of lexical recall. In the pro-verb uses of \textit{mashti} the referential target is uttered before the \textit{mashti} is and, thus, is not likely to signal hesitation in recalling a lexical item since the target was already produced and is thus more immediately accessible in the speaker’s memory.

The differences between a placeholder filler \textit{mashti} and a pro-verb \textit{mashti} is thus whether or not it occurs in an episode of disfluency, and the direction of reference. This can be sometimes difficult to tease apart. Consider \REF{ex:rice:16}, wherein a speaker was shown a cartoon animation and is asked to recount the events shown therein.\footnote{The cartoon animation in question was a public domain Tom and Jerry short (\citealt{Barbera1947}).} The \textit{mashtira}, as a finite verb in \REF{ex:rice:16b} could be a pro-verb referring back to the predicate, such as that of \REF{ex:rice:16a}: ‘the catfish followed the mouse’. Alternatively, \textit{mashtira} could be an independent placeholder, or a placeholder filler that buys time for the lexical retrieval of \textit{makara} ‘hit’ in \REF{ex:rice:16c}. In its morphology, \textit{mashtira} could be mirroring either the preceding \textit{katinakura} ‘[it] followed’ in \REF{ex:rice:16a} or the following \textit{makara} ‘[it] hit’ in \REF{ex:rice:16c}.

\ea%16
\label{ex:rice:16}
\ea  \label{ex:rice:16a} \gll {washa=lla}  {kati-naku-ra-${\emptyset}$} {\normalfont{(pause—632ms)}}  \\
{after=\textsc{lim}} {follow-\textsc{rcpr-pst-3sg}} {} \\
\medskip

\gll {chi} {atun} {bagri}\\
\textsc{dist.dem} {big} {catfish}\\
\glt ‘just after that, [he] followed [the mouse]… that big catfish’
\medskip

\ex \label{ex:rice:16b} \gll {\normalfont{(pause—1691ms)}} {chi} {misi} {a-u-shka-${\emptyset}$} {mayan-bi}\\
{} {\textsc{dist.dem}}  {cat}  {\textsc{cop-prog-perf-3sg}} {close-\textsc{loc}}\\
\medskip
        
\gll {\normalfont{(pause—699ms)}} {\textbf{mashti-ra-${\emptyset}$}}\\
{} {\textbf{\textsc{ph/prov-pst-3sg}}}\\
\glt ‘… close to where that cat was… \textbf{[the} \textbf{catfish]} \textbf{was} \textbf{there/did} \textbf{it’}
\medskip

\ex \label{ex:rice:16c}
\gll {\normalfont{(pause—2203ms)}} {chi} {bagri} {chupa-wan} {\uline{maka-ra-${\emptyset}$}}\\
{} {\textsc{dist.dem}} {catfish} {tail\textsc{{}-inst}} {hit-\textsc{pst-3sg}}\\
\glt‘… that catfish {\uline{hit}} [the cat] with its tail’ [qvz022:87--89]
\z\z


I presented \REF{ex:rice:16} as text and some contextual background information to a consultant and asked him what he thought \textit{mashtira} referred to. With only the text to work with, the consultant interpreted \REF{ex:rice:16b} as ‘[the catfish] was there close to where the cat was’, and specifically interpreted \textit{mashtira} as ‘[the catfish] was there’, and thus an independent placeholder. To another consultant, I presented an audio clip of \REF{ex:rice:16} with some contextual background information. She~-- by contrast~-- interpreted \textit{mashtira} as referring to \textit{makara} in \REF{ex:rice:16c}, and thus, as a placeholder filler.

Crucial to the difference in interpretations is the fact that one consultant had only the text to work with, while another had the audio clip. Note that in \REF{ex:rice:16} there are several pauses, some of which are around two seconds in length. Because the speaker was recounting the events of the cartoon animation she was just shown from memory, she likely experienced some difficulty in recalling exactly what occurred in the cartoon animation; hence the disfluency events. Thus, the second consultant who had access to the audio clip of \REF{ex:rice:16}, noticed the pauses which likely contributed to her interpretation of \textit{mashtira} as a placeholder. The first consultant, with only the textual representation of \REF{ex:rice:16} to work with (without indications of disfluency), assumed a non-hesitative use of \textit{mashtira}.  

As is shown by \REF{ex:rice:16} and the consultants’ interpretation of it, morphosyntactic context may not always be sufficient to distinguish between placeholder and pro-verb uses of \textit{mashti}. Pauses, as a correlate of disfluency, can be a useful factor. In \sectref{sec:rice:4} I expand further on this idea and present other correlates of disfluency that may assist in distinguishing between the different functions of \textit{mashti}.

\largerpage
\subsection{The form of \textit{mashti}}
\label{sec:rice:3.4}
Of the 157 tokens of \textit{mashti} in the corpus, 75\% take the form of \textit{mashti} (N=117) and the remaining 25\% as \textit{imashti} with an initial /i/ vowel (N=40). I have not determined what, if any, difference between the presence or lack of the initial /i/ vowel makes with regards to the function or distribution of \textit{mashti}. The examples in \REF{ex:rice:17} and \REF{ex:rice:18} come from the same speaker and same recording. In \REF{ex:rice:17} \textit{imashti} is used a hesitative filler, in \REF{ex:rice:18b} \textit{mashti} is also used as a hesitative filler.

\ea%17
\label{ex:rice:17}
\gll {chi} {ni-k} {a-nau-ra} {\textbf{imashti}} {amu-yuk}\\
{\textsc{dist.dem}} {say-\textsc{sbj.nmlz}} {\textsc{cop-3pl-pst}}  {\textbf{\textsc{hes}}} {owner-\textsc{prop}}\\
\glt ‘they call it, \textbf{um}, a (spirit) owner’ [tds\_qvz045:16]
\z


\ea%18
\label{ex:rice:18}
\ea \label{ex:rice:18a} 
\gll {chi} {punzhana-wa-ta} {api-sha=lla} {ni-ra-ni} {ñuka}\\
{\textsc{dist.dem}} {agouti-\textsc{dim-obj}} {catch-\textsc{1.fut-lim}}  {say\textsc{{}-pst-1sg}}  {1}\\
\glt   ‘“I will just catch that little agouti” I said’
\medskip

\ex \label{ex:rice:18b}
\gll {\textbf{mashti}}\\
\textbf{\textsc{hes}}\\
\glt \textbf{‘um’}
\medskip

\ex \label{ex:rice:18c} {\textit{kay}}\\
{\textsc{dist.dem}}\\
\glt ‘here’
\medskip

\ex \label{ex:rice:18d}
\gll {Montalvo-ma} {apa-mu-kpi}\\
{Montalvo-\textsc{all}}  {take\textsc{{}-trlc-ds}}\\
\glt ‘bringing it to Montalvo’ [tds\_qvz045:40--43]
\z
\z

The example given previously in \REF{ex:rice:9} (cf. \sectref{sec:rice:3.2}), shows \textit{mashti} and \textit{imashti} used as placeholders in the same utterance, which also comes from the same speaker and recording as \REF{ex:rice:13} and \REF{ex:rice:14}. The same speaker also uses both forms for pro-verbs in the same recordings. Compare \REF{ex:rice:15} in \sectref{sec:rice:3.3}, which uses \textit{mashti} as a pro-verb with the nominalized pro-verb \textit{imashti} in \REF{ex:rice:19}, which comes from the same recording.

\ea%19
\label{ex:rice:19}
\gll {chi-y} {kunguri-ra-nchi}  {kasna} \\
\textsc{dist.dem-loc}  {kneel-\textsc{pst-1pl}} {like.this} \\
\medskip

\gll {\textbf{imashti-shka-y}}\\
\textbf{\textsc{prov-perf.nmlz-loc}}\\
\glt ‘we kneeled there, and \textbf{when} \textbf{we} \textbf{were} \textbf{doing} \textbf{it} \textbf{[kneeling]} like this’ [tds\_qvz045:148]
\z


All of these examples come from the same speaker (BD), who uses \textit{mashti} and \textit{imashti} interchangeably. In fact, BD accounts for 38 of the 40 \textit{imashti} tokens, the other two come from LC, who also uses both forms. While I have not yet accounted for the variation between the two forms, the \textit{imashti} form with the initial /i/ likely represents the “older” or “original” variant which I discuss in \sectref{sec:rice:5}.

\section{Disfluency correlates of \textit{mashti}}
\label{sec:rice:4}
As shown in \sectref{sec:rice:3}, I generally consider uninflected instances of \textit{mashti} to be hesitative fillers, while inflected instances of \textit{masthi} can be categorized into placeholder filler (including use as anindependent placeholder), and pro-verb, based on the presence or absence of a referent and the direction in discourse to that referent (before or after the utterance of \textit{mashti}). However, these criteria are rough guidelines, as evidenced by several examples in \sectref{sec:rice:3}. In fact, the identification of a given use of \textit{mashti} in ambiguous morphosyntactic circumstances hinges on whether or not the speaker is experiencing a disfluency episode. This raises the following question: how do we know that the speaker is experiencing disfluency? This section explores the correlates of disfluency in the vocal-aural and visuo-spatial modalities and how those correlates interact with the use of \textit{mashti}.

For an initial illustration, consider the following examples in \figref{fig:rice:4} and \REF{ex:rice:20}, and \figref{fig:rice:5} and \REF{ex:rice:21}, which come from a recording session wherein a speaker is interpreting the designs in her and her sister’s ceramic art for two researchers. These examples show uninflected instances of \textit{mashti} that are ambiguous in terms of whether they are hesitative or placeholder fillers. However, the two examples are different in terms of the associated disfluency phenomena that co-occur with the use of the \textit{mashti} fillers.  

In still (i) of \figref{fig:rice:4} the speaker is holding a painted ceramic bowl that represents a type of plant and responds to questions from researchers about what the different painted patterns and shapes represent. In still (ii) of \figref{fig:rice:4}, she points to a painted flower in the center of the bowl but experiences a disfluency while trying to articulate that the arrangement of shapes represents a flower. In \REF{ex:rice:20a}, she produces a false start and \textit{mashti} during her disfluency episode. Then in \REF{ex:rice:20b} she resolves the disfluency and identifies the shape as the plant’s flower. The use of \textit{mashti} in \REF{ex:rice:20a} can be interpreted as either a hesitative filler, or as a placeholder targeting the uninflected noun \textit{sisa} ‘flower’ in \REF{ex:rice:20b}.

  
\begin{figure}
\includegraphics[width=\textwidth]{fig4.jpg}
\caption{\label{fig:rice:4} A speaker experiences disfluency while describing ceramic art (qvz017, 1:32--1:37), photo © 2019 Alexander Rice.}
\end{figure}


\ea%20 
\label{ex:rice:20}
\ea \label{ex:rice:20a}
\gl {{\hspace{5cm}{\normalfont{(ii)}}}}\\
\glt******************************************************
\gll {kay-} {kay} {\textbf{mashti}} {a-k=ka} \\
{{\textsc{prox.dem}}{\textbf{[false.start]}}} {\textsc{prox.dem}}  {\textbf{\textsc{hes/ph}}} {\textsc{cop-sbj.nmlz=top}} \\
\medskip

\glll*****\\
{a-u-n}\\
{\textsc{cop-prog-3sg}}\\
\glt ‘this- this \textbf{um/whatchamacallit} is being a…’\\
\medskip

\ex \label{ex:rice:20b}
\gl***************************\\
\gll{kay=ga} {pay-ba}  {\uline{sisa}}\\
{\textsc{prox.dem=top}}  {3-\textsc{gen}} {flower}\\
\glt    ‘this is its \uline{flower}’ [qvz017:23--24]
\z\z

The example in \figref{fig:rice:5} and \REF{ex:rice:21} is similar to the example of \figref{fig:rice:4} and \REF{ex:rice:21}. The speaker, with a different ceramic piece, is likewise pointing to one of the designs painted inside of the bowl and experiences a disfluency while explaining to the researcher what it represents. In this disfluency episode, the speaker uses an uninflected \textit{imashti} the final vowel of which is lengthened for 130ms. The use of \textit{imashti} here may be that of a hesitative filler, or a placeholder filler targeting the following uninflected noun \textit{apangura kiru} ‘crab claw’.

  
\begin{figure}
\includegraphics[width=\textwidth]{fig5.jpg}
\caption{\label{fig:rice:5} A speaker experiences disfluency while pointing to designs in ceramic art (qvz017, 1:58--2:01), photo © 2019 Alexander Rice.}
 \end{figure}


\ea%21
\label{ex:rice:21}
\gll {chi} {\textbf{imashtiii}} {apangura}  {kiru}  {m=a-u-n} \\
{\textsc{dist.dem}}  {\textbf{\textsc{hes/ph}}\textbf{(130ms)}}  {crab}    {claw}  {\textsc{ev.s=cop-prog-3sg}}\\
\medskip

\gll \textit{kay=ga}\\
\textsc{prox.dem=top}\\
\glt ‘that \textbf{um…/whatchamacallit}, is a crab claw, this one’ [qvz017:29]
\z

As is shown in \sectref{sec:rice:4.3}, some disfluency phenomenon like prosodic lengthening may be more associated with hesitative fillers compared to placeholder fillers and pro-verbs. Thus, the use of \textit{mashti} in \REF{ex:rice:20}~--  which is used without prosodic lengthening~-- may more likely be a placeholder filler compared to the \textit{mashti} in \REF{ex:rice:21}. 

This section goes over other correlates of disfluency in the interest of evaluating whether they can be used to better identify ambiguous uses of \textit{mashti} with regards to its function as filler or pro-verb, and hesitative versus placeholder. Crucially, this analysis is multimodal in that I consider disfluency phenomena articulated not only by the vocal tract but also as articulated by other parts of the body. \sectref{sec:rice:4.1} goes over the disfluency correlates of \textit{mashti} in the vocal-aural modality. \sectref{sec:rice:4.2} examines the visuo-spatial disfluency correlates of \textit{mashti}. In \sectref{sec:rice:4.3}, I look at how frequently each of the disfluency correlates occur with the different functions of \textit{mashti} and offer some discussion on the patterns observed therein.

\subsection{Disfluency correlates in the vocal-aural modality} 
\label{sec:rice:4.1}
The following disfluency correlates occur in the vocal-aural modality, which is to say is that the following disfluency phenomena are produced by the vocal tract: false starts (\sectref{sec:rice:4.1.1}), pauses (\sectref{sec:rice:4.1.2}), and prosodic lengthening (\sectref{sec:rice:4.1.3}).

\subsubsection{False starts}
\label{sec:rice:4.1.1}
\citet[710]{FoxTree1995} identifies false starts as a disfluency phenomenon that occurs “when speakers start to say something, but then decide to abort their utterances and begin again. \citet{FoxTree1995} refers to the aborted articulation as the “false start”, and the correctly articulated lexical item as the “fresh start”. In \REF{ex:rice:22}, the speaker begins to utter the word \textit{pasu} ‘mountain avocado’\footnote{The fruit of the tropical tree \textit{Gustavia macarenensis.}} and articulates the first syllable /pa/, which is then aborted. This is the false start. The word is fully articulated later as \textit{pasu}, and, thus, represents the fresh start.

\ea%22
\label{ex:rice:22}
\gll {\textbf{pa-}} {mashti} {\uline{pasu}} {\uline{muyu-ta}}  {ña}\\
{\textbf{[false start]}} {\textsc{hes}} {mountain.avocado} {fruit-\textsc{obj}} {\textsc{dsc.prt}}\\
\glt ‘\textbf{the} \textbf{pa-}, um, the \uline{\textit{pasu} fruit}’ [jbn\_qvz002:124]
\z

An uninflected \textit{mashti} follows the aborted articulation (false start) of \textit{pasu}. It is unlikely that this \textit{mashti} is a placeholder filler targeting \textit{muyu}, because the latter is inflected as an object, but it is possible that this \textit{mashti} is a placeholder targeting the noun \textit{pasu}, which acts as a modifier to \textit{muyu} ‘fruit’. \citet[59]{Navarretta2015} observes that hesitative fillers in Danish can occur in the same disfluency episode as false starts. Thus, the lack of clear syntactic integration plus the presence of a false start would make a better case for classifying this instance of \textit{mashti} as a hesitative filler instead of a placeholder filler.

\textit{Mashti} itself can also be subject to a false start. In \REF{ex:rice:23} the articulation of /ma/ is likely a false start of the hesitative \textit{mashti} that is correctly articulated afterwards.

\ea%23
\label{ex:rice:23}
\gll {\textbf{ma-}} {mashti} {wañu-shka} {mikya}\\
{\textbf{[false start]}} {\textsc{hes}} {die-\textsc{perf.nmlz}} {auntie}\\
\glt ‘uh- um, [my] deceased auntie’ [jbn\_qvz002:263]
    \z


Every instance of \textit{mashti} in the corpus was annotated for the presence or lack of a false start in the same intonational unit. The total counts are given in \tabref{tab:rice:7}.


\begin{table}
\begin{tabular}{lrr}
\lsptoprule
\textit{mashti} function & Occurs with false start & Occurs without false start\\
\midrule
hesitative filler & 10 & 109\\
placeholder filler & 0 & 22\\
pro-verb & 2 & 14\\
\lspbottomrule
\end{tabular}
\caption{\label{tab:rice:7}Occurrence of false starts with \textit{mashti} in corpus}
\end{table}

Overall, there are only 12 instances of false starts occurring in the proximity of \textit{mashti}. Most occurred with what I deemed to be hesitative fillers, while two occurred with pro-verbs.

\subsubsection{Pauses}
\label{sec:rice:4.1.2}
A silent pause~-- in which the speaker stops talking~-- is the “simplest way to hesitate” \citep[456]{Lickley2015}. However, as silence is a feature of fluent speech, it can be difficult to define how much silence constitutes a pause. Silence between intonational units in fluent speech may be longer than silence between words. \citet[456–457]{Lickley2015} gives an overview of the problems in distinguishing between “disfluent” pauses and “normal” pauses and shows that annotators make do with subjective perceptual judgement.\footnote{\citet[457]{Lickley2015} also points to making use of multiple annotators and estimating inter-rater reliability as the “safest approach” for identifying disfluent pauses. I did not take this approach for this research, due to the limited availability of Northern Pastaza Kichwa speakers and fieldwork funding constraints. Nevertheless, I acknowledge that future work in this vein would benefit greatly from including Northern Pastaza Kichwa speakers in annotating disfluency, and their ratings might differ from my own.} 

I likewise identified disfluent pauses based on my own subjective and perceptual judgement.\footnote{I am not a native speaker of Northern Pastaza Kichwa, but I do possess a moderate fluency in the language from having taken formal language classes in Ecuadorian Quechua and from working with Northern Pastaza Kichwa speakers for over a decade. I believe my ability to speak the language is sufficient enough to at least identify disfluent pauses.} The pauses I identified were between 500--2000ms in length. My identification of pauses also relied on the presence of other disfluency factors and overt signs of hesitation, which I discuss elsewhere in \sectref{sec:rice:4.2} (excessive blinking, gaze aversions, gesture freezes, etc.). I marked each instance of \textit{mashti} that occurred with a pause either in or adjacent to the \textit{mashti} bearing intonational unit.{\footnote{See \citet{chapters/ponsonnet} and \citet{chapters/rose} for the interaction between pauses and fillers/placeholders in Dalabon and Teko, respectively.}} An example is given in \REF{ex:rice:24b} where a pro-verb \textit{mashti} is followed by a 2.5 second pause.

\ea%24
\label{ex:rice:24}
\ea \label{ex:rice:24a} 
\gll {alku} {kishpi-sha} {ni-sha} {taula-guna-y} {api-ri-ra-${\emptyset}$}\\
{dog} {escape-\textsc{ss}} {say-\textsc{ss}} {plank-\textsc{pl-loc}}  {grab\textsc{{}-mid-pst-3sg}}\\
\glt ‘the dog, wanting to escape, grabs hold of the planks (at the dock)’
\medskip

\ex \label{ex:rice:24b}
\gll {chi} {aysa-sha} {\textbf{mashti-kpi}}  {\normalfont{{\textbf{(pause—2530ms)}}}} \\
{\textsc{dist.dem}} {pull-\textsc{ss}} {\textbf{\textsc{prov-ds}}} {} \\
\glt ‘and so pulling, \textbf{(he’s)} \textbf{doing} \textbf{that} while….’
\medskip

\ex \label{ex:rice:24c}
\gll {chi} {ukucha}\\
{\textsc{dist.dem}}  {mouse}\\
\glt ‘that mouse’
\medskip

\ex \label{ex:rice:24d}
\gll {ña} {mayan-bi} {a-u-shka-y} {pay-ta} \\
{\textsc{dsc.prt}}  {close-\textsc{loc}}  {\textsc{cop-prog-perf.nmlz}}  {3-\textsc{obj}} \\
\medskip

\gll {shuk} {kaspi-ta} {ku-gri-ra-${\emptyset}$}\\
{one}  {stick-\textsc{obj}} {give-\textsc{trlc-pst-3sg}}\\
\glt ‘being close by, went to give [the dog] a stick’ [qvz022:121--123]
\z\z

\tabref{tab:rice:8} shows how each of the \textit{mashti} functions occurs with a pause. Overall, there are more instance of pauses occurring with \textit{mashti} in the corpus than false starts, and hesitative uses of \textit{mashti} would appear to occur more with pauses than placeholders or pro-verbs do. This would point to the presence of pauses as a better indicator of disfluency compared to false starts. Although, as discussed in \sectref{sec:rice:4.3}, the distribution of the data severely limits quantitative assessments.

\begin{table}
\begin{tabular}{lrr}
\lsptoprule
\textit{mashti} function & Occurs with pause & Occurs without pause\\
\midrule
hesitative filler & 24 & 95\\
placeholder filler & 5 & 17\\
pro-verb & 1 & 15\\
\lspbottomrule
\end{tabular}
\caption{\label{tab:rice:8} Occurrence of pauses with \textit{mashti} in the corpus}
\end{table}

\subsubsection{Prosodic lengthening}
\label{sec:rice:4.1.3}
Prosodic lengthening is the non-phonemic prolongation of a syllable which can often occur in episodes of disfluency \citep[458]{Lickley2015}. In \REF{ex:rice:25b}, the final /i/ vowel of a hesitative filler \textit{mashti} is held for 418ms. 

\ea%25
\label{ex:rice:25}
\ea \label{ex:rice:25a}
\gll {chasna} {a-sha=mi} {pay-guna}  {libru-ta} \\
{like.that} {\textsc{cop-ss=ev.s}}  {3-\textsc{pl}} {book-\textsc{obj}}\\
\medskip

\gll {tiya-k-ta} {tupa-nau-ra}\\
{be-\textsc{sbj.nmlz-obj}} {find-\textsc{3pl-pst}}\\
\glt ‘being like that, they found the existing book’
\medskip

\ex \label{ex:rice:25b}
\gll {\textbf{mashtiii}} {\normalfont{(pause—7625ms)}}\\
{\textbf{\textsc{hes}}{\textbf{(418ms)}}} {}\\
\glt \textbf{“um...”}
\medskip

\ex \label{ex:rice:25c}
\gll {dos} {mil} {ocho}\\
{two} {thousand} {eight}\\
 \glt  ‘[in] 2008’ [qvz026:133--135]
\z\z

As is the case with silent pauses in disfluent speech versus fluent speech, determining whether the duration of a syllable in disfluent speech is actually “prolonged” is methodologically fraught \citep[458]{Lickley2015}.  As with the silent pauses, I used my own subjective and perceptual judgement to determine whether or not a given instance of \textit{mashti} was prosodically lengthened. However, I corroborated this approach by comparing the durations of the final vowel in each \textit{mashti} token with a sample of durations of the same vowel in a similar phonetic environment but in a non-hesitative context. This was especially necessary given that placeholder and pro-verb uses of \textit{mashti} can take a variety of suffixes and enclitics. The durations of non-prolonged and prolonged final /i/ vowels may be different than that of final /a/ vowels. Some suffixes attached to placeholder and pro-verb uses of \textit{mashti} also possess coda consonants and thus need to be compared to other phonetically similar material with the same coda consonant.

There are 18 different phonemic forms that represent the word final syllables of all the \textit{mashti} tokens in the corpus. To define the duration for non-prolonged vowels and codas of these syllables I semi-randomly selected ten words for each syllable type in contexts that I perceived to be fluent speech and measured the duration of the final vowel and coda consonant (if present). I then used the average of the ten values as the baseline non-prolonged duration. These baseline values are given in \tabref{tab:rice:9}.\footnote{The reference words I used and the individual durations can be found in the “ref\_durations.xlsx” document in the OSF drive: \url{https://osf.io/g84mv/}. The reference words were selected semi-randomly. In general I opted for words that occurred at the end of intonational units, which is the position that many but not all \textit{mashti} tokens occur in. I also tried to vary speakers as much as possible.}

\begin{table}
\begin{tabular}{lr}
\lsptoprule
Word final syllable & Average duration of final vowel and coda (ms)\\
\midrule
/t͡ʃun/ & 98.3\\
/da/ & 97.5\\
/ga/ & 111\\
/ka/ & 104.8\\
/kpi/ & 120.8\\
/na/ & 87.9\\
/ni/ & 80.9\\
/ɾa/ & 76.3\\
/ʃa/ & 123.5\\
/ʃi/ & 108.5\\
/ʃkai̯/ & 148.2\\
/ta/ & 94.9\\
/tas/ & 166.4\\
/ti/ & 73.7\\
/wa/ & 78.4\\
/wan/ & 144.3\\
/was/ & 211.2\\
/wi/ & 91.9\\
\lspbottomrule
\end{tabular}
\caption{\label{tab:rice:9} Reference durations for word final phonemic sequences in fluent speech}
\end{table}

I then compared the duration of the final syllable of each \textit{mashti} token with the corresponding value in \tabref{tab:rice:9}. If the duration was greater than the baseline value, I marked it as prolonged. To illustrate, the sampled average duration for the /a/ vowel in a word final /wa/ syllable is 78.4ms. In \REF{ex:rice:26} the placeholder filler \textit{imashti} carries the same-subject purpose clause suffix -\textit{ngawa}, the final syllable of which is /wa/. The duration of the /a/ vowel is 119ms, which is longer than the 78.4ms baseline, and is thus marked as “prolonged”.


\ea%26
\label{ex:rice:26}
\ea \label{ex:rice:26a}
\gll {mana} {chasna} {\textbf{imashti-ngawaaa}}\\
{\textsc{neg.intj}} {like.that} {\textbf{\textsc{ph-ss.purp}}\textbf{(119ms)}}\\
\glt ‘no, in order to \textbf{whatchamacallllit} like that’
\medskip

\ex  \label{ex:rice:26b}
\gll {ali-} {ali-chi-ngawa} {mana} {pay} {sagra-shka-${\emptyset}$}\\
{[false.start]} {good-\textsc{caus-ss.purp}} {\textsc{neg}} {3} {curse-\textsc{perf-3sg}}\\
\glt  ‘fix- in order to fix the curse that was done’ [qvz015:17--18]
\z
\z

\tabref{tab:rice:10} shows how many tokens of each \textit{mashti} function were prosodically lengthened. As is the case with silent pauses, the hesitative filler use of \textit{imashti} would appear to attract prosodic lengthening more so than the placeholder filler or pro-verb uses of \textit{mashti}. In fact, the majority of hesitative \textit{mashti} tokens were prosodically lengthened.


\begin{table}
\fittable{
\begin{tabular}{lrr}
\lsptoprule
\textit{mashti} function & With prosodic lengthening & Without prosodic lengthening\\
\midrule
hesitative filler & 98 & 21\\
placeholder filler & 13 & 9\\
pro-verb & 5 & 11\\
\lspbottomrule
\end{tabular}
}
\caption{\label{tab:rice:10} Occurrence of prosodic lengthening in \textit{mashti} tokens in the corpus}
\end{table}

\subsection{Visuo-spatial modality correlates of disfluency}
\label{sec:rice:4.2}

The following disfluency correlates occur in the visuo-spatial modality, which involves the use of bodily articulators outside of the vocal tract: the use of gaze (\sectref{sec:rice:4.2.1}), instances of excessive blinking (\sectref{sec:rice:4.2.2}), the use of manual gestures (\sectref{sec:rice:4.2.3}), and gestural disfluency (\sectref{sec:rice:4.2.4}).

\subsubsection{Gaze aversion}
\label{sec:rice:4.2.1}
Gaze and head movement are an active component of speech and serve a variety of functions such as turn management, attention signalling, attention-checking, and viewpoint shifting (\citealt{Rossano2012}; \citealt{Sweetser2016a}; \citealt{Thompson2014}). With regards to disfluency events, a shift in gaze away from the addressee (gaze aversion) can co-occur with a word search (\citealt{Goodwin1986}) and can invite/disinvite assistance from the interlocutor \citep{Hayashi2003body}. In particular, prior research has shown that gaze aversions often co-occur with fillers as a means of “holding the floor” during disfluency episodes (\citealt{Brone2017}; \citealt{Feyaerts2017}).

Speakers of Northern Pastaza Kichwa likewise use gaze aversions and fillers during disfluency events, as is shown in \figref{fig:rice:6} and \REF{ex:rice:27}. In \REF{ex:rice:27a} and (i) of \figref{fig:rice:6}, the speaker’s gaze is directed to the left (her right) whilst narrating. She then experiences a disfluency when attempting to introduce a new character to the narrative. The disfluency is initiated by using the discourse particle \textit{ña} as a hesitative filler and a pause in \REF{ex:rice:27a}. At the same time, she abruptly moves her gaze to the right (her left) and holds it there while hesitating, as shown in (ii) of \figref{fig:rice:6}. The use of the repeated “===” symbols in \REF{ex:rice:27} shows the approximate duration of the speaker’s gaze relative to discourse, and the carets show the approximate direction relative to the camera. The averted gaze co-occurs with the use of the hesitative filler \textit{mashti} in \REF{ex:rice:27b}. In \REF{ex:rice:27c} and (iii) of \figref{fig:rice:6}, the speaker recalls the name of the character she wants to introduce and moves her gaze back to the left (her right) to her interlocutor.

  
\begin{figure}
\includegraphics[height=.9\textheight]{fig6.jpg}
\caption{\label{fig:rice:6} A speaker averts her gaze while hesitating (qvz026, 5:06--5:11), photo © 2019 Alexander Rice}
\end{figure}

\ea%27
\label{ex:rice:27}
\ea  \label{ex:rice:27a}
\gllll {\normalfont{{(i)}}} {\normalfont{{(ii)}}} \\
{} {==============>} \\
{\textit{{chi-manda=mi}}} {{\textit{ña} {\hspace{1mm}}{\normalfont{{(pause—820ms)}}}}}\\
{\textsc{dist.dem-abl=ev.s}} {\textsc{dsc.prt}}\\
\glt ‘and from there, um…’
\medskip

\ex  \label{ex:rice:27b}
\glll {===>} \\
{\textbf{\textit{mashti}}} \\
{\textbf{\textsc{hes}}} \\
\glt \textbf{‘um’}
\medskip

\ex \label{ex:rice:27c}
\gl {{\normalfont{(iii)}}} \\
\glt <======== \\
\gll {Andrea} {Dagua} \\
{Andrea} {Dagua} \\
\glt ‘Andrea Dagua’ [qvz026:101--103]
\z\z

I annotated gaze aversion as an abrupt movement of the eyes and head away from one direction towards another. I did not measure the duration of each averted gaze, only whether or not it occurred, and I only annotated those that occurred in the presence of \textit{mashti}, either in the same intonational unit, or in an adjacent intonational unit (before or after) in the same disfluency event. \tabref{tab:rice:11} shows how many \textit{mashti} tokens of each type co-occurred with a gaze aversion. Across all types, gaze aversion. Overall, each type of \textit{mashti} occurred more with gaze aversion than without.


\begin{table}
\begin{tabular}{lrr}
\lsptoprule
\textit{mashti} function & With gaze aversion & Without gaze aversion\\
\midrule
hesitative filler & 81 & 37\\
placeholder filler & 16 & 6\\
pro-verb & 10 & 6\\
\lspbottomrule
\end{tabular}
\caption{\label{tab:rice:11}Occurrence of gaze aversions with \textit{mashti} tokens}
\end{table}

\subsubsection{Excessive blinking}
\label{sec:rice:4.2.2}
Research in neuroscience has shown that blinking can serve communicative functions in speech (\citealt{Mandel2014, Nakano2010}). With regards to the interaction between disfluency and blinking, \citet[382]{Lickley2017} briefly mentions that excessive blinking can be associated with stuttering.  \citet{Homke2017} show that addressees, when faced with speakers experiencing disfluency, can use long blinks to signal to the speaker that repair is not needed and additionally signal a disinterest in “taking the floor”. Beyond these studies, the interaction between disfluency and the blinking of the speaker has not yet been explored in detail as far I am aware. 

In the present corpus, disfluency episodes are sometimes accompanied by multiple blinks performed in rapid succession. I annotated instances of \textit{mashti} as being associated with excessive blinking when two or more subsequent blinks were performed in vicinity of the articulation of \textit{mashti}, either in the same or adjacent intonational unit in the same disfluency event. In \REF{ex:rice:24}, the speaker blinks four times while uttering \textit{mashti.} The “\^{}” symbol above the transcription line indicates one blink. Repeated blinking is somewhat difficult to show in a succinct way with static images so, \figref{fig:rice:7} shows only one of the four blinking sequences performed concurrently with the uttering of \textit{mashti} in \REF{ex:rice:28}. 

  
\begin{figure}
\includegraphics[width=\textwidth]{fig7.jpg}
\caption{\label{fig:rice:7} A speaker blinks several times in quick succession while uttering \textit{mashti} (lwc\_qvz002, 0:59--1:03), photo © 2016 Lisa Warren Carney.}
\end{figure}

\ea%28
\label{ex:rice:28}
\glll {} {} {\textbf{\^{}\^{}\^{}\^{}}} {} {}\\
{\textit{chi-ta}} {\textit{riku-sha}} {\textbf{\textit{mashtiii}}} {\textit{chi-ta}} {\textit{musku-k}}\\
{\textsc{dist.dem-obj}} {see-\textsc{ss}}    {\textbf{\textsc{hes}}\textbf{(395ms)}} {\textsc{dist.dem-obj}}  {dream\textsc{{}-sbj.nmlz}}\\
\medskip

\gll {likcha-ri-ra-ni}\\
{wake.up-\textsc{refl-pst-1sg}}\\
\glt ‘seeing that, \textbf{um,} dreaming (about) that, I woke up’ [lwc\_qvz002:13]
\z

\tabref{tab:rice:12} shows how many \textit{mashti} tokens occurred in the presence of an excessive blinking episode. Overall, excessive blinking does appear to occur as frequently in disfluency events involving \textit{mashti} as gaze aversions and prosodic lengthening.

\begin{table}
\begin{tabular}{lrr}
\lsptoprule
\textit{mashti} function & With excessive blinking & Without excessive blinking\\
\midrule
hesitative filler & 29 & 87\\
placeholder filler & 1 & 22\\
pro-verb & 2 & 14\\
\lspbottomrule
\end{tabular}
\caption{\label{tab:rice:12} Occurrence of excessive blinking with \textit{mashti} in corpus}
\end{table}

\subsubsection{Manual gestures}
\label{sec:rice:4.2.3}
Research in psychology and neuroscience has shown that manual gestures often occur during disfluency events, and more specifically, some have argued that manual gestures may aid in lexical retrieval (\citealt{Butterworth1978}; \citealt{Rauscher1996}; \citealt{Krauss1999}; \citealt{Pyers2021}; \citealt{Osorio2023}). Research in this vein has come under the banner of the “Lexical Retrieval Hypothesis” which, generally, posits that manual gestures facilitate lexical retrieval. This would suggest that filler words like \textit{mashti} and manual gestures might co-occur more often than not. 

The Lexical Retrieval Hypothesis is not a consensus position, however: other studies have presented challenges for some of its assumptions. For example, Christenfeld, Schachter, and Bilous \citeyear{Christenfeld1991} show that the use of a gesture during a disfluency episode may preclude the use of a filler word. Other research has challenged the Lexical Retrieval Hypothesis more directly in finding that gestures are not especially prevalent in disfluent speech versus fluent speech (\citealt{Graziano2018}; \citealt{Hoetjes2014}; \citealt{Kisa2022}).

In the interest of gaining additional insight into the relationship between gesture and disfluency, I marked each use of \textit{mashti} that contained a co-occurring manual gesture. One such example is given in \figref{fig:rice:8} and \REF{ex:rice:29}, wherein the speaker is describing how characters in a narrative made spears. She uses a gesture wherein her left arm is extended out in front her torso with the left palm oriented upwards. Her right hand is held in a loose cupping shape and she moves her right hand up and down over her left hand repeatedly. In the context of spear-making, this gesture is likely an iconic gesture wherein the speaker’s left hand represents the piece of wood being shaped into a spear, and the right hand depicts a hand holding a sharp instrument such as a knife or other scraping tool. The repeated movement of the right hand over the left thus represents a process of chamfering or scraping the piece of wood to create the shaft of the spear. The speaker begins to use the gesture in still (i) of \figref{fig:rice:8} and \REF{ex:rice:29b} and continues to perform the gesture when she experiences a disfluency and pauses after uttering \textit{sumakshi} ‘reportedly nice/nicely’.\footnote{\textit{Sumak} ‘nice/beautiful’ can modify both nouns and predicates, thus the aborted utterance may have been something like; “they had made reportedly nice spears” or “nicely they had made spears, reportedly”.} During the pause in speech, she continues to perform the spear-making gesture. 

  
\begin{figure}
\includegraphics[width=\textwidth]{fig8.jpg}
 \caption{\label{fig:rice:8} A speaker uses a manual gesture depicting the chamfering of a wooden spear shaft while hesitating in recalling the word for "spear" (jbn\_qvz003, 2:27--2:37), photo © 2015 Alexander Rice.}
\end{figure}

\ea%29
\label{ex:rice:29}
\ea   \label{ex:rice:29a}
\gll {shina} {piti-sha}\\
{like.this}  {cut-\textsc{ss}}\\
\glt ‘cutting like this’
\medskip

\ex  \label{ex:rice:29b}
\gllll {\normalfont{(i)}}\\
*************************\\
{{\textit{sumak=shi{\hspace{1mm}}}{\normalfont{(pause—642ms)}}}} \\
{nice=\textsc{ev.o}}\\
\glt  ‘nicely, reportedly…’
\medskip

\ex \label{ex:rice:29c} 
\glllll  {\normalfont{(ii)}} {}\\
{******} {\hspace{14mm}*******}\\
{} {˅==========}\\
{\textit{pay-guna}} {\normalfont{(pause—966ms)}}\\
{3-\textsc{pl}} {}\\
\glt  ‘they…’
\medskip

\ex  \label{ex:rice:29d}
\gl {\normalfont{(iii)}} {\hspace{5mm}\normalfont{(iv)}}\\
\glt **********************************\\
\glll {====>} {} {} \\
{\textbf{\textit{mashti}}} {\textit{lansa-ta}}  {\textit{{ra-shka-una}}}\\
{\textbf{\textsc{hes}}} {spear-\textsc{obj}} {do-\textsc{perf-3pl}}\\
\glt ‘um, they had made spears’ [jbn\_qvz003:31--35]
\z
\z

In \REF{ex:rice:29c}, she starts the narration again, continuing to perform the gesture, then pauses her speech and freezes her gesture. In (ii) of \figref{fig:rice:8}, she averts her gaze from her interlocutor to look down at her hands. Then she returns her gaze to her interlocutor and starts the gesture again, while still staying silent. In (iii) of \figref{fig:rice:8} and \REF{ex:rice:29d} she utters a hesitative filler \textit{mashti} and averts her gaze to the right (her left) while continuing to perform the manual gesture. Finally, in (iv) of \figref{fig:rice:8} and \REF{ex:rice:29d} she returns her gaze to her interlocutor and escapes the disfluency event by uttering \textit{lansata rashkauna} ‘they had made spears’.

In \figref{fig:rice:8} and \REF{ex:rice:29}, it would appear that the speaker’s hands essentially “knew” what the target utterance was before the speaker’s vocal tract did.\footnote{See \citet{chapters/doehler} for similar examples of gestures accompanying the place holder {\textit{bäne}} in Komnzo.} The spear-making gesture starts approximately 4.5 seconds before the speaker resolves the disfluency and articulates the spear-making event. Unable to recall a word or experiencing difficulty in articulating a phrase, a speaker may thus appear to use their hands to help “capture” or “organize” the target material, as is argued by the Lexical Retrieval Hypothesis. Additionally, gesture use during a disfluency may serve as an additional turn holding device and communicate the target material to the interlocutor visually.  

\tabref{tab:rice:13} shows the \textit{mashti} tokens that were uttered concurrently with a manual gesture. Overall, roughly half of the \textit{mashti} tokens for each category occurred with a manual gesture.

\begin{table}
\begin{tabular}{lrr}
\lsptoprule
\textit{mashti} function & Uttered with gesture & Uttered without gesture\\
\midrule
hesitative filler & 75 & 43\\
placeholder filler & 11 & 11\\
pro-verb & 9 & 7\\
\lspbottomrule
\end{tabular}
\caption{\label{tab:rice:13}The occurrence of manual gestures with \textit{mashti} in the corpus}
\end{table}

I should note, however, that I did not take into the account the types of gestures that occurred with \textit{mashti}. Nor does this data weigh in on the validity of the Lexical Retrieval Hypothesis since the rate or types of gestures in fluent speech were not considered. Here I am only interested in whether one of the different functions of \textit{mashti} co-occurs with manual gestures compared to the others. As is shown in \tabref{tab:rice:13}, the presence or absence of a gesture is roughly equal between the different \textit{mashti} types. 

\subsubsection{Gestural disfluency}
\label{sec:rice:4.2.4}

Like speech, manual gestures are also subject to disfluency effects \citep{Seyfeddinipur2006}. In an experimental setting, \citet{Betz2023} observed several of the following types of gestural disfluency, including cancellation (aborting a gesture), pauses (holds), slow-downs, and metaphoric gestures. In the corpus data, I found three types of manual gesture disfluency that occurred in proximity to a \textit{mashti} token.

\begin{itemize}
\item
Deactivation: The end of a manual gesture happens to coincide with a disfluency episode involving \textit{mashti} and no gestures are used until the disfluency is resolved.

\item 
Hold: A gesture is paused or “frozen” mid-stroke as a speaker experiences a disfluency episode. The gesture resumes when the disfluency is resolved.

\item 
Cycle: A gesture in which the hand or finger traces a circular path continuously while the speaker experiences a disfluency episode. The speaker stops the gesture or switches to a new gesture when the disfluency is resolved.

\end{itemize}

An example of gestural deactivation is given in \figref{fig:rice:9} and \REF{ex:rice:30}. In (i) and (ii) of \figref{fig:rice:9}, the speaker is actively gesturing while telling a story. In \REF{ex:rice:30b} she experiences disfluency and stops gesturing as shown by her hand dropping to her side in (iii) of \figref{fig:rice:9}. The gestural deactivation is concurrent with the use of the hesitative \textit{mashti} and the pause in speech in \REF{ex:rice:30b}. Note also that during the pause, no gestures are used while the speaker hesitates. Thus, there is a “gestural pause” parallel to the prosodic pause. When the disfluency is resolved in \REF{ex:rice:30c} the speaker begins gesturing again.

  
\begin{figure}
\includegraphics[width=.8\textwidth]{fig9.jpg}
 \caption{\label{fig:rice:9}A speaker deactivates her gesture while experiencing a disfluency episode (tds\_qvz034, 5:23--5:29), photo © 2015 Alexander Rice.}
\end{figure}

\ea%30
\label{ex:rice:30}
\ea \label{ex:rice:30a}
\gllll {} {\normalfont{(i)}} \\
{} {{\hspace{7mm}****}}\\
{\textit{chi}} {\textit{kaya-ndi}}\\
{\textsc{dist.dem}} {tomorrow-\textsc{add}}\\
\glt ‘and then the next day’
\medskip

\ex  \label{ex:rice:30b}
\gl {\normalfont{(ii)}} {\hspace{15mm}{\normalfont{(iii)}}} \\
\glt *********************\\
\gll {ri-ra-nchi} {\textbf{mashti}} {\normalfont{(pause—710ms)}}\\
{go-\textsc{pst-1pl}} {\textbf{\textsc{hes}}} {}\\
\glt ‘we went, \textbf{um…}’
\medskip

\ex      \label{ex:rice:30c}
\gl {} {\hspace{5mm}{\normalfont{(iv)}}}\\
\glt ********************************\\
\gll {riku-kpi} {chi} {muriti-y}\\
{see-\textsc{ds}} {\textsc{dist.dem}} {peach.palm.grove-\textsc{loc}}\\
\glt ‘and there in that peach palm grove we saw:’ [tds\_qvz045:72--74]
\z
\z

\figref{fig:rice:10} and \REF{ex:rice:31} show an example of a gesture hold associated with disfluency. The speaker tells how some characters in a narrative ran to a large tree and spreads his arms out to show the relative width of the trunk in (ii) of \figref{fig:rice:10}. However, the speaker also experiences a disfluency in recalling the name of the tree in \REF{ex:rice:31b}. While experiencing the disfluency, he holds the spread-arms gesture in (ii) of \figref{fig:rice:10} for nearly three seconds while struggling to recall the name of the tree in (\ref{ex:rice:31b}-\ref{ex:rice:31c}). In (iii) of \figref{fig:rice:10} and \REF{ex:rice:31d}, the speaker raises his hands and head slightly, which is another indicator of disfluency \citep{Ozkan2023}. The speaker then resolves the disfluency and recalls the name of the tree in \REF{ex:rice:31d} and lowers his arms in (iv) of \figref{fig:rice:10} to resume the spread-arm gesture indicating the width of the tree trunk.

  
\begin{figure}
\includegraphics[width=.8\textwidth]{fig10.jpg}
\caption{\label{fig:rice:10}A speaker holds a gesture while experiencing a disfluency (tds\_qvz031 9:46--10:50), photo © 2015 Alexander Rice.}
\end{figure}

\ea%31
    \label{ex:rice:31}
\ea \label{ex:rice:31a}
\gl {\hspace{22mm}{\normalfont{(i)}}} \\
\glt {\hspace{22mm}}{**************************************}\\
\gll {chasna} {ña} {chaulla-ta} {miku-sha} {ri-shka-ta} \\
{like.that} {then} {\textsc{ideo}:clean-\textsc{adv}} {eat-\textsc{ss}}   {go-\textsc{perf.nmlz-adv}}\\
\medskip

\gl********************\\
\gll {apa-naku-sha} {ri-kpi=ga}\\
{take-\textsc{rcpr-ss}} {go-\textsc{ds=top}}\\
\glt ‘then like that, having cleanly eaten (everything), they had gone, chasing (them)’
\medskip


\ex \label{ex:rice:31b}
\gl {\normalfont{(ii)}} \\
\glt********************************************(hold—2942ms)\\
\gll {atun} {\textbf{mashti}} {kay} {\normalfont{(pause—1019ms)}}\\
{big} {\textbf{\textsc{hes/ph}}} {\textsc{prox.dem}} {}\\
\glt ‘(there is) a big, umm, this…’
\medskip

\ex \label{ex:rice:31c}
\gl {*********************}\\
\gll {ima} {m=a-n} {kay}\\
{what} {\textsc{ev.s=cop-3sg}} {\textsc{prox.dem}}\\
\glt ‘what is this (one)?’
\medskip

\ex \label{ex:rice:31d}
\gll {\normalfont{(iii)}} {\hspace{3mm}{\normalfont{(iv)}}} \\
{******************} {\hspace{3mm}{***********}} \\ \linebreak 
\gll {ñuka-nchi} {ni-nchi} {wayra}  {kaspi} \\
{1-1\textsc{pl}} {say-\textsc{1pl}} {wind} {tree} \\
\glt ‘we call it “the wind tree”’ [tds\_qvz031:197--200]
\z
\z

\citet{Ladewig2011} shows that German speakers make use of repeated circular movement gestures (cycle gestures) with their fingers or hands in contexts of “word searches” (i.e. disfluency). Such cycle gestures would be considered as “metaphoric” gestures by \citet{Betz2023}. The form of the gesture~-- the circular motion of a hand or finger~-- is thought to represent “the continuous nature of the searching activity” \citep[8]{Ladewig2011}. 

 I take such circular gestures in the context of disfluency as additional indicators of disfluency in Northern Pastaza Kichwa discourse. \figref{fig:rice:11} and \REF{ex:rice:32} show a speaker using a cyclic gesture concurrently with the hesitative \textit{mashti}. The example comes from a historical narrative about the speaker’s ethnic background as a member of the Andwa ethnic group.\footnote{The Andwa territory is located on the Bobonaza river near the Ecuadorian military base in Montalvo of the Pastaza province of Ecuador. Historically the people in this territory spoke a now dormant Zaparoan language (probably Andoa [ISO 639-3: anb], endonym: Katsakáti).} In \REF{ex:rice:32}, she describes the conditions of one of the smallpox epidemics that occurred at the turn of the twentieth century and, more specifically, the smallpox symptoms that the Andwa people experienced. The speaker explains that the people often had to go naked because the pustules associated with smallpox that covered the body made it painful and unpleasant to wear clothes. More specifically, the pustules would stick to the clothes, and upon changing clothes, the pustules would be painfully ripped off the skin.

The speaker describes this by pointing to the back of her hand in (i) of \figref{fig:rice:11} to use her own skin as an example while describing the situation with the clothes in \REF{ex:rice:32a}. However, she experiences a disfluency in \REF{ex:rice:32b} which is characterized by the use of a hesitative \textit{mashti} with prosodic prolongation and a silent pause. While uttering \textit{mashti} and pausing, she circles her index finger on her hand twice while hesitating, which I take to be a cyclic gesture. In \REF{ex:rice:32c} she resolves the disfluency and moves ahead with the planned utterance and gesture of tapping her skin with her index finger to show how the pustules would stick.

  
\begin{figure}
\includegraphics[width=.8\textwidth]{fig11.jpg}
\caption{\label{fig:rice:11}A speaker cycles her finger index while hesitating (qvz026, 2:48--2:58), photo © 2019 Alexander Rice.}
\end{figure}

\ea%32
\label{ex:rice:32}
\ea \label{ex:rice:32a}
\gll {\hspace{65mm}{\normalfont{(i)}}}\\
{\hspace{65mm}{************************}}\\ \linebreak
\gll {dinu=ga} {kay} {llachapa-ta} {chura-ri-kpi=ga} {llachapa=ma}\\
{rather\textsc{=top}} {\textsc{prox.dem}} {clothes-\textsc{obj}} {put-\textsc{refl-ds=top}} {clothes-\textsc{all}}\\
\medskip

\glll {*****}\\
{\textit{kasna}}\\
{like.this}\\
\glt ‘rather, when putting on their clothes (it would stick) to their clothes’
\medskip

\ex \label{ex:rice:32b}
\gll {\normalfont{(ii)}} \\
{****************(2 repetitions)}\\ \linebreak
\gll \textbf{\textit{mashtiii}} {\normalfont{(pause—743ms)}}\\
    \textbf{\textsc{hes}}\textbf{(480ms)}\\
\glt    \textbf{‘um…’}
\medskip

\ex \label{ex:rice:32c}
\gl {\hspace{15mm}{\normalfont{(iii)}}} {\hspace{3mm}{\normalfont{(iv)}}}\\
\glt {\hspace{15mm}{*************}} \\
\gll {aglluña-sha=ga}  {tyapi} {tyapi=shi}  \\
{stick-\textsc{ss=top}} {\textsc{ideo}:sticking} {\textsc{ideo}:sticking=\textsc{ev.o}} \\
\medskip

\gll {api-ri-k} {a-shka-${\emptyset}$}\\
{grab-\textsc{refl-sbj.nmlz}} {\textsc{cop-perf-3sg}}\\
\glt ‘sticking, it was reportedly like \textit{tyapi! tyapi!} [the smallpox pustules stick to the clothes and are ripped off when clothes are taken off]’ [qvz026:55--57]
\z\z

\tabref{tab:rice:14} shows the \textit{mashti} tokens that were uttered while the speaker experienced a gestural disfluency. Note that the total sets of tokens are less than those of the other disfluency correlate counts. This is because for a gestural disfluency to occur with \textit{mashti}, there had to be a gesture present in the first place. Thus, the tokens in \tabref{tab:rice:14} represent a subset of the ‘Uttered with gesture’ tokens in \tabref{tab:rice:13}. 



\begin{table}
\begin{tabular}{lrr}
\lsptoprule
\textit{mashti} function & With gestural disfluency & Without gestural disfluency\\
\midrule
hesitative filler & 48 & 27\\
placeholder filler & 3 & 8\\
pro-verb & 0 & 9\\
\lspbottomrule
\end{tabular}
\caption{\label{tab:rice:14}Occurrence of \textit{mashti} with gestural disfluency}
\end{table}

\subsection{Summary}
\label{sec:rice:4.3}
\tabref{tab:rice:15} shows the frequency of all the considered multimodal indicators of disfluency with the different \textit{mashti} functions expressed as percentages. Due to the low \textit{N} values of the placeholder and pro-verb uses of \textit{mashti} the differences in percentages between the groups may be spurious. Additionally, recall that I use the correlates of disfluency to categorize ambiguous instances of \textit{mashti} (cf. \sectref{sec:rice:3} and \sectref{sec:rice:4}). Thus, evaluating the co-occurrence between the type of \textit{mashti} and the disfluency correlates is somewhat circular, because~-- for some \textit{mashti} tokens~-- the category was not assigned independently of the correlates. Nevertheless, there are some patterns in the data that can serve as jumping-off points for forming hypotheses for future investigations with more data and the use of independent and categorical criteria.


\begin{table}
\begin{tabular}{l rrr}
\lsptoprule
{\textit{mashti}} function & H ($N=119$) & PH ($N=22$) & PRO-V ($N=16$) \\
\midrule
False start                & 8\%       & 0         & 12\%\\
Pause                      & 20\%      & 22\%      & 6\% \\
Pros. lengthening          & 82\%      & 59\%      & 31\%\\
Gaze aversion              & 68\%      & 72\%      & 62\%\\
Excessive blinking         & 24\%      & 4\%       & 12\%\\
Gesture                    & 63\%      & 50\%      & 56\%\\
Gestural disfluency           & 40\%      & 13\%      & 0\\
\lspbottomrule
\end{tabular}
\caption{\label{tab:rice:15}Frequency of disfluency correlates with \textit{mashti} functions}
\end{table}

Looking across the columns, we can see that prosodic lengthening, gaze aversion, gesture, and gestural disfluency co-occur with the filler uses of \textit{mashti} more frequently than false starts, pauses, and excessive blinking. Upon comparing the \textit{mashti} categories (rows) it would appear that prosodic lengthening, excessive blinking, and gestural disfluency occur more with the hesitative filler \textit{mashti} compared with the placeholder and pro-verb uses of \textit{mashti}.\footnote{Note that a subset of the placeholder \textit{mashti} tokens are independent placeholders and not necessarily associated with disfluency like placeholder fillers are (cf. \sectref{sec:rice:3.2}). With more tokens, it would be instructive to compare independent placeholders to the other categories.} Keeping in mind the low \textit{N} value of the pro-verb \textit{mashti} tokens, gaze aversion and gesture would appear to occur more frequently in the pro-verb context than false starts, pauses, prosodic lengthening, excessive blinking, and gestural disfluency. 

The differences between the frequencies shown in \tabref{tab:rice:15} cannot be statistically tested due to the low \textit{N} values of the placeholder and pro-verb instances of \textit{mashti}. Additionally, the nature of the data presents the challenge of multicollinearity, because the tokens are drawn from a small number of speakers and some of the potential variables depend on other variables (i.e. gestural disfluency requires that a gesture is performed in the first place). 

With the caveat that the proportions of the multimodal disfluency correlates of \textit{mashti} as shown in \tabref{tab:rice:15} are likely not statistically significant, the suggestion that prosodic lengthening may be part of what distinguishes between the filler and non-filler uses of \textit{mashti} is not unwarranted. \citet{chapters/rose} shows that the (hesitative) filler \textit{baʔe} in Teko is more lengthened than the generic noun \textit{baʔe} ‘thing/non-human’.\footnote{Tupi-Guaranian, French Guiana [ISO 639-3: eme].} \citet{Vallejos-Yopán2023} also finds that the filler use of \textit{este} as a hesitative filler in Amazonian Spanish is more associated with prosodic lengthening compared to the use of \textit{este} as a demonstrative.\footnote{However, \citet{HenneckeMihatsch2022} find that using prosodic cues to disambiguate filler and placeholder functions in the French filler/placeholders \textit{truc} and \textit{machin} is problematic.} 

The results presented in this section provide the basis for some hypotheses that can be tested in future work with a larger corpus with higher \textit{N} values for placeholder and pro-verb uses of \textit{mashti}. Is the presence or absence of prosodic lengthening predictive of the filler or pro-verb use of \textit{mashti}? Is the same true of gestural disfluency or the other multimodal disfluency correlates?  If the trends presented here were to hold for a larger dataset and could be tested, I would argue overall that the case of \textit{mashti} in Northern Pastaza Kichwa shows that multimodality plays a bigger role in language function than is usually appreciated. \citet{Nuckolls2020e} shows that manual gestures are an essential component of defining ideophonic words in Northern Pastaza Kichwa. It may the be case that bodily indicators such as gaze aversion and gestural disfluency may be essential in distinguishing the different functions of \textit{mashti}.

\section{The origin and developmental pathways of \textit{mashti}}
\label{sec:rice:5}

In the previous sections, I have discussed the different linguistic functions of \textit{mashti} in discourse (as a hesitative filler, placeholder filler, and pro-verb), as well as the multimodal disfluency phenomena that may play a role in distinguishing between the discourse functions of \textit{mashti}. In this section I discuss the origin of \textit{mashti} and present the potential pathways of how \textit{mashti} became associated with the disparate functions it now has.

The use of \textit{mashti} as a filler is attested in other varieties of the Colombian-Ecuadorian branch of Quechua. \citet{Grzech2016a} attests of \textit{mashti} as a hesitative and placeholder filler in Upper Napo Kichwa~-- the sister variety of Northern Pastaza Kichwa~-- and it is used in varieties of Ecuadorian Highland Kichwa, Lower Pastaza Quechua (Peru), and Inga (Colombia) (p.c. Simeon Floyd, 2022). The form \textit{imasti} is also attested as a filler word in Santiago del Estero Quichua (Argentina) (\citealt{AlbarracindeAlderetes2009}), which suggests that \textit{mashti} has been used as filler word for centuries, at least as far back as pre-Spanish contact times in the fifteenth century.\footnote{The “Quechua II” branch includes northern and southern most extensions of Quechuan languages (in Colombia, Ecuador, Northern Peru, Southern Peru, Bolivia, Chile and Argentina), which correspond to the outermost regions of the former Inca Empire. The languages of the Quechua I branch inhabit the central and southern parts of Peru. The development and spread of the Quechua II varieties is thought to have occurred with the last major expansion of the Inca empire from Cusco in the early fifteenth century roughly a generation before Spanish colonizers arrived. The spread of Quechua into Southern Bolivia (and later into Argentina) is thought to have been approximately simultaneous with the arrival of Quechua in Ecuador. Thus, if \textit{mashti} is used in Quechuan varieties as far south as Argentina and as far north as Colombia, we can surmise that \textit{mashti} was used as filler as least as far back as this split in the fifteenth century. It does not appear that \textit{mashti} is used in the Quechua I varieties (p.c. Simeon Floyd, 2022)} \footnote{Consult \citet[180--191]{Adelaar2004} for an overview of the spread and development of the Quechuan language family.} The oldest attested source of \textit{mashti} in any Quechuan variety that I am aware of comes from a nineteenth century dictionary of a variety of Ecuadorian Quechua spoken in the Azuay province of Ecuador \citep{Cordero1892},\footnote{This Quechua variety would correspond to the present day Cañari Kichwa, a variety of Ecuadorian Highland Kichwa.} wherein \textit{imasti} and other, similar forms are listed with definitions that suggest filler functions. A scan of the page containing these entries is given in \figref{fig:rice:12}.\footnote{Credit goes to Simeon Floyd for teasing out the geographic distribution of \textit{mashti} as well as pointing out Cordero’s dictionary to me.}

  
\begin{figure}
\includegraphics[width=.8\textwidth]{fig12.png}
\caption{\label{fig:rice:12}Excerpt from Cordero's dictionary attesting the use of \textit{imasti} in Ecuadorian Highland Kichwa in the nineteenth century \citep[40]{Cordero1892}.}
\end{figure}

The English translations for the relevant entries in Cordero’s dictionary are as follows (English translations are my own):

\begin{itemize}
\item \textit{imasti}, n. “This, that, this one, whose name I cannot recall.”
\item \textit{imashuti}, n. “That; that one whose name escapes my memory; but which I vaguely indicate.”
\item \textit{imashutina,} v. “Doing something which I do not remember the name of in this instant; but I would like to make intelligible in some way.”
\end{itemize}

Some initial observations about the origin of \textit{mashti} can be drawn from these dictionary entries. The definition of the form \textit{imashuti} is very close to that of \textit{imasti}. The form \textit{imashuti} is transparently composed of the interrogative pronoun \textit{ima} ‘what’ and \textit{shuti} ‘name’. In the present Northern Pastaza Kichwa corpus, \textit{ima} and \textit{shuti} occur together only once and are used to ask someone’s name, as shown in \REF{ex:rice:30bis}. Coincidentally, the example in \REF{ex:rice:30bis} also has a hesitative \textit{imashti}, showing that \textit{ima shuti} and \textit{mashti} are now distinct lexical items.

\ea%30
\label{ex:rice:30bis}
\gll {\textbf{{ima}}} {\textbf{shuti=ta}} {a-ra-${\emptyset}$} {pay} {amu}  {kwinta-u-n=mi}  {ña}  \\
{\textbf{what}} {\textbf{name=\textsc{int}}} {\textsc{cop-pst-3sg}} {3} {owner} {tell-\textsc{prog-3sg=ev.s}} {\textsc{dsc.prt}}\\
\medskip

\gll {\textbf{imashti}} {chasna} {sukta} {wata-yuk} {a-sha} {pay=ga} {kasna=lla} \\
{\textbf{\textsc{hes}}} {like.that} {six} {year-\textsc{prop}} {\textsc{cop-ss}} {3=\textsc{top}}    {like.this=\textsc{lim}}\\
\medskip

\gll {kwinta-g} {a-ra-${\emptyset}$}\\
{tell-\textsc{sbj.nmlz}}  {\textsc{cop-pst-3sg}}\\
\glt \textbf{‘“what} \textbf{was} \textbf{[his]} \textbf{name?”} [we asked] “he says [that] he is the master” so \textbf{um}, like that, being six years old he would just tell [stories] like this’ [lwc\_qvz001:25]
\z

Thus, it is reasonable to conclude that the \textit{imasti} in \citegen[40]{Cordero1892} dictionary derives from \textit{ima shuti} ‘what name’, which~-- given its lexical semantic content of asking for a name~--  is roughly equivalent to the English ‘whatchamacallit’ and makes for a prime source from which a placeholder can be derived. 

Cordero’s dictionary also has an entry for a deverbal \textit{imashutina},\footnote{The \textit{{}-na} suffix is an infinitive nominalizer in the Colombian-Ecuadorian varieties of Quechua and is often used as the 'citation form' for a given verb.} which indicates that at least as far back as the nineteenth century \textit{ima shuti} ‘what name’ could serve as a verbal placeholder in addition to a nominal placeholder. The given definition “doing that for which I do not remember the name of in this instant” makes the placeholder function of \textit{imashutina} obvious but does not indicate if it was used as a pro-verb.

Concerning the historical derivation of the different functions of \textit{mashti}, the lexical phrase \textit{ima shuti} ‘what name’ clearly represents the starting point from which \textit{mashti} and its functions derived. Lexicalized constructions involving an interrogative and a noun are a common source of placeholders \citep[13]{Podlesskaya2010}. From this starting point, \textit{ima shuti} was phonetically reduced and took on new functions: placeholder filler, independent placeholder, hesitative filler, and pro-verb. A diagram representing this process is given in \figref{fig:rice:13}. The independent placeholder and placeholder filler functions are rolled into a single “placeholder” category for this discussion.

\begin{figure}[h]
% % % \includegraphics[width=0.75\textwidth]{fig13.jpg}
\begin{tikzpicture}[->, >={Triangle[]}]
\begin{scope}[minimum height=3\baselineskip, minimum width=2.5cm, rounded corners=5pt, align=center]
    \node[draw, align=center] (ima) {\textit{ima shuti}\\`what name'};
    \node[draw, right=1cm of ima] (placeholder) {placeholder};
    \node[draw, above right=2mm and 1cm of placeholder.east] (hesitative) {hesitative\\filler};
    \node[draw, below right=2mm and 1cm of placeholder.east] (verb) {pro-verb};
\end{scope}
\draw (ima) -> (placeholder);
\draw (placeholder.east) to [out=0, in=180] (hesitative);
\draw (placeholder.east) to [out=0, in=180] (verb);
\end{tikzpicture}
\caption{\label{fig:rice:13}Proposed development of \textit{mashti} functions in Northern Pastaza Kichwa}
\end{figure}

At the outset, the \textit{ima shuti} ‘what name’ construction was likely employed in “whatchamacallit” type contexts of disfluency and underwent a process of phonetic erosion (as evidenced by the loss of the /u/ vowel). Semantic bleaching was likely another factor as the propositional content (the interrogative force and the “name” noun) was lost over time. 

As a watchamacallit type placeholder, \textit{mashti} then took on new functions as a hesitative filler and pro-verb. To derive hesitative fillers, \citet{Heine2013} proposes a process of “cooptation” whereby lexical content is “co-opted” and redeployed at the “thetical” level of discourse (i.e. not syntactically integrated). Thus, \textit{mashti}~-- as a placeholder filler commonly deployed in situations of hesitation and disfluency~-- would have taken on a new role as a more general hesitative filler without needing to be syntactically integrated into the utterance.

For the pro-verb use of \textit{mahsti}, I propose that the placeholder filler \textit{mashti} is undergoing a process of (re)lexicalization. As I show in \sectref{sec:rice:3.3} and \sectref{sec:rice:4.3}, the pro-verb \textit{mashti} is less prone to occurring in contexts of disfluency and it does not target a yet unuttered referent. Instead, pro-verb \textit{imashti} makes anaphoric reference to a previously uttered predicate event similar to the English ‘do’. This represents an instance of lexicalization, wherein a new meaning that is paired with the form \textit{mashti} and is readily interpretable as a holistic unit (in this case, a verbal root) \citep{Lehmann2002}. Crucially, the new pro-verb meaning is more semantically specific than that of a placeholder. Lexicalization “repackages” a form with a new meaning. The new meaning is still semantically “light” as is the case with pro-forms and manner demonstrative verbs \citep{Guérin2015}, but it is somewhat more filled-in compared to a placeholder.

This brief discussion on the development of the different functions of \textit{mashti} represents cursory speculation on my part. A more dedicated diachronic analysis of the development of \textit{mashti} and fillers in general would make for a fruitful direction for future work.

\section{Summary and conclusions}
\label{sec:rice:6}

In this chapter I have shown how \textit{mashti} is used in Northern Pastaza Kichwa as a hesitative filler, placeholder filler, and resumptive pro-verb. Each category can be defined by their function and how they interact with surrounding discourse. Both the hesitative and placeholder filler uses of \textit{mashti} signal that the speaker is experiencing a disfluency event and serve to buy time to reorient their position in discourse. The hesitative filler \textit{mashti} is not syntactically integrated into its surrounding utterance. The placeholder filler \textit{mashti}, by contrast, is syntactically integrated into the surrounding utterance and mirrors the target lexical item that is eventually produced when the disfluency has been resolved. The pro-verb \textit{mashti} is not a filler. Instead, it targets a previously uttered predicate to reiterate and remind the interlocutor of a previously mentioned event.

I have also shown that there are ambiguous cases wherein the morphosyntactic properties of a given instance of \textit{mashti} may not be enough to identify its function. To this end, this analysis considers multimodal phenomena to help classify morphosyntactically ambiguous instances of \textit{mashti}. I show that multimodal elements such as pauses, false starts, prosodic lengthening, gaze aversion, and gesture are important correlates in identifying disfluency episodes, and I raise the possibility that such elements could be used to disambiguate the different functions of \textit{mashti} in a larger dataset. I have also discussed how \textit{mashti} likely arose from a lexical phrase and speculated on the developmental pathways that gave rise to the various extant functions of \textit{mashti} in Northern Pastaza Kichwa. 

There remain many outstanding questions and avenues of future investigation. First, a more careful consideration of the prosodic characteristics of disfluency such as the relevance of intonational unit boundaries and the lengths of hesitations and pauses might be useful in determining the function of uninflected instances of \textit{mashti}. Second, a larger corpus with more tokens of placeholder and pro-verb uses would be useful, given that 75\% of the tokens fall into the category of hesitatives in the present corpus. Third, there is some circular logic in my analysis, in that I analyzed the association between the correlates of disfluency and the different functions of \textit{mashti} while simultaneously using the correlates of disfluency to categorize the different functions of \textit{mashti} in the first place. A better approach would involve using a subset of a larger corpus whereby unambiguous tokens of \textit{mashti} could be categorized independently by morphosyntactic criteria and then the association between the categories and co-occurring multimodal phenomena could be better evaluated.

The development of the form of \textit{mashti} itself also warrants further investigation. I do not have an explanation for the seemingly optional presence of the initial /i/ vowel in some cases, nor for why in some other Quechuan varieties the form is \textit{masti} with an /s/ consonant instead of /ʃ/. It would also be useful to investigate whether \textit{mashti} in other varieties of Quechua shows the same tripartite range of functions that it possesses in Northern Pastaza Kichwa. Additionally, since the use of \textit{mashti} as a filler appears to be limited to varieties of the Quechua II branch, it would be worth investigating whether the Quechua I languages use it, and if not, what material is deployed as fillers in those varieties.

Finally, it is worth considering a broader view of the use of other fillers in Northern Pastaza Kichwa. In \sectref{sec:rice:2.3}, I listed the other items that are deployed as fillers in Northern Pastaza Kichwa. It would be illuminative to investigate how often these fillers are deployed in discourse and whether or not their functions and distributions mirror those of \textit{mashti}. 

This chapter thus contributes to the typological research on fillers and placeholders by describing a multifunctional filler in an Indigenous South American language, through a novel approach that takes into account the multimodal and prosodic signals associated to its different discourse functions. As stated in \sectref{sec:rice:1}, the investigation of “peripheral” linguistic phenomena like fillers is often ignored in language documentation and description. However, the study of said peripheral linguistic phenomena has important considerations for minority language revitalization and pedagogy. To date, minority language pedagogy has largely focused on samples of decontextualized and monologic language that show regular paradigms and predictable patterns of use. Such focus on “well-behaved” instances of language ignores the importance of metalinguistic awareness and conversation, which is the epicentre of language use (S. \citealt{Rice2021}). Indigenous minority languages in particular are often situated in primarily oral contexts, wherein face-to-face interaction is the place where language “lives” (S. \citealt{Rice2017}). Fillers and other “peripheral” linguistic phenomena are an essential part of interactional language use \citep{Ameka1992}. The documentation and description thereof is thus necessary in the pursuit of pedagogical development oriented towards natural and authentic language use. In other words, learning to be “fluently disfluent” should be an essential step in a minority language pedagogy program.


\section*{Acknowledgements}

This research was made possible by funding from the US Department of Education’s Foreign Language and Area Studies program (FLAS), and the Endangered Language foundation. Special thanks goes to the Amazonian Kichwa communities in Pastaza Ecuador for their time, stories, and friendship. In particular, I thank Bélgica Dagua Toqueton, Norma Ruiz Cadena, and Edwing Wilfrido Nuñez for their stellar translation work and insights. Lastly, I am grateful to the editors of this volume, Brigitte Pakendorf and Françoise Rose as well as the two anonymous reviewers of this chapter, Sally Rice, Jorge Rosés Labrada, and Simeon Floyd for their guidance, comments, and suggestions. All shortcomings of this work remain my own.


\section*{Abbreviations}

\begin{multicols}{2}
\begin{tabbing}
MMMMM \= first-person\kill
1             \> first-person     \\
3             \> third-person     \\
\textsc{all}  \> allative         \\
\textsc{ben}  \> benefactive      \\
\textsc{caus} \> causative        \\
\textsc{cop}  \> copula           \\
\textsc{conj} \> conjuntive       \\
\textsc{dem}  \> demonstrative    \\
\textsc{dim}  \> diminutive       \\
\textsc{dist} \> distal           \\
\textsc{ds}   \> different-subject\\
\textsc{ev}   \> evidential       \\
\textsc{fut}  \> future           \\
\textsc{gen}  \> genitive         \\
\textsc{hes}  \> hesitative       \\
\textsc{inf}  \> infinitive       \\
\textsc{imp}  \> imperative       \\
\textsc{intj} \> interjection     \\
\textsc{loc}  \> locative         \\
\textsc{mid}  \> middle           \\
\textsc{mir}  \> mirative         \\
\textsc{nmlz} \> nominalizer      \\
\textsc{int}  \> interrogative    \\
\textsc{lim}  \> limitative       \\
\textsc{o}    \> other-perspective\\
\textsc{obj}  \> object           \\
\textsc{prox} \> proximal         \\
\textsc{perf} \> perfective       \\
\textsc{ph}   \> placeholder      \\
\textsc{prog} \> progressive      \\
\textsc{prox} \> proximal         \\
\textsc{pst}  \> past             \\
\textsc{prov} \> pro-verb         \\
\textsc{purp} \> purpose          \\
\textsc{s}    \> self-perspective \\
\textsc{sbj}  \> subject          \\
\textsc{ss}   \> same-subject     \\
\textsc{top}  \> topic            \\
\textsc{trlc} \> translocative
\end{tabbing}
\end{multicols}

\printbibliography[heading=subbibliography,notkeyword=this]

\appendixsection{Summary of recordings in corpus}
\label{appendix:rice:I}

\begin{description}
\item[Consultants:]
     BD: Bélgica Dagua, DD: David Dagua, DE: Delicia Dagua, ED: Eulodia Dagua, LC: Luisa Cadena, SV: Sisa Viteri, TD: Teolinda Dagua
\item[Register:] C: Conversation, N: Narrative, SD: Stimulus description,
\item[Collector(s):] AR: Alexander Rice, Janis Nuckolls, LWC: Lisa Warren Carney, TS: Tod Swanson
\end{description}

\noindent
\fittable{\small%
\begin{tabular}{llllll}
\lsptoprule
           ID   & {\small Consultant} &  Duration & Register & {\small Collector(s)} & Source \\
            &    &  (mm:ss) & &  & \\
\midrule
jbn\_qvz002 & LC & 15:27 & N & JN & \citep{Nuckolls2017}\\
jbn\_qvz003 & LC & 14:08 & N & JN & \citep{Nuckolls2015}\\
jbn\_qvz005 & LC & 06:29 & N & JN & \citep{Nuckolls2015a}\\
lwc\_qvz001 & BD & 05:32 & N, C & LWC & \citep{WarrenCarney2016}\\
lwc\_qvz002 & ED & 02:58 & N, C & LWC & \citep{WarrenCarney2016a}\\
qvz001      & ED & 13:06 & N, C & AR & \citet{Ricea}\\
qvz002      & BD & 11:18 & N, SD & AR & \citet{Ricea}\\
qvz007      & BD & 14:40 & N & AR & \citet{Ricea}\\
qvz008      & BD & 06:38 & N & AR & \citet{Ricea}\\
qvz010      & BD & 15:05 & N, SD & AR & \citet{Ricea}\\
qvz011      & ED & 05:54 & N & AR & \citet{Ricea}\\
qvz014      & ED & 05:43 & C & AR & \citet{Ricea}\\
qvz015      & BD & 09:51 & C & AR & \citet{Ricea}\\
qvz017      & BD & 02:18 & C & AR & \citet{Ricea}\\
qvz018      & BD & 01:06 & C & AR & \citet{Ricea}\\
qvz020      & TD & 07:55 & N & AR & \citet{Ricea}\\
qvz022      & TD & 15:43 & N, SD & AR & \citet{Ricea}\\
qvz026      & BD & 14:04 & N, C  & AR & \citet{Ricea}\\
qvz042      & SV & 23:06 & C, N  & AR & restricted access\\
tds\_qvz003 & BD & 06:27 & N & TS & \citep{Swanson2020}\\
tds\_qvz010 & ED & 02:36 & N & TS & \citep{Swanson2021a}\\
tds\_qvz023 & LC & 09:53 & N & TS & private data\\
tds\_qvz024 & LC & 03:12 & N & TS & private data\\
tds\_qvz025 & LC & 11:41 & N & TS & \citep{Swanson2017}\\
tds\_qvz031 & DD & 15:28 & N & TS, AR & private data\\
tds\_qvz032 & DD & 26:22 & N & TS, AR & private data\\
tds\_qvz033 & DD & 07:47 & N & TS, AR & private data\\
tds\_qvz034 & BD & 05:30 & N & TS, AR & private data\\
tds\_qvz035 & BD & 05:47 & N, C &  TS, AR  & private data\\
tds\_qvz036 & BD & 05:19 & N    &  TS, AR  & private data\\
tds\_qvz037 & BD & 33:14 & N, C &  TS, AR  & private data\\
tds\_qvz039 & ED & 04:31 & N    &  TS, AR  & private data\\
tds\_qvz044 & DE & 15:31 & N & TS & \citep{Swanson2021a}\\
tds\_qvz045 & BD & 13:52 & N & TS & private data\\
\lspbottomrule
\end{tabular}}





\end{document}
