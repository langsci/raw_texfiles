\documentclass[output=paper]{langscibook}
\ChapterDOI{10.5281/zenodo.15697591}
\author{Maïa Ponsonnet\orcid{0000-0002-8879-9798}\affiliation{Laboratoire Dynamique du Langage (CNRS \& Université Lumière Lyon 2) ; University of Western Australia}}
\title[Individual preferences when using placeholders]{Individual preferences when using placeholders: The case of Dalabon (Australian, Gunwinyguan)}
\abstract{This article analyzes the placeholder \textit{keninjhbi} in Dalabon, a polysynthetic, non-configurational language of the Gunwinyguan family (non-Pama-Nyungan, \mbox{northern} Australia). Based on close to five hundred occurrences extracted from a 60-hour corpus, we observe that in line with the language’s general architecture, \textit{keninjhbi} is syntactically and morphologically flexible. It can replace verbal as well as nominal targets, and receive practically any of the morphology available to these targets. It can mirror the target’s marking either integrally or partially; some occurrences even carry informative morphology that does not recur on the target.

Reflecting this flexibility and the range of options this placeholder offers, speakers vary in the ways they use \textit{keninjhbi} – for instance in the word-classes they preferentially target, or in the amount of relevant information they package in the morphology allocated to the placeholder. The second part of the article compares speakers’ respective tendencies in using the placeholder. The comparison demonstrates how in a polysynthetic language like Dalabon, where placeholders inherit the morphological agility of the language, speakers can adopt distinctive disfluency-management styles.

\keywords{Dalabon, Australian languages, placeholders, polysynthesis, stylistics}
}
\IfFileExists{../localcommands.tex}{
  \addbibresource{../localbibliography.bib}
  % add all extra packages you need to load to this file

\usepackage{tabularx,multicol}
\usepackage{url}
\urlstyle{same}

\usepackage{listings}
\lstset{basicstyle=\ttfamily,tabsize=2,breaklines=true}

\usepackage{langsci-basic}
\usepackage{langsci-optional}
\usepackage{langsci-lgr}
\usepackage{langsci-osl}
% \usepackage{./langsci/styles/langsci-lgr}
% \usepackage{./langsci/styles/langsci-osl}
% \usepackage{langsci-gb4e}

\usepackage{tikz}
\usetikzlibrary{patterns,calc}
\pgfdeclarepatternformonly{south east lines}{\pgfqpoint{-0pt}{-0pt}}{\pgfqpoint{3pt}{3pt}}{\pgfqpoint{3pt}{3pt}}{
    \pgfsetlinewidth{0.6pt}
    \pgfpathmoveto{\pgfqpoint{0pt}{3pt}}
    \pgfpathlineto{\pgfqpoint{3pt}{0pt}}
    \pgfpathmoveto{\pgfqpoint{.2pt}{-.2pt}}
    \pgfpathlineto{\pgfqpoint{-.2pt}{.2pt}}
    \pgfpathmoveto{\pgfqpoint{3.2pt}{2.8pt}}
    \pgfpathlineto{\pgfqpoint{2.8pt}{3.2pt}}
    \pgfusepath{stroke}}
    
\usepackage{stmaryrd}
\usepackage{wasysym}
\usepackage{multirow}
\usepackage{caption}
\usepackage{subcaption}
\usepackage{mathrsfs}
\usepackage{qtree}

\usepackage{linguex}


  %pminos do not split footnotes
% \interfootnotelinepenalty=10000 %Footnote in Laporte chapters has to be split SN


%\DeclareIndexNameFormat{default}{%
%\nameparts{#1}%
%\usebibmacro{index:name}%
%{\index[names]}%
%{\namepartfamily}%
%{\namepartgiveni}%
% {}% L1
% {}% L2
%{\namepartprefix}% generates spurious space L3
%{\namepartsuffix}% generates spurious space L4
%}

%  {\DeclareIndexNameFormat{default}{%
%     \usebibmacro{index:name}{\index[names]}{#1}{#3}{#5}{#7}}}

%\DeclareIndexNameFormat{default}{%
%  \usebibmacro{index:name}{\sindex[nom]}{#1}{#3}{#5}{#7}}

%\DeclareIndexNameFormat{default}{%
%  \usebibmacro{index:name}{\sindex[person]}{#1}{#3}{#5}{#7}}
%\DeclareIndexNameFormat{default}{%
%\nameparts{#1} \usebibmacro{index:name}{\sindex[person]]}{\namepartfamily}{‌​\namepartgiven}{\nam‌​epartprefix}{\namepa‌​rtsuffix}}

%\newcommand{\smiley}{:)}

%\renewbibmacro*{index:name}[5]{%
%\usebibmacro{index:entry}{#1}%
%{\iffieldundef{usera}{}{\thefield{usera}\actualoperator}\mkbibindexname{#2}{#3}{#4}{#5}}}

% \newcommand{\noop}[1]{}

%remove for final
%\overfullrule=1mm

\newcommand{\tobi}[2]}}
\renewcommand{\S}[1]{\tobi{#1}{\textsc{*}}}

% this volume references
% puts: [this volume]
% already defined: \citetv
%\newcommand{\citepv}[1]{(\citeauthor{#1} \citeyear*{#1} [this volume])}
\newcommand{\citealtv}[1]{\citeauthor{#1} \citeyear*{#1} [this volume]}

%parentheses around example number
\newcommand{\pref}[1]{(\ref{#1})}

% in-text examples

\newcommand{\lnex}[1]{\textit{#1}} %target lang word
\newcommand{\lnlit}[1]{(lit.: `#1')} %literal reading
\newcommand{\lnlat}[1]{(#1)} % latinization
\newcommand{\lntrans}[1]{`#1'} %translation
\newcommand{\lnexl}[2]%
{\lnex{#1}{} \lnlat{#2}} % ex with latinization
\newcommand{\lnexlat}[3]{\lnex{#1}{} \lnlat{#2}{} \lntrans{#3}} % ex with latinization and tranl.

%ch01
\newcommand{\co}[1]{\mbox{\textbf{#1}}}

%ch09

\newcommand{\cyrbulg}[1]{\begin{otherlanguage*}{bulgarian}#1\end{otherlanguage*}}


%ch10
\newcommand{\nlp}{{\small NLP}}
\newcommand{\mwe}{{\small MWE}}
\newcommand{\rae}{{\small RAE}}
\newcommand{\lvc}{{\small LVC}}
\newcommand{\pos}{{\small P}o{\small S}}
%\newcommand{\todo}[1]{ \textcolor{red}{#1} }

%\renewcommand{\labelenumi}{\theenumi}
%\ainamefmt{{vv}{ll}{, ff}{, jj}} % fullname

\newcommand{\biberror}[1]{{\color{red}#1}}

\newcommand{\osenovaitem}{--~} 
  %% hyphenation points for line breaks
%% Normally, automatic hyphenation in LaTeX is very good
%% If a word is mis-hyphenated, add it to this file
%%
%% add information to TeX file before \begin{document} with:
%% %% hyphenation points for line breaks
%% Normally, automatic hyphenation in LaTeX is very good
%% If a word is mis-hyphenated, add it to this file
%%
%% add information to TeX file before \begin{document} with:
%% %% hyphenation points for line breaks
%% Normally, automatic hyphenation in LaTeX is very good
%% If a word is mis-hyphenated, add it to this file
%%
%% add information to TeX file before \begin{document} with:
%% \include{localhyphenation}
\hyphenation{
    Beck-man
    Ngu-yen
    back-chan-nel
    back-chan-nels
    mo-not-o-nous
    ste-reo-typ-i-cal
}

\hyphenation{
    Beck-man
    Ngu-yen
    back-chan-nel
    back-chan-nels
    mo-not-o-nous
    ste-reo-typ-i-cal
}

\hyphenation{
    Beck-man
    Ngu-yen
    back-chan-nel
    back-chan-nels
    mo-not-o-nous
    ste-reo-typ-i-cal
}
 
  \togglepaper[1]%%chapternumber
}{}

\begin{document}
\maketitle 


\section{Introduction}
\label{sec:ponsonnet:1}
The volume in which this article is published takes an interest in “fillers”, defined as “non-silent markers of hesitation in disfluency” (\citealt{chapters/intro}). Linguistic descriptions of fillers often assume, implicitly or explicitly, that since they reflect hesitation, fillers index speakers’ cognitive failures: \citet[485]{HayashiHayashi2006}, for instance, characterize them as words used “in contexts where speakers encounter trouble recalling a word or selecting the best word […]”. Linguists typically understand “disfluency” as a technical term stripped of prescriptive or evaluative connotations, and do not judge negatively what they may label speakers’ “trouble”. On the contrary, many linguists adopt a functional approach to disfluency, paying close attention to the broad range of pragmatic and discourse functions fillers fulfill beyond “disfluency repairs” (see for instance \citet{Keevallik2010}, or \citet{KosmalaCrible2022} for a review). Yet, beyond linguists’ relatively small networks, our choices of terms and definitions can also influence a less informed audience – or perhaps, in this case, simply encourage existing ideologies derived from prescriptive perspectives on language. 

Indeed, in many day-to-day interactions, hesitations and “failures” can be judged negatively – whether this judgement is conscious or not – because they seem to index lesser proficiency (\citealt{FehringerFehringer2007}, \citealt{Jong2016}). As demonstrated by \citet{Bourdieu2001} and many others, linguistic competence indexes and produces cultural capital and social power. Conversely, sounding linguistically incompetent incurs social costs. People across the world are well aware that they cannot escape this reality. As overt markers of disfluency, fillers are thus by nature caught up within a complex system of social signs, and we should expect speakers to handle them as such. They may try to avoid fillers altogether; or, on the contrary, recruit them to align or distance themselves from a norm, a group, or otherwise position themselves on a social chessboard \citep{Silverstein2003}. Illustrations can be found in the public sphere, for instance in former French President François Hollande’s (2012--2017) use of the hesitative \textit{euh}, and the perception of this strategy by his political adversaries\footnote{As illustrated by the title of this article in \textit{Le Point} during the 2012 campaign (22/01/2012): “L’UMP dégaine ‘le vrai discours de François euh... Hollande’”, “The UMP [Union for a People’s Movement] pulls out ‘the true speech of François er… Hollande’” –  \url{https://www.lepoint.fr/politique/l-ump-degaine-le-vrai-discours-de-francois-euh-hollande-22-01-2012-1422208_20.php\#11} (I am not aware of any linguistic study on Hollande’s speech specifially).} (see also \citealt{Duez1999}; \citealt{MareüilMareüil2013}). Of course, ordinary speech is rarely as thoroughly staged and planned as that of a presidential candidate. Yet, individual preferences in disfluency \mbox{management} are well attested, and inceptive stylistic manipulations of fillers are detectable in their day-to-day usage (\citealt{LasernaLaserna2014}; \citealt{BraunBraun2023}). In this article, I offer some considerations on how this plays out, and on which properties of fillers can lend themselves to such stylistic manipulations in Dalabon, a polysynthetic Australian language from the Gunwinyguan family.

As already mentioned, here I define fillers as “non-silent linguistic devices used in disfluencies”. Also in line with \href{chapters/intro}{ Pakendorf \& Rose}’s introduction to this special issue, among fillers I distinguish between hesitative interjections, which do “not occupy any specific syntactic slot within the structure of an unfolding utterance”; and placeholders, “referential expression[s] that [are] used as a substitute for a specific lexical item” and that hence “occup[y] a syntactic slot that would have been occupied by the target word” (\citet[507, 490]{HayashiHayashi2006}, respectively); see also \citet{BraunBraun2023} for a taxonomy of disfluencies and filled pauses). This article examines placeholders, which, being syntactically “active”, can, in many languages, be morphologically enriched. 

Dalabon, the language in focus here, is a member of the Gunwinyguan family (Australian, non-Pama-Nyungan), and is typical of this family in being polysynthetic and non-configurational. By definition, polysynthetic languages can aggregate a lot of morphology (\citealt{Evans2017}). In such languages, a great deal of information is therefore conveyed by adding affixes and clitics to word stems; and in Dalabon, many of these morphological tools are highly flexible distributionally \citep{Ponsonnet2015}. Dalabon has just one placeholder, but in line with the language’s architecture, it is syntactically and morphologically flexible: speakers can use it to replace words from most classes, and have at their disposal a large choice of items they can append to it. As a result, speakers’ preferred “styles” in using placeholders are easy to detect, even based on a relatively limited corpus. The language therefore offers an ideal arena to study variation in how different speakers use placeholders. So far, studies of placeholders in polysynthetic languages are rare (see \citetv{chapters/mithun} on Mohawk/Kanien’kéha’), and to my knowledge, none of them discusses the morphological flexibility of fillers in particular. 

After presenting the Dalabon language and the speakers who contributed to this study in \sectref{sec:ponsonnet:2}, in \sectref{sec:ponsonnet:3} I describe the syntactic and morphological behavior of the Dalabon placeholder, \textit{keninjhbi}. In \sectref{sec:ponsonnet:4}, I compare the distinctive tendencies or “styles” observed in the three speakers who contributed the most data to this study, and propose directions for further research in this respect. 

\section{Language and data}
\label{sec:ponsonnet:2}
\subsection{The Dalabon language}
\label{sec:ponsonnet:2.1}
Dalabon is an Australian language from the Gunwinyguan family, one of the largest non-Pama-Nyungan families. Prior to colonization, Dalabon was spoken by a few hundred semi-nomadic hunter-gatherers in the western part of the \mbox{Arnhem} Land region in the Northern Territory (center north) of Australia. At the time of writing, Dalabon is severely endangered, numbering a small handful of fluent speakers. The descendants of Dalabon speakers live mostly in remote Indigenous communities to the east of the town of Katherine (Barunga, Beswick, Bulman, Weemol). They have adopted Kriol, an English-lexified creole spoken by thousands of speakers across the center north of Australia.

There exists no full grammar of Dalabon at this stage, but there is a dictionary (\citealt{EvansEvans2004}), and a number of articles and theses describe various aspects of the language. This includes the verbal template (\citealt{evans2003dalabon}; \citealt{Ponsonnet2014a}: 61–64), morphological architecture (\citealt{Evans2017}; \citealt{PonsonnetPonsonnet2015}), word classes including nominal subclasses and noun incorporation \citep{Ponsonnet2015}, tense/aspect/mood categories (\citealt{evans2003dalabon}), person prefixes (\citealt{EvansEvans2001}), clause and argument marking (\citealt{Evans2006}; \citealt{LukLuk2019}; \citealt{Ponsonnet2021}), noun incorporation and nominal subclasses \citep{Ponsonnet2015}, demonstratives \citep{Cutfield2011}, prosody (\citealt{EvansEvans2008}; \citealt{Ross2011}). Regarding semantics and lexical semantics, \citet{BordlukBordluk2013} offers an inventory of the plant and animal lexicon, and \citet{Ponsonnet2014a} examines the emotion domain.

Like many non-Pama-Nyungan Australian languages, and all Gunwinyguan languages, Dalabon is highly polysynthetic and agglutinative. The language is predominantly head-marking, with clausal arguments systematically cross\hyp referenced by prefixes (and proclitics) on predicates. A discussion of the verbal template and a précis of Dalabon argument marking can be found in \citet[60–62, 152--154]{Ponsonnet2014a}. The full verbal template numbers 15 slots, although in ordinary speech, only a fraction of these are usually filled in a given utterance. In the first verb complex of example (\ref{ex:ponsonnet:1}), only obligatory slots (person and mood prefixes, and TAM suffixes) are filled in. The second verb complex illustrates a richer morphology, with two aspectual markers and an incorporated adverb (\textit{kakku-} ‘really’).

\largerpage
\ea
\label{ex:ponsonnet:1}
Dalabon (20120705b\_004 118 [Film])\\
\gll   Bunu   ka-h-na-ng,    \\
     3du  3sg/3du-\textsc{r}{}-see-\textsc{ppfv}\\
\gll     \textit{barra-h-dja-lng-kakku-yurd-minj.}\\
     3du-\textsc{r}{}-\textsc{foc}{}-\textsc{seq}{}-really-run-\textsc{ppfv}\\
\glt ‘She looked at the two of them, they were running fast.’
\z 

Dalabon noun phrases accommodate a set of case suffixes, as illustrated in (\ref{ex:ponsonnet:2}) with the locative marker \textit{{}-kah}. Noun incorporation, shown in (\ref{ex:ponsonnet:3}) with \textit{denge} ‘foot’, is frequent and in fact quasi-obligatory for certain classes of nouns like body-part nouns \citep{Ponsonnet2015}. 

\ea
{\label{ex:ponsonnet:2} Dalabon (20120720\_003 193 [Film])}\\
\gll Oh   klinik-kah     bala-h-bobo-n     ngale.\\
\textsc{intj}  clinic-\textsc{loc}   3pl-\textsc{r}{}-go:\textsc{redup-prs}  \textsc{intj}\\
\glt ‘Oh, they’re going to the clinic, right.’
\z 


\ea
{\label{ex:ponsonnet:3}
Dalabon (20110521a\_002 030 [El])}\\
\gll Nga-h-dengu-berderde-mu.\\
     1sg\textsc{{}-r-}foot-ache\textsc{{}-prs}\\
\glt ‘My foot aches.’
\z 

Apart from these generic typological properties, several other aspects of Dalabon grammar are worth keeping in mind to understand how placeholders pattern in the language. Overall, Dalabon exhibits a tendency towards non\hyp configurationality (like most Australian languages, see \citealt{AustinAustin1996}) and morphological flexibility. Clause-level word order is syntactically free and pragmatically determined. Within the noun phrase, word order is also relatively free. As is the case in many Australian languages, Dalabon noun phrases can be discontinuous. This is illustrated in (\ref{ex:ponsonnet:4}) where \textit{lord-boyenj-ko} ‘big-bodied-pair’ and \textit{yabbunh} ‘two’, both on the left of the verb, are part of the noun phrase headed by \textit{biyi} ‘men’, on the other side of the verb. Morphology occurs either on the head or on the modifier. In \REF{ex:ponsonnet:4}, the dyadic marker \textit{{}-ko} occurs on \textit{lord-boyenj} ‘big-bodied’. In \REF{ex:ponsonnet:5}, the third-person possessive marker \textit{{}-no} marks \textit{wadda} ‘camp’, which is the possessed head of the noun phrase \textit{ngal-Jane wadda-no-kah} ‘at Jane’s camp/place’. \textit{Wadda-no} carries the morphological mark of its dependency to the verb \textit{di} ‘stand’, of which it is a locative adjunct.
 
\ea
{ \label{ex:ponsonnet:4}
Dalabon (20111206a\_001 38 [El])}\\
\gll Lord-boyenj-ko   yabbunh   barra-h-bobo-n\\
body-big-\textsc{dyad}  two    3du-\textsc{r}{}-go:\textsc{redup:prs}\\
\gll biyi   burrkunh.\\
     man  two \\
\glt ‘Two fat (big-bodied) men are walking by.’
\z 


\ea
{\label{ex:ponsonnet:5} Dalabon (20120708b\_000 093 [Narr])}\\
\gll Ngal-Jane wadda-no-kah awudj buka-h-marnu-di.\\
\textsc{fem}-first.name camp-\textsc{poss-loc} 
house  3sg/3sg.h-\textsc{ben-}stand:\textsc{prs}\\
\glt ‘At Jane’s camp, stands her house.’
\z 

Generally, nominal morphology is distributionally flexible in Dalabon. That is, there is no categorical difference between the morphology nouns and adjectives allow. In addition, both these classes can receive some of the typically verbal affixes, for instance person prefixes marking the subject of a non-verbal predicate (see \citealt{EvansEvans2001} and \citealt{EvansEvans2004} for person prefix paradigms). In \REF{ex:ponsonnet:6} for instance, the first person singular prefix \textit{nga-} cross-references the subject of the nominal predicate \textit{wurdurd} ‘child’ (noun), which is also inflected for tense.

 
\ea
{ \label{ex:ponsonnet:6}Dalabon (30044/2007-4’ [Narr])}\\
\gll Ngey   mah   wurdurd   nga-h-wurdurd-ninj, \\
1sg   \textsc{conj}  child     1sg-\textsc{r}{}-child-\textsc{pimp}\\
\gll budj   nga-h-dja-ni-nj.\\
bush   1sg-\textsc{r}{}-\textsc{foc}{}-sit/be-\textsc{ppfv}\\
\glt ‘As for me, as a child, when I was a child, I lived in the bush.’
\z 

Conversely, typically nominal devices, such as possessive or case markers, can be appended to verbs. This is illustrated in (\ref{ex:ponsonnet:7}), where \textit{{}-burrng}, the standard possessive marker for third person dual (i.e. ‘their’, two people), is appended to the verb to cross-reference ‘them two’ as an oblique.\footnote{This is the only way to cross-reference oblique arguments on verbs, but this cross-referencing is very rare, and therefore a marginal use of the set of pronouns commonly used to mark possessed nouns.}  In (\ref{ex:ponsonnet:8}), it is the locative \textit{{}-kah}, which typically occurs on locative noun phrases, which marks the role of a subordinate verb complex.
 
\ea
{\label{ex:ponsonnet:7}Dalabon (20120713a\_000 181 [Film])}\\
\gll Kenbo kanh ka-h-lng-dje-bruH-minj-burrng        djarrkno.\\
then     \textsc{dem} 3sg-\textsc{r-seq}{}-nose-blow-\textsc{ppfv}{}-3du.\textsc{poss} together\\
\glt ‘That’s why she’s upset with them then, with both of them.’
\z 


\ea
{ \label{ex:ponsonnet:8}Dalabon (20100716\_002 [El])}\\
\gll Kenbo   bala-lng-mukka-ng \\
then      3pl-\textsc{seq}{}-cover-\textsc{ppfv}\\
\gll kanh   ka-ye-rakk-iyan-kah. \\
\textsc{dem}       3sg-\textsc{sub}{}-fall-\textsc{fut-loc}\\
\glt ‘Then they buried it, there where it would fall.’
\z 

Overall, Dalabon morphology offers multi-purpose, flexible tools, and this has consequences on what speakers can do with placeholders. 

\subsection{Corpus and data set}
\label{sec:ponsonnet:2.2}
The descriptions presented in this article draw on a 60-hour first-hand corpus collected between 2007 and 2012 in the First Nation communities called Weemol, Beswick and Barunga, with eight speakers. All but one speaker were above sixty years old at the time, and only the youngest is still alive as I write this article. Five of them, including the three who produced the vast majority of the material, were female. These Dalabon consultants expressed pride in working towards documenting their language. They were eager to share their language with me, and to see it disseminated beyond the Dalabon community. In the analysis and “portrayal” of Dalabon I present here, I have done my very best to remain faithful to the spirit in which these speakers shared their language with me in the late 2000s and early 2010s. They usually preferred to be cited by name in writings where their data was used, as I have done in other publications. Due to the evaluative connotations surrounding the question of disfluency and the use of fillers (see \sectref{sec:ponsonnet:1}), I make an exception here and use pseudonyms throughout the chapter. Family members and other Dalabon descendants are welcome to contact me, or my institutions if necessary, should they wish to access further details or the data.\footnote{Personal address at the time of writing: \href{mailto:maia.ponsonnet@cnrs.fr}{maia.ponsonnet@cnrs.fr}\\
CNRS team: \url{http://www.ddl.cnrs.fr/index.asp?Langue=EN};\\
AIATSIS collections items: \url{https://iats.ent.sirsidynix.net.au/client/en_AU/external/search/results?qu=ponsonnet&te=ILS}; ELAR collection: \url{https://www.elararchive.org/dk0071/}}

The corpus produced by these speakers is comprised of a range of audio and sometimes video recordings including narratives (mostly monological); free commenting on visual stimuli such as series of connected pictures, video clips, and thematically appropriate broad-audience films;\footnote{In particular \textit{Rabbit-Proof Fence} \citep{Noyce2002}, \textit{Ten Canoes} (\citealt{HeerHeer2006}) and \textit{Samson and Delilah} \citep{Thornton2009}, all of which focus on Indigenous Australians’ lives (each in different historical times). Most speakers were already familiar with these movies before commenting on them for the purpose of linguistic documentation.} as well as contextualized grammatical and semantic elicitation, particularly about body parts, bodily affects, and emotions (see \citealt{Ponsonnet2014b} for a discussion). This covers a range of discourse genres, including relatively spontaneous speech. Although no lengthy conversations were recorded, some exchanges occasionally occurred between participants during group elicitation sessions, or with myself, and this is included in the corpus. The corpus is entirely transcribed in ELAN and therefore fully searchable.

Searching for instances of the placeholder returned a set of 476 occurrences. These were coded for a number of properties concerning the speaker, context of occurrence, target constituent (realization, word class, semantics), morphological enrichments on the placeholder and whether they matched those on the target, as well as presence or absence of a pause. The spreadsheet is available here: \url{https://doi.org/10.5281/zenodo.13304541}. The analyses below are entirely based on this data set; I have not carried out focused elicitation on placeholders. 

\section{The Dalabon placeholder: \textit{keninjhbi}}
\label{sec:ponsonnet:3}
\subsection{Form and functions}
\label{sec:ponsonnet:3.1}
The most frequent form of the only Dalabon placeholder is \textit{keninjhbi} [ˈgeniɲʔˌbi]. Example (\ref{ex:ponsonnet:9}) presents relatively typical occurrences where the form is fully articulated. Alternatively, speakers often contract it as \textit{kenjhbi} [ˈgeɲʔˌbi] or \textit{kenhbi} [ˈgenʔˌbi], or reduce it to \textit{keninjh} [ˈgeniɲʔ].\footnote{The variants were also searched for and included in the data set.}  A distinct word, \textit{kenh} [ˈgenʔ], which presumably results from a truncation of \textit{keninjhbi}, has conventionalized as a self-correction interjection (‘oops’). 

The etymology of \textit{keninjhbi} is transparent: \textit{keninjh} is an interrogative pronoun, and \textit{bi} is probably related to \textit{biyi} [bi:], which in synchrony means ‘person’. As flagged by \citet[12]{Podlesskaya2010}, interrogatives are a common etymological source for placeholders cross-linguistically. The variant \textit{keninjhkun} [interrogative+genitive] is also used occasionally (see (\ref{ex:ponsonnet:11}) below). The \textit{keninjhkun} variant is about 10 times less frequent than \textit{keninjhbi}, and exhibits no obvious difference in semantics or behavior. In this study all variants are included in the same data set and analyzed together. 

\ea
{\label{ex:ponsonnet:9} Dalabon (20110518a\_002 082 [Narr])}\\
\gll Kanidjah   bala-lng-bo-ninj\\
\textsc{dem}    3pl-\textsc{seq}{}-go-\textsc{pimp}\\
\gll \textbf{keninjhbi-ngong}    [3.76s]     \uline{kunj-ngong}    \\
\textsc{ph}-group   {}   kangaroo-group\\
\gll bala-h-bo-ninj    \textbf{keninjhbi}     \uline{djakana}…\\
3pl-\textsc{r}{}-go-\textsc{pimp}    \textsc{ph}    bird.species\\
\glt ‘They were going there, \textbf{all} \textbf{the} \textbf{whatsit}… [pause] \uline{all the kangaroos} were going there, the \textbf{whatsit} \uline{jacanas}.’\\
\z 

\textit{Keninjhbi} is a placeholder, not a hesitative interjection: it is syntactically integrated into the clauses where it occurs, temporarily or permanently replacing the “target constituent” it stands for. Syntactic integration is evident where \textit{keninjhbi} carries morphology (as in the first occurrence in (\ref{ex:ponsonnet:9})). Where morphology is absent, prosody often indicates the syntactic role of \textit{keninjhbi}, because it gives clues to the clausal delimitation and information structure. Morphology and prosody usually make it easy to identify the target of the placeholder in the rest of the utterance. 

It is worth clarifying that in the vast majority of occurrences, \textit{keninjhbi} is used to fill in for temporary disfluencies, irrespective of pragmatic motivations such as avoidance or euphemistic effects (as with the English \textit{you-know-what} according to \citealt{Enfield2003}). \citet{Enfield2003} shows that in some languages, filler words can be recruited in contexts where pragmatic rules impose vagueness or indirectness (see \citealt{Seraku2024}, among others). A pragmatic preference for indirectness has been documented in many Australian languages, including Bininj Gun-Wok, a close relative and neighbor of Dalabon \citep{Garde2008}. Indeed, this preference applies in Dalabon too, however \textit{keninjhbi} is not used to realize indirectness or euphemism.\footnote{My impressionistic assessment is that demonstratives are used for this purpose, but this remains a question for future research.} This is evident given that the target constituents of \textit{keninjhbi} are realized more than 70\% of the time. Furthermore, nearly half the occurrences in my data set feature some mark of hesitation, such as a pause or an explicit comment on the search (e.g. “ah, I can’t find this word”). In line with its filler role, \textit{keninjhbi} frequently occurs in “lexically challenging” contexts, for instance when searching for relatively rare words like plant or animal species nouns (\ref{ex:ponsonnet:9}), or before borrowings from Kriol (\ref{ex:ponsonnet:10}), English, or other languages. When placeholders target borrowings, they most often represent words for objects or concepts that were not named in Dalabon, such as \textit{peismeika} ‘pacemaker’ (\ref{ex:ponsonnet:10}), \textit{alikopta} ‘helicopter’, \textit{djuwida} ‘sweater’, \textit{eidyukeishen} ‘education’, \textit{delebon} ‘telephone’, \textit{bens} ‘fence’, \textit{midjik} ‘music’, etc. While such words are not necessarily technical or rare, it is plausible that they may be somewhat less familiar to speakers, and/or less accessible when speaking Dalabon. The combination of occurrences in lexical elicitation of technical vocabulary, with those targeting natural species names, and occurrences involving borrowings, represent 40\% of all occurrences of \textit{keninjhbi} (and variants).

\ea
{\label{ex:ponsonnet:10}Dalabon (20120717\_002 108 [ConvEl])}\\
\gll Ka-h-yidjnja-n \textbf{keninjhbi} \uline{peismeika}. \\
3sg\textsc{/}3sg\textsc{{}-r-}have\textsc{{}-prs}  \textsc{ph}    pacemaker\\
\glt ‘He’s got a \textbf{what-you-call-it} \uline{pacemaker}.’
\z 

\subsection{Morphosyntactic properties} 
\label{sec:ponsonnet:3.2}
\subsubsection{Syntactic versatility}
\label{sec:ponsonnet:3.2.1}
In line with the tendencies observed in the rest of the grammar, Dalabon placeholders are syntactically flexible: they are attested with targets from a range of word classes. Perhaps reflecting the etymology of \textit{keninjhbi}, noun targets (\ref{ex:ponsonnet:11}) are most frequent, amounting to nearly 70\% of all occurrences. This can include proper names, mostly of people and locations. Verb targets (\ref{ex:ponsonnet:12}) are less frequent, yet by no means rare, as they represent 30\% of all occurrences. Predicative adjectives are also attested (\ref{ex:ponsonnet:13}), and there is no evidence that any word class may \textit{not} be targeted by \textit{keninjhbi} (or in fact, any constituent, including phrasal constituents).

\ea
{\label{ex:ponsonnet:11}Dalabon (20110614\_006~003 [Narr])}\\
\gll Bulu-ngan     ka-h-wrokeb-m-inji langa         \textbf{keninjhkun}… \textbf{kenhbi-kah}   \uline{Bikûri-yad}.\\
father-1sg.\textsc{poss}    3sg-\textsc{r}{}-work-\textsc{vblzr-pcust} {\textsc{loc}~\textup{(Kr)}} \textsc{ph} \textsc{ph}\textup{-}\textsc{loc} \textup{place.name}\\
\glt ‘My father used to work for \textbf{whatsit}… \textbf{at} \textbf{whatsit}, \uline{Pickery Yard}.’
\z 


\ea
{\label{ex:ponsonnet:12}Dalabon (20120705b\_001 011 [Stim])}\\
\gll Dohkardu  munu   kanh  barra-h-djarrk… \textbf{barra-h-djarrk-kenjhbi-mu} \uline{barra-h-djarrk-yenjhyenjdju-ng}   yang.\\
\textup{or.maybe} \textsc{lim} \textsc{dem} \textup{3du-}\textsc{r}\textup{{}-together} \textup{3du-}\textsc{r}\textup{{}-together}\textsc{{}-ph}\textup{{}-}\textsc{prs} \textup{3du-}\textsc{r}\textup{{}-together-talk:}\textsc{redup-prs} \textup{language}\\
\glt ‘Or else the two of them together may just be… \textbf{the} \textbf{two} \textbf{of} \textbf{them} \textbf{whatyoucallit} \textbf{together}… \uline{the two of them are talking together}.’
\z 

\ea
{\label{ex:ponsonnet:13}Dalabon (250909\_89OK 0105 [El])}\\
\gll \textbf{Ka-h-dorrung-kenjhbi} \uline{ka-h-dorrung-mondi-duninj}     yuno    pudiwan.\\
\textup{3sg-}\textsc{r}\textup{{}-body}\textsc{{}-ph} \textup{3sg-}\textsc{r}\textup{{}-body}\textup{{}-good-}\textsc{intens} \textsc{conj} \textup{pretty(Kr)}\\
\glt ‘\textbf{She’s} \textbf{whatsit} \textbf{in} \textbf{appearance}… \uline{she’s really good looking}… you know, pretty.’
\z 

\citet[14]{Podlesskaya2010} suggests that across languages, placeholders preferentially target nominals. Against this measure, the prevalence of noun targets in Dalabon appears less remarkable than the relative frequency of verb targets, which account for nearly one out of three occurrences. Overall, Dalabon \textit{keninjhbi} can be characterized as a syntactically flexible placeholder.  

\subsubsection{Morphological versatility}
\label{sec:ponsonnet:3.2.2}
Practically all morphological enrichments available to a target lexeme can be replicated on the placeholder that stands for this target. This is illustrated in (\ref{ex:ponsonnet:11}) and (\ref{ex:ponsonnet:13}) above for both nominal and verbal morphology. 

There is no evidence of restrictions on the use of nominal morphology when \textit{keninjhbi} targets a noun or an adjective. With verb targets, a notable exclusion is reduplication, but this could be due to the morphophonological make-up of \textit{keninjhbi}. Verbal reduplication is a relatively productive aspectual inflection for Dalabon verbs, with an iconic value around duration; however it is normally limited to one- or two-syllable verb roots. 

When targeting verbs, the placeholder is assigned to its own conjugation class. Morphologically, many Dalabon verbs combine a “prepound” with a “thematic” (see \citealt[chap. 8]{Evans2003bininjgunwok}) and \citet{Saulwick2003} for comparable analyses on the neighboring Bininj Gun-Wok and Rembarrga languages). This is illustrated in (\ref{ex:ponsonnet:14}), where the thematics appear in bold. Dalabon numbers about two dozen thematics. While each one tends to convey its own semantic “flavor”, they remain partly semantically opaque, or at least very versatile, and are therefore better analyzed as parts of the lexical verbs. They carry and determine TAM inflections, and therefore define the conjugation class to which each lexical verb belongs (\citealt{evans2003dalabon}): compare (\ref{ex:ponsonnet:14a}) to (\ref{ex:ponsonnet:14b}), where \textit{{}-mu} and \textit{{}-minj} are respectively the present and past-perfective forms of the same thematic; and (\ref{ex:ponsonnet:14c}) to (\ref{ex:ponsonnet:14f}), which illustrate other thematics.\footnote{Strictly speaking, the morpheme partition should be \textit{{}-m-u}, glossed -\textsc{thmc-prs}, \textit{{}-m-inj} glossed -\textsc{thmc-ppfv}, etc., but this would affect the readability of the glosses and add very little information. In this article, the neutral thematics \textit{{}-mu} is glossed solely for its TAM value, i.e. \textit{{}-mu} is glossed \textsc{prs}, \textit{{}-minj} is glossed \textsc{ppvf}, etc.}

\ea
{\label{ex:ponsonnet:14}Dalabon (personal knowledge)}\\
\ea \label{ex:ponsonnet:14a}
\gll \textit{nga-h-boled-}\textbf{\textit{mu}}\\
1sg-\textsc{r}{}-turn-\textsc{thmc1:\textsc{prs}}\\
\glt ‘I turn/change’


\ex \label{ex:ponsonnet:14b}
\gll \textit{nga-h-boled-}\textbf{\textit{minj}}\\
1sg-\textsc{r}{}-turn-\textsc{thmc1}:\textsc{ppfv}\\
\glt ‘I turn/changed’


\ex \label{ex:ponsonnet:14c}
\gll \textit{nga-h-buyh-}\textbf{\textit{won}}\\
1sg/3sg-\textsc{r}{}-show-\textsc{thmc2}:\textsc{prs} \\
\glt ‘I show (it) to him/her’

\ex \label{ex:ponsonnet:14d}
\textit{nga-h-buyh-}\textbf{\textit{wong}}\\
1sg/3sg-\textsc{r}{}-show-\textsc{thmc2}:\textsc{ppfv}\\
\glt ‘I showed (it) to him/her’\\

\newpage
\ex \label{ex:ponsonnet:14e}
\textit{nga-h-dadj-}\textbf{\textit{ka}}\\
1sg/3sg-\textsc{r}{}-cut-\textsc{thmc3}:\textsc{prs}\\
\glt ‘I cut it’

\ex \label{ex:ponsonnet:14f}
\textit{nga-h-ngadjdji-}\textbf{\textit{bun}}\\
1sg-\textsc{r}{}-sneeze-\textsc{thmc4}:\textsc{prs}\\
\glt ‘I sneeze’\\
\z
\z

When \textit{keninjhbi} targets a verb, TAM inflections are always appended using the default, semantically neutral, typically intransitive thematic, the present form of which is \textit{{}-mu} (\ref{ex:ponsonnet:14a} and \ref{ex:ponsonnet:14b}). In other words, irrespective of the target, \textit{keninjhbi} is consistently assigned to the same conjugation paradigm, and affords the full set of TAM marking following this conjugation. 

\subsubsection{Morphological mirroring}
\label{sec:ponsonnet:3.2.3}
The morphological flexibility of \textit{keninjhbi} enables it to “mirror” the morphology of its targets, i.e. the morphology of the target can be reproduced on the placeholder. Indeed, this is frequent, often with relatively complex morphological patterns. Mirroring is illustrated in (\ref{ex:ponsonnet:15}) for a noun target, and in (\ref{ex:ponsonnet:16}) for a verb target. Depending on speakers, full mirroring occurs on 30\% to 50\% of placeholders with nominal targets. Given that Dalabon verbal morphology is richer than nominal morphology, full mirroring concerns only 5\% to 10\% of occurrences of placeholders with verbal targets.

\ea
\label{ex:ponsonnet:15}Dalabon (20100724\_010 097 [Stim])\\
\gll {MT}  \textbf{djud-keninjhbi-no}   ke\\
   {} {neck}\textsc{-ph}-3sg.\textsc{poss} \textsc{emph}\\
   
\gll {MP}  \uline{djud-kon-no}     kardu\\
   {} neck-fin-3g.\textsc{poss} maybe\\
   
\gll \textup{MT}  nomo…   djud-kon-no\\
  {} {\textsc{neg}~(Kr)}  neck-fin-3sg.\textsc{poss}\\
  
\glt ‘MT  \textbf{Its} \textbf{dorsal-whatsit} again…\\
MP  Its dorsal fin maybe? \\
  MT  No… \uline{Its dorsal fin}.’
\z 


\ea
{\label{ex:ponsonnet:16}Dalabon (20110614\_009 15 [Stim])}\\
\gll Nunda   \textbf{dja-h-dje-kenhbi-minj}  \uline{dja-h-dje-boled-minj}.\\
\textsc{dem} \textup{2sg-}\textsc{r}\textup{{}-nose-}\textsc{ph}\textup{{}-}\textsc{ppfv} \textup{2sg-}\textsc{r}\textup{{}-nose-turn-}\textsc{ppfv}\\
\glt ‘On this one you \textbf{whatsit-ed} your face… \uline{you turned your face}.’
\z 

Mirroring can result in the placeholder carrying significant information. In example (\ref{ex:ponsonnet:12}) for instance (repeated here for convenience), where full mirroring is realized, based on the placeholder the addressee already knows that the event at issue involves two participants in a harmonic classificatory relationship,\footnote{Two people either in the same generation or two generations apart, i.e. siblings, cousins, or grandparent and grandchild (see \citealt{EvansEvans2001}).} with current, joint engagement. The retrieval of the target lexeme \textit{yenjdjung} ‘speak’ adds that the participants are talking together, but much of the scene was already decipherable on the basis of the placeholder and its morphology, prior to lexical specification.

\begin{exe}
\exr{ex:ponsonnet:12}
{Dalabon (20120705b\_001 011 [Stim])}\\
\gll Dohkardu  munu   kanh  barra-h-djarrk… \textbf{barra-h-djarrk-kenjhbi-mu} \uline{barra-h-djarrk-yenjhyenjdju-ng}   yang\\
\textup{or.maybe} \textsc{lim} \textsc{dem} \textup{3du-}\textsc{r}\textup{{}-together} \textup{3du-}\textsc{r}\textup{{}-together}\textsc{{}-ph}\textup{{}-}\textsc{prs} \textup{3du-}\textsc{r}\textup{{}-together-talk:}\textsc{redup-prs} \textup{language}\\
\glt ‘Or else the two of them together may just be… \textbf{the} \textbf{two} \textbf{of} \textbf{them} \textbf{whatyoucallit} \textbf{together}… \uline{the two of them are talking together}.’
\end{exe}

Full mirroring can be complex, especially with verbal targets, and instead the mirroring is often partial: some of morphological elements are matched on the placeholder, others not. In \REF{ex:ponsonnet:17} for instance, the placeholder carries the same person prefix as the target, and its suffix encodes the right TAM category (using the corresponding form from the placeholder conjugation, see (\ref{ex:ponsonnet:14}) in \sectref{sec:ponsonnet:3.2.2}). However, the target verb complex includes a benefactive applicative marker which is not mirrored on the placeholder. 


\ea
{\label{ex:ponsonnet:17}Dalabon (20120720\_003 170 [Film])}\\
\gll \textbf{Buka-h-kenjhbi-minj} \uline{bukah-marnu-yin-inj},     baw kakkak…\\
\textup{3sg/3sg.h}\textsc{-r-ph-ppfv} 3sg/3sg.h\textsc{{}-r}{}-\textsc{ben}{}-say/so-\textsc{ppfv}  \textsc{intj}  grandmother\\

\glt ‘\textbf{She} \textbf{did} \textbf{whatsit}… \uline{She told her}, nah, grandmother…’
\z 

Placeholders with noun targets are often free of any morphology. This may in fact mirror the target if it has no morphology either, as in the second token in example (\ref{ex:ponsonnet:9}) in \sectref{sec:ponsonnet:3.1} (\textit{keninjhbi} \textit{djakana} ‘whatsit \textit{djakana}’, where \textit{djakana} is a bird species). Sometimes it results in a mismatch, as in (\ref{ex:ponsonnet:18}), where we see that the ergative marker \textit{{}-yih} present on the target (\textit{sista} ‘nun’) is absent on the placeholder. 

\ea
{\label{ex:ponsonnet:18}Dalabon (20120721\_000 159 [Film])}\\
\gll Bala-h-yaw-djudju-mu. Kardu   \textbf{kenhbi}   \uline{sista-yih}   kardu.   \\
\textup{3pl-}\textsc{r-dim}\textup{{}-bath:}\textsc{redup-prs} \textup{maybe} \textsc{ph} \textup{nun-}\textsc{erg} \textup{maybe}\\
\glt \textup{‘The little cuties are having a shower. Like the} \textbf{\textup{whatsit}}\textup{, the \uline{nun} showers them maybe.’}\\
\z

Occasionally, a placeholder may carry informative morphology that does not appear on the target. In (\ref{ex:ponsonnet:19}), the speaker explains how a needle is used to pierce seeds when making necklaces. The placeholder standing for \textit{brikno} ‘needle’ carries an instrumental case marker \textit{-yih}.\footnote{Instrumental and ergative are both expressed by the suffix \textit{{}-yih}.} This case is indeed semantically appropriate in the context, but is not repeated on the target. This “transfer” of morphology from the target to the placeholder aligns well with the Dalabon propensity to “morphological flexibility” (\sectref{sec:ponsonnet:2.1}).


\ea
{\label{ex:ponsonnet:19}Dalabon (20100719a\_003~021 [ConvEl])}\\
\gll Yila-h-lng-dulubu-n           kardu   \textbf{keninjbi-yih} \uline{brikno}   yila-h-dulubu-n       kahnunh.\\
\textup{1pl.excl/3sg-}\textsc{r-seq}\textup{{}-pierce-}\textsc{prs} \textup{maybe} \textsc{ph}\textup{{}-}\textsc{inst} \textup{sharp}  1pl.excl/3sg-\textsc{r-seq}\textup{{}-pierce-}\textsc{prs} \textsc{dem}\\
\glt \textup{‘We pierce it} \textbf{\textup{with} \textbf{a} \textbf{whatsit}}\textup{… we pierce it with \uline{a needle} this one.’}\\
\z

Speakers differ in their tendencies to mirror more or less material from the targets onto placeholders, and this variation is discussed in \sectref{sec:ponsonnet:4}. 

Contrary to nouns, verbs in Dalabon always carry morphology: person prefixes and TAM suffixes are obligatory. Most frequently, both the person prefix and the TAM suffix are repeated, i.e. the pattern is “bilateral”. This results in the repetition of the entire verb complex, as illustrated above in (\ref{ex:ponsonnet:17}), for instance. This pattern accounts for 97\% of occurrences in the data set, yet two alternative patterns are observed. The first one is illustrated in (\ref{ex:ponsonnet:20}), where the speaker repeats the TAM suffixes after the verb roots (future \textit{{}-miyan} on the placeholder, future \textit{{}-ngiyan} on the target), but does not repeat the prefixes (person prefix \textit{ka-}, realis mood marker \textit{h-}, sequential \textit{lng-}). A last option, where no material is repeated at all and only the root is replaced, is illustrated in (\ref{ex:ponsonnet:21}).\footnote{In (\ref{ex:ponsonnet:21}), the pause and prosodic reset after the placeholder, as well as the context, clearly indicate that \textit{kenhbi} does not stand for an incorporated noun.} 

\largerpage
\ea
{\label{ex:ponsonnet:20}Dalabon (2220909\_70OK 0638 [El])}\\
\gll Ka-h-lng-\textbf{keninjbi}-miyan {[1.18 sec]}  \uline{yenjdju}-ngiyan.\\
\textup{3sg-}\textsc{r-seq-ph}\textup{-}\textsc{fut} {} \textup{speak-}\textsc{fut}\\
\glt \textup{‘He’ll} \textbf{\textup{whatsit}}\textup{… \uline{speak}.’}
\z

\ea
{\label{ex:ponsonnet:21}Dalabon (20110518b\_002 125 [ConvEl])}\\
\gll Kenbo-yah   dja-h-marnu-\textbf{kenbi}{}-{}-, \uline{bimbu}-yan.\\
\textup{then-just}     {3sg/2sg-}\textsc{r}\textup{{}-}\textsc{ben-ph}\textup{{}-{}-}   draw-\textsc{fut}\\
\glt \textup{‘She’s going to} \textbf{\textup{whatsit}} \textup{\uline{draw} it for you.’}
\z

These observations feed prior discussions on the delimitation of the prosodic word in Dalabon. In Gunwinyguan languages, verb complexes are typically analyzed as unitary words both from a grammatical and from a prosodic point of view – hence the representation of verb complexes as chains of morphemes separated by hyphens in the orthography (\citealt{McKay1975}; \citealt{Saulwick2003}; \citealt{Evans2003bininjgunwok}; \citealt{Baker2008}; \citealt{Kapitonov2021}). However, \citet{EvansEvans2008} observed that verbal complexes are phonologically less coherent in Dalabon than in some other polysynthetic languages in the Gunwinyguan family (e.g. Bininj Gun-Wok). Dalabon complexes allow for pauses between verbal prefixes and the verb root. Along with some other prosodic and syllable-structure properties, this suggests incipient modifications in the structure of the prosodic words. The attestation of non-bilateral patterns of morphological repetition on placeholder targets, illustrated in (\ref{ex:ponsonnet:20}) and (\ref{ex:ponsonnet:21}) above, seems to confirm the same tendency. Of the three patterns presented above, the “bilateral” pattern is the only one that treats the verb complex as a block, while the others split it. In my corpus, this “bilateral” pattern is not only the most frequent by far, but also the only one used by the speaker who has Dalabon as her dominant language, vs Kriol (see \sectref{sec:ponsonnet:4.1}). Increased familiarity with isolating languages like Kriol and English may influence the way speakers parse words in the polysynthetic language(s) in their repertoire. 

\section{Tendencies in use}
\label{sec:ponsonnet:4}
Given the wealth of optional morphological enrichments and variation described above, Dalabon placeholders can be used in many different contexts and many different ways. Indeed, the three speakers who provided the most data to the corpus under examination seemed to adopt individual tendencies or “styles”. In this section, I attempt to characterize these styles based on quantitative comparisons between these three speakers. Naturally, given the nature of the corpus, the observations I propose cannot be conclusive. However, this case study suggests that important research on “disfluency styles” could be carried out based on synthetic, morphologically flexible languages numbering more speakers than Dalabon. 

\subsection{Use of placeholders does not correlate with proficiency}
\label{sec:ponsonnet:4.1}
The three speakers who recorded the most data in the corpus – whom I will call S1, S2 and S3, see \sectref{sec:ponsonnet:2.2} – were all women above 60 years old at the time of recording. As I write, they are all deceased. Biographic information collected across a number of interviews revealed that each of them had a distinctive linguistic history, and Dalabon played a different role in their respective repertoires. A number of studies have shown that L2 speakers tend to use more fillers than L1 speakers (e.g. \citealt{FehringerFehringer2007}, \citealt{Jong2016}), yet others have shown that even for L2 learners, fillers endorse discourse functions beyond the management of hesitation (\citealt{KosmalaCrible2022}). With respect to the Dalabon speakers who provided data for this study, a first observation is that placeholder frequency does not correlate with lesser proficiency in Dalabon, at least not in any evident way. 

While I am not well placed to assess proficiency as such, I can comment on the different “ranks” the Dalabon language used to occupy in each of these speakers’ respective repertoires. S1, who was about 10 years older than the other two, spoke only Dalabon during her early childhood. She had spent the first years of her life in the bush, away from white settlements and influences, and did not attend Western-style school. In her teenage and early adulthood, she acquired Mayali (another Gunwinyguan language),\footnote{A variety of Bininj Gun-Wok.} Kriol, and English to a degree. During the years we worked together, Kriol was the language she used with her family and in most day-to-day contexts, yet as far as I can tell Dalabon remained her dominant language in terms of proficiency. She was regarded by other community members (as well as by linguists) as the primary authority for the Dalabon language.

S2 and S3, both born around 1950, belonged to the generation who “created” Kriol in their late childhood and teenage years \citep{Ponsonnet2010}. S2 had learnt Dalabon first with her parents and grandparents, before adopting Kriol. She acquired English at a quasi-native level via school. Throughout her life, she continued to use Dalabon with some of her ascendant relatives for many years, and remained a relatively balanced bilingual, with a slight advantage for Kriol as her main day-to-day language. 

S3’s linguistic biography was comparable to S2’s except that her parents were speakers of Kunwinjku (another Gunwinyguan variety\footnote{Also a variety of Bininj Gun-Wok.}). She was therefore a balanced bilingual speaker of Kunwinjku and Kriol, and also acquired native-like English proficiency via school. She learnt Dalabon as an adult, in a deliberate endeavor to extend her multilingual repertoire to her grandparents’ endangered language. She positioned herself as a proud “new speaker” of Dalabon \citep{Budrikis2021}. As far as I could judge, her Dalabon proficiency was not far from native-like, although she seemed to master less vocabulary than S1 and S2. 

Extrapolating proficiency on the basis of linguistic biography, S1 was more proficient than S2, who was more proficient than S3. This aligns with their own self and mutual assessments, as well as local reputation as Dalabon speakers (and with my experience of these three speakers’ speech). \tabref{tab:ponsonnet:1} compares this observation with the frequency of placeholders per speaker across the corpus, where each of them contributed a similar proportion of narratives, stimuli-based elicitation, and lexical elicitation. As we can see, S3, who acquired Dalabon later in life, is not the one who used placeholders the most, since S2 uses them twice as often as S3 on average (1 every 2.5 min vs 1 every 5 min). The disparity in frequency of placeholders between S1 and S2, who are both native speakers of Dalabon, is considerable. Therefore, for these speakers and in this data, the frequency of placeholders does not correlate with how well they master the language. This lack of correlation does not mean that proficiency may not, in fact, play a role in how or how much speakers use placeholders. But it does show that with respect to the data set under examination, we need to look elsewhere to explain these three speakers’ tendencies. While the observation that using more fillers does not correlate with lesser proficiency may seem trivial, the question is worth further examination given the existence of “deficit” perspectives on fillers.

\begin{table}
\begin{tabular}{llrr}
\lsptoprule
 & Status of Dalabon & nb of tokens & frequency\\
\midrule
S1 & Dalabon as only L1 & 140 & 1/16 min\\
S2 & balanced bilingual Kriol-Dalabon & 259 & 1/2.5 min\\
S3 & Dalabon as one of L2s & 73 & 1/5 min\\
\lspbottomrule
\end{tabular}
\caption{\label{tab:ponsonnet:1}Frequency of use by speaker}
\end{table}

\subsection{Differentiated styles}
\label{sec:ponsonnet:4.2}

Apart from the clear difference in how frequently S1 and S2 used \textit{keninjhbi}, in this corpus they generally used it in different contexts, targeting different parts of speech, and – consequently – with different morphological enrichments. In this section, I compare and characterize S1’s and S2’s respective styles in handling placeholders. S3’s style will be discussed briefly at the end of the section. Figures are recapitulated in \tabref{tab:ponsonnet:2}, also at the end of the section.

A first, and key difference is in the parts of speech targeted by each speaker’s placeholders. S1 and S2 both targeted nouns more often than verbs, but for S1 the difference was much greater: 75\% nouns vs 16\% verbs; while the distribution was 51\% vs 40\% for S2.\footnote{The figures do not add up to 100\%, because they also account for other parts of speech, irretrievable targets, etc.} This difference between S1 and S2 is statistically significant.\footnote{Based on chi-square test returning  $\chi^2 (1)$ = 24.179, \textit{p} < .001}

These figures reflect that S1’s placeholders occurred more frequently in contexts considered “lexically challenging” – i.e. in lexical elicitation, when discussing natural species, or with borrowings (see \sectref{sec:ponsonnet:3.1}). Natural species form a very large lexical class with hundreds of items, some of them rarely evoked in these speakers’ day-to-day lives. In (\ref{ex:ponsonnet:22}), S1 comments on a movie where she identifies a specific plant. The corresponding noun does not occur elsewhere in my corpus, and S1 might not have heard it for several years prior to this recording session. Contexts comparable to this one, or to example (\ref{ex:ponsonnet:10}) (in \sectref{sec:ponsonnet:3}) with the borrowing \textit{peismeika} ‘pacemaker’, represent 40\% of S1’s tokens, but only 12\% of S2’s (also a statistically significant difference\footnote{$\chi^2 (1)$ = 40.659, \textit{p} < .001}). 

\ea
{\label{ex:ponsonnet:22} Dalabon (20120713a\_000 072 [Film])}\\
\gll Bula-h-beyu-ngiyan  kanh   \textbf{keninjhbi}   \uline{kuladj}.\\
\textup{3pl/3sg-}\textsc{r}\textup{{}-fetch-}\textsc{fut} \textsc{dem} \textsc{ph} \textup{spike-rush}\\
\glt \textup{‘They are getting this} \textbf{\textup{whatyoucallit}}\textup{, this \uline{spike-rush}.’}\\
\z

By contrast, in this data set, placeholders with verbal targets often stood for relatively common verbs. This is illustrated in (\ref{ex:ponsonnet:23}), where S2’s verbal target, \textit{djawan}, is the standard verb for ‘ask’ and occurs at least two hundred times in the corpus. Overall, S1 used \textit{keninjhbi} more often in contexts like (\ref{ex:ponsonnet:22}), and S2 in contexts like (\ref{ex:ponsonnet:23}). This correlates with the stronger prevalence of noun targets for S1, and the comparatively large proportion of verb targets for S2.

\ea
{\label{ex:ponsonnet:23}Dalabon (20110605\_002 232 [Stim])}\\
\gll Kardu   \textbf{ka-h-kenjhbi-minj}   malung, \uline{buka-h-ye-djawa-nj}.\\
\textup{maybe}   {3sg/3sg-}\textsc{r-ph}\textup{{}-}\textsc{ppfv} \textup{first} \textup{3sg/3sg.h-}\textsc{r}\textup{{}-}\textsc{com}\textup{{}-ask-}\textsc{ppfv}\\
\glt \textup{‘Maybe} \textbf{\textup{he} \textbf{whatsit}} \textup{first, \uline{he asked him}.’}
\z

In line with the type of contexts in which S1 and S2 used placeholders, on average S1 realized the targeted lexical item more often than S2. Overall, they both realized it relatively often: 77\% of realizations for S1, and 69\% for S2. The difference between speakers is not very marked here (and indeed not statistically significant), but it converges with the speakers’ preferred contexts. In a context like (\ref{ex:ponsonnet:22}), preferred by S1, the form \textit{kuladj} is in fact part of what the speaker is communicating about – i.e. teaching the addressee (myself) the name of the plant in question. In this situation, the placeholder helps hold the space while searching for a key lexical element. In a context like (\ref{ex:ponsonnet:23}), on the other hand, eventually producing \textit{djawan} for ‘ask’ is less relevant, as long as the addressee can grasp the intended meaning based on the morphology on the placeholder, and contextual clues. In such contexts, the speaker may care less about ultimately uttering the target. 

Another difference between S1’s and S2’s placeholders is their morphological enrichments. Only 49\% of S1’s occurrences carry morphology, against 66\% of S2’s (another statistically significant difference).\footnote{$\chi^2 (1)$ = 11.132, \textit{p} < .001} This, again, correlates with S2’s preference for verb targets: these require some morphology, which placeholders with noun targets do not. This is reflected in the proportion of verbal versus nominal morphology in the speakers’ respective data sets, as highlighted in \tabref{tab:ponsonnet:2}. As a result, S2’s tokens generally convey more information. This is firstly because they carry more morphology; and secondly because the morphology they carry, being verbal morphology, conveys crucial information about clause participants. 

Bringing these observations together, we can portray S1’s and S2’s respective styles of placeholder management as follows. S1 used placeholders relatively “sparingly”, privileging lexically challenging contexts, which typically involve nouns. Aiming for lexical accuracy, she took care to eventually realize the lexical target more than three quarters of the time. S2, by contrast, made more “liberal” use of placeholders, in the sense that she recruited \textit{keninjhbi} more often, and in relatively unmarked contexts where the realization of the target mattered less. Relative to S1, S2 targeted more verbs, which were less lexically challenging. This in turn implied that S2’s placeholders carried more morphological information, and this meant that realizing the target was even less relevant. Overall, S1’s placeholders were geared towards lexical accuracy, while S2’s seemed to privilege fluid communication. 

This difference in orientation is evocative of \citegen{SeyfeddinipurSeyfeddinipur2008} distinction between using fillers for accuracy, i.e. to maintain accurate speech production; and using fillers for fluency, i.e. to avoid disruptions in the flow of speech. Their study examines these orientations in speech production. Here, the comparison between S1 and S2 suggests that a parallel distinction can be made at the level of discourse, i.e. with respect to speakers’ motivation to use placeholders for better communication of content. 

These differentiated styles of “placeholding” may be interpreted in the light of the speakers’ respective proficiencies in Dalabon. Indeed, it may be the case that for S2, managing Dalabon fluency required slightly more cognitive effort than for S1, which could explain her introducing \textit{keninjhbi} in place of frequent lexical items, including verbs. While this is very plausible, the comparison with S3 adds another perspective. 

As mentioned above, S3 learnt Dalabon as an adult, yet she used \textit{keninjhbi} less than S2 (see \tabref{tab:ponsonnet:1} in \sectref{sec:ponsonnet:4.1}). Like S1’s, S3’s placeholders targeted many more nouns than verbs; but contrary to S1, this did not result from a preference for lexically challenging contexts, as only 5\% of her tokens occurred in such contexts (vs 40\% for S1 and 12\% for S2). If we now compare S2 with S3, the higher rate of verb targets observed with S2 no longer seems to result solely from S2 being cognitively challenged when speaking Dalabon. Instead it could partly reflect a personal preference – and as pointed out above, this preference for verb targets allows her to make her placeholders morphologically richer, more informative, so that they optimize fluid communication. 

\begin{table}
\begin{tabular}{lrrr}
\lsptoprule
 & S1 & S2 & S3\\
\midrule
 Placeholder frequency & 1/16 min & 1/2.5 min & 1/5 min\\
Nominal targets & 75\% & 51\% & 60\%\\
Verbal targets & 16\% & 40\% & 20\%\\
Challenging contexts & 40\% & 12\% & 5\%\\
Target is realized & 77\% & 69\% & 67\%\\
PH has morphology & 49\% & 66\% & 51\%\\
of which verbal morphology & 32\% & 59\% & 37\%\\
of which nominal morphology & 65\% & 41\% & 62\%\\
\lspbottomrule
\end{tabular}
\caption{\label{tab:ponsonnet:2}
Distribution of features in speakers’ data}
\end{table}

Some clarification is in order here regarding the word “preference”. I am not suggesting that the styles sketched above involve a calculated, deliberate, or even conscious decision to use \textit{keninjhbi} in one way or another. What I am proposing, instead, is that a speaker like S2 may be less reluctant to use placeholders in contexts where some other speakers may deploy more effort to avoid them. In other words, where S1 and S3 seemed to filter out placeholders with verb targets, S2 did not, and from there automatically exploited the resulting communication benefits that followed. Without involving a conscious decision, aiming to filter out certain occurrences or not is a choice. In the case of placeholders particularly, such a choice is likely to be underpinned by a combination of contextual pressures (particularly in the context of linguistic elicitation), speakers’ personal perceptions of linguistic norms, and general orientation in linguistic communication (see \sectref{sec:ponsonnet:1}). All these questions call for investigation in future research in languages with larger populations of speakers. 

\section{Conclusions} 
\label{sec:ponsonnet:5}

The analysis of the morphosyntactic behavior and usage of placeholders in Dalabon I presented in this chapter has shown how disfluency-management devices can lend themselves to stylistic elaboration. This is particularly evident in a polysynthetic language like Dalabon, where rich morphology along with syntactic and morphological flexibility of the placeholder increase the range of options speakers can choose from when they use placeholders. Because the Dalabon placeholder, \textit{keninjhbi}, can target virtually any word class, and endorse almost any morphology, different speakers use it in different contexts, load it with different morphology, and as a result recruit it for different communicative purposes. For many reasons – among which primarily the fact that it is no longer extensively spoken – Dalabon cannot really help us answer the questions that flow from the above observations, but I hope that they can be further explored in future research involving morphologically rich vital languages.

\section*{Acknowledgements}

I am extremely grateful to all the Dalabon speakers who accepted to share their language with me and the broader scientific community. Most of the data analyzed here was collected thanks to a Individual Postdoctoral Fellowship (IPF0229) from the Endangered Languages Documentation Project.

\section*{Abbreviations}

\subsection*{Glosses}
Glosses not listed in the Leipzig Glossing Rules: 

\begin{multicols}{2}
\begin{tabbing}
MMMM \= conjunction\kill
\textsc{conj} \>  conjunction\\
\textsc{dim} \>  diminutive\\
\textsc{emph} \>  empathic\\
h \>  higher in animacy\\
\textsc{intens} \>  intensifier\\
\textsc{intj} \>  interjection\\
\textsc{lim} \>  limitative\\
\textsc{pcust} \>  customary past\\
\textsc{ph} \>  placeholder\\
\textsc{pimp} \>  past imperfective\\
\textsc{ppfv} \>  past perfective\\
\textsc{r} \>  realis\\
\textsc{redup} \>  reduplication\\
\textsc{seq} \>  sequential\\
\textsc{sub} \>  subordinate\\
\textsc{thmc} \>  thematics\\
\textsc{vblzr} \>  verbalizer\\
x/y \>  argument x acting upon \\ \> argument y (e.g. 2sg/3sg)\\
(Kr) \>  Kriol
\end{tabbing}
\end{multicols}

\subsection*{Data types}

\begin{tabularx}{\textwidth}{@{}lQ@{}}
{[ConvEl]}& conversation in elicitation\\
{[El]}  & elicitation\\
{[Film]}& comment on movie\\
{[Narr]}& narrative\\
{[Stim]}& comment on visual stimulus\\
\end{tabularx}

\printbibliography[heading=subbibliography,notkeyword=this]
\end{document}
