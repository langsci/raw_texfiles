\documentclass[output=paper,colorlinks,citecolor=brown]{langscibook}
\ChapterDOI{10.5281/zenodo.15697589}
\author{Christian Döhler\orcid{0000-0002-9659-5920}\affiliation{Berlin-Brandenburg Academy of Sciences and Humanities}}
%\ORCIDs{}

\title{\textit{That} placeholder in \uppercase{K}omnzo}

\abstract{Speakers of Komnzo have a number of conventionalised devices for situations of disfluency. In addition to silent pauses, there are hesitative and placeholder fillers. This contribution places a focus on the placeholder \textit{bäne} which is a pronoun, or pro-form in the language. The chapter consists of a description of the form, distribution, functions, functional extensions, frequency, and multimodality of \textit{bäne}. It therefore contributes to the emerging typology of placeholders.

\keywords{Yam languages, Southern New Guinea, fillers, placeholders, demonstratives}
}

\IfFileExists{../localcommands.tex}{
   \addbibresource{../localbibliography.bib}
   \usepackage{langsci-optional}
\usepackage{langsci-gb4e}
\usepackage{langsci-lgr}

\usepackage{listings}
\lstset{basicstyle=\ttfamily,tabsize=2,breaklines=true}

%added by author
% \usepackage{tipa}
\usepackage{multirow}
\graphicspath{{figures/}}
\usepackage{langsci-branding}

   
\newcommand{\sent}{\enumsentence}
\newcommand{\sents}{\eenumsentence}
\let\citeasnoun\citet

\renewcommand{\lsCoverTitleFont}[1]{\sffamily\addfontfeatures{Scale=MatchUppercase}\fontsize{44pt}{16mm}\selectfont #1}
  
   %% hyphenation points for line breaks
%% Normally, automatic hyphenation in LaTeX is very good
%% If a word is mis-hyphenated, add it to this file
%%
%% add information to TeX file before \begin{document} with:
%% %% hyphenation points for line breaks
%% Normally, automatic hyphenation in LaTeX is very good
%% If a word is mis-hyphenated, add it to this file
%%
%% add information to TeX file before \begin{document} with:
%% %% hyphenation points for line breaks
%% Normally, automatic hyphenation in LaTeX is very good
%% If a word is mis-hyphenated, add it to this file
%%
%% add information to TeX file before \begin{document} with:
%% \include{localhyphenation}
\hyphenation{
affri-ca-te
affri-ca-tes
an-no-tated
com-ple-ments
com-po-si-tio-na-li-ty
non-com-po-si-tio-na-li-ty
Gon-zá-lez
out-side
Ri-chárd
se-man-tics
STREU-SLE
Tie-de-mann
}
\hyphenation{
affri-ca-te
affri-ca-tes
an-no-tated
com-ple-ments
com-po-si-tio-na-li-ty
non-com-po-si-tio-na-li-ty
Gon-zá-lez
out-side
Ri-chárd
se-man-tics
STREU-SLE
Tie-de-mann
}
\hyphenation{
affri-ca-te
affri-ca-tes
an-no-tated
com-ple-ments
com-po-si-tio-na-li-ty
non-com-po-si-tio-na-li-ty
Gon-zá-lez
out-side
Ri-chárd
se-man-tics
STREU-SLE
Tie-de-mann
}
   \boolfalse{bookcompile}
   \togglepaper[23]%%chapternumber
}{}

\begin{document}
\maketitle
\graphicspath{{figures/doehler}}
\section{Introduction}\label{sec:doehler:intro}

This chapter describes and analyses fillers in Komnzo, a language of the Yam family spoken in the south-west of Papua New Guinea.\footnote{ISO 639-3: tci, Glottocode: komn1238} Fillers are linguistic devices that are used in situations of disfluency, i.e., they are filling a silent pause. They can be divided into hesitative fillers (e.g.\ English \textit{uhm}, Komnzo \emph{a}) and placeholder fillers (e.g.\ English \textit{whatchamacallit}, Komnzo \emph{bäne}). The former are non-referential and not syntactically integrated, while placeholders are referential and syntactically integrated (\cite{Hayashi:2010xh}).\footnote{Previous classifications have defined hesitative fillers as non-lexical but nonetheless conventionalized sounds, while placeholder fillers are lexical items (cf. \cite{Amiridze:2010aa}).} 

While I touch on hesitative fillers only in passing, the main focus of the chapter is on the placeholder \textit{bäne}/\textit{baf}. I show that this placeholder is best analyzed as a pro-form which has developed from a medial demonstrative. I argue that the use of \textit{bäne}/\textit{baf} goes well beyond filling a silent pause. It is used intentionally with communicative goals such as signalling a taboo context, or discourse managing goals such as gaining the floor. Moreover, I show that certain inflections of \textit{bäne} have evolved into conventionalized connectors for adverbial clauses.

In the remainder of this section, I introduce the sociolinguistic situation (\sectref{sec:doehler:socioling}), the text corpus  (\sectref{sec:doehler:corpus}), and some typological features of the languages (\sectref{sec:doehler:typo} and \sectref{sec:doehler:demonstratives}). \sectref{sec:doehler:overview} provides an overview of hesitative and placeholder fillers in Komnzo. The main body focusses on the placeholder \textit{bäne}/\emph{baf}, for which I describe its form (\sectref{sec:doehler:baneform}), distribution (\sectref{sec:doehler:banedist}), functions (\sectref{sec:doehler:banefunc}), functional extensions (\sectref{sec:doehler:banefuncext}), and multimodal aspects (\sectref{sec:doehler:gestures}). After addressing the problems in measuring the frequency of \textit{bäne}/\textit{baf} (\sectref{sec:doehler:freq}), I close with some final comments (\sectref{sec:doehler:concl}).

\subsection{Sociolinguistic background}\label{sec:doehler:socioling}

Komnzo is a small language even by the standards of Papua New Guinea, where language communities tend to be rather small. Komnzo is spoken by approximately 200–250 speakers in the villages of Rouku and Morehead Station. Genetically, the language belongs to the Tonda subgroup of the Yam languages. \figref{fig:doehler:yamfamily} shows a map of the language family.

\begin{figure}
    \includegraphics[width=\textwidth]{Yam-family-2.png}
    \caption{The Yam language family}
    \label{fig:doehler:yamfamily}
\end{figure}

Komnzo speakers live in a highly multilingual language ecology. Due to a marriage pattern of sister-exchange with exogamous groups based on clan, place, and -- epiphenomenally -- on language variety, virtually all children grow up with at least two languages. In reality, the portfolio of most children includes 4-5 languages by the time they reach adulthood.

Komnzo speakers live in a small-scale traditional society, i.e.\ what has been called ``society of intimates'' (\cite{Givon:2018sl}, \cite{Givon:2002nn}). In this type of social setting all generic knowledge is shared and almost all daily interactions take place between individuals who have known each other for a long time. This results in a large degree of common ground, thus, leading to higher informational predictability in face-to-face conversation.

\subsection{Text corpus and methodology}\label{sec:doehler:corpus}

The data discussed in this chapter is based on recordings made between 2010 and 2015, archived as (\cite{Dohler:2020tx}). The corpus used here comprises around 12 hours of various text genres, including both natural and stimuli-based narratives and conversations (\tabref{tab:doehler:corpus}). The overall size is around 55,000 word tokens, which makes the Komnzo text corpus a typical ``language documentation corpus'' (\cite{Mosel:2012ul}).

\begin{table}[H]
	\begin{tabular}{lc}
		\lsptoprule
		  Text type& hh:mm:ss\\
        \midrule
		  Conversations&01:01:55\\
		  Conversational tasks&01:49:51\\
		  Narratives&06:40:18\\
		  Procedural texts&02:11:36\\
		  Public speeches&00:42:38\\
        \midrule
		  Total&12:26:18\\
        \lspbottomrule
	\end{tabular}
	\caption{Corpus overview}	
	\label{tab:doehler:corpus}
\end{table}%corpus

All examples are referenced with a source code of the following format: [tci-YYYYMMDD-NN SSS \#\#]. The first part identifies the transcription file. Each session and the included files start with the ISO 639-3 code for Komnzo: tci. Next comes the date of the recording (YYYYMMDD) and the session number on that date (NN). The second part identifies the example within the transcription file. Transcription tiers are sorted by speaker (SSS). Intonation units on the respective transcription tiers are numbered (\#\#). Thus, example (\ref{ex:nzmär}) in this chapter has the source code [tci20150906-10 ABB 303--306]: it is the tenth recording session on September 6th, 2015, the speaker is Abia Bai (ABB), and on the speaker's text tier, the example shows the annotation units 303--306. The corpus of transcribed and interlinearised texts has been archived as (\cite{Dohler:2020tx}).\footnote{While (\cite{Dohler:2020tx}) contains a zipped file of all transcriptions, the audio-visual footage of each session can be found under: \url{https://zenodo.org/communities/komnzo}}

Most examples in this chapter include a figure showing the wave file, the pitch contour, and the transcription. Pauses in the example were measured in ELAN and are shown with three dots on the text tier (...) and in milliseconds on the gloss tier (ms). For producing the figures, the wav-files were exported from ELAN,  normalized and converted to mono track files in Audacity, and finally imported to Praat.\footnote{ELAN (version 6.7): \url{https://archive.mpi.nl/tla/elan}, Audacity (version 2.4.2): \url{http://audacity.sourceforge.net/}, Praat (version 6.4.04): \url{https://www.fon.hum.uva.nl/praat}} All processed wav-files, Praat pictures, and video screenshots can be downloaded from: \url{https://zenodo.org/doi/10.5281/zenodo.12032997}.

\subsection{Typological overview}\label{sec:doehler:typo}

Komnzo is a double-marking language, in which the verb indexes core arguments and noun phrases are flagged for case. The case marking is organised in an ergative-absolutive system. In addition to four core cases (absolutive, ergative, dative, possessive), there are 13 semantic cases. Verbs are by far the most complex part of speech in the language. Verbs mark person, number and gender of up to two participants, 18 TAM categories, valency, directionality and deictic status. Complexity lies not only in the amount of grammatical categories that can be expressed morphologically, but also in the way these categories are encoded (\cite[175ff.]{Dohler:2018qt}). This is best described by the term ``distributed exponence'' (\cite{Caballero:2012fq}, \cite{Carroll:2016bf}), a subtype of multiple exponence.

This aspect of the language is not the topic of this chapter, but it has a practical effect for the presentation of example sentences, in that I do not show the morpheme segmentation of verbs. Instead, I apply the word-and-paradigm approach (\cite{Matthews:1974yw}): In the morpheme tier, I separate the verb stem from affixal material by placing it between \textbackslash slashes/. In the gloss tier, I list the relevant grammatical categories (argument structure, TAM, directionality) followed by the lexeme translation.

\subsection{Demonstratives}\label{sec:doehler:demonstratives}

Before we proceed it is worth defining the notion of demonstratives adopted in this chapter and presenting some basics about the system of demonstratives in Komnzo. As a point of departure, I follow \textcite[2ff.]{Diessel:1999yp} in assuming that the most basic function of demonstratives is a spatial (or situational) use, but see \textcite{Himmelmann:1996sn} and \textcite{De-Mulder:1996gd} for a critique of this view. Based on this functional definition, we can identify the forms given in \tabref{tab:doehler:demonstratives} as demonstratives.

\begin{table}
    \begin{tabular}{ll lll l}
        \lsptoprule
            & \gl{pronoun /}&\multicolumn{3}{c}{\gl{adverb}}&\gl{clitic}\\\cmidrule(lr){3-5}
            & \gl{determiner}& \gl{neutral} & \gl{allative} & \gl{ablative}&\\
        \midrule
            \gl{prox} &\textit{zane}& \textit{zä} & \textit{zbo} & \textit{zba}&\textit{z=}\\
            \gl{med} &(\textit{bäne})& \textit{bä} & \textit{bobo} & \textit{boba}&\textit{b=}\\
            \gl{dist} &(\textit{ane})& \textit{fä} & \textit{fobo} & \textit{foba}&\textit{f=}\\
            \gl{q} &\textit{mane}& \textit{mä} & \textit{mobo} & \textit{moba}&\textit{m=}\\
        \lspbottomrule
    \end{tabular}
    \caption{Demonstratives in situational uses}
    \label{tab:doehler:demonstratives}
\end{table}%demonstratives

The demonstrative paradigm is organized in a four-way split into proximal, medial, distal, and (spatial) interrogative.\footnote{The latter category is glossed as \gl{q}, and it is used for questions that pertain to space: \textit{mane} `which one', \textit{mä} `where', \textit{mobo} `whither', and \textit{moba} `whence'.} \tabref{tab:doehler:demonstratives} shows that the distinction is signalled by the initial consonant: /z/ for proximal, /b/ for medial, /f/ for distal, and /m/ for interrogative. In this aspect, demonstratives are related to person deixis, i.e. personal pronouns. The first person singular pronouns start with /nz/ (\textit{nzä} \gl{1sg.abs})\footnote{Note that the Komnzo first person pronoun starts with /nz/, and not /z/. This is different in closely related varieties such as Wära, Anta, and Wèré where first person pronoun all begin with /z/.}, second person pronouns with /b/ (\textit{bä} \gl{2sg.abs}), and third person pronouns with /f/ (\textit{fi} \gl{3.abs}). The implications of this link, especially between second person and the placeholder \textit{bäne}, are addressed in \sectref{sec:doehler:baneform}.

Syntactically demonstratives belong to different parts of speech, which align with Diessel's classification: pronouns, determiners, adverbs, and identifiers (\cite{Diessel:1999yp}). The elements shown in the leftmost column can be used both adnominally and pronominally, i.e. they function as determiners (\textit{zane mni=me} [\gl{prox} fire=\gl{ins}] `with this fire') and pronouns (\textit{zane=me} [\gl{prox=ins}] `with this one'). The elements in the middle of the table function as adverbs (\textit{zä} `here', \textit{zbo} `hither', \textit{zba} `hence'). The elements in the right column are verbal proclitics, and their most frequent use is as part of a presentational (or identificational) construction.\footnote{The proclitics can attach to any inflected verb. In the presentational construction, they attach to the copula which follows the main verb of the clause (cf. \cite[109ff., 288]{Dohler:2018qt}).}

In the remainder of this section, I will focus on the forms in the leftmost column of \tabref{tab:doehler:demonstratives}. There are two elements in the table, namely \textit{bäne} and \textit{ane}, which formally belong in this paradigm, but are never used situationally, i.e. they do not point to something in space. The first is the anaphoric \textit{ane}, for which there is evidence that is has developed from an older form \textit{fane} (\cite[110ff.]{Dohler:2018qt}). \textit{Ane} no longer has the (distal) spatial reference that is suggested by its position in the paradigm. Instead, it is used anaphorically for referents or sometimes for a whole proposition that has been established in the preceding discourse, i.e.\ it is used for ``tracking'' (cf. \cite{Himmelmann:1996sn}). This is shown with both tokens in (\ref{ex:nzmär}). Note that, unlike \textit{ane}, the proximal \textit{zane} can be used both anaphorically and cataphorically. Syntactically, \textit{ane} functions as a pronoun (\ref{ex:nzmär}) and adnominally as a determiner (\ref{ex:selen}).

\ea \label{ex:nzmär}
    \gll ruga nzmär=me, yti \textbf{ane} thf\stem{konzr}mth fof, yti, \textbf{ane=me} za\stem{nänzütham}ath\\
    pig grease=\gl{ins} \gl{pn} \gl{dem} \gl{3pl>3pl:pst:dur}{\textbackslash}speak \gl{emph} \gl{pn} \gl{dem=ins} \gl{3pl>3sg.f:pst:pfv}{\textbackslash}paint\\
    \glt `with pig grease. They were really calling \textbf{this} \textit{yti}. They painted her \textbf{with} \textbf{this}.' \exsource{tci20150906-10 ABB 303-306}
\z

The second element is the placeholder \textit{bäne}/\textit{baf}. It has lost the spatial function that is suggested by its position in the paradigm (medial). Moreover, \textit{bäne}/\textit{baf} is almost never used adnominally, i.e. as a determiner. Such examples were not only assessed as ungrammatical during elicitation, but the corpus also points in this direction. Example (\ref{ex:gastol}) is the only example (out of more than 700 tokens in the corpus) that could be argued to be an adnominal use. This confirms \textcite{Hayashi:2010xh}, who mention that the majority of placeholder uses of demonstratives in their comparative study of Japanese, Korean and Mandarin were pronominal rather than adnominal. Note also that \textit{bäne}/\textit{baf} is never used anaphorically in Komnzo. For further elaboration on \textit{bäne}/\textit{baf}, I refer the reader to \sectref{sec:doehler:placeholdersov} and \sectref{sec:doehler:bane}.

\section{Overview of hesitators and placeholders}\label{sec:doehler:overview}

This section describes hesitative fillers (\sectref{sec:doehler:hesitatorsov}), placeholder fillers (\sectref{sec:doehler:placeholdersov}) and other fillers (\sectref{sec:doehler:stylistic}). In the section on placeholder fillers, I introduce three devices: the placeholder \textit{bäne}/\textit{baf}, the light verb \textit{-rä} `do', and the manner demonstrative \textit{nima}. These can be used to replace nominal elements, verbs or entire sections of discourse.

\subsection{Hesitative fillers}\label{sec:doehler:hesitatorsov}

Disfluency in the speech of Komnzo speakers can manifest itself as a stretch of silence, or it can be filled by a hesitator.\footnote{I use the term ``hesitator'' and ``hesitative filler'' interchangeably.} Hesitators are usually pronounced as open vowels of variable length, often [æ]$\sim$[ɐ]. The hesitator is usually followed by a very short pause before fluent speech continues. Examples from the Komnzo text corpus confirm what has been written about disfluencies elsewhere; e.g., \textcite{Clark:2002rw} schematize disfluencies into three phases: first, a \textit{suspension} of fluent speaking; secondly, a \textit{hiatus} in speaking, which may contain a stretch of silence or a hesitative filler (accompanied by other collateral actions like gestures); and thirdly, a \textit{resumption} of fluent speaking.

Example (\ref{ex:nomni}) shows a clause initial hesitator, produced as [æ]. The pause between the hesitator and the following \gl{np} \textit{no mni} is very short (70ms), the same length as the pause between the clause and the postposed \gl{np} \textit{kafsin} (cf. \figref{fig:doehler:nomni}).

\ea \label{ex:nomni}
    \gll \textbf{ä} ... no mni f=\stem{rä} ... kafsi=n\\
    \gl{hes} (70ms) water hot \gl{dist}=\gl{3sg.f:npst:ipfv}{\textbackslash}be (70ms) cup(\gl{e})=\gl{loc}\\
    \glt `uh there was tea there in a cup.' \exsource{tci20120924-01 TRK 44}
\z

\begin{figure}
    \includegraphics[width=\textwidth]{tci20120924-01_TRK_44.pdf}
    \caption{Audio analysis of [tci20120924-01 \textsc{trk} 44]}
    \label{fig:doehler:nomni}
\end{figure}

Another hesitator is shown in example (\ref{ex:selen}), produced as [ɐ], which is followed by a pause and a relative clause. The pause following the hesitator is much longer (320ms) than the pause in the previous example, but relatively speaking, it is still very short (cf. \figref{fig:doehler:selen}). This becomes evident when we compare it to the pause following the placeholder \textit{bafen} in the same example, which is twice as long (660ms).

\ea \label{ex:selen}
    \gll ane fam fof ŋa\stem{rä}r \textbf{a} ... monme san\stem{thb}ath bobo \textbf{baf=en} ... \uline{sel=en}.\\
    \gl{dem} thought \gl{emph} \gl{3sg:npst:ipfv}{\textbackslash}do \gl{hes} (320ms) how \gl{3pl>3sg.m:pst:pfv:venit}{\textbackslash}put\_in \gl{med:all} \gl{ph=loc} (660ms) cell(\gl{e})=\gl{loc}\\
    \glt `He is thinking of, uhm, how they put him \textbf{into the whatchamacallit}, \uline{into} \uline{the cell}.' \exsource{tci20111004 RMA 414--416}
\z

\begin{figure}
    \includegraphics[width=\textwidth]{tci20111004_RMA_414–416.pdf}
    \caption{Audio analysis of [tci20111004 \textsc{rma} 414--416]}
    \label{fig:doehler:selen}
\end{figure}

\subsection{Placeholder fillers}\label{sec:doehler:placeholdersov}

There are three kinds of placeholder fillers in Komnzo: the placeholder \textit{bäne}/\textit{baf}, the light verb \textit{-rä} `do', and the manner demonstrative \textit{nima}. These can be used to replace nominal elements, verbs or entire sections of discourse. I follow \textcite{Hayashi:2006aa} in their definition of placeholders as referential expressions that are used as a substitute for a specific lexical item. Under this definition, placeholders occupy ``a syntactic slot that would have been occupied by the target word'' (\citeyear[490]{Hayashi:2006aa}). For the present chapter, I expand the definition to include larger units, e.g.\ a whole clause or a proposition.

\subsubsection{The placeholder \textit{bäne}/\textit{baf}}\label{sec:doehler:baneintro}

The word that follows the above definition most closely is the placeholder \textit{bäne}/\textit{baf}. We have seen a rather typical example of the placeholder in (\ref{ex:selen}), in which the speaker has problems finding the appropriate word, and finally uses an English loan \textit{sel} `prison cell'. The placeholder is flagged with the locative case (\textit{=en}), and after a short pause the speaker produces the target \textit{sel} with the appropriate case flag. Thus, the placeholder \textit{bäne} is used pronominally. We will see in \sectref{sec:doehler:bane} that there are a number of functional extensions of this pattern.

Before moving on to introduce the other types of placeholders, I want to point to a recent change in my analysis of \textit{bäne}. Formally, \textit{bäne} patterns with the demonstratives (cf. \tabref{tab:doehler:demonstratives}), and based on its initial consonant /b/ it belongs in the medial category. Such a link is not surprising from a cross-linguistic perspective. \textcite{Hayashi:2006aa} report for Japanese, Korean, Mandarin and Indonesian that certain demonstratives also function as placeholders. This is akin to the analysis that I have adopted in the past, i.e., there is a distinction between the demonstrative \textit{bäne} and the placeholder \textit{bäne} (\cite[112]{Dohler:2018qt}). The analytic criteria for setting up this distinction were based on prosody and syntax. For example, a break in the intonation contour through a short pause signals a disfluency situation and following from this such examples were analysed as placeholders (e.g.\ \textit{bäne (.) kabe} `who's-that ... the man'). If there is no disfluency situation, \textit{bäne} is analysed as a demonstrative (e.g.\ used adnominally: \textit{bäne kabe} `that man'). Based on a rigorous inspection of corpus examples, I have now abandoned this analysis.

Despite its etymological origin in the system of demonstratives, I have come to the conclusion that \textit{bäne} is not used as an (exophoric/situational) demonstrative at all. The prosody of all inspected tokens points to disfluency situations, and what seemed like adnominal uses of a demonstrative can be accounted for by assuming a nominal compound in which the first noun is filled by a placeholder, as in example (\ref{ex:kwras}) below. I put this here as a first introduction to \textit{bäne}, and as a disclaimer, before we delve into the details in \sectref{sec:doehler:bane}.

\ea \label{ex:kwras}
    \gll kwras nima kam zä\stem{kwthef}a \textbf{bäne} \textbf{zawe} ... \uline{töna} \uline{zawe}\\
    brolga like\_this back \gl{sg:pst:pfv}{\textbackslash}turn \gl{ph} side (1350ms) high\_ground side\\
    \glt `The brolga turned its back to the \textbf{what's-that side} ... to the \uline{land side}.' \exsource{tci20130923-01 ALB 51--52}
\z

\subsubsection{The manner demonstrative \textit{nima}}\label{sec:doehler:nimaintro}

The second element in Komnzo to be discussed here is the manner demonstrative \textit{nima}, which I translate as `like this'. Sometimes it occurs with the instrumental case as \textit{nima=me} [like\_this=\gl{ins}].

This word is not a placeholder per se, but it can be used as a placeholder in certain contexts. \textit{Nima} is used to further elaborate on the manner in which some event was carried out. The manner component is understood from context in most corpus examples, i.e. it is often not spelled out. Additionally, it can be accompanied by a gestural component, as in (\ref{ex:kwras}) above, in which the speaker turns her body away to reenact the movement of the brolga. However, the manner component can also be verbalized, which provides a bridging context for the placeholder use of \textit{nima}. This is also a bridging context for another use of \textit{nima}, namely as a quotative marker. Especially in this latter function, \textit{nima} shares some characteristics with English \textit{like}. In its placeholder use, but also in the use as a quotative marker, \textit{nima} is always followed by a pause, as in (\ref{ex:sitau}).

There are two differences to the placeholder \textit{bäne}. First, \textit{nima} never substitutes a nominal element. As a manner demonstrative, it is used in place of a more complex event, often a stretch of discourse. What follows \textit{nima} in speech is usually a more fine-grained elaboration. In example (\ref{ex:sitau}), the speaker elaborates on the people who were present in the situation. In this placeholder use of \textit{nima}, there is a long pause (1060ms) followed by a whole clause (cf. \figref{fig:doehler:sitau}).

\ea \label{ex:sitau}
    \gll wati, ä\stem{kwa}thake \textbf{nima} ... sitau=aneme afa kwark b=ya\stem{r}a nafanm ä\stem{kwa}ne, nzenm=wä\\
    then \gl{1pl>3pl:pst:ipfv}{\textbackslash}cut\_meat like\_this (1060ms) \gl{pn=nsg.poss} father deceased \gl{med}=\gl{3sg.m:pst:ipfv}{\textbackslash}be \gl{3nsg.dat} \gl{1pl>3du:npst:ipfv}{\textbackslash}cut\_meat \gl{1nsg.dat}=\gl{emph}\\
    \glt `Then, we cut the meat \textbf{like this} ... Sitau's late father (and his wife) were there. We cut some for them, and also some for us.' \exsource{tci20120821-02 LNA 95--97}
\z
\begin{figure}
    \includegraphics[width=\textwidth]{tci20120821-02_LNA_95–97.pdf}
    \caption{Audio analysis of [tci20120821-02 \textsc{lna} 95--97]}
    \label{fig:doehler:sitau}
\end{figure}

Secondly, \textit{nima} can be used anaphorically and cataphorically. Example (\ref{ex:zanr}) shows a cataphoric use of \textit{nima}. In the text example, a man suddenly realises that his visiting relative has actually come as the vanguard of a group of headhunters. The cataphoric use (\ref{ex:zanr}) and the placeholder use (\ref{ex:sitau}) of \textit{nima} can be distinguished by the fact that only the latter occurs in disfluency situations. In the placeholder example (cf. \figref{fig:doehler:sitau}), there is a significantly longer pause than in the cataphoric example (cf. \figref{fig:doehler:zanr}): 1060ms versus 280ms.

\ea \label{ex:zanr}
    \gll fi miyatha sf\stem{rä}rm \textbf{nima} ... zan=r zä zf swan\stem{yak}\\
    \gl{3.abs} knowledgeable \gl{3sg.m:pst:dur}{\textbackslash}be like\_this (280ms) kill=\gl{purp} \gl{prox} \gl{eps} \gl{3sg.m:pst:ipfv}{\textbackslash}come\\
    \glt `He knew \textbf{it}: he had come here to kill (people).' \exsource{tci20111119-01 ABB 98--98}
\z
\begin{figure}
    \includegraphics[width=\textwidth]{tci20111119-01_ABB_98–98.pdf}
    \caption{Audio analysis of [tci20111119-01 \textsc{abb} 98--98]}
    \label{fig:doehler:zanr}
\end{figure}

(\ref{ex:zra}) is a typical example of anaphora with \textit{nima}. The speaker makes a resumptive comment summarizing a description of how to catch fish in the swamp. There is no pause or any other sign of disfluency in the audio of this example.

\ea \label{ex:zra}
    \gll zra=ma trikasi \textbf{nima=me} fof \stem{rä}\\
    swamp=\gl{char} story like\_this=\gl{ins} \gl{emph} \gl{3sg.f:npst:ipfv}{\textbackslash}be\\
    \glt `The story about the swamp is \textbf{just like this}.' \exsource{tci20120922-09 DAK 48}
\z

\subsubsection{The light verb \textit{-rä}}\label{sec:doehler:räintro}

The third element that can be used as a placeholder is the light verb \textit{-rä} `do'.\footnote{The verb `do' is heterosemous (cf. \cite{Lichtenberk:1991vt} and \cite{Evans:2010ct}) with the copula verb `be'. In one type of inflectional pattern \textit{-rä} means `do', and in another pattern it means `be' (cf. \cite{Dohler:2023rc}).} Komnzo has a number of light verbs, e.g. \textit{fiyoksi} `make', \textit{-kor} `become', \textit{wäsi} `happen'.\footnote{Some verbs lack an infinitival form. These are given here with a hyphen.} These are used in light verb constructions in which the verb carries the inflectional information, while a nominal element expresses the verbal semantics (cf. \cite[304]{Dohler:2018qt}). A typical context for such constructions is the integration of borrowed nouns (from a verb in the source language), as in (\ref{ex:zek}) below.

\ea \label{ex:zek}
    \gll no=r bobo \textbf{zek} \textbf{krä\stem{r}é}\\
    water=\gl{purp} \gl{med:all} check(\gl{e}) \gl{1sg:irr:pfv}{\textbackslash}do\\
    \glt `I would (go) there and check for water.' \exsource{tci20130903-03 MKW 146}
\z

There are certain contexts in which a speaker uses the verb \textit{-rä} not in a light verb construction, but because s/he has trouble finding the correct lexical entry. In corpus examples of this type, there is usually a short pause after the light verb, and then the correct full verb follows, sometimes the entire clause is repeated with the full verb. We can see this in examples (\ref{ex:bad}) and (\ref{ex:nzürna}) below. In both examples the light verb and the full verb carry the identical inflectional pattern in terms of alignment and TAM categories.

In example (\ref{ex:bad}), the speaker describes the final stage of the ritual destruction of a grave site. Since he has trouble finding the correct verb for `levelling the ground', he pauses (550ms), and then uses the light verb. After a longer pause (1800ms), he continues with the full verb \textit{frmzsi} `prepare, straighten' (cf. \figref{fig:doehler:bad}).

\ea \label{ex:bad}
    \gll wati ane bad kwot we ... \textbf{zf\stem{rä}rme} we ... \uline{zwa\stem{frmnzr}me} nima\\
    then \gl{dem} ground(\gl{abs}) properly also (550ms) \gl{1pl>3sg.f:pst:dur}{\textbackslash}do also (1800ms) \gl{1pl>3sg.f:pst:dur}{\textbackslash}straighten  like\_this\\
    \glt `Then we were properly doing the ground ... we were levelling it like this.' \exsource{tci20120805-01 ABB 831--832}
\z

\begin{figure}
    \includegraphics[width=\textwidth]{tci20120805-01_ABB_831–832.pdf}
    \caption{Audio analysis of [tci20120805-01 ABB 831--833]}
    \label{fig:doehler:bad}
\end{figure}

Example (\ref{ex:nzürna}) comes from the afterword of a recording. The speaker explains that his story is not a fictional story, but a real story. He substitutes the target word \textit{kwthenzsi} `change, turn' with the light verb \textit{-rä}. In this example, there is no pause preceding the light verb, and the following pause (550ms) is rather short (cf. \figref{fig:doehler:nzürna}).

\ea \label{ex:nzürna}
    \gll kabe zokwasi aha nzürna trikasi \textbf{za\stem{r}ath} ... \uline{za\stem{kwthef}ath}\\
    man story yes spirit story \gl{3pl>3sg.f:pst:pfv}{\textbackslash}do (550ms) \gl{3pl>3sg.f:pst:pfv}{\textbackslash}change\\
    \glt `A real story. Yes, they made it into a spirit story ... they changed it.' \exsource{tci20111119-06 MAB 146--147}
\z

\begin{figure}
    \includegraphics[width=\textwidth]{tci20111119-06_MAB_146–147.pdf}
    \caption{Audio analysis of an excerpt of [tci20111119-06 MAB 146--147]}
    \label{fig:doehler:nzürna}
\end{figure}

While Komnzo utilizes light verbs to function as placeholders, there are other Papuan languages that have dedicated placeholder verbs, for example \textit{məgi-} `to do whatever' in Manambu (\cite[576]{Aikhenvald:2008aa}).

\subsection{Other fillers}\label{sec:doehler:stylistic}

There is another element in Komnzo that can be used as a filler, but its main function is connected to narrative style rather than disfluency. Especially the passing of time or the distance that one has travelled is often indicated with a long stretched \textit{e}, realized as [ɛ:], which I gloss as `until'. While this function is not a case of disfluency, the long stretched \textit{e} is sometimes used like a hesitator, and it is the specific context that facilitates this use. In most corpus examples, the long stretched \textit{e} is followed by a place name, as in (\ref{ex:masu}) and (\ref{ex:büdisn}). But there is nothing in the prosody that would differentiate its use for narrative style from a disfluency situation. The fact that the speaker was indeed searching for the place name in the two examples below, only became clear during the transcription of the texts, in which the speakers themselves explained it in this context. 

The stretched \textit{e} can be quite short. In example (\ref{ex:masu}) it is 200ms long (cf. \figref{fig:doehler:masu}), not much longer than the pause between the clause and the postposed noun (180ms).

\ea \label{ex:masu}
    \gll fi we krän\stem{brim} \textbf{e} masu ... garda=me=nzo\\
    \gl{3.abs} also \gl{sg:irr:pfv:venit}{\textbackslash}return until \gl{pln} (180ms) canoe=\gl{ins}=\gl{only}\\
    \glt `He returned to Masu by canoe.' \exsource{tci20100905 ABB 81--82}
\z

\begin{figure}
    \includegraphics[width=\textwidth]{tci20100905_ABB_81–82.pdf}
    \caption{Audio analysis of [tci20100905 \textsc{abb} 81--82]}
    \label{fig:doehler:masu}
\end{figure}

It can also be very long, as in example (\ref{ex:büdisn}), where it is 1600ms and takes up almost half of the length of the intonation unit (cf. \figref{fig:doehler:büdisn}).

\ea \label{ex:büdisn}
    \gll foba fof kre\stem{far}é wi\stem{yak} \textbf{e} büdisn\\
    \gl{dist:abl} \gl{emph} \gl{1sg:irr:pfv}{\textbackslash}set\_off \gl{1sg:npst:ipfv}{\textbackslash}walk until \gl{pln}\\
    \glt `From there I set off and walked until Büdisn.' \exsource{tci20130903-03 MKW 23}
\z

\begin{figure}[h!]
    \includegraphics[width=\textwidth]{tci20130903-03_MKW_23.pdf}
    \caption{Audio analysis of [tci20130903-03 \textsc{mkw} 23]}
    \label{fig:doehler:büdisn}
\end{figure}

\section{The placeholder \textit{bäne}/\textit{baf}}\label{sec:doehler:bane}

This section describes the placeholder \textit{bäne}/\textit{baf} in its form (\sectref{sec:doehler:baneform}), syntactic distribution (\sectref{sec:doehler:banedist}), its various functions (\sectref{sec:doehler:banefunc}) and extensions thereof (\sectref{sec:doehler:banefuncext}). I show three examples of accompanying gestures in \sectref{sec:doehler:gestures} and discuss the problems of measuring the frequency of the placeholder in \sectref{sec:doehler:freq}.

I adopt the terminology of ``delayed constituent'' and ``target'' for the word or larger unit which the placeholder substitutes. In examples, I print the placeholder (and sometimes a larger unit to which the placeholder belongs) in \textbf{bold font} and the delayed constituent or target in \uline{underlined font}, as in (\ref{ex:tot}) below.

\ea \label{ex:tot}
    \gll zöbthé zwa\stem{wärez}é \textbf{bäne=me} ... \uline{kofä} \uline{tot=me}\\
    first \gl{1sg>3sg.f:rpst:pfv}{\textbackslash}aim \gl{ph=ins} (280ms) fish spear=\gl{ins}\\
    \glt `First I aimed at it \textbf{with the whatchamacallit} ... \uline{with the fish spear}.' \exsource{tci20130905-02 MKW 41--42}
\z

\subsection{Form}\label{sec:doehler:baneform}

As I have argued in \sectref{sec:doehler:baneintro}, the word \textit{bäne} belongs formally in the paradigm of demonstratives, more specifically in the medial category (cf. \tabref{tab:doehler:demonstratives}). Demonstratives in Komnzo can be used pronominally, and in this syntactic position they can be flagged with a subset of case markers, as is shown in \tabref{tab:doehler:bäne} for the proximal demonstrative \textit{zane}.

The placeholder \textit{bäne} is much more active in this respect. As can be seen in \tabref{tab:doehler:bäne}, it can also be flagged for the three spatial cases (\gl{loc}, \gl{all}, \gl{abl}) with inanimate referents. More importantly, \textit{bäne} has a second stem, \textit{baf}, which can be flagged for all cases that encode animate referents with the standard number distinction. In the Table, we can see some spillover of the two stems, i.e. \textit{bäne} for animates and \textit{baf} for inanimates, which I take as evidence that the two stems belong to the same underlying form. For example, \textit{bäne} appears as the absolutive case form for animate referents, while all other forms are built from \textit{baf}.\footnote{The lack of a number distinction is found with all absolutive case forms, e.g.\ personal pronouns and nominal enclitics.} Conversely, the locative case form for inanimate referents is built from \textit{baf}, not from the expected \textit{bäne}.

In terms of possible case flags, \textit{bäne}/\textit{baf} is not only more active than the demonstratives, but also more active than the person pronouns, which in turn cannot be flagged for cases that encode inanimate referents. In \tabref{tab:doehler:bäne}, I show the possibilities of personal pronouns with the 2\gl{sg} in the rightmost column. Thus, one can say that the placeholder \textit{bäne}/\textit{baf} is the ``most prototypical pronoun'' in the language because it can substitute all nouns inflected for all cases. However, the agnostic term ``pro-form'' is more suitable, since \textit{bäne}/\textit{baf} can also replace longer stretches of discourse.

\begin{table}
    \caption{Different case inflections}
    \label{tab:doehler:bäne}
    \small
    \begin{tabular}{l lll l l}
        \lsptoprule
        & \multicolumn{3}{c}{\gl{placeholder}}& \gl{dem} & \gl{pers. pron}\\
        & \gl{inanim} & \gl{anim (sg)} & \gl{anim (nsg)} & \gl{prox} & \gl{2sg}\\
        \midrule
        \gl{abs}  &   \textit{bäne}&\textit{bäne}&\textit{bäne}&\textit{zane}&\textit{bä}\\
        \gl{erg}  &   -&\textit{baf} & \textit{baf-a}&-&\textit{bné}\\
        \gl{dat}  &   -&\textit{baf-an} & \textit{baf-anm}&-&\textit{bun}\\
        \gl{poss} &   -&\textit{baf-ane} & \textit{baf-anme}&-&\textit{bone}\\
        \gl{char} &   \textit{bäne=ma}&\textit{baf-ane=ma} & \textit{baf-anme=ma}&\textit{zane=ma}&\textit{bone=ma}\\
        \gl{loc}  &   \textit{baf=en}&\textit{bafa-db=en} & \textit{baf-anme-db=en}&-&\textit{bun-db=en}\\
        \gl{all}  &   \textit{bäne=fo}&\textit{bafa-db=o} & \textit{baf-anme-db=o}&-&\textit{bun-db=o}\\
        \gl{abl}  &   \textit{bäne=fa}&\textit{bafa-db=a} & \textit{baf-anme-db=a}&-&\textit{bun-db=a}\\
        \gl{ic}   &   -&\textit{baf=rr} & \textit{baf=ä}&-&\textit{bn=rr}\\
        \gl{ins}  &   \textit{bäne=me} &-&-&\textit{zane=me}&-\\
        \gl{purp} &   \textit{bäne=mr}&-&-&\textit{zane=mr}&-\\
        \gl{prop} &   \textit{bäne=karä}&-&-&\textit{zane=karä}&-\\
        \gl{priv} &   \textit{bäne=mär}&-&-&\textit{zane=mär}&-\\
        \lspbottomrule
    \end{tabular}
\end{table}

While it is clear from the paradigm in \tabref{tab:doehler:demonstratives} that \textit{bäne} has developed directly from the demonstratives, we can only speculate on the origin of the second stem \textit{baf}. One hypothesis is that \textit{bäne} first merged with the ergative case enclitic \textit{=f} [\gl{erg.sg}], and then developed the case forms shown in \tabref{tab:doehler:bäne}. Also recall that -- based on the initial /b/ consonant -- the medial category in the deictic system is formally (and historically) related to second person pronouns. Therefore, there is a historical link between the second person pronouns, the medial category in the deictic system, and the placeholder. Synchronically, I do not analyse deixis in Komnzo as a person-based system in the sense of \textcite{Keenan:1985fm}.

\subsection{Distribution}\label{sec:doehler:banedist}

The placeholder can substitute any noun or noun phrase, whether it be case marked or zero marked (i.e.\ absolutive). A typical example is given in (\ref{ex:zaruyawi}), where the speaker uses \textit{bäne} followed by a very short pause of 160ms (cf. \figref{fig:doehler:zaruyawi}), and then continues with the delayed constituent \textit{zaru yawi}, on which he further elaborates.\footnote{\textit{Zaru} is the candlenut tree (\textit{Aleutrites mollucana}) and its fruit is a stone fruit. The speaker refers to the hard kernel in the example.} \textit{Bäne} here substitutes the whole nominal compound.

\ea \label{ex:zaruyawi}
    \gll fi \textbf{bäne} ... \uline{zaru} \uline{yawi} ... \uline{zaru} \uline{yawi} mane y\stem{konz}rth ... nimä=wä we fof\\
    but \gl{ph}(\gl{abs}) (160ms) \gl{pn} nut (2200ms) \gl{pn} nut which(\gl{abs}) \gl{3pl>3sg.m:npst:ipfv}{\textbackslash}speak (1200ms) like\_this=\gl{emph} also \gl{emph}\\
    \glt `But \textbf{whatchamacallit} ... \uline{the \textit{zaru} nut} ... (that) which they call \uline{\textit{zaru} nut}. (They did) the same thing (with it).' \exsource{tci20120818 ABB 36--38}
\z

\begin{figure}
    \includegraphics[width=\textwidth]{tci20120818_ABB_36.pdf}
    \caption{Audio analysis of an excerpt of [tci20120818 \textsc{abb} 36]}
    \label{fig:doehler:zaruyawi}
\end{figure}

In example (\ref{ex:skiskin}), the placeholder occurs in the position of the head of a nominal compound (`house whatchamacallit' > `(house) platform').\footnote{Houses are built on posts with a sitting platform called \textit{skiski} underneath.} In the delayed constituent only the head noun is repeated, not the whole compound. There are no measureable pauses in the example, but there is a break in the pitch contour before the delayed constituent.

\ea \label{ex:skiskin}
    \gll \textbf{mnz} \textbf{baf=en} boba \uline{skiski=n} y\stem{rakth}kwa\\
    house \gl{ph=loc} \gl{med:abl} platform=\gl{loc} \gl{sg>3sg.m:pst:ipfv}{\textbackslash}put\_on\_top\\
    \glt `I put it on top of \textbf{the house whatchamacallit}, \uline{the (house) platform}.' \exsource{tci20111119-03 ABB 34}
\z

\begin{figure}
    \includegraphics[width=\textwidth]{tci20111119-03_ABB_34.pdf}
    \caption{Audio analysis of [tci20111119-03 \textsc{abb} 34]}
    \label{fig:doehler:skiskin}
\end{figure}

In example (\ref{ex:gastol}), the speaker searches for the name of a particular fish species. Since the pause following the placeholder is very short (70ms), I analyse the placeholder as the first element (i.e.\ the modifying element) of a nominal compound (\textit{ane} \textit{bäne} \textit{kofä} > \textit{gastol}),\footnote{\textit{Gastol} is the striped snakehead (\textit{Channa striata}), which is an invasive species to the region. Hence, the word \textit{gastol} is a loanword from Papuan Malay.} rather than as a separate noun phrase (\textit{ane} \textit{bäne} > \textit{kofä} > \textit{gastol}). After a pause of 470ms, the speaker provides the name of the fish species \textit{gastol} (cf. \figref{fig:doehler:gastol}), but the anaphoric demonstrative \textit{ane} and the noun \textit{kofä} `fish' are not repeated (not: \textit{ane gastol kofä}).

\ea \label{ex:gastol}
    \gll komnzo thu\stem{kwthe}nzrm \textbf{ane} \textbf{bäne} ... \textbf{kofä} fof ... \uline{gastol}\\
    only \gl{sg>3pl:pst:dur}{\textbackslash}turn \gl{dem} \gl{ph} (70ms) fish \gl{emph} (470ms) gastol(\gl{ma})\\
    \glt `I was just turning \textbf{those} \textbf{whatchamacallit} \textbf{fish} ... \uline{(those)} \uline{gastol} \uline{(fish)}' \exsource{tci20130903-03 MKW 125--126}
\z

\begin{figure}
    \includegraphics[width=\textwidth]{tci20130903-03_MKW_125–126.pdf}
    \caption{Audio analysis of [tci20130903-03 \textsc{mkw} 125--126]}
    \label{fig:doehler:gastol}
\end{figure}

Example (\ref{ex:trikasi}) is clearer because all parts of the complex noun phrase are repeated (\textit{ane bäne} > \textit{ane trikasi}). At the same time, the example is interesting because the pause of 820ms (cf. \figref{fig:doehler:trikasi}) comes after the anaphoric demonstrative, i.e.\ in the middle of the delayed constituent. It is cross-linguistically common to have hesitation pauses after preposed function words (\cite[935]{Himmelmann:2014zj}).

The structure of (\ref{ex:trikasi}) is suggestive of another kind of analysis, namely a combination of the placeholder \textit{bäne}/\textit{baf} and a light verb placeholder (`be' in this case). The two clauses in (\ref{ex:trikasi}) are hanging topic constructions that I translate with `as for ...' (cf. \cite[337]{Dohler:2018qt}), and in the corpus the speaker continues to explain how the story was passed down from the ancestors. A more suitable analysis might be that the entire topic construction acts as a placeholder substituting the following clause which is also a topic construction (\textit{ane bäne mane rera} > \textit{ane trikasi mane nŋatrikwé}).

\ea \label{ex:trikasi}
    \gll watik \textbf{ane} \textbf{bäne} \textbf{mane} \textbf{re\stem{r}a} \uline{ane} ... \uline{trikasi} \uline{mane} \uline{n=ŋa\stem{trik}wé} \uline{fof}\\
    then \gl{dem} \gl{ph} which \gl{3sg.f:pst:ipfv}{\textbackslash}be \gl{dem} (820ms) story which \gl{ipst}=\gl{1sg:npst:ipfv}{\textbackslash}tell \gl{emph}\\
    \glt `Well, \textbf{as} \textbf{for} \textbf{this} \textbf{whatchamacallit}, \uline{as for this} ... \uline{story that I've just told}' \exsource{tci20131013-01 ABB 401--403}
\z

\begin{figure}
    \includegraphics[width=\textwidth]{tci20131013-01_ABB_401–403.pdf}
    \caption{Audio analysis of [tci20131013-01 \gl{abb} 401--403]}
    \label{fig:doehler:trikasi}
\end{figure}

In longer stretches of discourse, there is a framing structure in which the placeholder appears in an opening part, then there is some elaboration that reveals the substituted referent, and finally there is a closing part that repeats the structure of the opening part. An example of this is given in (\ref{ex:masen}) below. The speaker talks about a man who has no children of his own, but he has adopted the children of his wife, whose former husband died many years ago. The context here is one of taboo, because the speaker is in a taboo relationship with the deceased man, which means that he should not utter his name. We will see in \sectref{sec:doehler:banefunc} that this framing strategy is often used in taboo contexts.

\ea \label{ex:masen}
    \gll \textbf{wati} \textbf{baf=ane=nzo} \textbf{nagayé} \textbf{ä\stem{moneg}wr} ... mabata-a-fis kwark zöbthé mane ya\stem{r}a, kabe fof ... nafane kabe ... masen ... \uline{wati} \uline{nafane} \uline{nagayé=nzo} \uline{ä\stem{moneg}wr}\\
    then \gl{ph=poss.sg=only} children \gl{3sg>3pl:npst:ipfv}{\textbackslash}look\_after (2000ms) \gl{pn-poss-}husband deceased first who \gl{3sg.m:pst:ipfv}{\textbackslash}be man \gl{emph} (1650ms) \gl{3sg.poss} man (600ms) \gl{pn} (1450ms) then \gl{3sg.poss} children=\gl{only} \gl{3sg>3pl:npst:ipfv}{\textbackslash}look\_after\\
    \glt `\textbf{Well, he looks after whatsisname's children} ... Mabata's late, first husband  ... her man ... Masen ... \uline{Well, he looks after only his children}.' \exsource{tci20120814 ABB 217--221}
\z

The repeated part may involve more that just the delayed constituent. In example (\ref{ex:pictask-1}), the copula clause is repeated several times by two speakers who are trying to find and negotiate the correct expression for `mixed' or for `random order'. The example comes from a stimulus task, in which two speakers are asked to arrange a set of pictures into a story. In the final part of the task, the story is presented to a third participant. In (\ref{ex:pictask-1}), speaker \gl{rma} explains to the third participant that the set of pictures came in random order. He cannot think of the right word immediately and therefore uses a placeholder in a copula clause (\textit{bäne thfrä ane}). He then corrects himself and uses an English insertion (\textit{mix thfnrä ane}). Speaker \gl{tsa} corrects \gl{rma} by using a Komnzo verb instead of the English insertion (\textit{thafraksikaf thfrä ane}), and speaker \gl{rma} repeats this.\footnote{Note that in (\ref{ex:pictask-1}), \textsc{rma}'s response to \textsc{tsa} is not a perfect repetition, because there is a different proprietive case marker. The two variants, \textit{=karä} and \textit{=kaf}, however, do not differ in their semantics (\cite[161ff.]{Dohler:2018qt}).}

\ea \label{ex:pictask-1}
    \exi{\gl{rma:}} \label{ex:pictask-1a}
        \hspace*{-.5em}\gll nzäthe zöbthé mane nzwan\stem{ri}n \textbf{bäne} \textbf{thf\stem{rä}} \textbf{ane} ... \uline{mix} \uline{thfn\stem{rä}} \uline{ane}\\
        namesake first which \gl{3sg>1du:rpst:ipfv:venit{\textbackslash}give} \gl{ph}(\gl{abs}) \gl{3pl:rpst:ipfv}{\textbackslash}be \gl{dem} (.) mix(\gl{e)} \gl{3pl:rpst:ipfv:venit}{\textbackslash}be \gl{dem}\\
        \glt `When (our) namesake first gave us the (pictures), they were whatchamacallit ... they came mixed.' \exsource{tci20111004 RMA 305--307}
    \exi{\gl{tsa:}} \label{ex:pictask-1b}
        \hspace*{-.25em}\gll \uline{thafrak-si=kaf} \uline{thf\stem{rä}}\\
        mix-\gl{nmlz}=\gl{prop} \gl{3pl:rpst:ipfv}{\textbackslash}be\\
        \glt `They were mixed.' \exsource{tci20111004 TSA 206}
    \exi{\gl{rma:}} \label{ex:pictask-1c}
        \hspace*{-.5em}\gll \uline{thafrak-si=karä} \uline{thf\stem{rä}}\\
        mix-\gl{nmlz}=\gl{prop} \gl{3pl:rpst:ipfv}{\textbackslash}be\\
        \glt `They were mixed.' \exsource{tci20111004 RMA 308}
\z

The placeholder and the delayed constituent do not need to be adjacent. A frequent pattern in the corpus is that the placeholder remains in-situ, while the delayed constituent is postposed to the clause. One example is shown in (\ref{ex:gufiyar}) below. Note that there is a pause of 760ms between the clause and the delayed constituent (cf. \figref{fig:doehler:gufiyar}).\footnote{The audio quality of the example was not good enough to produce a meaningful pitch graph.}

\ea \label{ex:gufiyar}
    \gll kofä mane \textbf{baf=en} kwa\stem{thor}thrmth ... \uline{gufiyar=en}\\
    fish(\gl{abs}) which \gl{ph}=\gl{loc} \gl{3pl:pst:dur}{\textbackslash}enter (760ms) fish.trap=\gl{loc}\\
    \glt `and the fish was going \textbf{into the whatchamacallit} ... \uline{into the fish trap}.' \exsource{tci20110802 ABB 64--65}
\z

\begin{figure}[H]
    \includegraphics[width=\textwidth]{tci20110802_ABB_64–65.pdf}
    \caption{Audio analysis of [tci20110802 \textsc{abb} 64--65]}
    \label{fig:doehler:gufiyar}
\end{figure}

\subsection{Functions}\label{sec:doehler:banefunc}

Most of the examples of \textit{bäne}/\textit{baf} that we have seen so far involved cases of memory lapse or problems of accessing a lexical entry. It is those ``tip-of-the-tongue'' disfluency situations that are characteristic of placeholder uses. Hence, it is no surprise that many corpus examples also involve repair situations, as in (\ref{ex:radioma}). The speaker first produces a false start, the target of which is \textit{mobilema} `because of the mobile phone'. Next she inserts a placeholder (\textit{bänema}), and after a brief pause of 150ms (cf. \figref{fig:doehler:radioma}) she produces the delayed constituent: \textit{radioma} `because of the radio'.

\ea \label{ex:radioma}
    \gll watik -/mo/- \textbf{bäne=ma} ... \uline{radio=ma} noku=karä=nzo kwa\stem{fark}wrmth\\
    then \gl{fs} \gl{ph}=\gl{char} (150ms) radio(\gl{e})=\gl{char} anger=\gl{prop}=\gl{only} \gl{3pl:pst:dur}{\textbackslash}set\_off\\
    \glt `So they were leaving in anger because of the mo... \textbf{because} \textbf{of} \textbf{the} \textbf{whatchamacallit} ... \uline{because of the radio}.' \exsource{tci20131004-05 RNA 39}
\z

\begin{figure}
    \includegraphics[width=\textwidth]{tci20131004-05_RNA_39.pdf}
    \caption{Audio analysis of [tci20131004-05 \textsc{rna} 39]}
    \label{fig:doehler:radioma}
\end{figure}

The above example shows a context that is typical of placeholder uses, namely the occurrence of loanwords or ad-hoc insertions. This might be due to the fact that in some cases the speakers wish to express something for which there is no single word in the Komnzo lexicon, such as `radio' (\ref{ex:radioma}), `prison cell' (\ref{ex:selen}) or `Gastol fish' (\ref{ex:gastol}). Finding an ad-hoc insertion can lead to disfluency. An additional factor can be linguistic purism, in the sense that speakers try to avoid using words from another language. This is an important part of the linguistic ideology of Komnzo speakers. An example is (\ref{ex:pictask-1}) where the speaker is corrected by his interlocutor after using an ad-hoc insertion from English `mix'.

Another context for using placeholders is to hint at certain cultural sensitivities such as avoidance, taboo, or face-saving. This has been reported in the literature for different languages such as Lao (\cite[108ff.]{Enfield:2003rf}), Mandarin (\cite{Cheung:2015zo}), Korean, and Japanese (\cite[502ff.]{Hayashi:2006aa}). For Komnzo speakers, such sensitivities may be triggered by topic (e.g.\ substance abuse or sexual relationships) or more commonly by certain kinship relations which dictate name avoidance. Especially for affinal kin, name avoidance is seen as a way of showing respect, and therefore one should not pronounce the personal names of certain individuals. It follows that name avoidance can clash with the need to identify a particular individual. Komnzo speakers solve this problem by employing teknonyms (`X's father' or `X's husband') or by using the placeholder \textit{bäne}/\textit{baf}. While this strategy works well when interacting with people who share sufficient common ground, it is more difficult in recording situations that involve an outsider: a linguist fieldworker.

Consider example (\ref{ex:mabata}). The speaker talks about his sister, whose husband had died in an accident some 30 years ago. The speaker was in an avoidance relationship with the deceased husband, because he was his brother-in-law. On first mention, he uses a placeholder (\textit{bafane mezü} `whatsisname's widow'). After a pause of 1350ms, he pronounces the name (\textit{Masenane mezü} `Masen's widow'), and adds a teknonym (\textit{albertaŋafe kwark} `Albert's late father') for further elaboration. After another pause of 1580ms, he closes with a phrase that mirrors exactly the opening phrase (\textit{nafaŋafane mezü} `his father's widow'). The closing phrase is all the more peculiar, because he could have said simply \textit{nafane ŋame} `his mother' (i.e.\ `Albert's mother').

\ea \label{ex:mabata}
    \gll mabata fi mezü zwa\stem{m}nzrm ... \textbf{baf=ane} \textbf{mezü} re\stem{r}a ... \uline{masen=ane} \uline{mezü} ... albert-a-ŋafe kwark ... \uline{nafa-ŋaf=ane} \uline{mezü}\\
    \gl{pn} \gl{3sg.abs} widow \gl{3sg.f:pst:dur}{\textbackslash}stay (1350ms) \gl{ph=poss.sg} widow \gl{3sg.f:pst:ipfv}{\textbackslash}be (370ms) \gl{pn=poss.sg} widow (1580ms) \gl{pn-poss}-father deceased (2300ms) \gl{3sg.poss-}father=\gl{poss.sg} widow\\
    \glt `Mabata stayed as a widow. She was \textbf{whatsisname's widow} ... \uline{Masen's} \uline{widow} ... Albert's late father ... \uline{his father's widow}.' \exsource{tci20120814 ABB 39}
\z

Examples (\ref{ex:mabata}) and (\ref{ex:masen}) come from different sections of the same text. In both examples the speaker talks about his brother-in-law, and both examples show a striking similarity in that there is a kind of bracket structure with an opening and a closing part. The first mention of the brother-in-law is a placeholder in both examples, presumably signalling that this a taboo context.\footnote{A similar context is found in Kalamang \citep{chapters/visser} and Besemah \citep{chapters/mcdonnell_billings}.} What follows is a careful elaboration during which the name to be avoided is in fact uttered. Finally, the bracket is closed with a phrase in (\ref{ex:mabata}) -- or with a clause in (\ref{ex:masen}) -- that exactly mirrors the opening part. In Komnzo speech, this bracket structure is frequently found when talking about sensitive topics. It follows that the placeholder is used with a communicative goal rather than filling a disfluency. Figuratively speaking, the placeholder and the bracket structure set a stage on which is it permissible to break with the taboo. 

Another use outside of disfluency is for managing turn-taking, more specifically for ``gaining the floor'', which relates to conversational dynamics rather than conveying some communicative goal. Example (\ref{ex:pictask-2}) shows a short exchange from the picture task, in which speaker \gl{rma} comments on a picture card showing a man being dragged off by two police officers. The speaker \gl{tsa} adds his own thoughts on the state of affairs, which \gl{rma} agrees with.

\ea \label{ex:pictask-2}
    \exi{\gl{rma:}} \label{ex:pictask-2a}
        \hspace*{-.5em}\gll aiwa ... frisman=é kabe y\stem{thärku}nth\\
        oh\_no (500ms) policeman(\gl{e})=\gl{erg.pl} man(\gl{abs}) \gl{3du>3sg.m:npst:ipfv}{\textbackslash}drag\\
        \glt `Oh no, the two policemen are dragging away the man.' \exsource{tci20111004 RMA 109--111}
    \exi{\gl{tsa:}} \label{ex:pictask-2b}
        \hspace*{-.25em}\gll \textbf{bäne=ma} y\stem{thärku}nth \uline{ŋare} \uline{mane} \uline{nz=ü\stem{fn}zro}\\
        \gl{ph=char} \gl{3du>3sg.m:npst:ipfv}{\textbackslash}drag woman(\gl{abs}) which \gl{ipst}=\gl{sg>3sg.f:npst:ipfv:andat}{\textbackslash}hit\\
        \glt `\textbf{That's why} they are dragging him away: \uline{It was the woman who he hit} \uline{just before.}' \exsource{tci20111004 TSA 89}
    \exi{\gl{rma:}} \label{ex:pictask-2c}
        \hspace*{-.5em}\gll mh\\
        \gl{interjection}\\
        \glt `mh (okay).' \exsource{tci20111004 RMA 112}
\z

\gl{tsa}'s sucessful interruption is achieved by starting his turn with the placeholder \textit{bänema} `that's why' that has the entire following clause as its delayed constituent, as is indicated by the underlined font in (\ref{ex:pictask-2}). The initial placeholder creates a moment of anticipation that further elaboration is to come, thus, enabling \gl{tsa} to take over the floor. There are no pauses in \gl{tsa}'s turn, but there is a break in the intonation contour with a falling pitch on the last word of the first clause (\textit{ythärkunth}) and a rising pitch on the first word of the second clause (\textit{ŋare}), which separate the two clauses prosodically (cf. \figref{fig:doehler:pictask-2b}).

\begin{figure}
    \includegraphics[width=\textwidth]{tci20111004_TSA_89.pdf}
    \caption{Audio analysis of [tci20111004 \textsc{tsa} 89]}
    \label{fig:doehler:pictask-2b}
\end{figure}

\subsection{Functional extensions}\label{sec:doehler:banefuncext}

In most of the preceding examples, the placeholder mirrored the delayed constituent in its syntactic position and case marking. In this section, I argue that some of the inflected forms of \textit{bäne} have widened their functional scope to include a non-placeholder function, namely they are used as clausal connectors for adverbial clauses.

Consider example (\ref{ex:rfik}), in which the speaker explains how he stores different species of yam in his storage house. The placeholder \textit{bäne} is inflected with the purposive case (\textit{bänemr}), which has a temporal meaning in this example (`until'). The placeholder functions as a connector of two otherwise independent clauses. There is no sign of disfluency in the example. The falling pitch on \textit{bänemr} and subsequent reset of the pitch level separate the two clauses (cf. \figref{fig:doehler:rfik}). Thus, prosodically \textit{bänemr} belongs to the first clause.

\ea \label{ex:rfik}
    \gll ane fof e\stem{mig}wre \textbf{bäne=mr} fobo kwa thra\stem{rfik}wr\\
    \gl{dem} \gl{emph} \gl{1pl>3pl:npst:ipfv}{\textbackslash}hang \gl{ph=purp} \gl{dist:all} \gl{fut} \gl{3pl:irr:ipfv}{\textbackslash}grow\\
    \glt `We hang them up \textbf{until} (the shoots) will grow from there.' \exsource{tci20121001 ABB 24}
\z

\begin{figure}
    \includegraphics[width=\textwidth]{tci20121001_ABB_24.pdf}
    \caption{Audio analysis of [tci20121001 \textsc{abb} 24]}
    \label{fig:doehler:rfik}
\end{figure}

Tokens of \textit{bäne} such as (\ref{ex:rfik}) cannot be analysed as placeholders, nor can the following clause be analysed as a delayed constituent. Instead, they are part of a grammatical construction that connects two independent clauses, and it is the case marker on \textit{bäne} that signals the semantic relation that holds between them. There are adverbials of reason (\textit{bäne=ma} [\gl{ph=char}] `because'), manner (\textit{bäne=me} [\gl{ph=ins}] `thereby'), and purpose/time (\textit{bäne=mr} [\gl{ph=purp}] `in order to'/`until'). Note that Komnzo has additional strategies for adverbials including other types of connectors (e.g. \textit{fthé} `when', \textit{monme} `how') or nominalised verbs for non-clausal adverbials (\cite[321ff.]{Dohler:2018qt}).

(\ref{ex:gatham}) is a second example of \textit{bäne} as a clausal connector. This time it is inflected with the characteristic case, which I translate as `because'. Like in (\ref{ex:rfik}), \textit{bänema} has a falling pitch and the following clause resets the pitch level (cf. \figref{fig:doehler:gatham}).

Note that there is a second token of \textit{bäne} in the first clause of (\ref{ex:gatham}), inflected with the instrumental case (\textit{bäneme}). This one ticks all the boxes for a placeholder: there is a short pause (after the verb) signalling a disfluency, the placeholder mirrors the delayed constituent in terms of case marking and syntactic position. Note that there is no falling pitch on \textit{bäneme}.

\ea \label{ex:gatham}
    \gll zöbthé \textbf{bäne=me} kwa w\stem{rthaku}nzr ... \uline{zzarfa=me} \textbf{bäne=ma} gatha miyosé \stem{rä}\\
    first \gl{ph=ins} \gl{fut} \gl{3sg>3sg.f:npst:ipfv}{\textbackslash}sprinkle (200ms) ginger=\gl{ins} \gl{ph=char} bad taste \gl{3sg.f:npst:ipfv}{\textbackslash}be\\
    \glt `First, he will sprinkle it with \textbf{whatchamacallit} ... \uline{with ginger}, \textbf{because} it has a bad taste.' \exsource{tci20130903-04 RNA 63--64}
\z

\begin{figure}
    \includegraphics[width=\textwidth]{tci20130903-04_RNA_63–64.pdf}
    \caption{Audio analysis of [tci20130903-04 \gl{rna} 63--64]}
    \label{fig:doehler:gatham}
\end{figure}

Not all tokens of \textit{bäne} in three inflections (\gl{char}, \gl{purp}, \gl{ins}) are connectors, as we just saw with \textit{bäneme} in (\ref{ex:gatham}) or with \textit{bänema} in (\ref{ex:radioma}) and (\ref{ex:pictask-2}). Next to the meaning, it depends on syntactic and prosodic cues whether a specific example is best analysed as a placeholder or as a connector. Thus, it would be wrong to say that these inflections have grammaticalized to become connectors, but rather that they have widened their functional scope or their syntactic possibilities.

A reviewer of this chapter suggested that the development of the adverbial connector need not involve the additional step via a placeholder, but come directly from the medial demonstrative. Based on synchronic data, I cannot rule out this possibility. Moreover, \textcite[230]{Himmelmann:2014zj} noted already that demonstratives in the ``recognitional use'' are often found as connectors of relative clauses, and there is an obvious link between recognitional deixis and placeholders (cf. \cite{Enfield:2003rf}).

Nevertheless, I want to sketch out a scenario via the placeholder that seems to me more parsimonous. For this, let us think about the problem as bridging two ends of a spectrum: On the one hand, we have a clear placeholder use, which involves disfluency and some kind of mirroring of the delayed constituent. Most of the examples in this chapter fit this description. On the other end of the spectrum, we have examples like (\ref{ex:rfik}) and (\ref{ex:gatham}), where two clauses are connected, and where there is no disfluency.

\largerpage[1.5]
One kind of bridging construction are cases in which the placeholder has the entire following clause as its delayed constituent, as in (\ref{ex:pictask-2b}). A clearer example of this pattern is shown in example (\ref{ex:kukufia}), in which the speaker introduces the protagonist of a story. He uses a placeholder in the first clause, and after a pause of 1100ms he informs us about the protagonist (cf. \figref{fig:doehler:kukufia}).\footnote{Note that the characteristic case (\textit{=ma}) covers both meanings of reason and aboutness (\cite[157]{Dohler:2018qt}).} The function of the placeholder in (\ref{ex:pictask-2b}) and (\ref{ex:kukufia}) is to create a certain anticipation that the speaker has more to say.

\ea \label{ex:kukufia}
    \gll trikasi \textbf{bäne=ma} kwa na\stem{trik}wé ... \uline{kabe} \uline{tnz} \uline{yf} \uline{sf\stem{rä}rm} \uline{kukufia}\\
    story \gl{ph=char} \gl{fut} \gl{1sg>2sg:npst:ipfv}{\textbackslash}tell (1100ms) man short name \gl{3sg.m:pst:dur}{\textbackslash}be \gl{pn}\\
    \glt `I will tell you a story about \textbf{that one}: \uline{The short man's name was} \uline{Kukufia}.' \exsource{tci20100905 ABB 6--7}
\z

\begin{figure}
    \includegraphics[width=\textwidth]{tci20100905_ABB_6–7.pdf}
    \caption{Audio analysis of [tci20100905 \gl{abb} 6--7]}
    \label{fig:doehler:kukufia}
\end{figure}

\largerpage
All it takes for the second pillar of the bridge, is for the placeholder to occur after the verb, i.e. in final position. This happens frequently in a kind of afterthought expression which is introduced by \textit{bäne}. In (\ref{ex:gb}), the speaker explains the layout of yam tubers in his storage house. The placeholder (\textit{bäne=mr}), flagged with the purposive case, can be translated as `in order to' or `so that', and its delayed constituent is the following clause, the afterthought. The pauses preceding (850ms) and following (300ms) the placeholder signal a disfluency (cf. \figref{fig:doehler:gb}). Thus, the example has features from both ends of the spectrum: disfluency and clausal connector use.

\ea \label{ex:gb}
    \gll keke ŋa\stem{fsi}nzre komnzo e\stem{nak}wre ... \textbf{bäne=mr} ... \uline{gb} \uline{thra\stem{rfik}wr} \uline{zba}\\
    \gl{neg} \gl{1pl:npst:ipfv}{\textbackslash}count just \gl{1pl>3pl:npst:ipfv}{\textbackslash}put\_down (850ms) \gl{ph=purp} (300ms) shoot \gl{3pl:irr:ipfv}{\textbackslash}grow \gl{prox.abl}\\
    \glt `We don't count (them). We just put them down ... \textbf{so that} ... \uline{the shoots} \uline{grow from here}.' \exsource{tci20120805-01 ABB 33--35}
\z

\begin{figure}
    \includegraphics[width=\textwidth]{tci20120805-01_ABB_33–35.pdf}
    \caption{Audio analysis of [tci20120805-01 \textsc{abb} 33--35]}
    \label{fig:doehler:gb}
\end{figure}

I suggest here that afterthoughts like (\ref{ex:gb}) provide a bridging context from which the function of \textit{bäne} can be extended to include non-placeholder uses, i.e. to be used as different kinds of adverbial connectors. What drives this functional extension is the ``hiatus moment'' that is so typical of placeholders, i.e. the anticipation for further elaboration.

\subsection{Multimodality}\label{sec:doehler:gestures}

Multimodal aspects of placeholders have been largely neglected in the literature, but see the chapter on Northern Pastaza Kichwa \citep{chapters/rice}. An exception is \citeauthor{Navarretta:2016yf}'s (\citeyear{Navarretta:2016yf}) study on Danish hesitative fillers and simultaneously occurring gestures. In the examples from Komnzo, the placeholder \textit{bäne/baf} is often accompanied by a hand gesture, often a pointing gesture. This does not come as a surprise, as the placeholder has developed from a demonstrative.

Note that there has not been a detailed analysis or description of gestures in Komnzo, nor is the corpus currently annotated for gestures. For these reasons, I can only give a rough estimate of the frequency of gesture co-occurrence. One can find gestures in about two thirds of the placeholder tokens. I include here three examples from the Komnzo data.

The first example comes from a conversation in the garden. Speaker \gl{stk} refers to a road junction in the forest, but has problems finding the correct place name in his description. He uses a placeholder inflected with the allative case in (\ref{ex:fothrzfth}). The accompanying gesture is a pointing gesture consisting of a short jerk of the left hand in the corresponding direction. This is highlighted in the still image with the red circle (cf. \figref{fig:doehler:fothrzfth-still}). The audio analysis in Figure (\ref{fig:doehler:fothrzfth}) shows the time and length of the gesture with the grey overlay on the Praat picture. The gestural component is almost perfectly aligned with the placeholder, but not with the target word \textit{Fothr Zfthfo}, which is produced after a short pause. Hence, the gesture functions like a support during the retrieval of the correct place name. In fact, the gesture alone establishes the correct spatial relationships by pointing in the corresponding direction.

\ea \label{ex:fothrzfth}
    \gll fä mane \stem{rä} \textbf{bäne=fo} ... \uline{fothr\_zfth=fo}\\
    \gl{dist} which \gl{3sg}.\gl{f}:\gl{npst}:\gl{ipfv}{\textbackslash}be \gl{ph=all} (400ms) \gl{pln=all}\\
    \glt `where (the road) turns \textbf{to} \textbf{whatchamacallit} ... \uline{to Fothr Zfth.}' \exsource{tci20130823–06 STK 143--144}
\z

\begin{figure}
    \includegraphics[width=\textwidth]{tci20130823-06_STK_143–144.pdf}
    \caption{Audio analysis of [tci20130823–06 \gl{stk} 143--144]}
    \label{fig:doehler:fothrzfth}
\end{figure}

\begin{figure}
    \includegraphics[width=.95\textwidth]{tci20130823-06_STK_143–144.png}
    \caption{Still image of [tci20130823–06 \gl{stk} 143--144]}
    \label{fig:doehler:fothrzfth-still}
\end{figure}

The second example comes from a conversational narrative in which speaker \gl{mab} talks about an event that happened a long time ago. His interlocutor \gl{cam} asks whether he was married at the time, to which \gl{mab} replies ``No, I was just a boy''. To further emphasise his age, he adds that his beard had only just started to grow at the time. In (\ref{ex:fäkthäbu}), he uses a placeholder for the target word \textit{fäk thäbu} `beard'. The accompanying gesture is that he strokes his left cheek with the fingers of his right hand (cf. \figref{fig:doehler:fäkthäbu-still}). The gestural component overlaps with more than just the placeholder. The audio analysis in \figref{fig:doehler:fäkthäbu} shows that the gesture starts with the proximal demonstrative \textit{zane}, lasts through the false start and the placeholder, and stops in the pause. The speaker does not produce the target word (\textit{fäk thäbu} `beard'), because the gesture alone is sufficient to identify the referent.

\ea \label{ex:fäkthäbu}
    \gll komnzo kwa zane -/nzä/- \textbf{bäne} ... thf\stem{rfik}wrm\\
    just \gl{fut} \gl{prox} \gl{fs} \gl{ph} (480ms) \gl{3pl}:\gl{pst}:\gl{dur}{\textbackslash}grow\\
    \glt `\textbf{These} \textbf{watchamacallit} were just about to start growing.' \exsource{tci20130927-06 MAB 187}
\z

\begin{figure}
    \includegraphics[width=\textwidth]{tci20130927-06_MAB_187.pdf}
    \caption{Audio analysis of [tci20130927-06 \gl{mab} 187]}
    \label{fig:doehler:fäkthäbu}
\end{figure}

\begin{figure}
    \includegraphics[width=\textwidth]{tci20130927-06_MAB_187.png}
    \caption{Still image of [tci20130927-06 \gl{mab} 187]}
    \label{fig:doehler:fäkthäbu-still}
\end{figure}

In the third example, the gesture is more figurative. The example comes from the picture task, and it belongs to example (\ref{ex:pictask-1}) above: Speaker \gl{rma} explains to his interlocutor, who is off the videoframe to the right, that the picture cards came in random order, i.e. they were mixed up. He has problems in lexical retrieval for this concept. First, he uses the placeholder, then an insertion from English (\textit{mix}), and — after a prompt from the second participant \gl{tsa} — a Komnzo word (\textit{thafraksikarä}). The gesture consist of both hands spinning around each other on a horizontal left-to-right axis (cf. \figref{fig:doehler:pictask-1-still}), and he produces it two times. First, it occurs with the placeholder. Then he repeats the gesture with the English insertion, but not with the final expression in Komnzo. Here too, the gesture functions as a support channel during lexical retrieval.

\begin{figure}
    \includegraphics[width=\textwidth]{tci20111004_RMA_414–416.png}
    \caption{Still image of [tci20111004 \gl{rma} 305--307]}
    \label{fig:doehler:pictask-1-still}
\end{figure}

The three examples presented here are anecdotal, but we can still derive some generalisations from them. First, gestures provide a parallel support channel during disfluencies. This is shown by the fact that the gesture only accompanies the placeholder and not the target word. Secondly, there is a broad range of gestures for such situations. The three examples presented here involve different manual gestures, but non-manual gestures such as lip- and head-pointing are also common. Thirdly, gestures vary in their function: They can be used to point to a target that is not present, as in the first example. They can be used to identify the target directly, as in the second example. Finally, they can be used to re-enact the target, as in the third example.

\subsection{Frequency of \textit{bäne}/\textit{baf}}\label{sec:doehler:freq}

There has not been much mention in the literature on the frequency of placeholders. \textcite{Podlesskaya:2010aa} reports 5 placeholders per 1000 words in a corpus of informal elicited narratives in Russian, and \textcite{Zhao:2010sf} report 6.68 placeholders per 1000 words in a conversational corpus of Mandarin. However preliminary these figures are, on first inspection the placeholder \textit{bäne}/\textit{baf} is much more frequent in the Komnzo data. There are 723 tokens in a corpus of 53,678 words, which amounts to 13.47 placeholders per 1000 words. This high figure has to be treated cautiously.

The figure of 13.47 per 1000 words is certainly too high, because some the 723 tokens are not placeholders, but connectors (cf. \sectref{sec:doehler:banefuncext}). For example, inflections such as \textit{bäne=ma} [\gl{ph}=\gl{char}] can be used as a placeholder, as in example (\ref{ex:radioma}), and as a connector, as in (\ref{ex:gatham}). The corpus has not been annotated in such a way that these can be searched for separately. To get an idea of the error rate, I have therefore excluded all of the inflections of \textit{bäne} that can be used as connectors.\footnote{These are: \textit{bäne=ma} [\gl{ph}=\gl{char}] `because', \textit{bäne=mr} [\gl{ph}=\gl{purp}] `in order to', \textit{bäne=me} [\gl{ph}=\gl{ins}] `thereby'). They add up to 202 tokens.} This recount still results in a rather high figure of 9.71 placeholders per 1000 words. It follows that the true figure must lie somewhere between 9.71 and 13.47 per 1000 words.

A reviewer of this chapter suggested that the high frequency may be an artefact introduced by the recording situation. The presence of an outsider linguist may exert pressure on the speakers to find the correct word and to avoid English insertions, which are permissible in natural conversation. I acknowledge that these pressures are real for Komnzo speakers as for anyone else. However, a brief examination of the data shows that this aspect has little influence in my data. To investigate this claim, I selected four more natural recordings. These are either purely observational recordings (even without my presence), or recordings in which I was only a spectator of an ongoing conversation. If we compare the frequency of \textit{bäne}/\textit{baf} in this subset with the overall frequency in the text corpus, we see that the frequency is only slightly lower: 13.27 placeholders per 1000 words.\footnote{The four recordings were: tci20130823-06 (a conversation in the garden between two speakers), tci20130927v-06 (a conversational narrative between two speakers), tci20130901-04 (a conversational narrative between three speakers), and tci20131004-05 (a conversation between six speakers). The combined figures for these four recordings were 70 placeholders out of 5274 words.}

A third explanation for the high frequency might be that \textit{bäne}/\textit{baf} is used in disfluency situations in which other languages employ hesitative fillers. While this is confirmed by my general impression of Komnzo speech, there is no way to measure relative frequency, as hesitative fillers are not consistently coded in the corpus.    

\section{Conclusion}\label{sec:doehler:concl}

I hope that the chapter helps to push forward the emerging typology of fillers. I close the chapter by summarizing a few interesting observations in Komnzo.

As I pointed out in \sectref{sec:doehler:freq}, the figure of 13.47 placeholders per 1000 words is somewhat inflated. With a better coding of the data, this figure will come down, but not by much. The question of why Komnzo has a higher frequency than other languages will remain. Note that this observation extends to other languages in the Southern New Guinea region. For Bine, an unrelated Oriomo language spoken 200km to the East of Komnzo, I can report the staggering figure of 18.71 placeholders per 1000 words.\footnote{629 tokens of the placeholder \textit{nake} in a corpus of ca. 33,000 words. The coding problems are not an issue for Bine because (i) the placeholder \textit{nake} is unrelated to demonstratives, and (ii) it is not used as clausal connector. That being said, the description of Bine is still in its infancy (cf. \cite{Dohler:0tp}).} For Evenki, an endangered Tungusic language spoken in Russia, China and Mongolia, \textcite{Klyachko:2022ol} reports a token frequency of 12.6 placeholders per 1000 words\footnote{There are 350 tokens of the placeholder \textit{aŋi} in a corpus of about 27,700 words (\cite[213]{Klyachko:2022ol}).}, and she explains this with a ``lack of proficiency in some speakers'' (\citeyear[213]{Klyachko:2022ol}). This is not the case for Komnzo and Bine speakers. Although they are small languages -- Komnzo is spoken by about 250 people, and Bine is estimated to have 2000 speakers -- neither language is endangered, and the recorded speakers are fully competent. My general answer to the puzzle of the high frequency is that placeholders rather than hesitative fillers are the preferred strategy in disfluency situations.

The second point I want to raise here is the multi-functionality of \textit{bäne}/\textit{baf} in Komnzo. We have seen that its use goes well beyond that of a filler. It includes intentional uses with communicative goals such as signalling a taboo context. Moreover, it is used for interaction management in conversations. Lastly, in clause final position it has grammaticalized via an afterthought construction into a connector for adverbial clauses.

A third topic are co-occuring gestures, which open up a fascinating field of research for the study of fillers. As we have seen from the three examples in \sectref{sec:doehler:gestures}, Komnzo speakers use gestures as a visual support channel when they have problems retrieving the correct word from their mental lexicon.

\section*{Abbreviations}

Abbreviations in the gloss line follow the Leipzig Glossing Rules. Additional conventions are given below:\medskip\\
\noindent\begin{tabularx}{.5\textwidth}[t]{@{}lQ@{}}
... & pause\\
(\#\#\#ms) & measured pause in milliseconds\\
\gl{andat} & andative (`away')\\
\gl{anim} & animate\\
\gl{char} & characteristic\\
(\gl{e}) & loanword from English\\
\gl{emph} & emphatic\\
\gl{eps} & epistemic primacy\\
\gl{fs} & false start\\
\gl{hes} & hesitator\\
\gl{ic} & inclusory case\\
\gl{inanim} & inanimate\\
\end{tabularx}%
\begin{tabularx}{.5\textwidth}[t]{@{}lQ@{}}
\gl{ipst} & immediate past\\
(\gl{ma}) & loanword from Malay\\
\gl{med} & medial\\
\gl{npst} & non-past\\
\gl{nsg} & non-singular\\
\gl{only} & exclusive (`only X')\\
\gl{ph} & placeholder\\
\gl{pln} & place name\\
\gl{pn} & proper noun\\
\gl{prop} & proprietive (`having')\\
\gl{priv} & privative (`lacking')\\
\gl{rpst} & recent past\\
\gl{venit} & venitive (`hither')\\
\end{tabularx}

\section*{Acknowledgements}
I thank the speakers of Komnzo for welcoming me to their village and taking on the task of teaching their language to me. The fieldwork was funded by the \gl{dobes} program of the Volkswagen Foundation and the Australian National University. I thank both of these institutions for their support of language documentation and description. Over the years, I have received much useful feedback during presentations that have touched on the topic of fillers in one way or another. Moreover, I owe thanks to the two anonymous reviewers of this chapter. Lastly, I thank Françoise Rose and Brigitte Pakendorf for their patience with the manuscript and for their useful feedback.

{\sloppy\printbibliography[heading=subbibliography,notkeyword=this]}
\end{document}
