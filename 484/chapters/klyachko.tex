\documentclass[output=paper,colorlinks,citecolor=brown
\ChapterDOI{10.5281/zenodo.15697577}
% ,hidelinks
% showindex
]{langscibook}
\author{Elena Klyachko\orcid{}\affiliation{Higher School of
Economics, Moscow}}
\title{Placeholders versus general extenders in Tungusic languages}
\abstract{The paper describes two classes of discourse markers, namely placeholders and general extenders, in Tungusic languages (a family of endangered languages spoken in Russia and China). The aim of the paper is to provide an exhaustive list of stems, covering their etymology, mirroring features (i. e. copying morphology) and target stem restrictions. Although placeholders and general extenders both belong to the vague language category, they demonstrate different mirroring behaviour, which I will demonstrate. I argue that the use of the interrogative pronoun stem as a placeholder may have developed from the general extender use.

\keywords{vague language, Northeast Asia, whatchamacallit, and such}
}

%move the following commands to the "local..." files of the master project when integrating this chapter
\usepackage{tabularx,tabularray,graphicx}
\usepackage{langsci-optional}
\usepackage{langsci-gb4e}
% \bibliography{localbibliography}
\graphicspath{ {figures/klyachko} }

% \newcommand{\orcid}[1]{}

\IfFileExists{../localcommands.tex}{
 \addbibresource{../localbibliography.bib}
 \usepackage{langsci-optional}
\usepackage{langsci-gb4e}
\usepackage{langsci-lgr}

\usepackage{listings}
\lstset{basicstyle=\ttfamily,tabsize=2,breaklines=true}

%added by author
% \usepackage{tipa}
\usepackage{multirow}
\graphicspath{{figures/}}
\usepackage{langsci-branding}

 
\newcommand{\sent}{\enumsentence}
\newcommand{\sents}{\eenumsentence}
\let\citeasnoun\citet

\renewcommand{\lsCoverTitleFont}[1]{\sffamily\addfontfeatures{Scale=MatchUppercase}\fontsize{44pt}{16mm}\selectfont #1}
  
 %% hyphenation points for line breaks
%% Normally, automatic hyphenation in LaTeX is very good
%% If a word is mis-hyphenated, add it to this file
%%
%% add information to TeX file before \begin{document} with:
%% %% hyphenation points for line breaks
%% Normally, automatic hyphenation in LaTeX is very good
%% If a word is mis-hyphenated, add it to this file
%%
%% add information to TeX file before \begin{document} with:
%% %% hyphenation points for line breaks
%% Normally, automatic hyphenation in LaTeX is very good
%% If a word is mis-hyphenated, add it to this file
%%
%% add information to TeX file before \begin{document} with:
%% \include{localhyphenation}
\hyphenation{
affri-ca-te
affri-ca-tes
an-no-tated
com-ple-ments
com-po-si-tio-na-li-ty
non-com-po-si-tio-na-li-ty
Gon-zá-lez
out-side
Ri-chárd
se-man-tics
STREU-SLE
Tie-de-mann
}
\hyphenation{
affri-ca-te
affri-ca-tes
an-no-tated
com-ple-ments
com-po-si-tio-na-li-ty
non-com-po-si-tio-na-li-ty
Gon-zá-lez
out-side
Ri-chárd
se-man-tics
STREU-SLE
Tie-de-mann
}
\hyphenation{
affri-ca-te
affri-ca-tes
an-no-tated
com-ple-ments
com-po-si-tio-na-li-ty
non-com-po-si-tio-na-li-ty
Gon-zá-lez
out-side
Ri-chárd
se-man-tics
STREU-SLE
Tie-de-mann
}
 \boolfalse{bookcompile}
 \togglepaper[23]%%chapternumber
}{}

\begin{document}
\maketitle
\label{chap:klyachko}
\section{Introduction}
{Placeholders} (i. e. lexical forms that replace a {target} item) and {general extenders} (items used at the end of lists like ``and so on" or ``and so forth", \citealt{overstreet2021general}) are two categories of discourse markers, which have much in common. Both can be called \textit{vague language} (see \citealt{tarnyikova2019english} for the term regarding placeholders) utilized to avoid naming a specific word. The reasons for the avoidance are manifold. The speaker may either not be able to recall a specific word, or they do not want to pronounce it, or they would like to refer to some open-ended list without specifying the exact word.

The interaction between placeholders and general extenders has already drawn the attention of linguists. \citet{ganenkov2010interrogatives} describe placeholders and general extenders in Udi and Agul (both belonging to the Northeast Caucasian family). Their hypothesis is that in Udi and Agul, general extenders developed from placeholders, with the process being independent in both languages. The authors explain the similarity of the development by the common typological profile of Udi and Agul, which are both highly agglutinating, left-branching, and make great use of light verb constructions. It is therefore interesting if other languages with similar typological features exhibit similar patterns for placeholders and general extenders. The languages belonging to the Tungusic family are possible candidates for the research, being highly agglutinating and left-branching.

The following two Evenki (< Tungusic) examples demonstrate the parallels between placeholders and general extenders\footnote{In this paper, I use \textbf{bold} for the placeholder or the general extender in examples. In case of placeholders, I use \uline{underline} for the target word. In case of general extenders, I use \uline{underline} for the first members of the enumeration (which the general extender often mirrors).}:
\is{placeholder} 
\ea

\label{ex:klyachko:1.1}
\langinfo{Evenki}{}{\cite{EvenkiCorpusSiberianLang}}\\
\gll ə-wki:		minə		\textbf{e:-ra}		\uline{əɣi-fko:n-ə}\\
 \textsc{neg}-\textsc{ptcp.hab} \textsc{1sc.acc} what-\textsc{ptcp.neg} play-\textsc{caus}-\textsc{ptcp.neg}\\
\glt `She did not \textbf{ do whatchamacallit}, \uline{let me play}.'
\z
\il{Evenki}

\is{general extender} 
\ea
\label{ex:klyachko:1.2}
\langinfo{Evenki}{}{\cite{EvenkiCorpusSiberianLang}}\\
\gll toɣo-wo	o:kin=da:	e:-wko:	əru-t		\uline{jaŋu-w-rə} ə-wko:			\textbf{e:-w-ra=da:} \\
 fire-\textsc{acc} when=\textsc{foc} \textsc{neg}-\textsc{ptcp.hab} bad-\textsc{advz} talk-\textsc{pass}-\textsc{ptcp.neg} \textsc{neg}-\textsc{ptcp.hab} what-\textsc{pass}-\textsc{ptcp.neg}=\textsc{foc}\\
\glt `It should never \uline{be talked to fire badly} \textbf{or done similarly}.'
\z
\il{Evenki}

In both \REF{ex:klyachko:1.1} and \REF{ex:klyachko:1.2}, an \textit{e:(kun)-} (`what') stem is used. In the first case, it substitutes the lexeme the speaker could not produce at the moment of speaking whereas in the second case, it refers to an open-ended list of actions which are all not allowed when communicating with fire. In both cases, we can observe \textit{mirroring}. \citet{Podlesskaya2010} uses this term with regard to copying inflectional and/or derivational affixes from the \textbf{target}, i.~e. the constituent which is delayed or not pronounced at all. I think that the term can also be used for general extenders, when the general extender, used in an enumeration, copies the properties of the other members. However, the exact mirroring properties are different in examples \REF{ex:klyachko:1.1} and \REF{ex:klyachko:1.2}.
The wordform with the \textit{e:-} `what' stem is used in the negative construction in both examples, so it has the negative participle marker, which is obligatory in this construction. At the same time, the voice marker is only found in the general extender example \REF{ex:klyachko:1.2} but not in the placeholder one \REF{ex:klyachko:1.1}.\footnote{Anonymous reviewers also remark the repetition of the negative auxiliary in the general extend\-er case. Unfortunately, there are too few examples of such repetitions to draw any conclusions of whether the general extender use favours such repetitions or not.}

Mirroring does not always have to be full, with \citet{Podlesskaya2010} distinguishing between \textit{partial} and \textit{full mirroring}. Therefore, the question is whether the difference in the placeholder and general extender mirroring properties is a coincidence or a reflection of the underlying difference between placeholders and general extenders.

The aim of this paper is to describe the etymology, the grammatical features and functions of placeholders and general extenders in Tungusic languages as well as compare them. As a result, I would like to demonstrate the connections between these two categories, namely the derivation of one of the categories from the other one.

\subsection{Tungusic languages}
The Tungusic family is a family of endangered languages spoken nowadays over vast areas of Russia and China \citep{Hölzl2022}. The details of the internal classification of Tungusic languages are still a matter of debate (see \citealt{Oskolskayaetal2022} for a discussion). The sources used for this paper are summarized in \tabref{tab:klyachko:data}. These are mostly oral corpora, grammars and sometimes papers, not specifically on placeholders but containing language examples. Unfortunately, the original sentence translations sometimes lack words like ``whatchamacallit''. In such cases, I will use angle brackets to insert the translations for the placeholders. This paper considers nine Tungusic idioms. I do not take into account the Solon language as well as the dialects of Evenki and Nanai spoken in China due to the scarcity of data. Written idioms like Jurchen are not considered because little is known about spontaneous discourse.

\begin{table}
\begin{tabularx}{\textwidth}{ll>{\raggedright}p{50mm}Q}
 \lsptoprule
 Language & ISO & Corpora & Grammatical descriptions\\
 \midrule
 Evenki & evn & \cite{EvenkiCorpusSiberianLang} (mostly Northern and Southern dialect groups);
		\cite{EvenkiCorpusIEA} (Northern, Southern and Eastern dialect groups);
		\cite{EvenkiCorpusINEL} (Northern and Southern dialect groups); elicitation & \citet{klyachko2022}\\
\tablevspace
 Negidal & neg & \cite{NegidalCorpus} & \citet{khasanova2003}\\
\tablevspace
 Even & eve & \citet{nikolaevagarret} & \citet{matic2008} \\
\tablevspace
 Udihe & ude & \citet{nikolaevagarret} & \citet{Shneider1936}\newline
					\citet{Shneider1937}\newline
					\citet{nikolaeva2001grammar}\newline
					\citet{tolskaya2008question}\\
\tablevspace
 Oroch & oac & \citet{kazama1996} & \\
\tablevspace
 Nanai & gld & \citet{beldy2012} & \citet{kazama2007rhetorical}\newline
				    \citet{kazama1993}\\
\tablevspace
Kur-Urmi  Nanai&  & \cite{nanaicorpus} & \\
\tablevspace
 Ulch & ulc & \cite{ulchcorpus} & \citet{petrova1936}\\
\tablevspace
 Uilta & oaa & \citet{ikegami2010}, \citet{yamada2011uiruta}, \citet{yamada2015}, \citet{yamada2017}, \cite{uiltacorpus}, \cite{minlang_uilta} &\\
\tablevspace
 Sibe & sjo & \citet{kogura2021sibe}; \citet{kogura2022} & \citet{zikmundova2013spoken}
 \\
 \lspbottomrule
\end{tabularx}
 \caption{Tungusic languages and the sources used\label{tab:klyachko:data}}
\end{table}


\subsection{Challenges in studying placeholders and general extenders}
Studying placeholders and general extenders may be challenging due to several reasons. Firstly, placeholders, differently from general extenders, tend to be omitted in published texts and are often not mentioned in grammatical descriptions, having low prestige among both native speakers and linguists. It is also hard to perform elicitation experiments because in many cultures, including for instance Russian or Evenki cultures, placeholders seem to be a characteristic feature of ``poor" speech. Speakers tend to avoid pronouncing placeholders when they control the discourse situation, and drawing attention to placeholder may make them upset. At the same time, the whole Tungusic language family is endangered, with the majority of Tungusic speakers being at least bilingual. This influences greatly the way hesitation markers are used. For example, it has been noticed that code-switching in bilingual speech is often accompanied by various hesitation and monitoring markers (see \citealt{hlavac2011hesitation}, where Croatian-Australian English bilingualism is studied). We can find such examples in the oral corpora, too. A brief discussion of such examples will be provided in~\ref{Borrowed_placeholders}.

Mirroring phenomena themselves are also challenging. Firstly, the target is not always pronounced during the recording or clarified after the recording. For instance, in the Evenki language corpus (\cite{EvenkiCorpusSiberianLang}) the target was actually realized only in $60\%$ of all cases for nouns and in $56\%$ of all cases for verbs. Secondly, the investigation of mirroring is limited to the morphological features of the targets which occur in the corpora.

That said, I will still use mostly oral corpora for studying placeholders and general extenders. As I have already mentioned above, placeholders are often ``discriminated'' in literary texts, so using oral corpora is the best way to study them.


\section{Tungusic placeholders}
\subsection{Tungusic placeholder stems}
In Tungusic languages, several stems can be used as placeholders. All stems discussed below, except for the borrowed stems, are shown in \figref{fig:stems}.

\subsubsection{\textit{aŋə/aŋi/anu}}
This stem\footnote{I will gloss it as \textsc{ph1}.} can be found in Evenki (\ref{ex:klyachko:a.evk}) \citep{klyachko2022} , Udihe (\ref{ex:klyachko:a.ud}) (\citealt[16]{Shneider1936}, \citealt[361]{nikolaeva2001grammar}), Oroch (\ref{ex:klyachko:a.oroch}), and Uilta (\ref{ex:klyachko:a.uilta}) \citep[6]{ozolinya2001}:

\is{placeholder} 
\ea
\label{ex:klyachko:a.evk}
\langinfo{Evenki}{}{\cite{EvenkiCorpusSiberianLang}}\\
\gll həwəkiː hələ \textbf{aŋi-l-duk} \uline{sʲita-l-duk} o-dʲa-fkiː bi-sʲə bəjə-l-bə\\
God \textsc{intj} \textbf{\textsc{ph1}-\textsc{pl}-\textsc{abl}} clay-\textsc{pl}-\textsc{abl} make-\textsc{ipfv}-\textsc{ptcp.hab} be-\textsc{ptcp.ant} human-\textsc{pl}-\textsc{acc}\\
\glt `Well, God made people out of \textbf{whatchamacallit}, \uline{out of clay}.' 
\z
\il{Evenki}



\is{placeholder} 
\ea
\label{ex:klyachko:a.ud}
\langinfo{Udihe}{}{\cite[361]{nikolaeva2001grammar}}\\
\gll ge oloktoː-ni belie \textbf{aŋi-we} \uline{zakta-wa}\\
well cook.\textsc{pst}-\textsc{3sg} fairy \textsc{ph1}-\textsc{acc} porridge-\textsc{Acc}\\
\glt `Well, the fairy cooked this, \textbf{what it's called}, \uline{porridge}.'
\z
\il{Udihe}


\is{placeholder} 
\ea
\label{ex:klyachko:a.oroch}
\langinfo{Oroch}{}{\cite[text 12]{kazama1996}}\\
 \glll ӈэнэ-м=дээ	баа-ха-њи \textbf{аӈи-ва} \uline{омо кава}\\
 ŋene-m=dee	baa-xa-ɲi \textbf{aŋi-wa} ... omo kawa\\
 go-\textsc{cvb}=\textsc{foc} find-\textsc{pst}-\textsc{sg} \textsc{ph1}-\textsc{acc} ... one house\\
 \glt `Walking, he found \textbf{whatchamacallit}... \uline{one house.'}\footnote{Interestingly, this is the nominative form (\textit{kawa(n)} `cabin'), although the placeholder has the accusative suffix. However, the original audio recording shows that there was actually code-switching to Russian before pronouncing the word \textit{kawa}, when the speaker was trying to recall the word.}.
\z
\il{Oroch} 

\is{placeholder} 
\ea
\label{ex:klyachko:a.uilta}
\langinfo{Uilta}{}{\cite{minlang_uilta}}\\
 \gll čōtči tar šofʲor nari \textbf{anu} \textbf{bāru-ni} \uline{val} \uline{bāru-ni} zvoni-ri-ni val gasa-tai-ni\\
then that driver.R person \textsc{ph1} in.direction-\textsc{3sg.poss} Val in.direction-\textsc{3sg.poss} phone.R-\textsc{prs}-\textsc{3sg} Val settlement-\textsc{all}-\textsc{3sg.poss}\\
 \glt ‘Then that driver phones \textbf{in the direction of whatchamacallit}, \uline{in the direction of Val}, to the village of Val.’
\z
\il{Uilta}

The targets from the corpora include nouns ((\ref{ex:klyachko:a.evk}), (\ref{ex:klyachko:a.ud}), (\ref{ex:klyachko:a.oroch}), (\ref{ex:klyachko:a.uilta})), verbs ((\ref{ex.imp.evk}), (\ref{ex.allmarkers.oroch})) and, in rare cases, other parts of speech, such as adverbs (see \citealt{klyachko2022}). The frequency of this placeholder in the corpus of \citep{EvenkiCorpusSiberianLang} is $340$ per $37 108$ running words, which is about $9$ times per thousand words. It is also worth noting that in Evenki, the use of the \textit{aŋi} stem depends on the dialect, with its being rare or completely unknown in the Eastern Evenki dialects \citep{klyachko2022}. The corpus is not balanced regarding Eastern and non-Eastern data, with the non-Eastern data prevailing in it, and \textit{aŋi}'s frequency for the non-Eastern dialect subcorpus is higher (about $10$ per thousand words).
 
In~the Uilta corpus (\cite{uiltacorpus}), the frequency of \textit{anu} is $75$ per $18 838$ words, which is about $4$ per thousand words. The lower frequency as compared to the Evenki corpus is due to the fact that some texts in the Uilta corpus are published narratives from \citet{petrova1967}, which do not contain any placeholders, perhaps because of T.~Petrova's editing the texts or due to the conditions of how she was recording them (they may have been dictated, which reduced the level of spontaneity). The target is overtly expressed in about $70\%$ of the examples.
 
Unfortunately, I cannot measure the frequency of \textit{aŋi} for Udihe and Oroch due to the lack of large open-access corpora.

When substituting nominal and verbal stems, \textit{aŋə/aŋi/anu} can mirror the corresponding inflectional affixes. However, mirroring is not always full. Strikingly, the mirroring behaviour of \textit{aŋə/aŋi/anu} seems to be uniform throughout the languages which have the stem. As regards nouns, \textit{aŋə/aŋi/anu} stem usually mirrors case, number and possession, including indirect possession markers (sometimes called alienable possession markers, see \citet{AralovaPakendorf+2023+1563+1592} for an analysis). For example, in~(\ref{ex.all.markers.uilta}) the stem copies all possible nominal slots. In the first sentence, the speaker used the word \textit{dɵ̄kso-ŋu-l-ba-ni} (otter-\textsc{alien}-\textsc{pl}-\textsc{acc}-\textsc{3sg.poss}), which is what \textit{anu-ŋu-l-ba-ni} may stand for in my opinion, as the other names of the animals are repeated. However, it is still a matter of doubt whether the speaker used the placeholder because of suddenly forgetting the word she used previously (a typical function of a placeholder) or because she simply wanted to refer to the target without specifying it.


\is{placeholder} 
 \ea 
 \label{ex.all.markers.uilta}
 \langinfo{Uilta}{}{\cite{UiltaCorpusGusevToldova}}\\
 \gll no:ni duku-du-ni oro: suli-ŋu-l-ba-ni dɵ̄kso-ŋu-l-ba-ni səːpə-ŋu-l-ba-ni ujlle:-t͡ʃi əlləutə mon-ʒi-t͡ʃi <...> gə ilaːn-ʒi asi-lu ča bəiŋə-ŋu-l-ba-ni \textbf{anu-ŋu-l-ba-ni} səːpə-ŋu-l-ba-ni suli-ŋu-l-ba-ni t͡ʃipaːli monʒi-ɣa-t͡ʃi\\
\textsc{3sg} house-\textsc{loc}-\textsc{3sg.poss} \textsc{intj} fox-\textsc{alien}-\textsc{pl}-\textsc{acc}-\textsc{3sg.poss} otter-\textsc{alien}-\textsc{pl}-\textsc{acc}-\textsc{3sg.poss} sable-\textsc{alien}-\textsc{pl}-\textsc{acc}-\textsc{3sg.poss} work.\textsc{prs}-\textsc{3pl} very smoothen.\textsc{prs}-\textsc{3pl} <...>
well three-\textsc{ins} woman-\textsc{com} that animal-\textsc{alien}-\textsc{pl}-\textsc{acc}-\textsc{3sg.poss} \textsc{ph1}-\textsc{alien}-\textsc{pl}-\textsc{acc}-\textsc{3sg.poss} sable-\textsc{alien}-\textsc{pl}-\textsc{acc}-\textsc{3sg.poss} fox-\textsc{alien}-\textsc{pl}-\textsc{acc}-\textsc{3sg.poss} all smoothen-\textsc{pst}-\textsc{3pl}\\
 \glt `In his house, they work much, smoothen (the skins of) foxes, otters, sables. <...> With the three wives, they started to smoothen (the skin of) these animals, \textbf{and others}, sables, foxes.'
\z
\il{Uilta} 

In verbal forms, the stem usually mirrors mood (\ref{ex.imp.evk}), tense, person, and number markers (\ref{ex.allmarkers.oroch}). 

\is{placeholder} 
 \ea 
 \label{ex.imp.evk}
 \langinfo{Evenki}{}{\cite{EvenkiCorpusSiberianLang}}\\
 \gll t͡ʃaŋit	tar	t͡ʃaŋit-pa	tarə \textbf{aŋi-waːt}	\uline{t͡ʃok-naː-γaːt}\\
 bandit	that	bandit-\textsc{acc}	that.\textsc{acc} \textsc{ph1}-\textsc{imper.1pl.incl} kill-\textsc{and}-\textsc{pl.incl}\\
 \glt `\textbf{Let us do that thing}, \uline{let us go and kill} that bandit (=bear).'\\
 \z
\il{Evenki}



\is{placeholder} 
 \ea 
 \label{ex.allmarkers.oroch}
 \langinfo{Oroch}{}{\cite{kazama1996}}\\
 \glll тоjоо=чи мапачаа \textbf{аӈи-ва} \textbf{аӈи-ха-њи} \uline{лоо-хо-њи}\\
 tojoo=t͡ʃi mapat͡ʃaa \textbf{aŋi-wa} \textbf{aŋi-xa-ɲi} \uline{loo-xo-ɲi}\\
then=\textsc{foc} old.man \textsc{ph1}-\textsc{acc} \textsc{ph1}-\textsc{pst}-3\textsc{sg} hang-\textsc{pst}-3\textsc{sg}\\
 \glt `Then the old man \textbf{did whatchamacaillit} to \textbf{whatsitsname}, \uline{hung} [fish to dry it].'
 \z
\il{Oroch}

As we have already seen above \REF{ex:klyachko:1.1}, participial and converbial affixes can also be copied. Moreover, some other affixes, traditionally considered as derivational, are also copied, like the caritive marker in (\ref{ex.car.uilta}):

\is{placeholder} 
 \ea \label{ex.car.uilta}
 \langinfo{Uilta}{}{\cite{UiltaCorpusGusevToldova}}\\
 \gll biː məːnə orkin-duk-ki un-ʒi-ni darama \textbf{anu-lu} \textbf{ana} o-tči-mbi \uline{bəgʒi-lu} \uline{ana}\\
\textsc{1sg} \textsc{rfl} bad-\textsc{abl}-\textsc{rfl} say-\textsc{prs}-\textsc{3sg} back \textsc{ph1}-\textsc{car} \textsc{neg} become-\textsc{pst}-\textsc{1sg} leg-\textsc{car} \textsc{neg}\\
 \glt `Because of behaving badly I was left \textbf{without whatchamacallit}, \uline{without my legs}.'
 \z
\il{Uilta}


The question is which markers are \textbf{not} usually copied. The derivational markers that are mirrored are usually those which are both highly frequent and have little selection restrictions, i.~e. combine with a great number of stems often belonging to different parts of speech, like the caritive marker, the adjectivizer as well as the diminutive or the intensifier markers. More specialized markers, such as verbalizers or nominalizers, are not usually copied. Interestingly, the voice marker, which is traditionally viewed as inflectional, is also not copied, at least in the examples from the corpora. For instance, the voice marker is not copied in the placeholder case \REF{ex:klyachko:1.1}, although it is copied in the general extender case \REF{ex:klyachko:1.2}. As regards particles, they are also sometimes copied as in (\ref{ex.particle.evk}), where the \textit{=dVː} focus particle is used in the negation construction.


\is{placeholder} 
 \ea 
 \label{ex.particle.evk}
 \langinfo{Evenki}{}{\cite{EvenkiCorpusSiberianLang}}\\
 \gll \textbf{aŋi-n=daː} ɲi= \uline{əmkə-n=dəː} asʲin bi-soː-n\\
 \textsc{ph1}-\textsc{3sg.poss}=\textsc{foc} not.R.\textsc{slip} cradle-\textsc{3sg.poss}=\textsc{foc} \textsc{neg} be-\textsc{pst}-\textsc{3sg}\\
 \glt `There was no \textbf{(his) whatchamacallit}, \uline{(his) cradle}.'\\
 \z
\il{Evenki}


The etymology of the \textit{aŋə/aŋi/anu} stem is enigmatic. For Evenki and Udihe, it has been reported as an interrogative pronoun \citep[25]{BulatovaGrenoble1999} and an indefinite pronoun \citep[362]{nikolaeva2001grammar}. This would be a good etymology because interrogative and indefinite pronouns often become a source for placeholders \citep{Podlesskaya2010}. However, if we look at the examples provided in the grammars, they are hardly ``normal'' indefinite or interrogative pronoun examples. On the contrary, they can rather be interpreted as typical placeholder examples. For~instance, the authors remark that the following example (\ref{ex.evenki.bulatova_grenoble}) ``can be translated differently according to the context":


\is{placeholder} 
 \ea 
 \label{ex.evenki.bulatova_grenoble}
 \langinfo{Evenki}{}{\cite[25]{BulatovaGrenoble1999}}\\
 \gll \textbf{aŋiː-wa} nuŋan buː-rə-n\\
\textsc{ph1}-\textsc{acc} \textsc{3sg} give-\textsc{nfut}-\textsc{3sg}\\
 \glt `\textbf{What/who/how many/what kind} did he give?'
 \z
\il{Evenki}

They also emphasize the fact that \textit{aŋi} can also be used in assertive statements, ``acquiring the role of a substantivized pronoun". Moreover, 
\citet{bulatova2003} describes \textit{aŋi}'s ``more universal semantics" in comparison to the other Evenki interrogative pronouns. I think that the ``semantic universality'' of \textit{aŋi}, its context-defined translations, and, importantly, its use in assertive statements prove that it is actually a placeholder and not just an ordinary interrogative pronoun.

As regards \textit{aŋi} as an indefinite pronoun in Udihe, the situation is more difficult as the following Udihe example (\ref{ex.udihe.nikolaeva_tolskaya}) unfortunately lacks context.

\is{placeholder} 
 \ea \label{ex.udihe.nikolaeva_tolskaya}
 \langinfo{Udihe}{}{\cite[362]{nikolaeva2001grammar}}\\
 \gll \textbf{aŋi-le} ŋeneː-mi\\
 \textsc{ph1}-\textsc{loc} walk.\textsc{pst}-\textsc{1sg}\\
 \glt `I was walking \textbf{somewhere}.'
 \z
\il{Udihe}

Nevertheless, I have not found any case of \textit{aŋi} being used as an interrogative or an indefinite pronoun in the Evenki or Udihe text corpora. As~\citet{tolskaya2008question} notice, \textit{aŋi} ``is abundant in spontaneous speech... but it does not occur in the autobiographical book by A. Kanchuga", which would be odd for an ordinary indefinite pronoun. Next, the native speakers' estimation of \textit{aŋi} can also give a clue. Actually, many Evenki speakers' reluctance to seeing \textit{aŋi} in the transcription of their narratives is also typical of a low-prestige discourse marker but not of an ordinary interrogative pronoun. In addition, there exist standard interrogative and indefinite pronouns, which speakers use during elicitation experiments. Summing up, \textit{aŋi's} interrogative or indefinite pronoun meanings are in my opinion just the result of how the authors of the grammars treated placeholder examples. It is also worth noting that the Russian placeholders used to translate the corresponding Tungusic ones are derived from pronominal stems. Such translations could also have influenced the linguists in analyzing \textit{aŋi} as a pronoun. The form \textit{aŋi} may have actually evolved from an interrogative or an indefinite pronoun but the stem is certainly not the standard interrogative or an indefinite pronoun in any Tungusic language at the synchronic level.


Typically, placeholder stems are etymologically connected with interrogative or demonstrative pronouns or abstract nouns meaning `thing' \citep{Podlesskaya2010}. \citet{idiatov2007typology} supposes that the Evenki \textit{aŋi} can be a word originally meaning `thing' with the -\textit{ŋi} formant, which is a possessive marker. The disadvantage of this hypothesis is the lack of the original `thing' stem. I think that \textit{aŋi} can be also compared to the Amuric Nivkh \textit{aŋ}, which is an interrogative pronoun used to refer to people and anthropomorphic creatures (\citealt[253]{panfilov1965}, \citealt[33]{taksami1970}). However, I should admit that this pronoun is not used as a placeholder in the corpus of oral texts in Amuric and Sakhalin Nivkh \citep{smnnivkh} in contrast with other interrogative pronouns, which are abundant there and used as placeholders. Still, pronoun borrowing between Nivkh and Tungusic languages is not impossible. For example, two Nivkh interrogatives may have been borrowed into Uilta \citep[120]{holzl2018typology}. Therefore, it is theoretically possible that the Nivkh \textit{aŋ} was initally borrowed as an interrogative pronoun but then turned into a placeholder. Placeholder borrowing itself is quite common, too, as I will show below.

\subsubsection{\textit{uŋun/uŋ}}

The form \textit{uŋun/uŋ}\footnote{I will gloss it as \textsc{ph2.}} is another stem that can be found in some Evenki dialects \citep{klyachko2022}, in some Even dialects (\citealt{matic2008}, \citealt[271]{robbek2005}), and in Negidal (\cite{chapters/pakendorf}).
In Even and Negidal, it is a general purpose placeholder ((\ref{ex.ungun.neg}), (\ref{ex.ungun.eve})) for nominal or verbal stems, whereas in Evenki, it usually substitutes proper nouns like names of people or places ((\ref{ex.ungun.evk}), see discussion in \citet{klyachko2022}).


\is{placeholder} 
 \ea \label{ex.ungun.neg}
 \langinfo{Negidal}{}{\cite[AET\_village\_life 83]{NegidalCorpus}}\\
 \gll tos aːʨin bi-ʨa-n o-ta-s \textbf{uŋun-a} \uline{tosta-ja}\\
		salt \textsc{neg} be-\textsc{pst}-\textsc{3sg} \textsc{neg}-\textsc{neg}.\textsc{fut}-\textsc{2sg} \textsc{ph2}-\textsc{neg}.\textsc{cvb} salt-\textsc{neg}.\textsc{cvb}\\
		\glt ‘There was no salt, <so> you won’t \textbf{do this}, \uline{salt}.'
 \z
\il{Negidal}


\is{placeholder} 
 \ea \label{ex.ungun.eve}
 \langinfo{Even}{}{\cite{matic2008}}\\
 \gll min \textbf{uŋ-il-bu} \uline{\{eń-til-bu\}} ... amaskị badụ-sn-̣i-tan\\
 \textsc{1sg}.\textsc{obl} \textsc{ph2}-\textsc{pl}-\textsc{1sg.poss} mother-\textsc{pl}-\textsc{1sg.poss} ... back ride-\textsc{lim}-\textsc{pst}-\textsc{3pl}\\
 \glt `My \textbf{whatchamacallit} \uline{\{my parents\}} ... went back.'
 \z
 \il{Even}

\is{placeholder} 
 \ea \label{ex.ungun.evk}
 \langinfo{Evenki}{}{\cite{EvenkiCorpusSiberianLang}}\\
 \gll bəjə	\textbf{uŋun-mə} \uline{dəwit-pa} haː-∅-ndə\\
 person	\textsc{ph2}-\textsc{acc}	David-\textsc{acc}	know-\textsc{nfut}-2\textsc{sg}\\
 \glt `Friend, do you know \textbf{whatshisname}, \uline{David}?'
 \z
\il{Evenki}

Due to its proper name specialization in Evenki, \textit{uŋun} is infrequent in the Evenki oral corpora. Being on the contrary the main placeholder in the Negidal language corpus, it occurs $496$ times in a corpus of $41 195$ running words, that is about $12$ times per one thousand words.

Just like \textit{aŋi}, the \textit{uŋ(un)} wordform can copy inflectional and some frequent derivational affixes like the equative marker, as in (\ref{ex.ungun.neg.eqt}).

\is{placeholder} 
 \ea \label{ex.ungun.neg.eqt}
 \langinfo{Negidal}{}{\cite[APN\_lechenie\_ed 61]{NegidalCorpus}}\\
 \gll eeeeː net kipʲati-ja vari-ja kak \textbf{uŋun-gaʨin-mə} ti kak \uline{ɟokʨi-gaʨin} oː-knani-n\\
\textsc{intj} \textsc{no}.R boil.R-\textsc{nfut} cook.R-\textsc{nfut} like.R \textsc{ph2}-\textsc{eqt}-\textsc{acc} like like.R tar-\textsc{eqt} become-\textsc{cvb}-\textsc{3sg}\\
		\glt ‘No, one boils it, cooks it so that it becomes \textbf{like whatchamacallit}, \uline{like tar}.'
 \z
\il{Negidal}

The etymology of the stem is once again enigmatic as it does not correspond to any known interrogative pronoun. The word can be found in the Evenki language dictionary as \textit{ugun}\footnote{The interchangeability of intervocal \textit{ŋ} and \textit{g} is commonly attested in Evenki dialects.} \citep[625]{vasilevich1958}, with the following definition ``who? what? a question on an unknown person or animal". Despite the wording of this definition, which may once again suggest an interrogative function of this word, there are actually no examples of its being used as an interrogative in a text, a grammar or a dictionary. I think that, similarly to the \textit{aŋə/aŋi/anu} case, the Evenki \textit{uŋun} is not an interrogative pronoun at the synchronic level, with the dictionary definition being misleading. In~\citet[247]{cincius1975_1977}, the Evenki \textit{ugun}\footnote{\citet{cincius1975_1977} must have used \citet[625]{vasilevich1958} as a source, which explains why the form is \textit{ugun} and not \textit{uŋun}.} is compared to the Kur-Urmi Nanai \textit{uːnəkə-}, an interrogative verb meaning `what to do?'. Unfortunately, no other parallels, like the Even or the Negidal one, are mentioned there, and it is still a question whether Tungusic languages have this stem. In~this paper, I do not consider the Kur-Urmi Nanai \textit{uːnəkə-} because, according to the dictionary and corpus data, it does not function as a placeholder. Still, the existence of this verb may suggest that \textit{uŋ(un)-} used to function as an interrogative pronoun. Its proper name specialization in Evenki must have developed later. However, a difficulty with the Kur-Urmi Nanai is its mixed-language nature \citep[4]{Hölzl2022} with a lot of Evenki influence. One option is that \textit{uŋ(un)-} was a question word in the common ancestor of Evenki, Even, Negidal and Nanai and then developed into a placeholder in the Ewenic branch comprising Evenki, Even, and Negidal. Another option is that \textit{uŋ(un)-} functioned as a question word in the common ancestor of Evenki, Even, and Negidal and was then borrowed into the Kur-Urmi Nanai from Evenki. In this latter case, it may have functioned as a question word in Evenki, too, and later developed into a placeholder in all the languages of the Ewenic branch, losing its question word properties.


\subsubsection{Interrogative pronouns}


Several Tungusic interrogative pronoun stems can be used as placeholders, namely: \textit{xaj-}, \textit{e:(kun)-} and \textit{ịa-}(see \citealt[312--315]{holzl2018typology} for a detailed description).

Etymologically, \textit{xaj-} is reconstructed as \textit{*Kai} and is used as a placeholder in Nanai ((\ref{ex.xaj.gld.1}), (\ref{ex.xaj.gld.2})) and Ulch (\ref{ex.xaj.ulch.1}). The stems \textit{e:(kun)-} and \textit{ịa-} are reconstructed as \textit{*ja}. The form \textit{e:(kun)-} functions as a placeholder in some dialects of Evenki ((\ref{ex.ekun.evk}), sometimes with the \textit{=kanan} or \textit{=kana} particle (\ref{ex.ekun.evk.kanan})). The form \textit{ịa-} is a placeholder in some Even dialects (usually with the \textit{=kanan} particle, \citealt{matic2008}).



\is{placeholder} 
 \ea \label{ex.xaj.gld.1}
 \langinfo{Nanai}{}{\cite[216, 217]{beldy2012}}\\
 \glll эси=гдэ тэй бэюн туй туту-хэ-ни туй туту-хэ-ни туй элэ \textbf{хай-ва-ни} \uline{такто-ва-ни} иси-ха\\
 esi=gde tej bejun tuj tutu-xe-ni tuj tutu-xe-ni tuj ele \textbf{xaj-wa-ni} \uline{takto-wa-ni} isi-xa\\
now=\textsc{ptcl} that elk so run-\textsc{pst}-\textsc{3sg} so run-\textsc{pst}-\textsc{3sg} so already what-\textsc{acc}-\textsc{3sg.poss} barn-\textsc{acc}-\textsc{3sg.poss} reach-\textsc{pst}\\
 \glt `Now that elk was running so fast, so fast, reached \textbf{<her whatchamacallit>}, \uline{her barn}.'
 \z
\il{Nanai}

\is{placeholder} 
 \ea \label{ex.xaj.gld.2}
 \langinfo{Nanai}{}{\cite[36, 37]{beldy2012}}\\
 \glll \textbf{хай-ӈ-го-и} \uline{киалако-ӈ-го-и-ва} бугу\\
 \textbf{xaj-ŋ-go-i} \uline{kialako-ŋ-go-i-wa} bugu\\
what-\textsc{alien}-\textsc{desig}-\textsc{1sg.poss} cropped.bone-\textsc{alien}-\textsc{desig}-\textsc{1sg.poss}-\textsc{obl} give.back.\textsc{imp}\\
 \glt `Give back the <\textbf{whatsitsname}>, \uline{the cropped bone}!'
 
 \z
\il{Nanai}



\is{placeholder} 
 \ea \label{ex.xaj.ulch.1}
 \langinfo{Ulch}{}{\cite{ulchcorpus}}\\
 \gll na:n motor-wa \textbf{xaj-rɪ-nɪ} motor ŋən-i-wə-n \uline{ətəw-ri-ni}\\
 \textsc{3sg} engine.R-\textsc{acc} what-\textsc{prs}-\textsc{3sg} engine.R go-\textsc{prs}-\textsc{acc}-\textsc{3sg} watch-\textsc{prs}-\textsc{3sg}\\
 \glt `He \textbf{does whatchamacallit} with the engine, \uline{watches} how the engine is working.'
 
 \z
\il{Ulch}

\is{placeholder} 
 \ea \label{ex.ekun.evk}
 \langinfo{Evenki}{}{\cite{EvenkiCorpusSiberianLang}}\\
 \gll dʲukt͡ʃa	\textbf{eːkun-ma-n}	ham-na-∅	\uline{urkə-wə-n}	ham-na-∅\\
 tent	what-\textsc{acc}-\textsc{3sg.poss}	close-\textsc{nfut}-3\textsc{pl}	door-\textsc{acc}-\textsc{3sg.poss}	close-\textsc{nfut}-3\textsc{pl}\\
 \glt `They close the tent’s \textbf{whatsitsname}, \uline{its door}.' \\
 \z
\il{Evenki}


\is{placeholder} 
 \ea \label{ex.ekun.evk.kanan}
 \langinfo{Evenki}{}{\cite{EvenkiCorpusSiberianLang}}\\
 \gll nu oro-r-duk bagar tari-ŋ-i nan aja-ma-t \textbf{e:-ŋna-Ø-nni=kanan}\\
\textsc{intj} reindeer-\textsc{pl}-\textsc{abl} perhaps that-\textsc{alien}-\textsc{acc.rfl} again good-\textsc{ints}-\textsc{advz} \textbf{what-\textsc{hab}-\textsc{nfut}-\textsc{2sg}=\textsc{foc}}\\
\glt `Well, \textbf{you do whatchamacallit} <make a talisman> also very well from reindeer.' \\
 \z
\il{Evenki}

There are very few examples of the corresponding`what' stem used in a placeholder function in Negidal (\ref{ex.ekun.neg.ph}) and Udihe (\ref{ex.what.ude.ph}):


\is{placeholder} 
 \ea \label{ex.ekun.neg.ph}
 \langinfo{Negidal}{}{\cite[DIN\_rite 28]{NegidalCorpus}}\\
 \gll taj hujan \textbf{eːkun-du-n} \uline{səβəki-du-n} təje-s tiː\\
so forest what-\textsc{dat}-\textsc{poss.3sg} god-\textsc{dat}-\textsc{poss.3sg} treat-\textsc{nfut}.\textsc{2sg} so\\
\glt `You treat \textbf{<who>}, \uline{the spirit of the taiga}.\\
 \z
\il{Negidal}


\is{placeholder} 
 \ea \label{ex.what.ude.ph}
 \langinfo{Udihe}{}{\cite[Sisam Zauli and the hero]{nikolaevagarret}}\\
 \gll a: j’eu ute-bede ed’e-u kile diaŋ-ki-ni jeu=ke \textbf{jewe-ni}[uti uti] uti zoŋcula-ŋku \uline{o:-ni} su:kte-zi\\
\textsc{intj} what this-how become.\textsc{pf}-\textsc{3sg} seagull say-\textsc{pst}-\textsc{3sg} what=\textsc{indef} what-\textsc{3sg} this this this shoot-\textsc{nmlz} do.\textsc{pst}-\textsc{3sg} horsetail-\textsc{ins}\\
\glt `The seagull said: «Ah, how did it happen? They\footnote{The original clause is impersonal, this is why \textsc{3sg} verbal markers are used.} \textbf{<did whatchamacallit>}, \uline{made} an arrow out of a horse-tail \footnote{Despite its form, \textit{jewe-} is probably not an accusative form, see \citet[324]{holzl2018typology} for a discussion.}'\\
 \z
\il{Udihe}
 

The frequency of \textit{e:(kun)} as a placeholder is low in the Evenki text corpora. It is due to the fact that Southern and Northern dialect texts prevail in the corpora, whereas \textit{e:(kun)-} is a typical Eastern placeholder. Still, \textit{e:(kun)} does occur in non-Eastern texts but is far less frequent than \textit{aŋi}.

Apart from the interrogative pronoun stems mentioned above, there is also \textit{və} `who' in Sibe, which is used as a placeholder for ``a concrete proper name or
the specific designation of a person" \citep[111]{zikmundova2013spoken}, when the speaker has forgotten a name or does not want to pronounce it. For example, in~(\ref{ex.ve.sibe.ph}), ``the speaker is afraid of pronouncing a foreign name incorrectly".


\is{placeholder} 
 \ea \label{ex.ve.sibe.ph}
 \langinfo{Sibe}{}{\cite[111]{zikmundova2013spoken}}\\
 \gll məzə-i \textbf{və} … \uline{arien} əm jašqa\textsuperscript{n} ji-ɣ\textsuperscript{ə}i\\
\textsc{1pl.incl.gen} who … Arienne one letter come-PERF\\
\glt `\textbf{<whatchamacallit>} \uline{Arienne} wrote me a letter'\\
 \z
\il{Sibe}

\subsubsection{Demonstrative pronoun placeholders}


Demonstrative pronouns are typologically frequent sources for placeholders. For instance, in~Russian, one of the most frequently used placeholders is \textit{этот} (\textit{etot} `this'), which has been borrowed into Tungusic languages, too (see \sectref{Borrowed_placeholders}). However, among the Tungusic languages covered in this survey, it is only Sibe that makes great use of demonstratives (\textit{ək, səkei/skəi}) as placeholders, according to the detailed description in \citet{zikmundova2013spoken}, with (\ref{ex.sibe.ph}) as one of the placeholder examples:


\is{placeholder} 
 \ea \label{ex.sibe.ph}
 \langinfo{Sibe}{}{\cite[106]{zikmundova2013spoken}}\\
\gll ər χaʁəč da nan\textsuperscript{ə}-i favən-t dØš-kəŋ arq-aq\textsuperscript{ů} da \textbf{sk\textsuperscript{ə}i} \uline{xovə-i} diØrgit qotoN zə-m da dØš-k\textsuperscript{ə}i\\
this boy \textsc{ptcl} person-\textsc{gen} law-\textsc{dat} enter-\textsc{perf} method-\textsc{neg} \textsc{ptcl} \textsc{ph}
coffin-\textsc{gen} inside \textsc{onom} say-\textsc{cvb} \textsc{ptcl} enter-\textsc{perf}\\
 \glt `The boy had no other choice but to follow the (other person’s) order, and so he entered \textbf{whatchamacallit} \uline{coffin} with a thud.'
 \z
\il{Sibe}

\citet{zikmundova2013spoken} argues that Sibe placeholders can be used not only when the speaker has forgotten a word but also in case they do not want to pronounce it. For example, in~(\ref{ex.sibe.ph}) the word \textit{xovə-i} `coffin-\textsc{gen}' was disturbing for the speaker. This avoidance function is typical of placeholders, see \citet{cheung2015uttering}, \citet{seraku2024placeholders}.


\subsubsection{Borrowed placeholders}
\label{Borrowed_placeholders}
Placeholder borrowing is not a rare phenomenon. For example, in~\citet{chapters/visser}, placeholder borrowing in Kalamang is discussed.

Tungusic languages demonstrate some borrowed placeholders, such as \textit{eto} (`this' < Russian (borrowed into many languages in contact with Russian, cf. \citealt{chapters/ventayol_boada} on Kolyma Yukaghir); ((\ref{ex.eto.evk1}), (\ref{ex.eto.evk2}), (\ref{ex.eto.evk3})) or \textit{bolla} / \textit{bo:la} (<\textit{buolla}, which is a modal particle in Sakha, ((\ref{ex.bolla.evk}), ( \ref{ex.bolla.evn})).

These placeholders differ greatly from the ``native" ones because of the lack of mirroring. In Sakha, \textit{buolla} is a non-inflecting modal particle derived from a verbal stem. In Russian, \textit{eto} can be a non-mirroring filler but mirroring is also possible with the same stem (\textit{etot}). On the contrary, in the Evenki texts, \textit{eto} is never inflected. Not surprisingly, it is sometimes hard to say if the borrowing is used as placeholder or as a discourse marker from another category. For example, in~(\ref{ex.eto.evk1}), it is used at the beginning of the phrase as a hesitative marker. In~(\ref{ex.eto.evk2}), \textit{eto} is used together with the Russian borrowing \textit{ʃatun} `travelling bear', so the question is whether the use of these two Russian words is an example of borrowing or code-switching. In~(\ref{ex.eto.evk3}), it is perhaps a placeholder. Actually, in most examples from the corpus of \citep{EvenkiCorpusSiberianLang}, \textit{eto} is rather a hesitative at the beginning of the phrase or is used together with other Russian borrowings. Therefore, the lack of mirroring may be due to the fact that the word is a borrowing and not fully morphologically adapted or because it is actually not a borrowing but rather an example of code-switching.


\is{placeholder} 
 \ea \label{ex.eto.evk1}
 \langinfo{Evenki}{}{\cite{EvenkiCorpusSiberianLang}}\\
 \gll \textbf{eto} jakutskij-la: əmə-ksək tʲotʲa-m guni-pki: a aŋnu:pki:\\
this.R Yakutsk-\textsc{all} come-\textsc{cvb} aunt.R-\textsc{poss.1sg} say-\textsc{ptcp.hab} \textsc{intj} ask-\textsc{ptcp.hab}\\
\glt `\textbf{Uhm}, when I came to Yakutsk, my aunt said, oh, asked' \\
 \z
\il{Evenki}


\is{placeholder} 
 \ea \label{ex.eto.evk2}
 \langinfo{Evenki}{}{\cite{EvenkiCorpusSiberianLang}}\\
 \gll umno bi-t͡ʃoː-n \textbf{eːkun=ka} nu \uline{hawal-dʲa-riː} \textbf{eto} \uline{ʃatun}\\
once be-\textsc{pst}-\textsc{3sg} what=\textsc{foc} \textsc{intj}.R work.travel-\textsc{ipfv}-\textsc{ptcp} this.R travelling.bear.R\\
\glt `Well, once there was \textbf{whatchamacallit}, a \uline{travelling bear}' \\
 \z
\il{Evenki}



\is{placeholder} 
 \ea \label{ex.eto.evk3}
\langinfo{Evenki}{}{\cite{EvenkiCorpusSiberianLang}}\\
 \gll tadu: \textbf{eto} \uline{it͡ʃə-t-t͡ʃə-m=də:}\\
there this.R see-\textsc{dur}-\textsc{fut}-\textsc{1sg}=\textsc{foc}\\
\glt `There, \textbf{I will do whatchamacallit}, \uline{I will see}.' \\
 \z
\il{Evenki} 


As regards \textit{bolla}, it is used widely in the Evenki dialects of the Far East and in the dialects of Even. A native speaker of Evenki\footnote{V.~Sabrski, the settlement of Tugur, p. c.} even directly interpreted it as the borrowed Sakha word for the Evenki \textit{aŋi}. However, it is also hard to say whether it is a placeholder or a hesitation filler in examples like (\ref{ex.bolla.evk}) or the Lower Kolyma Even example (\ref{ex.bolla.evn}), where the Sakha borrowing \textit{bolla} is used together with the ``native" interrogative placeholder:

\is{placeholder} 
 \ea \label{ex.bolla.evk}
 \langinfo{Evenki}{}{\cite{EvenkiCorpusSiberianLang}}\\
 \gll a lamus o-ra-n \textbf{boːla}\\
\textsc{intj} deep.snow become-\textsc{nfut}-\textsc{3sg} \textsc{ptcl}.SAH\\
\glt `Oh, it was deep snow, \textbf{whatchamacallit}' \\
 \z
\il{Evenki}


\is{placeholder} 
 \ea \label{ex.bolla.evn}
 \langinfo{Even}{}{\cite[139]{Sharina2018}}\\
 \gll ə tar \textbf{ja-vra-ra-m} \textbf{bolla} \uline{ikə-vrə-rə-m}\\
\textsc{intj} that what-\textsc{hab}-\textsc{nfut}-\textsc{1sg} \textsc{ptcl}.SAH sing-\textsc{hab}-\textsc{nfut}-\textsc{1sg}\\
\glt `Oh, \textbf{<I do whatchamcallit>}, \uline{I sing}' \\
 \z
\il{Even}

\begin{figure}
 \centering
\includegraphics[width=\textwidth]{placeholder_map}
 \caption{Placeholder stems in Tungusic languages}
 \label{fig:stems}
\end{figure}

\subsection{Tungusic placeholder functions and formal features}

Judging from the corpora examples, the functions of the Tungusic placeholder stems are quite typical of placeholders in general. Tungusic placeholders are used in the following situations:
\begin{enumerate}
\item{The speaker cannot recall a word but wants to preserve fluency and keep the floor (most examples like (\ref{ex:klyachko:a.evk})).}
\item{The speaker does not want to pronounce a specific word due to extra-linguistic reasons (\ref{ex.sibe.ph}).}
\item{The speaker wants to draw the listener's attention due to some extralinguistic factors. For example, the speaker may want the listener to help them recall the target (like (\ref{ex.ungun.evk}), see discussion in~\citet[216]{klyachko2022}). The key point is the interaction between the speaker and the listener. In a way, it is an opposite strategy, as compared~to~1. In~1, the speaker wants the listener to ignore the disfluency and just reconstruct the possible target. In~3, the speaker wants the listener to actively help them recall the missing word. In Sibe \citep[106, 111]{zikmundova2013spoken}, placeholders are sometimes used by speakers to draw the listener's attention to a new participant of the situation or to emphasize the target because it is important (\ref{ex.attention.sibe}) or somewhat extraordinary, e. g. it is a foreign word}.
\end{enumerate}


\is{placeholder} 
 \ea \label{ex.attention.sibe}
 \langinfo{Sibe}{}{\cite[106]{zikmundova2013spoken}}\\
 \gll dučiurə-t \textbf{sk\textsuperscript{ə}i} tər \textbf{ək} \uline{miao} bi vaq na\\
fourth.banner-\textsc{dat} \textsc{ph} that \textsc{ph} temple be \textsc{neg} \textsc{inter}\\
\glt `In the Fourth Banner (a settlement in the Qapqal Xibe Autonomous County) there is \textbf{(that one, you know what)} \uline{temple}' \\
 \z
\il{Sibe}

It is not always possible to derive the actual pragmatic situation from a corpus example. Still, dialogues from the corpus of \citet{EvenkiCorpusSiberianLang} provide much insight into the speakers' interaction and show that speakers actively use placeholders to draw the listeners' attention and ask them for help.

Despite the seeming abundance of various placeholder stems across the Tungusic family, all the stems, apart from the borrowed ones, behave in a very similar way. They can all substitute both nominal and verbal stems, except for the Evenki \textit{uŋun}, which has a specialization for proper nouns. As regards nouns, placeholders usually mirror their case, number, and possession markers. In verbal forms, they usually mirror mood, tense, person, and number markers but not the voice marker. Interestingly, they can also copy frequent derivational markers, such as the equative marker, as in~(\ref{ex.ungun.neg.eqt}). For instance, in the Evenki language corpus (\cite{EvenkiCorpusSiberianLang}), nominal inflectional morphology is fully mirrored in about $76\%$ of all cases where it is possible to estimate mirroring at all (i. e. where the target is present), whereas verbal inflectional morphology is fully mirrored in about $67\%$ of the cases. There are only several sporadic examples of derivational morphology copying, although this may be explained by the comparative rarity of these markers.

The only language, which clearly has a separate placeholder with a specialized function (namely, \textit{uŋun} used specifically for proper nouns) is Evenki. In other cases, there is usually one most frequent placeholder for a dialect. For example, \textit{aŋi} is used in some Northern and Southern Evenki dialects whereas \textit{e:(kun)} is characteristic of most Eastern Evenki dialects.

\section{Tungusic general extenders}
\subsection{Tungusic general extender stems}

General extenders is originally a term suggested by M.~Overstreet for English phrases like ``and stuff like that" \citep{overstreet1999whales}. In~\citet{mauri2018linguistic}, they are defined more precisely as elements located at the
end of a non-exhaustive list, ``whose meaning is indexical with respect to some underlying property P". In most examples from this section, the original ``list" usually comprises only one element but there may be more elements in it, like in~(\ref{ex.ge.evk.list}).

Placeholder and general extender functions can sometimes be found for one stem (see \citealt{chapters/rose} for an analysis of \textit{baʔe} in Teko). General extenders derived from the interrogative pronoun stem are widely used for both nominal and verbal phrases in the Tungusic languages. In all Tungusic languages, the interrogative pronoun stem can also form an interrogative verb `what to do', so the verbal morphology is actually quite common for the stem. I will look at the mirroring properties of Tungusic general extenders and compare them with those of the placeholders.

In the Evenki example (\ref{ex.ge.evk.noun}), the interrogative pronoun \textit{e:(kun)} is used twice, once in its full form (with the \textit{-kun} suffix) as a placeholder, and once in its short form as a general extender \footnote{Actually, full-form general extenders can also be found in the corpus, so it is a question whether this short and full-form distinction is coincidental or not.}. As a general extender, it is used after \textit{oldo-ŋi-l-wa} (fish-\textsc{alien}-\textsc{pl}-\textsc{acc}) to denote other goods, apart from fish, which were transported to the settlement of Valyok. Interestingly, it mirrors the number and case but not the indirect possession suffix. (\ref{ex.ge.evk.list}) is one of the rare examples where the enumeration comprises more than one member, with the members of the enumeration all referring to various items a well-to-do person was once thought to possess. The general extender copies the adjectivizer suffix here. In~(\ref{ex.ge.evk.verb}), the same stem is used to refer to the skills that people usually learn at school, such as reading or writing. The members of the enumeration are thus verbal phrases.


\is{general extender} 
 \ea \label{ex.ge.evk.noun}
 \langinfo{Evenki}{}{\cite{EvenkiCorpusSiberianLang}}\\
\gll walok-tulaː toʐə \textbf{eːku-r-wa} \uline{oldo-ŋi-l-wa} \textbf{eː-l-wa} əmə-wu-pkiː-l bi-t͡ʃo-l\\
Valyok-\textsc{all} also what-\textsc{pl}-\textsc{acc} fish-\textsc{alien}-\textsc{pl}-\textsc{acc} what-\textsc{pl}-\textsc{acc} bring-\textsc{tr}-\textsc{ptcp.hab}-\textsc{pl} be-\textsc{ptcp.ant}-\textsc{pl}\\
\glt `They used to bring \textbf{whatchamacallit}, \uline{fish} \textbf{and so on} to Valyok' \\
 \z
\il{Evenki}


\is{general extender} 
 \ea \label{ex.ge.evk.list}
 \langinfo{Evenki}{}{\cite{EvenkiCorpusSiberianLang}}\\
\gll nu oro-sʲi o-sa dʲu:-sʲi \textbf{e:-sʲi} o-sa\\
\textsc{intj.R} reindeer-\textsc{adj} become-\textsc{ptcp.ant} house-\textsc{adj} what-\textsc{adj} become-\textsc{ptcp.ant}\\
\glt `Well, he became (a person) \uline{with reindeer}, \uline{with a house} \textbf{and stuff}.' \\
 \z
\il{Evenki}

\is{general extender} 
 \ea \label{ex.ge.evk.verb}
 \langinfo{Evenki}{}{\cite{EvenkiCorpusSiberianLang}}\\
\gll muldiː-ka-r ərəgərit eː-wa=da \uline{doku-dʲa-miː=da} \textbf{eː-dʲa-miː=da}\\
cannot.do-\textsc{nmlz}-\textsc{pl} at.all what-\textsc{acc}=\textsc{foc} write-\textsc{ipfv}-\textsc{cvb} what-\textsc{ipfv}-\textsc{cvb} \\
\glt `They could not do anything, could not \textbf{write} \uline{or do similar things}' \\
 \z
\il{Evenki}

The same \textit{eː(kun)} `what' stem is used as a general extender in Negidal ((\ref{ex.ge.neg.noun}), (\ref{ex.ge.neg.verb})). In~(\ref{ex.ge.neg.noun}), it refers to the objects that can be used to produce light, like a torch or matches. It does not only mirror the nouns's possessive marker but is also accompanied with the \textit{=dV} focus particle. \textit{=dV} is, on the one hand, typically used in the coordinating construction in Negidal and, on the other hand, marks negative polarity.


\is{general extender} 
 \ea \label{ex.ge.neg.noun}
 \langinfo{Negidal}{}{\cite[GIK\_2tatarskoe 39]{NegidalCorpus}}\\
\gll man-mi \uline{fonariki-β=de} \textbf{eːkun-mi=da} aːʨin\\
\textsc{rfl}-\textsc{1sg.poss} torch.R-\textsc{1sg.poss} what-\textsc{1sg.poss}=\textsc{foc} \textsc{neg}\\
\glt `I myself didn't have \uline{a torch} \textbf{or anything}.' \\
 \z
\il{Negidal}

In~(\ref{ex.ge.neg.verb}), the same `what'-stem is used as a verb referring to the actions people usually perform when going to the forest, like picking berries. It mirrors the converb affix but not the \textit{-lV} verbalizer, which derives `gather.berries' from `berry'.

\is{general extender} 
 \ea \label{ex.ge.neg.verb}
 \langinfo{Negidal}{}{\cite[GIK\_2tatarskoe 95]{NegidalCorpus}}\\
\gll taj tiː \uline{təβ-lə-jaːn} \textbf{eː-jaːn} əmə-dgi-ja-βun ɲan siksə\\
that like.this berry-\textsc{vblz}-\textsc{cvb} what-\textsc{cvb} come-\textsc{rep}-\textsc{nfut}-\textsc{1pl.excl} also in.the.evening\\
\glt `We \uline{picked} \textbf{and did something}, in the evening we came back' \\
 \z
\il{Negidal}

In~Even, \textit{ịa} `what' is an interrogative which is also used as a general extender. In~the Even example ((\ref{ex.ge.evn.noun1})), the `what'-stem refers to the various ways of ornamenting bedclothes. Interestingly, it does not copy the participial \textit{-tị} but copies the nominal proprietive affix (\textit{-lkan}). In \REF{ex.ge.evn.verb}, the verb with the `what'-stem refers to the actions connected with taking care of a reindeer. Interestingly, in \REF{ex.ge.evn.verb}, the proprietive suffix is mirrored but not the voice marker (see the discussion of voice marker mirroring in examples \REF{ex:klyachko:1.1} and \REF{ex:klyachko:1.2}).

\is{general extender} 
 \ea \label{ex.ge.evn.noun1}
 \langinfo{Even}{}{\cite[Glove and love]{nikolaevagarret}}\\
\gll ere-k teːden=de butun-ni [ọńaː] \uline{ọńaː-tị-lkan} \textbf{ịa-lkan}\\
this-\textsc{nmlz} bedclothes=\textsc{foc} all.SAH-\textsc{poss.3sg} paint paint-\textsc{ptcp}-\textsc{propr} what--\textsc{propr}\\
\glt `And the bed was all decorated \uline{with patterns} \textbf{and stuff}' \\
 \z
\il{Even}

\is{general extender} 
 \ea \label{ex.ge.evn.verb}
 \langinfo{Even}{}{\cite[Top09\_GNM\_4\_1.562]{EvenCorpus}}\\
 \gll nọŋman \uline{ọŋk-ụ-t-nịkan} \textbf{iạ-nikạn}\\
 \textsc{3sg.acc} graze-\textsc{tr}-\textsc{res}-\textsc{cvb} what-\textsc{cvb}\\
\glt `I \uline{feed} him \textbf{et cetera}.' \\
 \z
\il{Even} 

General extender constructions can also be found in Udihe. They make use of the interrogative \textit{je} `what'. In~(\ref{ex.ge.ude.noun}), the general extender refers to the products of hunting like meat, blood or marrow. In~(\ref{ex.ge.ude.verb}), the `what'-stem verb denotes the actions which are usually associated with taking care of a baby, like rocking it. Unlike the previous examples from this section, where the `what'-word directly succeeds the enumeration, in~(\ref{ex.ge.ude.verb}), there is a coordinating construction:
``rocked — crying, (general extender) — crying", which also resembles \REF{ex:klyachko:1.2}. The repeating of \textit{soŋoi} resembles ``recycling'', or repeating material, described by \citet{Podlesskaya2010} for placeholders.

\is{general extender} 
 \ea \label{ex.ge.ude.noun}
 \langinfo{Udihe}{}{\cite[Zabdala, an extraordinary snake]{nikolaevagarret}}\\
\gll uta-zi wa-i diga-i uta-zi diga-Ø-li \uline{ule:-we} \textbf{je-we} diga-i\\
 that-\textsc{ins} kill-\textsc{ptcp} eat-\textsc{ptcp} that-\textsc{ins} eat-\textsc{3sg} meat-\textsc{acc} what-\textsc{acc} eat-\textsc{ptcp}\\
\glt `Then they would eat what it had killed, \uline{the meat} \textbf{and all the rest}.' \\
 \z
\il{Udihe}


\is{general extender} 
 \ea \label{ex.ge.ude.verb}
 \langinfo{Udihe}{}{\cite[Zabdala, an extraordinary snake]{nikolaevagarret}}\\
 \gll \uline{emu-si:-Ø-ni} soŋo-i \textbf{i:-si-Ø-ni=de} soŋo-i\\
rock-\textsc{ipfv}-\textsc{nfut}-\textsc{3sg} cry-\textsc{ptcp} what-\textsc{ipfv}-\textsc{nfut}-\textsc{3sg} cry-\textsc{ptcp}\\
\glt `The old man \uline{rocked} him \textbf{and did everything}, but the child was still crying.' \\
 \z
\il{Udihe}

Due to the small amount of available Oroch data, I could only find a nominal example (\ref{ex.ge.oroch.noun}), where the `what'-stem refers to the kinds of traces one could see. The focus particle copying resembles the Negidal example (\ref{ex.ge.neg.noun}), where once again \textit{=dV} can be interpreted as a coordinating particle or as a negative polarity particle. Notice that in~(\ref{ex.ge.oroch.noun}), the hesitative \textit{aŋi} is used too, contrasting with \textit{jeu}. The former is used to fill in the pause, whereas the latter extends the enumeration started with the word \textit{xokto} `footprint'.

\is{general extender} 
 \ea \label{ex.ge.oroch.noun}
 \langinfo{Oroch}{}{\cite[text 10]{kazama1996}}\\
 \glll \uline{хокто=даа=мааки} \textbf{jэу=дээ} \textbf{аӈи}\\
 \uline{xokto=daa=maaki} \textbf{jeu=dee} \textbf{aŋi}\\
way=\textsc{foc}=\textsc{ptcl} what=\textsc{foc} \textsc{hes}\\
\glt `(There are) \uline{no footprints} \textbf{or anything} \textbf{<, well>}'. \\
 \z
\il{Oroch}

In~Uilta, the corresponding interrogative pronoun is \textit{xai} `what'. In~(\ref{ex.ge.uilta.noun}), the `what'-wordform refers to various kinds of food. Notice the structural parallelism of this example with (\ref{ex.ge.ude.verb}): in~(\ref{ex.ge.uilta.noun}), the verb \textit{dəp-tʃi-pu} `we eat' repeats after the general extender.

\is{general extender} 
 \ea \label{ex.ge.uilta.noun}
 \langinfo{Uilta}{}{\cite{UiltaCorpusGusevToldova}}\\
 \gll \uline{musi-l-ba} dəp-tʃi-pu \textbf{xai-l-ba} dəp-tʃi-pu un-ʒi-ni\\
 musi-\textsc{pl}-\textsc{acc} eat-\textsc{pst}-\textsc{1pl.excl} what-\textsc{pl}-\textsc{acc} eat-\textsc{pst}-\textsc{1pl.excl} say-\textsc{prs}-\textsc{3sg}\\
\glt `We ate \uline{musi} (kind of jelly made from skin fish), we ate \textbf{other things}, — he said.' \\
 \z
\il{Uilta}


In~(\ref{ex.ge.uilta.noun2}), the general extender refers to various parts of the object, mirroring the morphology on \textit{do:-} `inner part' . In~ (\ref{ex.ge.uilta.verb}), \textit{xai}- refers to the actions which a grown-up woman traditionally performed, such as sewing.

\is{general extender} 
 \ea \label{ex.ge.uilta.noun2}
 \langinfo{Uilta}{}{\cite{UiltaCorpusGusevToldova}}\\
 \gll təto:-ni \uline{do:-kke:-ni} \textbf{xai-kke:-ni=da} tʃipal bara: bara: i:-xə-tʃi\\
clothes-\textsc{3sg} inner.side-\textsc{prol}-\textsc{poss.3sg} what-\textsc{prol}-\textsc{poss.3sg}=\textsc{foc} whole very.much very.much enter-\textsc{pst}-\textsc{3pl}\\
\glt `They entered into her clothes and all over <lit.: \uline{into the inner part} \textbf{and other parts} of her clothes>' \\
 \z
\il{Uilta} 


\is{general extender} 
 \ea \label{ex.ge.uilta.verb}
 \langinfo{Uilta}{}{\cite{UiltaCorpusGusevToldova}}\\
\gll uže nari o-tči-ni \uline{ulp-i-ni} \textbf{xaj-ri-ni}\\
already.R person become-\textsc{pst}-\textsc{3sg} \uline{sew-\textsc{prs}-\textsc{3sg}} \textbf{what-\textsc{prs}-\textsc{3sg}}\\
\glt `She already grew up, \uline{sewed} \textbf{and did other things}.' \\
 \z
\il{Uilta} 


In Nanai, the cognate \textit{xaj} stem is used. In~(\ref{ex.ge.nanai.noun}), the general extender refers to the equipment one wears when going to the forest, such as skis. In~(\ref{ex.ge.nanai.verb}), the general extender refers to the set of actions a young man was traditionally thought to be able to do, such as shooting.


\is{general extender} 
 \ea \label{ex.ge.nanai.noun}
 \langinfo{Nanai}{}{\cite[36--37]{beldy2012}}\\
\glll эди-и пулэ сокта-ва-ни эди пулэ хай-ва-ни тэтугу-хэ-ни таваӈки дуйси то-ха-ни\\
edi-i pule \uline{sokta-wa-ni} edi pule \textbf{xaj-wa-ni} tetugu-xe-ni tawaŋki dujsi to-xa-ni\\
husband-\textsc{poss.rfl} extra ski-\textsc{acc}-\textsc{poss.3sg} husband extra \textbf{what-\textsc{acc}-\textsc{poss.3sg}} put.on-\textsc{pst}-\textsc{3sg} afterwards to.forest go-\textsc{pst}-\textsc{3sg}\\
\glt `She put on her husband's extra \uline{skis}, her husband's extra \textbf{stuff}, and she went to the forest ' \\
 \z
\il{Nanai}


\is{general extender} 
 \ea \label{ex.ge.nanai.verb}
 \langinfo{Nanai}{}{\cite[text 7]{kazama1993})}\\
\glll эӈ пиктэ-н(и)=тэнии эси=тэнии лэкээ-чи=й хай-рӥ=ӥ та-ло-ха-нӥ\\
eŋ pikte-n(i)=tenii esi=tenii \uline{lekee-tʃi-i} \textbf{xaj-ri=i} ta-lo-xa-ni\\
\textsc{intj} child-\textsc{poss.3sg}=\textsc{ptcl} now=\textsc{ptcl} arrow-\textsc{vblz}-\textsc{ptcp} what-\textsc{ptcp}=\textsc{ptcl} do-\textsc{inch}-\textsc{pst}-\textsc{3sg}\\
\glt `The child started \uline{to shoot} \textbf{and do other things like that}.' \\
 \z
\il{Nanai}

Unfortunately, due to the small amount of Ulch data I could not find a clear general extender example from Ulch, apart from (\ref{ex.ge.ulcha.noun}), which I analyze in more detail below.

The general extenders discussed in this section exhibit mirroring. They usually mirror inflectional morphology, often including verbal voice, as in \REF{ex:klyachko:1.2} (but not in \REF{ex.ge.evn.verb}). They sometimes mirror derivational morphology as well, as in (\ref{ex.ge.ude.adj}). However, we can see that full mirroring does not always take place, for example, in (\ref{ex.ge.evn.verb}), where the general extender refers to a set of actions akin to feeding. Naturally, these actions do not necessarily follow the causative model of `graze' > `feed'. This is why the transitive marker used in the `feed' wordform is not copied. In (\ref{ex.ge.ulcha.noun}), the general extender has the accusative marker lacking in the first member of the enumeration, which raises a question of how direct object marking works in case of general extenders. Unfortunately, at the time, there is too little data to study it.

Still, I suppose that general extender mirroring is generally fuller than placeholder mirroring. However, this should be proved using larger corpora for each language.

\subsection{Borrowed general extenders}

Borrowed general extenders seem to be harder to spot in the corpora. In~Russian, there are multiword expressions like \textit{i tak dalee} (`and so further') used as general extenders but they are rare in the Tungusic oral texts. In~the Sakha language dictionary (\cite{sakhatyla}), \textit{эҥин} (\textit{eŋin}), an adjective meaning `different', is also defined as a ``particle generalizing similar objects or actions appended to the main object or action". In this latter meaning, it is often written as \textit{игин} (\textit{igin}). This word can actually mirror the morphological features of the elements of the list as in~(\ref{ex.ge.sah.igin}).\footnote{This example was provided by S. Makarov.}:

\is{general extender} 
 \ea \label{ex.ge.sah.igin}
 \langinfo{Sakha}{}{S. Makarov, p. c.}\\
\glll \uline{оттуу-бут} \uline{сир} \uline{астыы-быт} \textbf{игин-нии-бит}\\
\uline{ottuu-but} \uline{sir} \uline{astɨɨ-bɨt} \textbf{igin-nii-bit}\\
mow-\textsc{prs}.\textsc{1pl} earth obtain-\textsc{prs}.\textsc{1pl} etc.SAH-\textsc{vblz}-\textsc{prs}.\textsc{1pl}\\
\glt `We \uline{mow grass, gather berries <lit. obtain earth food>} \textbf{etc}.' \\
 \z
\il{Sakha}

\textit{igin} occurs several times in the Even language corpus, as in~(\ref{ex.ge.even.igin}):

\is{general extender} 
 \ea \label{ex.ge.even.igin}
 \langinfo{Even}{}{\citealt[Krivoshapkin\_SP\_oxota 019]{EvenCorpus}}\\
\gll ịbga ŋịna-lkan \textbf{igin} bi-wre-n ta-la\\
good dog-\textsc{propr} etc.SAH be-\textsc{hab}-\textsc{3sg} that-\textsc{loc}\\
\glt `He used to have \uline{a good dog} \textbf{<etc>} there' \\
 \z
\il{Even}

Just like the borrowed placeholders, the borrowed general extender in Even does not mirror morphological features.


\subsection{Tungusic general extender functions}

The majority of the examples show that the main function of general extender is to refer to an open-ended list of objects and activities, which the speaker does not specify because they are a part of the background knowledge, as shown with the examples in the preceding section.

However, general extenders have a number of other functions, too. One of these ``extra" functions is the approximative one. In~(\ref{ex.ge.evn.noun}), \textit{ịa-} has an approximative function, because the speaker is not sure about the number of the days or perhaps it is actually not a strict rule of the rite. In~(\ref{ex.ge.ude.adj}), the function of \textit{xai} is also approximative because the speaker is not sure if the oriole's feathers can be called a ``crest" in Udihe and is looking for a right word.


\is{general extender} 
 \ea \label{ex.ge.evn.noun}
 \langinfo{Even}{}{\cite[Spirits]{nikolaevagarret}}\\
 \gll goli-ŋe-ndi \uline{ujun} \uline{dọlbanị-dụ-n=gụ} \textbf{ịa-dụ-n=gụ}\\
put.animal.bones.on.platform-\textsc{imper}-\textsc{2sg} nine night-\textsc{dat}-\textsc{3sg.poss}=\textsc{inter} what-\textsc{dat}-\textsc{3sg.poss}=\textsc{inter}\\
\glt `You should put its bones on the high platform, \uline{for nine nights} \textbf{or thereabouts}' \\
 \z
\il{Even}


\is{general extender} 
 \ea \label{ex.ge.ude.adj}
 \langinfo{Udihe}{}{\cite[The oriole grandson]{nikolaevagarret}}\\
 \gll küxoxi=tene läsi uligdiga činda xai dili-le-ni \uline{ogo-xi} \textbf{iː-xi} daktä-la-ni=de xai xutaligi-zi abdu bede iŋme-de-se tuː küxoxi mani-ni\\
oriole=\textsc{contr} very beautiful bird also head-\textsc{loc}-\textsc{poss.3sg} crest-\textsc{adj} what-\textsc{adj} wing-\textsc{loc}-\textsc{poss.3sg}=\textsc{contr} also red-\textsc{ins} bead like needle-\textsc{vblz}-\textsc{pass} all oriole flock-\textsc{poss.3sg}\\
\glt `Orioles are very beautiful birds: they have some kind of little crest on the head <lit. it is \uline{with a crest} or \textbf{with something of the kind}> and their wings look like they are embroidered with little red beads. That's a flock of orioles.' \\
 \z
\il{Udihe}

The question is whether these approximative usages of the `what'-stem are examples of placeholder use, or general extender use, or just of interrogative pronouns used as indefinites. I argue that the approximative constructions are derived from the general extender construction. Firstly, they are different from the normal placeholders formally. For example, in Even, the \textit{=kanan} particle must accompany the `what'-placeholder. In Udihe, the ordinary placeholder is \textit{aŋi} and not \textit{je}. Secondly, developing approximative function is common for general extenders (cf. \citealt{kim2020korean}). Finally, according to the grammars of Even \citep[12]{malchukov1995even}, Negidal \citep[22]{cincius1982} or Udihe \citep[353]{nikolaeva2001grammar}, interrogative pronouns are not generally used as indefinites without any other markers, requiring a particle like \textit{=dV}.

Another construction, which is in my opinion connected with general extenders is the alternative question construction, analyzed by \citet{tolskaya2008question}. In this specific case, the speaker uses `what' to denote an alternative to the main action.

\is{general extender} 
 \ea \label{ex.ge.ude.question}
 \langinfo{Udihe}{}{ex:klyachko:~$13$~from \citet{tolskaya2008question}}\\
\gll \uline{guli-zeŋe-fi=es} \textbf{ja-zaŋa-fi=es}\\
leave-\textsc{fut}-\textsc{1pl.incl}=\textsc{dis} what-\textsc{fut}-\textsc{1pl.incl}=\textsc{dis}\\
\glt `Shall we \uline{leave} \textbf{or what}?' \\
 \z
\il{Udihe}

\citet{overstreet2020english} calls English phrases like ``or something" disjunctive general extenders, because they refer to a set of alternatives. Although she does not specifically cover alternative questions, I think that such questions are actually close to non-interrogative usages like~(\ref{ex.ge.ulcha.noun}), where \textit{xaj-} stands for an alternative.

\is{general extender} 
 \ea \label{ex.ge.ulcha.noun}
 \langinfo{Ulch}{}{\cite{ulchcorpus}}\\
\gll tịmana=gdal ụm du:sə \uline{gịwụ=nụ} \textbf{xaj-wa=nụ} wa:-ra gaǯụ-xa-nị\\
tomorrow=\textsc{ptcl} one tiger roe=\textsc{inter} what-\textsc{acc}=\textsc{inter} kill-\textsc{cvb} bring-\textsc{pst}-\textsc{3sg}\\
\glt `In the morning, one tiger brought either \uline{a roe} \textbf{or something}.' \\
 \z
\il{Ulch}

To sum up, in my opinion, there are three types of situations, which can all be classified as general extender constructions. In the first one, general extenders are used in the classical way at the end of an enumeration to refer to a non-exhaustive list of objects or actions. In the second one (``approximatives'') they refer to a non-exhaustive list of situations (like ``nine days or something'', which can actually be unfolded as a list `nine days or eight days or ten days...'). In the third one (``alternative questions'') they refer to an non-exhaustive list of alternatives to the main situation. All these cases use the same `what'-pronouns, which contrasts with specialized placeholder stems in Evenki, Negidal, Even, Udihe, Oroch, and Uilta.

\section{Comparing placeholders and general extenders}

Tungusic placeholders and general extenders have somewhat similar functions, being used as ``vague language'', when the speaker cannot or does not want to specify the exact target. They sometimes have the same stems and demonstrate mirroring behaviour. However, there are also differences between them.

Firstly, Tungusic general extenders are uniformly derived from the `what'-stem. On the contrary, placeholders have various stems in various Tungusic languages, with their etymology often not clear, sometimes suggesting borrowings. Secondly, their mirroring properties are different: general extenders tend to perform full mirroring, including the derivational morphology, whereas for placeholders, the mirroring is often partial.
What is the reason for these differences? I think that the main explanation is just the fact that in case of general extenders, the first member(s) of an enumeration have already been pronounced. Therefore, copying is more mechanical. Furthermore, in case of placeholders, the target that is realized may be not the word the speaker was originally trying to recall.

As regards the stem uniformity vs. diversity, the explanation can be the fact that the use of `what'-stems for placeholders is influenced by the general extender use with `what'-stems drifting to the placeholder area and becoming the default vague language markers. This is opposite to the Udi and Agul case discussed in \citet{ganenkov2010interrogatives}. The sporadic examples from Negidal (\ref{ex.ekun.neg.ph}) and Udihe (\ref{ex.what.ude.ph}) show that `what'-stems can actually be used as placeholders. However, due to the moribund status of both languages and the scarcity of the language data, we cannot see the possible development of the process. An anonymous reviewer suggested another possible explanation, i. e. an independent development of general extenders and placeholders from the interrogative pronouns, which are also known to function as indefinites in the Tungusic languages. However, I think that this is less probable because in many Tungusic languages the indefinite pronouns are not just bare interrogative stems. On the contrary, they usually require additional particles, lacking in the placeholder or general extender examples shown above. Another possibility is an independent development of placeholders and general extenders from an interrogative stem but the aforementioned sporadic examples are in favour of the first hypothesis.

Was there a dedicated Tungusic placeholder stem? Word forms like \textit{aŋi-} and \textit{uŋun-} are good candidates since they can be found in several Tungusic languages. On the other hand, the data analysis shows that placeholders are quite often being borrowed. Therefore, we cannot exclude the fact that, as shown above, \textit{aŋi-} was simply borrowed from Nivkh into neighbouring Tungusic languages in the Amur area.

\section{Conclusions}

The paper shows that Tungusic placeholders and general extenders may be thought of as two very similar classes of discourse expressions, both dealing with uncertainty. However, they cannot be considered a continuous class. Firstly, placeholders exhibit much more diversity regarding the stems whereas general extenders are uniform. One possible explanation for the fact is that the original general extenders became placeholders in Nanai, Ulch and Uilta as well as some Even and Evenki dialects. The same process may be seen in several Udihe and Negidal examples but the languages are unfortunately moribund, with no grammaticalization (or pragmaticalization) process going on actively. Another peculiarity is the fact that placeholders seem to be much more prone to borrowing than general extenders. Specifically, there are borrowed Russian placeholders and Sakha particles in Evenki and Even. Moreover, \textit{aŋi}, a placeholder stem with no good etymology, may be a Nivkh borrowing. Placeholder borrowing in modern texts can be explained by the fact that in case of attrition placeholders help maintain fluency, whereas code-switching to Russian and Sakha is also frequent in case of attrition. 

Another difference between placeholders and general extenders is their mirroring behaviour. General extenders mirror more features, including e. g. verbal voice. An obvious explanation is that in case of general extenders, the speaker has already produced the wordforms in the enumeration, so it is easier for them to copy the affixes. As regards the placeholders, the speakers find themselves in a more difficult situation with the morphological features of the target not yet produced. The left-branching nature of the Tungusic languages must be crucial for placeholders: as \citet{ganenkov2010interrogatives} put it, ``in left-branching languages syntactic dependents can appear before the processing difficulties in the head’s nomination occur''. Still, partial mirroring means that the markers, which are not determined by the syntactic dependencies only, may be difficult to retrieve. The mirroring behaviour can thus give a clue to the affix retrieval process.


\section*{Abbreviations}
\begin{multicols}{2}
\begin{tabbing}
MMMMM \= ablative\kill
\textsc{abl} \> ablative\\
\textsc{acc} \> accusative\\
\textsc{adj} \> adjectivizer\\
\textsc{advz} \> adverbializer\\
\textsc{alien} \> alienable possession\\
\textsc{all} \> allative\\
\textsc{and} \> andative\\
\textsc{ant} \> anteriority\\
\textsc{car} \> caritive\\
\textsc{caus} \> causative\\
\textsc{com} \> comitative\\
\textsc{contr} \> contrast\\
\textsc{cvb} \> converb\\
\textsc{dat} \> dative\\
\textsc{desig} \> designative\\
\textsc{dis} \> disjunctive particle\\
\textsc{dur} \> durative\\
\textsc{eqt} \> equative\\
\textsc{excl} \> exclusive\\
\textsc{foc} \> focus particle\\
\textsc{fut} \> future tense\\
\textsc{gen} \> genitive\\
\textsc{hab} \> habitual\\
\textsc{hes} \> hesitative\\
\textsc{imper} \> imperative\\
\textsc{inch} \> inchoative\\
\textsc{incl} \> inclusive\\
\textsc{indef} \> indefinite\\
\textsc{ins} \> instrumental\\
\textsc{inter} \> interrogative particle\\
\textsc{intj} \> interjection\\
\textsc{ints} \> intensifier\\
\textsc{ipfv} \> imperfective\\
\textsc{lim} \> limitative\\
\textsc{loc} \> locative\\
\textsc{neg} \> negation\\
\textsc{nfut} \> non-future tense\\
\textsc{nmlz} \> nominalizer\\
\textsc{obl} \> oblique stem\\
\textsc{onom} \> onomatopoetic expression\\
\textsc{pass} \> passive\\
\textsc{perf} \> perfective\\
\textsc{ph1, ph2} \> placeholder stems\\
\textsc{pl} \> plural\\
\textsc{poss} \> possessive\\
\textsc{prol} \> prolative\\
\textsc{propr} \> proprietive\\
\textsc{prs} \> present tense\\
\textsc{pst} \> past tense\\
\textsc{ptcl} \> particle\\
\textsc{ptcp} \> participle\\
\textsc{rep} \> repetitive\\
\textsc{res} \> resultative\\
\textsc{rfl} \> reflexive\\
\textsc{sg} \> singular\\
\textsc{tr} \> transitive\\
\textsc{vblz} \> verbalizer\\
\textsc{R} \> Russian\\
\textsc{SAH} \> Sakha
\end{tabbing}
\end{multicols}

% % % \section*{Acknowledgements}

\printbibliography[heading=subbibliography,notkeyword=this]
\end{document}
