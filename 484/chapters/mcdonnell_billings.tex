\documentclass[output=paper,
\ChapterDOI{10.5281/zenodo.15697583}
colorlinks,
citecolor=brown]{langscibook}
\author{Bradley McDonnell\orcid{}\affiliation{University of Hawai‘i at Mānoa} and
Blaine Billings\orcid{}\affiliation{University of Hawai‘i at Mānoa}}
\title{Choosing fillers in Besemah}

\abstract{Studies of fillers have shown that while some languages, such as English, express (interjective) hesitator and placeholder functions by different means, other languages, such as Chinese, Japanese, and Korean, express these functions with the same word (or class of words). Against this backdrop, the Besemah dialect of South Barisan Malay presents an interesting case study because, like Chinese, Japanese, and Korean, Besemah does not make a formal distinction between hesitator and placeholder functions, meaning one and the same form may serve both functions. However, Besemah differs from these languages in that, despite having a dedicated filler, \textit{anu}, Besemah speakers make extensive use of the proximal demonstrative pronoun \textit{ini} -- and to a lesser extent the distal demonstrative pronoun \textit{itu} -- with the same range of functions as \textit{anu}. In analyzing these fillers in Besemah, we question some of the defining characteristics of fillers in general, including the strict separation of hesitator and placeholder functions and the degree to which repair is characteristic of placeholders.

\keywords{South Barisan Malay, hesitator, placeholder, repair, demonstrative}
}

% \usepackage{tabularx}
% \usepackage{langsci-optional}
% \usepackage{langsci-gb4e}
\usepackage{multirow}
% \bibliography{localbibliography}
% \newcommand{\orcid}[1]{}

\IfFileExists{../localcommands.tex}{
   \addbibresource{../localbibliography.bib}
   \usepackage{langsci-optional}
\usepackage{langsci-gb4e}
\usepackage{langsci-lgr}

\usepackage{listings}
\lstset{basicstyle=\ttfamily,tabsize=2,breaklines=true}

%added by author
% \usepackage{tipa}
\usepackage{multirow}
\graphicspath{{figures/}}
\usepackage{langsci-branding}

   
\newcommand{\sent}{\enumsentence}
\newcommand{\sents}{\eenumsentence}
\let\citeasnoun\citet

\renewcommand{\lsCoverTitleFont}[1]{\sffamily\addfontfeatures{Scale=MatchUppercase}\fontsize{44pt}{16mm}\selectfont #1}
  
   %% hyphenation points for line breaks
%% Normally, automatic hyphenation in LaTeX is very good
%% If a word is mis-hyphenated, add it to this file
%%
%% add information to TeX file before \begin{document} with:
%% %% hyphenation points for line breaks
%% Normally, automatic hyphenation in LaTeX is very good
%% If a word is mis-hyphenated, add it to this file
%%
%% add information to TeX file before \begin{document} with:
%% %% hyphenation points for line breaks
%% Normally, automatic hyphenation in LaTeX is very good
%% If a word is mis-hyphenated, add it to this file
%%
%% add information to TeX file before \begin{document} with:
%% \include{localhyphenation}
\hyphenation{
affri-ca-te
affri-ca-tes
an-no-tated
com-ple-ments
com-po-si-tio-na-li-ty
non-com-po-si-tio-na-li-ty
Gon-zá-lez
out-side
Ri-chárd
se-man-tics
STREU-SLE
Tie-de-mann
}
\hyphenation{
affri-ca-te
affri-ca-tes
an-no-tated
com-ple-ments
com-po-si-tio-na-li-ty
non-com-po-si-tio-na-li-ty
Gon-zá-lez
out-side
Ri-chárd
se-man-tics
STREU-SLE
Tie-de-mann
}
\hyphenation{
affri-ca-te
affri-ca-tes
an-no-tated
com-ple-ments
com-po-si-tio-na-li-ty
non-com-po-si-tio-na-li-ty
Gon-zá-lez
out-side
Ri-chárd
se-man-tics
STREU-SLE
Tie-de-mann
}
   \boolfalse{bookcompile}
   \togglepaper[]%%chapternumber
}{}

\begin{document}
\maketitle

\section{Introduction}\label{sec:intro}
In their seminal paper on fillers, \citet{hayashi2006crosslinguistic} provide defining characteristics of two different types of fillers: placeholders and (interjective) hesitators. They point out that, in languages like English, there are separate means of fulfilling these two different roles (i.e., the lexicalized phrase placeholder \textit{whatchamacallit} and hesitators \textit{uh} or \textit{uhm}). In other languages like Korean, Japanese, or Chinese, these roles are fulfilled by a single lexical item or a class of lexical items, namely demonstratives. Fillers of both types are defined as conventionalized forms, which deal with constraints in cognitive processes that are associated with trouble in word formulation as a means of delaying the next word or phrase \citep[see][]{hayashi2003joint,hayashi2006crosslinguistic,fox2010introduction}. They can be used in similar interactional contexts, such as word searches, different types of repair sequences, and holding or ceding the floor \citep{clark2002using}. Despite their use in similar contexts with similar purposes, \citet{hayashi2006crosslinguistic} define clear differences between the two types of demonstrative fillers, as summarized in \tabref{tab:filler}.

\begin{table}
\caption{Defining characteristics of placeholder and hesitator according to \citet{hayashi2006crosslinguistic}.}
\label{tab:filler}
\begin{tabularx}{1\textwidth}{X X}
\lsptoprule
Placeholder  & Hesitator\\
\midrule
``\ldots participates in the morpho-syntactic structure of an unfolding utterance.''
& ``\ldots are not produced as a syntactic constituent occupying a specific syntactic slot in an ongoing utterance.''\\[3em]

``\ldots is used as a referential expression.''
& ``\ldots used non-referentially, without making any referential contribution.''\\[2em]

``\ldots is subsequently replaced by a more specific lexical item that has become available to the speaker (and/or hearer(s)) as a result of word search.''
& ``\ldots little syntactic or semantic correspondence between the morphological forms \ldots and the `outcome' of word search.''\\
\lspbottomrule
\end{tabularx}
\end{table}

Essentially, \citet{hayashi2006crosslinguistic} distinguish placeholders from hesitators based on their referentiality (i.e., placeholders are referential, while hesitators are non-referential), syntactic integration in the ongoing utterance (i.e., placeholders are syntactically integrated, while hesitators are not), and repair of the filler (i.e., placeholders are ``replaced'' by a more specific word, whereas the hesitator and the ``outcome'' have little correspondence).  

Against this backdrop, Besemah, a dialect of South Barisan Malay, provides an interesting case study because like Chinese, Japanese, and Korean, Besemah does not make a formal distinction between hesitator and placeholder functions, meaning one and the same form may serve both functions. However, fillers in Besemah differ from these languages in that, despite having a dedicated, identifiable filler, \textit{anu}, Besemah speakers make extensive use of the proximal demonstrative pronoun \textit{ini} -- and to a lesser extent the distal demonstrative pronoun \textit{itu} -- with the same range of functions as the filler \textit{anu}. As this paper demonstrates, these fillers also share many of the same morphosyntactic properties, distributions, and frequencies.

In describing Besemah fillers, we problematize the notion that there are categorical differences between placeholders and hesitators as defined by \citet{hayashi2006crosslinguistic}. First, while clear examples of both functions can be identified in the corpus, in a number of cases the syntactic integration of the filler -- a distinguishing property of placeholders -- is indeterminate. For both analysts and participants, it is often impossible to decide whether a filler is employed as a hesitator or placeholder. Second, we find that the nature of repair is similarly not clear-cut. In Besemah, apparent placeholders are commonly not replaced, and even when they are replaced by a more specific lexical item, the two do not necessarily share the same morphology. Drawing on a sizeable documentary corpus of everyday conversations, we demonstrate how Besemah complicates the typology of fillers in the world's languages.

This chapter is organized as follows. \sectref{sec:besemah} introduces Besemah, the basic properties of its morphosyntax, and the corpus used in this study. \sectref{sec:fillers} discusses the etymologies of both the dedicated filler \textit{anu} in Besemah and the demonstrative pronouns as fillers before describing how speakers utilize them in interaction. It is further demonstrated how these fillers are used variably as placeholders and hesitators and, in a number of cases, how their role remains indeterminate. The section concludes with a discussion of the frequency counts of different morphological configurations of the fillers as found in the corpus. \sectref{sec:quantitative} presents a quantitative analysis of Besemah fillers in a subset of the corpus, describing the frequency of the different fillers (i.e., \textit{anu} versus demonstrative pronouns), their morphosyntactic properties, uses as hesitator or placeholder, properties related to disfluencies and repair, and distributions across speakers. \sectref{sec:conclusion} concludes the chapter.

\section{Besemah}\label{sec:besemah}
Besemah is a Malayic language spoken by approximately 330,000 people in the highlands of southwest Sumatra. It is considered a variety of South Barisan Malay, which comprises a complex network of 11 other named dialects spoken by an estimated 1.5 million people. Despite the relatively large number of speakers, South Barisan Malay remains under-described \citep[see][]{mcdonnell2016symmetrical}. Due to their diversity along several dimensions, Malayic languages have been categorized along sociolinguistic lines as vernacular Malay (i.e., varieties regularly inherited from Proto Malayic), literary Malay (i.e., varieties arising from the literary Malay tradition), and vehicular Malay (i.e., `trade' varieties that came about as a result of contact; \cite[for an overview of the complexities of these issues see][]{adelaar2005structural,mcdonnell2024malayic}). Unlike some well-known varieties of Malay, such as Standard Indonesian and Baba Malay, which fall into literary Malay and vehicular Malay categories, respectively, South Barisan Malay is considered a vernacular Malay variety, since these dialects appear to have been regularly inherited from Proto Malayic.

In the remainder of this section, we provide a brief overview of the basic structural properties of Besemah as a foundation for understanding the complexities of fillers in the language. We then follow in \sectref{sec:fillers} with a description of filler use, including competing forms of the fillers and their hesitator and placeholder functions, the use of fillers in trouble in word formulation, the morphological possibilities of fillers, and the indeterminacy of hesitator and placeholder functions. The examples and the data on which the analysis is based in this section come from a corpus of everyday conversations in Besemah that was compiled by the first author. The corpus contains 27 conversations, each involving two to five Besemah speakers. Sessions are each approximately one hour in length and have been transcribed using a simplified version of Discourse Transcription \citep{dubois1993outline}.\footnote{Transcription conventions are as follows: Segments are chunked into Intonation Units (IU) represented by a line break. End marks represent a `final' contour (`.'), a `continuing' contour (`,'), or an `appeal' contour (`?'). Dashes represent truncation: en dash (`--') for a word and em dash (`---') for an IU. Square brackets represent overlapping speech. Colons represent lengthening. Short pauses are represented by `..', while timed long pauses are represented as seconds within parentheses.} The recordings are archived with the Pacific and Regional Archive for Digital Sources in Endangered Cultures (PARADISEC; \cite{mcdonnell2008besemah}). The conversational corpus contains just over 250,000 words and just under 95,000 Intonation Units (IU).\footnote{The following sessions are included: BJM01-001, BJM01-002, BJM01-004, BJM01-008, BJM01-010, BJM01-011, BJM01-015, BJM01-028, BJM01-029, BJM01-032, BJM01-033, BJM01-035, BJM01-041, BJM01-043, BJM01-071, BJM01-073, BJM01-081, BJM01-086, BJM01-088, BJM01-093, BJM01-098, BJM01-116, BJM01-125, BJM01-129, BJM01-130, BJM01-131, BJM01-158.} The quantitative analysis in \sectref{sec:quantitative} draws on a subset of this corpus.

\subsection{Basic morphosyntactic properties}\label{sec:basic-morphosyntax}
Besemah is a mildly synthetic language, with a fairly robust set of nominal and verbal prefixes/proclitics and suffixes/enclitics in addition to a few circumfixes. The most common affixes include verbal prefixes that express voice and verbal suffixes that represent polyfunctional, causative/applicative markers. Nominal affixes derive nouns with various meanings (e.g., locative, objective). There are also a number of pronominal clitics that serve both as verbal arguments and possessors (see below). \tabref{tab:morphology} provides a brief overview of some of the more productive morphemes that occur on fillers, excluding morphology that marks transitive verbal predicates (presented later in this section). These affixes readily combine with fillers, as discussed further in \sectref{sec:fillers}. For a full listing of Besemah morphology, see \citet[][36--45]{mcdonnell2016symmetrical}.

\begin{table}
\caption{Several productive affixes in Besemah.}
\label{tab:morphology}
\begin{tabularx}{1\textwidth}{L{1cm} X}
\lsptoprule
 {Affix}         &  {Meanings with examples}\\
\midrule
\textit{be-} 
& Middle voice (e.g., \textit{jalan} `path' $\rightarrow$ \textit{bejalan} `walk') that also often derives denominal intransitive verbs with meaning of `have \textsc{noun}' (e.g., \textit{beghas} `unhusked rice' $\rightarrow$ \textit{bebeghas} `have unhusked rice') or `use \textsc{noun}' (e.g., \textit{pisau} `knife' $\rightarrow$ \textit{bepisau} `use a knife') with various other semantics associated with the root (e.g., \textit{kerite} `bike' $\rightarrow$ \textit{be-kerite} `ride a bike') \\
\textit{te-}
& Non-volitional voice (e.g., \textit{teghingat} `remember') that also derives intransitive verbs with accidental meaning (e.g., \textit{beli} `buy' $\rightarrow$ \textit{tebeli} `inadvertently buy')\\
\textit{-an}
& Objective nominalizer commonly derives nouns with meanings of the resulting action of a root (e.g., \textit{pecah} `break' $\rightarrow$ \textit{pecahan} `a broken piece') or the patient associated with the the root (e.g., \textit{buat} `make' $\rightarrow$ \textit{buatan} `something made'), although it also may result in the location where the root is carried out (e.g., \textit{mandi} `bathe' $\rightarrow$ \textit{mandian} `bathing place') and other conventionalized meanings (e.g., \textit{manis} `sweet' $\rightarrow$ \textit{manisan} `desserts').\\
\textit{se-}
& Numeral prefix meaning `one' (e.g., \textit{se-mubil} `one car') may also be used to mean `same \textsc{noun}' (e.g., \textit{se-mubil} `(in) the same car')\\
\lspbottomrule
\end{tabularx}
\end{table}

Nominal arguments and many intransitive predicates, however, need not occur with any additional morphology, as in the examples in (\ref{ex:intr}). 

\begin{exe}
    \ex\label{ex:intr}
    \begin{xlist}
        \ex\label{ex:intr-sv}
        \gll die empai duduk\\
        3 \textsc{rec.pst} sit\\
        \trans `she just sat down.'\\
        \hfill(BJM01-116-01, 01:00:52–01:00:53, Speaker: Susiana)
        \ex\label{ex:intr-vs} 
        \gll dindak duduk die Wi\\
        not.want sit 3 W.\\
        \trans `she didn't want to sit, Wi.'
    \end{xlist}
    \hfill (BJM01-081-01, 00:13:11–00:13:13, Speaker: Riska)
\end{exe}

The examples above demonstrate other properties of Besemah clause structure. Predicates are commonly preceded by auxiliary verbs that express tense, aspect, and/or mood. The single argument, S, may precede, as in (\ref{ex:intr-sv}), or follow, as in (\ref{ex:intr-vs}), the predicate.

With only a few exceptions, transitive predicates are marked as either A-Voice (AV) or P-Voice (PV) constructions. In AV, the verb is prefixed with the homorganic nasal \textit{N-} and the A argument is considered to be syntactically and pragmatically privileged, often functioning as the topic of discourse and able to undergo specific syntactic operations such as relativization. AV is exemplified in (\ref{ex:trans-av-avp}). In PV, the P argument is considered privileged in the same way, with the additional complication that verbal marking is based upon the person of the A argument. When A is first or second person, it is procliticized to the verb, as in (\ref{ex:trans-pv-pva-1}), whereas when A is third person, it is encliticized to the verb which is optionally prefixed with \textit{di-}, as in (\ref{ex:trans-pv-pva-3}). This voice system is said to be symmetrical as both constructions are equally transitive and neither construction appears to be derived from the other \citep{himmelmann2005austronesian,riesberg2014symmetrical,chen2019western}.

\begin{exe}
    \ex\label{ex:trans-av-avp} 
    \begin{xlist}
        % \exi{R:} 
        \exi{}
        \gll aku la udim .. ng-ambik mulan kawe di situ eh\\
        1\textsc{sg} \textsc{pfv} finish {} \textsc{av}-take seed coffee at there \textsc{tag}\\
        \trans `I already finished taking the coffee seeds there, right.'
    \end{xlist}
    \hfill (BJM01-001-01, 00:20:50–00:20:53, Speaker: Yawan)
\end{exe}

\begin{exe}
    \ex\label{ex:trans-pv-pva-1} 
    \begin{xlist}
        % \exi{R:} 
        \exi{}
        \gll nik abang-abang ku=ambik-i\\
        \textsc{n.li} red-\textsc{distr} 1\textsc{sg}=\textsc{pv}.take-\textsc{caus/loc}\\
        \trans `I took the red ones (i.e. peppers).'
    \end{xlist}
    \hfill (BJM01-158-01, 00:52:54–00:52:46, Speaker: Neti)
\end{exe}

\begin{exe}
    \ex\label{ex:trans-pv-pva-3} 
    \begin{xlist}
        % \exi{R:} 
        \exi{}
        \gll se-gale=nye di-tanam=e\\
        one-all=3 \textsc{pv}-plant=3\\
        \trans `she planted all of them.'
    \end{xlist}
    \hfill (BJM01-098-01, 00:03:44–00:03:46, Speaker: Ijir)
\end{exe}

Grammatical relations are not marked by case or agreement, but S, A in AV, and P in PV pattern together and appear to be the `privileged syntactic arguments' in the sense of \citet{vanvalin1997syntax}. \citet{mcdonnell2022universal} refers to this grammatical relation as the \emph{Primary Argument}. The other argument in transitive constructions (i.e., P in AV, A in PV), referred to as the \emph{Secondary Argument}, occurs adjacent to the verbal predicate and forms a single constituent with it. We refer to this constituent as the \emph{predicate complex}. Primary Arguments occur either before the predicate complex, as in the examples (\ref{ex:trans-av-avp})--(\ref{ex:trans-pv-pva-3}) above, or after the predicate complex, as in (\ref{ex:trans-av-pva}) for AV and (\ref{ex:trans-pv-avp}) for PV below.


\begin{exe}
    \ex\label{ex:trans-av-pva} 
    \begin{xlist}
        % \exi{R:} 
        \exi{}
        \gll m-beli guring-an agi aku\\
        \textsc{av}-buy fry-\textsc{nmz} again 1\textsc{sg}\\
        \trans `I bought fried snacks again.'
    \end{xlist}
    \hfill (BJM01-004-01, 00:59:31–00:59:32, Speaker: Peter)
\end{exe}


\begin{exe}
    \ex\label{ex:trans-pv-avp} 
    \begin{xlist}
        \ex
        \gll dide ku=ambik duit tu tadi\\
        \textsc{neg} 1\textsc{sg}=\textsc{pv}.take money \textsc{dem.dist} earlier\\
        \trans `I didn't take the money earlier.' \\
        \hfill (BJM01-028-01, 00:13:26–00:13:27, Speaker: Husni)
        \ex
        \gll la di-ambik=e due: {..} pelang\\
        \textsc{pfv} \textsc{pv}-take=\textsc{3} two {} dike\\
        \trans `he already took two dikes.'
    \end{xlist}
    \hfill (BJM01-033-01, 00:58:23–00:58:25, Speaker: Mariati)
\end{exe}

Polyfunctional applicative suffixes perform a number of functions. The suffix \textit{-ka} prototypically serves benefactive and instrumental applicative functions, and the suffix \textit{-i} prototypically serves locative and goal applicative functions. Both suffixes can additionally perform causative and other functions that derive transitive verbs from various types of intransitive verbal and nominal roots. They are also able to attach to transitive verbal roots with no apparent change in valency, often serving semantic or pragmatic functions (\cite[see][]{truong2022neglected,mcdonnell2024applicative} for discussion of the complexities of these suffixes in the languages of western Indonesia). In this paper, we gloss  \textit{-ka} as `\textsc{caus/ben}' and \textit{-i} as `\textsc{caus/loc}' to reflect their more prototypical functions.

\begin{exe}
    \ex\label{ex:appl-ka} 
    \begin{xlist}
        \exi{}
        \gll ambik-ka aku ayik Put\\
        \textsc{pv}.take-\textsc{caus/ben} 1\textsc{sg} water P.\\
        \trans `grab me some water, Put.'
    \end{xlist}
    \hfill (BJM01-073-01, 00:43:15–00:43:17, Speaker: Nisa)
\end{exe}

\begin{exe}
    \ex\label{ex:appl-i} 
    \begin{xlist}
        \exi{}
        \gll pinggir ni masih tanam-i ngah padi\\
        edge \textsc{dem.prox} still \textsc{pv}.plant-\textsc{caus/loc} with rice\\
        \trans `the perimeter was still planted with rice.'
    \end{xlist}
    \hfill (BJM01-011-01, 00:12:37–00:12:39, Speaker: Jamisah)
\end{exe}

Within the NP, numerals precede the head noun and other modifiers follow the head noun. NPs are often followed by a demonstrative determiner \textit{ni} `\textsc{dem.prox}' or \textit{tu} `\textsc{dem.dist}'. Consider both lines in the example in (\ref{ex:np-structure}).

\begin{exe}
    \ex\label{ex:np-structure} 
    \begin{xlist}[0\quad A:]
        \exi{1\quad L:}
        \gll ame ikan lele ni,\\
        \textsc{top} fish catfish \textsc{dem.prox} \\
        \trans `as for the catfish,'
        \exi{2\quad \hphantom{L:}}
        \gll di-untal-ka baih ayam putung mati-mati tu bik\\
        \textsc{pv}-toss-\textsc{caus/ben} just chicken cut die-\textsc{distr} \textsc{dem.dist} aunt\\
        \trans `throw to them butchered chickens that are dead, auntie.'
    \end{xlist}
    \hfill (BJM01-125-01, 00:15:23–00:15:26, Speaker: Lis)
\end{exe}

Besemah, like many other languages of the region, makes use of minimal structures where predicates occur without explicit arguments \citep{mcdonnell2016symmetrical}. The absence of explicit arguments is most commonly analyzed as a form of ellipsis. However, interactional approaches over the past decade have not made such a claim and do not assume that arguments are elided \citep[see][]{ewing2019predicate}. Instead, they take a what-you-see-is-what-you-get approach, where minimal structures are not outliers but the norm. Consider the example in (\ref{ex:minimal}) below. 


\begin{exe}
    \ex\label{ex:minimal} 
    \begin{xlist}[0\quad L:]
        \exi{1\quad L:} 
        \gll la beghape kali ng-gureng?\\
        \textsc{pfv} how.many times \textsc{av}-fry\\
        \trans `how many times did (you) fry (the fishcakes).'
        \exi{2\quad \hphantom{L:}} 
        \gll tujuh tujuh saje\\
        seven seven continuously\\
        \trans `seven, seven every time.'
        \exi{3\quad N:} 
        \gll dide tujuh saje Ci\\
        \textsc{neg} seven continuously C.\\
        \trans `not seven every time, Ci.'
        \exi{} (0.8)
        \exi{4\quad \hphantom{N:}} 
        \gll dapat beghape?\\
        get how.many\\
        \trans `how many (fishcakes) do (we) have?'
        \exi{5\quad \hphantom{N:}} 
        \gll dide pule ng-itung eh\\
        \textsc{neg} also \textsc{av}-count \textsc{tag}\\
        \trans `(we) didn't really count (them), right.'
    \end{xlist}
    \hfill (BJM01-158-01, 01:13:17–01:13:22, Speakers: Leksi, Neti)
\end{exe}

\section{Fillers in Besemah}\label{sec:fillers}
As mentioned in the introduction, Besemah makes use of two categories of fillers. The first is the dedicated filler \textit{anu}, which is often glossed as `whatchamacallit' or as `whatshername' or `whatshisname'. This filler is widespread in Malayic languages and is reconstructed to Proto Malayic *anuʔ `something; someone, so-and-so' as an indefinite pronoun \citep[128--129]{adelaar1992protomalayic}. At higher levels, \citet[516]{blust2013austronesian} reconstructs Proto Malayo-Polynesian *anu as an indefinite interrogative, providing the gloss `whatchamacallit', and, at the top level of the family, Proto Austronesian *anu as a general interrogative, providing the gloss `what'. Thus, the source of the filler in Besemah is likely the interrogative pronoun *anu that can be reconstructed all the way back to Proto Austronesian. What is remarkable is that the use of *anu as a filler can be reconstructed all the way back to Proto Malayo-Polynesian and that this form is today used as a filler in a diverse set of Malayo-Polynesian languages spoken throughout the Philippines and Indonesia (see \cite{blust2023acd} for a list of reflexes).\footnote{See \citet[][92--93]{nagaya2022tagalog} for an alternative analysis, wherein the interrogative pronoun developed from a placeholder.}

The second category is the demonstrative pronouns \textit{ini}, a proximal demonstrative, and \textit{itu}, a distal demonstrative, although the latter is much less often used as a filler. While these forms are familiar from other Malayic languages and are reconstructed to Proto Malayic \citep[129]{adelaar1992protomalayic}, Besemah actually contrasts two forms of the demonstrative pronouns with innovative forms \textit{tini}, for the proximal, and \textit{titu}, for the distal. In prototypical demonstrative (i.e. non-filler) functions, the two sets of forms appear to behave the same, and in discussions with Besemah speakers, the forms are completely interchangeable. However, when functioning as a filler, the original forms \textit{ini} and \textit{itu} are far more common. Short forms of these demonstratives, \textit{ni} and \textit{tu}, may only be used as demonstrative determiners, marking the end of a NP or certain subordinate clauses.

\subsection{Placeholder and hesitator functions}\label{sec:placeholder-filler}
Hesitator and placeholder functions are evident for both \textit{anu} and the demonstrative pronouns \textit{ini} `\textsc{dem.prox}' and \textit{itu} `\textsc{dem.dist}'. The examples below demonstrate how each type is used in both functions: (\ref{ex:anu-placeholder}) and (\ref{ex:ini-placeholder}) exemplify placeholders for \textit{anu} and \textit{ini}, respectively; (\ref{ex:anu-hesitator}) and (\ref{ex:ini-hesitator}) demonstrate hesitator functions for \textit{anu} and \textit{ini}, respectively.

In line 1 of (\ref{ex:anu-placeholder}), Jamisah is discussing a road on the downriver side of the village, and in line 2, she employs the filler \textit{anu} between an auxiliary, \textit{masih} `still', and a PP, \textit{di lembak ni} `on the downriver side'. This position is reserved for predicates and, as we will see below, is recognizable as such to participants. Thus, \textit{anu} functions here as a placeholder for a predicate, one which is repaired in line 3 along with the recycled auxiliary \textit{masih} `still'. In this and all following multi-line examples, an arrow (→) is used to indicate the line in which the filler is used.  

\begin{exe}
    \ex\label{ex:anu-placeholder} 
    \begin{xlist}[0\quad →A:]
        \exi{1\quad \hphantom{→}J:} 
        \gll ame nik di lembak ni,\\
        \textsc{top} \textsc{n.li} at downriver \textsc{dem.prox}\\
        \trans `as for the downriver one (i.e. road),'
        \exi{2\quad →\hphantom{J:}}
        \gll masih \textbf{anu} di lembak ni,\\
        still \textsc{fill} at downriver \textsc{dem.prox}\\
        \trans `still \textbf{whatchamacallit} on the downriver (side),'
        \exi{3\quad \hphantom{→J:}}
        \gll  masih \uline{rusak}\\
        still broken\\
        \trans `(it) is still \uline{broken down}.'
    \end{xlist}
    \hfill (BJM01-011-01, 00:31:06--00:31:09, Speaker: Jamisah)
\end{exe}

In the talk just prior to (\ref{ex:ini-placeholder}), the participants are airing their grievances about certain people passing through their fields, when in line 1 Partiwi seeks to bring the topic to a close by stating that they just need to worry about themselves and their families (i.e. `think about our children'). This example is similar to the previous one except that the speaker, Partiwi, employs in line 1 the proximal demonstrative \textit{ini} `\textsc{dem.prox}' just after an auxiliary \textit{nak} `want'. Again, this position is evidence that the filler is acting as a placeholder. The placeholder is repaired in line 2 with the predicate complex \textit{nginaki dai anak} `to think about our children (lit. to look at (our) child's face)'. What is particularly interesting here is that, while she recycles the auxiliary \textit{nak} `want' in the repair, the placeholder \textit{ini} does not share any of the verbal morphology with what is repaired. We return to this mismatch between the morphology of the placeholder and the repair in \sectref{sec:morphological-possibilities}.

\begin{exe}
    \ex\label{ex:ini-placeholder} 
    \begin{xlist}[0\quad →A:]
        \exi{1\quad →P:} 
        \gll tape ame diwik nak \textbf{ini} eh\\
        what \textsc{top} self want \textsc{dem.prox} \textsc{tag}\\
        \trans `because, as for us, (we) should \textbf{whatchamacallit}, right.'
        \exi{2\quad \hphantom{→P:}}
        \gll ka- nak \uline{ng-(k)inak-i\hphantom{.\textsc{loc} }dai\hphantom{e }anak}\\
        \textsc{trun} want {\textsc{av}-see-\textsc{caus/loc} face child}\\
        \trans `(we) should \uline{think of our children} (lit. want to look at (our) child's face),'
    \end{xlist}
    \hfill (BJM01-125-01, 00:40:02--00:40:06, Speaker: Partiwi)
\end{exe}

\textcite[][2]{fox2010introduction}, following terminology from Interactional Linguistics, describes placeholders as ``fulfill[ing] the syntactic projection of the turn so far.'' In the examples above, the placeholders \textit{anu} and \textit{ini} fulfill the syntactic projection of the predicate and predicate complex, respectively; the former is repaired by the predicate \textit{rusak} `broken' in (\ref{ex:anu-placeholder}) and the latter by the predicate complex \textit{nginaki dai anak} `to look at (our) child's face' in (\ref{ex:ini-placeholder}). \textcite{podlesskaya2010parameters} uses the term \textit{target}, which refers to the intended word (or phrase) that the filler is taking the place of. Thus, the predicate \textit{rusak} `broken' in (\ref{ex:anu-placeholder}) and the predicate complex \textit{nginaki dai anak} `to look at (our) child's face' in (\ref{ex:ini-placeholder}) are considered the targets. While notions of syntactic projection and target are conceptually different, in practice they can be used interchangeably when describing the syntactic properties of placeholders. However, it is important to note that these notions often assume some sort of repair (i.e., a replacement of the placeholder) or, at the very least that there is a particular, identifiable intended word that the placeholder takes the place of. As we show in \sectref{sec:repair}, the majority of placeholders in Besemah are not repaired and in a number of cases it is not entirely clear what the intended word is or that there is even an intended word at all. For this reason, we simply describe the projection that the placeholder fulfills or simply the syntactic position in which the placeholder occurs (see \sectref{sec:syntactic-position-placeholder}).

Turning to the hesitator functions of the same fillers, the example in (\ref{ex:anu-hesitator}) demonstrates how \textit{anu} is not syntactically integrated into the utterance and thus not recognizable as a placeholder. In line 1, Dewi is explaining that really muddy rice paddies are nice to work in. After a short pause, she adds the qualifier in line 2. She uses the filler \textit{anu} before \textit{sambil} `while', which marks simultaneous adverbial clauses. In the syntax of these simultaneous clauses, any elements occuring before \textit{sambil} `while' are not syntactically integrated into the clause. Thus, the filler here is analyzed as a hesitator.


\begin{exe}
\ex\label{ex:anu-hesitator} 
    \begin{xlist}[0\quad →A:]
        \exi{1\quad \hphantom{→}D:} 
        \gll anye ame kampung lacak tu lemak die\\
        but \textsc{top} village mud \textsc{dem.dist} pleasant 3\\
        \trans `but, if (it's) muddy, it's nice.'
        \exi{} (0.4)
        \exi{2\quad →\hphantom{D:}}
        \gll  \textbf{anu} sambil mundur\\
        \textsc{fill} while retreat\\
        \trans `\textbf{uhm} while going backwards.'
    \end{xlist}
    \hfill (BJM01-011-01, 00:12:16--00:12:20, Speaker: Dewi)
\end{exe}

In (\ref{ex:ini-hesitator}), Neti and Leksi are talking about how Leksi's husband fixed his health insurance card. Neti summarizes what Leksi had said in line 1, and then, after this fully articulated transitive clause, she uses the proximal demonstrative \textit{ini} as a filler in its own IU in line 2. The filler is followed by a pause and a self-addressed question before Neti asks her question to Leksi in the final line. This use of the filler is again not integrated into the clause and acts on its own as a hesitator. 

\begin{exe}
    \ex\label{ex:ini-hesitator} 
    \begin{xlist}[0\quad →A:]
        \exi{1\quad \hphantom{→}N:} 
        \gll Pinsi ng-iluk-i KIS tu\footnotemark\\
        P \textsc{av}-beautiful-\textsc{caus/loc} medical.card \textsc{dem.dist}\\
        \trans `Pinsi fixed the medical card.'
        \exi{} (0.6)
        \exi{2\quad →\hphantom{N:}}
        \gll  \textbf{ini},\\
        \textsc{dem.prox}\\
        \trans `\textbf{uhm},'
        \exi{} (0.4)
        \exi{3\quad \hphantom{→N:}}
        \gll  tape?\\
        what\\
        \trans `what?'
        \exi{} (0.7)
        \exi{4\quad \hphantom{→N:}}
        \gll  langsung s(e)=aghi Ci?\\
        direct one=day C.\\
        \trans `in a single day, Ci?'
    \end{xlist}
    \hfill (BJM01-158-01, 00:35:21--00:35:27, Speaker: Neti)
\end{exe}

\footnotetext{KIS stands for \textit{Kartu Indonesia Sehat} `Healthy Indonesia Card' which is used to access free health services.}

When either form of the filler takes affix or clitic morphology, we analyze the filler as performing a placeholder function, fitting with \citeauthor{hayashi2006crosslinguistic}'s (\citeyear{hayashi2006crosslinguistic}) criterion of morphosyntactic integration (see \sectref{sec:intro}). While the observed morphological forms of placeholders are discussed in the next subsection, the following examples demonstrate the placeholder functions of affixed fillers.

In (\ref{ex:anunye-placeholder}), Royani is telling her granddaughter that she should not throw away her rice. She uses \textit{anu} in line 2 with a third-person possessive enclitic \textit{=nye}. Aside from the enclitic marking, \textit{anu} is followed by the demonstrative determiner \textit{tu} and follows the PV-marked predicate, both of which provide further support for the syntactic integration of the filler (i.e., as a noun). Note that the enclitic \textit{=nye} in Besemah has a number of functions other than indicating third person possession and A arguments in PV constructions, including marking for definiteness. Similar uses of the third singular pronoun are found in Nasal (see \cite{chapters/billings_mcdonnell}).  

\begin{exe}
    \ex\label{ex:anunye-placeholder} 
    \begin{xlist}[0\quad →A:]
        \exi{1\quad \hphantom{→}R:} 
        \gll janga:n\\
        \textsc{neg.imp}\\
        \trans `don't.'
        \exi{2\quad →\hphantom{R:}}
        \gll  capak-i \textbf{anu}=\textbf{nye} tu,\\
        \textsc{pv}.discard-\textsc{caus/loc} \textsc{fill}=\textsc{3} \textsc{dem.dist}\\
        \trans `throw away \textbf{her whatchamacallit},'
        \exi{3\quad \hphantom{→R:}}
        \gll  \uline{nasik=(ny)e}\\
        rice=\textsc{3}\\
        \trans `\uline{her rice}.'
    \end{xlist}
    \hfill (BJM01-028-01, 00:34:01--00:34:04, Speaker: Royani)
\end{exe}

The same sort of situation is found with the demonstrative filler \textit{ini} in (\ref{ex:ininye-placeholder}). In this conversation, participants are weaving, and in what immediately precedes (\ref{ex:ininye-placeholder}), they are discussing how they go into the jungle to find particular plants that are good for weaving. In line 1, Riska uses the filler with the enclitic \textit{=nye} when stating that something is all gone. After a self-addressed question, the filler is repaired in line 3. Aside from the enclitic marking, the post-predicate position is recognizable as a position for Primary Arguments, providing support for a placeholder analysis.

\begin{exe}
    \ex\label{ex:ininye-placeholder} 
    \begin{xlist}[0\quad →A:]
        \exi{1\quad →R:} 
        \gll la abis die \textbf{ini=nye}\\
        \textsc{pfv} finish \textsc{ints} \textsc{dem.prox=3}\\
        \trans `the \textbf{whatchamacallit} was all gone.'
        \exi{2\quad \hphantom{→R:}}
        \gll  tape,\\
         what \\
        \trans `what,'
        \exi{3\quad \hphantom{→R:}}
        \gll  \uline{bemban=(ny)e}\\
        plant=3\\
        \trans `\uline{the \textit{bemban} plant}.'
    \end{xlist}
    \hfill (BJM01-081-01, 00:21:16--00:21:18, Speaker: Riska)
\end{exe}

As we will see further below, if placeholders occur with morphology, the morphology is most frequently verbal. In the example in (\ref{ex:beanu-placeholder}), the filler in line 1 is marked with the so-called middle voice prefix \textit{be-} (see \tabref{tab:morphology} for a description of this prefix). Its position between the negator \textit{dide} `\textsc{neg}' and the adverb \textit{agi} `again' is a recognizable predicate position. In line 2, the placeholder is repaired with recycling of the prefix and the adverb.

\begin{exe}
    \ex\label{ex:beanu-placeholder} 
    \begin{xlist}[0\quad →A:]
        \exi{1\quad →Y:} 
        \gll dide \textbf{be-anu} agi Sa eh\\
        \textsc{neg} \textsc{mid-fill} again \textsc{voc} \textsc{tag}\\
        \trans `(you) don't \textbf{use whatchamacallit} again, Sa, right.'
        \exi{2\quad \hphantom{→Y:}}
        \gll  \uline{be-LKS} agi\\
         \textsc{mid}-LKS again \\
        \trans `\uline{use LKS}\footnotemark (i.e. worksheets) anymore.'
    \end{xlist}
    \hfill (BJM01-073-01, 00:12:34--00:12:36, Speaker: Yanti)
\end{exe}

\footnotetext{\textit{Lembar Kerja Siswa} `Student Worksheets' are lessons and activities required by public schools for students.}

The excerpt in (\ref{ex:beini-placeholder}) is similar, with the demonstrative filler preceded by auxiliaries and prefixed with \textit{be-} `\textsc{mid}' in line 1. After Hendi initiates repair in line 2, the placeholder is repaired in line 3 with only the \textit{be-}marked predicate.

\begin{exe}
    \ex\label{ex:beini-placeholder}
    \begin{xlist}[0\quad →A:]
        \exi{1\quad →I:} 
        \gll masih nak \textbf{be:-ini} dighi\\
        still want \textsc{mid-fill} self\\
        \trans `we (lit. self) should \textbf{have whatchamacallit}.'
        \exi{2\quad \hphantom{→}H:}
        \gll {..} \#nak \#tuape?\footnotemark\\
        {} want what\\
        \trans `what do (you) want?'
        \exi{} (0.5)
        \exi{3\quad \hphantom{→}I:}
        \gll  \uline{be-rakit}\\
         \textsc{mid}-racket \\
        \trans `\uline{have a racket}.'
    \end{xlist}
    \hfill (BJM01-041-01, 01:07:01--01:07:05, Speakers: Iril, Hendi)
\end{exe}

\footnotetext{The \# refers to speech that is not clear. Whether Hendi says this phrase or not, it is fairly clear that Iril understood it as an initiation of repair.}


As many studies have discussed, fillers play an important role in word formulation trouble and are inextricably linked to repair \parencite[see, e.g.,][]{wouk2005syntax}. Many of the examples of fillers above have demonstrated this connection. Placeholder repair is by nature most often self-initiated, but can be repaired by either party. The previous excerpt in (\ref{ex:beini-placeholder}), for example, is a case of other-initiated self-repair.

However, this is not the most frequent type of repair. In \sectref{sec:repair}, we mention that the vast majority of repairs are instances of same-turn self-repair, as in (\ref{ex:anu-placeholder})--(\ref{ex:ini-placeholder}) and (\ref{ex:anunye-placeholder})--(\ref{ex:beanu-placeholder}). This may not be surprising given apparent preferences for self-repair in languages like English \citep{schegloff1977preference}. There are also instances of other-initiated other-repair, as in (\ref{ex:anu-other-other}).\footnote{See also the next example in (\ref{ex:ini-word-search}).} In this excerpt, Leksi and Neti are cooking together. Leksi is explaining what she plans to do with some seasoning when she uses the filler in line 1. Immediately after her use of the filler, Neti initiates and completes repair in line 2, and Leksi acknowledges the repair in line 3.

\begin{exe}
\ex\label{ex:anu-other-other}
\begin{xlist}[0\quad →A:]
\exi{1\quad →L:} 
\gll mangke aku pacak (1.0) \textbf{ng-anu-ka=nye}[:]\\
consequently 1\textsc{sg} can {} \textsc{av-fill-caus/ben=3}\\
\glt `so that I \textbf{whatchamacallit it}.' \\
\exi{2\quad \hphantom{→}N:}
\gll \uline{[ng-is]i=nye}\\
\textsc{av}-fill=3\\
\glt `\uline{fill it}.' \\
\exi{3\quad \hphantom{→}L:}
\gll  mm\\
 mhmm\\ 
\glt `mhmm.' \\
\end{xlist}
\hfill (BJM01-158-01, 00:21:13--00:21:16, Speakers: Leksi, Neti)
\end{exe}

Fillers also commonly trigger word searches with self-addressed questions, as seen in (\ref{ex:ini-hesitator}) and (\ref{ex:ininye-placeholder}). A longer word search sequence is provided in (\ref{ex:ini-word-search}), where the participants were discussing how expensive buying phone credits was. In the just prior talk, Riska asks how much the taxes are, and Dewi answers in line 1 using the filler. In line 3, Riska agrees with the amount and then initiates repair in line 4 with \textit{pajak} `taxes'. While Riska initiates repair, Dewi asks a self-addressed question in line 6 overlapping with Riska's repair. Dewi acknowledges the repair in line 7.


\begin{exe}
\ex\label{ex:ini-word-search}
\begin{xlist}[0\quad →A:]
\exi{1\quad →D:} 
\gll lapan ribu nah [\textbf{ini=nye} baih]\\
eight thousand well \textsc{fill}=3 just\\
\glt `well its \textbf{whatchamacallit} is just eight thousand.' \\
\exi{2\quad \hphantom{→}A:}
\gll \hspace{3.25cm}[m.]\\
\hspace{3.25cm}hmm\\
\glt \hspace{3.25cm}`mhmm.' \\
\exi{3\quad \hphantom{→}R:}
\gll  \hspace{3.25cm}[itu=la]:h\\
 \hspace{3.25cm}\textsc{dem.dist=emph}\\ 
\glt \hspace{3.25cm}`that's it.' \\
\exi{4\quad \hphantom{→R:}}
\gll  paj--\\
 \textsc{trun}\\ 
\glt `tax--' \\
\exi{} (0.5)
\exi{5\quad \hphantom{→R:}}
\gll  \uline{paj[ak=(ny)e} baih kan]\\
 tax=\textsc{3} just right\\ 
\glt `just \uline{its taxes}, \textsc{tag}.' \\
\exi{6\quad \hphantom{→}D:}
\gll  \hspace{0.35cm}[tape name=nye]?\\
 \hspace{0.35cm}what name=\textsc{3}\\ 
\glt \hspace{0.35cm}`what's it called?'\\
\exi{7\quad \hphantom{→D:}}
\gll au\\
yes\\
\glt `yeah.' \\
\end{xlist}
\hfill (BJM01-081-01, 00:30:20--00:30:26, Speakers: Riska, Dewi, (D)asmi)
\end{exe}

Finally, it should be noted that Besemah, as is the case for many insular Southeast Asian languages, generally lacks hesitative particles akin to English \textit{uh} or \textit{uhm}. Instead, other cues, such as lengthening and pause, also signal hesitation \citep[see][]{streeck_1996,himmelmann2014asymmetries}. An example of a sound stretch and subsequent pause is found in the example in (\ref{ex:sound-stretch}) below. We will see in \sectref{sec:quantitative} that these cues are co-present with fillers in a number of instances.

\begin{exe}
\ex\label{ex:sound-stretch} 
    \begin{xlist}
        % \exi{R:} 
        \exi{}
        \gll ku=kinak-i endik di: (1.2) Lubuk Saung\\
        1\textsc{sg}=see-\textsc{caus/loc} \textsc{n.li} at {} L. S.\\
        \trans `I saw the one in (the village of) Lubuk Saung.'
    \end{xlist}
    \hfill (BJM01-010, 00:05:47--00:05:50, Speaker: Aripin)
\end{exe}

In this example, the general locative preposition \textit{di} `at' is lengthened and followed by a fairly long pause before Aripin provides the name of the village \textit{Lubuk Saung}. There are many other examples where pauses and sound stretches accompany fillers. In (\ref{ex:ini-hesitator}), for example, the demonstrative hesitator \textit{ini} is preceded and followed by pauses, and in (\ref{ex:beini-placeholder}), the placeholder contains sound stretch.


As mentioned in \sectref{sec:intro}, it is not always possible to determine whether a filler is acting as a hesitator or as a placeholder. Consider the extended example in (\ref{ex:ini-anu-indeter}), wherein the demonstrative filler \textit{ini} is used in line 3 and the filler \textit{anu} is used in line 6.

\begin{exe}
\ex\label{ex:ini-anu-indeter}
\begin{xlist}[0\quad →A:]
\exi{1\quad \hphantom{→}L:} 
\gll petang kan aku ke situ\\
evening \textsc{tag} \textsc{1sg} to there\\
\glt `in the evening, right, I went there.' \\
\exi{2\quad \hphantom{→}N:}
\gll  au\\
 yes\\ 
\glt `yeah.' \\
\exi{3\quad →L:}
\gll mane \textbf{ini} kain panjang?\\
where \textsc{fill} cloth long\\
\glt `where is \textbf{uhm/whatchamacallit} the sarong?'\\
\exi{4\quad \hphantom{→L:}}
\gll  kain panjang ape kate Mili ni\\
  cloth long what word M. \textsc{dem.prox}\\ 
\glt `what sarong, said Mili.'\\     
\exi{5\quad \hphantom{→}N:}
\gll  mm\\
mhmm.\\ 
\glt `mhmm.' \\
\exi{} (1.8)
\exi{6\quad →L:} 
\gll uk- kalu: \textbf{anu},\\
\textsc{trun} maybe \textsc{fill}\\
\glt `maybe \textbf{uhm/whatchamacallit}' \\
\exi{7\quad \hphantom{→L:}}
\gll kain panjang Gina\\
cloth long G.\\
\glt `Gina's sarong.' \\
\exi{8\quad \hphantom{→L:}}
\gll  pakai=(ny)e nga Wit\\
 \textsc{pv}.use=\textsc{3} with W.\\ 
\glt `Wit wore (it)' \\
\end{xlist}
\hfill (BJM01-158-01, 00:55:27--00:55:38, Speakers: Leksi, Neti)
\end{exe}

In this excerpt, Leksi is relaying a story about a sarong, \textit{kain panjang} `(lit.) long cloth', to Neti. Before each instance of this word, Leksi uses a filler. In each case, the filler is unaffixed and occurs just before the word she is meaning to say. Under a hesitator analysis, Leksi employs the fillers \textit{ini} and \textit{anu} as delay devices that allow her time to think of the word. Under a placeholder analysis, she employs the placeholder until she selects the more specific word. Without any recycling, evidence from position in the clause, or morphological marking, it is not possible to determine whether these instances are acting as hesitators or as placeholders. In conversations, it is apparent from interlocutors' responses, however, that this presents no issue, and the filler's status as a hesitator or a placeholder makes little difference to the progressivity of the interaction \parencite{schegloff2007sequence}. Thus, recycling, position within the clause, and morphological marking are key criteria for identifying placeholders. We return to this issue in \sectref{sec:hesitators-v-placeholders}.


Examples like the one in (\ref{ex:ini-anu-indeter}) demonstrate that the line that \citet{hayashi2006crosslinguistic} draw between placeholder and hesitator is not always so clear. Despite any challenges that this presents to the analyst, however, it does not appear to make any noticeable difference to participants. 


\subsection{Morphological possibilities of fillers}\label{sec:morphological-possibilities}
In the vast majority of cases, fillers in Besemah have the same morphosyntactic properties as either noun roots or intransitive verb roots, both of which may appear as arguments or predicates without any further affixation. Consider the placeholders in (\ref{ex:ini-placeholder-np-p}) and (\ref{ex:ini-placeholder-intr}), where the placeholder in (\ref{ex:ini-placeholder-np-p}) expresses a nominal argument and the one in (\ref{ex:ini-placeholder-intr}) a verbal predicate.

In the excerpt in (\ref{ex:ini-placeholder-np-p}), Dewi and Riska are discussing weaving. In line 1, Dewi uses the placeholder \textit{ini}, which occurs as the P argument in an AV construction. It is then repaired with the noun \textit{wi} `rattan' by both Dewi and Riska in overlapping speech in lines 3 and 4 without any recycling.

\begin{exe}
    \ex\label{ex:ini-placeholder-np-p}
    \begin{xlist}[0\quad →A:]
        \exi{1\quad →D:} 
        \gll lemak ng-ambik \textbf{ini}\\
        pleasant \textsc{av}-take \textsc{fill}\\
        \glt `it's nice to take \textbf{whatchamacallit}.' \\
        \exi{} (0.8)
        \exi{2\quad \hphantom{→}R:}
        \gll au,\\
        yes\\
        \glt `yeah,' \\
        \exi{3\quad \hphantom{→}D:}
        \gll  [\uline{wi}.]\\
         rattan\\ 
        \glt `\uline{rattan}.' \\
        \exi{4\quad \hphantom{→}R:}
        \gll  [wi] kapuh ni,\\
         rattan \textsc{ext} \textsc{dem.prox}\\
        \glt `rattan and the like,' \\
    \end{xlist}
    \hfill (BJM01-081-01, 00:20:16--00:20:19, Speakers: Dewi, Riska)
\end{exe}

In (\ref{ex:ini-placeholder-intr}), Neti and Leksi are discussing bananas that they plan to use in their cooking. In line 1, Neti uses the proximal demonstrative as a filler in predicate position after the aspect marker \textit{la} `\textsc{pfv}'. In line 2, we see recycling of the conditional clause and repair of the placeholder with the predicate \textit{tue} `old'.

\begin{exe}
    \ex\label{ex:ini-placeholder-intr} 
    \begin{xlist}[0\quad →A:]
        \exi{1\quad →N:}
        \gll ame la \textbf{ini} seghempak die masak\\
        \textsc{top} \textsc{pfv} \textsc{fill} together 3 ripe\\
        \glt `if (they) are already \textbf{whatchamacallit}, they are both ripe.'\\
        \exi{2\quad \hphantom{→N:}}
        \gll ame (1.0) la \uline{tue nian}\\
        \textsc{top} {} \textsc{pfv} {old very} \\
        \glt `if (they) are \uline{really old}.'\\
    \end{xlist}
    \hfill (BJM01-158-01, 00:52:07--00:52:12, Speaker: Neti)
\end{exe}

If a filler is used to express a transitive verb, it appears, with few exceptions, with the causative/applicative suffix \textit{-ka}, which is frequently used to form transitive verbs from nominal or intransitive verbal roots (see \sectref{sec:basic-morphosyntax}). In (\ref{ex:anu-trans-placeholder}), the placeholder \textit{anu} is additionally marked by an AV prefix, and in (\ref{ex:ini-trans-placeholder}), the proximal demonstrative placeholder \textit{ini} in PV is both marked with a proclitic A argument and suffixed with the causative/applicative suffix \textit{-ka}. The small handful of exceptions are discussed below.


\begin{exe}
\ex\label{ex:anu-trans-placeholder} 
\begin{xlist}
\exi{}
\gll die galak \textbf{ng-anu-ka} pingging tu=lah\\
3 want \textsc{av-fill-caus/ben} butt \textsc{dem.dist=emph}\\
\trans `he always \textbf{whatchamacallited} (=\uline{push me to the side}) (with his) butt.'
\end{xlist}
\hfill (BJM01-004-01, 00:25:44--00:25:47, Speaker: Dian)
\end{exe}

% 

\begin{exe}
    \ex\label{ex:ini-trans-placeholder} 
    \begin{xlist}
        \exi{}
        \gll galak ng-guring bekayu ni \textbf{ku=ini-ka}\\
        want \textsc{av-}fry cassava \textsc{dem.prox} 1\textsc{sg=pv.dem.dist-caus/ben}\\
        \trans `(I) like to fry the cassava, (then) \textbf{I whatchamacallit} (=\uline{I mix it}).'
    \end{xlist}
    \hfill (BJM01-158-01, 00:23:54--00:23:56, Speaker: Leksi)
\end{exe}

These properties demonstrate that when \textit{anu} and the demonstrative pronouns are used as placeholders, they are best characterized as falling into one of two word classes, noun or intransitive verb, neither of which requires any additional morphology. Based on the frequencies of morphological patterns, this analysis largely holds in regards to morphological marking; however, there are a few instances wherein \textit{anu} holds the place of a transitive verb without any additional derivational morphology (most commonly the causative/applicative suffix \textit{-ka}) that would be otherwise required for nominal or intransitive verbal roots. As typical of transitive verbs, such a use of the filler can additionally be found with voice morphology \textit{N-} `\textsc{av}' or \textit{di-} `\textsc{pv}'.

The only example in the corpus where the AV prefix occurs with a placeholder not suffixed with \textit{-ka} is shown in line 3 of (\ref{ex:ini-morph-match}). In this excerpt, Rusmani is telling the other participants about some advice she received to not sell the vegetables she had been growing to an agent, who would then turn around and sell them for a higher price. While the placeholder is not repaired, it is fairly clear from the context that she means selling to an agent.

\begin{exe}
\ex\label{ex:ini-morph-match}
\begin{xlist}[0\quad →A:]
\exi{1\quad \hphantom{→}R:} 
\gll anye tape die tu,\\
but what 3 \textsc{dem.dist}\\
\glt `but he,' \\
\exi{2\quad \hphantom{→R:}} 
\gll dide kate=nye,\\
\textsc{neg} say=3\\
\glt `did not, she said,' \\
\exi{3\quad →\hphantom{R:}}
\gll \textbf{ng-anu} ke agin\\
\textsc{av-fill} to agent\\
\glt `\textbf{whatchamacallit} (=\uline{sell them}) to an agent,' \\
\end{xlist}
\hfill (BJM01-071-01, 00:17:00--00:17:04, Speaker: Rusmani)
\end{exe}

In (\ref{ex:dianu-without-ka}), Aripin is discussing the process of working with cocoa seeds. In line 2, he uses the placeholder \textit{anu} with the PV prefix \textit{di-} but without the causative/applicative suffix \textit{-ka}. This is only one of a few examples where this occurs. It is noteworthy that the PV prefix only combines with transitive verbal roots; intransitive or nominal roots must be made into transitive verbs with one of the causative/applicative suffixes \textit{-ka} or \textit{-i} in order to combine with the PV prefix. 

\begin{exe}
\ex\label{ex:dianu-without-ka}
\begin{xlist}[0\quad →A:]
\exi{1\quad \hphantom{→}A:} 
\gll tambah di-ghendam,\\
add \textsc{pv}-submerge\\
\glt `even better to submerge (the cocoa seeds).' \\
\exi{2\quad →\hphantom{A:}}
\gll bukane \textbf{di-anu}\\
\textsc{neg} \textsc{pv-fill}\\
\glt `not \textbf{whatchamacallit} (=\uline{dry them}).' \\
\end{xlist}
\hfill (BJM01-010-01, 00:26:04--00:26:07, Speaker: Aripin)
\end{exe}

These examples demonstrate that as placeholders, \textit{anu} and the demonstrative pronouns have some categorical flexibility, such that they show morphosyntactic properties of transitive verbal roots in addition to intransitive verbal roots and nouns. However, it is important to keep in mind that fillers that show properties of transitive verbs are marginal in the corpus. We should also note that there are instances, such as (\ref{ex:ini-placeholder}) above, where an unaffixed filler is repaired with a marked transitive verb.  

Utilizing the full corpus of Besemah conversations, we see clear patterns with the formal properties of fillers summarized in \tabref{tab:corpus-freq-filler-forms}. Since \textit{anu} exclusively functions as a filler, we can straightforwardly probe the full corpus for all possible forms and their frequencies. From this, we see that approximately 82\% of all forms of \textit{anu} (1,744 tokens) do not occur with any affixation while the remaining 18\% (374 tokens) occur with some sort of affixation or cliticization.\footnote{Clitics readily occur on both affixed and unaffixed forms of the fillers. In the corpus, the unaffixed filler \textit{anu} occurred with the emphatic clitic \textit{=lah} six times and with the polyfunctional third person pronominal clitic \textit{=nye} 96 times. With the unaffixed proximal demonstrative \textit{ini}, the third person pronominal clitic \textit{=nye} occurred 29 times and the first person pronominal clitic \textit{=ku} four times. These clitics also occur with affixed forms. Fillers with clitics but no prefixes or suffixes are included in the counts for affixed forms in \tabref{tab:corpus-freq-filler-forms} but  not presented in \tabref{tab:morphological-forms}.} Because filler is but one function of demonstrative pronouns, it is only feasible in this study to investigate the morphologically marked forms. As expected, demonstrative pronouns most often occur without any affixation, appearing with some 10,000 instances in the corpus. In the affixed forms of the demonstrative fillers, the proximal demonstrative is far more frequent (137 tokens) compared to the distal demonstrative (21 tokens). When comparing affixed forms of the dedicated filler \textit{anu} and the demonstrative fillers, \textit{anu} with 374 tokens is more than twice as frequent as the demonstratives with 158 tokens combined.

\begin{table}
\caption{Raw frequencies of Besemah fillers occurring with and without additional morphology in the full corpus.}
\label{tab:corpus-freq-filler-forms}
\begin{tabularx}{.8\textwidth}{X r  C c}
\lsptoprule
& & \multicolumn{2}{c}{\textbf{Demonstratives}} \\
\cmidrule{3-4}
& \textit{anu} &  {Proximal} \textit{ini} &  {Distal}  \textit{itu} \\
\midrule
Unaffixed   & 1,744 & \textit{\small (unknown)} & \textit{\small (unknown)} \\
Affixed     & 374   & 137   & 21  \\
\cmidrule{2-4}
Total       & 2,118 & \textit{\small (unknown)} & \textit{\small (unknown)} \\
\lspbottomrule
\end{tabularx}
\end{table}

\tabref{tab:morphological-forms} presents a breakdown of these frequencies for the most common forms of filler words in the full corpus. See \tabref{tab:morphology} in the previous section for descriptions of the functions of these affixes.


\begin{table}
\caption{Morphological forms of fillers in the full corpus.}
\label{tab:morphological-forms}
\begin{tabularx}{.8\textwidth}{llYrY}
\lsptoprule
\multicolumn{2}{c}{Affix}       &                       & \multicolumn{2}{c}{Demonstrative} \\
    \cmidrule{3-5}
        &       &   \textit{anu}        & \textbf{Proximal} & \textbf{Distal}\\
\midrule
\textit{N-} `\textsc{av}'  & \textit{-ka} `\textsc{caus/ben}' % \multirow{3}{*}{\textit{-ka}}
& 131
& 7
& 1 \\
\textit{di-} `\textsc{pv}' & \textit{-ka} `\textsc{caus/ben}'
& 61
& 25
& 5 \\
& \textit{-ka} `\textsc{caus/ben}'
& 42
& 35
& 5 \\
\textit{di-} `\textsc{pv}' &
& 6
& 1
& 0 \\
\textit{be-} `\textsc{mid}' & 
& 11
& 20
& 8 \\
\textit{te-} `\textsc{nvol}' & 
& 6
& 5
& 1 \\
\textit{se-} `one' & 
& 5
& 1
& 1 \\
\textit{si-} `\textsc{hon}' & 
& 5
& 0
& 0 \\
& \textit{-an} `\textsc{nmz}' 
& 5
& 10
& 0 \\
\lspbottomrule
\end{tabularx}
\end{table}

In addition to these more common forms, there are a number of forms that were used only once. For the filler \textit{anu}, this  includes a reduplicated form (\textit{anu-anu}), a transitive form with a PV prefix and a locative applicative suffix (\textit{dianui}), a form with the \textit{ke- -an} nominalizing circumfix (\textit{keanuan}), and a complex form with the nominalizing circumfix \textit{pe- -an} and the numeral prefix \textit{se-} (\textit{sepeanuan}). The AV prefix occurs without a transitivizing suffix only once on a filler (i.e., \textit{nganu}, as seen in (\ref{ex:ini-morph-match})). For the proximal demonstrative filler \textit{ini}, the following forms only occur once: \textit{ini} with the reciprocal circumfix \textit{se- -an} (\textit{seinian}) and \textit{ini} with the nominalizing circumfix \textit{ke- -an} (\textit{keinian}).

These patterns of affixation demonstrate several properties of placeholder fillers, which we revisit in \sectref{sec:quantitative}. First, transitive forms are almost invariably formed with the benefactive/instrumental applicative suffix \textit{-ka}; the locative/goal applicative \textit{-i} only occurs once in the corpus with \textit{anu} and never with a demonstrative pronoun. This distribution is partially explained by a morphophonological factor that prohibits \textit{-i} from being affixed to \textit{i}-final roots like \textit{ini} `\textsc{dem.prox}'. This does not, however, explain why it occurs so rarely with \textit{anu}. The very infrequent use of \textit{-i} with placeholders in the corpus is also not explained by its overall frequency. In a sample of four conversations, \textit{-ka} `\textsc{caus/ben}' is overall more frequent with 481 occurrences, but \textit{-i} `\textsc{caus/loc}' is still well represented with 307 occurrences.\footnote{These numbers are based on counts from the following four conversations, with counts for each conversation in parentheses: BJM01-004 (\textit{-ka} = 106; \textit{-i} = 71), BJM01-008 (\textit{-ka} = 145; \textit{-i} = 41), BJM01-010 (\textit{-ka} = 131; \textit{-i} = 137), BJM01-011 (\textit{-ka} = 99; \textit{-i} = 58).}

Second, as \tabref{tab:morphological-forms} shows, affixed forms of \textit{anu} are much more frequent than affixed forms of demonstrative fillers (272 instances of \textit{anu} compared to 104 instances of \textit{ini} and 21 instances of \textit{itu}). However, these differences in frequency are largely due to the disparity between fillers in AV transitive forms with the AV nasal prefix \textit{N-} and applicative suffix \textit{-ka}; this form alone accounts for just under half of all instances of affixed forms of \textit{anu}. It is also striking that with the demonstrative pronoun fillers, the AV form is the least frequent of the transitive verb forms, occurring a total of just eight times as either \textit{nginika} or \textit{ngituka}. To a lesser extent, PV transitive forms with the prefix \textit{di-} and suffix \textit{-ka} are a little more than twice as frequent with \textit{anu} than the demonstrative fillers.

Finally, other morphological forms of the fillers are far less frequent, making it difficult to interpret their relative frequencies. For example, while the verbal prefix \textit{be-} `\textsc{intr}' and the nominalizing suffix \textit{-an} `\textsc{nmz}' are more than twice as frequent with the demonstratives than with \textit{anu}, these numbers are likely too small to extrapolate any sort of conclusion.


Based on a limited typological sample, \citet[14]{podlesskaya2010parameters} hypothesizes that ``if a language has placeholders that can replicate morphology other than nominal, it also has placeholders that replicate nominal morphology, but not vice versa – simply because placeholders are by and large of proNOMINAL origin'' (emphasis in the original). The morphological properties of Besemah fillers support this notion, with both verbal and nominal morphology being replicated by placeholders. That nominal morphology is rare in Besemah fillers is simply a reflection of the properties of their source forms, since nominal morphology is largely lacking from Besemah in general. Simply put, verbal morphology is more frequent because affixed verbal fillers are more commonly derived, while nominal uses of the filler do not require the same sorts of derivation. 

Another parameter that \citet{podlesskaya2010parameters} proposes is the extent to which the morphology on fillers replicates the delayed word or constituent. Besemah appears to show mixed results. Filler forms in several cases are fully replicated, as in (\ref{ex:ini-morph-match-2}). In this example, Riska is asking Dasmi to help her by giving her child some water. In her request, Riska uses the filler in line 1 with a causative/applicative suffix \textit{-ka}. In line 2, Dasmi asks for clarification, and in line 3, Riska offers a repair whose morphological form exactly replicates that of the filler with the suffix \textit{-ka}.

\begin{exe}
    \ex\label{ex:ini-morph-match-2}
    \begin{xlist}[0\quad →A:]
        \exi{1\quad →R:} 
        \gll mintak tulung \textbf{anu-ka} dikit yuk\\
        request help \textsc{pv}.\textsc{fill}-\textsc{caus/ben} little.bit older.sister\\
        \glt `please help \textbf{whatchamacallit}, older sister.' \\
        \exi{} (0.3)
        \exi{2\quad \hphantom{→}D:}
        \gll luk ma[ne]?,\\
        how which\\
        \glt `how's that?' \\
        \exi{3\quad \hphantom{→}R:}
        \gll  \hspace{1.25cm}\uline{[en]juk-ka} dikit ngaghi Jep tu\\
         \hspace{1.25cm}\textsc{pv}.give-\textsc{caus/ben} little.bit with J. \textsc{dem.dist}\\ 
        \glt \hspace{1.25cm}`\uline{give} (it) to Jep.' \\
    \end{xlist}
    \hfill (BJM01-081-01, 00:26:04--00:26:07, Speakers: Riska, Dasmi)
\end{exe}


However, as mentioned above, the causative/applicative suffix \textit{-ka} is all but required to derive transitive verbs with fillers, and there are many examples where the repair does not have the applicative suffix, as in (\ref{ex:dianuka-mismatch}) and (\ref{ex:nginikahnye-mismatch}). In fact, repaired transitive verbs lack the causative/applicative suffix much more frequently than have it (although not universally; see (\ref{ex:ini-morph-match-2}) above). In (\ref{ex:dianuka-mismatch}), Aripin is talking about his cocoa field. In line 1, he uses the filler in a negative imperative construction. The filler is marked by a PV prefix and the causative/applicative suffix \textit{-ka}. In line 2, Damsi repairs the filler, recycling the negative imperative and PV prefix but not the suffix \textit{-ka}.

\begin{exe}
\ex\label{ex:dianuka-mismatch}
\begin{xlist}[0\quad →A:]
\exi{1\quad →A:} 
\gll jangan nian \textbf{di-anu-ka}\\
\textsc{neg.imp} very \textsc{pv-fill-caus/ben}\\
\glt `really don't \textbf{whatchamacallit}.' \\
\exi{} (0.4)
\exi{2\quad \hphantom{→}D:}
\gll jangan \uline{di-racu[n]}\\
\textsc{neg.imp} \textsc{pv}-poison\\
\glt `don't \uline{spray it with poison}.' \\
\exi{3\quad \hphantom{→}A:}
\gll \hspace{2.4cm}[a]u\\
\hspace{2.4cm}yes\\
\glt \hspace{2.4cm}`yeah.' \\
\end{xlist}
\hfill (BJM01-010-01, 00:21:19--00:21:22, Speakers: Aripin, Damsi)
\end{exe}

The example in (\ref{ex:nginikahnye-mismatch}) is similar, but the filler and repair are both in AV. In lines 1 and 2, Riska is responding to a question about why the item she is weaving is stiff. She uses a demonstrative filler with the AV prefix, causative/applicative suffix \textit{-ka}, and the P enclitic argument \textit{=nye} referring to the leaves she is using in her weaving. She immediately repairs the filler in line 3 with the AV prefix and P enclitic argument. In this case, the form of the filler and the repair are the same, save the suffix \textit{-ka}.

\begin{exe}
\ex\label{ex:nginikahnye-mismatch}
\begin{xlist}[0\quad →A:]
\exi{1\quad \hphantom{→}R:} 
\gll emk--\\
\textsc{trun}\\
\exi{2\quad →\hphantom{D:}}
\gll \textbf{ng-ini-ka=nye},\\
\textsc{av-fill-caus/ben}=3\\
\glt `(I) \textbf{whatchamacallit them},' \\
\exi{3\quad \hphantom{→A:}}
\gll nik \uline{me-lipat=e} kan,\\
for \textsc{av}-fold=\textsc{3} \textsc{tag}\\
\glt `\uline{folded them}, right,' \\
\end{xlist}
\hfill (BJM01-081-01, 00:25:17--00:25:20, Speaker: Riska)
\end{exe}

% Aripin: dide nian dianukah.
% Damsi: dik diurus.

\subsection{Interim summary}
In this section, we have seen that both the dedicated filler \textit{anu} and the demonstrative pronouns are used as either hesitators or as placeholders (\sectref{sec:placeholder-filler}). However, when the filler occurs without any morphology, it is not always possible to categorize its use as a hesitator or placeholder; such cases are thus considered indeterminate. In the corpus of Besemah conversations, fillers most commonly occur without any morphological marking. When they are morphologically marked, they are considered placeholders and may be affixed with a range of prefixes, suffixes, and circumfixes. The most common morphology by far is represented by those affixes that derive transitive verbs, as well as the polyfunctional enclitic \textit{=nye} that marks third person possession and definiteness among other functions (see \sectref{sec:morphological-possibilities}). Finally, we showed that fillers are commonly found in repair sequences. They often co-occur with disfluencies, such as cut-offs and sound stretches. The placeholder and its repair, however, do not always share the same form, especially in the case of transitive verbs. The next section takes a quantitative approach to the study of Besemah fillers to better understand the frequencies of the patterns described in this section.

\section{Quantitative analysis of fillers}\label{sec:quantitative}
In an effort to provide a more complete picture of the use of fillers in Besemah interaction, we quantify here the frequencies of many of the features described in \sectref{sec:fillers}. We do so by coding each use of the dedicated filler \textit{anu} and the demonstrative pronouns as fillers in a subset of the corpus of Besemah conversations. This section demonstrates several properties of fillers in Besemah. First, while demonstrative pronouns are fairly frequent as fillers, \textit{anu} is more frequent by a ratio of approximately 3:1. Second, fillers are far more frequently used as placeholders than hesitators; this is especially true of demonstrative fillers. Third, disfluencies associated with trouble in word formulation are significantly more common with hesitators than with placeholders. Finally, it is far more frequent for placeholders to be left unrepaired than to be repaired.

\subsection{Corpus of Besemah conversations}\label{sec:corpus}
Thus far, we have considered the full corpus of conversations in Besemah. In this section, we investigate a subset of the full corpus, including six conversations between two or three Besemah speakers each. These conversations were chosen as a fairly representative sample of the larger set of conversations, with an equal number of men and women ranging in age from 25 to 58 at the time of recording. The subcorpus contains just over 57,000 words in approximately 20,000 intonation units. Short descriptions of these six conversations are presented in Table~\ref{tab:corpus}. 

\begin{table}
\caption{Conversations included in subcorpus for the quantitative study}
\label{tab:corpus}
 \begin{tabularx}{1\textwidth}{l X}
  \lsptoprule
  ID         & Description \\
  \midrule
  BJM01-035
  & A conversation between two friends and neighbors--Ishan (M, 51) and Darsono (M, 58)--at Darsono's house.\\
  BJM01-041 
  & A conversation between three friends--Hendi (M, 28), Peter (M, 25), and Hairil (M, 31)--at a friend's rice paddy.\\

  BJM01-086 
  & A conversation between three friends--Jufri (M, 44), Fikri Tarnizi (M, 36), and Sasli (M, 43)--at Jufri's rice paddy.\\
 
  BJM01-116
  & A conversation between three women--Susianawati (F, 37), Neti (F, 37), and Rusmawati (F, 64)--at Susianawati's house.\\
  BJM01-125
  & A conversation between three women--Lismiana (F, 39), Meri (F, 40), and Partiwi (F, 51)--while collecting seeds from carrot flowers.\\

  BJM01-158 
  & A conversation between two friends--Neti (F, 43) and Leksi (F, 35)--while cooking.\\
  \lspbottomrule
 \end{tabularx}
\end{table}

Utilizing this subset of conversations, we present an in-depth analysis of the properties of fillers. These six conversations were coded following the schema presented in Appendix \ref{sec:coding}. The properties we coded for fall into the following areas: (i) the type of filler (\textit{anu}, proximal demonstrative pronoun, distal demonstrative pronoun), (ii) the use of the filler (hesitator, placeholder, indeterminate), (iii) its position in intonation units, (iv) the presence of disfluencies with fillers, (v) the morphological marking and syntactic position of placeholders, and (vi) properties related to repair. Qualitative descriptions of these properties were discussed in detail and exemplified in \sectref{sec:fillers}.

\subsection{Filler forms}
Overall, we coded 588 instances of fillers in the six conversations, as summarized in \tabref{tab:filler-forms}. The numbers outside of the parentheses represent the frequency count of all forms, while the numbers in parentheses represent the number of fillers that occur with some morphology. This table shows that the filler \textit{anu} is more frequent than the demonstrative pronouns. It also demonstrates that the vast majority of fillers do not occur with any morphology.


\begin{table}
\caption{Frequency counts of Besemah fillers in a subcorpus of six conversations. The total number of fillers found is 588.}
\label{tab:filler-forms}
\begin{tabularx}{.8\textwidth}{XrY}
\lsptoprule
& \multicolumn{2}{c}{ {Demonstratives}} \\ \cmidrule{2-3}
\textit{anu} &  {Proximal} \textit{ini} &  {Distal}  \textit{itu} \\
\midrule
439 (71)  & 120 (28)  & 29 (3) \\
\lspbottomrule
\end{tabularx}
\end{table}

If we compare the distribution of fillers in the subcorpus to the distribution in the full corpus presented in \sectref{sec:morphological-possibilities}, we see similar patterns in the forms that are available for comparison.\footnote{Recall that we were unable to determine (without detailed coding) whether the unaffixed demonstrative pronouns were being used as fillers in the full corpus.} For example, in comparing the frequency of affixed and unaffixed forms of \textit{anu}, we found that \textit{anu} occurred without any affixation in approximately 82\% of all tokens, while in the subcorpus we see that this number is only slightly higher, at 83\% of all instances of \textit{anu} being unaffixed. When comparing the proportion of affixed forms of the demonstrative pronouns and the dedicated filler \textit{anu} in the full corpus to the proportion in the subcorpus, we see that the affixed forms of \textit{anu} make up 70\% of all cases in the full corpus and just under 70\% of affixed forms in the subcorpus. Thus, based on just these two factors, it looks as if the subcorpus is a good representative of the larger corpus.


\subsection{Hesitators versus placeholders}\label{sec:hesitators-v-placeholders}
In the subcorpus of six conversations, we coded the primary distinction between \emph{hesitator} and \emph{placeholder} functions for each filler based on the three criteria outlined in \sectref{sec:placeholder-filler}, namely, recycling, position within the clause, and morphological marking. At the same time, we showed in the same section that there are a number of cases where it is not possible to determine whether the filler is being used as a placeholder or a hesitator. These cases were coded as indeterminate. The frequency counts of each are shown in \tabref{tab:hes-place-function}.

\begin{table}
\caption{Hesitator and placeholder frequency counts of Besemah fillers in the subcorpus.}
\label{tab:hes-place-function}
\begin{tabularx}{.8\textwidth}{lYYY}
\lsptoprule
& & \multicolumn{2}{c}{Demonstratives} \\
\cline{3-4}
& \textit{anu} & Proximal \textit{ini} & Distal  \textit{itu} \\
\midrule
Placeholder   & 302 & 99 & 21 \\
Hesitator     & 86 & 9 & 4 \\
Indeterminate & 51 & 12 & 4 \\
\lspbottomrule
\end{tabularx}
\end{table}



\tabref{tab:hes-place-function} demonstrates that while placeholder functions dominate the distribution for all forms, the hesitator use is especially infrequent for proximal demonstrative pronouns, occurring at just under 8\% of the proximal demonstrative fillers. For comparison, just under 20\% of all instances of \textit{anu} are hesitators. The distal demonstrative pronoun falls somewhere in between, occurring as a hesitator in approximately 14\% of instances. However, the relatively infrequent use of the distal demonstrative makes this number somewhat difficult to interpret. One may argue that these numbers are skewed due to the relatively large number of indeterminate cases. However, even if we considered all indeterminate cases to be hesitators, the same pattern holds: \textit{anu} is overall more frequent and the proportion of hesitators is higher in \textit{anu} than in the demonstrative pronouns.

Several other features are relevant to the distinction between hesitators and placeholders, including the position of the filler within an IU (\cite[see e.g.,][]{hennecke2022placeholder} who find weak correlations between function and prosodic features with fillers in French) and the presence of disfluencies. Hesitators are relatively evenly distributed within IUs across initial (17), medial (20), and final (26) positions. However, there is a clear preference for hesitators to be used in an independent IU (36). Indeterminate fillers are far more frequent in final position (37) than initial (9) or medial (7) positions, with a fewer number occurring in their own IU (12). Placeholder fillers show a different distribution, with clear preference for utterance medial (235) and final (130) positions. They are infrequently encountered initially (46) and only rarely in their own IUs (11). These distributions present a relatively coherent picture that is consistent with differences between placeholders and hesitators. That is, the frequent occurrence of hesitators in independent IUs is consistent with their lack of syntactic integration, and the fact that they are primarily employed as a delay device. The relatively low frequency of placeholders in IU initial position and in independent IUs is consistent with the fact that they are syntactically integrated into their clause.


Disfluencies associated with trouble in word formulation occurred immediately preceding or following the filler in 141 of the 588 occurrences, as shown in \tabref{tab:disfluency}. This table shows that despite hesitators constituting approximately a sixth of filler uses, around a third of the disfluencies coincided with the use of a hesitator (36 of 99) and more than a third with indeterminate cases (25 of 67). This distribution is consistent with analyses that view hesitators primarily as a delay device \citep{clark2002using}. It is also noteworthy that \textit{anu} is used in all but three instances of hesitators with disfluencies co-present. This provides further support that there is a clear preference for \textit{anu} to be used as a hesitator.


\begin{table}
\caption{Frequency counts of disfluencies immediately adjacent to the filler by filler type and form.}
\label{tab:disfluency}
\begin{tabular}{llrrr}
\lsptoprule
  & &  {Placeholder} &  {Hesitator} &  {Indeterminate} \\
\midrule
  {\textit{anu}} &  {Disfluent} & 66 & 33 & 21 \\
                 &  {Fluent}    & 236 & 53 & 30 \\
\tablevspace
  {Proximal \textit{ini}} &  {Disfluent} & 10 & 1 & -- \\
                          &  {Fluent} & 89 & 8 & 12 \\
\tablevspace
  {Distal \textit{itu}} &  {Disfluent} & 4 & 2 & 4 \\
                        &  {Fluent} & 17  & 2 & -- \\
%\cline{2-3}
  % Total: 384 = & 118 & 266 \\
\lspbottomrule
\end{tabular}
\end{table}

The most common disfluencies were pauses (\ref{ex:disfluency-pause}), truncation (\ref{ex:disfluency-trunc}), and lengthening (see (\ref{ex:beini-placeholder}) above) preceding or following the use of the filler. In a number of cases, these disfluences co-occur in the same utterance. Pauses were the most common, occurring 64 times, followed by truncation with 55 occurrences and lengthening with 29 occurrences in the subcorpus.



\begin{exe}
\ex\label{ex:disfluency-pause} \begin{xlist}
\exi{} \gll
la beghape kali die ni {..} \textbf{anu} {..} ny-(s)ayur ni \\
already many time 3 this {} \textsc{fill} {} \textsc{av}-vegetable this \\
\glt `he already multiple times, \textbf{umm}, planted these vegetables.' \\
\end{xlist}
\hfill (BJM01-086, 00:02:47–00:02:50, Speaker: Sasli) 
\end{exe}

\begin{exe}
    \ex\label{ex:disfluency-trunc} \begin{xlist}[0\quad →A:]
        \exi{1\quad →D:}
        \gll ka- katah alap=e: cu- [\textbf{anu=nye}]:\\
        \textsc{trun} how good=3 \textsc{trun} \textsc{fill}=3 \\
        \glt `how beautiful were her \textbf{whatchamacallit}.' \\
        \exi{2\quad \hphantom{→}I:}
        \gll \hspace{11em}[au] \\
        \hspace{11em}yes \\
        \glt \hspace{11em}`yeah.' \\
        \exi{3\quad \hphantom{→}D:}
        \gll {..} \uline{tumat=e} \\
        {} tomato=3 \\
        \glt `\uline{her tomatoes}.' \\
    \end{xlist}
    \hfill (BJM01-035-01, 00:28:22–01:07:05, Speakers: Darsono, Ishan) 
\end{exe}

\subsection{Placeholder syntactic properties}\label{sec:syntactic-position-placeholder}
The examples in \sectref{sec:fillers} demonstrate how placeholders in Besemah can be predicates, as in (\ref{ex:anu-placeholder}) and (\ref{ex:ini-placeholder}) above, or arguments, as in (\ref{ex:anunye-placeholder}) and (\ref{ex:ininye-placeholder}) above. It is also possible for placeholders to be the object in a PP, as in (\ref{ex:pp}), or a modifier in an NP, as in (\ref{ex:mod}).


\begin{exe}
    \ex\label{ex:pp} \begin{xlist}[0\quad →A:]
        \exi{1\quad \hphantom{→}D:}
        \gll jeme (H) mane,\\
        person {} which\\
        \glt `which person,' \\
        \exi{2\quad →\hphantom{Y:}}
        \gll jalan ke \textbf{anu} tu eh\\
        walk to which \textsc{dem.dist} \textsc{tag}\\
        \glt `walked to \textbf{whatchamacallit}, right.' \\
    \end{xlist}
    \hfill (BJM01-035-01, 01:07:00–01:07:03, Speaker: Darsono) 
\end{exe}

\begin{exe}
    \ex\label{ex:mod} \begin{xlist}[0\quad →A:]
        \exi{1\quad →U:}
        \gll mesin {..} \textbf{ini},\\
        machine {} \textsc{dem.prox}\\
        \glt `\textbf{whatchamacallit} machine,' \\
        \exi{2\quad \hphantom{→}J:}
        \gll {..} au,\\
        {} yes\\
        \glt  `yeah,' \\
        \exi{3\quad \hphantom{→}U:}
        \gll mesin \uline{padi kecik} eh\\
       machine {rice\hphantom{i} small} \textsc{tag}\\
        \glt `the \uline{small rice} (milling) machine.' \\
    \end{xlist}
    \hfill (BJM01-086-01, 00:35:44–00:35:47, Speakers: Ujang, Jupri) 
\end{exe}

In (\ref{ex:pp}), Darsono is launching into a story but having trouble locating the name of the person (line 1) and where this person is going (line 2). In line 2, the placeholder \textit{anu} occurs as the object of the preposition, occurring between the preposition \textit{ke} `to' and the demonstrative determiner \textit{tu}. It is eventually repaired by Darsono after a lengthy word search. In line 1 of (\ref{ex:mod}), Ujang is discussing rice milling and employs the noun \textit{mesin} `machine', which is followed by the proximal demonstrative placeholder \textit{ini}. In line 3, the head noun is recycled and the placeholder in modifier position is replaced. The resulting NP is \textit{mesin padi kecik} `small rice (milling) machine'.


The frequencies in which placeholders occupy these positions in the clause are shown in \tabref{table:place-rep-rep}. The numbers within the parentheses represent the morphologically marked forms, while the numbers outside of the parentheses represent the total number of forms irrespective of morphological marking.

\begin{table}
    \caption{Placeholder frequency counts of Besemah fillers by syntactic position.}
    \label{table:place-rep-rep}
    \begin{tabularx}{0.8\textwidth}{X@{}rrrr}
        \lsptoprule
        & Argument & Modifier & PP-Object & Predicate \\
        \midrule
        \textit{anu} & 80 (26) & 25 (1) & 19 (0) & 178 (43) \\ 
        Proximal \textit{ini} & 17 (4) & 6 (1) & 14 (1) & 62 (22) \\
        Distal \textit{itu} & 6 (1) & 2 (0) & 4 (1) & 9 (1) \\
        \midrule
        Total (422) & 103 & 33 & 37 & 249 \\
        \lspbottomrule
    \end{tabularx}
\end{table}

\tabref{table:place-rep-rep} demonstrates how Besemah placeholders have a wide syntactic distribution, which may be unsurprisingly similar to colloquial Indonesian as described in \textcite{wouk2005syntax} \parencite[see also][13--14]{podlesskaya2010parameters}. The syntactic positions we have coded in \tabref{table:place-rep-rep} largely overlap with traditional word classes. Where placeholders in our coding are repaired, arguments and objects of prepositions are exclusively nouns or NPs while predicates are almost exclusively verbs, with a single exception where the predicate in a cleft construction is repaired by a noun. Modifiers are repaired primarily by nouns or NPs (e.g. possessors), with a small number of instances repaired by verbs.\footnote{Besemah, like many Austronesian languages, does not distinguish adjectives from verbs \parencite[see ][]{mosel2023word}.}

While it is possible to identify these positions without morphological marking, certain affixes and clitics, if present, allow for rather transparent interpretation of the placeholder's syntactic status. As described in \sectref{sec:morphological-possibilities}, strictly nominal or verbal morphology most clearly provides evidence for this. There are only two enclitics that occur with both nominal and verbal forms: the focus enclitic \textit{=lah} and the enclitic \textit{=nye}, which can both attach to nominal arguments (marking third person possession or definiteness) and to verbal predicates (referencing the A argument in PV or the P argument in AV). In the latter case, the filler is always occupying the position of a transitive verb and is thus suffixed with the causative/applicative \textit{-ka} `\textsc{caus/ben}', so there is no confusion when coding these examples.

What is perhaps most striking about the syntactic positions of placeholders in Besemah is the skewing towards predicates in all forms; predicates make up roughly 60\% of all placeholders. Prototypically nominal positions, such as arguments and objects in PPs, make up the majority of the remaining positions of placeholders followed by nominal modifiers. While \textcite{podlesskaya2010parameters} proposes that ``[i]t is more common for placeholders to substitute nominal constituents'' across languages, it is interesting that in Besemah the most frequent positions are associated with verbal elements. 

Comparisons among the different placeholder forms do not reveal any significant differences in their morphosyntactic properties. Predicates occur with similar frequencies across the fillers, although with only 21 total occurrences, it is difficult to draw any definitive conclusions about distal demonstratives. The dedicated filler \textit{anu} occurs more frequently in argument and modifier positions, while the proximal demonstrative \textit{ini} occurs more often as the object of a preposition. These differences are not necessarily insignificant, but it is difficult to draw any strong conclusions based upon the distributions.

As expected, verbal morphology only occurred with placeholders in predicate positions, and of the placeholders in predicate position, a quarter of the instances of the dedicated filler \textit{anu} (43 of the total 178) occurred with verbal morphology, while a third of the proximal demonstratives \textit{ini} (22 of the total 62) occurred with verbal morphology. Distal demonstrative placeholders occur only once with explicit verbal morphology in the corpus. As was the case in the full corpus (\sectref{sec:corpus}), transitive verbal morphology was the most commonly employed among all fillers. The most frequent form is the dedicated filler \textit{anu} with both the AV prefix \textit{N-} and causative/applicative suffix \textit{-ka}, \textit{ng-anu-ka}.

Nominal morphology primarily included the third person possessive enclitic, and for demonstrative placeholders, this was the only morphological marking. For \textit{anu}, there were a few additional forms, but third person enclitics dominated the distribution.


\subsection{Repair properties}\label{sec:repair}
\tabref{table:place-repair} compares the relative frequencies of placeholders that are repaired with those that are not. Most striking is the fact that all forms are more likely to be left unrepaired than repaired. That is, repair only occurs in one-third of all instances of placeholders. The dedicated filler \textit{anu} is the least likely to be repaired. This is perhaps unsurprising, though, since, as opposed to the demonstratives, \textit{anu} is solely used as a filler. The demonstrative pronouns in their placeholder functions are more likely to be repaired, but are still outnumbered by instances where the placeholder is left unrepaired.


\begin{table}
\caption{Placeholder frequency counts of Besemah fillers by repair.}
\label{table:place-repair}
\begin{tabularx}{.8\textwidth}{l YY}
\lsptoprule
&  {Repaired} &  {Unrepaired}\\
\midrule
\textit{anu} & 87 & 215 \\ 
Proximal \textit{ini} & 35 & 64 \\
Distal \textit{itu} & 9 & 12 \\
\midrule
Total (422) & 131 & 291 \\
\lspbottomrule
\end{tabularx}
\end{table}

There is also a somewhat expected interaction between repair and disfluency: when disfluency is present, repair is more likely to occur. \tabref{tab:disfluency-repair} shows that placeholders from \textit{anu} are almost equally likely to be repaired as not when disfluency is present, compared to repair occurring just one-fourth of the time when there are no disfluencies. This pattern also appears to hold for demonstrative pronouns, although their use with disfluencies is relatively infrequent.

\begin{table}
\caption{Frequency counts of disfluencies with repaired and unrepaired placeholders in the subcorpus of Besemah.}
\label{tab:disfluency-repair}
\begin{tabularx}{.8\textwidth}{X@{} rr rr}
\lsptoprule
  & \multicolumn{2}{c}{ {Disfluent}} & \multicolumn{2}{c}{ {Fluent}} \\
  \cmidrule(lr){2-3}\cmidrule(lr){4-5}
  & Repaired & Unrepaired & Repaired & Unrepaired \\
  \midrule
  \textit{anu} & 32 & 36 & 57 & 178 \\ %\cline{2-3}
  Proximal \textit{ini} & 6 & 4 & 29 & 60\\
  Distal \textit{itu} & 3 & 1 & 6 & 11 \\
\lspbottomrule
\end{tabularx}
\end{table}


\subsection{Differences between \textit{anu} and the demonstrative pronoun fillers}
From the discussion above, there are few distinguishing patterns between the filler uses of \textit{anu} versus the demonstrative pronouns, aside from the overall higher frequency of \textit{anu}. One significant difference is that while all three forms are more frequently used as placeholders rather than hesitators, \textit{anu} is much more likely to be used as a hesitator than the demonstrative pronouns are. However, an important factor that has yet to be addressed is the individual preferences of speakers. The variation in these preferences is apparent from the frequencies of different forms in each of the conversations, as displayed in \tabref{tab:speakers-favoring}.


\begin{table}
\caption{Frequency counts of different forms by conversation. Conversations that are shaded favor demonstrative pronouns as fillers.}
\label{tab:speakers-favoring}
\begin{tabularx}{.8\textwidth}{l YYY}
\lsptoprule
& \textit{anu} & Proximal \textit{ini} & Distal \textit{itu}\\
\midrule
 BJM01-035-01 & 135 & -- & 14\\
 \shadecell BJM01-041-01 & \shadecell 29 & \shadecell 39 & \shadecell 2\\
 BJM01-086-01 & 34 & 7 & 1\\
 BJM01-116-01 & 103 & 2 & 2\\
 BJM01-125-01 & 102 & 12 & 3\\
 \shadecell BJM01-158-01 & \shadecell 36 & \shadecell 60 & \shadecell 7\\
\lspbottomrule
\end{tabularx}
\end{table}

This table demonstrates that in all but two conversations, the use of \textit{anu} dominates the distribution. In the two sessions where demonstrative pronouns are more frequent (BJM01-041-01, BJM01-158-01), \textit{anu} is still relatively frequent. 

Looking at the same frequencies among the 17 speakers in these six conversations reveals that a combination of age and time spent outside of the Besemah region appear to correlate with an increased use of demonstrative pronouns. Consider the frequency counts of filler use for each speaker in the subcorpus alongside some basic demographic factors as shown in \tabref{tab:subcorpus-speaker-freq}.\footnote{Although the principal number of participants in the subcorpus totaled 17, there were a total of 20 speakers who appeared in the subcorpus. The additional participants were family members or friends who momentarily joined the conversation. Two participants who fall into this category used a total of six fillers. In an effort not to skew the results with these small numbers, these participants are not included in \tabref{tab:subcorpus-speaker-freq}.} Younger speakers of the sample (born after 1975, the median birth year) tend to use more demonstrative pronouns than older speakers, especially if they spent significant time outside of the region. For older speakers (born before 1975), on the other hand, the use of \textit{anu} is far more frequent than the use of demonstrative pronouns. There are several apparent exceptions to these patterns.

\begin{table}
\caption{Frequency counts of the use of different fillers with rows sorted from the lowest proportion of \textit{anu} to the highest alongside speaker's name, year of birth, gender, and time outside of the Besemah region in years.}
\label{tab:subcorpus-speaker-freq}
\fittable{
\begin{tabular}{lllrrrrr}
\lsptoprule
   {Birth year}	&	 {Name}	&	 {Gender}	&	 {Outside region } & \textit{anu}	&	\textit{ini}	&	\textit{itu}	&	 {\% of \textit{anu}}\\
  \midrule
  1981	&	Hendi	&	m	& 5 & 1	&	20	&	1	&	5\%	\\ % 5 years for college
  1975	&	Neti	&  f &  4	&	 3	&	43	& 2	&	 6\%	\\ % 4 years in 1990s in Jakarta
  1979	&	Ujang	&	m	& 0	 & 2	&	6	&	1	&	22\%	\\ % 0 years
  1983	&	Iril	&	m	& 12 & 7	&	19	&	1	&	26\%	\\ % 12 years
  \midrule
  1983	&	Leksi	&	f	& 4	& 33	&	17	&	5	&	60\%	\\ % 4 years in Jakarta in 2000s
  %  1958	&	Danut	&	female	&	2	&	--	&	1	&	67\%	\\
  1964	&	Partiwi	&	f	& 0 & 34	&	7	&	2	&	79\%	\\ % 0? years
  1963	&	Ishan	&	m	& 0	& 50	&	--	&	12	&	81\%	\\ % 0 years
  1975	&	Lis	    &	f	& 0	& 30	&	3	&	--	&	91\%	\\ % 0 years
  1975	&	Meri	&	f	& 4	 & 38	&	2	&	1	&	93\%	\\ % 4 years (Bandung 1990s, Malaysia 2000s)
  1971	&	Jupri	&	m	& 1  &	14	&	1	&	--	&	93\%	\\ % 1 year in Jakarta 1980s
  1978	&	Susianawati	&	f	& 0 &	45	&	1	&	2	&	94\%	\\
  1951	&	Rusmawati	&	f	& 0 &	37	&	1	&	--	&	97\%	\\
  1956	&	Darsono	&	m	& 0	& 83	&	--	&	1	&	99\%	\\ % 0 years
  1978	&	Neti	&	f	& 0 & 	21	&	--	&	--	&	100\%	\\ %  0 years
  1984	&  Peter	& 	m &  0	&  18	& --	&	 --	&	 100\%	\\ % 0 years
  1972	&	Sasli	&	m	&	6 & 18	&	--	&	--	&	100\%	\\ % 6 years in the 1990s
  \lspbottomrule
\end{tabular}
}
\end{table}

Two interesting participants, Sasli and Peter, demonstrate the importance of both age and time outside the region. While Sasli was born before the median birth year, he lived outside of the region for six years, and still he exclusively uses \textit{anu}. Peter, despite being the youngest speaker of this sample, had not lived outside the region and also exclusively uses \textit{anu}. This supports the hypothesis that it is the combination of age and time outside of the region that correlates with the use of demonstrative pronouns. 

In the cases of Meri and Ujang, we see exceptional patterns. That is, Meri spent significant time outside of the region and her birth year is the same as the median. However, she very infrequently uses the demonstrative pronouns as fillers. This pattern is especially stark when compared with Neti, who was born in 1975. She spent roughly the same time outside of the region, and has the second lowest percentage of \textit{anu} use. Ujang shows the inverse of Meri, preferring to use demonstrative pronouns as fillers. He was born after the median year, but has not spent significant time outside of the region. However, with less than 10 instances of fillers, it is difficult to draw any reliable conclusions from his use of fillers.

The overall picture that these distributions produce is that the majority of speakers have a bias for the dedicated filler \textit{anu}, with a minority showing a preference for the demonstrative pronouns. Those that favored the demonstrative pronouns were born in 1975 or later with a mix of men and women. Those that favored \textit{anu} comprised a more diverse group with younger and older speakers. One possible explanation for these patterns is language contact. While the varieties of Malay/Indonesian with which Besemah is in contact have not yet been described, reports from colloquial varieties of Jakarta Indonesian show that demonstrative pronouns are most commonly employed as fillers \citep{wouk2005syntax,williams2010toward}. In fact, both \textcite{wouk2005syntax} and \textcite{williams2010toward} do not report any use of \textit{anu} in their data, even though it has been reconstructed for Proto Malayic \citep{adelaar1992protomalayic}.

\section{Conclusion}\label{sec:conclusion}
Studies of fillers, such as \textcite{enfield2003definition} and \textcite{hayashi2006crosslinguistic}, have rightly pointed out how important fillers are for interaction, serving a wide range of purposes for participants, none more important than dealing with trouble in word formulation. At the same time, interactional data are essential for understanding how fillers are employed by participants. It is especially important to investigate fillers in a range of typologically diverse languages. With few studies of Austronesian languages (with notable exceptions, such as \cite{nagaya2022tagalog,tanangkingsing2022cebuano}) , this chapter shows how fillers are used by speakers of Besemah in a relatively large corpus of conversations. In doing so, we have shown how several of the features that \citet{hayashi2006crosslinguistic} argue are defining features of fillers cannot be characterized as such in Besemah. For \citet{hayashi2006crosslinguistic}, the distinction between placeholder and hesitator fillers appears to be clear-cut. Placeholders are referential and syntactically integrated, and they are “replaced” by a more specific word. Hesitators, on the other hand, are non-referential and not syntactically integrated, and the `outcome' following the use of a hesitator (unlike repair) has little correspondence to it. 

It may be that Fox's (\citeyear[2]{fox2010introduction}) characterization that placeholders ``fulfill the syntactic projection of the turn so far, rather than simply delaying the next word due,'' better captures the distinction. We showed that `fulfilling a syntactic projection' and `delaying the next word due' are not necessarily clear in languages like Besemah with looser syntactic structures. Furthermore, such a distinction may not be consequential for speakers to carry out their interactions. Despite this, there are many instances where the use of hesitators is evident; they do not fulfill a syntactic projection and appear to simply delay the next word due. Based on this evidence, we would advocate for a position that recognizes prototypical placeholders that are morphologically marked and syntactically positioned to fulfill a projection. As fillers move farther from this prototype, losing morphological marking, occurring in less articulated syntactic positions, and being left unrepaired, they become less prototypical. Cases that we marked as indeterminate are positioned in this gray area between fillers that are clearly placeholders and those that do not fulfill a projection.

We have also shown that even though Besemah has a dedicated filler \textit{anu}, many speakers employ demonstrative pronouns as fillers with the same morphosyntactic properties and much of the same distributional properties. However, aside from the fact they are overall less frequent than \textit{anu}, there are intriguing differences in their frequencies along several dimensions. For example, (i) demonstrative pronouns rarely occur in AV, although this is common for \textit{anu}, (ii) they are less likely to occur as hesitators than placeholders, (iii) they are more likely to be repaired than \textit{anu} placeholders. While these properties are certainly interesting, the use of demonstrative pronoun fillers may be best explained by contact with other varieties of Malay/Indonesian, as evidenced by the preferences of younger speakers who have lived for some time outside of the Besemah region. In other words, few Besemah speakers utilize a roughly equal proportion of the different filler forms; they tend to strongly prefer \textit{anu} or the demonstrative pronoun \textit{ini}. 

\section*{Abbreviations}
\begin{tabularx}{.5\textwidth}{@{}lQ@{}}
1           & first person \\
2           & second person \\
3           & third person \\
\textsc{av} & A-Voice \\
\textsc{ben} & Benefactive Applicative\\
\textsc{caus} & Causative \\
\textsc{dem} & Demonstrative \\
\textsc{dist} & Distal \\
\textsc{distr} & Distributive \\
\textsc{emph} & Emphatic \\
\textsc{ext} & General Extender\\
\textsc{fill} & Filler \\
\textsc{hon} & Honorific\\
\textsc{imp} & Imperative \\
\textsc{ints} & Intensifier \\
\textsc{int} & Interrogative \\
\end{tabularx}%
\begin{tabularx}{.5\textwidth}{@{}lQ@{}}
\textsc{n.li} & Light Noun\\
\textsc{loc} & Locative Applicative\\
\textsc{mid} & Middle \\
\textsc{neg} & Negative \\
\textsc{nmz} & Nominalizer \\
\textsc{nvol} & Non-volitional \\
\textsc{pfv} & Perfective \\
\textsc{pl} & Plural \\
\textsc{prox} & Proximal\\
\textsc{pv} & P-Voice \\
\textsc{rdp} & Reduplication \\
\textsc{rec.pst} & Recent past \\
\textsc{sg} & Singular \\ 
\textsc{tag} & Tag question \\
\textsc{top} & Topic \\ 
\textsc{trun} & Truncation \\
\textsc{voc} & Vocative\\
\end{tabularx}

\section*{Acknowledgements}
We would like to thank two anonymous reviewers and the editors for valuable feedback on this chapter. We would also like to acknowledge Hendi Feriza who has provided his expertise in the transcription, translation, and analysis of the Besemah conversations for the purposes of this paper. We would like to thank our research counterpart in Indonesia, Yanti (Atma Jaya Catholic University of Indonesia), and to the National Research and Innovation Agency (BRIN) for allowing us to carry out this research. This research is based upon work supported by the National Science Foundation under Grant BCS–1911641. Any opinions, findings, and conclusions or recommendations expressed in this material are those of the author(s) and do not necessarily reflect the views of the National Science Foundation. The corpus of Besemah was compiled by the first author with funding support from a Fulbright scholarship, Blakemore Freeman Fellowship Language Grant, and Fulbright-Hays Doctoral Dissertation Research Abroad scholarship with permission to conduct research on Besemah from the Ministry of Research and Technology in Indonesia. We are especially grateful to all of the Besemah people who have participated in the conversations in the corpus and supported this research more generally, most especially, Asfan Fikri Sanaf, Kencana Dewi, Sarkani, and Hendi Feriza.

\sloppy
\printbibliography[heading=subbibliography,notkeyword=this]

\begin{paperappendix}

\section{Coding schema for quantitative study}\label{sec:coding}
\largerpage[3]
~\vspace*{-1em}
\begin{table}[h!]
\small
\label{tab:coding}
 \begin{tabularx}{1\textwidth}{>{\raggedright}p{5em} X >{\raggedright\arraybackslash}p{9em}}
  \lsptoprule
   {ID}         &  {Description} &  {Levels}\\
  \midrule
  Form 
  & What is the full morphological form of the filler? & e.g., \textit{ini=nye} \\[20pt]
  Root  
  & What is the root of the filler? & \textit{anu}, \textit{ini}, \textit{itu}\\[15pt]
  Function
  & What is the function of the filler? & {placeholder, hesitator, indeterminate}\\[20pt]
  IU {position}
  & What position within the Intonation Unit (IU) does the filler occupy? & initial, medial, final, sole\\[20pt]
  Syntactic position
  & What position in the clause does the filler occupy? & predicate, argument, modifier, PP object, adjunct\\[25pt]
  Disfluency
  & Are there any disfluencies immediately preceding or following the filler? & {truncation, sound stretch, pause, none}\\[25pt]
  Repair
  & Is the placeholder repaired? & {yes, no, NA}\\[15pt]
  Recycling 
  & Is any part of the phrase recycled? & {yes, no, NA}\\[15pt]
  Repair location 
  & How many turns after the placeholder is used is it repaired? & {0-5, NA}\\[25pt]
  Repair location 
  & How many IUs after the placeholder is used is it repaired? & {0-9, NA}\\[25pt]
  Repair initiator &
  Who initiates repair of the placeholder? & {self, other, NA}\\[20pt]
  Repairer &
  Who repairs the placeholder? & {self, other, NA}\\
  \lspbottomrule
 \end{tabularx}
\end{table}
\end{paperappendix}
\end{document}
