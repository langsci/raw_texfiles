\documentclass[output=paper,colorlinks,citecolor=brown
\ChapterDOI{10.5281/zenodo.15697581}
% ,hidelinks
% showindex
]{langscibook}
\author{Albert Ventayol-Boadaᵃᵇ\affiliation{ᵃUniversity of California, Santa Barbara  ᵇCurrent affiliation: Meta AI}}
\title{Copula or placeholder: A qualitative and quantitative analysis of \textit{ʎe}- in Kolyma Yukaghir discourse}
\abstract{Current grammatical descriptions of Kolyma Yukaghir analyze the form \textit{ʎe-} as one of the copula roots in the language.
Additionally, a variety of \textit{ʎe-}'s emerge in connected speech, which are often glossed as ``discourse particle" or ``that one", the latter suggesting a possible pronominal reading. 
Through an analysis of Kolyma Yukaghir narratives collected in the late 1990s, I show that the non-assertion uses of \textit{ʎe-} in fact represent a placeholder.
I also demonstrate that the distributions of both functions are distinct, and thus are best analyzed as being associated with different morphs rather than a single form with vaguely related meanings. 
Their synchronic resemblance, however, is probably not an accident, and the two uses are likely diachronically related. 
I suggest how the development of the placeholder from the copula is plausible through the reanalysis of the nominal interpretation of the copula as a referential expression standing in for another lexical item. 
The analysis of \textit{ʎe-} in Kolyma Yukaghir provides a new potential source of placeholders cross-linguistically and contributes to the study of discourse and grammar. 

\keywords{copula, placeholder, discourse and grammar, Kolyma Yukaghir\textsuperscript{‡}}
}


\IfFileExists{../localcommands.tex}{
   \addbibresource{../localbibliography.bib}
   % add all extra packages you need to load to this file

\usepackage{tabularx,multicol}
\usepackage{url}
\urlstyle{same}

\usepackage{listings}
\lstset{basicstyle=\ttfamily,tabsize=2,breaklines=true}

\usepackage{langsci-basic}
\usepackage{langsci-optional}
\usepackage{langsci-lgr}
\usepackage{langsci-osl}
% \usepackage{./langsci/styles/langsci-lgr}
% \usepackage{./langsci/styles/langsci-osl}
% \usepackage{langsci-gb4e}

\usepackage{tikz}
\usetikzlibrary{patterns,calc}
\pgfdeclarepatternformonly{south east lines}{\pgfqpoint{-0pt}{-0pt}}{\pgfqpoint{3pt}{3pt}}{\pgfqpoint{3pt}{3pt}}{
    \pgfsetlinewidth{0.6pt}
    \pgfpathmoveto{\pgfqpoint{0pt}{3pt}}
    \pgfpathlineto{\pgfqpoint{3pt}{0pt}}
    \pgfpathmoveto{\pgfqpoint{.2pt}{-.2pt}}
    \pgfpathlineto{\pgfqpoint{-.2pt}{.2pt}}
    \pgfpathmoveto{\pgfqpoint{3.2pt}{2.8pt}}
    \pgfpathlineto{\pgfqpoint{2.8pt}{3.2pt}}
    \pgfusepath{stroke}}
    
\usepackage{stmaryrd}
\usepackage{wasysym}
\usepackage{multirow}
\usepackage{caption}
\usepackage{subcaption}
\usepackage{mathrsfs}
\usepackage{qtree}

\usepackage{linguex}


   %pminos do not split footnotes
% \interfootnotelinepenalty=10000 %Footnote in Laporte chapters has to be split SN


%\DeclareIndexNameFormat{default}{%
%\nameparts{#1}%
%\usebibmacro{index:name}%
%{\index[names]}%
%{\namepartfamily}%
%{\namepartgiveni}%
% {}% L1
% {}% L2
%{\namepartprefix}% generates spurious space L3
%{\namepartsuffix}% generates spurious space L4
%}

%  {\DeclareIndexNameFormat{default}{%
%     \usebibmacro{index:name}{\index[names]}{#1}{#3}{#5}{#7}}}

%\DeclareIndexNameFormat{default}{%
%  \usebibmacro{index:name}{\sindex[nom]}{#1}{#3}{#5}{#7}}

%\DeclareIndexNameFormat{default}{%
%  \usebibmacro{index:name}{\sindex[person]}{#1}{#3}{#5}{#7}}
%\DeclareIndexNameFormat{default}{%
%\nameparts{#1} \usebibmacro{index:name}{\sindex[person]]}{\namepartfamily}{‌​\namepartgiven}{\nam‌​epartprefix}{\namepa‌​rtsuffix}}

%\newcommand{\smiley}{:)}

%\renewbibmacro*{index:name}[5]{%
%\usebibmacro{index:entry}{#1}%
%{\iffieldundef{usera}{}{\thefield{usera}\actualoperator}\mkbibindexname{#2}{#3}{#4}{#5}}}

% \newcommand{\noop}[1]{}

%remove for final
%\overfullrule=1mm

\newcommand{\tobi}[2]}}
\renewcommand{\S}[1]{\tobi{#1}{\textsc{*}}}

% this volume references
% puts: [this volume]
% already defined: \citetv
%\newcommand{\citepv}[1]{(\citeauthor{#1} \citeyear*{#1} [this volume])}
\newcommand{\citealtv}[1]{\citeauthor{#1} \citeyear*{#1} [this volume]}

%parentheses around example number
\newcommand{\pref}[1]{(\ref{#1})}

% in-text examples

\newcommand{\lnex}[1]{\textit{#1}} %target lang word
\newcommand{\lnlit}[1]{(lit.: `#1')} %literal reading
\newcommand{\lnlat}[1]{(#1)} % latinization
\newcommand{\lntrans}[1]{`#1'} %translation
\newcommand{\lnexl}[2]%
{\lnex{#1}{} \lnlat{#2}} % ex with latinization
\newcommand{\lnexlat}[3]{\lnex{#1}{} \lnlat{#2}{} \lntrans{#3}} % ex with latinization and tranl.

%ch01
\newcommand{\co}[1]{\mbox{\textbf{#1}}}

%ch09

\newcommand{\cyrbulg}[1]{\begin{otherlanguage*}{bulgarian}#1\end{otherlanguage*}}


%ch10
\newcommand{\nlp}{{\small NLP}}
\newcommand{\mwe}{{\small MWE}}
\newcommand{\rae}{{\small RAE}}
\newcommand{\lvc}{{\small LVC}}
\newcommand{\pos}{{\small P}o{\small S}}
%\newcommand{\todo}[1]{ \textcolor{red}{#1} }

%\renewcommand{\labelenumi}{\theenumi}
%\ainamefmt{{vv}{ll}{, ff}{, jj}} % fullname

\newcommand{\biberror}[1]{{\color{red}#1}}

\newcommand{\osenovaitem}{--~}
   %% hyphenation points for line breaks
%% Normally, automatic hyphenation in LaTeX is very good
%% If a word is mis-hyphenated, add it to this file
%%
%% add information to TeX file before \begin{document} with:
%% %% hyphenation points for line breaks
%% Normally, automatic hyphenation in LaTeX is very good
%% If a word is mis-hyphenated, add it to this file
%%
%% add information to TeX file before \begin{document} with:
%% %% hyphenation points for line breaks
%% Normally, automatic hyphenation in LaTeX is very good
%% If a word is mis-hyphenated, add it to this file
%%
%% add information to TeX file before \begin{document} with:
%% \include{localhyphenation}
\hyphenation{
    Beck-man
    Ngu-yen
    back-chan-nel
    back-chan-nels
    mo-not-o-nous
    ste-reo-typ-i-cal
}

\hyphenation{
    Beck-man
    Ngu-yen
    back-chan-nel
    back-chan-nels
    mo-not-o-nous
    ste-reo-typ-i-cal
}

\hyphenation{
    Beck-man
    Ngu-yen
    back-chan-nel
    back-chan-nels
    mo-not-o-nous
    ste-reo-typ-i-cal
}

   \boolfalse{bookcompile}
   \togglepaper[]%%chapternumber
}{}

% leipzig glosses





\renewcommand*{\glsmcols}{3}
%\newleipzig{advz}{advz}{adverbializer}
%\newleipzig{attr}{attr}{attributive}
\newleipzig{ass}{ass}{assertion}
\newleipzig{aug}{aug}{augmentative}
\newleipzig{ctx}{ctx}{contextual}
\newleipzig{dim}{dim}{diminutive}
\newleipzig{dp}{dp}{discourse particle}
\newleipzig{ef}{ef}{event-focus}
\newleipzig{ep}{(ep)}{epenthesis}
\newleipzig{ev}{ev}{evidential}
\newleipzig{freq}{freq}{frequentative}
\newleipzig{hab}{hab}{habitual}
\newleipzig{impf}{impf}{imperfective}
\newleipzig{inch}{inch}{inchoative}
\newleipzig{lnk}{lnk}{linker}
\newleipzig{pf}{pf}{participant-focus}
\newleipzig{ph}{ph}{placeholder}
\newleipzig{prop}{prop}{proprietive}
\newleipzig{seq}{seq}{sequential}
\newleipzig{unk}{unk}{unknown/unclear} 


%For the plot figures
%\usepackage{booktabs}

%\usepackage{subcaption}



\begin{document}
\maketitle
\shorttitlerunninghead{Copula or placeholder}

\graphicspath{{figures/ventayol_boada}}

% \begingroup
% \renewcommand\thefootnote{}\footnotetext{\textsuperscript{‡}The work presented here was carried out during my time at the University of California, Santa Barbara, and it is not associated with my current affiliation with Meta AI.}
% \endgroup

\section{Introduction} \label{sec_intro}

Analyzing grammatical morphs with identical phonological forms and distinct functions is a common linguistic task. 
According to \citet{Epps2008a}, at least three scenarios are possible in this context. 
First, the similar phonological shape might be a historical accident of two or more independent forms converging.
Second, the different functions might be vaguely related and thus constitute an example of polysemy. 
And third, the functions might be synchronically distinct but historically related and, thus, originate from the same source. 
This Chapter addresses the importance of considering discourse-level features when attempting to choose among these scenarios, as discourse explanations have been shown to account for a variety of grammatical forms (\citealt{Chafe1976, Clark&FoxTree2002, Davis2017, Diver1982, DuBois1987, Fox&Thompson1990, Hopper&Thompson1980, Mithun2008, Ono-et-al2000, Sankoff&Brown1976, Stern2006, Thompson&Mulac1991}; among others). 
Over time, salient discourse patterns may become grammatical \citep{Ariel2009, Couper-Kuhlen&Thompson2008, DuBois2003}. 

Specifically, I analyze the discourse distributions of \textit{ʎe}- in Kolyma Yukaghir, a form that is glossed with a variety of labels, including ``copula'', ``discourse particle'' and ``that one." 
After dividing spoken discourse into Intonation Units (IU; \citealt{Chafe1979, Chafe1992}), I analyze the semantic and pragmatic information of \textit{ʎe}- to establish two main functions: copula and placeholder. 
I then model the discourse features of each token to determine whether a polysemy analysis is possible given their distributions in discourse. 
Finally, I identify discourse contexts in which both interpretations are possible and can thus shed light into the historical development of the different uses of \textit{ʎe}-.

Overall, this Chapter shows, first, how dividing spoken discourse into IUs can illustrate the distinct discourse-dependent functions a morph carries out in context; second, how modeling the discourse features associated with each function can establish whether their distributions are distinct; and third, how identifying bridging contexts can explain the historical relationship between both functions. 
In so doing, this study contributes to the growing research on fillers and their discourse features, as well as the emergence of placeholders cross-linguistically.
It also provides a detailed account of a placeholder in Eurasia, and it offers methodological tools for linguists working with other endangered language communities in the documentation and description of their languages. 

This Chapter is organized as follows. 
\sectref{sec_lang} introduces the Kolyma Yukaghir community, provides some background information on the sociolinguistic landscape and offers a typological overview of the language.
\sectref{sec_qual} provides a qualitative discourse analysis of the form \textit{ʎe}-, and \sectref{sec_quant} models its distribution quantitatively with random forests.
\sectref{sec_origins} offers a hypothesis for the historical development of the placeholder in Kolyma Yukaghir, and \sectref{sec_conclusion} draws some conclusions.


\section{The language and its speakers} \label{sec_lang}

Kolyma Yukaghir (ISO 639-3: \textit{yux}) is one of the two extant Yukaghiric languages spoken in the Russian Arctic, the other being Tundra Yukaghir (ISO 639-3: \textit{ykg}). 
Both languages are spoken in the Kolyma River basin, in the northeastern part of the Republic of Sakha (Yakutia). 
Kolyma Yukaghir, also referred to as Southern Yukaghir, is spoken on the Kolyma mountain range, whereas Tundra Yukaghir, also known as Northern Yukaghir, is spoken on the Kolyma River mouth by the Arctic Ocean.
\figref{Map} shows the Yukaghirs' historical territory at the end of the 18\textsuperscript{th} century and their present-day locations.

\begin{figure}
\includegraphics[width=\textwidth]{Map.jpg}
\caption{Map of the current and historical territory of the Yukaghirs, adapted from \citet{Willerslev2004}. The image is copyrighted by the content publisher, John Wiley and Sons, and is reproduced here under license number 5761140519254}
\label{Map}
\end{figure}   

The origin of the exonym ``Yukaghir'' (Russian: Юкагир) is unknown \citep[21]{Maslova2003}, and it is the term that Yukaghirs typically use to refer to themselves when speaking in Russian, usually with a designation of their origin (i.e., Upper Kolyma or Lower Kolyma). 
Alternatively, the endonyms ``Odul'' and ``Wadul'' are used as self-designations in Kolyma Yukaghir and Tundra Yukaghir, respectively. 
%Both terms are likely connected to the root for `strong' and `powerful' \citep{Jochelson1926a, Nagasaki2010}. 
In what follows I use the term ``Yukaghir'' to refer to the overall ethnic group, and the term ``Yukaghiric'' as an encompassing denomination for both language varieties. 
%For each language, I use the endonyms and exonyms interchangably (i.e., `Kolyma Yukaghir' or `Odul', and `Tundra Yukaghir' or `Wadul').


\subsection{Language family affiliation}
 
%Until the 17\textsuperscript{th} century, the Yukaghirs inhabited a vast territory in the Arctic of around 600,000 square miles.
%During Russian colonization, they underwent the most rapid decline ever recorded among Siberian peoples, as a result of European-borne diseases and assimilation into neighboring groups such as the Evens, Evenks (formely Tungus), and Sakhas/Yakuts \citep{Willerslev2004}. 
%Currently, the Yukaghirs are one of the smallest ethnic groups in the area \citep{RussianFederalStateStatisticsService2022}, and the areas they inhabit have shrunk to the Kolyma Highlands on the mountain range and the Kolyma Lowlands by the Arctic Ocean. 
%\figref{Map} shows the Yukaghirs' historical territory at the end of the 18\textsuperscript{th} century and their present-day locations.

%\begin{figure}
%\centering
%\makebox[\textwidth][c]{\includegraphics{Map.jpg}}
%\caption{Map of the current and historical territory of the Yukaghirs, adapted from \citet{Willerslev2004}}
%\label{Map}
%\end{figure}   

The Yukaghiric languages are often grouped with Paleosiberian languages (termed Paleoasiatic in the Russian literature, cf. \citealp{Skorik1986, Volodin1997}). 
However, there is no linguistic cohesiveness across the languages in the group (i.e., Chukotko\hyp Kamchatkan, Nivkh, Yeniseian, and Yukaghiric). Rather, the term Paleosiberian is used to refer to the earlier presence of these groups in North Asia and the several  cultural traits shared amongst them \citep{Vajda2009}. 
%Paleosiberian peoples are typically foragers, hunters and fishers without domesticated animals other than dogs. 
%Reindeer herders, on the other hand, are considered `Neo-Siberians' as they are more recent migrants from the south, although some Paleosiberian groups have introduced reindeer herding to their lifestyle \citep{Vajda2009}.
It has been suggested that Yukaghiric might be related
to the Uralic language family \citep{Nikolaeva1988, Nikolaeva2006, Nikolaeva2020, Fortescue1998}.
This position, however, is not universally accepted \citep{Comrie1981, Abondolo1998, Aikio2014}, since establishing cognates and regular sound correspondences across Yukaghiric and Uralic remains a challenging task. 
This potential connection to the Uralic languages is part of a wider discussion about the Yukaghirs' origin and their relationship with their neighbors, both linguistically and culturally \citep{Grebenjuk&Fedorcenko2018, Gogolev-etal2022}.

Currently, the two Yukaghiric languages, Kolyma Yukaghir and Tundra Yukaghir, remain as the extant members of a linguistic group that is believed to have included the now dormant Chuvan and Omok languages \citep{Nikolaeva2008}.
Kolyma Yukaghir and Tundra Yukaghir are mutually unintelligible and exhibit significant differences in their morphosyntax, phonology, and lexicon, due in part to extensive language contact with neighboring groups and multilingualism \citep[27--28]{Maslova2003}.
%For example, loanwords for everyday vocabulary are common in Kolyma Yukaghir: ⟨оҕох⟩ \textit{oʁox} `stove' from Sakha/Yakut ⟨оһох⟩ \textit{ohox}; ⟨куриэ⟩ \textit{kurie} `fence' from Even ⟨куре⟩ \textit{kurie}; ⟨тэрикиэ⟩ \textit{terikie} `wife, old woman' from Russian ⟨старуха⟩ \textit{staruxa}).
In fact, lexical replacements are so extensive that differentiation between both language varieties may have started as far as 2,000 years ago \citep{Nikolaeva&Xelimskij1997}. 

%Writing conventions for the Yukaghiric languages did not materialize until the late 20\textsuperscript{th} century, unlike other languages of Russia for which orthographies were established much earlier. 
%It was only in 1987 that a Cyrillic-based orthography was first established for Wadul, which was later adapted for Odul \citep{Nagasaki2010}. 
%Both include additional characters and establish digraphs to represent sounds characteristic of each language (e.g., ⟨ӈ⟩ /ŋ/, ⟨ҕ⟩ /ʁ/; ⟨ль⟩ /ʎ/, ⟨нь⟩ /ɲ/ in Odul).
%Similarly, some graphs are only used in Russian loanwords (e.g. ⟨в⟩ /v/, ⟨ё⟩ /jo/).
%Examples in this Chapter are given in the practical orthography first, followed by an IPA transliteration.
%In numbered examples, the transliteration also includes morpheme boundaries. 
 
\subsection{Sociolinguistic background of Kolyma Yukaghir}

Kolyma Yukaghir is primarily spoken in the settlement of Nelemnoe, where  Yukaghirs constitute a majority \citep[20]{Maslova2003}, with a handful of speakers also living in Zyrjanka \citep{Nagasaki2010}. 
The Republic of Sakha (Yakutia) grants official status to the Yukaghir languages in the areas where Yukaghirs live, alongside with the official state languages, that is, Russian and Sakha/Yakut \citep{Grenoble2022}. 
The law, which also grants similar status to four other Indigenous languages in the Republic, aims to create the necessary conditions for their preservation and development \citep{Vinokurova-etal2022}.

Despite this legal recognition, Kolyma Yukaghir is considered critically endangered and remains the language of communication only among the oldest generation (\citealt{Grenoble2003}, but see \citealt{Kazakevich-etal2022} for a critique of vitality labels).
According to \citet[20]{Maslova2003}, in the late 1990s Russian was typically the first language for people under the age of 60, although many declared Kolyma Yukaghir as their mother tongue.
The youngest generation at the time was virtually monolingual in Russian, according to the author, although Kolyma Yukaghir had been taught in the local school in Nelemnoe since the mid 1980s. 
This pattern thus suggests a rapid decline in Kolyma Yukaghir language use and transmission. 

However, \citet[22--23]{Maslova2003} also points out that a similar pattern was already documented by \citet{Jochelson1926a} at the end of the 19\textsuperscript{th} century.
This similar attestation is striking and raises the question of the processes that have maintained language use over the span of a century, if the youngest generation were not to acquire the language, as suggested in both accounts. 
\citet{Maslova2003} hypothesizes that language users might underreport their own and others' language proficiency for various reasons.
Similarly, she observes that middle-aged community members who had previously claimed not to know the language in fact used it later in life. 

These complex sociolinguistic patterns are closely intertwined with several structural changes under way in Kolyma Yukaghir. 
These changes affect predominantly the language's morphosyntax, and they reflect a clear influence from Russian grammatical structures \citep{Matic2008influence}—an influence that is also documented in neighboring languages \citep{Grenoble2000, Pakendorf2024, Anderson2017, Kantarovich2020}.
Similarly, \citet[24]{Maslova2003} notes that the language used by each generation is clearly distinct, with younger language users making wide use of code-mixing and incorporating Russian loanwords without phonological adaptation. 
It is important to consider these changes because they are likely to involve word search processes, a common context in which speakers might deploy placeholders. 

%Put together, these structural changes can be understood as a survival strategy. 
%\citet[24]{Maslova2003} argues that each generation is faced with a choice either to reject speaking Kolyma Yukaghir because it is not their first language, or to simply speak it as they can. 
%Language users seem to favor the latter approach, where modification is accepted as a by-product. 
%This strategy is also embraced by the oldest speakers, who may have themselves made that very same decision decades ago and whose linguistic repertoire also differs from that of their parents' generation. 
%Overall, thus, language use seems to take precedence over language purity ideologies among the Yukaghirs. 



\subsection{Typological overview of Kolyma Yukaghir}
Like other languages in the Siberian linguistic area \citep{Anderson2006, Pakendorf2024}, Kolyma Yukaghir is strongly head-final, and it displays SV/AOV constituent order with nominative-accusative alignment. 
Additionally, Kolyma Yukaghir, like its sister language, is known for the grammaticalization of information structure in the languages' syntax, a property often referred to as \textit{grammatical focus}, \textit{focus system} or \textit{focus construction} \citep{Comrie1981, Krejnovich1982, Maslova1997, Maslova2003, Nagasaki2018, Nikolaeva&Xelimskij1997}.
Morphologically, Kolyma Yukaghir is a predominantly agglutinating, suffix-dominant language, with partially fusional morphology. 
Suffixes display some allomorphy due to residual vowel harmony and consonantal assimilation processes \citep{Krejnovich1982, Maslova2003, Nagasaki2010}.
Examples in this Chapter are given in the practical orthography first, followed by an IPA transliteration that also includes morpheme boundaries marked with hyphens.

The number and identity of parts of speech in Kolyma Yukaghir remains a contested issue \citep{Ventayol-Boada-et-al2023}. 
Morphologically, roots display a certain degree of flexibility and whether they are used ``verbally" (i.e., for predicating) or ``nominally" (i.e., for referring) depends on the affixes attached to them. 
The number and range of affixes with which a given root occurs varies for each use, and, thus, words exhibit different levels of morphological complexity. 
Roots used ``verbally'' have the largest number of affixal slots, some of which can be filled by a wide range of possible items (e.g., the slot for aspect). 
For the most part, boundaries between slots are well-defined, but some of the affixes for person have fused with the preceding morphemes. 
\figref{fig:verb-template} shows the template for words used ``verbally" to denote assertions;\footnote{I use the term \textit{assertion} for speech events in which ``the speaker describes or makes a claim concerning a general state of affairs' \citep[3]{Vatanen2014}. Speech events that express questions and commands are marked differently (cf. \citealt{Silverstein1976, VanValin2003}). \citet{Maslova2003} classifies the assertion-making morphemes according to valency (i.e., intransitive vs. transitive) and as to whether they highlight the event (i.e., verb-focus) or the event participants (i.e., subject-focus or object-focus). However, according to \citet{ventayol-boada-2023}, this description fails to account for a significant number of utterances in discourse in which ``intransitive'' forms are attested with two participants, and ‘transitive’ forms are found with a single participant. Since their different discourse functions remain unclear, I gloss the four forms as \Ass{} ``assertion."} only the root, assertion and person slots are obligatory.
Converbs and participles follow a different morphological template.

\begin{figure}
\small
\begin{tabular}{l|l|l|l|l|l|l}
% \lsptoprule
 {Root} & { {Aspect}} & { {Evidentiality}} & { {Number}} & { {Tense}} & { {Assertion}} & { {Person}} \\
 \hline
\multirow{6}{*}{Root} & \textit{-oo} (\Res{}) & \multirow{6}{*}{\textit{-ʎel} (\Ev{})} & \multirow{6}{*}{\textit{-ŋi} (\Pl{})} & \multirow{6}{*}{\textit{-te} (\Fut{})} &  & \multirow{6}{*}{Person}    \\
 & \textit{-ie/aa} (\Inch{}) &  &  &  & \textit{-je} & \\
 & \textit{-ej/aj} (\Pfv{}) &  &  &  & \textit{-l}  &  \\
 & \textit{-nu} (\Impf{}) &  &  &  & \textit{-me\textsubscript{E}}  &  \\
 & \textit{-nun} (\Hab{}) &  &  &  & \textit{-me\textsubscript{P}}  &  \\
 & etc. &  &  &  &  &  \\
%  \lspbottomrule
\end{tabular}
\caption{Verbal template in Kolyma Yukaghir, adapted from \citet{Maslova2001}}
\label{fig:verb-template}
\end{figure}

Roots used ``nominally'', on the other hand, have fewer slots. 
Typically, slots in the nominal template can be filled by fewer possible affixes than roots used ``verbally" and sometimes occur without affixes at all. 
In discourse, plurality is not obligatorily marked, and terms designating plural referents may not be marked with the plural suffix. 
%At the same time, the plural marker can be attested two or more times on the same root.
There is no case agreement between nominal referents and their modifiers. 
 \figref{fig:noun-template} shows the template for words used ``nominally."


\begin{figure}
\begin{tabular}{c|c|c|c|c}
% \lsptoprule
 {Root} &  {Number} &  {Possessor} &  {Size} &  {Case} \\
 \hline
\multirow{4}{*}{Root} & \multirow{4}{*}{\textit{-p/pe/pul} (\Pl{})} &  &  & \textit{-kele/gele} (\Acc{}) \\
 &  & \textit{-ki/gi} (\Third\Poss{}) & \textit{-die} (\Dim{}) & \textit{-ge} (\Loc{}) \\
 &  & \textit{-de} (\Third\Poss{})    & \textit{-tege/tke} (\Aug{}) & \textit{-get} (\Abl{}) \\
 &  &            &               & etc.  \\
% \lspbottomrule
\end{tabular}%
\caption{Nominal template in Kolyma Yukaghir, adapted from \citet{Maslova2003}}
\label{fig:noun-template}
\end{figure}

There are three demonstrative stems: \textit{tii}- (proximal), \textit{adaa}- (medial), and \textit{taa}- (distal, typically invisible) \citep[238--248]{Maslova2003}.
These stems can bear the linker -\textit{ŋ} when modifying a referent, and a different set of suffixes when used adverbially to denote source, direction or time. 
The anaphoric function is expressed by the forms \textit{tuøn}, \textit{aduøn}, and \textit{tamun}, respectively. 



\subsection{Data}
For this study I analyzed 34 of the 41 texts collected by \citet{Nikolaeva_Mayer2004} in the late 1990s. 
The texts were stripped of glosses, transliterated into Cyrillic orthography, and divided into Intonation Units (see \sectref{as-ph}) using the audio recordings.\footnote{All the corpus texts, annotations and script files are available on GitHub at \url{https://github.com/aventayolboada/placeholder_in_Odul.git}
}
In total, the corpus contains 10,258 words with 1,099 different types.
These 34 texts were narrated by seven different community members, although one speaker, Vasilij Šalugin, contributed the majority, with 22 texts.

All texts are monologic, although they differ slightly in genre. 
They include personal stories, folktales, legends, and what can be characterized as ``sharing knowledge'' narratives \citep{Smith1999}, which include knowledge and information about the Yukaghirs' worldview, ranging from traditions to games, to fortune-telling practices, among other topics.  
Many texts, however, combine elements from different genres. 
For example, legends often include retellings of historical events and events in the speaker's life, whereas knowledge narratives can incorporate personal experiences without them being the focus of the story. 


\section{Qualitative analysis: The functions of \textit{ʎe}-} \label{sec_qual}

\subsection{\textit{ʎe}- as copula}
Current grammatical descriptions of Kolyma Yukaghir analyze the form \textit{ʎe}- as the root for one of the two copulas in the language, the other being \textit{oo}- \citep{Krejnovich1982, Maslova2003, Nikolaeva2006, Nagasaki2010}.
Copulas are typically associated with connecting a referent to a predicational, identity or existential clause \citep{Citko2014}.
All three uses are attested in Kolyma Yukaghir: (\ref{ex-cop-pred}) shows a predicational use, in which the copula ascribes a property to a referent; (\ref{ex-cop-identity}) demonstrates an identity usage, by which the copula connects the terms for fishing rod in Kolyma Yukaghir and Russian; and (\ref{ex-cop-existential}) displays an existential meaning, in which the copula introduces the hero of the narrative.
Additionally, \textit{ʎe}- can function as an auxiliary following a converb, as shown in (\ref{ex-cop-aux}). 

\ea \label{ex-cop-pred}
    \glll Маҕилги атахун нугэнньэй чомоолбэн хаарэ \textbf{льэйбэдэк}. \\
    maʁil-gi ataq-u-n nugen-ɲe-j ʧomoolben qaar-e \textbf{ʎe-j-bed-ek}  \\
    coat-\Third\Poss{} two-\Ep-\Lnk{} arm-\Prop-\Ptcp{} elk skin-\Ins{} \textbf{\Cop-\Ptcp-\Nmlz-\Pred{}} \\
    \glt `His coat \textbf{was} (made) with two-finger (thick) elk skin' \\
     \hfill (\citealt{Nikolaeva_Mayer2004}, 36:10)
\z

\ea \label{ex-cop-identity}
    \glll Удочкадоон луусьии титэ \textbf{льэт}. \\
    udoʧka-doon luuçii tite \textbf{ʎe-t}  \\
    fishing.rod-\Nmlz{} Russian like \textbf{\Cop-\Cvb.\Ctx{}} \\
    \glt `Fishing rod is udochka in Russian' \\
    (\textit{Lit}. ``\textbf{Being} like an udochka in Russian") \hfill (\citealt{Nikolaeva_Mayer2004}, 50:17)
\z

\ea \label{ex-cop-existential}
    \glll Таат одун хаӈисьэ \textbf{льэльэл}. \\
    taa-t odu-n qaŋiçe \textbf{ʎe-ʎel-Ø}  \\
    there-\Adv.\Abl{} Yukaghir-\Lnk{} hunter \textbf{\Cop-\Ev-\Ass.\Intr.\Ef.\Tsg{}} \\
    \glt `There \textbf{was} a Yukaghir hunter' \hfill (\citealt{Nikolaeva_Mayer2004}, 21:1)
\z

\ea \label{ex-cop-aux}
    \glll Йугудин \textbf{льэйэ}. \\
    jug-u-din \textbf{ʎe-je}  \\
    kiss-\Ep-\Cvb.\Purp{} \textbf{\Cop-\Ass.\Intr.\Ef.\Fsg{}} \\
    \glt `I wanted to kiss (her)' (\textit{Lit}. `I \textbf{was} to kiss') \\
    \hfill (\citealt{Nikolaeva_Mayer2004}, 40:70)
\z

The copula typically occurs in clause-final position, as expected in a strongly head-final language.  
Additionally, the referent connected to the predicational, identity or existential clause typically bears no overt morphology, except when \textit{ʎe}- is marked with one of the participant-focus assertion forms. 
In these cases, the referent is typically marked with the predicative case marker, as shown in (\ref{ex-cop-ref-with-pred}). 

\ea \label{ex-cop-ref-with-pred}
    \glll Мит аархаа иркин шлюпкак \textbf{льэл}. \\
    mit aarqaa irk-i-n ʃlʲupka-k \textbf{ʎe-l}  \\
    \Fpl{} at one-\Ep-\Lnk{} boat-\Pred{} \textbf{\Cop-\Ass.\Intr.\Pf.\Tsg{}} \\
    \glt `There \textbf{was} a boat by us' \hfill (\citealt{Nikolaeva_Mayer2004}, 50:77)
\z

However, the analysis of \textit{ʎe}- as a copula is striking given some of its uses attested in discourse.
For example, in (\ref{ex2}) the first token of \textit{ʎe}- is marked with an assertion but it co-occurs with a case marking form on the neighboring pronoun \textit{tamun}, which refers to the personification of the devil into a girl, not typically associated with copulas (i.e., accusative case). 
Similarly, the second \textit{ʎe}- token does not signal (and it is not marked for) an assertion at all and takes case morphology, which is typically associated with the nominal class. 
%Example (\ref{ex2}) comes from the story ``The shaman Staryj", in which the devil has personified into a girl who is sitting and singing on top of a tree; that is the referent that \textit{tamun} alludes to. 
However, \citet{Nikolaeva_Mayer2004} gloss both uses as a copula. 

\ea \label{ex2}
    \glll Тамунгэлэ \textbf{льэльэлум} тудэ \textbf{льэгэлэ} истриэлагэлэ миндэллэ табудэ дьэ айиильэлум. \\
    tamun-gele \textbf{ʎe-ʎel-u-m} tude \textbf{ʎe-gele} istriela-gele min-delle tabud-e ʤe ajii-ʎel-u-m \\
    that-\Acc{} \textbf{\Cop?-\Ev-\Ep-\Ass.\Tr.\Ef.\Tsg{}} \Tsg.\Gen{} \textbf{\Cop?-\Acc{}} arrow-\Acc{} take-\Cvb.\Seq{} that-\Ins{} \Dp{} shoot-\Ev-\Ep-\Ass.\Tsg{}  \\
    \glt `He took an arrow and shot her' \hfill (\citealt{Nikolaeva_Mayer2004}, 23:5)
\z


Additionally, an identical form \textit{ʎe}- is glossed as ``that one'' by \citet{Nagasaki2010}, as shown in (\ref{ex3}) from the folktale ``Why the crow is black." 
This analysis suggests the existence of a pronominal \textit{ʎe}-, but she does not include the form in the list of pronouns  in the language.
Like the second \textit{ʎe}- in (\ref{ex2}), the token below is also marked with case (i.e., dative). 

\ea \label{ex3}
    \glll Пугэл, пугэл, пугэл, \textbf{льэӈин} миэстэӈин мэрэйӈи. \\
    puge-l puge-l puge-l \textbf{ʎe-ŋin} mieste-ŋin mer-ej-ŋi-Ø  \\
    be.hot-\Ptcp{} be.hot-\Ptcp{} be.hot-\Ptcp{} \textbf{that.one-\Dat{}} place-\Dat{} fly-\Pfv-\Third\Pl-\Ass.\Intr.\Ef \\
    \glt `They flew away to the hot, hot, hot place' \hfill (\citealt[252]{Nagasaki2010})
\z

Finally, there appears to be a discourse particle that also takes the shape \textit{ʎe}, based on \citet{Nikolaeva_Mayer2004}'s glossing. 
Example (\ref{ex4}) from the story ``Tobacco", in which the speaker narrates his experience with this substance from an early age, illustrates a case of such usage.
Unlike previous examples, \textit{ʎe} in (\ref{ex4}) bears no additional morphology. 


\ea \label{ex4}
    \glll Мэт чай \textbf{льэ} оожэ, дэӈдэйэ. \\
    met ʧaj \textbf{ʎe} ooʒe-Ø deŋ-de-je  \\
    \Fsg{} tea \textbf{\Dp{}} drink-\Ass.\Tr.\Ef.\Fsg{} eat-\Unk-\Ass.\Intr.\Ef.\Fsg{} \\
    \glt `I drank tea and ate' \hfill (\citealt{Nikolaeva_Mayer2004}, 34:64)
\z

The label `discourse particle', however, is not particularly insightful as to the functions \textit{ʎe} carries out in (\ref{ex4}).
Unraveling its functions is hampered by \citet{Nikolaeva_Mayer2004}'s additional list of four discourse particle forms that start with \textit{ʎe}.
These are: \textit{ʎege}, \textit{ʎegen}, \textit{ʎedegen}, and \textit{ʎelek}. 
It is plausible to assume that these four forms are morphologically complex.
What remains after singling out \textit{ʎe}- in these four forms coincides with affixes found in the nominal paradigm: \textit{-ge}, \textit{-gen} and \textit{-lek} are the locative, prolative and predicative cases, respectively; whereas \textit{-de-} marks 3\textsuperscript{rd} person possession. 
Thus, all these `discourse particles' can be seen as involving the form of \textit{ʎe}- followed, in some cases, by case marking, a description that is strikingly similar to what is attested in examples (\ref{ex2}-\ref{ex3}) above despite the different glosses offered: copula in (\ref{ex2}) and `that one' in (\ref{ex3}). 
These structural similarities suggest that these additional uses of \textit{ʎe}- are related but non-copular. 
The question as to what their functions might be, however, remains unanswered.

\subsection{\textit{ʎe}- as placeholder} \label{as-ph}
One possibility to investigate the non-copular uses of \textit{ʎe}- is to turn to discourse and analyze its semantic and pragmatic components through extended stretches of naturalistic discourse.
Discourse-level explanations have been shown to account for a variety of grammatical forms, including relative clauses \citep{Sankoff&Brown1976, Fox&Thompson1990}, transitivity \citep{Hopper&Thompson1980, Thompson&Hopper2001}, case and grammatical relations \citep{Diver1982, Ono-et-al2000}, ergativity \citep{DuBois1987}, epistemic parentheticals \citep{Thompson&Mulac1991}, hesitators \citep{Clark&FoxTree2002}, reflexives \citep{Stern2006}, dependency markers \citep{Mithun2008}, adverbial clitics \citep{Davis2017}, to name but a few. 

To do that, a useful first step is to divide texts into Intonation Units (IU; \citealt{Chafe1979, Chafe1992}), defined as “a stretch of speech uttered under a single coherent intonation contour” (\citealt[47]{DuBoisetal1993}), or as \citet[29]{Chafe1994} puts it, the “spurts of language” in which speakers typically produce speech.
The boundaries of IUs are defined in terms of prosodic features and, although they tend to overlap with the boundaries of syntactic units \citep{DuBois2003, Himmelmann2022}, IU boundaries are established by identifying a variety of phonetic cues. 
These include: pauses, pitch resetting, lengthening, and tempo shifts on the audio recordings \citep{Chafe1992, DuBoisetal1993}.
These criteria make IUs identifiable cross-linguistically \citep{Himmelmann-etal2018, Troiani2023}.

The role of \textit{ʎe}- indeed becomes more apparent when dividing \citet{Nikolaeva_Mayer2004}'s texts into IUs. 
Consider example (\ref{ex2}) above, reprinted below in IUs as (\ref{ex5-2inIUs}). 
Two features become apparent. 
First, the form \textit{ʎe}- appears in both cases at the end of an IU, and, in both cases, it is followed by a substantial pause: 0.4 and 0.5 seconds, respectively. 
The second important feature is that the morphology attested on \textit{ʎe}- is also found on two lexical items that appear later in the discourse. 
These are: \textit{istriela} `arrow' and \textit{ajii} `to shoot'. 
Put together, these two features suggest that \textit{ʎe}- in (\ref{ex5-2inIUs})\footnote{In the examples, each new line represents a new IU, and pauses are given in seconds within parentheses. Additionally, I use the following symbols from \citet{DuBoisetal1993}:
\begin{itemize}
\setlength\itemsep{0em}
\item[.] Final intonation
\item[,] Continuing intonation
\item[—] Truncated/abandoned IU
\end{itemize}
} is working as a placeholder. 
The same analysis can be made for (\ref{ex3}-\ref{ex4}), in which \textit{ʎe}- is also found with the same morphology as lexical items that appear immediately after in discourse: the dative case in \textit{mieste} `place' and assertion zero-morpheme in \textit{ooʒe} `to drink'.

\begin{exe}
\ex \label{ex5-2inIUs}
   \glll Тамунгэлэ \textbf{льэльэлум}, \\
    tamun-gele \textbf{ʎe-ʎel-u-m} \\
    that-\Acc{} \textbf{\Ph-\Ev-\Ep-\Ass.\Tr.\Ef.\Tsg{}} \\
    \glt `He \textbf{watchamacallited} that'

\sn (0.4)

\sn
   \glll тудэ \textbf{льэгэлэ}, \\
    tude \textbf{ʎe-gele} \\
    \Tsg.\Gen{} \textbf{\Ph-\Acc{}} \\
    \glt `His \textbf{watchamacallit}'

\sn (0.5)

\sn 
 \glll \uline{истриэлагэлэ} миндэллэ табудэ дьэ, \\
    \uline{istriela-gele} min-delle tabud-e ʤe \\
    \uline{arrow-\Acc{}} take-\Cvb.\Seq{} that-\Ins{} \Dp{} \\
    \glt `After taking (his) \uline{arrow}, with that'

\sn
 \glll \uline{айиильэлум}. \\
    \uline{ajii-ʎel-u-m} \\
    \uline{shoot-\Ev-\Ep-\Ass.\Tr.\Ef.\Tsg{}}  \\
    \glt `He \uline{shot} (her)' \hfill (\citealt{Nikolaeva_Mayer2004}, 23: IU 26--31)

\end{exe}

\citet[37]{Hayashi&Yoon2010} define placeholder in the following terms: ``(i) it is a referential expression that is used as a substitute for a specific lexical item that has momentarily eluded the speaker (and which is often specified subsequently as a result of a word search), and (...) (ii) it occupies a syntactic slot that would have been occupied by the target word, and thus constitutes a part of the syntactic structure under construction."
As mentioned, the uses of \textit{ʎe}- in (\ref{ex5-2inIUs}) fulfill both properties of this definition, since both tokens anticipate a word that is retrieved later in discourse (i.e., `arrow' and `shoot'), and both are integrated in their syntactic slot—so much so that they are already marked with the relevant morphology of the lexical item they stand in for.

Since roots in Kolyma Yukaghir display a certain degree of polycategoriality and can be used ``verbally" or ``nominally" depending on what affixes attach to them \citep{Ventayol-Boada-et-al2023}, it is not surprising that \textit{ʎe}- in (\ref{ex5-2inIUs}) can fill in the place of a word used for predicating or for referring. 
The ability of a single placeholder to stand in for multiple categories (especially nouns and verbs, but sometimes also adjectives or adverbs) has been found in other languages of Eurasia (e.g., Nganasan, Ulcha and Udihe; \citealt{Podlesskaya2010}), as well as Maliseet-Passamaquoddy \citep{LeSourd2003} and Indonesian \citep{Wouk2005}. 


More interesting is that \textit{ʎe}- is attested substituting demonstratives, since demonstratives are often the source of placeholders \citep{Hayashi&Yoon2010, Podlesskaya2010}. 
Demonstrative targets are not the most common, but the texts collected by \citet{Nikolaeva_Mayer2004} display 10 such cases. 
In example (\ref{ex6-dem}), \textit{ʎe}- bears the linker -\textit{ŋ}, which only attaches to demonstratives to grammatically mark their relationship with their referent.
 

\begin{exe}
\ex \label{ex6-dem}
    \glll \textbf{льэӈ}, \\
    \textbf{ʎe-ŋ} \\
    \textbf{\Ph-\Lnk{}} \\
    \glt `\textbf{Whatchamacallit}'

\sn   (1.2)

\sn
    \glll \uline{таӈ} пулут, \\
    \uline{ta-ŋ} pulut \\
    \uline{that-\Lnk{}} old.man \\
    \glt `\uline{That} old man'

\sn
    \glll отулгэ йахайэ,  \\
    otul-ge jaqa-je \\
    camp-\Loc{} reach-\Ass.\Intr.\Ef.\Fsg{} \\
    \glt `I arrived at the camp (of that old man)' \\
    \null \hfill (\citealt{Nikolaeva_Mayer2004}, 34: IU 201--204)
\end{exe}

In addition to the variety of categories \textit{ʎe}- can stand in for, examples (\ref{ex5-2inIUs}) and (\ref{ex6-dem}) also show that the placeholder fully mirrors the grammatical categories marked on its target. 
However, this need not be the case, since instances of zero or partial mirroring are also attested. 
Consider example (\ref{ex7-partial-mirroring}) below, from the story ``A game", in which the speaker narrates the rules of a four-hand game between two people. 
In this excerpt, the narrator is engaging in a word search while, at the same time, putting together the relevant morphosyntactic pieces as they relate to the ``pinkie finger." 
Specifically, he is trying to recall \textit{ɲumu}- `to press' or `to grab', which has eluded him temporarily, and to mark it with the relevant morphology. 
A first placeholder is uttered after a 0.9-second pause, but it does not bear any morphological marking.\footnote{The lack of morphology on the first \textit{ʎe}- in (\ref{ex7-partial-mirroring}) raises the question as to whether it could be best analyzed as a hesitative. The analysis is hampered by the existence of zero-morphemes both in the nominal case system and the verbal assertion paradigm (cf. the reanalysis of example (\ref{ex4}) at the beginning of \sectref{as-ph}). In total, I identified 26 examples of \textit{ʎe}- targeting a lexical item that does bear a zero-morpheme, which represents an 8.5\% of the total number of placeholders. Example (\ref{ex7-partial-mirroring}) is the only one with two \textit{ʎe}-'s produced sequentially.}
Immediately after, a second placeholder is uttered, this time with most morphological information included. 
At this point, the morphosyntactic scaffolding is almost complete, but the lexical retrieval has not succeeded yet. 
After a second-long pause, the lexical item \textit{ɲumu}- is finally produced, with the addition of the causative and perfective suffixes. 


\begin{exe}
\ex \label{ex7-partial-mirroring}
    \glll Ай тудэ йукоол пиэдисьэ, \\
    aj tude juk-oo-l piediçe \\
    also \Tsg.\Gen{} small-\Res-\Ptcp{} finger \\
    \glt `Also, his pinkie finger'

\sn (0.9)

\sn 
    \glll ай \textbf{льэ}, \\
    aj \textbf{ʎe} \\
    also \textbf{\Ph{}} \\
    \glt `Also, \textbf{whatchamacallit}'

\sn
    \glll \textbf{льэнульэлмэлэ}. \\
    \textbf{ʎe-nu-ʎel-mele} \\
    \textbf{\Ph-\Impf-\Ev-\Ass.\Tr.\Pf.\Tsg{}} \\
    \glt `He \textbf{whatchamacallits}'

\sn (1.0)

\sn 
    \glll \uline{ньумушэйнульэлмэлэ}. \\
    \uline{ɲumu-ʃ-ej-nu-ʎel-mele} \\
    \uline{press-\Caus-\Pfv-\Impf-\Ev-\Ass.\Tr.\Pf.\Tsg{}} \\
    \glt `He \uline{grabs} (his pinkie finger too)'  (\citealt{Nikolaeva_Mayer2004}, 43: IU 40--45)
\end{exe}


However, the search for the target lexical item is not always successful, either because the relevant item cannot be retrieved or because it simply does not exist.
Thus, speakers might sometimes deem the placeholder as a sufficient reference in a given communication context. 
\citet{Podlesskaya2010} calls this use of placeholders as an ``approximate naming" function. 
This use can be seen in (\ref{ex8-no-retrieval}) below.
This example comes from ``The singing girl" story, which retells the events also described in the story ``The shaman Staryj" mentioned above, albeit from the perspective of a different community member.
In this version, a statue was erected in the Popovka river basin where the demonic girl was killed. 
However, this statue was later lost. 
In (\ref{ex8-no-retrieval}), the placeholder stands in for the statue, but the lexical item is never retrieved. 
These are, in fact, the very last IUs in the narrative. 
Rather, the speaker considers the placeholder to be enough information given that the referent has been established and is the focus of discussion. 

\begin{exe}
\ex \label{ex8-no-retrieval}
    \glll Шаал оҕоой, \\
    ʃaal oʁ-oo-j \\
    tree stand-\Res-\Ass.\Intr.\Ef.\Tsg{} \\
    \glt `The tree stands [there]'

\sn (0.8)

\sn
    \glll это, \\
    eto \\
    this \\
    \glt `This'

\sn (1.0)

\sn 
    \glll \textbf{льэги} өйльэ. \\
    \textbf{ʎe-gi} øjʎe \\
    \textbf{\Ph-\Third\Poss{}} \Neg{} \\
    \glt `There is no \textbf{whatchamacallit}'

\sn (0.8)

\sn 
    \glll И всё. \\
    i vsjo \\
    and all \\
    \glt `That is the end' \hfill (\citealt{Nikolaeva_Mayer2004}, 28: IU 305--311)
\end{exe}

Most likely the use of the Russian demonstrative \textit{eto} (`this') in (\ref{ex8-no-retrieval}) is also an instance of deploying a placeholder, since that is the main placeholder used in Russian \citep{Podlesskaya2010}. 
The second-long pause following \textit{eto} suggests that a word search may have been carried out, but only the native placeholder was found.
At the time of speaking, however, it was deemed sufficient. 
An analysis of borrowed or code-switched placeholders and their relationship with native forms is clearly called for, but is beyond the scope of this chapter.
See \citet{chapters/visser} and \citet{chapters/klyachko} for more observations on borrowed placeholders in Kalamang and Tungusic languages, respectively. 

Example (\ref{ex8-no-retrieval}) also reveals that placeholders are not always followed by an IU boundary, as (\ref{ex5-2inIUs}) through (\ref{ex7-partial-mirroring}) might suggest. 
In fact, the placeholder and its target can be found in the same IU, as shown in (\ref{ex9-same-IU}), where \textit{ʎe}- is immediately followed by the target `Russian house' (i.e., the Yukaghir term for the town of Vekhnekolymsk). 
The co-occurrence of placeholder and target in the same IU is not entirely surprising; as \citet[63]{Chafe1994} points out, ``people sometimes revise their choice of wording while an intonation unit is already in progress." 

%This excerpt below is from ``The mouse" story, in which the speaker narrates the misfortunes that struck his family after a mouse was found eating their clothes. 


\begin{exe}
\ex \label{ex9-same-IU}
    \glll Мэт атахун, \\
    met ataq-u-n \\
    \Fsg{} two-\Ep-\Lnk{} \\
    \glt `My two'

\sn (0.3)

\sn
    \glll чаачаа эрэ, \\
    ʧaaʧaa ere \\
    elder.brother only \\
    \glt `Elder brothers only'

\sn (0.9)

\sn 
    \glll \textbf{льэгэ} луусьиин \uline{нумөгэ} модоӈи.  \\
    \textbf{ʎe-ge} luuçii-n \uline{numø-ge} modo-ŋi-Ø \\
    \textbf{\Ph-\Loc{}} Russian-\Lnk{} \uline{house-\Loc{}} live-\Tpl-\Ass.\Intr.\Ef{} \\
    \glt `Lived in \textbf{whatchamacallit}, in \uline{Verkhnekolymsk}' \\
    \null \hfill (\citealt{Nikolaeva_Mayer2004}, 49: IU 10--15)
\end{exe}

To summarize, dividing spoken discourse into IUs helps to reveal that the non-copular use of \textit{ʎe-} is indeed an instance of a placeholder. 
As such, it can stand in for a variety of lexical classes (i.e., at least nouns, verbs, and demonstratives) and typically mirrors, either partially or fully, the grammatical categories marked on the target when it is retrieved after a word search. 
However, speakers may deem at times that the placeholder is sufficient and do not seek further clarification of the intended target. 
Finally, the placeholder can be followed by an IU boundary and maybe a pause, although the target may be recovered immediately and thus be integrated in the IU in progress.  


\subsection{Characterizing \textit{ʎe-}}

At this point, a question remains: given their identical phonological form, what is the relationship between these distinct functions (i.e., copula and placeholder)? 
As \citet{Epps2008a} points out, this is the most recurrent issue that descriptive linguists face. 
According to her, there are three potential scenarios worth considering in this situation: is the resemblance a historical accident of two or more independent forms converging into the same phonological shape? 
Is it a case of polysemy (i.e., a single form with meanings vaguely related)? 
Or is it perhaps the case that, synchronically, their functions have diverged so much that there is no longer a discernible semantic relation, even though the forms are historically related?

%This is a nontrivial question, because both functions often co-occur in discourse, at times bearing the same morphology. 
%Consider the example below in (\ref{ex10-both-functions}) from the story ``The first lesson", in which the speaker narrates, among other things, his experience at the boarding school during his childhood. 
%In this excerpt, the narrative describes the activities that he and his fellow classmates would engage in after classes had finished, which included fishing and playing around the school premises. 

%In (\ref{ex10-both-functions}), the non-finite form \textit{ʎe-t} appears twice.
%Its function as a copula or placeholder can be disambiguated based on the prototypical role(s) of each category.  
%For the copula, these are connecting a referent to a predicational, identity or existential clause \citep{Citko2014}, whereas the placeholder role is to momentarily replace a target item in its syntactic slot \citep{Hayashi&Yoon2010}.  
%Based on that, the first token is best analyzed as a copula, where the speaker establishes the identity between referents by providing the Russian term (i.e., \textit{udoʧka}) of an introduced Kolyma Yukaghir lexical item (i.e., \textit{ʧumuçe}, `fishing rod').
%The second token, on the other hand, stands in for a lexical item that has momentarily eluded the speaker and is retrieved in a subsequent IU.
%The morphology in this example partially mirrors the morphology of the target: \textit{nug-u-nu-t}. 
%In this case, there are no predicational, identity or existential clauses for a copula analysis. 

%\begin{exe}
%\ex \label{ex10-both-functions}
%    \glll чумусьоорэт модаанундьиили, \\
%    ʧumuç-oo-re-t mod-aa-nun-ʤiili \\
%    fishing.rod-\Res-\Unk-\Cvb.\Ctx{} sit-\Inch-\Hab-\Ass.\Fpl{} \\
%    \glt `We used to sit fishing"
    
%\sn (1.0)

%    \glll удочкадоон, \\
%    udoʧka-doon \\
%    fishing.rod-\Nmlz{} \\
%    \glt `Udochka"
    
%\sn (0.6)

%    \glll луусьии титэ \textbf{льэт},\\
%    luuçii tite \textbf{ʎe-t} \\
%    Russian like \textbf{\Cop-\Cvb.\Ctx{}} \\
%    \glt `\textbf{Being} like (an udochka) in Russian"
    
%    \glll чумусьэ миндэллэ. \\
%    ʧumuçe min-delle \\
%    fishing.rod take-\Cvb.\Seq{} \\
%    \glt `After taking fishing rods" (i.e., ``We used to sit fishing with fishing rods, it is like an `udochka' in Russian")
    
%\sn (1.5)

%    \glll иркидьэ йуөт таатмиэй йуөлугэ, \\
%    irk-i-ʤe juø-t taa-t-mie-j juø-lu-ge \\
%    one-\Ep-\Freq{} see-\Cvb.\Ctx{} there-\Adv.\Abl-\Unk-\Ptcp{} see-\Fsg-\Loc{} \\
%    \glt `Once seeing, when I saw there"
    
%\sn (0.5)

%    \glll йоодоллэ чэмэйдэллэ. \\
%    joodo-lle ʧeme-j-delle \\
%    play-\Cvb.\Seq{} finish-\Pfv-\Cvb.\Seq{} \\
%    \glt `After having finished playing"
    
%\sn (1.1)

%    \glll наар, \\
%    naar \\
%    always \\
%    \glt `Always"
    
%\sn (1.4)

%    \glll \textbf{льэт}, \\
%    \textbf{ʎe-t} \\
%    \textbf{\Ph-\Cvb.\Ctx{}} \\
%    \glt `\textbf{Whatchamacallit}..."
    
%\sn (1.5)

%    \glll ньэ \uline{нугунут}. \\
%    ɲe \uline{nug-u-nu-t} \\
%    \Recp{} \uline{find-\Ep-\Impf-\Cvb.\Ctx{}} \\
%    \glt `\uline{Finding} one another"
    
%\sn (1.6)

%    \glll аҕи-д-уу-ну-т. \\
%    aʁi-d-uu-nu-t \\
%    hide-\Unk-\Pass-\Impf-\Cvb.\Ctx{}  \\
%    \glt `Hiding" (i.e. ``Once when we finished playing we got together and hid") \\
%    \null \hfill (\citealt{Nikolaeva_Mayer2004}, 50: IU 115-139)

%\sn (1.3)
%
%    \glll мит льэ — \\
%    mit ʎe \\
%    \Fpl{} \Ph{} \\
%    \glt `Our whatchamacallit"
%    
%    \glll мит йоодо-ллэ миэстэ, \\
%    mit joodo-lle mieste \\
%    \Fpl{} play-\Cvb.\Seq{} place \\
%    \glt `Our playing place"
%    
%\sn (1.0)
%
%    \glll школалэк, \\
%    ʃkola-lek \\
%    school-\Pred{} \\
%
%    \glt `It was the school"
%    
%    \glll клубэк, \\
%   klub-ek \\
%    club-\Pred{}\\
%    \glt `It was the club"
%    
%    \glll магазиинэк. \\
%    magaziin-ek \\
%    shop-\Pred{}\\
%    \glt `It was the shop"

%\end{exe}

%\begin{exe}
%\ex \label{ex10b-teasing-forms}
%    \glll тоуко чугө \textbf{льэ}дэйнэ, \\
%    touko ʧugø \textbf{ʎe}-de-jne \\
%    dog road \textbf{\Cop}-\Third\Poss-\Cvb.\Cond{} \\
%    \glt `If there were dog traces"
%    
%\sn (1.6)
%
%\sn
%    \glll то, \\
%    to \\
%    then \\
%    \glt `Then"
%    
%\sn (0.3)
%
%\sn 
%    \glll \textbf{льэ}ӈин одулӈин, \\
%    \textbf{ʎe}-ŋin odul-ŋin \\
%    \textbf{\Ph}-\Dat{} Yukaghir-\Dat{} \\
%    \glt `To whatchamacallit, to a Yukaghir"
%
%\sn (2.1)
%
%\sn 
%    \glll одулӈин хонтэйэк. \\
%    odul-ŋin qon-te-je-k \\
%    Yukaghir-\Dat{} go-\Fut-\Ass-\Ssg{} \\
%    \glt `You would get married to a Yukaghir' \\
%    \null \hfill (\citealt{Nikolaeva_Mayer2004}, 37: IU 54-60)
%\end{exe} %``Fortune telling"

As shown in examples (\ref{ex5-2inIUs}) through (\ref{ex9-same-IU}), the two uses of \textit{ʎe}- can be identified under the prototypical roles of copula and placeholder and, thus, are best described as synchronically distinct. 
To investigate whether a polysemy analysis is still possible in this scenario, one possibility is to analyze their distributions in discourse. 
Specifically, one can ask whether each function is associated with different discourse features, under the assumption that related senses of a word display similar distributions. 
%For example, in (\ref{ex10-both-functions}) both forms are followed by an IU boundary, and the overall intonation contour of the IU is continuing. 
%Both IUs where \textit{ʎe}- is present are preceded by a pause, although its length varies substantially (i.e., 0.6 seconds for the copula, and 1.4 seconds for the placeholder). 
%However, the placeholder is also followed by a pause, whereas the copula is not. 
%Similarly, the placeholder is the single element in its IU, while the copula co-occurs with two additional elements. 
Discourse features, such as intonation contour, preceding and following pause lengths or number of words in an IU, can offer predictions to characterize each use of \textit{ʎe}-. 
These predictions can be made explicit through the use of quantitative methods, which offer a probabilistic characterization of the discourse features that tend to co-occur with each of the functions of \textit{ʎe}-, and, crucially, evaluate whether or not the features in each case are statistically distinct from each other.
\sectref{sec_quant} provides an overview of the methods and data used in the quantitative analysis, as well as the variables and their predictions.

\section{Quantitative analysis} \label{sec_quant}

\subsection{Data} \label{quant_data}

From the texts collected by \citet{Nikolaeva_Mayer2004} that I analyzed, I extracted all 453 tokens of \textit{ʎe}-.
I annotated them for their function (i.e., copula vs. placeholder) through a qualitative analysis of the discourse context they appear in. 
The criteria to assign a token to a function were based on the prototypical role(s) associated with each function: for the copula, to connect a referent to a predicational, identity or existential clause \citep{Citko2014}; for the placeholder, to momentarily replace a target in its syntactic slot \citep{Hayashi&Yoon2010}. 
In total, the tokens include 148 copulas and 305 placeholders.  
The placeholder tokens show substantial variation in terms of their transparency as placeholders. 
Of the 305 placeholders, 219 tokens were followed by the target lexical item within the next five IUs (i.e., 71.8\%).
Of these, in 146 examples the morphological categories marked on the target were fully mirrored on the placeholder (i.e., 66.6\%). 

Overall, the copula occurs 14.43 times per 1,000 words, although with substantial differences across the seven speakers: from no copulas uttered by Anna Šadrina to 31.35 copulas per 1,000 words by Fevron'ja Šalugina.
As for the placeholder, it occurs 29.73 times per 1,000 words.
Here, too, are important differences across the seven speakers, ranging from 17.28 placeholders per 1,000 words by Ivan Dolganov to 46.23 by Dmitrij Djačkov. 
Similar interspeaker variation is attested in Besemah \citep{chapters/mcdonnell_billings}, Dalabon \citep{chapters/ponsonnet} Kalamang \citep{chapters/visser}, and Negidal \citep{chapters/pakendorf}. 

The placeholder frequency represents a much higher average than the reported frequencies in Russian elicited narratives (as reported in \citealt{Podlesskaya2010} from \citealt{Podlesskaya&Kibrik2006}) and Mandarin conversations \citep{Zhao&Jurafsky2005}, with 5 and 6.68 placeholders per 1,000 words respectively. 
The overall use of placeholders in Kolyma Yukaghir discourse, however, is presumably even higher, since these counts exclude the use of Russian placeholders, which are also attested, as shown in (\ref{ex8-no-retrieval}), but not analyzed systematically here. 
It is possible that this higher use of placeholders reflects the complex sociolinguistic patterns in the Kolyma Yukaghir community.
The attrition and interference from Russian underlying the strategy of ``survival through modification' \citep[24]{Maslova2003} creates shifting linguistic repertoires across generations.
In this context, words might temporarily elude speakers more often, thus explaining a higher need to deploy a placeholder when compared to a non-endangered language. 

%\begin{exe}
%\ex \label{ex11-former8}
%    \glll шаал оҕоой, \\
%    ʃaal oʁ-oo-j \\
%    tree stand-\Res-\Ass.\Tsg{} \\
%    \glt `The tree stands (there)"
    
%\sn (0.8)

%\sn
%    \glll это, \\
%    eto \\
%    this \\
%    \glt `This"
    
%\sn (1.0)

%\sn 
%    \glll \textbf{льэ}ги өйльэ. \\
%    \textbf{ʎe}-gi øjʎe \\
%    \textbf{\Ph}-\Third\Poss{} \Neg{} \\
%    \glt `There is no whatchamacallit"
    
%\sn (0.8)

%\sn 
%    \glll и всё. \\
%    i vsjo \\
%    and all \\
%    \glt `That is the end' \hfill (\citealt{Nikolaeva_Mayer2004}, 28: IU 305-311)
%\end{exe}


\subsection{Methods}

To predict the probability of \textit{ʎe}- functioning as a copula or placeholder based on its discourse context, I retrieved six discourse features for each token. 
These features include: the position of \textit{ʎe}- in the IU (hence \textit{match position}), the number of words in the IU, the IU intonation contour, the lengths of preceding and following pauses (if any), and the speaker of the text in which the token was found.
The match position was later normalized to a range between 0 and 1, where 0 indicates the match was in the first position, and 1 marks the last position. 
For tokens which were the only word in the IU, the match position was normalized to 1. 
While coding the match position in such examples as 0 is also possible, the alternative is more informative, because it captures that the match is followed by an IU boundary like other IU-final matches.
The goal is to measure how each function co-varies with each feature (i.e., to assess the relationship between the two uses of \textit{ʎe}- by examining how changes in the discourse features are associated with each function). 

These discourse features yield different expectations for each function. 
First, the copula is expected to occur later in an IU than the placeholder, and thus to be associated with a higher value of match position. 
Kolyma Yukaghir is strongly head-final, and thus the copula, as it bears an assertion-making form, is more likely to appear towards the end of an utterance, followed by an IU boundary.
The placeholder, too, is often found in an IU-final position: if a speaker is performing a lexical search, they might stall and project an IU boundary.
However, the placeholder can also be followed immediately by its target in an IU in progress, as shown in (\ref{ex9-same-IU}), or be deemed as a sufficient reference, as shown in (\ref{ex8-no-retrieval}). 
Since the placeholder can stand in for lexical items performing different functions (i.e., nouns, demonstratives, verbs, etc.) and these might appear in different positions within an IU, the overall expectation for the placeholder is to be associated with a lower value of match position.

Second, the copula is expected to appear in more fluent discourse and longer IUs than the placeholder, thus yielding a higher value for the number of words in an IU. 
The placeholder, on the other hand, is more likely to be involved in a word search process, and thus with disfluencies \citep{Zhao&Jurafsky2005} and overall shorter IUs.
As a result, the expectation is for the placeholder to be associated with a lower value for the number of words in an IU.

Third, the copula is expected to mark the end of an assertion, and thus to be associated with a drop in pitch and a falling intonation contour.  
On the other hand, the placeholder is expected to mark a word search problem, and therefore it is more likely to co-occur either with a less pronounced drop in intonation or with the speaker abandoning the IU altogether  
The expectation for the placeholder is thus to be associated with continuing and truncated intonations. 

Fourth, the copula is expected to co-occur with shorter delays than the placeholder since no repair is involved, thus yielding a lower value for preceding and following pauses. 
Longer following pauses are also possible, however, and may result from speech planning, as the speaker finishes the current chunk of discourse and moves on to the next item they want to communicate.
The placeholder, on the other hand, is more likely to co-occur with longer delays before and after it is deployed, as word search and repair processes correlate with unfilled pause lengths \citep{Clark&FoxTree2002}. 
Thus, the expectation is for the placeholder to be associated with a higher value for preceding and following pauses.

Finally, individuals are expected to vary in terms of how likely they are to use \textit{ʎe}- as a copula or a placeholder. 
As mentioned in \sectref{quant_data}, the number of copulas and placeholders varies substantially from person to person, which might reflect individual idiosyncratic preferences \citep{Clark&FoxTree2002}.
For example, both Anna Šadrina and Fevron'ja Šalugina contributed one text each, but Anna Šadrina's seven \textit{ʎe}-'s are all instances of the placeholder, wheareas Fevron'ja Šalugina's twenty \textit{ʎe}-'s are split evenly with ten tokens for each function. 

For the modeling, I used a classification random forest \citep{Breiman2001} with the function of \textit{ʎe}- (i.e., copula vs. placeholder) as the dependent variable, and the discourse features outlined above as well as the speaker\footnote{At this point, random forests do not have random effects capability, which are typically used to control for speaker effects. Treating the speaker information as a fixed effect can account for the attested variation in the data, but limits the model's generalizability to new speakers. In the context of language endangerment, such generalizability may be less concerning, as texts from new people are less likely to be included in a new analysis.} as the independent variables.
Random forests are a machine learning method that makes predictions by combining the results of many decision trees.
Each tree resembles a flowchart, where each decision point evaluates different attributes of the data. 
To build each tree, the data is split into two different parts: one trains the model, and the other tests it. 
The training data for each tree is sampled randomly with replacement (aka \textit{bootstrapping}), that is, some data points may appear multiple times while others may not appear at all. 
During training, the algorithm also randomly omits some predictors to determine which ones contribute the most. 
In the end, the predictions from all the trees are averaged together to produce the final result, and the prediction accuracy is calculated by comparing these predictions to the actual observations (i.e., the attested label of the dependent variable for each data point, in this case copula or placeholder).

Random forests are particularly suitable for the analysis because they can handle data structures with non-Gaussian (i.e., non-normal) distributions, and they are applicable to small-$n$ large-$p$ (i.e., few data points with many predictors) scenarios while avoiding problems of collinearity (i.e., the scenario in which two or more predictors are strongly related) \citep{Gries2021}.
However, detecting and interpreting interactions in random forests requires caution. 
The importance of a predictor as shown by the random forest does not necessarily reflect an individual effect, since its importance can be the result of a predictor’s participation in an interaction (i.e., the effect of two or more variables that is greater than or different from what would be expected from adding up their individual effects) \citep{Gries2021}. 

One solution to this problem is to include the combined effect of two or more predictors that one is interested in into the random forest modelling, which \citet{Gries2021} recommends following \citet{Forina-etal2009, Strobl-et-al2009} and \citet{ Baayen2011}. 
In this case, a potential interaction of interest is between the match position and the number of words in an IU: the copula's higher likelihood to appear towards the end of an IU and its occurrence in an IU with more elements are dependent on discourse fluency and the lack of a word search process. 
Thus, I created an interaction predictor by multiplying the normalized match position with the number of words in the IU and included it in the model.\footnote{The multiplicative interaction becomes 0 when the normalized match position is 0 (i.e., when the match is in the first position in an IU) regardless of IU length, thus suggesting that information about IU length is lost in such cases. However, modifying the first position in an IU to 0.1 in the normalization process to avoid multiplying by 0 and preserving IU length information in the interaction does not change the modeling results.} 
Tables \ref{categorical_predictors} and \ref{numeric_predictors} summarize the seven independent variables used to annotate each token of \textit{ʎe}-. 


\begin{table}
\centering
\begin{tabularx}{.8\textwidth}{Xll}
\lsptoprule
Categorical predictor  & Levels  \\
\midrule
Intonation contour       & Continuing, Falling, Truncated  \\ 
Speaker                  & Vasilij Šalugin, Anna Šadrina,  \\
                         & Ivan Dolganov, Fevron'ja Šalugina... \\ 
\lspbottomrule
\end{tabularx}
\caption{Categorical variables and their levels}
\label{categorical_predictors}
\end{table}


\begin{table}
\begin{tabularx}{.8\textwidth}{X rrrr}
\lsptoprule
Numeric predictor & Min & Max & Mean & Median\\ \midrule
Norm'd match position         & 0            & 1.0          & 0.83          & 1.0\\
Words in IU                   & 1            & 8.0          & 2.08          & 2.0\\
Interaction: Position \times Words & 0       & 5.0          & 1.60          & 1.0\\
Preceding pause (in sec)      & 0            & 7.9          & 0.81          & 0.5\\
Following pause (in sec)      & 0            & 5.0          & 0.80          & 0.5\\ 
\lspbottomrule
\end{tabularx}
\caption{Numeric variables and their summary statistics}
\label{numeric_predictors}
\end{table}




%The longest IU where \textit{ʎe}- is attested has 8 words, although the majority of the IUs retrieved are on the lower end of the scale ($Median=2$). 
%The lengths of preceding and following pauses are treated as continuous variables, where 0 indicates that there is no pause and 1.0 indicates a second-long pause.
%The longest preceding pause attested is 7.9 seconds, and the longest following pause is 5.0 seconds. 
%In both cases the average pause is significantly shorter ($Median = 0.5$).
Despite the long tail in the pause length distributions, their logs do not alter the results, since the log transformation is monotonic (i.e., it does not fundamentally alter the relationship between data points).
Therefore, the pause length measurements are left in seconds for simplicity. 

%Add model hyperparameters:
%10000 trees
%sampling with replacement
%Maybe where I mention random forest?

\subsection{Results} \label{sec_quant_results}
Due to the unbalanced distribution of the two forms, the baseline/no-information rate accuracy of the classification model is already at 67.33\%. 
That is, without any information the model would predict the placeholder 67.33\% of the time (i.e., its overall distribution with respect to the copula).
The model performs with a 74.17\% true prediction/out-of-bag accuracy, and with a 74.08\% as the out-of-bag Area Under Curve (\textsc{auc}, the equivalent of the \textit{C}-score in regression modelling).
The difference between the baseline and the prediction accuracy is statistically significant (\textit{p\textsubscript{binomial test}} < 0.001). 
\figref{fig:VarImpPlots} shows the Mean Decrease Accuracy plot, which indicates the importance of each predictor for model accuracy (although the scale is neither a percentage nor a count of observations).
From that, it can be inferred that the variable that contributes the most to the model is the interaction between the normalized match position and the number of words in the IU.

\begin{figure}
    \centering
    \includegraphics[height=.3\textheight]{VarImpPlot.png}
    \caption{Mean Decrease Accuracy plot from the Random Forest model}
    \label{fig:VarImpPlots}
\end{figure}


To zoom in on the effect of this predictor, I plotted the partial dependence (PD) scores, which represent the dependence between the target response (i.e., copula vs. placeholder) and the different levels of the variable, in this case the product of the normalized match position with the number of words in the IU.
That is, PD scores show the average prediction change between copula and placeholder as the chosen variable varies. 
%\figref{fig:dp.ph.inter} shows the partial dependence of the placeholder as a function of the interaction.
%The jittered rugs at the bottom shows the distribution of each value of the variable. 
PD scores marginalize out all of the other features in the model by averaging over predictions based on all combinations of those other features. 
Given the unbalanced distribution, this process creates a bias towards placeholders, since the combinations of the other features tend to be those that occur in placeholders. 
Thus, any PD score below 0.67 (i.e., the probability of the placeholder) can be interpreted as favoring the copula, whereas scores above 0.67 can be interpreted as favoring the placeholder (see \citealt[370]{Hastie-etal2009} for bias on PD scores). 
In a nutshell, due to the unbalanced distribution of the two functions (copula vs. placeholder), lower PD scores represent a preference for the copula, whereas higher PD scores represent a preference for the placeholder. 

\begin{figure}
    \centering
    %\makebox[\textwidth][c]{\includegraphics[scale=0.5]{dep.score_PH~INTERACTION_continuous.png}}
    \includegraphics[height=.3\textheight]{dep.score_PH~INTERACTION_continuous_probs.png}
    \caption{PD scores of the interaction, with dashed line representing the threshold above which the placeholder is predicted; the copula is predicted below the threshold line}
    \label{fig:dp.ph.inter}
\end{figure}

With this in mind, the plotting of the PD scores in \figref{fig:dp.ph.inter} shows that the placeholder is predicted for the lower values of the interaction predictor (i.e., 0 to 1), whereas the copula is predicted for higher values (i.e., roughly 1.3 and above).
The jittered rugs at the bottom show the distribution of each value of the variable. 
These results are consistent with the predictions that the copula is more likely to appear towards the end of an IU and in IUs with more elements, and that the placeholder is more likely to appear towards the beginning of an IU and in IUs with fewer elements. 

\newpage
Another way to represent the effect of the interaction is to plot the two individual predictors in the interaction (i.e., normalized match position and the number of words in an IU) in a heat map, with the predicted probability in the corresponding colored cell range.
\figref{fig:heatmap} shows higher predicted probabilities for the copula towards the top and the right sections, which represents longer IUs and the copula appearing towards the end of an IU.
The more purple and blue cells represent lower probabilities for the copula and, thus, higher probabilities for the placeholder.
These are mostly found towards the bottom and the left sides of the heat map, suggesting a higher likelihood for the placeholder to appear towards the beginning of an IU and in IUs with fewer elements.
Notably, IUs with a single word (which are normalized to a match position of 1) also show a stronger preference for the placeholder. 

\begin{figure}
    \centering
    \includegraphics[height=.3\textheight]{heatmap_COP.png}
    \caption{PD scores of the interaction as a function of both normalized match position and number of words in an IU plotted independently. Each cell represents the probability of occurrence of the copula based on the combination of the two variables, with red shades representing higher probabilities. The probabilities for the placeholder are complementary and thus are not displayed here}
    \label{fig:heatmap}
\end{figure}


According to the Mean Decrease Accuracy plot, the second variable that contributes the most to the model is the intonation contour. 
\figref{fig:dp.ph.inton} shows the PD scores for the placeholder as a function of the intonation contour, where bar width represents the proportion of observations in each intonation category and the horizontal dashed line represents the 0.67 threshold.
In this case the placeholder is predicted for the continuing and truncated intonation contours, whereas the copula is predicted for the falling intonation.  
This result is thus consistent with the expectations for each function. 


\begin{figure}
    \centering
    %\makebox[\textwidth][c]{\includegraphics[scale=0.4]{dep.score_PH~INTONATION_withline.png}}
    \includegraphics[height=.3\textheight]{dep.score_PH~INTONATION_withline_probs.png}
    \caption{PD scores of the intonation, with the dashed line representing the threshold above which the placeholder is predicted; the copula is predicted below the threshold line}
    \label{fig:dp.ph.inton}
\end{figure}


%As for the Mean Decrease Gini on the right panel in \figref{fig:VarImpPlots}, it shows how each variable contributes to the homogeneity of the leaves in the resulting random forest model.
%Here too the interaction is one of the best predictors, coming only after the preceding pause and before the following pause variables, which otherwise contribute little in terms of accuracy. 
%The mismatch for the pause predictors between accuracy and homogeneity is possibly related to the bias that Gini introduces with continuous variables: with many more split points, there is a higher probability that the variable happens to predict the outcome well by chance. 
%As mentioned, logging the pause variables does not alter the results. 


\subsection{Discussion} \label{sec_discussion}

The results from the random forest modeling show that the copula and the placeholder have slightly different distributions in discourse, with the placeholder displaying a stronger tendency towards the beginning of an IU and in IUs with fewer elements, and the copula displaying a tendency towards the end of an IU and in IUs with more elements.  
Additionally, the model shows a stronger preference for the placeholder with continuing and truncated intonation contours, and a stronger preference for the copula with falling intonation.
These results add to the qualitative evidence that these forms are synchronically distinct and suggest that a polysemy analysis is unlikely. 

%Moreover, the model shows a stronger preference for the placeholder with continuing and truncated intonation contours than with the falling contour.  
%It is surprising, nonetheless, that it also predicts the placeholder for the latter pattern, since the copula would be more consistent with it, at least theoretically. 
%I believe that there are two reasons for this result. 
%In cases where the placeholder is deemed sufficient and the target is not clarified, the expected drop in intonation after the assertion often occurs, as in (\ref{ex11-former8}) above. 
%A similar situation emerges when the target item is retrieved in the same IU, and the speaker carries out the the planned assertion until the end. 
%Example (\ref{ex12-former9}), shown above as (\ref{ex9-same-IU}), shows such an instance.
%Thus, the intonation information, while relevant, is more complex to interpret when the placeholder is not immediately followed by an IU boundary. 

%\begin{exe}
%\ex \label{ex12-former9}
%    \glll мэт атахун, \\
%    met ataq-u-n \\
%    \Fsg{} two-\Ep-\Lnk{} \\
%    \glt `My two"

%\sn (0.3)

%\sn
%    \glll чаачаа эрэ, \\
%    ʧaaʧaa ere \\
%    elder.brother only \\
%    \glt `Elder brothers only"

%\sn (0.9)

%\sn 
%    \glll \textbf{льэ}\uline{гэ} луусьиин нумө\uline{гэ} модоӈи.  \\
%    \textbf{ʎe}-\uline{ge} luuçii-n numø-\uline{ge} modo-ŋi-Ø \\
%    \textbf{\Ph}-\uline{\Loc{}} Russian-\Lnk{} house-\uline{\Loc{}} live-\Pl-\Ass.\Third{} \\
%    \glt `(Only my two elder brothers) lived in whatchamacallit, in Verkhnekolymsk' \\
%    \null \hfill (\citealt{Nikolaeva_Mayer2004}, 49: IU 10-15)
%\end{exe}


The results for the pause information and the speaker predictors suggest that these variables do not contribute substantially to predicting each function. 
At first, it might appear somewhat surprising that the preceding pause is not a contributing factor. 
However, \citet{Clark&FoxTree2002} show that preceding pause lengths explain the differences between two distinct fillers (i.e., English \textit{uh} and \textit{uhm}), whose main function is to signal a delay (minor for \textit{uh}, major for \textit{uhm}) while holding the floor. 
In the case of \textit{ʎe}-, both the copula and the placeholder are morphosyntactically integrated in the discourse in progress and, thus, are not directly comparable to \citet{Clark&FoxTree2002}'s analysis. 
These differences in discourse and morphosyntactic environments might help explain why the predicting power of the pause information is limited in this case (but see other papers in this volume). 

%A potential explanation is that the comparison in the case of \textit{ʎe}- is somewhat different from previous research.
%\citet{Clark&FoxTree2002} show that preceding pause lengths explain differences between distinct fillers (i.e., English `uh' and `uhm'), whereas the uses of \textit{ʎe}- involve a filler (i.e., the placeholder) and a fully-fledged lexical item (i.e., the copula). 

%As for the speaker variable, it is possible that its lack of predictive power lies in the very uneven distribution of texts. 

%Ask Stefan:
%Should I say something more about the speaker predictor not being helpful?
%Should I say something about the 74% accuracy? maybe that even if the diff is statistically significant, there is still a lot that is not predicted; maybe other discourse factors at play?


\section{Origins of the placeholder} \label{sec_origins}

The outcomes of the qualitative and quantitative analyses now allow us to attempt to answer \citet{Epps2008a}'s questions, i.e. whether the distinct functions of \textit{ʎe}- are a case of polysemy, a historical accident of independent forms converging into the same phonological shape, or a diachronic divergence from the same source. 
Both functions appear synchronically distinct and identifiable on the prototypical roles of copula and placeholder. 
Additionally, their different distributions in discourse suggest that a polysemy scenario seems unlikely, under the assumption that related senses of a word display similar distributions. 
Rather, the two uses of \textit{ʎe}- are best analyzed as being associated with different morphs. 
The question about their historical origins, however, remains open, that is, whether their phonological resemblance is a historical accident or whether the two are historically related.

To answer this question the best place to start is by analyzing the historical record of the copula. 
\citet{Nikolaeva2006, Nikolaeva2020} suggests that the copula \textit{ʎe}- can be reconstructed in Proto-Yukaghir as \textit{*ʎe}-, since an identical copular form is attested in Tundra Yukaghir. 
In fact, she argues that the copula is cognate with the Uralic \textit{*le}- `to exist, to become' (but see \citealt{Aikio2014} for a dissenting argument). 
Additional evidence for the copula \textit{ʎe}- being somewhat older material is that at least two suffixes can be traced back to it: the evidential -\textit{ʎel}- and the predicative case marker -\textit{lek} \citep{Nikolaeva2020}. 
The evidential arose as a combination of the copula \textit{ʎe}- with the participle -\textit{l} as a modal construction that was later reanalyzed as inflectional mood. As \citet{Nikolaeva2020} points out, other suffixes in Kolyma Yukaghir reflect a similar development, and the pathway from copula to evidential is attested cross-linguistically \citep{Aikhenvald2004}. 
As for the predicative case marker, \citet{Nikolaeva2020} argues that -\textit{lek} reflects the combination of the copula \textit{ʎe}- with the focus marker -\textit{k}, a possible development considering that the alternation between \textit{ʎ}\sim\textit{l} is not always phonologically conditioned in Yukaghiric \citep{Nikolaeva2020}.
Combined, \textit{ʎe-k} served as an existential predicate following a noun. 
Over time, the copula and the focus marker fused together, but the resulting nominal case marker maintained some of its properties, including the ability to signal assertions without a verbal predicate. 

%The development of the predicative -\textit{lek} seems plausible, especially considering that the alternation between \textit{ʎ}\sim\textit{l} is not always phonologically conditioned in Yukaghiric \citep{Nikolaeva2020}.
%An important challenge to this account, however, is posed by Wadul, in which the predicative takes the form -\textit{leŋ}.
%As \citet{Nikolaeva2020} admits, the role of the nasal is unclear. 
%A potential answer might be found in neighboring Chukchi, in which the verb \textit{ləŋ}- functions as a ``secondary equational predication' \citep[310]{Dunn1999} and can be considered a copula.
%It is thus possible that in Wadul the development of the predictive was influenced by a form in Chuckhi with a somewhat similar function, especially since Tundra Yukaghirs often spoke Chukchi whereas the opposite was less likely \citep{Jochelson1926a, Kantarovich2020}. 

%This hypothesis highlights the importance of considering contact when trying to disentangle the historical relationship between two forms. 
%It would appear from the discussion above that the copula \textit{ʎe}- is very productive in Odul. 
%However, its frequency is influenced by contact with Russian.
%\citet{Matic2008} notes that some verbal morphemes that are well-documented in \citet{Jochelson1926a}'s texts are virtually absent in modern Odul and have been instead replaced by constructions with the copula that mirror Russian morphosyntax more closely. 

Another avenue for unraveling the historical relationship between the copula and the placeholder is to consider cross-linguistic patterns.
As mentioned, demonstratives are a common source for placeholders cross-linguistically \citep{Hayashi&Yoon2010, Podlesskaya2010}, an unlikely source in Kolyma Yukaghir since the placeholder can target demonstratives.
\citet[54]{Nikolaeva2020}'s reconstruction of the demonstrative stems in Proto-Yukaghir confirms this suspicion, as they closely resemble the proximal \textit{tii}-, the medial \textit{adaa}- and the distal \textit{taa}-. 
However, \citet{Nagasaki2010}'s glossing of the placeholder as ``that one" captures the pronominal quality of the placeholder, unlike hesitators \citep{Hayashi&Yoon2010}. 
This referential flavor is probably amplified when the placeholder is used ``approximately' \citep{Podlesskaya2010} and is deemed as a sufficient referent without later clarification on the target.

Copulas, too, have been shown to grammaticalize from demonstratives \citep{Kuteva-etal2019}, but they are also the source of third person pronouns in what \citet{Katz1996} calls the ``copula cycle."
Hebrew is a good example of this cycle: the copula in Proto-Semitic gave rise to an independent third person pronoun in Biblical Hebrew, which in turn evolved to become a new copula in modern Hebrew \citep{Katz1996}. 
Additional examples of copulas turning into markers of third person include Turkish \citep{Katz1996}, Western Iranian languages \citep{Korn2011}, and Guinea-Bissau Creole \citep{Truppi2021}.

The connection of copulas and placeholders to demonstratives and pronominal forms does not seem accidental.
Rather, it is plausible to assume that referentiality is what connects the copula \textit{ʎe}- to the placeholder \textit{ʎe}- historically.
However, it is hard to posit one function as the source of the other given the copula's cyclic pattern with demonstratives and deictic pronouns. 
Theoretically, at least, both developments (i.e., from placeholder to copula, and from copula to placeholder) are possible.\footnote{A third scenario is also possible: both the placeholder and the copula emerged from an older demonstrative now lost.} 
It is somewhat of a chicken and egg situation. 

The development of the copula from the placeholder is plausible by extending \citet{Podlesskaya2010}'s ``approximate naming" function to contexts where the target would be the main assertion. 
Given \textit{ʎe}-'s ability to stand in for a target that functions ``nominally" and ``verbally", it is possible that speakers might not always clarify what their intended target is and deem the placeholder as a sufficient predicating device. 
When following lexical items deployed as converbs, such a structure can be reanalyzed as an analytic construction with an auxiliary. 
Such constructions are attested in Kolyma Yukaghir (e.g., the so-called periphrastic prospective, \citealt[178]{Maslova2003}; \citealt[244]{Nagasaki2010}).


\largerpage
Example (\ref{ex13-ambiguous}) shows an instance of a bridging context, where the use of the placeholder could be interpreted also as a copula.
The speaker here attaches the so-called verb-focus transitive morphology to \textit{ʎe}-, which suggests that \textit{ʎe}- is in fact standing in for a different lexical item; the copula is not otherwise attested with this assertion morpheme. 
However, the target is not retrieved and the predication is deemed sufficient, maybe because the preceding converb already provides enough lexical information. 
Without clarification of the target, this structure could be interpreted with the converb as the main lexical contributor to the discourse, where \textit{ʎe}- has a supporting role to fulfill the need for an assertion. 
This interpretation can give rise to a light-verb, copular or existential meaning of \textit{ʎe}-. 


\begin{exe}
\ex \label{ex13-ambiguous}
    \glll Хаӈсааги йуөлугэ, \\
    qaŋsaa-gi juø-lu-ge \\
    pipe-\Third\Poss{} see-\Fsg-\Loc \\
    \glt `When I saw his pipe'

\sn (0.5)

\sn
    \glll эл чубукньэ, \\
    el ʧubuk-ɲe \\
    \Neg{} chibouk-\Com{} \\
    \glt `Without chibouk'

\sn (1.4)

\sn 
    \glll тудэ таӈ чубуккэлэ чуму лэӈдэт,  \\
    tude ta-ŋ ʧubuk-kele ʧumu leŋ-de-t \\
    \Tsg.\Gen{} that-\Lnk{} chibouk-\Acc{} all eat-\Unk-\Cvb.\Ctx{} \\
    \glt `Eating all of that chibouk of his'

\sn (1.0)

\sn
    \glll \textbf{льэльэлум}. \\
    \textbf{ʎe-ʎel-u-m} \\
    \textbf{\Ph-\Ev-\Ep-\Ass.\Tr.\Ef.\Tsg{}} \\
    \glt `He \textbf{whatchamacallited}' \\
    Potential interpretation: ``He had smoked all his chibouk' \\
    \null \hfill (\citealt{Nikolaeva_Mayer2004}, 34: IU 264--270)
\end{exe}


The alternative is to posit the development of the placeholder from the copula. 
\citet{Katz1996} argues that an intermediate step in the evolution from the copula in Proto-Semitic to the third person pronoun in Biblical Hebrew is its interpretation in a more nominal manner (i.e., `a/the/his being'). 
This development is also plausible because of the interpretation of Kolyma Yukaghir roots in more nominal or more verbal ways depending on the context, sometimes without clear signs of derivation \citep{Ventayol-Boada-et-al2023}. 
The nominal interpretation might be part of the reason why \textit{ʎe}- is glossed as a copula in its placeholder function in \citet{Nikolaeva_Mayer2004}, as mentioned previously. 
The nominal reading of \textit{ʎe}- is later characterized in more specific terms.
Example (\ref{ex14-cop-as-ph}) shows an instance where such an interpretation would be possible. 
Over time, this structure can be reanalyzed as \textit{ʎe}- standing in for the characterization, especially if longer delays and unfilled pauses start occurring between the two, and thus giving rise to the placeholder function. 


\begin{exe}
\ex \label{ex14-cop-as-ph}
    \glll Пойнэй, \\
    pojne-j \\
    white-\Ptcp{} \\
    \glt `(Which) whites'

\sn (0.8)

\sn
    \glll ньаасьэньдьэ, \\
    ɲaaçe-ɲ-ʤe \\
    face-\Prop-\Ptcp{} \\
    \glt `Having a face'

\sn (2.6)

\sn 
    \glll \textbf{льэгэлэ},  \\
    \textbf{ʎe-gele} \\
    \textbf{\Cop-\Acc{}} \\
    \glt `\textbf{[A] being}'

\sn
    \glll көйгэлэ йуөӈидэ моннульэлӈи,  \\
    køj-gele juø-ŋide mon-nu-ʎel-ŋi-Ø \\
    boy-\Acc{} see-\Cvb.\Cond{} say-\Impf-\Ev-\Tpl-\Ass.\Intr.\Ef{} \\
    \glt `If they saw (a being), a boy (with a white face), they said,'

\sn (0.4)

\sn
    \glll ньаатлэбиэ монут ньуутиэльэлӈаа. \\
    ɲaatlebie mon-u-t ɲuu-t-ie-ʎel-ŋaa \\
    willow.ptarmigan say-\Ep-\Cvb.\Ctx{} call-\Unk-\Inch-\Ev-\Ass.\Tr.\Ef.\Tpl \\
    \glt `They called [him], namely a ptarmigan' \\
    \null \hfill (\citealt{Nikolaeva_Mayer2004}, 44: IU 90--97)
\end{exe}


At first, both scenarios seem plausible: the development of the copula from placeholder, and the development of the placeholder from copula.
I believe the latter is more likely in the case of Kolyma Yukaghir for two reasons.
First, the copula \textit{ʎe}- appears to be older linguistic material, as exemplified above by at least two suffixes having likely evolved from it. 
The fact that an identical copula exists in Tundra Yukaghir adds to this evidence. 
And second, the placeholder is followed by its target in 71.8\% of the time, as mentioned in \sectref{quant_data}.
It is unclear whether the frequency of ``approximate naming" placeholders as a predicating device would constitute enough grounds to trigger the development of the copula, especially since the remaining 28.2\% include both nominal and verbal targets. 
\tabref{grammaticalization_pathway} shows the potential stages of the placeholder development from the copula in Kolyma Yukaghir.

%It is also worth mentioning that the copula displays more variation in the vowel quality than the placeholder. 
%In \citet{Nikolaeva_Mayer2004}'s texts, the copula \textit{ʎe}- is realized phonetically both as [ʎie] and [ʎə], whereas the placeholder only exhibits the pronunciation with a schwa. 
%The vowel in the placeholder might thus represent a phonologically reduced form of the diphthong in the copula due to its higher use in discourse. 
%While vowel quality might not inherently reveal much about the relative age of the copula and the placeholder, it could suggest that the copula carries a higher functional load compared to the placeholder. 


\begin{table}
\begin{tabular}{@{}ccc@{}}
\lsptoprule
\textbf{Proto-Yukaghir} & \textbf{\textgreater{}} & \textbf{Kolyma Yukaghir}  \\ \midrule
\textit{ʎe}-            & \textit{ʎe}-            & \textit{ʎe}-              \\
``to be" (\Cop{})       & ``a/the/his being"     & ``whatchamacallit" (\Ph{}) \\ \lspbottomrule
\end{tabular}
\caption{Hypothetical copula-to-placeholder development in Kolyma Yukaghir}
\label{grammaticalization_pathway}
\end{table}


Additional evidence for the placeholder development from the copula in Kolyma Yukaghir might be found in Tundra Yukaghir. 
Given that both languages share the copula \textit{ʎe}-, it is possible to analyze its distribution in Tundra Yukaghir discourse. 
The lack of a placeholder use of \textit{ʎe}- or the presence of a placeholder with a different phonological shape would support the development from copula to placeholder in Kolyma Yukaghir. 
Alternatively, if \textit{ʎe}- is also attested as a placeholder in Tundra Yukaghir, the idea that the placeholder came first, from which the copula evolved, would still be possible, although a parallel development in both languages from copula to placeholder as outlined above could not be ruled out either. 
These questions remain for further research. 




\section{Conclusion} \label{sec_conclusion}

Analyzing grammatical morphs with identical phonological forms and distinct functions is a common linguistic challenge. 
In this context, at least three scenarios are possible: 1) the resemblance is a historical accident of two independent elements converging in form, 2) the different functions represent an example of polysemy (i.e., a single form with vaguely related meanings), or 3) the forms are synchronically distinct but historically related \citep{Epps2008a}.
This Chapter addresses the importance of considering discourse-level explanations when attempting to choose among these scenarios. 

The case of \textit{ʎe}- in Kolyma Yukaghir shows that dividing spoken discourse into IUs can help analyze the semantic and pragmatic components of a form and thus be more confident of its function(s).
When the same phonological material appears to be associated with multiple functions, as in the case of \textit{ʎe}-, discourse features offer the possibility to quantify and model what features are associated more closely with each function, thus allowing us to establish whether their distributions are distinct and rule out a polysemy analysis. 
Establishing the historical relationship between both morphs can be then attempted by identifying the bridging contexts in discourse, where multiple interpretations of a form are possible and the mechanisms of language change are set in motion.

I argue that the description of \textit{ʎe}- as a copula in Kolyma Yukaghir fails to account for a significant number of occurrences in naturalistic discourse, which can be better analyzed as a placeholder. 
Based on their distributions, I conclude that these two functions of \textit{ʎe}- can be best described as distinct morphs rather than a single polysemous morph with meanings vaguely related.
The copula appears to be more likely to appear towards the end of an IU and in IUs with more elements, whereas the placeholder is more likely to occur towards the beginning of an IU and in IUs with fewer elements.
Their resemblance, however, does not appear to be a historical accident but rather connected to referentiality. 
It seems plausible that the polycategorial nature of Kolyma Yukaghir roots has allowed a nominal interpretation of the copula \textit{ʎe}- (i.e., `a being') as a referential expression, a development attested cross-linguistically \citep{Katz1996}.
When followed by another referential expression that characterizes or delimits it in scope, it creates a morphosyntactic environment that can be reanalyzed as the first element standing in for the second, thus giving rise to the placeholder interpretation. 

Overall, this Chapter contributes to the study of discourse and grammar, and the grammaticalization of salient discourse patterns in particular \citep{Ariel2009, Couper-Kuhlen&Thompson2008, DuBois2003}. 
Additionally, it adds to the growing research on fillers and the emergence of placeholders cross-linguistically by establishing the copula as a new potential source. 
It does so by offering different methodological tools that can be easily applied in the context of endangered language description. 


\printglossaries



\sloppy
\printbibliography[heading=subbibliography,notkeyword=this]
\end{document}
