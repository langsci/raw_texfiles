\documentclass[output=paper]{langscibook}
\ChapterDOI{10.5281/zenodo.15697575}
\author{Brigitte Pakendorf\orcid{}\affiliation{Laboratoire Dynamique du Langage (CNRS \& Université Lumière Lyon 2)} and Françoise Rose\orcid{}\affiliation{Laboratoire Dynamique du Langage (CNRS \& Université Lumière Lyon 2)}}
\title{Fillers in the world’s languages: A refined typology}
\abstract{Interest in placeholders and related items in the languages of the world has burgeoned in recent years, both from a descriptive and a theoretical perspective. A particular focus of the recent literature has been on the pragmatic extensions of placeholders, demonstrating that these are not merely used in situations of disfluency to substitute for words that momentarily elude the speaker, but are frequently used intentionally to avoid terms for reasons of politeness, “conspirational” motivations, or rhetorical purposes. 

Fillers – a term that subsumes both placeholders and hesitatives – are cross-linguistically widespread, even though dedicated studies of such items are notably lacking for the languages of Africa. The distinction between placeholders and hesitatives is one of referentiality and morphosyntactic integration: placeholders are both referential and morphosyntactically integrated while hesitatives are neither. We distinguish five different types in our extended typology: 1) specific placeholders that substitute for a particular part of speech, 2) general placeholders that are not restricted to a particular part of speech, 3) general hesitatives such as the cross-linguistically common “pause vowels”, 4) specific hesitatives found in some Austronesian languages that are not referential or integrated, but are specific to particular delayed constituents, and 5) versatile fillers that can fulfill both placeholder and hesitative functions.

The in-depth studies of individual languages collected in this volume highlight the fact that it is not always straightforward to distinguish particular instances of a filler from those of its source form nor to distinguish hesitative from placeholder uses of versatile fillers, that placeholders are frequently not followed by an overt target, and that filler use can vary greatly among speakers.\\

\keywords{placeholders, hesitatives, filled pauses, disfluency, watchamacallit, avoidance}
}

\IfFileExists{../localcommands.tex}{
  \addbibresource{../localbibliography.bib}
  \usepackage{langsci-optional}
\usepackage{langsci-gb4e}
\usepackage{langsci-lgr}

\usepackage{listings}
\lstset{basicstyle=\ttfamily,tabsize=2,breaklines=true}

%added by author
% \usepackage{tipa}
\usepackage{multirow}
\graphicspath{{figures/}}
\usepackage{langsci-branding}

  
\newcommand{\sent}{\enumsentence}
\newcommand{\sents}{\eenumsentence}
\let\citeasnoun\citet

\renewcommand{\lsCoverTitleFont}[1]{\sffamily\addfontfeatures{Scale=MatchUppercase}\fontsize{44pt}{16mm}\selectfont #1}
   
  %% hyphenation points for line breaks
%% Normally, automatic hyphenation in LaTeX is very good
%% If a word is mis-hyphenated, add it to this file
%%
%% add information to TeX file before \begin{document} with:
%% %% hyphenation points for line breaks
%% Normally, automatic hyphenation in LaTeX is very good
%% If a word is mis-hyphenated, add it to this file
%%
%% add information to TeX file before \begin{document} with:
%% %% hyphenation points for line breaks
%% Normally, automatic hyphenation in LaTeX is very good
%% If a word is mis-hyphenated, add it to this file
%%
%% add information to TeX file before \begin{document} with:
%% \include{localhyphenation}
\hyphenation{
affri-ca-te
affri-ca-tes
an-no-tated
com-ple-ments
com-po-si-tio-na-li-ty
non-com-po-si-tio-na-li-ty
Gon-zá-lez
out-side
Ri-chárd
se-man-tics
STREU-SLE
Tie-de-mann
}
\hyphenation{
affri-ca-te
affri-ca-tes
an-no-tated
com-ple-ments
com-po-si-tio-na-li-ty
non-com-po-si-tio-na-li-ty
Gon-zá-lez
out-side
Ri-chárd
se-man-tics
STREU-SLE
Tie-de-mann
}
\hyphenation{
affri-ca-te
affri-ca-tes
an-no-tated
com-ple-ments
com-po-si-tio-na-li-ty
non-com-po-si-tio-na-li-ty
Gon-zá-lez
out-side
Ri-chárd
se-man-tics
STREU-SLE
Tie-de-mann
} 
  \togglepaper[1]%%chapternumber
}{}


\epigram{There are marginal phenomena that give a basis for very important lessons about the nature of language in general, about particular languages, and about the way we need to approach the examination and analysis of languages.}
\epigramsource{\citep[200]{Joseph1997}}

\begin{document}

\maketitle 
% ATTENTION: Diacritics on the following phonetic characters might have been lost during conversion: {'ɛ', 'ɨ', 'ə'}

\section{Introduction}\label{sec:intro:1}
\largerpage

This volume focuses on a marginal phenomenon that is starting to garner more and more attention, namely fillers.\footnote{The term ‘filler’ is used in different ways in the literature: it was introduced by \citet[75]{Clark2002} with reference to English \textit{uh} and \textit{uhm}, which had until then been mostly designated as ‘filled pauses’ (a label that continues to be used for items like \textit{uh} and \textit{uhm}, cf. \citealt{Lickley2015}: 458, \citealt{KosmalaCrible2022}, \citealt{Kirjavainen2022}, in addition to ‘filler’). Sometimes, however, the term ‘filler’ is used much more broadly as a label “… for all of the non-silence devices that can be deployed after the current word has been brought to completion to delay the next word due” \citep[2]{Fox2010}, i.e. it includes final vowel lengthening or discourse particles.} These are markers of hesitation like \textit{uh/uhm} and substitutes for items that elude a speaker or that she does not want to utter, like \textit{watchamacallit} or \textit{thingy}; as such, they belong to the many vague expressions found in natural languages \citep{Cutting2007}. In this volume we focus on conventionalized fillers\footnote{Note that some “pause vowels”, such as \textit{uh} in English or \textit{eh} in Spanish are conventionalized (cf. \citealt{Clark2002}, \citealt{Erker2022}).} that roughly coincide with what \citeauthor{Bloomfield1933} (\citeyear{Bloomfield1933}: 186, cited from \citealt[21]{Maclay1959}) called “special parenthetic hesitation forms”. \textit{Ad hoc} delaying devices, such as silent pauses, final syllable lengthening, or false starts and repetitions, as well as discourse particles that have a wide range of functions, are thus excluded from consideration (other than in comparison with conventionalized fillers). We define fillers as follows: 

\begin{exe}
\exi{(1)} A \textsc{filler} is an overt and independent marker of hesitation with a conventionalized form which is used in disfluency and which often has other discourse functions. Two strategies of use can be distinguished, namely \textsc{hesitatives} (or \textsc{hesitators}), elements that are non-referential and not morphosyntactically integrated into the unfolding utterance, and \textsc{placeholders}, which substitute for a particular element of the utterance and hence are morphosyntactically integrated and referential.
\end{exe}

For instance, French \textit{eum} is a hesitative: it is not referential and not morphosyntactically integrated \REF{ex:intro:1}. In contrast, Dalabon \textit{keninjhbi} \textup{is a placeholder, since it is referential and integrated within a noun phrase} \REF{ex:intro:2}.

\ea\label{ex:intro:1}
{French (\citealt[7]{KosmalaCrible2022})}\\
\gll une anecdote t’ en as pas une?\\
     an anecdote \textsc{2sg} \textsc{part.o} have[\textsc{prs.2sg}] \textsc{neg} one {}\\
\gll \textbf{eum} (1.120)\\
     \textbf{\textit{uhm}} {}\\
\gll ah, c’est bon!\\
     ah, that.is good {}\\
\glt ‘an anecdote don’t you have one? \textbf{uhm} (1.120) ah I have one!’\footnote{In the examples, fillers are highlighted in bold, while their targets, where relevant, are underlined.}
\z

\ea\label{ex:intro:2}
{Dalabon (\citealt[\pageref{ex:ponsonnet:9}]{chapters/ponsonnet}, ex.\REF{ex:ponsonnet:9})}\\
\gll Kanidjah   bala-lng-bo-ninj \textbf{keninjhbi-ngong}    [3.76s]    \uline{kunj-ngong} bala-h-bo-ninj    \textbf{keninjhbi}     \uline{djakana}…\\
     \textsc{dem}    3pl-\textsc{seq}-go-\textsc{pimp} \textbf{\textsc{ph}-group}   {}   kangaroo-group 3pl-\textsc{r}-go-\textsc{pimp}    \textbf{\textsc{ph}}    bird.species\\
\glt ‘They were going there, \textbf{all} \textbf{the} \textbf{whatsit}… [pause] \uline{all the kangaroos} were going there, the \textbf{whatsit} \uline{jacanas}.’
\z

Such fillers were at the margins of linguistic research for a considerable time, with notable exceptions being the seminal paper by \citet{Hayashi2006} and the edited volume by \citet{Amiridze2010}. Recently, however, interest has exploded, with a spate of papers exploring fillers in different languages appearing in the last few years (e.g. \citealt{Seraku2020}, \citealt{Hennecke2022}, \citealt{Klyachko2022}, \citealt{Nagaya2022}, \citealt{Vallejos-yopan2023}). Seraku, in particular, has explored the pragmatics of placeholders in the light of different semantic and formal approaches (\citealt{Seraku2022a}, \citeyear{Seraku2022b}, \citeyear{Seraku2023,Seraku2024}). However, the preliminary typology proposed by \citet{Podlesskaya2010} still remains the main cross-linguistic overview of morphosyntactic aspects of fillers (more precisely, of placeholders), and we believe the time has come to bring our knowledge on all aspects of fillers up to date. Hence we aim to provide a review of both functional and morphosyntactic aspects of fillers in a cross-linguistic perspective, basing ourselves on published literature as well as on the new insights that emerge from the chapters included in this volume. As a result we propose a refined morphosyntactic typology of fillers (\tabref{tab:intro:1}, see \sectref{sec:intro:4} for details).

% INSERT TABLE 1 here
\begin{table}
    \begin{tabular}{cc}
    \lsptoprule
        specific placeholders & specific hesitatives \\
        \midrule
       general placeholders  &  general hesitatives\\
       \midrule
    \multicolumn{2}{c}{versatile fillers}\\
    \lspbottomrule
    \end{tabular}
\caption{\label{tab:intro:1} Refined typology of fillers}
\end{table}

The rest of the article is structured as follows: \sectref{sec:intro:2} provides an overview of what is known about the cross-linguistic distribution of fillers, and \sectref{sec:intro:3} discusses the diachronic sources of these elements. In \sectref{sec:intro:4} our refined morphosyntactic typology of fillers is outlined, while \sectref{sec:intro:5} deals with the diverse functions of these items and \sectref{sec:intro:6} with their relation with general extenders. The subsequent sections briefly touch upon fillers in language contact (\sectref{sec:intro:7}) and what they can tell us about the nature of language (\sectref{sec:intro:8}), and \sectref{sec:intro:9} describes the volume and its main contributions to the study of placeholders and hesitatives.

\section{Fillers in the languages of the world}\label{sec:intro:2}

Fillers are known from practically every continent. The data ranges from brief mentions in grammars – possibly illustrated with a couple of examples – to in-depth studies of various aspects of their forms and uses. 

In fact, fillers are rarely discussed within grammars. A search was realized in a reduced but likely representative sample of grammars \citep{LahaussoisToAppear}. Within the table of contents of 237 grammars, our query resulted in only nine successful hits: six for “filler”, two for “hesitation”, and one for “placeholder”. The key terms  “hesitator”, “hesitative” and “filled pause” were not found at all. Although this search may miss grammars that do discuss fillers but use different labels, it does indicate that fillers are mostly considered a marginal question when describing a language. The hits refer to seven grammars in total (two grammars use two key terms in their table of contents), of which the oldest was published in 1993. This indicates that fillers have only recently emerged as a topic worthy of description in grammars. These seven grammars show a fair level of cross-linguistic coverage, with three languages from the Eurasian macro-area – Modern Welsh \citep{King2003}, Dumi (\citealt{Driem1993}), and Japhug \citep{Jacques2021} – one language from the Papunesian macro-area (Papuan Malay; \citealt{Kluge2017}), one from Australia (Kuuk Thaayorre; \citealt{Gaby2017}), one from South America (Movima; \citealt{Haude2006}), and one from Africa (Tiefo-D of Daramandugu; \citealt{Heath2021}). Only North America is not represented.

Dedicated studies on fillers also cover quite a number of languages, albeit with a bias towards the Eurasian macro-area. Fillers are well described in several East Asian languages: Japanese (\citealt{Kitano1999}, \citealt{Hayashi2006}, \citealt{Seraku2022a}, \citealt{Seraku2022b}), Yoron Ryukyuan \citep{Seraku2020}, Korean (\citealt{Hayashi2006}), and Mandarin Chinese (\citealt{Zhao2005}, \citealt{Hayashi2006}, \citealt{Cheung2015}). They are also fairly well described for various languages of Northern Asia: Kolyma Yukaghir (\citealt{chapters/ventayol_boada}), the Tungusic languages Even, Evenki, and Negidal (\citealt{Matić2008}, \citealt{Klyachko2022}, \citealt{chapters/klyachko}, \citealt{chapters/pakendorf}), and the Turkic language Dolgan \citep{Dabritz2018}, as well as for some languages of the Caucasus: the Nakh-Daghestanian languages Udi, Agul \citep{Ganenkov2010}, and Rutul \citep{Maisak2023}, as well as Armenian (\citealt{KhurshudyanPodlesskaya2006}) and Georgian, which has a placeholder verb \citep{AmiridzeGeorgian2010}. In European languages, they are known from English (e.g. \citealt{Enfield2003}, \citealt{Palacios_martinez2015}, \citealt{Tarnyikova2019}), which is often the benchmark for comparison with other languages, and have been described in some detail for French (\citealt{Mihatsch2006}, \citealt{Hennecke2022}, \citealt{Corminboeuf2023}), Russian (\citealt{Podlesskaya2009}, \citealt{Podlesskaya2022}), German \citep{Vogel2020}, Danish \citep{Navaretta2015}, and Estonian \citep{Keevallik2010}. As for Spanish, \citet{Bajo_perez2019} and \citet{Mihatsch2024} describe two different fillers, \textit{fulano/fulana} and \textit{chisme}, respectively, while \citet{Vallejos-yopan2023} analyses the filler \textit{este} in the variety of Spanish spoken in Amazonia. 

Other dedicated studies on fillers in South America cover Northern Pastaza Kichwa (\citealt{chapters/rice}) and the Tupí-Guaraní language Teko (\citealt{chapters/rose}). As for North America, fillers have been described in fair detail in a handful of languages: Maliseet\hyp Passamaquoddy \citep{LeSourd2003}, Sliammon Salish \citep{Watanabe2010}, and Mohawk (\citealt{chapters/mithun}). Very little work has been dedicated to fillers in Australian languages; a start has been made by \citet{chapters/ponsonnet}, who provides a detailed discussion of the Dalabon placeholder \textit{keninjhbi}. In contrast, Austronesian fillers have generally  been described in more detail, possibly because some of them show quite striking formal features (see \sectref{sec:intro:4.4} for details) and because a filler is reconstructed to the proto-language: *anu ‘whatchamacallit’ (\citealt[92]{Nagaya2022}, citing \citealt{BlustAustronesian2013})\footnote{In the web edition of the Austronesian Comparative Dictionary (\citealt{BlustTrussel2013}, \url{https://www.trussel2.com/ACD/acd-s_n1.htm?zoom_highlight=whatchamacallit}, accessed 26 July 2024), the authors define the Proto-Malayo-Polynesian form *a-nu as “thing whose name is unknown, avoided, or cannot be remembered: what?”.\label{footnote:rosepakendorf:4}}. Thus we have descriptions of fillers for Nahavaq spoken in Vanuatu \citep{Dimock2010}, Ilocano \citep{Rubino1996}, Western Subanon \citep{Blake2020}, and Tagalog \citep{Nagaya2022} of the Philippines, and for various languages spoken in Indonesia, such as Besemah (\citealt{chapters/mcdonnell_billings}) and Nasal (\citealt{chapters/billings_mcdonnell}) spoken on Sumatra, as well as Indonesian itself \citep{Wouk2005}. 

A surprising exception is Africa, for which no detailed descriptions of fillers seem (yet) to exist. Even though absence of evidence is not evidence of absence (all the more so since \citealt{Heath1999, Heath2008} and \citealt{Heath2021} briefly describe fillers in three unrelated languages of West Africa), the lack of dedicated descriptions of fillers in African languages is notable and possibly indicates a relative paucity of conventionalized placeholders and hesitatives in this region.

This brief overview demonstrates that – with the exception of some large and well-studied languages like Japanese or Mandarin Chinese – to date little is known about fillers in the languages of the world. The papers included in this volume thus fill in considerable gaps in our knowledge of these items, especially for the Americas, Australia, and the non-Austronesian languages of the Pacific. As mentioned above, the aim of the present chapter is to fill the gap of typological studies on fillers, which are notably rare (\citealt{Hayashi2006}, \citealt{Podlesskaya2010}, \citealt{Seraku2024}).

\section{The origins of the fillers}\label{sec:intro:3}

Some fillers do not have any other function in the current state of the language; these we call \textsc{dedicated}. In some cases, an erstwhile lexical or phrasal source was lost or modified in the language. This is the case for the French placeholder \textit{truc}, the origin of which is a word for theatre machinery \citep{Mihatsch2006}, unsuspected by most speakers. But the filler might also lack a lexical source, such as the Yoron Ryukyuan filler \textit{muna} \citep{Seraku2020}. This concerns especially the “pause vowels” (\citealt{Fox2010}: 2, \citealt{Candea2005}) used as hesitatives in many languages. Nevertheless, \citet[75]{Clark2002} “argue that \textit{uh} and \textit{um} are, indeed, English words. By words, we mean linguistic units that have conventional phonological shapes and meanings and are governed by the rules of syntax and prosody”. 

Dedicated fillers have been described in various languages around the world (though with labels different from ‘dedicated’ or ‘fillers’): Yoron Ryukyuan \citep{Seraku2020}, Ilocano \citep[656--661]{Rubino1996}, Nahavaq \citep[130--131]{Dimock2010}, Manambu \citep[571--578]{Aikhenvald2008}, and Evenki and other Tungusic languages (\citealt{Klyachko2022, chapters/klyachko}). Furthermore, the proto-Austronesian form *anu can be assumed to have been a dedicated filler (see footnote \ref{footnote:rosepakendorf:4}).

In contrast to dedicated fillers, the source of many fillers is still synchronically present in the language. The filler might be perfectly homonymous with its (lexical or phrasal) source form, or its form might have been modified through a process of pragmaticalization, yet the similarity with its source makes the diachronic relation obvious. An example of perfect homonymy is that of the Even filler \textit{iak}, which is indistinguishable from the interrogative ‘what’ \citep{Matić2008}. An example of a transparent source is that of the French placeholder \textit{machin}. It is obviously formally similar to \textit{machine} ‘machine’ (it originates in a word for war machinery, \citealt{Mihatsch2006}), even though it differs from it phonologically (/maʃẽ/ vs /maʃin/) and morphologically (\textit{machin} is masculine, \textit{machine} is feminine). Similarly, the Amazonian Spanish filler \textit{este} is clearly derived from the masculine singular proximal demonstrative, but has developed a different prosodic form \citep{Vallejos-yopan2023}.

Potential sources for placeholders were listed by \citet{Podlesskaya2010}. The most common source for placeholders are pronouns, especially demonstrative and interrogative pronouns.\footnote{Note that \citet[12]{Podlesskaya2010} includes personal and indefinite pronouns in her list of pronominal sources of placeholders. These seem to be extremely rare: the only examples we have come across (thanks to Wiltrud Mihatsch) are French pronouns such as \textit{quelque chose} ‘something’ or \textit{quelqu’un} ‘someone’ which are modified by determiners and which function as avoidance placeholders, as shown by the author’s description: “…the pronoun is the substitute for a noun one cannot or will not provide” and “In this respect, the designative use is not unlike \textit{truc}, \textit{machin}, \textit{bidule} […], and other \textit{thingummy} and \textit{whatsit}.” (\citealt[10, 13]{Larrivee2009}).} Demonstratives used as fillers are cross-linguistically common: among others, they are frequently used in Japanese and Korean (\citealt{Hayashi2006}, \citealt{Seraku2022a}), in Austronesian languages (\citealt{Wouk2005}, \citealt[388--389]{Kluge2017}, \citealt{chapters/billings_mcdonnell}, \citealt{chapters/mcdonnell_billings}), in Kuuk Thaayorre from Australia \citep[252--254]{Gaby2017}, and in Russian (\citealt{Podlesskaya2009}). Interrogative-based fillers, too, are prevalent in the world’s languages, having been described for Mandarin Chinese \citep{Cheung2015}, Austronesian languages (\citealt{Dimock2010}, \citealt{Kluge2017}, \citealt{Nagaya2022}, \citealt{chapters/billings_mcdonnell}), Even and Evenki (\citealt{Matić2008}, \citealt{Klyachko2022}), Dolgan \citep{Dabritz2018}, and Udi and Agul spoken in the Caucasus \citep{Ganenkov2010}. Another common source is that of a semantically bleached noun, often a general noun, such as the Teko hesitative, which originates in a general noun for non-humans (\citealt{chapters/rose}). In Armenian, too, a noun with a general meaning, \textit{ban}, can be used as a filler (\citealt{KhurshudyanPodlesskaya2006}), and in Kalamang the general noun functions as a placeholder with avoidance use (\citealt{chapters/visser}; see \sectref{sec:intro:5.1}). In contrast, French \textit{machin} and \textit{truc} are examples of fillers whose lexical sources were semantically specific to start with. A third common source of fillers are lexicalized constructions, often involving an interrogative and a naming noun or verb. A typical example is English \textit{whatchamacallit}; further fillers with a phrasal source are found in Russian (\citealt{Podlesskaya2022}), Dalabon (\citealt{chapters/ponsonnet}), and Northern Pastaza Kichwa (\citealt{chapters/rice}). A possible source that has not been mentioned before is a copula: Kolyma Yukaghir spoken in Siberia has an element \textit{ʎe}- that functions as a copula on the one hand and as a placeholder on the other. The functions can be distinguished on syntactic and intonational grounds, but the precise historical link between these two synchronically distinct items is unclear. However, \citet{chapters/ventayol_boada} argues in favour of the placeholder developing out of the copula.

\citeauthor{Podlesskaya2010} mentions that placeholders can also result from a combination of the potential sources mentioned above. In fact, in several languages it is not straightforward to identify which element functions as the placeholder \textit{per se}, since these consist of multiple items. For instance, the Mandarin Chinese complex phrase \textit{na-ge shenme} consists of the distal demonstrative \textit{na} combined with the neutral classifier \textit{ge} followed by the interrogative \textit{shenme} ‘what’. \citet[493--494]{Hayashi2006} include this in their discussion of the placeholder use of demonstratives, saying that the distal demonstrative used as a placeholder is frequently followed by the interrogative. In contrast, \citet{Cheung2015} includes the same expression in his discussion of “wh-placeholders” in Mandarin Chinese, stating that in “a wh-placeholder, the wh-word is typically preceded by the distal demonstrative marker \textit{na} and the generic classifier \textit{ge}” (p. 280). Apart from the fact that these complementary analyses demonstrate that the placeholder lies in the eyes of the beholder, they also indicate that it is probably the entire phrase that functions as a placeholder, rather than just one of its elements.

A similar issue concerns the Japanese adnominal demonstrative \textit{ano}. This is categorized as a “demonstrative-derived placeholder” by \citet[120--121]{Seraku2022a}, yet in the examples it occurs together with semantically vague nouns, e.g. ‘thing’ or ‘place’. It is thus open to debate whether it is the demonstrative which carries the placeholder function or the vague noun, and it might be preferable to view the entire demonstrative-noun phrase as the placeholder, rather than just one of its elements.

Lastly, it should be noted that whereas it is relatively straightforward to identify a dedicated filler as a filler, it can be very difficult to distinguish between the source item and the filler in languages that still have both. We discuss this issue in more detail in \sectref{sec:intro:9.2.1}. 

\section{A morphosyntactic typology of fillers}\label{sec:intro:4}

In this section, we outline our refined morphosyntactic typology of fillers, based on three criteria: 

\begin{itemize}
\item the referentiality of the filler;
\item the morphosyntactic integration of the filler;
\item the possible part of speech of the syntactic projection.
\end{itemize}

The morphosyntactic integration and referentiality of the filler let us distinguish between placeholders and hesitatives, and whether or not the syntactic projection is restricted to particular parts of speech distinguishes between specific and general fillers. We therefore obtain a four-way distinction between \textsc{specific} \textsc{placeholders} (\sectref{sec:intro:4.1}), \textsc{general} \textsc{placeholders} (\sectref{sec:intro:4.2}), \textsc{specific} \textsc{hesitatives}\footnote{ Note that \citet[136]{Dimock2010} already distinguishes between “syntactically specific delay fillers” and a “general interjection hesitator”.} (\sectref{sec:intro:4.4}), and \textsc{general} \textsc{hesitatives} (\sectref{sec:intro:4.3}). \textsc{Versatile} \textsc{fillers} (\sectref{sec:intro:4.5}) are used for both placeholder and hesitative strategies; in their placeholder function they can be either general or specific.   

It is important to note that many languages have several fillers, of either one or several types. In particular, Austronesian languages stand out typologically by their rich and varied repertoire of fillers. Thus, in Papuan Malay for example, apart from the placeholder \textit{siapa} dedicated to substituting for personal names, the interrogatives \textit{bagemana} ‘how’ and \textit{apa} ‘what’ function as placeholders, and the demonstratives \textit{ini} and \textit{itu} function as placeholders for nouns and verbs as well as – albeit infrequently – as hesitatives (\citealt[288--297, 388--389]{Kluge2017}).

\subsection{Specific placeholders}\label{sec:intro:4.1}

Fillers that are always referential and morphosyntactically integrated are placeholders. Some placeholders are restricted to particular parts of speech or a particular subset of projected nouns or verbs and are hence \textsc{specific}. Examples of placeholders that are restricted to particular parts of speech are the Georgian placeholder verb \citep{AmiridzeGeorgian2010} and the Komnzo placeholder \textit{bäne}/\textit{baf}, which is restricted to substituting for nouns and noun phrases (\citealt{chapters/doehler}), as exemplified in \REF{ex:intro:3}.

\ea \label{ex:intro:3}
{Komnzo (\citealt[\pageref{ex:tot}]{chapters/doehler}, ex.\REF{ex:tot})}\\
\gll Zöbthé zwa{\textbackslash}wärez/é \textbf{bäne=me} (280ms) \uline{kofä} \uline{tot=me}.\\
     first 1\textsc{sg}>\textsc{3sg.f:rpst:pfv}/aim \textbf{\textsc{ph=ins}}  {} fish spear=\textsc{ins}\\
\glt ‘First I aimed at it \textbf{with} \textbf{the} \textbf{whatchamacallit}... \uline{with the fish spear}.’
\z

Examples of placeholders that are restricted to a particular subset of projected nouns or verbs are the Jamsay placeholder \textit{cɛ̌ː} or \textit{cì gɛ́} ‘thing’ restricted to non-human nouns \citep[475]{Heath2008}, and the Manambu placeholder verb \textit{məgi}-, which is restricted to verbs of affect and process and cannot substitute for verbs of speech, emotion, or mental process, nor ditransitive verbs or stative verbs \citep[575]{Aikhenvald2008}. Several languages have specific placeholders for names or terms for people \citep{Vogel2020}. For example, in the Tungusic language Evenki there is a split between the placeholder \textit{aŋi}/\textit{aŋə}, which has a wide range of nominal and verbal targets, and the placeholder \textit{uŋun}, which is restricted to replacing proper names, i.e. the names of humans, anthropomorphized animals, and places \citep[214]{Klyachko2022}. 

Sometimes a specific placeholder for verbs is derived from a specific placeholder for nouns. For example, the Algonquian language Maliseet\hyp Passamaquoddy  has a placeholder element \textit{íy}{}- that is specific for nouns; from this, two specific verbal placeholders can be derived, namely \textit{iy-i}- (animate intransitive, illustrated in \REF{ex:intro:4}) and \textit{iy-ŭw}- (animate transitive; \citealt{LeSourd2003}). Similarly, in Dolgan the specific verbal placeholder \textit{kimneː}- is morphologically derived from the specific nominal placeholder \textit{kim}, itself identical to the interrogative pronoun ‘who’ \citep{Dabritz2018}. In contrast, in Jamsay the verbal placeholder \textit{cì gɛ kárná}{}- ‘do (a) thing’ is derived from the nominal placeholder with the help of a light verb construction \citep[475]{Heath2008}.

\ea\label{ex:intro:4}
{Maliseet-Passamaquoddy \citep[150]{LeSourd2003}}\\
\gll Nékom \textbf{’t-olŏmi=íy-i-n}, \uline{’kisahqé-wsa-n}.\\
     he \textbf{3-forward=\textsc{ph-ai-sub}} (3).uphill-walk-\textsc{sub}\\
\glt ‘He did something going forward, \uline{walked up the bank}.’
\z

\subsection{General placeholders}\label{sec:intro:4.2}

Other placeholders, i.e. morphosyntactically integrated fillers, are \textsc{general}: they are polycategorial and can substitute for nouns, verbs, adjectives and even phrases. The Kolyma Yukaghir placeholder \textit{ʎe}-, for example, can stand in for nouns, verbs, and demonstrative roots (\citealt{chapters/ventayol_boada}), and the Indonesian proximal and distal demonstratives \textit{ini} and \textit{itu} \citep{Wouk2005} can replace nouns, verbs, and adjectives. Example \REF{ex:intro:5} shows that the placeholder \textit{itu} can have a verbal target. It then takes verbal morphology (here patient trigger and applicative). 

\ea \label{ex:intro:5}
{Indonesian \citep[242]{Wouk2005}} \\
\gll Trus, ini-nya semua \textbf{di-itu-in} \uline{di-bilang-in}.\\
     then this-\textsc{gen} all \textbf{\textsc{pt-ph-appl}} \textsc{pt}-say-\textsc{appl}\\
\glt ‘Then everything \textbf{(will) be that} \uline{(will) be explained} (to him).’
\z

\subsection{General hesitatives}\label{sec:intro:4.3}

We consider fillers that are never referential nor morphosyntactically integrated as hesitatives. These tend to be \textsc{general}, occurring in word searches of any kind or when the speaker is pondering how to continue their discourse. They can either be dedicated, like English \textit{uh} and \textit{uhm}, Nahavaq \textit{a} \citep{Dimock2010}, and Komnzo \textit{ä} (\citealt{chapters/doehler}), or have a lexical source, like Korean \textit{ku}, \textit{ce} and \textit{ceki}, which are also demonstratives in that language (\citealt{Hayashi2006}). Similarly, the Kalamang word \textit{nain} ‘like’ functions as a general hesitative, as illustrated in example \REF{ex:intro:6} (\citealt{chapters/visser}).

\ea \label{ex:intro:6}
{Kalamang (adapted from \citealt[\pageref{exe:munan}]{chapters/visser}, ex.\REF{exe:munan})}\\
\gll Mu-nan \textbf{nain} opa \textbf{nain} neba-un me et kinkin=a saerak leng wa me.\\
     3\textsc{pl}{}-too \textbf{\textsc{hes}} \textsc{ana} \textbf{\textsc{hes}} what-\textsc{3poss} \textsc{top} canoe small=\textsc{foc} \textsc{neg.exist} village \textsc{prox} \textsc{top}\\
\glt ‘They too, \textbf{like} earlier, \textbf{like} whatsit, there are no small canoes in this village.’
\z

\subsection{Specific hesitatives}\label{sec:intro:4.4}

The definitions of placeholders and hesitatives found in the literature do not allow for hesitatives, i.e. fillers that are neither referential nor morphosyntactically integrated, that are \textsc{specific} for certain types of delayed constituent. However, this type of filler is attested in Austronesian languages. 

Nahavaq, a language spoken in Vanuatu, has numerous prefixal hesitatives that are “specific to the grammatical function of the word being retrieved” \citep[120]{Dimock2010} and that differ in form according to whether the delayed constituent is a personal name, a common noun, a location, or a verb \citep[121--127]{Dimock2010}. For example, the hesitative \textit{ni} is used when a speaker is trying to retrieve a common noun \REF{ex:intro:7}; this is formally related to the “nominal marker” prefixes \textit{nV}- and \textit{ni}- that occur with most nouns in the language and differs from the dedicated hesitative \textit{a} or \textit{ma} that occurs during a search for names or terms for people as well as from the hesitative \textit{e} that indicates trouble with retrieving a location. The verbal hesitatives are different yet again: they are identical to the prefixes that index subject and mood on verbs \REF{ex:intro:8} and differ according to the person and mood of the elusive verb; they can even carry extra prefixes, such as the negative \REF{ex:intro:9} or irrealis marker, which are found on the delayed constituent. They therefore convey information about the delayed constituent even while the word search is going on, although they cannot be considered placeholders as such. They might at first glance resemble disfluent repetitions of the verbal prefix, but the filler always retains the vowel \textit{e}, irrespective of the vowel of the prefix, while the vowel of the verbal subject-mood index assimilates to that of the base \REF{ex:intro:10}.

\ea\label{ex:intro:7}
{Nahavaq (adapted from \citealt{Dimock2010}: 122)}\\
\gll En vales tuwan ko-log ke-vini \textbf{ni} na-lambut. \\
     and time \textsc{indf} \textsc{3sg.irr}{}-go \textsc{3sg.irr}{}-shoot \textbf{\textsc{hes}} \textsc{nv}{}-rat.\\
\glt ‘And sometimes he would go and shoot a \textbf{uhm} rat.’  
\z

\ea\label{ex:intro:8}
{Nahavaq (adapted from \citealt[123]{Dimock2010})}\\
\gll ... gcen \textbf{ndu} \textbf{ndu} ndu-tig ha-haropw. \\
     {} because \textbf{\textsc{hes}} \textbf{\textsc{hes}} \textsc{1incl.du}{}-roast \textsc{dup}{}-quickly.over.flames \\
\glt ‘...because we’ve \textbf{uhm} cooked it over flames.’ 
\z

\ea\label{ex:intro:9}
{Nahavaq (adapted from \citealt[124]{Dimock2010})}\\
\gll I-noq re-vwer \textbf{mi-s} mi-s-makas veq, mi-koh tey. \\
     \textsc{3sg.r}-like \textsc{3pl-}say \textbf{\textsc{hes}}\textsc{-neg} \textsc{1excl.pl-neg}-come.out \textsc{neg} \textsc{1excl.pl}{}-be \textsc{foc} \\
\glt ‘(You know) we didn’t \textbf{uhm} come out, we just stayed.’ 
\z

\ea\label{ex:intro:10}
{Nahavaq (adapted from \citealt[124]{Dimock2010})}\\
\gll Oveh, kinag \textbf{ne} no-log siley. \\
     whoa \textsc{1sg} \textbf{\textsc{hes}} \textsc{1sg.r}{}-go far \\
\glt ‘Whoa, I have \textbf{uhm} come from far away.’ 
\z

The Austronesian language Patani spoken on Halmahera also has a set of subject-specific hesitatives. These have a status in between that of free subject pronouns and verbal subject prefixes and are used to gain time while a speaker searches for an elusive verb (\citealt[113--118]{Sjanes_rodvand2024}). Additionally, in an article containing a preliminary analysis of hesitation phenomena in Western Subanon, a language of the Philippines, \citet{Blake2020} describes three NP-specific hesitatives in addition to a general placeholder that can substitute for both nouns and verbs. The hesitatives are partially reduplicated forms of the syntactic pivot markers \textit{og} ‘pivot’, \textit{nog} ‘relativizer’ \REF{ex:intro:11}, and \textit{sog} ‘locative’.

\ea\label{ex:intro:11}
{Western Subanon (adapted from \citealt[14]{Blake2020})}\\
\gll Ongon dosop og laki:, \textbf{nogog} pokpanow bu mama' nog lolingitan.\\
     exist also \textsc{piv} man \textsc{\textbf{hes}} walking and looks \textsc{comp} angry \\
\glt ‘There’s also a man \textbf{who-ah} is walking and looks like (he’s) angry.’
\z

The syntactically specific hesitation markers in Nahavaq, Patani, and Western Subanon do not fit into the established typologies, which contrast morphosyntactically integrated placeholders that substitute for particular words with hesitatives that are not morphosyntactically integrated and function merely to stall for time. In contrast, these Austronesian forms carry grammatical meaning and are syntactically specific, but they do not hold the place for a particular word, they merely delay the completion of the utterance (cf. \citealt{Blake2020}: 16). It can only be hoped that further research on more Austronesian languages will uncover more such cases, so that the extent of variation in these syntactically specific hesitatives can be uncovered.

\subsection{Versatile fillers}\label{sec:intro:4.5}
\largerpage
In contrast to the fillers described above, some languages have a single \textsc{versatile} \textsc{filler}, which is used both as a hesitative and a placeholder (either specific or general). This is found for example in Ilocano, a language spoken in the Philippines, where the “versatile empty root” \textit{kua} is used as a hesitative, as in \REF{ex:intro:12}, as well as to substitute for nouns, verbs \REF{ex:intro:13}, or clauses \REF{ex:intro:14} \citep{Rubino1996}. A similar situation is found in the Tungusic language Negidal, where a single element functions as a general hesitative and a general placeholder (\citealt{chapters/pakendorf}). 


\ea\label{ex:intro:12}
{Ilocano (adapted from \citealt[657]{Rubino1996})}\\
\gll Ah, \textbf{kua}, bigla nga n-ag-idda ti kawayan.\\
     \textsc{hes} \textbf{\textsc{hes}} suddenly \textsc{lig} \textsc{pst-intr-}lie\textsc{:3}\textsc{sg.abs} \textsc{art} bamboo\\
\glt ‘Ah, \textbf{um}, suddenly he lay down on the bamboo…’
\z

\ea\label{ex:intro:13}
{Ilocano (adapted from \citealt[659--660]{Rubino1996})}\\
\gll \textbf{N-ag-kua} didiey. \uline{N-ag-waras-en} dayta nga balita-n.\\
     \textbf{\textsc{pst-intr}}\textbf{{}-}\textbf{\textsc{ph}} \textsc{dist:dem} \textsc{pst-intr}{}-spread-\textsc{compl:asp} that \textsc{lig} news-\textsc{part}\\
\glt ‘That stuff \textbf{did} \textbf{whatchamacallit}, that news \uline {spread}.’ \\
\z

\ea\label{ex:intro:14}
{Ilocano (adapted from \citealt[658]{Rubino1996})}\\
\gll Tatta n-ag-aramid-da ngay ti building nga \textbf{kua-..} nga \uline{kasla} \uline{Shoemart} \uline{ti} \uline{itsora-na}.\\
     now \textsc{pst-intr}-made-3\textsc{pl.abs}  \textsc{part}   \textsc{art} building \textsc{lig} \textbf{\textsc{ph}} \textsc{lig} like Shoemart:mall \textsc{art} appearance-3\textsc{sg.erg}\\
\glt ‘So then they made a building that \textbf{is} \textbf{whatchamacallit}.. that \uline{looks like Shoemart mall}.’
\z

In other languages, such as the Nakh-Daghestanian languages Udi and Agul or Armenian, a single form can be used as a hesitative \REF{ex:intro:15} as well as to substitute for nouns \REF{ex:intro:16} and – in Armenian – even adjectives, while a verbal placeholder is derived from this form with a light verb \REF{ex:intro:17} (\citealt[81]{Khurshudyan2006}, \citealt{KhurshudyanPodlesskaya2006}, \citealt{Ganenkov2010}).

\ea\label{ex:intro:15}
{Armenian (adapted from \citealt[7]{KhurshudyanPodlesskaya2006})}\\
\gll Na \textbf{ban} ibrev petkh a gar ēsor.\\
     \textsc{3sg} \textbf{\textsc{hes}} as.if must \textsc{aux.prs.3} come.\textsc{pst.}3 today \\
\glt ‘He, \textbf{uhm}, seems to have been supposed to come today.’
\z

\ea\label{ex:intro:16}
{Armenian (adapted from \citealt[9]{KhurshudyanPodlesskaya2006})}\\
\gll Hiš-um es \textbf{ban-ə} ber-el ēir \uline{aparat-ə}…\\
     remember-\textsc{ipfv} \textsc{aux.2sg} \textbf{\textsc{ph-def}} bring-\textsc{pfv} \textsc{aux.pst.2sg} camera-\textsc{def}\\
\glt ‘Do you remember, you brought \textbf{the} \textbf{whatchamacallit}, \uline{the camera}, ….’
\z

\ea\label{ex:intro:17}
{Armenian (adapted from \citealt[13]{KhurshudyanPodlesskaya2006})}\\
\gll Yes ēl mi angam \textbf{ban} \textbf{ar-echi} \uline{gn-achi} ēd sayth-ə spyware-i…\\
     \textsc{1sg} also one time \textbf{\textsc{ph}} \textbf{do-\textsc{aor.1sg}} go-\textsc{aor.1sg} this site-\textsc{def} spyware-\textsc{gen}\\
\glt ‘I also once \textbf{did} \textbf{whatchamacallit}, \uline{went} onto this site “spyware”…’
\z

Surprisingly, such versatile fillers are not uncommon in the world’s languages: apart from the languages mentioned above, they are found in Even and Evenki (\citealt{Matić2008}, \citealt{Klyachko2022}), Mandarin Chinese (\citealt{Hayashi2006}), Tagalog \citep{Nagaya2022}, Nasal (\citealt{chapters/billings_mcdonnell}), Besemah (\citealt{chapters/mcdonnell_billings}), Papuan Malay \citep[388--389]{Kluge2017}, Manambu \citep[573--574]{Aikhenvald2008}, Amazonian Spanish \citep{Vallejos-yopan2023}, Northern Pastaza Kichwa (\citealt{chapters/rice}), Sliammon Salish \citep{Watanabe2010} and French (\citealt{Hennecke2022}), and further languages listed in \citet{Seraku2025}. It is possible that versatile fillers have been overlooked by researchers biased by the terminological dichotomy of placeholders vs. hesitatives, and that close examination of naturally occurring fillers will reveal even more examples.

\section{Extended functions of fillers}\label{sec:intro:5}

Fillers are mainly discussed in the light of disfluencies, e.g. in \citet{Hayashi2006}, who focus on “demonstratives as ‘filler words’ in contexts where speakers \textbf{encounter trouble} recalling a word or selecting the best word to use to designate some entity during the course of producing an utterance” (485, our highlighting). Similarly, the paper by \citet{Podlesskaya2010} “focuses on a type of discourse marker that \textbf{signals production difficulties} in spontaneous spoken discourse” (11, our highlighting). However, it was pointed out early on (e.g. \citealt{Enfield2003}) that placeholders can be used intentionally by speakers, with a variety of functions, namely socially motivated avoidance of a particular term (\sectref{sec:intro:5.1}), to manage interactions (\sectref{sec:intro:5.2}), and to refer to arbitrary referents (\sectref{sec:intro:5.3}). However, it should be noted that in studies based on oral corpora it is not always clear what motivates particular instances of filler use. 

\subsection{Socially motivated functions}\label{sec:intro:5.1}

Socially motivated intentional uses have been given different labels in the literature: “therapeutic” vs. “diplomatic” \citep{Tarnyikova2019}, “communicative” vs. “social” \citep{Seraku2020}, and “PH\textsubscript{A}” for placeholders motivated by the speakers’ ability vs. “PH\textsubscript{P}” for uses motivated by the speakers’ preference \citep{Seraku2024}. Among the socially motivated intentional uses, two major sub-functions can be distinguished, namely the  “avoidance” of socially sensitive terms (aka taboo) and the “conspiratorial” function, used “to prevent potentially overhearing third parties from understanding, and/or to create a collusive air between interlocutors” (\citealt[106]{Enfield2003}, see also \citealt[501--507]{Hayashi2006}, \citealt{Keevallik2010}, \citealt{Cheung2015}). Example \REF{ex:intro:18} shows the use of a placeholder to avoid embarrassment due to the explicitness of ‘die’. Example \REF{ex:intro:19} shows the use of a placeholder to prevent dinner guests from understanding the hosts’ plan for a surprise.

\ea\label{ex:intro:18}
{Mandarin Chinese (adapted from \citealt[276]{Cheung2015})}\\
\gll Na ge lao taitai yijing na-ge \textbf{shenme-le}.\\
     \textsc{dem} \textsc{clf} old lady already \textsc{dem}-\textsc{clf} \textsc{\textbf{ph-pfv}}\\
\glt ‘The old lady has already … \textbf{you-know-what-ed} (= died).’
\z

\ea\label{ex:intro:19}
{English \citep[106]{Enfield2003}}\\
 \textup{[After dinner, the host says to his wife]} I think it’s time to serve the \textbf{you-know-whats} (=peaches).\\
 \z

\citet{Palacios_martinez2015} list several pragmatic meanings of English placeholders, such as when they are used derogatorily, as insults, with the goal of not sounding pretentious, as euphemisms, and informally to build in-group identity. A further intentional strategy of placeholders are “rhetorical” uses, found for instance with Japanese demonstrative-derived placeholders used in internet articles and blog posts: these are written not in the \textit{hiragana} script standardly used for demonstratives, but in the distinct \textit{katakana} script, thus visibly signalling their special function to the reader \citep{Seraku2022a}. Furthermore, it is possible that in the Tupí-Guaraní language Teko a particular construction involving the filler has a euphemistic or suspense-creating function (\citealt{chapters/rose}); however, this cannot be asserted with certainty due to the small number of examples. 

Placeholders can be functionally specific, with forms used only in contexts of disfluency, such as \textit{uŋun} in Negidal (\citealt{chapters/pakendorf}), or forms used only in socially motivated contexts, where the target is mostly omitted, such as Nahavaq \textit{na-lan} \citep[130--131]{Dimock2010}. In several languages, both types of functionally specific placeholders co-exist. For example, in Kuuk Thaayorre, a Pama-Nyungan language of northern Australia, the distal demonstrative \textit{yuunhul} serves as a filler in situations where the speaker wants to “signal that the inaccessibility of the target lexeme […] is disruptive to the flow of speech, and must be repaired before proceeding” \citep[253]{Gaby2017}, whereas the proximal demonstrative \textit{inhul} is used when the speaker feels that it is unnecessary or even undesirable to fill in the target, such as in taboo situations \citep[254]{Gaby2017}. English makes a similar distinction between \textit{whatchamacallit}, used when a speaker cannot retrieve or doesn’t know the target word, and \textit{you-know-what}, used when the speaker does not want to utter the target for particular social reasons \citep[105--107]{Enfield2003}. Similarly, Kalamang has distinct placeholders used for word retrieval and taboo substitution (\citealt{chapters/visser}). Thus, in these languages the distinction between “ability” and “preference” \citep{Seraku2024} is formally marked. Such functional specificity is far from being generalized, however: speakers of the Papuan language Komnzo and the Austronesian language Nasal use the same filler in taboo situations as in situations of word search (\citealt{chapters/doehler, chapters/billings_mcdonnell}).

\subsection{Interaction management}\label{sec:intro:5.2}

Several studies have demonstrated that fillers (both hesitatives and placeholders) are used not only in contexts of word search or trouble with speech planning, but also to manage interactions in discourse (e.g. \citealt{Huang2005}, \citealt[162--167]{Keevallik2010}, \citealt[178]{Watanabe2010}, \citealt{JehoulEtAl2016}, \citealt{KosmalaCrible2022}). \citet[90]{Clark2002} list many interpretations of the English fillers that are related to interaction management, like “speakers want to keep the floor”, “speakers want to cede the floor”, “speakers want the next turn”, among others. \citet{AllwoodEtAl2005} therefore prefer to use the term “Own Communication Management” to cover phenomena otherwise referred to as “hesitation”, “disfluency”, or “self-repair”, and \citet[220]{KosmalaCrible2022} introduce the term “fluenceme” for elements like fillers, in order to underline their “potential to serve both fluent and disfluent functions”. Example \REF{ex:intro:20} shows how a filler (the Estonian pronominal demonstrative \textit{see}) is used turn-initially, here specifically to initiate a “reason-for-the-call turn”.

\ea\label{ex:intro:20}
{Estonian (adapted from \citealt[164]{Keevallik2010})\footnote{The equal sign in the text line indicates “latching of turns or words”, and .h indicates in-drawn breath \citep[171]{Keevallik2010}.}}\\
\gll \textbf{=see}, .h ma tahsin seda küsida et\\
     \textbf{\textsc{fil}} {} I want:\textsc{pst:1sg} this:\textsc{prt} ask:\textsc{inf} that\\
\glt ‘\textbf{Uhm}, I wanted to ask you’
\z

\subsection{Arbitrary reference}\label{sec:intro:5.3}

Even though it has been posited that the prototypical use of a placeholder is to signal to the hearer that a specific referent should be looked for in the context (e.g. \citealt[300]{Hennecke2022}), \citet{Seraku2022b} identifies arbitrary reference as a function of Japanese and Korean interrogative-derived and Romanian and Bulgarian demonstrative-derived placeholders \REF{ex:intro:21}. In such uses, the placeholder is not used to substitute for a specific referent, but precisely to fill in for some arbitrary entity. 

\ea\label{ex:intro:21}
{Japanese \citep[436]{Seraku2022b}}\\
\gll Boku jitsuwa mainichi nikki kai-teru-ndesu. Kyoo-wa \textbf{nani-ga} at-ta-toka \textbf{nanishi-ta-toka} \textbf{nani} tabe-ta-toka.\\
     1\textsc{sg} in.fact every.day diary write-\textsc{ipfv-mm.hon} today-\textsc{top} \textbf{\textsc{ph-nom}} happen-\textsc{pst}-etc. \textbf{\textsc{ph-pst}-etc}. \textbf{\textsc{ph}} eat-\textsc{pst}-etc.\\
\glt ‘[In this post, the writer conveys that writing up a diary every day reduces his stress.] In fact, I write in my diary every day, like “\textbf{Such-and-such} happened today,” “I did \textbf{such-and-such},” “I ate \textbf{such-and-such},” and so on.’ 
\z

\subsection{Issues with extended functions}\label{sec:intro:5.4}

This section has presented several functions of fillers beyond their use in disfluency. It should be noted, however, that the extended functions of placeholders (avoidance and conspiratorial use, interaction management or arbitrary reference) have not been identified for dedicated placeholders with opaque etymology, but only for placeholders derived from demonstratives, interrogative pronouns, or general/semantically empty nouns. This can partly be explained by the fact that certain functions have only recently been identified, such as the rhetorical or the arbitrary reference function (\citealt{Seraku2022a}, \citeyear{Seraku2022b}). It is thus possible that some dedicated fillers might be found to have these functions if more attention is directed to finding them. Nevertheless, the paucity of dedicated filler items with extended functions raises the question whether these extensions are actually functions of the filler, or whether they aren’t rather functions of the base form, i.e. the demonstrative, interrogative, or general noun, which developed into a filler on the one hand and into a marker of arbitrary reference or rhetorical flag on the other.

\section{Fillers and general extenders}\label{sec:intro:6}

Several descriptions of fillers include general extenders in the scope of their discussion (e.g. \citealt[573--575]{Aikhenvald2008}, \citealt{Ganenkov2010}, \citealt{Maisak2023}, \citealt{chapters/klyachko}, \citealt{chapters/rose}). Such general extender expressions add an item with non-specific reference to an existing (even minimal) list, thus creating an “ad hoc category” (\citealt{Mauri2018}). They are optional structures which typically consist of a conjunction and a noun phrase, occur in phrase- or clause-final position (\citealt[1]{Overstreet2021}), and can be translated as ‘and so on’ or ‘et cetera’. What fillers, and notably placeholders, and general extenders have in common is that they both belong to the category of vague language. In addition, both fillers and general extenders frequently develop out of general nouns or interrogatives (\citealt{Hayashi2006}; \citealt[28]{Mauri2018}), and are therefore formally related. For instance, the \textit{baʔe} root in Teko is used as a noun for non-humans \REF{ex:intro:22}, as a hesitative \REF{ex:intro:23}, and as a general extender \REF{ex:intro:24}.

\ea\label{ex:intro:22}
{Teko (\citealt[\pageref{ex:rose:4}]{chapters/rose}, ex.\REF{ex:rose:4})}\\
\gll O-ho-pa  \textbf{baʔe}-kom-a=nam  o-apɨg=o  kupa=o.\\
    3-go-\textsc{compl}  \textbf{thing}-\textsc{pl}{}-\textsc{ref}=when  3-sit=\textsc{cont}  \textsc{pl}.\textsc{s}=\textsc{cont}\\
\glt ‘When all the \textbf{animals} had left, they (the men) sat down.’ \exsource{23.084w}
\z 

\ea\label{ex:intro:23}
{Teko (\citealt[\pageref{ex:rose:30}]{chapters/rose}, ex.\REF{ex:rose:30})}\\
\gll Kob (0.3) pitaŋ-am (2.3)  \textbf{baʔe} (1.2)  kɨto-ɾ=ehe  e-iba.\\
    \textsc{exist} {} child-\textsc{transf} {} \textbf{\textsc{hes}} {} frog-\textsc{reln}=with  3-pet\\
\glt ‘There is a child with his um… pet frog.’ \exsource{13.001}
\z 

\ea\label{ex:intro:24}
{Teko (\citealt[\pageref{ex:rose:36}]{chapters/rose}, ex.\REF{ex:rose:36})}\\
\gll Dati  aɾakapusa (0.8),   dati  ʃort,   t-ɨɾu, \textbf{\textit{baʔe-kom}}.\\ 
    \textsc{exist.neg} gun {} \textsc{exist.neg} shorts \textsc{nsp}-clothes thing-\textsc{pl}\\
\glt ‘There was no gun, no shorts, no clothes, \textbf{and} \textbf{so} \textbf{on}.’ \exsource{30.012}
\z 

The diachronic link between fillers and general extenders remains to be investigated; three scenarios are found in the literature. Firstly, the general extender could have developed out of the filler, as argued for Udi and Agul by \citet[111--114]{Ganenkov2010}. Secondly, the general extender may have been the base from which the filler developed. For example, \citet{chapters/klyachko} suggests that in some Tungusic languages the interrogative-based placeholders developed under the influence of the use of the interrogative stem as a general extender found in the entire family. Thirdly, it is possible that the source element developed separately into a filler and into a general extender, as with the extended functions of fillers discussed in \sectref{sec:intro:5.4}; this is the analysis provided by \citet{chapters/rose} for Teko.

\section{Fillers in language contact}\label{sec:intro:7}

Very little is known about the impact of language contact on the use of fillers. Data from Spanish-English bilinguals in Boston show that the age of arrival in the US and the amount of Spanish-exclusive communication have an impact on the phonetic shape of the hesitative \textit{uh}: individuals who arrived before late adolescence ({\textasciitilde}15 years of age) or who were born in the US and who have relatively few interlocutors with whom they use exclusively Spanish are far more likely to use a central vowel ([a] or [ə]) as their hesitative when speaking Spanish than the form commonly found in Spanish, namely [e] (\citealt{Erker2022}). This difference is likely to be due to the influence of English and demonstrates that language contact can have an impact on the form of the hesitative. Spanish borrowed a placeholder with arbitrary reference, \textit{fulano}/\textit{fulana}, from Arabic, with earliest attestations from the Middle Ages (\citealt{Gerhalter2020}). In both languages the placeholder was predominantly used in juridical and ritual texts where it replaced proper nouns, such as in prescriptions as to what the groom and bride were to say during a marriage ceremony. 

Borrowed fillers are further mentioned for Indonesian (placeholder \textit{anu}, from Javanese; \citealt[9]{Williams2009}), Kalamang (filler \textit{apa} borrowed from Indonesian; \citealt{chapters/visser}), Kolyma Yukaghir and Tungusic languages (demonstrative-derived filler \textit{èto} borrowed from Russian; \citealt{chapters/ventayol_boada}, and \citealt{chapters/klyachko}, respectively), and Northern Pastaza Kichwa (demonstrative-derived hesitative \textit{este} from Spanish; \citealt{chapters/rice}). In Besemah, contact influence from Jakarta Indonesian might have increased the frequency of use of the demonstrative-derived filler \textit{ini} over the dedicated filler \textit{anu} (\citealt{chapters/mcdonnell_billings}). However, while these brief descriptions mention the presence of these elements, which were mostly borrowed from the sociopolitically dominant language, they do not provide information on the relative frequency of the borrowed vs. the indigenous forms,\footnote{An exception is the presentation by \citet{Egorova2021} on borrowed fillers in Evenki, who mention that in oral recordings native fillers are preferred over borrowed ones, even by semi-speakers (with indigenous fillers occuring more than 4.5 times as often as borrowed ones).} and they also leave open the question to what extent these fillers of foreign origin are really integrated into the language as opposed to their occurring as nonce borrowings or in code-switches. It is thus clear that there is still a lot of scope for research in the domain of filler forms and functions in language contact situations.

\section{Fillers and the nature of language}\label{sec:intro:8}

Fillers are not only important from a descriptive point of view, but also for their potential to enlighten us on how speech is produced and how it functions in interaction. The fact that placeholders mirror the morphology of their targets, such as both case and evidential markers in Northern Pastaza Kichwa (\citealt{chapters/rice}), aspect and mood plus subject marking in Evenki, as illustrated in \REF{ex:intro:25}, or case and possessive marking in Negidal (\citealt{chapters/pakendorf}), provides evidence for the fact that the syntactic structure of an utterance is in place before the target lexeme is accessed. 

\ea\label{ex:intro:25}
{Evenki (adapted from \citealt[209]{Klyachko2022})}\\
\gll D'əm-muː-l-mi \textbf{aŋi-ŋna-kal} guː-səː əri-ŋ-mə-w tugeː \uline{sʲiwu-ŋna-kal}.\\
     eat-\textsc{des-inch-cvcond} \textbf{\textsc{ph-hab-imp.2sg}} say-\textsc{pant} this-\textsc{indr.poss-acc-1sg.poss} so lick-\textsc{hab-imp.2sg}\\
\glt ‘If you get hungry, he said, \textbf{do} \textbf{that} \textbf{thing}, \uline{lick} this one (paw) of mine.’
\z

Mismatches between the placeholder and the target also demonstrate that the syntactic specifications of an utterance are put in place early on. Thus, in Negidal mismatches between case morphemes on the placeholder and the target tend to be semantically congruent, involving cases that can both be used to mark goals or direct objects, for example (\citealt{chapters/pakendorf}). Similarly, the Russian complex placeholder \textit{ètot… kak ego} (literally ‘this one …. how is it (called)’, or ‘what is its (name))’ carries the same gender and number as the target.\footnote{The first element, the proximal demonstrative \textit{ètot}{\textasciitilde}\textit{èta}{\textasciitilde}\textit{èto} also takes the case of the target; it can, however, also stand in the invariant neuter gender form \textit{èto}.} However, in some cases there is a mismatch in gender between the placeholder and the target, such as in \REF{ex:intro:26}, where the placeholder carries feminine gender (also found on the modifier \textit{gosudarstvennuju} ‘State’), but the target is neutral gender. Here, it is probable that the initial target of the placeholder was the feminine-gender word \textit{premija} ‘bonus’, which was replaced with the final target \textit{voznagraždenie} ‘reward’. This shows that in speech production the grammatical specifications of an item, such as grammatical gender, are accessed independently and in advance of the actual lexeme (\citealt[68--69]{Podlesskaya2022}).

\ea\label{ex:intro:26}
{Russian (adapted from \citealt[68]{Podlesskaya2022})}\\
\gll Nu ladno polučiš’ gosudarstvennuju... \textbf{èto... kak eë...} \uline{voznagraždenie}~[…]\\
     well OK you.will.get State[\textsc{adj.f}] \textbf{\textsc{ph}}[\textsc{f}] reward[\textsc{n}] {}\\
\glt ‘Well, OK, you will receive a State \textbf{whatchamacallit}, reward…’
\z

\section{Scope and contribution of the volume}\label{sec:intro:9}

This section first presents the scope of the volume in terms of the languages discussed (\ref{sec:intro:9.1}), then discusses major issues (\ref{sec:intro:9.2}) and emerging topics (\ref{sec:intro:9.3}). In our discussion, we restrict ourselves to the languages of the volume.

\subsection{Languages discussed in the volume}\label{sec:intro:9.1}
As summarized in \tabref{tab:intro:2}, fillers in 16 languages are discussed in this volume, with 10 chapters being devoted to individual languages and the chapter by \citeauthor{chapters/klyachko} providing a comparison of fillers and general extenders in the Tungusic family as a whole. The table lists the languages in geographical order, from west to east and north to south. This is also the order of inclusion in the volume.

\begin{table}
\begin{tabularx}{\textwidth}{QlllQ}
\lsptoprule
Language & {Glottocode} & {Macro-area} & {Family} & {Author}\\
\midrule
Evenki & even1259 & Eurasia & Tungusic & \citeauthor{chapters/klyachko}\\
\tablevspace
Even & even1260 & Eurasia & Tungusic & \citeauthor{chapters/klyachko}\\
\tablevspace
Udihe & udih1248 & Eurasia & Tungusic & \citeauthor{chapters/klyachko}\\
\tablevspace
Oroch & oroc1248 & Eurasia & Tungusic & \citeauthor{chapters/klyachko}\\
\tablevspace
Nanai & nana1257 & Eurasia & Tungusic & \citeauthor{chapters/klyachko}\\
\tablevspace
Kur-Urmi Nanai & kuro1242 & Eurasia & Tungusic & \citeauthor{chapters/klyachko}\\
\tablevspace
Negidal & negi1245 & Eurasia & Tungusic & \citeauthor{chapters/pakendorf}, \citeauthor{chapters/klyachko}\\
\tablevspace
Kolyma Yukaghir & sout2750 & Eurasia & Yukaghir & \citeauthor{chapters/ventayol_boada}\\
\tablevspace
Besemah & cent2053 & Papunesia & Austronesian & \citeauthor{chapters/mcdonnell_billings}\\
\tablevspace
Nasal & nasa1239 & Papunesia & Austronesian & \citeauthor{chapters/billings_mcdonnell}\\
\tablevspace
Kalamang & kara1499 & Papunesia & West Bomberai & \citeauthor{chapters/visser}\\
\tablevspace
Komnzo & komn1238 & Papunesia & Yam & \citeauthor{chapters/doehler}\\
\tablevspace
Dalabon & ngal1292 & Australia & Gunwinyguan & \citeauthor{chapters/ponsonnet}\\
\tablevspace
Mohawk & moha1259 & N. America & Iroquoian & \citeauthor{chapters/mithun}\\
\tablevspace
Northern Pastaza Kichwa & nort2973 & S. America & Quechuan & \citeauthor{chapters/rice}\\
\tablevspace
Teko & emer1243 & S. America & Tupian & \citeauthor{chapters/rose}\\
\lspbottomrule
\end{tabularx}
\caption{Languages discussed in the present volume}
\label{tab:intro:2}
\end{table}

The languages represented in this volume are found in different macro-areas (\figref{fig:intro:1}). Papunesia is the best represented, with descriptions of fillers in both Austronesian and Papuan languages, while African languages are unfortunately not included at all. Thus the volume unintentionally mirrors the general state of cross-linguistic descriptions (\sectref{sec:intro:2}).

\begin{figure}
\includegraphics[width=\textwidth]{figures/Fillersrefinedtypologyfinal-img001.png}
\caption{Map of the languages studied in the volume}
\label{fig:intro:1}
\end{figure}

All studies are based on (mostly first-hand) corpora of oral recordings, sometimes supplemented with video recordings which allowed the authors to include observations on the gestures that accompany particular uses of fillers (see \sectref{sec:intro:9.3.3}). While most chapters provide synchronic descriptions of fillers, three chapters deal with diachronic aspects to various extents: \citeauthor{chapters/ventayol_boada} suggests that the Kolyma Yukaghir placeholder may have developed out of the copula \textit{ʎe}-, and \citeauthor{chapters/rice} proposes that the Northern Pastaza Kichwa filler \textit{mashti} developed out of the phrase “what name”, with the placeholder uses developing first via phonetic erosion and semantic bleaching. The placeholder later evolved into both a hesitative (via cooptation) and a pro-verb (a case of (re)lexicalization). \citeauthor{chapters/rose} discusses the development of discourse functions of the general noun with non-human reference \textit{baɁe} in Teko: the general noun probably developed into a hesitative via placeholder uses, in a process that involved semantic bleaching, reanalysis, extension, and prosodic changes. The hesitative may have been the source for the general verb ‘do’, though that remains somewhat speculative due to the lack of bridging contexts in the corpus. In contrast, the general noun independently developed interrogative, general extender, rhetoric and nominalization functions.

\subsection{Major issues}\label{sec:intro:9.2}

This section highlights four major issues identified on the basis of the chapters within this volume: 1) the distinction between fillers and related elements; 2) the distinction between the hesitative and the placeholder uses of a given filler; 3) the rather low frequency of overt targets after placeholders in several languages; and 4) individual speaker variation.

\largerpage
\subsubsection{Methodological issues with identifying fillers}\label{sec:intro:9.2.1}
Dedicated fillers can straightforwardly be identified as such; however, disentangling filler uses from uses of the source item (i.e. demonstrative, interrogative, or general noun) or from future developments\footnote{In this volume, fillers are described as developing into predicates (verb ‘make’ in Teko and pro-verb in Mashti), and connectors (in Komnzo) or markers of dependency between clauses (in Mohawk).}  can be problematic in languages where fillers co-exist with their sources or their extensions. Criteria used by the authors of the chapters in the present volume to distinguish fillers from their sources are the following:

\begin{itemize}
\item form
\item prosody
\end{itemize}
\begin{itemize}
\item position within utterances
\item function in the discourse structure
\item semantics
\item morphological combinability
\item syntactic distribution
\end{itemize}

The chapter by \citeauthor{chapters/rose}, for example, shows how the syntactic, morphological, semantic and prosodic characteristics of \textit{baʔe} in Teko are sufficient to distinguish three lexical units and identify particular occurrences as either nouns, verbs or hesitatives. As a noun for non-human referents, \textit{baʔe} is found in the syntactic position of nominals, takes nominal morphology, refers to non-human entities, and is fully integrated in the phrase in which it occurs. As a verb, \textit{baʔe} is found in the syntactic position of verbs and takes verbal morphology. As a hesitative, \textit{baʔe} does not show a restricted syntactic distribution, it never takes any morphology, can be found before delayed constituents referring to human entities, and is often lengthened and surrounded by pauses. 

However, there are limits to the use of formal criteria to distinguish fillers from their sources or their future developments. For example, in Teko, the identification of placeholder uses of the same form \textit{baʔe}, distinct from the basic general noun, is not straightforward. The reverse might also be true: \citeauthor{chapters/doehler} explains how in a first analysis of Komnzo \textit{bäne} he considered it to be both a demonstrative and a placeholder. However, a careful examination of all text examples leads him to the conclusion that all occurrences are actually linked to disfluency, and he does not consider \textit{bäne} as a demonstrative any more. In Mohawk, the proximal and distal demonstratives are used both in contexts of disfluency as hesitatives and especially as placeholders, but they are also found in contexts without disfluency: first, in a discourse structure where they basically allow the speakers to hold the floor, and second in a complex syntactic structure, marking the fact that a complement or relative clause follows. However, the different uses cannot be distinguished by their prosody, their morphology (they do not take any), or their position. In the lack of formal evidence for the speaker’s communicative intention, the analysis is based solely on the investigator’s interpretation of the discourse function. 

\subsubsection{Indeterminacy of placeholder and hesitative use in versatile fillers}\label{sec:intro:9.2.2}

Hesitatives and placeholders are often portrayed as being separate elements that are clearly distinguishable (most notably in \citealt{Hayashi2006}). This is also found by \citeauthor{chapters/visser} for Kalamang, where hesitatives and placeholders are distinct lexical items with no functional overlap. In other languages the filler can straightforwardly be analysed as being mostly a hesitative, as in Teko (\citeauthor{chapters/rose}), or a placeholder, as in Kolyma Yukaghir (\citeauthor{chapters/ventayol_boada}), Dalabon (\citeauthor{chapters/ponsonnet}), and Mohawk (\citeauthor{chapters/mithun}). However, there are also languages with versatile fillers which are used both as hesitatives and placeholders (\sectref{sec:intro:4.5}), namely Negidal (\citeauthor{chapters/pakendorf}), Besemah (\citeauthor{chapters/mcdonnell_billings}), Nasal (\citeauthor{chapters/billings_mcdonnell}), and Northern Pastaza Kichwa (\citeauthor{chapters/rice}). The relative frequency of one or the other strategy differs across the languages: while in Nasal the demonstrative-derived fillers predominantly function as placeholders and the interrogative mainly functions as a hesitative, in Besemah placeholder uses dominate for all filler forms. Similarly, in Negidal placeholder uses are about twice as frequent as hesitative uses; in contrast, in Northern Pastaza Kichwa the hesitative use is by far the most frequent in the corpus, even though it can be assumed to have developed out of the placeholder use. 

What is striking, however, is that in languages with versatile fillers it is at times hard or even impossible to determine what function a particular token of the filler might have, in spite of analyses of prosody and intonation. These studies thus demonstrate that in some languages the filler is indeed a single polyfunctional item with a continuum of uses from hesitative to placeholder. Whether this indeterminacy also holds for the other languages discussed in \sectref{sec:intro:4.5} would need to be verified with prosodic analyses. 

\subsubsection{Low frequency of overt targets}\label{sec:intro:9.2.3}

A third striking result is that targets are regularly not present after a placeholder in several languages discussed in the volume, namely in Negidal (\citeauthor{chapters/pakendorf}), Kolyma Yukaghir (\citeauthor{chapters/ventayol_boada}), Besemah (\citeauthor{chapters/mcdonnell_billings}), Nasal (\citeauthor{chapters/billings_mcdonnell}), Kalamang (\citeauthor{chapters/visser}), and Dalabon (\citeauthor{chapters/ponsonnet}). For instance, the Negidal placeholder is followed by an overt target in 70\% of its occurrences as a nominal placeholder, but in only 50\% of its occurrences as a verbal placeholder. Even more surprising, in Besemah and Kalamang targets are actually more frequently absent than present after a placeholder. \citeauthor{chapters/mcdonnell_billings} indicate that the Besemah placeholders are followed by a target (a repair, in their terminology), in only one third of all occurrences. 

This is an important finding, since there is a bias in the literature towards examples with overt targets for obvious expository reasons (openly acknowledged by \citealt{Dimock2010}). This bias generates an implicit expectation that a placeholder will of necessity be followed by its target – a situation that is explicit in the terminological choice made by \citet{KhurshudyanPodlesskaya2006}, who distinguish between placeholders (\textit{preparativnaja podstanovka}, literally “preparatory replacement”), which are followed by their target, and “approximate nominalizations” (\textit{priblizitel’naja nominalizacija}), which lack an overt target. The corpus-based studies included in the volume, with meticulous coding of each occurrence of the placeholders in a given language, thus provide important insights into the actual nature of placeholders. For instance, it is possible that placeholders do not always function as substitutes of specific targets that elude the speaker (and hence do not necessarily have the function to “signal that the speaker is not able or willing to provide a more specific target expression” (\citealt[300]{Hennecke2022})), but that these semantically vague expressions maintain or develop some referential uses. 

\subsubsection{Variation between speakers}\label{sec:intro:9.2.4}

Finally, several studies in the volume (\citeauthor{chapters/pakendorf}, \citeauthor{chapters/ventayol_boada}, \citeauthor{chapters/mcdonnell_billings}, \citeauthor{chapters/visser}, and \citeauthor{chapters/ponsonnet}) find notable differences in the frequency of use of fillers between individual speakers of a language, as also found for hesitatives among German women \citep{Braun2023} and placeholders among English speakers of various ages (\citealt{Palacios_martinez2015}). Such differences among speakers can partly be explained by their sociolinguistic profile: \citeauthor{chapters/mcdonnell_billings} show that younger speakers of Besemah, and especially those who have spent time outside of the region, strongly prefer the demonstrative pronoun \textit{ini} over the dedicated filler \textit{anu}. The authors explain this preference by contact with Jakarta Indonesian, which employs demonstrative pronouns as fillers. In contrast, \citeauthor{chapters/ponsonnet} finds that it is individual speakers’ preferences, which she calls ‘styles’, that determine placeholder use in Dalabon, with one speaker favouring lexical accuracy and hence using the placeholder to substitute for a (mostly nominal) target that is subsequently supplied, while another speaker favours fluidity of speech and therefore does not necessarily supply the target, with the placeholder standing in with nearly equal frequency for verbs and nouns. These styles are not strictly correlated with the speakers’ proficiency. \citeauthor{chapters/pakendorf} also suggests that the differences in the frequency of use of the Negidal filler \textit{uŋun}, as well as the differences in choice between hesitatives and placeholders, can only be partly explained by the proficiency of the speakers: while a semi-speaker of Negidal indeed uses it most frequently, a very proficient speaker also uses fillers very often (see \sectref{sec:intro:9.3.4} for other explanations).

\subsection{Emerging topics}\label{sec:intro:9.3}
This section broaches several topics that have to date been rarely discussed in the literature, but which are touched upon in the studies included in the volume, and which we consider worth exploring in greater detail and in diverse languages in the future: the interaction of fillers 1) with prosody, 2) with other markers of disfluency or other fillers, 3) with gestures, and 4) the frequency of fillers in discourse as well as 5) the impact of language contact on fillers. 


\subsubsection{Fillers and prosody}\label{sec:intro:9.3.1}
Most chapters in the volume include some discussion of prosody, a topic with little prior coverage in the cross-linguistic literature on fillers (but see \citealt{Dimock2010}, \citealt{Podlesskaya2022}, \citealt{Hennecke2022}, and \citealt{Vallejos-yopan2023}). In the chapters of the volume, the discussion mostly focuses on duration (i.e. lengthening of fillers) and pauses, but sometimes also deals with prosodic contours. The prosodic studies may have two different goals. Firstly, some authors use prosody to distinguish fillers from homonymous forms. \citeauthor{chapters/rose} uses prosodic features to investigate the distinction between the Teko hesitative and its source noun \textit{baʔe}. The hesitative is prosodically more salient via final lengthening, a higher frequency of pauses preceding and following it, and longer duration of following pauses. In contrast, \citeauthor{chapters/ventayol_boada} does not find pauses to be relevant for distinguishing the filler and copula uses of \textit{ʎe}- in Kolyma Yukaghir; rather, it is the position of the filler and the number of words in the intonation unit as well as the intonation contour that discriminate the two uses. In Komnzo, the close examination of the prosody of all occurrences of \textit{bäne} leads \citeauthor{chapters/doehler} to the conclusion that it is not a proper demonstrative in synchrony anymore, but always a placeholder. Secondly, some authors investigate how prosody correlates with different uses of the same filler. \citeauthor{chapters/rice} compares the features of prosodic lengthening and pauses in relation to the different uses of the Northern Pastaza Kichwa \textit{mashti} element, namely as a hesitative, a placeholder, and a pro-verb. In general, the hesitative use attracts more pauses and prosodic lengthening. \citeauthor{chapters/mcdonnell_billings} also show that hesitative uses of Besemah fillers are more frequently associated with other disfluency cues than their placeholder uses. In contrast, \citeauthor{chapters/pakendorf} and \citeauthor{chapters/mithun} do not find any correlation between prosodic patterns and uses of fillers as hesitatives or placeholders in Negidal and Mohawk, respectively, mirroring the lack of distinction between the hesitative and placeholder strategies of fillers in French \textit{truc} and \textit{machin} (\citealt{Hennecke2022}). The lack of a cross-linguistically clear pattern concerning the role of prosody with respect to fillers motivates our call for further research on that question. 

\subsubsection{Fillers and other markers of disfluency}\label{sec:intro:9.3.2}
Several authors (\citeauthor{chapters/mcdonnell_billings}, \citeauthor{chapters/doehler}, \citeauthor{chapters/rice}) also touch upon other markers that accompany hesitations, essentially non-lexical ones like false starts, glottalization, and repetitions, in addition to pauses and lengthening. Furthermore, some authors mention the existence of other fillers in the language they study (\citeauthor{chapters/visser}, \citeauthor{chapters/doehler}, \citeauthor{chapters/rice}). How the different lexical fillers and other types of filled pauses combine or complement each other is worth being investigated in more detail in more languages.

\subsubsection{Fillers and co-gestures}\label{sec:intro:9.3.3}
Three chapters (\citeauthor{chapters/pakendorf}, \citeauthor{chapters/doehler}, \citeauthor{chapters/rice}) include some discussion of gestures associated with fillers, a topic rarely approached for languages other than English (but see \citealt{Hayashi2003body} on Japanese, \citealt{Navaretta2015} on Danish, and \citealt{Graziano2018} and \citealt{Kosmala2024} for comparative studies). \citeauthor{chapters/rice} offers a systematic study of three types of gestures accompanying the Northern Pastaza Kichwa filler \textit{mashti}: gaze aversion, excessive blinking and manual gestures. Manual gestures can themselves undergo disfluency, such as being “on hold”, repeated, or corrected. In general, it seems that gaze aversion, manual gestures, and gestural disfluency are often associated with the filler. Just like prosodic cues, excessive blinking and gestural disfluency occur more with the hesitative than with the placeholder use of \textit{mashti}. \citeauthor{chapters/doehler} finds that about two thirds of the Komnzo placeholders are accompanied by a gesture, either hand gestures, or lip- or head- pointing gestures, which thus provide a “parallel support channel”, helping the speaker find and the hearer identify the target. Hand gestures are the most common, often pointing to the referent (in line with the demonstrative origin of the placeholder). Other gestures identify a visible target or re-enact it. In Negidal, \citeauthor{chapters/pakendorf} finds the same type of gestures accompanying the versatile filler \textit{uŋun} as those found in Northern Pastaza Kichwa and in Komnzo: gaze aversion, pointing gestures, and re-enacting gestures. She calls for a systematic study of gestures in future investigations of fillers in the languages of the world, with a close look at the timing of the gesture and the utterance of the filler, and the different possible functions of the filler.


\subsubsection{Frequency of fillers}\label{sec:intro:9.3.4}
As much as possible, the authors of the chapters included frequency counts of the fillers under study in their corpus (remember that all studies in the volume are corpus-based). The frequency of fillers can vary depending on various parameters, related to the speaker (\sectref{sec:intro:9.2.4}), content and context of the utterance, as well as the speech situation (\citealt{Corley2008}). For example, differences in the frequency of occurrences of fillers can partly be explained by genre and recording situation. Thus, \citeauthor{chapters/visser} finds that placeholders are more common in conversations than in narratives, probably because the former are less planned. Also, \citeauthor{chapters/visser} interprets differences in the individual speakers’ uses of fillers in her Kalamang corpus as being mainly related to the topic of the conversation: one speaker used very many fillers in a recording in which he was urged to talk about herbal medicine and where he had to search his memory for names of plants. Similarly, \citeauthor{chapters/pakendorf} finds that an excellent Negidal speaker uses the filler very frequently while telling traditional fairy tales; here, it might be the cultural pressure to tell such tales in a particular manner that led him to hesitate frequently.

A more general question is whether there are cultural differences in the overall frequency of fillers. It is striking that in all the chapters of the volume, the frequency of fillers in the various languages is generally higher than that found in preceding studies (\citealt{Zhao2005}, \citealt{Podlesskaya2009}). This high frequency of use of fillers found in typologically and geographically disparate languages indicates that they are far from being the marginal phenomenon they have been considered to date, leading us to call for more dedicated studies on fillers in the languages of the world.


\subsubsection{Borrowed fillers}\label{sec:intro:9.3.5}
Finally, as briefly discussed in \sectref{sec:intro:7}, there is sporadic mention of borrowed fillers in some chapters included in the volume (\citeauthor{chapters/klyachko}, \citeauthor{chapters/ventayol_boada}, \citeauthor{chapters/visser}, \citeauthor{chapters/rice}). However, detailed studies of the effect of language contact on the form, frequency, and usage patterns of fillers are still missing. Given that discourse markers such as “fillers, tags, interjections, and hesitation markers” are among the most contact-sensitive elements of language \citep[193]{Matras2009}, this would constitute a fruitful field for further research. 

\section*{Acknowledgements}

Some of the ideas developed here were presented at the workshop on “Placeholders in East and West” at the University of Tübingen in October 2024; we thank the audience for their questions. We also thank Martin Haspelmath, Eline Visser and Albert Ventayol-Boada for reading and commenting on the paper, and most especially Wiltrud Mihatsch for her expert feedback. Needless to say, any errors are entirely our responsibility.

We are furthermore extremely grateful to Rémi Anselme for his assistance with eliminating the numerous formatting and cross-referencing errors that arose as a result of conversion of the initial Word file to \LaTeX, and thank Sébastien Flavier for creating the map in \figref{fig:intro:1}.

\section*{Abbreviations}
This list includes only the glosses which are not found in the Leipzig Glossing Rules.

\begin{multicols}{2}
\begin{tabbing}
MMMMM \= animate\kill
\textsc{ai} \> animate intransitive\\
\textsc{ana} \> anaphoric demonstrative\\
\textsc{aor} \> aorist\\
\textsc{asp} \> aspect\\
\textsc{compl} \> completive\\
\textsc{cont} \> continuative\\
\textsc{cvcond} \> conditional converb\\
\textsc{des} \> desiderative\\
\textsc{dup} \> reduplication\\
\textsc{exist} \> existential\\
\textsc{fil} \> filler\\
\textsc{hab} \> habitual\\
\textsc{hes} \> hesitative\\
\textsc{hon} \> honorific\\
\textsc{inch} \> inchoative\\
\textsc{indr.poss} \> indirect possession\\
\textsc{lig} \> ligature\\
\textsc{mm} \> modal marker\\
\textsc{nsp} \> non-specific possessor\\
\textsc{nv} \> nV- nominal prefix\\
\textsc{pant} \> anterior participle\\
\textsc{part} \> particle\\
\textsc{part.o} \> partitive object\\
\textsc{pf} \> particle-final\\
\textsc{ph} \> placeholder\\
\textsc{pimp} \> past imperfective\\
\textsc{piv} \> pivot\\
\textsc{prt} \> partitive\\
\textsc{pt} \> patient trigger\\
\textsc{r} \> realis\\
\textsc{ref} \> referential\\
\textsc{reln} \> relational\\
\textsc{rpst} \> recent past\\
\textsc{seq} \> sequential\\
\textsc{sub} \> subordinative\\
\textsc{transf} \> transfer
\end{tabbing}
\end{multicols}

\printbibliography[heading=subbibliography,notkeyword=this]
\end{document}
