\documentclass[output=paper]{langscibook}
\ChapterDOI{10.5281/zenodo.15697593}
\author{Marianne Mithun\orcid{}\affiliation{University of California, Santa Barbara}}
\title{Fillers and beyond}
\abstract{In important work, \citet{HayashiYoon2006} distinguish types of fillers, elements used “in contexts where speakers encounter trouble recalling a word or selecting the best word” (\citeyear{HayashiYoon2006}: 485). Their placeholders are referential expressions used as a substitute for a specific lexical item. Their interjective hesitators are used to merely delay the production of the next element. Both roles are often filled by demonstratives, but while placeholders are produced as a syntactic constituent of a sentence, hesitators are not (\citeyear[489]{HayashiYoon2006}). \citet{HayashiYoon2006} propose that demonstratives functioning as interjective hesitators have been “pragmaticized for the discourse function of monitoring the flow of the production of an utterance to such an extent that they have diverged from ordinary demonstratives in terms of syntactic distribution, and referentiality” (\citeyear{HayashiYoon2006}: 488). Their examples are drawn from a range of languages, with special focus on Japanese, Korean, and Mandarin. Striking parallels can be seen in a typologically quite different language in another part of the world. In Mohawk, an Iroquoian language indigenous to North America, demonstratives are used similarly not only as placeholders and interjective hesitators, but also as devices for structuring the flow of information.\medskip\\
\keywords {demonstratives, prosodic structure, Mohawk, complement constructions, relative clause constructions}
}

\IfFileExists{../localcommands.tex}{
  \addbibresource{../localbibliography.bib}
  \usepackage{langsci-optional}
\usepackage{langsci-gb4e}
\usepackage{langsci-lgr}

\usepackage{listings}
\lstset{basicstyle=\ttfamily,tabsize=2,breaklines=true}

%added by author
% \usepackage{tipa}
\usepackage{multirow}
\graphicspath{{figures/}}
\usepackage{langsci-branding}

  
\newcommand{\sent}{\enumsentence}
\newcommand{\sents}{\eenumsentence}
\let\citeasnoun\citet

\renewcommand{\lsCoverTitleFont}[1]{\sffamily\addfontfeatures{Scale=MatchUppercase}\fontsize{44pt}{16mm}\selectfont #1}
   
  %% hyphenation points for line breaks
%% Normally, automatic hyphenation in LaTeX is very good
%% If a word is mis-hyphenated, add it to this file
%%
%% add information to TeX file before \begin{document} with:
%% %% hyphenation points for line breaks
%% Normally, automatic hyphenation in LaTeX is very good
%% If a word is mis-hyphenated, add it to this file
%%
%% add information to TeX file before \begin{document} with:
%% %% hyphenation points for line breaks
%% Normally, automatic hyphenation in LaTeX is very good
%% If a word is mis-hyphenated, add it to this file
%%
%% add information to TeX file before \begin{document} with:
%% \include{localhyphenation}
\hyphenation{
affri-ca-te
affri-ca-tes
an-no-tated
com-ple-ments
com-po-si-tio-na-li-ty
non-com-po-si-tio-na-li-ty
Gon-zá-lez
out-side
Ri-chárd
se-man-tics
STREU-SLE
Tie-de-mann
}
\hyphenation{
affri-ca-te
affri-ca-tes
an-no-tated
com-ple-ments
com-po-si-tio-na-li-ty
non-com-po-si-tio-na-li-ty
Gon-zá-lez
out-side
Ri-chárd
se-man-tics
STREU-SLE
Tie-de-mann
}
\hyphenation{
affri-ca-te
affri-ca-tes
an-no-tated
com-ple-ments
com-po-si-tio-na-li-ty
non-com-po-si-tio-na-li-ty
Gon-zá-lez
out-side
Ri-chárd
se-man-tics
STREU-SLE
Tie-de-mann
} 
  \togglepaper[1]%%chapternumber
}{}

\begin{document}
\maketitle 
%\shorttitlerunninghead{}%%use this for an abridged title in the page headers

\graphicspath{{figures/mithun}}

\section{Introduction}
\label{sec:mithun:1}

Fillers have usually been missing from language documentation, in good part because speakers do not typically consider them integral elements of the language; in fact, they are rarely even aware of them. Before the easy accessibility of audio or video recording technology, it was nearly impossible to document their natural use; they would rarely make it into the record when speakers were creating written materials, translating, or dictating word by word for transcription. Advances in recording technology, however, have now made such documentation possible, and the results are revealing aspects of language processing which might fill in some detail in our understanding of certain pathways by which discourse patterns and even grammatical constructions can emerge.

Here structures involving a common type of filler, demonstratives used as placeholders, are described for Mohawk, a language of the Iroquoian family indigenous to northeastern North America. The same pattern is used on occasion for interjective hesitators. It occurs much more pervasively, however, in contexts not triggered by disfluency. The development of the pattern can be attributed to cognitive factors underlying the processing of information. In what follows, \sectref{sec:mithun:2} describes the general structure of Mohawk, \sectref{sec:mithun:3} the demonstratives, \sectref{sec:mithun:4} their use as placeholders, \sectref{sec:mithun:5} their use as hesitators, \sectref{sec:mithun:6} their use in a conventionalized discourse structure, and \sectref{sec:mithun:7} their appearance in contexts where complex syntactic structures are used in other languages.

\section{Mohawk}
\label{sec:mithun:2}

Mohawk or Kanien’kéha’ (\textsc{moh}) is a language of the Northern branch of the Iroquoian family, spoken in six main communities in modern Quebec, New York State, and Ontario. Material cited here is drawn from a corpus of approximately 70,000 words of unscripted speech in a variety of genres, both conversation and monologue, from over 80 speakers. All six communities are represented.

The language is typologically polysynthetic, with potentially elaborate templatic verb morphology and extensive noun incorporation. It is head-marking: core arguments are represented by pronominal prefixes in the verb, one argument for intransitives and two for transitives. Nouns are unmarked for case, gender, or number. Alignment follows a basically agent/patient pattern. Constituent order is primarily pragmatically determined: constituents are arranged essentially in descending order of newsworthiness.

Three lexical categories are distinguished by their morphological structure: particles, nouns, and verbs. Particles by definition have no internal structure, though they may be compounded. They serve a variety of discourse, syntactic, and adverbial functions. Morphological nouns serve as referring expressions. Morphological verbs can function not only as predicates but also as referring expressions, often without any overt nominalization, and as clauses and even sentences on their own. Kinship terms, a subset of verbs more often used referentially, are sometimes classified as an additional category. Verbs are noticeably more pervasive in speech than in many other languages.

Some of these structures can be seen in example \REF{ex:mithun:1}. Transcription is in the community orthography. Digraphs <en> and <on> represent nasalized vowels [ʌ̨] and [ų], and <i> before a vowel represents a glide [j]. The layout here and throughout reflects prosody: each line represents a separate intonation unit (prosodic phrase). Intonation units are characterized by a coherent prosodic contour, typically beginning with a full or partial pitch reset, and often but not always preceded by a pause. In the transcription here, commas indicate non-final intonation unit boundaries, periods terminal prosodic boundaries, and dashes \mbox{<-{}->} truncated units. All examples are drawn from unscripted speech.

\ea%1
    \label{ex:mithun:1}
    Mohawk polysynthesis and head marking\\
    \textsc{verb}\\
   \gllll \textit{Iahonwaia’ténhawe’,} \\
    i-a-honwa-ia’t-enhaw-e’  \\
    \textsc{trl-fac-m.pl>m.sg-}body-take-\textsc{pfv}\\
    {they~physically~took~him~away}\\
    \glt ‘They took him away,\\
    ~\\
    \textsc{noun}        \\
    
\glll \textit{karihtòn:ke,} \\
    ka-rihton=’ke    \\
    \textsc{n}-police=place\\
   \glt police place\\
     ~\\
    
 \glllll
    \textup{\textsc{particle}}  \textup{\textsc{verb}}\\
    \textit{tánon’}    \textit{wahonwahnhó:ton’.}\\
    tanon’    wa-honwa-nho-ton-’\\
    and      \textsc{fac-m.pl>m.sg-}opening-cover-\textsc{pfv}\\
    and      {they~door~closed~him}\\
    \glt and put him in jail.\\
~\\
    \glllll \textup{\textsc{particle}} \textup{\textsc{particle}}  \textup{\textsc{verb}}                \textup{\textsc{noun}}   \textup{\textsc{verb}}\\
    Ó:nen   kati’   iontate’kèn:’a                Wá:ri   iontátiats,\\
    onen   kati’   iontate-’ken’=a              Wari   iontat-iat-s\\
    then    in.fact   \textsc{fi>fi-}have.as.ygr.sibling=\textsc{dim}  \textsc{name}   \textsc{fi>fi}{}-call-\textsc{hab}\\
    then    {in~fact}   {they~have~her~as~ygr~sibling}    Mary  {one~calls~her}\\
     \glt Their sister Mary
~\\
\glllll \textup{\textsc{verb}}\\
 \textit{taionkwatewennátahse’.}\\
    t-a-ionkwa-ate-wenn-at-ahs-e’\\
    \textsc{csl-fac-fi>1sg-mid-}voice-insert-\textsc{ben-pfv}\\
    {she~her~voice~inserted~to~me}\\
    \glt phoned me.’\\
    \exsource{Watshenní:ne’ Sawyer, speaker}
\z

\section{Mohawk demonstratives}
\label{sec:mithun:3}

Mohawk contains two basic demonstratives: a proximal \textit{kí:ken} ‘this, these’, and distal \textit{thí:ken} ‘that, those’, as well as a discourse demonstrative \textit{né:,} which refers either anaphorically or cataphorically to something mentioned in the discourse. Examples are in \REF{ex:mithun:2}, \REF{ex:mithun:3}, and \REF{ex:mithun:4}.

\ea%2
    \label{ex:mithun:2}
    Proximal demonstrative\\
    \glll \textit{Iakwaterohrókha’}            \textit{kí:ken.}\\
 iakwa-ate-rohrok-ha’          kiken\\
    \textsc{1excl.pl.agt-mid-}watch-\textsc{hab}  this\\
    \glt ‘We would watch this.’\\
    \exsource{Watshenní:ne’ Sawyer, speaker}
\z

\ea%3
    \label{ex:mithun:3}
    Distal demonstrative\\
    \glll \textit{Takónhewe’}                  \textit{thí:ken.}\\
    ta-k-onhew-e’                thiken\\
    \textsc{csl.fac-1sg.agt-}sweep-\textsc{pfv}    that\\
    \glt ‘I swept that up.’\\
    \exsource{Kahentoréhtha’ Marie Cross, speaker}
\z

\ea%4
    \label{ex:mithun:4}
    Anaphoric discourse demonstrative\\
    {[}Following description of how to make traditional cornbread]\\
    \gllll \textit{Né:’}     \textit{eniákwake’}                \textit{ohrhon’kè:ne.}\\
    ne:=’e    en-iakwa-k-e’              o-rhon’ke-hne\\
    that=is  \textsc{fut-1excl.pl.agt}{}-eat-\textsc{pfv}    \textsc{n}{}-dawn-place\\
    that      {we~will~eat~it}                morning\\
    \glt ‘That is what we would eat in the morning.’\\
\exsource{Watshenní:ne’ Sawyer, speaker}
\z

The first two demonstratives can occur in apposition with a nominal, as in \REF{ex:mithun:5} and \REF{ex:mithun:6}. Note that the demonstratives and nouns here were spoken in separate intonation units.

\ea%5
    \label{ex:mithun:5}
    Appositive proximal demonstrative\\
    \gll \textit{Sok}    \textit{ki’}   \textit{kí:ken,}\\
     {then~so}  this one\\
    \glt ‘So then this one,\\
    ~\\
    \glll \textit{o-{}-}  \textit{okwáho,}\\
    {} \textit{o-kwáho,}\\
    {} \textsc{n}{}-wolf\\
    \glt {the~wolf,}\\
    ~\\
   \gllll \textit{wahatakia’takéhnha’}          \textit{wahanòn:take’.}\\
    wa-ha-at-ia’t-akenha-’          wa-ha-non’t-a-k-e’\\
    \textsc{fac-m.sg.agt-mid-}body-help-\textsc{pfv}    \textsc{fac-m.sg.agt}{}-milk-\textsc{lk-}consume\textsc{{}-pfv}\\
    {he~helped~himself}            {he~consumed~milk}\\
   \glt helped himself to milk.’ \\                   \exsource{Nancy Jackson, speaker}
\z

\ea%6
    \label{ex:mithun:6}
    Appositive distal demonstrative\\
    \glll \textit{Tsi}    \textit{nòn:wa’}    \textit{takhará:tate’}                \textit{thí:ken,}\\
    tsi    nonhwa’    ta-k-haratat-e’    thiken\\
    at      now     \textsc{csl.fac-1sg.agt-}lift-\textsc{pfv}    that\\
    \glt ‘And then I lifted that thing,\\
    ~\\
    \glll    \textit{ne}    \textit{kaiarahrónnion’.}\\
    ne    ka-iar-a-hronnion-’\\
    \textsc{art}    \textsc{n}{}-bag-\textsc{lk}{}-be.on-\textsc{st}\\
     \glt the     mattress.’   \\             \exsource{Marie Kahentoréhtha’ Cross, speaker}
\z

These two demonstratives also occur, though more rarely, within the same intonation unit as a following nominal.

\newpage
\ea%7
    \label{ex:mithun:7}
Prosodic integration\\
\glll \textit{Í:se’}  \textit{ken}    \textit{sá:wen}                \textit{kí:ken}  \textit{truck?}\\
ise’    ken    sa-wen                kiken  truck\\
2        \textsc{q}    \textsc{2sg.al.poss}{}-possession  this    truck\\
\glt ‘Is this truck yours?’ \\               \exsource{Watshenní:ne’ Sawyer, speaker}
\z

\ea%8
    \label{ex:mithun:8}
Prosodic integration\\
\glll \textit{Ioterihsionhátie’}              \textit{thí:ken}  \textit{o’tá:ra’.}\\
io-ate-rihsi-on-hatie’          thiken    o-’tar-a’\\
\textsc{n.pat-mid-}disappear-\textsc{st-prog}  that      \textsc{n-}clan-\textsc{ns}\\
\glt ‘Those clans are disappearing.’       \\ 
\exsource{Watshenní:ne’ Sawyer, speaker}
\z

The discourse demonstrative \textit{né:} does not occur with a nominal in this way. An article \textit{ne} has developed from it, however, which does precede a referring expression. The article is short and unstressed and can fuse as a proclitic to a following vowel-initial nominal. Unlike its English counterpart, it also occurs with possessed nominals and proper names. Traditionally it has meant ‘the aforementioned’, but it is coming to be used as a more general definite article. Both the anaphoric demonstrative \textit{né:} and the article \textit{ne} can be seen in \REF{ex:mithun:9}, as well as fusion of the article with the following vowel-initial first person pronoun:  \textit{ne ì:’i > ní:}.

\ea%9
\label{ex:mithun:9}
Discourse demonstrative \textit{né:’} and article \textit{ne} \\
\textit{Tóta.}
\glt ‘Gramma.’\\~\\

\glll \textit{Né:’}      \textit{ní:’}      \textit{iakhina’tónhkha’}               \textit{ne}    \textit{Gramma.}\\
ne:=’e    ne=i’i    iakhi-na’tonhkw-ha’           ne    Gramma\\
that=is  \textsc{art=1}    \textsc{1excl.pl>fi}{}-call.by.name-\textsc{hab}   \textsc{art}    Gramma\\
\glt ‘That is what we called Gramma.’\\
\exsource{Watshenní:ne’ Sawyer, speaker}
\z

The proximal and distal demonstratives can co-occur with the article.

\ea%10
\label{ex:mithun:10}
Demonstrative combination\\
\glll \textit{Ok}    \textit{ò:ni’}  \textit{sénha’}    \textit{kí:ken}     \textit{ne} \textit{thotiièn:sa.}\\
ok    ohni’  senha’    kiken     ne  t-hoti-ien’sa\\
and    also    more    this       \textsc{art}  \textsc{csl-m.pl.pat-}be.young\\
\glt ‘And it’s also more these young people.’\\
\exsource{Joe Awenhráthon Deer, speaker}
\z

All three of the demonstratives, \textit{kí:ken}, \textit{thí:ken}, and \textit{né:} are morphological particles with no internal structure, though the first two can be traced historically to sequences \textit{ken’ í:ken} ‘here it.is’ and \textit{tho í:ken} ‘there it.is’. The modern forms are often shortened to \textit{ki:} and \textit{thi:} respectively. They can refer to persons, animals, objects, places, situations, ideas, etc., but, like nouns, they do not distinguish gender, number, person, or grammatical role.

The demonstratives are remarkably pervasive in speech. They serve a variety of purposes. As in many languages, they are used to distinguish previously mentioned referents, as in the examples above, and to introduce new referents, as in \REF{ex:mithun:11}.

\ea%11
\label{ex:mithun:11}
New referent\\
\glll \textit{Tánon’}  \textit{ki:}    \textit{Saksárie},\\
    tanon’    kiken  Saksárie\\
     and      this    \textsc{name}\\
~\\~\\
\glll \textit{o’nó:wa’}  \textit{ne:}        \textit{thaterennótha’}\\
o-’now-a’  ne:        t-ha-ate-renn-ot-ha’    \\
guitar      that.one    \textsc{dv-m.sg.agt-mid-}song-stand-\textsc{hab}\\
~\\~\\
\glll \textit{sok}    \textit{wahorihónnien’}\\
sok    wa-ho-rihw-onni-en-’\\
    then    \textsc{fac-m.sg>m.sg}{}-matter-make-\textsc{ben-pfv}\\
\glt ‘And he taught this guy Franklin to play the guitar.’\\
\exsource{Joe Awenhráthon Deer, speaker}
\z

Demonstratives are also used pervasively as antitopics at the ends of clauses to confirm the identity of referents and often to mark conclusion of a thought.

\ea%12
    \label{ex:mithun:12}
Antitopic\\
{[}‘Maybe you remember, they gave out horses to various people.   Some were good animals, some were bad animals. You couldn’t   hitch them up.’]\\
~\\
\glll \textit{Iáh}    \textit{ki’}      \textit{tekarihonnié:ni}              \textit{thí:.}\\
iah    ki’      te-ka-rihw-onni-en-i        thiken\\
not    actually  \textsc{neg-z}{}-matter-make-\textsc{ben-st}  that\\
\glt ‘They weren’t trained, those [horses].’\\
\exsource{Joe Tiorhakwén:te’ Dove, speaker}
\z

\section{Placeholders}
\label{sec:mithun:4}
The Mohawk demonstratives also occur as fillers. \citet{HayashiYoon2006} characterize \textit{fillers} as elements used “in contexts where speakers encounter trouble recalling a word or selecting the best word to use to designate some entity during the course of producing an utterance” (\citeyear[485]{HayashiYoon2006}). One type of filler is the \textit{placeholder}, which they define as follows.

\begin{itemize}
\item It is a referential expression that is used as a substitute for a specific lexical item that has momentarily eluded the speaker (and which is often specified subsequently as a result of a word search).
\item It occupies a syntactic slot that would have been occupied by the target word, and thus constitutes a part of the syntactic structure under constructions. (\citealt[490]{HayashiYoon2006})
\end{itemize}

\citet{HayashiYoon2006} and others have observed that the sources of placeholders are often demonstratives.

\largerpage
Both of the Mohawk demonstratives \textit{kí:ken} ‘this one, these’ and \textit{thí:ken} ‘that one, those’ are used as placeholders in just the contexts described by \citet{HayashiYoon2006} when speakers are searching for how to designate an entity. Example \REF{ex:mithun:13} shows the placeholder \textit{kí:ken}, ‘this one’, which fills the same syntactic slot as a referring expression which follows, ‘the boy going along with the stolen fruit’. (In transcriptions here shorter pauses are represented by sequences of two dots < . . >, and longer pauses by three dots < . . . >.) 

\ea%13
\label{ex:mithun:13}
Proximal demonstrative placeholder\\
\glll \textit{Rononhwaro’tsheróntion} \textbf{\textit{kiken:}} \textit{. .}  \textit{um,}    \textit{. . .}\\
ro-nonhwaro-’tsher-onti-on          kiken {} {} {}\\
\textsc{m.sg.pat}{}-scalp.cover-\textsc{nmlz}{}-lose-\textsc{st}    this {} {} {}\\
\glt ‘He lost his hat \textbf{this}  . . um,    . . .\\
~\\

\glll \uline{\textit{raksà:’a}}          \uline{\textit{tsi}}  \uline{\textit{niká:ien’}}    \uline{\textit{ne:,}}  \textit{. . .}\\
ra-ksa’=a        tsi  ni-ka-ien-’    ne {}\\
\textsc{m.sg-}child=\textsc{dim}  as    \textsc{prt-n-}lie-\textsc{st}  the.aforementioned {}\\
\glt \uline{the [aforementioned] boy who}  . . .,  \\
~\\
\glll \uline{\textit{rohianenskwenhátie’.}}\\
ro-ahi-a-nenskw-en-hatie’\\
\textsc{m.sg.pat}{}-fruit-\textsc{lk}{}-steal-\textsc{st-prog}\\
\glt \uline{the one who was going along with the stolen fruit}.’
\exsource{Annette Kaia’titáhkhe’ Jacobs, speaker}
\z

Example \REF{ex:mithun:14} shows the use of \textit{thí:ken}, ‘that one’ as a placeholder, standing in for the nominal phrase \textit{Variety Show}.


\ea%14
\label{ex:mithun:14}
Distal demonstrative as placeholder\\
\gllll \textup{MB}   \textit{Shaià:ta}                    \textit{ka’níhsera’} \\
{} s-ha=ia’t-at                ka-’ni-hser-a’  \\
{} \textsc{rep-m.sg.agt-}body-be.one  \textsc{n}{}-be.father.to-\textsc{nmlz-ns}  \\
{} {‘One~father}\\~\\~\\

\gllll \textcolor{white}{MB} \textit{kwáh}  \textit{ia’thahséntho’,}    \textit{. . .}\\
{} kwah  ia’-t-ha-ahsentho-’\\
{} just    \textsc{trl-fac-dv-m.sg.agt-}cry-\textsc{pfv}\\
{} {started~to~cry~there,}\\~\\~\\

\gll \textcolor{white}{MB} \textbf{\textit{thiken:}}\textit{{}-{}-}  \textit{. .}\\
{} \textbf{that~one} {}\\~\\~\\

\gllll \textcolor{white}{MB} \textit{nahò:ten’}          \textit{na’—}\\
{} na-h-o’ten-’        na’—\\
{} \textsc{prt-n-}be.a.kind.of-\textsc{st} {}\\
{} {what~is~it—'}\\~\\~\\

\gl \textup{WG}  \textit{\uline{Variety show}.}\\~\\
\gll \textup{MB}   \textit{\uline{Variety show}.}\\
{} {‘at~\uline{the~Variety~show}.’}\\~\\~\\
\exsource{Mina Tewateronhiáhkhwa’ Beauvais, Warisóse Gabriel, speakers}
\z

The demonstratives in these examples show all the characteristics of placeholders described by \citet{HayashiYoon2006}:

\begin{quote}
The use of a placeholder demonstrative appears to be motivated by constraints in cognitive processes, such as difficulty in remembering or “accessing” an appropriate lexical item when it should be articulated during the course of utterance production. There, it is often accompanied by those features that are typically observed during word searches, such as \textbf{intra-turn} \textbf{pauses}, \textbf{sound} \textbf{stretches}, repetitions, \textbf{hesitation} \textbf{signals} (equivalents of \textbf{uh/um}), \textbf{self-addressed} \textbf{questions}, etc. (\citealt[500]{HayashiYoon2006})
\end{quote}

The demonstrative \textit{kí:ken} in \REF{ex:mithun:13} was followed by a pause, then the hesitator \textit{um}, then another pause. The demonstrative \textit{thí:ken} in \REF{ex:mithun:14} was truncated, then followed by a pause, then by \textit{nahò:ten’ na’—}‘what is it wha-{}- ’, another sign of disfluency, as the speaker searched for a word.

The demonstratives in this function also show characteristic prosody: stressed penultimate syllable with rising pitch and added length, followed by final syllable with falling pitch. Basic word stress in Mohawk is penultimate. (Penultimate stress was established before the time of Proto-Northern-Iroquoian. Since then epenthetic vowels have been added in certain phonological and morphological contexts in Mohawk, which do not affect stress.) The primary acoustic correlates of stress are pitch and duration. Short stressed syllables show relatively high pitch, represented by an acute accent <á>. Open stressed syllables are lengthened. Lengthened stressed syllables show either rising pitch, represented by an acute accent and colon <á:>, or a rapid rise then fall in pitch to below the baseline, represented by a grave accent and colon <à:>. The basic pronunciation of the proximal demonstrative is thus \textit{kí:ken}, as in \figref{fig:mithun:1}, and that of the distal demonstrative is \textit{thí:ken}, as in \figref{fig:mithun:2}, shown here with their conventional spellings.


\begin{figure}
\includegraphics[width=\textwidth]{Figure_1_600-dpi.PNG}
\caption{Basic proximal demonstrative \textit{kí:ken}: ‘This is my son.’}
\label{fig:mithun:1}
\end{figure}

\begin{figure}
\includegraphics[width=\textwidth]{Figure_2_600-dpi.PNG}
\caption{Basic distal demonstrative \textit{thí:ken}:  ‘It’s that one.’}
\label{fig:mithun:2}
\end{figure}

In examples \REF{ex:mithun:13} and \REF{ex:mithun:14} above, in which the demonstratives were used as placeholders, there is little if any length or rise in pitch on the first syllable of the demonstratives, but, significantly, there is added length on the second syllable: \textit{kiken:} ‘this one’, \textit{thiken:} ‘that one’. The prosodic contour of \REF{ex:mithun:13} is shown in \figref{fig:mithun:3}.



\begin{figure}
\includegraphics[width=\textwidth]{Figure_3_600-dpi.PNG}
\caption{Placeholder \textit{kí:ken} ‘this’, example \REF{ex:mithun:13}}
\label{fig:mithun:3}
\end{figure}

\largerpage
A closer view of the prosody of the demonstrative is in \figref{fig:mithun:4}. The second syllable of the demonstrative ‘this’ is noticeably longer than the first.


\begin{figure}
\includegraphics[width=.66\textwidth]{Figure_4_600-dpi.PNG}
\caption{Closer view of demonstrative placeholder and following pause, example \REF{ex:mithun:13}}
\label{fig:mithun:4}
\end{figure}

A similar pattern can be seen in the prosody of the placeholder construction with \textit{thí:ken} ‘that’ from example \REF{ex:mithun:14} in \figref{fig:mithun:5}. Here, too, the second syllable of the demonstrative is noticeably longer than the first. A closer view shows the contour in more detail.

  
\begin{figure}
\includegraphics[width=\textwidth]{Figure_5_600-dpi.PNG}
\caption{\label{fig:mithun:5}:  Placeholder \textit{thí:ken} ‘that’, example \REF{ex:mithun:14}}
\end{figure}


\begin{figure}
\includegraphics[width=.66\textwidth]{Figure_6_600-dpi.PNG}
\caption{Closer view of demonstrative placeholder and following pause, example \REF{ex:mithun:14}}
\label{fig:mithun:6}
\end{figure}

Like \citet{HayashiYoon2006}, Podlesskaya notes that “a placeholder may fully or partially mirror the grammatical shaping of its target” (\citeyear[11]{Podlesskaya2010}). She adds (\citeyear[18]{Podlesskaya2010}) that languages may:

\begin{itemize}
\item exactly replicate the full grammatical marking of the delayed constituent
\item not replicate the grammatical marking of the constituent
\item allow partial replication
\end{itemize}

Neither Mohawk demonstratives nor nominals are inflected for number, gender, or case: this is a strictly head-marking language. The demonstratives in these constructions thus would not show the kind of mirroring of such features observable in some other languages in any case. The choice between the proximal and distal demonstratives in this Mohawk construction does, however, mirror the status of the referent of the following constituent in terms of distance in place, time, discourse mention, etc. The proximal demonstrative in \REF{ex:mithun:13}, ‘He lost his hat \textbf{this} . . . [\uline{aforementioned boy going along with the stolen fruit}]’ referred to the central protagonist at that point in the narrative, a core argument of most sentences leading up to this one. The distal demonstrative in \REF{ex:mithun:14}, ‘One father started to cry at \textbf{that} . . . [\uline{Variety Show}]’, referred to a location that had not been mentioned before and was more remote in place and time. Beyond this, the only basis for judging whether an item is a placeholder is \citegen{HayashiYoon2006} characterization, “a referential expression that is used as a substitute for a specific lexical item”.


\section{Interjective hesitators}
\label{sec:mithun:5}

A second type of filler described by \citet{HayashiYoon2006} is the \textit{interjective hesitator}. It is distinct from the placeholder in that the demonstrative does not point to a particular kind of referent. While placeholders are produced as a syntactic constituent of a sentence, hesitators are not (\citeyear[489]{HayashiYoon2006}). Mohawk demonstratives occasionally occur in such contexts of hesitation.

\ea%15
\label{ex:mithun:15}
Demonstrative as interjective hesitator\\
\gll \textit{Tánon’}  \textit{sewatié:rens}  \textit{ò:ni’}  \textit{né:’}    \textit {kwáh} \textbf{\textit{ki:}} \textit{a:,}\\
and      sometimes    also    it.is    just    \textbf{this}    ah\\~\\~\\
\glll \textit{enthotinà:khwen’.}\\
en-t-hoti-na’khwen-’\\
\textsc{fut-csl-m.pl.pat}{}-get.angry-\textsc{pfv}\\
\glt ‘And sometimes they’ll just get mad.’\\      \exsource{Warisóse’ Gabriel, speaker}
\z

The demonstrative occurred before the filler \textit{a:} then a pause, before the speaker continued with the verb ‘they’ll get mad’, not a coreferential nominal.

\begin{figure}
\includegraphics[width=\textwidth]{Figure_7_600-dpi.PNG}
\caption{Interjective hesitator \textit{kí:ken} ‘this’, example \REF{ex:mithun:15}}
\label{fig:mithun:7}
\end{figure}

Another example is in \REF{ex:mithun:16}. The demonstrative occurred before the filler \textit{en} then a false start \textit{wa-{}-} and another filler \textit{en}, and a breath, before the speaker continued with the verb ‘they finished this’. 

\ea%16
    \label{ex:mithun:16}
Demonstrative as interjective hesitator\\
\glllll \textit{Ó:nen} \textit{ken} \textbf{\textit{kiken:}} \textit{en} \textit{wa-{}-en,}\\
now        \textsc{q}    \textbf{this}       ah       wa-{}-ah\\
\textit{wahatihsa’}                  \textit{kí:ken?}\\
wa-hati-ihsa-’             kiken\\
\textsc{fac-m.pl.agt-}finish-\textsc{pfv}  this  \\
\glt ‘Now \textbf{this} ah ah, have they finished this?’\\
\exsource{Doris White, speaker}
\z

The prosody can be seen in \figref{fig:mithun:8} (p. \pageref{fig:mithun:8}).

\begin{figure}
\includegraphics[width=\textwidth]{Figure_8_600-dpi.PNG}
\caption{Interjective hesitator \textit{kí:ken} ‘this’, example \REF{ex:mithun:16}}
\label{fig:mithun:8}
\end{figure}

In neither \REF{ex:mithun:15} nor \REF{ex:mithun:16} does the demonstrative anticipate the syntactic category of what follows.

In these interjective hesitator constructions, the demonstrative shows the same prosody as in the placeholder constructions, with added length on the final syllable and sometimes a continuing rise in pitch. This pattern is characteristic of continuing intonation unit patterns across the language when the penultimate syllable of the final word is open. In \figref{fig:mithun:8} above, for example, the initial particle \textit{ó:nen} ‘now’ was pronounced with added length and rising pitch on the final syllable (\textit{ó:nen} -> \textit{onén}:) as was the final syllable of the demonstrative (\textit{kí:ken -> ki:kén}:). 

Similar prosodic differences between basic demonstratives and fillers have been observed by \citet{Vallejos-Yopán2023} with a filler \textit{este} in Peruvian Amazonian Spanish:

\begin{quote}
Filler-\textit{este}, which originally evolved from demonstrative\textit{{}-este}, serves to deal with word-formulation delays during spontaneous speech production. Analyses of conversations reveal that \textit{este} primarily functions as a filler: 70\% of the tokens of \textit{este} are either fillers serving as placeholders, which replace lexical items in specific syntactic slots, or fillers serving as hesitators, which are non-referential and distributionally free. Further, phonetic analyses show that demonstrative-\textit{este} and filler-\textit{este} exhibit different phonetic shapes. Demonstrative-\textit{este} patterns with disyllabic words with penultimate stress – the first vowel is longer than the second vowel. Filler-\textit{este} shows the opposite configuration – the second vowel is significantly longer than the first vowel. \citep[651]{Vallejos-Yopán2023}
\end{quote}

\citeauthor{Vallejos-Yopán2023} and others have pointed out that the development of fillers from demonstratives is not entirely surprising. “One scenario for the emergence of the filler use [of demonstratives] could be the pronominal use with cataphoric reference, if the identity of \textit{este} is specified in the following discourse.” (\citeyear{Vallejos-Yopán2023}: 671) She further proposes the following:

\begin{quote}
The repetition of \textit{este} ‘this one’ in this context seems natural. This path of development entails extending the use of \textit{este} from pointing to entities in space, entities presented in prior discourse, or those to be specified cataphorically, to indicate issues with lexical retrieval. Next, the form employed as a placeholder becomes invariant, it no longer holds gender or number agreement with the referent. Truncated placeholders over time become signs of hesitation and loose referentiality, thus becoming interjective hesitators. (\citeyear[671]{Vallejos-Yopán2023})
\end{quote}

The use of demonstratives as hesitators in Mohawk is actually not as common as their use in another function, however.

\section{Beyond fillers}
\label{sec:mithun:6}

This structure can be seen in \REF{ex:mithun:17}. As in the filler constructions, the demonstrative here showed extra lengthening of the final syllable and a non-terminal pitch contour, then was followed by a pause before the next intonation unit, which began with a partial pitch reset.

\ea%17
    \label{ex:mithun:17}
    Discourse structure\\
    \glll \textit{Thó}    \textit{ki’}      \textit{wahonwéntskaron’se’} \textbf{\textit{ki\uline{kén:}}}\textit{,}~\textit{. . .}\\
    there  in.fact    wa-honwa-itskar-on-’s-e’          \textbf{kiken}\\
    there  {in~fact}    \textsc{fac-fi>m.sg}{}-rug-make-\textsc{ben-pfv} \textbf{this}\\
    \glt ‘There she made him a pallet,  ...\\~\\

   \glll \textit{tsi}    \textit{tsiiotékha’.}\\
    tsi    io-atek-ha’\\
    place  \textsc{n.pat}{}-burn-\textsc{hab}\\
   \glt by the fire.’\\
   \exsource{Sadie Sesír Smoke Peters, speaker}
\z

The prosody can be seen in \figref{fig:mithun:9} (p. \pageref{fig:mithun:9}).

The same pattern can be seen in \REF{ex:mithun:18}. The first intonation unit ended in a demonstrative with added length on the final syllable and non-terminal pitch contour, followed by a pause and partial pitch reset on the following intonation unit.

\ea%18
    \label{ex:mithun:18}
    Discourse structure\\
    \glll \textit{Wa’tiakwatska’hòn:ne’} \textbf{\textit{kikén:}}\textit{,}    \textit{. .}\\
    wa’-t-iakwa-atska’hon-hne’        \textbf{kiken}\\
    \textsc{fac-dv-1excl.pl.agt}{}-dine-\textsc{purp}    \textbf{this}\\~\\~\\
    \textit{Joyce Mitchell}.
    \glt ‘We all were going to eat with Joyce Mitchell.’\\
    \exsource{Watchenní:ne’ Sawyer, speaker}
\z

The prosody can be seen in \figref{fig:mithun:10} (p. \pageref{fig:mithun:10}).

\begin{figure}[p]
\includegraphics[width=\textwidth]{Figure_9_600-dpi.PNG}
\caption{Discourse construction with \textit{kí:ken} ‘this’, example \REF{ex:mithun:17}}
\label{fig:mithun:9}
\end{figure}

\begin{figure}[p]
\includegraphics[width=\textwidth]{Figure_10_600-dpi.PNG}
\caption{Discourse construction with \textit{kí:ken} ‘this’, example \REF{ex:mithun:18}}
\label{fig:mithun:10}
\end{figure}

Like \citegen{HayashiYoon2006} interjective hesitators, the demonstratives here are not necessarily coreferential with what follows. Speakers confirm that the demonstrative \textit{kí:ken} ‘this’ in \REF{ex:mithun:19} did not refer to the week.

\ea%19
    \label{ex:mithun:19}
    Mohawk discourse pattern\\
 \glll \textit{Ó:nare’,  . . .}\\
    onen  á:re’\\
    now   again\\~\\~\\
\glll \textit{saionkenikè:tohte’} \textbf{\textit{ki\uline{kén:}}},\\
    sa-ionkeni-ke’toht-e’        \textbf{kiken}\\
    \textsc{rep.fac-1pl.pat}{}-appear-\textsc{pfv}  \textbf{this}\\~\\~\\
    \glll \textit{ó:nen}  \textit{tsiahià:kshera}      \textup{{[}.~.~.{]}}\\
    onen  ts-i-ahia’k-sher-at\\
    now  \textsc{rep-n-}week-\textsc{nmlz-}be.one  \\
    \glt ‘Now we’re back again after one week.’\\    
    \exsource{Joe Awenhráthon Deer, speaker}
\z

But these uses differ in a significant way from those of fillers: they are not associated with disfluency. The speakers were not searching for words or hesitating about how to formulate the next idea. They are part of a larger phenomenon.

A number of authors have observed that speech is not produced sentence by sentence, but in spurts, termed by \citet{Halliday1967} “information units”, by \citet{Crystal1975} “tone units”, by \citet{Grimes1975} “information blocks”, by \citet{Kroll1977} “idea units”, and by Chafe (\citeyear{Chafe1980}, \citeyear{Chafe1994}, \citeyear{Chafe2018}) “intonation units”. These units are characterized by patterns of pitch, intensity, rhythm, voice quality, and pausing. Chafe pointed out that “these spurts of language are linguistic expressions of focuses of consciousness” (\citeyear[15]{Chafe1980}) and that “each intonation unit verbalizes the information active in the speaker’s mind at its onset”. (\citeyear[63]{Chafe1994}) 

\citet{HayashiYoon2006} proposed that the use of demonstratives as interjective hesitators stems from cognitive factors much like those described by Chafe:

\begin{quote}
We consider the possibility that demonstratives in this usage have undergone the process of “pragmaticization”. We will suggest that the demonstratives used as interjective hesitators have been pragmaticized for the discourse function of monitoring the flow of the production of an utterance to such an extent that they have diverged from ordinary demonstratives in terms of syntactic distribution, referentiality, and correspondence between morphological forms and semantic meanings. (\citealt[488]{HayashiYoon2006})
\end{quote}

The Mohawk discourse structure has developed as a conventionalized way of packaging information for both the speaker and the hearer. It can facilitate interaction. The demonstrative at the end of an intonation unit, pronounced with a non-terminal prosodic contour, signals to listeners that more is to come. It can thus serve as a floor-holding device. It can also facilitate processing by listeners, allowing them to take in one new idea at a time.

\citet{Keevallik2010} describes strikingly similar effects in Estonian. The Estonian demonstrative \textit{see} serves as a placeholder during word searches, but \citeauthor{Keevallik2010} observes that it also does more. She points to its role in facilitating smoother interaction:

\begin{quote}
Why a delay needs to be announced has generally been explained from the perspective of the speaker and her cognitive limitations. The reasons have included a planning or formulation problem, memory search, doubt, uncertainty or hesitance (see the summary in \citealt[90]{Tree2002}). It has been assumed that the speaker is in general unable to proceed with talk at the current moment. In contrast, this study will explore the possibility that a filler may be implemented as a conscious strategy for achieving certain interjective ends. \citep[139]{Keevallik2010}
\end{quote}

She also points to the facilitation of processing for the listener:

\begin{quote}
There may be interpersonal advantages in the delay of key items in turns, such as easing the processing for the recipient, announcing structural boundaries in conversation, and displaying orientation to the sensitiveness of the action. (\citeyear[139]{Keevallik2010})
\end{quote}

As she notes, both \citet[526--527]{HayashiYoon2006} and \citet{Schegloff2010} observe that fillers can mark larger structural boundaries in conversation, often prefacing an introduction to a new topic. She finds similar patterns in her Estonian data, drawn primarily from a corpus of 324 telephone calls, augmented with some face-to-face conversation: 

\begin{quote}
Topic initiations in general tend to be accompanied by delays, and reasons for the call will as a rule initiate new topics in conversations. Accordingly, the Estonian \textit{see} is regularly used when presenting reasons for the call. (\citeyear[170]{Keevallik2010})
\end{quote}

\citet[216]{KosmalaCrible2022} found similar patterns with fillers \textit{euh} and \textit{eum} in spoken French: “Although they [filled pauses] are commonly associated with hesitation, disfluency, and production difficulty, it has also been argued that they can serve more fluent communicative functions in discourse (e.g., turn-taking, stance-marking).” They cite previous work indicating that in addition to their association with hesitation, filled pauses can also mark discourse structure \citep{Swerts1998} and manage turn-taking (\citealt{Kjellmer2003}, \citealt{Ben̆uš.Štefan2009}).

Just such effects can be seen with the Mohawk demonstrative structure. The speaker in \REF{ex:mithun:20} was opening a brand new topic of discussion, his garden. The first intonation unit ended with the demonstrative with characteristic prosody, with higher pitch and length on the stressed penultimate syllable \textit{kí:} then falling pitch on the ultimate syllable. The demonstrative did not refer cataphorically to the following constituent, the summer; if it referred to anything, it was the whole new topic of conversation.

\ea%20
\label{ex:mithun:20}
Mohawk discourse pattern\\
\glll \textit{Né:} \textbf{\textit{kik\uline{en:}}},\\
    ne:’=e    kiken\\
    that=is  this \\~\\~\\
\glll \textit{nòn:wa’}    \textit{akenhnhá:te’}        \textit{né:ne,}\\
    nonhwa’    akenh-at-e’        ne’ne\\
    now        summer-stand-\textsc{st}    which \\~\\~\\
 \glll \textit{tiotieriénhton} \\
    t-io-at-ierenht-on   \\
    \textsc{csl-n.pat-}be.first-\textsc{st}  \\~\\~\\
\glll \textit{iáh}    \textit{teiotòn:’on} \textit{thé:nen}    \textit{aontió’ten’.}\\
        iah    te-io-aton’-on           othenen    aa-wak-io’t-en’\\
        not  \textsc{neg-n.pat}{}-be.poss-\textsc{st}  anything  \textsc{irr-1sg.pat}{}-work-\textsc{pfv}\\
\glt    ‘This summer was the first time I wasn’t able to work.’   \\  

    \exsource{Joe Awenhráthon Deer, speaker}
\z

This Mohawk pattern, in which one intonation unit ends in a demonstrative pronounced with added length on the final syllable, a non-terminal prosodic contour, and often a pause, is now pervasive in speech, but the frequency varies widely across speakers, topics, and contexts. In one conversation involving five speakers, 50\% of the demonstratives \textit{kí:ken} ‘this/these’ and \textit{thí:ken} ‘that/those’ occurred in these structures. In another conversation involving two of the same speakers plus another, the figure was 39\%. In a third, involving two other speakers, the figure was 30\%. In some others, it was considerably less. An important point is that there is not a sharp distinction between the filler pattern and the discourse pattern; technically the only difference would be whether it is a result of disfluency.

Similar patterns can be seen in related Northern Iroquoian languages, though the demonstratives recruited are not necessarily cognate. The Tuscarora counterparts \textit{kyè:ní:kę:} ‘this/these’ and \textit{hè:ní:kę:} ‘that/those’ are used in the same ways, as are the Cayuga counterparts \textit{nę:gyęh} ‘this/these’ and \textit{to:gyęh} ‘that/those’.

\section{Toward syntax?}
\label{sec:mithun:7}

It is well known that cross-linguistically, complementizers and relativizers are often descended from demonstratives (\citealt{Brugmann1904}, \citealt{Bühler1934}, \citealt{HopperTraugott2003}, \citealt{HarrisCampbell1995}, \citealt{Diessel1999,Diessel2006}, \citealt{HeineKuteva2007}, \citealt{DiesselBreunesse2020}, among many). These are typically analyzed synchronically as part of the dependent clauses: \textit{It is regrettable [}\textbf{\textit{that}} \textit{our language is disappearing], I remembered [}\textbf{\textit{that}} \textit{we are striving to speak], It is different for the children [}\textbf{\textit{that}} \textit{grew up there].} The sources of these constructions have been hypothesized to be sequences of sentences, one of which contained a demonstrative. A continuing topic of discussion has been whether the demonstrative was originally part of what became the matrix clause or the dependent clause, and whether it was cataphoric or anaphoric. \citet[308--310, 319--320]{DiesselBreunesse2020} trace some of the history of discussion.

In many cases, it is not possible to identify the precise source construction on the basis of written records alone. Mohawk provides a glimpse of a point along one possible pathway of development. Complement and relativization constructions have not crystallized in Mohawk to the same extent as those in some other languages. In fact their counterparts have the same structure as that of the placeholder, hesitator, and discourse-structuring pattern seen in the previous sections.

\subsection{Toward complement constructions?}
\label{sec:mithun:7.1}

The Mohawk pattern seen in previous sections is often used in situations in which complement constructions would be used in other languages. An idea is presented in one intonation unit ending in a demonstrative with non-final prosody, which indicates that further elaboration is to follow. Some occurrences of this structure are comparable to subject complement constructions in other languages, like that in \REF{ex:mithun:21}. In the Mohawk, the demonstrative is grouped prosodically with what would be the matrix clause, however, while in the English, it is analyzed as part of the complement.

\ea%21
    \label{ex:mithun:21}
    Toward a subject complement?\\
    \gllll \textit{Sénha’}  \textit{ki’}     \textit{iet}\textbf{\textit{io}}\textit{hnhá:ten} \textbf{\textit{ki\uline{kén:}}}\textit{,}    \textit{. . .}\\
 senha’    ki’      ie-t-\textbf{io}{}-nhat-en            \textbf{kiken}\\
    more    in.fact  \textsc{trl-dv-}\textbf{\textsc{n.pat}}{}-regret-\textsc{st}    \textbf{this}\\
    more    {in~fact}    {\textbf{it}~is~regrettable}            \textbf{this}\\~\\~\\
\gllll    \textit{onkwehonwehnéha’}      \textit{onkwawén:na’}      \textit{iohtentionhátie’.}\\
    onkweh=onweh=neha’  onkwa-wenn-a’      io-ahtenti-on-hatie’\\
    person=trad=style      \textsc{1pl.al.poss}{}-lg-\textsc{ns}  \textsc{n.pat}{}-move-\textsc{st-prog}\\
    Native                  {our~language}        {it~is~moving~along}\\
\glt    ‘It is all the more regrettable [that our Indian language is getting        lost].’\\
    \exsource{Watshenní:ne’ Sawyer, speaker}
\z

The prosody can be seen in \figref{fig:mithun:11}, in which the first intonation unit ended in the proximal demonstrative \textit{kiken:} ‘this one’ with added length and rising pitch on the final syllable. The next clause began after a pause.

\begin{figure}
\includegraphics[width=\textwidth]{Figure_11_600-dpi.PNG}
\caption{Toward a complement construction? Example \REF{ex:mithun:21}}
\label{fig:mithun:11}
\end{figure}

The sentence in \REF{ex:mithun:22} shows a similar pattern. The first intonation unit ended in the distal demonstrative \textit{thí:ken} ‘that one’ with final lengthening, indicating that more was to follow. The next began after a pause and provided further elaboration.

\newpage
\ea%22
    \label{ex:mithun:22}
    Toward a subject complement?\\
   \gllll \textit{Ó:nen}    \textit{ni’}    \textit{énska}    \textit{ia’}\textbf{\textit{ká}}\textit{:ienhte’}  \textbf{\textit{thiken:}}\textit{,}  \textit{. . .}\\
    onen    ohni’  enskat    i-a’-\textbf{ka}{}-ienht-e’          \textbf{thiken}\\
    now      also    one      \textsc{trl-fac-}\textbf{\textsc{n.agt}}{}-hit-\textsc{pfv}  \textbf{that}\\
    now      also    one      {there~\textbf{it}~hit}              \textbf{that}\\~\\~\\

    \gllll \textit{thó}    \textit{ki’}      \textit{á:re’}  \textit{ní:ioht} \\
    tho    ki’      are’    ni-io-ht        \\
    there  in.fact    again  \textsc{prt-n.pat}{}-be.so  \\
    there  {in~fact}    again  {so~it~is}\\~\\~\\

\gllll         \textit{ó:ia’}        \textit{ia’tekohtáhrhohs.}\\
          o-i-a’        ia’-te-k-ohtarho-hs\\
          \textsc{n-}other-\textsc{ns}    \textsc{trl-dv-1sg.agt}{}-clean-\textsc{hab}\\
          other        {I~was~cleaning~there}\\
\glt    ‘Then once again \textbf{it} happened 

      [\textbf{that} I was cleaning in there again].’

        \exsource{Kahentoréhtha’ Marie Cross, speaker}

\z

\begin{figure}
\includegraphics[width=\textwidth]{Figure_12_600-dpi.PNG}
\caption{Toward a complement construction? Example \REF{ex:mithun:22}}
\label{fig:mithun:12}
\end{figure}

The same pattern is used in contexts where object complement constructions are used in some other languages. One intonation unit ends in a demonstrative showing added length and sometimes rising pitch on the final syllable, indicating that further elaboration is to follow. The next begins after a pause with a partial pitch reset. An example is in \REF{ex:mithun:23}. (Neuter arguments are not marked explicitly in the pronominal prefixes on verbs unless there is no other argument, but transitivity is implied by the verb.)

\ea%23
    \label{ex:mithun:23}
    Toward an object complement?\\
   \glll \textit{Né:}    \textit{ki’k}          \textit{wa’kehià:ra’ne’} \textbf{\textit{kik\uline{en:}}}\textit{,}  \textit{. . .}\\
    ne’e    ki’=ok        wa’-k-ehiahr-a’n-e’               \textbf{kiken}\\
    it.is    in.fact=just  \textsc{fac-1sg.agt}{}-remember-\textsc{inch-pfv} \textbf{this}\\~\\~\\

  \glll  \textit{tsi}  \textit{niió:re’}          \textit{tsi}    \textit{kiótkon}    \textit{ionkwahskéhnhen,}\\
    tsi  ni-io-r-e’        tsi    tiotkon    ionkwa-ahskehnh-en\\
    as    \textsc{prt-n-}extend-\textsc{st}  how    always    \textsc{1pl.pat}{}-strive-\textsc{st}\\~\\~\\

 \glll   \textit{nonkwawén:na’,}\\
    ne=onkwa-wenn-a’\\
    \textsc{art=1pl.al.poss}{}-language-\textsc{ns}\\~\\~\\
    
\glll    \textit{onkwehonwehnéha’,}\\
    onkweh=onweh=neha’\\
    person=traditional=style\\~\\~\\

\glll    \textit{aonsetewatá:ti’.}\\
    a-ons-etewa-atati-’\\
    \textsc{irr-rep-1incl.pl.agt}{}-speak-\textsc{pfv}\\

\glt    ‘I just remembered [how we are always striving\\
        to still speak our Native language].’

    \exsource{Sadie Sesír Smoke Peters, speaker}
\z

The prosody can be seen in \figref{fig:mithun:13}.


\begin{figure}
\includegraphics[width=\textwidth]{Figure_13_600-dpi.PNG}
\caption{Toward a complement construction? Example \REF{ex:mithun:23}}
\label{fig:mithun:13}
\end{figure}

The grouping of a demonstrative prosodically with the intonation unit before a clause which could function as a complement is not unique to Mohawk. In his discussion of asymmetries in the prosodic phrasing of function words, \citeauthor{Himmelmann2014} points out that “truncations after preposed function words are probably possible and attested in many, if not all, languages that have such function words” (\citeyear[936]{Himmelmann2014}) and, furthermore, that “preposed function words are often chunked with the preceding word rather than with their morphosyntactic host, which follows the discontinuity” (\citeyear[937]{Himmelmann2014}). Such chunking makes sense when prosody is understood in terms of foci of consciousness. \citeauthor{Himmelmann2014} cites the example in \REF{ex:mithun:24} from German, which he characterizes as evincing a type of disfluency. The complementizer \textit{dass} ‘that’ appeared in the same intonation unit as the matrix verb rather than the complement.

\ea%24
    \label{ex:mithun:24}
    German prosody:  \citealt[937]{Himmelmann2014}\\
\gll     un:    dann  hab’  ich  plötzlich    von    weitem,\\
      and    then    have  I    suddenly  from  afar\\~\\~\\
  \gll  (0.5)  \textit{gesehen} \textbf{\textit{dass}},\\
     {}     seen    that\\~\\~\\
\gll    (0.8)  en  Teil    von  der—\\
        {}  a    part    of    the\\~\\~\\
\gll    (0.8)  öh  öh (0.4)  Strecke,\\
         {} uh  uh     {}   road\\~\\~\\
\gll    öh  mit    Schnee,\\
    uh  with    snow\\~\\~\\

\gll \textit{öh}  (0.8)  \textit{ähm}  (1.1)\\
    uh     {}    um  {} \\~\\~\\

\gll \textit{also}  (0.4)  \textit{mit}    \textit{Schnee} \textit{bedeckt} \textit{war.}\\
    well     {}   with    snow    covered    was\\
\z

\subsection{Toward relative clause constructions?} %7.2. /
\label{sec:mithun:7.2}

Somewhat less common in Mohawk is the use of the pattern in contexts where relative clauses might occur in other languages. One intonation unit ends in a demonstrative with added length and a non-terminal pitch contour, indicating that more is to follow. The next begins after a pause and provides further elaboration.

\ea%25
    \label{ex:mithun:25}
    Headless relative clause?\\
    
\gllll Shakorihonnién:ni          kwi’ \textbf{thi\uline{kén:},}  {.~.~.}\\
shako-rihw-onni-enni        ki’=wahi’    \textbf{thiken}\\
\textsc{m.sg>fi-}matter-make-\textsc{ben}  in.fact=\textsc{tag}  \textbf{that}\\
{he~makes~words~for~us}      {in~fact}        \textbf{that}\\~\\~\\

\gllll \textit{shonkwahnhà:’on.}\\
shonkwa-nha’-on\\
\textsc{m.sg>1pl-}hire-\textsc{st}\\
{he~hired~us}\\
\glt ‘The teacher is the one [\textbf{that} told us to].’           \\                                        \exsource{Kahentoréhtha’ Marie Cross, speaker}
\z

The prosody can be seen in \figref{fig:mithun:14}.

  
\begin{figure}
\includegraphics[width=\textwidth]{Figure_14_600-dpi.PNG}
 \caption{\label{fig:mithun:14} Toward a headless relative clause? example \REF{ex:mithun:24}}
\end{figure}

Though less frequent, examples occur that might approach the form of headed relative clauses. 

\ea%26
    \label{ex:mithun:26}
    Relative clause?\\
      \gllll Né:    aorì:wa’ \textbf{ki:k\uline{én:}},\\
      ne:    ao-rihw-a’            kiken\\
      it.is    \textsc{n.al.poss-}reason-\textsc{ns}  this\\
      {it~is}    {its~reason}            this\\~\\~\\

\gllll      kkaraienté:ri     akwé:kon.\\
      k-kar-a-ienteri              akwekon\\
      \textsc{1sg.agt}{}-story-\textsc{lk-}know.\textsc{st}  all\\
      {I~know~the~story}            all\\

\glt      ‘That is the reason [\textbf{that} I know the whole story].’    

     \exsource{Lazarus Jacob, speaker}
\z

\ea%27
    \label{ex:mithun:27}
    Relative clause?\\
\glll      \textit{Wakerihwaiè:wahse’                    se’s   ka’  nón:} \textbf{\textit{ki:kén:}},\\
 wake-rihw-a-iehwahs-e’                  se’s   ka’   nonwe   kiken\\
      \textsc{1sg.pat-}reason-\textsc{lk}{}-be.unable.to.find-\textsc{st}  then   what place this\\~\\~\\

\glll thonathéhtaien’                    né:    ki:    né:,\\
t-hon-at-heht-a-ien-’                ne’e    kiken  ne’e\\
      \textsc{csl-m.pl.pat-mid}{}-field-\textsc{lk-}have-\textsc{st}    it.is    this    it.is\\~\\~\\

 \glll     \textit{rake’níha                  tsi    nonkwá:ti.}\\
      rake-’ni=ha                tsi    nonkwati\\
      \textsc{m.sg>1sg}{}-be.father.to=\textsc{dim}  place  side\\

      \glt      ‘I couldn’t figure out the place\\
          {[}\textbf{where} their garden was on my father’s side].’

    \exsource{Joe Tiorhakwén:te’ Dove, speaker}
\z

\section{Conclusion}
\label{sec:mithun:8}
\largerpage
Many of the features observable in the placeholder and hesitator structures in Mohawk are also characteristic of a more general discourse pattern in the language. \citet{HayashiYoon2006} cited similar features of their interjective hesitators, which they described as “pragmaticized for the discourse function of monitoring the flow of the production of an utterance to such an extent that they have diverged from ordinary demonstratives in terms of syntactic distribution, and referentiality” (\citeyear[488]{HayashiYoon2006}). The Mohawk discourse construction serves as a general strategy for managing the flow of information. It consists of an intonation unit ending in a demonstrative with added length on the final syllable and a non-terminal pitch contour, often followed by a pause, then followed by another intonation unit beginning with a partial pitch reset which provides some elaboration. Each intonation unit expresses a new idea, a single focus of consciousness in the sense of \citet{Chafe1980,Chafe1994}. In some cases, in the placeholder and filler uses, the pattern appears with observable disfluency, but in many more, there is no indication of disfluency at all. 

The demonstrative in this discourse construction may be understood to refer cataphorically to what follows or not, reflecting the divergence cited by \citet{HayashiYoon2006} for hesitators “from ordinary demonstratives in terms of syntactic distribution and referentiality”. Reference is cataphoric in the placeholder uses and some general discourse uses, including those comparable to complement and relative clauses in other languages, but not (simply by definition) in hesitator uses and in many other discourse uses.

It is perhaps not surprising that such constructions should develop in languages in general. They facilitate processing for both speaker and listener. As is well known, speakers tend to present one significant new idea at a time prosodically, in successive intonation units. Such phrasing can not only allow speakers time to formulate the next idea, but also to organize the information for their listeners. It can signal larger discourse structure, occurring at major discourse breaks. It can also facilitate interaction, signaling to listeners that the speaker wishes to hold the floor. As more audio documentation of unscripted, interactive speech in more languages becomes available, it is likely that similar patterns will be observed more widely.

\section*{Abbreviations}

\begin{multicols}{2}
\begin{tabbing}
MMMM \= grammatical\kill
\textsc{agt} \> grammatical agent\\
\textsc{al} \> alienable\\
\textsc{art} \> article\\
\textsc{ben} \> benefactive applicative\\
\textsc{caus} \> causative\\
\textsc{coin} \> coincident\\
\textsc{csl} \> cislocative\\
\textsc{dim} \> diminutive\\
\textsc{dv} \> duplicative\\
\textsc{excl} \> exclusive\\
\textsc{fac} \> factual\\
\textsc{fi} \> feminine/indefinite gender\\
\textsc{fut} \> future tense\\
\textsc{hab} \> habitual aspect\\
\textsc{inch} \> inchoative\\
\textsc{incl} \> inclusive\\
\textsc{irr} \> irrealis\\
\textsc{lk} \> linker\\
\textsc{m} \> masculine\\
\textsc{mid} \> middle\\
\textsc{n} \> neuter gender\\
\textsc{nmlz} \> nominalizer\\
\textsc{ns} \> noun suffix\\
\textsc{pat} \> grammatical patient\\
\textsc{pfv} \> perfective aspect\\
\textsc{pl} \> plural\\
\textsc{poss} \> possessive\\
\textsc{prog} \> progressive\\
\textsc{proh} \> prohibitive\\
\textsc{prt} \> partitive\\
\textsc{purp} \> purposive\\
\textsc{q} \> question particle\\
\textsc{rep} \> repetitive\\
\textsc{sg} \> singular\\
\textsc{st} \> stative aspect\\
\textsc{trl} \> translocative
\end{tabbing}
\end{multicols}

\sloppy\printbibliography[heading=subbibliography,notkeyword=this]
\end{document} 
