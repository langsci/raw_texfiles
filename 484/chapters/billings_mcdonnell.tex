\documentclass[output=paper,colorlinks,citecolor=brown
\ChapterDOI{10.5281/zenodo.15697585}
% ,hidelinks
% showindex
]{langscibook}


\author{Blaine Billings\orcid{}\affiliation{University of Hawai'i at Mānoa} and  Bradley McDonnell\orcid{}\affiliation{University of Hawai'i at Mānoa}}

\title{Fillers in the gray: Indeterminacy in their use in everyday conversation}[Fillers in the gray: Indeterminacy in their use in everyday conversation] 

\abstract{Previous studies that focus on the use of demonstrative and interrogative pronouns as fillers have identified two primary functions, a placeholder function and an (interjective) hesitator function. At the same time, these studies have demonstrated both that placeholders and hesitators can often be clearly distinguished and that the majority of placeholders are eventually repaired. In this chapter, we describe how demonstrative and interrogative pronouns are used as fillers in everyday conversations in Nasal, an Austronesian language of Western Indonesia. Instead of clear-cut boundaries between these functions, we show (i) how distinctions between placeholder and hesitator functions are often indeterminate and (ii) how many instances of the placeholder function do not result in repair. Despite this indeterminacy and lack of placeholder repair, there are no apparent issues for participants in the course of their interaction.

\keywords{Austronesian, fillers, conversation, repair, word-formulation trouble}
}

%move the following commands to the "local..." files of the master project when integrating this chapter
% \usepackage{multirow}
% \usepackage{tabularx}
% \usepackage{langsci-optional}
% \usepackage{langsci-gb4e}
% \usepackage{booktabs}
% \usepackage{rotating}
% \usepackage{subcaption}
% \usepackage{afterpage}
% \let\eachwordone=\itshape
% \bibliography{localbibliography}
% \newcommand{\orcid}[1]{}

\IfFileExists{../localcommands.tex}{
   \addbibresource{../localbibliography.bib}
   % add all extra packages you need to load to this file

\usepackage{tabularx,multicol}
\usepackage{url}
\urlstyle{same}

\usepackage{listings}
\lstset{basicstyle=\ttfamily,tabsize=2,breaklines=true}

\usepackage{langsci-basic}
\usepackage{langsci-optional}
\usepackage{langsci-lgr}
\usepackage{langsci-osl}
% \usepackage{./langsci/styles/langsci-lgr}
% \usepackage{./langsci/styles/langsci-osl}
% \usepackage{langsci-gb4e}

\usepackage{tikz}
\usetikzlibrary{patterns,calc}
\pgfdeclarepatternformonly{south east lines}{\pgfqpoint{-0pt}{-0pt}}{\pgfqpoint{3pt}{3pt}}{\pgfqpoint{3pt}{3pt}}{
    \pgfsetlinewidth{0.6pt}
    \pgfpathmoveto{\pgfqpoint{0pt}{3pt}}
    \pgfpathlineto{\pgfqpoint{3pt}{0pt}}
    \pgfpathmoveto{\pgfqpoint{.2pt}{-.2pt}}
    \pgfpathlineto{\pgfqpoint{-.2pt}{.2pt}}
    \pgfpathmoveto{\pgfqpoint{3.2pt}{2.8pt}}
    \pgfpathlineto{\pgfqpoint{2.8pt}{3.2pt}}
    \pgfusepath{stroke}}
    
\usepackage{stmaryrd}
\usepackage{wasysym}
\usepackage{multirow}
\usepackage{caption}
\usepackage{subcaption}
\usepackage{mathrsfs}
\usepackage{qtree}

\usepackage{linguex}


   %pminos do not split footnotes
% \interfootnotelinepenalty=10000 %Footnote in Laporte chapters has to be split SN


%\DeclareIndexNameFormat{default}{%
%\nameparts{#1}%
%\usebibmacro{index:name}%
%{\index[names]}%
%{\namepartfamily}%
%{\namepartgiveni}%
% {}% L1
% {}% L2
%{\namepartprefix}% generates spurious space L3
%{\namepartsuffix}% generates spurious space L4
%}

%  {\DeclareIndexNameFormat{default}{%
%     \usebibmacro{index:name}{\index[names]}{#1}{#3}{#5}{#7}}}

%\DeclareIndexNameFormat{default}{%
%  \usebibmacro{index:name}{\sindex[nom]}{#1}{#3}{#5}{#7}}

%\DeclareIndexNameFormat{default}{%
%  \usebibmacro{index:name}{\sindex[person]}{#1}{#3}{#5}{#7}}
%\DeclareIndexNameFormat{default}{%
%\nameparts{#1} \usebibmacro{index:name}{\sindex[person]]}{\namepartfamily}{‌​\namepartgiven}{\nam‌​epartprefix}{\namepa‌​rtsuffix}}

%\newcommand{\smiley}{:)}

%\renewbibmacro*{index:name}[5]{%
%\usebibmacro{index:entry}{#1}%
%{\iffieldundef{usera}{}{\thefield{usera}\actualoperator}\mkbibindexname{#2}{#3}{#4}{#5}}}

% \newcommand{\noop}[1]{}

%remove for final
%\overfullrule=1mm

\newcommand{\tobi}[2]}}
\renewcommand{\S}[1]{\tobi{#1}{\textsc{*}}}

% this volume references
% puts: [this volume]
% already defined: \citetv
%\newcommand{\citepv}[1]{(\citeauthor{#1} \citeyear*{#1} [this volume])}
\newcommand{\citealtv}[1]{\citeauthor{#1} \citeyear*{#1} [this volume]}

%parentheses around example number
\newcommand{\pref}[1]{(\ref{#1})}

% in-text examples

\newcommand{\lnex}[1]{\textit{#1}} %target lang word
\newcommand{\lnlit}[1]{(lit.: `#1')} %literal reading
\newcommand{\lnlat}[1]{(#1)} % latinization
\newcommand{\lntrans}[1]{`#1'} %translation
\newcommand{\lnexl}[2]%
{\lnex{#1}{} \lnlat{#2}} % ex with latinization
\newcommand{\lnexlat}[3]{\lnex{#1}{} \lnlat{#2}{} \lntrans{#3}} % ex with latinization and tranl.

%ch01
\newcommand{\co}[1]{\mbox{\textbf{#1}}}

%ch09

\newcommand{\cyrbulg}[1]{\begin{otherlanguage*}{bulgarian}#1\end{otherlanguage*}}


%ch10
\newcommand{\nlp}{{\small NLP}}
\newcommand{\mwe}{{\small MWE}}
\newcommand{\rae}{{\small RAE}}
\newcommand{\lvc}{{\small LVC}}
\newcommand{\pos}{{\small P}o{\small S}}
%\newcommand{\todo}[1]{ \textcolor{red}{#1} }

%\renewcommand{\labelenumi}{\theenumi}
%\ainamefmt{{vv}{ll}{, ff}{, jj}} % fullname

\newcommand{\biberror}[1]{{\color{red}#1}}

\newcommand{\osenovaitem}{--~}
   %% hyphenation points for line breaks
%% Normally, automatic hyphenation in LaTeX is very good
%% If a word is mis-hyphenated, add it to this file
%%
%% add information to TeX file before \begin{document} with:
%% %% hyphenation points for line breaks
%% Normally, automatic hyphenation in LaTeX is very good
%% If a word is mis-hyphenated, add it to this file
%%
%% add information to TeX file before \begin{document} with:
%% %% hyphenation points for line breaks
%% Normally, automatic hyphenation in LaTeX is very good
%% If a word is mis-hyphenated, add it to this file
%%
%% add information to TeX file before \begin{document} with:
%% \include{localhyphenation}
\hyphenation{
    Beck-man
    Ngu-yen
    back-chan-nel
    back-chan-nels
    mo-not-o-nous
    ste-reo-typ-i-cal
}

\hyphenation{
    Beck-man
    Ngu-yen
    back-chan-nel
    back-chan-nels
    mo-not-o-nous
    ste-reo-typ-i-cal
}

\hyphenation{
    Beck-man
    Ngu-yen
    back-chan-nel
    back-chan-nels
    mo-not-o-nous
    ste-reo-typ-i-cal
}

   \boolfalse{bookcompile}
   \togglepaper[23]%%chapternumber
}{}

\begin{document}
\maketitle
\shorttitlerunninghead{Fillers in the gray}

\graphicspath{{figures/mcdonnell}}
\section{Introduction}
Demonstrative and interrogative pronouns are among the most common -- if not \emph{the} most common -- source for fillers cross-linguistically \citep[]{podlesskaya2010parameters}, and the uses of different types of pronouns in these roles appear to be well motivated. In dealing with trouble in word-formulation, speakers commonly employ them as fillers, which according to \textcite{hayashi2006crosslinguistic}, fall into two categories: placeholders and (interjective) hesitators. Placeholders, as the name suggests, take the place of an unspecified constituent, be it an individual lexical item or phrase, which is typically described as being subsequently ``replaced'' by a more specific lexical item or phrase. This characterization leads \citet[11]{podlesskaya2010parameters} to refer to placeholders as performing a ``preparatory substitute'' function. Placeholders are thus syntactically integrated and often carry some or all of the requisite morphology \parencite[][489]{hayashi2006crosslinguistic}. Hesitators, on the other hand, are used in sequences of word-formulation trouble to delay ``the production of the next item due in the course of the unfolding utterance'' \parencite[][507]{hayashi2006crosslinguistic}. They are not syntactically integrated and thus not repaired by a more specific lexical item.

An important factor that motivates the use of demonstrative pronouns as fillers is their focusing/pointing role, which draws the speaker's and hearer's focus on the same, yet unspecified referent. This property helps to explain how demonstrative pronouns are commonly taken up as a resource for dealing with word-formulation trouble because, as \citet[515]{hayashi2006crosslinguistic} put it, ``a placeholder demonstrative creates a prospective link to a subsequent specification of the referent and focuses the hearer's attention on it.'' For hesitators, \citet{hayashi2006crosslinguistic} propose that demonstratives lose referential features via a process of pragmaticization, developing not only into hesitators but also discourse markers with a wide variety of functions (\cite[see][525-529]{hayashi2006crosslinguistic} for discussion of discourse functions of demonstratives in Mandarin, Japanese, and Korean as well as \cite[][]{Næss2020} for a broader discussion of discourse functions of demonstratives).

Interrogative pronouns have received less attention, but in a description of placeholders derived from interrogative pronouns in two Caucasian languages, Udi and Agul, \citet{ganenkov2010interrogatives} point out close similarities between the use and form of self-addressed questions on the one hand and placeholders on the other. In fact, they state that there are cases where distinguishing the two is impossible. Such contexts help explain how interrogative pronouns may have developed non-interrogative functions as placeholders. The development of hesitators from interrogative pronouns could operate through a similar process of pragmaticization as with demonstrative pronouns, although it is not difficult to imagine that interrogative pronouns follow a distinct pathway of change where self-addressed questions need not become a placeholder before developing hesitator functions.

Although placeholder and hesitator are treated as two distinct functions in the descriptions of Mandarin, Japanese, and Korean by \citet[][]{hayashi2006crosslinguistic}, in practice it is unclear to what extent these functions can be distinguished in the world's languages. In Nasal, for example, the lack of obligatory morphological marking and frequent lack of repair for placeholders renders two of the otherwise more reliable criteria for delineating these two functions useless (see also \cite{chapters/mcdonnell_billings} for a similar issue in Besemah). Indeterminacy is even more pronounced with apparent hesitators, where the line between a hesitator and a self-addressed question used as a delay device is often blurry (see \cite[][33]{ganenkov2010interrogatives} for a similar indeterminacy in the use of the interrogative pronoun as a placeholder and in self-addressed questions). In a broader respect, there is a third level of indeterminacy in analyzing whether a demonstrative or interrogative pronoun is used in a filler function at all. When morphological marking does not appear on one of these function words, it is often difficult to determine whether the function word is serving one of its more typical deictic functions (not filler), taking the place of some yet unspecified referent (placeholder), or simply delaying speech (hesitator).

Despite their presence in a wide variety of typologically diverse languages, there are relatively few studies on the use of demonstrative and interrogative pronouns as fillers in everyday conversations \citep[see][58]{hayashi2010crosslinguistic}. This is particularly true in Austronesian languages where there are few studies dedicated to the use of fillers (\cite{nagaya2022tagalog} for Tagalog, \cite{dimock2010fillers} for Navahaq, and \cite{williams2010toward} for Indonesian being notable exceptions). There are several studies that describe how Austronesian languages use demonstratives in word-formulation trouble. For instance, \citet{wouk2005syntax}, in discussing repair in Indonesian conversation, has shown that demonstrative placeholders play a significant role in same-turn self-repair. \citet{ewing2005grammar}, in discussing demonstrative pronouns in Cirebon Javenese conversation, describes how they are used as hesitation markers. In a grammar of Papuan Malay, \citet{Kluge2017} briefly describes how demonstrative pronouns are utilized as both placeholders and hesitators. 

Across the Austronesian family, however, interrogative pronouns are more commonly described as a source for fillers and have even been reconstructed as fillers for higher-level subgroups, such as Proto Malayo-Polynesian *a-nu `thing whose name is unknown, avoided, or cannot be remembered; what?' (\cite[516]{blust2013austronesian}, \cite{blust2023acd}). In quite a few languages, this filler has lost its interrogative function, and the form has become a dedicated filler word (see \cite{chapters/mcdonnell_billings}). Within this context, Nasal (glottocode: nasa1239), an Austronesian language of Indonesia, presents a particularly interesting case wherein both the demonstrative pronouns and the interrogative pronoun are used in placeholder and hesitator functions. Unlike well-known Austronesian languages like Indonesian that use demonstrative pronouns as fillers, Nasal has a rather complex three-way demonstrative system, and the filler functions of the interrogative pronoun \textit{api} `what' have not fully lexicalized. Instead, it shows how \textit{api} `what' still occurs in word-formulation trouble both by itself as a lexical hesitator as well as in phrasal expressions of self-questioning in word searches. Potentially more surprising is that, unlike the examples examined in \citet{hayashi2006crosslinguistic}, many instances of fillers in Nasal conversation are indeterminate in terms of their placeholder or hesitator function. That is, in a number of instances the function of the filler cannot be straightforwardly analyzed, either because of lack of morphological marking, lack of repair, or homophony with other forms. While the indeterminacy between these two functions does not present any issues for the participants in the course of their interaction, it raises a number of issues in the analysis and calls into question whether the divide between these functions is a sharp as \citet{hayashi2006crosslinguistic} describe.

This chapter is organized as follows. The following section, \sectref{sec:Nasal}, provides a brief introduction to Nasal, its morphosyntax, and the corpus used for this study. \sectref{sec:Fillers} gives an overview of those lexical items that operate as fillers in Nasal and their various uses. \sectref{sec:HesitatorFillers} and \sectref{sec:Placeholders} describe hesitator and placeholder uses of these fillers, respectively. \sectref{sec:Fillers in the larger corpus} discusses the distribution of these fillers in the larger corpus. Finally, \sectref{sec:implications} and \sectref{sec:Conclusion} provides some broader implications of our analysis and concluding remarks.

\section{Nasal}\label{sec:Nasal}
Nasal is an underdocumented language of the Sumatran subgroup within the Malayo-Polynesian branch of the Austronesian language family and is spoken by approximately 3,000 people along the southwestern coast of Sumatra \parencite{billings2024sumatran}. Documentation is limited to a few wordlists and texts (\cite{stokhof_1987}, \cite{southsumatralanguagesmap_1987}, \cite{anderbeckaprilani_2013}) along with the authors' ongoing documentation project \parencite{mcdonnell2017elar,mcdonnell2019documentation}. Nasal is surrounded by much larger speaker populations of neighboring Malayic and Lampungic languages. While the languages of these three subgroups of Malayo-Polynesian—Sumatran, Malayic, and Lampungic—share a number of the morphosyntactic properties found in the languages of the western Indonesia region, they also show considerable variation from one another in nearly all domains, from phonology to lexicon to syntax \parencite[see][]{mcdonnell2024malayic,mcdonnell2024nonmalayic}. At the same time, neighboring Lampungic and Malayic languages have considerably affected Nasal, likely as the result of widespread multilingualism \parencite{mcdonnelltoappeardocumenting}. This has led to the development of a grammar that is strikingly similar to both neighboring Lampungic and Malayic languages (i.e. South Barisan Malay, Kaur) with a high number of loan words from both subgroups. For the discussion here, this is perhaps most relevant in the domain of demonstratives and interrogatives, where the Nasal system stands in stark contrast to Malayic but has clear affinities with Lampungic. Although there remains a distinctively Nasal core, these multiple layers of borrowing complicate the picture of Nasal. 

In the remainder of this section, we provide a brief overview of the basic structural properties of Nasal as a foundation for the study of Nasal fillers and introduce the corpus used for this study.

\subsection{Basic morphosyntactic properties}\label{sec:BasicMorpho}
% Nasal is a mildly synthetic language with a fairly robust set of nominal and verbal prefixes and suffixes in addition to a few circumfixes. The most common of these affixes are verbal and mark for voice – the nasal substitution \textit{N-} for `\textsc{av}' and \textit{di-} for `\textsc{nv}' – or fulfill typical applicative functions – \textit{-kun} `\textsc{caus/ben}' and \textit{-i} `\textsc{loc/dist}'.

% In addition to affixes, Nasal employs pronominal proclitics and enclitics. Proclitics are only found on verbal stems in the form \textit{ku=} `1\textsc{s}' and \textit{mu=} `2\textsc{s}' where they indicate an \textsc{nv} agent. The enclitics \textit{=ku} `1\textsc{s}', \textit{=mu} `2\textsc{s}', and \textit{=nyo} `3\textsc{s}' occur with nouns to indicate possession. The enclitic \textit{=nyo} can furthermore act as a determiner when attached to a noun or indicate an \textsc{av} object or \textsc{nv} agent when attached to a verb. \textit{\tabref{tab:billings:morphemes}} provides a brief overview of all of these and other common morphemes.

% \begin{table}
%     \caption{Common affixes in Nasal}
%     \label{tab:billings:morphemes}
%     \begin{tabularx}{.8\textwidth}{l X}
%         \lsptoprule
%         Morpheme         & Meaning \\
%         \textit{N-} & Agent voice (\textsc{av})\\
%         \textit{di-} & Non-agent voice (\textsc{nv})\\
%         \textit{be-} & Middle voice (\textsc{mid})\\
%         \textit{te-} & Non-volitional \textsc{nonvol}\\
%         \textit{-kun} & Applicative suffix with typically causative or benefactive meaning (\textsc{caus/ben})\\
%         \textit{-i} & Applicative suffix with typically locative or distributional meaning (\textsc{loc/dist})\\
%         \textit{peN-} & Agent nominalizer\\
%         \textit{ke-} & Ordinal\\
%         \textit{-an} & Patient nominalizer\\
%         \textit{ke- -an} & Action nominalizer; adversative\\ \hline
%         \textit{ku=} & 1\textsc{s} agent in \textsc{nv} constructions\\
%         \textit{=ku} & 1\textsc{s} possession \\
%         \textit{mu=} & 2\textsc{s} agent in \textsc{nv} constructions\\
%         \textit{=mu} & 2\textsc{s} possession \\
%         \textit{=nyo} & 3\textsc{s} object in \textsc{av} constructions; 3\textsc{s} agent in \textsc{nv} constructions; 3\textsc{s} possession; determiner \\
%         \lspbottomrule
%     \end{tabularx}
% \end{table}

In Nasal, as in other languages of Western Indonesia, predicates commonly occur without explicit arguments, and while the absence of arguments has traditionally been analyzed as a form of ellipsis, recent studies that take an interactional approach to these languages do not assume that arguments are elided \citep[e.g.][]{ewing2019predicate}. Instead, these approaches show how such minimal structures are the norm, as in (\ref{ex:wysiwyg}), where the verbal predicate in line 3 occurs without any arguments.\footnote{We use a simplified and updated version of Discourse Transcription \citep{dubois1993}. Each new line represents an Intonation Unit (IU) with end marks on each representing different contours: `.' is a `final' contour, `,' is a `continuing' contour, and `?' is an `appeal' contour. Brackets represent overlapping speech. An en dash `--' represents a truncated word, an em dash `---' represents a truncated IU, `:' represents lengthening, `..' represents a short pause, (TSK) represents a click of the tongue, and (\%) represents glottalization.}

\begin{exe}
    \ex\label{ex:wysiwyg} \begin{xlist}[0\quad A.:]
        \exi{1\quad A:} \gll
        nyak so hago m-inum \\
        1\textsc{sg} this want \textsc{intr}-drink \\
        \glt `I want to drink (something).' \\
        \exi{2\quad U:} \gll
        akuk=do \\
        take=\textsc{emph} \\
        \glt `take (it).' \\
        \exi{3\quad A:} \gll
        mak pandai ng-akuk, \\
        \textsc{neg} able \textsc{av}-take \\
        \glt `(I) can't get (anything),' \\
    \end{xlist}
    \hfill (BJM02-002, 01:07:53–01:07:56, Speakers: Arma, Upik) 
\end{exe}

As in many other Austronesian languages \parencite[see][]{mosel2023word}, word classes can be difficult to distinguish in Nasal where a number of roots may be considered flexible or unspecified for word class. The word class of lexical roots is thus entirely determined by their syntactic role in a given utterance. While nominal and verbal morphology draws a clear word class divide in complex forms, affixes may operate on roots from either class (i.e. verbal affixes may combine with both nominal and verbal roots and nominal affixes may combine with verbal and nominal roots). In the examples, therefore, verbal roots may appear with a nominal gloss and vice-versa.

Nouns take no grammatical marking for gender, number, or case, and noun phrases minimally consists of a noun. Numerals and numeral classifiers typically precede the noun, with other modifiers, like demonstratives, following the noun. Possessors follow the head noun, and possession can also be indicated by pronominal enclitics \textit{=ku} `1\textsc{sg}', \textit{=mu} `2\textsc{sg}', and \textit{=nyo} `3\textsc{sg}'. As is common for languages of the region, the third-person singular enclitic \textit{=nyo} additionally expresses definiteness among a number of other functions (see, for example, \citeauthor{gil2005riauindonesian}'s \citeyear{gil2005riauindonesian} analysis of Riau Indonesian =\textit{nya}). Relative clauses follow the head noun and are marked by the relativizer \textit{sai}; they are commonly headless.

Intransitive verbs are often unmarked (\ref{ex:nsy-intr-unmarked}) but may take one of several prefixes, such as the non-volitional prefix \textit{te-} `\textsc{nvol}' or the middle prefix \textit{be-} `\textsc{mid}' (\ref{ex:nsy-intr-be}). These prefixes derive verbal predicates from nominal or verbal roots.

\begin{exe}
    \ex\label{ex:nsy-intr-unmarked}
        \gll yo khatung dijo jo, \\
        3\textsc{sg} come here earlier \\
        \glt `he came here before,' \\
    \hfill (BJM02-007, 00:48:44–00:48:45, Speaker: Eka)
\end{exe}

\begin{exe}
    \ex\label{ex:nsy-intr-be} \gll
        sawah kito be-wayil \\
        field 1\textsc{pl.incl} \textsc{mid}-water \\
        \glt `our field has water.' \\
    \hfill (BJM02-005, 00:34:28–00:34:30, Speaker: Mas) 
\end{exe}

Transitive verbs in Nasal make use of a two-way voice distinction between an A-Voice (AV), marked by the so-called nasal prefix \textit{N-} `\textsc{av}' (\ref{ex:nsy-av}), and a P-Voice (PV), marked by the prefix \textit{di-} `\textsc{pv}' or a bare verbal stem (\ref{ex:nsy-pv}).\footnote{Given that bare verbal transitive roots almost exclusively mark for PV, they will be glossed in the examples as PV.} Similar to intransitive prefixes, these prefixes derive verbal predicates from both nominal and verbal roots. In both voice constructions, the Primary Argument (A in AV, P in PV) has privileged access to certain syntactic operations (e.g. relativization), and the Secondary Argument (A in PV, P in AV), depending on the referent, can be indexed by a pronominal clitic on the verb (\textit{=nyo} `3\textsc{sg}' in AV and \textit{ku=} `1\textsc{sg}' or \textit{mu=} `2\textsc{sg}' in PV). The Secondary Argument frequently forms a tight association with the verb, disallowing any intervening elements and forming a constituent that we refer to as the predicate complex.

\begin{exe}
    \ex\label{ex:nsy-av} \gll
        antakan ni nyak ng-gulai khetak telung, \\
        before \textsc{dem.addr} 1\textsc{sg} \textsc{av}-soup long.bean eggplant\\
        \glt `yesterday, I made long bean soup with eggplant,' \\
    \hfill (BJM02-048-02, 00:05:20–00:05:22, Speaker: Ita) 
\end{exe}

\begin{exe}
    \ex\label{ex:nsy-pv} \gll
        sawah wo Ros ni kak di-akuk=nyo lih Satimo, \\
        field sister R. \textsc{dem.addr} already \textsc{pv}-take=3\textsc{sg} by S. \\
        \glt `Ros's field has already been bought by Satimo,' \\
    \hfill (BJM02-005, 00:35:36–00:35:38, Speaker: Ari) 
\end{exe}

The suffixes \textit{-kun} `\textsc{caus/ben}' and \textit{-i} `\textsc{loc/plur}' most often function as valency increasing affixes that have both causative and applicative functions \citep[see][]{mcdonnell2024applicative}. The use of one of these suffixes allows, for example, for typically intransitive verbal roots to become transitive or for monotransitive roots to become ditransitive with benefactive, instrumental, or goal applied arguments. They are also used to derive transitive verbs from noun roots and even trigger semantic changes on the verb, such as iterative and pluractional meanings \citep[see][]{truong2022neglected}. 

Aside from the fact that the Secondary Argument often forms a predicate complex with the verb, word order in Nasal is otherwise variable, as is the case with many other languages of Western Indonesia. Argument fronting frequently occurs for emphasis regardless of voice construction. Because of this, both the marked word order, in which the Primary Argument follows the predicate complex (\ref{ex:var-order}), and the unmarked order, in which the Primary Argument precedes the predicate complex (\ref{ex:ditrans-order}), are common.

\begin{exe}
    \ex\label{ex:var-order} \gll
        pakai sintikh udi nyak \\
        use flashlight \textsc{dem.dist} 1\textsc{sg} \\
        \glt `I used that flashlight.' \\
    \hfill (BJM02-012, 00:37:18–00:37:19, Speaker: Een) 
\end{exe}

\begin{exe}
    \ex\label{ex:ditrans-order} \gll
        nyak ng-ucap-kun {tekhimo kasih} lawan kai, \\
        1\textsc{sg} \textsc{av}-say-\textsc{caus/ben} {thank you} with 2\textsc{pl} \\
        \glt `I say thank you to you all,' \\
    \hfill (BJM02-056-01, 00:13:11–00:13:13, Speaker: Arifin) 
\end{exe}



\subsection{Corpus of conversational data}
This study reports on the use of fillers in six transcribed conversations -- totalling approximately eight hours of speech -- taken to be representative of both men and women ranging from 20 to 60 years old from a larger corpus of 24 conversations. We revisit this larger corpus in \sectref{sec:Fillers in the larger corpus}. These conversations were coded for the various uses of placeholders and hesitators in Nasal along with other relevant information regarding repair, disfluencies, and syntactic position, among others. This coding allows us to take a comprehensive look at the uses of fillers in Nasal (without cherry picking), reporting on the relative frequencies of fillers and their functions. These recordings included both audio and video, but while the video assisted some in determining a given word's function (for example, pointing to indicate a prototypical demonstrative function), syntactic and prosodic cues were most reliable in determining the role of a placeholder and, thus, coding only applied to recorded speech. A brief description of each of the six conversations is provided in \tabref{tab:billings:corpus}, and the complete schema used for coding this dataset is shown in Appendix \ref{app:coding}. The remainder of the coded examples cited in this paper are taken from these six conversations.

\begin{table}
    \caption{Conversations included in the corpus.}
    \label{tab:billings:corpus}
    \begin{tabularx}{1\textwidth}{l X}
        \lsptoprule
        ID         & Description \\
        \midrule
        BJM02-012
        & A conversation between two cousins – Een (F, 23) and Nera (F, 29) – and their friend Metia (F, 26) in Gedung Menung about their families, weddings and marriages, and the weather \\
        BJM02-014
        & A conversation between three friends – Bahrul (M, 39), Hermansyah (M, 37), and Karnain (M, 45) – in Tanjung Baru primarily about fishing and keeping livestock \\
        BJM02-022
        & A conversation between three friends – Yogi (M, 26), Johan (M, 57), and Sumardianto (M, 35) – in Tanjung Betuah about finding work, growing various plants, and the prices of things \\
        BJM02-044
        & A conversation between three friends – Susi (F, 36), Seni (F, 37), and Mihar (F,  42) – in front of a house in Gedung Menung about planning going to the beach, cooking, looking for snails, and selling things online  \\
        BJM02-050
        & A conversation between three friends – Surnila (F, 50), Ros (F, 35), and Resta (F, 31) – in Tanjung Betuah about family happenings, cooking and food, and using technology \\
        BJM02-058
        & A conversation between three friends – Bahuri (M, 51), Redo (M, 29), and Wawan (M, 36) – in Tanjung Betuah discussing Nasal history and stories about Nasal ancestors \\
        \lspbottomrule
    \end{tabularx}
\end{table}

%%%
%%%
%%%
%%% 
%%% SECTION: Overview of Fillers in Nasal
%%%
%%%
%%%
%%%
\section{Overview of fillers in Nasal}\label{sec:Fillers}

Unlike some other languages of Western Indonesia, Nasal does not have a dedicated filler (compare, for example, Karo Batak \textit{kadih} in \cite[][118]{woollams_1996}; South Barisan Malay \textit{anu} in \cite{chapters/mcdonnell_billings}). Instead, demonstrative and interrogative pronouns frequently fulfill this role and are variously employed as both hesitators and placeholders.

Nasal employs a three-way contrast between demonstratives for indicating objects near the speaker (\textit{ajo} `\textsc{dem.sp}'), near the addressee (\textit{ani}/\textit{heni} `\textsc{dem.addr}'), and far from both the speaker and addressee (\textit{udi} `\textsc{dem.dist}'). Each of these can variously be used as a demonstrative pronoun and as a demonstrative determiner, which is the last element in an NP. Such a three-way contrast is common in Austronesian languages, but, as noted above, the complexity of the entire demonstrative system as a whole is rather remarkable (see \tabref{tab:billings:demonstratives} below). All three demonstratives have an emphatic form with initial \textit{ng-} as well as a determiner form that occurs at the right edge of a noun phrase and is otherwise semantically equivalent to the determiner uses of the independent forms. This demonstrative series also forms the roots for locative and manner demonstratives. Although Nasal has a dedicated set of manner adverbs based on these roots, the emphatic forms with initial \textit{ng-} are also frequently utilized in this role. An overview of this demonstrative system is provided in \tabref{tab:billings:demonstratives}, and examples of pronominal and determiner use of demonstratives are provided in (\ref{ex:intro-ajo}) and (\ref{ex:intro-sudi}) below, respectively. In these and following examples, an arrow (→) is used to indicate the line in which the referenced demonstrative, interrogative, or filler is used.

\begin{table}
    \caption{Nasal demonstrative system.}
    \label{tab:billings:demonstratives}
    \begin{tabularx}{.8\textwidth}{l c c c c c}
        \lsptoprule
        & Ind. & Emph. & Det. & Loc. & Manner \\
        \midrule
        \textsc{dem.sp} & \textit{ajo} & \textit{ngajo} & \textit{sijo} & \textit{(d)ijo} & jeujo \\
        \textsc{dem.addr} & \textit{ani/heni} & \textit{ngani} & \textit{ni} & \textit{(d)isan} & \textit{jeusan} \\
        \textsc{dem.dist} & \textit{udi} & \textit{ngudi} & \textit{sudi} & \textit{(d)udi} & \textit{jeudi} \\
        \lspbottomrule
    \end{tabularx}
\end{table}

\begin{exe}
    \ex\label{ex:intro-ajo} \begin{xlist}[0\quad →A:]
        \exi{1\quad \hphantom{→}S:} \gll
        na:, \\
        well \\
        \glt `ah,' \\
        \exi{2\quad →\hphantom{S:}} \gll
        \textbf{ajo} khetuk \\
        \textsc{dem.sp} delicious \\
        \glt `these are delicious.' \\
    \end{xlist}
    \hfill (BJM02-044-01, 00:09:56–00:09:57, Speaker: Seni) 
\end{exe}

\begin{exe}
    \ex\label{ex:intro-sudi} \begin{xlist}[0\quad →A:]
        \exi{1\quad \hphantom{→}S:} \gll
        mak nihan njuk=nyo: \\
        \textsc{neg} very \textsc{pv}.give=3\textsc{sg} \\
        \glt `she really didn't give (it to her).' \\
        \exi{2\quad \hphantom{→S:}} \gll
        mak njuk=nyo:, \\
        \textsc{neg} \textsc{pv}.give=3\textsc{sg} \\
        \glt `she didn't give (to her),' \\
        \exi{3\quad →\hphantom{S:}} \gll
        suluh \textbf{sudi} \\
        firewood \textsc{dem.dist} \\
        \glt `that firewood.' \\
    \end{xlist}
    \hfill (BJM02-044-01, 00:13:39–00:13:43, Speaker: Susi) 
\end{exe}

All three of the demonstrative pronouns can operate as both hesitators and placeholders. Although the near-speaker demonstrative \textit{ajo} is not found to be used as a hesitator in the corpus, its absence may simply be due to its lower frequency as a filler overall (as reflected by only a single clear use as a placeholder being found in the corpus) rather than an inability of the near-speaker demonstrative to be used as a hesitator. This stands in contrast to Indonesian \citep{wouk2005syntax} and Besemah (\cite{chapters/mcdonnell_billings}) in which the proximal demonstrative is much more frequent.

In addition to the demonstrative pronouns, the interrogative pronoun \textit{api} `what' is also frequently found operating as a filler and, as with the demonstratives, is used as both a hesitator and a placeholder. In its typical non-filler functions, \textit{api} `what' is used as an interrogative pronoun (\ref{ex:api-intro-inter}), an indefinite pronoun (\ref{ex:api-intro-indef}, \ref{ex:api-intro-indef2}), and an alternative correlative conjunction (\ref{ex:api-intro-conj}). The other interrogatives – \textit{sapo} `who', \textit{kebilo} `when', \textit{(d)ipo} `where', \textit{jeupo} `how', \textit{sipo} `which' – are not used as fillers. In theory, the interrogative \textit{ngapi} `why' could be homophonous with the morphologically complex \textit{ng-api} `\textsc{av}-what', although it is not attested in our corpus. 

\begin{exe}
    \ex\label{ex:api-intro-inter} \gll
        n-(t)akhuk \textbf{api} tian? \\
        \textsc{av}-plant what 3\textsc{pl} \\
        \glt `what did they plant?' \\
    \hfill (BJM02-012-01, 00:43:23–00:43:24, Speaker: Metia) 

    \ex\label{ex:api-intro-indef} \gll
        sekh:ah ngio ng-(k)icik \textbf{a:pi} \\
        surrender \textsc{emph} \textsc{av}-say what \\
        \glt `it's up to us, (we can) talk about whatever.' \\
    \hfill (BJM02-012-01, 00:18:14–00:18:16, Speaker: Een) 
\end{exe}

\begin{exe}
    \ex\label{ex:api-intro-indef2} \gll
        wat \textbf{api-api} untuk ayuk \\
        exist what for older.sibling \\
        \glt `there's something for him.' \\
    \hfill (BJM02-046-01, 00:18:03–00:18:05, Speaker: Lili)
\end{exe}

\begin{exe}
    \ex\label{ex:api-intro-conj} \gll
        sebenakhnyo husnul \textbf{api} khusnul? \\
        actually h. what k. \\
        \glt `actually (how is it pronounced), \textit{husnul} or \textit{khusnul}?'\footnote{In this example, the speaker is asking how to pronounce an Arabic word in the the phrase \textit{husnul khatimah} `a beautiful conclusion'.}\\
    \hfill (BJM02-012-01, 01:05:27–01:05:30, Speaker: Een) 
\end{exe}


Although Nasal demonstratives and the interrogative \textit{api} operating in their prototypical functions are unable to take affixes or proclitics, both of these types of fillers can appear with a variety of verbal and nominal affixes as well as the pronominal proclitics \textit{ku=} `1\textsc{sg}' and \textit{mu=} `2\textsc{sg}'. Since hesitators by definition do not occur with any  morphological marking, morphologically complex forms are entirely restricted to the placeholder use of the fillers. As opposed to affixes and proclitics, enclitics are found both when demonstratives and \textit{api} are used in their typical function and when used as a placeholder.

For the remainder of this chapter, general claims made about hesitators or placeholders, unless otherwise specified, should be taken in reference to all four function words as fillers under investigation, both demonstrative and interrogative. Other general distributional tendencies or formal differences between any of the function words when used as fillers not mentioned here were not apparent in our coding and require additional data for analysis.

\subsection{The indeterminacy of placeholders and hesitators}
Given their polyfunctional nature, it is often difficult to determine whether a given filler in Nasal is operating as a hesitator or as a placeholder. This is made even more difficult by the fact that the typical diagnostics for distinguishing between these two filler types -- such as case, agreement, and other morphological marking -- are largely absent in Nasal. The only help in this matter comes from voice, applicative, and nominalizing affixes, although even these are not always obligatory; see, for example, (\ref{ex:indet-nom}) and (\ref{ex:indet-verb}), both of which lack morphology in the placeholder but fill the syntactic position of a noun and a verb respectively. Even clitics are not of any immediate help. While first- and second-person enclitics only attach to these function words when they are used as placeholders, they only rarely attach to fillers (see \tabref{tab:billings:filler-clitic} further below). Although similarly disambiguating placeholders from hesitators, the third-person enclitic (as used in \ref{ex:api-intro-place}), on the other hand, does not necessarily disambiguate filler from non-filler uses of function words since it can freely attach to function words to mark definiteness. In (\ref{ex:api-intro-place}), the repair provides clear evidence of its placeholder status.

\begin{exe}
    \ex\label{ex:indet-nom} \begin{xlist}[0\quad →A:]
        \exi{1\quad →B:} \gll
        duwik .. \textbf{ani:} \\
        many {} \textsc{dem.addr} \\
        \glt `(there are) many \textbf{whatchamacallit}.' \\
        \exi{2\quad \hphantom{→B:}} \gll
        anak .. api kio? \\
        child {} what \textsc{emph} \\
        \glt `baby .. what is it?' \\
        \exi{3\quad \hphantom{→B:}} \gll
        \uline{ipun hukhang} \\
        {baby shrimp} \\
        \glt `\uline{baby shrimp}.' \\
    \end{xlist}
    \hfill (BJM02-014-01, 00:49:39–00:49:43, Speaker: Baidi) 

    \ex\label{ex:indet-verb} \begin{xlist}[0\quad →A:]
        \exi{1\quad →J:} \gll
        nyo sudi hago \textbf{ani:} \\
        3\textsc{sg} \textsc{dem.dist} want \textsc{dem.addr} \\
        \glt `he wants \textbf{whatchamacallit}.' \\
        \exi{2\quad \hphantom{→J:}} \gll
        \uline{be-variasi:} \\
        \textsc{mid}-variety \\
        \glt `\uline{(for it) to have variety}.' \\
        \exi{3\quad \hphantom{→J:}} \gll
        pajuhan=nyo sudi \\
        food=3\textsc{sg} \textsc{dem.dist}. \\
        \glt `his food.' \\
    \end{xlist}
    \hfill (BJM02-014-01, 01:01:56–01:02:00, Speaker: Edwar) 
\end{exe}

\begin{exe}
    \ex\label{ex:api-intro-place} \begin{xlist}[0\quad →A:]
        \exi{1\quad \hphantom{→}A:} \gll
        pada saat ani, \\
        at time \textsc{dem.addr} \\
        \glt `at that time,' \\
        \exi{2\quad →\hphantom{A:}} \gll
        mimang wat \textbf{api=nyo}, \\
        really exist what=3\textsc{sg} \\
        \glt `there really was \textbf{whatchamallit},' \\
        \exi{3\quad \hphantom{→A:}} \gll
        \uline{sejarah=nyo} \\
        history=3\textsc{sg} \\
        \glt `\uline{a story}.' \\
    \end{xlist}
    \hfill (BJM02-058-01, 00:11:04–00:11:09, Speaker: Ahmad) 
\end{exe}

As with many roots in Nasal, however, it is not always possible to distinguish the word class of the filler, whether demonstrative or interrogative pronoun. That is, without specific morphological marking on the affixed word indicating one class or the other (for example, the verbal AV prefix \textit{N-} or the nominalizing suffix \textit{-an}), a root can operate in its normal function as a demonstrative\slash interrogative, as a placeholder noun, or as a placeholder intransitive verb (for placeholder transitive verbs, the valency\hyp increasing applicative \textit{-kun} is present in virtually every case, as will be seen later).

Two additional factors complicate matters. As mentioned in \sectref{sec:Nasal}, Nasal, like other languages of Western Indonesia, tends to omit arguments without any indexing on the verb. When arguments have been omitted, an otherwise evident verbal placeholder can appear very similar to a hesitator or, at times, a prototypical demonstrative use, given that all three can utilize a bare root. Second, placeholders in Nasal frequently are not repaired. In the absence of a word search or recycling, the placeholder can again look very similar to a demonstrative pronoun. Of the coded data, more than a third of placeholder fillers are not repaired (see \sectref{sec:Placeholders.Repair}).

For the examples given above (and many others in the corpus), the lack of preceding context for the intended meaning of \textit{ani} `\textsc{dem.addr}' -- whether verbal or gestural -- in (\ref{ex:indet-nom}) and (\ref{ex:indet-verb}) eliminates the possibility of it being used anaphorically. The disambiguation of hesitator and placeholder role for these two examples can be found in the repair, although this does not always occur. In cases lacking repair, participation in the syntax of the clause and continuation of normal intonation are key markers for placeholder uses. For example, in (\ref{ex:api-intro-place}) above, \textit{api=nyo} could be, as is often the case, a self-addressed question. However, the participation in the clause as Primary Argument, the continuing intonation, and the recycled \textit{=nyo} all point to the interrogative being used as a placeholder. Such contextual, syntactic, and intonational clues allow us to reliably code many of the hesitator and placeholder uses of fillers, but a large number remain indeterminate.

\tabref{tab:billings:hes-vs-place} contains frequency counts from the six-conversation corpus (approximately 60,000 words) for both the filler and prototypical uses of each function word. While the 18 speakers varied in their use of hesitators and fillers, there were no clear discrepancies or inconsistencies in the use of each across speakers in our sample. The demonstrative \textit{udi} `\textsc{dem.dist}' as a hesitator (6 tokens) and the interrogative pronoun as a placeholder (10 tokens) were each only used by a single speaker. However, these uses were checked against additional participants and conversations in the larger corpus and it was found that many additional speakers outside of these six conversations use \textit{udi} `\textsc{dem.dist}' and the interrogative in the same functions. Thus, these numbers are likely a result of low frequency, and not simply speaker preference.

Demonstrative and interrogative pronouns more frequently function in their prototypical demonstrative or interrogative use. As fillers, the demonstratives have a clear preference to be placeholders, and the interrogative has a preference to be a hesitator. Each of these uses is discussed further in the following sections. Rather than forcing a particular analysis, a filler was coded as indeterminate where either its role as a placeholder or hesitator was unclear or whether it should be considered a filler at all was unclear. The number of these cases is also shown in \tabref{tab:billings:hes-vs-place}.


\begin{table}
    \caption{Hesitator and placeholder frequencies of Nasal fillers.}
    \label{tab:billings:hes-vs-place}
    \begin{tabular}{l r r r r}
        \lsptoprule
        & \multicolumn{2}{c}{Filler}\\
        \cmidrule(lr){2-3}
        & {Hesitator} & {Placeholder} &{Indeterminate} & {Prototypical Use} \\
        \midrule
        \textsc{dem.sp}   & -  & 1  & 1 & 237 \\
        \textsc{dem.addr} & 24 & 77 & 27 & 496 \\
        \textsc{dem.dist} & 6  & 18 & 3 & 289 \\
        Interrogative     & 22 & 10 & 2 & 546 \\
        \lspbottomrule
    \end{tabular}
\end{table}

%%%
%%%
%%%
%%% 
%%% SECTION: Hesitators
%%%
%%%
%%%
%%%
\section{Hesitator fillers}\label{sec:HesitatorFillers}

As seen in \tabref{tab:billings:hes-vs-place}, the filler most likely to be used as a hesitator rather than a placeholder is the interrogative pronoun \textit{api} `what', as in (\ref{ex:api-hes}).

\begin{exe}
    \ex\label{ex:api-hes} \begin{xlist}[0\quad →A:]
        \exi{1\quad \hphantom{→}N:} \gll
        Bakakh m-ulang jenu \\
        B. \textsc{intr}-return earlier\\
        \glt `Bakakh returned then.' \\
        \exi{2\quad →\hphantom{N:}} \gll
        angan=ku hago ng-(k)ikit di .. \textbf{api} disan, \\
        thought=1\textsc{sg} want \textsc{av}-fish at {} what there \\
        \glt `I thought he wanted to fish at, \textbf{uh}, there,' \\
        \exi{3\quad \hphantom{→N:}} \gll
        di Saudi tian ni \\
        at S. 3\textsc{pl} \textsc{dem.addr} \\
        \glt `where Saudi and them were.' \\
    \end{xlist}
    \hfill (BJM02-014-01, 00:24:16–00:24:20, Speaker: Nain) 
\end{exe}

In this example, Nain uses \textit{api} in line 2 to hold the floor when retelling how Bakakh had to return home to get his fishing net. However, when used to express hesitation, \textit{api} is much more often encountered in self-addressed questions (63 instances in the 6-conversation corpus, not included in the frequency counts of \tabref{tab:billings:hes-vs-place}) compared to \textit{api} on its own (22, as shown in \tabref{tab:billings:hes-vs-place}). In (\ref{ex:api-sudi}), for example, Ahmad is about to recount a story and uses the phrase \textit{api sudi} `what's that?' in line 2 to delay while remembering the story's content and main character.

\begin{exe}
    \ex\label{ex:api-sudi} \begin{xlist}[0\quad →A:]
        \exi{1\quad \hphantom{→}A:} \gll
        amun: cekhito:, \\
        if story \\
        \glt `if it's a story,' \\
        \exi{2\quad →\hphantom{A:}} \gll
        yang \textbf{api} \textbf{sudi}, \\
        \textsc{rel} what \textsc{dem.dist} \\
        \glt `that, \textbf{what's it},' \\
        \exi{3\quad \hphantom{→A:}} \gll
        yang pintakh, \\
        \textsc{rel} smart \\
        \glt `that (is about somebody who's) smart,' \\
        \exi{4\quad \hphantom{→A:}} \gll
        amun cekhito Arab sudi kan? \\
        if story Arab \textsc{dem.dist} \textsc{tag} \\
        \glt `if it's one of those Arab stories, right?' \\
        \exi{5\quad \hphantom{→A:}} \gll
        ido Abu Nawas: \\
        true A. N. \\
        \glt `then that (is going to be one about) Abu Nawas.' \\
    \end{xlist}
    \hfill (BJM02-058-02, 00:14:54–00:15:00, Speaker: Ahmad) 
\end{exe}

Similarly, in (\ref{ex:api-kio-namonyo}), Nera is telling what she had recently cooked and uses the phrase \textit{api kio namonyo} `what's it called?' while trying to recall in line 2. Many such similar self-addressed questions in which the common element is always \textit{api} are frequently used to express hesitation.

\begin{exe}
    \ex\label{ex:api-kio-namonyo} \begin{xlist}[0\quad →A:]
        \exi{1\quad \hphantom{→}N:} \gll
        kito:--- \\
        1\textsc{pl.incl} \\
        \glt `we' \\
        \exi{2\quad →\hphantom{N:}} \gll
        \textbf{api} \textbf{kio} \textbf{namo=nyo}? \\
        what \textsc{emph} name=3\textsc{sg}\\
        \glt `\textbf{what's it called?}' \\
        \exi{3\quad \hphantom{→N:}} \gll
        ny-(s)ani:k, \\
        \textsc{av}-make \\
        \glt `made,' \\
        \exi{4\quad \hphantom{→N:}} \gll
        (TSK) gulai talus sudi \\
        {} soup taro \textsc{dem.dist} \\
        \glt `that taro soup.' \\
    \end{xlist}
    \hfill (BJM02-012-01, 00:09:54–00:09:59, Speaker: Nera) 
\end{exe}

The distinction between a hesitator use of \textit{api} and a self-addressed question appears on the surface to be clear-cut. The former is a non-referential interjection vocalizing hesitation in speech while the speaker attempts to formulate or complete an utterance. The latter is a fully-formed and referential interrogative, externally expressing the speaker's uncertainty during an internal recollection of a difficult to remember or lost referent. However, as with other matters involving fillers, distinguishing these two in practice is difficult. First, both the hesitator and the self-addressed question may be a single word or an entire phrase. For interjective hesitators, both \textit{api} and \textit{api kio} (with the emphatic discourse marker \textit{kio} `\textsc{emph}') are often employed. For self-addressed questions, the same can be said, with \textit{api?} `what?', \textit{api kio?} `what?', \textit{api kio gelakhnyo?} `what's its name?', or \textit{api namonyo?} `what's its name', all being common choices. Thus, unlike English `what do you call it?' and `whatchamacallit' which are distinguished phonologically (\cite{enfield2003definition}), the structure of the hesitator utterance in Nasal does not immediately help in distinguishing the two.

Second, while there are some instances of clear-cut intonational differences between an interjective hesitator \textit{api} and a self-addressed question, this is not always the case. Other non-lexical markers, if they occur, may help distinguish the two, such as lengthening or truncation of the preceding IU leading into a self-addressed question. In \figref{fig:spect-ques} and \figref{fig:spect-hes} we see two such distinguishable cases.

        \begin{figure}
            \includegraphics[width=\linewidth]{praat2_edited.png}
            \caption{Self-addressed question \textit{api gelakh ni?} used for hesitation (BJM02-014-01, 01:07:24–01:07:27, Speaker: Edwar).}
            \label{fig:spect-ques}
        \end{figure}
        
        \begin{figure}
            \includegraphics[width=\linewidth]{praat_edited.png}
            \caption{Interjective hesitator \textit{api} in continuous speech (BJM02-058-01, 00:14:38–00:14:41, Speaker: Ahmad).}
            \label{fig:spect-hes}
        \end{figure}

In \figref{fig:spect-ques}, Edwar is talking about people from a certain area but forgets the name of the area, first using \textit{ani} as a placeholder and then hesitating while trying to remember the name. This attempt at recall is expressed with the self-addressed question \textit{api gelakh ni?} `what's this called?' where \textit{ani} appears yet again in a reduced form \textit{ni} to reference the sought referent. That this final element \textit{ni} of the IU immediately preceding \textit{api} is lengthened, an independent IU is started, and distinctly interrogative intonation is used all indicate that this is a self-addressed question. In \figref{fig:spect-hes}, Ahmad is telling a story in which somebody catches a deer. He starts saying the wrong word for deer (\textit{kijang}), cuts himself off, hesitates briefly with an interjective \textit{api}, and then continues normally. No independent IU is started, intonation is continuous over the whole utterance, and truncation occurs, but only as a means to prevent production of the wrong word.

While such distinguishable cases do exist, many are ambiguous for an interjective hesitator or a self-addressed question, contributing further to the indeterminacy in establishing the filler uses of function words in Nasal.

The demonstratives \textit{ani/heni} `\textsc{dem.addr}' (\ref{ex:ani-hes}) and \textit{udi} `\textsc{dem.dist}' (\ref{ex:udi-hes}) are also often used to signal hesitation. In (\ref{ex:ani-hes}), Edwar is talking about how it would be easier to use fishing nets with two people and uses \textit{ani} `\textsc{dem.addr}' as a hesistative in line 3 when he cannot remember the location of where the second person would be standing. This hesitation is immediately followed up by a self-addressed question to hold the floor within the same IU.

\begin{exe}
    \ex\label{ex:ani-hes} \begin{xlist}[0\quad →A:]
        \exi{1\quad \hphantom{→}E:} \gll
        nyo sudi kan, \\
        3\textsc{sg} \textsc{dem.dist} \textsc{tag} \\
        \glt `he, right,'
        \exi{2\quad \hphantom{→E:}} \gll
        khampusan=nyo ni, \\
        shallows=3\textsc{sg} \textsc{dem.addr} \\
        \glt `(can go into) these shallows,' \\
        \exi{3\quad →\hphantom{E:}} \gll
        khamp-- \textbf{ani:}, \\
        {} \textsc{dem.addr} \\
        \glt `\textbf{uh},' \\
        \exi{4\quad \hphantom{→E:}} \gll
        api gelakh ni? \\
        what name \textsc{dem.addr}\\
        \glt `what's it called?' \\
        \exi{5\quad \hphantom{→E:}} \gll
        khelum pinggikh \\
        deep edge \\
        \glt `(go to the part that's) deep at the edge.' \\
    \end{xlist}
    \hfill (BJM02-014-01, 00:21:35–00:21:41, Speaker: Edwar) 
\end{exe}

Johan's use of \textit{udi} multiple times in (\ref{ex:udi-hes}) in lines 3 and 4 shows his hesitation in trying to find the right description for a kind of bread eaten during war.

\begin{exe}
    \ex\label{ex:udi-hes} \begin{xlist}[0\quad →A:]
        \exi{1\quad \hphantom{→}J:} \gll
        khuti so duwik telul=nyo susu=nyo \\
        bread \textsc{dem.sp} many egg=3\textsc{sg} milk=3\textsc{sg} \\
        \glt `this bread has a lot of eggs and milk.' \\
        \exi{2\quad \hphantom{→J:}} \gll
        e-- kung ngam--- \\
        {} not.yet {} \\
        \glt `not yet---' \\
        \exi{3\quad →\hphantom{J:}} \gll
        \textbf{udi} khuti hulun, \\
        \textsc{dem.dist} bread person \\
        \glt `\textbf{uh}, the bread that people,' \\
        \exi{4\quad →\hphantom{J:}} \gll
        pe-- \textbf{udi}, \\
        {} \textsc{dem.dist} \\
        \glt `\textbf{uh}' \\
        \exi{5\quad \hphantom{→J:}} \gll
        pekhang sudi kidah, \\
        war \textsc{dem.dist} \textsc{emph} \\
        \glt `those ones (that you eat) in war' \\
        \exi{6\quad \hphantom{→J:}} \gll
        mbak i:jo nihan \\
        same.size \textsc{dem.sp} very \\
        \glt `it's the same size as this.' \\
    \end{xlist}
    \hfill (BJM02-022-01, 01:09:48–01:09:55, Speaker: Johan) 
\end{exe}

While it is most common to encounter hesitators on their own, other markers of hesitation (e.g. pause, truncation, lengthening) are frequently used in conjunction with them. In fact, in languages like Nasal where hesitator particles akin to English \textit{uh} and \textit{um} are far less frequent (although not entirely absent, as with \textit{e} in (\ref{ex:place-ani}) further below), it is more common to find these additional means of expressing hesitation, as has also been noted, for example, for Ilokano and Lauje (\cite{streeck_1996}) as well as Tagalog (\cite{himmelmann2014prosodicphrasing}). Those that were identified in the coding are shown in \tabref{tab:billings:non-lex-hes}. In some cases, more than one of these co-markers of hesitation were found to co-occur with the hesitator.

\begin{table}[h]
    \caption{Co-markers of hesitation.}
    \label{tab:billings:non-lex-hes}
    \begin{tabularx}{\textwidth}{l Yrrrr}
        \lsptoprule
        & {None} & {Pause} & {Truncation} & {Lengthening} & {Glottalization} \\
        \midrule
        Demonstrative & 16 & 9 & 7 & 2 & 1 \\
        Interrogative & 14 & 2 & 6 & - & - \\
        %Demonstrative & 90 & 13 & 10 & 16 & \\
        %Interrogative & 10 & - & - & - & \\
        \lspbottomrule
    \end{tabularx}
\end{table}

The most frequent co-marker of hesitation is a brief pause (\ref{ex:api-hes}) or truncation of a lexical item or IU (\ref{ex:ani-hes}, \ref{ex:udi-hes}). Lengthening of the word immediately preceding the hesitator (\ref{ex:api-kio-namonyo}) is also present but not common, and we identified one case where hesitation is signalled additionally by glottalization (\ref{ex:api-hes-glottal}). In this last example (\ref{ex:api-hes-glottal}), Johan and Yogi are discussing finding work in Nasal. When Johan asks what work Yogi has found in Nasal, Johan expresses hesitation in finishing his response through both glottalization and the demonstrative \textit{ani}.

\begin{exe}
    \ex\label{ex:api-hes-glottal} \begin{xlist}[0\quad →A:]
        \exi{1\quad \hphantom{→}J:} \gll
        api lukak di dijo so? \\
        what work at here \textsc{dem.sp}? \\
        \glt `what work is there (for you) here?' \\
        \exi{2\quad \hphantom{→}Y:} \gll
        mak deduk igo:, \\
        \textsc{neg} exist too.much \\
        \glt `there's not too much,' \\
        \exi{3\quad →\hphantom{Y:}} \gll
        amun wat .. \%lukak .. \%\textbf{ani} \%tukhu:k \\
        if exist {} {\  }work {} {\  }\textsc{dem.addr} {\  }join.in \\
        \glt `if there is .. work .. \textbf{uh}, I do it.' \\
    \end{xlist}
    \hfill (BJM02-022-01, 00:04:02–00:04:07, Speakers: Johan, Yogi) 
\end{exe}

Proportionate to their frequency of use, there do not appear to be any major differences between the uses of these non-lexical co-markers with demonstratives or interrogatives. Although a brief pause is seemingly more commonly found before a demonstrative used as a hesitator, a larger sample would need to be taken in order to confirm this correlation.

Most other methods of signalling hesitation in Nasal that can be used independently of lexical hesitators are either interjective (as with \textit{na} `well' in (\ref{ex:place-udi}) or \textit{e} `um' in (\ref{ex:place-affix-verb}) below) or self-addressed questions (see above). Questions containing \textit{api} -- as in (\ref{ex:api-sudi}) and (\ref{ex:api-kio-namonyo}) above -- remain the most common of these, even more so than the use of a demonstrative or \textit{api} on its own.

\section{Placeholder fillers}\label{sec:Placeholders}

Placeholders are often described in reference to their target (e.g. word class, grammatical relation), which according to \textcite{podlesskaya2010parameters} refers to the word (or presumably phrase) that the speaker intended. One issue with the term \textit{target}, however, is that it requires the analyst to know what the speaker intended, which is only truly possible when the placeholder has been repaired. Since the majority of forms are not repaired in Nasal, we follow \textcite{fox2010introduction}, who frames her discussion of fillers using an Interactional Linguistics approach. She describes placeholders as ``fulfill[ing] the syntactic projection of the turn so far'' \parencite[][2]{fox2010introduction}. By describing fillers in this way, we need not make any claim about the word or phrase the speaker intended in the absence of repair. Instead, we simply describe the syntactic position of the placeholder. In practice, however, fulfilling a syntactic projection and target can be considered rough equivalents with the caveat that we make no claim about what the speaker intended.

In the six conversations analyzed, there are a total of 106 placeholder uses of fillers. As mentioned in \sectref{sec:HesitatorFillers}, there is a tendency for demonstratives in Nasal to be used as placeholders rather than as hesitators (see \tabref{tab:billings:hes-vs-place}). The most common demonstrative to be used as a placeholder (as with hesitators) is \textit{ani/heni} `\textsc{dem.addr}' (\ref{ex:place-ani}), followed then by \textit{udi} `\textsc{dem.dist}' (\ref{ex:place-udi}), and finally \textit{ajo} `\textsc{dem.sp}' (\ref{ex:place-ajo}). Metia uses \textit{ani} in line 1 of (\ref{ex:place-ani}) as a placeholder when recommending that Een buy an album for her wedding photos. There, \textit{ani} fulfills the syntactic projection of a clausal argument, and the target for the placeholder is \textit{bingkai foto} `picture frame', as demonstrated by the repair in line 2.

\begin{exe}
    \ex\label{ex:place-ani} \begin{xlist}[0\quad →A:]
        \exi{1\quad →M:} \gll
        beli-kun pai e \textbf{ani} wu:i \\
        buy-\textsc{caus/ben} first um \textsc{dem.addr} \textsc{emph}\\
        \glt `buy, uh, a \textbf{whatchamacallit} please.' \\
        \exi{2\quad \hphantom{→M:}} \gll
        \uline{bingkai foto} \\
        {frame photo} \\
        \glt `\uline{a picture frame}.' \\
    \end{xlist}
    \hfill (BJM02-012-01, 00:58:43–00:58:48, Speaker: Metia) 
\end{exe}

% Although they differ somewhat conceptually, fulfilling syntactic projection can be understood as rough equivalent to target in the sense that a placeholder can be said to fulfill a predicate projection or its target is a predicate. ``Repair'' is used to mean the attempted replacement of a missing or placeholder-substituted referent. If, for example, truncated speech or a placeholder is not repaired, it said to be left ``unrepaired''.

% whose syntactic position is being held by the placeholder. th The placeholder in this examples occurs in an argument position. Following \textcite{fox2010introduction}, who frames her discussion of fillers using an Interactional Linguistics, placeholders fulfilling a syntactic projection, a concept developed in Conversation Analysis but common in Interactional Linguistics. Projection here simply ``means that the earlier part of a structure foreshadows its later trajectory and thus makes its completion predictable'' \parencite[][39]{couper-kuhlen2018interactional}. For the purposes of this paper, placeholders may occur in various syntactic positions, which we refer to as fulfilling syntactic projections. Elsewhere in this volume, the term target from \textcite{podlesskaya2010parameters} is used to refer to the word (or presumably phrase) that the speaker intended whose syntactic position is being held by the placeholder. 


% The terminology used here follows that of the Interactional Linguistics literature (\cite{fox2010introduction}). ``Projection'' is a concept developed in Conversation Analysis \parencite{} but common in Interactional Linguistics \parencite{}. For the purposes of this paper, it can be understoodis to beused as a rough equivalent to ``target'' \parencite{} in the sense that a placeholder can be said to ``fulfill a syntactic projection'' or ``project'' a specific target (e.g., noun, verb, predicate, etc.). 

In (\ref{ex:place-udi}), Johan is talking about what he had recently planted and uses \textit{udi} to hold the place while he recalls \textit{kacang buncis} `green beans'. In this case, the placeholder \textit{udi} fulfills the syntactic projection of a nominal argument.

\begin{exe}
    \ex\label{ex:place-udi} \begin{xlist}[0\quad →A:]
        \exi{1\quad →J:} \gll
        na n-(t)akhuk .. \textbf{udi:} nyak, \\
        well \textsc{av}-plant {} \textsc{dem.dist} 1\textsc{sg} \\
        \glt `well, I plant \textbf{whatchamacallit},' \\
        \exi{2\quad \hphantom{→J:}} \gll
        api?\\
        what \\
        \glt `what?' \\
        \exi{3\quad \hphantom{→J:}} \gll
        \uline{kacang buncis} \\
        {green bean} \\
        \glt `\uline{green beans}.' \\
    \end{xlist}
    \hfill (BJM02-022-01, 00:57:30–00:57:34, Speaker: Johan) 
\end{exe}

Finally, Edwar uses both \textit{ajo} and \textit{ani} in (\ref{ex:place-ajo}) for somebody he had seen earlier, while he tries to remember his name, fulfilling a nominal projection in both instances. 

\begin{exe}
    \ex\label{ex:place-ajo} \begin{xlist}[0\quad →A:]
        \exi{1\quad \hphantom{→}E:} \gll
        pikho=do m-beli \\
        how.much=\textsc{emph} \textsc{av}-buy \\
        \glt `however much (it is, he will) buy (it),'
        \exi{2\quad →\hphantom{E:}} \gll
        d- .. d- dang \textbf{ajo} \\
        {} {} {} mister \textsc{dem.sp} \\
        \glt `Mr. \textbf{whatchamacallit}.' \\
        \exi{3\quad →\hphantom{E:}} \gll
        dang: \textbf{ani} .. \uline{Brad} jenu \\
        mister \textsc{dem.addr} {} B. earlier \\
        \glt `Mr. \textbf{whatchamacallit} .. \uline{Brad} from earlier.' \\
    \end{xlist}
    \hfill (BJM02-014-01, 01:10:28–01:10:32, Speaker: Edwar) 
\end{exe}

Unlike hesitators, placeholders may take morphological marking both via affixation and cliticization. Affixation has a clear tendency to occur when the placeholder is operating in a verbal function, as in (\ref{ex:place-affix-verb}), accounting for 18 of the 20 affixed forms. The remaining two are nominal placeholders, as in (\ref{ex:place-affix-nom}). In (\ref{ex:place-affix-verb}), Susi and Seni are talking about somebody else's kid being sick. Susi asks about how to treat the sickness, and Seni describes the process, using \textit{di-udi-kun} to hold the place of an unrepaired verb, understood from context to be rubbing on oil. 

\begin{exe}
    \ex\label{ex:place-affix-verb} \begin{xlist}[0\quad →SU:]
        \exi{1\quad \hphantom{→}SU:} \gll
        a:pi ubat=nyo? \\
        \textsc{dem.int} medicine=3\textsc{sg} \\
        \glt `what's the treatment for it?' \\
        \exi{2\quad \hphantom{→}SE:} \gll
        e, \\
        \textsc{hes} \\
        \glt `uh,' \\
        \exi{3\quad \hphantom{→SE:}} \gll
        yang panas-pana:s \\
        \textsc{rel} \textsc{rdp}-hot \\
        \glt `that hot stuff.' \\
        \exi{4\quad →\hphantom{SE:}} \gll
        \textbf{di-udi-kun} minyak {kayu puti:h}, \\
        \textsc{pv}-\textsc{dem.addr}-\textsc{caus/ben} oil {cajuput} \\
        \glt `cajuput oil is \textbf{whatchamacallit-ed} (\uline{rubbed on}),' \\
        \exi{5\quad \hphantom{→SE:}} \gll
        adu tapal-i busung, ... \\
        after \textsc{pv}.apply-\textsc{loc/dist} belly \\
        \glt `and then it's applied to the belly,' \\
    \end{xlist}
    \hfill (BJM02-044-02, 00:04:34–00:04:40, Speakers: Susi, Seni) 
\end{exe}

Edwar uses the nominalizing suffix \textit{-an} in \textit{ani-an} `\textsc{dem.addr-nmlz}' in the place of `difference' in (\ref{ex:place-affix-nom}) while remarking that there's no difference between the two types of bamboo he and Nain are discussing. This interpretation is reinforced by his following IU.

\begin{exe}
    \ex\label{ex:place-affix-nom} \begin{xlist}[0\quad →A:]
        \exi{1\quad \hphantom{→}N:} \gll
        bet:i:k m-belah=nyo, \\
        nice \textsc{av}-cut=3\textsc{sg} \\
        \glt `it's easy to split it,' \\
        \exi{2\quad \hphantom{→}E:} \gll
        beti:k \\
        nice \\
        \glt `easy.' \\
        \exi{3\quad →\hphantom{E:}} \gll
        mak deduk [\textbf{ani-an}{]} \\
        \textsc{neg} exist \textsc{dem.addr}-\textsc{nmlz} \\
        \glt `there isn't any \textbf{whatchamacallit} (=\uline{difference}).' \\
        \exi{4\quad \hphantom{→}N:} \gll
        \hspace{4.8em}[m-belah][$_2$=nyo] \\
        \hspace{4.8em}\textsc{av}-cut=3\textsc{sg} \\
        \glt\hspace{4.8em}`(it's easy to) split it.' \\
        \exi{5\quad \hphantom{→}E:} \gll
        \hspace{8.5em}[$_2$samo] gawuh \\
        \hspace{8.5em}same just \\
        \glt\hspace{8.5em}`they're just the same.' \\
    \end{xlist}
    \hfill (BJM02-014-01, 00:37:25–00:37:29, Speakers: Nain, Edwar) 
\end{exe}

In the data considered here, cliticization on a placeholder is only encountered with the enclitic \textit{=nyo} `3\textsc{sg}' and primarily among nominals (seven occurrences as in (\ref{ex:place-nyo-nom})) rather than verbals (two occurrences as in (\ref{ex:place-nyo-verb})). This reflects the more common usage of \textit{=nyo} `3\textsc{sg}' as a possessive marker or marker of definiteness rather than as an indexed argument in transitive verb constructions. Baidi uses \textit{heni=nyo} `\textsc{dem.addr=3s}' in (\ref{ex:place-nyo-nom}) when discussing the necessity of clipping the wings of domesticated ducks. The placeholder \textit{heni} `\textsc{dem.addr}' holds the place for \textit{kepi} `wing' while Baidi thinks of the right word, \textit{=nyo} here referring to the ducks. 

\begin{exe}
    \ex\label{ex:place-nyo-nom} \begin{xlist}[0\quad →A:]
        \exi{1\quad \hphantom{→}B:} \gll
        amun kak se=khatus, \\
        if already one=hundred \\
        \glt `if (you've) already (done) one hundred (of them),' \\
        \exi{2\quad \hphantom{→B:}} \gll
        kak n-(t)anggung, \\
        already \textsc{av}-tiring \\
        \glt `it's already tiring,' \\
        \exi{3\quad →\hphantom{B:}} \gll
        n-(t)etuk-i \textbf{heni=nyo}, \\
        \textsc{av}-cut-\textsc{loc/dist} \textsc{dem.addr}=3\textsc{sg} \\
        \glt `cutting off \textbf{their whatchamacallits},' \\
        \exi{4\quad \hphantom{→B:}} \gll
        \uline{kepi=nyo} \\
        wing=3\textsc{sg} \\
        \glt `\uline{their wings}.' \\
    \end{xlist}
    \hfill (BJM02-014-01, 01:02:09–01:02:11, Speaker: Baidi) 
\end{exe}

In (\ref{ex:place-nyo-verb}), Gadis is joking about Resta looking sleepy on the recording. She uses \textit{ng-udi-kun} to hold the place for a verb referring to watching and \textit{=nyo} to refer to the recording itself.

\begin{exe}
    \ex\label{ex:place-nyo-verb} \begin{xlist}[0\quad →A:]
        \exi{1\quad \hphantom{→}G:} \gll
        betik amun dengi \\
        nice if night \\
        \glt `it's nice at night.' \\
        \exi{2\quad \hphantom{→G:}} \gll
        ai, \\
        ah \\
        \glt `ah,' \\
        \exi{3\quad \hphantom{→G:}} \gll
        khaso=ku kepayaha:n yang se=bigi sudi la=nyo, \\
        feel=1\textsc{sg} tired \textsc{rel} one=\textsc{class} \textsc{dem.dist} say=3\textsc{sg} \\
        \glt `I feel like he's going to say, that one's tired,' \\
        \exi{4\quad →\hphantom{G:}} \gll
        keliyakan .. nti tian \textbf{ng-udi-kun=nyo} \\
        seem {} later 3\textsc{pl} \textsc{av}-\textsc{dem.dist}-\textsc{caus/ben}=3\textsc{sg} \\
        \glt `looking .. later when they \textbf{whatchamacallit} (=\uline{watch the video recording}).' \\
    \end{xlist}
    \hfill (BJM02-050-03, 00:03:03–00:03:08, Speaker: Gadis) 
\end{exe}

The following subsections describe how these placeholders operate both in repair sequences and the syntactic positions they occupy (i.e. the syntactic projections they fulfill).

\subsection{Repair}\label{sec:Placeholders.Repair}

As seen in \tabref{tab:billings:place-repair}, more than a third of all cases of placeholder fillers are not repaired. There appears to be no clear correlation between repair and the form of the placeholder that would not be otherwise attributable to general distribution of the fillers. We first consider repaired placeholders.

\begin{table}
    \caption{Placeholder frequencies of Nasal fillers by repair.}
    \label{tab:billings:place-repair}
    \begin{tabularx}{.8\textwidth}{lYY}
        \lsptoprule
 %       & \multicolumn{2}{c}{Repair} \\ \cline{2-3}
        & {Repaired} & {Unrepaired} \\
        \midrule
        \textsc{dem.sp} & 1 & - \\
        \textsc{dem.addr} & 47 & 30 \\
        \textsc{dem.dist} & 14 & 4 \\
        Interrogative & 8 & 2 \\\midrule
        Total: 106 = & 70 & 36 \\
        \lspbottomrule
    \end{tabularx}
\end{table}

\subsubsection{Repaired placeholders}
As has been noted cross-linguistically (\cite{nemeth2012repair}), the primary repair strategy in Nasal is self-initiated self-repair, but both other-initiated and other-repair are attested in our corpus, as in (\ref{ex:repair-other}) and (\ref{ex:place-repair-cooperative}) below. When Ahmad cannot remember the word he used for `gang' in his retelling of a Nasal story in (\ref{ex:repair-other}), Johan helps repair the filler \textit{api} in line 4.

\begin{exe}
    \ex\label{ex:repair-other} \begin{xlist}[0\quad →A:]
        \exi{1\quad \hphantom{→}A:} \gll
        habis--- \\
        gone \\
        \glt `gone---' \\
        \exi{2\quad \hphantom{→A:}} \gll
        a, \\
        ah \\
        \glt `uh,' \\
        \exi{3\quad \hphantom{→A:}} \gll
        belo=do yang, \\
        gone=\textsc{emph} \textsc{rel} \\
        \glt `they were gone,' \\
        \exi{4\quad →\hphantom{A:}} \gll
        \textbf{api} so nu, \\
        what this earlier \\
        \glt `these \textbf{whatchamacallit} I just mentioned,' \\
        \exi{5\quad \hphantom{→}J:} \gll
        \uline{gekhumbulan} so nu \\
        gang this earlier \\
        \glt `\uline{the gang}.' \\
        \exi{6\quad \hphantom{→}A:} \gll
        gekhumbulan \\
        gang \\
        \glt `the gang.' \\
    \end{xlist}
    \hfill (BJM02-058-01, 00:07:21–00:07:25, Speakers: Ahmad, Johan) 
\end{exe}

Same-turn self-repair in Nasal typically occurs in the immediately following IU. Neither the choice of filler nor the syntactic position of the placeholder appears to have any correlation with what material, if any, is recycled during repair. Word searches are not altogether infrequent but rarely involve participation from other participants. One of the few cases of this can be seen in (\ref{ex:place-repair-cooperative}) where Baidi assists Edwar in trying to recall the name for a specific kind of fish.

\begin{exe}
    \ex\label{ex:place-repair-cooperative} \begin{xlist}[0\quad →A:]
        \exi{1\quad →E:} \gll
        nju:k \textbf{ani} [api?] \\
        like \textsc{dem.addr} what \\
        \glt `(it's) like \textbf{whatchamacallit}, what?' \\
        \exi{2\quad \hphantom{→}B:} \gll
        \hspace{6.3em}[iwo bela]to? \\
        \hspace{6.3em}\hphantom{[}fish b. \\
        \glt \hspace{6.3em}`\textit{belato} fish?' \\
        \exi{3\quad \hphantom{→}E:} \gll
        ayin=nyo iwo belato \\
        \textsc{neg}=3\textsc{sg} fish {kind of fish} \\
        \glt `it's not \textit{belato} fish.' \\
        \exi{4\quad \hphantom{→E:}} \gll
        anak-anak tungkul \\
        child-\textsc{rdp} mackerel \\
        \glt `young mackerel.' \\
        \exi{5\quad \hphantom{→E:}} \gll
        njuk anak tungkul ni \\
        like child t. \textsc{dem.sp} \\
        \glt `it's like young mackerel.' \\
        \exi{6\quad \hphantom{→E:}} \gll
        tapi [\%gemuk] \\
        but \hphantom{[\%}fat\\
        \glt `but fat.' \\
        \exi{7\quad \hphantom{→}B:} \gll
        \hspace{1.7em}[o \uline{iwo dinci][$_2$s}] \\
        \hspace{1.7em}oh {fish sardine} \\
        \glt \hspace{1.7em}`oh, \uline{sardine}.' \\
        \exi{8\quad \hphantom{→}E:} \gll
        \hspace{6.2em}[$_2$o]o, \\
        \hspace{6.2em}yeah \\
        \glt \hspace{6.2em}`yeah,'
        \exi{9\quad \hphantom{→E:}} \gll
        mungkin=do ani=do \\
        maybe=\textsc{emph} \textsc{dem.addr}=\textsc{emph} \\
        \glt `maybe that's it.' \\
    \end{xlist}
    \hfill (BJM02-014-01, 00:27:49–00:27:59, Speakers: Edwar, Baidi) 
\end{exe}

The majority of word searches, however, involve only the speaker who utilized the placeholder, as in (\ref{ex:place-repair-wordsearch}) and (\ref{ex:place-repair-wordsearch2}). Yogi searches for the right word for applying poison to protect his plants in (\ref{ex:place-repair-wordsearch}), and Johan searches for the right word for a type of banana whose seedlings somebody had recently picked for planting in (\ref{ex:place-repair-wordsearch2}).

\begin{exe}
    \ex\label{ex:place-repair-wordsearch} \begin{xlist}[0\quad →A:]
        \exi{1\quad \hphantom{→}J:} \gll
        (\%) adak beli-kun tiodan \\
        {} or \textsc{pv.}buy-\textsc{caus/ben} poison \\
        \glt `or buy poison.' \\
        \exi{2\quad \hphantom{→}Y:} \gll
        oo \\
        yeah \\
        \glt `yeah.' \\
        \exi{3\quad \hphantom{→}J:} \gll
        [sempekhut tiodan] \\
        spray poison \\
        \glt `(you can) spray the poison.' \\
        \exi{4\quad →Y:} \gll
        [\textbf{ani-kun} nih]an \\
        \textsc{pv.}\textsc{dem.addr}-\textsc{caus/ben} very \\
        \glt `(you can) \textbf{whatchamacallit}.' \\ 
        \exi{5\quad \hphantom{→Y:}} \gll
        pik-kun, \\
        \textsc{pv.}put-\textsc{caus/ben} \\
        \glt `put it,' \\
        \exi{6\quad \hphantom{→Y:}} \gll
        \uline{suntik-kun} \\
        \textsc{pv.}inject-\textsc{caus/ben} \\
        \glt `\uline{inject it}.' \\
    \end{xlist}
    \hfill (BJM02-022-01, 00:25:48–00:25:54, Speakers: Johan, Yogi) 
\end{exe}

\begin{exe}
    \ex\label{ex:place-repair-wordsearch2} \begin{xlist}[0\quad →A:]
        \exi{1\quad \hphantom{→}J:} \gll
        dijo so kak .. akuk-i Tris anak=nyo \\
        here \textsc{dem.sp} already {} \textsc{pv}.take-\textsc{loc/dist} T. child=3\textsc{sg} \\
        \glt `here Tris's mom already took their seedlings (lit. their children).' \\
        \exi{2\quad \hphantom{→J:}} \gll
        Ambon \\
        {kind of banana} \\
        \glt `Ambon (bananas).'
        \exi{3\quad →\hphantom{J:}} \gll
        khan \textbf{udi}, \\
        and \textsc{dem.dist} \\
        \glt `and \textbf{whatchamacallit},' \\
        \exi{4\quad \hphantom{→J:}} \gll
        api gelakh=nyo:? \\
        what name=3\textsc{sg} \\
        \glt `what's it called?' \\
        \exi{5\quad \hphantom{→J:}} \gll
        ajo:, \\
        \textsc{dem.sp} \\
        \glt `whatchamacallit,' \\
        \exi{6\quad \hphantom{→J:}} \gll
        {pisang huwai:}, \\
        {kind of banana} \\
        \glt `\textit{huwai} bananas,' \\
        \exi{7\quad \hphantom{→J:}} \gll
        \uline{jantan} \\
        {kind of banana} \\
        \glt `\uline{\textit{jantan} bananas}.' \\
    \end{xlist}
    \hfill (BJM02-022-01, 00:23:17–00:23:24, Speaker: Johan) 
\end{exe}

If a clitic is used in conjunction with a placeholder, it is more often found recycled also in the repair, whether in reference to a verbal or a nominal, as in (\ref{ex:place-nyo-nom}) above, but, again, this is not obligatory. Similarly, if a verb is repaired, typically both the voice and transitivity affixes of the placeholder will be recycled, as in (\ref{ex:place-repair-wordsearch}). However, given the hesitative nature of placeholders, this, too, is not always the case, as discussed further below with (\ref{ex:place-no-kun-obj-av}) and (\ref{ex:place-no-kun-obj-pv}).

\subsubsection{Unrepaired placeholders}
In more than a third of the coded placeholders, no explicit repair is made and conversation continues without any apparent issue, as in (\ref{ex:place-unrepair}) where Edwar's placeholder use of the second \textit{ani} in line 2 is understood by Baidi and Nain from its contrast with the referential \textit{ani} (referring to a previously mentioned `dragnet') immediately preceding it in line 1. 

\begin{exe}
    \ex\label{ex:place-unrepair} \begin{xlist}[0\quad →A:]
        \exi{1\quad \hphantom{→}E:} \gll
        man mak hago ani, \\
        if \textsc{neg} want \textsc{dem.addr} \\
        \glt `if you don't want that one (=dragnet),' \\
        \exi{2\quad →\hphantom{E:}} \gll
        m-beli=do \textbf{ani}, \\
        \textsc{av}-buy=\textsc{emph} \textsc{dem.addr} \\
        \glt `buy a \textbf{whatchamacallit} (=\uline{different kind of net}),' \\
        \exi{3\quad \hphantom{→E:}} \gll
        api gelakh=nyo ni? \\
        what name=3\textsc{sg} \textsc{dem.sp} \\
        \glt `what's this one called?' \\
        \exi{4\quad \hphantom{→}B:} \gll
        genggung gawuh Manulah madah \\
        bring just M. let's \\
        \glt `let's just bring it to Manulah.' \\
        \exi{5\quad \hphantom{→}N:} \gll
        incang-kun gawuh jeujo \\
        \textsc{pv.}raise-\textsc{caus/ben} just like.that \\
        \glt `we can fish just like that.' \\
    \end{xlist}
    \hfill (BJM02-014-01, 00:17:04–00:17:11, Speakers: Edwar, Baidi, Nain) 
\end{exe}

While in some cases placeholders are used for taboo reasons and thus not repaired, as in both (\ref{ex:place-unrepair-taboo}) and (\ref{ex:place-unrepair-taboo2}) where \textit{ani} is used to refer to dying, the majority of cases where they are not repaired appear to simply be understood from context, as in (\ref{ex:place-affix-verb}) above.\footnote{In all three examples of avoidance in our coding, \textit{ani} is the filler used. However, this simply could be a result of frequency, and we suspect that other fillers are able to fulfill the same function.} The use of a placeholder for taboo reasons does not appear to have any marked differences with other placeholder uses that are not repaired.

\begin{exe}
    \ex\label{ex:place-unrepair-taboo} \begin{xlist}[0\quad →A:]
        \exi{1\quad →E:} \gll
        semenjak \textbf{ani} pekdang so, \\
        since \textsc{dem.addr} uncle \textsc{dem.sp} \\
        \glt `since my uncle \textbf{whatchamacallit} (=\uline{passed away}),' \\
        \exi{2\quad \hphantom{→E:}} \gll
        mak pekhenah .. ngekhe- ny-(c)akak-kun henil-henilan agi \\
        \textsc{neg} ever {} {} \textsc{av}-raise-\textsc{caus/ben} rancid again \\
        \glt `I haven't brought any more rancid things (=fish) again.' \\
    \end{xlist}
    \hfill (BJM02-014-01, 00:28:30–00:28:34, Speaker: Edwar) 
\end{exe}

\begin{exe}
    \ex\label{ex:place-unrepair-taboo2} \begin{xlist}[0\quad →A:]
        \exi{1\quad →R:} \gll
        ilmu=nyo sudi kak bel:o masih jugo kung \textbf{ani} \\
        magic=3\textsc{sg} that already used.up still also not.yet \textsc{dem.addr} \\
        \glt `her magic is already gone, but she also still hasn't yet \textbf{whatchamacallit} (=\uline{passed away}).' \\
        \exi{2\quad \hphantom{→R:}} \gll
        kak pacul-kun \\
        already \textsc{pv}.free-\textsc{caus/ben} \\
        \glt `she's already been freed.' \\
    \end{xlist}
    \hfill (BJM02-050-01, 00:20:17–00:20:21, Speaker: Resta) 
\end{exe}

\subsection{Syntactic position}

Of the 106 instances of placeholders, more than two thirds occur in the syntactic position (i.e. fulfill the syntactic projection) of nominals, either as arguments, objects of prepositions, or nominal modifiers (e.g. possessors). This is consistent with the cross-linguistic tendency for placeholders to occur in nominal positions \citep{podlesskaya2010parameters}. Although, as discussed previously, demonstrative and interrogative forms do not necessarily require marking for particular word classes, it is likely their origin in referential and pronominal functions that reflects this greater trend to take the place of nominals. Nominal placeholders rarely take any morphology (as in, for example, (\ref{ex:place-affix-nom}) above), and such morphology is by no means obligatory.

\begin{table}
    \caption{Placeholder frequencies of Nasal fillers by syntactic position.}
    \label{tab:billings:place-rep-rep}
    \begin{tabularx}{.8\textwidth}{lYY}
        \lsptoprule
        & {Nominal} & {Verbal} \\\midrule
        \textsc{dem.sp} & 1 & - \\
        \textsc{dem.addr} & 55 & 22 \\
        \textsc{dem dist} & 14 & 4 \\
        Interrogative & 6 & 4 \\\midrule
        Total: 106 =  & 76 & 30 \\
        \lspbottomrule
    \end{tabularx}
\end{table}

When a placeholder fulfills the syntactic projection of an intransitive verb, explicit morphology is equally likely as being morphologically unmarked (4/8). It does, however, occur in virtually every instance of a placeholder fulfilling the projection of a transitive verb (11/12), as in (\ref{ex:place-kun-pv}).\footnote{Placeholders fulfill intransitive and transitive projections 20 times in the corpus while placeholders occur in predicate positions a total of 30 times. The 10 remaining instances are those that were not repaired and cannot be clearly coded as fulfilling the syntactic projection of an intransitive or transitive verb.} In this sequence, Edwar asks about what was done with the fish he had seen earlier, using PV- and applicative-marked  \textit{di-ani-kun} `\textsc{pv}-\textsc{dem.addr}-\textsc{caus/ben}' to fulfill the syntactic projection of a transitive verb describing `putting them in the freezer'. In such a case, the valency-increasing suffix \textit{-kun} allows for the otherwise intransitive placeholder to take a Secondary Argument.

\begin{exe}
    \ex\label{ex:place-kun-pv} \begin{xlist}[0\quad →A:]
        \exi{1\quad \hphantom{→}E:} \gll
        dipo iwo nu? \\
        where fish earlier \\
        \glt `where's the fish from earlier?' \\
        \exi{2\quad →\hphantom{E:}} \gll
        kan \textbf{di:-ani-kun} tian di kulkas kan? \\
        \textsc{tag} \textsc{pv}-\textsc{dem.addr}-\textsc{caus/ben} 3\textsc{pl} in freezer \textsc{tag} \\
        \glt `right, they already \textbf{whatchamacallit-ed} (=\uline{put}) it in the freezer, right?' \\
    \end{xlist}
    \hfill (BJM02-014-01, 00:18:35–00:18:39, Speaker: Edwar) 
\end{exe}

 If a Secondary Argument is implicit or only stated in the repair, the placeholder need not have the applicative, as in (\ref{ex:place-no-kun-obj-av}) for AV and (\ref{ex:place-no-kun-obj-pv}) for PV where the repair contains a P argument which is unrealized in the clause with the placeholder.\footnote{Although voice changes in the repaired clause, the transitivity does not -- the placeholder is occupying the syntactic position of a verb indicating `pulling a net', an action which certainly requires two arguments, as implied by the following PV transitive verb \textit{takhik} `pull'.} In (\ref{ex:place-no-kun-obj-av}), Edwar is describing how he would use a kind of net, using the AV-marked placeholder \textit{ng-ani} `\textsc{av-dem.addr}' to fill the syntactic position of a verb meaning `pulling it in the shallows'. In (\ref{ex:place-no-kun-obj-pv}), Gadis uses the distal demonstrative \textit{udi} as a placeholder while she determines how to describe what Ros's child is doing. Both cases are clearly verbal placeholders, the former because of the verbal marking and the latter because of the immediately preceding aspectual marker \textit{masih} `still'.

\begin{exe}
    \ex\label{ex:place-no-kun-obj-av} \begin{xlist}[0\quad →A:]
        \exi{1\quad \hphantom{→}E:} \gll
        amo kito pandai: ng-uyun ni agi, \\
        if 1\textsc{pl.incl} able \textsc{av}-control \textsc{dem.sp} again \\
        \glt `if we can control it again,' \\
        \exi{2\quad →\hphantom{E:}} \gll
        nyak \textbf{ng-ani}, \\
        1\textsc{sg} \textsc{av}-\textsc{dem.addr} \\
        \glt `I'm going to \textbf{whatchamacallit},' \\
        \exi{3\quad \hphantom{→E:}} \gll
        \uline{takhik di khampusan} \\
        {\textsc{pv}.{pull} at shallows} \\
        \glt `\uline{pull (it) in the shallows}.' \\
    \end{xlist}
    \hfill (BJM02-014-01, 00:20:10–00:20:15, Speaker: Edwar) 
\end{exe}

\begin{exe}
    \ex\label{ex:place-no-kun-obj-pv} \begin{xlist}[0\quad →A:]
        \exi{1\quad \hphantom{→}G:} \gll
        kung \\
        not.yet \\
        \glt `not yet.' \\
        \exi{2\quad →\hphantom{G:}} \gll
        masih \textbf{u:di}, \\
        still \textsc{dem.dist} \\
        \glt `(he's) still \textbf{whatchamacallit},' \\
        \exi{3\quad \hphantom{→G:}} \gll
        masih \uline{jeujo-kun=nyo} \\
        still \textsc{pv}.like.that-\textsc{caus/ben}=3\textsc{sg} \\
        \glt `still \uline{do them like that} (=\uline{holding his hands up}).' \\
        \exi{4\quad \hphantom{→G:}} \gll
        kung ber-arti pedum na \\
        not.yet \textsc{mid}-meaning sleep well \\
        \glt `that means that he hasn't fallen asleep yet.' \\
    \end{xlist}
    \hfill (BJM02-050-03, 00:04:35–00:04:38, Speaker: Gadis) 
\end{exe}

%%%
%%%
%%%
%%% 
%%% SECTION: Discussion
%%%
%%%
%%%
%%%
\section{Fillers in the larger corpus}\label{sec:Fillers in the larger corpus}
A cursory look at our larger corpus of 24 transcribed conversations ($\sim$30 hours of speech) largely reflects what we have seen above. A superficial analysis of this set provides a broad overview of the range and various functions of each of the function words without respect to their polyfunctional uses, whether hesitators, placeholders, or within their typical demonstrative/interrogative contexts.

Frequency counts for each function word with various affixes are shown in \tabref{tab:billings:filler-morph}. The nasal prefix is excluded from the discussion here since a large portion of forms affixed with \textit{N-} must be attributed to polysemy/homophony – \textit{ng-ajo} `\textsc{av-dem.sp}', \textit{ng-ani} `\textsc{av-dem.sp}', and \textit{ng-udi} `\textsc{av-dem.sp}' with the emphatic demonstratives and \textit{ng-api} `\textsc{av-}what' with the interrogative \textit{ngapi} `why'. When not used in their bare form, Nasal placeholders are most frequently found occurring with verbal affixes. The most common (excluding the AV nasal prefix) is the applicative suffix \textit{-kun} `\textsc{caus/ben}'. Other typically verbal affixes such as \textit{di-} `\textsc{pv}', \textit{te-} `\textsc{nvol}', and \textit{be-} `\textsc{mid}' are also found on Nasal placeholders, although these are rare in our corpus. The same can also be said of the nominalizing affixes \textit{peN-}, \textit{ke-}, and \textit{-an}.

\begin{table}
    \caption{Morphological marking on Nasal placeholders in the larger corpus.}
    \label{tab:billings:filler-morph}
    \begin{tabular}{l r r r r  r r r}
        \lsptoprule
        & \multicolumn{4}{c}{Verbal} & \multicolumn{3}{c}{Nominal} \\ \cmidrule(lr){2-5} \cmidrule(lr){6-8}
        & {\textit{di-}} & {\textit{te-}} & {\textit{be-}} & {\textit{-kun}} & {\textit{peN-}} & {\textit{ke-}} & {\textit{-an}} \\
        & {\textsc{pv}} & {\textsc{pass}} & {\textsc{mid}} & {\textsc{appl}} & {\textsc{agent}} & {\textsc{nom}} & {\textsc{nom}} \\
        \midrule
        \textsc{dem.sp} & 1 & - & - & 1 & - & - & - \\
        \textsc{dem.addr} & 11 & 3 & 3 & 55 & 1 & 2 & 15 \\
        \textsc{dem.dist} & 13 & 2 & 4 & 57 & 1 & - & 10 \\
        Interrogative & 5 & 1 & 3 & 22 & - & - & 4 \\
        \lspbottomrule
    \end{tabular}
\end{table}

Frequency counts for the fillers occurring with each of the proclitics and enclitics in our corpus are shown in \tabref{tab:billings:filler-clitic}. The use of \textit{=nyo} `3\textsc{sg}' far exceeds that of the other clitics, unsurprisingly so given its polysemous nature as a possessive with nouns, A and P indexes in transitive verbs, the preference for third-person over first- and second-person referents in discourse, and the fact that \textit{=nyo} `3\textsc{sg}' in Nasal additionally operates as a marker of definiteness among other functions (see the discussion in \sectref{sec:BasicMorpho}). Because of this polysemy between verbal and nominal functions, \textit{=nyo} is marked as `Ambiguous' in the table.

\begin{table}
    \caption{Cliticization on Nasal placeholders in the larger corpus.}
    \label{tab:billings:filler-clitic}
    \begin{tabular}{l r r  r r  r}
        \lsptoprule
        & \multicolumn{2}{c}{Verbal} & \multicolumn{2}{c}{Nominal} & Ambiguous \\ \cmidrule(lr){2-3} \cmidrule(lr) {4-5} \cmidrule(lr){6-6}
        & {\textit{ku=}} & {\textit{mu=}} & {\textit{=ku}} & {\textit{=mu}} & {\textit{=nyo}} \\
        & {`\textsc{1sg}'} & {`\textsc{2sg}'} & {`\textsc{1sg}'} & {`\textsc{2sg}'} & {`\textsc{3sg}'} \\
        \midrule
        \textsc{dem.sp} & - & - & - & - & 5 \\
        \textsc{dem.addr} & 3 & - & 1 & - & 44 \\
        \textsc{dem.dist} & 1 & - & 1 & 1 & 57 \\
        Interrogative & - & - & 1 & 1 & 95 \\
        \lspbottomrule
    \end{tabular}
\end{table}

While all of the most common affixes and clitics used with placeholders have been discussed above, those not already discussed but shown in \tabref{tab:billings:filler-clitic} (e.g. \textit{be-} or \textit{peN-}) do not play a major role in the morphosyntax. Apart from clearly indicating placeholder rather than hesitator or demonstrative/interrogative uses of the function words discussed here, these affixes give little insight into the operation of fillers.

%%%
%%%
%%%
%%% 
%%% SECTION: Conclusion
%%%
%%%
%%%
%%%
\section{Some implications}\label{sec:implications}
The investigation of fillers in a corpus of six conversations has allowed us to see that, while all four function words investigated have the ability to take on hesitator and placeholder functions, the demonstrative and interrogative pronouns have complementary tendencies. Demonstrative fillers favor the placeholder functions, while the interrogative pronoun favors hesitator functions. \citet{hayashi2006crosslinguistic} have demonstrated how demonstratives have come to function as placeholders through their focusing/pointing function and as hesitators through a separate process of pragmaticization. Nasal provides additional support for their analysis of demonstrative fillers that serve as both placeholders and hesitators.

The person-oriented and tripartite division of demonstratives makes Nasal stand out against neighboring Malay varieties. Because Nasal lacks a dedicated filler (cf. Besemah \textit{anu}, \cite{chapters/mcdonnell_billings}), demonstratives play a larger functional role in dealing with word-formulation trouble. Furthermore, Malay varieties most commonly have a two-way proximal-distal division in demonstratives, and when employed as a filler, it is the proximal that is most frequent (see \cite{chapters/mcdonnell_billings}; \cite{wouk2005syntax}). However, in Nasal the near-addressee demonstrative is by far the most frequent filler. This is perhaps due to the fact that the pragmatic role of placeholders (the most common role of demonstrative fillers in Nasal) is to signal contextual knowledge purported to be shared with the addressee or to invite the addressee in repair and word-search functions (see the discussion of participant access in \cite{hayashi2006crosslinguistic}). % As opposed to the addressee-oriented demonstrative, the speaker-oriented demonstrative might indicate context only known by the speaker and the distal would indicate knowledge distant to both speaker and addressee.

The origin of the use of the interrogative pronoun as a placeholder and as a solitary hesitator is not as clear as with the demonstratives, but parallels can be found elsewhere both within the Austronesian family (see \cite[][]{blake2020westernsubanon} for Western Subanon, \cite[][]{tanangkingsing2022cebuano} for Cebuano, \cite{nagaya2022tagalog} for Tagalog) and outside the family (see \cite[][]{buskunbaeva2021bashkir} for Bashkir, \cite[][]{klyachko2022evenki} for Evenki). Given, however, that interrogatives, like Nasal \textit{api}, frequently participate in the development of indefinite pronouns (\cite{haspelmath_2001}), the referentiality of the indefinite pronoun (similar to that of a demonstrative as described above) could have given rise to its use in a placeholder function. It is possible that the placeholder uses of \textit{api} derived from this use as an indefinite pronoun and its hesitator use from self-addressed questions, but further study, both for Nasal and cross-linguistically, is needed to understand the development of fillers from interrogative pronouns.

Placeholders are designed to fulfill the function of the word that a speaker intends to say, and they often do so by mirroring not only the syntactic position but the morphology of the word that is delayed. \citet[6]{fox2010introduction} states this position clearly: ``In languages that mark their placeholder fillers with complex morphology, that morphology may be part of the adequacy of the promise for fulfillment.'' However, as \citet{podlesskaya2010parameters} shows in her typological survey of fillers, placeholders need not precisely mirror the item they stand in for. In fact, given the hesitator nature of placeholders, it should not surprise us that affixation and cliticization differ between a placeholder and its target. Furthermore, such `fulfillment' need not even take place. In opposition to \citeauthor{hayashi2010crosslinguistic}'s (\citeyear{hayashi2010crosslinguistic}: 42) remark that ``Frequently (but not invariably), a placeholder demonstrative is subsequently replaced'', repair in Nasal can scarcely be called frequent, occurring in less than two thirds of placeholder instances in our corpus. Despite this, Nasal speakers proceed without any apparent issue (see examples (\ref{ex:place-affix-verb}) and (\ref{ex:place-unrepair})–(\ref{ex:place-kun-pv})).

As is the case elsewhere in Austronesia, we have seen how the lack of explicit arguments and explicit marking based on word class in combination with the variable word orders leads to indeterminacy in analysis of fillers as placeholders or hesitators. However, that the division between typical interrogative and demonstrative pronoun functions and their filler functions is not as clear-cut as those discussed by \textcite{hayashi2010crosslinguistic} demonstrates both the great variety of fillers and the lack of depth of our knowledge about how fillers operate cross-linguistically. We suspect that with other languages of Austronesia (as, for example, is also the case with South Barisan Malay, \cite{chapters/mcdonnell_billings}) and beyond (as in Negidal, \cite{chapters/pakendorf}), such pronouns may frequently vary in their usage and clear divisions between uses of hesitator and placeholder functions may not exist.

\newpage
\section{Conclusion}\label{sec:Conclusion}

In this chapter, we described the various ways hesitator and placeholder fillers are utilized in everyday conversations in Nasal. We demonstrated how demonstratives and the interrogative \textit{api} `what' fulfill these functions. While there are other phrasal and non-lexical means of accomplishing the same two functions, these polyfunctional lexical items are frequently encountered in interaction as a means for dealing with trouble in word-formulation. 

\appendixsection{Coding schema for conversational corpus}
\label{app:coding}

\begin{xltabular}{1\textwidth}{p{\widthof{IU Position}} Q p{8em}}
  \lsptoprule   ID         & Description & Levels\\\midrule\endfirsthead
  \midrule      ID        & Description & Levels\\\midrule\endhead
  \endfoot\lspbottomrule\endlastfoot
  Form      & What is the full morphological form of the filler? & e.g. \textit{ani=nyo} \\
  Root      & What is the root of the filler? & e.g. \textit{ani}\\
  Function  & What is the function of the filler? & {placeholder, hesitator, indeterminate}\\
  IU Position  & What position within the Intonation Unit (IU) does the filler occupy? & initial, medial, final, sole\\
  Disfluency  & Are there any disfluencies immediately preceding or following the filler? & {truncation, sound stretch, pause, none}\\
  Syntactic position  & What position in the clause does the filler occupy? & predicate, argument, modifier, PP object, adjunct\\
  Repair  & Is the placeholder repaired? & {yes, no, NA}\\
  Recycling   & Is any part of the phrase recycled? & {yes, no, NA}\\
  Repair location (turn)  & How many turns after the placeholder is used is it repaired? & {0-5, NA}\\
  Repair location (IU)  & How many IUs after the placeholder is used is it repaired? & {0-9, NA}\\
  Repair initiator &  Who initiates repair of the placeholder? & {self, other, NA}\\
  Repairer &  Who repairs the placeholder? & {self, other, NA}\\
 \end{xltabular}

\section*{Abbreviations}
\begin{tabularx}{.5\textwidth}[t]{@{}lQ@{}}
\textsc{addr} & Near-addressee\\
\textsc{av} & A-Voice \\
\textsc{ben} & Benefactive \\
\textsc{caus} & Causative \\
\textsc{dem} & Demonstrative\\
\textsc{dist} & Distal\\
\textsc{emph} & Emphatic \\
\textsc{loc} & Locative \\
\end{tabularx}%
\begin{tabularx}{.5\textwidth}[t]{@{}lQ@{}}
\textsc{mid} & Middle \\
\textsc{neg} & Negative \\
\textsc{nvol} & Non-volitional \\
\textsc{pv} & P-Voice \\
\textsc{rdp} & Reduplication \\
\textsc{rel} & Relativizer\\
\textsc{sp} & Near-speaker\\
\end{tabularx}

\section*{Acknowledgements}
We would like to thank two anonymous reviewers and the editors for valuable feedback on this chapter. We would also like to acknowledge Johan Safri and Wawan Sahrozi, who have provided their expertise in the transcription, translation, and analysis of the Nasal conversations, as well as the Nasal people who have participated in the conversations in the corpus. We would like to thank our research counterpart in Indonesia, Yanti (Atma Jaya Catholic University of Indonesia), and the National Research and Innovation Agency (BRIN) for allowing us to conduct research on Nasal. This research is based upon work supported by the National Science Foundation under Grant BCS–1911641. Any opinions, findings, and conclusions or recommendations expressed in this material are those of the author(s) and do not necessarily reflect the views of the National Science Foundation.
%\section*{Contributions}
%John Doe contributed to conceptualization, methodology, and validation. 
%Jane Doe contributed to writing of the original draft, review, and editing.

\printbibliography[heading=subbibliography,notkeyword=this]
\end{document}
