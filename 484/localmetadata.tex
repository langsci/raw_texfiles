\title{Fillers}
\subtitle{Hesitatives and placeholders}
\BackBody{Fillers are non-silent linguistic devices used in disfluencies to gain time while searching for words. In addition, they are frequently used intentionally to avoid words for reasons of politeness, ‘conspirational’ motivations, or rhetorical purposes. Two syntactically distinct types of conventionalized fillers can be distinguished: placeholders and hesitatives (also called hesitators). Placeholders are referential and morphosyntactically integrated, while hesitatives are neither. Strikingly, even though fillers are cross-linguistically widespread, dedicated studies of such items in particular languages are still largely lacking.

This collective volume comprises in-depth descriptions of conventionalized fillers in a substantial variety of languages from Eurasia, Papunesia, Australia, and the Americas, hoping to stimulate typological research on fillers, both hesitatives and placeholders. The book aims to contribute to a better visibility of the topic among general linguists, to make data and analyses accessible that will be useful for further typological studies on the topic, and to provide models for descriptive linguists.

The introductory chapter discusses issues emerging from the previous literature and offers a new typology of fillers. It also highlights the major findings of the eleven remaining chapters. Each of these contains a detailed and typologically informed analysis of fillers in one or several underdescribed languages, based on corpora of natural speech and focusing on lexical fillers rather than on phenomena below the word-level (phonetic lengthening, truncation) or above the word-level (such as idioms and discourse markers like ‘you know’, or rhetorical questions like ‘what’s the word for that?’). The chapters cover a large amount of diversity, both in terms of languages and with respect to the type of filler. They focus on (i) the criteria for identification of the various types of fillers and the terminology used, keeping in mind that the domain is still largely under construction, (ii) a detailed analysis in terms of morphosyntactic distribution and, if possible, (iii) frequency in speech, and (iv) some reflection on the diachronic development of these disfluency markers.}

\author{Brigitte Pakendorf and Françoise Rose} 
 

\renewcommand{\lsISBNdigital}{978-3-96110-526-7}
\renewcommand{\lsISBNhardcover}{978-3-98554-146-1}
\BookDOI{10.5281/zenodo.15632051}
\typesetter{Sebastian Nordhoff}
\proofreader{Carmen Jany,
David Carrasco Coquillat,
Elisa Roma,
Eva Schultze Berndt,
Jeroen van de Weijer,
Killian Kiuttu,
Lachlan Mackenzie,
Maria Onoeva,
Nicoletta Romeo,
Jean Nitzke,
Rainer Schulze,
Mary Ann Walter,
Yvonne Treis
}
% \lsCoverTitleSizes{51.5pt}{17pt}% Font setting for the title page


\renewcommand{\lsSeries}{rcg}
\renewcommand{\lsSeriesNumber}{5}
\renewcommand{\lsID}{484}
