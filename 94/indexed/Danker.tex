% 4
\documentclass[output=paper]{LSP/langsci}
\author{Kathleen Danker}
\title{Ba-be-bi-bo-ra: {R}efinement of the {Ho-Chunk} syllabary in the nineteenth and twentieth centuries}
\shorttitlerunninghead{\mbox{Refinement of the Ho-Chunk syllabary in the 19\textsuperscript{th} and 20\textsuperscript{th} centuries}}

\abstract{In 1885, a man from the Nebraska Winnebago Reservation learned the Great Lakes syllabary writing system from the Sac and Fox in Iowa and began adapting it to Ho-Chunk, the only Siouan language to be written in this syllabary. During the first half of the twentieth century, the Ho-Chunk syllabary continued to be refined by writers who created new symbols necessary for the additional Ho-Chunk vowels and consonants and discarded unnecessary Sac and Fox characters. Increasing correspondence to the phonemic characteristics of the Ho-Chunk language can be seen by comparing the version of it used by its original adapter in Nebraska published in 1890, a text composed in 1938 by Sam Blowsnake\ia{Blowsnake, Sam} in Wisconsin, and one written in the 1970's by Felix White, Sr.\ia{White, Sr., Felix}, of Winnebago, Nebraska, who referred to the Ho-Chunk syllabary as the \emph{ba-be-bi-bo-ra}. 
% KEYWORDS: [Ho-Chunk, Winnebago, syllabary, orthography]
}
\ChapterDOI{10.17169/langsci.b94.123}

\maketitle

\begin{document}

\section{Introduction}
The members of the Ho-Chunk, or Winnebago, tribe, with reservations in Wisconsin and Nebraska, are the only speakers of a Siouan language to have developed a phonemic written language system\is{orthography}. Generally referred to as a \isi{syllabary}, this type of \isi{orthography} is more specifically termed an \emph{abugida} as defined by Peter T. Daniels in his \citeyear{Daniels1990} \isi{typology} of writing systems. Rather than employing separate symbols for all possible syllabic combinations of consonants and vowels in a language, as does a classic \isi{syllabary}, an abugida consists of a phoneme-specific consonant symbol followed by a secondary, also phoneme-specific, vowel symbol. In the case of the \ili{Ho-Chunk} system, the vowel /a/ was usually inherent or unmarked, but all other vowels were marked and connected to the preceding consonant in a syllable. In this article, I will use the more familiar term \isi{syllabary} for the \ili{Ho-Chunk} system. The Ho-Chunk people were introduced to this type of writing in the 1880's by members of the Sac and Fox tribe, with whom they had historical ties. 

Little is known about the origins of the Great Lakes Algonquian Syllabary developed by the \ili{Algonquian}-speaking Sac and Fox, Potawatomi, Kickapoo, and possibly Ottawa tribes in the late nineteenth century. First mentioned in print in 1880, its characters were based on cursive European handwriting, likely \ili{French} \citep[169]{Walker1996}. In 1884 or 1885, a man from the Nebraska Winnebago Reservation learned it while visiting the Sac and Fox in Iowa and adapted it to \ili{Ho-Chunk} \citep[354]{Fletcher1890b}. According to the ethnographer Alice Fletcher, by 1888 knowledge of the \isi{syllabary} writing system had ``spread rapidly among the Winnebagoes of Nebraska, and also to that part of the tribe living in Wisconsin, so that at the present time the principal correspondence of the tribe takes place by means of these characters'' \citeyearpar[299]{Fletcher1890a}. Considerable adaptation was necessary to convert the \ili{Sac and Fox} system to \ili{Ho-Chunk}. \il{Sauk}Sac (Sauk) and \il{Mesquakie}Fox (Mesquakie) have four vowels, all oral, while \ili{Ho-Chunk} has five oral vowels and three nasalized\is{nasalization} ones. There are nearly twice as many consonants in \ili{Ho-Chunk} as in \ili{Sac and Fox}. The \il{Sauk}Sac, however, do have one consonant, /θ/, not found in \ili{Ho-Chunk}, and the \il{Mesquakie}Fox included a single character in their \isi{syllabary} for a consonant \isi{cluster}, /kw/ \citep{Jones1906}. (See Tables \ref{hochunkphonemes}, \ref{mesquakiesaukvowels} and \ref{mesquakiesaukconsonants} below.) Examples of the \ili{Ho-Chunk} \isi{syllabary} from the nineteenth and twentieth centuries show that it took some time for its users to discard unnecessary \il{Sauk}Sac and \il{Mesquakie}Fox characters and to develop the new ones required for their Siouan language. 

Initial enthusiasm for the \isi{syllabary} may have declined among the Ho-Chunk by 1912 when anthropologist Paul Radin reported that it ``was known to very few Indians and was used only for the writing of letters'' (\citealt[21]{Radin1954}; cited in \citealt[161]{Walker1981}). However, this writing of letters between Nebraska and Wisconsin did continue until correspondence in \ili{Ho-Chunk} became outmoded because of general \ili{English} literacy in the second half of the twentieth century. Up to that time, tribal members in Nebraska and Wisconsin worked on altering the \isi{orthography} and the syllabification conventions of their writing system to better represent the \ili{Ho-Chunk} language. Because this took place primarily within families, the \isi{syllabary} inevitably developed into somewhat different versions. Nonetheless, an overall evolution of the \ili{Ho-Chunk} \isi{syllabary} can be seen by comparing and contrasting a version of it published by Alice Fletcher in 1890 (Tables \ref{fletchercv} and \ref{fletchercvc} below); a \isi{syllabary} text composed in 1938 by Sam Blowsnake\ia{Blowsnake, Sam} in Wisconsin that was included in an unpublished manuscript written by Amelia Susman in \citeyear{Susman1939} (\tabref{blowsnakesyllabary} below); and a fragment of a text written by Felix White, Sr.\ia{White, Sr., Felix}, of Winnebago, Nebraska, in the late 1970's when he taught me his version of the \isi{syllabary}, which he called the \emph{ba-be-bi-bo-ra} (\tabref{whitesyllabary} below). The published 1890 version of the \ili{Ho-Chunk} \isi{syllabary} contained additions to its \ili{Sac and Fox} model as well as unnecessary retentions of some of its features. The texts of Blowsnake and White in the twentieth century conformed more closely to the phonemic characteristics of the \ili{Ho-Chunk} language. The few ways in which these later texts continued to differ from these characteristics are of particular interest because of what they may reveal about how \ili{Ho-Chunk} speakers conceptualize the phonemes of their language.

\section{The Fletcher publications}

Alice Fletcher\ia{Fletcher, Alice~C.} published the first account of the \ili{Ho-Chunk} \isi{syllabary}, actually two slightly different accounts, both in 1890, one in the \emph{Proceedings of the American Association for the Advancement of Science} and the other in the \emph{Journal of American Folklore}. By chance, Fletcher had been on the Nebraska Winnebago reservation to witness the earliest inception of the writing system. In May 1883, she had been moved there from the adjacent Omaha Reservation to recuperate from a serious illness incurred while she was apportioning Omaha land allotments \citep[90]{Mark1988}. Fletcher lay bedridden at the Winnebago Agency in the winter of 1883--1884 when some fifteen to twenty members of the Sac and Fox tribe living in Iowa came to visit the reservation. The Fox and the Ho-Chunk had been neighbors in Wisconsin since at least 1600 as on-again, off-again allies and enemies -- although primarily allies after 1737, when the Ho-Chunk helped persuade the French not to exterminate the Fox at the end of the Fox Wars \citep[75]{Bieder1995}. Fletcher wrote that the travelers from Iowa ``were in full Indian costume and bent on enjoying old-time pleasures. I met these visitors several times; they came not infrequently to see me when I lay upon my sick bed'' \citeyearpar[354]{Fletcher1890b}. They told her that someone in the tribe had recently invented a writing system for their language, but none of them then visiting in Nebraska had learned it \citeyearpar[299]{Fletcher1890a}.

In 1884 or 1885, a group of Ho-Chunk paid a return visit to the Sac and Fox, and one of them studied the \ili{Sac and Fox} writing system. Once back at the Nebraska Winnebago reservation, he began to adapt the \ili{Sac and Fox} \isi{orthography} to the \ili{Ho-Chunk} language and to teach it to others. According to Fletcher, he ``became before long quite expert in its use, to his own amusement and that of his friends'' \citeyearpar[299]{Fletcher1890a}. Fletcher received a letter from the Winnebago agent in August 1885 stating:

\begin{quote}
The tribe have suddenly taken to writing their own language, and people who have never learned \ili{English} have acquired this art. The people claim they took the basis of it from the Sauk and elaborated it themselves. It is a very suggestive sight to see half a dozen fellows in a group with their heads together working out a letter in these new characters; it illustrates the surprising facility with which they acquire what they want to learn. \citeyearpar[299]{Fletcher1890a}
\end{quote}

In 1887--1888, Fletcher returned to the Winnebago Reservation, having been hired to oversee allotment there as she had done on the Omaha Reservation \citep[162]{Mark1988}. The Ho-Chunk man who had brought the \ili{Sac and Fox} \isi{syllabary} back to Nebraska arranged a table of his \ili{Ho-Chunk} \isi{syllabary} characters for Fletcher, which she then published in her 1890 articles about the writing system. Unfortunately, she did not report her informant's name in these articles \citep[300]{Fletcher1890a}.

\section{The phonemes of the {Ho-Chunk} and Sac and Fox languages}\il{Sauk}\il{Mesquakie}\largerpage


\tabref{hochunkphonemes} shows the vowels and consonants of the \ili{Ho-Chunk} language as organized by Amelia Susman in her dissertation on the \ili{Ho-Chunk} accentual system \citeyearpar[15]{Susman1943}. Her listing does not distinguish between long and short vowels or glottalized and unglottalized consonants as do more recent descriptions of the language such as that in \emph{Hoc\k{a}k Teaching Materials, Volume I} by Johannes Helmbrecht and Christian Lehmann \citeyearpar[5--7]{HelmbrechtLehmann2010}. However, Susman's arrangement is particularly useful for this article because of the way it divides Category I consonant stops, affricatives, and spirants into two columns: voiced (sonant) in column A and voiceless (surd) in column B. 

Tables \ref{mesquakiesaukvowels} and \ref{mesquakiesaukconsonants} show the vowels and consonants found in the closely related Mes\-qua\-kie (Fox) and \ili{Sauk} (Sac) languages (\ili{Mesquakie}-\ili{Sauk}). To the right of each vowel in \tabref{mesquakiesaukvowels}, and of each consonant in \tabref{mesquakiesaukconsonants}, I have added the corresponding \il{Mesquakie}Fox \isi{syllabary} characters as described by William Jones in \citeyear{Jones1906}. Length\is{vowel length} is phonemic in both \ili{Ho-Chunk} and \il{Sauk}Sac and \il{Mesquakie}Fox vowels, something indicated in \tabref{mesquakiesaukvowels}, but not marked in either the \ili{Ho-Chunk} or the \ili{Sac and Fox} syllabaries. Jones did not record a \isi{syllabary} symbol for the consonant /h/ found in \il{Sauk}Sac and \il{Mesquakie}Fox as well as in \ili{Ho-Chunk}, nor one for the consonant /θ/ found only in \il{Sauk}Sac. As can be seen by comparing Tables \ref{hochunkphonemes}, \ref{mesquakiesaukvowels} and \ref{mesquakiesaukconsonants}, \ili{Ho-Chunk} has the four oral vowels /a/, /e/, /i/ and /o/ used in \il{Sauk}Sac and \il{Mesquakie}Fox as well as a fifth \ili{Ho-Chunk} oral vowel /u/. There are also nasalized\is{nasalization} /\k{a}/, /\k{i}/ and /\k{u}/ vowels in \ili{Ho-Chunk}. \ili{Ho-Chunk} shares the consonants /č/, /h/, /k/, /m/, /n/, /p/, /s/, /\v{s}/, /t/, /w/ and /y/ with this \ili{Algonquian} language, but in addition has the consonants /b/, /g/, /\v{j}/, /r/, /x/, /ɣ/, /z/, /\v{z}/ and a \isi{glottal} stop.\largerpage

\begin{table}[H]
\begin{tabularx}{\textwidth}{lXXXlXl}
\lsptoprule
\multicolumn{7}{l}{\ili{Ho-Chunk} Vowels}\\
\midrule
Oral & a & i & u & e && o\\
Nasal & ą & į & ų\\
\\
\multicolumn{7}{l}{\ili{Ho-Chunk} Consonants}\\
& (Voiced) & (Voiceless)\\
I & A & B & II & & III & \\
\midrule
stops & b & p & nasals & m & stops & t\\
& g & k && n && ' \\
af\isi{fricative} & ǰ & č & trill & r & breath & h\\
spirants & z & s & semi-vowels & w &&\\
& ž & š && y &&\\
& ɣ & x &&&&\\
\lspbottomrule
\end{tabularx}
\caption{\ili{Ho-Chunk} phonemes \citep[15]{Susman1943}.}
\label{hochunkphonemes}
\end{table}


%\begin{table}
%\begin{tabular}{lllllll}
%\lsptoprule
%\multicolumn{3}{c}{Vowels} && \multicolumn{3}{c}{Consonants}\\
%\midrule
%Alphabetic & Syllabic & Americanist && Alphabetic & Syllabic & Americanist\\
%\cline{1-3} \cline{5-7}
%a & \multirow{2}{*}{\includegraphics{figures/Danker2a}} & a, ʌ && ch & \includegraphics{figures/Danker2ch} & č\\
%â && aˑ && h && h\\
%e & \multirow{2}{*}{\includegraphics{figures/Danker2e}} & e, ɛ && k & \includegraphics{figures/Danker2k} & k\\
%ê && eˑ, æˑ && m & \includegraphics{figures/Danker2m} & m\\
%i & \multirow{2}{*}{\includegraphics{figures/Danker2i}} & i, ɪ && n & \includegraphics{figures/Danker2n} & n\\
%î && iˑ && p & \includegraphics{figures/Danker2p} & p\\
%o & \multirow{2}{*}{\includegraphics{figures/Danker2o}} & o, ʊ && s & \includegraphics{figures/Danker2s} & s (only in Mesquakie)\\
%ô && oˑ && sh & \includegraphics{figures/Danker2sh} & š\\
%&&&& t & \includegraphics{figures/Danker2t} & t\\
%&&&& th & \includegraphics{figures/Danker2th} & θ (only in \ili{Sauk})\\
%&&&& w & \includegraphics{figures/Danker2w} & w\\
%&&&& y & \includegraphics{figures/Danker2y} & y\\
%&&&& kw & \includegraphics{figures/Danker2kw} & kw\\
%\lspbottomrule
%\end{tabular}
%\caption{Mesquakie and \ili{Sauk} phonemes (adapted from \citealt{NatLangMeskSauk}; \citealt{Susman1939}; and \citealt{Jones1906}).}
%\label{mesquakiesaukphonemes}
%\end{table}

\begin{table}[H]
\begin{tabular}{lll}
\lsptoprule
Alphabetic & Syllabic & Americanist\\
\midrule
a & \multirow{2}{*}{\includegraphics{figures/Danker2a}} & a, ʌ\\
â && aˑ\\ \\
e & \multirow{2}{*}{\includegraphics{figures/Danker2e}} & e, ɛ\\
ê && eˑ, æˑ\\ \\
i & \multirow{2}{*}{\includegraphics{figures/Danker2i}} & i, ɪ\\
î && iˑ\\
o & \multirow{2}{*}{\includegraphics{figures/Danker2o}} & o, ʊ\\
ô && oˑ\\
\lspbottomrule
\end{tabular}
\caption{\ili{Mesquakie} and \ili{Sauk} vowel phonemes (adapted from \citealt{NatLangMeskSauk}; \citealt{Susman1939}; \citealt{Jones1906}).}
\label{mesquakiesaukvowels}
\end{table}



\begin{table}[H]
\begin{tabular}{lll}
\lsptoprule
Alphabetic & Syllabic & Americanist\\
\midrule
ch & \includegraphics{figures/Danker2ch} & č\\
h && h\\
k & \includegraphics{figures/Danker2k} & k\\
m & \includegraphics{figures/Danker2m} & m\\
n & \includegraphics{figures/Danker2n} & n\\
p & \includegraphics{figures/Danker2p} & p\\
s & \includegraphics{figures/Danker2s} & s (only in \ili{Mesquakie})\\
sh & \includegraphics{figures/Danker2sh} & š\\
t & \includegraphics{figures/Danker2t} & t\\
th & \includegraphics{figures/Danker2th} & θ (only in \ili{Sauk})\\
w & \includegraphics{figures/Danker2w} & w\\
y & \includegraphics{figures/Danker2y} & y\\
kw & \includegraphics{figures/Danker2kw} & kw\\
\lspbottomrule
\end{tabular}
\caption{\ili{Mesquakie} and \ili{Sauk} consonant phonemes (adapted from \citealt{NatLangMeskSauk}; \citealt{Susman1939}; \citealt{Jones1906}).}
\label{mesquakiesaukconsonants}
\end{table}

				
\section{The Fletcher syllabary}

\begin{table}
\begin{tabular}{lllllll}
\lsptoprule
Syll. = Engl. && Syll. = Engl. && Syll. = Engl. && Syll. = Engl.\\
\midrule
Ka = gah && Ke = gay && Ki = gee && Ko = go\\
da = jah && de = jay && di = g && do = jo\\
wa = wä && we = w\underline{e} && wi = wï && wo = wo\\
xa = xä && xe = x\underline{e} && xi = xï && xo = xo\\
ta = tdä && te = tde && ti = tdï && to = tdo\\
ma = mä && me = me && mi = mï && mo = mo\\
na = nä && ne = ne && ni = nï && no = no\\
La = Rä && Le = Ray && Li = Ree && Lo = Row\\
ga = gwar && ge = Gway && gi = gwee && go = gwo\\
ra = Sah && re = say && ri = see && ro = So\\
Tha = Thä && The = They && Thi = The && Tho = Tho\\
Ya = yä && Ye = yea && Yi = Ye && Yo = Yo\\
ba = pah && be = pay && bi = pee && bo = po\\
a = ä && e = \underline{e} && i = ï && o = o\\
đa = shar && đe = shay && đi = shee && đo = sho\\
Aa = hah && Ae = hay && Ai = hee && Ao = ho\\
\lspbottomrule
\end{tabular}
\caption{Fletcher's representations of CV characters in the \ili{Ho-Chunk} \isi{syllabary} with \ili{English} pronunciations \citep[adapted from][300]{Fletcher1890a}.}
\label{fletchercv}
\end{table}

\begin{table}
\begin{tabular}{lllllll}
\lsptoprule
Syll. = Engl. && Syll. = Engl. && Syll. = Engl. && Syll. = Engl.\\
\midrule
Ka\emph{m} = gark && Ke\emph{m} = gake && Ki\emph{m} = geek && Ko\emph{m} = goke\\
da\emph{m} = jark && de\emph{m} = jake && di\emph{m} = geek && do\emph{m} = joke\\
wa\emph{m} = wärk && we\emph{m} = w\underline{e}rk && wi\emph{m} = week && wo\emph{m} = woke\\
xa\emph{m} = xärk && xe\emph{m} = x\underline{e}rk && xi\emph{m} = xeek && xo\emph{m} = xörk\\
ta\emph{m} = tdärk && te\emph{m} = td\underline{e}rk && ti\emph{m} = tdeek && to\emph{m} = tdörk\\
ma\emph{m} = märk && me\emph{m} = make && mi\emph{m} = meek && mo\emph{m} = moke\\
na\emph{m} = närk && ne\emph{m} = nake && ni\emph{m} = neek && no\emph{m} = noke\\
La\emph{m} = Rark && Le\emph{m} = Rake && Li\emph{m} = Reek && Lo\emph{m} = Roke\\
ga\emph{m} = Gwark && ge\emph{m} = Gwake && gi\emph{m} = Gweek && go\emph{m} = Gwooke\\
ra\emph{m} = sark && re\emph{m} = sake && ri\emph{m} = seek && ro\emph{m} = soke\\
Tha\emph{m} = Thark && The\emph{m} = Thake && Thi\emph{m} = Theek && Tho\emph{m} = Thoke\\
Ya\emph{m} = Yark && Ye\emph{m} = Yake && Yi\emph{m} = Yeek && Yo\emph{m} = Yoke\\
ba\emph{m} = park && be\emph{m} = pake && bi\emph{m} = peek && bo\emph{m} = poke\\
a\emph{m} = ark && e\emph{m} = ake && i\emph{m} = eek && o\emph{m} = oke\\
đa\emph{m} = shark && đe\emph{m} = shake && đi\emph{m} = sheek && đo\emph{m} = shoke\\
Aa\emph{m} = hark && Ae\emph{m} = hake && Ai\emph{m} = heek && Ao\emph{m} = hoke\\
\lspbottomrule
\end{tabular}
\caption{Fletcher's representations of CVC characters in the \ili{Ho-Chunk} \isi{syllabary} with \ili{English} pronunciations \citep[adapted from][300]{Fletcher1890a}.}
\label{fletchercvc}
\end{table}

An early stage of how \ili{Ho-Chunk} letter writers went about adapting the \ili{Sac and Fox} \isi{syllabary} to their own language can be seen in Tables \ref{fletchercv} and \ref{fletchercvc}, the \ili{Ho-Chunk} \isi{syllabary} table published by Fletcher in 1890. For easier display on the page, Fletcher's original table has been divided into two halves, one consisting of four vertical columns (one each for the vowels /a/, /e/, /i/ and /o/) of CV syllables and the other of four column of the same vowels in CVC syllables having the last consonant of /k/. This table appears odd to anyone familiar with the \ili{Ho-Chunk} language, but its anomalies can be explained as resulting from a combination of the following factors: the arrangement of the table by syllables rather than by vowels and consonants; the retention of two \ili{Sac and Fox} \isi{syllabary} characters not corresponding to \ili{Ho-Chunk} phonemes; the lack of \isi{syllabary} characters for four \ili{Ho-Chunk} vowels and seven \ili{Ho-Chunk} consonants (counting the \isi{glottal} stop); the difficulty of  representing handwritten symbols in print; the misreading of handwriting; the use of a separate character for the consonant /k/ when it followed a vowel to end a syllable; and the ad hoc spelling of the \ili{English} sound equivalents employed.

The arrangement of Tables \ref{fletchercv} and \ref{fletchercvc} into syllables is the system used by the Sac and Fox and the Ho-Chunk, who put spaces between these syllables when they combined them into words and sentences. The two characters retained from the \ili{Sac and Fox} \isi{syllabary}, although not phonemes of the \ili{Ho-Chunk} language, were a \emph{Th}-shaped \isi{syllabary} character standing for the \il{Sauk}Sac phoneme /θ/, which was not represented in Jones' \il{Mesquakie}Fox \isi{syllabary}; and a \emph{g}-shaped character used to represent a consonant \isi{cluster} -- although not the \il{Mesquakie}Fox consonant \isi{cluster} /sk/, but one combining /g/ and /w/ into /gw/. Although it includes these unnecessary characters, Fletcher's table lacks characters for the \ili{Ho-Chunk} oral vowel /u/, its three nasalized\is{nasalization} vowels /\k{a}/, /\k{i}/ and /\k{u}/, and its \isi{glottal} stop /'/. The table also omits six other \ili{Ho-Chunk} consonants, because it gives \isi{syllabary} characters for only one of each of the paired voiced and voiceless stops, affricatives, and spirants shown in columns A and B of Category I in \tabref{hochunkphonemes}. There is a character for the voiceless /p/ but not the voiced /b/; the voiced /g/ but not the voiceless /k/; the voiced /ǰ/ but not the voiceless /\v{c}/; the voiceless /s/ but not the voiced /z/; the voiceless /\v{s}/ but not the voiced /\v{z}/; and the voiceless /x/ but not the voiced /ɣ/. 

As far as can be told from the printed letters used in the table to stand for cursive handwriting, the \ili{Ho-Chunk} \isi{syllabary} in 1890 used the same or similar characters as the \il{Mesquakie}Fox \isi{syllabary} for the shared vowels and consonants /e/, /i/, /o/, /p/, /m/, /n/, /\v{s}/, /w/, /y/, and /t/. The \ili{Ho-Chunk} symbol for the vowel /a/ looked like a handwritten lower case <a> rather than the handwritten lower case <u> that stood for /a/ in the \il{Mesquakie}Fox \isi{syllabary}. The \ili{Ho-Chunk} character for the consonant /g/, an upper case <K>, seems to be the same as that used for the \ili{Sac and Fox} consonant /k/. However, the \ili{Ho-Chunk} consonant /\v{j}/ was not represented by the <tt> used by the Fox for the consonant /\v{c}/, but by a printed <d>. Fletcher's informant's \isi{syllabary} used another printed <d>, this one italicized, to stand for the consonant /\v{s}/. Fletcher's informant also deviated from the \ili{Sac and Fox} model by using a printed lower case <r> to stand for the consonant /s/ although the \il{Mesquakie}Fox \isi{syllabary} used an <s>. Perhaps when the Ho-Chunk adapter of the \isi{syllabary} studied \ili{Sac and Fox} \isi{syllabary} writing, he saw a lower-case handwritten <s> that was not closed at the bottom, mistaking it for an <r>, because some Fox users of their \isi{syllabary} wrote <s> that way \citep[170]{Walker1996}.  

The last four vertical columns of Fletcher's table, here reproduced in \tabref{fletchercvc}, contain some of its oddest-looking material, in both the \ili{Ho-Chunk} syllables to the left of the equal signs and the \ili{English} sound equivalents to their right. The italicized lower case <\emph{m}> in each CVC \ili{Ho-Chunk} syllable appears to represent an experiment with using a special character to stand for the consonant /k/ when it follows a vowel to end a syllable, /k/ being the consonant found most frequently in this position in \ili{Ho-Chunk}. Fletcher indicated that, when handwritten, this <\emph{m}> character was ``more like a wavy line'' \citeyearpar[300]{Fletcher1890a}. Of the \ili{English} sound equivalents given for the \ili{Ho-Chunk} syllables, the <ar> constructions in the first column of \tabref{fletchercv}, the <ark> constructions in the first column of \tabref{fletchercvc} and the three cases of <\underline{e}rk> in the second column of \tabref{fletchercvc} are the hardest to make sense of, because the \ili{Ho-Chunk} language has no /r/-controlled vowels. It seems to me that the only possible explanation for these constructions is that someone mistook someone else's handwritten <h> for <r>. That would make the top horizontal progression in the CVC table through the four oral vowels for a syllable beginning with /g/ and ending in /k/ read as follows: ``Ka\emph{m} = gahk'' (rather than gark), ``Ke\emph{m} = gake'', ``Ki\emph{m} = geek'', and ``Ko\emph{m} = goke''. The third \ili{English} sound equivalent here would be pronounced like the word in \ili{English} spelled the same way, and the second and fourth \ili{English} sound equivalents would each end in a silent <e> \citep[300]{Fletcher1890a}. 

Despite its shortcomings, Fletcher's table reproduced in Tables \ref{fletchercv} and \ref{fletchercvc} shows that by 1890 the Ho-Chunk adapter of the \ili{Sac and Fox} \isi{syllabary} had instigated two important innovations to make it more useful for writing his language. He had created a character based on an upper-case handwritten <L> to represent the \ili{Ho-Chunk} consonant /r/ and one resembling an upper-case handwritten <A> for the consonant /h/. The idea for this <A> character came originally from the Fox, but rather than using it for the consonant /h/, the Fox had put it before the vowel /a/ to represent that vowel when used in the initial position \citep[170]{Walker1996}. Further steps toward making the \ili{Ho-Chunk} \isi{syllabary} accurately reflect the language can be seen in the writings of Sam Blowsnake\ia{Blowsnake, Sam} and Felix White, Sr.\ia{White, Sr., Felix}, in the twentieth century. While Fletcher printed only a few words in \isi{syllabary} characters in her 1890 articles, both Blowsnake and White wrote manuscripts which can be examined for the way the \isi{syllabary} worked in extended \isi{discourse}. 

\section{The Blowsnake syllabary}

\begin{table}
\begin{tabular}{lllllllllll}
\lsptoprule
\multicolumn{11}{l}{\ili{Ho-Chunk} Vowels}\\
\multicolumn{2}{l}{Oral}\\
\cline{1-2}
a & \includegraphics{figures/Danker4a}\\
e & \includegraphics{figures/Danker4e}\\
i & \includegraphics{figures/Danker4i}\\
o & \includegraphics{figures/Danker4o}\\
u & \includegraphics{figures/Danker4o}, \includegraphics{figures/Danker4u}\\
\\
\multicolumn{11}{l}{\ili{Ho-Chunk} Consonants}\\
& \multicolumn{2}{l}{(Voiced)} & \multicolumn{2}{l}{(Voiceless)}\\
I & A && B && II &&& III\\
\midrule
stops & b & \includegraphics{figures/Danker4b} & p & \includegraphics{figures/Danker4p} & nasals & m & \includegraphics{figures/Danker4m} & stops & t & \includegraphics{figures/Danker4t}\\
& g & \includegraphics{figures/Danker4g} & k & \includegraphics{figures/Danker4k1}, \includegraphics{figures/Danker4k2} && n & \includegraphics{figures/Danker4n} && ' \\
af\isi{fricative} & ǰ & \includegraphics{figures/Danker4j} & č & \includegraphics{figures/Danker4ch} & trill & r & \includegraphics{figures/Danker4r} & breath & h & \includegraphics{figures/Danker4h}\\
spirants & z & \includegraphics{figures/Danker4z} & s & \includegraphics{figures/Danker4s} & semi- & w & \includegraphics{figures/Danker4w}\\
& ž & \includegraphics{figures/Danker4zh} & š & \includegraphics{figures/Danker4sh} & vowels & y & \includegraphics{figures/Danker4y}\\
& ɣ & \includegraphics{figures/Danker4gh} & x & \includegraphics{figures/Danker4x}\\
\lspbottomrule
\end{tabular}
\caption{\ili{Ho-Chunk} \isi{syllabary} characters used by Sam Blowsnake in 1938 \citep{Susman1939}, added to Susman's \citeyearpar[15]{Susman1943} arrangement of \ili{Ho-Chunk} phonemes.}
\label{blowsnakesyllabary}
\end{table}

\begin{figure}
\includegraphics[width=1\textwidth]{figures/DankerBlowsnakeText}
\caption{\ili{Ho-Chunk} \isi{syllabary} text written by Sam Blowsnake in 1938 \citep{Susman1939}.}
\label{blowsnaketext}
\end{figure}

\tabref{blowsnakesyllabary} adapts \tabref{hochunkphonemes} by adding the \isi{syllabary} characters written by Sam Blowsnake\ia{Blowsnake, Sam} in 1938 to Susman's \ili{Ho-Chunk} phonemes. We can see in \tabref{blowsnakesyllabary} that, although Blowsnake still did not differentiate nasalized\is{nasalization} from oral vowels, he did sometimes use the character <u> for the fifth oral \ili{Ho-Chunk} vowel /u/. This may have been a recent innovation for him, because he still usually wrote /u/ with an <o> as he did the vowel /o/, conforming most of the time to the four-vowel \isi{syllabary} systems of the Sac and Fox and of Fletcher's informant. Blowsnake did not mark \isi{glottal} stops, but he had a character for every other \ili{Ho-Chunk} consonant. Along with the symbols for /m/, /n/, /r/, /w/, /y/, /t/ and /h/ seen in Fletcher's \isi{syllabary}, he used a regularized system for indicating the paired stops, affricatives, and aspirants that Susman divided into columns A and B of her Category I consonants. He wrote the voiceless consonants in Column B as combinations of the characters for the voiced consonants in Column A with the symbol for /h/ (labeled by Susman\ia{Susman, Amelia~L.} as ``breath'') that the Fletcher\ia{Fletcher, Alice~C.} \isi{syllabary} had printed as an <A>. For voiceless /k/, he used two different characters -- one of them the upper-case character <K> for a voiced /g/ plus the symbol <A>, and the other a single upper-case <K> written in a slightly different way than the <K> representing /g/. Blowsnake wrote the \ili{Sac and Fox} symbol <tt> for /\v{j}/ rather than the <d> character employed by Fletcher's informant, and he did not use a wavy-line <\textit{m}> to indicate a /k/ following a vowel in a CVC syllable. That aspect of the \isi{syllabary} recorded by Fletcher had apparently been abandoned.

\figref{blowsnaketext} shows a page of a \isi{syllabary} text written by Blowsnake that Susman appended to the end of her \emph{Winnebago\il{Ho-Chunk} Syllabary} manuscript. It consisted of a description of traditional child-rearing instructions given to boys, entitled ``Child Teaching.'' Apparently, Blowsnake\ia{Blowsnake, Sam} wrote the \isi{syllabary} portion of it, and Susman typed the \isi{transcription}, the \ili{English} translation, and the two footnotes at the bottom. I have numbered the lines. In this text, we can see that the vowel /a/ was omitted after consonants in \ili{Ho-Chunk} \isi{syllabary} writing except when it followed a \isi{glottal} stop as it did at the end of line 10 in the iterative suffix \emph{-s'a}. The \isi{glottal} stop itself was not marked in Blowsnake's \isi{syllabary}, but its presence in this morpheme was indicated by the writing out of the usually unmarked vowel character <a>. Blowsnake did not mark \isi{glottal} stops occurring before other vowels. Fletcher's informant's \isi{syllabary} table listed syllables containing /a/ along with those using the other three marked \ili{Ho-Chunk} oral vowels. Without the evidence of \ili{Ho-Chunk} \isi{syllabary} texts from the nineteenth century, I do not know whether or not the earliest \ili{Ho-Chunk} letter writers left out the vowel /a/ when writing syllables. The \ili{Sac and Fox} \isi{syllabary} most closely resembling the \ili{Ho-Chunk} one did not omit /a/ or any other vowel, but other early forms of \ili{Sac and Fox} syllabaries have been recorded, one of which omitted /a/ after a consonant and used a dot\is{diacritics} to indicate /e/ after a consonant, a raised dot to indicate /i/, and two dots to indicate /o/ \citep[158--159]{Walker1981}.

The lower-case handwritten <r> that Blowsnake\ia{Blowsnake, Sam} used to write the /s/ in \emph{-s'a} at the end of line 10 in \figref{blowsnaketext} should be a <rA>, but he frequently wrote the voiced rather than the voiceless \isi{syllabary} characters for voiceless consonants, especially when they were not phrase initial or were part of a consonant \isi{cluster}. Halfway through line 4, we can see that Blowsnake used his \emph{y}-shaped \isi{syllabary} character for the glide sound in the phrase \emph{\v{c}iiokisak} (`center of the lodge'), which he spelled <ttAiyo Ki rAK>. However, in the middle of line 9, he also used this \isi{syllabary} character for the diphthong in the first syllable of the clause \emph{n\k{a}\k{a}\k{i}re\v{z}e} (`they slept, it is said', pronounced like the \ili{English} word \emph{nigh}), which he spelled <ny Le de>.  

Another characteristic of Blowsnake's \isi{syllabary} style seen in \figref{blowsnaketext} was his regular insertion of the same vowel between the two consonants of a consonant \isi{cluster} as the vowel that followed the \isi{cluster}. If the syllable ended in a consonant, he repeated the vowel after that consonant, too. This can be seen in \emph{w\k{a}\k{a}k\v{s}ik} (`Indians' or `people'), the third word in line 1, which he spelled <w KidiKi>. In this word, he omitted, as he often did, the <A> from the symbol standing for /k/. When he did write the <A> to make a consonant voiceless, he sometimes inserted a vowel between the character for the consonant and the <A>. For example, in the middle of line 3, he spelled the morpheme \emph{p\k{i}\k{i}'\k{u}} (`make good') <biAi o>. At the ends of lines 1, 4, 5, 7 and 10, and elsewhere in \figref{blowsnaketext}, we can see that Blowsnake\ia{Blowsnake, Sam} employed punctuation in his writing, in that he marked the end of sentences with periods.

In her unpublished analysis of Blowsnake's \isi{syllabary} text, \citeauthor{Susman1939} came to the conclusion that it employed ``a fairly consistent method of representing syllables by symbols derived from \ili{English} script.'' However, she noted three inefficiencies: 


\begin{enumerate}
\item{Duplication of symbols for a single sound, most notably the two symbols for /k/.}
\item{Ambiguity of symbols, such as in the use of <o> for both /o/ and /u/ sounds. She found especially troubling the ambiguity caused by the insertion of extra vowels in CCV and CVC syllables that sometimes made it impossible to tell them apart.}
\item{Some inconsistency of syllabification.}
\end{enumerate}

\section{The White syllabary}

The \isi{syllabary} writing system used by Felix White, Sr.\ia{White, Sr., Felix}, shown in \tabref{whitesyllabary} and \figref{whitetext} was quite similar to that of Sam Blowsnake\ia{Blowsnake, Sam}, but it addressed some of the inefficiencies mentioned by \citeauthor{Susman1939}. When Mr. White first began giving \ili{Ho-Chunk} \isi{language classes} at the Little Priest Community College in Winnebago, Nebraska, in the early 1970's, he used the \isi{syllabary} in teaching the language. He also taught me the \isi{syllabary} characters in the late 1970's when I expressed an interest in learning \ili{Ho-Chunk}. He told me that he and his aunt Florence Mann had made several changes to the \isi{syllabary}, including marking \isi{nasalization}. He used his \isi{nasalization} marks fairly often, but not always. They consisted of lines\is{diacritics} extending off of the top of vowels to the right. He indicated a nasalized unmarked /a/ by drawing a line from one consonant to the next over the space where /a/ would have appeared had it been marked. (See the second syllable in the first word, \emph{hoo\v{c}\k{a}k}, in line 4 of \figref{whitetext} below.)

\begin{table}
\begin{tabular}{lllllllllll}
\lsptoprule
\multicolumn{11}{l}{\ili{Ho-Chunk} Vowels}\\
\multicolumn{2}{l}{Oral} & \, & \multicolumn{2}{l}{Nasal}\\
\cline{1-2} \cline{4-5}
a & \includegraphics{figures/Danker6a} && ą & \includegraphics{figures/Danker6an}\\
e & \includegraphics{figures/Danker6e}\\
i & \includegraphics{figures/Danker6i} && į & \includegraphics{figures/Danker6in}\\
o & \includegraphics{figures/Danker6o}\\
u & \includegraphics{figures/Danker6o}, \includegraphics{figures/Danker6u} && ų & \includegraphics{figures/Danker6un}\\
\\
\multicolumn{11}{l}{\ili{Ho-Chunk} Consonants}\\
& \multicolumn{2}{l}{(Voiced)} & \multicolumn{2}{l}{(Voiceless)}\\
I & A && B && II &&& III\\
\midrule
stops & b && p && nasals & m & \includegraphics{figures/Danker6m} & stops & t & \includegraphics{figures/Danker6t}\\
& g & \includegraphics{figures/Danker6g} & k & \includegraphics{figures/Danker6k} && n & \includegraphics{figures/Danker6n} && ' & \includegraphics{figures/Danker6glottal}\\
af\isi{fricative} & ǰ && č && trill & r & \includegraphics{figures/Danker6r} & breath & h\\
spirants & z && s & \includegraphics{figures/Danker6s} & semi- & w & \includegraphics{figures/Danker6w}\\
& ž && š && vowels & y & \includegraphics{figures/Danker6y}\\
& ɣ & \includegraphics{figures/Danker6gh} & x & \includegraphics{figures/Danker6x}\\
\lspbottomrule
\end{tabular}
\caption{\ili{Ho-Chunk} \isi{syllabary} characters used by Felix White, Sr., in the 1970's, added to Susman's \citeyearpar[15]{Susman1943} arrangement of \ili{Ho-Chunk} phonemes.}
\label{whitesyllabary}
\end{table}

\begin{figure}
\includegraphics[width=1\textwidth]{figures/DankerWhiteText}
\caption{\ili{Ho-Chunk} \isi{syllabary} text written by Felix White, Sr., in the 1970's (Danker papers).}
\label{whitetext}
\end{figure}

Unlike Blowsnake\ia{Blowsnake, Sam}, White\ia{White, Sr., Felix} used only one character for the consonant /k/, but like Blowsnake, he employed a new symbol, in his case <oo>, to stand for the oral vowel /u/. However, also like Blowsnake, he almost never used it, much more frequently writing <o> for both /o/ and /u/. \figref{whitetext} shows that White marked \isi{glottal} stops, and that he wrote a more open form of the <A> character standing for /h/ than Blowsnake did. He said that this character was not supposed to represent an <A>, but a star shape. Like Susman\ia{Susman, Amelia~L.}, he also called it a ``breath'' symbol. He wrote the character he used for the semi-vowel /y/, not like Blowsnake's lower-case \ili{English} <y>, but more like the \ili{Mesquakie}Fox symbol recorded by \citet{Jones1906} for this phoneme, which looked something like the front part of a number <2>. One case in which White used less regularized \isi{syllabary} characters than Blowsnake had to do with the voiced consonant /z/ and the voiceless consonant /s/. White wrote /s/ with the character <d> and /z/ with the character <dA> instead of the other way around. 

These and other aspects of the \isi{syllabary} as written by White\ia{White, Sr., Felix} can be seen in \figref{whitetext}. It is the first page of a \isi{transcription} White made of an audio tape of a story told by a friend of his named Jim Frenchman\ia{Frenchman, Jim}. I remember that in \ili{English} White called this story ``The Killing of the Body Outright.'' White transcribed the story using the \isi{syllabary} and annotated it himself based on his studies of linguistics. I found this single page in some of my papers a few years ago and have added the line numbers. Line 6 shows an example of \isi{nasalization} marked on the second and third syllables of its last word \emph{ho\v{c}\k{i}\v{c}\k{i}}, which means `boys'. White followed the practice of appending an <A> to a voiced Category I consonant to make it voiceless more consistently than Blowsnake\ia{Blowsnake, Sam} did, though he did not always do it, and never in the case of /s/. Except in one instance on this manuscript page, White did not insert extra vowels between consonant \isi{cluster}s or after syllable-ending consonants. The exception occurs at the beginning of line 6, where he put an <o> with a \isi{nasalization} line\is{diacritics} (standing for /\k{u}/) between the consonant \isi{cluster} beginning the first syllable of the dubitive enclitic \emph{\v{s}g\k{u}n\k{i}}, which he spelled <dAoko ni>. 

In the middle of line 8, in the iterative marker \emph{-s'a}, we can see that, like Blow\-snake, White wrote out the usually unmarked vowel <a> to indicate a \isi{glottal} stop preceding it, but he also included a mark like an apostrophe for the stop itself. He also used this glottal-stop marker when the \isi{glottal} stop occurred with vowels other than /a/. White employed his \emph{2}-shaped \isi{syllabary} character for the semi-vowel /y/ where it was appropriate at the beginning of the second syllable of \emph{\v{c}iyo}, the first phrase in line 10. However, he did not use it at the beginning of line 9 to represent the diphthong pronounced much like the \ili{English} word \emph{why} in the second to the last syllable of the clause \emph{\v{c}\k{a}\k{a}grogiwaire} (`they went outside'), as Blowsnake would have done. Instead, White indicated this diphthong by writing out the vowels <a> and <i> and extending a curved line upward\is{diacritics} from the end of the <i>. In this way, he differentiated the semi-vowel /y/ from a diphthong that sounded nothing like it. Like Blowsnake, White marked the end of sentences in his \isi{syllabary} writing with periods, but he also used commas to separate \isi{clauses} within sentences, making the structure of these sentences easier to determine when read. Nonetheless, it remained difficult for the recipients of letters written using the \isi{syllabary} to always tell how to group syllables into the words their writers intended. White told me that people would puzzle over some parts of a letter they received because of this, but they could usually figure out the correct words and meanings in the context of the rest of the message. 

\section{Conclusion}

Comparing the \ili{Ho-Chunk} \isi{syllabary} of Alice Fletcher\ia{Fletcher, Alice~C.}'s unnamed informant with the writings of Sam Blowsnake\ia{Blowsnake, Sam}, and Felix White, Sr.\ia{White, Sr., Felix} reveals steady progress from the 1880's into the last half of the twentieth century toward correspondence with the phonemic characteristics of the \ili{Ho-Chunk} language. The users of this writing system would no doubt have continued to refine it had it not become unnecessary for communication between Wisconsin and Nebraska when almost all tribal members learned to speak and write \ili{English}. Examination of these syllabaries also reveals some instances in which their practitioners deviated in similar ways from strict phonemic fidelity to \ili{Ho-Chunk}. Perhaps these reoccurring ``errors'' indicate something about how \ili{Ho-Chunk} speakers conceptualized the phonemes of their language. 

That Fletcher's informant wrote down syllables containing only the four oral vowels /a/, /e/, /i/ and /o/ in his table could be attributed to the influence of the \ili{Sac and Fox} model from which he worked. However, it could also reflect a perception that the three nasalized \ili{Ho-Chunk} vowels represented only a subset of their oral counterparts. Of the three \isi{syllabary} writers, only White decided to mark \isi{nasalization}. Perhaps there was also a sense that the vowels /o/ and /u/ were related in some way that made their differentiation unnecessary. It is interesting that while Blowsnake and White each developed a new character to represent /u/ as a separate vowel, they almost always conformed in their writing to the pattern of Fletcher's informant by writing down the same character for /u/ as for /o/. Even the word that White used to refer to the \isi{syllabary}, \emph{ba-be-bi-bo-ra}, is made up of syllables containing only the vowels /a/, /e/, /i/ and /o/, followed by the definite enclitic \emph{ra}. 

The manner in which the three \isi{syllabary} versions examined in this paper handled the voiced and voiceless consonants listed by \citeauthor{Susman1939} under Category I in \tabref{hochunkphonemes} also invites speculation. The \ili{Sac and Fox} \isi{syllabary} contained characters for all of the consonants that Susman\ia{Susman, Amelia~L.} listed in Categories II and III for \ili{Ho-Chunk} in \tabref{hochunkphonemes} except for /r/ and the \isi{glottal} stop, which are not found in the \ili{Sac and Fox} language, and /h/, which the \ili{Sac and Fox} \isi{syllabary} did not mark. Fletcher's\ia{Fletcher, Alice~C.} informant virtually copied the \ili{Sac and Fox} \isi{syllabary} characters for these shared Category II and III phonemes (/m/, /n/, /w/, /y/ and /t/) in his \ili{Ho-Chunk} \isi{syllabary}, yet he treated Susman's Category I consonants differently. \ili{Sac and Fox} shares five of these consonants with \ili{Ho-Chunk}, all of them voiceless (/\v{c}/, /k/, /p/, /s/ and /\v{s}/). \ili{Ho-Chunk} has, in addition, the voiced counterparts of these consonants (/\v{j}/, /g/, /b/, /z/ and /\v{z}/) as well as the voiceless and voiced pair /x/ and /ɣ/.  Fletcher's informant, however, did not simply retain in his \ili{Ho-Chunk} \isi{syllabary} the five voiceless consonants for which the Sac and Fox had invented \isi{syllabary} characters.  Instead, he kept three of these voiceless consonants (/p/, /s/ and /\v{s}/) and added two voiced ones (/\v{j}/ and /g/) and the unvoiced /x/. In doing so, he devised a system which included a character for only one of each of the paired \ili{Ho-Chunk} voiced and voiceless stops, affricatives, and spirants in Category I. I assume that in writing he used the same character for both the voiced and voiceless consonants in a pair. Perhaps we can infer from this that he was aware that these consonants formed linked pairs and thought each pair so closely related that one \isi{syllabary} character would suffice for both of its parts. This system of having one \isi{syllabary} symbol stand for two different consonants when they consist of voiced and voiceless pairs was also practiced by the Potawatomi in their Great Lakes \ili{Algonquian} \isi{syllabary}. The \ili{Potawatomi} language has phonemes for all of the voiced and voiceless consonant pairs listed by Susman for \ili{Ho-Chunk} except for /ɣ/ and /x/. The writers of the \ili{Potawatomi} \isi{syllabary} used the character <l> for /b/ and /p/, <s> for /z/ and /s/, <sh> for /\v{z}/ and /\v{s}/, <tt> for /\v{j}/ and /\v{c}/, and <K> for /g/ and /k/. Their language also has a voiced and voiceless pair /d/ and /t/ that they wrote with a <t> \citep[172, Fig. 11]{Walker1996}.  

There can be no doubt that Blowsnake\ia{Blowsnake, Sam} and White\ia{White, Sr., Felix} knew of the relationship between the paired voiced and voiceless consonants in \ili{Ho-Chunk}, because of the way they sometimes, at least, wrote them with a character for each voiced consonant to which they added the <A> symbol to indicate its voiceless counterpart. When White started teaching me \ili{Ho-Chunk} in the 1970's, he had me make a set of flash cards with which to practice recognition that had the \isi{syllabary} characters on one side and their \ili{English} equivalents on the other. These cards clearly showed White's \isi{syllabary} system for marking voiced and voiceless consonants as well the fact that White switched that system around in the case of /z/ and /s/. This reversal appears to be a reversion to the way Fletcher's informant wrote the symbol for /s/ as <r>. In a way, it goes along with Blowsnake's frequent, and White's own occasional, practice of writing voiceless consonants as voiced. This inconsistency seems to indicate a persistent perception, reminiscent of those of Fletcher's informant and the \ili{Potawatomi} \isi{syllabary} writers, that paired voiced and voiceless consonants are so closely united with each other that it is not crucial to differentiate them in writing, that they are interchangeable. Perhaps this sense of interchangeability comes from the way that voiceless consonants do sometimes change to voiced ones, and vice-versa, at the intersection of morphemes in spoken \ili{Ho-Chunk}. For example, the voiceless consonant at the end of the word \emph{waa\v{c}} (`boat') becomes voiced when the definite enclitic \emph{-ra} (`the') is added to it, forming \emph{waa\v{j}ra} (`the boat'). Conversely, the voiced consonant at the beginning of the causal suffix \emph{-ge\v{j}\k{i}n\k{i}} becomes voiceless when it is added to the clause \emph{h\k{i}wuus}, `I was dry,' to become \emph{h\k{i}wuuske\v{j}\k{i}n\k{i}} (`because I was dry') \citep[33 \& 41]{Lipkind1945}. 

In their efforts to preserve and teach their language today, the Ho-Chunk in Wisconsin have collaborated with linguists to devise an \isi{orthography} that is more completely phonemic, more easily printed, and more comprehensible to a wider range of readers than was their handwritten \isi{syllabary}. However, the native-language \isi{syllabary} the Ho-Chunk developed and refined from the late nineteenth to the mid twentieth centuries was important. It provided families and friends with a way to communicate and maintain ties between Nebraska and Wisconsin through years of forced separation, persistent poverty, and cultural suppression. It contributed to the Ho-Chunk tribe's survival as a people. 

\printbibliography[heading=subbibliography,notkeyword=this]

\end{document}
