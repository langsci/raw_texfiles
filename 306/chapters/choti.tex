\documentclass[output=paper]{langscibook}
\ChapterDOI{10.5281/zenodo.6393740}
\author{Jonathan Choti\affiliation{Michigan State University}}
\title{The augment in Haya and Ekegusii}

\abstract{This article examines the behavior of the augment in Haya (E22) and Ekegusii (E42), two Bantu Zone E languages, revealing many similarities and a few differences between the Haya and Ekegusii augment. In both languages, morphosyntactic, semantic, and pragmatic requirements regulate the behavior of the augment. The common shape of the augment is a vowel (V, namely /a/, /e/, and /o/). Besides, Ekegusii has the CV shape in \textit{ri-} and \textit{chi-}6 of class 5 and 10, respectively. Augmented nouns in both languages are the default but are ambiguous between a specific and non-specific reading. In Haya and Ekegusii, the augment is not marked on proper names, most kinship terms, and vocative nouns because these pick out specific referents. Nouns used as adverbs of location, time, and manner omit the augment in both languages. The two languages require the augment in predicative and associative constructions. In complex nouns, both elements require the augment but in compound nouns, only the first is augmented in both languages. The two languages allow the augment in gerunds but not in infinitives. Most pronominals require the augment in the two languages. Haya and Ekegusii disallow the augment in interrogative and negative constructions, proverbs, and nouns modified by ‘any’ to signal non-specific reference. In both languages, affirmative declaratives require the augment. Emphatic nouns in topic and contrastive focus positions require the augment to mark emphasis and specificity even in negative contexts. These features of the augment in Haya and Ekegusii confirm that the so-called augment is actually a bound article.}

\begin{document}
\SetupAffiliations{mark style=none}
\maketitle

\section{Introduction}
In a number of Bantu languages, common nouns and other nominals contain a stem-initial prefix that precedes the class prefix. This prefix is commonly known as the augment (or initial vowel or preprefix). However, some Bantu languages such as Swahili do not have the augment. \Citet{blois1970augment} presented a typological survey of the behavior of the augment in which he demonstrated its cross-linguistic variation and the semantic, syntactic, and other factors that regulate its behavior. He however concluded that his investigation was incomplete because of insufficient data. Subsequent studies on the augment have focused on its behavior in individual languages such as Dzamba \citep{bokamba1971specificity}, Haya \citep{chagas1977}, Luganda  \citep{ashton1987luganda,hyman1993augment,ferrari2009,mould1974syntax}, Kinande \citep{progovac1993non}, Kagulu \citep{petzell2003function}, IsiXhosa \citep{visser2008definiteness}, Kirundi \citep{ndayiragije2012augment}, and Nata \citep{gambarage2013pre,gambarage2019belief}. Some of these accounts maintained that the behavior of the augment is regulated by the semantics, i.e. definiteness and specificity (e.g. \citealt{bleek1869comparative}, \citealt{bokamba1971specificity}, \citealt{gambarage2013pre,gambarage2019belief}, \citealt{givon1972studies}, \citealt{meeussen1959essai}, \citealt{mould1974syntax}). Other studies argued that morphosyntactic requirements are the key determinants of its behavior (e.g. \citealt{dewees1971role}, \citealt{hyman1993augment}). The current study develops \citeauthor{blois1970augment}’ typological account by comparing the behavior of the augment in Haya and Ekegusii of Bantu Zone E \citep{guthrie196771}. The data in \REF{hayagusii1} and \REF{hayagusii2} illustrate the marking of the augment in the two languages:\footnote{Any undocumented Haya data in this study stems from from the Yoza dialect and was provided by Abdul Mutashobya. The Ekegusii data was provided by the author who is a native speaker of the language. The author is thankful to Abdul Mutashobya for the Haya data.}\largerpage

\ea
 Augment marking in Haya\footnote{The numerals in the glosses indicate the noun class of the noun.} \label{hayagusii1}
    \ea\label{hayagusii1a} o-mu-ana\hphantom{`}  	‘1-child’
    \ex\label{hayagusii1b}   	a-ba-ana\hphantom{o`}  	‘2-children’
    \ex\label{hayagusii1c}  	e-ki-imba\hphantom{i}  	‘7-bean’
    \ex\label{hayagusii1d}  	o-ru-limi\hphantom{n}	‘11-tongue’
    \z
\z

\ea Augment marking in Ekegusii\label{hayagusii2}
    \ea\label{hayagusii2a}  o-mo-nto\hphantom{`} 	‘1-person’
    \ex\label{hayagusii2b}   a-aba-nto\hphantom{`}	 ‘2-people’
    \ex\label{hayagusii2c}  o-mo-te\hphantom{ln}		‘3-tree’
    \ex\label{hayagusii2d}  ri-i-timo\hphantom{n}		‘5-spear’
    \z
\z
In (\ref{hayagusii1}), the Haya augment occurs as \textit{o}- (\ref{hayagusii1a}, \ref{hayagusii1d}), \textit{a}- (\ref{hayagusii1b}), and \textit{e}- (\ref{hayagusii1c}) and is immediately followed respectively by the class prefixes -\textit{mu}- (\ref{hayagusii1a}), -\textit{ba}- (\ref{hayagusii1b}), -\textit{ki}- (\ref{hayagusii1c}) and -\textit{ru}- (\ref{hayagusii1d}). In (\ref{hayagusii2}), the Ekegusii augment is realized as \textit{o}- (\ref{hayagusii2a}, \ref{hayagusii2c}), \textit{a}- (\ref{hayagusii2b}), and \textit{ri}- (\ref{hayagusii2d}) before the class prefixes -\textit{mo}- (\ref{hayagusii2a}, \ref{hayagusii2c}), -\textit{ba}- (\ref{hayagusii2b}), and -\textit{i}- (\ref{hayagusii2d}), respectively. The third element in the examples is the nominal root.

The goal of this article is three-fold. The first is to compare and contrast the behavior of the augment in Haya and Ekegusii. The second is to determine the semantic, pragmatic, and morphosyntactic properties of the augment in both languages. The third is to show that the behavior of the augment is consistent with that of articles in other languages. The rest of this article proceeds as follows. A review of previous accounts of the augment is presented in \sectref{sec:choti:2}, formal properties of the Haya and Ekegusii augments in \sectref{sec:choti:3}, and their grammatical properties in \sectref{sec:choti:4}. The semantic/pragmatic account of the augment in Haya and Ekegusii is presented in \sectref{sec:choti:5}, the augment’s article properties in \sectref{sec:choti:6}, and the summary and conclusion in \sectref{sec:choti:7}.

\section{Review of previous accounts of the augment}\label{sec:choti:2}
In the literature, three aspects of the augment appear to be prominent: Semantic, morphosyntactic, and typological properties. This review focuses on the aforementioned characteristics of the augment that are presented in three different subsections.

\subsection{Semantic properties of the augment}
The primary characteristic of the augment is that it occurs in some contexts but not in others. Some of the earlier accounts of the augment maintain that it's (non)-occurrence is determined by the contrast between (in)definiteness and/or (non-)specificity, as in Bemba \citep{givon1972studies},  Luganda \citep{ashton1987luganda,ferrari2009,mould1974syntax}, Dzamba \citep{bokamba1971specificity},  Kagulu \citep{petzell2003function}, Kinande \citep{progovac1993non}, Xhosa \citep{visser2008definiteness},  and Nata \citep{gambarage2013pre,gambarage2019belief}. In Dzamba, \citet[220]{bokamba1971specificity} concluded that the behavior of the augment is determined by the semantics, describing it as a referentiality and definiteness marker. The Dzamba examples in \REF{hayagusii3} demonstrate that the absence of the augment in \textit{moibi} ‘thief’ \REF{ex:choti:3a} and its occurrence in \textit{omoibi} ‘the thief’ \REF{ex:choti:3b} makes a semantic distinction.

\begin{exe}
\ex \label{hayagusii3}
\begin{xlist}
\ex\label{ex:choti:3a} mo-ibi (mɔɔ) anyɔlɔki  ondaku
\glt ‘A thief entered the house.’
\ex\label{ex:choti:3b}o-mo-ibi (*mɔɔ) anyɔlɔki    ondaku
\glt ‘The thief entered the house.’
\end{xlist}
\end{exe}
Moreover, the Dzamba augment is obligatory in topicalized NPs and NPs modified by relative clauses and adjectives. Bokamba concluded that in these contexts, the augment marks specificity.

\subsection{Morphosyntactic properties of the augment}
Some of the previous studies of the augment analyzed it as part of inflectional morphology and attributed its behavior to syntactic or a combination of syntactic and morphological requirements  (e.g. \citealt{dewees1971role,hyman1993augment,mould1974syntax}). \citet[224]{hyman1993augment} showed that in Luganda, two operators of negation and focus license bare nouns (those without the augment) while nouns with the augment are self-licensing, as in the examples in \REF{hayagusii4}:

\begin{exe}
\ex Augment in Luganda\label{hayagusii4}
\begin{xlist}
\ex\label{hayagusii4a} \gll	tè-bááwà               báànà    bìtábó \\
            NEG-they gave    children    books \\
      \trans     ‘They didn’t give children books.’\\
           (*à-báànà  è-bìtábó, *à-báànà  bìtábó, *báànà  è-bìtábó)
\ex\label{hayagusii4b} 	yàgúlà  bìtábó  (bìnó) \\
        ‘He bought (these) books.’  [Postverbal focus]
\end{xlist}
\end{exe}
In \REF{hayagusii4a}, the nouns \textit{báànà} ‘children’ and \textit{bìtábó} ‘books’ appear without the augment because they occur within the scope of negation. In \REF{hayagusii4b}, the noun \textit{bìtábó} ‘books’ appears as a bare noun in a postverbal focus position where it may co-occur with the demonstrative \textit{bìnó} ‘these’. The scopal relations between the bare NPs and the operators led \citet{hyman1993augment} to attribute the absence of the Luganda augment to the syntax. They further argued that the Luganda augment cannot have any semantic correlates because it is an inflectional category with all the properties of inflectional morphology such as those proposed by \citet{anderson1988inflection}.\footnote{The relevant properties of inflectional morphology include: (a) configurational properties (i.e., sensitivity to syntactic configurations), (b) agreement properties, (c) inherent properties, and (d) phrasal properties \citep{anderson1988inflection}.} \citet{hyman1993augment} claim that only syntax (and not semantics or pragmatics) conditions the augment in Luganda is too strong. Besides, the fact that an augment is an inflectional category does not mean that it has no semantic or pragmatic correlates. Furthermore, negation and focus are semantic principles as well. Moreover, other studies such as \citet[30]{ashton1987luganda}, \citet{mould1974syntax}, and \citet{ferrari2009} determined a semantic factor in the behavior of the Luganda augment. In addition, (non-)specificity in opaque contexts such as negative and interrogative construction involve scopal relations that draw on syntactic and semantic principles (e.g. \citealt{abusch1993scope,lyons1999definiteness,winter1997choice}). The current study reconciles the morphosyntactic and semantic accounts of the augment and shows that the properties of the augment range from morphological, phonological, syntactic, and semantic to pragmatic ones.

\subsection{Typological properties of the augment}
\Citet{blois1970augment} categorized augment languages into three classes based on the factors that determine the behavior of the augment in the language. These factors include formal grammatical conditions, definable semantic function, and special functions. He attributed the absence of the augment in locatives, vocatives, compound nouns, kinship terms, proper names, predicative constructions, etc. to formal grammatical conditions. In languages where the augment has a definable function, it appears in nouns with determinate or particularized referents, but does not occur in other conditions. Languages in which augment has special functions include Tswa, Ronga, Thonga-Shangaan, and Tonga. In these languages, it was observed that faster speakers used the augment while slower speakers did not. The data in (\ref{hayagusii5}--\ref{hayagusii7}) show \citeauthor{blois1970augment}’ categories of augment languages:

\ea Formal grammar: kinship terms, titles
\label{hayagusii5}
  \ea\label{hayagusii5a} 	Nyakyusa: 	nsoko\hphantom{tiru} 		‘your mother’
  \ex\label{hayagusii5b} 	Rundi:\hphantom{iusa}		nyookûru 	‘my grandmother’
	\ex\label{hayagusii5c}  Nande:\hphantom{usa} 		tatâ\hphantom{okuru}		‘my father’
  \ex\label{hayagusii5d} 	Ganda:\hphantom{usa}		ssebo\hphantom{turu}		‘sir’
  \z
\z

\ea Semantic function
\label{hayagusii6}
  \ea\label{hayagusii6a} 	Sumbwa: 	a-maguta matimbu 	‘the oil is good’
  \ex\label{hayagusii6b} 	Sumbwa: 	tuagula maguta\hphantom{nbu} 	‘we bought some oil’
  \z
\z

\ea Special functions\\
\label{hayagusii7}
  \gllll Tonga:   {bazyala ciindi comwe \ldots}  \\
     {} {‘they simultaneously produced \ldots'} \\
     {} {i-bana bakozyanya} \\
     {} {`children who look like each other’}\\
\z

In (\ref{hayagusii5}), the kinship terms \textit{nsoko, nyookûru, tatâ}, and the title \textit{ssebo} occur without the augment. In \REF{hayagusii6a}, the augment \textit{a-} in \textit{amaguta} ‘the oil’ marks definiteness while its absence in \textit{maguta} ‘some oil’ \REF{hayagusii6b} signals indefiniteness. In \REF{hayagusii7}, \textit{i-bana} ‘children’ occurs with the augment \textit{i}- after a pause. I noted earlier that both semantics and syntax have a bearing on the behavior of the augment in Luganda. Thus, \citeauthor{blois1970augment}’ categorization of augment languages into three distinct classes based on its function is questionable. The rest of this article shows that the behavior of the augment may vary intra- and cross-linguistically due to syntactic, semantic, and pragmatic requirements.

\section{Formal properties of the augment in Haya and Ekegusii} \label{sec:choti:3}
This section presents basic facts about the shapes of the augment in Haya and Ekegusii, respectively. In Haya, the augment occurs consistently as a vowel (V), which is also the most common shape of the Ekegusii augment. The other shape of the augment in Ekegusii is a consonant-vowel sequence (CV) that occurs in class 5 and 10.  The Ekegusii class 5 augment has two allomorphs, V and CV. In both languages, the augment is not realized in class 1b nouns (kinship terms) and class 21 nouns (proper names). The other similarity between the Haya and Ekegusii augment is that the V shape involves the three non-high vowels /a/, /e/, and /o/. However, the vowel of the CV augment in Ekegusii is /i/.  The data in \tabref{hayagusii8} show the respective shapes of the augments in Haya \citep[35]{chagas1977} and Ekegusii (adapted from \citealt[199]{cammenga2002phonology}).

\vfill
\begin{table}[H]
\caption{Augments in Haya and Ekegusii\label{hayagusii8}}
\begin{subtable}[t]{.5\textwidth}\centering
\caption{Augment in Haya\label{hayagusii8:a}}
\begin{tabular}{ll}
\lsptoprule
Class &	Augment+prefix\\\midrule
1  &	o-mu-\\
2  &	a-ba-\\
3  &	o-mu-\\
4  &	e-mi-\\
5  &	e-li-\\
6  &	a-ma-\\
7  &	e-ki-\\
8  &	e-bi-\\
9  &	e-N-\\
10 &	e-N-\\
11 &	o-ru-\\
12 &    a-ka-\\
13 &	o-tu-\\
14 &	o-bu-\\
15 &	o-ku-\\
16 &	a-ha-\\
17 &	o-ku-\\
18 &	o-mu-\\
\lspbottomrule
\end{tabular}
\end{subtable}\begin{subtable}[t]{.5\textwidth}\centering
\caption{Augment in Ekegusii\label{hayagusii8:b}}
\begin{tabular}{ll}
\lsptoprule
 Class	&	Augment+prefix \\\midrule
   1	&	o-mo- \\
   2	&	a-ba- \\
   3	&	o-mo- \\
   4	&	e-me- \\
   5	&	e-ri-, ri-i- \\
   6	&	a-ma-\\
   7	&	e-ke-\\
   8	&	e-bi- \\
   9	&	e-N- \\
   10	&	chi-N- \\
   11	&	o-ro- \\
   12	&	a-ka- \\
   13	&	e-bi- \\
   14	&	o-bo- \\
   15	&	o-ko- \\
\lspbottomrule
\end{tabular}
\end{subtable}
\end{table}
\vfill\hbox{}\pagebreak

In \tabref{hayagusii8:a}, the Haya augment vowel exhibits harmony with the prefix vowel. The augment appears as the back vowel /o/ when the prefix vowel is the back vowel /u/, as in class 1, 3, 11, 13, 14, 15, 17, and 18. The Haya augment vowel occurs as the front vowel /e/ whenever the prefix vowel is the front vowel /i/, as in class 4, 5, 7, 8, 9, and 10. There is perfect harmony in the Haya augment and prefix vowel involving /a/ in class 2, 6, 12, and 16. In \tabref{hayagusii8:b}, the Ekegusii augment vowel exhibits perfect harmony with the prefix vowel in most of the cases as well. For example, it appears as /o/ whenever the prefix vowel is /o/, as in class 1, 3, 11, 14, and 15. The Ekegusii augment vowel appears as /a/ before class prefixes with the vowel /a/, as in class 2 and 6. The Ekegusii augment /e/ occurs before prefixes with front vowels /e/ and /i/, as in class 4, 5, 7, 8, and 13. The Ekegusii vowel in the CV augment of class 5 appears as /i/ before the prefix vowel /i/\footnote{A reviewer suggested that the Ekegusii class 5 allomorph ri- of the augment might be due to a metathesis rule /i-ri-/ \rightarrow\ [ri-i-] to get rid of word-initial /i/. The lowering of Proto-Bantu initial /i/ to [e] created the other allomorph /e/.} . It is possible to relate the Ekegusii augment vowels /e/ and /i/ of class 9 and 10 to Proto-Bantu /e-Ni-/ \citep[35]{chagas1977}. Apparently, Haya and Ekegusii exhibit many similarities in the formal properties of their augments despite the CV shape observed in Ekegusii. Besides, Haya locative classes 16, 17, and 18 have an augment but the single locative class in Ekegusii (i.e. class 16) does not, and thus is not included in \tabref{hayagusii8:b}. \tabref{tab:1l} summarizes the formal shapes of the augment in Haya and Ekegusii.\largerpage[-2]

\begin{table}
    \begin{tabular}{ccc}
    \lsptoprule
      Shape   &  Haya & Ekegusii\\\midrule
        V & a- & a- \\
        V & e- & e- \\
        V & o- & o- \\
    CV & - & ri- \\
    CV & - & chi- \\
    \lspbottomrule
    \end{tabular}
    \caption{Shapes of the augment in Haya and Ekegusii}
    \label{tab:1l}
\end{table}

\section{Grammatical properties of the augment in Haya and Ekegusii}\label{sec:choti:4}\largerpage
This section addresses the behavior of the augment in Haya and Ekegusii across different morphosyntactic environments especially those identified in \citet{blois1970augment} and elsewhere. The relevant contexts include proper names and kinship terms, complex nouns, compound nouns, predicative constructions, adverbial nouns, verbal nouns (gerunds and infinitives), associative constructions, nouns modified by determiners such as ‘other’ and ‘every’, adjectives, relative clauses, and vocatives. Morphosyntactic contexts that are ambiguous between syntax and semantics appear in \sectref{sec:choti:5} and include nouns modified by ‘any’, and nouns in negative and interrogative constructions. This section is organized into six subsections.

\subsection{Augment in proper names and kinship terms}\label{sec:choti:4.1}
Kinship terminology refers to “the system of names applied to categories of kin standing in relationship to one another” \citep{britannica2017}. This definition implies that kinship terms, similar to proper names, pick out specific referents because these terms denote categories of kin standing in relationship with one another. Well-known examples of kinship terminology include ‘father’, ‘mother’, ‘sister’, ‘brother’, ‘wife’, and ‘husband’. In both Haya and Ekegusii, many kinship terms do not take the augment. Additionally, the augment is not marked on proper names even those derived from common nouns that take the augment. The examples in (\ref{hayagusii9a}--\ref{hayagusii9c}) illustrate these facts.


\ea Augment in proper names and kinship terms
\label{hayagusii9}
  \ea\label{hayagusii9a} Absence of augment in proper names \\
    \glllll Haya	 {}				Ekegusii {} \\
      Ntale 		{(e-ntale ‘lion’)}	         Sese  {(e-sese ‘9-dog’)} \\
      Milembe 	{(e-milembe`blessings')}	 Kerandi  {(e-kerandi ‘7-gourd’)} \\
      Burungi 	{(o-burungi ‘beauty’)}		Sigara  {(e-sigara‘9-cigarette’)} \\
      Mukama 	{(o-mukama ‘king’)}		Nyanchera  {(e-nchera ‘9-path’)} \\

  \ex\label{hayagusii9b} Absence of augment in kinship terms \\
  \glllllll Haya {}  Ekegusii {} \\
  mae  {`my mother'} 		{baba, mama}  {`my mother'} \\
  tata 	 {`my father'} 		tata  {`my father'} \\
	isho  {`your father'} 	iso   {`your father'}\\
  nyoko    {`your mother'}   	nyoko  {`your mother'} \\
  {mae enkuru}	 {`my grandma'} 	magokoro  {`my grandma'} \\
  {tata enkuru}	 {`my grandpa'} 	sokoro  {`my grandpa'}\\

  \ex\label{hayagusii9c} Absence vs. marking of augment on kinship terms \\
   \gllllllll Haya   Ekegusii  Gloss \\
   {o-munyanya wange}		{moiseke ominto}    {`my sister'} \\
   {} {(o-moiseke ‘1-girl’)} {}\\
   {o-munyanyazi wange}		{momura ominto}         {`my brother'}\\
   {} {(o-momura ‘1-boy’)} {}\\
   {o-mushaija wange}	 	{o-mosaacha one}     {`my husband'}\\
   {} {(o-mosaacha ‘1-man’)} {}\\
   {o-mukazi wange}		mokaane   {`my wife'}\\
   \z
\z

\begin{sloppypar}\noindent In \REF{hayagusii9a}, the Haya augment is marked on common nouns \textit{entale} ‘lion’, \textit{emilembe} ‘blessings’, \textit{oburungi} ‘beauty’, and \textit{omukama} ‘king’. However, the proper names derived from these common nouns omit the augment. They include \textit{Ntale}, \textit{Milembe}, \textit{Burungi}, and \textit{Mukama}. In Ekegusii, proper names \textit{Sese}, \textit{Kerandi}, \textit{Sigara}, and \textit{Nyanchera} occur without the augment but their corresponding common nouns do, i.e. \textit{e-sese} ‘dog’, \textit{e-kerandi} ‘gourd’, \textit{e-sigara} ‘cigarette’, and \textit{e-nchera} ‘path’. In \REF{hayagusii9b}, Haya kinship terms that occur without the augment include \textit{mae} ‘mother’, \textit{tata} ‘father’, \textit{isho} ‘your father’, \textit{nyoko} ‘your mother’, \textit{mae enkuru} ‘my grandma, and \textit{tata enkuru} ‘grandpa’. Ekegusii examples include \textit{tata} ‘my father’, \textit{baba}, \textit{mama} ‘my mother’, \textit{iso} ‘your father’, \textit{nyoko} ‘your mother’, \textit{magokoro} ‘grandma, and \textit{sokoro} ‘grandpa’. The data in \REF{hayagusii9c} show that some kinship terms do take the augment in both Haya and Ekegusii. Note that these kinship terms are modified by possessive pronouns to identify the two kins involved. Thus, the grammars of the two languages dictate that proper names and most kinship terms occur without the augment and augmented kinship terms take a possessive modifier. Moreover, the absence of the augment in these nouns is also conditioned by the semantics since kinship terms and proper names have identifiable referents. Hence, the absence of the augment in such nouns signals their referential role. \citet{longobardi1994reference} showed a similar situation regarding the behavior of Italian articles in proper names. I return to the semantic and pragmatic properties of the augment in \sectref{sec:choti:5}.\end{sloppypar}


\subsection{Augment in complex and compound nouns}
This subsection explores the behavior of the augment in complex and compound nouns in Haya and Ekegusii. The relevant complex nouns are those denoting the idea `having/possessing' while compound nouns comprise an agentive noun and its object or complement. Haya uses the phrase ‘owner(s) of x’ for a complex noun denoting the idea `having/possessing'. On the other hand, Ekegusii expresses the same idea with the structure ‘owner(s) x’. Nonetheless, in both languages the augment is required on both elements of the complex noun. The data in \REF{hayagusii10} illustrate the marking of the augment in Haya and Ekegusii complex nouns:

\ea Augment in complex nouns\smallskip
\label{hayagusii10}
  \ea\label{hayagusii10a}	Haya:\hphantom{isit} 		o-mukama w’e-nju\hphantom{mu} 		‘house-owner’
  \ex\label{hayagusii10b} 	Haya:\hphantom{isit}		o-mukama w’e-mbwa\hphantom{i}		‘dog-owner’
  \ex\label{hayagusii10c} 	Haya:\hphantom{isit}		o-mukama w’eichumu 	‘spear-owner’
  \ex\label{hayagusii10d} 	Ekegusii: 	o-monyene e-nyomba	 	‘house-owner’
  \ex\label{hayagusii10e} 	Ekegusii: 	o-monyene e-sese\hphantom{tmu} 		‘dog-owner’
  \ex\label{hayagusii10f} 	Ekegusii: 	o-monyene ri-itimo\hphantom{tu}	 	‘spear-owner’
  \z
\z
The apostrophe (’) in (\ref{hayagusii10a}--\ref{hayagusii10c}) indicates vowel coalescence (or vowel deletion) that occurs across a morpheme boundary as a hiatus resolution strategy in Haya and other Bantu languages. The compound nouns examined consist of an agentive deverbal noun and its common noun complement. In these structures, the first element of the compound (i.e. agentive noun) takes the augment but the second element loses the augment in both Haya and Ekegusii as seen in \REF{hayagusii11}:

\ea\label{hayagusii11} Augment in compound nouns\smallskip
  \ea\label{hayagusii11a} 	Haya:\hphantom{isit}		o-mulye njoka	(*e-njoka = snake)\hphantom{oats}		 ‘snake-eater’
  \ex\label{hayagusii11b} 	Haya:\hphantom{isit}		o-muteme miti	 (*e-miti = trees)\hphantom{ntoats}		 ‘tree-cutter’
  \ex\label{hayagusii11c} 	Haya:\hphantom{isit}		o-mukame mbuzi (*e-mbuzi = goats)\hphantom{s}		 ‘goat-milker’
  \ex\label{hayagusii11d} 	Ekegusii: 	o-mori ng’iti (*chi-ng’iti = snakes)\hphantom{ats}	 	‘snake-eater’
  \ex\label{hayagusii11e} 	Ekegusii: 	o-motemi mete (*e-mete = trees)\hphantom{ioats}		 ‘tree-cutter’
  \ex\label{hayagusii11f} 	Ekegusii: 	o-mokami mbori (*chi-mbori = goats)	‘goat-milker’
  \z
\z

\subsection{Augment in predicative and associative constructions}
A predicative construction is a noun or adjective that follows a linking verb and provides information about the subject of the sentence \citep{aarts2011oxford}. In Bantu languages, the associative construction refers to the structure ‘of + noun/possessive pronoun’ that functions to express possession or association between two nouns, one being the head noun and the other a modifier in a prepositional phrase. The modifier noun acts as a complement of the preposition ‘of’. In this subsection, we examine the behavior of the augment in nouns that follow linking verbs and ‘of’ in predicative and associative constructions, respectively. The data in \REF{hayagusii12} illustrate the marking of the augment in Haya and Ekegusii predicative constructions:

\ea Augment in predicative constructions\smallskip\\
\label{hayagusii12}
  \ea\label{hayagusii12a}	 Haya:\hphantom{isit} 		Wenene n’o-mukama.		‘He/she is a king.’
  \ex\label{hayagusii12b} 	Haya:\hphantom{isit} 		Muta n’o-mushaija.\hphantom{na}		‘Muta is a man.’
  \ex\label{hayagusii12c} 	Haya:\hphantom{isit} 		abaana n’a-bageni.\hphantom{ma}		‘The children are visitors.’
  \ex\label{hayagusii12d} 	Ekegusii: 	Ere n’o-morwoti.\hphantom{amal}		‘He/she a king.’
  \ex\label{hayagusii12e} 	Ekegusii: 	Sese n’o-mosaacha.\hphantom{na}		‘Sese is a man.’
  \ex\label{hayagusii12f} 	Ekegusii: 	Abaana n’a-bageni.\hphantom{na}		‘The children are visitors.’
  \z
\z
In \REF{hayagusii12}, the linking verb `to be (is, are)' occurs as \textit{n}\,+\,vowel, but this vowel is deleted or coalesces with that of the augment whereas in slow speech, this vowel is realized as a replica of the augment vowel. In the literature, the augment in this environment is called a \textit{latent augment} (e.g. \citealt{blois1970augment}). The data in \REF{hayagusii12} show that the augment occurs in predicative constructions in both languages.

The data in \REF{hayagusii13} show that the augment in Haya and Ekegusii is also retained in associative constructions. Both the head noun (first noun) and the modifier (second noun and complement of the preposition) must take the augment. In the second noun, it is realized as a latent augment in Ekegusii (this is also the case in Haya but in faster speech):

\ea Augment in associative constructions\smallskip
\label{hayagusii13}
  \ea\label{hayagusii13a} 	Haya:\hphantom{isit} 		{o-muti gwa a-matunda}\hphantom{na} 	‘3-a/the fruit tree’
  \ex\label{hayagusii13b} 	Haya:\hphantom{isit} 		{e-nju ya a-bageni}\hphantom{imwana}		‘9-a/the guest house’
  \ex\label{hayagusii13c} 	Haya:\hphantom{isit} 		{e-kyakulya kyo o-mwana}	 ‘7-a/the child’s food’
  \ex\label{hayagusii13d} 	Ekegusii: 	{o-mote bw’a-matunda}\hphantom{ina}		‘3-a/the fruit tree’
  \ex\label{hayagusii13e} 	Ekegusii: 	{e-nyomba y’a-bageni}\hphantom{iana}		‘9-a/the guest house’
  \ex\label{hayagusii13f} 	Ekegusii: 	{e-ndagera y’omwana}\hphantom{tana}		‘7-a/the child’s food’
  \z
\z


\subsection{Augment in adverbial nouns}
In the literature, adverbial nouns are nominals that normally function grammatically as adverbs to modify verbs. Normally, such nouns provide a range of information including location, time, and manner. This subsection deals with the behavior of the augment in these nominals. First, I will consider locative nouns. The data in \REF{hayagusii14} demonstrate the behavior of the augment in locative nouns in Haya and Ekegusii, respectively:\pagebreak

\ea Absence of augment in locatives\smallskip
\label{hayagusii14}
  \ea\label{hayagusii14a} 	Haya:\hphantom{isit} 		omukitanda (e-kitanda = bed)\hphantom{neld}		‘in bed’
  \ex\label{hayagusii14b} 	Haya:\hphantom{isit} 		aamwiga (o-mwiga = a river)\hphantom{nield}	 	‘at the river’
  \ex\label{hayagusii14c} 	Haya:\hphantom{isit} 		ommukibanja (e-kibanja = a field) 	‘on the field’
  \ex\label{hayagusii14d} 	Ekegusii: 	borere (o-borere = a bed)\hphantom{i= a field}		‘in bed’
  \ex\label{hayagusii14e} 	Ekegusii: 	nyomba (e-nyomba = a house)\hphantom{itld}	 ‘in the house’
  \ex\label{hayagusii14f} 	Ekegusii: 	rooche (o-rooche = a river)\hphantom{ia field} 		‘at the river’
  \z
\z
In \REF{hayagusii14}, Haya and Ekegusii locative nouns drop the augment, but Haya substitutes the augment with locative prefixes that express such meanings as ‘in’, ‘on’, and ‘at’. There are no such locative prefixes in Ekegusii locative nouns.

As for the grammatical function of expressing time or temporal information, the data in \REF{hayagusii15} show that both Haya and Ekegusii adverbial nouns of time drop the augment as well. Note that Haya examples have adverbial prefixes that are lacking in the Ekegusii data:

\ea Absence of augment in temporal nouns\smallskip
\label{hayagusii15}
  \ea\label{hayagusii15a}	 Haya:\hphantom{isit} 		ombwankya (o-bwankya = a morning)\hphantom{g} 	‘in the morning’
  \ex\label{hayagusii15b} 	Haya:\hphantom{isit} 		ombwaigoro (o-bwaigoro = an evening)	‘in the evening’
  \ex\label{hayagusii15c} 	Haya:\hphantom{isit}		omukiro (e-kiro = a night)\hphantom{t= an evening}			‘at night’
  \ex\label{hayagusii15d} 	Ekegusii: 	mambia (e-mambia = a morning)\hphantom{vening}		‘in the morning’
  \ex\label{hayagusii15e} 	Ekegusii: 	morogoba (o-morogoba = an evening)\hphantom{ig}	‘in the evening’
  \ex\label{hayagusii15f} 	Ekegusii: 	botuko (o-botuko = a night)\hphantom{tan evening} 			‘at night’
  \z
\z

In Haya and Ekegusii, adverbial nouns that express manner omit the augment. Unlike locative and temporal nouns, nouns that denote manner in Haya do not have adverbial prefixes. Consider the data in \REF{hayagusii16}:

\ea Absence of augment in adverbial nouns of manner\smallskip
\label{hayagusii16}
  \ea\label{hayagusii16a} 	Haya:\hphantom{isit} 		bwango (o-bwango = speed)\hphantom{ftculty}		‘quickly’
  \ex\label{hayagusii16b} 	Haya:\hphantom{isit} 		gumisye (o-kuguma = difficulty)\hphantom{iy)}	 ‘firmly’
  \ex\label{hayagusii16c} 	Haya:\hphantom{isit} 		burikiro (e-kiro = a day)\hphantom{a difficulty}		‘daily’
  \ex\label{hayagusii16d} 	Ekegusii:  	bwango (o-bwango = speed)\hphantom{iiculty}		‘quickly’
  \ex\label{hayagusii16e} 	Ekegusii: 	botambe (o-botambe = length)\hphantom{utty} 	‘always’
  \ex\label{hayagusii16f} 	Ekegusii: 	bokong’u (o-bokong’u = difficulty) 	‘firmly, hard’
  \z
\z

\subsection{Augment in deverbal nouns (gerunds and infinitives)}
	Deverbal nouns are nouns derived from verbs, for example, gerunds and infinitives. In the literature, a gerund is a noun derived from a verb that retains some verb-like properties such as taking a direct object and adverbial modifiers. The infinitive is the form of the verb with \textit{to}, or its equivalent, in front of the verb or prefixed to the verb. In Haya and Ekegusii, the equivalent of \textit{to} is the prefix \textit{ku}- and \textit{ko-/go}-, respectively. The gerund in the two languages takes the augment /o/ before the infinitive prefix \textit{ku}-. Thus, in Haya and Ekegusii grammars, the augment occurs in gerunds but not in infinitives. Both gerunds and infinitives in the two languages occupy grammatical roles of nouns such as subject and object in a sentence. The gerunds in \REF{hayagusii17} take the augment in subject position but the infinitives in \REF{hayagusii18} occupying the same position do not:

\ea Augment in subject gerunds\smallskip
\label{hayagusii17}
  \ea\label{hayagusii17a} 	Haya:\hphantom{isit} 		o-kunyama ni kurungi. 	‘Sleeping is good.’
  \ex\label{hayagusii17b} 	Haya:\hphantom{isit} 		o-kuimba kwawa.\hphantom{rungi}		‘Singing has ended.’
  \ex\label{hayagusii17c} 	Haya:\hphantom{isit} 		o-kushoma kwabanza\hphantom{ii}		‘Studying has started.’
  \ex\label{hayagusii17d} 	Ekegusii:  	o-korara n’okuya.\hphantom{iungi}		‘Sleeping is good.’
  \ex\label{hayagusii17e} 	Ekegusii: 	o-goteera kwaerire.\hphantom{ngt}		‘Singing has ended.’
  \ex\label{hayagusii17f} 	Ekegusii: 	o-gosoma gwachakire.\hphantom{i} 	‘Studying has started.’
  \z
\z

\ea Absence of augment in subject infinitives\smallskip
\label{hayagusii18}
  \ea\label{hayagusii18a} 	Haya:\hphantom{isit}	    	kuimba ge ne kipaji\hphantom{vanchetre}			‘to sing well is a talent’
  \ex\label{hayagusii18b} 	Haya:\hphantom{isit}	    	kunyama ni kurungi\hphantom{iancheire}			‘to sleep is good’
  \ex\label{hayagusii18c} 	\gll Haya:\hphantom{isit}	    	{kushoma omusauti}\hphantom{wanchetre} 	{‘to read aloud is acceptable’}\\
                            {}  nkuikirizibwa {}\\
  \ex\label{hayagusii18d} 	Ekegusii: 	goteera buya n’ekeegwa\hphantom{heire}		‘to sing well is a talent'
  \ex\label{hayagusii18e} 	Ekegusii: 	korara mbuya\hphantom{riogi wancheire}				‘to sleep is good’
  \ex\label{hayagusii18f} 	Ekegusii: 	gosoma n’eriogi ngwancheire		‘to read aloud is acceptable’
  \z
\z
The data in (\ref{hayagusii17}--\ref{hayagusii18}) indicate that the augment is obligatory in gerunds but dropped in infinitives. I appears that the behavior of the augment distinguishes between gerunds and infinitives in Haya and Ekegusii. It is possible to argue that the variation of the augment in \REF{hayagusii17} vs. \REF{hayagusii18} is due to the syntactic position of subject. However, the data in (\ref{hayagusii19}--\ref{hayagusii20}) reveal the same pattern when gerunds and infinitives occur in object position. The augment is obligatory in object gerunds and dropped in infinitives parallel to the forms in (\ref{hayagusii17}--\ref{hayagusii18}) involving subject position.

\ea Augment is object gerunds\smallskip
\label{hayagusii19}
  \ea\label{hayagusii19a} 	Haya:\hphantom{isit}		tata nayenda o-kuchumba kwange 	‘Dad likes my cooking’
  \ex\label{hayagusii19b} 	Haya:\hphantom{isit}		nayenda o-kuzina kwawe\hphantom{i kwangi}		‘I’ve liked your singing’
  \ex\label{hayagusii19c} 	Haya:\hphantom{isit}		ninyenda o-kunyama muno\hphantom{wange}		‘I like a lot of sleeping’
  \ex\label{hayagusii19d} 	Ekegusii: 	tata nanchete o-koruga kwane\hphantom{nge} 		‘Dad likes my cooking’
  \ex\label{hayagusii19e} 	Ekegusii: 	nanchire o-gotera kwao\hphantom{ia kwange}		‘I’ve liked your singing’
  \ex\label{hayagusii19f} 	Ekegusii: 	ning’ncheti o-korara okonge\hphantom{tange}		‘I like a lot of sleeping’
  \z
\z

\ea Absence of augment in object infinitives\smallskip
\label{hayagusii20}
  \ea\label{hayagusii20a} 	Haya:\hphantom{isit} 		Samia nayenda kuzina burikiro. 	‘Samia likes to sing daily.’
  \ex\label{hayagusii20b}	Haya:\hphantom{isit} 		Samia nayenda kunyama.\hphantom{rikiro.} 		‘Samia likes to sleep.’
  \ex\label{hayagusii20c} 	Haya:\hphantom{isit} 		Samia yabanza kushoma.\hphantom{iiikiro.} 		‘Samia has started to read.’
  \ex\label{hayagusii20d} 	\gll Ekegusii: 	{Moraa nanchete gotera}\hphantom{burtktro} 	{‘Moraa likes to sing}\\
                                {} botambe. {always.’}\\
  \ex\label{hayagusii20e} 	Ekegusii: 	Moraa natagete korari.\hphantom{burokiro}		‘Moraa want to sleep’
  \ex\label{hayagusii20f} 	Ekegusii: 	Moraa ochakire gosoma.\hphantom{orikiro} 		‘Moraa has started to read.’
  \z
\z

The data sets in (\ref{hayagusii17}--\ref{hayagusii20}) affirm that Haya and Ekegusii gerunds take the augment while infinitives do not. In addition, gerunds take determiners such as possessive pronouns (\ref{hayagusii19a}--\ref{hayagusii19b}, \ref{hayagusii19d}--\ref{hayagusii19e}) while infinitives take adverbial modifiers such as \textit{burikilo} ‘daily’ \REF{hayagusii20a} and \textit{botambe} ‘always’ \REF{hayagusii20d}.  This means that gerunds exhibit more noun-line properties while infinitives exhibit more verb-like properties. Yet, both deverbal nouns may occupy subject and object positions.

\subsection{Noun+modifier ‘one’, ‘two’, ‘each’, ‘every’, ‘other’/‘another’, ‘all’}
The behavior of the augment in nouns and/or their determiner or modifiers may also reveal some of its typological properties \citep{blois1970augment}. I examine the behavior of the Haya and Ekegusii augments in nouns and  modifiers such as numerals ‘one’ and ‘two’, ‘each’ or ‘every’, ‘(an)other’, ‘all’, ‘many’, ‘few’, and ‘whole’. The examples in \REF{hayagusii21} illustrate the status of the augment in these forms in both Haya and Ekegusii:

\ea Augment in noun+modifier constructions\smallskip\\
\label{hayagusii21}
              \hphantom{abc }Haya				\hspace{1ex}\hphantom{eimbe e-kindi }Ekegusii			\hphantom{ie e-kemo }Gloss
  \ea\label{hayagusii21a}	e-kikombe kimo\hphantom{edi}\hspace{1ex}		e-gekombe e-kemo     		‘one cup’
  \ex\label{hayagusii21b}	e-bikombe bibili\hphantom{edi}\hspace{1ex} 		e-bikombe bibere\hphantom{tt} 		‘two cups’
  \ex\label{hayagusii21c}	e-kikombe e-kindi\hspace{1ex}\hspace{1ex}	 	e-gekombe kende\hphantom{ii}	 	‘another cup’
  \ex\label{hayagusii21d}	e-bikombe e-bindi\hspace{1ex}\hspace{1ex}		e-bikombe binde\hphantom{nt}	 	‘other cups’
  \ex\label{hayagusii21e} buli kikombe\hphantom{k kera}\hspace{1ex}			kera e-gekombe\hphantom{mt}		‘every cup’
  \ex\label{hayagusii21f}	e-bikombe byona\hphantom{ia}\hspace{1ex}		e-bikombe bionsi\hphantom{ni} 		‘all cups’
  \ex\label{hayagusii21g}	e-bikombe bingi\hphantom{mi}\hspace{1ex} 		e-bikombe e-binge\hphantom{i}		‘many cups’
  \ex\label{hayagusii21h}	e-bikombe bike\hphantom{iiaa}\hspace{1ex} 		e-bikombe bike igo		‘few cups’
  \ex\label{hayagusii21i}	e-kikombe kyona\hphantom{ii}\hspace{1ex} 		e-gekombe gionsi\hphantom{io}		‘whole cup’
  \z
\z
The forms in \REF{hayagusii21} show four similarities and five differences in the behavior of the Haya augment and Ekegusii augment. In both, the augment is marked only on the head noun and not on the modifiers ‘two’ \REF{hayagusii21b}, ‘all’ \REF{hayagusii21f}, ‘few’ \REF{hayagusii21h}, and ‘whole’ \REF{hayagusii21i}. However, in the noun+‘one’ \REF{hayagusii21a} and noun+‘many’ \REF{hayagusii21g} NPs, the Ekegusii augment appears on both the noun and the modifier whereas the Haya augment occurs only on the noun. Additionally, in (\ref{hayagusii21c}--\ref{hayagusii21d}), the Haya augment occurs on both the noun and ‘(an)other’ while the Ekegusii augment occurs only on the noun. In \REF{hayagusii21e}, the noun and ‘every’ omit the augment in Haya while only ‘every’ omits it in Ekegusii.

The other noun+modifier constructions pertinent to the analysis of the augment involve adjectives and relative clauses for which examples are given in \REF{hayagusii22} show the behavior of the augment in these contexts.

\ea Augment in noun+adjective, noun+relative clause constructions\smallskip\\
\label{hayagusii22}
  \hphantom{abc }Haya				\hphantom{ombe e-kili enja }Ekegusii 			\hphantom{be e-kenene }Gloss
  \ea\label{hayagusii22a}	e-kikombe e-kiango\hphantom{ji}\hspace{.5ex}		e-gekombe e-kenene\hphantom{i}		‘big cup’
  \ex\label{hayagusii22b} e-bikombe e-biango\hphantom{ji}\hspace{.5ex}		e-bikombe e-binene\hphantom{ti}		‘big cups’
  \ex\label{hayagusii22c}	 e-kikombe e-kili enja\hspace{.5ex}		e-gekombe kere isiko		‘the cup that is outside’
  \ex\label{hayagusii22d} e-bikombe e-bili enja\hspace{.5ex}		e-bikombe bire isiko\hphantom{a}		‘the cups that are outside’
  \z
\z
In \REF{hayagusii22}, the augment is compulsory on the noun and adjective in both languages (\ref{hayagusii22a}--\ref{hayagusii22b}). However in noun+relative clause NPs (\ref{hayagusii22c}--\ref{hayagusii22d}), the Ekegusii augment is marked only on the noun but not on the relative clause while the Haya augment appears on both the noun and the relative clause.

	In Haya and Ekegusii, modifiers or determiners may function as pronouns, occuring in an NP without the head noun. Thus, the modifier acts as the head of the noun phrase in the absence of the noun. These forms are also significant in understanding the behavior of the augment. The data in \REF{hayagusii23} show the behavior of the augment in Haya and Ekegusii pronominals:

\ea Augment in pronominals\smallskip\\
\label{hayagusii23}
              \hphantom{abc }Haya			\hspace{1ex}\hphantom{a byona}Ekegusii 		\hphantom{ons}Gloss
  \ea\label{hayagusii23a} byona\hphantom{ibyona}\hspace{1ex}			bionsi\hphantom{bande}			‘8-all’
  \ex\label{hayagusii23b} 	e-bingi\hphantom{byont}\hspace{1ex}			e-binge\hphantom{onsa}			‘8-many’
  \ex\label{hayagusii23c}	e-bike\hphantom{ byona}\hspace{1ex}			e-bike igo\hphantom{aa} 		‘8-few’
  \ex\label{hayagusii23d}	e-biango\hphantom{ttnp}\hspace{1ex} 		e-binene\hphantom{naa}		‘8-big’
  \ex\label{hayagusii23e}	e-bili enja\hphantom{ipa}\hspace{1ex}		e-bire isiko\hphantom{a} 		‘8-the ones that are outside’
  \ex\label{hayagusii23f}	e-bindi\hphantom{byont}\hspace{1ex}		 	e-binde\hphantom{ionsi}			‘8-other’
  \ex\label{hayagusii23g}	bibili\hphantom{a byona}\hspace{1ex}			bibere\hphantom{bionsi}			‘8-two’
  \ex\label{hayagusii23h}	byona byona\hspace{1ex}		binde bionsi 		‘8-any’
  \ex\label{hayagusii23i}	e-bi\hphantom{na byona}\hspace{1ex}			e-bi\hphantom{ie bionsi}			‘8-these’
  \ex\label{hayagusii23j}	e-byange\hphantom{ona}\hspace{1ex}		ebiane\hphantom{iionsi}			‘8-mine’
  \z
\z
In \REF{hayagusii23}, Haya and Ekegusii exhibit the same pattern. In both languages, ‘all’ \REF{hayagusii23a}, ‘two’ \REF{hayagusii23g}, and ‘any’ \REF{hayagusii23h} do not take the augment while the rest of the pronominals do. It is noteworthy that Ekegusii ‘an/other’ and the relative clause do not take the augment in the presence of the head noun in (\ref{hayagusii21c}--\ref{hayagusii21d}) and \REF{hayagusii22d} but as pronominals they do as in \textit{e-bire} \textit{isiko} \REF{hayagusii23e} and \textit{e-binde} ‘other’ \REF{hayagusii23f}. In \REF{hayagusii23}, the augment signals specificity since pronominals have identifiable antecedents. However, some nominals occur without the augment in \REF{hayagusii23} due to idiosyncratic properties.

\section{Semantic and pragmatic properties of the augment in Haya and Ekegusii}\label{sec:choti:5}
I noted earlier that the retention vs. deletion of the augment in other Bantu languages correlates with definiteness/specificity vs. indefiniteness/non-specificity, as in Dzamba (e.g. \citealt{bokamba1971specificity}). The relevant contexts include negative constructions, questions, noun+`any’ constructions, vocatives, emphatic nouns, focalized nouns, and proverbs. Some of these contexts create clear non-specific readings while others create both non-specific and specific interpretation, depending on the context (i.e. pragmatics). A specific interpretation obtains when a noun denotes a particular referent and a non-specific reference when the noun refers to a general class \citep[§4]{lyons1999definiteness}. The two kinds of readings are possible in both transparent and opaque contexts. In transparent contexts, ambiguity between a specific and non-specific reading does not involve scope relations. Conversely, opaque contexts involve scope relations created by operators such as negation, questions, verbs of propositional attitude (e.g. want, believe, hope, intend), conditionals, modals, and future tense \citep[166--78]{lyons1999definiteness}. This investigation includes negation and interrogation as opaque contexts. This next subsection focuses on these contexts.

\subsection{Augment in affirmative vs. negative constructions}\label{sec:choti:5.1}
Negation is one of the operators that create opaque contexts, i.e. contexts in which a specific and non-specific interpretation are possible \citep{lyons1999definiteness}. To determine the behavior of the augment in negative constructions, we must also examine its behavior in affirmative contexts as well. In Haya and Ekegusii, the augment is obligatory in affirmative constructions (\ref{hayagusii24a}--\ref{hayagusii24c}) but absent in negative constructions (\ref{hayagusii24d}--\ref{hayagusii24f}). The data in \REF{hayagusii24} illustrate this variation of the augment:

\ea Augment in affirmative vs. negative constructions\smallskip\\
\label{hayagusii24}
             \hphantom{abc }Haya			\hphantom{ombe kiliyo }Ekegusii			\hphantom{be nkere oo }Gloss
  \ea\label{hayagusii24a}	e-kikombe kiliyo\hspace{1ex} 	e-gekombe nkere oo		‘there is a cup’
  \ex\label{hayagusii24b}	n’e-kikombe\hphantom{iliyo}\hspace{1ex}		n’e-gekombe\hphantom{kere oo}			‘it’s a/the cup’
  \ex\label{hayagusii24c}	nina o-muyo\hphantom{iliyi}\hspace{1ex} 		nimbwate o-moyio\hphantom{tt}		‘I have a/the knife’
  \ex\label{hayagusii24d}	taliyo kikombe\hphantom{yi}\hspace{1ex} gekombe nkeiyo\hphantom{i oo}		‘no cup’
  \ex\label{hayagusii24e}	ti kikombe\hphantom{kiloyo}\hspace{1ex}		tari gekombe\hphantom{keri oo}			‘not a cup’
  \ex\label{hayagusii24f}	tiina muyo\hphantom{kiloyo}\hspace{1ex}		timbwati moyio\hphantom{it ot}		‘I don’t have any knife’
  \z
\z
In affirmative constructions (\ref{hayagusii24a}--\ref{hayagusii24c}), the augment is required in Haya nouns such as \textit{e-kikombe} ‘cup’ and \textit{o-muyo} ‘knife’ and Ekegusii nouns \textit{e-gekombe} ‘cup’ and \textit{o-moyio} ‘knife’ take the augment. The same nouns drop the augment in corresponding negative constructions (\ref{hayagusii24d}--\ref{hayagusii24f}). The Haya and Ekegusii facts in (\ref{hayagusii24d}--\ref{hayagusii24f}) align with the Kinande data that led \citet{progovac1993non} to conclude that bare nouns in Kinande behave as negative polarity items (NPIs). Beyond the NPI view, the bare forms in \REF{hayagusii24} receive a non-specific interpretation in that they describe a class of entities as opposed to specific entities while the augmented forms are ambiguous between a non-specific and specific interpretation. Disambiguation of the augmented forms will depend on the context and are thus subject to the principles of pragmatics. For example, if Ekegusii speaker A asks B, “Which one between the cup and the knife do you want?” B may reply, “It’s the cup” (\textit{n’egekombe}). In this context, \textit{e-gekombe} `the cup' refers to a particular cup, identifiable to both A and B.  However, if A is in a different room, hears an object fall in the kitchen where B is and asks B, ``What fell?” B may respond by saying, “It’s a cup” (\textit{n’egekombe}). In this context, B describes the type of object that fell but not a specific cup identifiable to both speakers. Therefore, augmented forms in Haya and Ekegusii may describe types of entities or pick out particular ones. In \sectref{sec:choti:5.3} below, I show that the augment is required in emphatic nouns occurring in negative constructions because they refer to specific referents and occupy syntactically salient positions such as topic and contrastive focus.

\subsection{Augment in interrogatives vs. declaratives}\label{sec:choti:5.2}
Besides exploring the behavior of the augment in interrogative constructions, I will also examine its behavior in declarative constructions.  The data in \REF{hayagusii25} demonstrate that the augment is omitted in Haya and Ekegusii interrogatives, respectively:

\ea Absence of the augment interrogatives\smallskip\\
\label{hayagusii25}
             \hphantom{abc }Haya				\hphantom{azi ki? (*o-musigazi) }Ekegusii			\hphantom{ ki? (*e-gekombe) }Gloss
  \ea\label{hayagusii25a}	mbuzi ki? (*e-mbuzi)\hphantom{sigazi}		mbori ki? (*e-mbori)\hphantom{iombe} 		‘which goat?’
  \ex\label{hayagusii25b}	kikombe ki? (*e-kikombe)\hphantom{i} 	gekombe ki? (*e-gekombe) 	‘which cup?’
  \ex\label{hayagusii25c}	mwana ki? (*o-mwana)\hphantom{itzi} 	mwana ki? (*o-mwana)\hphantom{tbe} 	‘which child?’
  \ex\label{hayagusii25d}	musigazi ki? (*o-musigazi) 	momura ki? (*o-momura)\hphantom{t} 	‘which boy?
  \ex\label{hayagusii25e}	ichumu ki? (*e-ichumu)\hphantom{azi} 	itimo ki? (*ri-itimo)\hphantom{komoe}		‘which spear?’
  \ex\label{hayagusii25f}	nju ki?	(*e-nju)\hphantom{o-musigazt}		nyomba ki? (*e-nyomba)\hphantom{io}	‘which house?’
  \z
\z
In \REF{hayagusii25}, the interrogative morpheme is \textit{ki} ‘which’ in both languages. Haya nouns \textit{mbuzi} ‘goat’, \textit{kikombe} ‘cup’, \textit{mwana} ‘child’, and \textit{musigazi} ‘boy’ drop the augment in interrogative constructions similar to the Ekegusii nouns \textit{mbori} ‘goat’, \textit{gekombe} ‘cup’, and \textit{momura} ‘boy.' Speakers use the questions in \REF{hayagusii25} to seek information on the identity of the nouns’ referents. Thus, the referents of the non-augmented nouns are not known by the speaker, i.e. are non-specific. I showed in \REF{hayagusii24} that augmented forms are ambiguous between a specific and non-specific reading. The forms in \REF{hayagusii26} show that the augment is compulsory in positive declaratives.\largerpage

\ea Augment in positive declaratives\smallskip\\
\label{hayagusii26}
             \hphantom{abc }Haya				\hphantom{t e-kikombe kiange}\hspace{1ex}Ekegusii			\hphantom{ekombe kiane}Gloss
  \ea\label{hayagusii26a}	egi ne e-mbuzi yange\hphantom{tae}\hspace{1ex} 		eye n’e-mbori	yaane\hphantom{ne}		‘this is my goat’
  \ex\label{hayagusii26b}	eki ne e-kikombe kiange\hspace{1ex}	 eke n’e-gekombe kiane	 ‘this my cup’
  \ex\label{hayagusii26c}	ogu no o-mwana wange\hphantom{i}\hspace{1ex} 	oyo n’o-mwana one\hphantom{ine}		‘this my child’
  \ex\label{hayagusii26d}	ogu no o-muyo wange\hphantom{ge}\hspace{1ex}   	oyo n’o-moyio one\hphantom{ame}		‘this is my knife’
  \ex\label{hayagusii26e}	eli ne e-ichumu lyange\hphantom{te}\hspace{1ex} 	eri ne-ri-itimo riane\hphantom{ine}		‘this is my spear’
  \ex\label{hayagusii26f}	egi ne e-nju yange\hphantom{iiange}\hspace{1ex}		eye n’e-nyomba yane\hphantom{m}	 	‘this is my house’
  \z
\z
In the literature, demonstratives and possessives are analyzed as characteristically definite (e.g. \citealt[§3]{lyons1999definiteness}). Hence, the use of ‘this’ and ‘my’ in \REF{hayagusii26} implies that the augmented nouns are definite and refer to specific referents. Consequently, the two kinds of determiners help disambiguate the augmented forms. In \REF{hayagusii26}, the augment combines with the demonstrative ‘this’ and possessive ‘my’ to mark definiteness or specificity in Haya and Ekegusii. In this context, definite NPs refer to specific referents.



\subsection{Augment in emphatic nouns}\label{sec:choti:5.3}
Constituents of a sentence or utterance that occupy topic and contrastive focus positions are treated as emphatic in the literature. Therefore, these elements receive linguistic prominence of different kinds depending on the language (e.g. \citealt{gundel2004topic}). The general view is that \textit{topic} is \textit{given information} that ranks higher in the referentiality or specificity scale. The same is true for constituents in the contrastive focus position. Contrastive focus refers to material that the speaker calls to the hearer’s attention and that normally stands in contrast with other entities that might fill the same position \citep[181]{gundel2004topic}. Therefore, constituents in topic and contrastive focus positions receive linguistic and referential emphasis. \citet{zimmermann2008contrastive} explains that speakers “use additional grammatical marking, e.g., intonation contour, syntactic movement, clefts, or morphological markers to signal contrastive focus.” In addition, this special marking corresponds with emphatic marking in descriptive and typological accounts in some languages. Topic and contrastive focus are relevant to the analysis of the behavior of the augment. I showed earlier that the augment in Haya and Ekegusii undergoes deletion in negative constructions. However, nouns in topic and contrastive focus positions retain the augment in the context of negation. I posit that the two languages use the augment as an emphatic marker and argue that emphatic nouns express specific reference. The data in \REF{hayagusii27} illustrate this pattern:\largerpage

\ea Augment in topic and contrastive focus NPs\smallskip\\
\label{hayagusii27}
 \hphantom{abc }Haya				\hphantom{ia eliinya}\hspace{1ex}Ekegusii		\hphantom{-ngoko,i}Gloss
  \ea\label{hayagusii27a}	\gll e-mbuzi,		e-mbori,  	{‘the goat, I have not seen it’}\\
                              tinkagiboine\hphantom{ta}\hspace{1ex} tindanyerora\hphantom{oko} {}\\
  \ex\label{hayagusii27b}	\gll e-kikombe, 		e-gekombe,\hphantom{goko}  	{‘the cup, it is not broken’}\\
                              tikyatikile\hphantom{inya}\hspace{1ex} tikerateka {}\\
  \ex\label{hayagusii27c}	\gll o-mwana, 		o-mwana,  	{‘the child, it is not up yet’}\\
                              takaimukile\hphantom{ya}\hspace{1ex} tarabooka\hphantom{ingoko} {}\\
  \ex\label{hayagusii27d}	\gll takalesile  	taragora  	‘s/he has not bought that dog’\\
                                {embwa eliinya\hspace{1ex}} {e-sese	 eria}\hphantom{ngoko} {}\\
  \ex\label{hayagusii27e}	\gll {taina enkoko,}\hphantom{a}\hspace{1ex} 	{tabwati e-ngoko,} 	{‘s/he does not have a chicken, s/he}\\
                                        {aine embwa}         {esese abwate}               {has a dog’}\\
 \z
\z
In (\ref{hayagusii27a}--\ref{hayagusii27c}), the nouns for ‘goat’, ‘cup’, and ‘child’ function as discourse topics and for this reason must retain the augment in the context of negation. These nouns are also left dislocated to show that they are in topic position. Examples (\ref{hayagusii27d}--\ref{hayagusii27e}) illustrate the retention of the augment in contrastive focus positions in spite of negation. In \REF{hayagusii27d}, the nouns for ‘dog’ are in contrastive focus with something else not included in the discourse. In \REF{hayagusii27e}, the nouns for ‘chicken’ and ‘dog’ are in contrastive focus positions. The data in \REF{hayagusii27} confirm that Haya and Ekegusii use the augment as a morphological marker of topic and contrastive focus. In these contexts, the augment encodes emphasis in the relevant constituents. Besides morphological and syntactic marking, these emphatic nouns receive phonological prominence in the two languages (this point is not explored).

\subsection{Augment in vocatives}
Vocatives refer to “phrases used in direct address” \citep[152]{lyons1999definiteness}. Proper nouns denoting persons, kinship terms, and second person pronouns typically function as vocatives. Some accounts treat vocative as grammatical case and many languages have special vocative forms. Lyons explained that there is a great tendency for vocatives to be bare or exhibit morphological minimality. This tendency appears to obtain in Haya and Ekegusii where common nouns in vocative case drop the augment. The data in \REF{hayagusii28} illustrate this behavior in the augment:\largerpage

\ea Absence of augment in vocatives\smallskip\\
\label{hayagusii28}
             \hphantom{abc }Haya			\hphantom{nusigazi, ija aa}\hspace{.5ex}Ekegusii			\hphantom{wa, inchwo aa }Gloss
  \ea\label{hayagusii28a}	 {iwe mwana, ija aa}\hphantom{ip}\hspace{.5ex} 	{aye mwana, inchwo aa}\hphantom{a} 	{‘you child, come here’}\\
  \ex\label{hayagusii28b}	\gll {inywe baana, ija aa}\hphantom{i}\hspace{.5ex}  	{inwe baana, inchwo aa}\hphantom{a} 	{‘you children, come}\\
                                  {} {} here’\\
  \ex\label{hayagusii28c}	iwe musigazi, ija aa\hspace{.5ex} 	aye momura, inchwo aa 	‘you boy, come here’
  \ex\label{hayagusii28d}	iwe, mwisiki, ija aa\hphantom{i}\hspace{.5ex} 	aye moiseke, inchwo aa 	‘you girl, come here’
  \z
\z
In \REF{hayagusii28}, the Haya nouns \textit{mwana} ‘child’, \textit{baana} ‘children’, \textit{musigazi} ‘boy’, and \textit{mwisiki} ‘girl’ and the Ekegusii mwana ‘child’, \textit{baana} ‘children’, \textit{momura} ‘boy’, and \textit{moiseke} ‘girl’ lose the augment in the vocative function. Given that referents of vocative phrases are contextually identifiable, the logical conclusion is that the augment is not needed in this context to signal specificity. The behavior of the augment in vocatives resembles its behavior in proper names and kinship terms derived from common nouns (see \sectref{sec:choti:4.1}).

\subsection{Augment in noun+`any’ constructions}
The determiner or pronoun ‘any’ is used to refer to one or some of a thing or number of things and to express a lack of restriction in selecting one of a specified class. This means that nouns modified by ‘any’ receive non-specific interpretation. In English, \textit{any} is also used in questions (e.g., \textit{Do you have any money?}) and negative constructions (e.g., \textit{I don’t have any money}). Both contexts imply non-specificity. I also showed that negative (\sectref{sec:choti:5.1}) and interrogative (\sectref{sec:choti:5.2}) constructions disallow the augment in Haya and Ekegusii on the same grounds. The non-specific interpretation inherent in ‘any’ makes it significant from the perspective of semantics. The augment in Haya and Ekegusii behave alike in noun+‘any’ phrases. In both languages, the nouns modified by ‘any’ lose the augment, as in the examples in \REF{hayagusii29}:

\ea Augment in noun+‘any’ constructions\smallskip\\
\label{hayagusii29}
             \hphantom{abc }Haya			\hphantom{nbe kyona kyona}\hspace{1ex}Ekegusii			\hphantom{ikende gionsi}Gloss
  \ea\label{hayagusii29a}	mbuzi yona yona\hphantom{yona}\hspace{1ex}		mbori ende yonsi\hphantom{ionsi}		‘any 9-goat’
  \ex\label{hayagusii29b}	kikombe kyona kyona\hspace{1ex} 	gekombe kende gionsi		‘any 7-cup’
  \ex\label{hayagusii29c}	mwana wena wena\hphantom{ina}\hspace{1ex}		mwana onde bwensi\hphantom{si}		‘any 1-child’
  \ex\label{hayagusii29d}	mugeni wena wena\hphantom{mt}\hspace{1ex} 		mogeni onde bwensi\hphantom{si}		‘any 1-guest’
  \ex\label{hayagusii29e}	ichumu lyona lyona\hphantom{na}\hspace{1ex}		itimo rinde rionsi\hphantom{ionsi}		‘any 5-spear’
  \ex\label{hayagusii29f}	nju yona yona\hphantom{a kyona}\hspace{1ex}			nyomba ende yonsi\hphantom{nsi}		‘any 9-house’
  \z
\z
In \REF{hayagusii29}, the Haya nouns \textit{e-mbuzi} ‘goat’, \textit{e-kikombe} ‘cup’, \textit{o-mwana} ‘child’, and \textit{o-mugeni} ‘guest’ drop the augment similar to their Ekegusii counterparts \textit{e-mbori} ‘goat’, \textit{e-gekombe} ‘cup’, \textit{o-mwana} ‘child’, \textit{o-mogeni} ‘guest’, \textit{ri-itimo} ‘spear’ and \textit{e-nyomba} ‘house’. The omission of the augment suggests that the bare nouns are non-specific in their reference, which is reinforced by ‘any’. Note that in both languages, ‘any’ occurs after the head noun and takes agreement prefixes. The data in \REF{hayagusii29} typify a kind of non-specificity agreement between the bare noun and ‘any’.

\subsection{Augment in proverbs}
The inclusion of proverbs in this study stems from the fact that nouns used in proverbs do not have specific referents. Instead, such nouns denote a class of entities. In both Haya and Ekegusii, the augment is absent in nouns used in proverbs. Consider the data in \REF{hayagusii30}:

\ea Absence of augment in proverbs\smallskip
\label{hayagusii30}
  \ea\label{hayagusii30a} Haya:\hphantom{isit}		njubu elagile teyata bwato \\
\hphantom{Ekegusii: }‘a hippo that is full does not break a boat’
  \ex\label{hayagusii30b}	Haya:\hphantom{isit}		balezi babili baliza mwana \\
\hphantom{Ekegusii: }‘two baby sitters make the baby to cry’
  \ex\label{hayagusii30c}	Ekegusii: 	mominchoria imi tang’ana mosera ibu \\
\hphantom{Ekegusii: }‘one who braves the dew is incomparable with one who\\
\hphantom{Ekegusii: }stirs ash’
  \ex\label{hayagusii30d}	Ekegusii: 	mwana obande mmamiria makendu\\
\hphantom{Ekegusii: }‘someone’s child is cold mucus’
  \z
\z

In (\ref{hayagusii30a}--\ref{hayagusii30b}), the Haya nouns \textit{njubu} ‘hippo’, \textit{bwato} ‘boat’, \textit{balezi} ‘baby sitters’, and \textit{mwana} ‘child’ occur without the augment in the two proverbs. Similarly, Ekegusii nominals \textit{mominchoria imi} ‘one who braves dew’, \textit{mosera ibu} ‘one who stirs ash’, \textit{mwana} ‘child’, \textit{mamiria} ‘mucus’, and \textit{makendu} ‘cold’ omit the augment. The bare nouns in \REF{hayagusii30} do not have specific referents. Therefore, the data in \REF{hayagusii30} provide additional evidence that the absence of the augment reflects the non-specific interpretations in particular contexts.

\section{Augment as an article}\label{sec:choti:6}
The previous sections have shown that the behavior of the augment in Haya and Ekegusii is consistent with articles in other languages. There is no evidence in the Bantu literature that negates this observation. Therefore, the terms \textit{augment}, \textit{initial vowel}, and \textit{preprefix} used to describe this formative are a misnomer and thus misleading. In this section, I explain some of the key properties of articles evident in the augment. In reference to the two English articles \textit{the} and \textit{a}, \citet[36]{lyons1999definiteness}  defines an article thus:

\begin{exe}
\ex Definition of \textsc{article}\label{31}\\
    The basic unmarked nature of \textit{the} and \textit{a}, with their minimal semantic content [+Def] and [+Sg] respectively, reflected in their phonological weakness and default behavior, I shall take to be what defines the term article.
\end{exe}
In \REF{31}, the features [+Def] and [+Sg] represent respectively \textit{definite} and \textit{singular} and thus characterize English \textit{the} and \textit{a} as definite and cardinality articles. \citeauthor{lyons1999definiteness}’s definition also identifies phonological weakness as a core property of articles. This property means that phonologically, articles are dependent on adjacent elements and are mostly monosyllabic \citep{giusti1997categorial}. Besides, articles are also morphologically dependent or bound (i.e., clitics, affixes), form closed classes, inflect for number, gender and case, occur in NPs, and correlate with (in)definiteness and/or (non-)specificity \citep{giusti1997categorial}. The augment reveals these traits in Haya and Ekegusii.

I highlight five properties of articles evident in the Haya and Ekegusii augments. First, in both languages, the augment occurs as a monosyllabic prefix whose phonological shape depends on the vowel of the class prefix. This underlines the fact that the augment is phonologically weak and dependent on the adjacent host similar to articles in other languages. Cross-linguistic evidence shows that articles are prone to phonological reduction processes, with the article and the host forming word-like units that function as a full lexical form. Second, the augment in Haya and Ekegusii occurs as a bound morpheme, not an infrequent quality in articles across languages. \citet[63]{lyons1999definiteness} explains that articles may exist as either independent words (e.g. English the) or bound morphemes. Bound articles may occur as clitics (e.g. Spanish \textit{el} in \textit{el hombre} ‘the man’) or affixes (e.g. Romanian \textit{-ul} in \textit{om-ul} ‘man-the’). Therefore, the augment shares the property of bound morphemes with articles of languages such as Spanish, Romanian, Arabic, and Hausa \citep{lyons1999definiteness}. Therefore, the augment is best treated as a bound article. These facts reveal the flaw in using the imprecise terms augment, preprefix, and initial vowel to describe this formative, making it look like a foreign element that has no equivalents in other human languages.

The third characteristic of articles observed in the behavior of the augment is that it occur as a single formative with variants that constitute a closed class in both Haya and Ekegusii. This is a common property of functional categories such as determiners. In Haya, the augment has three variants /a/, /e/, and /o/ and in Ekegusii five variants /a/, /e/, /o/, /ri/, and /chi/. The variants of the augment exhibit identical behavior across various contexts similar to variants of articles in other languages such as a and an of the English indefinite, cardinality article. The limited number of variants of the Haya and Ekegusii augments parallels that of articles in languages that have them. While some languages have no articles (e.g. Swahili, Latin, and most Slavic languages), other languages have one article (usually the definite article, e.g. Bulgarian and Modern Greek) and yet others have two (e.g. English)  \citep{giusti1997categorial,lyons1999definiteness}. The fourth property of articles found on the augment is that it inflects for number, gender, and case; the augment in Haya and Ekegusii alternates to indicate singular/plural distinctions, noun class (or gender), and case (e.g. locative and vocative). The fifth attribute of articles apparent on the augment is its association with (non-)specificity and (in)definiteness. The augment is absent in proper names, kinship terms, and vocatives because these nouns are definite and specific as they pick out specific referents. Elsewhere, the omission of the augment signals a non-specific reading, as in negative constructions, questions, proverbs, and nouns modified by ‘any’. The augment is compulsory in emphatic nouns in topic and contrastive focus positions to mark specificity. In neutral or transparent contexts, the marking of the augment is ambiguous between a specific and non-specific reading but contextual variables help disambiguate the NPs in question.

\section{Summary and conclusion}\label{sec:choti:7}
This article has shown many similarities and a few differences between the Haya and Ekegusii augments, determined the grammatical, semantic, and pragmatic properties of the augment, and highlighted the properties of article found in the augment. The common shapes of the augment are vowels (V) /a/, /e/, and /o/ in both languages though Ekegusii also has the CV shape in /ri/ and /chi/. In both languages, the augment is not marked on proper names, most kinship terms, and vocative nouns because these are definite and specific in reference. Augmented nouns in both languages are ambiguous between a specific and non-specific reading in transparent contexts. Adverbial nouns of location, time, and manner omit the augment in both languages. They also require the augment in predicative and associative constructions. In complex nouns, both elements of the NP take the augment in Haya and Ekegusii. However, in compound nouns, only the first element is augmented in both languages. Haya and Ekegusii allow the augment in gerunds but not in infinitives. The languages exhibit some similarities and differences in the marking of the augment in the head noun and its modifiers. Most pronominals require the augment in the two languages, but omit it in interrogative and negative constructions to signal non-specific reference. In both languages, affirmative statements require the augment but the meaning of the noun varies between a specific and non-specific interpretation. Emphatic nouns in topic and contrastive focus positions require the augment to mark emphasis and specificity. Nouns in proverbs and those modified by ‘any’ drop the augment to express non-specific reference inherent in these contexts.

The findings from Haya and Ekegusii confirm that the augment is indeed a bound article. The augment in these languages exhibits five properties found in articles in other languages. First, it occurs as a monosyllabic prefix whose phonological shape depends on the vowel of the class prefix. Second, as a prefix, the augment is morphologically dependent on its host. Third, the various shapes of the augment in Haya and Ekegusii constitute a closed class. Fourth, both languages use the augment to mark the grammatical properties of number, gender, and case. Lastly, the augment interacts with semantic and pragmatic principles to express respectively definite and (non-)specific interpretations.

\printbibliography[heading=subbibliography,notkeyword=this]
\end{document}
