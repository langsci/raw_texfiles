\documentclass[output=paper]{langscibook}
\ChapterDOI{10.5281/zenodo.6393748}
\author{Yan Cong\affiliation{Michigan State University} and Deo Ngonyani\affiliation{Michigan State University}}
\title{A syntactic analysis of the co-occurrence of stative and passive in Kiswahili}
\abstract{This study concerns the co-occurrence of stative and passive in Kiswahili. The co-occurrence is only possible with an intervening applicative suffix and in the order \textsc{st-appl-pass}. There are two readings of the stative extension in Kiswahili, potential and resultative. The study seeks to account for the co-occurrence, the order of the suffixes, and the two interpretations of the stative. Our findings are consistent with the [VoiceP [ApplP  [vP [VP]]]] structure. We argue that passive and stative share the same essential structure [Voice, Appl, v]. As to the derivation, we propose syntactic head movement where V moves to the stative head resulting in [V-\textsc{st}], which moves to the applicative yielding [V-\textsc{st-appl}], and finally moves to voice to form [V-\textsc{st-appl-pass}]. Last, but not least, our account connects stative with \textit{patient-manner} predicates to derive resultative reading, and \textit{agent-manner} predicates to derive potential reading.}


\begin{document}
\tikzset{every tree node/.style={align=center,anchor=north}}
\SetupAffiliations{mark style=none}
\maketitle

\section{Introduction}
This study examines the co-occurrence of stative \textit{-ik-} and passive \textit{-w-} in Kiswahili. The two are part of derivational morphology commonly known in Bantu linguistics as verb extensions \citep{Guthrie1962a}. The following sentences illustrate the contrast between active, passive and stative clauses.  


 \begin{exe}
    \ex\label{sellmilk}{
    \begin{xlist}
    \ex{
    \gll m-toto a-li-mwag-a ma-ziwa.\\
    1-child \textsc{1sm-pt}-spill-\textsc{fv} 6-milk\\
    \glt `the child spilled the milk' \hfill(\textsc{active})}
    \ex{
    \gll ma-ziwa ya-li-mwag-\textit{w}-a na m-toto.\\
    6-milk \textsc{6sa-pst}-spill-\textsc{pass-fv} with 1-child\\
    \glt `the milk was spilled by the child' \hfill(\textsc{passive})}
    \ex{
    \gll ma-ziwa ya-li-mwag-\textit{ik}-a.\\
    6-milk \textsc{6sa-pst}-spill-\textsc{st}-\textsc{fv}
    \\
    \glt `the milk was spilled' 
    \hfill(\textsc{stative})}
    \end{xlist}}
\end{exe}

Sentence (\ref{sellmilk}a) is the active sentence with two arguments, the agent \textit{mtoto} `child' in subject position, and the theme \textit{maziwa} `milk' in object position. The verb in the passive construction (\ref{sellmilk}b) is marked with the passive suffix \textit{-w-}, and has the theme in subject position while the agent is an oblique object. The stative suffix \textit{-ik-} marks the verb in (\ref{sellmilk}c) where the theme is in the subject position, and there is no agent. 

The passive has two allomorphs in Kiswahili, \textit{-ew-} and \textit{-w-}. The stative may be realized as \textit{-ik-}, \textit{-ek-}, or \textit{-k-}\footnote{In this paper we do not include \textit{-ik-} in Chewa that is identified by \citet{Simango:2009aa} as causative disguised as stative as in \textit{gona} `sleep' and \textit{goneka} `lay someone down'. This increases the valency. It is also different from Kiswahili impositive according to \citet{8603069}, as in \textit{choma} `stab' and \textit{chomeka} `insert'. This affix in Kiswahili does not change the argument structure.}. These morphosyntactic changes are summarized in \tabref{con_ngo:table1}.\largerpage

\begin{table}
\begin{tabularx}{\textwidth}{l QQQ}
 \lsptoprule
  & Active & Passive & Stative\\
\midrule
 Valence & 2 & −1 & −1\\
  Thematic selection & [agent, theme]  & [theme, agent] & [theme]\\
 Allomorphs   &   &\textit{ew,w} & \textit{ik,ek,k}\\
 External argument &  & suppressed & deleted\\
 Two similarities   &  & (i) logical object becomes subject   (ii) logical object triggers subject agreement & (i) logical object becomes subject   (ii) logical object triggers subject agreement\\
  \lspbottomrule
 \end{tabularx}
    \caption{Features of passives and statives\label{con_ngo:table1}}
\end{table}

As indicated in \tabref{con_ngo:table1}, the active verb selects two arguments: agent and theme. Both passives and statives change the verb's valence by reducing the external argument (henceforth EA). They both select the theme. Notice that here theme refers to the prototypical object. The theme becomes the subject of the derived sentence. In examples (\ref{sellmilk}b) and (\ref{sellmilk}c) above, the subject triggers the subject prefix \textit{ya-} which agrees with \textit{maziwa} `milk’.  

On the surface, passives and statives share similarities in that they “eliminate” EA, promote the logical object to the subject position, and trigger subject agreement. However, there are fundamental differences that call for a closer look. One of the differences is in interpretation. There are two readings of the stative extension \textit{-ik-} in Kiswahili: “potential” denoting activity or state; and “resultative” indicating accomplishment or achievement (\ref{sellmilk}c) (\citealt{Simango:2009aa,Levin:1993aa,vendler67}, a.o.). Such readings are not available for passives. Illustrated in (\ref{swahili1a}) is an example showing the co-occurrence.

\begin{exe}
\ex\label{swahili1a}
\gll a-li-mwag-\textit{ik}-i-\textit{w}-a   maji\\
\textsc{1sm-pst}-spill-\textsc{st-app-pass-fv}  water\\
\glt `he got water spilled on him'
\end{exe}

Another difference is that the passive does not delete the external argument and allows it to be expressed as an oblique object while the stative appears to eliminate the EA altogether. Thus, it appears the passive selects the theme and the agent, while the stative selects the theme only. Also intriguing is the fact that the passive and the stative can co-occur, as in (\ref{swahili1a}). Since both are derived from transitive verbs, the derivation of both calls for an explanation. 

This paper's main claim is that although both passives and statives appear to eliminate EA, this happens in different steps of the derivation. This claim is built upon the assumption that various mergers occur at different steps. In other words, where you get passive merger, stative merger is not expected. The different readings of statives are also derived from different statives. Furthermore, it is only the resultative stative but not the potential stative that can co-occur with passives in Kiswahili. Resultative statives are, in fact disguised causatives. With this fine-grained sub-categorization in mind, a larger amount of data can provide better predictions. By adopting \citet{Hale:2002aa}, we propose that the stative extension licensing subject promotion is captured as patient-manner predication; while the stative extension blocking subject promotion is analyzed as agent-manner predication. 
 
\section{Previous studies}
The differences between passive and stative have been a subject of much interest among Bantuists. In this section, we tap into the insights from previous studies on meaning, modularity,  argument structure, and the co-occurrence of passive, stative, and applicatives. 

\subsection{Meanings}
While the passive does not present a range of meanings in Bantu, the stative generates a range of meanings that have led to its being referred to also as neuter or neutro-passive according to \citet[179]{8603069}. The stative in Kiswahili is associated with two meanings, namely, state and potential \citep{Ashton-1947,Polome-nd,8603069}. These are illustrated by the following examples from \citet[227--228]{Ashton-1947}.

\begin{exe}
\ex\label{ash}
\begin{xlist}
\ex{ 
\gll ki-kombe ki-me-vunj-ik-a\\
7-cup 7\textsc{sm}-\textsc{prf}-break-\textsc{st}-\textsc{fv}\\
\glt `The cup is broken' \hfill stative
}
\ex{ 
\gll Kazi hii ya-fany-ik-a\\
9.work9 this 9\textsc{sm}-do-\textsc{st}-\textsc{fv}\\
\glt `This work can be done' \hfill potential
}
\end{xlist}
\end{exe}
(\ref{ash}a) denotes a state of affairs resulting from some event. (\ref{ash}b) refers to the possibility or potential of the event taking place. While the past tense and perfect generally generate stative readings, present tense often leads to potential reading. The present tense often leads to ambiguity due to the availability of both interpretations. 

The interpretation of the stative is also sensitive to aspect. Using Vendler's aspectual types of verbs \citep{vendler67}, it is possible to discern which verbs receive which interpretation. Vendler classified verbs according to two dimensions, namely, whether or not the verb denoted a process event and whether or not the event had an end point.

\begin{itemize}
    \item States such as \textit{know, want}, and \textit{understand}, are characterized by duration of time with no change. There is no inherent endpoint. 
    \item Activities such as \textit{carry, run, dance} also take place over a duration of time and do not have an inherent endpoint.
    \item Accomplishments like \textit{build (a house), draw (a figure), make (a structure)} are processes that change a state over time with an endpoint. 
    \item Achievements, e.g., \textit{break, arrive, win}, are events that happen at a particular moment. They involve a clear endpoint, but they are not processes.
\end{itemize}
These types are presented in Table \ref{con_ngo:table2}. 

\begin{table} 
\fittable{\begin{tabular}{lll}
\lsptoprule
  & +process & −process \\
 \midrule
 +end point & \textsc{accomplishment}  & \textsc{achievement}  \\
            & \hspace{1ex} (e.g., \textit{kujenga} `to build') & \hspace{1ex} (e.g., \textit{kuvunja} `to break')\\
 −end point &  \textsc{activity} (e.g., \textit{kubeba} `to carry') & \textsc{state} (e.g., \textit{kujua} `to know') \\
 \lspbottomrule
\end{tabular}}
    \caption{Aspectual types of verbs (based on \citealt{vendler67})\label{con_ngo:table2}}
\end{table}

Potential readings are associated with verbs denoting events that do not have an endpoint \citep{Dubinsky:1996}. Therefore, verbs expressing activities and state have potential reading in terms of event types when the stative is attached. By contrast, state readings are derived from stativized verbs denoting events with endpoints. Those readings are related to accomplishments and achievements, both of which express a change of state.  We shall refer to such reading as “resultative" and use “state" for Vendler's type of verb. The “stative" is the verbal suffix or the construction with such a verb. In the stative construction, the agent of the action does not appear, as examples in (\ref{simango091}) demonstrate in Chichewa, a Bantu language related to Kiswahili. 

\begin{exe}
\ex Chichewa \citep[122]{Simango:2009aa} \label{simango091}

\begin{xlist}
\ex{
\gll Shuko a-dza-thyol-a ndodo \\
Shuko \textsc{1.Agr-fut}-break-\textsc{fv} 9.stick\\
\glt `Shuko will break a stick'
}
\ex{
\gll Ndodo i-dza-thyo-k-a\\
9.stick \textsc{9.agr-fut}-break-\textsc{st-fv} \\
\glt `A stick will break' 
 }
\end{xlist}

\end{exe}
While (\ref{simango091}a) asserts that someone will break a stick, (\ref{simango091}b) merely asserts that the event will come to pass without implying that an agent will be involved in bringing about the event, which is what the stative suffix \textit{-ik-} means. 

\citet{SeidlDimitriadis2003} present evidence from aspect and argue that the stative generates a middle construction or impersonal construction.\largerpage

\begin{exe}
\ex\label{sd03}\citep[248]{SeidlDimitriadis2003}
\begin{xlist}
\ex{ 
\gll Ch-akula ki-li-kuwa ki-me-pik-ik-a sana\\
7-food 7\textsc{sm-pst}-be 7\textsc{sm-prf}-cook-\textsc{st-fv} very\\
\glt `The food was being much cooked'
}
\ex{ 
\gll Ki-tabu ki-li-kuwa ki-me-zungumz-ik-a sana katika mi-aka ya 70\\
7-book 7\textsc{sm-ps}-be 7\textsc{sm-prf}-discuss-\textsc{st-fv} very in 4-year 4.of 70s\\
\glt `The book was being discussed much in the seventies.'\\
}
\end{xlist}
\end{exe}
The eventive verbs \textit{pika} `cook' and \textit{zungumza} `speak' are combined with the stative suffix to create attributive readings. \citet{Dubinsky:1996} examine stative and passive constructions in Chichewa. The Chichewa facts draw parallels with the differences between English adjectival passives and verb passives. The stative is similar to adjectival passives. As in English, the question is whether the two are derived similarly. The interaction with tense and aspect provide some clues regarding its derivational history. 

\subsection{Argument structure}
One of the insights provided by the previous studies contrasting passive and stative in Bantu is that although the passive and the stative are considered detransitivizing affixes, they differ in a very fundamental way in that while the passive suppresses the logical subject, the stative deletes it \citep{Dubinsky:1996,Mchombo:1993aa,SeidlDimitriadis2003}. The deletion is observed in several constructions. One such feature is the availability of the agent or logical subject. The passive allows for the agent to be expressed as an oblique object. No such oblique object is available for stative, as illustrated in (\ref{mchombo93-7}).

\begin{exe}
\ex Chichewa \citep[7]{Mchombo:1993aa} \label{mchombo93-7}
\begin{xlist}
\ex{
\gll Mbûzi zi-na-pínd-á mǎta.\\
10.goats 10\textsc{sm-pst}-bend-a 6.bows \\
\glt `The goats bent the bows.'
}
\ex{ 
\gll Ma-ǔta a-na-pindidw-á ndí mbûzi. \\
6-bows 6-\textsc{sm-pst}-bend-\textsc{pass-fv} by 10.goats    \\
\glt `The bows were bent by the goats.'
}
\ex{ 
\gll Ma-ǔta a-na-pínd-ík-a (*ndí mbûzi).\\
6-bows 6-\textsc{sm-pst}-bend-\textsc{pass-fv} (*by 10.goats)\\
\glt `The bows got bent (*by the goats).'
}
\end{xlist}
\end{exe}

Both passive (\ref{mchombo93-7}b) and stative (\ref{mchombo93-7}c) have the theme as the subject. But only the passive allows the \textit{by}-phase. A \textit{by}-phrase would be ungrammatical as (\ref{mchombo93-7}c) shows. 

\citet{Mchombo:1993aa} further demonstrates that passive constructions bear an implicit argument while there is not such implicit argument in stative constructions. This difference can also be shown in Kiswahili using subject-oriented adverbs and purpose clauses.  

\begin{exe}
\ex\label{collapse}
\begin{xlist}
\ex[]{ 
\gll Wa-li-vunj-a jengo makusudi\\
2\textsc{sm-pst}-demolish-\text{fv} 5.building deliberately\\
\glt `They deliberately demolished the building'
}
\ex[]{ 
\gll Jengo li-li-vunj-w-a makusudi\\
5.building 5\textsc{sm-pst}-demolish-\textsc{pass-fv} deliberately\\
\glt `The building was deliberately demolished.'
}
\ex[*]{ 
\gll Jengo li-li-vunj-ik-a makusudi\\
5.building 5\textsc{sm-pst}-demolish-\textsc{st-fv} deliberately\\
\glt `The building deliberately collapsed.'
}
\end{xlist}
\end{exe}
Sentence (\ref{collapse}b) with the adverb `deliberately' implies an agent exists. The adverb relates to the agent. On the contrary (\ref{collapse}c) is ungrammatical precisely because there is no agent associated with the event. Often the stative is used in impersonal constructions which have no agent.  

In his study of the stative in Chichewa, \citet{Mchombo:1993aa} claims that the stative construction seems to fall in with the class of unaccusatives. Supporting evidence comes from its tolerance of locative inversion, as illustrated in (\ref{mchombo30}). 

\begin{exe}
\ex\label{mchombo30}\citep[18]{Mchombo:1993aa}
\begin{xlist}
\ex{
\gll Zitseko zi-a-pind-\textit{ik}-a mu chitsime.\\
8doors \textsc{8sm-prf}-bend-\textsc{st-fv} 18in 7waterhole\\
\glt `The doors have got bent in the waterhole (well)' }
\ex{
\gll M'chitsime mu-a-pind-\textit{ik}-a zitseko.\\
18.7.waterhole \textsc{18sm-prf}-bend-\textsc{st-fv} 8:doors\\
\glt `In the waterhole some doors have got bent'
}
\end{xlist}
\end{exe}

The two sentences are built on the stative verb \textit{pindika} `get bent'. The subject in (\ref{mchombo30}a) is \textit{zitseko} `doors,' which is the theme of the verb. This Class 8 noun has triggered the subject agreement on the verb (8SM). In (\ref{mchombo30}b), on the other hand, the subject is not the theme. The location \textit{m'chitsime} `in the waterhole'  (Class 18) is the subject triggering the subject marker 18SM on the verb. The theme is in the postverbal position.  
 
In a sense, this locative inversion example makes the stative less of an isolated phenomenon. This is because it shows that the stative does behave like an unaccusative. To the extent that the inverted locative functions as the subject, the grammatical subject needs not correspond with the logical subject. Thus, it appears advisable to subsume the stative into the phenomenon of unaccusativity, extending to it whatever formal apparatus the theory of grammar has deployed for dealing with unaccusatives. 
 
However, the unaccusatives are intransitive in their basic underived forms whereas the stative is a derived form. Therefore, while the statives can be accommodated within the unaccusativity phenomena, it is their derivation from the transitive verbs and their relation to them that demands the appeal to formal devices other than those employed in the analysis of unaccusativity. 

Further insights into the passive construction are obtained from \citet{Dubinsky:1996} on Chichewa. The Chichewa facts draw parallels with the differences between English adjectival passives and verb passives. The stative is similar to adjectival passives. As in English, the question is whether the two are derived similarly. Using a modular theory of grammar, they suggest that although the two derivations are thematically similar, they are produced in two formally distinct operations. They examine passive and stative constructions in Chichewa and seek to motivate two distinct types of verbal extension. In their analysis, passive alters mapping from arguments to grammatical functions (GFs) while stative performs an entirely analogous operation on the lexical conceptual structure (LCS) itself. The stative has changed the lexical conceptual structure of the verb. The passive, on the other hand, does not change the lexical conceptual structure of the verb. It is an operation on the mapping of arguments to the grammatical functions. 

The important lesson from this study is that although passives and statives look somewhat similar on the surface, they exhibit syntactic properties that point to the operations being carried out in different modules. We believe that this has a bearing on the syntax and the semantics of the stative morpheme \textit{-ik-}. 

\subsection {The ordering of the suffixes}
The wealth of extensions and combinations raise the questions of how the affixes are ordered and what principles underlie the affixes' ordering. \citet{Ngonyani:2016aa} addresses two questions on Kiswahili verb extensions: (i) what is the order of the extensions in relation to the applicative, and (ii) how can the order be accounted for. The article seeks to establish the positions of extensions relative to the applicative in Kiswahili;  and to determine the extent to which the semantic scope can account for the pairwise combinations with applicatives. Using data from the Helsinki Corpus of Kiswahili, the following combinations are discovered.  

\begin{table}
\begin{tabular}{ llll }
 \lsptoprule
  \multicolumn{2}{c}{Applicative first} & \multicolumn{2}{c}{Applicative last}   \\
 \midrule
 \textsc{appl-caus} & Yes & \textsc{caus-appl} & Yes \\
 \textsc{appl-pass} & Yes & \textsc{pass-appl} & No\\
 \textsc{appl-rec} & Yes & \textsc{rec-appl} & Yes\\
 \textsc{appl-rev} & No & \textsc{rev-appl} & Yes\\
 \textsc{appl-stat} & No & \textsc{stat-appl} & Yes\\
 \lspbottomrule
\end{tabular}
    \caption{A summary of the attested and unattested pairwise combinations \citep[65]{Ngonyani:2016aa}\label{table3}}
\end{table}

Table \ref{table3} shows that the search for pairwise combinations of extensions with the applicative revealed three distinct patterns. Pattern one shows that the applicative can appear in a variable affix order with the causative and reciprocal extensions. The second pattern shows that the applicative appears after the reversive and does not precede the reversive. The third pattern shows  that the applicative appears after the stative suffix and before the passive. The study further supports the Mirror Principle \citep{Baker1985} and the semantics scope hypothesis \citep{rice2000}. 

These three empirical observations provide three arguments supporting the syntactic-semantic account. First, the variable order is attributable to variable scopal relationships. Further, regarding the reversive, it must appear between the root and the applicative. Last, the third argument results from the different positions of the stative and applicative extensions, both of which suppress the agent. Essentially, the passive promotes the applied object. By contrast, the stative promotes the direct object. This corresponds to the passive scope, including the applicative verb, whereas the stative has a narrower scope and falls under the applicative scope. 

To sum up, previous studies have shown that there are two readings of the stative, state or result, and potentiality. The meanings are subject to the aspectual type of the verbs and tense. The studies have also shown that while the passive suppresses the external argument and making it an implicit argument, the stative eliminates it altogether. This makes it similar to middle constructions. With respect to the order of the verbal suffixes, the passive appears after the applicative, while the stative appears before the applicative.  

\section{Assumptions}
We follow the proposal by \citet{Folli:2007aa}, \citet{legate14}, and \citet{pylkkanen08} that there are two functional projections within the verb phrase, namely, VoiceP and vP. The three layers are shown in \figref{fig:rmtree}.

\begin{figure}
\caption{\label{fig:rmtree}Three layers of verb phrase}
\begin{tikzpicture}[>=latex']
\tikzset{level 1+/.style={sibling distance=0.01mm}}
\Tree [.TP [.DP ] [.T\1 [.T ] [.VoiceP [.DP ] [.Voice\1 [.Voice ] [.vP [.v ] [.VP [.V ] [.DP ] ] ] ] ] ] ]
\end{tikzpicture}
\end{figure}

The internal argument is introduced in the VP where it is assigned its theta role. The vP is the locus of causative semantics. VoiceP is the domain of the external $\theta$-role where the feature passive or active is specified. In this structure, the theme is generated in the VP, while the external argument is generated in the specifier of the vP.

We also assume that theta roles are specified in syntactic positions. We adopt the uniformity of theta assignment hypothesis (UTAH) regarding the positions of arguments \citep{Baker:1988aa}. The hypothesis states:

\begin{exe}
\ex\label{deo2}
Identical thematic relationships between items are represented by identical structural relationships between those items at the level of D-structure. \citep[46]{Baker:1988aa}
\end{exe}

According to this hypothesis, the theme in the following two sentences are generated in the same position.\largerpage[2]

\begin{exe}
\ex\label{deoIce}
\begin{xlist}
\ex John melted the ice.
\ex The ice melted away.
\end{xlist}
\end{exe}

The verb \textit{melt} has as its theme \textit{the ice} in both sentences. However, in (\ref{deoIce}a), \textit{the ice} is the object of this transitive verb. In (\ref{deoIce}b), \textit{the ice} is the subject. The subject in (\ref{deoIce}a) is the cause of the state that is expressed in (\ref{deoIce}b). The sentence with only the theme as its argument is anticausative. Anticausative constructions express a change of state. They are intransitive and are characterized by the elimination of the agent, promotion of the theme to subject position \citep{domEtal2016,Haspelmathl2016,Heidinger2015,Kulikov2011}.

The stative in Bantu languages exhibit features of anticausative \citep{domEtal2016,Gluckman2016,MallyaEtal2019}. The differences between passives and anticausatives also distinguish passives and statives. For this reason, our discussion will treat the stative as anticausative. Furthermore, we assume that the derivational suffixes for passive, stative, and applicative are syntactic heads that merge in the syntax. Several studies of Bantu verbal derivations \citep{Baker1985,Baker:1988aa,Harley2013,MallyaEtal2019,Ngonyani:2016aa,pylkkanen08,SeidlDimitriadis2003} have made a similar argument about the status of the verb extensions. 

\section{Derivation}\largerpage
This section provides details of the derivations. In particular, we attempt to address the following questions: 

\begin{enumerate}[label=(\alph*)]
    \item How the passive and stative co-occur on the verb if they both act on the same external argument?
    \item How can we explain the affix order in the co-occurrence?
    \item How do the different interpretations arise?
\end{enumerate}

In order to account for (a), we must first establish the positions of the syntactic heads. The features of \textit{v} are consistent with the stative. It selects VP and specifies causation or volition. When the passive is specified on the Voice, it suppresses the external argument. Voice selects a complement that is [+transitive]. 

Since Voice selects only transitive, it cannot select a constituent that is [−trans\-i\-tive]. The passive and the stative should not co-occur. However, consider an example where such a co-occurrence does occur. 

\begin{exe}
\ex\label{swahili1}
\gll a-li-mwag-\textit{ik}-i-w-a   maji\\
\textsc{1sm-pst}-spill-\textsc{st-app-pass-fv}  water\\
\glt `he got water spilled on him'
\end{exe}
In this example (\ref{swahili1}), the stative and passive co-occur but with an intervening applicative. The goal of the spill appears as the subject of the sentence. 

The Kiswahili example shows parallels with the English sentences built on the verb \textit{melt}. While in (\ref{deoIce}a) the verb takes the theme \textit{ice} as its object and \textit{the heat} as the causer, in (\ref{deoIce}b) the verb takes only the theme as its argument. The theme appears in the subject position in (\ref{deoIce}b). Previous studies of applicatives have establiished that the applied object is generated in a position higher than the theme or patient based on c-command relations and ellipsis \citep{Marantz1993,Ngonyani1996,pylkkanen08}. The position of the applicative in relation to the passive and the stative is shown in \figref{fig:rmtree2}.

\begin{figure}
\caption{\label{fig:rmtree2}The position of the applicative in relation to the passive and the stative} 
\begin{tikzpicture}[>=latex']
\tikzset{level 1+/.style={sibling distance=0.01mm}}
\Tree [.TP [.DP ] [.T\1 [.T ] [.VoiceP [.DP ] [.Voice\1 [.Voice\\{[Pass]}\\\textit{-w} ] [.ApplP [.DP ] [.Appl' [.Appl\\\textit{-i} ] [.vP [.v\\\textit{-ik} ] [.VP [.V\\\textit{mwag} ] [.DP\\\textit{maji} ] ] ] ] ] ] ] ] ]
\end{tikzpicture}
\end{figure}

The derivation begins with the merger of the verb \textit{mwag} `spill' with the object \textit{maji} `water'. Next, this VP merges with the v stative \textit{-ik}, moving the \textit{V} to attach to the left of the stative head forming \textit{mwag-ik}. This new stative head merges with the Appl \textit{-i} to create \textit{mwag-ik-i}. This applicative has introduced the applied object \textit{3sg} in the specifier of ApplP. This \textit{mwag-ik-i} complex moves to the Voice to pick the [−active] feature, attaching on the left of the passive head \textit{-w} to create \textit{mwag-ik-i-w}. The applied object then moves to SpecTP to satisfy the EPP requirement. The derivations is shown in \figref{fig:rmtree3}.

\begin{figure}
\caption{\label{fig:rmtree3}Derivation details}
\begin{tikzpicture}[>=latex']
\tikzset{level 1+/.style={sibling distance=0.01mm}}
\Tree [.TP [.\node(x){DP\\3sg} ; ] [.T\1 [.T ] [.VoiceP [.DP ] [.Voice\1 [.\node(a){Voice[−active]\\\textit{mwag-ik-i-w}} ; ] [.ApplP [.\node(x0){DP\\\textit{t}} ; ] [.Appl\1 [.\node(b){Appl\\\textit{t}} ; ] [.vP [.\node(c){v\\\textit{t}} ; ] [.VP [.\node(d){V\\\textit{t}} ; ] [.DP\\\textit{maji} ] ] ] ] ] ] ] ] ]
\draw[semithick, dashed, <-]  [bend right=70] (c) to (d) ;
\draw[semithick, dashed, <-]  [bend right=70] (b) to (c) ;
\draw[semithick, dashed, <-]  [bend right=70] (a) to (b) ;
\draw[semithick, dashed, <-]  [bend right=80] (x) to (x0);
\end{tikzpicture}
\end{figure}

This structure does not show the details of tense \textit{li-} and mood \textit{-a} in order not to crowd the presentation of the derivations. 

The derivation is consistent with two features highlighted in the beginning, namely, (a) the passive and stative both thought to act on the external argument may appear on a verb; and (b) the order of the verbal suffixes.  

\section{Implications}
The interpretation of the stative indicates some interesting parallels with the argument structure of transitivity alternation \citep{Hale:2002aa}. In their theory, Hale and Keyser assert that structures are characterized by two kinds of relations: head-complement relations and specifier-head relations. These relations are projected from the lexical entry of each head. Such lexically determined relations are responsible for the difference between a verb that takes a complement prepositional phrase, as in (\ref{smear}), and adjunct PP in (\ref{spill}).\largerpage

\begin{exe}
\ex\label{smear}Agent-manner
\begin{xlist}
\ex[ ]{They smeared \textit{mud} on the wall.}
\ex[*]{\textit{Mud} smeared on the wall.}
\end{xlist}
\ex\label{spill}Patient-manner
\begin{xlist}
\ex{The puppy spilled \textit{water} on the floor.}
\ex{\textit{Water} spilled on the floor.}
\end{xlist}
\end{exe}
\citet{Hale:2002aa} characterized \textit{smear} as agent-manner verb taking a complement PP with the reading `smear X on Y.' The verb includes information regarding its adverbial feature describing what the external argument does. On the other hand, the verb \textit{spill} is a patient-manner verb with semantic features expressing motion, distribution, dispersal, or attitude of the patient. This is the alternating type because its features are associated with the internal argument. 

\begin{sloppypar}
Swahili does not permit such alternation. However, the stative derivation yields patterns of readings reflecting the split between agent-manner verbs and patient-manner verbs. We use the verb \textit{ziliba} `smear' and \textit{mwaga} `spill'. 
\end{sloppypar}

\begin{exe}
\ex\label{mango}
\begin{xlist}
\ex[]{
\gll Wa-toto wa-li-zilib-a ma-tope.\\
2-child \textsc{2sm-pst}-smear-\textsc{fv} 6-mud\\
\glt `The children smeared mud on the wall.'}
\ex[]{
\gll Ma-tope ya-li-zilib-ik-a u-kuta-ni.\\
6-mud 6\textsc{sm-pst}-smear-\textsc{st-pass-fv} 11-wall-\textsc{loc}\\
\glt `Mud got smeared on the wall.'}
\ex[*]{ 
\gll Ma-tope ya-li-zilib-ik-i-a u-kuta-ni.\\
6-mud 6\textsc{sm}-\textsc{pst}-smear-\textsc{st-appl-fv} 11-wall-\textsc{loc}\\
\glt `Mud got smeared on he wall.'
}
\end{xlist}

\ex\label{mango2}
{ 
\begin{xlist}
\ex[]{ 
\gll Mbwa wa-li-mwag-a ma-ji sakafu-ni.\\
2.dog 2\textsc{sm-pst}-spill-\textsc{fv} 6-water 9.floor-\textsc{loc}\\
\glt `The dogs spilled water on the floor.'
}
\ex[]{ 
\gll Ma-ji ya-li-mwag-ik-a sakafu-ni.\\
6-water 6\textsc{sm-pst}-spill-\textsc{st-fv} 9.floor-\textsc{loc}\\
\glt `Water spilled on the floor.'
}
\ex[]{ 
\gll Ma-ji ya-li-mwag-ik-i-a sakafu-ni.\\
6-water 6\textsc{sm-pst}-spill-\textsc{st-appl-fv} 9.floor-\textsc{loc}\\
\glt `Water spilled onto the floor.'
}
\end{xlist}
}
\end{exe}
The stative in (\ref{mango}b) creates a potentiality reading for the agent-manner verb, while (\ref{mango2}b) has a resultative reading for the patient-manner verb. The addition of the applicative does result in ungrammatical form for the agent-manner verb (\ref{mango}c), and a grammatical form for the patient-manner verb (\ref{mango2}c). The applicative constructions show clearly that subject promotion is possible with patient-manner predication and not possible with agent-manner predication.
           
In the light of Hale and Keyser’s proposal, it is clear at this point that the argument structure of the stative derivation in Swahili calls for further investigation. The study of the interaction of these two verb types and Vendler’s aspectual types is likely to lead to a better understanding of the stative readings. 


\section{Conclusions}
This paper set out to examine the co-occurrence of stative \textit{-ik} and passive \textit{-w} in Kiswahili. Both reduce the valency of the verb by either suppressing the external argument or eliminating it altogether. They co-occur when there is an intervening applicative affix, and in the \textsc{st-appl-pass} order. We offer an analysis of the non-canonical argument realization and explore what this shows and explore how we can conceptualize the alternation from a cross-linguistic perspective. 

This paper argues that voice projects on top of v, and v is the head that is interacting with the external argument. This is in line with \citet{Collins:2005aa}, and \citet{johnson04} maintaining that Voice is independent of agent-hood. Voice selects for a particular vP carrying certain properties. But this does not mean actives select for vP with an external argument. Voice distinction is independent of the presence/absence of an EA. This explains the empirical observation that unaccusatives is still active in a voice sense despite of the lack of an EA, given that Voice is not the head that is responsible for any agenthood arguments.

A prediction of the current analysis is that there are two manners to derive a middle. In a regular middle where only one argument gets realized, first get the agent argument removed and then derive the middle built upon a subjectless vP. In the middle varieties where both of the arguments get realized, Appl introduces the recipient goal argument inside a low-applicative structure and then the middle is built upon a vP that gets two arguments realized via ApplP.

\section*{Abbreviations}
\begin{multicols}{3}
\begin{tabbing}
\textsc{appl}\hspace{1ex}\= applicative\kill
\textsc{appl} \> applicative\\
\textsc{fv} \> final vowel\\
\textsc{inf} \> infitival\\
\textsc{loc} \> locative \\
\textsc{om} \> object marker\\
\textsc{pass} \> passive\\
\textsc{prf} \> perfective aspect\\
\textsc{pst} \> past tense\\
\textsc{sm} \> subject marker\\
\textsc{st} \> stative
\end{tabbing}
\end{multicols}

\section*{Acknowledgments}
The language consultant is Professor Deogratias Ngonyani. Starting from (\ref{sellmilk}), all Kiswahili data points (except for those that are explicitly marked by citations) were collected in class ``FS17 LIN881 The structure of Kiswahili''. Asante Mwalimu Deo Ngonyani! We are extremely grateful to ACAL\,49 for their comments and engagement in our Q\&A session. This project has also benefited from early discussions with LIN881 classmates.

The first author would like to express deep gratitude to Professor Deogratias Ngonyani. Xiayimaierdan Abudushalam, Yuankai Chen, Adam Smolinski, and Rachel Stacey also deserve thanks, for their invaluable comments at various points of the writing of this paper. All errors are ours. 

\printbibliography[heading=subbibliography,notkeyword=this]

\end{document}
