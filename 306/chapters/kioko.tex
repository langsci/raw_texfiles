\documentclass[output=paper,colorlinks,citecolor=brown]{langscibook} 
\ChapterDOI{10.5281/zenodo.6393768}
\author{Angelina Nduku Kioko\affiliation{US International University-Africa} and Josephat Rugemalira \affiliation{Tumaini University College Dar es Salaam}}
\title{The future of the indigenous languages of Kenya and Tanzania}  
\abstract{This paper examines the language policies and practices in Kenya and Tanzania and argues that, in spite of the observable differences between these neighbouring countries, the ethnic community languages face an uncertain future. Although language policies play a role in determining this future, there are stronger forces that defy language policy, viz. population movements, urbanization, technological changes affecting mass communication, and the structure of the economies.}

\IfFileExists{../localcommands.tex}{
  \addbibresource{localbibliography.bib}
  \usepackage{langsci-optional}
\usepackage{langsci-gb4e}
\usepackage{langsci-lgr}

\usepackage{listings}
\lstset{basicstyle=\ttfamily,tabsize=2,breaklines=true}

%added by author
% \usepackage{tipa}
\usepackage{multirow}
\graphicspath{{figures/}}
\usepackage{langsci-branding}

  
\newcommand{\sent}{\enumsentence}
\newcommand{\sents}{\eenumsentence}
\let\citeasnoun\citet

\renewcommand{\lsCoverTitleFont}[1]{\sffamily\addfontfeatures{Scale=MatchUppercase}\fontsize{44pt}{16mm}\selectfont #1}
   
  %% hyphenation points for line breaks
%% Normally, automatic hyphenation in LaTeX is very good
%% If a word is mis-hyphenated, add it to this file
%%
%% add information to TeX file before \begin{document} with:
%% %% hyphenation points for line breaks
%% Normally, automatic hyphenation in LaTeX is very good
%% If a word is mis-hyphenated, add it to this file
%%
%% add information to TeX file before \begin{document} with:
%% %% hyphenation points for line breaks
%% Normally, automatic hyphenation in LaTeX is very good
%% If a word is mis-hyphenated, add it to this file
%%
%% add information to TeX file before \begin{document} with:
%% \include{localhyphenation}
\hyphenation{
affri-ca-te
affri-ca-tes
an-no-tated
com-ple-ments
com-po-si-tio-na-li-ty
non-com-po-si-tio-na-li-ty
Gon-zá-lez
out-side
Ri-chárd
se-man-tics
STREU-SLE
Tie-de-mann
}
\hyphenation{
affri-ca-te
affri-ca-tes
an-no-tated
com-ple-ments
com-po-si-tio-na-li-ty
non-com-po-si-tio-na-li-ty
Gon-zá-lez
out-side
Ri-chárd
se-man-tics
STREU-SLE
Tie-de-mann
}
\hyphenation{
affri-ca-te
affri-ca-tes
an-no-tated
com-ple-ments
com-po-si-tio-na-li-ty
non-com-po-si-tio-na-li-ty
Gon-zá-lez
out-side
Ri-chárd
se-man-tics
STREU-SLE
Tie-de-mann
} 
  \togglepaper[1]%%chapternumber
}{}

\begin{document}
\maketitle

\section{Introduction}\label{sec:kioko:1}

This paper examines the language policies and practices in Kenya and Tanzania in order to determine their impact on the past and future fortunes of the indigenous languages. The agenda is two-fold. First, it is of interest to determine which policies (if any) are more significant in the endeavour to enhance the fortunes of less powerful languages, i.e. which policies or mix of policies are more effective in the business of language promotion and preservation. Second, we consider the open possibility that the language policies notwithstanding, the powerful forces for change in the social and political systems of the world are set against the long term preservation of linguistic diversity. We show that in spite of the observable differences in language policies and practices between Kenya and Tanzania, the threats to the indigenous languages in both countries are the much more formidable forces beyond the tinkering of politicians and bureaucrats of small states. These forces include population movements, urbanization, technological changes affecting mass communication, and the structure of the economies.

\section{Hostile language policies and practices in Tanzania}\label{sec:kioko:2}

In this section we show that overall there are more policies and practices hostile to the indigenous languages in Tanzania than in Kenya. After presenting comparative language demographics for the two countries, we proceed to examine the policies and practices that appear to place a stranglehold on the local languages of Tanzania while Kenyan languages appear to get a friendlier environment.

\subsection{Demographic basics}
\tabref{tab:kioko:1} shows that only 13.3 million people, i.e. 38\% of the population of Tanzania (which was 35m in 2009) speak a language among the top ten. By contrast, 32.8 million Kenyans, i.e.  85\% of the Kenyan population (which was 38.6m in 2009) speak one of the top ten languages. This means that it is more difficult for a Tanzanian ethnic community language to break out of the large pack of small community languages -- 150 -- in total and gain respect and power. In Kenya the top ten languages constitute a quarter of the total number of languages spoken in the country – about forty altogether. Such relatively respectable numerical strength provides a base for public recognition of the languages.

\begin{table}
    \begin{tabular}{lrlr}
    \lsptoprule
    \multicolumn{2}{c}{Kenya} & \multicolumn{2}{c}{Tanzania}\\
        \midrule
         Kikuyu & 6.6 &  Sukuma & 6\phantom{.0} \\
          Luhya & 5.3 & Ha & 1.2 \\
          Kalenjin & 4.6  & Gogo & 1.0 \\
          Kamba & 3.9 & Maasai & 0.8 \\
          Luo & 4.0 & Haya & 0.8 \\
          Somali & 2.4 & Makonde & 0.8 \\
          Kisii & 2.2 & Nyakyusa & 0.7 \\
          Meru & 1.7 & Hehe & 0.7 \\
          Turkana & 1.0 & Fipa & 0.7 \\
          Maasai & 0.8 & Iraqw & 0.6 \\
        \midrule
    {Total} & {32.8} &  & {13.3} \\
    \lspbottomrule
    \end{tabular}
    \caption{Top ten languages in Kenya and Tanzania (millions of speakers). Sources: \cite{Kenya-National-Bureau-of-Statistics2009, Mradi-wa-Lugha-za-Tanzania2009}.\label{tab:kioko:1}}
\end{table}

\subsection{Languages of government} 
There is no formal recognition of indigenous languages in Tanzania. Indeed such recognition is regarded as inappropriate and undesirable for the unity of the nation. The first attempt to mention language matters in the aborted 2014 draft constitution only mentions Kiswahili as national and official language and English as the other official language. By contrast, the 2010 Kenya constitution guarantees space for indigenous languages. The primary language of government business in Tanzania is Kiswahili; in oral communication English is rather minimal. Even in formal government documents Kiswahili is edging out English. It is indeed arguable that English may still hold on in written communication because some of the government documents need to be presented to foreign governments for consultations on aid.  

Parliament is a good illustration of the struggle for turf between Kiswahili and English and a good indicator of the slow pace of change within the legal profession.  While all written laws (except the constitution which is in Swahili and English) are drafted in English, they are orally presented in summary form in Kiswahili and debated in Kiswahili; they are enacted in the English version. Yet MPs need not demonstrate competence in English, although representatives in regional assemblies, such as the East African Community \& SADC, defend their candidacy in English within parliament and may lose votes if they cannot express themselves properly. In the courts the language of oral communication is Kiswahili; oral proceedings and the verdict are in Kiswahili, but English is used in written submissions and final judgement.

All this is in stark contrast to the Kenyan scene where the primary language of government business is English and, though Kiswahili is an official language and the national language, its use in government is usually prompted by issues of access, rather than issues of policy preference. It is not normal for a Kenya government document to be written in Kiswahili. In parliament, the constitution allows a choice of either Kiswahili or English; in practice English is the language of parliamentary proceedings. Elected politicians in parliament and county assemblies must demonstrate competence in English and Kiswahili.  All written laws including the constitution are in English; bills are drafted and debated in English. In the courts English reigns supreme and Kiswahili may be available via an interpreter; in lower courts an indigenous language may be allowed via an interpreter.

The dominance of Kiswahili in Tanzania government business may have become a major obstacle to the promotion of indigenous languages because of the assumed near total accessibility of the national language. By contrast, the dominance of English in Kenya government business may be seen as a factor helping promote the indigenous languages because of the need to ensure access to government communication.

\subsection{Language in education}\largerpage
In Tanzania Kiswahili is the language of instruction (LOI) in the seven years of primary school (except for a very small section of the population who can access English medium primary schools). English is the LOI at secondary and tertiary levels;\footnote{This policy has been reaffirmed over the past several decades in spite of the perennial debate about its appropriateness and effectiveness, cf. \citet{Tanzania-Government1995, Tanzania-Government2014, QorroEtAl2012}.} it is a compulsory subject from the third year of school. Indigenous languages are banned in school.

Kenya is more accommodating of the indigenous languages, allowing them to be used as languages of instruction in the first three years, alongside with Kiswahili. English takes over as LOI from fourth grade. This friendly policy arrangement is largely thwarted in practice because English has become the preferred language of school from nursery level. 

Language in other public domains: In Tanzania’s mass media, Kiswahili predominates over English. English is sporadic in print media and insignificant in radio and TV. Indigenous languages are banned in the media \citep{TCRA2005}.\footnote{The government has not registered any indigenous language newspaper since independence even though the relevant laws have not explicitly banned such outlets \citep[68--69]{Rugemalira2013}. Papers registered before independence eventually died for various reasons – including an environment hostile to indigenous languages \citep[92]{Madumulla2007}.} Kenya, on the other hand, has a vibrant body of broadcasting in indigenous languages even though English is dominant overall in both print and electronic media. 

Kiswahili is dominant in the various forms of public entertainment (music, soap opera, stand-up comedy) in both countries, which is a good reflection of the sensitivity to market forces. English may come in second ahead of indigenous languages in this regard. The same pattern of paying attention to the structure of the market may be observed in electoral politics (campaigns for office). In both countries Kiswahili is the language for getting votes: Kenya’s political campaigns are mainly in Kiswahili in urban and cosmopolitan areas; in Kiswahili or indigenous languages in the rural areas (except presidential campaigns). English is only used on TV talk shows. In Tanzania Kiswahili is the language of politics, whereas English, while permitted, is virtually unusable. The indigenous languages however are banned, being covered by the ``divisive language'' label of campaign regulations \citep{Tanzania-Government2010}.

Language choice in product packaging, user manuals, and advertising can be a good indicator of market structure, but there is considerable sophistication here. English appears to be dominant in both countries, which may be a sign of external dependencies of the economies (in imports and exports, and historical links of the particular industry). Bilingual labeling in English and Kiswahili is sporadic – apparently without being required by government regulations.\footnote{How often have you come across a Swahili label on a bottle of water or soda, a bag of cement, or a packet of medicines?}~A study to determine the level of sophistication in audience targeting might clarify the advertizing scene somewhat: for instance, how do the cell-phone companies choose the language for their myriad promotional drives?\footnote{It is kind of fashionable to choose an English trade name: “the linguistic market is strongly influenced by English in ads, business,trade and commerce, banks, technology, TV and in other domains that are a concomitant of Western lifestyle.” \citep[64]{Legere2010}}~In this domain, in both countries, the indigenous languages are at the losing end even if more so in Tanzania. 

In the sphere of organized religion, Kiswahili is in dominant use in Tanzania – even in rural congregations.  Translations of the bible, catechism/religious instruction, and hymns in the indigenous languages have fallen into disuse.  Popular gospel music is in Kiswahili – another nod to market forces. While there are no government regulations on this, it is notable that the churches and mosques have promoted Kiswahili services. In Kenya, English and Kiswahili religious services are common in urban and cosmopolitan areas. In the rural areas indigenous languages are more prevalent while some services in Kiswahili or English may be found. Hymnal and bible translations into indigenous languages are still in popular usage even in urban areas.

\subsection{Language attitudes} 
The stigma attached to indigenous languages as “dangerously tribalistic and retrogressive” was most explicitly articulated by \citet{Nyerere1995} at his party’s conference to nominate a presidential candidate for the first multi-party elections in Tanzania. Although the point he was making about the imperative of nominating a candidate on grounds of leadership merit rather than narrow ethnic belonging is crucial, the damage to the cause of indigenous languages and cultures was colossal: in that speech he maintained that the only use left for “tribal languages” was in tribal religious ritual (“\textit{kutambika}”) and that in this age only fools would engage in tribal-based associations. Kiswahili on the other hand is the language of national unity, progress and political correctness. As for English, the ambivalence in attitudes is oppressive: it is the language of the erstwhile colonial master and the imperialist powers; but it is also the language of individual power and prestige, and anyone who amasses sufficient linguistic capital in this form is sure to go places and climb higher on the social ladder. This would account for the “show-off” phenomenon whereby some speakers engage in uncontrolled code-mixing involving English and Kiswahili, in very formal contexts regardless of the needs of the audience.\footnote{“As for language choices, a {laissez-faire} approach is observed in the recent past. Swahili still holds a strong position, but there is a tendency in public that some people tend to demonstrate their (mostly rudimentary) knowledge of English by inserting English words (articulated with a terrible Swahili accent) whenever possible. They produce a speech variety that is basically Swahili, but intended to demonstrate the status of the speaker who is eager to distinguish him/herself from others who do not dispose of an English vocabulary.” \citep[54]{Legere2010}} Still conditions for the mastery of English in the education system are tough and in communication encounters the advice for many would be avoid it as best you can lest you face humiliation.

In Kenya, there is a relatively strong and positive attachment to indigenous languages; but you speak English to colleagues and the boss at the office, Kiswahili to the house servant, and the indigenous language to your tribesman to signal belonging.

\section{Emerging patterns of promise}\label{sec:kioko:3}

There are significant differences between Kenya and Tanzania regarding the way the ethnic community languages (ECLs) are treated. In Kenya the ECLs appear to have far more habitable space, both in public and private domains than is the case in Tanzania where official policies and practices as well as private preferences appear to disregard or actively suppress these languages. The notable legal constitutional and language in education policy openings that provide relief to Kenya’s ECLs have created three potential growth poles for the languages. 

The first more vibrant pole revolves around the use of the local languages in the media. As noted in the previous section, ECL print media in Kenya has been quickly overshadowed by the fast growth of digital channels – TV and FM radio. It would be safe to say that most ECLs in Kenya can boast of at least a radio FM station. In any case the government-owned Kenya Broadcasting Corporation maintains scheduled radio programmes in many of the ECLs in Kenya. Many of the larger communities have access to several TV and radio stations. According to the records from the Communication Authority of Kenya [CAK], there are currently 64 registered “vernacular” radio FM stations and 20 registered “vernacular” TV stations (Personal Communication with CAK officials).  The growth, especially of the TV stations, is quite recent but very rapid.  This is in spite of the initial persecution of promoters of vernacular radio stations in the 1990s and early 2000, and the perception of some politicians that the vernacular radio stations are a threat to national unity.  It is important to note that there has not been any legislation prohibiting the use of ECLs in media; on the contrary, the Kenyan constitution guarantees the citizens of linguistic rights:

\begin{itemize}
    \item The constitution retains the status of Kiswahili as the National language and further elevates it to official status in addition to English.  
    \item It articulates the commitment to “promote and protect the diversity of languages of the people of Kenya [and to] promote the development and use of indigenous languages, Kenyan sign language, Braille and other communication formats and technologies accessible to persons with disabilities” (Kenya Constitution, Chapter 2, Article 7, Clauses 3a and 3b; \citealt{Republic-of-Kenya2010})

\end{itemize}
Chapter 4 has the following provisions:

\begin{itemize}
    \item The right to information held by the State or by another person which is required for the exercise or protection of any right or fundamental freedom (Article 35).
    \item The right to use the language, and to participate in the cultural life of the person’s choice (Article 44).
    \item The right to enjoy that person’s culture and use that person’s language (Article 50).
    \item The right to form, join and maintain cultural and linguistic associations and other organs of civil society (Article 54).
    \item The right of an accused person to have assistance of an interpreter without payment if the accused person cannot understand the language used  at the trial (Article 56).
\end{itemize}

The competitive atmosphere among ethnic groups promotes these outlets and the languages they serve. In the rural areas, in particular, these stations are the primary sources of information and entertainment.  In addition to the usual entertainment and popular adverts, these stations run programs with diverse content including civic education, instructions on relevant farming methods, information on health related issues, and religious instructions. 

These media outfits/firms create jobs and encourage entrepreneurs to thrive.\footnote{What may be described as the YouTube of Kenya, viz. Viusasa, provides access to original videos/films in some ECLs as well as foreign language films with subtitles or voicing in ECLs.} They are seen as part of the creative economies. They make literacy in these languages meaningful and may become a catalyst that creates demand for ECL courses in school beyond third grade. The young speakers and programmers in these stations are quite good in their use of the ECLs even though it is not clear where they polished these skills, since ECLs are not taught in the schools beyond primary three.

The second growth pole for the ECLs in Kenya is rooted in the language education policy. The policy states that during the first three years children should be taught in the language of the catchment area. In the homogeneous rural areas, where the majority of Kenyans currently live, the language of the catchment area is an ECL. In addition to use as languages of instruction in these early years, the ECLs are taught as subjects in the first three years of school.  It is expected that learners in the rural areas will develop basic literacy skills within that time.  The Kenya Institute of Curriculum Development has a department that deals with “mother tongue”.  With the liberalization of the production of learning materials, this department is expected to evaluate materials for teaching the various mother tongues.  The department works with language committees from the various language groups to ensure the development and use of standard orthographies.  

The third growth pole pertains to the religious sphere. In this respect, comparisons between Kenya and Tanzania ought to be much closer because of the absence of government regulation on the Tanzanian side, and because in both countries the relevant organized religion is Christianity since Islam does not appear to have transcended its Arabic and Kiswahili avenues. 
 
The production of new or revised versions of the bible in various ECLs is considerably vibrant, often with active community involvement. The significance of these initiatives in setting standards for orthography and formality, as well as language boundaries, has long been recognized even if inter-denominational and inter-ethnic rivalry may have created unnecessary confusion and duplication in some instances \citep[242]{Kioko2017}. For many ECLs in both Kenya and especially Tanzania, religious materials [bible, hymn book, prayer book, religious instruction book/catechism], are available mainly in print but also in digital form on smartphones or “audio proclaimers”, and religious gatherings provide the main public domain of use. 

\section{Limitations in growth poles}\label{sec:kioko:4}

Kenya’s ECL mass media, in spite of the growth promise they hold out, have to contend with the fact that they are circumscribed within an ideological frame that regards the languages as a third tier means of communication in a restricted range of domains at best, or as mere symbols for a glorified rural existence.  The restriction in terms of audience is apparent amongst ECL speakers in the city.  In the urban centres the audience consists largely of speakers above 40 years of age, but for the younger generation, these are the stations to turn on to keep visiting grandparents informed/entertained.  

Even more disturbing is the lingering suspicion that ECL can easily be transformed into lethal tools for rubble-rousers and perpetrators of genocides. Hence one radio station is accused of having fomented the 2008 killings in the Rift Valley in similar fashion as the \textit{Rwandese Radio Television Libre des Mille Collines} did during the 1994 killings in Rwanda. One of the accused six who appeared before the International Criminal Court in The Hague after the 2008 ethnic clashes in Kenya was a broadcaster with one of the ECL radio stations. Vociferous debates calling for the ban of the “vernacular” stations have also been witnessed in the Kenyan parliament despite the constitutional protection of the ECLs. For example \citet[123]{Kioko2013} notes: 

\begin{quotation}
Introducing a bill in parliament  to ban the use of indigenous languages (other than Kiswahili) in official settings on the 8th of June 2011, a law maker made the following remarks “Article 7(2) of the Constitution recognises Kiswahili and English as the official languages of the Republic of Kenya; aware this provision will address ethnic disharmony in public offices if implemented to the letter...concerned that the use of indigenous languages in public offices and national institutions is a major contributor to disharmony, suspicion, and discomfort in public offices in the country, this house urges the government to ban the use of indigenous languages in all offices”. \hfill \citep[21]{Kenya-National-Assembly-Official-Records2011}  
\end{quotation}

The language committees as the gate keepers of ECL development have generally been made irrelevant by the fact that the policy requiring the use of the mother tongue in the first three years of school is not strictly adhered to: 

\begin{quote}\sloppy
Both formal research and informal observations indicate that, in some countries of Africa, national policy regarding language of instruction is not being followed. In Kenya, for example, the national policy calls for the mother tongue or “language of the catchment area” to be used as the medium of instruction through Grade 3. However, Piper (2010) shows that in fact, the language used between 70 per cent and 80 per cent of the classroom time in Grades 1–3 is English – not Swahili and not the mother tongue of the students. This is true even in rural environments, where fluency in English was extremely low among the students.  \hfill \citep[156]{Trudell2013}
\end{quote}

Thus not much writing and publication is happening in the ECLs in spite of the policy and the measures to ensure standardization.  It would appear that the raison d’être for the Kenya language committees, viz. evaluation of school materials in ECLs, is largely disappearing. Even where a strong language committee is still operational, as is the case with the Kikuyu committee, the engagement is more on language development and modernization than on evaluating learning materials because very few pedagogical materials are getting published. 

This situation is surprisingly similar to that found in the rather popular English medium primary schools in Tanzania where children in the first two years of school are not even allowed to learn to read and write Kiswahili. The new curriculum requires that children focus on the three Rs in Standard One and Two – using Kiswahili \emph{or} English, and no other subjects are allowed:

\begin{quote}
Therefore, the development of competence in the 3Rs in English-medium schools in Standards I and II will be carried out in English. The teaching of other subjects, including Kiswahili, will be introduced in Standard III. \hfill \citep[2]{Tanzania-Government2016} 
\end{quote}\largerpage

Religious literature in ECLs is available in a very restricted domain even as it is supposed to appeal to the heart. It is understandable that in urban areas such literature would be of very limited use because of the mixed nature of congregations – except perhaps for rather ethnically skewed denominations like the Presbyterian Church of East Africa which is predominantly Kikuyu in Kenya and runs Kikuyu services even in the city. In the rural areas there ought to be more room for using ECL versions of the religious materials. In Tanzania this is demonstrably not the case \citep{Madumulla2007, MuzaleRugemalira2008, Rugemalira2013}.\footnote{“There are many languages in Uganda and Tanzania with Scriptures available in them, but it is a challenge to discern exactly why they are not being used” (Liz Thomson of SIL, personal communication, 2010)} Even in rural Kenya the ECLs are not used exclusively; functions where speakers from other communities are present get conducted in English, even when the focus is on a particular ECL. Consider the example of the launches of the Meru bible\footnote{\url{https://www.youtube.com/watch?v=mLVq88X7eIE}} and the Kikuyu bible\footnote{\url{https://www.youtube.com/watch?v=1TYgLTC6X38}}. On both occasions held in the respective heartlands of the speech communities concerned, proceedings were conducted mainly in English, and prominent guests (including church leaders, government ministers and civil servants from the focus ECL) addressed the audience in English. 

It is possible that these choices of language were driven by logistical considerations [deference to invited non-ECL guests and benefactors] or the desire by public\slash political figures to project a national perspective\slash image. Whatever the case, there appears to be an underlying uneasiness when it comes to using ECLs in high level formal meetings. The people concerned appear to be anxious to show, even without being prompted, that they are nationalists rather than tribal chauvinists. Such behavior on the part of the Kenyans tempts one to regard it as “unity envy”\footnote{“In the former Yugoslavia, Serbia and Croatia used to have one common language. But since Croatia became an independent republic, interviews with or statements made by Serbs are subtitled in ‘pure’ Croatian before they are broadcasted on the Croatian national TV. The suggestion is: we don’t understand this strange, foreign language; our language is different from that of the Serbs. The wider context for this peculiar phenomenon of creating differences out of similarities is that of nation building and radical nationalism” \citep[1]{Blommaert2014}.} as they cast an eye at neighbouring Tanzania.\footnote{Why would Kenyans be “envious” of their southern neighbour?  “… many will vehemently oppose any move to consider their speech variety as a ‘dialect’ of another ‘language’ … even prominent linguists join hands with politicians to agitate for and celebrate the production of written material in their speech variety, even when for years they have read and understood current material written in a related variety. It is an issue of ethnic identity, [which explains] the obsession with emphasizing the differences”  \citep[244]{Kioko2017}}  And considering the precarious state of ECLs in Tanzania, it is arguable that Kenya may be starting to tread the same path.

\section{Killer languages}\label{sec:kioko:5}

Some language choices made by Kenyans surprise their Tanzanian friends. Why would a father and his son speak in English at the funeral of the wife\slash mother? Why would parents address the guests at their son’s wedding in English? These questions raise the issue of whether English is the “killer language” encroaching on the space of the ECLs in Kenya.  \citet{MuthwiiKioko2002}, surveyed the language choice and language use patterns in both urban and rural areas among five main ECLs.  The research found that both Kiswahili and English are used in the home domain even in the rural areas at varying levels among the five ethnic groups.  Of significant mention are the leading ethnic groups in the choices between ECL, Kiswahili and English as the language in the home: The Kikuyu community was found to use ECL at home even in the urban areas; the Luhya community led in the use of Kiswahili at home even in the rural areas; the Luo community was leading in the use of English at home even in the rural areas; the Kalenjin community significantly used Kiswahili at home and the Akamba community significantly used ECL at home even in the urban areas.   This is a clear indication that even in the home domain, the languages of wider communication have significantly made inroads.

\begin{sloppypar}
Because Kiswahili in Kenya was not until more recently as aggressively promoted as it was in Tanzania, the local languages appear to have more space in the public domain than is the case in Tanzania. Unlike in Tanzania, in Kenya Kiswahili became a compulsory subject in the schools much later in 1980s. This lapse deprived Kenya the chance of creating a common national language that would neutrally be available to both the elite (the highly educated) and the masses (with only primary education).  That gap has for a while been filled by English for a significant section of the Kenyan [educated] population.  Kiswahili is coming from behind to claim that role, particularly for the section of the population with less than secondary education, but the competition for space involving English, Kiswahili and the ECLs is stiff. In the long term it may be safe to say that Kiswahili is going to expand its turf at the expense of both English and the ECLs.  
\end{sloppypar}

In Tanzania, Kiswahili is much more clearly a greater danger for the ECLs and is bound to replace them as the language of normal everyday communication, partly driven by the education system, and other regulatory arrangements that limit ECL domains of use. However, the forces that threaten the ECLs appear to be too strong for normal treatment via the education system or other regulatory policy. It is easy to overestimate the impact of language policy in education, citing for instance, the apparent growth of Swahili in Kenya since the study of the language was made compulsory in the schools. But consider related negative evidence in the Tanzanian context. First, the dismal state of English in Tanzania persists (deteriorates) in spite of the fact that it is a compulsory subject at virtually all levels of the education system (including university), and the language of instruction at all levels except the initial seven years. Second, in spite of the laisez-faire situation regarding the languages of worship, the organized religions in Tanzania have largely shunned the ECLs.\footnote{This is partly understandable in the wider context that \citet[8]{Blommaert2014} describes: “Swahili was swept up in a wave of massive nation building exercises in the late 1960s and 1970s, driven by and incorporated in the state ideology of \textit{Ujamaa}.  … the Tanzanian state made a successful attempt (successful, at least, for some time) at ideological hegemony, … Swahili was given a prominent role in this process of homogenization. Swahili was, thus, deliberately constructed, manufactured, and not ‘just’ as a language, but as an overdetermined emblem of national belonging and ideological rectitude.”}

\section{The forces against indigenous languages}\label{sec:kioko:6}

\subsection{The structure of the economies} 
It has been argued that languages do not easily appear or disappear by legislative fiat. Rather languages whose speakers are dominant in the “production and consumption interdependencies” will become dominant languages and attract speakers of marginalized communities who may thereby shift linguistic allegiances over time  \citep[218]{Mufwene2004}. The dominance of English in East Africa as the language of the state goes back to British conquest and to current American global dominance. It is the language of choice of a small section of the population (the elite) whose production and consumption patterns are more closely tied to the English nations abroad. The second class position of Kiswahili [relative to English] parallels its being a language of a larger section of the population that is struggling to find a precarious foothold in the modern economy as rather mobile labour with a limited\slash slippery say in the relations of production. Yet in relation to the ECLs, Kiswahili is a dominant language. Its position as a national and official language in Kenya and Tanzania acknowledges its power as a lingua franca without a rival among the ECLs.\footnote{“Language shift [is] an adaptive response to changing socioeconomic conditions [which have] undervalued and marginalized” indigenous languages \citep[207]{Mufwene2004}.
}~These latter are largely spoken by a rural agricultural\slash pastoral population that has an even weaker hold on the surplus product of the land. As a consequence, Kiswahili is an attractive alternative over the ECLs and is within reach for a bigger section of the communities. English by contrast is beyond the reach of most people and so cannot be a realistic target for shift by the majority of ECL speakers.

Parents’ language preferences for their children in the schools provide a good illustration of the patterns of language shift in progress. In Tanzania it would be very difficult to find parents who want their children to be taught in the ECL. Instead, in the rural areas, they support the children’s immediate immersion into Kiswahili on the very first day of school and would regard any ECL instruction as a retrogressive measure – possibly a conspiracy to deprive their children a chance to forge ahead in the national economic and social network.  Furthermore, the phenomenon of English medium primary schools, not just in the urban areas but also in relatively rural settings, attests to the desire to participate in that imagined international community that communicates in English and is visibly represented by local elites with relatively good jobs and considerable power. This wish is realized to a greater extent in Kenya where even the official policy on LOI in the first three years of school is disregarded. As already noted, of course, in both Kenya and Tanzania, the reality is that the majority of people will find Kiswahili to be the realistic target for language shift as they get sorted into their respective economic and social slots.

Technological changes affecting mass communication: The ease with which messages are created and disseminated via the smartphone, as well as the language processing technology underlying the gadgets, will have a profound impact on future of the ECLs. Social media operate across local language boundaries; they amplify exposure to dominant languages and other forms of self-expression that undermine ECLs by targeting a wider audience; radio and TV programming will be targeting wider  audiences so that the local village radio station [where permitted] may have rather limited impact.

\subsection{Urbanization} 
The large cities of Kenya and Tanzania are growing at a phenomenal, perhaps uncontrollable, rate. Placed in the global context such growth is not peculiar. The world population is already 50\% urban and is forecast to be 66\% urban by the year 2050 \citep{URBANET2019}. The approximately 5 million inhabitants of Dar es Salaam constitute 10\% of the population of Tanzania -- which is 32\% urban. Similarly the 4 million inhabitants of Nairobi make 8\% of the Kenyan population – which is 26\% urban. The bigger picture is that even the numerous small settlements along major highways or rural roads are a threat to the ECLs because their inhabitants operate in Kiswahili or English. In particular, the children born in such contexts, even if their parents have a common language, are likely to be Kiswahili first language speakers. 

\subsection{Population movements}
Besides the rural to urban migration already discussed, there is another wave of population movements which may be largely rural and still threaten ECL vitality. Traditionally each language would be conceived as inhabiting a clearly demarcated geographical area. However, increasing populations and diminishing resources (land, water) have been forcing different speech communities to live in the same space. 

In Kenya the Rift Valley province is a prime example of such co-existence with speakers of Kikuyu, Kalenjin, Luhya, and Kisii living in close proximity.  The more economically and politically dominant Kikuyu are often accused of not wanting to learn the language of the “natives”, but it is fair to note that no community learns the language of the other community. As a result, the dominant lingua franca, viz. Kiswahili, has developed fairly well. Similarly, the Kamba speakers have dispersed out of their cradle land to various parts of the Coastal province. Their Kiswahili has prospered as a result. In both cases the threat to the ECL is growing because it is confined to the home even in such rural contexts. 

In Tanzania large rural movements are associated with speakers of Maasai and Sukuma. From their Kenya\slash Tanzania border in the north, the Maasai have ventured as far south as the border with Mozambique\slash Malawi\slash Zambia. Similarly the Sukuma are no longer confined to the south shores of Lake Victoria but have moved all the way south to Mbeya, Iringa and Morogoro, just like the Maasai \citep{MuzaleRugemalira2008}. The scale of these movements has gained constant attention particularly via the frequent media reports about clashes between the pastoralist Maasai and the settled agriculturalists in Morogoro and Coast regions. Similarly, government operations to remove large herds of cattle from wetlands and reserve lands\slash forests have involved these cattle keeping communities. As in Kenya, the migrations do not foster the ECLs; rather they create conditions for the lingua franca, viz. Kiswahili, to prosper.

\section{Conclusion}\label{sec:kioko:7}
\begin{sloppypar}
It may be prudent to make a rough distinction between language promotion\slash revitalization endeavours on the one hand and language documentation\slash conservation initiatives on the other.  Activities that make a significant contribution to the active use and promotion of ECLs in Kenya include the teaching of these languages in the school system, and the publication and dissemination of printed materials in various civic educational campaigns pertaining to such matters as health, agriculture, animal husbandry, governance and human rights. The active use of the ECLs in the mass media and in worship may help some of the major speech communities to develop and hold onto these forms of expression for much longer than others.  This paper poses a pertinent question in relation to the promotion endeavours:  who holds the key to the promotion of ECLs? Are these initiatives that language researchers can contribute to or are these initiatives that only the speakers of the language can engage in? Would language researchers’ engagement with the promotion endeavours in the Kenyan setting help in the maintenance or even development of the ECLs in the country? Or is the downward drift without a turn-around button?
\end{sloppypar}

Documentation\slash conservation efforts focus on the gathering and preservation of records of instances of language use, museum style, as part of humanity’s intangible cultural heritage. This is the main thrust of a number of scholarly initiatives where the funding authority specifies strict criteria for identifying an endangered language, using an index of language vitality \citep{UNESCO2003}. Typical products would traditionally include a descriptive grammar and a word list\slash dictionary. Modern technology has made it easier to capture audio and visual records of speakers. Should this be the main thrust of scholarly initiatives with regard to endangered languages?\largerpage

The Kenyan context contains a number of activities that still keep the ECLs in the public domain. How long it will take before these activities become part of museum records is a matter for considerable debate. The Tanzanian context suggests that ECL promotion is a lost cause, and given the close parallels between the two countries, Kenya cannot be far behind in the relegation of the ECLs to the museum. This is not a judgement on the desirability of linguistic diversity or on the moral grounds of the linguistic human rights movement. Rather it is an attempt to answer the question whether speech communities can turn or fight the tidal wave\footnote{“There are currently 7000 languages spoken in the world, and at least half are projected to disappear in this century. The Endangered Language Fund is helping to stem the tide”. Endangered Language Fund, \url{http://www.endangeredlanguagefund.org/}} of dominant languages and reverse the misfortunes of an endangered language.\footnote{“Ikiwa vikwazo hivyo havitatafutiwa ufumbuzi, basi juhudi hizi za kujaribu kuziinua zitakuwa ni sawa na mateke ya punda afaye, hazitasaidia. Tafiti, maandiko ya istilahi, sarufi na makamusi, pamoja na tafsiri zinazofanywa hivi sasa – bila ya kuvikabili vikwazo hivyo kwanza – zitakuwa ni amali za kuzipeleka kwenye majumba ya kumbukumbu tu ili zikapewe jina la nyaraka kuukuu kwa ajili ya kukoleza tafiti na simulizi za vizazi vijavyo kuhusu zama za wahenga wao” [`If these obastacles do not get a solution, then attempts to promote the indigenous languages will be like the kicks of a dying donkey, they will  be useless. Research, publication of technical terms, grammars and dictionaries, together with translations currently being produced, -- without addressing the obstacles first – they will amount to materials bound for museums to be regarded as archives for the enrichment of research and conversations of future generations regarding the era of their ancestors.'] \citep[99]{Madumulla2007}} And given the forces at play in Kenya and Tanzania the answer seems to be negative. In many of the cases, minority language advocates are viewed as parochial tribalists or sentimental “small is beautiful” enthusiasts. Promotional efforts have even been construed as attempts to deny the weak a chance to advance and catch up with dominant groups (in education, political power, economic advancement) by keeping the dominant language out of their reach \citep{Mkude2002}. Hence the “suicidal” wish of marginalized speech communities is not an irrational psychological malady that requires psychotherapy and counseling. Rather it is a rational\slash shrewd assessment of the best interests of such communities, particularly the future fortunes of their offspring.

% 
% \section*{Abbreviations}
% 
% \section*{Acknowledgements}


{\sloppy\printbibliography[heading=subbibliography,notkeyword=this]}

\end{document}
