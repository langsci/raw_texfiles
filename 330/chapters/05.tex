
\chapter[The first grammar of a South Australian language]{The first grammar of a South Australian language: Teichelmann \& Schürmann (\citeyear{teichelmann_outlines_1840})}
\label{chap:key:5}



This chapter discusses the grammar of Kaurna, the language of the Adelaide Plains, written by the Lutheran missionaries C. G. Teichelmann and C. W. Schürmann (\citeyear{teichelmann_outlines_1840}). The size of Teichelmann \& Schürmann’s case paradigms and the way in which cases are named is shown to be entirely different from the methods employed earlier by Threlkeld (\chapref{chap:key:3}) and by Günther (\chapref{chap:key:4}). Teichelmann \& Schürmann’s description of ergativity (\sectref{sec:key:5.4.2}) as well as a number of other descriptive practices (\sectref{sec:key:5.3}) influenced later grammars of South Australian languages, many of which were written by Lutheran missionaries (\chapref{chap:key:6}, \ref{chap:key:8} \& \ref{chap:key:9}).

\section{Historical context}
\label{sec:key:5.1}

The earliest grammars of South Australian languages were written by men who were ordained and sent to Australia by the Evangelisch-Lutherische Missions-Gesell\-schaft zu Dresden (Evangelical Lutheran Mission Society of Dresden, Dresden Mission Institute), henceforth DMI. These “Dresdner” grammars are of languages spoken in the earliest settled coastal districts of the South Australian colony and were made in the decade after the Colony of South Australia was established (1836). They are a grammar of Kaurna spoken on the Adelaide Plains (\citealt{teichelmann_outlines_1840}; this Chapter), a grammar of Ramindjeri, a variety of Ngarrindjeri, spoken on the south coast of the Fleurieu Peninsula at Encounter Bay (\citealt{meyer_vocabulary_1843}; \sectref{sec:key:6.1}), and a grammar of Barngarla spoken on the Eyre Peninsula (\citealt{schurmann_letter_1844}; \sectref{sec:key:6.2}). These three “Dresdner” grammars were all published in English in Adelaide. \citet[viii]{teichelmann_outlines_1840} acknowledged M. Moorhouse (1813--1876) as having revised their work, since “English … [was] not the vernacular tongue of the authors”, and Meyer acknowledged ``Mr Lindsay'', whose identity is uncertain, as having assisted him with the English translation (\citeyear[vi]{meyer_vocabulary_1843}). Original German manuscript grammars are not known to have survived.

Together with a grammar of Ngayawang, spoken upstream from Ngarrindjeri on the Murray River, written by M. Moorhouse (\citeyear{moorhouse_vocabulary_1846}; \sectref{sec:key:6.4}), and a brief five-page grammar of Kaurna written by \citet{moorhouse_report_1841}, the three Dresdner grammars form a previously recognised sub-school of Australian linguistic description. These works have been termed “the Adelaide School of language researchers” by Simpson (\citeyear[410]{simpson_notes_1992}; see also \citealt[123--126]{simpson_i_2008}).

This body of work has been assessed positively in comparison with other early grammatical source materials (\citealt[2]{ray_aboriginal_1925}; \citealt[667]{capell_history_1970}), although it is important to recognise that Moorhouse’s grammar (\citeyear{moorhouse_vocabulary_1846}), written within the intellectual sphere of the Dresdner missionaries' grammars, is \textit{not} of the same calibre as the other works.

With the exception of \citet{moorhouse_annual_1843}, these grammars are shorter and less detailed than Threlkeld’s earlier published work (\citeyear{threlkeld_australian_1834}). As the Dresdner grammarians pointed out (\citealt[v]{teichelmann_outlines_1840}; \citealt[v]{meyer_vocabulary_1843}), their grammars were written shortly after first hearing the language: “Eighteen months is but a short period for the study of an unwritten language, where no means of instruction exist, and where all information must be gleaned from casual and trivial conversation” \citep[v]{teichelmann_outlines_1840}. Threlkeld had by comparison studied the language for nine years before producing his largest grammatical description (\citeyear{threlkeld_australian_1834}). His 1827 work (\sectref{sec:key:3.2}), written within roughly the same short two-year time frame as the Dresdner grammars, is much less substantial by comparison.

Each of the grammars produced by the three Dresdner missionaries was followed by an ethnographic publication describing the “manners and customs” of the people speaking the language (\citealt{teichelmann_letter_1841}; \citealt{schurmann_aboriginal_1846}; \citealt{meyer_vocabulary_1843}). A climate of optimism characterised the relation between Aboriginal and European relations in the very early period of South Australian colonisation. In the introduction to their grammar (\citeyear[v]{teichelmann_outlines_1840}), Teichelmann and Schürmann explained that although postponing the publication would have allowed “greater maturity, and certainty of statements”, they were advised by colonial authorities to publish quickly for “the good which might arise from it to the natives”. Correlation between the degree linguistic and ethnographic investigation into Australian Aboriginal people and the climate of intercultural relations has previously observed by \citet[12]{dixon_languages_1980}.

The prompt publication of grammars and ethnographic descriptions by the Dresdners contrasts with the relatively slow publication of material at later South Australian Lutheran missions, as well as with the type of materials that were first published. None of the missionary grammars of Diyari written over a period of three decades by Lutheran missionaries at Bethesda were published until decades after the closure of the mission in 1915 (\sectref{sec:key:8.2.1}), and the first grammar of Arrernte, written by Lutheran missionary Kempe (\citeyear{kempe_grammar_1891}; \sectref{sec:key:9.1.2}), was published fourteen years after he co-established the Hermannsburg mission.

At both of these later inland Lutheran missions the first published mission material was a primer (\citealt{homann_1870}, \citealt{kempe_intalinja_1880}), which included translations of religious texts into the vernacular for use in mission schools. The same is also true of Ridley’s missionary effort among the Gamilaraay (\citealt{ridley_gurre_1856}; \sectref{sec:key:4.5}).

Schürmann attempted but struggled to translate the Ten Commandments and the Lord’s Prayer into Barngarla \citep[78--81]{rathjen_difficult_1998}. Teichelmann and Schürmann translated the Commandments \citep[111]{amery_warrabarna_2016}, six hymns, a school prayer, and Biblical truths \citep[78]{amery_warrabarna_2016} into Kaurna, and Meyer translated prayers, the Commandments, and some hymns into Ramindjeri \citep[75]{gale_nothing_2011}. None of the Dresdners' translations were published, and the Barngarla and Ramindjeri MSS have not been located.

The support of Sir G. Grey, the Governor of South Australia (1841--1845) underwrites much of the material published by the ``Adelaide School''. Grammars by \citet[v: iii]{schurmann_vocabulary_1844}, \citet[iv]{meyer_vocabulary_1843}, and \citet[v]{moorhouse_vocabulary_1846} are dedicated to Grey, who took an interest in Indigenous languages throughout his career -- first as an explorer in the northwest of Australia (1837) and in the southwest of Australia (1839) and then while Magistrate at King George Sound (1839--1840) and later as Governor of South Australia (1841--1845), New Zealand (1845--1853 and 1861--1868) and of Cape Colony, South Africa (1854--1861). Grey published \textit{Vocabulary of the dialects spoken by the Aboriginal Races of South-Western Australia} (\citeyear{grey_vocabulary_1839}). Grey later published a map of Australian languages (\citeyear{grey_languages_1845}) in the Royal Geographical Society Journal (\figref{fig:map2}). Grey’s map, which showed five Australian “dialects” spoken across the southern portion of the continent, was informed by his own enquiry into the languages of Western Australian and by the work of the Lutheran missionaries in South Australia (\citealt{teichelmann1940}; \citealt{meyer_vocabulary_1843}; \citealt{schurmann_vocabulary_1844}) and by the reports of M. Moorhouse (\sectref{sec:key:6.4}).

\begin{figure}
\includegraphics[width=8cm]{figures/FigMap2.png}
\caption{Grey's map of Aboriginal dialects \citeyearpar{grey_languages_1845}}
\label{fig:map2}
\centering
\end{figure} 


The initial goodwill towards Aboriginal people in the colony of South Australia, which saw the prompt publication of Dresdners' linguistic and ethnographic work, soon soured and was replaced by a degree of hostility. European anxiety over the perception of Aboriginal moral indecency and their unwillingness to adopt a European ``work ethic'' caused social tension \citep[94--95]{scrimgeour_colinizers_2007}. Teichelmann wrote to the German philologist H. C. von der Gabelentz:

\begin{quote}
    [T]his collection [of words] was published because we and our friends believed that it would make intercourse between the Aborigines and the Europeans easier. There does not seem to be very much interest in that, though, since a large number of the English would just like to hang or shoot all the Aborigines, rather than having them in the country. That is the old way of the English in their colonies. \citep{teichelmann_aboriginals_1841}
\end{quote}
 
Each of the Dresdner missions -- Teichelmann \& Schürmann’s at Pirltawardli (1839--1848), Meyer’s at Encounter Bay (1840--1848), and Schürmann’s at Port Lincoln (1840--1845) -- was short-lived. Each operated in poverty. With little and uncertain financial support from colonial authorities, or from Dresden, the enterprises strove to be economically self-sufficient. Missionaries struggled to feed and clothe themselves, let alone provide for the people who they had difficulty attracting permanently to the missions. The missionaries became despondent, as their linguistic efforts seemed increasingly futile: “[T]hose who speak our language are scattered all over the country and will probably not return within the coming months, if ever” (Teichelmann diary 10/11/1844).

As at Wellington Valley (\sectref{sec:key:4.1}) and at the later Lutheran missions (\chapref{chap:key:8} \& \ref{chap:key:9}), the Dresdner missionaries focussed their evangelical efforts on the children \citep[22]{gale_nothing_2011}, perceiving the adults to be beyond redemption (although note that Teichelmann took a different position and persisted in working with adults; see \citealt[12]{lockwood_vision_2007}). Yet not a single Aboriginal person was baptised at any of the four Dresdner missions before the Lutheran mission in South Australia was closed in 1848 in an atmosphere of disappointment and exhaustion.

The failures and successes of the Dresdner missionaries have been the subject of recent evaluations (\citealt{amery2004}; \citeyear{amery_warrabarna_2016}; \citealt{scrimgeour_colinizers_2007}; \citealt{gale_nothing_2011}; \citealt{lockwood_vision_2007}; \citeyear{lockwood_two_2014}). Although the Dresdners' failed to baptise a single Aboriginal person before the missions' closures, the primacy of their work within Kaurna and Barngarla language reclamation programmes and within the revitalisation of Ngarrindjeri is a measure of their success.

A number of factors have been identified as contributing to the early closure of the Dresdner missions. Meyer believed that linguistic diversity and the “wandering habit” of Aboriginal people contributed to the closure and he perceived these factors as impediments to further mission work in South Australia \citep[43--45]{zweck_for_2012}. But these factors were also experienced at later Lutheran missions that endured for decades. The Dresdners encountered a specific range of difficulties resulting from the proximity of each of the missions to heavily settled districts, which was contrary to the missionaries' desire to work among people who were removed from the influence of European vices. Proximity to European settlers was widely perceived as counterproductive to mission activity. The Church Missionary Society in New South Wales was, for example, aware of the need to establish missions beyond the frontier, although this was not achieved \citep[283]{bridges_church_1978}. The site of the mission at Encounter Bay was chosen by colonial authorities against Meyer’s wish \citep[16]{lockwood_vision_2007}. By contrast missionaries at later inland missions worked in the prototypical mission context as an elite racial minority in political control of an Indigenous majority who had little opportunity to contact other Europeans.\footnote{In 1844, the Aboriginal population in Adelaide fluctuated from 300 to 500 people (Governor Grey’s estimate, \citealt[151]{scrimgeour_2006}). In the same year, the first South Australian Colonial census estimated the European population in Adelaide to be ten thousand.}  

Further, the establishment of the earliest South Australian mission in Adelaide, the colony’s capital, to where surrounding groups of Aboriginal people soon flocked, resulted in the outnumbering of the original Kaurna owners of the country by other Aboriginal people who traditionally may not have visited that territory regularly or for extended stays (see Moorhouse’s report, in \citealt[59--60]{foster_two_1990}). The marginalisation of the Kaurna in Adelaide is illustrated by the alteration of Teichelmann’s terminology used to refer to their language. Initially he referred simply to “the language” (\textit{die Sprache}) but by 1844 he discussed “our language” (\textit{unsere Sprache}), reflecting the increasing numbers of speakers of other Aboriginal languages in Adelaide. By August 1845 he refers to “any who speak our branch of language” (\textit{einige unseres Sprachstamms sprechen}; Teichelmann diary 24/08/1845). 

\subsection{The naming of Kaurna}
\label{sec:key:5.1.1}

The first record of the term “Kaurna” to refer to a group of people appears in W. Wyatt’s vocabulary of the “Adelaide and Encounter Bay Tribes”, in Woods, \textit{The Native Tribes of South Australia} \citeyearpar[180]{woods1879}. Under the heading “Names of Tribes” (ibid.: 180) appear two entries: ``Kaurna Encounter Bay Bob's'' and ``Meeyuna Onkaparinga Jack's''.

Here the term “Meeyuna” refers to the people from Adelaide who are associated with Onkaparinga Jack (Mullawirraburka, King John) and the term “Kaurna” refers to the name of people from Encounter Bay. Both words mean “men, people’. 

\ea
\gll korn-ar \\
	  man-\textsc{pl} \\
\z

\ea
\gll miyu-rna \\
	  man-\textsc{pl} \\
\z


\textit{Korn(e)} is recorded as the translation for “man” by Meyer at Encounter Bay (\citeyear[12]{meyer_vocabulary_1843}) and Meyu is given as “man” by \citet[23]{teichelmann_outlines_1840}. 

The next use of the term “Kaurna” with reference to the people from South Australia was given in a key accompanying a map drawn by Charles Richards (no date) in 1892 (see \citealt[18--19]{knapman_mapping_2011}). The label ``Ka Orna'' (item 194) is assigned to the area near Encounter Bay. Richards assigns the label ``Mei Orna'' (item 192) to the area around Adelaide. The caption Richards gave to the map reads: “Map showing the location of 208 Tribes, with their correct names – gathered during three years constant travel among them (over the whole area represented) and from available data”. It is likely that Wyatt (\citealt{woods1879}) was Richards' available source data, although in 1891 there were still some speakers of the language, and Richards may have sourced his material independently (see \citealt[78--80]{gara_life_1990}). Richards does not conform to any earlier spelling conventions in naming any of the tribes on the map.

The next known recorded usages of the term “Kaurna” are also given on maps. In these sources the term is used to refer to the people from Adelaide, rather than Encounter Bay. A. W. \citet[opp. p. 44]{howitt_native_1904} placed the name across an area of land just north of Adelaide, extending towards the Murray River. He used the term “Narrinyeri” to refer to the area east of the Murray River, along the coast. Carl Strehlow (1907--1920, vol. 2), following Howitt, gave “Kaúrna” and ``Nàrrinjèri'' in the same locations.

That Howitt’s and Strehlow’s maps follow Wyatt’s spelling \citep[180]{woods1879} indicates that this was the source of the term. It seems that \citet{howitt_native_1904} misinterpreted Wyatt when choosing a name by which to refer to the Adelaide language. There is no evidence that the term \textit{Kaurna} meaning “man” in a neighbouring language was used to refer to the people and language of the Adelaide Plains exonymically. Note that W. Schmidt’s 1919 map (Figure 8.17) refers to the same group as the “Meyu-Sprachen” (\citeyear[41--42]{schmidt_gliederung_1919}). The use of the term referring to the people and language of the Adelaide Plains became entrenched after the authoritative mapping of tribal boundaries and their “proper names” made by N. B. Tindale (\citeyear[213]{tindale_aboriginal_1974}; \citealt[4]{amery_warrabarna_2016}), although Tindale had used the term as early as 1926 \citep[Appendix B, 8]{Amery1998}. It seems probable that Tindale followed \citet{howitt_native_1904} and/or \citet{strehlow_notitle_1910}, both of whom he listed as sources.

\subsection{Training} 
\label{sec:key:5.1.2}

The Dresden missionaries were initially trained at Germany’s first mission school, the Jänickesche Missionsschule in Berlin (Jänicke Mission Institute) in preparation for missionary work in India. The institute had been established by J. Jänicke (1748--1827) in 1800. Here Teichelmann, Schürmann and Meyer received a scholarly education that focused on the study of theology and foreign languages. They studied Latin, English, Greek and Hebrew (\citealt{schurmann_diaries_1838}, in \citealt[256]{schurmann_id_1987}). Schürmann had also studied some Chinese. 

Jänicke trained missionaries to work for larger non-denominational mission societies in Basel, Rotterdam and London. Teichelmann and Schürmann were initially offered mission work with the Church of England in India. Refusing to be ordained in the Church of England (\citealt{schurmann_diaries_1838}, in \citealt[256]{schurmann_id_1987}), Teichelmann and Schürmann declined the offer and in 1836 became the first students of the Evangelisch-Lutherischen Missions-Gesellschaft zu Dresden (DMI), which had been specifically established to allow Teichelmann and Schürmann to complete their training. Here they received further training in English, Hebrew, Greek and the exegetic exposition of the Biblical text. Teichelmann and Schürmann were ordained as Lutheran missionaries in 1838 in Altenburg, where they became acquainted with H. C. von der Gabelentz. Under the patronage of George Fife Angas (1789--1879), they were sent to Adelaide.


\section{C. G. Teichelmann \& C. Schürmann’s grammar of Kaurna (\citeyear{teichelmann_outlines_1840})}
\label{sec:key:5.2}

C. G. Teichelmann (1807--1888) and C. Schürmann (1815--1893) arrived in South Australia in 1838, two years after the colony’s official settlement. In collaboration with the Kaurna people, they established a settlement and a garden on the north bank of the Torrens River at a site named “the native location” or “Pirltawardli” (possum house). Within two years of arriving in Adelaide, Teichelmann and Schürmann had published a twenty-four-page grammar of the language of the Adelaide Plains, later referred to as Kaurna. The grammar appeared as the first component in a one hundred and eight-page book, which also included a Kaurna to English vocabulary “of 1816 head entries, a phraseology of 141 entries, two short passages illustrating dialect differences and five short song lines” \citep[87]{amery_warrabarna_2016}.\footnote{Note that the pagination begins again in the vocabulary. Pages cited here are in the initial grammatical section of the work, unless shown as 1840V, which designates pages in the vocabulary}  The grammar was the second published account of an Australian language, preceded only by \citet{threlkeld_australian_1834} and the unpublished grammars of Wiradjuri by Günther (\citeyear{gunther_native_1838}; \citeyear{gunther_lecture_1840}) and those now lost by Handt and Watson (\sectref{sec:key:4.1}).

In their introduction (\citeyear[vii-viii]{teichelmann_outlines_1840}), Teichelmann and Schürmann situated the work within existing knowledge of Australian languages and their known relatedness to one another. They offered their work as confirmation that “all Australian languages are derived from one root”, a fact that, as the authors pointed out, had been suggested by Threlkeld (\citeyear[10]{threlkeld_australian_1834}) and by Grey (\citeyear{grey_vocabulary_1839}; \citeyear[365--366]{grey_journals_1841}). The contextualisation of the material within existing research is a feature of works of the Adelaide School, which stands in contrast to the later Lutheran descriptions of Diyari, and with most of the corpus of early grammars. The features that were seen as indicating common ancestry were reiterated by Schürmann (\citealt{schurmann_aboriginal_1846}; \sectref{sec:key:6.2.1}) and by Moorhouse (\citealt{moorhouse_vocabulary_1846}; \sectref{sec:key:6.4.1}).
    
Teichelmann and Schürmann were content to express uncertainty about some of the structures they encountered and to exemplify complex structures that they were unable to provide an account of. This trait is shared with some other more detailed, longer analyses (e.g., \citealt{kempe_grammar_1891}; \sectref{sec:key:9.1.2}). \citet[106]{amery_warrabarna_2016} points out: “[T]here are indications that Teichelmann and Schürmann tried as best as they could to base their analysis entirely on what they actually heard.” After the initial presentation of nominal case paradigms in three numbers (\citeyear[5--6]{teichelmann_outlines_1840}), the authors stated that they are unable to account for morphophonemic variation in the shape of the inflection for ergative/instrumental “active or ablative” case, and of inflection for dual and plural number: “[A]s yet, no fixed rule can be given for those letters by which the dual termination is joined to the root” (ibid.: 5). Importantly, they exemplified the variation. This stance differs from the certainty with which some grammarians with less grammatical insight into the languages they described, and here R. H. Mathews comes to mind (\sectref{sec:key:4.3}), presented their material as authoritative.

Nevertheless, there is also evidence that Teichelmann and Schürmann \textit{did} engage in the “filling-in” of paradigms, for the sake of regularising the description, as Günther had admitted to doing (\citeyear[350]{gunther_lecture_1840}). The amendments that Teichelmann later made to the published case paradigms (\citeyear{teichelmann_verb_1858}; \sectref{sec:key:5.2.1}) suggest that this original analysis \citep{teichelmann_outlines_1840} had been regularised.

The grammar does not include reference to the word-class ``article'', not even to state that it does not exist. Perhaps alerted to the needlessness of the category through reading Threlkeld’s description “of the \textit{substitute} for the article” (\citeyear[9]{threlkeld_australian_1834}; emphasis added), the absence of this grammatical category signals that the authors intended to compose their description in response to the structure of the language. Yet, like all other early grammarians, Teichelmann and Schürmann adopt aspects of the traditional descriptive framework which were not motivated or appropriate to the morphosyntactic structure. The missionaries stated that the purpose of their work was to 

\begin{quote}
    keep up good understanding and facilitate the intercourse, between Aborigines and Europeans; to give to the latter a medium of communication, and, especially, assistance to those who may be inclined to acquire the language. \citep[iv]{teichelmann_outlines_1840}
\end{quote}

The inclusion of familiar descriptive schemata may have been seen as helpful in ensuring the grammar remained accessible to its readers. 

The grammatical component of the work commences with a description of “nouns” divided into three categories: ``substantives'', “adjectives” and ``pronouns''. The traditional division between substantives and adjectives, which the missionaries did not describe as functionally motivated, is maintained. They state: “declension of substantives also applies to adjectives” \citep[4]{teichelmann_outlines_1840}. \citet{simpson2021} shows that Teichelmann and Schuurmann’s use of the term “noun” parallels the classical use of the term ``nomen'', as a superclass covering both noun-substantive and noun-adjective (see \citealt[33--37]{arnold1781}). The missionaries used the term “noun” in the sense of the modern term ``nominal''. The usage is repreated by \citet[2]{moorhouse_vocabulary_1846}.

Substantives and adjectives are each divided into three classes: ``primitive'', i.e., underived nominal roots, “derivative”, i.e., mostly nouns derived from verbs, and ``compound forms''. Each is illustrated with a dozen or so examples showing a range of derivational and inflectional morphology, some of which is not overtly presented elsewhere in the grammar. Derivational application of the privative suffix \textit{–tina}, for example, is shown under the heading ``derivative adjectives'': “\textit{yangarutanna}, unmarried, from \textit{yangarra} wife” (\citeyear[6]{teichelmann_outlines_1840}): 

\ea\label{ex:5:7}
  yangarutanna \\
\gll yangarra-tina \\
    wife-\stockcaps{PRIV}                            \\
\z

Under the heading “derivative nouns”, the process through which words for items of European clothing were formed from existing lexical items is illustrated. Newly formed words for European items of clothing were commonly included by early grammarians in discussions of derivational morphemes (e.g., \citealt[19]{meyer_vocabulary_1843}). \citet[4]{teichelmann_outlines_1840} gave: \textit{Mukartiana} `hat' formed from \textit{mukarta} `head' suffixed with what may be inflection for allative case \textit{–ana} \citep[122, 175]{amery_kulurdu_2013}: 

\ea\label{ex:5:8}
 mukartiana \\
\gll mukarti-ana\\
    hat-\stockcaps{ALL}\\
\z

They also gave (ibid.: 4) \textit{turtiana} `jacket' formed through the same process from \textit{turti} `upper arm'. 

Although the inclusion of distinct categories of “substantives” and “adjectives” provided schemata in which the missionaries could illustrate a wealth of morphological processes, it is not clear that the processes shown as occurring on nouns and on adjectives are distinct or that the missionaries perceived a syntactic motivation for maintaining a distinction between the two classes of word. The ``derivative adjective'' \textbf{yangarutanna}, for instance, most probably functioned also as a noun referring to a man who is not married. 

In the year of the publication of the Kaurna grammar, Schürmann commenced work with Barngarla people on the Eyre Peninsula, and two additional DMI graduates arrived in South Australia. S. G. Klose assumed responsibility for teaching the Kaurna school children at Pirltawardli, although Teichelmann continued to teach religious education in Kaurna. H. A. E. Meyer was sent to establish a new mission at Encounter Bay on the southeastern tip of the Fleurieu Peninsula.

In 1844 a government school in which English was the sole language of instruction was opened in Adelaide to cater for the increasing numbers of Aboriginal people visiting Adelaide from the areas adjacent to the Murray River. In 1845, Pirltawardli was amalgamated with the government school and in 1846 Klose’s contract was terminated from Adelaide and from Dresden.

\subsection{Teichelmann’s later linguistic work}
\label{sec:key:5.2.1}

While Schürmann may have been the more affable of the two earliest Dresdner missionaries \citep{kneebone_language_2005}, Teichelmann was the most linguistically active, if not astute, of the pair. Of the six Kaurna hymns which Klose sent to Dresden in 1843 (reprinted in \citealt[27--30]{graetz2002}), the two translated by Schürmann are by far the shortest, having two and three verses each. The remaining four hymns translated by Teichelmann are up to seven verses long.

In the same year in which Teichelmann produced his thirteen-page ethnographic description of “The South Australian Aborigines” (\citeyear{teichelmann_letter_1841}), he co-authored a much more substantial “Report on the Aborigines of South Australia” with (\citealt{moorhouse_report_1841}; in \citealt[38]{foster_two_1990}). 

During the early 1840s Teichelmann collected specimens for the Na\-tur\-for\-schen\-de Gesellschaft des Osterlandes (Nature Research Society of the Osterland), based in Altenburg, and in 1841 commenced correspondence with H. C. von der Gabelentz, to whom he sent a Kaurna vocabulary and grammar. Gabelentz also had a copy of Meyer’s grammar, from which he quoted. It is not known if this had also been supplied by Teichelmann, or whether Meyer corresponded independently with Gabelentz.

After the closure of the Lutheran missions in South Australia in 1848, Teichelmann continued to refine his analysis of Kaurna. In the face of the demise of the Kaurna, and while ministering to German settlers in the Bremer Valley, Teichelmann continued to revise his Kaurna linguistic data. At the request of Sir George Grey, then Governor of Cape Colony, South Africa, Teichelmann produced MSS analyses of Kaurna, which he sent to Cape Town, where they remain, held by the South African Public Library. They were catalogued by \citet[40]{bleek_library_1858}. The “Dictionary of the Adelaide Dialect” (1857) was described by \citet[40]{bleek_library_1858} as containing 2400 “words” for which “the meanings are given much fuller and illustrated more copiously than in the [1840] Vocabulary”. More recently the MS has been assessed as ``provid[ing] an additional source of grammatical data, entries are given with some comment on morpho-syntactic structure and often with illustrative sentences'' \citep[411--12]{simpson_notes_1992}. 

Teichelmann also sent Bleek a three-page description of the verb, which he completed in 1858 \citep{teichelmann_verb_1858}. \citet[40]{bleek_library_1858} observed that the work “treats the formation of seven different kinds of verbs”.

In 1858, Teichelmann’s own copy of the 1840 Kaurna grammar was also sent to Cape Town (\citeyear{teichelmann_verb_1858}). \citet[40]{bleek_library_1858} described this copy as having notes that “extend over the whole grammatical part”. The annotations are of exceptional interest to tracing Teichelmann’s developing understanding of Kaurna structure. However, note that since this MS remained completely unknown in Australia, it was Teichelmann’s very early analysis (\citeyear{teichelmann_outlines_1840}) that came to be most influential on later PN descriptions. Teichelmann’s alterations and additional comments concentrate on some of the analytically challenging morpho-syntactic structures that are examined in detail in this study: processes of clause subordination (\sectref{sec:key:5.6.2}) and the naming and presentation of the range of dative and possessive case functions (\citealt{stockigt_2017} §5.4).

\subsection{Threlkeld’s influence}
\label{sec:key:5.2.2}

Prior to arriving in Adelaide, Teichelmann and Schürmann became acquainted with PN structure through copying by hand a MS version of Threlkeld’s grammar(s) (presumably 1834) while in London \citep[68]{rathjen_difficult_1998} and later by studying a copy of the 1834 publication which they had been lent \textit{en route} to Australia in 1838 by fellow passenger Governor Gawler \citep[65]{amery_warrabarna_2016}. Teichelmann and Schürmann’s work was to some degree influenced by Threlkeld’s grammar. They stated that they followed the system of spelling “adopted by Rev. Mr. Threlkeld … and other missionaries experienced in the Polynesian languages” (\citeyear[v]{teichelmann_outlines_1840}). Their division of the document into three sections -- grammar, vocabulary and phraseology -- may also follow Threlkeld’s tripartite arrangement. 

That the Lutheran missionaries did not, however, simply copy Threlkeld’s format is well recognised \citep[87]{amery_warrabarna_2016}. Teichelmann and Schürmann did not provide interlinear-style translation of clauses, nor use hyphens to mark the boundaries of word-internal syllables or sub-word meaningful units as Threlkeld had done. Other than sharing structural similarities that are indicative only of commonly inherited influences, Teichelmann \& Schürmann’s grammar (\citeyear{teichelmann_outlines_1840}) shows a high degree of independent response to PN morphosyntax. In comparison to the way in which some later corpus grammarians borrowed previously developed descriptive templates, Threlkeld’s influence on Teichelmann and Schürmann’s analysis was marginal.

\section{The beginning of a new descriptive tradition}
\label{sec:key:5.3}

Teichelmann and Schürmann’s earliest grammar of a South Australian language had a major descriptive impact on later grammars written by Lutherans and other grammarians in South Australia. 
Comparison of later grammars of Diyari (\chapref{chap:key:8}) and Arrernte (\chapref{chap:key:9}) with Teichelmann and Schürmann’s presentation of case (\sectref{sec:key:5.3.1}), postpositions (\sectref{sec:key:5.3.2}), the declension of possessive pronouns (\sectref{sec:key:5.3.3}), ergativity (\sectref{sec:key:5.4.2}), description of the sytax of complex clauses (\sectref{sec:key:5.6}), and the division of verbs into “transitive” and “intransitive” classes (\sectref{sec:key:6.2.1.6}) shows that “the Adelaide School of language researchers” \citep[410]{simpson_notes_1992} generated a larger and more enduring school of descriptive practice than has previously been recognised.

\subsection{Case paradigms}
\label{sec:key:5.3.1}

Unlike the paradigms of nominal case given by \citet{threlkeld_australian_1834} and Günther (\citeyear{gunther_native_1838}; \citeyear{gunther_lecture_1840}), which extend the classical paradigms so as to embrace the larger systems of morphological case in PN (\sectref{sec:key:3.3.3}; \sectref{sec:key:4.4.3}), \citet{teichelmann_outlines_1840} presented conservative five-place Latinate case paradigms (see \figref{fig:5:78}). Only case forms that were translated into SAE languages using a case-inflected word, rather than a prepositional phrase, were included in the paradigm and in the discussion of nominal case. While it is not clear that the case system of Kaurna was as large as that documented by Threlkeld for Awabakal, Teichelmann and Schürmann’s presentation of case marks a radical departure from earlier PN descriptions.
 
\begin{figure}
\includegraphics[width=8cm]{figures/Fig78.png}
\caption{\label{fig:5:78}Teichelmann \& Schürmann’s case paradigm of a noun \citeyearpar[5]{teichelmann_outlines_1840}}
\end{figure}


That Teichelmann \& Schürmann presented a paradigm that was so radically different from Threlkeld’s, with which they were well acquainted, shows that they approached the description of Australian languages with confidence in their own descriptive ability, and did not feel the need to replicate the descriptions from the Colony of New South Wales. 

Subsequently, small Latinate paradigms commonly appeared in grammars of PN languages by later Lutheran grammarians such as \citet{koch_untitled_1868}, \citet{schoknecht_grammar_1947}, \citet{flierl_dieri_1880}, Reuther (\citeyear{reuther_dieri_1894}, \citeyear{reuther_ms_1899}), \citet{kempe_grammar_1891}, C. \citet{strehlow_untitled_1931}, and by T. G. H. \citet{strehlow_aranda_1944}. Symmons (\citeyear{symmons_grammatical_1841}; \figref{fig:key:130}) is the only non-Lutheran to produce such a paradigm.

But Teichelmann and Schürmann were the only grammarians of the Adelaide School to present such a conservative paradigm. Meyer (\citeyear{meyer_vocabulary_1843}; \sectref{sec:key:6.1.2.1}), Schürmann (\citealt{schurmann_letter_1844};  \sectref{sec:key:6.2.1.1}), and Moorhouse (\citealt{moorhouse_vocabulary_1846}; \sectref{sec:key:6.4.1.1}) each presented case systems differently. Teichelmann and Schürmann’s inaugural description of case in a South Australian language (\citeyear{teichelmann_outlines_1840}) was independently influential on later Lutheran works.

Teichelmann and Schürmann’s paradigms of nominal case only include suffixes that they perceived to carry the same or a comparable function to those carried by one of the Latin cases. The notable exception is the inclusion of the ergative case, which despite marking a function that is not carried by SAE case systems, nevertheless attracted a case label. Current analysis of the suffixes described as case inflections are shown in \tabref{fig:5:79}.

\begin{table}
	\small
    \begin{tabularx}{\textwidth}{lQlQ}
        \lsptoprule
        \multicolumn{2}{l}{Teichelmann \& Schürmann (1840)} & \multicolumn{2}{l}{ Amery \& Simpson (2013)} \\
        \cmidrule(lr){1-2}\cmidrule(lr){3-4}
        Form & Function & Form & Function \\
        \midrule
        \textit{-ø} & marks nominative case & -ø & marks nominative case \\
        \textit{-ko(nna)} & marks genitive case & \textit{-ku, -rna} & marks possessive case \\
        \textit{-nni} & marks dative case & \textit{-ni} & marks dative case (p. 120) \\
        \textit{-ø} & marks accusative case & -ø & marks accusative case \\
        \textit{-lo (-rlo, dlo)} & marks active and ablative cases & \textit{-rlu, -dlu} & marks ergative/ instrumental case \\
        \lspbottomrule
    \end{tabularx}
    \caption{\label{fig:5:79} Current analysis of inflections described by Teichelmann and Schürmann as marking nominal case}
    \label{tab:chap5:inflections}
\end{table}


The form and function of nominal inflections marking cases in Kaurna with functions that are \textit{not} associated with SAE case systems, but which are instead carried by prepositional phrases, were listed towards the end of the grammar under the heading ``postpositions''. These include inflections now analysed as marking purposive, ablative, allative, locative, comitative and perlative cases. These case inflections also tend to be illustrated inadvertently in other sections of the grammar where they are not overtly described. 

Note here that Threlkeld had \textit{not} shown inflections for case as separate units unattached to a nominal stem, as did Teichelmann \& Schürmann (\figref{fig:2:19}) and Günther in MS grammars of Wiradjuri (\citeyear{gunther_native_1838}; \citeyear{gunther_lecture_1840}; \figref{fig:2:18}). There is no indication that Teichelmann and Schürmann had seen copies of the MS Wiradjuri grammars. Teichelmann \& Schürmann may have innovated this type of description in response to the agglutinative structure of the language, without influence from other grammarians. It is also possible, however, that this style of presentation, which is not a feature of the traditional description of classical European languages, was suggested to the Dresdners through reading Rhenius' grammar of Tamil (\citeyear{rhenius_grammar_1836}).

\subsection{Postfixa and postpositions}
\label{sec:key:5.3.2}

Teichelmann and Schürmann differentiated two classes of postposition on structural grounds. Their innovation was subsequently employed by later grammarians of Diyari (\sectref{sec:key:8.6.1}), and of Arrernte (\sectref{sec:key:9.3.3.1}), and in Taplin’s last grammar of Ngarrindjeri (\citeyear[8]{taplin_grammar_1878}).


The treatment of the word-class “pre/post-positions” by the corpus grammarians is, like the representation of case-systems broadly, diagnostic of schools of descriptive practice. It is \citegen[vi]{roth_ethnological_1897} unconventional division of “prepositions” into the four categories of ``motion'', ``rest'', ``purpose, reason and means'' and ``time'' (\sectref{sec:key:10.1.1}) that, in part, establishes Roth’s influence on the grammars of Guugu-Yimidhirr written by Neuendettelsau-trained Lutheran missionaries (\citealt{schwarz_koko_1900}; \sectref{sec:key:10.1.2}).

\begin{figure}[b]
\includegraphics[width=8cm]{figures/Fig80.png}
\caption{\label{fig:5:80}Teichelmann and Schürmann’s division of “postpositions” into two classes \citeyearpar[21]{teichelmann_outlines_1840}}
\end{figure}


Making sense of Teichelmann and Schürmann’s rationale for the division (\figref{fig:5:80}) is difficult, and one gets the impression that this section would have read better in the authors' first language. Postpositions are listed as being of two types, confusingly termed “postfixa” \citep[21--22]{teichelmann_outlines_1840} and “postpositions” (ibid.: 22--23). Entries presented under each type were given because they are seen to serve the same grammatical and semantic functions as SAE prepositions. The first class was described as “affixes” and the second class as ``words''.



Items belonging to the first class of affixes, termed “postfixa” (\tabref{tab:5:81}) were described as ``particles affixed to the words''. The class “postfixa” included case inflections mostly marking local case functions, which translated into a SAE language using a prepositional phrase.

\begin{table}[b]
	\small
    \begin{tabularx}{\textwidth}{llQ}
        \lsptoprule
        Original form & Current orthography &  Analysis \citep[122--123]{amery_kulurdu_2013}\\
        \midrule 
        \textbf{-anna, -kanna} & \textit{-(k)ana} & allative: motion towards a place\\
        \textbf{-arra, -tarra} & \textit{-(t)arra} & perlative\\
        \textbf{-illa} & \textit{-ila, -illa**} & locative\\
        \textbf{-ngga} & \textit{-ngka, -ngga†} & locative\\
        \textbf{-itya*} & \textit{-itya} & allative: motion towards a person, and purposive (dative)*\\
        \textbf{-ityangga} & \textit{-ityangka} & comitative\\
        \textbf{-unungko} & \textit{-unangku} & ablative (used with inanimates)\\
        \textbf{-ityarnungko} & \textit{-ityanungku} & ablative (used with animates)\\
        \lspbottomrule
    \end{tabularx}
	\legendbox[1]{$^\ast$The analysis of the suffix \textit{–itya} taken here differs from \citet[122]{amery_kulurdu_2013}. Rather than analysing the suffix \textit{–itya} as marking allative function towards an animate being as well as purposive function, the suffix is said here to mark the dative case (see \citealt{stockigt_2017} §5.4). $^{\ast\ast}$Note that the form\textit{–ila}, described by Teichelmann \& Schürmann as a “postfixa” was re-analysed by F. \citet[97]{muller_grundris_1882} as the \textit{localsuffix} which was listed among a group of other case suffixes in other Australian languages.$^\dagger$In reclaimed Kaurna the orthographic representation of the locative suffixes on placenames has been maintained as \textit{–ngga} and \textit{–illa} because these representations are fixed in accepted spellings of many placenames. Compare for instance the name of the suburb Noarlunga with \textit{Nurlungga} and *\textit{Nurlungka} `on the bend'. }
    \caption{\label{tab:5:81} Analysis of nominal inflections listed by Teichelmann \& Schürmann as ``postfixa''}
    \label{tab:chap5:ortho}
\end{table}


    
    
Items belonging to the second class were termed “postpositions” (\tabref{tab:5:82}) and were described as a class of noun to which postfixa can attach. Here Teichelmann and Schürmann included nouns inflected with their ``postfixa''. Many are the locational words inflected for locative case common to Australian languages \citep[68]{dixon_preface_2002}. Some are nouns derived from body parts in locative case. The reason for the inclusion of other items in this category remains unclear. Some translations remain semantically opaque. Each is translated by the missionaries as having either spatial or causal function.

\begin{table}[b]
	\small
    \begin{tabularx}{\textwidth}{lp{6cm}Q}
        \lsptoprule
        Form & Original translation &  Analysis \\
        \midrule 
        \textit{wattingga} & In the midst of, between, on account of & \textit{warti-ngka} \newline middle-\stockcaps{LOC}\\
        \textit{wattewattingga} & On account of & \textit{warti-warti-ngka} \newline middle-\stockcaps{REDUP}-\stockcaps{LOC}\\
        \textit{wattedrukkungga} & In the midst of, the centre, amongst & \textit{warti-trruku-ngka}* \newline middle-centre-\stockcaps{LOC}\\
        \textit{wirrawirrangga} & On account of & \textit{wirra-wirra-ngka} \newline forest-\stockcaps{LOC}\\
        \textit{worngangga} & Before, in front of & \textit{warnka-ngka} \newline omentum-\stockcaps{LOC}\\
        \textit{tangkangga} & In the entrails” within & \textit{tangka-ngka} \newline liver-\stockcaps{LOC}\\
        \textit{trukkungga} & In the centre, amidst & \textit{trruku-ngka} \newline centre-\stockcaps{LOC}\\
        \textit{ngurrungga} & In the back, behind & \textit{ngurru-ngka} \newline back-\stockcaps{LOC}\\
        \textit{marrangga} & In or on the hand, alongside, with (accompanying) & \textit{mara-ngka} \newline hand-\stockcaps{LOC}\\
        \textit{martungga} & In the smell or taste, for, instead, in place of & \textit{martu-ngka} \newline taste-\stockcaps{LOC}\\
        \textit{martuity [?sic]} & For the smell or taste, in [sic] behalf, on account of & \textit{martu-itya} \newline taste-\stockcaps{PURP}*\\
        \textit{mikangga} & In the eye, before, in the presence of & \textit{mika-ngka} \newline ?eye-\stockcaps{LOC}\\
        \textit{minkaara} & Along the eye, before, in the presence of & \textit{miina-(k)-arra} \newline eye-\stockcaps{PERL}\\
        \lspbottomrule
    \end{tabularx}
    \legendbox[1]{$^\ast$The form of “centre” shown here and below in \figref{fig:key:95} follows \citet{amery_kulurdu_2013}. It is, however, possible that the onset is \stockcaps{CVC}, rather than \stockcaps{CC}. In favour of the \stockcaps{CC} onset analysis is the form of the locative}
    \caption{\label{tab:5:82} Analysis of nominal inflections listed by Teichelmann \& Schürmann as ``postpositions''}
    \label{tab:chap5:nominal}
\end{table}

\begin{table}
	\small
    \begin{tabularx}{\textwidth}{QQQ}
        \lsptoprule
        Form & Original translation &  Analysis \\
        \midrule
        \textit{birra} & on account of, about, for & case suffix \textit{–pira} \stockcaps{AFTER}*\\
        \textit{ngundarta} & behind & \textit{nguntarta}\\
        \textit{pulyo} & without & case suffix – privative**\\
        \textit{parnatta} & on this side & \textit{parnata}\\
        \lspbottomrule
    \end{tabularx}
	\legendbox[1]{$^\ast$This form is glossed AFTER, following Wilkins's (\citeyear[210]{wilkins_mparntwe_1989}) analysis of the Arrernte case suffix \textit{–iperre, -ipenhe}. $^{\ast\ast}$Two other suffixes are described with privative function: \textit{-tina} and \textit{-marraka.} The difference is unclear.}
    \caption{\label{tab:5:83} Analysis of additional ``postpositions'' given by Teichelmann \& Schürmann}
    \label{tab:chap5:postpo}
\end{table}


An additional list of four “postpositions” was given (\tabref{tab:5:83}). These were seen to differ because they “cannot be derived from a noun” \citep[22]{teichelmann_outlines_1840}. Two are locational words: \textit{ngundarta}, `behind' and \textit{parnatta} `on this side'. Two are probably nominal inflections, possibly marking case, which have been analysed by the authors as words.




\citet[21]{teichelmann_outlines_1840} likened the attachment of their “postfixa” to their “postpositions” to the structure of Hebrew. While the explanation of postpositions and postfixa (\figref{fig:5:80}) remains slightly opaque, it is likely that the perceived similarity between Kaurna and Hebrew relates to placement of the postpositions marked with a postfix after the head noun that they qualify, as in the following NP:

\ea\label{ex:5:9}
Worli    worngangga \\
Before, in front of the house \\
\citep{teichelmann_dictionary_1857} \\
\gll warli warnka-ngka \\
house   omentum-\stockcaps{LOC} \\
\glt `In front of the house'
\z

In Hebrew and in Kaurna adjectives follow the noun they qualify, differing from SAE languages. The adjective in Hebrew agrees with the noun in gender, number and definiteness, Hebrew having lost grammatical case. Teichelmann and Schürmann may also have intimated the construct-state \citep[247]{gesenius_gesenius_1910}, in which the possessed noun \textit{nomen rectum} follows the head it qualifies \textit{nomen regens} and the relationship is marked by alteration to stress patterns and vowel quality of the first constituent. In constructions in which there are two dependent nouns -- e.g., “the sons of David and his daughters” \citep[414]{gesenius_gesenius_1910} -- or where a second genitive noun qualifies the \textit{nomen regens} -- e.g., “the hill of my holiness” or “my holy hill” \citep[50]{weingreen_construct-genitive_1954} -- the second \textit{nomen rectum} is morphologically marked with a suffix to agree with the head \textit{nomen regens}. Note that \citep[21]{teichelmann_outlines_1840} stated, ``as \textit{sometimes} in the Hebrew languages'' (emphasis added), suggesting that it was to this less frequently occurring construction that they referred. 

The division of “postpositions” into two structurally distinct classes was adop\-ted by later generations of Lutheran missionaries, but was not employed by other grammarians of the Adelaide School. Teichelmann and Schürmann’s schema was subsequently used by missionary-grammarians describing Diyari (1868--1899; see \sectref{sec:key:8.6.1}), by Kempe (\citealt{kempe_grammar_1891}; see \sectref{sec:key:9.3.3.1}), and by T. G. H. \citet{strehlow_aranda_1944} in a description of Arrernte \citep{Stockigt2021a}. The schema was also employed by \citet[8]{taplin_grammar_1878}, but only in his final analysis of Ngarrindjeri (see \sectref{sec:key:7.3.3}).

\subsection{Declension of ``possessive or adjective pronouns''}
\label{sec:key:5.3.3}

In some PN languages the marking of “phrasal” or “adnominal” case -- e.g., possessive or proprietive functions (\citealt[9]{dixon_introduction_1976}; \citeyear[141]{dixon_preface_2002}) -- occurs directly before clausal case marking, resulting in double case marking on a single word (see \citealt{austin_double_1995}). Examine the following Awabakal clause (\ref{ex:5:10}) from Romans 8:14 in which the phrase-final noun in possessive case – \textit{Eloi-kupa} `God’s' – receives ergative marking. Both constituents of the possessive NP standing in the ergative case – `God’s spirit' – are marked with different ergative allomorphs (see \citealt[26]{lissarrague_salvage_2006}):

\ea\label{ex:5:10}
Yantin   barun            yemmam-an         Marai -to             Eloi-kupa-ku \\
\glt `God’s spirit leads them all' \\
\citep{threlkeld_australian_1834} \\
\gll Ya\stockcaps{NT}iyn   paraN            ?-N                  Maraye-tju      \textbf{Eloi}-kupa-ku \\
        all           3\textsc{pl}.\stockcaps{ACC} 	  lead-pres           spirit-\stockcaps{ERG}       God-poss-\stockcaps{ERG}\\
\z


\hspace*{-2.7pt}Nouns marked as possessive receive double case marking in languages in which clausal case is either marked on each constituent of the NP or is marked on the last constituent where the possessive noun follows the head it qualifies. In Kaurna case can be marked on each constituent of a discontinuous NP \citep[115, 132]{amery_kulurdu_2013} and possessive pronouns can be marked for the clausal case.

Teichelmann \& Schürmann accounted for this phenomenon by presenting a paradigm of “possessive or adjective pronouns” (\citeyear[11--12]{teichelmann_outlines_1840}; \figref{fig:5:84}). This declension of possessive pronouns marked for clausal case reflects the importance of the word as the unit of analysis in the traditional descriptive framework. A modern representation would discuss the case marking of the noun-phrase. Teichelmann \& Schürmann’s innovation was to have an enduring influence on some later grammars of PN languages.

The paradigm declines possessive pronouns for two cases, termed ``genitive'' and ``dative''. The form labelled 2sgGEN \textit{ninkuitya} in \figref{fig:5:84} might translate as `for/to/towards your X' (\ref{ex:5:11}), since the suffix \textit{-itya} marks both dative and allative function \citep[122]{amery_kulurdu_2013}. There are no example clauses given in the Kaurna of possessive pronouns marked for these cases.

\ea\label{ex:5:11}
ninkuitya \\
\gll ninku-itya \\
{}[2\textsc{sg}\stockcaps{POSS}]-\stockcaps{DAT}/\stockcaps{ALL}                         \\
\z

      	  

          
Teichelmann \& Schürmann did not present an ergative case form of the possessive pronouns, labelled “active” in the paradigm. Such ergative forms are, however, exemplified. In the answer to the question “Whose child gave it to you?”, for example, the 1sgPOSS pronoun is marked for ergative case: 

\ea\label{ex:5:12}
Ngangko   		wakwakurlo 	  niinanni   yüngki?	      Ngaityurlo \\
\glt `Whose child gave it to you?'						Mine \\
\citep[12]{teichelmann_outlines_1840} \\
\gll ngangku    		wakwaku-rlu 	 ninna-ni            yungk-i?	       ngatiyu-rlu \\
 \stockcaps{INTER}.\stockcaps{POSS}    	  	child-\stockcaps{ERG}   	 2sg-\stockcaps{DAT}          give-\stockcaps{PAST}      1\textsc{sg}.\stockcaps{POSS}-\stockcaps{ERG}                                \\
\z

       	

       	
As \citet[14]{moorhouse_vocabulary_1846} later explained regarding his declension of Ngayawang ``pronominal adjectives'', i.e. possessive pronouns, they “are formed from the genitive of the personal pronouns”. The pronoun termed “genitive” in the table of personal pronouns -- i.e., a pronoun in possessive case -- is the zero-marked form which is termed “nominative” in paradigms of ``possessive or adjective pronouns''. Observe that in Teichelmann \& Schürmann’s presentation of Kaurna the 1sgGEN form in the declension of personal pronouns \textit{Ngaityo} (\figref{fig:5:85}) is the 1sgNOM form in the declension of ``possessive or adjective pronouns'' (\figref{fig:5:84}).

Dual and plural forms of “nominative possessive pronouns” in each number and person are shown as regularly formed from the singular with the suffixes \textit{–rla} and \textit{–rna}, which mark the dual and plural respectively on other nominals in Kaurna \citep[123]{amery_kulurdu_2013}. The function of these forms is not clear. It is probable that they resulted from the filling in of the paradigm. No later works that presented paradigms of possessive pronouns in each person and number show an additional axis marking number. Some later grammarians -- such as Koch and Reuther in Diyari, and Taplin in Ngarrindjeri (\sectref{sec:key:8.2}) -- did, however, confuse the number reference of the possessive pronoun, and incorrectly translated the forms with the number reference referring to the head noun. Reuther, for example (\figref{fig:key:146}), translated the nominative form of 1dlPOSS as “my two” instead of “belonging to us two”. It is possible that Teichelmann \& Schürmann were also somewhat confused.
 
% continue here
 
\begin{figure}
\includegraphics[width=7.5cm]{figures/Fig84a.png}
\includegraphics[width=7.5cm]{figures/Fig84b.png}
\caption{\label{fig:5:84}Teichelmann \& Schürmann’s “declension of possessive or adjective pronouns” \citeyearpar[11-12]{teichelmann_outlines_1840}}
\end{figure}

 
 
\begin{figure}
\includegraphics[width=8cm]{figures/Fig85a.png}
\includegraphics[width=8cm]{figures/Fig85b.png}
\caption{\label{fig:5:85}Teichelmann \& Schürmann’s pronominal case paradigm \citeyearpar[7-8]{teichelmann_outlines_1840}}
\end{figure}


Teichelmann \& Schürmann’s method of accounting for the additional clausal case marking of a pronoun in possessive case became a feature of early South Australian description, being employed by \citet[25]{meyer_vocabulary_1843}, \citet[14--18]{moorhouse_vocabulary_1846}, Taplin (\citeyear[no pag]{taplin_vocabulary_1867}; \citeyear[12]{taplin_grammar_1880}), but not by \citet{schurmann_letter_1844}. The earliest grammars of Diyari (\sectref{sec:key:8.3}) state that the declension of possessive pronouns is regular and follows that of the noun (\citealt[no pag]{koch_untitled_1868}; \citealt[5]{schoknecht_grammar_1947}; \citealt[18--20]{flierl_christianieli_1880}). Reuther (\citeyear{reuther_dieri_1894}; \citeyear{reuther_ms_1899}; \figref{fig:key:146}), however, included full case paradigms of Diyari possessive pronouns, and Homann (in \citealt[44]{fraser_australian_1892}) stated, “The possessive pronouns, which are the personal pronouns of the genitive case, are declined like substantives”, and presented first-person singular Diyari forms. Case paradigms of possessive pronouns appear in later South Australian Lutheran descriptions of Arrernte, in Kempe (\citeyear[8]{kempe_galtjintana-pepa_1891}; \figref{fig:5:86}) and C. Strehlow (\citeyear[68--71]{strehlow_untitled_1931}; \citeyear[14--16]{strehlow_notitle_1910}). Case paradigms of possessive pronouns were later engaged by T. G. H. Stehlow (\citeyear[95--96]{strehlow_aranda_1944}; see \citealt{Stockigt2021a}), extending Teichelmann \& Schürmann’s influence into the twentieth century.

 
\begin{figure}
\includegraphics[width=8cm]{figures/Fig86.png}
\caption{\label{fig:5:86}Kempe’s declension of possessive pronouns \citeyearpar[8]{kempe_grammar_1891} (Arrernte)}
\end{figure}


The declensions of possessive pronouns had \textit{not} been presented by Threlkeld. Paradigms of possessive pronouns are not shown in any grammars of languages spoken in New South Wales. 

\newpage
\hspace*{-1pt}Other grammarians who described double case marking include Mathews, who liked to point out that he was the first to report this grammatical feature \citep[181]{koch_r_2008}, and Roth (\citeyear[7]{roth_ethnological_1897}; \sectref{sec:key:10.1}).

\section{Teichelmann \& Schürmann’s description of syntactic cases}
\label{sec:key:5.4}

\subsection{Split system of marking syntactic case and recognition of an “absolutive” case}
\label{sec:key:5.4.1}

Teichelmann and Schürmann’s description of Kaurna and Schürmann’s description of Barngarla (\sectref{sec:key:6.2}) present all singular nominals as showing ergative alignment (\stockcaps{A/SO}) and all non-singular nominals as undifferentiated for the syntactic cases (\stockcaps{ASO}). Of ergative nouns, Teichelmann and Schürmann wrote, “for dual and plural of this case no termination is known” (\citeyear[6]{teichelmann_outlines_1840}). Non-singular nouns are undifferentiated in the syntactic cases (\stockcaps{ASO}).\footnote{Reclamation of Kaurna \citep[136]{amery_kulurdu_2013} has rectified this seemingly unfeasible system by offering distinctly marked ergative forms, marked with \textit{–rlu}, in all numbers. The situation, which is not supported in the source material, has been proposed in order to overcome potential ambiguities.}  Accusative forms are identical with the nominative on all nominal-types. 

Like other early Australian grammarians, Teichelmann and Schürmann maintained a three-case analysis for nominals with an ergative system (\stockcaps{A/SO}) and included the accusative case in all paradigms. Its inclusion stands in contrast to the frequent omission of the ergative case from paradigms of nominals showing accusative alignment (\stockcaps{AS/O}; \figref{fig:3:38}; \figref{fig:key:127}).
In some paradigms, however, rather than filling in the accusative forms, Teichelmann \& Schürmann stated “same as the nominative” (\tabref{tab:5:81}). Further, when declining “possessive or adjective pronouns” (\citeyear[11--12]{teichelmann_outlines_1840}) they wrote: “the accusative forms have been omitted because they are like the nominative”. In this way Teichelmann and Schürmann’s presentation approaches an analysis of an “absolutive” case.
In a significant rearrangement of the pronominal case paradigm in response to the ergatively aligned system, \citet[49--53]{moorhouse_report_1841} placed nominative and accusative pronouns in a single position at the top of the paradigm (\figref{fig:5:87}). It is presumably this aspect of the grammar that \citet[108]{amery_warrabarna_2016} describes as being “set out with greater clarity”.
 
\begin{figure}
\includegraphics[width=8cm]{figures/Fig87.png}
\caption{\label{fig:5:87}Teichelmann and Moorhouse’s later representation of case forms of Kaurna pronouns, 1841 (in \citealt[49]{foster_two_1990}). Note the typographical error in 1NOM, shown elsewhere as \textit{ngaii}}
\end{figure}

In one sense this paradigm might be seen simply as a rearrangement motivated by a more economical presentation of identically marked forms, in which the two cases remain theoretically intact. Nevertheless, the presentation realises a necessary step towards a two-case analysis of ergatively aligned languages, which treats the roles of \stockcaps{S} and \stockcaps{O} as a single case, in the same way that grammars of accusatively aligned SAE languages treat the roles of \stockcaps{A} and \stockcaps{S} as the ``nominative''.

The absolutive case is otherwise only suggested in early descriptions of PN languages made in Germany. The absolutive case was described and labelled “absolutive” in the paradigms presented in Planert’s grammars of Arrernte (\citeyear{planert_australische_1907}) and of Diyari (\citeyear{planert_australische_1908}; \sectref{sec:key:2.6.1}). An absolutive case was suggested in F. Müller’s re-presentation (\citeyear[19]{muller_grundris_1882}) of Threlkeld’s Awabakal case paradigms (\citeyear[13--16]{threlkeld_australian_1834}; \figref{fig:3:37}), in which cases of nominals showing ergative alignment are re-ordered. The accusative case is taken out of one of its traditional positions in the paradigm and placed at the top of the paradigm, bracketed with the identical nominative form and termed ``nominative subject''. Note that there are two traditional orderings of case, one being nom > voc > acc > gen > dat (> abl) and the other being nom > gen > dat > (abl >) acc > voc.

\subsection{The ergative case}
\label{sec:key:5.4.2}

Teichelmann and Schürmann’s grammar is atypical of the corpus grammars in not describing the function of overtly marked ergative nominals in the body of the text. The only explanation is given as the note stating “the agent” in the pronominal paradigms (\figref{fig:5:85}).

Unlike \citet{threlkeld_australian_1834}, Günther (\citeyear{gunther_native_1838}; \citeyear{gunther_lecture_1840}), and all later grammarians of languages spoken in New South Wales (\citealt{ridley_kamilaroi_1866}; \citeyear{ridley_kamilaroi_1875}; \citealt{livingstone_grammar_1892}), who placed ergative case forms in second position alongside the nominative case at the top of case paradigms, Teichelmann \& Schürmann consistently placed ergative nominals in last position (\figref{fig:5:78}; \figref{fig:5:88}).

By placing the ergative case in last position, Teichelmann and Schürmann instigated a descriptive practice that had not been influenced by Threlkeld, and which was to become subsequently influential. 

Teichelmann and Schürmann placed ergative forms in last paradigmatic position, the traditional position of the ablative case in Latinate paradigms, because the ergative inflection usually took the same shape as the inflection marking Kaurna case functions that are marked by the Latin ablative.

Later grammarians who placed ergative case forms in positions at the bottom of case paradigms, which were separated by other cases from the nominative case form placed at the top of the paradigm, were influenced by Teichelmann \& Schürmann’s paradigm. The placement of the ergative case in the position of the Latin ablative may have been innovated only once in Australia, by these two missionaries, although Symmons' (\citeyear{symmons_grammatical_1841}) grammar of Nyungar (\sectref{sec:key:7.1}), belonging to country in southwest Western Australia, more than 2500 kilometers distant from Adelaide, also placed ergative nouns, termed ``ablative'', at the bottom of Latinate paradigms. Symmons' work, which was of no further influence, may have been influenced by Teichelmann \& Schürmann, although there is no other indication that Symmons had access to the publication. Symmons aside, Teichelmann \& Schürmann’s placement of ergative case forms in this unusual position had two trajectories of influence, both within Lutheran description.

Missionary Meyer, Teichelmann \& Schürmann’s colleague from the Dresden Mission Institute, also placed the ergative case in this position, but \textit{reinterpreted} the term ``ablative''. Meyer’s template became subsequently influential on later grammarians working in the southeast of the country (\sectref{sec:key:2.6}). But Teichelmann and Schürmann’s most substantial impact on the placement of ergative case forms paradigmatically is found in Lutheran grammars of Diyari (\sectref{sec:key:8.5.1}), and this influence reached as far as W. Planert’s 1908 Diyari grammar published in German (\sectref{sec:key:8.5.2.1}).

In paradigms of nouns, Teichelmann \& Schürmann gave the labels “active” \textit{and} “ablative” to the ergative form (\figref{fig:5:78}), while in pronominal paradigms the ergative form is termed simply “active” (\figref{fig:5:85}).

In Kaurna, ergative and instrumental case functions appear to have been identically marked on all types of nominals except interrogative pronouns. The missionaries' interrogative paradigm (\figref{fig:5:88}) shows how the missionaries used the terms “active” and “ablative” to refer to distinct functions. The term “active” is used to label the ergative form, and the term “ablative” is used to label the instrumental form. Distinct “active” and “ablative” case forms of the interrogative pronoun are placed side by side at the bottom of the case paradigm (\figref{fig:5:88}).\footnote{ \citet[328]{dixon_australian_2002} assesses Kaurna as among a minority of PN languages in having a single form for both “who?” and ``what?''. Teichelmann \& Schürmann, however, provide clauses suggesting that the language had distinct forms for these interrogatives (see \citeyear[141--148]{amery_kulurdu_2013}).}
 
\begin{figure}
\includegraphics[width=8cm]{figures/Fig88.png}
\caption{\label{fig:5:88}Teichelmann \& Schürmann’s case paradigm of interrogative pronouns \citeyearpar[9]{teichelmann_outlines_1840}}
\end{figure}


\largerpage
Syncretism between ergative and instrumental cases on some nominal types is widespread in PN languages, as is the marking of this functional range with a single suffix on \textit{all} nominal types \citep[135]{dixon_australian_2002}.

Note here that despite the fact that ergative and instrumental functions are commonly marked with the same form in Australia, the term “instrumental” is not used to label the ergative case in early PN languages written in the country. 

A later era of PN description sees the engagement of the term “operative” to name the case marking both ergative and instrumental functions (\citealt{smythe1949}; \citealt{hercus_languages_1969}; \citealt{blake_pitta-pitta_1971}). In 1956, Arthur Capell wrote:

\begin{quote}
    The term ``operative'', first suggested to me by Dr.C. M. Churchward some ten years ago, is preferable to the “agentive” because it covers more neatly two different usages of the suffix – the first to express the instrument by which an act is performed and the second the person who performs it. \citep[63--64]{capell_new_1956}
\end{quote}
 
Note that Churchward was a missionary and a linguist, who wrote grammars of Tongan, Fijian and Rotuman. His communiucation with Smythe marks an instance of Oceanic influence on Australian descriptive practice.

The term “instrumental” was rarely used in Australia, and always to refer to peripheral case functions (\sectref{sec:key:9.3.4.1}). The term “instrumental” was, however, used to describe the ergative case in grammars written outside Australia. \citet{muller_grundris_1882} renamed the ergative case “instrumental” in grammars of South Australian languages. G. \citet{gabelentz_sprachwissenschaft_1891} used the term “Activo-instrumentalis” to name the ergative case and Ray employed the terms “instrumental” (\citealt{ray1897}) and “active instrumental” (\citeyear{ray_linguistics_1907}).

\hspace*{-2.9pt}The missionaries demonstrated their understanding that interrogatives showed distinct marking for these two case functions when stating, “the active or ablative case has here two forms” \citep[9--10]{teichelmann_outlines_1840}. They juxtaposed the following clauses in order to clarify the different function of the interrogative \textbf{ngando} \textit{nganthu} labelled ``active'' and the interrogative \textbf{ngannarlo} \textit{ngana-rlu} labelled ``ablative'':\footnote{The form of ergative interrogative pronoun shown here follows that given by Amery \& Simpson (\citeyear[141]{amery_kulurdu_2013}). The expected form according to reconstruction of Proto-PN \citep[67]{koch_languages_2014} is ngantu.}


\ea\label{ex:5:13}
Ngando    aityo  		 mudlinna                       metti \\
Who has taken away my implements? \\
\citep[10]{teichelmann_outlines_1840} \\
\gll  Nganthu-aityu  		 mudli-rna  	          mitti \\
\stockcaps{INTER}.\stockcaps{ERG}-1\textsc{sg}.\stockcaps{DAT}  	 implement-[\stockcaps{ACC}]-\textsc{pl}    steal-[\stockcaps{PAST}]	                         \\
\glt `who has taken away the implements on me' \footnote{See \sectref{sec:key:7.3.3} for further discussion of this clause. }
\z

\ea\label{ex:5:14}
Ninna 	ngannarlo 	minkarni \\
By what have you been wounded? \\
\citep[10]{teichelmann_outlines_1840} \\
\gll Niina	 ngana-rlu            minka-rni	 \\
2sg\stockcaps{ACC}        \stockcaps{INTER}-\stockcaps{INST} 	wound-\stockcaps{INCH}-[\stockcaps{PAST}]                                \\
\z

      	       


In other paradigms Teichelmann \& Schürmann engaged the term ``active'' in order to capture a form’s ergative function and employed the term ``ablative'' to capture its instrumental function, recognising that the two functions were marked by the same form on nouns. 

The labelling of these pronouns with the sole descriptor ``active'' reflects the fact that they are likely to function as agents but unlikely to function as instruments. 

Teichelmann \& Schürmann’s use of the term ``active'' to name the ergative case in Australia was new to the description of PN languages, although Threlkeld had used “active nominative” when describing interrogative pronouns, and Günther had labelled all ergative forms ``active nominative''. The term “active” was later used by H. A. E. \citet{meyer_vocabulary_1843} – but only in discussion of antipassive construction (\sectref{sec:key:6.1.2.6}) – and in most grammars of Diyari (\citealt{koch_untitled_1868}; \citealt{schoknecht_grammar_1947}; \citealt{flierl_dieri_1880}; \sectref{sec:key:8.5}). It is the terms “active” and “active nominative” that have the widest currency in the earliest decades of PN description (\tabref{tab:chap3:ergativepm}).

\subsection{The use of the term “ablative’}
\label{sec:key:5.4.3}

The case labelled “ablative” in traditional grammars of Latin is a conflation of three historically distinct cases -- the ablative, the locative and the instrumental \citep[7]{blake1994} -- and assumes a broad range of oblique functions (\sectref{sec:key:3.3.3}). The functions that the early grammarians termed “ablative” differ in the corpus grammars.

Teichelmann and Schürmann assigned the term “ablative” to a different range of case functions than did the grammarians who preceded them (\citealt{threlkeld_australian_1834}; \citealt{gunther_native_1838}; \citeyear{gunther_lecture_1840}) and many grammarians who came after them (e.g., \citealt{kempe_report_1881}). They did not assign the term “ablative” to the suffix marking motion-away, despite the fact that this “ablative of separation” function is in traditional practice primarily associated with the case termed ``ablative''. It is the function for which the case is named – from the Latin \textit{ablātīv-us}, “to carry away”. In Kaurna this function is marked by the suffixes \textit{–nangku} and \textit{–ityanungku}, which the missionaries described as “postfixa” (\sectref{sec:key:5.3.2}).\footnote{The functional difference between the ablatives is reclaimed as depending on whether the marked nominal is a place or a person \citep[122]{amery_kulurdu_2013}. }  Teichelmann and Schürmann were motivated to assign the case label “ablative” to a different range of functions by the distinctive syncretism in the marking of case functions in Kaurna.

Teichelmann \& Schürmann explained the range of functions that they labelled “ablative”, stating: 

\begin{quote}
    [T]he ablative case, which has the same termination as the active case, is put not only where the medium of an action is an instrument, but also in cases where merely shall be expressed by what means something is to be performed; as, \textit{Parndarlo ngatto wodli taieta} – I shall build a house with bricks. \citep[24]{teichelmann_outlines_1840}
\end{quote}

\ea\label{ex:5:15}
Parnda-rlo 		ngatto 	  wodli                 taie-ta.  \\
`I will build the house with bricks' \\   	        
\citep[24]{teichelmann_outlines_1840} \\
\gll Parnta-rlu  		 ngathu     wardli              tayi-tha \\
limestone-\stockcaps{INST}   	1\textsc{sg}.\stockcaps{ERG}   house-[\stockcaps{ACC}]   build-\stockcaps{FUT}                               \\
\z
      	         

      	         
In Kaurna, the instrumental suffix, which was termed “ablative” by the missionaries, and which takes the same shape as the ergative suffix on all nominal types except interrogatives, marks two functions of the Latin ablative. It marks the Latin ablative-of-means \citep[257--259]{gildersleeve_latin_1895} or the weapon or instrument used to carry out the verb action, which is the primary function of the instrumental case in PN languages \citep[135]{dixon_preface_2002}. The same suffix also marks the Latin ablative-of-material \citep[254--255]{gildersleeve_latin_1895} or the material out of which something is made. This syncretism between Latin ablative-of-means and the Latin ablative-of-material (\tabref{fig:5:89}) which occurs in some PN languages but is not typical \citep[136]{dixon_preface_2002} gave Teichelmann and Schürmann reason to assign the term “ablative” to the suffix that marked a different range of functions from the Latin ablative of origin.


\begin{sidewaystable}
\small
\begin{tabularx}{\textwidth}{p{3.5cm}QQQQ}
\lsptoprule
Latin                                       & \textbf{Ablative of personal agent i.e.,
                                                the agent of a passive construction
                                                                          `by X'} & \textbf{Ablative of means
                                                                                    or instrument ‘made
                                                                                          with X’}        & \textbf{Ablative of material ‘made from X’}
                                                                                                                                        & \textbf{Ablative of separation `from X'}\\
\tablevspace
Pama-Nyungan                                &  \textbf{Agent of a transitive verb} & \textbf{Instrument used to carry out verb action} &  {\textbf{\mbox{Material out of which} something is made}} & \textbf{Motion from a place}\\
\midrule\midrule
Kaurna  \citep{TeichelmannSchürmann1840} &  active                             & \textbf{ablative}     & { ‘postfixa’}               & { ‘postfixa’}\\
\tablevspace
Ngarrindjeri \citep{meyer_vocabulary_1843}              &  \textbf{ablative}                  & \textbf{ablative}     & \multicolumn{2}{c}{uncertain and variable See \sectref{sec:key:8.2}}  \\
\tablevspace
 Nyungar\newline  \citep{Symmons1841}               &  \textbf{ablative}                  &  \textbf{ablative}    & uncertain                 & uncertain  \\
\tablevspace
Diyari Koch, Schoknecht, Flierl  (1868-1880)  &  {active}                           & \textbf{ablative}     &  postposition             & postposition\\
\tablevspace
Diyari \citep{reuther_ms_1899}                  &  \textbf{ablative}                  & \textbf{ablative}     &   postposition            & postposition\\
\tablevspace
\mbox{Awabakal  \citep{Threlkeld1834}   }           & nominative 2                        & nominative 2          & \textbf{ablative 2}       & \textbf{ablative 2}\\
\tablevspace
\mbox{Wiradjuri \citep{Günther1838} }             &  nominative agentive                &   \textbf{ablative 4} &  {uncertain}              & \textbf{ablative 1}\\
\tablevspace
\mbox{Wiradjuri \citep{gunther_lecture_1840}   }          &  nominative agentive                & instrumental          &  {uncertain}              & locomotive\\
\tablevspace
Arrernte \citep{Kempe1891}                  &  nominative                         &   postposition        & ablative                  & ablative \\
\lspbottomrule
  \end{tabularx}
    \caption{\label{fig:5:89} The mapping of various functions of the Latin ablative case onto PN case functions
 in some early grammars of Australian languages}
    \label{tab:chap5:mapping}
\end{sidewaystable}


Later grammarians -- \citet{symmons_grammatical_1841}, \citet{meyer_vocabulary_1843}, and \citet{reuther_dieri_1894} -- employed the term “ablative” to describe the ergative case. It is not always entirely clear whether in doing so they were invoking another function of the Latin ablative -- the-ablative-of-personal-agent \citep[272]{gildersleeve_latin_1895}, which marks the agent by whom the action of a passive verb is performed -- or had simply failed to appreciate that Teichelmann \& Schürmann had used the terms “ablative” and “active” with distinct reference.

Because the term “ablative” had been “used up” to describe instrumental function, Teichelmann \& Schürmann did not include the suffixes \textit{–nangku}, and \textit{–ityanungku} which mark the function of “motion away from” a place and a person in their description of the case system. Nor do later Lutheran grammars of Diyari, upon which Teichelmann \& Schürmann’s grammar was influential. The description of ablative form and function in grammars of these languages occurs \textit{only} in a discussion of “postpositions” (\sectref{sec:key:8.7.1}). In this, these works differ from the other early descriptions of PN languages.


\section{Bound pronouns}
\label{sec:key:5.5}

Because there was no existing schema in which to describe the form and function of bound pronouns, and because their function and structure was not always initially recognised, bound pronouns tended to either not be described, or to be accounted for in a ways that may not seem immediately intuitive to a modern reader. 

The presentation of bound pronouns given by Teichelmann and Schürmann tells of a dawning recognition of the existence of the system and of its function. The missionaries first \textit{mention} the presence of pronominal affixes without explicitly describing their form or function in relation to the ``conditional mood of the verb'': “Whether this mood is changed in its significance when the personal pronouns are affixed must remain for farther enquiry” \citep[19]{teichelmann_outlines_1840}.

At the end of the grammar, in a section appearing as an addendum headed ``Grammatical Remarks'', the authors wrote: “The nominative is frequently put twice, the answering pronoun being affixed to the verb” (\citeyear[23]{teichelmann_outlines_1840}). They gave two examples. The first presents 1plS as a free and a bound form pronoun:

\ea\label{ex:5:16}
Niina	   narta     padne-ota,    ngadlu   yaintya   wandi-adlu \\
\glt `You are going, (but) we, we shall sleep here' \\
\citep[23]{teichelmann_outlines_1840} \\
\gll niina          narta    padni-utha,   ngadlu    yaintya   wanti-adlu \\
2\textsc{sg}.\stockcaps{NOM}    now     go-\textsc{fut}           1\textsc{pl}.\textsc{s}        here        sleep-1\textsc{pl}.\textsc{s} \\
\z


Teichelmann \& Schürmann (ibid.: 24) then wrote, “The same takes place for the accusative (the object)”, and gave a second example (\ref{ex:5:17}).

\ea\label{ex:5:17}
Tidnarla     	nguiyuatto      	purla \\
\glt `The feet, I will warm them' \\
\citep[23]{teichelmann_outlines_1840} \\
\gll Tidna-rla   	nguyu-athu      	purla \\
feet-3\textsc{dl}.\stockcaps{ACC}       warm-1\textsc{sg}.\stockcaps{ERG}     3\textsc{dl}.\stockcaps{ACC}                                    \\
\z



They included an additional bracketed construction (\ref{ex:5:18}), explaining (ibid.: 24): “The contracted form in the parenthesis is the usual way of speaking; the separate forms have been chosen for the sake of illustration”. In this example the bound 3dlACC pronoun is attached to the bound 1sgA pronoun, which is attached to the verb:

\ea\label{ex:5:18}
(nguiyuatturla)	\\
\citep[23]{teichelmann_outlines_1840} \\
\gll nguyu-athu-rla \\
warm-1\textsc{sg}.\stockcaps{ERG}-3\textsc{dl}.\stockcaps{ACC}                              \\
\glt `I warm them'
\z

That these comments and illustrations appear at the end of the grammar, almost as an afterthought, suggests that the missionaries' unpreparedness to describe bound pronominal forms resulted in less coverage in the body of the grammar than their function warranted. 

Teichelmann \& Schürmann’s most comprehensive presentation of bound pronouns occurred within descriptions of the optative and imperative moods of the verb. The presentation suggests that the missionaries were alerted to the existence of bound pronouns when hearing these short utterances. The optative and imperative moods provided the missionaries with a schema in which bound pronouns were accommodated while maintaining the traditional descriptive format. The same strategy was used by Threlkeld (\citeyear[51]{threlkeld_australian_1834}; \sectref{sec:key:3.3.6.1}), who showed bound pronouns under the heading ``the imperative mood''.

\citegen[18--19]{blake_nineteenth-century_2015} unravelling of Spieseke’s (\citeyear{spieseke1878}) and Mathews' (\citeyear{mathews_language_1903}) descriptions of bound pronouns in Wergaya and in Bunganditj respectively, and \citegen{koch_r_2008} overview of Mathews' description in multiple languages, confirm that even when bound forms could appear on other clausal constituents, the forms were more likely to be depicted as verb inflections. This occurred both when a grammarian recognised that the forms were pronominal, and in instances when this was apparently not understood. Of Spieseke’s analysis of the bound pronouns in Wergaya (\citealt{spieseke1878}; \figref{fig:5:90}), Blake writes:

\begin{quote}
    Although verbs in many Australian languages appear to inflect for person … and number …the person number forms are actually enclitics, abbreviated, unstressed pronouns that can be attached to words other than the verb. Spieseke gives past and “perfect” forms of the verb \textbf{who-räg} `speak'. … His perfect forms are the same as the past forms with the addition of \textbf{mala} `then', but note how the person marking \textbf{–n} and \textbf{-r} now appears on \textbf{mala}. The third-person is often unmarked. Spieseke, presumably unfamiliar with this practice, has filled in the demonstrative \textbf{kinya} `this one'. \citep[18--19]{blake_nineteenth-century_2015}
\end{quote}
 
\begin{figure}
\includegraphics[width=8cm]{figures/Fig90.png}
\caption{\label{fig:5:90}Spieseke's presentation of bound pronouns marking number and person on the verb (\citeyear[57]{Spieseke1878}) (Wergaya)}
\end{figure}


Teichelmann \& Schürmann showed all bound first-person pronouns, except 1sgACC, attached to the verb in “optative” mood (\citeyear[18]{teichelmann_outlines_1840}; \figref{fig:5:91}), which was described as expressing “the wish or the will of a person”. Importantly, the authors recognised that inflection for optative mood was pronominal, stating (ibid.: 18) that the mood “is not marked by a particular termination; but the personal pronouns are affixed to all the tenses of the indicative, and form, in this manner, a new mood.”
 
\begin{figure}
\includegraphics[width=8cm]{figures/Fig91.png}
\caption{\label{fig:5:91}Teichelmann \& Schürmann’s presentation of bound pronouns as the “optative mood” of the verb \citeyearpar[18]{teichelmann_outlines_1840}}
\end{figure}


Most second and third-person dual and plural bound pronouns are shown under the heading “the imperative mood” (\citeyear[17]{teichelmann_outlines_1840}; \figref{fig:5:92}). Ergative forms are shown attached to the “active” transitive verb \textit{kundo}- `to beat', and the nominative forms are attached to the “neuter” intransitive verb \textit{tikka}- `to sit'. Again, the authors explained that the mood inflection is pronominal:

\begin{quote}
    There occurs no common termination for the imperative, neither does there appear to be any distinction of time in it; the following may give to the reader an idea of the formation of this mood … It will be seen that each person of this mood is formed, in most instances by the last or more syllables of the answering pronoun[.] \citep[17]{teichelmann_outlines_1840}
\end{quote}

 
\begin{figure}
\includegraphics[width=8cm]{figures/Fig92.png}
\caption{\label{fig:5:92}Teichelmann \& Schürmann’s presentation of bound pronouns as the “imperative mood” of the verb \citeyearpar[17]{teichelmann_outlines_1840}}
\end{figure}


The missionaries' understanding that the “inflection for imperative mood” was pronominal was sufficiently advanced to enable them to point to instances of suppletion in the paradigm. They noted that the 3sg forms \textbf{kundaingki} and \textbf{tikkaing\-ko} were marked by “other forms”, describing that here the imperative was marked by \textbf{–ki} on the transitive verbs and by \textbf{–ko} on intransitive verbs \citep[17]{teichelmann_outlines_1840}. These forms are probably cognates of the Thura-Yura imperative suffix -ka \citep[279]{hercus_tragedy_2001}, although an intervening segment \textit{–ing}, which as consonant-final is phonotactically aberrant, remains unexplained. These 3sg imperative forms appear to be marked with an imperative verbal suffix \textit{–ing-kV}. Third-person singular bound pronouns commonly have zero-realisation in PN languages \citep[343]{dixon_preface_2002}.

\ea\label{ex:5:19}
kundaingki \\
\glt       `let him beat' \\
\citep[17]{teichelmann_outlines_1840} \\
\gll kunda-ing-kV-Ø \\
beat-?-\stockcaps{IMP}-3\textsc{sg}.\stockcaps{ERG}\\
\z

\ea\label{ex:5:20}
tikkaingko \\
\glt `let him sit' \\
\citep[17]{teichelmann_outlines_1840} \\
\gll tikka-ing-kV-Ø\\
sit-?-\stockcaps{IMP}-3\textsc{sg}.\stockcaps{NOM}\\
\z


\tabref{tab:5:93} shows the section of Teichelmann and Schürmann’s grammar in which each bound pronoun was exemplified. The morpheme \textit{–ing} in the 3sgA, 3sgS/O, 2dl, 3dl and 2pl forms precedes the bound pronoun in the imperative mood.

\begin{table}[t]
	\begin{tabularx}{\textwidth}{lQQQQ}
		\lsptoprule
		\textbf{Free-forms} & \multicolumn{4}{c}{\textbf{Bound forms}} \\
		\cmidrule(lr){2-5}
		& \textbf{A} & \textbf{Shown as:} & \textbf{S/O} & \textbf{Shown as:} \\
		\midrule
		1sg & \textit{-athu\textbackslash begin} & Optative mood & \textit{-ai} & (Example \ref{ex:5:21}) \\
		\textit{ngathu} (A) & & & & \\
		\textit{ngai} (S,O) & & & & \\
		2sg &\shadecell \textit{\bfseries -nthu}$^\ast$ & Imperative mood & \textit{-iina} & (Example \ref{ex:5:22}) \\
		\textit{ninthu} (A) & & & & \\
		\textit{niina} (S,O) & & & & \\
		3sg & \shadecell \textit{\bfseries -ing-ki-Ø} & Imperative mood &\shadecell \textit{\bfseries -ing-ku-Ø} & Imperative mood \\
		\lspbottomrule
	\end{tabularx}
	\begin{tabularx}{\textwidth}{lQQ}
		\lsptoprule
		\textbf{Free-forms} & \textbf{Bound form ASO} & \textbf{Shown as:} \\
		\midrule
		1dl \textit{ngadli} & \textit{-adli} & Optative mood \\
		2dl \textit{niwa} &\shadecell \textit{\bfseries -ing-wa} & Imperative mood \\
		3dl \textit{purla} & \textit{-ing-rla} & Imperative mood \\
		1pl \textit{ngadlu} & \textit{-adlu} & Optative mood \\
		2pl \textit{na} & \shadecell \textit{\bfseries -ing-a} & Imperative mood \\
		3pl \textit{parna} & \textit{-rna} & Imperative mood \\
		\lspbottomrule
	\end{tabularx}
	\legendbox[1]{$^\ast$The form of the 2sgERG pronoun shown here, \textit{ninthu}, follows \citet[136]{amery_kulurdu_2013}. The expected form according to reconstruction of Prono-PN \citep[62]{koch_languages_2014} is \textit{nintu}.\label{fn:5:60}}
	\caption{\label{tab:5:93}Bound pronouns in Kaurna. The forms shown in bold are not reclaimed as bound pronouns in Kaurna
	\citep[138]{amery_kulurdu_2013}.}
\end{table}

1sgS/O and 2sgS/O bound pronouns were not given in the imperative and optative paradigms, but their existence is attested elsewhere in the corpus. The anticipated form of the 1sgS/O bound pronoun \textit{–ai} is shown in the following clauses:

\ea\label{ex:5:21}
Kauwitya        kundowarponendi ai \\
\glt `I wish to have water' \\
\citep{teichelmann_outlines_1840} \\
\gll kauwi-tya       kuntuwarpu-rni-nthi-ai \\
water-\stockcaps{DAT}    {chest bone-\stockcaps{INCH}-\stockcaps{PRES}-1\textsc{sg}.\stockcaps{NOM}\footnotemark}                               \\
\glt `I long for water'
\footnotetext{Where \textit{kuntuwarpu-rni-} `chest bone-\textsc{inch}' is a metaphor for “desire” \citep[165]{amery_kulurdu_2013}}
\z


\ea\label{ex:5:22}
 Kundo       punggorendaii   	 ngaityo 		yungakko \\
\glt     	`I am concerned about, or long for my elder brother' \\
\citep{teichelmann_dictionary_1857} \\
\gll Kuntu	  pungku-rri-nth-ai   	  ngaityu		 yunga-ku \\
    chest	  hit-\stockcaps{REF}-\stockcaps{PRES}-1\textsc{sg}.\stockcaps{NOM}   1\textsc{sg}.\stockcaps{POSS}             {older brother-\stockcaps{DAT}}	\\
\glt `I am concerned about my older brother'
\z


Regarding the 2sgS/O form, the missionaries \citep[17]{teichelmann_outlines_1840} explained that the form of the 2sg imperative, when used with the intransitive verb \textbf{tikka} `to sit' was “the pure root of the verb”. While this bound pronoun was apparently not used in imperative commands, the anticipated form of the 2sgS/O bound pronoun \textit{–iina} is shown attached to an interrogative in the following example:

\ea\label{ex:5:23}
Ngando inna 		pulyunna    meyurlo	 anto-kartando 		yungk-i?  \\
\glt     `What black man has given you the kangaroo skin?' \\
\citep[68]{teichelmann_outlines_1840} \\
\gll Nganthu-iina                    pulyuna    miyu-rlu  	nantu-kartantu		yungki? \\
who.\stockcaps{ERG}-2\textsc{sg}.\stockcaps{ACC}          black        man-\stockcaps{ERG}  kangaroo-skin-[\stockcaps{ACC}]       give-\stockcaps{PAST}\\
\z


Current reclamation of the language \citep[138]{amery_kulurdu_2013} does not propose a complete set of bound pronouns, but rather suggests that the bound system was partially fused with the marking of imperative mood. The form 2pl \textit{–inga} attached to the verb in the following example is, for instance, currently interpreted as a single portmanteau morpheme marking mood and agreeing with the number and person of the agent. 

\ea\label{ex:5:24}
Itto		 ngaityo 		yungaitya	     kattinga \\
\glt `these bring, carry to my elder brother'
       
\gll itu                 ngaityu      	 yunga-itya	    kat-inga \\
       \stockcaps{DEM}.\textsc{pl}         1\textsc{sg}.\stockcaps{POSS}  	{older brother-\stockcaps{ALL}}   bring-\stockcaps{IMP}.2.\textsc{pl}                            \\
\z

The analysis taken here proposes that the verb is better glossed:

\ea
\gll Kat-ing-a \\
              Bring-?\stockcaps{IMP}-2\textsc{pl}.\stockcaps{ERG}                    \\

\z


The forms that are not reclaimed by \citet{amery_kulurdu_2013} as bound pronouns (shaded in \figref{tab:5:93}) are:

\begin{enumerate}
	\item Zero realisation of 3sg forms
	\item 2sgERG –\textit{nthu}.\footnotemark
	\footnotetext{The form of 2sgERG pronoun shown here follows that given by Amery \& Simpson (\citeyear{amery_kulurdu_2013}). The expected form according to reconstruction of Proto-PN \citep[62]{koch_languages_2014} is \textit{ngantu.}\label{chap5:fn10}}
	\item Those which follow the segment \textit{–ing} in imperative mood, 2dl \textit{–wa} and 2pl –\textit{a}. Although note that 3dl \textit{–rla} is reclaimed.
\end{enumerate}

While the bound form may commonly have occurred in an optative or an imperative construction, bound pronouns are also found in declarative statements. The 2sgACC form \textit{–iina} appears only to have been used in declarative clauses (Example \ref{ex:5:23}). There is ample evidence for the 2sgERG bound form \textit{–nthu} (see Examples \ref{ex:5:25} \& \ref{ex:5:26}).

While no examples of the bound pronouns 2dl –\textit{wa} and 2pl–\textit{a} have been located in a declarative clause, example \ref{ex:5:18} shows the 3dl form -\textit{rla} operating in a declarative clause without the preceding imperative marker–\textit{ing}.

It is possible that the forms 2dl –\textit{wa}, 2pl –\textit{a} have not been as well recorded, or as properly retrieved from the record, as were the other bound pronouns which have an obstruent onset. The initial liquid in 3dl –\textit{rlu} may also have been more easily discerned. 

\section{The syntax of complex clauses}
\label{sec:key:5.6}

As with other parts of their grammatical description, grammarians of the South Australian Lutheran school explained the processes of marking clausal dependency using methods that differed from those utilised by \citet{threlkeld_australian_1834}. Lutheran grammarians presented processes of clause subordination in discrete and sometimes unexpected sections of their grammars. Like Threlkeld and other early grammarians, they presented subordinating morphology within a discussion of “conditional” moods of the verb (\sectref{sec:key:5.6.2}). The description subordinate clauses as apprehensional constructions (\sectref{sec:key:5.6.1}) and under the heading “the relative pronoun” (\sectref{sec:key:5.6.3}) is, however, unique to the Lutherans' description.

\subsection{Teichelmann \& Schürmann’s “negative optative or preventative” mood}
\label{sec:key:5.6.1}

The apprehensional, lest, or aversive construction (Example \ref{ex:5:23}) is common to many PN languages. Verbs marking apprehensional constructions have a subordinating, non-finite usage \citep[87]{dixon_preface_2002} and often also a modal, finite usage (ibid.: 210). Apprehensional constructions are described in Teichelmann \& Schürmann’s earliest grammar of a South Australian language (\citeyear{teichelmann_outlines_1840}) in a listing of moods of the verb, where they were named the ``negative optative or preventative mood''. Later South Australian grammarians termed this “verbal mood” the “denunciative” \citep[no pag.]{koch_untitled_1868}, the prohibitive \citep[32]{taplin_1879a_nodate}, and the ``metutiv'' \citep[692]{planert_australische_1908}. Teichelmann and Schürmann’s work provides the most succinct analysis, and instructive account of the construction, which was not subsequently matched in the early descriptive era, nor by T. G. H. Strehlow’s account (\citeyear{strehlow_aranda_1944}) written on the cusp of the modern descriptive era \citep{Stockigt2021a}.

\citet[18--19]{teichelmann_outlines_1840} described the Kaurna apprehensional, or aversive, morpheme \textit{–tuwayi}, which attaches to a verb marking it as subordinate and indicating that the main verb action should be executed in order to avoid the action denoted by the verb that is marked as subordinate. They described the function of their “negative optative or preventative mood” well, stating: “This termination expresses that something will, may, or shall not take place, in consequence of another action” \citep[18]{teichelmann_outlines_1840}. They described the verb marked with the apprensional morpheme as non-finite when judiciously observing, “since this mood always depends on the proposition, there is no need for any tense in it, being always expressed by the tense of the proposition” (ibid.: 18).

Teichelmann \& Schürmann exemplified the construction within entire matrix clauses (Examples \ref{ex:5:25} and \ref{ex:5:26}), showing both main and dependent constituents. In each, the apprehensional clause immediately follows the main clause. It is not known whether other orderings were possible. Their description and exemplification of the subordinating usage of the apprehensional construction is good compared with that of many other early grammarians. 

\ea\label{ex:5:25}
Tarralyoanna   mutyertanna    wondando,		 yerta   buttonettoai \\
\glt `Put the clothes on the table, lest they be (or become) spoiled by the earth'\\
\citep[18]{teichelmann_outlines_1840} \\
\gll Tarraly-ana    mutyarta-rna         wanta-nthu,\footnotemark{}	 yarta-purtu-rni-tuwayi \\
 table-\stockcaps{ALL}     clothes-\textsc{pl}              put-2\textsc{sg}.\stockcaps{ERG}              earth-full-\stockcaps{INCH}-\stockcaps{AVERS}                            \\
\footnotetext{For discussion of the form of the 2sgERG pronoun, in this and the following example, see \fnref{chap5:fn10}}
\z



\ea\label{ex:5:26}
Yurrepaiaiandunna,                           kundattoai             parna. \\
\glt `You must pay attention to them (the goats) lest they kill (them)' \\
\citep[18]{teichelmann_outlines_1840} \\
\gll Yuri-payi-nthu-rna,     		kuntu-tuwayi	  parna\\
ear-examine-2\textsc{sg}.\stockcaps{ERG}-3\textsc{pl}.\stockcaps{ACC}	kill-\stockcaps{AVERS}	  3\textsc{pl}.\stockcaps{ACC}\\
\z

They also noted that “sometimes the first sentence is omitted, and must be supplied by the hearer”.
Teichelmann and Schürmann also exemplified the structurally similar purposive construction, which is common to PN languages marking “an action which happens by virtue of some earlier action, referred to in a previous clause” \citep[71]{dixon_preface_2002} as another mood of the verb. Under the heading ``infinitive mood'', \citet[19]{teichelmann_outlines_1840} explained: “No exclusive termination is yet known for this mood. Sometimes when an intention or purpose of an action is to be expressed, the termination \textbf{titya} (i.e., \textbf{itya}) is affixed”. They gave two examples: “I came in order to hear you speak” and example \ref{ex:5:27}. As in the apprehensional constructions, the purposive morpheme was shown attached to a verb in a subordinate clause that was exemplified alongside a main clause. It is not known whether the Kaurna purposive morpheme \textit{–titya} had modal non-finite function. 

\ea\label{ex:5:27}
Ngatto       punggetitya     wārpunna               pingga \\
\glt `I have made the dagger for the purpose of stabbing' \\
\citep[20]{teichelmann_outlines_1840} \\
\gll ng.athu    pungku-titya	warpu-rna	 pingka  \\
   1\textsc{sg}.\stockcaps{ERG}     kill-\stockcaps{PURP}  	dagger-\textsc{pl}-[\stockcaps{ACC}]   make-[\stockcaps{PAST}]                              \\
\z




\subsection{“The conditional or potential mood of the verb’}
\label{sec:key:5.6.2}

\citet[19]{teichelmann_outlines_1840} exemplified two types of conditional constructions in a discussion of the “conditional or potential” mood of the verb. Although \citet[40]{bleek_library_1858} described the annotations Teichelmann made to the copy of the 1840 grammar of Kaurna he sent to \citet{teichelmann_verb_1858} as extending “over the whole grammatical part”, it is this section that Teichelmann marked most heavily. 

The first construction exemplified in a discussion of verbal moods (Example \ref{ex:5:28}) is fairly certainly a hypothetical finite construction. The suffix \textit{–ma} is shown attaching to verb roots in two adjacent clauses. \citet[19]{teichelmann1940} perceived that the morpheme \textit{–ma} did not mark clausal dependency when stating: “this termination expresses not only the condition, but at the same time, the consequence”. In 1858, \citet[19]{teichelmann_verb_1858} described this inflection as “a particle expressing the possibility or if you like what might be the case [illegible]”. The suffix \textit{–ma} appears to be a finite verb inflection that occurs in the place of tense and marks the verb as conditional but not as subordinate. Following \citet[233]{wilkins_mparntwe_1989}, \textit{-ma} is glossed as hypothetical, after the Arandic morpheme \textit{–mere}, which marks the verb as irrealis but has no subordinating function. That this construction was firmly described by the missionaries as finite is relevant to later descriptions of Diyari made by Lutheran missionaries in South Australia, whose analysis appears to have been influenced by this section of Teichelmann and Schürmann’s work. 

\ea\label{ex:5:28}
Ninna  ngattaityangga  wānggama,          nindaitya aii            budnama.  \\
\glt  `If you had spoken to me, I would have come to you.' \\
\citep[19]{teichelmann_outlines_1840} \\
\gll Niina   ngathaityangka   wangka-ma,         ninthaitya-ai\footnotemark{}             pudna-ma \\
2\textsc{sg}.\textsc{s}    1\textsc{sg}.\stockcaps{COM}	           speak-\stockcaps{HYPO},      2\textsc{sg}.\stockcaps{DAT}-1\textsc{sg}.\stockcaps{ACC}     arrive-\stockcaps{HYPO}                               \\
\footnotetext{The form of the 2sgDAT pronoun, \textit{ninthaitya}, shown here follows \citet[137]{amery_kulurdu_2013}. The form is likely to be \textit{nintaitya} (see \citet[62]{koch_languages_2014}}
\z

The missionaries also provided this example elsewhere in the grammar: 

\ea\label{ex:5:29}
Niwa        yakko   ngarkoma,     niwa         yakko      padloma \\
\glt `If you two had not eaten,   you would not have died' \\
\citep[67]{teichelmann_outlines_1840} \\
\gll Niwa          yaku   ngarku-ma,    niwa          yaku     padlu-ma \\
    2\textsc{dl}.\stockcaps{NOM}    \stockcaps{NEG}   eat-\stockcaps{HYPO},    2\textsc{dl}.\stockcaps{NOM}    \stockcaps{NEG}  die-\stockcaps{HYPO}\\
\z

In both of examples (\ref{ex:5:28} \& \ref{ex:5:29}) two juxtaposed finite hypothetical clauses are translated as a conditional complex clause construction.

The second “affix” that Teichelmann \& Schürmann presented as marking the “conditional or potential mood of the verb” was introduced by the following statement: 

\begin{quote}
    Besides \textbf{ma}, another affix occurs … \textbf{ntyidla}, which, when added to the verb renders it either as a participle of the present tense, or a verbal substantive, but is frequently used in the sense of this mood. \citep[19]{teichelmann_outlines_1840}
\end{quote}

They supplied Example \ref{ex:5:30}, which shows \textit{-ma} attached to the verb of the second component and \textit{–ntyidla} to the verb of the first. The suffix \textit{-ntyidla} appears to mark the verb \textit{nguri}- “to throw” as both conditional and dependent. The main verb action \textit{yungku}- “to give”, marked with \textit{–ma}, might occur, if the dependent verb action were to occur.

\ea\label{ex:5:30}
Ngatto     ngurrintyidla,     ninna           yungkoma \\
\glt Were I permitted to throw, I would give (the bird) to you \\
\citep[19]{teichelmann_outlines_1840} \\
\gll Ngathu	 nguri-ntyidla,      niina           yungku-ma \\
1\textsc{sg}.\stockcaps{ERG}  throw-\stockcaps{COND},     2\textsc{sg}.\stockcaps{ACC}     give-\stockcaps{HYPO}\\
\z

Again, Teichelmann had more to say about these constructions in the annotations he sent to Grey: “\textbf{intyidla}, I do not doubt any more, is an affix which gives the verb the character of the participle, which may be affixed to any tense” \citep[22]{teichelmann_annotations_1858}. He added the following note comparing and clarifying the different syntactic functions of \textit{-ntyidla} and \textit{-ma}:

\begin{quote}
    The termination – \textbf{ntyidla} expresses that if a certain thing had taken place, another would have been the consequence, but because the one did not take place, the other would not be expected, or not follow. But the above instance where the condition is expressed by \textbf{–ma} the consequence too, is a simple conditional case which might or might not have happened. \citep[19]{teichelmann_annotations_1858}
\end{quote}

These processes of marking clausal dependency that \citet{teichelmann_outlines_1840} discussed under the heading “the conditional or potential mood of the verb” influenced the development of ideas about clausal subordination in the Lutheran sub-corpus (\sectref{sec:key:8.7.3}).

\subsection{``Relative pronouns''}
\label{sec:key:5.6.3}

The category “relative pronoun” is generally not included in the corpus grammars, other than to note that the forms were not to be found. \citet[327]{mathews_arranda_1907}, for example, wrote that there were no relative pronouns “in the Aranda tongue and in this respect it resembles all other Australian languages with which I am acquainted.” However, by functional analogy with the SAE relative pronoun, Lutheran missionaries, commencing with Teichelmann and Schürmann, and after them T. G. H. Strehlow (Stockigt, in Preparation), categorised certain subordinating clause types under the heading “relative pronouns.” 
Teichelmann \& Schürmann discussed other subordination processes under the heading ``relative pronoun'', based on their perception that those structures were functionally equivalent to the relative pronoun in European languages. Their presentation is a classic example of how foreign PN structures were presented as traditional grammatical categories, which conventionally convey structures that were perceived as functionally equivalent to the newly encountered PN structure (\sectref{sec:key:2.3.3}). The vacant slot “relative pronoun” was colonised by processes of clause subordination that Teichelmann \& Schürmann struggled to understand.

\citet[13]{teichelmann_outlines_1840} initially observed the absence of a relative pronoun – “if there be any, they are hitherto unknown” – before providing examples (\ref{ex:5:31}, \ref{ex:5:34}, \ref{ex:5:35}) to show how “the relation between two nouns is expressed.” Here the missionaries simply illustrated the construction without providing accompanying analysis. It is fortunate that they did so, because these examples add to our understanding of how complex clauses in Kaurna may have been formed, although the subordinating processes they exemplified remain poorly understood. With limited data, assigning function to morphemes remains speculative. The first example they gave was:

\ea\label{ex:5:31}
Ngurluntya      ai                   kunda,           tikkandi	urlo \\
\glt `That man struck me, who is sitting there' \\
\citep[13]{teichelmann_outlines_1840} \\
\gll ngurlu-ntya-ai		 kurnta,          tika-nthi-urlu \\
\textsc{dem}.\stockcaps{ERG}-\textsc{indef}-1\textsc{sg}.\stockcaps{ACC}    hit-[\stockcaps{PAST}]    sit-\stockcaps{PRES}-?\\
\glt `Someone hit me, the one sitting'
\z


Here the verb \textit{tika} `to sit', in the ostensible relative clause, is marked with the present tense morpheme \textit{–nthi}, which is in turn followed by a morpheme, \textit{–urlu}. Presumably the comma in the original shows two separate independent clauses, and the morpheme \textit{–urlu} is in some capacity marking some sort of clausal relation that the missionaries understood as relative in nature and which they conveyed though their awkward English translation. The function of \textit{-urlu} is not entirely clear (see \citealt[191--192]{amery_kulurdu_2013}).

\citet{teichelmann_verb_1858} later showed that \textit{–urlu} was an encliticised form of the ergative distal demonstrative \textit{ngurlu} \citep[9]{teichelmann_outlines_1840} that marked a temporal or causal relationship, which he translated as English relative constructions: ``when X, then…''. Examine the following seemingly dependent non-finite examples: 

\ea\label{ex:5:32}
manyarendurlo (manyarendi ngurlo) \\
when it rains then …  \\
\citep{teichelmann_verb_1858} \\
\gll manya-rri-nthi-urlu \\
rain-\stockcaps{RECIP}/\stockcaps{REFL}-\stockcaps{PRES}-?                        \\
\z

\ea\label{ex:5:33}
wakwakurlo ngurretti urlo (ngurlo) \\
\glt just when the boy had thrown then … \\
\citep{teichelmann_verb_1858} \\
\gll wakwaku-rlu     ngurru-thi-urlu \\
child-\stockcaps{ERG}         throw-\stockcaps{PAST}-?                    \\
\z

		

		
While this evidence does not show that the contracted form of \textit{ngurlu} was necessarily anaphoric, it is significant that \citet{teichelmann_verb_1858} translated these additional examples of \textit{–urlu} in dependent temporal clauses. 

However, in Example \ref{ex:5:31}, there appears to be no such temporal or causal relationship. The ergative enclitic –\textit{urlu} could be seen to be referencing the transitive subject of the initial clause \textit{ngurlu-ntya} `someone'. The same strategy seems to be at work in Example \ref{ex:5:34}.

The next clauses (Examples \ref{ex:5:34} \& \ref{ex:5:35}) are supplied under the heading ``relative pronoun'', given without explanation.

\ea\label{ex:5:34}
Idlo  atto          numa     nakkoma,       padlo        ngai      turnki              yungkoma        idlo \\
\glt `him I would love, who would give me clothing' \\
\citep[13]{teichelmann_outlines_1840} \\
\gll idlu-athu              numa   naku-ma,     padlu     ngai     turnki        yungku-ma  idlu \\
\stockcaps{DEM}.\stockcaps{ERG}-1\textsc{sg}.\stockcaps{ERG}  well        look-\stockcaps{HYPO},    3\textsc{sg}.\stockcaps{ERG}   1\textsc{sg}.\stockcaps{ACC}  cloth-[\stockcaps{ACC}]  give-\stockcaps{HYPO}    \stockcaps{DEM}.\stockcaps{ERG}                              \\
\glt `This one, I might love [him], he might give me clothing, this one'
\z


\ea\label{ex:5:35}
Ngatto  pa           wadli    nakko-ndi,     ngai         turnki             padlo       yakko      yungkondi \\
\glt `but him I hate who gives me no clothing' \\
\citep[13]{teichelmann_outlines_1840} \\
\gll Ngathu    pa           waadli	 naku-nthi,     ngai         turnki             padlu       yaku    yungku-nthi \\
1sg\stockcaps{ERG} 3\textsc{sg}.\stockcaps{ACC}  dislike    see-\stockcaps{PRES},   1\textsc{sg}.\stockcaps{ACC}   cloth-[\stockcaps{ACC}]  3\textsc{sg}.\stockcaps{ERG}   \stockcaps{NEG}   give-\stockcaps{PRES}                                \\
\z

The co-referentiality of the object of the first clause, “I hate him”, with the subject of the second clause, “he gives me no clothing”, is represented in translation as a relative clause: “but him I hate who gives me no clothing”. There is, however, no morphology marking clausal dependency in the original, and both verbs remain finite. A literal translation would therefore read: “Him I view badly. He does not give me clothes”. The co-referentiality is simply understood from the juxtaposition of the two clauses, and from the context, particularly from the fact that this matrix (\ref{ex:5:35}) follows an initial utterance (\ref{ex:5:34}) translated as: “Him I would love who would give me clothing”.

In this first construction (\ref{ex:5:34}) both verbs are marked with the morpheme –\textit{ma}, marking the verb as conditional without subordinating function (\sectref{sec:key:5.6.2}). The ergative demonstrative \textit{idlu} `this' \citep[9]{teichelmann_outlines_1840}, at the beginning of the utterance and then right at the end, appears to be coreferential with 3sgA pronoun \textit{padlu}, the agent of the second clause “he might give me clothing”, and to be establishing its referent as a salient argument. That the ergative form is used to track this argument suggests that the role of the agent in the second clause is being focussed on, and that the first clause is providing qualifying information.
Although it is difficult to be sure of the structures that Teichelmann \& Schürmann presented without explanation, it is clear that they perceived that demonstrative pronouns played a role in marking clausal dependency in Kaurna. Of the closely related Thura-Yura language Barngarla, of which Schürmann sadly provided no exemplification of clausal dependency, Schürmann wrote: 
\begin{quote}
    Relative pronouns there are none in this language, their place being supplied partly by demonstrative pronouns, partly by repetitions and circumlocutions. \citep[10]{schurmann_letter_1844}
\end{quote}


The notion that demonstrative pronouns, or pronouns more generally, had subordinating function recurs in later descriptions of South Australian languages, notably in \citegen{meyer_vocabulary_1843} description of Ramindjeri (\sectref{sec:key:6.1.2.9}) as does the description of other subordinating processes under the heading “relative pronoun” (see \sectref{sec:key:8.7.3.3}, \sectref{sec:key:9.3.5}).

\section{Concluding remarks}
\label{sec:key:5.7}

Teichelmann \& Schürmann’s grammar of Kaurna provides a detailed morphosyntactic analysis in comparison with other grammars in the corpus. It is similar in length to Symmons' grammar of Nyungar, which appeared the following year (\citeyear{symmons_grammatical_1841}; \sectref{sec:key:7.1}) but contains less detail than the published grammar that immediately preceded it (\citealt{threlkeld_australian_1834}; \chapref{chap:key:3}) or followed it (\citealt{meyer_vocabulary_1843}; \sectref{sec:key:6.1}). The presentation of paradigms of possessive pronouns (\sectref{sec:key:5.3.3}) demonstrates that Teichelmann and Schürmann had engaged deeply in the structure of the language in a relatively short space of time, as do the missionaries' recognition of irregularities in the way that bound pronouns marked imperative verbs (\sectref{sec:key:5.5}) and their inclusion of processes of marking dependent clauses (\sectref{sec:key:5.6}).

This earliest South Australian grammar of the language of the Adelaide Plains (1840) had a decisive and extended influence on subsequent grammars of languages spoken in South Australia, and indirectly elsewhere in the country. Teichelmann \& Schürmann’s descriptive influence is shown in following chapters to have persisted beyond the previously recognised “Adelaide School” \citep[410]{simpson_notes_1992}. Echoes of their descriptive responses to the case system (\sectref{sec:key:5.3.1}), nominal suffixes, “postpositions” (\sectref{sec:key:5.3.2}), the marking of clausal case on pronouns (\sectref{sec:key:5.3.3}), ergative case (\sectref{sec:key:5.4.2}), and clause subordination (\sectref{sec:key:5.6}) are found throughout the South Australian Lutheran sub-corpus, as well as in grammars written by Congregationalist missionary G. Taplin (\sectref{sec:key:7.3}). Importantly, Teichelmann \& Schürmann’s description of these areas of grammar differs from that given by Threlkeld in 1834, whose influence on Teichelmann \& Schürmann’s analysis was minimal.
