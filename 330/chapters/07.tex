\chapter{Other grammars of languages spoken in southern Australia}
\label{chap:key:7}

This chapter investigates remaining grammars written in the southern regions of Australia, albeit of a language belonging to the extreme southwestern region of the continent and of languages belonging to southeastern regions, and separated by some 3000 kilometres.

The chapter begins with an overview of C. Symmons' under-researched grammar of Nyungar (\citeyear{symmons_grammatical_1841}), spoken in southwestern Western Australia (\sectref{sec:key:7.1}). It then provides background to the grammatical material provided by missionaries working in Victoria that is presented in Volume II of R. \citegen{smyth_aborigines_1878} \textit{The Aborigines of Victoria}, which is devoted to language (\sectref{sec:key:7.2}).

The remainder of the chapter investigates the grammars of Ngarrindjeri, spoken near the mouth of the Murray River in South Australia, which were written by Congregationalist missionary G. Taplin (\citeyear{taplin_vocabulary_1867}; \citeyear{taplin_notes_1872}; \citeyear{taplin_native_1874}; \citeyear{taplin_grammar_1878}; \sectref{sec:key:7.3}). The discussion considers the influence of the Adelaide School (\chapref{chap:key:5} \& \ref{chap:key:6}) on Taplin’s analysis, especially the grammar of the closely related language Ramindjeri, described by Meyer (\citeyear{meyer_vocabulary_1843}; \sectref{sec:key:6.1}). Section \sectref{sec:key:7.3.4} presents Taplin’s case paradigms, which were reproduced by Moravian missionaries in Brough Smyth, and details the earliest usage of the term “ergative” (\sectref{sec:key:2.6}).

\section{Symmons' grammar of Nyungar \citeyearpar{symmons_grammatical_1841}}
\label{sec:key:7.1}\label{bkm:Ref336779457}\label{bkm:Ref74069483}

The earliest grammar of a language spoken in Western Australia (\citeyear{symmons_grammatical_1841}) was written by C. A. J. Symmons (1804–1887), the Protector of Aborigines, twelve years after the establishment of the Swan River Colony (\citeyear{symmons_grammar_1892}; renamed Perth in 1832). It is the only early grammar of a language spoken in Western Australia, other than R. H. Mathews' short publication on the same language (\citeyear{mathews_notes_1910}), which is based on Symmons, and Carl Strehlow’s rather superficial presentation of Luritja (\sectref{sec:key:9.2.3.3}), completed in the same year (\citeyear{mathews_notes_1910}).

\newpage
Symmons came from a colourful family of Welsh clergymen, scientists and politicians.\footnote{C. A. J. Symmons' uncle, Rev. C. Symmons (1749--1826), was a well-known poet and scholar, who obtained a Bachelor of Divinity at Cambridge University (1786). His brother-in-law, Sir Anthony Carlisle (1767--1840), was a fellow of the Royal Society, Surgeon to King George IV, curator of the Hunterian Museum, and the likely author of the gothic novel \textit{The Horrors of Oakendale Abbey} (1797).}  His father J. Symmons III (1745–1831) was a scientist, barrister and collector of books, art and botanical specimens, and a founding member of the Royal Society \citep{Conole2013}. Symmons arrived in the Western Australian colony in 1839 to take up an appointment as the Protector of Aborigines in 1840. He combined the position with the duties of magistrate until 1873. The grammar of Nyungar was printed in the \textit{Perth Gazette and Western Australian Journal} (1841) and in the \textit{Western Australian Almanac} in 1842. It was later republished by \citet{fraser_australian_1892}, who noted: “This short sketch of the Grammar of the language of Western Australia is the only one that I can find anywhere” \citep[48]{fraser_australian_1892}.

\citet[i]{symmons_grammatical_1841} acknowledged that the material for his grammar was “furnished by the Native Interpreter”, F. F. Armstrong (c. 1813--1897), who had arrived in the colony in 1829 as a teenager and had befriended members of the local population and learnt their languages. Armstrong became superintendent of the Native Institution (1834--1838). The Wesleyan Native School (1840--1845) was initially established in his home. In addition to informing Symmons' article (\citeyear{symmons_grammatical_1841}), Armstrong wrote a piece describing Aboriginal society in the \textit{Perth Gazette and Western Australian Journal} \citeyearpar{armstrong_manners_1836}. A “Perth” wordlist published in Curr (\citeyear{curr_australian_1886} vol. I: 334--335) was probably also collected by F. F. Armstrong. That it is attributed to C. F. Armstrong is likely to be a misprint.

\subsection{Symmons' analysis of Nyungar \citeyearpar{symmons_grammatical_1841}}
\label{sec:key:7.1.1}

There is some evidence that Symmons' analysis was written in collaboration with the then Governor of Western Australia, J. Hutt (1795--1880; \citealt[305]{ferguson_bibliography_1951}), who supported the education of Aboriginal people in the colony, and helped fund G. F. Moore’s descriptive vocabulary of Nyungar (\citeyear{moore_descriptive_1842}). It is not known whether Symmons or Hutt had access to the two published PN grammars existing at the time (\citealt{threlkeld_australian_1834}; \citealt{teichelmann_outlines_1840}).

Symmons' grammar evinces a sound understanding of the language, giving more example clauses than many other early Australian grammarians, including Taplin, Moorhouse and Ridley. The work gives the earliest description of pronominal sensitivity to kinship (\citealt[xiv-xv]{symmons_grammatical_1841}; \sectref{sec:key:6.2.1.4}) and describes what appears to have been an unusual and complicated pattern of ergative marking.

Symmons' work has received virtually no attention within the discipline of linguistics, and a biographical sketch of Symmons \citep{conole_nodate} fails to appreciate the significance of his grammatical analysis, describing his grammar as an “aboriginal language dictionary”.

The work is among the few comprehensive grammatical descriptions of Nyungar, the most widely available being Wilfred Douglas’s \citeyear{douglas_aboriginal_1968} grammar of the variety he called ``Neo-Nyungar'', which gives a synchronic description of the variety already having shifted syntactically under influence from English. Two years before Douglas' description, \citet[30]{ogrady_languages_1966} described that the “number of active speakers of Nyungic dialects was close to zero”. Remarkably, Symmons' 1841 grammar is the only published grammatical description which attempts to describe the language as it was spoken at the time of colonisation.

\subsubsection{Passive interpretation of ergative constructions}
\label{sec:key:7.1.1.1}\label{bkm:Ref73466178}

Like other early grammarians, Symmons described transitive constructions with an elided agent as passive (\sectref{sec:key:2.3.1}), explaining (ibid.: xix) that the passive could be conveyed by “an elliptical or defective form of the sentence” (\figref{fig:key:129}).


\begin{figure}
\includegraphics[width=.8\textwidth]{figures/stockigt-img075r.png}
\caption{Symmons' exemplification of “passive” constructions (\citeyear[xx]{symmons_grammatical_1841}; Nyungar)}
\label{fig:key:129}
\end{figure}

In his paradigm of nouns, Symmons named the ergative case in Nyungar “ablative” and placed forms marked with the ergative suffix \textit{-al} in the last position of conservative Latinate case paradigms (\citeyear{symmons_grammatical_1841}; \figref{fig:key:130}). It is possible that in doing so, he was influenced by \citet{teichelmann_outlines_1840}. Note, however, that Symmons treated pronouns in ergative case differently (below).

\begin{figure}
\includegraphics[width=.8\textwidth]{figures/stockigt-img076r.png}
\caption{Symmons' case paradigm of a noun (\citeyear[ix]{symmons_grammatical_1841}; Nyungar)}
\label{fig:key:130}
\end{figure}

Like \citet{meyer_vocabulary_1843}, \citet{symmons_grammatical_1841} translated ergative NPs as ``by X''. He consistently translated transitive clauses with an overtly marked agent, an object and a verb, which appears to show no derivational morphology (Examples \ref{ex:key:50} \& \ref{ex:key:51}) as English passive constructions (\figref{fig:key:131}). Like Meyer, he also described the transitive verb in a clause with an overtly marked agent as a ``participle''.

\newpage
\ea \label{ex:key:50}
	Bud-jor Yag-go-ål   bi-an-a-ga \\
	\glt The ground was dug by the woman \\
	(\citealt{symmons_grammatical_1841}: x, xxii) \\
	\gll \textbf{Bud-jor}          \textbf{Yaggo-ål}      \textbf{biana-ga}\\
	ground-[\stockcaps{ACC}]    woman-\stockcaps{ERG}   dig-\stockcaps{PAST}\\
	\glt `The woman dug the ground'
\z

\ea\label{ex:key:51}
	Ngan-ya   wau-gål-ål   bak-kan-a-ga
	\glt I was bitten by the snake \\
	(\citealt{symmons_grammatical_1841}: xiv) \\
	\gll \textbf{Nganya}   \textbf{waugål-ål}           \textbf{bakkana-ga}\\
	1\textsc{sg}.\stockcaps{ACC}  snake-\stockcaps{ERG}         bite-\stockcaps{PAST}\\
	\glt The snake bit me
\z

\begin{figure}
\includegraphics[width=.8\textwidth]{figures/stockigt-img077r.png}
\caption{Symmons' explanation of the “ablative” case (\citeyear[ix]{symmons_grammatical_1841}; Nyungar)}
\label{fig:key:131}
\end{figure}

While \citet{meyer_vocabulary_1843} did not interpret transitive clauses as passive constructions, Symmons did. Example \ref{ex:key:50} was given twice. First in order to demonstrate ``the use of the ablative'' \citep[x]{symmons_grammatical_1841}, i.e., ergative case, marked with -\textit{ål} on the stem \textit{Yag-go} `woman', and second (ibid.: xxii) to illustrate passive constructions.

Like Symmons, other corpus grammarians, including R. H. Mathews in some of his earlier grammars, interpreted AOV clauses with overtly marked agents as passive constructions. Grammarians who did this usually assigned the label “ablative” to the ergative case. \citet[193]{koch_r_2008} reproduces the following sections from two of the scores of Mathews' grammars. He points out that while Mathews' logic is topsy-turvy, he used the term “ablative” to name the agent of a transitive clause, because “the Ablative Case in Latin is used to express the agent of a passive clause” (\sectref{sec:key:5.4.3}):

\begin{quote}
The sense of the ablative is often obtained by means of the accusative case, thus, instead of saying, “The man was bitten by a snake,” a native says, a snake bit the man. (Thurrawal; \citealt{mathews_gundungurra_1901}: 133)
\end{quote}

\begin{quote}
The sense of the ablative is often obtained by means of the objective: “\textbf{Wud\-dungurr-a koongara buddhal}’–the dog the opossum bit; that is, the opossum was bitten by the dog. (Thoorga; \citealt{mathews_aboriginal_1902}: 53)
\end{quote}

Moorhouse (\citealt[24]{moorhouse_vocabulary_1846}; \sectref{sec:key:6.4}) also argued that an AOV clause was equivalent to the SAE passive construction, although he did \textit{not} label the ergative case ``ablative''. He wrote: “The English passive voice is not expressed by the inflection of the verb, but by the application of the active nominative case”, and that “[t]he existence of an active nominative supersedes the necessity of having a form for the passive voice”:

\ea\label{ex:key:52}
	Purnangunnanna laplapnanna ngape mukkarna
	\glt “Large knife me did wound” or its equivalent \\ `With a large knife I was wounded' \\
	\citep[24]{moorhouse_vocabulary_1846} \\
	\gll \textbf{Purnangu-nnanna} \textbf{laplap-nanna} \textbf{ngape} \textbf{mukk-arna}\\
	large-\stockcaps{ERG} knife-\stockcaps{ERG}  1\textsc{sg}.\stockcaps{ACC} wound-?\\
	\glt `The large knife wounded me'
\z

Taplin occasionally provided passive translations of transitive clauses. In this regard, Taplin followed Meyer (\sectref{sec:6.1.2.6}):

\ea\label{ex:key:53}
	Kile    yan          pettir
	\glt `by him it was stolen' \\
	\citep[88]{taplin_notes_1872} \\
	\gll Kili-yan             peth-ur\\
	3\textsc{sg}.\stockcaps{ERG}-3\textsc{sg}.\stockcaps{ACC}       steal-\stockcaps{PAST}\\
\z

But more frequently Taplin gave a passive “interlinear-style” translation and an active free translation:

\ea\label{ex:key:54}
	\gll {Ngate   yan   ellani} \\
	{By me it will be done} \\
	\glt `I will do it' \\
	\citep[88]{taplin_notes_1872} \\
	\gll ngati-yan      el-ani\\
	1\textsc{sg}.\stockcaps{ERG}-3\textsc{sg}.\stockcaps{ACC}    do-\stockcaps{FUT}\\
\z

\subsubsection{Symmons' account of ergativity}
\label{sec:key:7.1.1.2}\label{bkm:Ref74322244}

Symmons' description of ergativity in Nyungar is the most opaque of the early PN sources. Like sections of other early grammars which are difficult to decipher – for instance, Threlkeld’s description of pronouns in Awabakal (\sectref{sec:key:3.3.6.1}) – there are currently different interpretations of the system Symmons attempted to describe. \citet[54--55]{thieberger_linguistic_2004} describes the language as non-ergative, showing accusative alignment on all nominal types, with agents and subjects both marked with the suffix \textit{-il/-al.} \citet[65]{blake_case_1977} gives the ergative/locative/instrumental/dative suffix on nouns as \textit{-ak.} \citet[64]{douglas_aboriginal_1968} gives it as \textit{-il} in Neo-Nyungar, and \citet[313]{dixon_australian_2002} suggests that the ergative suffix on nouns was \textit{-al,} and shows 1sg and 2nonsg pronouns as tripartite (A/S/O). It is not clear to what extent these different interpretations are based on Symmons' account. Reliance on data provided in the \textit{Descriptive Vocabulary} (\citeyear{moore_descriptive_1842}) of G. Moore, who (ibid.: 72) described the grammatical structure of the Western Australian language as “simple - rudimentary - and not very copious” may also have contributed to the discrepancies.

But Symmons' discussion has more to offer a reconstruction of the system of marking syntactic case in the Whadjuk variety of Nyungar, spoken close to Perth soon after colonisation, than has currently been reclaimed. His interpretation of transitive clauses as passive constructions and resultant burying of the exemplification of ergative constructions in sections of the grammar unanticipated by a philologically ill-equipped modern reader has rendered his material thus far inaccessible. As with the different readings of Threlkeld’s description of pronouns in Awabakal, the difficulties in interpreting Symmons' account appear also to be the upshot of a system that was uniquely complicated, and which may not now be entirely retrievable.

Symmons material supports a fairly incontrovertible analysis in which animate nouns show tripartite marking (A/S/O) and inanimate nouns are ergatively aligned (A/SO). Examine clauses \ref{ex:key:50}, \ref{ex:key:51}, \ref{ex:key:55} and \ref{ex:key:56}.

\newpage
\ea\label{bkm:Ref516554518}\label{ex:key:55}
	Ya-go         my-ak-al        yu-gow                 bar-da-ga
	\glt `The woman has come to the house' \\
	\citep{symmons_grammatical_1841} \\
	\gll Yaku            miyak-al          yuka-w              parta-ka\\
	woman\textbf{{}-[\stockcaps{NOM}]} house-\stockcaps{ALL} stand-?\stockcaps{PRES}      house-\stockcaps{PAST}\\
\z

\ea\label{bkm:Ref516554630}\label{ex:key:56}
	Ngad-jo         yån-gor-in                   ngan-gow bru
	\glt `I do not see the kangaroo' \\
	\citep{symmons_grammatical_1841} \\
	\gll Ngatyu                 yankor-iny          ngana-?w-buru\\
	1sgA        kangaroo\textbf{{}-}\stockcaps{ACC}          see-\stockcaps{PRES}-\stockcaps{NEG}\\
\z

Making sense of Symmons' attempt to describe the system of marking the syntactic cases on pronouns is, however, more difficult. And here it is judicious to note that the current understanding of a linguistically interesting relationship between tense and syntactic case alignment in Pitta-Pitta (\citealt{blake_pitta-pitta_1971}: 84--90; \citealt{blake_australian_1979}: 193--196; \sectref{bkm:Ref514607076}) is partially dependent on the nineteenth century record left by W. E. Roth (\citealt{roth_ethnological_1897}; see \citealt[135--136]{breen_we_2008}). Symmons made numerous references to the role that the tense of the verb had to play in determining the forms of the 1sg and 2sg pronouns in Nyungar (see \citealt{Stockigt2017}). That tense may be a controlling factor has not, however, informed the reclamation of Nyungar. It is important to remember that Symmons had no impetus to record a sensitivity of case marking to verb tense. Since the likelihood of such a phenomenon would have been unknown to him, the factor motivating such an account is likely to have been an astute perception of the structure of the language.

\section{R. Brough Smyth \citeyearpar{smyth_aborigines_1878}}
\label{sec:key:7.2}\label{bkm:Ref456083856}

R. Brough Smyth (1830--1889), a civil servant and mining engineer, was appointed honorary secretary to the Board for the Protection of Aborigines in Victoria in 1860. He collated materials about Victorian languages that he published in the second volume, devoted to language, of the two-volume work \textit{The Aborigines of Victoria} (\citeyear{smyth_aborigines_1878}). The work, described by O’Grady \textit{et al} (\citeyear{ogrady_languages_1966}: 5) as “quantitively impressive but qualitatively appalling”, is among a handful of such materials produced within the first “survey-era of linguistics” in Australia (\citealt{ogrady_languages_1966}: 5; \citealt{mcgregor_introduction_2008}), of which Barry (1867) is the earliest and \citet{taplin_1879a_nodate} and \citet{curr_australian_1886} the most renowned. While Curr’s \textit{The Australian Race} spread the broadest net and is the most renowned of these survey works, Curr did not specifically elicit grammatical material from his informants as did Taplin, and as Brough Smyth appears also to have done.

Of the twenty-three informants whose work is included in Brough Smyth, only a small portion described morphosyntactic structure. Other than the grammatical description of Woiwurrung \citep[118--120]{smyth_aborigines_1878} from Melbourne’s Yarra River drainage by the Protector, W. Thomas (1774--1867), and the grammatical description of Ganai (ibid.: 24--31) recorded at Lake Tyers mission in Gippsland by Church Missionary Society missionary, J. Bulmer (1833--1913; \citeyear{smyth_aborigines_1878}: 24--31), grammatical material in Brough Smyth detailing languages from Victoria was contributed by Moravian missionaries. The Moravian grammatical contribution to Brough Smyth includes description of Wergaya from country close to the Ebenezer Mission (Lake Hindmarsh) in Western Victoria by A. Hartmann, F. W. Spieseke and F. A. Hagenauer (\citeyear{smyth_aborigines_1878}: 50--52, 56--58, 39--43 respectively).

The Moravian missionary presence in the colonies of Australia had been instigated by Charles J. LaTrobe (1801--1875), the first Lieutenant Governor of Victoria, who came from an influencial family of British Moravians \citep[57--62]{jensz_german_2010}. Moravian missions were first established in Victoria at Lake Boga (1850--56), Ebenezer (1858--1904), and Ramahyuck (Lake Wellington 1862--1907). After a failed attempt to establish a mission in South Australia – at the same location and at the same time (1866--1868) as the Lutheran’s ultimately successful missionary endeavour to the east of Lake Eyre (\sectref{bkm:Ref74134584}) – the Moravians set their sight on Northern Queensland (\sectref{bkm:Ref76120332}), where they established Mapoon (1891--1919), Weipa (1898--1932), and Aurukun (1898--1919).

The grammatical materials in Brough Smyth do not, however, constitute complete grammatical sketches. \citet{Blake2016} assesses Bulmer, Spieseke and Hagenauer’s analyses, and observes that the material Bulmer supplied in Brough Smyth does “not inspire any confidence in his ability to have mastered the language”. Bulmer’s paradigm of nouns (\figref{fig:key:7-143}) appears to show case marked by prepositions, a situation that does not accord with PN languages, or with \citeapo{mathews_aboriginal_1902} record of languages from the region. The grammatical materials contributed by Hagenauer (1829–1909) and Bulmer are nevertheless considered in this study because they include case paradigms that are significant to the history of the development of the term “ergative” (\sectref{sec:key:2.6}).

Hagenauer commenced missionary training at the Moravian institution Herrnhut in 1851. He arrived in Melbourne in 1858 with F. W. Spieseke (1820--1877), who was making his second journey to Australia, having returned to Germany in 1856 after the closure of the Moravian Lake Boga mission in western Victoria. Hagenauer moved to Gippsland in 1862, at the request of the Presbyterian Church, in order to establish a new mission at Lake Wellington, which he named Ramahyuck.

Bulmer arrived in Melbourne in 1853 and after working in the Victorian goldfields, where he witnessed the mistreatment of Aboriginal people \citep[166]{harris_one_1994}. He established Yelta (1854--1866), a Church of England mission, near the junction of the Murray and Darling rivers. In {1862} Bulmer was sent by the Church Mission Society to establish a mission on the south-eastern Victorian coast at Lake Tyers \citep[5--7]{shaw_194_nodate}.

In addition to his grammatical material in Brough Smyth, Bulmer provided a wordlist and some phrases in “Murray'', Marrawarra, a dialect of Baagandji spoken at Yelta \citep[33--37]{bulmer_language_1878}. Some of this material is presented alongside ``Gippsland'', Ganai spoken at Lake Tyers for comparison.

The presentation of Bulmer’s Ganai material (\citeyear{smyth_aborigines_1878}: 24--26) suggests that it was structured in response to a questionnaire. Content is given in answer to the questions: “In what way is the article expressed…?”, “Is there such a thing as gender …?”, “How is the plural formed?” etc. It is likely that Brough Smyth circulated a questionnaire, as both Taplin and Curr did in the same era. The original has, however, not been located.\footnote{\citet{blake_nineteenth-century_2015} suggests
    that Brough Smyth sent Bulmer a questionnaire in 1863.
}  That Bulmer and Hagenauer’s case paradigms (\citeyear{bulmer_language_1878}) (\figref{fig:key:7-142}; \figref{fig:key:7-143}) are identical, and reproduce Taplin’s paradigm for nouns (\textsc{nom} > \textsc{gen} > \textsc{dat} > \textsc{acc} > \textsc{voc} > \textsc{abl} > \textsc{exat} > \textsc{ergat}) published in 1872 and 1874, suggests that this lost questionnaire included a template paradigm taken from Taplin’s publications. It is, however, also possible, that this case paradigm was originally drawn up the Moravians and was then appropriated by Taplin (below).

\section{Taplin’s grammars of Ngarrindjeri (\citeyear{taplin_vocabulary_1867}; \citeyear{taplin_notes_1872}~[1870]; \citeyear{taplin_narrinyeri_1874}; \citeyear{taplin_grammar_1878})}
\label{sec:key:7.3}\label{bkm:Ref74069502}\label{bkm:Ref389146214}\label{bkm:Ref389145322}\label{bkm:Ref514608340}\label{bkm:Ref514608058}\label{bkm:Ref514607800}\label{bkm:Ref514606937}\label{bkm:Ref514606622}\label{bkm:Ref514606594}\label{bkm:Ref514605178}

Close to the time of the Lutheran missionaries' earliest encounters with Diyari (\chapref{chap:key:8}), spoken in the Lake Eyre Basin in northern South Australia, and when W. Ridley was publishing about central New South Wales languages (\sectref{sec:key:4.5}), Congregationalist missionary G. Taplin (1831--1879) was compiling his first MS grammar of Ngarrindjeri \citep{taplin_vocabulary_1867}, belonging to the estuarine regions of the Murray River in South Australia. A related variety of the same language had previously been described by Meyer (\citealt{meyer_vocabulary_1843}; \sectref{bkm:Ref74068587}).

With a private school education and intent on working as a missionary, Taplin arrived in Adelaide in 1849. He commenced work as a gardener for Rev. T. Q. Stow (1801--1862), founder of the Congregationalist Church in South Australia, from whom he received some training for ministry. In {1854} Taplin opened a school at Port Elliot in the Congregationalist Chapel. Here he worked among the Ngarrindjeri people, who were divided into numerous regional groups speaking related linguistic varieties (\citealt{yallop_narinjari_1975}: 2--3). Lutheran missionary Meyer had ceased missionary work and closed a nearby school at Encounter Bay eleven years before Taplin opened his school at Port Elliot.

\hspace*{-.8pt}In 1859, Taplin was appointed as “missionary agent” by the Aborigines' Friends' Association, an organisation founded the previous year. He established a mission on the eastern shores of Lake Alexandrina at Point McLeay (sometimes ``Point Macleay'', or Raukan), a site removed from other centres of European industry and which Taplin had himself recommended. The Congregationalist mission was a success in comparison with previous South Australian missionary endeavours, closing not until 1916, when control was assumed by the state. Fifteen years after the mission’s establishment, Taplin wrote: “The Narrinyeri possess (for Aborigines) a remarkable vitality, and I do not fear their speedy extinction as a nation” (\citeyear{taplin_taplin_1874}: 7).\footnote{Taplin’s remark was presumably made in response to a comment made by \citet[6]{bleek_inquiries_1874} published in the same report:  ``I thought it my duty to put aside for the time the, to me, very important work of a Comparative Grammar of the South African Languages, and to try to rescue, while it was still possible, something of the language and literature of this dying-out nation.''}

Taplin was visited by Moravian missionary G. Meißel for a period of eight months in 1865--1866. Meißel was among a group of four Moravians – the others being C. C. W. Kramer, W. Kühn, and H. Walder – who arrived in Melbourne at the end of 1864 in order to establish an inland mission on Coopers Creek in South Australia, beyond the northern edge of agriculturally viable land (\sectref{bkm:Ref74134584}).

During the nearly two years in which time the Moravians' inland journey was frustrated by permissions and unfavourable rainfall, the missionaries worked among Aboriginal people in Victoria and South Australia. Meißel travelled first to Ebenezer mission in Western Victoria with Walder, where Spieseke and Hartmann were stationed. In May 1865 the pair arrived in South Australia by ship and were invited by the Aborigines' Friends' Association to Point McLeay, where they spent two weeks with Taplin (\figref{fig:key:1.1.3-9}). Meißel visited again a second time alone \citep[210--212]{edwards_moravian_2007}, and it was presumably on this visit that he produced two watercolours titled “Mr Taplins School” (\textit{sic}; \figref{fig:key:7-132} \& \figref{fig:key:7-133}).

\begin{figure}[p]
% \includegraphics[width=.8\textwidth]{figures/stockigt-img078.tif}
\includegraphics[width=.8\textwidth]{figures/stockigt-img078-compressed.png}
\caption{“Süd Australien (1868)'', produced by Moravian missionaries showing “Walders Reisen 1865--1866'', with permission from Unitätsarchiv (Bd.30.11.a)}
\label{bkm:Ref77325656}\label{fig:key:7-77}
\end{figure}

\begin{figure}[p]
\includegraphics[width=.7\textwidth]{figures/stockigt-img079.jpg}
\caption{“Mr Taplins School Point Macleay'', watercolour, signed “G. Meißel'', with permission from Unitätsarchiv (Bd.30.14.b)}
\label{bkm:Ref76459178}\label{fig:key:7-132}
\end{figure}


\begin{figure}
\includegraphics[width=.8\textwidth]{figures/stockigt-img080.jpg}
\caption{“Mr Taplins School Point Macleay'', watercolour signed “G. Meißel(1866)'', with permission from Unitätsarchiv (Bd.30.17.a) Taplin’s analyses of Ngarrindjeri (\citeyear{taplin_vocabulary_1867}; \citeyear{taplin_notes_1872}~[1870]; \citeyear{taplin_native_1874}; \citeyear{taplin_grammar_1878}).}
\label{bkm:Ref76459209}\label{bkm:Ref73602819}\label{fig:key:7-133}
\end{figure}

Meißel stayed with Taplin the year before Taplin produced his earliest-known MS Ngarrindjeri grammar in 1867 and eight years after he established the mission. It is therefore possible that Meißel encouraged and had input into the structure of Taplin’s first Ngarrindjeri grammar. Since Meißel had earlier visited Ebenezer, where Spieseke, Hartmann and earlier Hagenauer were posted \citep[109--112]{edwards_moravian_2007}. It is possible that the Moravians” lost analyses of Wergaya presented a case paradign for nouns (nom > gen > dat > acc > voc > abl > exat > ergat) that influenced Taplin’s work through Meißel’s cross-fertilisation.

Ngarrindjeri is the only language from the Southern districts of South Australia to have survived colonial pressure long enough for missionaries to make substantial translations (see \citealt[122--123]{simpson_i_2008}). Within five years, Taplin had published extracts from the scriptures (1864) which were the first scriptures published in an Australian Aboriginal language. In {1874} his \textit{Native Book of Worship} was printed at the Southern Argus Printing Office in Strathalbyn. Coincidentally, the Southern Argus building, ``Argus House'', is currently owned by the present author and is where this study was largely researched and written.



\largerpage[-1]
Taplin made successive grammatical descriptions of Ngarrindjeri, producing four distinct representations of the language (\citeyear{taplin_vocabulary_1867}; \citeyear{taplin_notes_1872}; \citeyear{taplin_native_1874}; \citeyear{taplin_grammar_1878}). Like Ridley he published similar, but non-identical, analyses in different locations. His earliest grammar was produced as a MS nearly a decade after the mission opened (1867). The next analysis was first published in 1872, appended to Taplin’s \textit{Comparative Table of Australian Languages} (below). This published grammatical material is almost identical to an earlier MS “philological notes” (\citeyear{taplin_notes_1870}), which was attached to some comparative Australian vocabulary, both of which were reprinted by \citet[132--144]{grimwade_george_1975}. The {1872} publication omits discussion of some word-classes given in the 1867 MS, including adverbs, adjectives or prepositions. An even shorter version of this analysis, which omits an extended description of the verb, was republished 1874 \citep{taplin_narrinyeri_1874} and 1879 \citep{taplin_narrinyeri_1879}. Taplin’s final representation of the language was first published in 1878, and posthumously in 1880, with identical page numbers. At seventeen pages this is Taplin’s most comprehensive work.

Taplin’s last analysis has previously received detailed examination by Yallop \& Grimwade \citeyearpar{grimwade_george_1975}, who “attempt to widen our present understanding of the language” \citep[8]{grimwade_george_1975} by supplementing historical information collated from the speech of James Kartinyeri, described as “probably the last speaker of Narinjari” \citep[1]{grimwade_george_1975}. The authors' understanding of the language would have further been widened had they also utilised Meyer’s description.

Taplin sought to improve his understanding of the structure of the language until his death. In his last work he wrote: “I know that I am always discovering something in the language which I did not know before” (\citeyear{taplin_grammar_1880}: 6[1878]). Taplin’s successive grammars show that he was constantly reconsidering the best way to present grammatical structures. He produced three different case paradigms (\citeyear{taplin_vocabulary_1867}; \citeyear{taplin_notes_1872}; \citeyear{taplin_grammar_1878}). The alterations Taplin made in his successive grammars stand in contrast to other analyses made at the time. The analysis of Diyari made by missionaries at Bethesda mission, and Ridley’s grammars of Gamilaraay, show much less alteration over decades of reproduction.

\subsection{Taplin the philologist}
\label{sec:key:7.3.1}\label{bkm:Ref514609035}\label{bkm:Ref514607617}

In addition to publishing broadly on Ngarrindjeri language and culture while engaged in the demands of running a mission, Taplin took it upon himself to collate and publish comparative information about Aboriginal people in the South Australian colony. Aware that the linguistic material he was able to record as a missionary was of value to those tracing the history of human development, he compiled a comparative vocabulary of twenty Australian languages prior to the publication of any of his own work (\citeyear{taplin_notes_1870}, reprinted in \citealt{grimwade_george_1975}). When in 1870 Taplin sought to have his material published, he wrote to the Governor of South Australia, J. Fergusson (1832--1907):

\newpage
\begin{quote}
It has for some time been my conviction that some of the most difficult questions in Ethnology can only be answered when a very extended study of Aboriginal languages has been accomplished by scientific comparative Philologists in Europe. (Taplin to Governor Fergusson, quoted in \citealt{grimwade_george_1975}: 118)
\end{quote}

Fergusson considered Taplin’s comparative vocabulary with attached “philological notes” – i.e., with an attached MS grammar of Ngarrindjeri – as “worth transmission to England” \citep[119]{grimwade_george_1975}, resulting in Taplin’s earliest philological publication (\citeyear{taplin_notes_1872}). It is clear that Taplin was researching comparative philology in the 1860s, referring to the belief of the German philologist Max Müller that “the savage languages are of as much importance as those of more civilised races” (1870, quoted in \citealt{grimwade_george_1975}: 117). With the exception of Ridley (\sectref{sec:key:4.5}), who may be seen as Taplin’s counterpart in New South Wales, Taplin’s philological research was commenced at a time when little other enquiry into Australian Aboriginal languages was being undertaken elsewhere in the country.

\citet[84]{taplin_notes_1872} described his comparative table of Australian languages as having been “constructed so as to correspond as nearly as possible with the comparative table of Polynesian and Melanesian dialects found in Dr George Turner’s work”. In {1861} the London Missionary Society missionary Turner (1818--1891) had published a comparative vocabulary of Polynesian languages at the back of an authoritative work, \textit{Nineteen years in Polynesia} (\citeyear{turner_nineteen_1861}: opposite 536). The words Taplin sought from respondents in his 1874 circular (below) were also taken from \citet{turner_nineteen_1861} with some modification.

When in 1874 the then South Australian Governor, Sir A. Musgrave (1828--1888), received a request from W. Bleek for information concerning the manners, customs and folklore of the natives of the colony, Musgrave (quoted in \citealt{bleek_inquiries_1874}: 6) recommended Taplin as “one of the best informed men in the Colony on all subjects respecting the natives”. Musgrave forwarded Bleek’s initial letter to Taplin, whose letter of response was published with notes added by Bleek in the 1874 “Report of the sub-protector of Aborigines [South Australia]”. It followed an article by Bleek titled: “On enquiries into Australian Aboriginal Folklore”.

Prompted by the interest from this internationally acclaimed scholar, Taplin drew up a questionnaire, which was circulated throughout the colony to those in contact with Aboriginal people: missionaries, police troopers and pastoralists. The 1875 report of the Sub-Protector stated:

\begin{quote}
In the previous report from this Department, reference was made to a communication from Dr. Bleek of Cape Town … suggesting the desirability of steps being taken to collect aboriginal folk-lore and information of an ethnographic nature … With this view … a series of questions on these subjects were proposed by Mr Taplin, adopted, and embodied in a circular, about 100 copies of which were distributed early last year … affording a prospect that eventually valuable contributions will be … arranged in a collected form, and published.
\end{quote}

Taplin’s questionnaire was circulated close to the time that Ridley published his largest and final investigation of Gamilaraay \citeyearpar{taplin_report_1875} and close to the time that Brough Smyth was compiling data for inclusion in Volume II of \textit{The Aborigines of Victoria} (\citeyear{smyth_aborigines_1878}). Taplin’s survey collated the translations of English words taken from \citet{turner_nineteen_1861} and answers to a list of questions seeking specific information about Aboriginal cultural practices and language (\sectref{sec:key:2.3.1}) The results were published as \textit{The folklore, manners, customs, and languages of the South Australian Aborigines} \citeyearpar{taplin_1879a_nodate} that appeared \textit{after} Woods \citeyearpar{taplin_1879a_nodate}, in which Taplin also published (\citealt{taplin_narrinyeri_1879}; see \citealt{taplin_1879a_nodate}: 109).

\subsection{Influence from Meyer \citeyearpar{meyer_vocabulary_1843} and the Dresdners}
\label{sec:key:7.3.2}\label{bkm:Ref336429071}

Comparison of Meyer’s 1843 grammar of Ramindjeri with Taplin’s grammars of Ngarrindjeri show unequivocal influence of the former on the latter, not least in the paradigms of bound pronouns (Compare \figref{fig:key:99} with \figref{fig:key:7-134}).


\begin{figure}
\includegraphics[width=.8\textwidth]{figures/stockigt-img081.png}
\caption{Taplin’s paradigm of bound pronouns \citeyearpar[86]{taplin_notes_1872}}
\label{bkm:Ref456016370}\label{fig:key:7-134}
\end{figure}

In his final comment on Ngarrindjeri structure (\citeyear{taplin_grammar_1878}: 6), Taplin is unduly eager to marginalise the importance of Meyer’s influence on his work, describing the earlier missionary’s work as “a brave attempt to master the grammar of this language in 1843, and with some success, but yet his attempt presents a great number of ludicrous mistakes to one better acquainted with it.” Dixon has assessed the relative merit of the two grammars very differently:

\begin{quote}
Meyer’s grammar is full of wonderful insights; for example, he clearly recognises an antipassive … Meyer’s work was followed by \citet{taplin_1879a_nodate} an Anglican missionary of considerably lesser intelligence. Taplin criticised Meyer’s work but that did not hinder him from plagiarising large portions of it (those bits he could understand; he ignored other bits). \citep[7]{dixon_australian_2002}
\end{quote}

Unlike Meyer, Taplin did not analyse or exemplify the anti-passive construction (\sectref{bkm:Ref73965175}), presumably the section Dixon perceived that Taplin ignored. Note, however, that \citet[19]{taplin_grammar_1878} did discuss derivational morphology on the verb. He recognised the difference between a verbal root and one inflected with the detransitivising morpheme \textit{-el}, but did not demonstrate the alteration to the arguments of the verb (see \figref{fig:key:7-135}).


\begin{figure}
\includegraphics[width=.8\textwidth]{figures/stockigt-img082.png}
\caption{Taplin’s discussion of derivational morphology on the verb \citeyearpar[19]{taplin_grammar_1878}}
\label{bkm:Ref456016347}
\label{fig:key:7-135}
\end{figure}

In his final analysis \citeyearpar{taplin_grammar_1878}, Taplin made one major improvement to his earlier works. He included a description of a set of kinship terms that are “used in conjunction with the propositus term to refer to people in the third-person” (\citealt[304–305]{GabySinger2014}; \figref{fig:key:7-136}). In this, Taplin was probably guided by Morgan, via Fison, who made contact with Taplin in 1872 (\citealt{gardner_southern_2015}: 125). Question eleven of the survey of Australian Aboriginal people that Taplin circulated in 1874 (\sectref{sec:key:2.3.1}) asked: “What is the system of kinship in the tribe? Give names for following relationships” (\citeyear{taplin_1879a_nodate}: 6). An attached note further stipulated: “It is also desirable to discover whether there is not a slight variation of the word according as it is borne or attributed to the speaker; for instance, a variation for \textit{my} father, \textit{your} father, \textit{his} father, \& c”. Taplin had listed \textit{some} of these terms in his 1867 MS grammar (no pag.).

But even this addition to Taplin’s later publication (\figref{fig:key:7-136}) had previously been more extensively described by Meyer (\citeyear[11, 34--36]{meyer_vocabulary_1843}; \sectref{bkm:Ref330037209}).


\begin{figure}
\includegraphics[width=.8\textwidth]{figures/stockigt-img083.png}
\caption{Taplin’s paradigm of kinship terms \citeyearpar[10]{taplin_grammar_1878}}
\label{bkm:Ref456016447}\label{fig:key:7-136}
\end{figure}

Taplin’s grammars did not substantially add to Meyer’s 1843 analysis of the closely related variety Ramindjeri, other than suggesting some dialectal difference. None of Taplin’s grammars are as long or as intricately detailed as Meyer’s work, nor do they include the grammatically and culturally rich exemplification given by Meyer.

When tracing Taplin’s developing ideas about the best way to convey the Ngarrindjeri case system, it is striking that his final grammar (\citeyear{taplin_grammar_1878}) has more in common with the Dresdners' works than do his earlier grammars. Taplin’s descriptions of case, given in each of his different Ngarrindjeri grammars (\citeyear{taplin_vocabulary_1867}; \citeyear{taplin_notes_1872}; \citeyear{taplin_grammar_1878}) were differently influenced by aspects of the earlier grammars of South Australian languages.

One significant alteration to the presentation of case that Taplin made to his last grammar (\citeyear{taplin_grammar_1878}: 8) is the inclusion of an informal listing of the functions of case suffixes, without assigning them case-labels (\figref{fig:key:7-137}). In this he followed \citet{meyer_vocabulary_1843} and \citet{schurmann_letter_1844}.


\begin{figure}
\includegraphics[width=.8\textwidth]{figures/stockigt-img084.png}
\caption{Taplin’s later listing of case suffixes \citeyearpar[8]{taplin_grammar_1878}}
\label{bkm:Ref456016469}\label{fig:key:7-137}
\end{figure}

While Taplin’s earlier conventional paradigmatic presentation may have appeared more concise and pleasingly familiar to the general reader, the choice to present traditional paradigms lost the descriptive flexibility of Meyer’s prose-like presentation, which was able to illustrate different functions of the same suffix through exemplification on different nominal types. Given that the marking of Ngarrindjeri case functions appears to have been asymmetrical on different nominal types, the rigidity of Taplin’s earlier framework struggled to encapsulate the system. Taplin’s later presentation (\figref{fig:key:7-137}), while gaining some additional power to assign variant function to a single suffix, still lacked the multiplicity of example clauses that is characteristic of the richness of Meyer’s grammar. Of this new presentation of case, Taplin wrote:

\begin{quote}
	It is really difficult to say how many cases the nouns have, all prepositions are joined as affixes to the nouns to which they relate, but only some of them change their form, according as the noun is in the singular, dual or plural number. The following list of prepositional affixes and prepositions, shewing where they change in the dual and plural, and where they do not. \citep[8]{taplin_grammar_1878}
\end{quote}

In the above passage, Taplin suggests the classically conservative definition of case in which a true case suffix must also be a portmanteau morpheme conveying other categories. The marking of number appears to be modified by the marking of case in Ngarrindjeri (\citealt{horgen_languages_2004}: 94, 101). Whether Australian case systems constituted classes of declension had previously been considered by \citet[4]{schurmann_letter_1844} and by \citet[5]{moorhouse_vocabulary_1846}.

The passage also introduces another schema of the Adelaide School, but one that had been instigated by Teichelmann \& Schürmann and had \textit{not} been employed by Meyer. Taplin’s “prepositional affixes” and “prepositions” are Teichelmann \& Schürmann’s “postfixa” and “postpositions” (\sectref{sec:key:5.3.2}).

The explanation of ergativity that Taplin gave in his last grammar also shows an influence from the Dresdners that is not apparent in his earlier works.

\subsection{Ergativity}
\label{sec:key:7.3.3}\label{bkm:Ref516244764}

Taplin used the term “causative” to name the ergative marking on all types of nouns in some works (\citeyear{taplin_vocabulary_1867}; \citeyear{taplin_grammar_1878}) but only on pronouns in others (\citeyear{taplin_notes_1870}; \citeyear{taplin_notes_1872}; \citeyear{taplin_1879a_nodate}). Although Threlkeld had discussed the “agent causative of an action” (\citeyear{threlkeld_australian_1834}), the term “causative” was seldom used in early PN description. The use of the term “causative” to name the ergative case appears \textit{not} to have occurred outside Australia. Mathews later used the term in grammars of \citet{mathews_wiradyuri_1904} and Arrernte (\citeyear{mathews_arranda_1907}), as did J. M. \citet{black_vocabularies_1920} to name the ergative case on pronouns in vocabularies of South Australian languages. Both were probably influenced by Taplin, who innovated a range of case terminology.

In his two earliest grammatical analyses, Taplin provided no clear description of ergative function. He did not convey that he understood that subject nominals might be differently marked according to argument predication of intransitive and transitive verbs. He clarified the function of overtly marked nominals in ergative case only by translating them as `by X' in paradigms and in interlinear and free translations. That Taplin may not have clearly understood ergativity, at least initially, is suggested by the fact that he assigned different case labels to nouns and pronouns that are morphologically marked for ergative case. The only other corpus grammarian to assign different labels to ergative nouns and ergative pronouns was \citet{moorhouse_vocabulary_1846}. Taplin labelled ergative nouns “ablative'', but ergative pronouns “causative'', a difference which may have been intended to capture the instrumental function of nouns in “ablative” case. One of the linguistic questions which Taplin included in a survey which he circulated in 1874, however, indicates that by this stage he clearly understood how ergative function was likely to be morphologically marked:

\begin{quote}
What is the form of the declension of nouns? In the case of a word for “man,” how do they say … “by a man” (\textit{as an agent}) “by a man” (situated near a man)? (\citeyear{taplin_1879a_nodate}: 6; emphasis added).
\end{quote}


Taplin explained the difference between the nominative and ergative case for the first time in his last analysis of the language (1878: 14--19) under the heading “the verb''. Here he employed a descriptive strategy innovated by \citet{teichelmann_outlines_1840}, refined by Schürmann (\citeyear{schurmann_letter_1844}; \sectref{bkm:Ref336283270}), followed by \citet{moorhouse_vocabulary_1846}, but \textit{not} employed by Meyer. Taplin presented phrases showing the case forms of pronouns in different moods and tenses for intransitive verbs and then transitive verbs. He wrote:

\begin{quote}
The transitive verbs are distinguished from the intransitive by the former using the causative case of the pronoun; whereas the latter uses the simple nominative. This will be seen in the following conjugation of the verb. \citep[18]{taplin_grammar_1878}
\end{quote}

\subsection{Case paradigms}
\label{sec:key:7.3.4}\label{bkm:Ref456259132}


\tabref{tab:key:83} shows the different names assigned to case forms of singular nouns in grammars of Ngarrindjeri. From Taplin’s earliest analysis, his naming of the ergative case and his placement of ergative forms in the case paradigms was influenced by the Adelaide School. Like \citet{teichelmann_outlines_1840} and \citet{meyer_vocabulary_1843}, Taplin \textit{sometimes} labelled the ergative case “ablative” and placed ergative case forms in the traditional paradigmatic position of the Latin ablative. Like \citet{meyer_vocabulary_1843}, Taplin presented an extended seven case paradigm, excluding the vocative (\figref{fig:key:7-140}; \figref{fig:key:7-141}).

Taplin thought hard about the labelling of cases, reconsidering his options over time and inventing his own terminology to label cases not included in the classical case paradigm (\tabref{tab:key:83}). In relation to his re-presentation of Moorhouse’s Ngayawang case paradigm for instance, he wrote:

\begin{quote}
It has often lately suggested itself to me that, in the cases where we put ablative to these forms of words, the word locative would more exactly express the shade of meaning of the inflection. \citep[32]{taplin_1879a_nodate}
\end{quote}

While Taplin never engaged the case label “locative” in his grammars of Ngarrindjeri, the statement indicates his continuing search for better ways to present PN case systems.

\begin{figure}
\includegraphics[width=.4\textwidth]{figures/stockigt-img085.png}
\caption{Meyer's Ngarrindjeri case paradigm \citeyearpar[24]{meyer_vocabulary_1843}}
\label{fig:key:7-139}
\end{figure}

\begin{figure}
\includegraphics[width=.8\textwidth]{figures/stockigt-img086.png}
\caption{Taplin's earliest Ngarrindjeri case paradigm \citeyearpar{taplin_vocabulary_1867}}
\label{bkm:Ref456016644}\label{fig:key:7-140}
\end{figure}

\begin{figure}
\includegraphics[width=.7\textwidth]{figures/stockigt-img087.png}
\caption{Taplin's later case paradigm (1975~[\citeyear{taplin_notes_1870}]: 123; \citeyear{taplin_notes_1872}: 85; \citeyear{taplin_narrinyeri_1879}: 123)}
\label{bkm:Ref456016650}\label{fig:key:7-141}
\end{figure}

\begin{sidewaystable}
\footnotesize
\begin{tabularx}{\textwidth}{QQQQQQQQQQ}
\lsptoprule
\multicolumn{2}{c}{}
 & \multicolumn{8}{c}{ \textbf{Singular nouns}}\\
\multicolumn{2}{c}{{ \citet{meyer_vocabulary_1843}}

 1SG PRONOUNS}
 & \multicolumn{2}{c}{{ \textbf{Taplin}}

 \textbf{1867}}
 & \multicolumn{2}{c}{{ \textbf{Taplin}}

 \textbf{1870, 1872, 1879}}
 & \multicolumn{2}{c}{ \textbf{\citealt{Taplin1878,Taplin1880,Taplin1878}}}
 & \multicolumn{2}{c}{ \textbf{Taplin in \citealt{fraser_australian_1892}}}\\

Ngāpe & Nom

NOM & Korni & Nom

ABS & Korni & Nom

ABS & Porle & Nom

ABS & Korn-i

Porl-y & Nom 1

ABS\\
Ngān & Acc

ACC & Kornald & Gen

GEN & Kornald & Gen

GEN & Porlald & Gen

GEN & Korn-ald

Porl-ald & Gen

GEN\\
Ngāte & Abl

ERG & Kornank & Dat

DAT & Kornank & Dat

DAT & Porlangk

Porlangai & Dat

DAT & Korn-aģk

Porl-aģk

Porl-uģar & Dat 2

DAT\\
Ngañ-auwe & Gen & Korni & Acc

ABS & Korn & Acc

ABS & Porlil & Caus

ERG & Korn

Porl-y & Acc

ABS\\
Ngañangk & Dat

†† & Kornil & Caus*

ERG & Kornil & Abl

ERG & Porlenend & Abl

ABL & Korn-il

Porl-il & Abl 1

ERG\\
Ngañ-anyir & From

** & Kornanmant & Exative

ABL & Kornanmant & Exative

ABL &  &  & Korn-anmant

Porl-inend & Abl 2

ABL\\
Nan-āmbe & For

† & Kornanyir & Abl

ABL & Kornanyir

Kornald & Ergative

ABL &  &  & Korn-anyir

– & Abl 6

ABL\\
\lspbottomrule
\end{tabularx}
\legendbox{$^\ast$Crossed out ``ablative''.$^{\ast\ast}$\textit{-anyir} is reclaimed in this study as “associative''. The form is shown by Horgen as both ablative and locative.$^\dagger$This form is shown as allative by Horgen, \citet[24]{meyer_vocabulary_1843} suggests its function on pronouns is dative.$^{\dagger\dagger}$This form is shown by Horgen as dative on pronouns and locative elsewhere. \citet[24]{meyer_vocabulary_1843} suggests its function on pronouns is allative.}
\caption{\label{bkm:Ref329599964}\label{tab:key:83} Table showing names assigned to case forms of singular nouns in grammars of Ngarrindjeri. Forms are given in their original order. Cases in upper case designate the label assigned in the reclamation by \citet[95--96]{horgen_languages_2004}.}
\end{sidewaystable}


\hspace*{-1.4pt}In Taplin’s earliest case paradigm (\citeyear{taplin_vocabulary_1867}; \figref{fig:key:7-140}), he initially followed Meyer and named the ergative case “ablative” but then reconsidered the choice and crossed out “ablative” and inserted instead “causative''. The term “ablative” was thus reserved for a slot further down the paradigm, to label a peripheral case form translated as “with X” and marked with the suffix \textit{{}-anyir.} In this earliest case paradigm Taplin coined the term “exative” to name the case marked by a suffix –\textit{anmant}. The form had been included in Meyer’s informal paradigm of nouns (\figref{fig:key:95}) where it was translated as “from, out of X''.

In 1870, Taplin rearranged his paradigm (\figref{fig:key:7-141}) and returned to his original inclination to follow Meyer in naming the ergative case “ablative''. In need of a new label to name the case form suffixed with \textit{{}-anyir}, Taplin invented the term “ergative''. The case is still translated as “with X” but is now shown as additionally marked by the suffix \textit{{}-ald(e)}.

The identical paradigm for nouns (nom > gen > dat > acc > voc > abl > exat > ergat) was subsequently published by Moravians in Brough Smyth; by Hagenauer in a grammar of Wergaya, spoken in Western Victoria (\citeyear[43]{hagenauer_language_1878}; \figref{fig:key:7-142}) and by Bulmer in a grammar of Ganai spoken in Eastern Victoria (\citeyear[31]{hagenauer_language_1878}; \figref{fig:key:7-143}). It is, however, possible that the paradigm originates with Moravians and that Taplin accessed it via Meißel (\sectref{bkm:Ref456083856}; \sectref{bkm:Ref74069502}).

\begin{figure}
\includegraphics[width=.8\textwidth]{figures/stockigt-img088.png}
\caption{Hagenauer’s case paradigm (\citeyear[43]{hagenauer_language_1878}; Wergaya)}
\label{bkm:Ref456016978}\label{fig:key:7-142}
\end{figure}

\begin{figure}
\includegraphics[width=.8\textwidth]{figures/stockigt-img089.png}
\caption{Bulmer’s case paradigm (\citeyear[31]{bulmer_language_1878}; Ganai)}
\label{bkm:Ref456016988}\label{fig:key:7-143}
\end{figure}

In both grammars, nouns labelled “ergative” stand in peripheral case function and are translated as ``with X''. The same paradigm of Ngarrindjeri was also later republished in German by \citet[92]{eylmann_eingeborenen_1908}.


Taplin’s 1872 paradigm was reproduced in Fraser’s Ngarrindjeri material (\citeyear{fraser_australian_1892}), which also takes content from Taplin’s 1878 grammar. Fraser presents a paradigm not only for \textbf{korni} `man’, but also for \textbf{porle} `child’, which had only appeared in Taplin’s 1878 grammar. \citet[30--32]{fraser_australian_1892} abandoned the case terminologies developed by Taplin and renamed the cases using the system he had developed based on Threlkeld’s method of supplying numbered “ablative” and “dative” cases (\sectref{sec:key:3.3.3}; \figref{fig:key:7-144})


\begin{figure}
\includegraphics[width=.8\textwidth]{figures/stockigt-img090.png}
\caption{Fraser’s Ngarrindjeri case paradigm (\citeyear[30]{fraser_australian_1892}, from Taplin)}
\label{bkm:Ref456016861}\label{bkm:Ref336432860}\label{fig:key:7-144}
\legendbox{Note that Fraser showed the velar nasal as ``\.{g}''}
\end{figure}

\largerpage[2]
The asterisk that Fraser placed next to the “nominative 1” form refers to a passage (\figref{fig:key:7-145}) in which \citet[15]{fraser_australian_1892} gave an explanation of the case functions to which he assigned case labels. Note, however, that the ergative case form \textbf{korn-il} is still labelled “ablative” (ablative 1) by Fraser, who appears not to have recognised that this was the “nominative active'', ergative case form.


\begin{figure}
\includegraphics[width=.8\textwidth]{figures/stockigt-img091.png}
\caption{Fraser’s key to the functions of his case labels \citeyearpar[15]{fraser_australian_1892}}
\label{bkm:Ref456016832}\label{fig:key:7-145}
\end{figure}
