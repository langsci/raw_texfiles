\chapter{The peculiar nature of the language}
\label{chap:key:1}

When the Australian continent was colonised by the British in 1788, what is arguably a single family of languages \citep{bowern_computational_2012} termed Pama-Nyungan\footnote{The status of Pama-Nyungan as a phylogenetic entity has been much debated by Australian linguists. The notion has been contested by Dixon who, while recognising the group typologically, believes that diffusion of features between languages that have existed side by side for exceptionally long periods of time renders the comparative linguistic method invalid for Australia (\citeyear[225--226, 226--227]{dixon_languages_1980}, \citeyear[xix, 48, 53]{dixon_australian_2002}). His claims have been widely refuted (\citealt{bowern_anotherlook_2006, bowern_computational_2012}; \citealt{evans_enigma_1998}; \citealt[40]{koch_historical_2014}) and sometimes vehemently opposed. \citet*[69]{ogrady_coherence_2004} write: 

\begin{quote}
For decade after decade, Dixon … has persisted in the same wrong-headed assessment of the phylogenetic status of the large Pama-Nyungan group of Australian Aboriginal languages. His claim, which is extravagantly and spectacularly erroneous, is that it has no genetic significance in the wider Australian linguistic context. Moreover he denies that the Comparative Method can be applied to Australian languages.
This approach is so bizarrely faulted, and such an insult to the eminently successful practitioners of Comparative Method Linguistics in Australia, that it positively demands a decisive riposte.
\end{quote}} was spoken across a large proportion -- ninety percent -- of the landmass, from the islands off the tip of the continent's northeastern extremity, Cape York Peninsula, down across the entire southern portion of the continent. The estimated 290 Pama-Nyungan (henceforth PN) languages and their dialects \citep[817]{bowern_computational_2012} show a degree of grammatical similarity that stands in contrast to the linguistic diversity found in the remaining and comparatively much smaller areas in Australia’s north, northwest and Tasmania (see \tabref{tab:chap1:map} on p. \pageref{tab:chap1:map}).\footnote{The languages spoken in this smaller northerly region are referred to as non-Pama-Nyungan, a category defined by the features these languages do not share with Pama-Nyungan. The present day boundary between the Pama-Nyungan and non-Pama-Nyungan languages, although still subject to conjecture, was largely established by the American linguist Ken Hale (\citeyear{Hale_1961}). Hale coined the term Pama-Nyungan, by compounding words meaning “man” in the languages spoken in the recognised extremities of the Pama-Nyungan region; \textit{Pama} “man” in Cape York and \textit{Nyungar} “man” in the southwest of Western Australia. In employing this “man-man” naming method, Hale followed the practice employed by Schmidt when naming Australian linguistic subgroups \citep[20]{koch_arandic_2004}.}

This study assesses what was discovered about the morphology and syntax of Australian Aboriginal languages prior to the era of institutionalised academic investigation into Australian linguistic structure. By defining the body of historical grammatical records and examining the nature of the intellectual network in which appreciation of PN morphosyntactic structure evolved, this study traces the developing understanding and ability to describe Australian languages. The extent to which missionary-grammarians working in remote and diverse locations across the vast Australian continent had access to their predecessors’ descriptions, or alternatively, worked in intellectual isolation, has not previously been well understood.

The title of this chapter, “The peculiar nature of the language'' (\citealt[vii]{meyer_vocabulary_1843}; \citealt[1]{kempe_galtjintana-pepa_1891}), is drawn from two early grammars of Australian Aboriginal languages by Lutheran missionaries. These authors, and their fellow early Australian grammarians, encountered languages with word and phrase structures that were beyond what was considered linguistically possible. Australian Aboriginal languages were commonly described as “peculiar” (\citealt[x]{threlkeld_australian_1834}; \citealt[vii]{meyer_vocabulary_1843}; \citealt[57]{gunther_grammar_1892}; \citealt[lxii]{smyth_aborigines_1878}; \citealt[315]{haeckel_history_1876}). Haspelmath explains:

\begin{quote}
It was only towards the end of the 20\textsuperscript{th} century, as more and more had become known about the grammatical properties of the languages of the rest of the world, that linguists realised how peculiar the core European languages are in some ways when seen in the world-wide context. \citep[1492]{haspelmath_european_2001}
\end{quote}

Eurocentric linguistic understanding, informed largely by knowledge of standard average European languages (\citealt{whorf_relation_1941}; henceforth SAE), skewed early nineteenth-century perspectives of unfamiliar structures. When accounting for the distortion of Inuktitut syntactic structures in early grammatical descriptions, \citet{nowak_through_1993} puts it metaphorically, describing the language as having been viewed through a “looking glass”. 

In appraising lineages of PN grammatical description, this historiographic investigation focuses particularly on aspects of morphology and syntax that the early grammarians perceived as peculiar. It is the description of these areas of grammar, in which an early grammarian was theoretically and descriptively out of his depth, that evince reliance on the work of predecessors. The description of case systems, of ergative forms and function, and of processes of clause subordination have been found to be fruitful areas of historiographic enquiry. 

In seeking to “reconstruct the linguistic thought of earlier times” \citep[1]{mcgregor_introduction_2008}, this examination of early morphosyntactic description of PN languages is essentially an epistemological task. Although early wordlists, which were sometimes illustrated with sentence examples, and textual material contain much morphosyntactic content, these are excluded from this study, which confines itself to assessing the early \textit{analysis} of Australian morphosyntax.

Interest in the historiography of Australian languages is a fairly recent phenomenon. Due to Australia’s relatively late colonisation, much missionary linguistic effort in the country postdates the era before 1850, upon which global missionary linguistic historiography has focussed \citep[vii]{zwartjes_forward_2014}. In the introduction to the only edited volume dealing with the historiography of Australian languages, McGregor writes: 

\begin{quote}
[R]ather little has been written on the historiography of Australian Aboriginal linguistics. One might say that the subject has barely been born, though it has at least been conceived. \citep[2]{mcgregor_introduction_2008}
\end{quote}

While historiographic studies of Australian languages are becoming less rare \citep[122--123]{mcgregor_missionary_2008}, few works investigate the development of morpho\-syntactic description. As both \citet{newton_more_1987} and \citet[13]{mcgregor_encountering_2008} have pointed out, Australian linguistic histories have tended to be chronological over\-views of existing materials. When they have ventured into the historiographic realm, the focus has largely remained on the developing understanding of phonology and its orthographic manifestation (e.g., \citealt{austin_gamilaraay_2008}) rather than on the developing understanding of morphology and syntax. \citegen{newton_more_1987} own study of developing ideas about Australian Aboriginal languages before 1860 similarly compares early orthographies but does not probe the nature of the early grammatical analyses.

 A few works are unusual in illuminating how and why particular PN structures were misconstrued within developing understandings of morphosyntactic structure. They include \citegen{koch_r_2008} assessment of R. H. Mathews’ large body of grammatical material and Blake's (\citeyear{blake_nineteenth-century_2015}, \citeyear{blake_nineteenth-century_2016}) discussions of the early linguistic work carried out in the southeast of the country. But even these works, and others like them (e.g., \citealt{simpson_i_2008}; \citealt{wafer_waiting_2011}), do not attempt to place the analysis made by individual grammarians, or groups of grammarians, within a broader picture of early PN description.
 
Further, the role that Australian missionaries’ grammars played in German philologists’ understandings about ergativity in the late nineteenth and early twentieth centuries has remained unnoticed, indicating the infancy of Australian linguistic historiographic investigation. H. C. von der Gabelentz’s and F. Müller’s passive analyses of Australian ergativity \citep{stockigt_early_2015} pre-date Schuchardt’s 1896 passive reading of ergativity in Caucasian languages by some decades, but unlike Schuchardt’s work their role has been overlooked. Similarly, the earliest world-wide usages of the term “ergative” to describe both a peripheral case (\sectref{sec:key:7.3.4}) and the syntactic case (\sectref{sec:key:2.6}) occurred with the description of Australian languages and have only recently been recognised.

Within the near vacuum of Australian linguistic historiography, the influence of the earliest grammar of a PN language \citep{threlkeld_australian_1834} on later work has been assumed (e.g., \citealt[264--269]{carey_lancelot_2004}; \citealt[91]{simpson_i_2008}). Threlkeld’s published work is described as having been: 

\begin{quote}{}
[…] essential in establishing a framework for the study of his fellow-mis\-sion\-ary linguists including Watson and Günther at Wellington, and Meyer, Schürmann and Teichelmann in South Australia. He set a standard for other missionary work in the field which followed soon after. \citep[269]{carey_lancelot_2004}
\end{quote}

But without properly comparing the terminology and descriptive practices that Threlkeld employed with later descriptions of Pama-Nyungan languages, it is impossible to know which standards he may have set, and which aspects of his description were – and were \textit{not} – followed by later grammarians. 

McGregor’s \citeyearpar[2--13]{mcgregor_introduction_2008} “history of the histories of Australian languages” shows many Australian linguists agreeing that the 1930s was a watershed decade in Australian linguistic thought (\citealt{ogrady_languages_1966}; \citealt{capell_history_1970}; \citealt{capell_1972}; \citealt{dixon_languages_1980}; \citealt{blake_australian_1981}; \citealt{blake_introduction_1991}). It was then that a wave of urgency to record fast disappearing linguistic and ethnological data from people and cultures generally considered to be doomed to extinction (\citealt[325]{haeckel_history_1876}; \citealt[28]{harris_one_1994}; \citealt[21]{anderson_miserablest_2007}) activated systematic recording of Australian languages by the country’s academic institutions. McGregor establishes three periods of Australian linguistic description, the first before 1930, the second 1930--1960 and the third 1960--present (\citeyear[9--20]{mcgregor_introduction_2008}). The pre-contemporary era of Australian linguistic description (henceforth `early description') considered in this study coincides roughly with McGregor’s first era. This study examines works written between 1834, the year missionary Threlkeld published the first PN grammar, and 1910, when C. Strehlow wrote a comparative Arrernte and Luritja grammar. Not a single new grammatical analysis of a PN language was written between 1910 and 1930. Coincidentally, the first works to appear in the 1930s were of the same, or closely related, languages recorded by C. \citet{strehlow_notitle_1910}. These are a grammar of Western Arrernte, written by T. G. H. \citet{strehlow_aranda_1944}, who was C. Strehlow’s son, and R. Trudinger’s grammar of Pitjantjatjara (\citeyear{trudinger_grammar_1943}), which was based, in part, on J. R. B. \citegen{love_notes_1937} manuscript grammar of the language. The only other PN grammar known to have been produced during the same period is  a first grammar of Wik-Mungkan by U. McConnel (1888--1957), written close to 1940 (Peter Sutton, pers. comm.) based on fieldwork conducted between 1927 and 1934 \citep{sutton_explorations_nodate}.

Thoroughgoing linguistic investigation did not commence in Australia until the 1960s, when inaugural chairs of linguistics were established at Australian universities. Stephen Wurm (1922–2001) had been appointed senior fellow in linguistics at the University of Sydney in 1957, and Ulf Göran Hammarström (1922–2019) was appointed professor of linguistics at Monash University in 1965. Some academic investigation of Australian languages did, however, commence on the cusp of the first and second descriptive eras, within the discipline of anthropology at the University of Sydney, where A. R. Radcliffe-Brown (1881–1955) had been appointed Australia’s first chair of the discipline in 1926, and a little later at the University of Adelaide, by a multi-disciplinary team that in 1930--1931 formed “a small language committee ... [which formulated] after much consideration, a working list of phonetic symbols applicable to the general study of central Australian languages” \citep[261]{tindale_legend_1935}. The University of Adelaide language committee consisted of N. Tindale (1900--1993), C. Chewings (1859--1937), and J. A. FitzHerbert (1872–1970).

Fewer than ten percent of PN languages were grammatically described before 1930. European philologists commonly mentioned the scarcity of Australian linguistic material (e.g., \citealt[489]{gabelentz_uber_1861}; \citealt[241]{muller_reise_1867}, \citeyear[2]{muller_grundris_1882}; \citealt[403]{gabelentz_sprachwissenschaft_1891}). The Austrian linguist and ethnologist W. Schmidt (1868--1954; \citeyear[941]{schmidt_review_1946}) described “a desolate lack of scientifically recorded materials for most Australian languages”.\footnote{ “De[r] trostlose[] Mangel an wissenschaftlich aufgenommem Material […] bei dem grössten Teil der […] australischen Sprachen” \citep[941]{schmidt_review_1946}.} In a review of Schmidt’s classification of Australian languages (\citeyear{schmidt_personalpronomina_1919}) the American anthropologist A. L. Kroeber observed that ``the international contribution'' to Australian linguistics stood in contrast to the apathy of linguistic researchers in Australia:

\begin{quote}
    It is remarkable that there does not exist a single first-class monograph or body of material on any one of the native languages of this continent. This distressing fact should burn into the minds of all who profess interest in learning and science. Perhaps the realisation that the first scholarly attempt to deal seriously with these tongues was made in German by an Austrian priest will stir Australians into effort. \citep[226]{kroeber_australian_1921}
\end{quote}

The commencement of the second descriptive era of linguistic research in Australia (c. 1930--1960) is characterised as having focussed “strong attention on structural and typological features” \citep[17]{wurm_languages_1972}. Its onset is demarcated by three typological studies: the study of Australian languages made by the British linguist S. Ray (1858--1939; \citeyear{ray_aboriginal_1925}), who had previously written grammars of Paman languages (\citeyear{ray_study_1893}, \citeyear{ray_linguistics_1907}), and the studies produced in 1937 by the Australian linguistic researchers A. Capell (1902--1986) and A. P. Elkin (1891--1979), both of whom steered Australian linguistics towards the modern era (\tabref{tab:chap1:early} on p. \pageref{tab:chap1:early}). Ranging freely over the early descriptions of Australian languages, these three works collated and synthesised data presented in the primary sources examined in this study. They provide valuable insight into the understanding of PN structure that was discoverable at the time.

The early descriptions of Australian morphology and syntax examined in this study initially received little attention within the most recent era of grammatical description (1960--). Assessment of the value of the data contained in these antique records was understandably not prioritised by linguists in the modern era, who concentrated on documenting languages from the last generations of fluent speakers. The early grammars were most likely to be mentioned in accounts of “earlier work on the language” given in the introductions to modern grammars. The brevity of summations like the following assessment of \citegen{schwarz_koko_1900} and \citegen{roth_structure_1901} descriptions of Guugu{\hyp}Yimidhirr is not uncommon:

\begin{quote}
    All this work [in early sources] suffered from a basic misunderstanding of the sound system of the language […] and from a heavy reliance on grammatical categories derived from the study of European languages and decidedly inappropriate for an analysis of Guugu{\hyp}Yimidhirr. \citep[35]{haviland_guugu_1979}
\end{quote}

The grammatical description of dying Australian languages made during the last decades of the twentieth century by linguists such as Haviland now brings a wealth of valuable insight with which to assess the early grammars (\sectref{sec:key:2.1}).

Nevertheless, there has been a tendency within the third era of Australian linguistic description to dismiss the body of work considered in this study as having been produced by “amateurs” (e.g., \citealt[15]{dixon_languages_1980}; \citealt[4]{blake_introduction_1991}). The tendency to ignore, or to downplay, the contribution that missionaries made to the understanding of Australian linguistic structures sits within a general disregard of missionary linguistics, both within Australia \citep[260--261]{carey_lancelot_2004} and globally \citep[121]{mcgregor_missionary_2008}. 

Considering the training and preparation received by many of the early mis\-sion\-ary-grammarians, it is difficult to conceive of an intended meaning of the term “amateur” which is not overly anachronistic. Rev. W. Ridley (\sectref{sec:key:4.4}), who published on Gamilaraay spoken in New South Wales between the 1850s and the 1870s, was trained at King's College London, held an MA from the University of Sydney, and had learnt Gaelic. His first publication on Gamilaraay appeared in the London-based \textit{Transactions of the Philological Society}. Ridley is no better described as “amateur” than are some of the well-read “armchair” philologists in Europe, such as S. Ray (\sectref{sec:key:10.2}) and H. C. von der Gabelentz (1807--1874), who similarly described Australian languages with no formal philological or linguistic training, but who are not usually described as amateurs.\footnote{Ray’s first description of a language \citep{ray_study_1893} was written \textit{before} he visited the region.} While it is perhaps true that \textit{some} of the early grammars produced in Australia are not as professionally compiled as \textit{some} of the primary descriptions produced outside the country by men of academic standing – \citet{hale_languages_1846}, \citet{ray_study_1893}, and \citet{ray_linguistics_1907} – the insightful and sometimes \textit{inaugural} description of PN structures given in many of the Australian sources evince skilful analyses that are better described as “pioneering” works written by “forerunners” on a pre-theoretical descriptive frontier. To appropriate Elkin’s \citeyearpar[1]{elkin_r_1975} description of early anthropologists in Australia, we can say that the early grammarians may have been amateurs, but they were not necessarily amateurish.

The historiography of Australian linguistics complements recent investigations of the development of anthropological understandings of Aboriginal people (\citealt{weit_2004}; \citealt{kenny_arandas_2013}; \citealt{gardner_southern_2015}; \citealt{kelly_mcconvell_2018}). Regarding the historical treatment of Threlkeld’s legacy, Roberts observes: 

\begin{quote}
    Historians have made extensive use of the larger body of material, but have tended to gloss over [Threlkeld’s] linguistic works … their significance and their relevance have not been well understood by historians. \citep[108]{roberts_language_2008}
\end{quote}

This observation that Threlkeld’s linguistic achievements have not received adequate historical attention equally applies to other early Australian grammarians considered in this study. It is, for instance, surprising that despite making detailed study of diaries, letters and anthropological work, the major biographical studies of the missionary-grammarian C. Strehlow (\citealt{strehlow_tale_2011}; \citealt{kenny_arandas_2013}) and of his son T. G. H. Strehlow (\citealt{mcnally_aborigines_1981}; \citealt{hill_broken_2002}) do not enquire into the nature of their subjects’ grammatical analyses in order to properly probe their linguistic work for signs of influence and collaboration. While it is true that “there has been increasing fascination with the social and cultural evidence that missionary linguistics provides about interaction with indigenous peoples across the contact zone” \citep[13]{wafer_waiting_2011}, these early grammatical descriptions warrant closer interdisciplinary examination. Tracing philological influence through the comparative study of the early grammatical sources is of broad historical relevance.

Due to the high rate of linguistic extinction in Australia,\footnote{Australia’s indigenous languages have suffered a high rate of extinction since colonisation. The second National Indigenous Languages Survey (NILS 2) puts the number of “strong” Aboriginal languages that continue to be acquired by children in 2014 at 13, a decrease of five languages since the first NILS report in 2005 \citep[xii]{marmion_community_2014}.} the historic record of PN morphosyntax is of increasing significance within a number of fields of philological enquiry. Grammars that stand as the sole record of a lost language have received considerable attention from within the language reclamation and revitalisation movement that has gained considerable momentum among Aboriginal descendants of speakers \citep{hobson_re-awakening_2010}. For example, the revival of Kaurna \citep[2000]{amery_warrabarna_2016}, the language of the Adelaide plains, would have been impossible without the grammar produced by Lutheran missionaries Teichelmann and Schürmann (\citeyear{teichelmann_outlines_1840}). Similarly, the reconstitution of Awabakal (\citealt{lissarrague_salvage_2006}; \citealt{oppliger_phonology_1984}) is based on Threlkeld (\citeyear{threlkeld_specimens_1927}; \citeyear{threlkeld_australian_1834}; \citeyear{threlkeld_key_1850}).

Beyond the revival context, the optimal reclamation of material from the early grammars is crucial in the reconstruction of the Australian pre-contact linguistic landscape. These documents play a role in the description of endangered languages as linguists are “forced to turn to work by these linguistically naïve recorders to augment their own inadequate corpora” \citep[445]{mcgregor_early_2000}. Philological investigation of these early materials continues to contribute to the understanding of Pama-Nyungan typology (e.g., \citealt{dixon_australian_2002}) and to the ongoing internal classification of Pama-Nyungan languages (e.g., \citealt{bowern_computational_2012}). The documents considered in this study are precious and rare resources of increasing relevance. As such, they deserve careful and close scrutiny. 

One might conceive of a fourth era of Australian linguistic studies, commencing towards the end of the twentieth century, with a focus on the description of language contact varieties (\citealt[365--366]{meakins_2014}; \citealt{dickson_inpress}) and on the philological investigation of nineteenth-century records.

\section{The corpus of early PN description}
\label{sec:key:1.1}

\tabref{tab:1:1} summarises the corpus of early PN grammatical description, which covers twenty-four languages. It shows that most early grammars of PN languages were made by missionaries. With the exception of the otherwise anomalous grammars by the self-taught philologist and ethnologist R. H. Mathews (1841--1918), who published in numerous Australian and international journals (see \citealt{koch_r_2008}), only about one quarter of the grammars in the corpus (henceforth corpus grammars) were not written by missionaries. Although missionaries produced a substantial proportion of the corpus, it is important to realise that their efforts were unusual. Australian missionaries tended \textit{not} to investigate Aboriginal languages \citep[805]{harris_one_1994}. The more general disinterest in Aboriginal languages from missionaries sits within what missionary Threlkeld described as
\begin{quote}
    almost sovereign contempt with which the Aboriginal language of New South Wales has been treated in this Colony, and the indifference shown toward the attempt to gain information on the subject, are not highly indicative of the love of science in this part of the globe, and for which it is difficult to account. \citep[10]{threlkeld_key_1850}
 \end{quote}

With exception again to Mathews, whose numerous works substantially enlarge the body of “non-missionary” description, the grammars written by non-missionaries were either written outside the country (\citealt{ray_study_1893}; \citealt{ray_linguistics_1907}; \citealt{planert_australische_1907}, \citeyear{planert_australische_1908}; \citealt{gatti_lingua_1930}) or by men appointed to the office of Protector of Aborigines (\citealt{symmons_grammatical_1841}; \citealt{moorhouse_vocabulary_1846}; \citealt{roth_structure_1901}). While the duties of Protector extended to learning Aboriginal languages (\citealt[50]{jones_box_1996}; \citealt{blake_nineteenth-century_2016}), grammatical study was made only by those who happened to possess the required aptitude and inclination. 

\begin{table}\footnotesize
    \caption{The corpus of early grammatical descriptions of Pama-Nyungan languages. Shaded works are those that were dependent on earlier missionary analysis.}
    \label{tab:1:1}
    \begin{tabularx}{\textwidth}{QllQQQ}
        \lsptoprule
        Language & Year & Published & Author & Author's vocation & Language written in \\
        \midrule
        Awabakal & 1834 & ✔ & L. E. Threlkeld (1788--1859) & Missionary, London Mission Society & English \\
        Wiradjuri & c. 1835--1838 & Lost \stockcaps{MS} & W. Watson (1798--1866) & Missionary, Church Mission Society & English \\
         & c. 1835 & Lost \stockcaps{MS} & J. S. C. Handt (1783--1863) & Basel-trained, Church of English missionary & Unknown, probably German \\
         & 1838 \& 1840 & \stockcaps{MS} & W. J. Günther (1806--1879) & Basel-trained, Church of English missionary & English \\
         Kaurna & 1840$^{\dagger}$ & ✔ & C. G. teichelmann (1807--1888) \& C. W. Schürmann (1815--1893) & Missionaries, Lutheran, Dresden Mission Institute & English \\
         Nyungar & 1841 & ✔ & C. Symmons (1804--1887) & Sub-guardian of Natives, Swan River Colony (Perth) & English \\
         Barngarla & 1844a$^{\dagger}$ & ✔ & C. W. Schürmann (1815--1893) & Missionary, Lutheran, Dresden Mission Institute & English \\
         Ramindjeri & 1843$^{\dagger}$ & ✔ & H. A. E. Meyer (1813–1862) & Missionary, Lutheran, Dresden Mission Institute & English \\
         \lspbottomrule
         \end{tabularx}
        \legendbox{$^\dagger$These grammars are of languages spoken in South Australia, which until 1911 included the Northern Territory and were made by Lutheran missionaries.}
\end{table}

\begin{table}\footnotesize
    \begin{tabularx}{\textwidth}{QQlQQQ}
        \lsptoprule
        Language & Year & Published & Author & Author's vocation & Language written in \\
        \midrule
        Ngarrindjeri & 1867 & \stockcaps{MS} & G. Taplin (1831--1879) & Missionary, Congregationalist & English \\
        & 1870, 1872, 1874, 1879 & \stockcaps{MS} & & & \\
        & 1878, 1880 & ✔ & & & \\
         Ngayawang & \shadecell 1846 &\shadecell  ✔ & \shadecell  M. Moorhouse (1813--1876) & Protector of Aborigines, South Australia & English \\
         Gamilaraay and Turrubul & 1866 & ✔ & W. M. Ridley (1819–1878) & Missionary, Presbyterian & English \\
         Gamilaraay & 1856b 1855a & ✔ \stockcaps{MS} & & & \\
         & 1875 & ✔ & & & \\
         Diyari & 1868$^{\ddagger}$ $^{\dagger}$ & \stockcaps{MS} & W. Koch (1848--1869) & Teacher at the Lutheran mission & German \\
         & 1947 [1872]$^{\dagger}$ & \stockcaps{MS} & C. Schoknecht (1841--1905) & Missionary, Lutheran, Hermannsburg Mission Society & German \\
         & 1880$^{\dagger}$ &	\stockcaps{MS}	& J. Flierl (1858–1947) & Missionary, Lutheran, Neuendettelsau & German \\
         & 1981 [1899]$^{\dagger}$ &	\stockcaps{MS}	& J. G. Reuther (1861–1914)	& Missionary, Lutheran, Neuendettelsau & German \\
         & \shadecell  1908 & \shadecell  ✔	& \shadecell  W. Planert (1882–post 1940) & Student of Prof. Luschan, at the Royal Museum for Ethnology, Berlin & German \\
         \lspbottomrule
         \end{tabularx}
        \legendbox[.8]{$^\dagger$These grammars are of languages spoken in South Australia, which until 1911 included the Northern Territory and were made by Lutheran missionaries. $^\ddagger$This grammar is assigned to W. Koch, a teacher at the mission, on internal historical evidence (\sectref{sec:key:8.3.4}). The document does not attribute authorship.}
\end{table}

\begin{table}\footnotesize
    \begin{tabularx}{\textwidth}{QllQQQ}
        \lsptoprule
        Language & Year & Published & Author & Author's vocation & Language written in \\
        \midrule
        &\shadecell  1930 &\shadecell  ✔ & \shadecell G. Gatti (dates unknown) & Italian linguist & Italian \\
        Wangkangurru & 1880$^{\dagger}$ & \stockcaps{MS} & J. Flierl & See above & German \\
        & 1981 [1901]$^{\dagger}$ & \stockcaps{MS} & J. G. Reuther & See above & \\
        Yandrruwandha & 1981 [1901]$^{\dagger}$ & \stockcaps{MS} & J. G. Reuther & See above & German \\
        Ganai & 1878 & ✔ & J. Bulmer (1833–-1913) & Missionary, Church of England & English \\
         Wergaya & 1878 & ✔ & F. A Hagenauer (1829–-1909) & Missionary, Moravian & English \\
         Minjangbal & 1892 & ✔ & H. Livingstone (dates unknown) & Missionary, Presbyterian & English\\    
         Western Arrernte & 1891$^{\dagger}$ & ✔ & H. Kempe (1844–1928) & Missionary, Lutheran, Hermannsburg Mission Society & German \\
         & \shadecell 1907 &\shadecell  ✔ & \shadecell  W. Planert & See above & German \\
         & 1910	& \stockcaps{MS} & C. Strehlow (1871–1922) & Missionary, Lutheran, Neuendettelsau & German \\
         & 1931 [c.1907]$^{\dagger}$	& \stockcaps{MS} & C. Strehlow & See above & German \\
         & 1931 [c.1923]$^{\dagger}$	& \stockcaps{MS} & J. Riedel (1885–1961) & Missionary, Lutheran, Neuendettelsau & \\
         & 1908$^{\dagger}$ & ✔ & C. Strehlow & See above & German \\     
         Luritja & 1910 & \stockcaps{MS} & C. Strehlow & See above & German \\
         Pitta-Pitta & 1897 & ✔ & W. E. Roth (1861–1933) & Medical doctor & English \\
         \lspbottomrule
         \end{tabularx}
        \legendbox[.8]{$^\dagger$These grammars are of languages spoken in South Australia, which until 1911 included the Northern Territory and were made by Lutheran missionaries.}
\end{table}

\begin{table}\footnotesize
    \begin{tabularx}{\textwidth}{QllQQQ}
        \lsptoprule
        Language & Year & Published & Author & Author's vocation & Language written in \\
        \midrule
          {Kalaw Lagaw Ya}, Western Torres Straits &\shadecell  1893$^{\ast}$ &\shadecell  ✔ &\shadecell  S. Ray (1858–1939)  \& A. Haddon (1855–1940) & Linguist (Ray) Cambridge biologist (Haddon) & English \\
        \midrule
         & 1907$^{\ast}$ & ✔ & S. Ray &\multirow{3}{=}{\raggedright Linguist, Cambridge Anthropological Expedition to the Torres Straits} &English \\
         Yadhaykenu	& 1907$^{\ast}$	& ✔	& S. Ray &  & English \\
        \cmidrule{1-4}\cmidrule{6-6}
         {Guugu{\hyp}Yimidhirr} \& Nggerrikwidhi	&\shadecell  1907$^{\ast}$ &\shadecell  ✔	& S. Ray &   & English\\
        \midrule
         {Guugu{\hyp}Yimidhirr} & 1900$^{\ast}$\ $^{\ast}$ & \stockcaps{MS} & G. H. Schwarz (1868–1959) \& W. G. F. Poland (1866–1955) & Missionaries, Lutheran, Neuendettelsau & German \& English \\
        \midrule
        &\shadecell  1901 &\shadecell  ✔ &\shadecell  W. E. Roth & Protector of Aborigines for the Northern District of Queensland & English \\
        Nggerrikwidhi  & 1903 & ✔ & N. Hey (1862–1915) & Missionary, Moravian & English \\
        \hline
        \multicolumn{6}{c}{\parbox{\textwidth}{R. H. Mathews’ grammars of some dozen Pama-Nyungan languages, predominantly from the southeast of the continent.}} \rule{0pt}{12.5pt}\rule[-8.5pt]{0pt}{0pt}\\ % The two \rule commands are struts -E.
        \hline
        \bottomrule
    \end{tabularx}
    \legendbox[.8]{$^\ast$Grammars by \citet{ray_study_1893} and \citet{ray_linguistics_1907} are presented within an extensive classification of languages of the Torres Strait and thus differ from other works considered in the corpus by being written for historical and comparative purposes. ${^\ast}{^\ast}$ Earlier grammatical notes on Guugu{\hyp}Yimidhirr written in German by Poland in a letter sent to Neuendettelsau (18/08/1889; \citealt[133]{haviland_how_1980}) have, unfortunately, not been sourced for the purpose of this study.}
\end{table}

A relatively small number of early grammatical descriptions are excluded from the corpus on the grounds that they do not contain detailed enough description to warrant comparison with other sources. The criterion for inclusion is that a work contains a reasonably comprehensive description – for the era and relative to other grammars – of \textit{both} nominal and verbal morphology. Notable exclusions include a description of Dharuk (\citeyear{dawes_grammatical_1790}; \citeyear{dawes_vocabulary_1790}) spoken in Sydney made by W. Dawes (1762--1836), C. Smith’s description of Bunganditj spoken in the southeast corner of South Australia (\citeyear{smith_booandik_1880}) – the material for which was supplied by Smith's son, Duncan Stewart – J. Dawson’s descriptions of Jab-Wurrung and Peek-Whurrung spoken in western Victoria (\citeyear{dawson_australian_1881}), and W. Thomas’ analysis of Woiwurrung spoken in Melbourne \citep[118--120]{hagenauer_language_1878}. These works tend to be sketchy grammatical notes appended to larger vocabularies or ethnographic descriptions. 

It should be noted, however, that Dawes’ grammatical notebook (\citeyear{dawes_grammatical_1790}; \citeyear{dawes_vocabulary_1790}) is remarkable within the history of Aboriginal language description, in having been written so soon after the colonisation of New South Wales and decades before any subsequent grammatical description is known to have occurred. While Dawes \textit{does} attempt to conjugate numerous verbs, the work does \textit{not} make any description of case morphology on nouns or on pronouns. Thus, Threlkeld’s grammar of Awabakal (\citeyear{threlkeld_australian_1834}) is described in this study as the earliest Australian grammar. 

Although T. G. H. Strehlow’s MA thesis, “An Aranda grammar” \citep{strehlow_aranda_1938}, published in \textit{Oceania} as part of “Aranda Phonetics and Grammar” (\citeyear{strehlow_aranda_1944}), sits outside the timeframe of this study (1834--1910), the work is difficult to ignore. There are seven grammars of Arrernte written by five different authors that predate T. G. H. Strehlow’s long and detailed analysis, including three by his own father C. Strehlow (\citeyear{strehlow_grammatik_1931}; \citeyear{strehlow_einige_1908}; \citeyear{strehlow_notitle_1910}), who was the missionary at Hermannsburg mission where T. G. H. Strehlow grew up (\citealt{hill_broken_2002}; \citealt{strehlow_tale_2011}). Arrente has a longer analytical history than any other Australian language \citep[159]{green_altyerre_2012}. The time depth of Diyari grammatical description, commenced by Lutheran missionaries in the late 1860s \citep{koch_untitled_1868}, is shorter than that of Arrernte, commenced by Lutheran missionaries in the late 1870s \citep{kempe_grammar_1891}, only because Arrernte has continued to be spoken and grammatically described throughout the modern descriptive era, whereas the last speakers of Diyari were recorded in the 1970s \citep{austin_grammar_1978}. Arrernte is in fact the only corpus language that continues to be acquired by children. As such, the language provides a rare opportunity to make a longitudinal study of the history of description of a single language. \citet{Stockigt2017,Stockigtinpress} examines the ways in which T. G. H. Strehlow’s analysis of Arrernte articulates into a tradition of descriptive practice established by his Lutheran forefathers.

\subsection{The naming of languages}
\label{sec:key:1.1.1}

The naming of Aboriginal languages and the way they are spelled is not straightforward or apolitical (\citealt{sutton_australian_1979,RosenbergBowerninPress}). The terminology employed in this study is chosen primarily for ease of reference. Many of the names used to refer to languages – Awabakal, Kaurna, Ngarrindjeri and Arrernte – are the product of post-colonial linguistic and anthropological investigation. They were not recorded in the early sources. The language Threlkeld described was first named “Awabakal” by J. \citet[v]{fraser_australian_1892}, the term being derived from the name of Lake Macquarie \textit{Awaba} marked with the associative suffix –\textit{kal}. The language is nowadays also referred to as “Hunter River and Lake Macquarie language” or “HRLM” (\citealt{lissarrague_salvage_2006}; \citealt{wafer_waiting_2011}). The title of Threlkeld’s earliest grammar: \textit{Australian grammar comprehending the principles and natural rules of the language spoken by the Aborigines in the vicinity of Hunter’s River, Lake Macquarie, \&c }(\citeyear{threlkeld_australian_1834}) is typical of many works in the corpus in identifying the language by the location in which it was spoken. Examine, for example, the titles of Threlkeld (\citealt{threlkeld_specimens_1927}, \citeyear{threlkeld_key_1850}), \citet{teichelmann_outlines_1840}, \citet{meyer_vocabulary_1843}, \citet{moorhouse_vocabulary_1846}, \citet{hagenauer_language_1878}, and \citet{kempe_grammar_1891}. The mechanisms by which many language names – with all their variant spellings\footnote{See for instance, \citet[8]{austin_reference_1993} for a list of the twenty-one spellings of the language name “Gamilaraay”, to which this investigation adds “Gammilurai” (\sectref{sec:key:4.1.1}).}  – came to be the accepted descriptors of languages, and of the people speaking them, can be difficult to retrieve from the historical record and remain generally not well understood. \citet[89]{sutton_australian_1979} comments: “[T]he question of what language names or labels actually refer to, and how they function in Aboriginal societies, has in general been neglected”. So too have the processes by which the nomenclature has developed since colonisation and has been assumed by Aboriginal people, who now identify as owners of newly named varieties, which are themselves sometimes post-colonial constructs. Section 5.1.1 discusses the evolution of the term “Kaurna” (see \citealt{Stockigt2021a,Stockigtinpress} for discussion of the evolution of the term Arrernte).

While the study examines only Pama-Nyungan languages, the corpus languages may be referred to as ``Australian'' rather than ``Pama-Nyungan'' in order to better reflect the early grammarians’ point of view. Although the division of mainland Australian languages into two high level groups – \textit{Sprachstämme} – was recognised in very early philological publications (\citealt{bleek_library_1858}; \citealt[241]{muller_reise_1867}) and later by W. Schmidt (\citeyear{schmidt_personalpronomina_1919}; see \citealt{koch_historical_2014} for a history of ideas about the internal relations of Australian languages), scarcely anything was known about non-Pama-Nyungan languages \citep{Stockigt2021a,Stockigtinpress}. The term ``Pama-Nyungan'', and a grammatical distinction between languages in the far north and those featured in the corpus was meaningless to the early grammarians.

\subsection{Lutheran grammarians}
\label{sec:key:1.1.2}

A large proportion of the missionary grammars were written by Lutherans. Lutheran missionaries made comparatively detailed grammatical descriptions of seven Aboriginal languages spoken in South Australia before the turn of the twentieth century: Kaurna, Ramindjeri, Barngarla, Diyari, Wangkangurru, Yandrruwandha and Arrernte (see \tabref{tab:1:1}).\footnote{Koonibba, established in 1901, was the last Lutheran mission to Aboriginal people established in South Australia. Pastor C. A. Wiebusch, who ministered at Koonibba from 1910, compiled a “Julbara” (Wirangu) wordlist, and other vocabularies of Wirangu, Mirning and Kokatha were collected at the mission \citep{hoff_hoff_2004}. No grammatical description appears to have been made at Koonibba.}  The swiftness with which missions were established in the Colony of South Australia, after its founding in 1836, is atypical of the wider Australian experience.\footnote{The 1834 South Australian colonisation act was passed relatively late within British imperial history. Secretary of State to the Colonies, Lord Glenelg, was among a group of humanitarians who insisted in letters issued to the Colonisation Commission in 1836 that the welfare and rights to land of Aboriginal people in South Australia be formally protected by the Colonial Office \citep[94--102]{reynolds_law_1987}. Missions to Aboriginal people in South Australia were subsequently better considered and supported, albeit still inadequately, than in other states} In 1838, the founding chairman of the South Australian Company, George Fife Angas (1789--1879), assisted the passage to the colony of graduates from the Evangelical-Lutheran Mission Society of Dresden to work among the Aboriginal population of the South Australian colony \citep[61--65]{lockwood_two_2014}. Consequently, Lutheran missionaries dominated the early grammatical description of South Australian languages – including those spoken within the present-day Northern Territory, which was not separated from South Australia until 1911.

The earliest wave of Lutheran description in South Australia (\chapref{chap:key:5} \& \chapref{chap:key:6}) has previously been recognised as belonging to a descriptive school termed “the Adelaide School” \citep[410]{simpson_notes_1992}. While S. Ray described “the existing material for the study of Aboriginal languages” as “of a very unsatisfactory manner” (\citeyear[2]{ray_aboriginal_1925}), he lists these earliest South Australian Lutheran grammars among the “the best of the early grammars” (\citeyear[2]{ray_aboriginal_1925}). The calibre of grammars produced in South Australia was also noted by Capell (\citeyear[667]{capell_history_1970}).\footnote{Capell (\citeyear[668]{capell_history_1970}) incorrectly describes Teichelmann and Schürmann as Moravian, and he confuses the publication dates of Teichelmann and Schürmann’s and Schürmann’s grammars with one another.}   

The accuracy with which Lutheran missionaries described languages spoken in South Australia has, however, been overlooked by Carey when positioning the important linguistic contribution made by Threlkeld: 

\begin{modquote}
    Early missions to the Australian Aborigines …were rarely successful whether success is measured in terms of conversions and baptisms, or the more common linguistic coin of wordlists, grammars and scripture translations. Only a handful of missionaries produced published or unpublished records of their linguistic work that necessarily proceeded prior to evangelisation. \citep[258--259]{carey_lancelot_2004}
\end{modquote}

\citet[110]{roberts_language_2008} similarly unwittingly extrapolates the sparseness of early grammatical description of languages spoken in the earliest settled colony of New South Wales to Australia more broadly. So too does \citet[12]{dixon_languages_1980}, who delineates a period of linguistic research in Australia running between the mid-1840s and mid-1870s and characterises it as exhibiting a scarcity of linguistic description with “just a little new material coming from the missionaries”. Here Dixon might be forgiven for failing to notice the unpublished Lutheran grammars of Diyari written in German during the 1860s and 1870s, but dismissing the handful of high quality, published South Australian grammars that were written in the early 1840s, places his assessment within a broader trivialisation of missionary description. 

Recent treatments of the Lutherans’ grammatical legacy in Australia have rightly emphasised that it was a conviction that the “heathen” would be better converted to Christianity in their mother tongue that necessitated the Lutherans’ acquisition and description of Aboriginal languages (\citealt[9]{graetz_open_1988}; \citealt[87]{kenny_arandas_2013}). Teichelmann, for instance, attributed the resistance to Christian conversion entirely to his own inadequate mastery of the language and his pietistic conviction that conversion would \textit{necessarily} follow his ability to preach in the vernacular. He wrote (19/01/1840): “when we have their language in our power, the lord will through his Word perform signs and wonders on these natives however low they have sunk”. The case is made with reference to Luther having given “the German people scriptures in their own language” (\citealt[9]{graetz_open_1988}; \citealt[523--527]{hill_broken_2002}) and to the related conviction that conversion should occur through the free will provided by understanding the scriptures in one’s first languages \citep[87]{kenny_arandas_2013}. Lutheran mission activity has been characterised as resting in German philosophical traditions hailing from J. G. Herder (1744--1803), which asserted that understanding a people’s language provided a window into their \textit{Volksgeist} that was necessary for successful conversion \citep[99]{kenny_arandas_2013}. 

However, such accounts obscure the fact that the same conviction was held by other evangelical denominations, as well as by Catholics working with non-PN languages at Beagle Bay and in the Daly River. Lutheran missionaries did not hold a monopoly on the belief that the “heathen” were best converted to Christianity in their mother tongue (\citealt[21--26]{Stockigt_2020}). They were certainly not “singular in their embracing of Indigenous language in the service of mission” \citep[362]{kneebone_language_2005}. Protestant missionaries of differing denominations in Australia were expected to learn the Indigenous language spoken by the people they wished to convert. Missionary Threlkeld (\chapref{chap:key:3}), for example, “shared a devout belief that the scriptures alone \textit{(sciptura sola)} were sufficient for salvation” \citep[116]{wafer_waiting_2011} and his linguistic achievements were made as part of an established tradition of London Mission Society Bible translation \citep[107]{roberts_language_2008}. The acquisition of Indigenous languages was seen as one of the principal tasks of London Mission Society missionaries in the South Pacific \citep[xxiii]{hughes_introduction_1998}. Missionary Watson (\sectref{sec:key:4.1}) of the Church Mission Society, who established the Wellington Valley Mission in 1832, was instructed to “learn the language and reduce it to writing” \citep[297]{bridges_church_1978} and Archdeacon Broughton in New South Wales encouraged the writing of grammars as part of missionary activity (ibid., 275). Congregationalist missionary Taplin (\sectref{sec:key:7.3}) similarly wasted no time in advancing the grammatical descriptions of Ngarrindjeri, and Moravian missionaries in Australia (\sectref{sec:key:7.2}, \sectref{sec:key:10.1.3}) were trained and also expected to learn the local language \citep[319]{edwards_moravian_2007}. The importance of linguistic analysis in order to translate religious texts is not even an exclusively Christian mission philosophy \citep[33]{ostler_social_2004}.

Note here also that Queensland is the only state other than South Australia in which there were nineteenth-century Lutheran missions: Zion Hill (Nundah) (1838–1848), Nerang Creek (1869--1879), Cape Bedford (Elim, Hopevale) (1886--1942), Bethesda (Queensland) (1886--1881), Bloomfield (1887--1901), and Mari Yam\-ba (1887--1902). Yet Queensland’s Aboriginal languages were scarcely recorded before the twentieth century. Manuscript grammatical descriptions of Guugu{\hyp}Yi\-mi\-dhirr written by Neuendettelsau missionaries \citet{schwarz_koko_1900} at Cape Bedford, the longest enduring Queensland mission, are the only surviving Lutheran analyses of a Queensland language.

At Protestant missions around the country grammars and vocabularies were collated in order to carry out two interrelated tasks essential to evangelism: the translation of religious texts and the preparation of materials for use in vernacular literacy programmes. The Lutherans’ work can be said to differ from that of missionaries from other denominations only in terms of the \textit{extent} to which the Lutherans produced vernacular literacy materials and in terms of their \textit{success} in teaching young Diyari, Arrernte, and Guugu{\hyp}Yimidhirr Christian converts to read and write in their own language. Both factors were enabled more by circumstantial opportunity for missionaries to interact with Aboriginal people at missions that lasted for uncharacteristically long periods of time in Australia (\sectref{sec:key:8.1}) than by particular missiological convictions.

\subsection{Collaborations with Aboriginal people}
\label{sec:key:1.1.3}

The production of these early grammars of Australian languages necessitated exchanges that were exceptional to the brutal impacts of colonisation: the ensuing theft of land, devastation from disease, relocation, murder, institutionalisation and the deterioration of biotas that had sustained and been sustained by Aboriginal people for tens of thousands of years.  The alliance between Aboriginal people and missionary-grammarians required to produce the grammars tells of dedicated intellectual collaborations and concerted attempts to bridge a sociolinguistic divide which otherwise characterises early colonial encounters. Some degree of reciprocity between Aboriginal informants and missionary-grammarians was necessary to produce the grammars. The exchange was aptly expressed by Schürmann, writing from Port Lincoln to the Committee of the Evangelical Lutheran Mission Society in Dresden:
\begin{quote}
    The learning of the language, which must precede any immediate and specific mission activity, is a difficult and laborious task, which can be less perfected by application and enthusiasm, than by patience and endurance[…] When the physical sphere of the language is exhausted and one then comes to the spiritual side of abstract concepts and ideas, all devices and urging are fruitless. \textit{Here the missionary stands on one side} pondering and striving for the meaning of a spoken word, or ferreting out a relevant word for a given concept; \textit{with the aborigine on the other} often just as eager to make himself understood...Patient observation, quick comprehension and lively fantasy alone can produce a result in such cases. (\citealt{schurmann_letter_1844}; emphasis added)
\end{quote}

Counter to the still widely received notion which construes as amoral the missionaries’ avowed intention to Christianise Aboriginal people stands the irony that missionaries, more than any other group, came to grips with the complexity and diversity of the languages spoken in Australia. That said, it is important to remember that only a small minority of missionaries in Australia learned and described Aboriginal languages \citep[805--806]{harris_one_1994}, although most of the earliest missionaries made some attempt at translating liturgical texts. The popular trope that Aboriginal people were punished for using their own languages at the missions is perhaps informed by the experience and memory of twentieth-century government policy. Accounts which cast missionary engagement with Aboriginal people as solely culturally destructive and as having facilitated genocide \citep[77--79]{dixon_languages_1980} have recently been contested in a growing body of literature. A more balanced story of missionary relations with Aboriginal people has emerged which places the missionaries’ role in providing welfare, medical assistance, and physical protection against the backdrop of neglect and abuse from other quarters of colonial society (\citealt{kneebone_teichelmann_2005}; \citealt{harris_one_1994}; \citealt{edwards_moravian_2007}; \citealt{strehlow_tale_2011}).

\hspace*{-1.6pt}Opening pages of nineteenth-century missionary grammars often contain qualifying remarks about the complexity of the described language which are framed to counter the opinion, still prevalent today, that Aboriginal languages were simple and less evolved than European languages. G. Taplin observed:

\begin{quote}
    The Narrinyeri have a language, and do not, as an English farmer once told me[…] only make noises, like beasts of the field. They have a highly organised one too, possessing inflections which ours do not. \citep[123]{taplin_narrinyeri_1879}
\end{quote}

It was not usual for early missionary-grammarians to acknowledge their Aboriginal informants, nor document their method of fieldwork. The identity of the numerous Aboriginal informants who chose to share their language with early missionary-grammarians is frequently unknown. The colonial power imbalance and the inherent sense of superiority held by the European recorders created a situation in which it was generally thought unnecessary to acknowledge Aboriginal collaborators. That the Aboriginal people who enabled the production of the grammars tend to remain unacknowledged, even as nameless identities, is characteristic of the rather disembodied corpus of early descriptions of Australian languages.

Early Lutheran missionaries in South Australia were explicitly instructed to “take on a teacher and study the grammar, as well as meet the people in order to grasp the spoken language in their day-to-day life” \citep[681]{rheinwald_acta_1840}.\footnote{ ``Einen Lehrer annehmen und die Grammatik studieren, als auch unter den Volk gehen werden, um den mündlichen Ausdruck aus Leben aufzufassen'' \citep[681]{rheinwald_acta_1840}.} In reality the productive relationships formed between missionaries and Aboriginal people were likely to have been initiated by Aboriginal people choosing to engage in an interchange of cultural ideas and practices with the Europeans, as detailed by Sutton \citeyearpar[166--167, 178]{sutton_explorations_nodate}.

In the case of most, if not all, of the missions examined in the timeframe of this study, Aboriginal people were under no obligation to work and stay at the mission. Aboriginal people were generally free to leave, and did, judging by the missionary-grammarians’ frequently aired frustration at their “wandering habits” (\citealt[xi]{threlkeld_australian_1834}; \citealt[v]{meyer_vocabulary_1843}; \citealt[1]{kempe_grammar_1891}). In 1888, at the Elim mission in north Queensland, Lutheran missionary Poland commented:

\begin{quote}
    While it was certainly difficult enough to keep the young people at the station, it seemed to be virtually impossible to persuade older people to stay. Adults did not settle at Elim, at best they made Elim a staging-post. This usually occurred during the wet (Summer) season when the food that sustained their nomadic existence became scarce. \citep[16]{Poland_1998}
\end{quote}

Drawing on the autobiographical account of Moses Tjalkabota (below), who was a boy when Lutheran missionaries arrived as the first wave of Europeans in Western Arrernte country in Central Australia in 1877, \citet[25--26]{latz_2014} explores the reasons that Tjalkabota’s father Tjita may have led his family away from the Hermannsburg mission, including desecration of sacred sites, and fear of losing influence over his children. Rations of food, tobacco, blankets, and at Hermannsburg, music, particularly singing (ibid.: 23--25), were used to entice adults to stay at the mission. 

\subsubsection{The relationship between informant and missionary }
\label{sec:key:1.1.3.1}

In some instances, the relationship between the missionary-grammarian and his Aboriginal informant might aptly be characterised as one of friendship. In his earliest grammar, Ridley \citeyearpar[76]{ridley_kamilaroi_1855-1} stated: “The permissive voice of \textbf{buma} is \textbf{bumanbilla}, which I learned from a black fellow, who, at my request, was explaining his idea of friendship.'' 

\ea\label{ex:1:1}
\gll {Kamil Yarri} {ngununda bumanabilla.} \\
Harry {will not \textit{allow-any-one-to-beat} me.} \\
\glt \citep[76]{ridley_kamilaroi_1855-1}\\
\gll Gamil Yarri nganunda buma-na-bi-li\\
\textsc{neg} Harry-\stockcaps{[ERG]} 1sg\stockcaps{LOC} hit-\stockcaps{VD}?-\stockcaps{LET}\footnotemark{}\\
\z\footnotetext{See Giacon (\citealt[355, 363]{giacon_2014}) for account of permissive voice glossed “LET”}

The “Yarri” to whom Ridley refers is “Harry of Bungulgully”, who \citet[170]{ridley_kamilaroi_1875} had initially met on the Upper Paterson, close to Newcastle, in 1851 when Harry was engaged as a farm labourer some 600km south of his own country. It was Harry who initially re-inspired Ridley’s calling for mission work \citep{gunson_australian_2016-1}. At this stage Ridley wrote that Harry 
\begin{quote}
    was pleased at being recognised as one for whom the minister cared: and I found that by merely acting on the rule – “honour all men” – treating him as a fellow creature, I had won his friendship.
\end{quote}

Four years later, Ridley was reacquainted with Harry back on his own country at Bungulgully in the Namoi River catchment in northern New South Wales:
\begin{quote}
    He had heard of my coming and went out on the track to meet me. His countenance expressed joy. He gave me help in learning the Kamillaroi, and listened with earnest attention …
\end{quote}

Ridley provides this detail not in any of his Kamilaroi grammars (\citeyear{ridley_kamilaroi_1855-1}, \citeyear{ridley_gurre_1856}, \citeyear{ridley_kamilaroi_1866}, \citeyear{ridley_kamilaroi_1875}; see \sectref{sec:key:4.5}), but in a closing section headed “Random illustrations of Aboriginal life and character” \citeyearpar[166--170]{ridley_kamilaroi_1875}. Although Aboriginal informants are rarely acknowledged in the grammars, their identity and contribution are sometimes evident in other documents including missionary-grammarians’ letters and journals. 

Not all missionary-grammarians, however, benefitted from the willingness of Aboriginal people to interact freely. Teichelmann attributed an early inability to discuss important religious matters as resulting from the language being “withheld” (\textit{zurückgehalten}) by its speakers (1839--1846, 09/01/1840). Moravian missionary A. F. C. Täger (1811--1870) at Lake Boga (1850--1856), the short-lived and earliest Moravian mission in Victoria (\sectref{sec:key:4.5}), described how Aboriginal people controlled their communication with the missionaries:

\begin{quote}
    We have not yet been able to spread the word of God, because we are still missing too many words … we are totally convinced that some Aborigines are starting to speak a different language in our presence, when they are speaking about something that we should not know. (Täger, quoted in \citealt[81]{jensz_german_2010})
\end{quote}

Nor were all missionary-grammarians able to forge relationships with people who were established bi-cultural negotiators (see further in the next section). At the Wellington Valley mission in New South Wales, for instance, misgivings about the missionaries’ intentions preceded the arrival of Watson and Handt in the region, possibly affecting the nature of the relationships formed with local Aboriginal people. White settlers had told the Aboriginal occupants that the missionaries intended on stealing and imprisoning their children (\citealt{handtpapers}, 30/09/1932; 24/09/1832). The settlers’ perception of the missionaries’ method of Christianising Aboriginal people was not entirely incorrect. Handt writes:

\begin{quote}
    One woman had a half cast [sic] little infant. I had been informed that they had a girl of about four years old among them of the same description. It was said that she had lost her mother, and was taken care of by an old woman. This child I should have endeavoured to obtain by giving the old woman a trifle, had she not gone into the bush with it the day previous. (Handt \citeyear{handtpapers}, 30/03/1834)
\end{quote}

And in 1838, missionary Gunther at Wellington Valley presented the following Wiradjuri clause. 

\ea \label{ex:1:2}
\gll Ngungu-dhi nhila buraay-nhu ngadhu-nhu; Minyaminyambul ngum-biya-girri. \\
Give-! That child-your I-for-you; Something-something give-constantly-will\\
\glt `Give me that child and I will give you plenty of compensation' \\
(Transcription, gloss and translation, \citealt[39]{grant_introducing_2000})\footnote{Note that Rudder \& Grant’s translation of the clause, “… plenty of compensation” differs from the ``plenty to eat'' given in the 1838 grammar (\tabref{tab:chap3:ergativepm} on p. \pageref{tab:chap3:ergativepm}). Their translation presumably comes from another version of the same clause.}
\z 

\begin{figure}
\includegraphics[width=8cm]{figures/figa.png}
\caption{Page from \citet[270]{gunther_native_1838}}
\end{figure} 

Schürmann’s encounter with speakers of Barngarla on the Eyre Peninsula (\sectref{sec:key:6.2}) stands in contrast to the ease with which he, Teichelmann and Meyer had earlier garnered the trust of Kaurna and Ramindjeri men. Acting as Deputy Protector of Aborigines, Schürmann had little time for mission work, and was immediately swept into police investigations of a series of murders of Aboriginal people and Europeans, in which he felt morally and professionally compromised. Frontier hostility on the Eyre Peninsula was better recorded than in most areas of Australia thanks to Schürmann’s elicitation from Aboriginal people of their version of events, which he detailed in letters and in his journal \citep{schurmann_diaries_1838}. It took Schürmann almost two months to establish any significant contact with Aboriginal people from the district, and he related (in \citealt[113]{schurmann_id_1987}) that “everyone discourages the natives as much as possible, that they are regarded as a nuisance, and their presence could cause animosity towards me”. Perhaps people were choosing to keep clear of the small and remote European settlement at Port Lincoln for good reason. The morning after establishing peaceful contact with nine adult men, with whom Schürmann hoped to form an ongoing association, he discovered that these people had been imprisoned and physically mistreated by police \citep[113]{schurmann_id_1987}

Missionaries interacting with Aboriginal people close to the frontier witnessed hostilities between the local land-owning people and the European settlers. Schürmann’s account of frontier violence on the Eyre Peninsula includes one of at least two recorded incidents in which an Aboriginal man who had forged a trusting relationship with a missionary-grammarian was slain by colonial authorities for a crime he did not commit. Schürmann (in \citealt[151--152]{schurmann_id_1987}) mourns the execution of Nummalta, who had previously acted as his guide. Just over a decade later, in Western Victoria, Moravian missionary Spieseke laments the execution of Bonaparte, a Wemba Wemba cattle drover, described as “an esteemed linguistic informant” \citep[85]{jensz_german_2010}, who was killed in 1854 by police as retribution for a crime committed by another man. 

In the case of the parties of Lutheran and Moravian missionaries who in 1867 established themselves in Diyari country at Lake Killalpaninna and Lake Kopperamanna respectively (\sectref{sec:key:8.3.1}), concurrent with the expansion of pastoralism into the unsettled north of South Australia, the missionaries were themselves threatened by frontier hostility. Both parties were forced to retreat to southern, settled districts due to imminent attack from local men \citep[72--83]{proeve_work_1952}. The missionaries were informed of the threat to their lives by the Diyari man Pikali, “Macky”, whose name appears in Koch’s 1868 \textit{Deklination der Eigenamen} (Declension of proper nouns; \figref{fig:key:1.1.3-2}). Gößling recorded that: 

\begin{quote}
    an old man, whom we have working for clothing food and tobacco (they do not know any other pay), and who adheres to us really faithfully, informed us of it [the planned attack]. He told us, the blacks in Perigundi are very angry, and they want to murder all the whites here in the North at Lake Hope. (Gößling, quoted in \citealt[72--83]{proeve_work_1952})
\end{quote}

\begin{figure}
\includegraphics[width=8cm]{figures/figb.png}
\caption{Koch’s declension of a male personal name \citep[no pag.]{koch_untitled_1868}}
\label{fig:key:1.1.3-2}
\end{figure} 

\subsubsection{Aboriginal informants}
\label{sec:key:1.1.3.2}

Where missionaries operated soon after initial contact, the Aboriginal men who attached themselves to the mission were often already recognised and respected identities within colonial society, having acted as intermediaries in other capacities before embarking on more dedicated linguistic work. Several Aboriginal men who worked with the missionary-grammarians had already established themselves as bicultural frontier figures of notoriety. Sometimes their activities are described in other colonial records and additional biographical detail is retrievable.

A case in point is Biraban, “Eaglehawk” (John McGill; c. 1800--c. 1846), who had acted as a respected cultural broker before commencing work with Threlkeld in 1825 (\sectref{sec:key:3.2}). As servant to an officer in the military barracks, Biraban had helped establish a penal settlement at Port Macquarie and had been recognised as the “tribal King” by L. Macquarie, Governor of New South Wales. Transitioning into the role of the missionary’s main linguistic informant was a natural progression. An early portrait of Biraban, titled “Magill” (\figref{fig:key:1.1.3-3}) had been taken by the convict artist R. Brown in 1819 \citep{Gunson2021}, prior to Threlkeld’s arrival.

\begin{figure}
\includegraphics[width=7.5cm]{figures/figc.png}
\caption{Richard Brown’s portrait of Biraban McGill (c. 1819). Kerry Stokes collection, Perth}
\label{fig:key:1.1.3-3}
\end{figure} 

Similarly, G. Taplin’s anthropological and linguistic publications (\sectref{sec:key:7.3}) were underpinned by the knowledge of James Unaipon (1835--1907), who had earlier acted as an interpreter and been baptised by the travelling missionary Rev. J. Reid. After Reid’s drowning in a boating accident, James attached himself to the Point McLeay mission in 1864, intending to “improve himself in reading and writing”. Here he was known by his Potawolin name, Ngunaitpon, of which “Unaipon” is a phonological approximation. By 1865 Taplin was training Unaipon as a teacher and native evangelist, enabling him to preach to outlying Ngarrindjeri people along the Coorong, southeast of the mission \citep[114]{Lane1997} and in 1871 he was appointed the first Ngarrindjeri church deacon (\citealt{Jones2005}). An image of his son, David Unaipon (1872--1967), who was born at the mission, appears on the Australian fifty dollar note. 

\begin{figure}[b]
\includegraphics[width=7.5cm]{figures/figd.jpg}
\caption{“Kertamaroo, a Native of South Australia”, wax portrait by Theresa Walker (c. 1840)}
\label{fig:key:1.1.3-4}
\end{figure}

\begin{figure}[b]
\includegraphics[width=8cm]{figures/fige.jpg}
\caption{“A warrior of the Adelaide tribe …” George French Angas (1847) Plate 22, no.1}
\label{fig:key:1.1.3-5}
\end{figure}


Two of the several Aboriginal men of the Adelaide Plains who informed Teichelmann and Schürmann’s grammar of Kaurna (\sectref{sec:key:5.2}), Mulla\-wirra\-burka (c. 1811--1845; “King John”, “Onkaparinga Jack”), Kadlitpinna (“Captain Jack”) were renowned cultural brokers before working with the missionaries. In 1838, Mullawirraburka and Kadlitpinna had been appointed as honorary constables by the Governor of South Australia, and Mullawirraburka had been sketched by W. H. Leigh in 1837 \citep[91]{Gara1998}, where he is not named \citep[84–88]{Leigh1839}. A wax portrait by T. Walker (\figref{fig:key:1.1.3-4}) of an individual named “Kertamaroo” (\textit{kartamiru} meaning “first born”) is most probably Mullawirraburka \citep[92]{Gara1998}. A painting by G. F. Angas (\figref{fig:key:1.1.3-5}) of “a warrior of the Adelaide tribe” is of Kadlitpinna \citep[96]{Gara1998}. In October 1838 these two men along with Ityamaiitpinna (“King Rodney”) are depicted in Martha Berkeley’s watercolour ``The first dinner given to the Aborigines'' (1838; \figref{fig:key:1.1.3-6}), where they are recorded using their European names \citep[60]{Hylton2012}. Teichelmann and Schürmann, who had arrived in South Australia just weeks before the ceremony, attended, and Schürmann (\citeyear{schurmann_diaries_1838}, 28/10/1838) describes the dressing of King John and Captain Jack, who had been marked out for special distinction due to their intelligence and ability:

\begin{quote}
    Their right shoulders were decorated with a corsage of yellow silk tassels, the right side of their cap with a bunch of multi-coloured bands.  Likewise, on the right side of the breast and on the right arm, each of them had three stripes like a Hanoverian sergeant major.  On the left side of their cap, dark red aiguillettes hung down to their shoulders, where tassels of similar colour served as epaulets. The outfit gave them a kind of military appearance.  Contributing to this effect, each of them carried his spear (\textit{wieda}), his club (\textit{waddi}), and his woomera (\textit{womarra}) in his hands. (\citealt{schurmann_diaries_1838}, 28/10/1838; translation by G. Lockwood)\footnote{ ``Unter ihnen hatte man zwei d[urch] Klugheit u Tüchtigkeit hervorragende Männer, König Johann u Capitain Jack, besonders ausgezeichnet.  Ihre rechte Schulter schmückte ein Bändel gelbseidener Quasten, die rechte Seite ihrer Mütze ein Strauss von verschiedenfarbigen Bändern, eben so die rechte Seite der Brust u auf dem rechten Arm hatte jeder 3 Striche, wie ein hannöverscher Wachtmeister.  An der linken Seite ihrer Mütze hingen dunkel rothe Fangschnüre auf die Schultern herab und Troddeln von ähnlicher Farbe dienten hier als Epaulets. [54] Die Ausstaffirung gab ihnen eine Art von kriegerischen Ansehen.  Dazu kam dass jeder seinen Speer (Wieda), seine Keule (Waddi) und seine Womarra in den Händen trug.''} 
\end{quote}




\begin{figure}[t]
\includegraphics[width=.8\textwidth]{figures/figf.png}
\caption{Detail from M. Berkeley “The first dinner given to the Aborigines” (1838). From left: Kadlitpinna Mullawirraburka, and Ityamaiitpinna \citep[103]{Gara1998}}
\label{fig:key:1.1.3-6}
\end{figure} 

Mullawirraburka and Kadlitpinna are generally recognised as Teichelmann \& Schürmann’s main linguistic informants (e.g., \citealt[64]{amery_warrabarna_2016}), yet it is another Kaurna man who is most frequently named as a linguistic informant in Schürmann’s diaries. Wauwitpinna told Schürmann the names of the constellations (\citealt{schurmann_diaries_1838}, 05/06/1839, and \citealt{schurmann_diaries_1838}, 21/08/1839) shared a creation story in which the deity \textit{Nganno} ``named the places of the country as we know them today''. Schürmann here observes that

\newpage
\begin{quote}
    [t]he language that Wauwitpinna used in the speech of the characters introduced varied from what is now in use. According to him, it was the language of the \textbf{Munana Meyu} (ancestors). (Translation G. Lockwood).\footnote{  ``Die Sprache, welche Wauwitpinna in den Reden der eingeführten Personen gebrauchte, wich von der jetzigen ab, und war nach seiner Aussage, die Sprache der Munana Meyu (Vorfahren).'' (\citealt{schurmann_diaries_1838}, 21/08/1839)}
\end{quote}
  
Although these Kaurna men are not formally acknowledged in Teichelmann and Schürmann’s grammar (\citeyear{teichelmann_outlines_1840}), they are among eight people named in the vocabulary (\citeyear[36]{teichelmann_outlines_1840}) by their Aboriginal name, which were first recorded by the missionaries. The name Mullawirraburka means `senior man from the dry forest county' (\textit{murla-} `dry', \textit{wirra-} `forest' and \textit{-purka} `senior man'), and Kadlitpinna means `father of dog' (\textit{kadli}- `dingo / dog', \textit{-itpina}, `father of'). Each of these names are likely to be a means of identifying and referring to an individual while avoiding the use of another culturally sensitive name. 

Under the entry \textit{Pangkarra}, `district or tract of country belonging to an individual which he inherits from his father',\footnote{Schürmann’s observation is an early incisive statement about patrilineal acquisition of country estate. It was first made in a letter to G. F. Angas (12/06/1839, in \citealt[50]{schurmann_id_1987}).}  Teichelmann and Schürmann illustrate methods of forming male names with the segment \textit{-burka} `old man' and \textit{-itpinna} `father of'.

\begin{figure}
\includegraphics[width=8cm]{figures/figg.png}
\caption{Teichelmann \& Schürmann’s illustration of methods of forming male names, giving the names of some of their collaborators \citep[36]{teichelmann_outlines_1840}}
\label{fig:key:1.1.3-7}
\end{figure} 

The names of the Aboriginal people who informed the missionaries’ work also sometimes appear in grammatical paradigms. In 1834, Threlkeld used Biraban’s name when exemplifying the case marking of proper nouns (\sectref{sec:key:3.3.5}, \figref{fig:3.11}), and half a century later at the inland Lutheran mission, Bethesda, Reuther used the name of the Diyari evangelist Pingilina, discussed below (\figref{fig:key:1.1.3-8}).

\begin{figure}
\includegraphics[width=8cm]{figures/figh.png}
\caption{Reuther’s declension of a male personal name \citep[13]{reuther_dieri_1894}}
\label{fig:key:1.1.3-8}
\end{figure} 

By contrast, the missionaries’ records of interaction with Wiradjuri people at Wellington Valley in New South Wales in the 1830s (\sectref{sec:key:4.1}) shows that missionaries Watson and Handt tended to refer to their Aboriginal informants generically using expressions such as `the Blacks’ or `a Native'. Handt (\citeyear{handtpapers}, 22/19/1832), for instance, writes that he “[w]as endeavouring to get some words from the Natives”, and “[e]ngaged in getting words and phrases, and in talking to the Blacks” (\citealt{handtpapers}, 07/01/1835). While several Aboriginal people, mostly children, are named by the Wellington Valley missionaries in their writings, there is no evidence of enduring collaborations with adults. 

Aboriginal people sometimes accompanied missionaries with whom they had formed an attachment to new mission fields remote from their own country and language. Moravian missionary Spieseke at Lake Boga in Western Victoria (\sectref{sec:key:7.2}) collaborated with the Wotjobaluk man Nathaniel Pepper (c. 1841--1877), who had taken his surname from the Irish colonist John Pepper \citep{DeAraugo2005}. The son of a respected senior Wotjobaluk man of the Wergaia people, Pepper would have been approximately 18 years old when the Moravians established themselves in the area in 1859. Baptised ``Nathanael'' by Spieseke in 1860, as the first Moravian convert at Ebenezer, the missionaries recognised Pepper as a talented student. From 1865 he received a mission salary for his evangelical work, preaching in Wergaya and in English at the mission and in the bush (\citealt[136--137]{jensz_german_2010}). In 1869, Pepper travelled with missionary Hagenauer to Ramahyuck in eastern Victoria \citep{jensz_german_2010}, where he also taught at the mission school and preached to adult Kurnai people. Hagenauer’s Western Victorian Wergaya material, spoken by the “Pine Plains tribe” in Brough Smyth (\citealt[39]{smyth_1876}, vol. II.: 39) was collected off-country in east Gippsland from Aboriginal men relocated from the Moravian Ebenezer mission to the Ramahyuck Mission (\figref{fig:key:1.1.3-9}). It is probable that at least one of these informants was Nathaniel Pepper. 

\begin{figure}[t]
\includegraphics[width=8cm]{figures/figi.png}
\caption{Map of Moravian mission in Australia, 1869 (“Mission der Brüder Unitaet in Australien 1857--1869”), with permission from Unitäts\-archiv (Bd. 30.13)}
\label{fig:key:1.1.3-9}
\end{figure}

Pepper operated at roughly the same time and in the same capacity as James Unaipon in South Australia, and the pair may be seen as precursors to the Diyari and Arrernte evangelists who preached to Aboriginal people away from their own country decades later in Central Australia. The Diyari evangelist Johannes Pingilina (birth date unknown but c. 1904), who had attached himself to the Bethesda mission in the outback desert regions of South Australia, travelled to Cape Bedford in far north tropical Queensland in 1886 with C. A. Meyer, before returning to Bethesda in 1892. Pingilina assisted the Neuendettelsau missionaries in Queensland in learning Guugu{\hyp}Yimidhirr, and Kuku-Yalanji spoken at the Bloomfield mission. The important linguistic contribution made by the Diyari evangelist Johannes Pingilina was noted by the Protector of Aborigines and grammarian W. E. \citet[8]{roth_structure_1901}, who stated that the use of Guugu{\hyp}Yimidhirr in the school was achieved with Pingilina’s assistance. The Western Arrernte man Moses Tjalkabota (1869--1964), ``the blind evangelist'', was baptised by missionary Kempe on Christmas day in 1890. With his wife Sofia Ingkamala, Moses travelled by foot, donkey and camel, often providing the first point of Christian contact with remote Arrernte and Luritja speaking peoples. Moses Tjalkabota and A. Heidenreich translated the appendix to the 1924 \textit{Arrernte Christian Instruction Book}, compiled by C. Strehlow and published posthumously (\sectref{sec:key:9.2.3.3}).


\subsubsection{Field work methods}
\label{sec:key:1.1.3.3}

As in other productive relationships between ethnographers and Aboriginal people in Australia (see \citealt[163--193]{Sutton_2009}), the missionaries who learned Aboriginal languages lived alongside Aboriginal people sometimes for decades. The Aboriginal people with whom they worked linguistically were integral to the missionaries’ survival as they strove to sustain themselves in the fledgling colonies and, in some instances, beyond the frontier. Aboriginal people acted as guides when missionaries ventured beyond the settled districts, directed them to water, acted as interpreters, and sometimes (e.g., Schürmann, in \citealt[132]{schurmann_id_1987}) saved the missionaries from life-threatening circumstances. In such situations, language was learnt embedded in the field, while communicating with Aboriginal people about everyday needs. When arriving at Cape Bedford mission in far north Queensland in 1888, missionary Poland was advised by his Lutheran brother Schwarz (\citealt[16]{Poland_1988}) to
\begin{quote}
    [t]ry and pick up as much [of the language] as you can through personal contact with the Aborigines. That is the best way of acquiring correct pronunciation and sentence-structure. What you hear will leave an indelible impression on your memory.
\end{quote}

In a rare instance illustrating the means by which a missionary learnt a grammatical structure, Schürmann’s Adelaide diary entry titled “Fünf Tage mit den Eingeborenen im Busch” (Five days with the natives out bush) evinces this fieldwork method:
\begin{quote}
    What I promised myself from this journey has been more than confirmed: a closer acquaintance with the life and the language of the natives. My progress in the language consists not only in a number of new words but also in the consolidation and more fluent use of what I already knew. In particular, the discovery of a \textbf{modus conjuctivus} which is formed by attaching the little syllable \textbf{ma} to the stem of the verb, and is used very regularly. (\citealt{schurmann_diaries_1838}, 13/09/1839; translation by Lois Zweck)
\end{quote}

The closeness that developed between some missionaries and Aboriginal people is sometimes evident in the illustrative material given in the grammars. In the following Ngarrindjeri example, given by Meyer in 1843, the Aboriginal voice is in the first person:

\ea\label{ex:1:3}
\gll Ngate pant-ir porle, balb-êmb-itye \\
{By me (a)} {bringing forth has been} child, {white was it} \\
\glt “I brought forward a child and it was white: or, My child was white when it was born” \\
\citep[36]{meyer_vocabulary_1843} \\
\gll Ngati pant-ir po:rli palp-emb-itji\\
1sg\stockcaps{ERG} bear-\stockcaps{PAST} child-[\stockcaps{ACC}] white-\stockcaps{RPAST}-3sg\stockcaps{NOM}\\
\z

The clause is starkly revealing of early colonial contact and a carries rare instance of Aboriginal perspective. Teichelmann and Schürmann present the following Kaurna clauses:

\ea
\gll {Pulyunna meyu tittappe-urti,} {pindi  meyu nurru-ttoai} \\
{`Don’t hang the black man,} {that the European be not charmed [ensorcelled]'} \\
\glt \citep[69]{teichelmann_outlines_1840}\\
\gll Pulyurn miyu tita-api-rti {pinti miyu} nuru-tuwayi\\
black man hang-\stockcaps{CAUSE}-\stockcaps{NEG}.\stockcaps{IMP} European curse-\stockcaps{AVERS}\\
\z
	       
\ea
\gll {Pulyunna meyurloyakko yailtyapindi meyubudnitina} \\
{`The black man did not think that the white man would come'} \\
\citep[69]{teichelmann_outlines_1840} \\
\gll Pulyurn miyu yaku yailtya {pinti miyu} pudni-tina\\
Black man \stockcaps{NEG} think European come-\stockcaps{PRIV}\\
\z

Missionary-grammarians also engaged more formal fieldwork methods, sometimes resulting in frustration at the unwillingness of Aboriginal people to play their required role. 

Meyer is known to have engaged in a style of enquiry that made Ramindjeri people uncomfortable and resulted in them avoiding contact with him \citep[66--74]{gale_nothing_2011}. The missionaries’ initial, and perhaps ongoing, insensitivity to polite conversational pragmatics practised in Australia probably contributed to ineffective formal elicitation strategies. Differences between Australian and Western conversational strategies of silence, turn-taking, deference, ellipsis, and questioning (\citealt{Walsh_1997}, \citealt{BlytheMushininpress}) are likely to have been misinterpreted by both parties, thereby causing communication breakdown. Culturally different expectations about how knowledge is acquired \citep{Eades1982} may also have played into an uneasy exchange of information. In 1841 Moorhouse and Teichelmann were, nevertheless, successful in quickly eliciting dialectal pronominal variation within neighbouring groups holding country around the lower Murray River in South Australia (\sectref{sec:key:6.4}). They had evidently navigated communicative barriers with people who had had little or no previous contact with Europeans and had developed an effective method of formal elicitation.

As a reminder of the ontological divide between the scholarly European mis\-sion\-ary-grammarians and the non-literate hunter-gatherer people whom they sought to understand, consider the astonishment expressed by missionaries upon discovering that there was no Indigenous grammatical tradition. \citet[v]{meyer_vocabulary_1843} described his Australian informants as “incapable of answering or even comprehending, grammatical questions”. Similarly, Congregationalist missionary G. Taplin (\sectref{sec:key:7.3}) lamented that “enquiries are useless when addressed to minds upon whom the idea of grammar has never dawned” \citeyearpar[6]{taplin_grammar_1880}, and Lutheran missionary Kempe (\citeyear[4]{kempe_grammar_1891}; \sectref{sec:key:9.1}) expressed the same frustration, appearing almost exasperated that the “natives” could not explain the difference between Arandic past tense suffixes \textit{-ke} and \textit{–kele} \citeyearpar[1]{kempe_grammar_1891}. Teichelmann wrote:

\begin{quote}
    We have had to collect the language from the mouth of a people who do not have the faintest idea of the grammar and etymology embodied in their language, and who are even not capable of giving us a minimum of lessons, but for whom every question about their language seems to present an insoluble puzzle. … Let me give the following example to illustrate their [Aboriginal people’s] behaviour when we want to learn something from them: If we ask about the first person pronoun, they answer in the second person, and when we ask about the second person, they answer in the first. If we ask about the meaning of a word, they add either a noun or an adjective to it and produce an expression with a narrower meaning. Or they give us an example that describes a situation in their lives, in which the word that we asked for appears. Or they say: “There is only one word for that,” or “yes, that’s what it’s called,” and then they repeat the word. (Teichelmann to H. C. von der Gabelentz, 6/1/1841; translation by J. McElvenny)
\end{quote}

\newpage
Teichelmann and Schürmann \citeyearpar[v]{teichelmann_outlines_1840} did, however, describe a “\textit{natural} inability” (my emphasis, C.S.) to answer grammatical questions. The average German peasant would not have been very different.

It is possible that such comments were published to raise awareness among an international audience that missionaries in Australia were presented with this specific difficulty that was additional to that experienced in mission fields where there existed some Indigenous grammatical tradition utilised by missionaries. In 1836, C. T. E. Rhenius had published a grammar of Tamil, in which he described using the existing tradition of grammatical description for the language (Rhenius \citeyear[i]{rhenius_grammar_1836}; \sectref{sec:key:2.4}). Rhenius had been trained for missionary work in India by J. Jänicke in Berlin before 1814, as had Teichelmann and Schürmann, and Schürmann is known to have studied “dictionaries” of Tamil and Malay while in Australia \citep[67]{rathjen_difficult_1998}.

A distinctive and somewhat questionable formal method of fieldwork engaged by some authors involved eliciting material from Europeans who had “learnt” an Australian language. In some instances, the European informant became acquainted with the language as a child growing up with Aboriginal children. C. Symmons’ (1841) grammar of Nyungar was informed by Francis Armstrong (\sectref{sec:key:7.1}), W. Ridley’s (1866) grammar of Turrubul was informed by Thomas Petrie (\sectref{sec:key:4.5.2}), and C. Smith’s (1880) description of Bunganditj spoken in the southeast corner of South Australia \citeyearpar{smith_booandik_1880} was informed by her son, Duncan Stewart, all of whom are said to have acquired the language naturally as children while mixing freely with their Aboriginal playmates \citep{gunson_australian_2016}. Thomas Petrie’s daughter, when writing her father’s reminiscences, recalls Ridley’s method of learning Turrubul:

\begin{quote}
    In the early days the Rev. W. Ridley came to Brisbane to learn what he could about the Queensland aborigines, and he sought out my father [Thomas Petrie], who was quite a lad at the time, to get information from him. He seemed very clever, and as fast as the boy [Thomas Petrie] could speak the language he [Ridley] was able to write it down. He took a part of the Bible and read out verse after verse, and the lad followed in the black’s tongue. Afterwards reading out the aboriginal version for his young companion’s approval, it was almost as though a blackfellow spoke. \citep[140]{petrie_tom_1904}
\end{quote}

More dubiously, the linguistic data upon which some grammarians based their description of morphosyntactic structure were translations of liturgical texts made by previous missionaries. This dubious practice of drawing up a grammar based on the structure of the language presented in a Bible translation was utilised by G. Gatti, whose grammar of Diyari \citeyearpar{gatti_lingua_1930} was largely informed by the structure of the language presented in Lutheran missionaries G. Reuther and C. Strehlow’s translation of the New Testament (\citeyear{reuther_testamenta_1897}; \sectref{sec:key:8.4.2}). S. Ray’s (1893) description of Kalaw Lagaw Ya, spoken in the western Torres Strait, similarly used a missionary’s translation of the Gospel of St Mark. \citet[119]{ray_study_1893} were at least aware of the limitations of the method, describing his source as “the only text available for the elucidation of the Saibai grammatical forms”. 

Grammarians of the Adelaide School (\sectref{sec:key:5.1}) may have been the first grammarians in Australia to elicit data about a second Aboriginal language via bilingual Aboriginal speakers. Relationships with speakers who were bilingual in the target Aboriginal language and a language with a previously described structure provided some missionary-grammarians with rapid and direct access to material. When in Adelaide, Schürmann commenced learning the Ramindjeri dialect of Ngarrindjeri from Tammuruwe Nankanere, “Encounter Bay Bob”, compiling a comparative “Adelaide/Encounter Bay” wordlist \citep[68]{amery_warrabarna_2016}. Tammuruwe spoke both his own language, Ramindjeri, as well as Kaurna, and had learnt English while working in the whaling industry, which operated off the coast of South Australia prior to colonisation. A sought-after guide and interpreter on government expeditions \citep{Lockwood2017}, Tammuruwe later assisted missionary Meyer linguistically at Encounter Bay. 

The grammar of Ngayawang published by M. Moorhouse (\citeyear[v]{moorhouse_vocabulary_1846}; \sectref{sec:key:6.4}) acknowledges that the work was dependent upon an unnamed bilingual speaker of Kaurna and Ngayawang. Moorhouse described his field-work methodology in the following terms: 

\begin{quote}
    I have produced the material, mainly, through the aid of an interpreter, who knows the Adelaide and Murray dialects; and had it not been for his assistance, I could not have gathered many of the grammatical remarks, which are now given, in the few months that I have been engaged with this dialect. \citep[v]{moorhouse_vocabulary_1846}
\end{quote}

He does not name his interpreter, and it is not clear whether the material informing \citegen{moorhouse_vocabulary_1846} Ngayawang grammar and vocabulary was collected in Adelaide or in the Murray lands where the language belonged.\footnote{When describing having been “engaged with the dialect” for only a few months, Moorhouse presumably implied that the period was intermittent and interspersed over the few years since his 1843 report.}  

\subsection{Secondary source material}
\label{sec:key:1.1.4}
\largerpage
The nature of linguistic work carried out in Australia during the nineteenth century runs counter to the \textit{Zeitgeist} of genealogical and typological linguistic classification. While similarities between words in Australia and those from around the world were observed by researchers in Australia (\citealt{grey_languages_1845}; \citealt{taplin_narrinyeri_1879}; \citealt{curr_australian_1886}; \citealt{fraser_australian_1892}) as part of what \citet[667]{capell_history_1970} described as an “endemic of origin hunting”, the early PN grammars written in the country were predominantly synchronic, non-comparative, non-classificatory works made mostly by missionaries for largely evangelistic purposes. 

Studies of comparative linguistics, historical linguistics and, to a lesser extent, of linguistic typology probed the central question of linguistic and human origins within the nineteenth century preoccupation with speculative historicism. As scientific theories of evolution revolutionised mid-nineteenth century thinking, the focus of philologists concentrated on genealogical classification of languages in order to determine the origins of people \citep{gregorio_reflections_2002}. The empirical, non-classificatory and synchronic grammatical research produced in Australia informed a body of historical and classificatory literature which was overwhelmingly produced outside the country. The primary sources produced by missionaries in Australia, describing individual languages, provided fodder for the philological study of “Language” made overseas. There is, however, as pointed out by Newton (\citeyear[365--366]{newton_more_1987}), a curious neglect of Australian Aboriginal languages in some of the comparative philological works produced in Europe during the second half of the nineteenth century, for example, \citet{pott_einleitung_1884} and F. Max \citet{muller_lectures_1861}.

Table~\ref{tab:chap1:early} shows the major secondary studies of Australian linguistic structure that were informed by the corpus grammars. Following are brief overviews of some secondary works referred to throughout the study.

\begin{table}
\fittable{\small
    \begin{tabularx}{\textwidth}{QQQQQ}
        \lsptoprule
        Author, year of publication & Based on the earlier grammars by:  & Type of work & Metalanguage & Author's vocation \\
        \midrule 
        \multicolumn{5}{c}{Early secondary materials}\\
        \hline
        H. Hale, 1846 &	Threlkeld, 1834;\newline Watson, no date &	Classificatory	& English &	Philologist on the United States Exploring Expedition 1838--1842\\
        J. C. Prichard, 1847 &	Grey, 1841\newline Meyer, 1843 Schürmann, 1844 &	Classificatory. Prichard’s Australian material was presented in the fifth volume of \textit{Physical History of Mankind,} which was added to the polymath’s life’s work the year before his death.	& English &	British ethnologist and physician.\\
       W. Bleek, 1858, 1872 &	Extensively sourced material contained in Sir G. Grey’s library & 1858, Discussion of Australian linguistic structure 1872, Classificatory 	& English &	German linguist with expertise in African languages and curator of the library of Sir George Grey held in Cape Town, South Africa\\
        \lspbottomrule 
    \end{tabularx}
    }
    \caption{Secondary sources based on the corpus}
    \label{tab:1:2}
    \label{tab:chap1:early}
\end{table}

\begin{table}
	\fittable{\small
		\begin{tabularx}{\textwidth}{QQQQQ}
			\lsptoprule
			Author, year of publication & Based on the earlier grammars by:  & Type of work & Language written in & Author's vocation \\
			\midrule     
			F. Müller, 1867 &	Threlkeld, 1834;
			Watson, no date, Teichelmann and Schürmann, 1840, Schürmann, 1844a; Meyer, 1843; Moorhouse, 1846 & Linguistic report of the Voyage of the Austrian Frigate \textit{Novara}. Discussion of Australian linguistic structure. Classificatory & German & Viennese professor of Oriental languages and later of Sanskrit, and member of the Imperial Academy of Science in Vienna\\
			F. Müller, 1882 &	Threlkeld, 1834;
			Watson, no date,\newline Teichelmann and Schürmann, 1840, \newline Symmons, 1841,\newline Schürmann, 1844a;\newline
			Meyer, 1843; Ridley, 1866 &	Classificatory. Contains an edited collection of previously published \stockcaps{PN} grammars given within a four volume classification of the world’s languages (1876--1888)	& German &\\
			J. Fraser, 1892: xi-lxiv &	Threlkeld, 1834, Günther, 1840, \newline Symmons, 1841,  \newline Taplin, 1879 &	Classificatory and typological. Contains an edited collection of \stockcaps{PN} grammars	& English &	Interested individual\\
			W. Schmidt, 1919a, 1919b &	Vast range of extensively sourced lexical and grammatical material &	Classificatory	& German &	Eminent Viennese linguist, ethnologist and priest\\
			\lspbottomrule
		\end{tabularx}
	}
%     \caption{Secondary sources based on the corpus}
\end{table}

\begin{table}
\fittable{\small
    \begin{tabularx}{\textwidth}{QQQQQ}
        \lsptoprule
        Author, year of publication & Based on the earlier grammars by:  & Type of work & Language written in & Author's vocation \\
        \midrule
        S. Ray 1925 &	Most available material &	Discussion of Australian linguistic structure	& English &	British comparative and descriptive linguist\\
        \midrule
        \multicolumn{5}{c}{Later secondary materials}\\
        \hline
        A. P Elkin, 1937 &	Most available material &	Discussion of \newline Australian linguistic structure	& English &	Anglican clergyman, professor of anthropology at the University of Sydney\\
        Capell 1937 &	Most available material &	Discussion of \newline Australian linguistic structure	& English &	Anglican clergyman and professional Australian linguist\\
        \lspbottomrule
    \end{tabularx}
    }
%     \caption{Secondary sources based on the corpus}
\end{table}


As previously noted, the production of such material by the Australian academics \citet{capell_structure_1937} and \citet{elkin_nature_1937} heralded the arrival of a new descriptive era.

\citegen{fraser_australian_1892} edited volume republished grammars from the primary corpus -- \citet{threlkeld_australian_1834}, Günther (\citeyear{gunther_native_1838}, \citeyear{gunther_lecture_1840}), \citet{taplin_notes_1872} -- and presented Livingstone’s grammar of Minjangbal (\citeyear{livingstone_grammar_1892}) for the first time. In the introduction, Fraser also presented a classificatory study (ibid.: xi-lxiv) and a typological study of the phonology (ibid.: 1--8) of Australian languages drawing from a range of primary material. Both studies are anomalous within the body of nineteenth-century secondary literature in being written and published in Australia. Another early anomaly is the material presented by \citet{threlkeld_key_1850}, which compares Awabakal with predominantly Polynesian languages. 

\subsubsection{Wilhelm Bleek (\citeyear{bleek_library_1858}, \citeyear{bleek_position_1872})}
\label{sec:key:1.1.4.1}

In 1858, the German philologist W. H. I. Bleek (1827--1875), an authority on Khoi\-san languages, prepared a catalogue of the library of Sir George Grey (1812--1898), Governor of Cape Colony (1854--1861), which was held in Cape Town. As the Governor of South Australia (1841--1845), Grey had supported the publication of missionary grammars (\sectref{sec:key:5.1}), and as an explorer in Western Australia had himself published about Australian Aboriginal languages (\citeyear{grey_vocabulary_1839}). Upon Grey’s departure from Cape Town to take up the position of Governor of New Zealand (1861--1868) – for the second time, the first being 1845--1853 – Bleek was appointed curator of Grey’s library, a position he held from 1862 until his death \citep{gregorio_reflections_2002}. The library contained an unusually comprehensive collection of linguistic material from around the world. Bleek’s (\citeyear{bleek_library_1858}) catalogue presents the Australian material in Vol. II Part I, with a short \textit{addenda} in Part III of the same volume.

Bleek had received a doctorate in linguistics from the University of Bonn and had spent time at the University of Berlin, where he studied under K. R. Lepsius (1810--1884). Bleek collaborated intellectually with his cousin E. Haeckel (1834--1919), professor of comparative anatomy at the University of Jena.

\hspace*{-0.8pt}Grey’s continuing patronage allowed Bleek to broadly pursue philological studies \citep[170]{gilmour_grammars_2006}. Bleek later authored “On the position of Australian languages” (\citeyear{bleek_position_1872}).

\subsubsection{Friedrich Müller (\citeyear{muller_reise_1867}, \citeyear{muller_grundris_1882})}
\label{sec:key:1.1.4.2}

The most renowned early classification of Australian languages was published in German (\citeyear{schmidt_gliederung_1919}) by the Viennese linguist, anthropologist and Roman Catholic priest Father Wilhelm Schmidt (1868--1954; see \citealt[18--25]{koch_methodological_2004}). Much less well-known, but equally as informed for their time, are the classifications made by another Viennese philologist, Friedrich Müller (1834--1898), who was professor of Sanskrit and comparative philology at the University of Vienna. 

Note that Friedrich Müller is easily confused with the German-born Oxford Professor of comparative philology, Friedrich Max Müller (1823--1900), generally referred to as Max Müller, who in 1854 \citep[158]{muller_letters_1854} classified languages from the “Great Southern Continent” within the southern branch of the putative Turanian family. 

As a member of the Imperial Academy of Science in Vienna, F. Müller authored the linguistic (\citeyear{muller_reise_1867}) and ethnographic (\citeyear{muller_reise_1868}) reports of the Voyage of the Austrian Imperial “Novara” Expedition. The frigate circumnavigated the world between 1857 and 1859, and docked in Sydney for a month in 1858. Müller did not, however, take part in the expedition or ever visit Australia (contra \citealt[367]{newton_more_1987}). Müller sourced his material from Bleek, with whom he corresponded. With twenty-six pages of the report devoted to the description of Australian languages (\citeyear[241--66]{muller_reise_1867}), this Viennese publication introduced European philologists to a large amount of information about Australian morphosyntactic structure.

Between 1876 and 1888, F. Müller published \textit{Grundriss der Sprachwissenschaft} (Outline of Linguistics) in four volumes, which presented grammars of over one hundred languages from around the world. The material is presented according to the ``race'' of the people speaking the languages. Müller, Bleek, and Haeckel theorised about the origin of language within biological evolutionary frameworks. Their ideas form a sub-school of comparative linguistics that was particularly well-developed in Germany \citep{gregorio_reflections_2002}. Müller classified and ranked races according to hair-type (\citeyear[24]{muller_18761887_nodate}), and on this matter he referred to Haeckel (\citeyear[72--73]{haeckel_history_1876}). The Australian material is presented in Vol. 2 (\citeyear[1--98]{muller_grundris_1882}) “Der schlichthaarigen Rassen” (of the smooth-haired races).

F. Müller’s works have been overlooked in histories of Australian linguistics, including \citet{elkin_nature_1937}, Dixon (\citeyear[8--17]{dixon_languages_1980}, \citeyear{dixon_australian_2002}), \citet{koch_methodological_2004}, and \citet{mcgregor_introduction_2008}. His work has, however, been reviewed by \citet[2]{ray_aboriginal_1925} and by \citet[366--367]{newton_more_1987}. F. Müller’s (\citeyear{muller_grundris_1882}) work is significant to this study because this republication and translation of Australian grammars into German \textit{reanalyses} some of the data presented in the original. 

In this way, F. Müller’s work differs from the translation into German and publication of Australian grammatical material by the self-funded German ethnographer E. Eylmann (1860--1926; \citeyear{eylmann_eingeborenen_1908}). Eylmann visited the Hermannsburg mission for three months on his first inland Australian expedition of 1896--1898. He spent six weeks at Bethesda on his second expedition in 1900, and in the same year visited Point McLeay for eight days \citep[193--194]{schroder_erhard_2011}. In 1908 Eylmann republished much of \citegen{kempe_grammar_1891} Arrernte grammar (ibid.: 84--92), \citegen{taplin_notes_1872} Ngarrindjeri grammar (ibid.: 92--93), and Reuther’s (\citeyear{reuther_dieri_1894}, \citeyear{reuther_reuther_1899}) Diyari grammar \citep[93--98]{reuther_ms_1899}. 

\section{The corpus languages} 
\label{sec:key:1.2}

\figref{fig:key:1-3} shows the location of the Australian languages that were grammatically described in the pre-contemporary descriptive era. It shows the lower level PN subgroups in which each language is classified \citep[820]{bowern_computational_2012}. \tabref{tab:chap1:map} provides a key to the map and shows the higher-level PN subgroup in which each language is classified \citep{bowern_computational_2012}, as well as the source material for each language.

\begin{figure}
% \includegraphics[width=\textwidth]{figures/Au_map.png}
\includegraphics[width=\textwidth]{figures/StockigtMap.pdf}
\caption{Map showing location of languages described in the corpus, and lower-level PN subgroups in areas not shaded (after \citealt[820]{bowern_computational_2012}, used with permission of the authors). See key in \tabref{tab:chap1:map}.}
\label{fig:key:1-3}
\end{figure} 

\begin{table}
{\footnotesize
    \begin{tabularx}{\textwidth}{lp{2.5cm}llQ}
        \lsptoprule
        & Language  & Higher level & Lower-level & Early source material\\
        &			& \stockcaps{PN} sub-group$^{\dagger}$ & \stockcaps{PN} subgroup & \\
        \midrule
        1 &	Awabakal\newline (\stockcaps{HRLM} language) &	South-eastern	& Yuin-kuri & \citealt{Threlkeld1834}, \citealt{hale_languages_1846}\\
        2 &	Wiradjuri &	South-eastern	& Central \stockcaps{NSW} & Watson no date (lost) \citealt{Günther1838,gunther_lecture_1840,hale_languages_1846,mathews_wiradyuri_1904}\\
        3 &	Kaurna &	Central	& Thura-Yura & \citealt{TeichelmannSchürmann1840}\\
        4 &	Nyungar &	Western	& Nyungic & \citealt{Symmons1841}\\
        5 &	Ngarrindjeri  &	South-eastern	& Lower Murray & \citealt{meyer_vocabulary_1843,taplin_vocabulary_1867,taplin_notes_1872,Taplin1880}\\
        6 &	Barngarla &	Central	& Thura-Yura & \citealt{schurmann_vocabulary_1844} \\
        7 &	Ngayawang  &	South-eastern	& Lower Murray & \citealt{Moorhouse1846}\\
        8 &	Gamilaraay &	South-eastern	& Central \stockcaps{NSW} & 
        \citealt{ridley_kamilaroi_1855, ridley_kamilaroi_1855-1, ridley_kamilaroi_1856, ridley_kamilaroi_1866,ridley_kamilaroi_1875,mathews_languages_1903}\\
        9 &	Turrubul &	South-eastern	& Durubalic & \citealt{ridley_kamilaroi_1866}\\
        10 &	Diyari  &	Central	& Karnic & \citealt{koch_untitled_1868,Schoknecht1872,Flierl1880,reuther_ms_1899,planert_australische_1908,gatti_lingua_1930}\\
        11 &	Wergaya  &	South-eastern	& Kulin & \citealt{hagenauer_language_1878,mathews_aboriginal_1902} (Djadjala)\\
        12 &	Ganai  &	South-eastern	& Eastern Victoria & \citealt{bulmer_language_1878}\\
        13 &	Western Arrernte  &	Central	& Arandic & \citealt{Kempe1891}, C. \citealt{strehlow_grammatik_1931,strehlow_einige_1908,strehlow_notitle_1910,mathews_arranda_1907,planert_australische_1907,riedel_notitle_1931}, T. G. H. \citealt{strehlow_aranda_1944} \\
        14 &	Minjangbal  &	South-eastern	& Bandjalangic & \citealt{livingstone_grammar_1892}\\
        15 &	\stockcaps{WTS}  &	Northern 	& Paman & \mbox{\citealt{ray_study_1893,ray_linguistics_1907}} \\
        16 &	Pitta Pitta   &	Central	& Karnic & \citealt{roth_ethnological_1897} \\
        17 &	Guugu{\hyp}Yimidhirr  &	Northern 	& Paman & \citealt{schwarz_koko_1900,roth_structure_1901,ray_linguistics_1907}\\
        18 &	Wangkangurru & Central	& Karnic & \citealt{reuther_ms_1901b} \\
        19 &	Yandrruwandha &	Central	& Karnic & \citealt{reuther_ms_1901b} \\
        20 &	Nggerrikwidhi  & Northern & Paman & \citealt{hey_elementary_1903,ray_linguistics_1907}\\
        21 &	Yadhaykenu  &	Northern & Paman & \citealt{ray_linguistics_1907} \\
        22 &	Luritja  &	Western	& Wati & \citealt{mathews_languages_1907}, C. \citealt{strehlow_notitle_1910}\\
        \lspbottomrule
    \end{tabularx}
    }
    \legendbox[.8]{$^\dagger$According to \citet[820]{bowern_computational_2012}}
    \caption{Map and key showing location of languages described in the corpus, and lower-level PN subgroups (after \citealt[820]{bowern_computational_2012}, used with permission of the authors).}
    \label{tab:chap1:map}
\end{table}

Only twelve of the thirty-two currently recognised lower level PN subgroups \citep[820]{bowern_computational_2012} are sampled in the corpus. The four higher-level subgroups – Western, Northern, Southeastern and Central (\citealt[837--838]{bowern_computational_2012}) – are unevenly represented (\tabref{tab:chap1:map} \& \tabref{tab:chap1:consonantphonemes}). Only two corpus languages belong to the Western group. There is a small cluster of descriptions of languages from the Northern group. 

The high proportion of Southeastern languages, none of which survive, is partly attributable to that fact that these languages were spoken in the earliest-settled regions, and to the extensive work of R. H. Mathews, whose grammatical descriptions have been previously assessed by \citet{koch_r_2008}. Mathews’ large body of work \citep[211--216]{koch_r_2008}, while broad in scope, is narrow in depth. Mathews developed his own schema of Australian language description based on the traditional framework (\sectref{sec:key:4.3}).

The high proportion of languages from the Central group – all of which were spoken in South Australia, including areas now in the Northern Territory, annexed by South Australia from 1863 until 1911 – is entirely due to an active and prolonged Lutheran missionary effort administered from Adelaide and the Barossa Valley (\sectref{sec:key:8.1}).

\subsection{Linguistic structure}
\label{sec:key:1.2.1}

The following brief overview of the structure of PN languages attends specifically to areas of the grammar treated in this study. See Dixon (\citeyear{dixon_languages_1980}, \citeyear{dixon_australian_2002}) and \citet{koch_historical_2014} for broader overviews of PN structure, and grammars cited in \tabref{tab:chap2:modern} for further description of individual languages.

\subsubsection{Phonology}
\label{sec:key:1.2.1.1}

PN languages have fairly similar phonemic systems by cross-linguistic standards. Systems with three vowel phonemes are common, although some Arandic varieties are analysed as having only two \citep{breen_wonders_2001}. Consonants typically show a limited number of manner contrasts but a more extensive set of place of articulation contrasts. \figref{tab:chap1:consonantphonemes} shows a maximally contrastive inventory, although Arrernte and Diyari show additional distinctions: a series of pre-stopped and rounded consonants in Arrernte \citep{breen_wonders_2001} and a voicing distinction in Diyari restricted to apico-alveolar and retroflex stops in non-word-initial positions \citep[13]{austin_grammar_2013}. Guugu{\hyp}Yimidhirr follows the pattern of many languages in the country’s eastern third in having no retroflex series.

\begin{table}\footnotesize
\fittable{
    \begin{tabularx}{\textwidth}{llllllp{2cm}}
        \lsptoprule
         & Labial & Inter-dental & Apico-alveolar & Retroflex, Apico-domal & Palatal & Velar \\
        \midrule
        Stop & p & th	& t & rt & ty & k \\
        Nasal & m & nh	& n & rn & ny & ng \\
        Lateral & – & lh	& l & rl & ly & – \\
        Trill & – & – & rr & – & – & – \\
        Glide & w & –& (r) & r & y & (h) \\
        \lspbottomrule
    \end{tabularx}
     }
    \caption{The inventory of consonant phonemes common to many PN languages.}
    \label{tab:chap1:consonantphonemes}
\end{table}

\hspace*{-2.1pt}Early orthographic treatments of Australian phonologies tended to give a broad transcription of vowel quality, while phonemic articulation contrasts of consonants tended to be under-represented. The filter of the European ear saw the orthographic collapse of coronal consonant phonemes, with the letters: “t”, “n”, “l” used generically to represent stops, nasal and laterals at all coronal articulatory places. The velar nasal was often also undifferentiated, or not represented word-initially. Palatal stops were more likely than other coronal consonants to be distinguished, and were represented as “ty”, “tj” or “ch”. While the velar nasal was often undifferentiated from other nasals, or not represented word-initially, some very early grammatical sources (\citealt[1ff.]{dawes_grammatical_1790}; \citealt{hale_languages_1846}) represented the phone using engma (\ng). Rhotic phonemes were sometimes, but always inconsistently, distinguished. Assessments of individual grammarians’ orthographies -- for example, Koch’s (\citeyear[183--186]{koch_r_2008}) study of the system employed by Mathews -- point out exceptions to this general collapse of the system. Impressionistically, the phonemes least likely to be distinguished in the corpus grammars are the retroflex series and interdental nasals and laterals. A comprehensive comparative study of the early representation of Australian phonology remains to be done. \citet{newton_more_1987} makes a comparative study of orthographies before 1860, and \citet[15--17]{Stockigt2021a} investigates the periphonemic analysis and orthography developed by the University of Adelaide language committee in the 1930s.

\citet[vi]{threlkeld_australian_1834} adopted the spelling system employed in the description of languages from Polynesia, where he had spent six years at London Mission Society missions prior to coming to Australia, because he sensed propriety in adopting the “same character to express the same sounds used in countries which are adjacent”. Although \citet[v]{teichelmann_outlines_1840}, \citet[338]{gunther_lecture_1840}, and \citet[290]{ridley_kamilaroi_1856} were subsequently able to reassure their readers that they followed Threlkeld’s “method of spelling words”, early researchers were aware of, and frustrated by, the inadequacy of writing systems.

Phonological science was slow to enter Australian description. The earliest presentations of the sounds of Australian languages in systematic diagrams that set out consonant inventories in tables mapping place of articulation against manner of articulation, and vowels in triangular displays mapping height against backness occur in descriptions of Australian languages published in Europe (\citealt[64]{lepsius_allgemeine_1855}, \citeyear[226]{lepsius_standard_1863}; \citealt{muller_reise_1867}, \citeyear{muller_grundris_1882}; \citealt{planert_australische_1907}; \citeyear{planert_australische_1908}; \citealt[1]{gatti_lingua_1930}; \citealt[42, 45]{Sommerfelt_1938}; \tabref{tab:chap1:early}), by men who never visited Australia or heard an Australian language. 

The earliest such representation of an Australian language is in Lepsius (\citeyear[64]{lepsius_allgemeine_1855}, \citeyear[226]{lepsius_standard_1863}), which refers specifically to Teichelmann \& Schürmann’s grammar (\citeyear{teichelmann_outlines_1840}; \figref{fig:key:Lepsius}) but mentions no other Australian work. Lepsius’ representation of Kaurna phonology was first published in 1855 in German, and then in 1863 in English translation. It is possible that the 1840 grammar from Adelaide came to the attention of Lepsius via the missionaries’ connection with H. C. von der Gabelentz, whom the missionaries had met at the time of their ordination in Altenburg, and with whom Teichelmann corresponded. Lepsius’ publication predates \citet{bleek_library_1858}, which first introduced many European philologists to information about Australian languages.

\begin{figure}
\includegraphics[width=.8\textwidth]{figures/Figure 5.png}
\caption{Lepsius’ representation of Kaurna \citep[266]{lepsius_standard_1863}}
\label{fig:key:Lepsius}
\end{figure}

Informed by the missionaries’ grammar, \citet[226]{lepsius_standard_1863} produced a table of consonants and a vowel triangle purporting to represent Australian sound systems. Unlike modern practice, however, manner of articulation is shown on the horizontal axis instead of the vertical, and the vowel triangles are inverted, with the low vowels at the highest, rather than lowest, point of the diagram.


Soon after Lepsius’ presentation, F. \citet[245]{muller_reise_1867} presented a very similar grid of Australian consonants. Later graphic representations of Australian vowels and/or consonants appear in \citegen{muller_grundris_1882} publication in German, in W. \citegen[551--552]{planert_australische_1907} publication in German (\figref{fig:key:Planert}), in an Italian publication by G. \citet[1]{gatti_lingua_1930}, and in a French publication by A. \citet[42, 45]{Sommerfelt_1938}. 

\begin{figure}
\includegraphics[width=.8\textwidth]{figures/Figure 6.png}
\includegraphics[width=.4\textwidth]{figures/Figure 6b.png}
\caption{Planert's two-dimensional representations of Australian phonology \citep[551–552]{planert_australische_1907}}
\label{fig:key:Planert}
\end{figure}

These presentations of both consonants and vowels based on articulatory parameters made by European philologists appear not to have been read by grammarians in Australia, and if they were, not understood or assimilated into Australian practice. Aside from a confused attempt to show consonants in a grid made by Fraser (\citeyear[8]{fraser_introduction_1892}; \figref{fig:key:Fraser}), the earliest graphic representation of consonants published in Australia was \citet[8]{capell_new_1956}, and the earliest two-dimension\-al representation of vowel shape given in a grammar produced in Australia appears in T. G. H. Strehlow’s grammar of Western Arrernte (\citeyear[4]{strehlow_aranda_1944}).

\begin{figure}
\includegraphics[width=8cm]{figures/Figure 7.png}
\caption{Fraser's grid of Australian consonants \citeyearpar[8]{fraser_introduction_1892}}
\label{fig:key:Fraser}
\end{figure}

\subsubsection{Morphology}
\label{sec:key:1.2.1.2}

PN languages are synthetic, agglutinative and suffixing. Words can be multi-morphemic. Suffixes attach to the root or to the stem and tend to be mono-morphemic rather than portmanteau. 

The languages share a similar range of derivational processes, exhibiting productive nominalising processes, which form verbs from nouns, and processes which alter verb valency. T. G. H. \citet[62]{strehlow_aranda_1944} described such processes as “a grean \textit{(sic)} [great] boon to the missionaries translating the New Testament into the native language.” Examine, for example, the following Arrernte term, which shows the inchoative suffix \textit{-irre} attached to a nominal root \textit{tyelke} `flesh' to derive a verb meaning `to become flesh'. The verb is then nominalised with the suffix \textit{-ntye} to derive a lexeme used to translate `incarnation'.

\ea
\textbf{Tjálkeríntja} = incarnation \\
\glt \citep[62]{strehlow_aranda_1944} \\
\gll tyelke-irre-ntye \\
flesh-\stockcaps{INCH}-\stockcaps{NOM} \\
\z

\hspace*{-2.5pt}Verbs are morphologically complex, generally inflecting for tense, aspect, mood and a range of language-specific and wider areal categories, such as the category of associated motion in Arrernte (\citealt[270]{wilkins_mparntwe_1989}; \citealt[23]{koch_category_1984}; \sectref{sec:key:9.3.2}). Tense is marked word finally, although inflection for tense is sometimes followed by subordinating morphemes.

The degree to which the shape of inflectional suffixes is phonologically conditioned by the preceding stem varies between languages \citep{baker_word_2014}. Guugu{\hyp}Yimidhirr for example shows a high degree of allomorphy \citep[43--47]{haviland_guugu_1979}, while Arrernte, which historically has lost vowel-final distinctions (\citealt{breen_wonders_2001}; \citealt{koch_pama-nyungan_1997}), shows virtually none.

Nouns are not obligatorily marked for number, but in some instances may optionally be marked for dual and plural.\footnote{ Wangkangurru nouns may also be marked for trial number (\citealt[64]{hercus_grammar_1994}). }  Nouns that are unmarked for number have no specific or default number reference, and number is instead determined through context \citep[77]{dixon_australian_2002}.

Nominals do not generally exhibit grammatical gender, with a relevant exception found in Minjangbal (\citealt{livingstone_grammar_1892}; \citealt{crowley_middle_1978}; \sectref{sec:key:4.6.1}).

\subsection{Case systems}
\label{sec:key:1.2.2}

The Australian continent presents “the richest large-scale concentration of inflectional case languages anywhere in the world” \citep[xv--xvi]{blake_case_2001}. Pama-Nyungan languages have sophisticated inflectional case systems that are entirely synthetic \citep[9]{blake_case_2001}, with case inventories that are large by world standards \citep{iggesen_number_2013}. 

\subsubsection{Syntactic case}
\label{sec:key:1.2.2.1}

The core arguments of the verb – agent, subject, and object – are morphologically differentiated using split ergative systems (\citealt[6]{blake_case_1977}; \citealt[72]{dixon_preface_2002}) that are sensitive to an animacy hierarchy \citep{silverstein_hierarchy_1976}. The conception of case taken in this paper assumes a universal distinction between agent (A) marked by ergative case, intransitive subject (S) in nominative case and the object (O) in accusative case. \textit{Pronouns} often exhibit an accusative system (AS/O) in which the O is marked by an overt inflection, while the A and S remain unmarked. \textit{Nouns,} however, generally exhibit an ergative system (A/SO) in which the subject and the object are both unmarked and the agent in ergative case is morphologically differentiated. 

This study maintains a three-case analysis of syntactic case (\citealt{goddard_case_1982}; \citealt{wilkins_mparntwe_1989}; \citealt[224--226]{nordlinger_constituency_2014}) in which the nominative and the accusative cases are taken to remain underlyingly intact for classes of nouns on which they are identically marked. Correspondingly, the nominative and ergative cases are taken to exist even when these cases are identically marked. 

The three-case analysis upholds the conception of case maintained by traditional grammar, in which case is seen as a substitution class in which different classes of nominals may be marked differently in the same syntactic environment (\citealt{baerman_case_2002}). Note that under this three-case analysis the reference of the term “nominative” differs from its traditional usage in referring \textit{only} to a nominal acting as the subject of an intransitive clause. The traditional usage of the term, developed to describe the entirely accusative systems of SAE languages, refers to the relation carried by a nominal functioning in the role of subject as well as agent. 

\tabref{tab:chap1:split} shows the idiosyncratic and sometimes complex splits of the languages under consideration. Guugu{\hyp}Yimidhirr exhibits the canonical Australian ergative split. The situation in Diyari is shown at \tabref{fig:key:8-174}. For languages such as Kaurna, Barngarla and Ngayawang, in which all nominal types show either ergative (\stockcaps{A/SO}) or undifferentiated (\stockcaps{ASO}) systems and thus have \textit{no} overt accusative marking, a two case, ergative / absolutive analysis is sufficient. Core arguments are nevertheless consistently glossed as standing in ergative, nominative and accusative cases in all languages in this study.

\begin{table}\footnotesize
\fittable{
    \begin{tabularx}{\textwidth}{QQQQQ}
        \lsptoprule
        Language & Ergative alignment \stockcaps{A/SO} & Accusative alignment \stockcaps{AS/O} & Tripartite marking \stockcaps{A/S/O} & Undifferentiated \stockcaps{ASO} \\
        \midrule
        Awabakal & Common nouns & Pronouns, Proper nouns, Some nouns referring to people	&  Personal interrogative, Personal names & – \\
        \tablevspace
        Kaurna, Barngarla \& Ngayawang & Sg personal pronouns, All Sg nouns,  Demonstratives, Interrogatives & – & – &  All non-singular nouns, Non singular pronouns.\\
        \tablevspace
        Ngarrindjeri & Proper nouns & All 2\textsuperscript{nd} person pronouns, 1dl \& 1pl pronouns &  1sg, 3sg, 3dl \& 3pl pronouns, Demonstratives, Interrogative pronouns & Common nouns* (see \citealt[24--27]{bannister_longitudinal_2004})\\
        \tablevspace
        Guugu{\hyp}Yimidhirr & Nouns, Interrogatives & Personal Pronouns & – & – \\
        \tablevspace
        Western Arrernte & Common nouns, Interrogatives & All pronouns except 1sg & 1sg pronouns, Higher animate nouns &  – \\
        \lspbottomrule
    \end{tabularx}
    }
    \caption{The split in marking the syntactic cases in the languages treated in the corpus.}
    \label{tab:chap1:split}
\end{table}

\subsubsection{Peripheral cases}
\label{sec:key:1.2.2.2}

In addition to the three “core”, “syntactic” or “core clausal” \citep[132]{dixon_australian_2002} cases – nominative, ergative and accusative – PN case systems commonly have dative, ablative, allative, locative, instrumental, genitive, and comitative cases \citep[158]{blake_case_2001}. It is important to bear in mind that some peripheral cases, especially the dative, have a syntactic function when marking “the adjunct of intransitive verbs or the complement of semi-transitive verbs” \citet[330--331]{blake_australian_1979}.

Dixon identifies fourteen ``case functions'' commonly carried by nominals in Australian languages (\citeyear[132--143]{dixon_australian_2002}). These functions are always carried by a lesser number of \textit{case forms,} usually between eight and ten, in any given language (\citeyear[152]{dixon_australian_2002}). Because a greater number of case functions are usually carried by a smaller number of case markers, a single case inflection may carry a range of case functions. While there are strong, shared tendencies in the way that case functions group together to be marked by a single case form cross-linguistically, there are also regional and individual idiosyncrasies in the syncretism of case functions. 

\subsubsection{Pronouns}
\label{sec:key:1.2.2.3}

Sets of pronouns in singular, dual and plural number and in first, second and third-person are common. In some languages sets of third-person pronouns are replaced by sets of demonstratives (\citealt[5]{ray_aboriginal_1925}; \citealt[276--277]{dixon_languages_1980}, \citeyear[243]{dixon_australian_2002}).

Forms of pronouns marked for syntactic case are sometimes suppletive. Pronouns in peripheral cases are often marked by the suffix that attaches to nouns to mark the same function. The case of the pronominal stem to which peripheral case inflection attaches varies between languages. 

A number of languages examined in this study are among languages covering a continuous bloc of the continent’s southeast that have a set of bound personal pronouns, or pronominal enclitics (\citealt[337--401]{dixon_australian_2002}; \sectref{sec:key:3.3.6.1}; \sectref{sec:key:4.4.6}; \sectref{sec:key:5.5}; \sectref{sec:key:6.3}).

\subsubsection{Syntax}
\label{sec:key:1.2.2.4}

Clausal word order is generally free, although unmarked pragmatic word order tends to be AOV. Phrase-internal word order tends to be stricter. A NP may be continuous or non-continuous. The particular phrasal constituents on which case is marked is language-specific. Case can be marked on the final constituent, as in Arrernte (\citealt[102]{wilkins_mparntwe_1989}; \citealt[14]{henderson_topics_2013}) or on each constituent of a continuous NP, as in the following Pitta-Pitta clause. Note here that the adjective follows the noun. 

\ea
\gll {Machoomba-lo wapa-lo pooriti-na pokara-na tichea} \\
 {`the kangaroo’s pup is eating all the grass'} \\
\glt \citep[12]{roth_ethnological_1897}\\
\gll matyumpa-lu warrpa-lu ?-nha pukarra-nha thatyi-ya \\
kangaroo-\stockcaps{ERG} young-\stockcaps{ERG} ?all-\stockcaps{ACC} grass-\stockcaps{ACC} eat-\stockcaps{PRES}\\
\z

Complex sentence constructions vary between languages \citep{nordlinger_constituency_2014}. Commonly, relative clause constructions are conveyed through the “adjoined relative clause” \citep{hale_1976}, as in the following Ramindjeri example: 

\ea
\gll {Ngāte nakk-ir korne, yarn-…..ir an-ang–itye watañgrau} \\
{`I saw the man, he spoke to me yesterday'} \\
\glt \citep[33]{meyer_vocabulary_1843} \\
\gll Ngati nak-ir ko:rni yan-ir-anangk-itji watanggrau \\
1sg.\stockcaps{ERG} see-\stockcaps{PAST} man-[\stockcaps{ACC}], speak-\stockcaps{PAST}-1\textsc{sg}.\stockcaps{DAT}-3\textsc{sg}.\stockcaps{NOM} yesterday \\
\z

Arrernte and Diyari are among an areal group of languages that exhibit a system of ``switch reference'', in which subordinating suffixes attaching to a finite verb vary according to whether the subordinate verb has the same or a different subject from the main verb (\citealt{austin_notitle_1981}; \sectref{sec:key:8.7.3}, \sectref{sec:key:9.3.5}). Arrernte has fully embedded relative clauses (\citealt[414--423]{wilkins_mparntwe_1989}; \sectref{sec:key:9.3.5}).

\section{Outline of the study}
\label{sec:key:1.3}

Following this chapter, which establishes the corpus of early morphosyntactic description of PN languages as a previously under-researched and valuable field of historiographic enquiry, \chapref{chap:key:2} presents the philological methodology by which the grammars are investigated, and discusses challenges and opportunities presented by the data. It concludes with a review of the role descriptions of case in Australian languages play in the historiography of the terms “ergative” and “absolutive” (\sectref{sec:key:2.6}). Chapters \ref{chap:key:3}–\ref{chap:key:10} examine the corpus grammars in roughly chronological order. The study is arranged primarily by language, rather than by morphosyntactic topic, and is structured around schools of descriptive practice. The study identifies three descriptive schools: a New South Wales school (Chapters \ref{chap:key:3} \& \ref{chap:key:4}), a South Australian school (Chapters \ref{chap:key:5}, \ref{chap:key:6}, \ref{chap:key:8} \& \ref{chap:key:9}), and a Queensland school (\chapref{chap:key:10}). Each school is delineated by the constitutionally independent Australian colonies in which the described languages were spoken. Australian federation occurred in 1901, and the Northern Territory, where Arrernte is spoken, was not separated from South Australia until 1911.

Each early grammar is assessed in terms of its relative strength compared with other works in the corpus. Examined morphosyntactic categories are \textit{not} presented following conventional practice but are rather chosen with regard to the extent that the examined account is descriptively innovative or evinces influence from an earlier Australian source. For each grammar, the size of nominal case paradigms, the ordering of cases, the case names assigned to case functions, and the description of ergative function are observed. Emphasis is given to morphosyntactic categories which evince lineages of descriptive descent, including bound pronouns and processes of clause subordination. Descriptive breakthroughs, or instances of an author successfully accounting for a previously undescribed grammatical category are thus highlighted, as are the shortcomings of individual analyses in relation to the examined corpus. Brief biographical material is presented for most grammarians. More detailed material is available in the cited sources.

Chapters \ref{chap:key:3}–\ref{chap:key:7} discuss grammars of languages of which there is no, or little, modern record. The discussion in these chapters sometimes diverges to investigate the missionary-grammarians’ description of grammatical features of which there are different contemporary analyses, or of which the interpretation taken in this study differs from that usually made today based on the available source material. This type of investigation is made of Threlkeld’s presentation of compound pronouns (\sectref{sec:key:3.3.6.2}), of bound pronouns in Awabakal (\sectref{sec:key:3.3.6.1}) and in Kaurna (\sectref{sec:key:5.5}), and of Symmons’ description of ergativity in Nyungar (\sectref{sec:key:7.1.1.1}). The intention is to highlight the suppositions that have led to different interpretations.

\chapref{chap:key:3} investigates Threlkeld’s descriptive responses to the structure of Awabakal (\citeyear{threlkeld_australian_1836}) given in this first grammar of an Australian language, spoken in the earliest-settled colony of New South Wales. Since Threlkeld’s grammars have been described \citep[269]{carey_lancelot_2004} as “essential in establishing a framework” for later description, the assessment of his description of case marking on Awabakal nouns and pronouns and his response to the large Awabakal case systems, as well as his description of case allomorphy, establishes a baseline from which later corpus grammars are measured. 

\chapref{chap:key:4} investigates other languages spoken in New South Wales, commencing with Günther’s grammars of Wiradjuri (\citeyear{gunther_native_1838}, \citeyear{gunther_lecture_1840}) written at the Wellington Valley Mission, showing that similarities between Threlkeld’s grammar (\citeyear{threlkeld_australian_1834}) and Günther’s grammars (\citeyear{gunther_native_1838}, \citeyear{gunther_lecture_1840}) beyond those engendered by the inherited descriptive framework the authors shared are few. By comparing Günther’s grammars with later descriptions of Wiradjuri (\citealt{hale_languages_1846}; \citealt{mathews_wiradyuri_1904}), the study helps establish the provenance of early works emanating from Wellington Valley mission and discusses reasons for discrepancies in the sources. The discussion then turns to Ridley’s grammars of languages spoken in northern New South Wales and southern coastal Queensland, predominantly Gamilaraay (1855--1875). It then examines the grammar of Minjangbal (\citeyear{ridley_gurre_1892}) written by Rev. Hugh Livingstone, for whom virtually no biographical information has been discovered. This section establishes that materials held by Museum Victoria (\citealt{livingstone_notitle_1876}; \citealt{livingstone_notitle_1876-1}) contain part of the manuscript Minjangbal grammar, which was later published by Fraser \citep{livingstone_grammar_1892}. Both Ridley’s and Livingstone’s grammars are found to have had little influence on later works in the corpus, with the possible exception of one tenuous link between \citet{ridley_kamilaroi_1875} and \citet{roth_ethnological_1897} regarding the presentation of peripheral cases. 

\chapref{chap:key:5} and \chapref{chap:key:6} consider grammars of languages spoken in South Australia -- Kaurna (1840), Ramindjeri (1843), Barngarla (1844a) and Ngayawang (1846) -- which were made by Lutheran missionaries or written in the realm of their grammatical influence. While much of this part of the corpus has previously been subsumed under the label of the Adelaide descriptive school \citep[410]{simpson_notes_1992}, the discussion here shows that these texts nevertheless employ a range of divergent descriptive techniques when accounting for case systems. These strategies are compared with one another in order to demonstrate the influence that different aspects of these grammarians’ analyses exerted on later grammars.

\chapref{chap:key:5} considers Teichelmann \& Schürmann’s grammar of Kaurna \citeyearpar{teichelmann_outlines_1840}, spoken on the Adelaide Plains. This earliest grammar of a South Australian language is found to employ schemata that are substantially different from those used in earlier grammars of languages spoken in New South Wales. The discussion concentrates on the description of features that proved to be influential on later PN description. 

\chapref{chap:key:6} presents Meyer’s description of Ramindjeri (\citeyear{meyer_vocabulary_1843}), spoken south of Adelaide, observing a new descriptive strategy for accommodating the case system. The chapter focuses on Meyer’s presentation of ergative NPs, which differs from that of most other early grammarians. Following is an examination of Schürmann’s grammar of Barngarla (\citeyear{schurmann_vocabulary_1844}), spoken on the Eyre Peninsula, which focusses on Schürmann’s modification of descriptive practices made in his second grammar of an Australian language. The discussion briefly diverts to an overview of middle-era understandings (\citealt{ray_aboriginal_1925}; \citealt{capell_structure_1937}; \citealt{elkin_nature_1937}) of bound pronouns in Australian languages, based on the sources thus far examined, since these describe the only corpus languages known to exhibit systems of bound pronouns. The chapter concludes with a study of Moorhouse’s grammar of Ngayawang (\citeyear{moorhouse_vocabulary_1846}), the last and least detailed grammar of the “Adelaide school”. 

\chapref{chap:key:7} investigates other grammars written in the southern portions of the continent, commencing with Symmons’ grammar of Nyungar (\citeyear{symmons_grammatical_1841}), spoken in the Swan River Colony (Western Australia, Perth), which depicted ergativity in a manner similar to Meyer’s account. It then turns to Congregationalist missionary G. Taplin’s substantial descriptions of Ngarrindjeri (\citeyear{taplin_vocabulary_1867}, \citeyear{taplin_notes_1872}, \citeyear{taplin_grammar_1878}), a language closely related to Ramindjeri, which had previously been described by Meyer (\citeyear{meyer_vocabulary_1843}). Taplin’s earliest analysis is shown to have been strongly influenced by Meyer. Taplin, whose grammars were written for an international academic audience, constantly adapted his framework in order to make better representations of the language, and his last analysis (\citeyear{taplin_grammar_1878}) shows fresh influences from Teichelmann \& Schürmann. The chapter then turns to the grammatical sketches of languages from southeast of the country that appeared in Brough Smyth (\citeyear{thomas_succinct_1878}), Bulmer’s Ganai material from eastern Victoria (\citeyear[24--26]{bulmer_language_1878}) and the grammatical sketches of Wergaya, from western Victoria, by the Moravian missionaries A. Hartmann, F. W. Spieseke and F. A. Hagenauer (\citeyear[50--52, 56--58, 39--43]{hagenauer_language_1878} respectively). This body of material reproduces Taplin’s second to last case paradigm (\citeyear{taplin_notes_1872}), which is shown to play an important part within the historiography of the term “ergative”.

Chapters \ref{chap:key:8}–\ref{chap:key:9} assess the grammars written by Lutheran missionaries of Diyari and Arrernte, for which there is good modern description. The accuracy of early records of Diyari (1868--1901; \chapref{chap:key:8}) are assessed in terms of \citegen{austin_grammar_2013} analysis, and the early records of Arrernte (1891--1910; \chapref{chap:key:9}) in terms of \citegen{wilkins_mparntwe_1989} and \citegen{henderson_topics_2013} analyses. Both chapters attend closely to the provenance of analyses of the same language made by successive generations of missionaries at each mission and examine the ways in which each analysis sits within the larger South Australian Lutheran descriptive school that had been instigated by \citet{teichelmann_outlines_1840}. \chapref{chap:key:8} presents the earliest grammar of Diyari (\citeyear{koch_untitled_1868}), a manuscript written by W. Koch, which was located during the course of this study. \chapref{chap:key:9} presents manuscript grammars written by C. Strehlow (\citeyear{strehlow_untitled_1931}, \citeyear{strehlow_grammatik_1931}, \citeyear{strehlow_notitle_1910}) which have not previously been critically examined. 

\chapref{chap:key:10} investigates the description of languages spoken in Queensland, importantly Roth’s grammar of Pitta-Pitta (\citeyear{roth_ethnological_1897}; \sectref{sec:key:10.1.1}) showing that relatively late in the pre-academic time frame of Australian linguistic study, Roth instigated a new set of descriptive practices that showed little or no influence from earlier grammarians. Roth’s novel descriptive template was subsequently implemented in later grammars of languages spoken in Queensland, including his own grammar of Guugu{\hyp}Yimidhirr (\citeyear{roth_structure_1901}), Hey’s description of Nggerrikwidhi (\citeyear{hey_elementary_1903}), which Roth edited, and the manuscript grammar of Guugu{\hyp}Yimidhirr written by Lutheran missionaries \citet{schwarz_koko_1900}. The chapter then presents the context in which \citet{ray_study_1893} and \citet{ray_linguistics_1907} wrote grammars of Western Torres Strait (WTS) and languages spoken in far north Queensland.
