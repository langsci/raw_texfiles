\documentclass{langscibook}
\usepackage{langsci-gb4e}
\usepackage{langsci-optional}
\usepackage{langsci-gb4e}
\usepackage{langsci-lgr}

\usepackage{listings}
\lstset{basicstyle=\ttfamily,tabsize=2,breaklines=true}

%added by author
% \usepackage{tipa}
\usepackage{multirow}
\graphicspath{{figures/}}
\usepackage{langsci-branding}


\newcommand{\sent}{\enumsentence}
\newcommand{\sents}{\eenumsentence}
\let\citeasnoun\citet

\renewcommand{\lsCoverTitleFont}[1]{\sffamily\addfontfeatures{Scale=MatchUppercase}\fontsize{44pt}{16mm}\selectfont #1}
  
\begin{document}
\ea\label{ex:1:1}
\glll Kamil Yarri ngununda bumanabilla. \\
Gamil Yarri nganunda buma-na-bi-li\\
\textsc{neg} Harry-[\stockcaps{ERG}] 1sg\stockcaps{LOC} hit-\stockcaps{VD}?-\stockcaps{LET}\footnotemark{}\\
\glt Harry will not \textit{allow-any-one-to-beat} me. \\
\citep[76]{ridley_kamilaroi_1855-1}
\z\footnotetext{See Giacon (\citealt[355, 363]{giacon_2014}) for account of permissive voice glossed ‘LET’}




--
\ea \label{ex:1:2}
\gll Ngungu-dhi nhila buraay-nhu ngadhu-nhu; Minyaminyambul ngum-biya-girri. \\
Give-! That child-your I-for-you; Something-something give-constantly-will\\
\glt `Give me that child and I will give you plenty of compensation' \\

\z 





--
\ea\label{ex:1:3}
\glll Ngate pant-ir porle, balb-êmb-itye \\
Ngati pant-ir po:rli palp-emb-itji\\
1sg\stockcaps{ERG} bear-\stockcaps{PAST} child-[\stockcaps{ACC}] white-\stockcaps{RPAST}-3sg\stockcaps{NOM}\\
\glt By me (a) bringing forth has been child, white was it \\
“I brought forward a child and it was white: or, My child was white when it was born” \\
\citep[36]{meyer_vocabulary_1843}
\z



\ea
\glll Pulyunna meyu tittappe-urti, pindi  meyu nurru-ttoai \\
Pulyurn miyu tita-api-rti pinti miyu nuru-tuwayi\\
black man hang-\stockcaps{CAUSE}-\stockcaps{NEG}. \stockcaps{IMP} European curse-\stockcaps{AVERS}\\
\glt `Don’t hang the black man, that the European be not charmed [ensorcelled]' \\
\citep[69]{teichelmann_outlines_1840}
\z
	       
\ea
\glll Pulyunna meyurloyakko yailtyapindi meyubudnitina \\
Pulyurn {miyu yaku} {yailtya pinti} {miyu pudni-tina}\\
Black man \stockcaps{NEG} think European come-\stockcaps{PRIV}\\
\glt `The black man did not think that the white man would come' \\
\citep[69]{teichelmann_outlines_1840}
\z




--
\ea
\textbf{Tjálkeríntja} = incarnation \\
\gll tyelke-irre-ntye \\
flesh-\stockcaps{INCH}-\stockcaps{NOM} \\
\glt \citep[62]{strehlow_aranda_1944}
\z





--
\ea
\glll Machoomba-lo wapa-lo pooriti-na pokara-na tichea \\
matyumpa-lu warrpa-lu ?-nha pukarra-nha thatyi-ya \\
kangaroo-\stockcaps{ERG} young-\stockcaps{ERG} ?all-\stockcaps{ACC} grass-\stockcaps{ACC} eat-\stockcaps{PRES}\\
\glt `the kangaroo’s pup is eating all the grass' \\
\citep[12]{roth_ethnological_1897}
\z



\ea
\glll Ngāte nakk-ir korne, yarn-…..ir an-ang–itye watañgrau \\
Ngati nak-ir ko:rni yan-ir anangk-itji watanggrau \\
Isg\stockcaps{ERG} see-\stockcaps{PAST} man-[\stockcaps{ACC}], speak-\stockcaps{PAST}-1\textsc{sg} \stockcaps{DAT}-3\textsc{sg}.\stockcaps{NOM} yesterday \\
\glt `I saw the man, he spoke to me yesterday' \\
\citep[33]{meyer_vocabulary_1843}
\z



\section{Outline of the study}
--
\ea
\glll Mochoomba wapa \\
Matyumpa warrpa\\
Kangaroo-[\stockcaps{NOM}] pup-[\stockcaps{NOM}]\\
\glt `a kangaroo’s pup'  \\
\citep[8]{roth_ethnological_1897}
\z


--
\ea
\gll nhandru nguyama-yi marla ngakunga \\
3\textsc{sg}.\textsc{f}.\stockcaps{ERG} know-\stockcaps{PRES} more 1\textsc{sg}.\stockcaps{LOC}\\
\glt `she knows more than me' \\
\citep[112]{austin_grammar_2013}
\z





--
\ea
\gll Nyundu ganaa? Ganaa \\
2\textsc{sg}.\stockcaps{NOM} \stockcaps{OK} \stockcaps{OK}\\
\glt `how are you [=are you OK]? I’m fine [=OK]' \\
\citep[152]{haviland_guugu_1979} 
\z

\ea
\gll Nyundu mayi buda-y ganaa, ngali dhada-a \\
2\textsc{sg}.\stockcaps{NOM} food-[\stockcaps{ACC}] eat-\stockcaps{PAST} \stockcaps{OK} 1\textsc{dl}.\stockcaps{NOM} go-\stockcaps{NONPAST}\\
\glt `When you have eaten [enough?], then we’ll go' \\
\citep[152]{haviland_guugu_1979}
\z





--
\ea
\gll kuri-kupa \textit{or} kuri-ku\\
man-\stockcaps{POSS} {} man-\stockcaps{DAT}\\
\glt `of a man” {} “of a man'
\z



\ea
\gll artwe-nhe\\
man-\stockcaps{ACC}\\
\glt `to a man'
\z

But ‘to a man’ used with a verb of motion may have elicited an allative form:

\ea
\gll artwe-werne\\
man-\stockcaps{ALL}\\
\glt `to a man'
\z






--
\ea\label{ex:key:2-8}
\glll Jerlauandru\\
yarla-wa-ndru\\
elsewhere-\stockcaps{DIST}-\stockcaps{ABL}\\
\glt irgend woher, von weit her [from anywhere, from far away] \\
\citep[no pag.]{koch_untitled_1868}
\z

\ea\label{ex:key:2-9}
\glll Jerlauanni\\
yarla-wa-nhi\\
elsewhere-\stockcaps{DIST}-\stockcaps{ALL}\\
\glt irgend wohin von Entfernung [to anywhere distant] \\
\citep[no pag.]{koch_untitled_1868}
\z




--
\ea
\glll Ngan lakkir\\
Ngan lak-ur\\
1sg\stockcaps{ACC} spear-\stockcaps{PAST}\\
\glt I am speared \\
\citep[17]{taplin_grammar_1878} \\
“speared me”
\z



--
\ea
\glll Ngani demateraia\\
nganhi dama-tharri-yi\\
1\textsc{sg}.\stockcaps{NOM}   cut-\stockcaps{REFL}-\stockcaps{PRES}\\
\glt Ich schneide mich \\
\citep[26]{flierl_dieri_1880} \\
“I cut myself”

\z

\ea
\glll Ngaiani {antjama laia}\\
ngayani ngantya-mali\\
1\textsc{pl}.\stockcaps{EXCL} love–\stockcaps{RECIP}\\
\glt Wir lieben einander \\ 
\citep[26]{flierl_dieri_1880} \\
“We love one another”
\z
    


--
\ea
 \glll umma-lo uttapeukka-pityiri–wara-na\\
 ngama-lu ngathapiyaka-pityiri-wara-nha\\
 mother-\stockcaps{ERG}	offspring-\stockcaps{PL}-\stockcaps{POSS}-\stockcaps{ACC}\\
 \glt `a mother [is striking] her children' \citep[8]{roth_ethnological_1897}\\

\z




--
\ea\label{ex:key:2-13}\label{ex:2:13}
\glll Ganguru warka nulu goda dabelbi\\
gaangurru warrga nyulu gudaa dhabil-?dhi\\
kangaroo-[\stockcaps{ERG}] tall-[\stockcaps{ERG}] 3\textsc{sg}.\stockcaps{ERG} tame-dog-[\stockcaps{ACC}] kick-\stockcaps{PAST}\\
\glt `tall kangaroo was kicking the dog' \\
\citep[23]{roth_structure_1901}
\z

\ea\label{ex:key:2-14}\label{ex:2:14}
\glll Peba-ngato nulu kalka dirainggur-be uma\\
biibi-ngadhu nyulu galga dyiirraanhgurr-bi wu-maa\\
father-1\textsc{sg}.\stockcaps{POSS}-[\stockcaps{ERG}] 3\textsc{sg}.\stockcaps{ERG} spear-[\stockcaps{ACC}] man-\stockcaps{POSS} give-\stockcaps{PRES}\\
\glt father-my he the spear old man’s gives\\
\glt `my father gives the old man’s spear' \\
\citep[29]{roth_structure_1901}
\z



--
\ea\label{ex:key:2-15}
\gll Dyidyii-nda ngani dyindal-y ngaabaay\\
Bird-\stockcaps{ERG} 1\textsc{sg}.\stockcaps{ACC} peck-\stockcaps{PAST} head-[\stockcaps{ACC}]\\
\glt `the bird pecked me [in the] head' \\
\citep[58]{haviland_guugu_1979}
\z

\ea\label{ex:key:2-16}
\gll Yarrga-aga-mu-n gudaa gunda-y biiba-ngun\\
Boy-\stockcaps{POSS}-mu-\stockcaps{ERG}	dog-[\stockcaps{ACC}] hit-\stockcaps{PAST} father-\stockcaps{ERG}\\
\glt `The boy’s father hit the dog' \\
\citep[57]{haviland_guugu_1979}
\z





--
\ea\label{ex:3:3-17}
\begin{xlist}
\ex
\gll Tah-ke-le-ko \\                                 
eat-be-to-for \\
\glt "for to be eat" \\
\ex
\gll Tjaki-li-ku \\
eat-\stockcaps{NMSR}-\textsc{purp} \\
\glt  “for eating” \\
\citep[11]{threlkeld_specimens_1927}
\z
\z

--
\ea\label{ex:3:3-18}
\begin{xlist}
\ex
\gll Won-tah ko lahng bulah \\
Whither for do ye two \\
\glt `Whither are ye two going?' \\
\ex
\gll wantja-kulang pula \\
\stockcaps{INTER}-\stockcaps{ALL} 2\textsc{dl}.\stockcaps{NOM}  \\
\glt \citep[8]{threlkeld_specimens_1927}
\end{xlist}
\z

--
\ea\label{ex:3:19}
\begin{xlist}
\ex Ke-kul ko-ri-en un-ni yan-ti un-noa ki-lo-a \\
\glt This is not so sweet as that \\
\citep[17]{threlkeld_australian_1834}
\ex 
\gll kaykal-kuriyaN aNi ya\stockcaps{NT}i aNuwa-kiluwa \\
sweet-\stockcaps{PRIV}-[\stockcaps{NOM}] this-[\stockcaps{NOM}] thus that-\stockcaps{SEMB} \\
\glt \citep[62]{lissarrague_salvage_2006} 
\end{xlist}
\z
--
\ea\label{ex:3:3-20}
\begin{xlist}
\ex Ngali-noa \\
\glt This is he who
\ex    
\gll \textit{ngali-nyuwa} \\
this-[\textsc{erg}]-3\textsc{sg}.\stockcaps{ERG} \\
\glt `this one' \\
\citep[22]{threlkeld_australian_1834}
\end{xlist}
\z
--
\ea\label{ex:3:3-21}
\glll Ngan-umba       unni      	      wonnai?             Biriban-umba       unni   	               wonnai \\
ngan-ampa       aNi         	      waNay               Pirapan-ampa     aNi      	waNay \\
\stockcaps{INTER}-\stockcaps{POSS}   this-[\stockcaps{NOM}]    child-[\stockcaps{NOM}]     Pirapan-\stockcaps{POSS}      this-[\stockcaps{NOM}]       child-[\stockcaps{NOM}] \\
\glt Whose child is this? Biriban’s, this child. \\
(transcription \citealt[42]{lissarrague_salvage_2006}) 
\z

\ea\label{ex:3:3-22}
\glll Minnaring kopa	 unni?  \\
minyaring -kupa 	 aNi  \\
\stockcaps{INTER}-\stockcaps{POSS}	 this-[\stockcaps{NOM}] \\
\glt What does this belong to? \\
(transcription \citealt[51]{lissarrague_salvage_2006})
\z




--
\ea\label{ex:3:23}
\glll {Ammoung be }	{weah-lah.}	{Weah-lah be teah} \\
Amuwang-pi 	wiya-la.          	wiya-la-pi-tja \\
1\textsc{sg}.\stockcaps{ACC}-2\textsc{sg}.\stockcaps{NOM} 	speak-\stockcaps{IMP} 	speak-\stockcaps{IMP}-2\textsc{sg}.\stockcaps{NOM}-1\textsc{sg}.\stockcaps{ACC} \\
\glt `Speak to me. Do tell me' \\
(Transcription \citealt[46]{lissarrague_salvage_2006}) 
\z




--
\ea\label{ex:3:24}
\glll {Karai tia}      		 nguwa   	 emmoung    	 takilli ko \\
Karay-tja  		 ngu-wa  	 amuwang             tjaki-li-ku \\
Flesh-[\stockcaps{ACC}]-1\textsc{sg}.\stockcaps{ACC}	 give-\stockcaps{IMP} 	1\textsc{sg}.\stockcaps{ACC}		 eat-\stockcaps{NMSR}-\stockcaps{PURP} \\
\glt `Give me flesh to eat' \\
(Gloss and transcription from \citealt[42]{lissarrague_salvage_2006})
\z




--
\ea\label{ex:3:25}
\glll Minnung 	bunnun 	        ngaiya 	biloa?  \\
minyang 	wupa-NaN    ngaya 	piN-luwa \\
\textsc{interr:abs}	do-\textsc{fut} 	      then 	2\textsc{sg}.\textsc{acc}-3.\textsc{sg}.\textsc{nom}.\textsc{m} \\
\glt `What will he do to you?' \\
(Gloss and transcription from \citealt[42]{lissarrague_salvage_2006})
\z




--
\ea\label{ex:3:26}
\begin{xlist}
\ex Wi-yán    no-a   ba \\
\glt `While he speaks' \\
\ex
\gll Wiya-N		nyuwa-pa \\
speak-\stockcaps{PRES}	3\textsc{sg}.\textsc{m}.\stockcaps{NOM}-\stockcaps{SUB} \\
\glt (\citealt[70]{threlkeld_australian_1834})
\end{xlist}
\z

--
\ea\label{ex:key:1}
\gll Gamil-araay \\
no-\stockcaps{COM} \\
\glt “the language with the word ‘gamil’"
\z



\ea\label{ex:key:2}
\gll Wirraay-dhurraay  \\
no-\stockcaps{COM} \\
\glt “the language with the word ‘wirraay’"                                 
\z
    





--
\ea\label{ex:key:3}
\begin{xlist}
\ex buraugundi
\glt `to my boy' \\
\ex 
\gll Burai-gu-N-dhi \\
boy-\stockcaps{DAT}-?-1\textsc{sg}.\stockcaps{POSS} \\
\glt \citep[355]{gunther_lecture_1840}
\end{xlist}
\z

\ea\label{ex:key:4}
\begin{xlist}
\ex Buraigunu
\glt `to your boy' \\
\ex
\gll Burai-gu-nhu \\
boy-\stockcaps{DAT}-2\textsc{sg}.\stockcaps{POSS} \\
\glt \citep[355]{gunther_lecture_1840}
\end{xlist}
\z

\ea\label{ex:key:5}
\begin{xlist}
\ex  Buraigugula
“to his boy” \\
\ex
\gll Burai-gu-gula \\
boy-\stockcaps{DAT}-3\textsc{sg}.\stockcaps{POSS} \\
\glt \citep[355]{gunther_lecture_1840}
\end{xlist}
\z

--
\ea\label{ex:key:6}
\gll Murruba inda \\
ood 2\textsc{sg}.\stockcaps{NOM} \\
\glt `you are good' \\
\citep[39]{ridley_kamilaroi_1875}
\z 	



--
\ea
\gll Kaurna     {Encounter Bay Bob’s} \\
	  Meeyuna {Onkaparinga Jack’s} \\
\z



\ea
\gll korn-ar	miyu-rna \\
	  man-\textsc{pl}   man-\textsc{pl} \\
\z







--
\ea\label{ex:5:7}
  yangarutanna \\
\gll yangarra-tina \\
    wife-\stockcaps{PRIV}                            \\
\z



\ea\label{ex:5:8}
 mukartiana \\
\gll mukarti-ana\\
    hat-\stockcaps{ALL}\\
\z




--
\ea\label{ex:5:9}
Worli    worngangga \\
Before, in front of the house \\
\citep{teichelmann_dictionary_1857} \\
\gll warli warnka-ngka \\
house   omentum-\stockcaps{LOC} \\
\glt `In front of the house'
\z



--
\ea\label{ex:5:10}
Yantin   barun            yemmam-an         Marai -to             Eloi-kupa-ku \\
\glt `God’s spirit leads them all' \\
\citep{threlkeld_australian_1834} \\
\gll Ya\stockcaps{NT}iyn   paraN            ?-N                  Maraye-tju      \textbf{Eloi}-kupa-ku \\
        all           3\textsc{pl}.\stockcaps{ACC} 	  lead-pres           spirit-\stockcaps{ERG}       God-poss-\stockcaps{ERG}\\
\z




--
\ea\label{ex:5:11}
ninkuitya \\
\gll ninku-itya \\
{}[2\textsc{sg}\stockcaps{POSS}]-\stockcaps{DAT}/\stockcaps{ALL}                         \\
\z

      	  

          


\ea\label{ex:5:12}
Ngangko   		wakwakurlo 	  niinanni   yüngki?	      Ngaityurlo \\
\glt `Whose child gave it to you?'						Mine \\
\citep[12]{teichelmann_outlines_1840} \\
\gll ngangku    		wakwaku-rlu 	 ninna-ni            yungk-i	       ngatiyu-rlu \\
 \stockcaps{INTER}.\stockcaps{POSS}    	  	child-\stockcaps{ERG}   	 2sg-\stockcaps{DAT}          give-\stockcaps{PAST}      1\textsc{sg}.\stockcaps{POSS}-\stockcaps{ERG}                                \\
\z

       	

       	
--
\ea\label{ex:5:13}
Ngando    aityo  		 mudlinna                       metti \\
Who has taken away my implements? \\
\citep[10]{teichelmann_outlines_1840} \\
\gll  Nganthu-aityu  		 mudli-rna  	          mitti \\
\stockcaps{INTER}.\stockcaps{ERG}-1\textsc{sg}.\stockcaps{DAT}  	 implement-[\stockcaps{ACC}]-\textsc{pl}    steal-[\stockcaps{PAST}]	                         \\
\glt `who has taken away the implements on me' \footnote{See \sectref{sec:key:7.3.3} for further discussion of this clause. }
\z

\ea\label{ex:5:14}
Ninna 	ngannarlo 	minkarni \\
By what have you been wounded? \\
\citep[10]{teichelmann_outlines_1840} \\
\gll Niina	 ngana-rlu            minka-rni	 \\
2sg\stockcaps{ACC}        \stockcaps{INTER}-\stockcaps{INST} 	wound-\stockcaps{INCH}-[\stockcaps{PAST}]                                \\
\z

      	       


--
\ea\label{ex:5:15}
Parnda-rlo 		ngatto 	  wodli                 taie-ta.  \\
“I will build the house with bricks” \\   	        
\citep[24]{teichelmann_outlines_1840} \\
\gll Parnta-rlu  		 ngathu     wardli              tayi-tha \\
limestone-\stockcaps{INST}   	1\textsc{sg}.\stockcaps{ERG}   house-[\stockcaps{ACC}]   build-\stockcaps{FUT}                               \\
\z
      	         

      	         

--
\ea\label{ex:5:16}
Niina	   narta     padne-ota,    ngadlu   yaintya   wandi-adlu \\
\glt `You are going, (but) we, we shall sleep here' \\
\citep[23]{teichelmann_outlines_1840} \\
\gll niina          narta    padni-utha,   ngadlu    yaintya   wanti-adlu \\
2\textsc{sg}.\stockcaps{NOM}    now     go-\textsc{fut}           1\textsc{pl}.\textsc{s}        here        sleep-1\textsc{pl}.\textsc{s} \\
\z




\ea\label{ex:5:17}
Tidnarla     	nguiyuatto      	purla \\
\glt `The feet, I will warm them' \\
\citep[23]{teichelmann_outlines_1840} \\
\gll Tidna-rla   	nguyu-athu      	purla \\
feet-3\textsc{dl}.\stockcaps{ACC}       warm-1\textsc{sg}.\stockcaps{ERG}     3\textsc{dl}.\stockcaps{ACC}                                    \\
\z




--
\ea\label{ex:5:18}
(nguiyuatturla)	\\
\citep[23]{teichelmann_outlines_1840} \\
\gll nguyu-athu-rla \\
warm-1\textsc{sg}.\stockcaps{ERG}-3\textsc{dl}.\stockcaps{ACC}                              \\
\glt `I warm them'
\z




--
\ea\label{ex:5:19}
kundaingki \\
\glt       ‘let him beat’ \\
\citep[17]{teichelmann_outlines_1840} \\
\gll kunda-ing-kV-Ø \\
beat-?-\stockcaps{IMP}-3\textsc{sg}.\stockcaps{ERG}\\
\z

\ea\label{ex:5:20}
tikkaingko \\
\glt ‘let him sit’ \\
\citep[17]{teichelmann_outlines_1840} \\
\gll tikka-ing-kV-Ø\\
sit-?-\stockcaps{IMP}-3\textsc{sg}.\stockcaps{NOM}\\
\z




--
\ea\label{ex:5:21}
Kauwitya        kundowarponendi ai \\
\glt `I wish to have water' \\
\citep{teichelmann_outlines_1840} \\
\gll kauwi-tya       kuntuwarpu-rni-nthi-ai \\
water-\stockcaps{DAT}    {chest bone-\stockcaps{INCH}-\stockcaps{PRES}-1\textsc{sg}.\stockcaps{NOM}\footnotemark}                               \\
\glt `I long for water'
\footnotetext{Where \textit{kuntuwarpu-rni-} ‘chest bone-INCH is a metaphor for ‘desire’ \citep[165]{amery_kulurdu_2013}}
\z


\ea\label{ex:5:22}
 Kundo       punggorendaii   	 ngaityo 		yungakko \\
\glt     	“I am concerned about, or long for my elder brother” \\
\citep{teichelmann_dictionary_1857} \\
\gll Kuntu	  pungku-rri-nth-ai   	  ngaityu		 yunga-ku \\
    chest	  hit-\stockcaps{REF}-\stockcaps{PRES}-1\textsc{sg}.\stockcaps{NOM}   1\textsc{sg}.\stockcaps{POSS}             {older brother-\stockcaps{DAT}}	\\
\glt `I am concerned about my older brother'
\z



--
\ea\label{ex:5:23}
Ngando inna 		pulyunna    meyurlo	 anto-kartando 		yungk-i?  \\
\glt     “What black man has given you the kangaroo skin?” \\
\citep[68]{teichelmann_outlines_1840} \\
\gll Nganthu-iina                    pulyuna    miyu-rlu  	nantu-kartantu		yungki? \\
who.\stockcaps{ERG}-2\textsc{sg}.\stockcaps{ACC}          black        man-\stockcaps{ERG}  kangaroo-skin-[\stockcaps{ACC}]       give-\stockcaps{PAST}\\
\z




\ea\label{ex:5:24}
Itto		 ngaityo 		yungaitya	     kattinga \\
\glt `these bring, carry to my elder brother'
       
\gll itu                 ngaityu      	 yunga-itya	    kat-inga \\
       \stockcaps{DEM}.\textsc{pl}         1\textsc{sg}.\stockcaps{POSS}  	{older brother-\stockcaps{ALL}}   bring-\stockcaps{IMP}.2.\textsc{pl}                            \\
\z

The analysis taken here proposes that the verb is better glossed:

\ea
\gll Kat-ing-a \\
              Bring-?\stockcaps{IMP}-2\textsc{pl}.\stockcaps{ERG}                    \\

\z


--
\ea\label{ex:5:25}
Tarralyoanna   mutyertanna    wondando,		 yerta   buttonettoai \\
\glt `Put the clothes on the table, lest they be (or become) spoiled by the earth'\\
\citep[18]{teichelmann_outlines_1840} \\
\gll Tarraly-ana    mutyarta-rna         wanta-nthu,\footnotemark{}	 yarta-purtu-rni-tuwayi \\
 table-\stockcaps{ALL}     clothes-\textsc{pl}              put-2\textsc{sg}.\stockcaps{ERG}              earth-full-\stockcaps{INCH}-\stockcaps{AVERS}                            \\
\footnotetext{For discussion of the form of the 2sgERG pronoun, in this and the following example, see \fnref{fn:5:60}}
\z



\ea\label{ex:5:26}
Yurrepaiaiandunna,                           kundattoai             parna. \\
\citep[18]{teichelmann_outlines_1840} \\
\glt `You must pay attention to them (the goats) lest they kill (them)'
\gll Yuri-payi-nthu-rna,     		kuntu-tuwayi	  parna\\
ear-examine-2\textsc{sg}.\stockcaps{ERG}-3\textsc{pl}.\stockcaps{ACC}	kill-\stockcaps{AVERS}	  3\textsc{pl}.\stockcaps{ACC}\\
\z


\ea\label{ex:5:27}
Ngatto       punggetitya     wārpunna               pingga \\
\glt `I have made the dagger for the purpose of stabbing' \\
\citep[20]{teichelmann_outlines_1840} \\
\gll ng.athu    pungku-titya	warpu-rna	 pingka  \\
   1\textsc{sg}.\stockcaps{ERG}     kill-\stockcaps{PURP}  	dagger-\textsc{pl}-[\stockcaps{ACC}]   make-[\stockcaps{PAST}]                              \\
\z




--
\ea\label{ex:5:28}
Ninna  ngattaityangga  wānggama,          nindaitya aii            budnama.  \\
\glt  ‘If you had spoken to me, I would have come to you.’ \\
\citep[19]{teichelmann_outlines_1840} \\
\gll Niina   ngathaityangka   wangka-ma,         ninthaitya-ai\footnotemark{}             pudna-ma \\
2\textsc{sg}.\textsc{s}    1\textsc{sg}.\stockcaps{COM}	           speak-\stockcaps{HYPO},      2\textsc{sg}.\stockcaps{DAT}-1\textsc{sg}.\stockcaps{ACC}     arrive-\stockcaps{HYPO}                               \\
\footnotetext{The form of the 2sgDAT pronoun, ninthaitya, shown here follows \citet[137]{amery_kulurdu_2013}. The form is likely to be nintaitya (see \citet[62]{koch_languages_2014}}
\z


\ea\label{ex:5:29}
Niwa        yakko   ngarkoma,     niwa         yakko      padloma \\
\glt `If you two had not eaten,   you would not have died' \\
\citep[67]{teichelmann_outlines_1840} \\
\gll Niwa          yaku   ngarku-ma,    niwa          yaku     padlu-ma \\
    2\textsc{dl}.\stockcaps{NOM}    \stockcaps{NEG}   eat-\stockcaps{HYPO},    2\textsc{dl}.\stockcaps{NOM}    \stockcaps{NEG}  die-\stockcaps{HYPO}\\
\z




--
\ea\label{ex:5:30}
Ngatto     ngurrintyidla,     ninna           yungkoma \\
\glt Were I permitted to throw, I would give (the bird) to you \\
\citep[19]{teichelmann_outlines_1840} \\
\gll Ngathu	 nguri-ntyidla,      niina           yungku-ma \\
1\textsc{sg}.\stockcaps{ERG}  throw-\stockcaps{COND},     2\textsc{sg}.\stockcaps{ACC}     give-\stockcaps{HYPO}\\
\z




--
\ea\label{ex:5:31}
Ngurluntya      ai                   kunda,           tikkandi	urlo \\
\glt `That man struck me, who is sitting there' \\
\citep[13]{teichelmann_outlines_1840} \\
\gll ngurlu-ntya-ai		 kurnta,          tika-nthi-urlu \\
\textsc{dem}.\stockcaps{ERG}-\textsc{indef}-1\textsc{sg}.\stockcaps{ACC}    hit-[\stockcaps{PAST}]    sit-\stockcaps{PRES}-?\\
\glt ?“Someone hit me, the one sitting”
\z



--
\ea\label{ex:5:32}
manyarendurlo (manyarendi ngurlo) \\
when it rains then …  \\
\citep{teichelmann_verb_1858} \\
\gll manya-rri-nthi-urlu \\
rain-\stockcaps{RECIP}/\stockcaps{REFL}-\stockcaps{PRES}-?                        \\
\z

\ea\label{ex:5:33}
wakwakurlo ngurretti urlo (ngurlo) \\
\glt just when the boy had thrown then … \\
\citep{teichelmann_verb_1858} \\
\gll wakwaku-rlu     ngurru-thi-urlu \\
child-\stockcaps{ERG}         throw-\stockcaps{PAST}-?                    \\
\z

		

		
--
\ea\label{ex:5:34}
Idlo  atto          numa     nakkoma,       padlo        ngai      turnki              yungkoma        idlo \\
\glt `him I would love, who would give me clothing' \\
\citep[13]{teichelmann_outlines_1840} \\
\gll idlu-athu              numa   naku-ma,     padlu     ngai     turnki        yungku -ma  idlu \\
\stockcaps{DEM}.\stockcaps{ERG}-1\textsc{sg}.\stockcaps{ERG}  well        look-\stockcaps{HYPO},    3\textsc{sg}.\stockcaps{ERG}   1\textsc{sg}.\stockcaps{ACC}  cloth-[\stockcaps{ACC}]  give-\stockcaps{HYPO}    \stockcaps{DEM}.\stockcaps{ERG}                              \\
\glt `This one, I might love [him], he might give me clothing, this one'
\z


\ea\label{ex:5:35}
Ngatto  pa           wadli    nakko-ndi,     ngai         turnki             padlo       yakko      yungkondi \\
\glt `but him I hate who gives me no clothing' \\
\citep[13]{teichelmann_outlines_1840} \\
\gll Ngathu    pa           waadli	 naku-nthi,     ngai         turnki             padlu       yaku    yungku-nthi \\
1sg\stockcaps{ERG} 3\textsc{sg}.\stockcaps{ACC}  dislike    see-\stockcaps{PRES},   1\textsc{sg}.\stockcaps{ACC}   cloth-[\stockcaps{ACC}]  3\textsc{sg}.\stockcaps{ERG}   \stockcaps{NEG}   give-\stockcaps{PRES}                                \\
\z




--
\ea\label{ex:key:36}
\begin{xlist}
	\ex
	\gll Ram- ... inyeri -ap     porl \\
	Ramong  {}   of       I       child \\
	\glt  `I am a native of Ramong' \\
	\citep[63]{meyer_vocabulary_1843}
	\ex 
	\gll Ram-inyeri-ap        po:rli\\
	Ram-\stockcaps{ASSOC}-1\stockcaps{SG.NOM}       child \\
\end{xlist}
\z
--
\ea \label{ex:key:37}
\glll Yarn-ir-an-angg-itye \\
yarn-ir-anangk-itye \\
speak-\stockcaps{PAST}-3\textsc{sg}.\stockcaps{DAT}-3\textsc{sg}.\stockcaps{NOM}\footnotemark \\

\z

--
\ea\label{ex:key:38}
\begin{xlist}
	\ex Ngate nakk-ir korne
	\glt `by me seeing has been a man' \\
	\citep[33]{meyer_vocabulary_1843}
	\glt `I have seen the man'
	\ex
	\gll ngati nak-ir ko:rni \\
	1\textsc{sg}.\stockcaps{ERG}   see-\stockcaps{PAST}   man-[\stockcaps{ACC}] \\
\end{xlist}
\z
--
\ea\label{ex:key:39}
\begin{xlist}
	\ex Ngand-im memp-ing 
	\glt `by whom have you been beaten' \\
	\citep[33]{meyer_vocabulary_1843}
	\ex
	\gll  ngand-im memp-ing\\
	\stockcaps{inter}.\stockcaps{ERG}-2\textsc{sg}.\stockcaps{ACC} hit-\stockcaps{PAST} \\
\end{xlist}
\z

--
\ea\label{ex:key:40}
\glll Korn-il lakk-in māme \\
Ko:rn-il lak-un  ma:mi \\
man-\stockcaps{ERG} spear-\stockcaps{PRES} fish-[\stockcaps{ACC}] \\
\glt  `The man spears the fish.'
\z

is

\ea\label{ex:key:41}
\glll Korne laggel-in mām-il \\
Ko:rni lak-el-in ma:m-il \\
man-[\stockcaps{NOM}] spear-\stockcaps{ANTIP}-\stockcaps{PRES} fish-\stockcaps{INST}\footnotemark]] \\
\glt `The man spears the fish.'

\z




--
\ea\label{ex:key:42}
\begin{xlist}
	\ex Nganto wiyan? 
	\glt `Who speaks?' \\
	\citep[127]{Threlkeld1834}
	\ex
	\gll NgaN-Tu wiya-n \\
	\stockcaps{INTER}-\stockcaps{ERG}   speak-pres \\
\end{xlist}
\z

\ea\label{ex:key:43}
\begin{xlist}
	\ex Ngan                  unnung                wiyellin                       yong 
	\glt Who there talking out there \\
	\citep[127]{Threlkeld1834}
	\ex
	\gll NgaN                   aNang       wiya-li-N          yung \\
	\textsc{inter}.\stockcaps{NOM}      that\stockcaps{NOM}        speak-\stockcaps{DTR}-\stockcaps{PRES}    there\footnotemark\\

\end{xlist}
\z
--
\ea\label{ex:key:44} % gloss needs re-alignment, confusing in orig
\begin{xlist}
	\ex
	\gll Ngāte {} nakk- ir korne, yarn-… ir an-ang-itye watañgrau \\
	{By me} (a) seeing {has been} man, speaking was {me to he} yesterday \\
	\glt `I have seen the man who spoke to me yesterday' \\
	\citep[33]{meyer_vocabulary_1843}
	\ex
	\gll Ngati nak-ir ko:rni, yarn-ir-anang-itye watangrau \\
	1\textsc{sg}.\stockcaps{erg} see-\stockcaps{past} man-[\stockcaps{acc}]   speak-\stockcaps{past}-1\textsc{sg}.\stockcaps{all}/\stockcaps{dat}-3\textsc{sg}.\stockcaps{nom}       yesterday \\
\end{xlist}
\z
--
\ea\label{ex:key:45}
\begin{xlist}
	\ex
	\gll Nāiye lēw-…in mant-angg-an \\
	That {living  is} {house at my} \\
	\glt `He who lives at my house' \\
	\citep[32]{meyer_vocabulary_1843}
	\ex
	\gll Naiyi le:w-in mant-angk-an \\
	DEM sit-\stockcaps{PRES} house-\stockcaps{LOC}-1\stockcaps{POSS} \\
	\glt `that one lives at my house'
\z
\z
--
\ea\label{ex:key:46} % unclear
\begin{xlist}
	\ex
	\gll  Ngande-….m-..angg-…..engg-ul {}  ram-……ing? \\
    {By whom you to them two}  {by (a)} {speaking was}? \\
	\glt `Who are the two who told you?' \\
	\citep[32]{meyer_vocabulary_1843}
	\ex
	\gll ngandi-mangk-engkul ram-ing \\
	\stockcaps{inter}.\stockcaps{erg}-2\stockcaps{dat}-3\textsc{dl}.\stockcaps{erg} speak-\stockcaps{past}\\
	\glt `which two spoke to you?'\footnote{The number of the 2DAT -\textit{mangk} bound pronoun is unspecified. See \sectref{sec:key:7.3.2}.}
\z
\z
--
\ea % needs re-formatting
\begin{xlist}
	\ex Mangalla yurarri innamatta
	\glt `friendly men (are) these' \\
	\citep[10]{Schürmann1844}
	\ex
	\gll Mangarla yurha-rri inha-madla \jambox{(Gloss and transcription \citealt[161]{Clendon2014})} \\
	Friendly man-\stockcaps{hum}. \stockcaps{PL}    this-\stockcaps{pl}\\
\end{xlist}
\z

--
\ea\label{ex:key:48}
Marrályinga ngai píttanarrù kányanga
\glt `The boy me did hit with a stone'\\
\citep[5]{schurmann_vocabulary_1844} \\
\gll Marralyi-nga ngayi birda-nga-aru ganya-nga \\
Boy-\stockcaps{ERG} 1\textsc{sg}.\stockcaps{ACC}  hit/pelt-\stockcaps{PAST}-3\textsc{sg}.\stockcaps{ERG} stone-\stockcaps{INST} \\
\z




--
\ea\label{ex:key:49} % formatting
\glll Karpanga   {iridningutu adli} \\
garrba-nga   iridni-ngu-dhu-w-adli \\
house-\stockcaps{ERG}   separate-\stockcaps{CAUS}-\stockcaps{PRES}-\stockcaps{EP}-1\textsc{dl}.\stockcaps{ACC}\\
\glt `the house/room separates us' \\
\hfill (\citealt[V:8]{schurmann_vocabulary_1844}; Gloss and transcription from \citet[44]{Clendon2015})
\z




--
\ea \label{ex:key:50}
\begin{xlist}
	\ex Bud-jor Yag-go-ål   bi-an-a-ga
	\glt The ground was dug by the woman \\
	(\citealt{symmons_grammatical_1841}: x, xxii)
	\ex 
	\gll \textbf{Bud-jor}          \textbf{Yaggo-ål}      \textbf{biana-ga}\\
	ground-[\stockcaps{ACC}]    woman-\stockcaps{ERG}   dig-\stockcaps{PAST}\\
	\glt `The woman dug the ground'
\end{xlist}
\z
--
\ea\label{ex:key:51}
\begin{xlist}
	\ex Ngan-ya   wau-gål-ål   bak-kan-a-ga
	\glt I was bitten by the snake \\
	(\citealt{symmons_grammatical_1841}: xiv)
	\ex 
	\gll \textbf{Nganya}   \textbf{waugål-ål}           \textbf{bakkana-ga}\\
	1\textsc{sg}.\stockcaps{ACC}  snake-\stockcaps{ERG}         bite-\stockcaps{PAST}\\
	\glt The snake bit me
\end{xlist}
\z
--
\ea\label{ex:key:52}
\begin{xlist}
	\ex Purnangunnanna laplapnanna ngape mukkarna
	\glt ‘Large knife me did wound’ or its equivalent \\ `With a large knife I was wounded' \\
	\citep[24]{moorhouse_report_1841}
	\ex
	\gll \textbf{Purnangu-nnanna} \textbf{laplap-nanna} \textbf{ngape} \textbf{mukk-arna}\\
	large-\stockcaps{ERG} knife-\stockcaps{ERG}  1\textsc{sg}.\stockcaps{ACC} wound-?\\
	\glt `The large knife wounded me'
\end{xlist}
\z
--
\ea\label{ex:key:53}
\begin{xlist}
	\ex Kile    yan          pettir
	\glt `by him it was stolen' \\
	\citep[88]{taplin_notes_1872}
	\ex
	\gll Kili-yan             peth-ur\\
	3\textsc{sg}.\stockcaps{ERG}-3\textsc{sg}.\stockcaps{ACC}       steal-\stockcaps{PAST}\\
\end{xlist}
\z

--
\ea\label{ex:key:54}
\begin{xlist}
	\ex
	\gll Ngate   yan   ellani \\
	{By me it will be done} \\
	\glt `I will do it' \\
	\citep[88]{taplin_notes_1872}
	\ex
	\gll ngati-yan      el-ani\\
	1\textsc{sg}.\stockcaps{ERG}-3\textsc{sg}.\stockcaps{ACC}    do-\stockcaps{FUT}\\
\end{xlist}
\z
--
\ea\label{bkm:Ref516554518}\label{ex:key:55}
\begin{xlist}
	\ex Ya-go         my-ak-al        yu-gow                 bar-da-ga
	\glt `The woman has come to the house' \\
	\citep{symmons_grammatical_1841}
	\ex
	\gll Yaku            miyak-al          yuka-w              parta-ka\\
	woman\textbf{{}-[\stockcaps{NOM}]} house-\stockcaps{ALL} stand-?\stockcaps{PRES}      house-\stockcaps{PAST}\\
\end{xlist}
\z

\ea\label{bkm:Ref516554630}\label{ex:key:56}
\begin{xlist}
	\ex Ngad-jo         yån-gor-in                   ngan-gow bru
	\glt `I do not see the kangaroo' \\
	\citep{symmons_grammatical_1841}
	\ex \gll Ngatyu                 yankor-iny          ngana-?w-buru\\
	1sgA        kangaroo\textbf{{}-}\stockcaps{ACC}          see-\stockcaps{PRES}-\stockcaps{NEG}\\
\end{xlist}
\z


--
\ea\label{bkm:Ref329630469}
\begin{xlist}
	\ex kuba natu nandrai \\
	\glt `ich schlage den Knaben'\\
	(\citealt{koch_untitled_1868}: no pag.)
	\ex
	\gll kupa              ngathu     nandra-yi\\
	boy-[\stockcaps{NOM}]  1\textsc{sg}.\stockcaps{ERG}   hit-\stockcaps{PRES}\\
	\glt `I hit the boy'
\end{xlist}
\z
--
\ea\label{bkm:Ref329630485}
\begin{xlist}
	\ex Kintella kuballi nandrai \\
	\glt Der Knabe schlägt den Hund” \\
	(\citealt{koch_untitled_1868}: no pag.)
	\ex
	\gll Kinthala       kupa-li    nandra-yi\\
	dog-[\stockcaps{ACC}]   child-\stockcaps{ERG}    hit-\stockcaps{PRES}\\
	\glt `The child hits the dog'
\end{xlist}
\z
--
\ea\label{bkm:Ref329630508}
\begin{xlist}
	\ex nanni wappai \\
	\glt Ich gehe \\
	(\citealt{koch_untitled_1868}: no pag.)
	\ex
	\gll nganhi wapa-yi\\
	1\textsc{sg}.\stockcaps{NOM} go-\stockcaps{PRES}\\
	\glt `I go'
\end{xlist}
\z
--
\ea\label{bkm:Ref329630522}
\begin{xlist}
	\ex Kuba tikai \\
	\glt Der Knabe kehrt zurück \\
	(\citealt{koch_untitled_1868}: no pag.)
	\ex
	\gll kupa thika-yi\\
	boy-[\stockcaps{NOM}] return-\stockcaps{PRES}\\
	\glt `The child returns'
\end{xlist}
\z
--
\ea
worakunaru\\
\gll waRakunha-ru\\
\stockcaps{INTER}.\stockcaps{POSS}-\stockcaps{ERG}\\
\z


--
\ea
\glll {kintella ulo} \\
*kinthala-wurlu \\
dog-\textsc{dl}-[\stockcaps{ACC}]\\
\glt ‘two dogs’
\z





--
\ea\label{bkm:Ref329626776}
\begin{xlist}
	\ex kalkauralu, \\
	\glt `until the evening' \\
	\citep[52]{flierl_dieri_1880}
	\ex
	\gll Kalkawarra-ya-rlu\\
	evening-\stockcaps{ALL}-\stockcaps{STILL}\\
\end{xlist}
\z

\ea\label{bkm:Ref329626810}
\begin{xlist}
	\ex jidni ko ja nga ni koldra \jambox{\citep[52]{flierl_dieri_1880}}
	\glt `You are ignorant and so am I'
	\ex
	\gll Yini          kuwu         nganha          kuwu-lda\\
	2\textsc{sg}.\stockcaps{NOM}    ignorant    1\textsc{sg}.\stockcaps{NOM}        ignorant-\textsc{addinf}\\
\end{xlist}
\z


--
\ea\label{bkm:Ref329626419}
\begin{xlist}
	\ex kupanto
	\glt `with a child' \\
	\citep[52]{flierl_dieri_1880}
	\ex
	\gll {Kupa-nthu}\\
	child-\stockcaps{PROP}\\
\end{xlist}
\z

\ea\label{bkm:Ref329626478}
\begin{xlist}
	\ex waru warula
	\glt of old. Formerly \\
	\citep[52]{flierl_dieri_1880}
	\ex
	\gll waru-waru-lha \\
	{long ago-\stockcaps{REDUP}-\stockcaps{CHAR}} \\
	\glt `of old'
\end{xlist}
\z
--
\ea\label{bkm:Ref329627527}
mitandru {out of the Earth}\\
\gll  {mitha-ndru}\\
ground-\stockcaps{ABL}\\
\glt \citep[52]{flierl_dieri_1880}
\z

\ea\label{bkm:Ref329627545}
widlandru {on account of the woman}\\
\gll  {wilha-ndru}\\
woman-\stockcaps{ABL}\\
\glt \citep[52]{flierl_dieri_1880}
\z


\ea\label{bkm:Ref329627573}
marani {in or on the hand}{}
\gll   {mara-nhi}\\
hand-\stockcaps{LOC}\\
\glt \citep[52]{flierl_dieri_1880}
\z


\ea\label{bkm:Ref329627599}
nghapaia towards water
\gll  {ngapa-ya}\\
water-\stockcaps{ALL}\\
\glt \citep[52]{flierl_dieri_1880}
\z



\ea
\label{bkm:Ref329627843}
pitali with the stick
\gll  {pirta-li}\\
stick-\stockcaps{ERG}/\stockcaps{INST}\\
\glt \citep[52]{flierl_dieri_1880}
\z

--
\ea\label{bkm:Ref329628202}
Mochoomba wapa\\
a kangaroo’s pup
\gll Matyumpa warrpa\\
Kangaroo-[\stockcaps{NOM}]    pup-[\stockcaps{NOM}]\\
\glt  \citep[8]{Roth1897}
\z




--
\ea\label{bkm:Ref329628748}
Teraia         kalti{} {\textit{nicht}}   teri kalti,
\gll    \textit{tari-ya        kalthi} not  \textit{tari   kalthi}\\
boy-\stockcaps{POSS}  spear       \textit{not}    boy-[\stockcaps{NOM}] spear\\
\glt (\citealt{koch_untitled_1868}: no pag.)
\z

\ea   \label{bkm:Ref329628762}
kintella     milki   {\textit{nicht}}{} Kintellaia milki
\gll    kinthala      milki not  \textit{kinthala-ya   milki}\\
dog-[\stockcaps{NOM}]   eye   \textit{not} dog-\stockcaps{POSS}   eye\\
\glt (\citealt{koch_untitled_1868}: no pag.)
\z



\ea          mataraia kalti, {\textit{der mannes speer,}} {\textit{aber nicht}} mataraia milki, {\textit{der mannes augen, sondern}}{} matari milki\\
 \textit{mathari-ya kalthi} the man’s spear, but not      \textit{mathari-ya milki} the man’s eye,      rather       \textit{mathari  milki}

\citep[12]{flierl_christianieli_1880}

                         man-POSS spear, ‘the man’s spear; but not:    man-POSS eye, ‘the man’s eye’, rather     man-[NOM] eye
\z

(\citealt{koch_untitled_1868}: no pag.) also exemplified the construction in instances in which the possessor is pronominal:


--
\ea\label{bkm:Ref339715937}
Ninkidani     wapamai,      nato     nandraiati!\\
\glt `Come here, otherwise I strike!' \\
(\citealt{schoknecht_grammar_1947}: 10[1872]) \\
\textit{nhingki-rda-nhi    wapa-mayi,      ngathu    nandra-yathi} \\
here-vicin-ALL     go.IMP-EMPH,   1sgERG   hit-AVERS
\z



\ea
\label{bkm:Ref339715961}Jesu       antjanimai,                jura             paliati!\\
\glt `Love Jesus, or you will die!'

(\citealt{schoknecht_grammar_1947}: 10[1872])

\textit{Yesu       ngantya-ni-mayi,           yura           pali-yathi}

    Jesus-[\stockcaps{ACC}]    love.\stockcaps{IMP}-\stockcaps{NM}-\stockcaps{EMPH}    2pl\stockcaps{NOM}    die-\stockcaps{AVERS}\\
\glt `love Jesus, lest you die!'
\z
--
\ea\label{bkm:Ref339719815}
\gll \textit{pantha-ma-mayi          kilthi               ngakarni     ngathu     wayi-rna    wara-rnanhi} \\
smell-\textsc{tr}.\stockcaps{IMP}-\stockcaps{EMPH}   stew-[\stockcaps{ACC}]  1\textsc{sg}.\stockcaps{POSS}    1\textsc{sg}.\stockcaps{ERG}    cook-\textsc{ptcp}    \textsc{aux}-\stockcaps{IMPERF}. \stockcaps{DS}\\
\glt `smell my stew, that I cooked' \\
(\citealt{austin_grammar_2013}: 214[1981a])
\z



\ea\label{bkm:Ref339728750}
judla ~   ~taji-nani,             ~judla ~              pali-nani\\
\glt    `If you eat, you will die' \\
(\citealt{schoknecht_grammar_1947}: 12[1872])
\z

\ea
\gll yula           thayi-rnanhi,   yula                 pali-rnanhi\\
2\textsc{dl}.\stockcaps{NOM} ~  eat-\stockcaps{COND},     2\textsc{dl}.\stockcaps{NOM}          die-\stockcaps{COND}\\
\z






\ea
\gll \label{bkm:Ref340429075}pali-lha    ngana-yi\\
die-\stockcaps{FUT}  \textsc{aux}-\stockcaps{PRES}\\
\glt ‘will die’
\z




\label{bkm:Ref340429028}\label{fig:key:8-186}

--
\ea   Kaparajai, ~ ~ jidni ~ ~      ninkida ~ ~     ngana-nani,       neji ~ ~      ngakani ~  ~ wata ~ ~    pali-nani\\
\glt `Lord if thou had been here, my brother would not have died' \\
\citealt{reuther_testamenta_1897})
\gll Kaparra-yayi      yini           nhingkirda  ngana{}-rnanhi,   nhiyi      ngakarni       wata       pali-rnanhi\\
boss-\stockcaps{EMPH}  2\textsc{sg}.\stockcaps{NOM}   here     be-\stockcaps{COND}       brother   1\textsc{sg}.\stockcaps{DAT}    \stockcaps{NEG}     die-\stockcaps{COND}\\
\z





--
\ea\label{bkm:Ref339729473}
\gll nganhi        wakara-rna     wara-yi,     yundru     matya    nganthi            wayi-rna      wara-ni\\
1\textsc{sg}.\stockcaps{NOM}   come-\textsc{ptcp}     \textsc{aux}-\textsc{pres}   2\textsc{sg}.\stockcaps{ERG}   already   meat-[\stockcaps{ACC}]  cook-\textsc{ptcp} \textsc{aux}-\textsc{seq}.\stockcaps{DS}\\
\glt  ‘I came after you had already cooked the meat’ \\
(\citealt{austin_grammar_2013}: 227[1981a])
\z

\ea
\label{bkm:Ref340480092}
\gll pantha-ma-mayi          kilthi               ngakarni     ngathu     wayi-rna    wara-rnanhi\\
smell-\textsc{tr}.\stockcaps{IMP}-\stockcaps{EMPH}   stew-[\stockcaps{ACC}]  1\textsc{sg}.\stockcaps{POSS}    1\textsc{sg}.\stockcaps{ERG}    cook-\textsc{ptcp}    \textsc{aux}-\stockcaps{IMPERF}.\stockcaps{DS}\\
\glt `smell my stew, that I cooked' \\
(\citealt{austin_grammar_2013}: 214[1981a])
\z



% Double check
--
\ea
   \label{bkm:Ref339729609}Ninnaia     anxale     wappaia,     nanna   anxanalli\\
    Den liebe ich, welcher mich liebt,\\

\citep{koch_untitled_1868}

 \gll nhinha-ya              ngantya-lha\footnotemark{} wapa-ya,        nganha       ngantya-?\\
    3\textsc{sg}.\stockcaps{ACC}-near        love-\textsc{ptcp}/\stockcaps{FUT}  \textsc{aux}-\stockcaps{PAST}      1\textsc{sg}.\stockcaps{ACC}        love-?\\
\glt    “I loved the one who loves me’
\z

--
\ea
\label{bkm:Ref516470384}
\gll   Ninaia                   ngato         antjai,            ngana        antjani,\\
Den liebe ich, welcher mich liebt,\\
\citep[26]{flierl_dieri_1880}
\gll nhinha-ya            ngathu       nganta-yi,     nganha     nganta-ni\\
3\textsc{sg}.\stockcaps{ACC}-near      1\textsc{sg}.\stockcaps{ERG}     love-\stockcaps{PRES},  1\textsc{sg}.\stockcaps{ACC}    love-\textsc{seq}.\stockcaps{DS}\\
\glt `I loved the one who loves me'
{nunkangu}      {ngani}            {kalakalariai,}           {ngakangu}     {kalakala}  {rina}\\
und den hasse ich welcher mich haßt.\\
\citep[26]{flierl_dieri_1880}
\z
--
\ea        Ngato     ninaia                  ngantjai,      nulia                   ngakangu   ngumu          nganka-na  warai\\
\glt `Ich liebe den, der mir gut gewesen ist' \\
\citep[29]{reuther_dieri_1894}
\gll ngathu    nhinha-ya           nganta-yi,   nhulu-ya              ngakangu  ngumu         ngank-nha  wara-yi\\
1sg\stockcaps{ERG}      3sg\stockcaps{ACC}-near           love-\stockcaps{PRES},      3sgnf\stockcaps{ERG}-near         1sg\stockcaps{LOC}         good                make-\stockcaps{PART} \stockcaps{AUX}-\stockcaps{PRES}\\
\glt `I love him who has been good to me'
\z




--
\ea
 Ninna          ngannaitya           yellarra   yakko      pepaitya      budni\\
\glt `why do you not come to school today'

\citep{teichelmann_dictionary_1857}

\gll niina           ngana-itya            yalarra   yaku        pepa-itya     pudn-i\\
         2\textsc{sg}.\stockcaps{NOM}     \stockcaps{INTER}-\stockcaps{PURP}     today      \stockcaps{NOT}       paper-\stockcaps{ALL}       come-\stockcaps{PAST}\footnotemark\\

\z

--
\ea
    Galtjintana-Pepa\\
\gll kaltye-         anthe-   nhe-        pepe\\
knowledge   give      \stockcaps{NOM}     paper\\
\glt      ‘book that gives knowledge”
\z



\ea
Kristianirberaka\\
\gll Kristian-   iRpeRa-   ke\\
Christian   \textsc{pl}             \stockcaps{DAT}/\stockcaps{POSS}\\
\z






--
\ea            lunaluna = redeemer \\
\gll             irlwe- nhe- irlwe -nhe\\
           to{ }loosen,{ }untie -\stockcaps{NMSR}. \textsc{hab}.\textsc{rdp}\\
\glt (T. G. H. \citet{strehlow_aranda_1944}: 363[1938])
\z\todo{glosses do not match}




--
\ea
   \label{bkm:Ref340586104}tutyalbuma\\
return to beat\\

(\citealt{kempe_grammar_1891}: 20; \citealt{mathews_languages_1907}: 334)

\gll Twe-ty=alpe-me\\
hit-\stockcaps{RETURN}\& \stockcaps{DO}-\stockcaps{PRES}\\
\z


--
\ea
            \label{bkm:Ref340586632}Tulabuma\\
\glt `beating, by walking about'

\citep[19]{kempe_grammar_1891}

 \gll Twe –rle=pe- me\\
                    hit-\stockcaps{CONT}\& \stockcaps{MOT}-\stockcaps{PRES}\\
\z


--
\ea
\label{bkm:Ref340587267}Ilkutjinjama\\
\glt `to eat in the morning, to breakfast'

\citep[20]{kempe_grammar_1891}

\gll irlkwe-ty=antye-me\\
eat – \stockcaps{DO} \stockcaps{UPWARDS} – \stockcaps{PRES}\\

\glt `to eat while the sun rises'
\z\todo{glosses do not match}
--
\ea
\label{bkm:Ref340587421}Ilkutjikalama\\
\glt `to eat in the evening, to sup'

\citep[20]{kempe_grammar_1891}

\gll irlkwe-tye=kerle-me\\
eat - \stockcaps{DO} \stockcaps{DOWNWARDS} – \stockcaps{PRES}\\
\glt `to eat while the sun goes down'
\z\todo{glosses do not match}

--
\ea\label{bkm:Ref340591178}
Era          atunga         ulara        tnama \\
\glt `he             of man         in front    stands' \\
\citep[4]{Kempe1891}

\gll Re          artwe-nge    lwarre    irtna-me \\
3\textsc{sg}.\stockcaps{NOM}    man-\stockcaps{ABL}    in{ }front  stand-\stockcaps{PRES}\\
\glt `he stands in front of the man'
\z


--
\ea{}
[X-rle (Y) Vb finite – (rle)]\textsubscript{SREL} (3pnDEF) – CASE  \\

\z




--
\ea
\label{bkm:Ref339739442}Atua   nala        nana             tmurka       albuka     worana      tukala\\
\glt `the man who went away yesterday has beaten the boy'

\citep[11]{kempe_grammar_1891}

\gll Artwe    nhale,          nhenhe        apmwerrke   alpe-ke,          werre-nhe   atwe-ke-le\\
Man      that.\stockcaps{ERG}    that.\stockcaps{NOM}    yesterday     return-\stockcaps{PAST}   boy-\stockcaps{ACC}    hit-\stockcaps{PAST}-\stockcaps{SS}\\
\glt `the man, who returned yesterday, is the one who hit the boy'
\z

--
\ea
\label{bkm:Ref339787713}Ilupa   tera   iltala    mbakatnarakala   jiraka\\
\glt `the two axes, which were leaning on the house disappeared'

  \citep[11]{kempe_grammar_1891}
\z





--
\ea
Ilepe   therre       ilthe(-le)-rle               ampeke-tne-rre-ke-rle                         uyerre-ke\\
\gll {}[ilepe  therre]\textsubscript{HD}    [ilthe(-le)-rle              ampeke-tne-rre-ke-rle]\textsubscript{SREL}                uyerre-ke\\
     axe   two-[\stockcaps{NOM}]                        [shelter(-\stockcaps{LOC})-\stockcaps{REL}           lean-stand-\textsc{dl}.S/A-\stockcaps{PAST}-\stockcaps{REL}]-[\stockcaps{NOM}]  disappear-\stockcaps{PAST}\\
\glt `the two axes, which were leaning against the house disappeared'
\z





--
\ea
\gll  Ilepe   atherre             ilthe-le         ampeke-tne-rre-ke-le               uyerre-ke\\
Axe    two-[\stockcaps{NOM}]       shelter-\stockcaps{LOC}    lean-stand-duS/A{}-\stockcaps{PAST}-\stockcaps{SS}      disappear-\stockcaps{PAST}\\
\glt `the two axes were leaning against the house before they disappeared'
\z





--
\ea
\label{bkm:Ref339794050}Atua       lena,                    nana              lata               pitjikala,          mara       nama\\
\glt `mann dieser, der da heute gekommen ist, gut ist'

 (C \citealt{strehlow_untitled_1931} [c.1907]: 39--40)

\gll Artwe      lanhe,                nhanhe          lyete      petye-ke-rle,             marre    ane-me\\
Man        that.\stockcaps{NOM}          this.\stockcaps{NOM}      today    come-\stockcaps{PAST}-\stockcaps{REL}      good      sit-\stockcaps{PRES}\\
\glt `That man, the one who came today, is good'
\z

--
\ea
     \label{bkm:Ref339788704}Ara     nana         unta              nguruka         ntainatnala (part.perf.)   andere    naka\\
\glt `kaenguru dies da, du gestern gespeert habend  fett war'

(C. \citealt{strehlow_untitled_1931} [c.1907]: 39--40)

\gll Aherre     nhenhe      unte              ngwerreke     irntarne-rtne-rle              antere      ane-ke\\
Kangaroo   this         2sg\stockcaps{ERG}      yesterday       spear-?-\stockcaps{REL}      fat             sit-\stockcaps{PAST}\\
\glt `this kangaroo, that you speared yesterday, was fat'
\z

--
\ea
\gll    \label{bkm:Ref339788405}Korne    naiye    lewin    mantangg  an\\
Mann      der       wohnte im Haus  mein\\
\glt `Der Mann , der in meinem Hause wohnte'

(C. \citealt{strehlow_grammatik_1931} [c.1907] after \citet{meyer_vocabulary_1843}: 32)
\z



\ea
\gll    \label{bkm:Ref77507345}ngato    ninaia    ngantjai,    nulia   ngakangu    nguma   nankara   warai\\
Ich ihn liebe, er mir gutes getan hat\\
\glt `Ich liebe den, der mir gutes getan hat'

(C. \citealt{strehlow_grammatik_1931} [c.1907] after \citealt{reuther_dieri_1894}: 29)
\z




--
\ea
      \label{bkm:Ref77507401}Ngani   Gabrieli   nganai,          nauja    Godani    terkai\\
 \glt io sono Gabriele che sto innanzi a Dio

\citep[67]{gatti_lingua_1930}

\gll Nganhi Gabriel    ngana-yi,     nhawa   Goda-nhi tharka-yi\\
    1sg\stockcaps{NOM} Gabriel  be-\stockcaps{PRES},     3\textsc{sg}.\stockcaps{NF}   God-\stockcaps{LOC}  stand-\stockcaps{PRES}\\
\glt `I am Gabriel, who stands in the presence of God'
\z

--
\ea
iwu\c{n}a              aţa      arugula   altaraka,      la\c{n}a    aţa      jusem-ila-ka
\glt `what I first found, that I used'

(T. G. H. \citealt{strehlow_aranda_1944}: 101[1938])


\gll Iwenhe             the     arrekwele    arltare-ke,       lenhe   the     usem-ile-ke\\
 \stockcaps{DEM}              1\textsc{sg}.\stockcaps{ERG}      first              find-\stockcaps{PAST}      \stockcaps{DEM}   1\textsc{sg}.\stockcaps{ERG}   use-\textsc{tr}-\stockcaps{PAST}\\
\z

--
\ea
\label{bkm:Ref465086262}Kana-lo nopo-na nungkarea ; kooyungo-ngooro\\
\glt `The man takes care of his wife; (he is) a good fellow'

(\citealt[24]{roth_ethnological_1897}; Pitta-Pitta)

       \label{bkm:Ref465086269}bama diral nangu(go) na-(mal)-ma : nulu bodan\\
\glt `The man who looks after his wife is a good fellow' \\

(\citealt[no pag.]{schwarz_koko_1900}; Guugu-Yimidhirr)
\z
--
\ea
        \label{bkm:Ref339469326}bama diral nangu-gobantchen-chil: nulu bodan\\
\glt `The man who nurses his own wife is a good fellow' \\
(\citealt[18]{roth_structure_1901}; Guugu-Yimidhirr)
\z

and in Hey’s description of Nggerrikwidhi:

\ea
\gll     ma  endranana-nu   nguno-ma    yi    sea  : ma tanko\\
    man woman-to   his-very-own food gives : man good\\
\glt `The man who gives food to his own wife is good man' \\
(\citealt[13]{hey_elementary_1903}; Nggerrikwidhi)
\z





--
\ea
l Machoomba-lo   wapa-lo          pooriti-na          pokara-na      tichea\\
\glt `the kangaroo’s pup is eating all the grass'

                  \citep[12]{roth_ethnological_1897}

\gll matyumpa-lu       warrpa-lu       ?-nha       pukarra-nha   thatyi-ya\\
kangaroo-\stockcaps{ERG}    young-\stockcaps{ERG}     ?all-\stockcaps{ACC}        grass-\stockcaps{ACC}      eat-\stockcaps{PRES}\\
\z


--
\ea
wungata\footnote{The form \textbf{wungata} ‘woman’ is not recorded by \citet{blake_pitta_1979}}{}-ngo    kanari-ko      moorra-ngo           pite\\
\glt `The woman will hit the girl with the stick'

(\citealt{roth_ethnological_1897}: 16, Pitta-Pitta)

\gll wangatha-ngu                  kiniyari-ku    murra-ngu             pithi\\
woman-\stockcaps{FUT}/\stockcaps{ERG}      girl-\stockcaps{DAT}       stick-\stockcaps{FUT}/\stockcaps{INST}   hit-[\stockcaps{FUT}]\\
\z

--
\ea
Ngando-ngai         dana            ngundar \footnotemark{} dogar-en               baitchare\label{bkm:Ref77508300}n\\
the-woman             they            the-plums           with-sand            covered

(\citealt{roth_structure_1901}: 30, Guugu-Yimidhirr)

\gll ngaandhu-ngay       dhana              ?                      dyuugaar-\$inh        baydya-rrin\\
     woman-\stockcaps{PL} 3\textsc{pl}.\stockcaps{ERG} plum-[\stockcaps{ACC}] sand-\stockcaps{INST} cover-\stockcaps{PAST}\\

\z

--
\ea \label{bkm:Ref77508359}
\gll   Lante    yi     agoi-be       aentchina\\
girl      food     sand-with   cover{ }up\\
\citep[21]{hey_elementary_1903}
\z

\end{document}
