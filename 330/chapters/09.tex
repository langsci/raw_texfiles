\chapter{Grammars of Arrernte (1891--1938)}
\label{chap:key:9}

This chapter presents the six grammars of Arrernte written at, or emanating from, the Lutheran mission at Hermannsburg, west of Alice Springs. The linguistic work done by missionaries at the Hermannsburg mission was the culmination of a tradition of Lutheran description of South Australian languages. Western Arrernte, the Arandic variety spoken at the mission, was the fifth and last language to be grammatically described by Lutherans administered from Adelaide. The Arrernte grammars were produced over a similar time frame (1891--1923) as the grammars of Diyari (1868--1899) but are more diverse in structure.

\section{The HMS phase of Hermannsburg mission (1877--1891)}
\label{sec:key:9.1}\label{bkm:Ref514690108}
The Hermannsburg mission station was established on the Finke River west of Alice Springs in 1874 by the Immanuel Synod (ELIS), based in Langmeil (Tununda), South Australia, and the Hermannsburg Mission Society (HMS), in Lower Saxony, Germany. The first HMS trained missionaries sent to Adelaide in order to establish the mission were A. H. Kempe (1844--1928) and W. F. Schwarz (1842--1920). The arduous journey to the continent’s interior lasted two years, and the party rested for several months at Bethesda on the way. At the time of the missionaries' stopover, Bethesda mission was staffed by C. A. Meyer and the Vogelsang and Jacob families. Schoknecht, the last HMS-trained missionary to work at Bethesda, had already left and J. Flierl, the first Neuendettelsau missionary, had not yet arrived. Kempe and Schwarz reached Hermannsburg on the Finke River in 1877. They were joined by HMS missionary L. G. Schulz (1851--1924) the following year. These three men had entered the HMS seminary together in 1870.

Kempe felt their linguistic training was insufficient for the task of translation and described the difficulty:


\begin{quote}
[N]o one can imagine how difficult it is in the initial stages to reach the point where it is possible to proclaim even the basic truths of Christianity to the heathen - and this especially in view of the fact that no white people had ever been there before us and that we often had to drag every single word out of them. (Kempe, quoted in \citealt{scherer_joiners_1973}: 13)
\end{quote}

A school for Aboriginal children was opened in 1879. Mirroring the pattern of the production of printed material in Diyari by HMS missionaries at Bethesda, the first work in Arrernte produced at Hermannsburg was a primer for use in the school, \textit{Intalinja Nkenkalalbutjika Galtjeritjika} \citep{kempe_intalinja_1880}, published within three years of the mission’s establishment and a decade before the publication of an Arrernte grammar. The prompt publication of these primers at both inland missions reflects the focus of evangelistic effort on the children \citep[55]{harms_traume_2003}.

The Hermannsburg mission was abandoned close to the time of the publication of Kempe’s substantial grammar (\citeyear{kempe_grammar_1891}) in much the same circumstances in which Bethesda had been abandoned by the last HMS missionaries two decades earlier (\sectref{bkm:Ref456005923}). Lack of fresh food and water, illness and death again coincided with doctrinal disputes that threatened the continuity of the mission. Kempe was the last HMS missionary to leave after burying his wife beside their young son in 1891.

\subsection{The first Arrernte primer}
\label{sec:key:9.1.1}
\largerpage
As with the earliest materials produced at Bethesda by HMS missionaries, assessing different missionaries' contribution to the production of the first Arrernte primer is not straightforward. This twenty-one-page work, which was printed in Adelaide, is attributed to Kempe \citep[103]{graetz_open_1988}, who is usually credited with having done the linguistic work during the HMS phase of Hermannsburg mission. The primer was followed a decade later by the Christain instruction booklet \textit{Galtjintana-Pepa Kristianirberaka Mbontala} \citep{kempe_galtjintana-pepa_1891}, which was published in Hanover. Kempe is recognised as having collated and translated this material \citep[103]{graetz_open_1988}, although the MS held at LAA attributes the work to both Kempe and Schwarz. A collaboration seems likely since Schwarz is known to have “completed an exegesis of Isaiah in 1881” (J. \citealt{strehlow_tale_2011}: 368). The 160-page work contained Old and New Testament stories, Psalms, Luther’s Small Catechism, occasional prayers and 53 hymns.

The use of the term \textit{Pepa} in the title – a phonological approximation to “paper” – to translate “book'', had previously been used in the title of the first Neuendettelsau publication in Diyari \citep{flierl_christianieli_1880}. The root \textit{pepa} had earlier been incorporated into the Kaurna lexicon, as recorded by Teichelmann:

\ea
 Ninna          ngannaitya           yellarra   yakko      pepaitya      budni\\
\glt `why do you not come to school today'

\citep{teichelmann_dictionary_1857}

\gll niina           ngana-itya            yalarra   yaku        pepa-itya     pudn-i\\
         2\textsc{sg}.\stockcaps{NOM}     \stockcaps{INTER}-\stockcaps{PURP}     today      \stockcaps{NOT}       paper-\stockcaps{ALL}       come-\stockcaps{PAST}\footnotemark\\
\footnotetext{The suffix \textit{{}-itya} is reclaimed as marking the allative case in Kaurna when motion is towards a person (\citealt{amery_kulurdu_2013}:122), although it appears here marking motion towards a place.}
\z

Aboriginal languages frequently employ the same term to name an object and the material from which it is made \citep[117]{dixon_languages_1980}. The application of the word \textit{pepa} `paper' to refer to a book, or to school, is likely to have been independently coined by Kaurna, Diyari and Arrernte people who incorporated the English word “paper” into their languages.

\citet[156]{graetz_open_1988} translates the title of the Arrernte primer: \textit{Galtjintana-Pepa Kristianirberaka Mbontala} as “instruction book for the Christians on the wide open plains”. The first segment is

\ea
    Galtjintana-Pepa\\
\gll kaltye-         anthe-   nhe-        pepe\\
knowledge   give      \stockcaps{NOM}     paper\\
\glt      `book that gives knowledge'
\z

The noun, or adjective, \textit{kaltye} meaning `knowledgeable' is compounded with the stem of the verb \textit{anthe-} `to give', which is nominalised, with the suffix \textit{-nhe}: “used in names for some things that describe some action associated with them” (\citealt{henderson_eastern_1994}: 499). This noun appears to be compounded with \textit{pepa.} Regarding the next segment:

\ea
Kristianirberaka\\
\gll Kristian-   iRpeRa-   ke\\
Christian   \textsc{pl}             \stockcaps{DAT}/\stockcaps{POSS}\\
\z

\citet[4]{kempe_grammar_1891} described \textbf{{}-irbera} as marking plural number, although T. G. H. Strehlow (\citeyear{strehlow_aranda_1944}: 77~[1938]) described this plural suffix as “antiquated and archaic”. The plural noun “Christians” is then marked for dative/possessive with the suffix \textit{-ke.}

The next segment of the title \textbf{Mbonta} (\textit{Mpurnte}) is cognate with \textit{Mparntwe} in Eastern/Central dialects (Gavan Breen pers. comm., 10/09/2012) which is marked for locative case with the suffix \textit{-le}.\footnote{The use of the locative suffix here in natural Arrernte is unlikely. The sense would rather be expressed through affixation of associative suffix \textit{-arenye} \citep[202]{wilkins_mparntwe_1989} to \textit{Mpurnta}, deriving a lexeme meaning “people associated with/hailing from the wide open plains''.}  \textit{Mpurnta}, which occurs in the name given to the linguistic variety spoken in Alice Springs, “Mparntwe Arrernte”, is translated by \citet[42]{strehlow_no_1979} as `wide timbered plains'. The word probably refers to the type of habitat that occurs around major watercourses in which the settlements of Hermannsburg and Alice Springs are situated.

The title of this Arrernte primer (1891) is morphologically similar to part of the title of the second Diyari primer (\citealt{flierl_christianieli_1880}): \textit{Christianeli ngujangujara-pepa}, demonstrating a cross-mission linguistic influence. \textbf{ngujangujara} appears to be a reduplication of \textit{nyuyama}{}- `to know', which is compounded with \textit{pepa}, and placed alongside ``Christian'', although the function of what looks like an ergative suffix on “Christian” is uncertain.

Of the first Hermannsburg missionaries, Kempe was the most prolific. In addition to publishing the first grammatical description of Arrernte \citeyearpar{kempe_grammar_1891}, Kempe compiled a list of native plants and supplied samples to Ferdinand von Müller (1825--1896). He also wrote the first major survey of Central Australia and its people, published in the \textit{Hermannsburger Missionsblatt} \citep{kempe_report_1881}.

Kempe’s \citeyear{kempe_report_1881} publication gives the earliest published reference to the \textit{Altyerre} (ibid.: 55–56), \textit{contra} \citet{green_altyerre_2012}, who attributes the earliest written reference to the term to \citet{schulze_aborigines_1890}.\footnote{The term, which denotes a multifaceted abstract concept intrinsic to Arandic ontological belief, was chosen by the HMS missionaries to denote the Christian God. Interestingly, in Strehlow’s earliest grammar (\citeyear{strehlow_untitled_1931}: 19~[c.1907]) he gave \textbf{Altjira} as an example of a concrete noun (``Concrete Substantive'') along with \textbf{inkata} `master', \textbf{kwatja} `water' and \textbf{alarkna} `mud'!} Kempe’s record is significant to this present study because he spelled \textit{Altyerre} as \textbf{alxira}, representing the term’s lamino-palatal stop with the letter ``x''. This orthographic treatment of the phone shows that Kempe had read and assimilated the HMS missionary’s analysis of Diyari, in which “x” was used to represent the lamino-palatal stop (\sectref{bkm:Ref326948006}), which Kempe would have encountered during his sojourn at the Bethesda mission to the Diyari \textit{en route} to Hermannsburg.

While Kempe’s initial use of the letter “x” in 1881 and the structure of the title of the 1891 Christain instruction booklet show a cross-mission and cross-linguistic descriptive influence from Bethesda to Hermannsburg, the HMS missionaries' analysis of Arrernte otherwise shows little influence from the Lutherans' earlier analyses of Diyari, despite the considerable movement of staff between the two missions.

\subsection{Kempe’s grammar of Arrernte (\citeyear{kempe_grammar_1891})}
\label{sec:key:9.1.2}\label{bkm:Ref74837211}

After thirteen years at Hermannsburg, Kempe published the first grammar of Arrernte (\citeyear{kempe_grammar_1891}). The thirty-six-page work remained the most comprehensive published grammatical description of Arrernte until missionary C.~Strehlow’s son, T. G. H. Strehlow, wrote a Masters thesis submitted to the University of Adelaide in 1938 titled “An Aranda Grammar”. T. G. H. Strehlow’s grammar remains the most comprehensive published of Western Arrernte morphology and syntax.

Unlike the MS grammars of Diyari written in German during the HMS phase of mission at Bethesda (\citealt{koch_untitled_1868}; \citealt{Schoknecht1872}), Kempe’s grammar was translated into English, and was published in Australia. The essay, which was read to the Royal Society of South Australia in December 1890, was edited by R. Tate (1840--1901),  {Elder} Professor of {natural science} at the {University of Adelaide} and president of the Royal Society. The publication of the work in the Society’s \textit{Transactions} \citeyearpar{kempe_grammar_1891} was to a degree a collaboration with Tate, the journal’s editor. Since no earlier MS grammars of Arrernte have been located, the nature of Tate’s edit is unknown. Tate was a natural scientist, as were other early scholars -- such as such as W. Dawes, R. Brough Smyth and C. Chewings -- who investigated Australian Aboriginal people and languages before the emergence of departments of linguistics and anthropology in Australia.

As a geologist, Tate travelled through Central Australia in 1882 preparing geological and mineralogical reports. He visited Hermannsburg, after Kempe’s departure, as a member of the Horn Expedition, and contributed to the preparation of palaeontological, botanical and geological reports of the expedition.

Among the early grammars of Australian languages there is a tendency for the earliest descriptions of a language – \citet{threlkeld_australian_1834}, \citet{teichelmann_outlines_1840}, \citet{meyer_vocabulary_1843}, \citet{koch_untitled_1868}, and \citet{roth_ethnological_1897} – to be more morphologically wide-ranging and less formulaic than later works. Kempe’s inaugural grammar of Arrernte \citeyearpar{kempe_grammar_1891} provides a wealth of illustrative clauses that are socio-linguistically richer than those ubiquitous to missionary craft. The work also demonstrates a willingness to express linguistic complexities that were beyond Kempe’s descriptive capacity. Like Teichelmann \& Schürmann (\sectref{sec:key:5.6.3}), Kempe (\sectref{bkm:Ref339746464}) illustrates processes of clause subordination under the heading ``relative pronoun'', without providing an analysis of the structures.

\section{The Neuendettelsau phase of Hermannsburg mission (1894--1923)}
\label{sec:key:9.2}\label{bkm:Ref456881682}

In 1894 the Evangelical Lutheran Immanuel Synod purchased the Hermannsburg station in Central Australia, after which both inland South Australian missions were staffed by missionaries trained at Neuendettelsau. C.~Strehlow, who had been working at Bethesda for two years, was appointed as the new missionary to Hermannsburg. On his initial journey from Bethesda to Hermannsburg, Strehlow was accompanied by Reuther, with whom he was in the process of translating the New Testament into Diyari. Once stationed at Hermannsburg, Strehlow continued working on the final drafts of \textit{Testamenta Marra}, published in 1897.

Strehlow remained at Hermannsburg until his death in 1922, leaving Central Australia only three times during his twenty-eight years of service, once in 1903 during which time missionary N. Wettengel was left in charge. Like Strehlow, Neuendettelsau-trained Wettengel had worked at Bethesda (1896--1902) before being transferred to Hermannsburg, where he remained until being dismissed by the Immanuel Synod Mission Committee in 1906 (\tabref{tab:key:157}). Wettengel then returned to Germany, where he met the Berlin-based philologist W. Planert. Planert published a grammar of Arrernte (\citeyear{planert_australische_1907}; \sectref{bkm:Ref464826218}) and a grammar of Diyari (\citeyear{planert_australische_1908}; \sectref{bkm:Ref456334295}) based on information supplied to him by Wettengel.

\subsection{C.~Strehlow’s retranslation of Kempe’s materials}
\label{sec:key:9.2.1}\label{bkm:Ref326948956}

Two years after his arrival at Hermannsburg, C. \citet{strehlow_letter_1896} wrote to his brother-in-law C. Keysser (1877--1961) about the structure of Arrernte. He discussed the morphological negation on the verb (\citealt{wilkins_mparntwe_1989}: 235; \citealt{henderson_topics_2013}: 356) and marking of past tense.\footnote{D. Wilkins (pers. com. 26/08/2016) suggests that the Western Arrernte ``past tense'' inflection -\textbf{kala} \textit{-kele,} which is not documented in other Arandic varieties, is in fact \textbf{{}-ka} \textit{-ke} PAST found in Western Arrrernte and in other Arandic varieties, suffixed with subsequent inflection either marking the same subject \textit{-le} or with \textit{-rle} the relative clause marker (\sectref{sec:key:9.3.5}). See Examples \ref{bkm:Ref339739442} \& \ref{bkm:Ref339794050}}

\begin{quote}
	Learning the language is a huge amount of work; it is much more difficult and complicated than the Dieri language. One must, for example, observe that for every positive form there is also a negative form […]. For example, \textbf{Jinga lama} = I go, \textbf{jinga litjikana} = I do not go, \textbf{Jinga laka} or \textbf{lakala} – I went, \textbf{jinga litjina} = I'll go \textbf{jinga litji-gunia} I will not go. In the first two negative-forms the word not = \textbf{itja} is placed in the middle of the word = \textbf{l-itji-kana}, \textbf{l-itji-makana} (before certain consonants, especially for “tj", the a is turned into i). By contrast, in the negative future tense there is another word gunia that is placed at the end of the verb.
	
	I used to think, as Missionary Kempe writes in his grammar, that there is to be found only a single perfect-form in the Aranda language. But I think that I can now safely assume that there are at least 2 of these forms. I have had, that is, for example, \textbf{inakala}, but I have often heard the form \textbf{initjita}. I now believe that the form \textbf{inakala} is used for the past several days. Should this opinion prove to be true, then the entire Bible history should have to undergo a thorough revision.\footnote{“Die Sprache zu lernen, ist eine Riesenarbeit; sie ist viel schwieriger und verwickelter, als die Dieri-Sprache. So z.B. muß man zu jeder Positv-form auch eine Negativ-form […] merken, z.B. Jinga lama = ich gehe, jinga litjikana = ich gehe nicht, jinga laka od. lakala - ich ging, jinga litjina = ich werde gehen jinga litji-gunia ich werde nicht gehen. In den beiden ersten Negatio-formen ist das Wort nicht = itja mitten in das Wort hineingesetzt = l-itji-kana, l-itji-makana [das a wird vor gewissen Consonanten, besonders for “tj-“ in i verwandelt] in der Negativ-Futurform steht dagegen ein anderes Wort gunia am Schluß des Zeitworts. Früher dachte ich, wie auch Miss. Kempe in seiner Grammatik schreibt, daß nur eine Perfekt-Form in der Arandasprache sich findet. Doch glaube ich jetzt mit Sicherheit annehmen zu dürfen, daß es mindestens 2 solcher Formen gibt. Ich habe gehabt heißt z.B. inakala, doch habe ich dafür schon öfter die Form initjita gehört. Ich glaube nun, daß die Firm inakala für die Vergangenheit mehrerer Tage gebraucht wird. Sollte sich diese Meinung als wahr herausstellen, so müßte die ganze bibl. Geschichte einer gründlichen Revision unterzogen werden.” \citep{strehlow_letter_1896}}
\end{quote}

In addition to showing Strehlow’s engagement with the morphology and morphophonology of Arrernte, this passage shows that reasonably soon after arrival, Strehlow contemplated retranslating the HMS missionaries' work.

By {1904} C.~Strehlow had prepared a revision and expansion of Kempe and Schwarz’s Christian instruction book \citeyearpar{kempe_galtjintana-pepa_1891}. Strehlow’s 264-page work \citeyearpar{strehlow_galtjindinjamea-pape_1904} included the second of five translations of Luther’s Catechism into Arrernte, prayers, and an enlarged hymnal with 100 hymns \citep[104]{graetz_open_1988}. Strehlow became adept at employing morphological processes internal to the language’s structure to coin new words expressing esoteric Christian concepts. His son, T. G. H. Strehlow, described his father’s rewriting of his predecessor’s work as a process of eliminating Latin terms \citeyearpar[42]{strehlow_no_1979}.

Ironically the missionaries' pronunciation of Arrernte was so poor that the ingenious derivations upon which they coined new terms were sometimes lost on the Arrernte, who imitated the European’s pronunciation. The word \textbf{lunaluna}, for example, was coined to translate “redeemer'', derived from the verb root –\textit{irlwe}, `to let go, to untie', through a process of nominalisation marked with the suffix \textit{-nhe} and reduplication, producing a noun that refers to an entity habitually involved in the action of the verb (see \citealt{wilkins_mparntwe_1989}: 139--141). Of this created lexeme, T. G. H. Strehlow (\citeyear{strehlow_aranda_1944}: 363~[1938]) wrote: “[T]he natives copy the white missionaries' mispronunciation of the two original n sounds of this word!” That is, their speech failed to differentiate a phonemic interdental and alveolar nasals:

\ea            lunaluna = redeemer \\
(T. G. H. \citet{strehlow_aranda_1944}: 363~[1938]) \\
\gll             irlwe- nhe- irlwe -nhe\\
           {to loosen, untie}- \stockcaps{NMSR}- \textsc{hab} -\textsc{rdp}\\
\z

C.~Strehlow’s retranslations of the HMS missionaries' works caused tension at the mission. Missionary Wettengel refused to use Strehlow’s retranslations on theological grounds, and detailed the aspects of the texts to which he objected in correspondence with the mission board. Strehlow was subsequently forced to defend each translational change against charges of teaching false doctrine, maintaining that he tried to make as few changes as possible to Kempe’s manual but aimed to:

\begin{quote}
 [s]ubstitute Aranda words for terms which were imported from the Latin, Greek, English and German languages and only leave those foreign terms we also use in German. \citep{strehlow_galtjindinjamea-pape_1904}\footnote{“Die aus der lateinischen, griechischen, englischen u. deutschen Sprache herübergenommenen Wörter durch Aranda-Ausdrücke zu ersetzen u. nur die fremdsprachlichen Wörter stehen zu lassen, die wir auch im Deutschen gebrauchen.''}
\end{quote}

Strehlow’s retranslations of Kempe’s Arrernte catechism and the commandments (\citeyear{strehlow_1904b}) play an important role within the historiography of the description of ergativity, because it was Wettengel’s refusal to teach from Strehlow’s 1904 translations that ultimately resulted in Wettengel’s dismissal, and his return to Germany, where he subsequently met W. Planert. Planert’s published grammars of Arrernte (\citeyear{planert_australische_1907}; \sectref{bkm:Ref464826218}) and of Diyari (\citeyear{planert_australische_1908}; \sectref{bkm:Ref340578477}) based on missionary Wettengel’s materials, gave very early global usages of the terms “ergative” and “absolutive” to name the syntactic cases, and conceived of split ergative systems in a way that was not repeated in Australia until the modern descriptive era (\sectref{sec:key:2.6.1}).

\subsection{C.~Strehlow’s German editor, Moritz von Leonhardi}
\label{sec:key:9.2.2}\label{bkm:Ref456272403}\label{bkm:Ref74834598}

C.~Strehlow (1871--1922) is best known for his seven-volume German work \textit{Die Aranda- und Loritja-Stämme in Zentral-Australien}, which details Arrernte kinship, totemism, social life and religion (1907--1920; see \citealt{kenny_arandas_2013}). Regular communication with his German editor, M. von Leonhardi (1856--1910), an armchair anthropologist and private scholar, during the course of its production placed Strehlow at the epicentre of early twentieth-century European thinking, while conducting his research at Hermannsburg. Through correspondence between Hermannsburg and Germany, Strehlow was kept informed of the reception in Europe of Australian and European publications describing Australian Aboriginal languages and culture, which Leonhardi sent to Strehlow. The pair appraised a range of material with an eye to the relative value to their own pending publications. When assessing Basedow’s Arrernte vocabulary (1908) published in Germany, von Leonhardi demonstrates how conversant the pair was with available material describing Australian languages:

\begin{quote}
I had already anticipated that you would not be satisfied by Basedow’s work. Our periodicals always accept such works; since – with very, very few exceptions – we have no other vocabularies. The vocabularies in the 3-volume work by Curr are hardly any better and, yet, we still have to work with them. And that is a great shame. With regard to phonetics, there are no correctly recorded Austr[alian] Languages whatsoever to be found in the literature, even the works of Threlkeld, Günther, Meyer are inadequate. \citep{leonhardi_letter_1908-2}
\end{quote}

In the year that \citeapo{mathews_languages_1907} grammar of Arrernte was published, von Leonhardi assessed Mathews' scholarly contribution unfavourably:

\begin{quote}
That Mr RH. Matthews (\textit{sic}) in Paramatta churns out a terrible lot of writing and is an awful muddle-head who does not just flood Australian and American journals with his essays, but also French, Austrian and German ones. \citep{leonhardi_letter_1907}
\end{quote}

Leonhardi’s correspondence indicates that Strehlow’s seven-volume publication was intended to contain more linguistic material than eventuated. Leonhardi often refers to the importance of the inclusion of comparative linguistic material, writing \citeyearpar{leonhardi_letter_1908}, for instance, “One could have the linguistic work published at later stage – possibly separately –, but in my view it is very important that a good grammar and dictionary of Aranda, Loritja and Dieri comes out together.”

\subsection{C.~Strehlow’s Arrernte grammars}
\label{sec:key:9.2.3}

At least three different grammatical descriptions of Arrernte written by Strehlow are known to survive (\citeyear{strehlow_untitled_1931} \& \citeyear{strehlow_grammatik_1931}~[c.1907]; \citeyear{strehlow_einige_1908}; \citeyear{strehlow_notitle_1910}). All are written in German, and one was published \citeyearpar{strehlow_einige_1908}. The circumstances of their production are considered in the following sections.

\subsubsection{C. \citet{strehlow_untitled_1931} \& \citeyearpar{strehlow_grammatik_1931} [c.1907]}
\label{sec:key:9.2.3.1}\label{bkm:Ref456272021}

One of Strehlow’s three analyses of Arrernte survives as two very similar but non-identical copies of a lost original. The date of the original lost document from which the copies were made is not known. It is, however, probable that the lost original was Strehlow’s earliest analysis of Arrernte, since the copies resemble Kempe’s publication (\citeyear{kempe_grammar_1891}) much more closely than do either Strehlow’s \citeyear{strehlow_einige_1908} published grammar, or his \citeyear{strehlow_notitle_1910} MS grammar. In this earliest work Strehlow expresses surprise that in Arrernte there is no third-person pronominal gender distinction.

\begin{quote}
Strangely gender of the personal pronouns cannot even be seen in the third-person. era serves to indicate he, she and it. Era pitjima as well as meaning he comes also means “she comes” and “it comes''. (\citealt[30--31]{strehlow_grammatik_1931}~[c. 1907])
\end{quote}

It is likely that this observation was made by Strehlow soon after encountering Arrernte, since he was aware that Diyari \textit{did} make this distinction.

The original analysis, from which the copies were made, is referred to here as [c.1907], the work’s latest likely date, although it is possible that this earliest grammar by C.~Strehlow is least a decade older, since Strehlow arrived at the mission in 1894.

The grammar makes some comparative phonological and grammatical study of Arrernte, Diyari and Ramindjeri, the latter referred to as the \textit{Encounter Bay Sprache}, for which \citet{meyer_vocabulary_1843} is acknowledged as the source. Comparative case paradigms for nouns and pronouns are given in each number for the three languages.

One copy of the lost original ``Grammar of the Aranda language'' was made by the self-appointed missionary E. Kramer (1889--1958) and is dated as completed 12 May 1931. The notebook is held by the South Australian Museum. Strehlow is \textit{not} named as the original author. Through comparative study of the MS with other Arrernte grammars (\citealt{Stockigt2017}; §9.3.5), the analysis is attributed here to C.~Strehlow and is referred to as \citet{strehlow_untitled_1931}~[c.1907].

1931, the year Kramer completed copying Strehlow’s grammar, was also the year in which T. G. H. Strehlow, C.~Strehlow’s son, made his first return journey to Alice Springs from Adelaide. Kramer provided Strehlow with important support on this initial field trip, supplying him with camels \citep[150]{hill_broken_2002}. Having completed his honours degree in English at the University of Adelaide, T. G. H. Strehlow was encouraged by his classics professor J. A. FitzHerbert to apply to the Australian National Research Council to make a comparative survey of dialects of Arrernte. 1931 was also the year that T. G. H. Strehlow’s mother, Frieda Strehlow (1875--1957), returned to Germany, at which stage C.~Strehlow’s MSS came into T. G. H. Strehlow’s possession (J. Strehlow pers. comm., 10/08/2013). In 1934, when lecturing in Old and Middle English, \citet{strehlow_letter_1934} applied to have the title of his approved Masters thesis altered from “The Elizabethan conception of tragedy” to “Primitive elements in Old Icelandic Mythology and in Old Heroic verse, in the light of Aranda myths and legends''.

Kramer’s copy of C.~Strehlow’s grammar refers to T. G. H. Strehlow, in a note \textit{am Rand}, “in the margin” (\figref{fig:key:9-187}). It appears that Kramer copied the grammar from a document on which T. G. H. Strehlow had added additional notes to the margins. It is possible T. G. H. Strehlow lent Kramer C.~Strehlow’s now lost original grammar, on which he had made notes while familiarising himself with his father’s grammatical analysis at the commencement of his own academic study of the language.

\begin{figure}
\includegraphics[width=.8\textwidth]{figures/stockigt-img136.png}
\caption{E. Kramer’s copy of C.~Strehlow's grammar (\citeyear{strehlow_untitled_1931} [c.1907: 36]) showing reference to “Th. Str.''}
\label{bkm:Ref456020911}\label{fig:key:9-187}
\end{figure}

This notebook into which Kramer copied Strehlow’s grammar also contains a second Arrernte grammar: “An abbreviated grammar according to Pastor Riedel”, which Kramer completed copying the following day (13/05/1931). \footnote{The South Australian Museum pagination of the two grammars, which is used here, is not straightforward. The Riedel grammar starts on page 87 and runs on double page spreads until the end of the notebook. From there it continues on the left hand side only of double page spreads at the beginning of the notebook (pp. 6, 8, 10) with the Strehlow grammar appearing on the right hand page. Kramer’s copy of C.~Strehlow’s work is on pages 3, 5, 7, 9, 11--86.} J. Riedel (1885--1965), who studied at Neuendettelsau (1904--08), was chairman of the Finke River mission board (1926--1950) and served at Hermannsburg for six months after C.~Strehlow’s death in 1923. His brother W. Riedel was missionary at Bethesda between 1908 and 1914.

A second mimeographed copy of C.~Strehlow’s earliest analysis titled \textit{Die Grammatik der Aranda-Sprache} was made by an unknown typist. It is nearly identical to the copy made by Kramer but contains some additional material. The sections that are contained in this copy, but which are absent in the Kramer copy, mostly describe Diyari and Ramindjeri. This mimeograph document also makes more frequent reference to “Th. Str.” (Theodore Strehlow). Again, this material is noted as \textit{Am Rand} (in the margin) where equivalent Luritja is also supplied. The work is held at the Lutheran Archives, where a previous archivist has attributed it to missionary Wettengel. The analysis bears no resemblance to Wettengel’s grammatical description of Arrernte, which is contained in Planert (\citeyear{planert_australische_1907}; \sectref{bkm:Ref464826218}). While it is difficult to disregard any possible unrecorded factors upon which the previous archivist attributed the grammar to Wettengel, the work is here attributed to C.~Strehlow (\citeyear{strehlow_grammatik_1931} [c.1907]).

\subsubsection{C.~\citet{strehlow_einige_1908} and \citet{planert_australische_1907}}
\label{sec:key:9.2.3.2}\label{bkm:Ref464826218}

In 1907 W. Planert published “Australische Forschungen I. Aranda-Grammatik” in the prestigious Berlin-based German ethnological journal \textit{Die Zeitschrift für Ethnologie.} Material for the publication was furnished by missionary N. Wettengel, who had returned to Germany the previous year, having been dismissed by the mission committee because he refused to teach from Strehlow’s retranslation (\citeyear{strehlow_letter_1904}) of Kempe’s catechism (\citeyear{kempe_grammar_1891}).

In his introduction Planert acknowledged Wettengel as having informed the work, and he outlined the limitations of his source:

\begin{quote}
Herr Wettengel has lived many years in Central Australia and has familiarised himself with both languages to the extent that he could preach in them fluently. It is therefore hoped that his statements are for the most part correct, although a lack of previous education leaves something to be desired. I have eliminated errors, as far as possible, and re-worked the grammar according to linguistic principles. Since Herr Wettengel only remained in Berlin for two weeks, my results will not of course properly convey the grammars of Aranda and Dieri. \citep[551]{planert_australische_1907}\footnote{``Hr. Wettengel hat mehrere Jahre in Zentralaustralien gelebt und sich mit den beiden Sprachen soweit vertraut gemacht, dass er darin geläufig predigen konnte. Es its daher zu hoffen, dass seine Angaben im wesentlichen richtig sind, obschon der Mangel an Vorbildung in ihm mancherlei irrige Anschauungen über sprachliche Erscheinungen entstehen liess. Ich habe die Fehler nach Möglichkeit ausgemerzt und die Grammatik nach sprachwissenschaftlichen Prinzipien ausgearbeitet. Da Hr. W. sich nur zwei Wochen in Berlin aufgehalten hat, so sind natürlich meine erzielten Resultate nicht dazu angetan, die Grammatik der Aranda und Dieri zu erschöpfen.''}
\end{quote}

Shortly after the publication of the work, von Leonhardi wrote to Strehlow:

\begin{quote}
I suspected that you would not be impressed by the Wettengel-Planert grammar. It appeared immediately thin to me. The same is likely to apply to the Dieri grammar…. At the moment I do not want to make any comment on your judgment of the Planert Aranda grammar, which you would like to publish as a response. However, protest in one or another form will be voiced. (\citealt{leonhardi_letter_1908}; translation by \citealt[348]{kenny_missionary_2008})
\end{quote}


The publication of Planert’s grammar of Arrernte sparked a response publication from Strehlow, “Einige Bemerkungen über die von Dr. Planert auf Grund der Forschungen des Missionars Wettengel veröffentlichte Aranda-Grammatik” (Some remarks on the grammar published by Dr. Planert based on the research of missionary Wettengel; \citeyear{strehlow_einige_1908}). Strehlow’s response appeared in the same journal as Planert’s grammar of Diyari \citeyearpar{planert_australische_1908}. In the opening passage, Strehlow stated:

\begin{quote}
In the year 1890 missionary H. Kempe published a grammar and a dictionary of Wonkaranda, which may still, despite its many mistakes, be regarded as a good work. When after some 16 years Dr. Planert undertakes to write a new grammar of Aranda – based on material supplied by missionary Wettengel, one might expect it to be an improvement on the earlier work…Unfortunately one is in this regard mistaken…the Planert-Wettengel grammar is the poorer and more meagre of the two. \citep[698]{strehlow_einige_1908}\footnote{“Im Jahre 1890 veröffentlichte Missionar H. Kempe eine Grammatik und ein Wörterbuch der Wonkaranda,[…] eine Arbeit, die trotz mancher Fehler doch als eine gute bezeichnet werden kann. Wenn nach etwa 16 Jahren Dr. Planert auf Grund ihm vom Missionar Wettengel gelieferten Materials es unternommen hat eine neue Aranda-Grammatik zu schreiben, […] so könnte man erwarten, dass dieselbe gegen die frühere Arbeit einen Fortschritt bedeute, […]. Leider sieht man sich in diesen Erwartungen getäuscht  […]  die Planert-Wettengelsche Grammatik [ist] dürftiger und unvollständiger, als die Kempesche.”}
\end{quote}

As suggested by the title, the work is not a complete grammar. Strehlow explained:

\begin{quote}
It is beyond my current intention to attempt to correct all the mistaken forms and words contained in the new Aranda grammar; I here limit myself to pointing out a few of its more significant mistakes, while hoping in time to publish myself a comprehensive work on the Aranda language.\citep[698]{strehlow_einige_1908}\footnote{“Es kann nicht in meiner Absicht liegen und würde zu weit führen, wollte ich hier den Versuch machen, alle unrichtigen Formen und Wörter der neue Aranda-Grammatik zu berichtigen; ich beschränke mich daher darauf, auf einige gröbere Fehler hinzuweisen und hoffe in einiger Zeit selbst eine zusammenhängende Arbeit über Aranda-Sprache veröffentlichen zu können.”}
\end{quote}

Despite Strehlow’s hope, the work was the most extensive grammatical material that he published.

\subsubsection{C. \citet{strehlow_notitle_1910}}
\label{sec:key:9.2.3.3}\label{bkm:Ref335827573}

A MS comparative grammar of Arrernte and Luritja dated 1910, written by C. Strehlow, is held at the Strehlow Research Centre (Alice Springs, Northern Territory). It is the only early description of a language belonging to the Western Desert family of PN languages, many of which are still spoken today. It and Symmons' grammar of Nyungar \citeyearpar{symmons_grammatical_1841} are the only early grammars of languages spoken in Western Australia. It is, however, a rather superficial analysis. It does not, for instance, suggest the existence of classes of verbs which are found in Western Desert languages, but which are absent in Arrernte and in Diyari, with which C.~Strehlow was much more familiar. The work gives little discussion, and hardly any German translation of the Luritja or Arrernte material. The work presents Arrernte material on the left of the page and the parallel Luritja material on the right. The same comparative format had previously been employed by \citet{flierl_dieri_1880} and by \citet{hale_languages_1846}.

Strehlow probably wrote this grammar for his own evangelistic purpose as an aid for the preparation of sermons into Luritja. This Western Desert language spoken to the west of Hermannsburg came into the sphere of mission activity as people migrated east towards the mission and towards centres of European industry. \textit{Galtindjinjamea-Pepa} (\citeyear{Strehlow1924}), compiled by C.~Strehlow and published posthumously, contained an appendix with a Catechism and hymns in Luritja, which were translated by teacher A. Heinrich and M. Tjalkabota (Moses; 1869--1964), the Arrernte ``blind evangelist''.

In spite of the existence of these MSS grammars by C.~Strehlow (\citeyear{strehlow_untitled_1931} [c.1907]; \citeyear{strehlow_grammatik_1931} [c.1907]; \citeyear{strehlow_notitle_1910}), which describe not only Arrernte but Diyari, Ramindjeri and Luritja, a recent major biography of Strehlow written by his grandson, J. Strehlow, asserts (p. 1064) that C.~Strehlow “seems not to have compiled a Dieri grammar at all, nor to have improved on the sketchy Loritja grammar Mathews published” \citep[1064]{strehlow_tale_2011}. This oversight is characteristic of the tendency, observed of historical accounts of Threlkeld \citep[108]{roberts_language_2008}, for historians to “gloss over” their subjects' linguistic achievements.

\section{Lutheran traditions of Arandic description 1891--1938}
\label{sec:key:9.3}\label{bkm:Ref326948586}

Kempe's grammar of Arrernte, made without recourse to earlier work and produced after thirteen years of mission work, has to an extent been overshadowed by Strehlowian renown. It has not previously been recognised that C.~Strehlow’s earliest grammar of Arrernte (\citeyear{strehlow_untitled_1931} [c.1907]) reproduced much of Kempe’s work. Regarding the choice of the term “postposition” rather than the more conventional term “preposition'', for example, \citet[4]{kempe_grammar_1891} had written: “Concerning Prepositions, there are none in the language. The relative word always stands behind the noun as either a bound or stand-alone form. For this reason the word-class would better be termed `post-position’.” C.~Strehlow (\citeyear{strehlow_untitled_1931}: 17) wrote: “There are no prepositions in the Aranda language. The relationship word always stands behind the noun, whether it is attached, or is a stand-alone word. For this reason, the word-class would better be called `postposition’\thinspace”.\footnote{“Praepositionen gibt es in der Aranda Sprache nicht. Das verhältnis Wort steht immer hinter dem nomen, entweder verbunden mit ihm, oder als selbständiges Wort, deshalb wird diese Wortklasse besser mit den Wort `Postposition' bezeichnet.”} In the spirit of collaborative research made towards the common aim of proselytisation, grammars written by Neuendettelsau men at Hermannsburg reproduced the analysis given by earlier Hermannsburg missionaries, as had previously occurred in grammars of Diyari written at Bethesda.

Moreover, \citet{kempe_grammar_1891} is not recognised by linguistic historians as having presented the earliest description of kin-dyadics (\sectref{bkm:Ref463530241}) or the category of associated motion (\sectref{bkm:Ref456085056}).

\subsection{{}-\textit{nhenge} ``kin-{dyadic}''}
\label{sec:key:9.3.1}\label{bkm:Ref463530241}

Many PN languages mark pronouns and other terms with a dyadic suffix showing a reciprocal relationship between the referents \citep{MerlanHeath1982}. In Arrernte members of these pairs are marked with \textit{{}-nhenge}. Wilkins describes the structure in Mparntwe Arrernte:

\begin{quote}

 {}-\textit{nhenge} “kin-dyadic” attaches to certain kin terms…to form a lexeme which refers to a group of people (usually only two) who are related to each other in such a way that one member of the group would call the other member of the group by the kin term which is the root of the formation. \citep[136]{wilkins_mparntwe_1989}
\end{quote}

\citet[3]{kempe_grammar_1891} described a “particular form of the dual, which is only used personally by annexing the particle \textbf{nanga}, as wora, `the boy'; \textbf{worananga}, `the two boys’\thinspace”. Following Kempe, C.~Strehlow wrote:

\begin{quote}
In instances in which the people stand in a close and friendly relationship with one another, the dual is commonly expressed with the ending \textbf{nanga}, especially in sentences which are questions. For example, \textbf{kwarananga nta\-na}? Where are both the girls? (literally girls both where?) \textbf{tjinananga}, the two friends. \citep[25]{strehlow_untitled_1931}\footnote{“Bei Personen, die in einem näheren Freundschaftsverhältnis zu einander stehen, wird der Dual häufig besonders in Fragesätzen durch die Endung “\textbf{nanga}” ausgedrückt zB: \textbf{kwaranga ntana}? Wo sind die beiden Maedchen? (Woertlich: MAEDCHEN BEIDE WO?) Tjinananga = die beiden Freunde.”}
\end{quote}

T. G. H. Strehlow’s (\citeyear{strehlow_aranda_1944}: 61[1938]) later account of the suffix did not explain that the suffix marked a reciprocal relationship as effectively as had his predecessors' descriptions. He firstly described \textbf{naŋa} (\textit{nhenge}) as alternative dual suffix, proposing that the form was the original dual suffix that has been replaced by \textbf{tara} (\textit{therre}). He also exemplified the suffix in a list of items given under the heading ``collective nouns'': \textbf{tjóanaŋa} `twins', \textbf{mánaŋa} `mother and baby', \textbf{njínaŋa} `father(s) and son(s)' and \textbf{kąnaŋa} `two brothers'.

In an overview of the category in Australian languages, Evans overlooks the pre-contemporary descriptions of the category:

\begin{quote}
To our knowledge the first discussion of the term “dyadic” in the sense used here … was given by \citet{MerlanHeath1982}, though other Australianist scholars had discussed the same or related phenomena under other names, such as “kinship proprietives” (e.g. \citealt{breen_gugadj_1976}; \citealt{blake_pitta_1979}), “kinship duals” \citep[234--5]{dixon_dyirbal_1972}, “reciprocal plurals” \citep[104--5]{donaldson_ngiyambaa_1980}, “collective nouns” (\citealt{hercus_perception_1973}) and “kinship pairs” \citep{hercus_bagandji_1982}. \citep[2]{evans_interesting_2003}
\end{quote}

The oversight of the pre-contemporary descriptions of the structure is characteristic of a discontinuity in the tradition of Australian grammatical description. Structures are described in the modern descriptive era as if for the first time, without recognition of the descriptive breakthroughs that had been made earlier.

\subsection{Verb morphology and the category of associated motion}
\label{sec:key:9.3.2}\label{bkm:Ref456085056}

Kempe’s grammar of Arrernte delved boldly into the complexity of verbal morphology in a way that is uncharacteristic of the corpus. Allusions to the complexity of verb structure are widespread within the corpus works (\citealt{teichelmann_outlines_1840}: 13; \citealt{schurmann_letter_1844}: 16; \citeyear{schurmann_aboriginal_1846}: 30; \citealt{ridley_kamilaroi_1855}: 76; \citeyear{ridley_kamilaroi_1866}: 63), and Ray, in his 1925 overview of Australian languages observed that verbal morphology remained under-investigated \citep[6]{ray_aboriginal_1925}. Grammarians frequently qualify this area of their description as incomplete, conveying that they are aware of a much greater complexity than their presentation attempted to account for. \citet[25]{kempe_grammar_1891} wrote “[n]ow there are many forms, which have been considered in the foregoing chapters, which can be brought into all the moods and voices… there are 30 at the least which can be thus regularly conjugated, that means, from one mood can be made 9,000 different phrases.”

Kempe assigned the tables of verb conjugation, which are among the most comprehensive materials relating to the verb considered in this study, to a ten-page appendix \citep[26--36]{kempe_grammar_1891} and instead devoted the section describing verbs given in the main section of the grammar (ibid.: 17--24) to the description of inflection and derivation of a more complex nature. Chapter XIII (ibid.: 23--24) “Sundry moods and forms …” presents subordinating morphology, including a discussion of the translation of the counterfactual construction from John 11: 21, “if thou hadst come here my brother would not have died''. The detail given here surpasses that provided in the grammars of Diyari written over three decades of mission work at Bethesda.

Arrernte is among a group of languages spoken in Central Australia with verbs specifying “that the verb action is associated in some way with a motion event” \citep[270]{wilkins_mparntwe_1989}. The morphological system was first described as a “category of associated motion” by \citet[23]{koch_category_1984} for the Arandic language Kaytetye. 

Wilkins' discussion (\citeyear{wilkins_mparntwe_1989}: 270--298) of associated motion in Mparntwe Arrernte appraises T. G. H. Strehlow’s account of some associated motion forms, which were listed by Strehlow in a discussion of “periphrastic verbs” (\citeyear{strehlow_aranda_1944}: 171--174~[1938]). Wilkins writes:

\begin{quote}
Strehlow himself never separates out the individual morphemes nor indicates in what manner the verb form is derived. Indeed, … he lists reduplicated aspectual forms as well as combining reduplicated aspectual forms and associated motion forms. \citep[273--274]{wilkins_mparntwe_1989}
\end{quote}

Some of the forms Strehlow gave had first been described by \citet[19--22]{kempe_grammar_1891}, and subsequently by \citet[334]{mathews_languages_1907}. The morphemes controlling different categories of verbal morphology were not identified. Kempe for instance, included Example (\ref{bkm:Ref340586104}), in which the verb root \textit{-twe} “to hit” is marked with the morphological compound -\textit{ety=alpe-,} a category of associated motion indicating that the “subject returns to a place and then performs the action described by the verb stem” \citep[241]{henderson_topics_2013}, before word final inflection for present tense.

\newpage
\ea
   \label{bkm:Ref340586104}tutyalbuma\\
return to beat\\

(\citealt{kempe_grammar_1891}: 20; \citealt{mathews_languages_1907}: 334)

\gll Twe-ty=alpe-me\\
hit-\stockcaps{RETURN}\&\stockcaps{DO}-\stockcaps{PRES}\\
\z

Kempe also supplied Example (\ref{bkm:Ref340586632}) showing inflection for continuous aspect, which fills the slot in the Arrernte verb directly after the marking of associated motion \citep[276]{henderson_topics_2013}. In this example the verb is inflected with the morpheme complex –\textit{rle=pe-} : `do continuously while moving along', indicating “an action which is done repeatedly or continuously while moving along” (\citealt{henderson_topics_2013}: 248; \citealt{wilkins_mparntwe_1989}: 252).

\ea
            \label{bkm:Ref340586632}Tulabuma\\
\glt `beating, by walking about'

\citep[19]{kempe_grammar_1891}

 \gll Twe–rle=pe-me\\
                    hit-\stockcaps{CONT}\&\stockcaps{MOT}-\stockcaps{PRES}\\
\z


\citet[20]{kempe_grammar_1891} also recognised the associated motion forms: \textit{ty=antye} DO UPWARDS and \textit{tye=kerle} DO DOWNWARDS \citep[272--273]{wilkins_mparntwe_1989}. He described \textit{ty=antye} DO UPWARDS as \textbf{injama} `to rise or ascend' (Example \ref{bkm:Ref340587267}) and recognised that verbs containing this form implied motion upwards. While his analysis of the constituents of \textit{tye=kerle} DO DOWNWARDS, \textbf{ikalama} (Example \ref{bkm:Ref340587421}) was flawed, he demonstrated how both forms altered the meaning of verbs with which they co-occurred. Both illustrative examples convey the time of the day implied by the motion of the sun:

\ea
\label{bkm:Ref340587267}Ilkutjinjama\\
\glt `to eat in the morning, to breakfast'

\citep[20]{kempe_grammar_1891}

\gll irlkwe -ty=antye -me\\
eat -{\stockcaps{DO} \stockcaps{UPWARDS}} -\stockcaps{PRES}\\

\glt `to eat while the sun rises'
\z

\ea
\label{bkm:Ref340587421}Ilkutjikalama\\
\glt `to eat in the evening, to sup'

\citep[20]{kempe_grammar_1891}

\gll irlkwe -tye=kerle -me\\
eat -{\stockcaps{DO} \stockcaps{DOWNWARDS}} -\stockcaps{PRES}\\
\glt `to eat while the sun goes down'
\z

Like T. G. H. Strehlow, Kempe interspersed the verbs inflected for associated motion with verbs exemplifying other functions marked by aspect or reduplication.

\citet[334]{mathews_languages_1907} listed forms illustrating that the Arrernte verb showed “repetition or continuance of the act described, and many complexities, which must only briefly be mentioned in this article” (\figref{fig:key:9-188}). In keeping with his propensity to assert that his analysis was the first to present Australian linguistic structures, he likened the Arrernte forms to verbs in “Kamilaroi, Wiradjuri, Thurrawal and other Australian tongues, the grammars of which have been published by me.”

Each of the forms Mathews listed (\figref{fig:key:9-188}) were picked from Kempe’s four-page analysis (\citeyear{kempe_grammar_1891}: 20--23). Mathews, however, failed to appreciate that some of the examples he picked from Kempe’s grammar described a category that was not shared by any of the languages he mentioned.

\begin{figure}
\includegraphics[width=.8\textwidth]{figures/stockigt-img137.png}
\caption{Mathews' exemplification of verbal complexity \citeyearpar[334]{mathews_arranda_1907}}
\label{bkm:Ref456021260}\label{fig:key:9-188}
\end{figure}

\subsection{Early representations of Arandic case}
\label{sec:key:9.3.3}

\subsubsection{Prepositions}
\label{sec:key:9.3.3.1}\label{bkm:Ref331424200}

Rather than presenting the word-class “preposition” in its conventional position towards the back of the grammar, Kempe discussed “prepositions” (\citeyear[4--5]{kempe_grammar_1891}) straight after ``substantive case'', and before “adjectives” and “pronouns''. This placement, which had previously been employed by Moorhouse (\citeyear[2]{moorhouse_vocabulary_1846}; \sectref{bkm:Ref333854849}), reflects an awareness that some members of this class performed the same grammatical function as suffixes deemed to mark case. He explained that the term ``postposition'' was more appropriate than ``preposition'', since the particles were placed at the end of a word, but nevertheless maintained the term ``preposition''.

Kempe (ibid.: 4) divided ``prepositions'' into two classes: “one consisting of separate words” (\tabref{tab:key:9-189}) and the other “consisting only of small particles annexed to the substantives, to which they belong as suffixes” (\tabref{tab:key:9-190}). This division had been established by Teichelmann \& Schürmann (\citeyear{teichelmann_outlines_1840}; \sectref{sec:key:5.3.2}) and was followed by Lutheran missionary-grammarians describing Diyari (\sectref{bkm:Ref456339501}). Following \citet[4--6]{kempe_grammar_1891}, the division was maintained in later descriptions of Arrernte, although the classes were assigned different labels, by \citet[335]{mathews_languages_1907} and by T. G. H. Strehlow (\citeyear{strehlow_aranda_1944}: 199--207~[1938]). This division of pre/postpositions into two functionally distinct classes is unique to the South Australian sub-corpus of description.

Kempe’s first class (\tabref{tab:key:9-189}) corresponds to Teichelmann \& Schürmann’s “postpositions” (\tabref{tab:5:82}). The class includes what are mostly locational words and spatial adverbs. The forms given by Kempe are mostly the same as those given by T. G. H. Strehlow as ``postpositional suffixes''.

Kempe’s second class (\tabref{tab:key:9-190}), which corresponds to Teichelmann \& Schürmann’s “postfixa” (\tabref{tab:5:82}), are termed “postpositional suffixes” by T. G. H. Strehlow. These are the peripheral case suffixes that were not included in the conservative Arrernte case paradigms, marking cases that are now called allative, instrumental/comitative, locative, proprietive and AFTER. They attach to the unmarked stem of nouns and to the dative stem of pronouns or, in the words of T. G. H. Strehlow (\citeyear{strehlow_aranda_1944}: 199~[1938]), “to the nominative cases of nouns and adjectives, and to the possessive cases of personal pronouns.”


\begin{table}
\begin{tabularx}{\textwidth}{@{}l@{}lQQQQ}
\lsptoprule
 & \textbf{Form}  & \textbf{Original translation}  & \textbf{Form}  & \textbf{Currently \mbox{described as:}}  & \textbf{Source}\\
\midrule
\multicolumn{6}{l}{“governed by the ablative case''}\\
\midrule
& \textbf{ulara}  & in front, against & \textit{lwarre}  & in front, front & \citet{breen_introductory_2000}\\
\tablevspace
& \textbf{topala}  & behind & \textit{irtepe-le}  & back-\stockcaps{LOC} & \citet{breen_introductory_2000}\\
\tablevspace
& \textbf{gatala}  & outside & \textit{kethe-le}  & outside-\stockcaps{LOC} & \citet{breen_introductory_2000}\\
\tablevspace
& \textbf{mbobula}  & between & \textit{mpwepe-le}  & Middle-\stockcaps{LOC}, between & \citet{breen_introductory_2000}\\
\tablevspace
& \textbf{ntuara}  & other side & \textit{ntwarre}  & other side & \citet{breen_introductory_2000}\\
\tablevspace
& \textbf{nankara}  & this side & \textit{nhenh-ankwerr /} \textit{nhangkwarr}  & in this direction & \citet[714]{Green2010}\\
\tablevspace
& \textbf{nkelala}  & beside &  & ? & \\
\tablevspace
& \textbf{itinjawara}  & close by & \textit{itenye-(lwarre)}  & close by -?) & \citet{breen_introductory_2000}\\
\tablevspace
& \textbf{ntuarintjirka}  & though & \textit{ntwarre-(?)}  & ?on the other side & \citet{breen_introductory_2000}\\
\tablevspace
\multicolumn{6}{l}{“governed by the accusative case''}\\
\midrule
& \textbf{katningala}  & upon & \textit{kertnengele}  & above & \citet{breen_introductory_2000}\\
\tablevspace
& \textbf{katningalagana}  & over & \textit{Kertnegele-?}  &  & \\
\tablevspace
& \textbf{kwanakala}  & down & \textit{kwanakerle}  & downwards & \citet{breen_introductory_2000}\\
\tablevspace
& \textbf{kwanala}  & inside & \textit{kwanale}  & inside & \citet{breen_introductory_2000}\\
\lspbottomrule
\end{tabularx}
\caption{Analysis of Kempe’s first class of “preposition''}
\label{bkm:Ref456020954}\label{tab:key:9-189}
\end{table}

\begin{sidewaystable}
\begin{tabularx}{\textwidth}{lQlQp{3cm}l}
 \textbf{Form}
 & \textbf{Original translation}
 & \textbf{Form}
 & \textbf{Currently described as marking:}  & \textbf{Source}
 & \\

\lsptoprule
%\hhline%%replace by cmidrule{-----~}
\textbf{{}-una}
 & “on, upon, into” & \textit{{}-werne}
 & allative case & \citet{wilkins_mparntwe_1989}: 189 & \\
%\hhline%%replace by cmidrule{-----~}
\textbf{{}-lela}
 & “the instrument with which anything is made, or the person by whom anything is accompanied” & \textit{{}-LeLe}
 & instrumental and comitative cases & Wilkins pers. com. 26/08/2016 & \\
%\hhline%%replace by cmidrule{-----~}
\textbf{{}-la}
 & “the place where someone is” & \textit{{}-le}
 & locative case & \citet{wilkins_mparntwe_1989}: 174 & \\
%\hhline%%replace by cmidrule{-----~}
\textbf{{}-gata}
 & “with” (Latin \textit{cum}) & \textit{{}-kerte}
 & proprietive case & \citet{wilkins_mparntwe_1989}: 197 & \\
%\hhline%%replace by cmidrule{-----~}
\textbf{{}-raba}
 & ``without” & \textit{repe}
 & ADVERB

“going along without a particular thing or person” & \citet{henderson_eastern_1994}
 & \\
%\hhline%%replace by cmidrule{-----~}
\textbf{{}-gitjala}
 & “for in exchange for” & \textit{{}-ketye-le}
 & ? &  & \\
%\hhline%%replace by cmidrule{-----~}
\textbf{{}-kaguia}
 & “for the sake of” &{ unattested} & unattested
 &  & \\
%\hhline%%replace by cmidrule{-----~}
\textbf{kiaka}
 & “on, to” &  &  &  & \\
%\hhline%%replace by cmidrule{-----~}
\textbf{{}-ibera}

\textbf{{}-ibena}
 & “from, out of” & \textit{{}-iperre}
 & \stockcaps{AFTER} & \citet{wilkins_mparntwe_1989}: 210 & \\
%\hhline%%replace by cmidrule{-----~}
 &  & \textit{{}-ipenhe}
 &  &  & \\
\multicolumn{6}{c}{}\\
\lspbottomrule
\end{tabularx}
\caption{Analysis of Kempe’s second class of “preposition''}
\label{bkm:Ref456020961}\label{tab:key:9-190}
\end{sidewaystable}

Kempe divided his first class of ``prepositions'', i.e., locational words, into two classes, those “governed” by the accusative case, and those “governed” by the ablative case. By “governed” he meant that the locational word was postposed to a nominal in a particular case, as in the NP \textit{artwe-nge lwarre} `in front of the man', given in Example (\ref{bkm:Ref340591178})

\ea\label{bkm:Ref340591178}
Era          atunga         ulara        tnama \\
\glt `he             of man         in front    stands' \\
\citep[4]{Kempe1891}

\gll Re          artwe-nge    lwarre    irtna-me \\
3\textsc{sg}.\stockcaps{NOM}    man-\stockcaps{ABL}    in{ }front  stand-\stockcaps{PRES}\\
\glt `he stands in front of the man'
\z

Clauses given by Kempe to exemplify “postpositions” governed by the ablative case resemble the Mparntwe Arrernte “relative location construction'', in which a subclass of spatial adverbs “enter into a special construction in which the NP representing the ground is suffixed with \textit{-nge} ABL” \citep[314]{wilkins_mparntwe_1989}. They are also shown in \citegen[64]{PfitznerSchmaal1991} description of Western Arrernte.

\subsubsection{Case paradigms}
\label{sec:key:9.3.3.2}\label{bkm:Ref339538382}

\tabref{tab:key:9-191} summarises the labels given to nominal case markers in the early grammars of Arrernte and contrasts them with Wilkins' more recent analysis.


\begin{sidewaystable}
\small
 \begin{tabularx}{\textwidth}{QQQQQQQQQQ}
\lsptoprule
 & {  {Wilkins} }  {1989}  & {  {Kempe} }  {1891}  & {  {Mathews} }  {1907b} &  {C.~Strehlow}  {C 1907} & {  {Planert} }   {1907}  & {  {C.~Strehlow} }  {1908}  & {  {C.~Strehlow} }  {1910**} & {  {Riedel} }  {1931 [c.1923]**}  & {  {T. G. H.} } {  {Strehlow} }   {1944[1938]} \\
\midrule
 \textit{{}-Ø}  & Nom /

Acc & { {Nom} }

 {Acc}  & { {Nom} }

 {Acc}  & { {Nom} }

 {intrans}
 & {Abs}
 & { {Nom} }

 {Acc}  & { {Nom} }

 {intrans}
 & { {Nom} }

 {intrans}
 & { {Nom I} }

 {Obj} \\
 \textit{{}-le}  & Erg & {Nom}  & {Caus}  & { {Nom}}

 {trans}
 & {Erg}  & {Erg}  & { {Nom}}

 {trans}
 & { {Nom}}

 {trans}
 & {Nom \stockcaps{II}} \\
 \textit{{}-nhe}  & Acc

Pronouns & {Dat}  & {Dat}  & {Acc}
 & {All}  & { {Dat} }

 {Acc}  & { {Dat}}

 {Acc}
 & {Dat}
 & {Obj} \\
 \textit{{}-ke}  & Dat & {Gen}  & {Gen}  & {Gen}
 & {Gen}  & {Gen}  & {Gen}
 & {Gen}
 & {Poss} \\
 \textit{{}-kenhe}  & Poss &  &  &  &  &  &  &  & {Poss} \\
 \textit{{}-nge}  & Abl & {Abl}  & {Abl}  & {Abl}
 & {Abl}  & {Abl}  & {Abl}
 & {Abl}
 & {Abl} \\
 \textit{{}-werne}  & All &  &  & {Dat}
 &  & {All}  & {Loc II}
 & {Loc 2}
 & \\
 \textit{{}-le}  & Loc &  &  &  &  & {Loc}  & {Loc 1}
 & {Loc 1}
 & \\
 \textit{{}-}{lela}
 & * &  & {Inst}  &  & {\_}
 & {Inst}
 & {Inst}
 & {Inst}
 & \\
{ \textit{{}-ipenhe/}}

 \textit{{}-iperre}
 & AFTER &  &  &  &  & {Causalis}
 & {Causalis}
 & {Causalis}
 & \\
\lspbottomrule
\end{tabularx}
\caption{Nominal inflections on nouns in Arrernte which have been described as marking case.}
\label{bkm:Ref456347505}
\label{tab:key:9-191}
\label{fig:key:191}
\legendbox{
*Instrumental function is marked by –\textit{le} in Mparntwe Arrernte, which also marks the ergative and locative cases. Instrumental function is described as distinctly marked in Western Arrernte with the form \textbf{-lela}, which also marks the comitative case.\\
** Riedel and C.~Strehlow(1910) give an additional case form –\textbf{ngibera} termed “Loc 3''. It is not
 given by other early grammarians. It appears to be –\textit{nge} ABL + \textit{{}-iperre} AFTER.
 }
\end{sidewaystable}


Following the missionary grammarians of Diyari (\figref{fig:key:165}) and Teichelmann \& Schürmann’s description of Kaurna (\citeyear{teichelmann_outlines_1840}; \figref{fig:5:78}), Kempe (\figref{fig:key:9-192}) produced conservative Latinate case paradigms when describing Arrernte. He \citeyearpar[3]{kempe_grammar_1891} wrote, “There are six cases – nominative, genitive, dative, accusative, ablative and vocative”.


\begin{figure}
\includegraphics[width=.8\textwidth]{figures/stockigt-img138.png}
\caption{Kempe’s case paradigm of singular nouns \citeyearpar[3]{kempe_grammar_1891}}
\label{bkm:Ref456021315}\label{fig:key:9-192}
\end{figure}

While the case labels, and the ordering of cases that Kempe showed in his paradigms, are ostensibly the same as his Lutheran predecessors’, Kempe’s presentation of ergative morphology and function differ from that instigated by Dresdners. The practices Kempe employed were later followed by C.~Strehlow (\citeyear{strehlow_untitled_1931}~[c.1907]) and mark an alteration to the South Australian Lutheran descriptive tradition.

\largerpage
Ergative case forms are no longer placed at the bottom of the paradigm in the position of the Latin ablative. Kempe is among the few early grammarians who described the ergative case as “nominative” and placed ergative forms along side “other” nominative forms at the top of case paradigms (\citealt{symmons_grammatical_1841}; \citealt{roth_ethnological_1897}; \citealt{ray_linguistics_1907}; \figref{fig:key:9-192}). Kempe’s explanation of ergative function is perfectly adequate: \footnote{Note, however, that Kempe’s (\citeyear{kempe_grammar_1891}: 8) declension of possessive pronouns (\figref{fig:5:86}) did not list both a nominative and an ergative form under the heading ``nominative''. The omission is odd.}
\clearpage

\begin{quote}
The nominative is expressed in a double manner. The noun is unchanged if connected with an intransitive verb, as in \textbf{atua indama}, “the man sleeps”, or \textbf{katjia lima}, “the child goes;” but it is formed by adding the particle \textbf{la}, when it is unconnected (\textit{sic}) with a transitive verb, as, \textbf{worala gama}, “the boy cuts,” \& c., or \textbf{apmala utnuma}, \& c., “the snake bites,” \& c. (\citealt{kempe_grammar_1891}: 3)
\end{quote}


Unlike generations of Lutheran missionaries before him, who had used the term “ablative” to name the ergative case (\figref{fig:5:89}), Kempe used the term “ablative” to describe the suffix marking a range of functions currently associated with the case called “ablative” in Arrernte \citep[185--187]{wilkins_mparntwe_1989}. Kempe gave a brief discussion of the marking and function of each case form and provided example clauses for each. He described the ablative suffix \textit{-nge} as marking:

\ea
\begin{xlist}
\ex {`the direction where a thing comes [sic]':} \\
{Ta              kwatja     nanjinga          inama}\\
\glt `I fetch water from the well' \\
                \citep[3]{kempe_grammar_1891}\\
\gll     The          kwatye      ngentye-nge   ine-me\\
1\textsc{sg}.\stockcaps{ERG}    water-[\stockcaps{ACC}]    well-\stockcaps{ABL}        get-\stockcaps{PRES}\\

\ex {`the material from which a thing is made':} \\
      {Era           ulbainja          ititjinga         erbuma}\\
\glt `he a boomerang from mulga makes' \\
                \citep[3]{kempe_grammar_1891} \\
\gll Re                ulperrenye irtetye-nge     arrpe-me\footnotemark\\
3\textsc{sg}.\stockcaps{NOM}    boomerang-[\stockcaps{ACC}]     mulga-\stockcaps{ABL}   make-\stockcaps{PRES}\\
\glt `He makes a boomerang from mulga'
\footnotetext{While the Western Arrernte term for “boomerang” is listed as \textit{ulperrenye} \citep{breen_introductory_2000}, Kempe’s spelling \textbf{Ulbainja} suggests the term was \textit{ulpeyenye.}
}

\ex {`the cause for which a thing is done':} \\
{Jinga        woringa     tarama}\\
\glt `I laugh for the sake of the boy' \\
                \citep[3]{kempe_grammar_1891}\\
\gll Yenge        werre-nge    therre-me\\
1\textsc{sg}.\stockcaps{NOM}    boy-\stockcaps{ABL}     laugh-\stockcaps{PRES}\\
\glt `I laugh because of the boy'
\end{xlist}
\z

While C.~Strehlow’s earliest case paradigms (\citeyear{strehlow_untitled_1931}: 50--51~[c.1907]; \figref{fig:key:193}) resemble Kempe’s, C.~Strehlow’s treatment of the ergative case differed. Ergative forms are now called ``nominative transitive'', and are assigned their own position in the paradigm, occurring after the nominative form termed “nominative intransitive” (\figref{tab:key:9-191}; \figref{fig:key:193}; \figref{fig:key:194}):


\begin{figure}
\includegraphics[width=\textwidth]{figures/stockigt-img139.png}
\caption{E. Kramer’s copy of C.~Strehlow’s comparative case paradigm of singular nouns in three languages (1931a: 50--51~[c.1907])}
\label{bkm:Ref456021367}
\label{fig:key:9-193}
\label{fig:key:193}
\end{figure}

\begin{figure}
\includegraphics[width=\textwidth]{figures/stockigt-img140r.png}
\caption{E. Kramer’s copy of C.~Strehlow’s comparative case paradigm of dual pronouns in three languages (1931a: 64--65~[c.1907])}
\label{bkm:Ref331499049}
\label{fig:key:194}
\end{figure}

Strehlow’s use of the term \textit{nominativ transitiv}, given here in this comparative grammar of Arrernte and Luritja (\citeyear{strehlow_notitle_1910}) is unique in the Australian literature. Although Strehlow reproduced Kempe’s conservative five-case paradigm – excluding the vocative – Strehlow (ibid.: 17) did consider the idea of inserting other forms termed \textit{locativ}, \textit{instrumentalis} and \textit{causalis} to name case forms in this earliest analysis. Strehlow’s discussion of these terms, and his choice of \textit{nominativ transitiv} to name the ergative case may have been introduced via communication with Leonhardi. The term \textit{Nomitivus transitivus} had previously been employed by Fabricius (\citeyear{fabricius_forsog_1801}: 78--79) in descriptions of Greenlandic (see Lindner, \citeyear{lindner_komposition_2013}: 186, 198).

Like Kempe, C.~Strehlow accounted for the different marking of agents and subjects in terms of verb transitivity. In a note accompanying his case paradigm, Strehlow reproduced in German part of Kempe’s (\citeyear{kempe_grammar_1891}: 3; quoted above) English explanation:

\begin{quote}
The nominative has a double form, depending upon whether the noun is connected with an intransitive or a transitive verb. If the subject nominative is connected with an intransitive verb, then the pure nominative is placed. That is, the word is not altered. If however, the word is connected with a transitive verb, then the syllable “la” is added to the word. (Strehlow \citeyear{strehlow_untitled_1931}: 27~[c.1907])\footnote{“Der Nominativ hat eine doppelte Form, je nachdem das Substantivum mit einem intransitiven oder transitiven Verb verbunden ist. Ist das Subjecktive Nomen mit einem verbum intransitivum verbunden, so wird der Nominativus purus gesetzt; d.h. das Wort erleidet keine Veränderung. Ist dagegen das Wort mit einem transitiven Verb verbunden, so wird die Silbe “la” dem Wort angefügt.”}
\end{quote}

\subsection{C.~Strehlow’s later representations of case}
\label{sec:key:9.3.4}

When responding to Planert’s Arrernte case paradigm (\citeyear{planert_australische_1907}: 555; \figref{fig:2:26}) -- which presented the same six case forms as had Kempe, albeit with different case labels -- C.~Strehlow upped the \textit{ante}.

Strehlow’s 1908 case paradigms differ radically from those given in his earlier MS, and from the homogeneity of earlier Lutheran case paradigms of South Australian languages. C.~Strehlow’s later case paradigms (\citeyear{strehlow_einige_1908}; \citeyear{strehlow_notitle_1910}) are the last in the corpus, although his 1910 paradigm was later replicated by Riedel (\citeyear{riedel_notitle_1931}~[c.1923]). They mark a radical departure from the Lutheran paradigmatic template established by \citet{teichelmann_outlines_1840} that had been followed in the Lutheran descriptions of Diyari and by Kempe.

After criticising Planert’s listing of cases as “incomplete and inadequate” (\textit{dürf\-tig und unvollkommen}), Strehlow (\citeyear[699]{strehlow_einige_1908}) stated: “one must differentiate eleven cases” (\textit{Man muss elf Kasus unterscheiden}; \figref{fig:key:195}).


\begin{figure}
\includegraphics[width=\textwidth]{figures/stockigt-img141.jpg}
\caption{C.~Strehlow’s later extended case paradigm of nouns, showing the term \textit{ergativ} \citeyearpar[699--700]{strehlow_einige_1908}}
\label{bkm:Ref456021393}
\label{fig:key:195}
\end{figure}

C.~Strehlow’s 1908 paradigm includes case suffixes marking locative, instrumental, allative and a case marked by alternative suffixes \textit{-iperre} and –\textit{ipenhe,} glossed by \citet[210]{wilkins_mparntwe_1989} as AFTER. These had previously been described as “prepositions” (\citealt{kempe_grammar_1891}; C. \citealt{strehlow_untitled_1931}~[c.1907]) and as “postpositional suffixes” by T. G. H. Strehlow (\citeyear{strehlow_aranda_1944}: 199--204~[1938]). Leonhardi (\citeyear{leonhardi_letter_1909}) noted that three of that cases Strehlow identified in the 1908 publication had been “included in Dr. P[lanert]’s postpositions”.

C.~Strehlow’s response assimilated the new term \textit{ergativ} from Planert’s grammar, but not the term \textit{absolutiv} (\sectref{sec:key:2.6.1}). Strehlow’s paradigm shows the accusative case as overtly marked with \textbf{-na} \textit{{}-nhe} (A/S/O), as had his earlier work (\citeyear{strehlow_untitled_1931} [c.1907]), differing from Kempe’s representation (\figref{fig:key:9-192}). From this it can be concluded that Planert’s material was based on Wettengel’s adherence to Kempe’s first representation of the language.

\subsubsection{Case labels}
\label{sec:key:9.3.4.1}\label{bkm:Ref74834433}

C.~Strehlow’s 1908 paradigm employs a range of other case terminology not generally found in the corpus: \textit{allativ}, \textit{instrumentalis}, \textit{locativ}, and \textit{causalis}. His stimuli were from a diverse range of sources.

He imported the term \textit{allativ} from Planert. It was used with modern reference to indicate motion towards the marked nominal. This term is not found elsewhere in the corpus. C.~Strehlow’s MS comparative grammar of Arrernte Luritja (\citeyear{strehlow_notitle_1910}; \figref{fig:key:9-196}) maintained the large paradigm but did not utilise the term “allative'', opting instead for a numbered locative case. Parallel forms are given in Luritja. This paradigm was reproduced by Riedel (\citeyear{strehlow_untitled_1931}~[c.1923]; \figref{fig:key:9-197}), probably after Strehlow’s death in 1923, indicating that Riedel copied his material from Strehlow’s \citeyear{strehlow_notitle_1910} MS.


\begin{figure}
% \includegraphics[width=\textwidth]{figures/stockigt-img142.emf}
\includegraphics[width=.8\textwidth]{figures/stockigt-img142.png}
\caption{C.~Strehlow’s comparative case paradigm of Arrernte and Luritja nouns \citeyearpar[11]{strehlow_notitle_1910}}
\label{bkm:Ref456021579}
\label{fig:key:9-196}
\end{figure}

\begin{figure}
\includegraphics[width=.6\textwidth]{figures/stockigt-img143.png}
\caption{Riedel’s case paradigm of nouns (\citeyear{riedel_notitle_1931}: 104 [c.1923])}
\label{bkm:Ref456021650}
\label{fig:key:9-197}
\end{figure}

The term \textit{causalis} was presented with a footnote (\citealt[700]{strehlow_einige_1908}) stating: “I am unclear of the correct signification of this case. The naming is therefore only provisional.”\footnote{“Über die rightige Bezeichnung dieses Kasus bin ich nicht im Klaren, die Benennung Causalis ist daher nur eine vorläufige.”} In March 1908 Leonhardi wrote to Strehlow: “About the causalis or sublative, I should probably best ask Father W. Schmidt in Mödlingen, when the occasion arises” \citep{leonhardi_letter_1908}. Strehlow’s connection to Leonhardi allowed him to seek advice on the naming of Arrernte cases from the European linguistic intelligentsia. By contrast, C.~Strehlow’s naming of other cases appears to have been influenced by Australian grammarians.

C.~Strehlow’s inclusion of suffixes marking functions not associated with the case system of SAE languages is reminiscent of the paradigms given in the earliest grammars of PN languages by Threlkeld (\citeyear{threlkeld_australian_1834}; \sectref{sec:key:3.3.3}) and by Günther (\citeyear{gunther_native_1838}; \citeyear{gunther_lecture_1840}; \sectref{sec:key:4.4.3}). Strehlow was aware of \citegen{fraser_australian_1892} publication. Leonhardi (\citeyear{leonhardi_letter_1908-1}) mentioned both of these grammarians when writing to Strehlow, as if he was also conversant with their grammatical descriptions. It is possible that \citegen{fraser_australian_1892} publication of these earliest two PN grammars influenced the shape of C.~Strehlow’s paradigms and his choice of terminology.

The term “locative” had only been previously used in Australia by \citet[346--347]{gunther_lecture_1840} and by \citet[10]{livingstone_grammar_1892}, although \citet[31]{taplin_1879a_nodate} contemplated using the term when describing Ngayawang. It is likely that in choosing the term \textit{locativ} Strehlow was guided by grammars in \citet{fraser_australian_1892}.

The term “instrumental” had previously been used by Günther, “instrumentative” \citeyearpar[346--347]{gunther_lecture_1840} as well as by \citet[6]{moorhouse_vocabulary_1846} to name a “particle'', and by \citet[9]{livingstone_grammar_1892}. Mathews, with whom C.~Strehlow corresponded, also used the term (see \citealt{koch_r_2008}: 192), including in his description of Arrernte (\citeyear{mathews_languages_1907}). Strehlow’s stimuli for using the term are likely to have come from either \citet{gunther_grammar_1892} or \citet{mathews_languages_1907}, of which Strehlow is most likely to have become aware through his communication with Leonhardi.

\subsection{Processes of clause subordination}
\label{sec:key:9.3.5}\label{bkm:Ref73463231}

Mparntwe Arrernte is described as having fully embedded relative clauses of the structure:

\ea{}
[X-rle (Y) Vb finite – (rle)]\textsubscript{SREL} (3pnDEF) – CASE  \\
(see \citealt{Wilkins1989}: 414--423).\footnote{An alternative analysis of the structure (\citealt{henderson_eastern_1994}: 202) suggests that the marking with SREL \textit{-rle} is optional on both the verb and the first constituent.}
\z

The first constituent of the relative clause, in which the verb is finite, is marked with the relative clause marker \textit{-rle}, which occurs again optionally attached to the finite verb. The relative clause usually sits after the head that it modifies. The final constituent of the clause is marked to agree with the case of the head. This may be a third-person pronoun. The Western Arrernte fully embedded relative clause was described by T. G. H. Strehlow (\citealt{strehlow_aranda_1944} [1938]) and by Capell (\citeyear{capell_outline_1958}; see \citealt{Stockigt2017}; §10.5.1).

\subsubsection{Kempe’s descriptions of subordinate clauses}
\label{sec:key:9.3.5.1}\label{bkm:Ref339746464}\label{bkm:Ref74837096}\label{bkm:Ref339794217}

In the first grammar of Arrernte, \citet[11]{kempe_grammar_1891} wrote: “There are no Relative Pronouns in the language; they are expressed either by repetition of the demonstrative pronouns \textbf{nana} and \textbf{tana} \textit{…} or else by the participle of the verb” (\figref{fig:key:9-198}). Thus, Kempe exemplified two processes under the ``relative pronoun'' heading.


\begin{figure}
\includegraphics[width=.8\textwidth]{figures/stockigt-img144.png}
\caption{Kempe’s description of clause subordination \citeyearpar[11]{kempe_grammar_1891}}
\label{bkm:Ref339787226}
\label{fig:key:9-198}
\label{fig:key:198}
\end{figure}

The first process of clause subordination Kempe exemplified under the heading “relative pronoun” showed “the repetition of demonstratives” (Example \ref{bkm:Ref339739442}).

\ea
\label{bkm:Ref339739442}Atua   nala        nana             tmurka       albuka     worana      tukala\\
\glt `the man who went away yesterday has beaten the boy'

\citep[11]{kempe_grammar_1891}

\gll Artwe    nhale,          nhenhe        apmwerrke   alpe-ke,          werre-nhe   atwe-ke-le\\
Man      that.\stockcaps{ERG}    that.\stockcaps{NOM}    yesterday     return-\stockcaps{PAST}   boy-\stockcaps{ACC}    hit-\stockcaps{PAST}-\stockcaps{SS}\\
\glt `the man, who returned yesterday, is the one who hit the boy'
\z

The clause that is given the relative translation in Kempe’s first example – \textit{nhenhe apmwerrke alpe-ke}, “who went away yesterday” – is marked by a free-standing demonstrative pronoun, \textit{nhenhe}, which is the first constituent. Although the structure of the construction given by Kempe to show that a demonstrative pronoun acted to mark clausal dependency remains uncertain,\footnote{Note that in Example \ref{bkm:Ref339739442}, the clause \textit{Artwe nhale […]werre-nhe atwe-ke-le,} which is translated as the declarative clause “The man […] has beaten the boy”, is itself marked as subordinate. The past tense of the verb \textit{atwe-ke} `hit' is inflected with the same subject switch reference marker. This analysis follows D. Wilkins' (pers. comm., 26/08/2016) suggestion that the Western Arrernte inflection -\textbf{kala} \textit{-kele,} which the missionaries analysed as a second past tense inflection (\sectref{bkm:Ref326948956}) and which is \textit{not} documented in other Arandic varieties, is in fact \textbf{{}-ka} \textit{-ke} PAST found in Western Arrernte and in other Arandic varieties subsequently marked with either the same subject inflection \textit{-le}, or with \textit{-rle,} the relative clause marker. See also Example \ref{bkm:Ref339794050}.} it resembles the structure of the SAE relative clause constructions given in English translation. What is important for this historiographical investigation is that \citegen{kempe_grammar_1891} discussion of demonstrative pronouns acting in a relativising capacity may have influenced \citegen{reuther_dieri_1894} analysis of Diyari, which differed from that offered in previous Diyari grammars (\sectref{bkm:Ref74838296}), and that Kempe’s analysis was probably influenced by the process Meyer had perceived and illustrated as occurring in Ramindjeri in 1843 (\sectref{bkm:Ref73381962}; Examples \ref{ex:key:44}, \ref{ex:key:45} \& \ref{ex:key:46} on pages \pageref{ex:key:44}, \pageref{ex:key:45} and \pageref{ex:key:46}). It is clear that Kempe was acquainted with Meyer’s published grammar (\sectref{bkm:Ref456096204}).

The second example provided by Kempe under the “relative pronoun” heading (Example \ref{bkm:Ref339787713}) showed that the relative pronoun was “expressed” by the verb participle:

\ea
\label{bkm:Ref339787713}Ilupa   tera   iltala    mbakatnarakala   jiraka\\
\glt `the two axes, which were leaning on the house disappeared'

  \citep[11]{kempe_grammar_1891}
\z

Both of Kempe’s two examples were republished by Elkin (\citeyear{Elkin1937}: 164) to show that “[t]he absence … of certain parts of speech [i.e., relative pronouns] does not necessarily mean the absence of the process of thought which we express through them”.

Interpretation of the morphological process used in Kempe’s second example (Example \ref{bkm:Ref339787713}) is complicated by the under-differentiation of lateral phonemes in the missionaries' orthography. There are two possible interpretations of the clause Kempe provided, depending on whether the suffix –\textbf{la} attached to \textit{ilthe} “shelter” is \textit{-le} marking locative case, or is \textit{-le} marking locative case followed by \textit{-rle,} the relative clause marker, where the segment \textit{-le-rle}, may have been reduced or simply not heard, and was represented as \textbf{-la}.

A relative clause interpretation would be:

\newpage´
\ea
Ilepe   therre       ilthe(-le)-rle               ampeke-tne-rre-ke-rle                         uyerre-ke\\
\gll {}[ilepe  therre]\textsubscript{HD}    [ilthe(-le)-rle              ampeke-tne-rre-ke-rle]\textsubscript{SREL}                uyerre-ke\\
     axe   two-[\stockcaps{NOM}]                        [shelter(-\stockcaps{LOC})-\stockcaps{REL}           lean-stand-\textsc{dl}.S/A-\stockcaps{PAST}-\stockcaps{REL}]-[\stockcaps{NOM}]  disappear-\stockcaps{PAST}\\
\glt `the two axes, which were leaning against the house disappeared'
\z

where the relative cause \textit{ilthe-le-rle ampeke-tne-rre-ke-rle}, “which were leaning against the house'', is of the structure [X-rle Vb finite] and is unmarked for nominative case in agreement with the head “two axes''.

Alternatively, Example \ref{bkm:Ref339787713} might be given a switch reference interpretation, in which the finite verb in a dependent clause is marked to indicate that it either shares or does not share the subject of verb in the main clause (see \citealt{wilkins1988}; \citeyear{wilkins_mparntwe_1989}: 454--470). In Western Arrernte the dependent verb is suffixed word finally with \textit{-le} (SS) to indicate that its subject is shared with that of the main clause or the dependent verb is marked with\textit{-nge} (DS) to indicate that the subjects of the two clauses have different identity.

The compound \textit{ampeke-tne} `to lean against (standing)' might be marked with \textit{-le} as having the same subject as the verb \textit{uyerre-} `to disappear'. This would give a relative tense interpretation \citep[462]{wilkins_mparntwe_1989}, where the dependent verb event occurs prior to the time of the man verb event:

\ea
\gll  Ilepe   atherre             ilthe-le         ampeke-tne-rre-ke-le               uyerre-ke\\
Axe    two-[\stockcaps{NOM}]       shelter-\stockcaps{LOC}    lean-stand-duS/A{}-\stockcaps{PAST}-\stockcaps{SS}      disappear-\stockcaps{PAST}\\
\glt `the two axes were leaning against the house before they disappeared'
\z


Whatever the structure, Kempe correctly perceived that clause subordination was signalled on the verb and exemplified the process.

\subsubsection{C.~Strehlow’s descriptions of subordinate clauses}
\label{sec:key:9.3.5.2}\label{bkm:Ref516470592}

Carl Strehlow’s earliest analysis of Arrernte (1931a~[c.1907]) utilised the previous analysis of Arrernte written by HMS missionary \citet{kempe_grammar_1891}. C.~Strehlow reproduced Kempe’s explanation of clause subordination under the heading “relative pronouns” (ibid.: 39--40). Strehlow’s explanation (\figref{fig:key:199}) is almost a word-for-word translation into German of Kempe’s English publication (\figref{fig:key:198}).

\begin{figure}
% \includegraphics[width=\textwidth]{figures/stockigt-img145.emf}
\includegraphics[width=.9\textwidth]{figures/stockigt-img145r.png}
\caption{E. Kramer’s copy of C.~Strehlow’s description of relative pronouns (\citeyear{strehlow_untitled_1931} [c.1907]: 39--40)}
\label{bkm:Ref339787839}
\label{fig:key:199}
\end{figure}

Like \citet{Kempe1891}, C.~Strehlow (1931a~[c.1907]) provided examples of two processes of clause subordination in this section of the grammar, but he provided different example clauses. Example (\ref{bkm:Ref339794050}) was used to show how repetition of a demonstrative acted to relativise a clause.

\ea
\label{bkm:Ref339794050}Atua       lena,                    nana              lata               pitjikala,          mara       nama\\
\glt `mann dieser, der da heute gekommen ist, gut ist'

 (C \citealt{strehlow_untitled_1931} [c.1907]: 39--40)

\gll Artwe      lanhe,                nhanhe          lyete      petye-ke-rle,             marre    ane-me\\
Man        that.\stockcaps{NOM}          this.\stockcaps{NOM}      today    come-\stockcaps{PAST}-\stockcaps{REL}      good      sit-\stockcaps{PRES}\\
\glt `That man, the one who came today, is good'
\z

Here the demonstrative, \textit{nhanhe} `this', which is the head of the parenthetical relative clause \textit{nhanhe lyete petye-ke-rle} `the one who came yesterday' makes anaphoric reference to the S of the main clause, \textit{artwe lanhe} `that man'.

Strehlow also illustrated a different process of clause subordination. He provided another example (Example \ref{bkm:Ref339788704}) to show that where there was no repetition of the demonstrative; the participle form of the verb acted to relativise the clause. He stated: “The repetition does not occur and the predicative verb occurs in the participle form.'' (\citealt[39--40]{strehlow_untitled_1931}~[c.1907])\footnote{“Die Wiederholung unterbleibt und das praedikative Verb tritt in der Partizip-Form auf.”}

\ea
     \label{bkm:Ref339788704}Ara     nana         unta              nguruka         ntainatnala (part.perf.)   andere    naka\\
\glt `kaenguru dies da, du gestern gespeert habend  fett war'

(C. \citealt{strehlow_untitled_1931} [c.1907]: 39--40)

\gll Aherre     nhenhe      unte              ngwerreke     irntarne-rtne-rle              antere      ane-ke\\
Kangaroo   this         2sg\stockcaps{ERG}      yesterday       spear-?-\stockcaps{REL}      fat             sit-\stockcaps{PAST}\\
\glt `this kangaroo, that you speared yesterday, was fat'
\z

In his earliest grammar of Arrernte (\sectref{bkm:Ref456272021}), C.~Strehlow provided comparative Diyari and \textit{Encounter Bay Sprache} (Ramindjeri) material. The typescript copy of Strehlow’s grammar held at the Lutheran Archives (\citeyear{strehlow_grammatik_1931} [c.1907]) contains a more extensive note on the comparison of clause subordination in three languages (\figref{fig:key:9-200}) than does the copy made by Kramer held at the South Australian Museum.

\begin{figure}
\includegraphics[width=.8\textwidth]{figures/stockigt-img146.png}
\includegraphics[width=.8\textwidth]{figures/stockigt-img147r.png}
\caption{Copy of C.~Strehlow's analysis of Arrernte (\citeyear{strehlow_grammatik_1931} [c.1907]) made by an unknown typist (no page). Annotations made by unknown reader}
\label{bkm:Ref339787928}
\label{fig:key:9-200}
\end{figure}


Here Strehlow reproduced the Ramindjeri example published by Meyer in 1843 (Example \ref{ex:key:45}, on page \pageref{ex:key:45}) with slight variation (Example \ref{bkm:Ref339788405}). He provided interlinear and free translation:

\ea
\gll    \label{bkm:Ref339788405}Korne    naiye    lewin    mantangg  an\\
Mann      der       wohnte im Haus  mein\\
\glt `Der Mann, der in meinem Hause wohnte'

(C. \citealt{strehlow_grammatik_1931} [c.1907] after \citet{meyer_vocabulary_1843}: 32)
\z

\largerpage
He reiterated Reuther’s description of clause subordination in Diyari, which had differed from all four previous missionary-grammarians descriptions of that language -- “personal pronouns replace the missing relative in Diyari” -- and he reproduced the clause which had been given by Reuther (Example \ref{bkm:Ref77507345}):

\ea
\gll    \label{bkm:Ref77507345}ngato    ninaia    ngantjai,    nulia   ngakangu    nguma   nankara   warai\\
Ich ihn liebe, er mir gutes getan hat\\
\glt `Ich liebe den, der mir gutes getan hat'

(C. \citealt{strehlow_grammatik_1931} [c.1907] after \citealt{reuther_dieri_1894}: 29)
\z

Drawing on the work of Meyer and Reuther, Strehlow concluded that the process of relativisation occurs similarly in all three languages. Meyer’s erroneous examples are recycled to substantiate the suggestion that personal or demonstrative pronouns function to relativise clauses in these Australian languages. While the year in which C.~Strehlow made this analysis is unknown, it seems likely that these parallels between Meyer’s account of clause subordination in “Encounter Bay” and the structure of Diyari were drawn before Reuther completed his 1894 grammar, within C.~Strehlow’s first two years in Australia. It is also likely that C.~Strehlow was further alerted to the possibility of such structures occurring by \citegen{kempe_grammar_1891} published description of demonstratives acting to relativise clauses in Arrernte (Example \ref{bkm:Ref339739442}).

The subordinating construction that Strehlow and Kempe described may have reflected a new method of marking subordinate clauses that developed as a syntactic calque from German and English, and which became a feature of the variety of language used at the mission. It may have been a feature of what Siebert described as \textit{Küchen-Dieri} (Kitchen Diyari) at Bethesda (quoted in \citealt{kneebone_teichelmann_2005}: 372--373) and of Arrernte.

Gatti’s grammar (\citeyear{gatti_lingua_1930}) of Diyari (\sectref{bkm:Ref73350344}), based on the language used in Reuther and Strehlow’s translation of the New Testament, shows the third-person pronoun acting to mark subordinate clauses in the variety used by the missionaries in liturgical translation. Gatti (ibid.: 67) quoted Luke 1: 19: “I am Gabriel, who stands in the presence of God” (Example \ref{bkm:Ref77507401}). In presenting the construction, he specifically indicated that the Diyari third-person pronoun \textbf{nauja} corresponded functionally to the Italian relativiser \textit{che}:

\ea
      \label{bkm:Ref77507401}Ngani   Gabrieli   nganai,          nauja    Godani    terkai\\
 \glt io sono Gabriele che sto innanzi a Dio

\citep[67]{gatti_lingua_1930}

\gll Nganhi Gabriel    ngana-yi,     nhawa   Goda-nhi tharka-yi\\
    1sg\stockcaps{NOM} Gabriel  be-\stockcaps{PRES},     3\textsc{sg}.\stockcaps{NF}   God-\stockcaps{LOC}  stand-\stockcaps{PRES}\\
\glt `I am Gabriel, who stands in the presence of God'
\z

Here it is important to observe that T. G. H. Strehlow (\citeyear{strehlow_aranda_1944}: 101~[1938]) described the shift to marking subordinate clauses with a pronoun in Arrernte as having occurred among Arrernte speakers at the Hermannsburg mission. He suggested that Bible translation “necessitated the continual use of the interrogative pronouns (\textbf{ŋu\c{n}a}, \textbf{iwu\c{n}a)} and of the demonstrative pronoun \textbf{na\c{n}a}, by the white missionaries in a strictly relative sense”, and he saw this usage as having influenced the speech of younger Arrernte generations. \citet[13]{capell_outline_1958} observed the same shift in function of Western Arrernte demonstrative pronouns. T. G. H. Strehlow gave this example:

\ea
iwu\c{n}a              aţa      arugula   altaraka,      la\c{n}a    aţa      jusem-ila-ka
\glt `what I first found, that I used'

(T. G. H. \citealt{strehlow_aranda_1944}: 101[1938])


\gll Iwenhe             the     arrekwele    arltare-ke,       lenhe   the     usem-ile-ke\\
 \stockcaps{DEM}              1\textsc{sg}.\stockcaps{ERG}      first              find-\stockcaps{PAST}      \stockcaps{DEM}   1\textsc{sg}.\stockcaps{ERG}   use-\textsc{tr}-\stockcaps{PAST}\\
\z


Further philological investigation of religious texts in Diyari and in Arrernte would help establish the extent to which the linguistic variety used by the missionaries was the data source for their grammatical analyses.

\section{Conclusion}
\label{sec:key:9.5}

The Lutheran school of South Australian linguistic description, which originated with the Dresdner missionaries' descriptions (\citealt{teichelmann_outlines_1840}; \citealt{meyer_vocabulary_1843}; \citealt{schurmann_letter_1844}) of languages made in the decade following the establishment of the Colony of South Australia held sway over the description of Arrernte into the twentieth century.

Kempe’s inaugural Arrernte grammar (\citeyear{kempe_grammar_1891}), which is a remarkably comprehensive work describing verbs inflected for associated motion (\sectref{bkm:Ref456085056}) and kin-dyadic terms (\sectref{bkm:Ref463530241}), does not follow Teichelmann and Schürmann’s placement of the ergative case towards the bottom of the case paradigm. Teichelmann and Schürmann’s format was maintained over six decades of descriptions made in Australia, by Taplin (\citeyear{taplin_vocabulary_1867}; \citeyear{taplin_notes_1872}~[1870]; \citeyear{taplin_native_1874}; \citeyear{taplin_grammar_1878}) (\sectref{bkm:Ref514608340}) and by those who replicated his paradigms in Brough Smyth (\sectref{bkm:Ref456083856}) and in all grammars of Diyari (\chapref{chap:key:8}), including \citegen{planert_australische_1908} grammar published in Germany. Kempe’s choice to present ergative forms as an alternative “nominative” form was atypical, and marked a departure from established practice, but is characteristic of the descriptive independence displayed by many of the corpus grammarians, who confidently presented new and alternative representations of PN structures. Nevertheless, many aspects of Kempe’s analysis show that he did borrow descriptive methods instigated by his South Australian Lutherans with whose work he was familiar.

Three descriptive features define the tightly-knit Lutheran school of descriptive practice. First, the division of post-positions into two classes (\sectref{bkm:Ref331424200}), made originally by Teichelmann \& Schürmann (\sectref{sec:key:5.3.2}) and followed in grammars of Diyari (\sectref{bkm:Ref456339501}). Second, the inclusion of paradigms declining possessive adjectives (\sectref{sec:key:5.3.2}), which had similarly been initiated by Teichelmann \& Schürmann and followed by the Diyari missionary-grammarians and by Kempe (\figref{fig:5:86}). Third, the description of theoretically challenging processes of clause subordination under the heading “relative pronoun” (\sectref{bkm:Ref73381962}).
