\chapter{Theoretical considerations}
\label{chap:key:2}
\label{sec:2}
This chapter first classifies the corpus grammars according to the types of other documentation available for the languages they describe (\sectref{sec:key:2.1}). The early grammars of languages of which there is no other comprehensive grammatical record can be difficult to assess, while grammars of languages of which there is a good modern record present opportunities to understand the early grammarians' descriptive intent. The traditional grammatical framework, which was used in all the early sources, to different degrees and for different reasons, is then characterised (\sectref{sec:key:2.2}). The discussion focuses on the processes by which PN structures were overlooked and misrepresented, thereby expounding a philological methodology for retrieving morphosyntactic data from the early sources (\sectref{sec:key:2.3}).
The chapter concludes with a discussion of how the traditional word and paradigm descriptive model (\citealt{hockett_two_1954}; \citealt{robins_1959}) is ill-suited to the representation of PN agglutinative typology and shows that some early grammarians innovated more appropriate models (\sectref{sec:key:2.4}).

\section{Methodological limitations}
\label{sec:key:2.1}

It is not possible to assess each corpus language from an equally well-informed point of view, since the quality of morphosyntactic description varies. For some languages there is reasonably good old and modern description, while the record of others survives only in early skeletal sketches. Thus, the corpus languages present different types of methodological challenges.

Without a good modern description of the language sketched in the early sources, the early grammatical description requires a degree of initial interpretation in order to reconstruct the putative original form of the language. This can be problematic. It can be difficult to decide whether the absence of the description of a particular grammatical construction in an early source that is known to be common in surrounding PN languages reflects a true absence in the described language or a gap in the data. The identical marking of inalienable and alienable possessed NPs (\sectref{sec:key:8.7.2}) in reclaimed Awabakal \citep[33]{lissarrague_salvage_2006} is one such instance. PN languages commonly show a difference in the marking of alienable and inalienable possessive constructions \citep[59]{dixon_australian_2002}. When the possessed nominal is implicit to the relationship between the two items within a possessive NP, the dependent modifier remains unmarked. Thus, inalienable possessive constructions are marked through juxtaposition as in the following Pitta-Pitta clause in which the possessor \textbf{mochoomba} `kangaroo' is morphologically unmarked:

\ea
\gll {Mochoomba wapa} \\
{`a kangaroo’s pup'}  \\
\glt \citep[8]{roth_ethnological_1897} \\
\gll Matyumpa warrpa\\
Kangaroo-[\stockcaps{NOM}] pup-[\stockcaps{NOM}]\\
\z

The juxtaposition of the inalienably possessed constructions contrasts with the morphological marking of alienably possessed NPs. Given that Awabakal ceased to be spoken at a time when the unmarked inalienably possessed NP was undescribed in Australia, it is difficult to be sure whether Awabakal followed the common practice of indicating an inalienably possessed NP through juxtaposition rather than morphological marking \citep[59]{dixon_australian_2002}. 

Further, it can be difficult to distinguish which elements of an early description relate to idiosyncratic structure and which might result from poor recording. Livingstone (\citeyear[8]{dixon_australian_2002}; \sectref{sec:key:4.6.1}), for example, gave \textit{bula}, a common PN form marking the dual (\citealt[116--117]{dixon_australian_2002}) as the numeral two in Minjangbal, but he did not describe morphological marking of dual number on nouns. Regarding pronouns, he stated: “MinyuG has no simple dual, although there are compound terms and phrases denoting number” \citep[6]{dixon_australian_2002}. Livingstone described the plural pronoun, suffixed with the dual morpheme \textit{bula} as ``dual compounds''. It might be tempting to assume that the description is an error, since Livingstone’s description does not conform to the PN norm. But in this instance, modern descriptions of nearby languages confirm Livingstone’s analysis (\citealt[15]{Sharpe_1992}; \citealt[258]{smythe_bandjalang_1978}). Crowley (\citeyear[78]{crowley_middle_1978}) explains the unusual situation: “Note that the dual pronouns of most Australian languages are absent. What seems to have happened is that the earlier plural forms were lost and that the original duals were generalised to become plurals”.

That all early sources underdifferentiated coronal consonant phonemes (\tabref{tab:chap1:consonantphonemes}, p. \pageref{tab:chap1:consonantphonemes}) hampers the reclamation of inflectional and derivational morphology from the early records. The extent of the difficulty is highlighted when considering the early description of languages that have been described well in recent times. The missionaries' orthography developed for Diyari, for example, failed to distinguish apico-alveolar, apico-domal and lamino-dental nasals. Consequently, inflections marking the locative case \textit{-nhi}, the dative/possessive case \textit{-rni} and nominative case \textit{-ni} on different nominal types were all represented as \textbf{-ni} (\tabref{fig:key:8-174}). The complexity of the Diyari case system could not be reclaimed from the historical record.

Consider also the Western Arrernte suffix marking accusative case on animate nouns \textit{–nhe}, and the suffix marking the allative case –\textit{werne}. The difference between these forms was not noted in the earliest grammars (\citealt{kempe_grammar_1891}; \citealt{mathews_arranda_1907}) and consequently the functions marked by formally distinct suffixes were described as being marked by a single case with \textit{–na}. This phonemic underdifferentiation of interdental and retroflex nasals contributed to misrepresentations of the marking of allative, accusative and dative functions in the early descriptions of Arrernte (\citealt{Stockigt2017}). That allative function is marked by a distinct case form in Western Arrernte would not be retrievable from the earliest sources alone.

\tabref{tab:chap2:modern} tabulates the corpus languages according to the type of contemporary description available for that language.

\begin{table}
\small
    \begin{tabularx}{\textwidth}{lQQQ}
        \lsptoprule
        Category & Language & Modern source material & Modern source material based on:  \\
        \midrule
        1 & Nggerrikwidhi & None, other than references in \citet{Dixon2002} & \\
        \tablevspace
        2 & Ganai & \citet{Fesl1985,Gardner1996} & The early source material \\
        & Awabakal & \citet{Oppliger1984,Lissarrague2006} & \\
        & Wiradjuri & \citet{GrantRudder2001,GrantRudder2014} & \\
        & Kaurna & \citet{AmerySimpson2013,Amery1998,amery_warrabarna_2016} & \\
        & Barngarla & \citet{Clendon2015} & \\
        & Ngayawang & \citet{Horgen2004} & \\
        \tablevspace
        3 & Nyungar & \citet{Douglas1968} & Both the early sources material and the speech of remembering speakers \\
        & Ramindjeri Ngarrindjeri & \citet{YallopGrimwade1975,Cerin1994,McDonald2002,Bannister2004,Gale2009,Gale2010} & \\
        & Gamilaraay & \citet{Austin1993,Giacon2014} & \\
        & Pitta-Pitta & \citet{BlakeBreen1971,Blake1979b} & \\
        & Yadhaykenu & \citet{Crowley1981} & \\
        & Wergaya & \citet{Hercus1969} & \\
        \lspbottomrule
    \end{tabularx}
    \caption{The modern grammatical descriptions of the languages described by early grammarians in Australia }
    \label{tab:chap2:modern}
    \label{tab:2.1}
\end{table}	

\begin{table}
	\begin{tabularx}{\textwidth}{lQQQ}
		\lsptoprule
		Category & Language & Modern source material & Modern source material based on:  \\
		\midrule
		4 & Diyari & \citet{austin_grammar_2013} & The speech of fluent speakers \\
		& Wangkangurru & \citet{hercus_grammar_1994} & \\
		& Yandrruwandha & \citet{breen_innamincka_2004} & \\
		& Arrernte & \citet{capell_outline_1958} (W. Arrernte), \citet{wilkins_mparntwe_1989} (Mparntwe Arrernte), \citet{henderson_topics_2013} (Eastern \& Central Arrernte), \citet{PfitznerSchmaal1991} (W. Arrernte) & \\
		& Luritja & \citet{hansen_pintupi_1978} & \\
		& Guugu-Yimidhirr & \citet{haviland_guugu_1979} & \\
		& Minjangbal & \citet{smythe_bandjalang_1978}, \citet{Cunningham1969}, \citet{geytenbeek1971}, \citet{crowley_middle_1978} & \\
		\lspbottomrule
	\end{tabularx}
\end{table}	

Of the languages that ceased to be spoken before being described in the modern descriptive era (categories 1 \& 2) most have, to some degree, been reclaimed within language revival programmes (category 2). \citegen{hey_elementary_1903} grammar of Nggerrikwidhi appears to be a different variety from that recorded in the region in more recent time by \citet{hale_linngithigh_1966} and Crowley (\citeyear{crowley_mpakwithi_1981}; \sectref{sec:key:10.1.3.1}) and this grammar has received little modern scrutiny.

Contemporary understanding of languages in category 3 has been obtained both from early records and from elicitation from twentieth-century speakers, who have retained different degrees of spoken competency. For these languages it can be difficult to determine whether discrepancies between early and contemporary records result from descriptive misconstrual or oversight in the early sources, or from diachronic language change. Indeed, some of the modern source material for this group of languages aims specifically to document language shift. Bannister’s description of Ngarrindjeri (\citeyear{bannister_longitudinal_2004}) examines language shift under pressure from English by comparing twentieth-century speech with the structure recorded in the early sources, noting loss of case allomorphy, loss of morphological marking of dual on nouns, loss of dual pronouns, and loss of ergative marking. \citet{douglas_aboriginal_1968} makes a synchronic study of “Neo-Nyungar” spoken by twentieth-century speakers and does not attempt to reconstruct the variety described in the earliest grammatical record by \citet{symmons_grammatical_1841}.

Posing further methodological difficulties for this study is the fact that the source material on which a language has been reclaimed, and the reasoned decisions which must sometimes be made without definitive substantiation, are often not clearly documented. For instance, Grant and Rudder’s grammar of Wiradjuri (\citeyear{grant_grammar_2014}) does not divulge the extent to which the presented structures are based on historic documents, calquing from better-remembered neighbouring languages, or from the remembered speech of the community.

At the other end of the spectrum are languages that have been described from fluent speakers in the modern era (category 4). One can be fairly certain that the structures recently described are the same as those encountered by nineteenth-century grammarians. The contemporary analysis of these languages provides an ideal analytical platform from which to view the relative merit of an early grammar. 

Modern grammars of some of the languages treated in this study are among the best and most informative materials written on PN languages. These are the grammars of Pitta-Pitta (\citealt{blake_pitta-pitta_1971}; \citealt{blake_pitta_1979}), Guugu-Yimidhirr \citep{haviland_guugu_1979}, Diyari \citep{austin_grammar_2013}, Yadhaykenu \citep{crowley_mpakwithi_1981}, Wangkangurru \citep{hercus_grammar_1994}, and Arrernte (\citealt{wilkins_mparntwe_1989}; \citealt{henderson_topics_2013}). These works drew on an eclectic range of theories in order to accommodate newly encountered linguistic structures during the last quarter of the twentieth century. All but the earliest (\citealt{blake_pitta-pitta_1971}) are characterised by an approach emanating from the Australian National University school of Australian grammatical description (see \citealt[58--72]{wilkins_mparntwe_1989}), which produced works that “evolved in parallel to the evolution of theory and practice in linguistics” (\citealt[59]{wilkins_mparntwe_1989}) and have shaped the contemporary understanding of PN structures.

\hspace*{-2.1pt}Comparative reading of these sophisticated modern grammars with early grammatical description \textit{of the same language} provides training in the type of philological method required to interpret morphosyntactic data from the early sources. One becomes attuned to the particular mistakes or obscure representations that are likely to be made, and alerted to the possibility of their occurrence in languages that did not survive to be described in the modern era. Making sense of an early grammar of a language for which there is good contemporary analysis is like cheating in a puzzle. It is not necessary to struggle to discover what exactly the early missionary-grammarian was attempting to describe because the structure is clearly explained in modern terminology and using currently accepted linguistic conventions. Reading old grammatical material while informed about the structure of the language from a contemporary record brings into sharp focus the particular limitations of language reconstitution based solely on old material.

Note that most comprehensive, contemporary descriptions of Arrernte (\citealt{henderson_topics_2013}; \citealt{wilkins_mparntwe_1989}) are of different varieties from that recorded in the early sources at the mission. The less extensive manuscript grammars by \citet{capell_outline_1958} and \citet{pfitzner_1991} are of the same Western Arandic variety. The assessment of the early description of Western Arrernte presents some of the same methodological difficulties as languages which ceased to be spoken before a contemporary record was made.

Making sense of the description of a language of which there is no modern description (categories 1 \& 2) depends on consideration of material from three sources: the historical record, the surviving material of closely related and/or contiguous languages, and the contemporary understanding of other PN morphosyntax and its likely historical evolution. This philological method is described by Oppliger: 

\begin{quote}
    [A]n analysis of Awabakal grammar is limited to one source … Comparative material from nearby languages … is however helpful as supportive evidence … Also commonly occurring Australian pronominal features sometimes suggest an analysis as well as lending support to certain hypotheses. \citep[64]{oppliger_phonology_1984}
\end{quote}

The process is one of triangulation \citep[33, 147]{amery_warrabarna_2016}, since it involves the weighing up of the relative value of three source of information, and some degree of educated guesswork.

But this process of triangulation is dependent upon \textit{initially recognising} the structures that are described in the early sources. Reclaiming morphosyntactic systems and structures from early sources in which their description is obscured by an absence of appropriate frameworks, accepted terminology, and phonemic orthographies is not straightforward. The potential difficulty is epitomised in the three conflicting accounts of the pronominal system of marking syntactic case in Awabakal based on Threlkeld’s grammar (\citeyear{threlkeld_australian_1834}; \citealt{oppliger_phonology_1984}; \citealt{lissarrague_salvage_2006}; \citealt{dixon_australian_2002}; \sectref{sec:key:3.3.6.1}). While these divergent analyses probably result from a complex morphosyntactic idiosyncrasy that is now irretrievable, attempting to account for the differences requires viewing the structure of Awabakal through the looking glass of Threlkeld’s nineteenth-century expectations.

This type of hermeneutic investigation of the source material pinpoints the processes by which morphosyntactic data might be skewed when cast in an inappropriate descriptive framework and reveals the type of filters through which obscure explanations should be screened if the source material is to be accurately reclaimed. Rather than seeing the source material as a foundation upon which the triangular process of language reclamation rests, any philological methodology of language reclamation from historical sources must also articulate processes of extracting data from pre-contemporary grammars (\sectref{sec:key:2.3}).

A. P. Elkin, professor of anthropology at the University of Sydney in the 1930s, assessed the early grammatical description of Australian languages as generally inadequate. He did, however, perceive that careful scrutiny of early Australian grammars yielded valuable linguistic material, and described the process of retrieving material from the early sources as “careful sieving” (\citeyear[9]{elkin_nature_1937}). By stripping back the veil of arcane terminology and inappropriate descriptive frameworks, and by recreating the author’s logic, the close and comparative study of the early grammars is sometimes “punctuated by the occasional sudden realisation of the point of a piece of writing, an understanding of what the writer is really on about” \citep[2]{mcgregor_introduction_2008}. Despite the fact that language reclamation processes currently underway across Australia profit from proper historiographical investigation of the source materials, Elkin’s process of “careful sieving” remains un-theorised.

\section{The traditional grammatical framework }
\label{sec:key:2.2}

This section investigates the processes by which morphosyntactic features were likely to be obscured when fitted into the schemata of an ill-suited grammatical framework.

The traditional descriptive practices that missionary-grammarians tended to employ when describing PN languages evolved from the study of Greek, particularly the ancient writings of Dionysius Thrax (c. 100 BC; English translation in \citealt{thrax_tekne_1986}) and subsequently of Latin, in the writings of Varro (c.100 BC; \citealt{Varro_1996}) and Priscian (c. 500 AD; see \citealt{luhtala_amsterdam_2005}). As the system became codified and was applied to the description of other European languages, the grammatical categories conveyed within the schemata came to be seen as atheoretical universal categories that required no introduction or clarification. Koch, in an assessment of R. H. Mathews' grammatical descriptions, characterised the traditional grammatical framework as having:
\begin{quote}
    emerged from Greek and Roman grammarians [and which] was further developed in Western Europe during the Middle Ages, Renaissance and subsequent centuries, and inherited into nineteenth-century Britain … This is the system that underlay the pedagogy of not only Latin and Greek but also modern languages and English itself. Its basic framework can be seen most easily in nineteenth century textbooks of Latin and Greek that have been used into the twentieth century. \citep[187]{koch_r_2008}
\end{quote}

The term “traditional grammar” is used here to invoke both the schemata and descriptive model of grammars that developed to best capture the structure and typology of classical European languages. It entails firstly the conventional arrangement of headings and subheadings in which the existence of certain structures was anticipated (\sectref{sec:key:2.3.1}) as well as the word and paradigm model of description (\sectref{sec:key:2.4}) that had developed to convey the fusional morphology of SAE languages.

Beyond the ubiquitous schoolboy Latin, which provided all early grammarians with a ready-made scaffold on which to hang nascent awareness of PN structures, the grammarians' exposure to nineteenth-century grammars of Latin, Greek and Hebrew varied according to their education and their training for mission. Lutheran missionary-grammarians were likely to have learnt Greek and Hebrew, which were seen as important source languages for translation of the Scriptures (\citealt[67--78]{rathjen_difficult_1998}; \citealt[332]{strehlow_tale_2011}). The Lutheran seminary in Adelaide, for example, holds \textit{Grammatik des neutestamentlichen Sprachidioms als sichere Grundlage der neutestamentlichen Exegese} (\citealt{winer_grammatik_1844}; Grammar of the idiomatic speech of the New Testament as a sound basis for New Testament Exegesis), which belonged to missionary Schoknecht, who wrote a grammar of Diyari in 1872 (\sectref{sec:key:8.3.3}). Copies of other books held in their collection, including \textit{Ausführliches Lehrbuch der Hebräischen Sprache des Alten Bundes} (\citealt{ewald_ausfuhrliches_1844}; A detailed textbook of old Testament Hebrew) and \textit{Elementargrammatik der lateinischen Sprache} (\citealt{kuhner_elementargrammatik_1841}; An elementary grammar of Latin) were probably donated to the seminary upon the death of their owner. Kühner had earlier produced \textit{Ausführliche Grammatik der griechischen Sprache} (1834--1835; A detailed grammar of the Greek language; \figref{fig:key:2-10}).

\hspace*{-.9pt}Later missionaries from a British background were likely to have studied works such as B. H. Kennedy’s grammars of Latin (\citealt{kennedy_public_1879}; \figref{fig:key:2-11}, \figref{fig:key:2-13}).

The content of works such as these shaped the missionary-grammarians' expectations about how a language should work and equipped them with the tools to describe Australian languages, whose morphosyntax had never previously been analysed.
 
\begin{figure}
\includegraphics[width=.8\textwidth]{figures/Figure 10.png}
\caption{Kühner’s categories of pronoun in Classical Greek (1890: xxiii [\citeyear{raphael_ausfuhrliche_1834}])}
\label{fig:key:2-10}
\end{figure} 

\begin{figure}
\includegraphics[width=.8\textwidth]{figures/Figure 11.png}
\caption{Kennedy's declension of Latin relative pronouns \citep[140]{kennedy_public_1879}}
\label{fig:key:2-11}
\end{figure}

\subsection{Categorical particularism and the absence of appropriate schemata}
\label{sec:key:2.2.1}

In 1844, Schürmann perceived a tension between the premises underlying received descriptive linguistic schemata and the new linguistic structures he encountered. He advised that the description of Australian languages required authors to:

\begin{quote}
   divest their minds as much as possible of preconceived ideas, particularly of those grammatical forms which they may have acquired by the study of ancient or modern languages. (\citealt[vi]{schurmann_vocabulary_1844})
\end{quote}

\hspace*{-.2pt}Schürmann and some other nineteenth century grammarians in Australia were aware that the description of Australian languages might be compromised by ``categorical particularism'', defined by \citet[2]{haspelmath_comparative_2010} as “one of the major insights of structuralist linguistics of the 20th century (especially the first half) that languages are best described in their own terms … rather than in terms of a set of pre-established categories that are assumed to be universal”. The realisation that the study of language should be non-aprioristic is widely associated with Boas (\citeyear[81]{boas_handbook_1911}; \citealt[4ff.]{haspelmath_comparative_2010}) but is traceable to the writings of Wilhelm von Humboldt (1767--1835), who observed:

\begin{quote}
    Normally we come to the study of an unknown language from the point of view of a known language, be it our mother tongue or Latin, we try to see the grammatical relationships of this language expressed in the new one …; to avoid this mistake we must study each language in its peculiarities. (Wilhelm von Humboldt 1827, quoted in \citealt[105]{morpurgo-davies_language_1975})
\end{quote}

Some early Australian description was similarly made with awareness that linguistic principles deduced from the study of classical languages did not have universal application. Threlkeld was aware that the structural complexities he encountered could not adequately be described by the existing descriptive framework: 
\begin{quote}
    The arrangement of the grammar now adopted, is formed on the natural principles of the language, and not constrained to accord with any known grammar of the dead or living languages. The peculiarities of its structure being such, as to totally prevent the adaptation of any one as a model. \citep[x]{threlkeld_australian_1834}
\end{quote}
 
But even with this awareness, the early grammarians' description of PN languages was hampered by the absence of appropriate frameworks and terminology to describe the foreign structures. With reference to Threlkeld’s grammar of Awabakal (\citeyear{threlkeld_australian_1834}), H. Hale appreciated the difficulty in framing a “mass of information which is entirely new” (\citealt[482]{threlkeld_1846}) without appropriately developed descriptive tools: 
\begin{quote}
    It is not surprising that the novelty and strangeness of the principles on which the structure of the language was found to rest, should have rendered a clear arrangement, at first a matter of difficulty; and some degree of obscurity and intricacy in this respect have caused the work to be less appreciated than its merits deserved. (\citealt[482]{threlkeld_1846})
\end{quote}

Investigation of the early analysis of PN languages shows that the missionary-grammarians' ability to use the language sometimes outstripped their descriptive ability. \citet{wafer_waiting_2011} assess the language used by Threlkeld in translation of scriptural texts by examining processes of clause subordination, concluding (ibid.: 132) that “Threlkeld’s command of the language was surprisingly good.” They observe, however, (ibid.: 114) that “Threlkeld’s handling of this cryptic feature of the language [i.e., processes of clause subordination with the clitic \textit{-pa}] was surprisingly idiomatic, \textit{in spite of the fact that he was able to unravel only a small part of it in his grammatical analysis}” (but see \sectref{sec:key:3.3.8}). The observation is also relevant to other PN grammarians who managed to engage deeply with the structure of the language. \citet[13]{teichelmann_outlines_1840}, for example, described “observing” subordinating structures in Kaurna, which they illustrated without attempting to provide any accompanying analysis (see \sectref{sec:key:5.6.2}).

In other instances, some of the more astute early PN grammarians liberated aspects of their grammatical description from the stranglehold of the traditional grammatical framework and innovated pre-theoretical descriptive solutions and invented new terminology and schemata to better convey foreign PN structures.

\section{Philological methodology}
\label{sec:key:2.3}

By reimagining the authors' logic when trying to capture previously undescribed structures, this study articulates a philological methodology for optimally reclaiming the structure of the languages, which was the target of the early grammarians' descriptive attempts. 

Just as histories of Australian linguistic description have focussed on the developing understanding of phonology, existing studies treating the methodology of reclaiming Australian languages from historical sources (e.g., \citealt{thieberger_paper_1995}) have tended to concentrate on phonological rather than morphosyntactic content. The emphasis on phonology is partly due to the dearth of grammatical analysis in comparison with more easily collected vocabularies, a fact noted before the end of the nineteenth century \citep[xvi]{fraser_australian_1892}.\footnote{ The greater number of vocabularies are, however, unevenly distributed across Pama-Nyungan languages and consequently there are also unsatisfactory lexical records of many languages.}  The necessary dependence of language reclamation on more commonly available wordlists has sidelined the importance of the systematic study of the early representations of Pama-Nyungan morphosyntax. Koch’s (\citeyear{koch_george_2011}) treatment of G. A. Robinson’s linguistic legacy demonstrates the kind of “philological methods … that need to be done by anyone engaged in the recovery of language material known only through old sources” (\citealt[141]{koch_george_2011}). While this work is rare in showing how morphological data can be extracted from early wordlists (\citealt[157]{koch_george_2011}), Koch nevertheless concentrates on the methodology of reclaiming phonology from pre-phonemic orthographies. Relatively little has been written about the method of retrieving morphological data from early documents.

Recognising the morphosyntactic categories that are likely to be overlooked or disguised in the early sources requires awareness of the grammatical divergence between the language under investigation and the languages informing the inherited descriptive framework. When considering the nature of the looking glass through which PN structures were viewed, it is helpful to recognise two interrelated processes by which morphosyntactic categories are likely to be obscured when cast in the schemata of traditional grammar. \footnote{ Historically it has been the grammatical traditions of Classical Greek and subsequently Latin into which the structures of other languages have been framed, although a tradition of Sanskrit description has similarly resulted in the imposition of an inappropriate grammatical tradition on the non-Indo-European languages from the Indian subcontinent. The problem was identified by R. Caldwell of the London Missionary Society in a grammar of Dravidian (\citealt[203]{caldwell_1861}).}   

The first occurs when categories that were \textit{not} present in the target language were nevertheless described because they were a standard feature of the received traditional framework. This results from the failure of a grammarian to “divest their minds as much as possible of preconceived ideas” \citep[vi]{schurmann_vocabulary_1844} and consequently from the unnecessary and inappropriate specification of grammatical categories (\sectref{sec:key:2.3.1}).

The second process occurred when grammatical structures in the target language were overlooked because the traditional grammatical schema did not readily describe them. It occurred when authors failed to formulate the description around the “natural principles of the language'', as \citet[x]{threlkeld_australian_1834} had advised (\sectref{sec:key:2.3.2}).

\subsection{The description of unnecessary categories}
\label{sec:key:2.3.1}

Many of the categories of traditional grammar that are unnecessary when applied to PN languages are among those that establish SAE as a \textit{Sprachbund} \citep{haspelmath_european_2001}. Haspelmath describes twelve morphosyntactic Europeanisms that are found in “the great majority of core European languages” but “are not found in the majority of the world’s languages” \citep[1493]{haspelmath_european_2001}. Many are not common to PN morphosyntax,\footnote{Some PN languages, including Diyari have a passive construction (\citealt[530]{dixon_australian_2002}). }  and four are pertinent to this examination of the early description of PN morphosyntax:

\begin{enumerate}
    \item Both definite and indefinite article
    \item The comparative marking of adjectives
    \item Passive constructions\footnote{Haspelmath describes the passive construction formed with a participle as defining of the SAE \textit{Sprachbund} (\citeyear[1496--1497]{haspelmath_european_2001}). Passive constructions are reasonably common in the world’s languages (\citealt{siewierska_passive_2013}).}
    \item Relative clauses signalled by relative pronouns
\end{enumerate}

The corpus grammarians' treatment of these features tells much about the authors' approach and perspective. The first three are discussed below and the description of “relative pronouns” is discussed within the relevant sections.

The corpus grammars are standardly arranged under chapter headings describing the word classes, or parts of speech that are functionally motivated in SAE languages. These provided the “fundamental organisational principle of descriptions” \citep[187]{koch_r_2008}. In the more detailed grammars up to eleven parts of speech were given in roughly the following order: Articles, Nouns, Adjectives, Numerals, Pronouns, Verbs, Adverbs, Pre/post-positions, Particles, Conjunctions and Interjections (\figref{fig:key:2-12}). Functionally motivated classes of words in a modern grammar of a PN language are likely to include: nominal classes (common nouns, proper nouns, locational nouns, adjectives, pronouns, demonstratives), verbs, adverbs, particles, conjunctions and interjections \citep[271]{dixon_languages_1980}. Chapter headings detailing SAE word-classes such as “articles” and “postpositions” given in many early PN grammars are particularly conspicuous instances of unnecessary specification.

\begin{figure}
\includegraphics[width=.8\textwidth]{figures/Figure 12.png}
\caption{Kramer's 1931 copy of C. Strehlow's grammar of Western Arrernte \citep{kramer_ernst_1931}}
\label{fig:key:2-12}
\end{figure}

The common inclusion of case forms termed “vocative” in the early grammars similarly results primarily from a need to fill a slot in the traditional schema. In Latin the vocative form differs from the nominative in the singular of the second declension and is consequently included in case paradigms for structural reasons. Forms that are labelled “vocative” in the early grammars and included in case paradigms do not mark a case relation, and vocative expressions are not considered to mark case in Australian languages \citep[8]{blake_case_2001}. In Arrernte, the suffix \textit{–aye} termed “vocative” by the early grammarians (\figref{fig:key:9-192}) is a more general emphatic morpheme \citep[353]{wilkins_mparntwe_1989}. Similarly, in Diyari, the missionaries' vocative forms (\figref{fig:key:8-180}) described by \citet[no pag.]{koch_untitled_1868} as “carelessness of speech”,\footnote{“Nachlässigkeit des Sprechens'' \citep[no pag.]{koch_untitled_1868}} are currently analysed as a shouted speech phenomenon \citep[39]{austin_grammar_2013}.

Discussion of comparative and superlative adjectival degrees under the prescribed traditional heading “the comparison of adjectives”(\figref{fig:key:2-13}) epitomises the sway of traditional grammar over early Pama-Nyungan description.

\begin{figure}
\includegraphics[width=.8\textwidth]{figures/Figure 13.png}
\caption{Kennedy’s paradigm of Latin comparative and superlative degrees of comparison of adjectives \citep[132]{kennedy_public_1879}}
\label{fig:key:2-13}
\end{figure}

The existence of a particle in the SAE comparative construction: “X is bigger than Y” and the marking of the adjective for comparative and superlative (the \textit{groß, größer, am Größten} paradigm in German) \citep[1499, 1501--1502]{haspelmath_european_2001} stand in contrast to the constructions presented in many PN languages. The semantics of comparison and extreme are often conveyed lexically rather than through morphology: 

\ea
\gll nhandru nguyama-yi marla ngakunga \\
3\textsc{sg}.\textsc{f}.\stockcaps{ERG} know-\stockcaps{PRES} more 1\textsc{sg}.\stockcaps{LOC}\\
\glt `she knows more than me' \\
\citep[112]{austin_grammar_2013}
\z

Alternatively, the semantics of comparison and extreme may be conveyed through the syntactic frame of juxtaposition (X big, Y little). The description of a morphological process was sometimes falsely specified in early descriptions of Australian languages. 

In descriptions of Arrernte, for example, T. G. H. \citet[86--87]{strehlow_aranda_1944} – following C. Strehlow (\citeyear[28--30]{strehlow_untitled_1931}, \citeyear[8]{strehlow_notitle_1910}), following \citet[7]{kempe_grammar_1891} – showed derivations of the adjective with the morpheme \textit{–alkura} to denote the comparative and \textit{-indora} to denote the superlative. Both are currently analysed as ``free-forms and not as suffixes'' in the Western Arrernte dictionary \citep{breen_introductory_2000}. The first, \textit{alkwerre} functions as a quantitive adjective, translated as `more'. The second, \textit{nthurre} is listed by \citet[587]{wilkins_mparntwe_1989} as a word meaning true, proper, exact, real, which “in modifying adjectives means `very'”. The structures are not formally equivalent to the \textit{–er} and \textit{–est} of English. All early authors expressed some awareness that the forms they represented as suffixes attached to the adjective were in fact just words meaning `more' and `very' but nevertheless represented their structure in order to conform to the prescribed traditional schema. While an adjective in an Arrernte NP that is modified by \textit{alkwerre} `more' has a similar semantic function to the SAE morphological comparative construction, an adjective modified by \textit{nthurre} is not correctly described as marking superlative degree.

Similarly, the Lutheran missionaries Poland \& Schwarz’s description of a morphological superlative in Guugu-Yimidhirr (\citeyear[no pag.]{schwarz_koko_1900}) repeated by Roth (\citeyear[26]{roth_structure_1901}; \sectref{sec:key:10.1.4}) represented the form \textit{kana} as a superlative adjectival prefix. The form \textit{kana} (\textit{ganaa}) is currently listed as a particle meaning `alright', `OK' \citep[169]{haviland_guugu_1979}. Constructions formed with \textit{ganaa} are not functionally equivalent to the superlative category predicted by the traditional grammatical framework:

\ea
\gll Nyundu ganaa? Ganaa \\
2\textsc{sg}.\stockcaps{NOM} \stockcaps{OK} \stockcaps{OK}\\
\glt `how are you [=are you OK]? I’m fine [=OK]' \\
\citep[152]{haviland_guugu_1979} 
\z

\ea
\gll Nyundu mayi buda-y ganaa, ngali dhada-a \\
2\textsc{sg}.\stockcaps{NOM} food-[\stockcaps{ACC}] eat-\stockcaps{PAST} \stockcaps{OK} 1\textsc{dl}.\stockcaps{NOM} go-\stockcaps{NONPAST}\\
\glt `When you have eaten [enough?], then we’ll go' \\
\citep[152]{haviland_guugu_1979}
\z

Yet \citet[26]{roth_structure_1901} translated \textit{ganaa} preceding the adjective meaning `weak' as `weakest' (\figref{fig:key:2-14}). Haviland does not describe this construction. It appears that it was constructed as a morphological and semantic equivalent to the SAE superlative by either Roth or by Poland \& Schwarz.

\begin{figure}
\includegraphics[width=\textwidth]{figures/Figure 14.png}
\caption{Roth’s description of the superlative degree of adjectives (Guugu-Yimidhirr; \citealt[26]{roth_structure_1901})}
\label{fig:key:2-14}
\end{figure} 

Based on Roth’s analysis of Guugu-Yimidhirr, the idea that some Australian languages have a morphological superlative later entered into mainstream Australian linguistic thought (\citealt[165]{elkin_nature_1937}; \citealt[55]{capell_structure_1937}).

In 1874, G. Taplin (\sectref{sec:key:7.3}) circulated a questionnaire in order to gather information about Aboriginal customs and languages. The material was published in \textit{Manners, customs and Languages of the South Australian Aborigines gathered from Enquiries made by authority of South Australian Government, Edited by the late Rev. G. Taplin, of Point Macleay }(\citeyear{taplin_1879a_nodate}). Fourteen of the questions he asked relate to language (\citeyear[6]{taplin_1879a_nodate}). Several of Taplin’s respondents provided no answer to questions eliciting linguistic data. In answer to question 31, which requested information about nominal declension, Police-trooper Provis, for instance, reported that “Ku-ka-tha’: “is altogether too crude and meagre to admit of these nice grammatical distinctions” (Provis in \citeyear[97]{taplin_1879a_nodate}). The nature of the questions Taplin formulated is, however, revealing of his own developing understandings of PN structure and of the type of issues that he perceived as important or in need of clarification. Six questions sought information about potentially unnecessary categories:

\begin{enumerate}
    \setcounter{enumi}{29}
    \item Has the language any articles? If so, what are they? Are forms of the pronouns used as articles?
    \setcounter{enumi}{34}
    \item Is there any gender to pronouns?
    \setcounter{enumi}{37}
    \item How is the passive form of the verb constructed?
    \item Is there any verb “to be,” or “to have” in the language?
    \item Is the letter s used in the language, or f, v, z ?
    \item What are the numerals? How high can a native count in their own language?
\end{enumerate}

Grammatical gender,\footnote{The term “gender” is used here to refer to both noun class systems and pronominal gender systems, which are taken to exhibit grammatical gender on the grounds of agreement because “the control of anaphoric pronouns by their antecedent (the girl … she) [is seen] as part of agreement” (\citealt{corbett_number_2013}). Note that this usage differs from that used by \citet[452]{dixon_australian_2002}.}  about which Taplin enquired, is another feature inherent to the traditional descriptive framework but generally not applicable to the description of PN languages. Since the term “gender” is used within traditional grammar to describe both the category held by nouns with which other word classes agree as well as the lexical marking for biological gender (e.g., \citealt[10--11]{gildersleeve_latin_1895}; \citealt[19--32]{ramshorn_lateinische_1824}), the category gender is maintained in a body of early Pama-Nyungan grammars of languages with no system of gender. 	

Some of the earliest grammarians working in South Australia were aware that these categories were not pertinent to their analyses. Many of these features are among a list described as absent in Australian languages by Schürmann (\citeyear[250--251]{schurmann_aboriginal_1846}; \sectref{sec:key:6.2.1}) and by Moorhouse (\citeyear{moorhouse_vocabulary_1846}; \sectref{sec:key:6.4.1}). Yet they continued to be included in PN descriptions for the following century.

It is, however, likely that the classical grammatical rubric was the preferred vehicle to carry PN structures, for reasons that were not linguistic in nature, even when a grammarian realised that it was not a good morphological fit. The traditional framework may have been chosen for the utilitarian reason that it was the simplest and most easily understood way to convey grammatical structure. Whether employed by missionaries in unpublished MS grammars intended for circulation only within the mission, or by grammarians writing for international publication in prestigious philological journals, the framework was the only available framework that could render the material immediately accessible to a linguistically trained reader. Such pedagogical motivation for engaging the traditional framework has previously been suggested for the missionary grammatical analysis of Polynesian languages: 
\begin{quote}
    It is not … certain that it always and only was ignorance and lack of insight into the grammatical structure of Polynesian languages that dictated the descriptive solutions of the missionary grammar. There are several indications that there were other and more pedagogical reasons behind their choices (\citealt[109]{hovdhaugen_beginning_1993}). 
\end{quote}

Further, it has been argued that when describing Awabakal, Threlkeld was motivated to show that the language was spoken by intelligent and sophisticated people (\citealt{roberts_language_2008}). Many of the corpus grammarians stressed that the structure they described evinced intelligence on the part of the speakers (\citealt[iv]{teichelmann_outlines_1840}; \citealt[24]{kempe_grammar_1891}). Missionary-grammarians appear to have also sought to elevate the status of Aboriginal languages not only by drawing attention to grammatical structure but also by showing that the Aboriginal language was capable of being construed using the same terminology and framework as Classical Greek and Latin, the languages placed at the pinnacle of human potential linguistic achievement. When, for example, T. G. H. Strehlow – son of missionary C. Strehlow – introduced “the verb” in a grammar of Arrernte (\citeyear{strehlow_aranda_1944}) he explained that “the tenses and moods given below all bear familiar and easily intelligible names not very much different from those borne by the moods and tenses in Latin and Greek.” This tendency has similarly been observed in colonial Africa \citep{gilmour_grammars_2006}, and elsewhere.

The traditional grammatical framework also continued to be engaged when a grammarian recognised that the inherited framework was inappropriately designed to capture the morphosyntactic structure of the target language. Early grammarians continued to present the traditional grammatical schema even when providing evidence that the predicted categories did not apply to the described language. Schemata were sometimes presented as vacuous headings with an accompanying apologetic note that the feature had not been found. Grammarians commonly note the lack of grammatical gender in their introductions to the “substantive” where the category “gender” is conventionally given (\citealt[4]{teichelmann_outlines_1840}; \citealt[10]{meyer_vocabulary_1843}; \citealt[2]{schurmann_vocabulary_1844}; \citealt[7]{taplin_grammar_1880}; \citealt[2]{kempe_grammar_1891}; \citealt{Strehlow_n.d}). Similarly, under the heading “the article”, \citet{symmons_grammatical_1841} and Roth (\citeyear{roth_ethnological_1897}, \citeyear{roth_structure_1901}) state that there is none.

\subsubsection{The description of foreign PN structures}
\label{sec:key:2.3.1.1}

An array of “foreign” PN features that the traditional grammatical framework could not readily deal with was described by the early grammarians. The description of categories that were not integral to the inherited descriptive framework required a grammarian to either extend himself beyond what was descriptively familiar or to borrow techniques innovated by previous grammarians. Early grammarians were most likely to look for guidance from their predecessors' descriptions when venturing to describe structures that the traditional grammatical framework was powerless to convey. These areas of the grammar for which the early missionary-grammarians were theoretically and terminologically ill equipped provide particularly rich fields of historiographic study, evidencing lineages of descriptive practice. They include: 

\begin{itemize}
    \item The marking and function of the ergative case (throughout)
    \item The large case systems of Pama-Nyungan languages (throughout)
    \item Systems of bound pronouns 
    \item The juxtaposition of constituents in inalienably possessed phrases (\sectref{sec:key:8.7.2})
    \item The inclusive and exclusive pronominal distinction (\citealt{Stockigt2017})
    \item The morphological marking of clause subordination
\end{itemize}
    
Note here that Hebrew has both “separate” pronouns and “pronominal suffixes” (\citealt[105--109]{gesenius_gesenius_1910}). It might therefore be expected that grammarians of Australian languages who were familiar with the structure of Hebrew would be better placed to describe the bound pronouns of some Australian languages. 

Five of the questions Taplin (\citeyear[6]{taplin_1879a_nodate}) circulated sought information about points of PN grammar for which the traditional framework is deficient: 

\begin{enumerate}
    \setcounter{enumi}{10}
    \item What is the system of kinship in the tribe? Give names for following relationships.
\end{enumerate}

A note to this item states: “It is also desirable to discover whether there is not a slight variation of the word according as it is borne or attributed to the speaker; for instance, a variation for my father, your father, his father, \&c”. 

\begin{enumerate}
    \setcounter{enumi}{30}
    \item What is the form of the declension of nouns? In the case of a word for “man,” how do they say “of a man,” “to a man,” “by a man” [as an agent], “by a man” [situated near a man], “from a man,” or “a man” objectively? (Parentheses original).
    \item Is there a dual form of the noun – i.e., is there not only a word for man and men but a word for two men?
    \setcounter{enumi}{33}
    \item Is there an abbreviated form of the pronoun, for the sake of euphony, used in composition?
    \setcounter{enumi}{35}
    \item Has the verb any indicative mood? Or has the verb only a participle construction? Is the form in which the verb is used in the indicative the form in which the same word is used adjectivally? Give a specimen.
\end{enumerate}

Thirty-five years after Threlkeld had written the first grammar of an Australian language (\citeyear{threlkeld_australian_1834}), these questions show that Taplin was aware that Australian languages were likely to exhibit pronominal morphology sensitive to kinship (Qu. 11) (\sectref{sec:key:6.2.1.4}), systems of bound pronouns (Qu. 34), and that he was fairly well informed about the type of arguments that were likely to be morphologically marked, and about ergative morphology (Qu. 31). See how ergative forms are elicited with the prepositional phrase: “by a man'', which Taplin anticipates will be different from the locative or comitative form elicited with the phrase “by a man [situated near a man]''.

The inclusion of these questions seeking information about foreign structures marks a change in perspective from both Moorhouse’s and Schürmann’s earlier typologies of Australian languages, which were essentially deficit models, listing features that the languages lacked.

Taplin’s elicitation of examples of verb participles in question 36 relates to a particular analysis of syntax given by missionary \citet{meyer_vocabulary_1843} in Ramindjeri (\sectref{sec:key:6.1.2.7}), which Taplin subsequently assumed in descriptions of Ngarrindjeri (\citeyear{taplin_vocabulary_1867}; \citeyear{taplin_notes_1872}; \citeyear{taplin_grammar_1878}).

Note that dual number is not strictly a deficient category because it occurs in Homeric and Classical Greek and in Sanskrit and is reconstructed for Proto-Indo-European. It is described in all of the corpus grammars. Dual pronouns were shown by \citet[30]{dawes_grammatical_1790} and were described in Threlkeld’s earliest work (\citealt[4--8]{threlkeld_specimens_1927}; \sectref{sec:key:3.2}), published the same year as W. Humboldt’s treatise \textit{Über den Dualis} (On the dual form; GS Vol. 6, S. 4–30), which surveyed the occurrence of dual morphology in known languages. \citet[276]{prichard_1847} noted that both Australian and Polynesian languages had ``three numbers, singular, dual, and plural''. Threlkeld was familiar with the morphological marking of dual number on pronouns from his knowledge of Polynesian languages. In 1834 he described dual number as a regional feature: “[I]n this part of the hemisphere, all the languages in the South Seas in common with New South Wales, possess a dual number, and so essential is it to the languages, that conversation could not be carried on without this form of speech” (\citealt[viii]{threlkeld_australian_1834}).

\subsection{Difficulties in describing the case system}
\label{sec:key:2.3.2}

Taplin’s \citeyearpar[6]{taplin_native_1874} method of seeking case forms in Question 31 of the circular he distributed highlights the difficulties associated with identifying and naming Australian case functions that are evident throughout the corpus. 

Taplin does not seek the nominative form because it was known to be the unmarked root. Only the accusative and ergative forms are requested using traditional grammatical terminology, viz. “objectively” and “as an agent''. For other cases Taplin seeks a form that translates an English prepositional phrase. The translation of PN inflected case forms with prepositional phrases in English, German, and sometimes Latin, is common to the corpus, and reflects the partially analytic case systems of SAE languages \citep[9]{blake_case_2001}.

Case forms marking functions carried synthetically in SAE languages were readily assigned case labels, whereas case forms marking functions carried analytically in SAE languages tended to be translated with prepositional phrases and described as prepositions.

Both strategies depicted case forms in a way that was potentially ambiguous, and both were problematic in assuming an isomorphic correspondence of case functions between SAE and PN case systems. The methodological limitations of representing the function of PN case forms has serious consequences for the reclamation of case systems based on historical materials. 

Many of Taplin’s prepositional phrases could have elicited more than one case suffix. The prepositional phrase “of a man'', for example, would have elicited either an NP marked for possessive case or for dative case in a language like Awabakal, in which this range of functions is formally differentiated (see \tabref{tab:chap3:Awabakal}).

\ea
\gll {kuri-kupa} \textit{or} {kuri-ku}\\
{man-\stockcaps{POSS}} {} {man-\stockcaps{DAT}}\\
\glt {`of a man'} {} {`of a man'}
\z

Further, Taplin’s prepositional phrases are likely to have elicited differently marked NPs depending on the predicating verb. For example, “to a man” used with the verb “to give” would have elicited an accusative NP in a language like Arrernte (\citealt[169]{wilkins_mparntwe_1989}; \citealt[294]{henderson_topics_2013}) in which the second argument of this di-transitive verb stands in accusative case: 

\ea
\gll artwe-nhe\\
man-\stockcaps{ACC}\\
\glt `to a man'
\z

But “to a man” used with a verb of motion may have elicited an allative form: 

\ea
\gll artwe-werne\\
man-\stockcaps{ALL}\\
\glt `to a man'
\z

In some languages, like Arrernte, the form marking allative function would have been dedicated to marking this function alone. In other languages (see \citealt[60]{blake_case_1977}), the allative form might also have been used to translate “at a man'', “for a man” and “of a man'', if the allative case showed syncretism for locative, dative, and genitive functions respectively. Taplin’s prepositional phrases used with the same verb in different languages would also have elicited NPs marking the same role but standing in different cases. Used with the verb “to speak'', for example, “to a man” would have been translated using a locative NP in Diyari (\citealt[131]{austin_grammar_2013}) but a dative NP in Arrernte (\citealt[179]{wilkins_mparntwe_1989}). 

The organisation of PN case systems varies between languages and consequently “[c]omparing cases across languages is problematic” (\citealt[155]{blake_case_2001}). The functional range marked by a suffix designating a particular case-label in one language may only partially overlap with the functional range of a suffix bearing the same label in a different language. 

Another factor contributing to the difficulty in describing case is a lack of clarity about the primary function that should attract a certain case label. Traditional case labels cannot be accepted at face value (\citealt[155]{blake_case_2001}). This is true especially of the cases labelled “dative” and “ablative”(\sectref{sec:key:3.3.3} \& \sectref{sec:key:5.4.3}), both of which mark a diverse range of functions in Latin (\citealt[218--230, 246--265]{gildersleeve_latin_1895}; \citealt[157--162]{blake_1994}). That the particular function that early grammarians named “dative” and “ablative” differ had ramifications for the representations of Australian case systems.

\subsection{Appropriation of the traditional framework}
\label{sec:key:2.3.3}

While maintaining the traditional grammatical framework, early grammarians employed techniques by which the traditional framework was subverted and used to construe foreign PN “peculiarities''. Section headings inherent to the traditional schema which accommodated Europeanisms that were not to be found in PN languages provided a vacant schema into which foreign structures for which the traditional framework was deficient could be conveyed. This occurred both when it was strikingly apparent that the two structures bore no formal equivalence, as well as when a grammarian was unaware that the described PN structure was not equivalent to SAE structure associated with the schema in which it was shown. In these ways foreign structures were colonised by the traditional framework.

For example, the allative case suffix tended to be under-specified in the corpus grammars because allative function is not marked by the morphological case systems of the languages in which the missionaries were trained. Diyari grammarians, however, exemplified allative case marking under the heading “correlative pronouns'', a morphological category inherent to the descriptive framework of Classical Greek (\figref{fig:key:2-10}) but unnecessary when applied to PN languages.\footnote{Correlative pronouns or “correlative pronominal adverbs” are sets that correspond with one another in both form and meaning, for instance, “whither?” and “hither'', “whence?” and “hence''. }  Each Diyari grammarian (\citealt{koch_untitled_1868}; \citealt[8]{schoknecht_grammar_1947}; \citealt[28]{flierl_dieri_1880}; \citealt[18]{reuther_three_1981}) provided the following pair (\ref{ex:key:2-8}, \ref{ex:key:2-9}) under the heading “indefinite correlative pronouns''. They showed the marking of allative and ablative case on the spatial location nominal \textit{yerla} `elsewhere'. Spatial locational nominals are a small, closed class of nouns which inflect only for allative \textit{–nhi} and ablative \textit{–ndru} cases and are unmarked in locative case (\citealt[41]{austin_grammar_2013}; \citealt[54--56]{austin_grammar_2013}). There is, however, no suggestion in these early sources that the suffixes marking these cases were pronominal.

\ea\label{ex:key:2-8}
\gll {Jerlauandru}\\
{irgend woher, von weit her [from anywhere, from far away]} \\
\glt \citep[no pag.]{koch_untitled_1868} \\
\gll yarla-wa-ndru\\
elsewhere-\stockcaps{DIST}-\stockcaps{ABL}\\
\z

\ea\label{ex:key:2-9}
\gll {Jerlauanni}\\
{irgend wohin von Entfernung [to anywhere distant]} \\
\glt \citep[no pag.]{koch_untitled_1868} \\
\gll yarla-wa-nhi\\
elsewhere-\stockcaps{DIST}-\stockcaps{ALL}\\
\z

Other instances include the construal of ergative morphology as marking passive (\sectref{sec:key:7.1.1.1}) constructions, the depiction of bound or enclitic pronouns as verbal inflections for number and person (\sectref{sec:key:5.5}), and the description of deictic forms as third-person “neuter” pronouns (\citealt{threlkeld_australian_1834}; \sectref{sec:key:3.3.2.1}).

Foreign PN structures were slotted into the traditional grammatical framework under traditionally prescribed section headings when the PN structure was perceived to be \textit{functionally rather than structurally} equivalent to the SAE structure traditionally described under that heading. For example, under the heading “passive verbs'', grammarians frequently noted that there were none, before going on to describe their perception of the way the European passive function is carried in PN languages. Grammarians commonly described active clauses with no overt subject as passive constructions. In the first grammar of Diyari, for example, \citep[no pag.]{koch_untitled_1868} stated, “Passive verbs are missing in Diari”, before explaining that if one wanted to say “my father was slain'', one would place \textbf{aperi nakani} “my father” in accusative case and leave out the subject \textbf{tarnalia} “they”.

Similarly, Taplin also described a transitive clause with an elided agent as passive (\figref{fig:2:15}).

\begin{figure}
\includegraphics[width=.8\textwidth]{figures/Figure 15.png}
\caption{Taplin’s exemplification of “passive” constructions \citep[17]{taplin_grammar_1878}}
\label{fig:2:15}
\end{figure}

\ea
\gll {Ngan} {lakkir}\\
{I am} {speared} \\
\glt \citep[17]{taplin_grammar_1878} \\
\gll Ngan lak-ur\\
1sg\stockcaps{ACC} spear-\stockcaps{PAST}\\
\glt `speared me'
\z

Much later, \citet[20]{roth_structure_1901} also presented a syntactically equivalent Guugu-Yimidhirr construction (\figref{fig:2:16}) as “passive” within a discussion of verbs after stating: “There is no special form of the verb to express the Passive but it is rendered by the person passive being placed in the objective case, the individual whence the action proceeds being understood”. Threlkeld also showed a transitive clause with an elided agent as the “passive”(\citeyear[28]{threlkeld_australian_1834}), as did \citet[xx]{symmons_grammatical_1841} and \citet[494]{hale_languages_1846}. The constructions carry the same discourse function as European passives in de-emphasising the agent.

\begin{figure}
\includegraphics[width=.8\textwidth]{figures/Figure 16.png}
\caption{Roth’s illustration of passive constructions (Guugu-Yimidhirr; \citealt[20]{roth_structure_1901})}
\label{fig:2:16}
\end{figure}

Another prominent instance where a foreign PN structure is presented as a prescribed traditional category by virtue only of its \textit{functional} equivalence is found in the Lutheran missionaries' exemplifications of reciprocal and reflexive verb morphology under the heading “reciprocal and reflexive pronouns''. Flierl and Meyer (\citeyear[26]{flierl_christianieli_1880}; \figref{fig:2:17}), for example, gave the following Diyari reciprocal and reflexive constructions showing valency altering derivational verbal morphology under the heading “pronoun'':

\ea
\gll {Ngani} {demateraia}\\
{Ich} {schneide mich} \\
\glt \citep[26]{flierl_dieri_1880} \\
\gll nganhi dama-tharri-yi\\
1\textsc{sg}.\stockcaps{NOM}   cut-\stockcaps{REFL}-\stockcaps{PRES}\\
\glt `I cut myself'
\z

\ea
\gll {Ngaiani} {antjama laia}\\
{Wir} {lieben einander} \\ 
\glt \citep[26]{flierl_dieri_1880} \\
\gll ngayani ngantya-mali\\
1\textsc{pl}.\stockcaps{EXCL} love–\stockcaps{RECIP}\\
\glt `We love one another'
\z
    
\begin{figure}
\includegraphics[width=.8\textwidth]{figures/Figure 17.png}
\caption{Flierl’s discussion of reflexive and reciprocal verb morphology under the word-class heading “pronoun”(Diyari; \citealt[26]{flierl_dieri_1880})}
\label{fig:2:17}
\end{figure}

Lastly, when describing “the article'', T. G. H. Strehlow wrote:
\begin{quote}
    While there is no separate word in Aranda corresponding to the English “the” the French “le” or “la” or the German “der,”“die,”“das” the third-personal pronoun (era) is very frequently put after the noun in an Aranda sentence, and then undergoes a change in meaning until its force is practically identical with that of the definite article in modern European languages. \citep[57]{strehlow_aranda_1944}
\end{quote}

In referring to the French and German articles, Strehlow indicates that his knowledge of European vernacular languages informed his expectations about language structure as much as did the Classical languages.

That grammarians described PN grammatical categories in sections of the traditional grammatical framework that accounted for SAE structures that were perceived to be \textit{functionally rather than structurally} equivalent helps to account for the range of functionally diverse content that is given under the word-class heading pre/post-position in the early grammars. Many grammarians are unperturbed by the contradiction in describing what they represent as “affixes” as the \textit{class of word} “preposition''. For example, as a note attached to his declension of Diyari nouns, \citet[no pag.]{koch_untitled_1868} wrote: “Additional mention must also be made of various adjuncts, so-called postpositions, being attached instead of the ending. However, we will deal with those separately as a particular class of words”.\footnote{Original: ``Nachträglich sei noch bemerkt, daß verschiedene Anhängsel an Sustantive s.g. Postpositionen statt der Endung angehängt werden, welche wir jedoch für sich als besondere Wortklasse behandeln werden.''}  Structures that are included under the word class heading pre/postposition are those that are \textit{functionally} equivalent to SAE prepositions regardless of the structure. As \citet[13]{roth_ethnological_1897} explained: “[P]repositions, or what would correspond to them in our language, are signified in the Pitta-Pitta language by various suffixes … \textit{or by separate words}'' (emphasis added).

\section{Traditional descriptive models}
\label{sec:key:2.4}

Blevins notes that the traditional word-based model of European grammatical description:

\begin{quote}
    project[ed] morphological analysis primarily upwards from the word, and treat[ed] the association of words with paradigms or other sets of forms as the most fundamental morphological task. \citep[375]{blevins_word-based_2013}
\end{quote}

The recognition of sub-word units – roots, stems, prefixes and suffixes – as well as accounts of word-internal morpheme constituent order were not part of the Greek or Latin grammatical tradition. The word and paradigm (henceforth WP) model of description (\citealt{hockett_two_1954}; \citealt{Robins_1959}), which developed to best convey the fusional and synthetic typology of SAE languages, takes the word, rather than the morpheme, as the minimal unit of analysis.

The WP model, which is implicit to the traditional framework, was widely applied to the description of PN languages because, like categories inherent to traditional grammar, it was the only available framework.

The WP model developed to accommodate the case systems of classical European languages in which the marking for case was frequently fused with the marking for number and gender. The model is able to simultaneously represent three grammatical categories within a single word. This paradigmatic presentation of words was suited to the fusional morphology of SAE languages, where multiple categories might be carried by a single portmanteau morpheme.

Alternative descriptive models recognising word-internal constituents post-date the early description of Australian languages. \citet[383]{blevins_word-based_2013} describes A. Schleicher’s (1821--1868) analysis of \textit{wurzeln} “roots” and \textit{beziehungslauten} “inflec\-tions” – literally “relational sounds'', given in an 1859 description of Lithuanian as ``\textit{almost entirely} without precedent in the classical tradition'' (emphasis added). Note that it was in this work that Schleicher coined the term “morphology” on a biological analogy (\citeyear[35]{schleicher_zur_1859}). The term “morpheme” was not coined until 1895, by the Polish linguist J. Baudouin de Courtenay (1845--1929). 

Inspection of the corpus grammars shows that sub-word units were commonly recognised and represented by the earliest PN grammarians. Missionary grammarians in Australia, faced with the pre-theoretical challenge of describing the structure of agglutinative languages, \textit{combined} the word and paradigm descriptive model with presentations of sub-word units. Suffixes were referred to as “signs'', “terminations'', “terminating syllables'', “affixes'', “postfixes'', and “particles''. In 1838, for example, Günther presented the “terminations of cases” showing case inflections as sub-word units (\figref{fig:2:18}). In 1840 Teichelmann and Schürmann presented “affixes or terminating syllables” (\figref{fig:2:19}), which like Günther’s earlier Wiradjuri grammar conveyed the functional load carried by word internal constituents by assigning case labels to them.

\begin{figure}
\includegraphics[width=.8\textwidth]{figures/Figure 18.png}
\caption{Günther’s presentation of case suffixes as unattached morphemes (Wiradjuri; \citealt[45]{gunther_native_1838})}
\label{fig:2:18}
\end{figure} 

\begin{figure}
\includegraphics[width=.8\textwidth]{figures/Figure 19.png}
\caption{Teichelmann \& Schürmann’s presentation of case suffixes (Kaurna; \citealt[5]{teichelmann_outlines_1840})}
\label{fig:2:19}
\end{figure} 

In 1843 Meyer described bound pronouns in Ngarrindjeri in the following terms: “the inseparable [forms] are fragments of the separable pronouns, attached as affixes to other words” (\citeyear[22]{meyer_vocabulary_1843}). 

The missionaries also stated rules explaining the attachment of inflectional morphology to the roots. \citet[11]{meyer_vocabulary_1843} wrote: “The relations expressed by the Latin and Greek cases are in this language expressed by particles added to the root”. \citet[8]{roth_ethnological_1897} stated: “when the possessor is a person, \textbf{-ng-ā} is suffixed” and “the noun objective takes the suffix \textbf{–nā} in the present or past time, and \textbf{–kō} in future time”. 

Similarly, London Mission Society missionary J. Davies (1722--1855) had earlier discussed “prefixing” constituents and “adding particles” to words in a grammar of Tahitian (1851: 16[\citeyear{davies_grammar_1823}]). He did this despite also creating maximal word boundaries when representing a language in which the concept of the word is “notoriously difficult” (\citealt[108]{hovdhaugen_beginning_1993}).

The missionaries' presentations of sub-word units and their discussion of the ways in which these attached to other word-internal constituents given in a-theoretical, synchronic, pedagogical grammatical descriptions is presumably the marginal type of work leading Blevins (\citeyear[383]{blevins_word-based_2013}) to qualify Schleicher’s widely read, academically based discovery as being “almost entirely” without precedent. There were earlier precedents. 

\largerpage
It is possible that German missionaries who described PN pronominal suffixes – Günther (\sectref{sec:key:4.4.6}), Teichelmann and Schürmann (\sectref{sec:key:5.5}), and Meyer (\sectref{sec:key:6.1}) – did so by analogy with Hebrew. Gesenius' account of suffixing pronominal morphology in Hebrew was first published in German in 1813 (see ibid.: 158).

\largerpage
Further, Lutheran missionaries who trained at the Jänicke Mission Institute (\sectref{sec:key:5.1.2}) are likely to have been acquainted with methods of describing agglutinative morphology through reading the grammar of Tamil written by the Church Missionary Society missionary C. T. E \citet{rhenius_grammar_1836}. Tamil, like PN languages, is agglutinative and is largely suffixing. The language has its own classical tradition of grammatical description. Rhenius wrote:

\begin{quote}
    [I]n constructing the chapters on Orthography and Etymology [morphology], I have followed more the order of the native Grammars, than that of the European languages, because I judged it expedient to introduce the student at once to the native manner of treating the subject; and to facilitate the study of native grammars. I have, however, everywhere noticed the difference between the Tamil and the European languages. (\citealt[i]{rhenius_grammar_1836})
\end{quote}

Rhenius identified and described discrete word-internal units of meaning. Like grammarians in Australia, Rhenius discussed nominal declension (ibid.: 44--49) in terms of “terminations” attaching to the nominative form. He also discussed verb morphology in terms of “roots” and “affixes”(ibid.: 76). 

\citet{blevins_word-based_2013} accounts for the theoretical recognition by academic European philologists of models and terminology accommodating the description of agglutinative morphology. More research is required to trace the development of practices in far-flung missionary fields and to establish how innovations that were made in response to non-SAE structures were influenced by, and influenced, schools of SAE description.

\subsection{The word and paradigm descriptive model }
\label{sec:key:2.4.1}

The application of the WP descriptive model to PN languages was reasonably well suited to the synthetic character of PN languages. The WP model of grammatical description anticipated that the nominal inflectional categories of case, number and gender and the verbal inflectional categories of tense, and agreement with the number of the subject would be carried \textit{within} the word. The application of the WP descriptive model to PN structure resulted in an effective representation of an important range of Pama-Nyungan morphosyntactic data. A Diyari pronominal paradigm, for example, supplied by Missionary Homann but rearranged by Fraser (\figref{fig:2:20}) shows person on the horizontal axis, case on the vertical axis and number through separate structurally identical paradigms. A fourth category, natural gender in the third person, is also presented as a separate division of the horizontal axis (\citealt[43--44]{fraser_australian_1892}).

\begin{figure}[t]
\includegraphics[width=.8\textwidth]{figures/Figure 20.png}
\caption{Homann’s use of the Word and Paradigm model. (Diyari; in \citealt[43--44]{fraser_australian_1892})}
\label{fig:2:20}
\end{figure}

Like Homann, all the corpus grammarians used some sort of WP model to represent pronominal forms. 

But unlike the synthetic and fusional morphology of SAE, where the marking of gender, case and number is fused within single morphemes, PN nominal morphology is generally both synthetic and agglutinative. Individual grammatical categories carried within the word are generally inflected discretely. While the agglutinative structure of PN words made the earlier grammarians' analytical task easier, because the shape of a morpheme was easily associated with the marking of a single function, the wholesale application of the WP model resulted in descriptions that were unnecessarily repetitive. 

\subsubsection{Declension}
\label{sec:key:2.4.1.1}

Some early grammarians considered that the agglutinative marking of case and number of nouns did not constitute “declension''. The term “declension”, from the Latin \textit{declinere} `to lean', has its origin in a conception of the marking for “case” – from the Latin \textit{cadere} `to fall', \textit{casus} `falling' – as falling away from the nominative. 

\citet[4]{teichelmann_outlines_1840} wrote: “there is no declension of substantives in the common sense of the word”, and Moorhouse stated:

\begin{quote}
    It is difficult to determine whether the terminations in the above examples strictly form declensions, or whether they might not be considered particles added to the roots, to indicate the relations expressed. I have preferred giving them as declensions as the terminating syllable of the root is always changed. (\citealt[5]{moorhouse_vocabulary_1846})
\end{quote}

\citet[8]{taplin_grammar_1880} noted a similar concern. The idea, which was expressed solely by early South Australian grammarians, was later pursued at some length by \citet[49--50, 56]{capell_structure_1937}, who described the Arrernte case paradigm he provided (ibid.: 50) as “not quite true declension, because the same suffixes are found in the dual and plural as well as in the singular”. 

The application of the word and paradigm model to Pama-Nyungan agglutinative structure resulted in descriptions that, while reasonably accurate, are not always economical. This is less true of languages like Ngarrindjeri, in which the ordering of number and case suffixes varies according to whether the case is syntactic or non-syntactic (\citealt{meyer_vocabulary_1843}; \citealt[101]{horgen_languages_2004}), or of languages like Diyari with complicated splits in the systems of marking syntactic case (\tabref{fig:key:8-174}).

But the wholesale application of the WP descriptive model to Australian languages, such as that made by T. G. H. Strehlow to Arrernte \citep[§2.4.1.2]{Stockigt2017}, with their predictable agglutinative morphology, minimal nominal classes, and no morphophonemic variation between the stem and inflectional and derivational morphology, resulted in repetitive and superfluous paradigmatic description.
 
\subsection{Alternative descriptive models}
\label{sec:key:2.4.2}

Most early PN grammarians, especially those who wrote detailed grammars and had learnt the language relatively well, blended different descriptive models when describing PN languages. Additional to descriptive strategies imported from grammars of vernacular and Classical European languages, from grammars of Hebrew (\sectref{sec:key:5.3.2}), and from Polynesian languages (\sectref{sec:key:3.1}), missionary-grammarians developed \textit{in situ} models in direct descriptive response to previously undescribed categories, which in turn influenced later grammarians of Australian languages.

The WP model, inherited from the classical description of fusional European languages was often blended with a word-internal model befitting the agglutinative typology of the languages at hand. The presentation of case inflections independent of the stem to which they attach (\figref{fig:2:18}, \figref{fig:2:19}), and the labelling of these word-internal morphemes as markers of case, was an efficient way of conveying that these forms always mark the same case irrespective of the number and gender of the nominal. Missionary grammarians in Australia innovated \textit{pre-theoretical common-sense strategies} to better describe the agglutinative morphological structure.

Ancient constituency-based grammatical traditions are indigenous to the Indian subcontinent and culminated in Pāṇini’s fourth century BC description of Sanskrit (\citealt[375]{blevins_word-based_2013}). Blevins attributes the “extraordinary success of the Neogrammarian School” to the “consolidation of …`external' and `internal' perspectives on word structure” (\citealt[382]{blevins_word-based_2013}). He suggests that it was a knowledge of the descriptive tradition of Sanskrit that first brought syntagmatic description – statements about the relative ordering of sub-word constituents – to the Neo-grammarians' attention, writing: “While the classical word-based model would continue to serve as the basis of Greek and Latin pedagogy from Priscian’s time until the present, the Western rediscovery of Sanskrit ensured that it did not remain unchallenged as a general model of linguistic description” (\citealt[382]{blevins_word-based_2013}). Further, Rocher describes how in the nineteenth century: 

\begin{quote}
    Sanskrit was first taught to Europeans … according to the tradition of rigorous analysis by Pāṇini and other Indian grammarians and phoneticians. The identification of the root as the smallest common denominator of derived forms, vocalic alternation, derivational and inflectional suffixes, substitution rules, zeroing … were the procedures according to which Europeans learned Sanskrit from pandits. (\citealt[748]{rocher_sanskrit_nodate})
\end{quote}

Syntagmatic analysis, which had been partially introduced to the European Neogrammarians in the late nineteenth century through familiarisation with traditions of Sanskrit grammar, was later developed in the work of Bloomfield (\citealt[382--5]{blevins_word-based_2013}). Alternative descriptive models, the Item and Process (henceforth IP; \citealt[128]{hockett_two_1954}) and the Item and Arrangement (henceforth IA) models (\citealt[114]{hockett_two_1954}) differ from the WP model in \textit{not} according centrality “to the word as a fundamental unit in the grammar as a whole and as the basic syntactic unit” (\citealt[118]{Robins_1959}). The IP model developed in America in response to the structure of Native American languages in the early decades of the twentieth century. In the 1940s a refined IA descriptive model emerged (\citealt[112]{hockett_two_1954}). Hockett (ibid.: 111) writes: ``\textit{in this country} [America] \citet[27f.]{boas_handbook_1911} established IP, and Sapir (\citealt{sapir_1921}, esp. Ch. 4) elaborated it'' (emphasis added). 

Concurrent with Neogrammarians' theoretical responses to the descriptive requirements of Sanskrit, some nineteenth century grammarians in Australia not only recognised and represented sub-word units of meaning but also innovated pre-theoretical practical descriptive responses that described the relative ordering of word-internal constituents either in terms of process or arrangement, without of course using these terms. This descriptive response to agglutinative structure occurred \textit{before} Boas and Sapir were active in American linguistics. 

Examine, for example, W. E. Roth’s description of Pitta-Pitta (\citeyear[8]{roth_ethnological_1897}) in which he instigated a distinctive method of conveying the relative ordering of inflections for number and case on nominals that was more efficient than the traditional exposition of the forms in lengthy paradigms. Like \citet{meyer_vocabulary_1843} and Schürmann (\citeyear{schurmann_vocabulary_1844}; \sectref{sec:key:6.1.2.1} \& \sectref{sec:key:6.2.1.1}), Roth chose not to construct a case paradigm for nouns in different number. Under a subsequent heading “number'', Roth presented an intelligently arranged set of examples designed to clearly exemplify the relative ordering of inflection for case and number with the noun (\figref{fig:2:21}).

\begin{figure}
\includegraphics[width=.8\textwidth]{figures/Figure 21.png}
\caption{Roth’s descriptive response to agglutinative morphology (Pitta-Pitta; \citealt[8]{roth_ethnological_1897})}
\label{fig:2:21}
\end{figure} 

The clauses he gave illustrated in turn: 

\begin{enumerate}
    \item An NP unmarked for number standing in the unmarked nominative case
    \item An NP inflected for plural number standing in the unmarked nominative case
    \item An NP inflected for plural number subsequently inflected for ergative case
    \item An NP inflected for plural number subsequently inflected for accusative case
    \item An NP inflected for plural number and subsequently marked as being possessed by a third-person (\citealt[200]{blake_pitta_1979}) and then subsequently inflected for accusative case.
\end{enumerate}

This final form is:

\ea
 \gll \mbox{umma-lo uttapeukka-pityiri–wara-na}\\
 \mbox{`a mother [is striking] her children'} \\
 \glt \citep[8]{roth_ethnological_1897}\\
 \gll ngama-lu ngathapiyaka-pityiri-wara-nha\\
 mother-\stockcaps{ERG}	offspring-\stockcaps{PL}-\stockcaps{POSS}-\stockcaps{ACC}\\
 \glt `A mother  (hits) her children'\footnote{The Pitta-Pitta word for “mother” is \textit{ngamaRi} according to \citet[234]{blake_pitta_1979}. The form \textit{ngathapiyaka}, which Roth translates as “children”, is recorded by \citet[235]{blake_pitta_1979} as only being used by a man speaking of his offspring. }
\z

Roth then stated an IA-type rule regarding the marking of dual number: “the dual is expressed by \textbf{pa-koo-lā} = `two,' which is used in exactly the same manner as \textbf{pityiri }[plural]” (\citealt[8]{roth_ethnological_1897}). Such predictive syntagmatic statements about word-internal constituents made in terms of either process or arrangement occur rarely in the corpus grammars, but they do occur.

\hspace*{-.4pt}Explicit demonstrations of and statements about word-internal productive constituency are theoretically in conflict with the word and paradigm descriptive model, which takes the word as the smallest analysable unit.\footnote{Contemporary discussion of the word and paradigm model has developed different senses of the idea of the word and has introduced the notion of the lexeme (\citealt{koch_australian_1990}; \citealt{aronoff_morphology_1994}).} The earliest descriptions of PN languages show that the drive to make better presentations of agglutinative morphology prompted new methods of linguistic analysis in Australia. Unlike the similar developments in America described by Hockett (\citealt[111]{hockett_two_1954}), these developments in Australia did not impact on a theoretical development of novel descriptive models more appropriate to the description of agglutinative morphology.

\section{The nature of recorded varieties}
\label{sec:key:2.5}

When assessing the authenticity of the historical record of PN languages, it is important to keep a number of factors in mind. First, missionary-grammarians may have filled in the slots of grammatical paradigms with forms they had not actually heard, but which were anticipated through the identification of patterns. Missionary \citet[350]{gunther_lecture_1840}, for example, wrote: “It must be understood as a matter of course that these words are carried out into all the principal cases for the sake of example, though not every word may be used in every case”. Almost a century later T. G. H. \citet[171]{strehlow_aranda_1944} included forms in his verb paradigms which he described as “grammatically possible; but I cannot recollect for certain having heard them.”

It is also important to realise that the data provided in the descriptions may not always reflect “native speaker usage''. This is likely to be the case where a grammarian had not “mastered” the language. In other instances, “native speaker usage” may not have been the linguistic variety the missionary-grammarian aimed to describe. 

Many missionaries began preaching in the vernacular very soon after their arrival at the mission, at a time when their grasp of the language was at best minimal. Authors of grammars frequently stated that their understanding of the language was undeveloped, especially with regards to verb morphology (\citealt[28]{threlkeld_australian_1834}; \citealt[13]{teichelmann_outlines_1840}; \citealt[16]{schurmann_vocabulary_1844}). Dresdener missionary S. G. Klose (1802--1889), for example, described the Kaurna hymns sung at the Pirltawardli School in 1843 as being “written in the first years and because of that still in a very imperfect language. Up to now time has not permitted a revision” (in \citealt[27]{graetz_open_1988}). Similarly, the publication of service books, containing translations of hymns, the catechism and Bible stories written in the vernacular occurred at the very commencement of the missionaries' linguistic studies at the Bethesda and Hermannsburg missions. The missionary-linguists' first attempts to produce grammatically correct texts were thus frozen in the published print. It is probable that the language printed in these early mission publications was a “missionese'', i.e., a simplified variety of the Aboriginal language used by the missionaries that reflected their limited grasp of the language.

Indeed, Austin (\citeyear[246--247]{austin_grammar_2013}) assesses the language used to translate the New Testament into Diyari \textit{Testamenta Marra} (1897), thirty years after the Lutherans started investigating the structure of the language, as “clearly not typical of Diyari especially in the relative clause structure [showing] little \textit{attempt} … to produce `natural' Diyari'' (emphasis added). It is pertinent that Austin perceives that the missionaries did not \textit{try} to produce natural Diyari. The \textit{target} variety of the Diyari grammars may have been a “missionese” i.e., as used in scriptural translations, sermons, and possibly in the mission domain by Aboriginal people, rather than the language spoken away from the mission in a broad range of contexts. Missionary Siebert termed the variety used at the mission \textit{Küchen-Dieri} (Kitchen Diyari; quoted in \citealt[372--373]{kneebone_language_2005}) and described how it was structurally simple compared with the variety used away from the mission (\sectref{sec:key:8.4.2.3}).

The findings of this study suggest that the variety recorded in the corpus grammars may be dependent upon the purpose for which a grammar was written. Pedagogical grammars written at missions solely for evangelistic purpose appear to be more likely to describe a “missionese” than do grammars written for publication, which were aimed at philologically educated audiences. 

\subsection{Mission Guugu-Yimidhirr}
\label{sec:key:2.5.1}

Evidence of linguistic varieties developed by missionaries is found in the grammars of Guugu-Yimidhirr written by Lutheran missionaries at Cape Bedford mission (\citealt{schwarz_koko_1900}), and following them by Roth (\citeyear{roth_structure_1901}; \sectref{sec:key:10.1.5.2}). The ergative marking of NPs recorded in these grammars is dissimilar to the language recorded by \citet{haviland_guugu_1979}.

These early grammars of Guugu-Yimidhirr, along with \citegen{hey_elementary_1903} grammar of Nggerrikwidhi (\sectref{sec:key:10.1.3}) give the poorest account of ergative function and marking in the corpus. Despite the fact that \citet{roth_ethnological_1897} had previously described the unusual sensitivity of ergative morphology to verb tense in Pitta-Pitta on nouns (ibid.: 7) and on pronouns (ibid.: 10; see \citealt[84--90]{blake_pitta-pitta_1971}; \sectref{sec:key:10.1.5.1}), ergative morphology is not accounted for in Roth’s later grammar of Guugu-Yimidhirr (\figref{fig:2:22}) or in \citegen{schwarz_koko_1900} MS grammar of the same language. These grammars make no reference to the ergative marking of the agent of a transitive clause (\figref{fig:2:22}).

\begin{figure}
\includegraphics[width=.8\textwidth]{figures/Figure 22.png}
\caption{The absence of description of ergative morphology on nouns in Roth’s grammar of Guugu-Yimidhirr \citep[16]{roth_structure_1901}}
\label{fig:2:22}
\end{figure} 

An exceptional complexity of ergative allomorphy \citep[51]{haviland_guugu_1979} and optional ergative marking on intransitive subjects \citep[154--156]{haviland_guugu_1979} presented these early grammarians with a considerable descriptive challenge, and apparently also motivated the missionaries to invent a strategy that avoided the need to mark any noun as ergative. \citet[133]{haviland_guugu_1979} says that the missionaries “clearly never grasped basic grammatical structure”.

In \textit{all} example sentences supplied in the early Guugu-Yimidhirr grammars, a noun acting in the role of agent is \textit{always} followed by a third-person pronoun, which shows accusative alignment (\stockcaps{AS/O}). Examine, for instance, examples (\ref{ex:key:2-13}) and (\ref{ex:key:2-14}). The final constituent of the ergative NP is in each clause a pronoun, which carries the case marking for the entire NP. In example (\ref{ex:key:2-13}), the noun \textit{gaangurru} `kangaroo' and adjective \textit{warrga} `tall' are both unmarked in ergative case, but are followed by the 3sg pronoun, which takes the same form in ergative and nominative cases. Similarly, in example (\ref{ex:key:2-14}) the possessed kin-term \textit{biibi-ngadhu} `my father' is unmarked in ergative case and is followed by the 3sg pronoun.

\ea\label{ex:key:2-13}\label{ex:2:13}
\gll {Ganguru warka} {nulu goda dabelbi}\\
{`tall kangaroo} {was kicking the dog'} \\
\glt \citep[23]{roth_structure_1901} \\
\gll gaangurru warrga nyulu gudaa dhabil-?dhi\\
kangaroo-[\stockcaps{ERG}] tall-[\stockcaps{ERG}] 3\textsc{sg}.\stockcaps{ERG} tame-dog-[\stockcaps{ACC}] kick-\stockcaps{PAST}\\
\z

\ea\label{ex:key:2-14}\label{ex:2:14}
\gll Peba-ngato nulu kalka dirainggur-be uma\\
father-my he {the spear} {old man’s} gives\\
\glt `my father gives the old man’s spear' \\
\citep[29]{roth_structure_1901} \\
\gll biibi-ngadhu nyulu galga dyiirraanhgurr-bi wu-maa\\
father-1\textsc{sg}.\stockcaps{POSS}-[\stockcaps{ERG}] 3\textsc{sg}.\stockcaps{ERG} spear-[\stockcaps{ACC}] man-\stockcaps{POSS} give-\stockcaps{PRES}\\
\z

All example clauses in the early Guugu-Yimidhirr grammars showing an NP acting in the role of A have a pronoun as the final NP constituent. 
\citet[102--5]{haviland_guugu_1979} describes how in Guugu-Yimidhirr case is marked optionally on each NP constituent but usually on the last constituent. Unlike the early grammarians he gives numerous examples, such as (\ref{ex:key:2-15}) and (\ref{ex:key:2-16}), in which a noun is overtly marked for ergative case: 

\ea\label{ex:key:2-15}
\gll Dyidyii-nda ngani dyindal-y ngaabaay\\
Bird-\stockcaps{ERG} 1\textsc{sg}.\stockcaps{ACC} peck-\stockcaps{PAST} head-[\stockcaps{ACC}]\\
\glt `the bird pecked me [in the] head' \\
\citep[58]{haviland_guugu_1979}
\z

\ea\label{ex:key:2-16}
\gll Yarrga-aga-mu-n gudaa gunda-y biiba-ngun\\
Boy-\stockcaps{POSS}-mu-\stockcaps{ERG}	dog-[\stockcaps{ACC}] hit-\stockcaps{PAST} father-\stockcaps{ERG}\\
\glt `The boy’s father hit the dog' \\
\citep[57]{haviland_guugu_1979}
\z

Haviland (ibid.: 104) also describes how “if a referent of the noun phrase is an animate being, especially a human … it is normal for the whole NP to begin with an appropriate person pronoun'' (emphasis added). His account of Guugu-Yimidhirr NP structure does not describe pronominal-final NPs. In natural Guugu-Yimidhirr an animate NP in ergative case is likely to have appeared with a third-person pronoun, which would have preceded other NP constituents taking ergative case marking. It appears that the missionaries invented a variety in which a correctly marked case form of the ergative pronoun was placed in NP-final position, eliminating the necessity to mark the noun as overtly ergative.

Two other factors further suggest that the missionaries employed third-person pronouns to avoid marking case on ergative nouns. The first relates to the structure of the example clauses supplied in the early grammars. Neither of the accusative noun phrases in Roth’s examples (\ref{ex:key:2-13}) and (\ref{ex:key:2-14}) \textit{gudaa} `tame dog' and \textit{galga} \textit{dyiirraanhgurr-bi} \textit{gudaa} `old man’s spear' appears with a personal pronoun. Yet Haviland (ibid.: 104) states: “Not all animate NPs are adjoined to a personal pronoun, but most animate NPs in A, S or O function are”. The early grammarians recorded third-person pronouns in ergative NPs but not in animate accusative NPs, which Haviland says would occur in fluent native speaker usage.

The second factor indicates that \citet{roth_structure_1901} \textit{accurately recorded} a variety that had developed at the mission in which non-pronominal constituents of ergative NPs did not receive ergative morphology. In Guugu-Yimidhirr the ergative case shows syncretism with the instrumental case (\citealt[47]{haviland_guugu_1979}). Although Roth did not account for ergative marking of nouns in the role of agent, he did record a range of ergative/instrumental allomorphy in Guugu-Yimidhirr (\citeyear[30]{roth_structure_1901}; \sectref{sec:key:10.1.5.2}), which he exemplified as only occurring on NPs with instrumental function. That Roth identified a list of suffixes that mark ergative/instrumental case functions in Guugu-Yimidhirr, but only exemplified this morphology in instrumental function, and made no mention of ergativity, suggests that the variety he recorded, either independently or by reproducing the missionaries' data, was that used by the missionaries rather than fluent native speaker usage.
When assessing the early grammatical materials examined in this study, it is important to remain mindful that the target language of an early grammar might not necessarily have been the natural variety used away from the mission.

\subsubsection{Kneebone’s (\citeyear{kneebone_language_2005}) account of mission Diyari} 
\label{sec:key:2.5.1.1}

\citet[28,36]{kneebone_language_2005} argues that the phonologically altered, standardised and scripted variety of Diyari used by Lutheran missionaries and Diyari Christians at the Bethesda mission substantially influenced the structure of Diyari spoken away from the mission. She argues that the variety of language which was frozen in print, and which was used by the missionaries for proselytisation, and perhaps by a few score of literate Diyari Christians, became a prestige form, which out-lived other varieties. In support of her argument she seeks to identify features of the mission idiom that have been retained into the twentieth century language recorded by \citet{austin_grammar_1981}. Kneebone writes: 
\begin{quote}
    The question must be asked as to the nature of the Diyari spoken retrieved by Austin in the 1970s and whether this language is in fact significantly influenced by mission written forms, vocabulary and grammatical organisation. \citep[72]{kneebone_language_2005}
\end{quote}
Kneebone’s argument, however, overemphasises the importance of the mission in people’s lives and underestimates their ability to code switch as they moved between mission habitation and a semi-traditional lifestyle. While she views the demise of Diyari as consistent with the high rate of language loss in South Australia, one of the central tenets of her thesis (ibid.: 23) is that the path of Diyari to extinction, and its loss of sociolect diversity, were atypical of other languages. She writes: 

\begin{quote}
    One of the major unintentional effects of the development of a mission idiom in the context of unwritten languages was the promotion of custodianship of the language, from tribal elders and traditional oral transmission to the mission as educator and authority on the standard written form of the language. \citep[4]{kneebone_language_2005}
\end{quote}

Kneebone draws attention to two grammatical features. The first concerns the semantic organisation of a vocabulary compiled by the missionaries J. Flierl and C. A. Meyer (\citealt{flierl_wonini-pepa_1883}). Observing that the vocabulary is organised under a set of semantic head-words, which are similar to a small closed set of generic nouns described by Austin (\citeyear[44]{austin_grammar_2013}; \figref{fig:2:23}), Kneebone (\citeyear[83]{kneebone_language_2005}) suggests that “it is quite possible that these generic terms were at least in part the product of the categories set in place by the earliest missionaries”. She observes that the following categories given by Flierl and Meyer (\citeyear{flierl_1843}) – \textbf{anti} `food', \textbf{paia} `birds', \textbf{tjo tjo} `worms, reptiles' and \textbf{pita} `woods and grasses' – were later reported by Austin as \textit{nganthi}, \textit{paya}, \textit{thutyu}and \textit{pirta} respectively.

\begin{figure}
\includegraphics[width=.8\textwidth]{figures/Figure 23.png}
\caption{Generic noun classifiers in Diyari \citep[44]{austin_grammar_2013}}
\label{fig:2:23}
\end{figure} 

But the marking of a noun class through juxtaposition of a generic noun belonging to a small, closed word set with a specific noun belonging to a large open set of nominals is a widespread feature of Australian languages (\citealt[102]{dixon_languages_1980}) and many of the semantic fields designated by the nine Diyari generic nouns listed by Austin are widespread in PN languages. Diyari’s northeasterly, and grammatically close neighbour Ngamani has a similar system (Austin pers. comm. 11/08/2013) and the missionaries did not teach or support this language. Wangkangurru (\citealt[102]{hercus_grammar_1994}) has the “prefixing noun” \textit{paya} `bird', which was used almost obligatorily when referring to birds that were ancestral beings, which presumably pre-dated missionary influence. In Mparntwe Arrernte (\citealt[107]{wilkins_mparntwe_1989}), which came under no Lutheran missionary pressure, generic nouns denote similar semantic fields as those noted by Kneebone: \textit{merne} `vegetable food', \textit{kere} `animal food', \textit{thipe} `birds', \textit{tjape} `grubs' \textit{arne} `trees and bushes', \textit{name} `long grasses', etc. 

The second piece of evidence Kneebone gives to support her claim that the scripted missionary variety of Diyari survived at the expense of more natural forms is equally seriously flawed. Kneebone (\citeyear[72]{kneebone_language_2005}) suggests that in twentieth century Diyari “one would expect to find … the `overuse' of auxiliary verbs, which accords nicely with the system of tense forming auxiliaries in German”.\footnote{\textit{können}, \textit{müssen}, \textit{wollen}, \textit{mögen}, \textit{dürfen} and \textit{sollen}.}  Without clarifying what an “over-use” might look like, she suggests that the Diyari auxiliaries, which are described by \citet[13, 91--94]{austin_grammar_2013} as a set of six optional, non-lexical auxiliary verbs that interact with verb suffixes to express fine tense and modal distinctions (\figref{fig:2:24}) were used in Diyari through a process of morphosyntactic calquing from German.

\begin{figure}
\includegraphics[width=.8\textwidth]{figures/Figure 24.png}
\caption{Auxiliary verbs in Diyari \citep[92]{austin_grammar_2013}}
\label{fig:2:24}
\end{figure} 

However, Kneebone fails to notice that, while this auxiliary system is a “fairly unusual feature for an Australian language …[that] seems to have developed [from main verbs (\citealt[13]{austin_grammar_2013})] fairly recently” (\citealt[70]{austin_grammar_2013}), the auxiliary verbal structure is an areal/genetic feature that predates any European contact in the Lake Eyre Basin. Indeed, some Diyari auxiliary verbs were recorded by S. Gason, an English-speaking policeman at Lake Hope, as early as 1874 (\citealt[39]{austin_grammar_2013}), when one would expect Kneebone’s proposed process of calquing from German at the mission to be still underway. Similar auxiliary sets also occur in Ngamini (Austin pers. comm 11/08/2013), for which no calquing from German can be invoked. It is unlikely that the system of auxiliaries has diffused from Diyari to Ngamini subsequent to the development that Kneebone suggests.

\section{Historiography of the term ``ergative''}
\label{sec:key:2.6}
\label{sec:2.6}

Early worldwide usages of the term “ergative” to describe nominals marked as agents of transitive clauses in Australian languages occur in the pre-academic era of Australian language description. This fact has until recently remained unnoticed in historiographies of the term, as have the earliest worldwide usage of the term “ergative” to describe a peripheral case in Taplin’s grammar of Ngarrindjeri (\citeyear{taplin_notes_1870}; \sectref{sec:key:7.3.1}), and the earliest oppositional usages of the terms “ergative” and “absolutive” in Planert’s grammars of Arrernte (\citeyear{planert_australische_1907}) and Diyari (\citeyear{planert_australische_1908}; \sectref{sec:key:2.6.1}).

\begin{figure}
\includegraphics[width=.8\textwidth]{figures/Figure 25.png}
\caption{Planert’s case paradigm showing the terms “absolutive” and “ergative” \citep[555]{planert_australische_1907}}
\label{fig:2:25}
\end{figure} 

The historiography of the term “ergative” has altered considerably over the last decades. \citet[191]{seely_ergative_1977} followed \citet[363]{regamey_propos_1954} in assuming that the first usage of the term “ergative” was made by the German linguist and ethnographer Adolf Dirr (1867–1951), who in 1912 described the marking of the agent of transitive clauses in Caucasian languages. \citet[3]{dixon_ergativity_1994} similarly suggests that Dirr’s 1912 usage was the earliest. Manaster-Ramer (\citeyear[211]{manaster-ramer_origin_1994}) showed that usages of the term “ergative” antedate Dirr’s usage and marks the first usage of the term at Ray and Haddon’s (\citeyear{ray_study_1893}; \citealt{ray_haddon_1897}) grammatical descriptions of languages of the Torres Straits – the Papuan languages Miriam (Meryam Mir) and Daudai (Kawai) and the PN language Saibai (WTS). Note that these works were published at the time when Ray had been engaged by Haddon, who was then a Cambridge biologist, and before Ray had visited the area (\sectref{sec:key:10.2}).

As \citet[212]{manaster-ramer_origin_1994} pointed out, Ray and Haddon did not apply the term to the syntactic case to which it currently refers but rather they use the term to describe a peripheral case function. The functions of the Western Torres Straits (henceforth WTS) case, initially called “ergative” by Ray (\citealt[130--8]{ray_haddon_1897}) but which he later named “locative of motion”(\citeyear[19]{ray_linguistics_1907}) are translated using the prepositions `with, by alongside' and using the verb `to have’. The case is said to “express the doing of a thing by means of, or at the same time with, another … but the exact meaning seems difficult to define” (\citeyear[138]{ray_linguistics_1907}). The described case suffix \textit{-ay} marks the comitative case (\citealt[19]{alpher_western_2008}) which is common to the large case inventories of PN languages (\citealt[9]{dixon_introduction_1976}). \citet[213]{manaster-ramer_origin_1994} proposed that the source etymology informing Ray’s coinage of the case label “ergative” to name this case was taken from “the Latin preposition \textit{ergā}, rendered in English as `right against, next to'.”

In an article titled “Ergative Historiography Revisited” (\citeyear{lindner_ergative_2014}), Lindner’s thorough investigation of the historical records of PN grammatical description showed that usages of “ergative” as a descriptor of case antedate Ray’s 1893 description. The term “ergative” as a descriptor of peripheral case originates in Taplin’s first published grammar of the South Australian language Ngarrindjeri (\citeyear{taplin_notes_1872}) and in MS (\citeyear{taplin_notes_1870}; \citealt[188--189]{lindner_ergative_2014}). It is probable that Ray, who is known to have collated large collections of linguistic data prior to collaborating with Haddon (\citealt[183]{shnukal_at_1998}) borrowed the terminology from Taplin (\citeyear{taplin_notes_1872}; \citeyear{taplin_native_1874}; \citeyear{taplin_1879a_nodate}), or from \citet{thomas_succinct_1878} (\sectref{sec:key:7.2}).

Contrary to \citet[189]{lindner_ergative_2014}, however, the forms that Taplin labelled “ergative” (\figref{fig:key:7-141}) were not given as the case called “For” by Meyer (\figref{fig:key:7-139}). Nor can the preposition “for” be said to “correspond closely with Meyer’s English glosses” \citep[190]{lindner_ergative_2014}. Taplin’s perception of the functions of the Ngarrinyeri case suffix to which he assigns the term “ergative” are now irretrievable, the situation being confounded by now non-traceable dialectic differences in recorded Ngarrindjeri varieties (see \citealt[364--368]{stockigt_early_2015}).

After Ray’s 1897 usage, the next known usage of “ergative” is soon made by \citet[88]{schmidt_sprachlichen_1902} in a description of the Papuan languages Kai and Miriam (Meryam Mir) and again of the PN language Saibai (WTS). Schmidt’s source for Meryam Mir and for WTS is Ray (\citealt[211]{manaster-ramer_origin_1994}). In this work Schmidt engages the term “Casus ergativus” to name the syntactic case marking the agent of a transitive verb designated by the term today. He does so, however, falteringly, choosing also the terms “ablative” and “transitive nominative” to name the case in other languages in the same essay. Schmidt’s first tentative usage of the term “ergative” to name the ergative case in languages including the PN language Saibai (WTS; \citeyear{schmidt_sprachlichen_1902}) marks the genesis of the term “ergative” with modern reference. 

Manaster-Ramer has suggested that Ray’s usage of the term “ergative” to describe a syntactic case occurred through a misinterpretation by Schmidt of Ray’s 1893 usage. Remembering the term, but forgetting its original reference, Schmidt is said to have reused the term “ergative” utilising an etymology that, instead of being based in Latin, is based on the Greek root \textit{erg} `work' (verbal), thus applying the term to the subject of a transitive verb (\citealt[213]{manaster-ramer_origin_1994}). Without an alternative explanation as to how the term “ergative” evolved to have such different reference within such a short space of time when describing ergative languages spoken in the same region, Manaster-Ramer’s supposition is viable. It is, however, also possible that Schmidt re-etymologised the term “ergative” deliberately rather than accidentally.

The next usage of the term “ergative” to name the ergative case in an Australian language occurs in Planert’s Arrernte grammar (\citeyear{planert_australische_1907}) published in Berlin (\sectref{sec:key:9.2.3.2}).  As an associate of F. von Luschan (1854--1924), then the Director of the Africa and Oceania Department at the Königliches Museum für Völkerkunde, Planert’s grammar was written close to the intellectual origin of Schmidt’s coinage of the term with contemporary reference (\citealt{schmidt_sprachlichen_1902}). The term was subsequently also used by C. Strehlow’s grammar of the same language (\citeyear{strehlow_einige_1908}), written in reponse to Planert’s publication, and also published in German. The following year (\citeyear{planert_australische_1908}), Planert published a grammar of Diyari (§8.5.3). \citegen{planert_australische_1907} work also marks the earliest usage in the world of “ergative” in opposition to the “absolutive” with modern reference (\sectref{sec:key:2.6.1}).

During the following half-century the use of “ergative” to describe the ergative case in PN languages is maintained in European publications by \citet{schmidt_personalpronomina_1919}, \citet{gatti_lingua_1930} and by \citet{holmer_history_1963}. Note that Gatti, who names the case “nominativo agente” in case paradigms but describes the forms as ``ergativo'', does not refer to the work of Planert (\citeyear{planert_australische_1908}). Gatti’s well-referenced discussion of Australian languages and particular investigation of Diyari probably inherits the use of the term from Alfredo Trombetti (1866–1929), professor of linguistics at the University of Bologna, who had first used the term in 1903 (\citealt[211]{manaster-ramer_origin_1994}). 

Nils Holmer reintroduced “ergative” into the Australian literature, at first tentatively (\citeyear{holmer_history_1963}; \citeyear{holmer_attempt_1966}), but later (\citeyear[7]{holmer_notes_1971}) firmly: “Other names for this typically Aboriginal case are agentive […] agent case, active and operative”. It is \citet[59--60]{dixon_dyirbal_1972}, however, who established the term as the accepted descriptor of the case in which stands the agent of a transitive verb, at least for classes of nominals for which this function is formally differentiated from the marking of the subject of an intransitive verb. He recalled having first heard the term “ergative” from Michael Halliday after returning from field work in Queensland to University College London in 1964 (\citealt[127]{dixon_searching_1983}).

\subsection{Planert’s usage of the terms “ergative” and “absolutive” (\citeyear{planert_australische_1907}, \citeyear{planert_australische_1908})}
\label{sec:key:2.6.1}

In 1907, Wilhelm Planert (1882–post 1940), a student of linguistics associated with the Königliches Museum für Völkerkunde (Royal Museum for Ethnology) in Berlin, published a grammar of Arrernte (\citeyear{planert_australische_1907}; \sectref{sec:key:9.2.3.2}) in the well-regarded German ethnographic journal \textit{Die Zeitschrift für Ethnologie.} This Arrernte grammar appeared as “Australische Forschungen I. Aranda-Grammatik” and was followed in 1908 by a grammar of Diyari which appeared as “Australische Forschungen II. Dieri-Grammatik” (\sectref{sec:key:8.4.3}). Planert’s PN grammars of Arrernte (\citeyear{planert_australische_1907}) and of Diyari (\citeyear{planert_australische_1908}) are anomalous to the corpus in being synchronic descriptions published outside the country.

Planert had previously studied Khoisan languages spoken in German South West Africa (present day Namibia) publishing \textit{Über die Sprache der Hottentotten und Buschmänner} (\citealt{planert_1905}) and \textit{Handbuch der Nama-sprache in Deutsch-Südwestafrika} (\citealt{planert_1905}). His description of languages spoken in the German colonies continued with a study of Samoan, \textit{Einige Bemerkungen zum Studium des Samoanischen} (\citealt{planert_1906}), and the publication of his Ph.D. dissertation, \textit{Die Syntaktischen Verhältnisse des Suaheli} (\citeyear{planert_syntaktischen_1907}), which appeared in the same year as his first Australian grammar. Given this academic background, it is not surprising that \citet[245]{austin_grammar_2013} assesses Planert’s 1908 Diyari grammar as showing “a much keener awareness of linguistics and an insight into the workings of the Diyari language [than the missionaries' grammars]”. Written within the heart of German philological discourse, these works sit within a different theoretical framework from grammars of PN languages written in Australia. 

Planert’s descriptions of case in Arrernte (\citeyear[555]{planert_australische_1907}) and in Diyari (\citeyear[689]{planert_australische_1908}) present very early usages of the terms “ergative” and “absolutive” with modern reference, and are significant within the global history of ergativity in giving the earliest paired oppositional use of the terms “ergative” and “absolutive''.

Planert’s grammar of Arrernte (\citeyear[555]{planert_australische_1907}) set up the same small six case forms as the Lutheran commencing with Kempe (\citeyear{kempe_grammar_1891}; \figref{fig:key:9-193}), but altered the naming of the cases and, importantly, the conception of syntactic case. Planert presented the ergatively aligned nouns (\stockcaps{A/SO}) as standing in two cases, which he named “ergative” and “absolutive” (\figref{fig:2:26}; \sectref{sec:key:2.6.1}), stating: “the absolutive can represent our nominative and accusative cases and the ergative refers to the actor” (\citeyear[555]{planert_australische_1907}).\footnote{“[…] der Absolutiv unseren Nominativ und Akkusativ vertreten kann, der Ergativ den Täter bezeichnet” \citep[555]{planert_australische_1907}.} This section was presumably among those that \citet[551]{planert_australische_1907} described as having been “re-worked … according to linguistic principles”.\footnote{“nach sprachwissenschaftlichen Prinzipien ausgearbeitet” \citep[551]{planert_australische_1907}.}

\begin{figure}
\includegraphics[width=.8\textwidth]{figures/Figure 26.png}
\caption{Planert’s Arrernte case paradigm showing the terms “absolutive” and “ergative” \citep[555]{planert_australische_1907}}
\label{fig:2:26}
\end{figure} 

The term “absolute” had been used with grammatical reference from the sixteenth century to refer to the form of a word that is “not inflected to indicate relation to other words in a sentence” (OED 1933 vol. I: 37). The first application of the term “absolutive” to a case realised by nominals that remain unmarked in the role of subject and object was made by William Thalbitzer (\citeyear[242--243]{thalbitzer_phonetic_1904}) in a description of Kalaallisut (West Greenlandic; \citealt[198]{lindner_komposition_2013}; \citealt[231, 226]{lindner_2015}). In this work, Thalbitzer named the ergative case “relative''. 

In the Diyari work (\citeyear[689]{planert_australische_1908}), Planert presented a four-case analysis of a split syntactic case system \citep[132]{dixon_australian_2002}. He employed the terms “ergative'', “absolutive'', “accusative” and “nominative” with identical reference to that used in the most complete modern description of the language \citep[52]{austin_grammar_2013}. Planert’s paradigm of Diyari interrogatives, which shows ergative alignment (\stockcaps{A/SO}) lists no accusative case but gave an ergative and an absolutive (\figref{fig:2:27}).

\begin{figure}
\includegraphics[width=.5\textwidth]{figures/Figure 27.png}
\caption{Planert’s case paradigm of Diyari interrogatives \citep[689]{planert_australische_1908}}
\label{fig:2:27}
\end{figure} 

\hspace*{-3.8pt}Because non-singular first and second-person Diyari pronouns show accusative alignment (\stockcaps{AS/O}) and other pronouns show tripartite marking (\stockcaps{A/S/O}), pronominal paradigms necessarily listed all three syntactic cases (\citealt[689]{planert_australische_1908}; \citealt[52]{austin_grammar_2013}, \figref{fig:2:28}).

\begin{figure}
\includegraphics[width=.8\textwidth]{figures/Figure 28.png}
\caption{Planert's case paradigm of Diyari personal pronouns \citep[689]{planert_australische_1908}}
\label{fig:2:28}
\end{figure}

Under this four-case analysis, nominals showing accusative alignment (\stockcaps{AS/O}), like the non-singular first and second-person pronouns, are shown standing in either the nominative or the accusative case, and the term nominative is used in its traditional sense. Nominals showing tripartite marking, like the first and second-person singular, and third-person pronouns, are shown standing in either nominative, accusative or ergative cases. The subsequent dual reference of the term “nominative”– used to refer only to subject where the pronoun shows tripartite marking, but referring to both agent and subject where the pronoun shows accusative alignment – is a consequence of Planert’s four-case dichotomous scheme.\footnote{See \citet[163--166]{wilkins_mparntwe_1989} for a discussion of other problems arising from a four-case analysis of syntactic case in Arrernte.} 

An absolutive case had earlier been implied in the description of Australian languages by the Austrian philologist F. \citet[19]{muller_grundris_1882} when rearranging Threlkeld’s Awabakal case paradigms (\citeyear[13--16]{threlkeld_australian_1834}; \figref{fig:3:37}). The only suggestion of an absolutive case in the Austalian literature is given by Teichelmann and Moorhouse in 1841 in a rearranged Kaurna paradigm (\sectref{sec:key:5.4.1}).

Planert’s theoretical and terminological innovations, made in a German academic journal before the First World War, have not been noticed by modern linguistic theorists or historians. \citet[9]{dixon_dyirbal_1972}, who reached the same four-case analysis independently when describing Dyirbal more than sixty years after Planert, was unaware of Planert’s analysis, as was \citet{austin_grammar_2013} when reaching the same analysis of Diyari. M. \citet[112]{silverstein_hierarchy_1976} who reintroduced the term “absolutive” into grammatical descriptions of PN languages, does not refer to Planert’s earlier usage of the term.
