\chapter{Later grammars of languages from New South Wales}
\label{chap:key:4}

This chapter examines bodies of work describing languages that belonged to regions of what became New South Wales that were written after Threlkeld’s grammar of Awabakal. These languages are Wiradjuri, Gamilaraay and Minjangbal.
Section 4.1 identifies the grammars of Wiradjuri written at the Wellington Valley Mission by W. Günther (\citeyear{gunther_native_1838}, \citeyear{gunther_lecture_1840}) and discusses the existence of earlier, but now lost, grammars written by missionaries W. Watson and J. S. C. Handt. These works were written at the time when Threlkeld continued to describe Awabakal. Günther’s analyses are compared to Threlkeld’s earlier work and are assessed in terms of the similarity of the descriptive frameworks they employ, and their relative insight. Section 4.2 investigates of the comparative grammar of Wiradjuri and Awabakal written by the American ethnologist H. \citet{hale_languages_1846}, which helps establish the provenance of early works emanating from Wellington Valley Mission.

Sections \ref{sec:key:4.4}, \ref{sec:key:4.5} and \ref{sec:key:4.6} assess Günther’s analysis of Wiradjuri, Ridley’s grammars of Gamilaraay (\citeyear{ridley_kamilaroi_1875}, \citeyear{ridley_kamilaroi_1866}, \citeyear{ridley_kamilaroi_1855}, \citeyear{ridley_kamilaroi_1855-1}, \citeyear{ridley_kamilaroi_1856}) and Livingstone’s grammar of Minjangbal (1876--1886) respectively. 

\section{Descriptions of Wiradjuri and the Wellington Valley Mission (1832--1842)}
\label{sec:key:4.1}

The Wellington Valley Mission was established in 1832 by the Church Mission Society on the “newly opened” agricultural and pastoral frontier on the western side of the Great Dividing Range in New South Wales \citep[324]{bridges_church_1978}. The abandoned infrastructure of what had previously been a remote convict station (1823--1830) was chosen as the site in an area that had been encroached upon by hopeful squatters ahead of official colonial planning \citep[285]{bridges_church_1978}. Hence the earliest missionaries encountered Aboriginal people who were accustomed to Europeans \citep[326]{bridges_church_1978}. The brutal contact with lawless Europeans experienced by Wiradjuri people in the decade before the missionaries' arrival is typical of post-invasion encounter across Australia. The situation does, however, contrast with the pre-mission experience of some Aboriginal people in the south of South Australia where the nature of the encounter was at least monitored and discussed by colonial officials \citep{scrimgeour_colinizers_2007}, and with the situation in the north of South Australia (\chapref{chap:key:8} \& \chapref{chap:key:9}), where the missionaries were among the earliest European people to encroach upon Aboriginal territory. 

Wiradjuri, the language described at the mission, was spoken by several thousand people \citep[317]{Krzywieki1934} and covered an area of central New South Wales that was large by PN standards. The mission lasted for twelve years. Grammatical description of Wiradjuri was not published in Australia until fifty years after the missions' closure, when the 1838 and 1840 analyses made by missionary J. W. Günther (1806--1879) appeared in \citet{gunther_grammar_1892}. Wiradjuri grammatical material emanating from the mission was, however, published in America by H. \citet{hale_languages_1846}.

Extant original grammatical documentation of Wiradjuri is contained in two of missionary Günther’s notebooks held by the State Library of New South Wales. The grammars are dated 1838 and 1840. Both MS grammars appear to be works in progress, containing crossed-out passages and blank pages where more detail was presumably intended to be added later. As a Lutheran who initially trained at Basel Mission Institute in Switzerland, Günther received further training at the Church Missionary Society College in London and was ordained in the Anglican ministry in London in 1833. He arrived at the Wellington Valley Mission in 1837, five years after the mission’s establishment. Considerable linguistic work had been undertaken at Wellington Valley prior to Günther’s arrival \citep{bridges_church_1978} by W. Watson (1798–1866) and by Günther’s fellow Basel-trained, German-speaking missionary J. C. S. Handt (1783--1863).

Watson had entered the Church Missionary Society in Yorkshire in 1829, after working as a teacher and a grocer. He had received little training at the time of his selection to establish a Church Missionary Society mission in Australia, having been quickly ordained as deacon by the Church of England in 1830 and as a priest in 1831 in order to secure his services as a missionary \citep[256--263]{bridges_church_1978}. By contrast, the training received by Günther and Handt at Basel “consisted of 40\% theology, 22\% linguistics and 38\% `skills' which were often far from practical, e.g., calligraphy, anatomy and botany” \citep[45]{allen_2011}. Handt had entered Basel in 1822, where he was trained by Rev. Theophilus Blumhardt \citep[264]{bridges_church_1978}.

Watson arrived in New South Wales in 1832 and was met by Handt, who had arrived the previous year. A Lutheran minister, who was never ordained by the Church of England, Handt had between 1827 and 1830 worked as a Church Missionary Society missionary in Liberia, before returning to London for health reasons and being sent to Australia \citep[263]{bridges_church_1978}. While waiting for Watson, Handt commenced a “vocabulary of Wiradjuri from informant(s) who had spent time at the mission site” \citep[292]{bridges_church_1978}. 

During this time Handt married Mary Crook (1804--1844), the daughter of the London Missionary Society missionary Rev. W. P. Crook (1775--1846). The company in which this marriage placed Handt is tantalisingly suggestive of an alternative, but unsubstantiated path of development of linguistic ideas about Australian languages. Crook had been the first London Missionary Society missionary in the Marquesas Islands. Having left the crew of the \textit{Duff} at Tahuata in 1797, he spent a harrowing year alone, before being rescued by the \textit{Betsy} and being moved to Taioha’e. In nineteen months, before returning to London, Crook collected material for “An essay towards a Dictionary and Grammar of the Lesser-Australian Language, According to the Dialect Used at the Marquesas” (\citeyear{crook_essay_1799}; see \citealt{hughes_introduction_1998}). The data is described \citep[xxiv]{hughes_introduction_1998} as “perhaps the finest to emerge from anywhere in Polynesia in the eighteenth century”. In 1803, Crook settled in New South Wales, where he held a number of prestigious positions, including Chaplain of the Colony, before emigrating to Tahiti in 1816, accompanied by his eleven-year-old daughter Mary, Handt’s future wife, where they stayed for the following fifteen years \citep[308]{bridges_church_1978}.

By 1834, within less than two years of the mission’s establishment, Watson and Handt reported that they had prepared a vocabulary of 4000 “words” and had translated the Apostles' Creed, the Lord’s Prayer, the Ten Commandments, portions of Genesis and the Gospel of St Matthew \citep[12]{allen_2011}. 

The relative contribution of Watson and Handt to these translations is not known. Given what is known about Handt’s background it does, however, seem likely that Handt’s input to these translations and to the analysis of Wiradjuri made before Günther’s arrival in 1837 outweighed Watson’s contribution. Handt is known to have commenced learning Wiradjuri before Watson’s arrival, and to have produced his own grammar (below). He had also had a more thorough linguistic training, and his previous years in West African mission fields, and marital connection to LMS missionary W. P. Crook are likely to have exposed him to missionary grammars of non-Australian languages. Handt was also later involved with the description of Turrubul spoken around Moreton Bay, where he worked with C. Eipper (1813--1894; \sectref{sec:key:4.5.2}). In a letter written from Moreton Bay, Handt described to Günther at Wellington Valley the similarity between the 1\textsc{sg} and 2\textsc{sg} pronouns in Turrubul and Wiradjuri \citep[175]{newton_more_1987}.

Without any positive evidence, it is possible to summise that during the sixteen months in which Handt bided his time in Sydney waiting for instructions and for Watson to arrive from London, that he assisted Threlkeld in preparing the grammar of Awabakal (\citeyear{threlkeld_australian_1834}) that was a finished MS in 1832. The men moved in the same institutional circles within the small colonial capital. 

Both Watson and Handt independently wrote Wiradjuri grammars, which have now been lost. How similar they were is not known \citep[414--415]{bridges_church_1978}. \citet[xii]{fraser_australian_1892} claimed that Watson’s MS grammar was sold “as waste paper”, but \citet[799]{bridges_church_1978} asserts that it was sold “by his widowed wife to the New South Wales government in 1871 when the government collected Aboriginal language materials for the British linguist Prof. Max Müller … [Mrs Watson] believed the grammar to be perfect”. The fate of Handt’s grammar is unknown. 

Rev. W. Ridley, who later described Gamilaraay, spoken to the north of Wiradjuri (\sectref{sec:key:4.5}) did, however, hold copies of both Watson’s and Günther’s grammars and vocabularies, which he returned to the Colonial Secretary of New South Wales some time before 1873 \citep[275--276]{ridley_report_1873}. He described Watson’s work as being in two volumes, and stated that Watson had “entered on the work with the hope of making a much more comprehensive collection of words than he succeeded in getting”. Ridley also described how:
\begin{quote}
    the amount of information furnished by Messrs. Watson and Günther concerning the grammatical structure of the language, especially the modifications of the verbs and pronouns, is remarkable. Mr Watson’s manuscript includes dialogues illustrative of the modes of thought and expression in use among the aborigines. \citep[276]{ridley_report_1873}
\end{quote}

Missionary work at Wellington Valley was hampered by personal disagreement between the missionaries, initially between Watson and Handt, who left in 1836, and later between Watson and Günther. Watson was dismissed early in 1840 and left in October. The mission was closed in 1842 and Günther remained until the following year \citep[7--8]{allen_2011}. 

There is conflicting evidence in the primary sources concerning the extent to which Günther’s acquisition of Wiradjuri was made independently of Watson’s \citep[483--490]{bridges_church_1978} or of Handt’s. The bitterness between missionaries extended to a disagreement about the intellectual ownership of the grammatical analysis. Bridges writes: 
\begin{quote}
    After Watson’s expulsion from the station Günther reported that Watson had avoided assisting him to any extent, giving only very little help with the vocabulary, but that he had almost completed a grammar on his own. Watson, on the other hand, claimed that he had lent Günther his manuscript grammar and that it was not returned for upwards of two years. \citep[485]{bridges_church_1978}
\end{quote}

By comparing the descriptions of Wiradjuri contained in Gunther’s notebooks (\citeyear{gunther_native_1838}; \citeyear{gunther_lecture_1840}) and that published in \citet{hale_languages_1846}, this chapter establishes that Günther’s analysis of Wiradjuri did not replicate Watson’s work. 

\subsection{W. Günther’s grammars of Wiradjuri (\citeyear{gunther_native_1838}; \citeyear{gunther_lecture_1840})}
\label{sec:key:4.1.1}

The notebook containing Günther’s earliest grammar (\citeyear[5--89]{gunther_native_1838}; \figref{fig:4:56}) also contains translations of The Creed (ibid.: 317--318), The Ten Commandments (ibid.: 319--321), and The Lord’s Prayer (ibid.: 322).\footnote{The pagination given for Günther’s MSS is that of the State Library of New South Wales.} It is not known whether these are replicas of, or improvements on, Handt and Watson’s lost translations. This notebook also contains short vocabulary of “Wanngaibuoan” (Wangaaybuwan) (ibid.: 309--313) and a slightly longer vocabulary of “Gammilurai” (Gamilaraay) (ibid.: 229--306). 

\begin{figure}[t]
\includegraphics[width=.8\textwidth]{figures/Fig56.jpg}
\caption{Title page of Günther's first Wiradjuri grammar (\citeyear{gunther_native_1838}: 5)}
\label{fig:4:56}
\end{figure}

Note that this is the earliest, and previously unrecognised, written record of the language name Gamilaraay (see \citealt[8--10]{austin_reference_1993}; \citeyear[40]{austin_gamilaraay_2008}). It is not known from whom Günther collected this material. Although the word for ``no'', is given as \textit{kamil}, which is diagnostic of the variety referred to as Gamilaraay, spoken to the north of Wiradjuri, not all of the entries are part of what is now identified as the corpus Gamilaraay vocabulary (John Giacon, pers. comm.). Soon after first encountering the language Ridley stated: 

\begin{quote}
    The language I refer to is called by those who speak it ``Kamilaroi'' … The languages are named generally after the negative adverb ; thus, in Kamilaroi … \textit{kamil} means “no” : in Wolaroi, \textit{wol} is “no” : in Wailwun, \textit{wail} is “no” … From a lecture delivered in Melbourne, I see that the same plan of naming languages prevails in Victoria. (\citealt[73]{ridley_kamilaroi_1855-1})
\end{quote}

\ea\label{ex:key:1}
\gll Gamil-araay \\
no-\stockcaps{COM} \\
\glt `the language with the word ``gamil'''
\z

As also noted by \citet[275]{ridley_report_1873}, the etymology of the name “Wiradjuri” is similarly formed, although the missionaries at Wellington Valley are not known to have observed this: 

\ea\label{ex:key:2}
\gll Wirraay-dhurraay  \\
no-\stockcaps{COM} \\
\glt `the language with the word ``wirraay'''
\z
    
The notebook containing Günther’s later grammar (\citeyear{gunther_lecture_1840}; \figref{fig:4:57}) commen\-ces with a section titled “Lecture on the Aborigines of Australia” (ibid.: 4--139). Following are \textit{four separate} Wiradjuri vocabularies. The first and longest is dated 1837. The following two collections of “most essential words” and a “supplementary” vocabulary are all dated 1840. Interspersed between the vocabularies is a section headed “phrases in the optative, subjunctive” (ibid.: 308--309) and one describing demonstratives (ibid.: 310). There is also a translation of the first chapter of Genesis (ibid.: 324--327) and a folded slip of paper upon which is written, “To be preserved / A comparison between Celtic and Aboriginal words”, which lists thirty-two Wiradjuri words “bearing affinity to the Celtic Language” (ibid.: 312, 314). This item is an early example of the “philological imagining” \citep[633]{capell_history_1970} that pervades the lexical study of Australian languages in the nineteenth century. The work concludes with the grammar, titled “An attempt of a Grammar of the Aboriginal Dialect Wirradurrei spoken in the Wellington District \&\& [\textit{sic}] of New Holland by James Günther 1840” (\figref{fig:4:57}).

\begin{figure}[t]
\includegraphics[width=.8\textwidth]{figures/Fig57.jpg}
\caption{Title page of Günther second Wiradjuri grammar (\citeyear{gunther_lecture_1840}: 337)}
\label{fig:4:57}
\end{figure}

Günther used multiple ampersands in the title of both grammars, after the location of the language, in order to convey “etc.’. Given the vast area in which Wiradjuri was spoken, it is probable that during twelve years of missionary activity, the missionaries were contacted by people speaking different regional varieties of the language. This fact may explain some of the variation in the early sources of some of the illustrated forms, especially in the description of pronouns (\citealt{Stockigt_2017}; §4.4.5).

Günther’s 1838 and 1840 grammars are fairly similar, although the 1840 grammar contains additional sections called “syllabication” (\textit{sic}) and “accentuation”, but does not contain sections describing the “formation of words” or, importantly, pronouns. The 1838 grammar declines personal pronouns, demonstratives, “relatives’, and interrogative pronouns. This entire section is absent in the 1840 grammar. In the earliest grammar the language name is given as “Wirradurri’, and in the latter as ``Wirradhurrei''. Otherwise, the orthography used in each work is similar.

The works are, however, different analyses, rather than incomplete versions of the same work. The way in which nominal classes of declension are set up in each work differs, as does the labelling of cases in tables of nominal declensions. 

The grammars contain some clausal exemplification of the structures they describe. Most textual material is, however, contained in a section following the vocabulary headed “Sentences or phrases of the Wiradurri dialect” (\citealt[249]{gunther_native_1838}). Like other linguistic descriptions written soon after initial contact, this section provides rare insight into the intimate nature of Aboriginal contact with missionaries and Europeans (see, e.g., Example \ref{ex:1:2} on p. \pageref{ex:1:2}).

\subsection{W. Günther in Fraser (\citeyear{gunther_grammar_1892})}
\label{sec:key:4.1.2}

Günther’s MSS analyses were brought into the public arena through publication by Fraser who in his introduction to the publication of Günther’s Wiradjuri material \citep{gunther_grammar_1892} wrote: 

\begin{quote}
    I consider myself fortunate in having secured a publication of the Grammar and Vocabulary of so important a tribe. The following manuscript … is especially reliable because of its author’s character and experience, and because, at the time, the tribe had not yet begun to decay, and its language was entire … The MSS. are the property of the late Mr. Günther’s son … who has lent them to me for this purpose. In editing them I have retained the author’s mode of spelling the native words, and have made only some slight alterations in the form of the matter of the Grammar and Vocabulary, with the view of securing greater symmetry throughout. \citep[56]{gunther_grammar_1892}
\end{quote}

Günther’s 1840 grammar is written in black ink. Many pages are annotated in pencil. Comparison of the MS (\citealt{gunther_lecture_1840}) with \citet{gunther_grammar_1892} shows that the notes made in pencil are Fraser’s “alterations in the form of the matter” required to secure the “greater symmetry” Fraser desired. The pencilled alterations to the ordering and naming of cases in Günther’s MS grammar (\figref{fig:4:58}) equate to that presented in Fraser’s publication (\figref{fig:4:59}). See also \figref{fig:4:66}.

\begin{figure}
\includegraphics[width=.8\textwidth]{figures/Fig58.jpg}
\caption{Günther's Wiradjuri case paradigm (\citeyear{gunther_lecture_1840}: 347), showing notes made by Fraser in preparation for his 1892-edited publication. The numbers accord with those presented in 1892 (\figref{fig:4:59})}
\label{fig:4:58}
\end{figure}

\begin{figure}
\includegraphics[width=.8\textwidth]{figures/Fig59.png}
\caption{Fraser’s Wiradjuri case paradigm \citep[57]{gunther_grammar_1892}}
\label{fig:4:59}
\end{figure}


Comparison of the two Günther MMS and \citet{gunther_grammar_1892} shows that Fraser also had access to the 1838 grammar, although this work is not annotated. Fraser’s published work contains portions of analysis which were not included in 1840 but which were given in 1838, most notably the entire description of pronouns and the translations of the Lord’s Payer etc. 

\section{H. Hale’s comparative grammar of Wiradjuri and Awabakal (\citeyear{hale_languages_1846})}
\label{sec:key:4.2}

Horatio Hale (1817--1896), the American ethnologist and later mentor of Franz Boas, published a comparative grammar of two Australian languages in the \textit{Reports of the United States Exploring Expedition} (\citeyear{hale_languages_1846}). The work was made early in Hale’s career after graduation from Harvard (1833--1837), where he had studied Algonquian. An early exponent of the study of linguistics within ethnology, Hale was selected as the philologist on the United States Exploring Expedition (1838--1842) and subsequently authored Vol. VI of the expedition report, \textit{Ethnology and Philology} (\citeyear{hale_languages_1846}). The work detailed people and languages of Oceania, Australia, South America and northwest America, and included the Australian comparative grammar (ibid.: 479--531).

Hale advocated that the examination of grammatical structure was equally important as lexical comparison in determining a relationship between languages, and hence people. In relation to the position of Australian languages he stated:
\begin{quote}
Besides the similarity of words … it was considered important to ascertain whether an equal degree of resemblance was apparent in the grammatical structure of the different languages. With this view it was thought best to select two dialects as widely separated as possible, and determine … their leading characteristics (\citeyear[481]{hale_languages_1846}).
\end{quote}

Echoing Hale is the statement given the following year by the British ethnologist and physician James Prichard (\citealt[272]{jamesprichard_1847}) in his well-informed overview of existing research into Australian languages: “correspondences in vocabularies would not have afforded by themselves sufficient proof of a family relation between the Australian languages. But the evidence afforded by it has been confirmed by grammatical researches”.
    
The two languages Hale “selected” – in reality, probably the only two he had good access to via established missions – were Awabakal and Wiradjuri. Hale visited New South Wales between November 1839 and March 1840 and stayed with missionary Threlkeld at Lake Macquarie mission and with missionary Watson at Wellington Valley (\citealt{Wilkes_1845}). Hale referred to the language that Threlkeld had not named, but had identified by location, as “Kãmilarai’. His reasons for doing so are not clear. Hale does not refer to the three Lutheran grammars of South Australian languages (\citealt{teichelmann_outlines_1840}; \citealt{meyer_vocabulary_1843}; \citealt{schurmann_vocabulary_1844}) or to Symmons' description of Nyungar (\citeyear{symmons_grammatical_1841}) from Western Australia, which were all published in the years between the expedition’s visit to New South Wales and the publication of the Reports.

In the same year as the publication of the Expedition report, both \citet[249]{schurmann_aboriginal_1846} and \citet[vi]{moorhouse_vocabulary_1846} discussed grammatical structures that they believed indicated that Australian languages belonged to a single family (\sectref{sec:key:6.2}, \sectref{sec:key:6.4.1}). These South Australians identified a greater number of shared features from a larger sample of languages than did \citet[479]{hale_languages_1846}, who himself stated that “his field of inquiry did not extend beyond the limits of New South Wales”.

The Wiradjuri and Awabakal linguistic material Hale presented is almost entirely based on the missionaries' analyses. In recollection of his time at Wellington Valley, Hale recalls that Wiradjuri grammatical description was supplied to him by Watson, who: 

\begin{quote}
not only gave every assistance in obtaining a vocabulary from the natives, but did us the unexpected favour of drawing up an account of the most important peculiarities of the language, modelled as nearly as possible on Mr Threlkeld, for the purpose of comparison. This here is given with only a slight change of form. (\citealt[482]{hale_languages_1846})
\end{quote}

While it is not clear what process Hale implied when stating that Watson “drew up an account” of the language, it is probable that it involved copying out existing grammatical material, in the way that later missionaries at Bethesda and Hermannsburg are known to have reproduced the existing analyses of Diyari and Arrernte (\sectref{sec:key:8.2}, \sectref{sec:key:9.3}). It is not known whether “Watson’s grammar” reproduced the analysis by the Lutheran, Church Missionary Society missionary Handt, who had left the mission four years before Hale’s arrival, or whether Handt’s and Watson’s materials differed.

It is odd that Hale makes no mention of the German-speaking Basel-trained missionary Günther who was also at the mission in 1840. Günther had produced at least one MS grammar and vocabulary (\citeyear{gunther_native_1838}) at the time of Hale’s visit. The oddity is probably explained by the coincidence of Hale’s visit with Watson’s dismissal in early 1840, when tension between the two missionaries was likely to have been most heightened. 

\citet[666]{capell_history_1970} suggests Hale’s “guiding hand” is evident in the spelling system employed in Günther’s grammars, which he describes as “reasonably phonemic”. This, however, cannot be the case, since Günther’s 1838 grammar was completed well before Hale visited Australia, and because the orthography Günther used in the 1838 MS is not substantially different from that employed in 1840. Note that although the spelling of the language name used in Günther’s MSS differs from (\figref{fig:4:56}, \figref{fig:4:57}) Hale’s representation, “Wiradurei” differs again.
Of the Awabakal material, \citet[482]{hale_languages_1846} stated: “The grammar … which follows is therefore entirely due to Mr Threlkeld, the only changes being in the orthography, the arrangement, and some of the nomenclature”. A small proportion of the analysis may, however, be Hale’s own. \citet[482]{hale_languages_1846} also wrote that while at Lake Macquarie he “received [from Threlkeld] many useful explanations on the points not sufficiently elucidated in the grammar, together with free access to his unpublished notes, and the advantage of reference, on doubtful points to the natives”.

Hale presented the “Wiradurei” and “Kãmilarai” (Awabakal) material in adjacent columns giving equivalent structures for each language side by side (\figref{fig:key:4-60}). Hale described the format as favourable because “the points of resemblance and dissimilarity may be seized at once … [and because] the necessity of repeating many explanations is avoided” (\citealt[484]{hale_languages_1846}). 

The same format was later engaged by \citet{flierl_christianieli_1880} in a comparative grammar of Diyari and Wangkangurru (\sectref{sec:key:8.4.1.1}) and by C. \citet{strehlow_notitle_1910} in a comparative grammar of Arrernte and Luritja (\sectref{sec:key:9.2.3.3}). Note that while Hale commented that the format allowed for an efficient presentation of points of grammatical similarity \textit{and} dissimilarity, the later missionaries' use of this method tended to assume that the structure of the two languages presented side-by-side would necessarily be equivalent. Nineteenth-century corpus grammarians tended to overestimate the structural homogeneity of Australian linguistic structures, in keeping with conclusions drawn by Grey (\citeyear{grey_languages_1845}; \figref{fig:map2}; \sectref{sec:key:6.4.1}), \citet[v-vi]{moorhouse_vocabulary_1846}, and Ridley (\citeyear[293]{ridley_kamilaroi_1856}).

\begin{figure}
\includegraphics[width=.8\textwidth]{figures/Fig60.png}
\caption{Page showing tense terminology employed by H. Hale in a comparative grammar of Awabakal and Wiradjuri (\citeyear{hale_languages_1846}: 498)}
\label{fig:key:4-60}
\end{figure} 

\citet[485]{hale_languages_1846} was the first to suggest that orthographic differentiation between voiced and unvoiced stops was superfluous, \textit{contra} Koch (\citeyear[154]{koch_explaining_2011}; \citeyear[184]{koch_r_2008}) and \citet{blake_nineteenth-century_2016}, who attribute this discovery to Mathews' works, made over half a century later. That Hale’s practice has remained historically unrecognised tells of the infancy of Australian linguistic historiography. Hale also used engma (ŋ) to represent the velar nasal, before Ridley (\citeyear{ridley_kamilaroi_1855}; \sectref{sec:key:4.5}), to whom \citet[41]{austin_gamilaraay_2008} attributes the accolade. Note, however, that Dawes (\citeyear{dawes_grammatical_1790}: 1\textit{ff}.; \citeyear{dawes_vocabulary_1790}) had used engma well before either of these nineteenth century grammarians.

Since the Wiradjuri material presented in \citet{hale_languages_1846} was produced specifically for Hale by missionary Watson, the comparison of the Günther MSS (\citeyear{gunther_native_1838}; \citeyear{gunther_lecture_1840}) with \citet{hale_languages_1846} shows the extent to which Günther’s grammars are a result of his own analysis. While both show some influence from \citet{threlkeld_australian_1834}, Hale’s presentation of Wiradjuri is significantly different from either of the Günther MSS, suggesting that Günther and Watson made independent analyses of the language. The works differ from one another not only in the presentation of the material but also in terms of the data given to illustrate what appear to be the same structures. 

\section{R. H. Mathews}
\label{sec:key:4.3}

This short section provides some background for R. H. Mathews (1841--1918), whose grammars of Wiradjuri (\citeyear{mathews_wiradyuri_1904}) and Gamilaraay (\citeyear{mathews_languages_1903}) are mentioned in this chapter. Like Wiradjuri, Gamilaraay was spoken over a vast area of New South Wales \citep[2--3]{austin_dictionary_1992}. Like Wiradjuri, Gamilarray was described by a missionary close to the time of first contact (\citealt{ridley_kamilaroi_1875}; \sectref{sec:key:4.5}) and was subsequently described by Mathews early in the twentieth century.

Mathews' grammatical descriptions have been previously assessed by \citet{koch_r_2008}. His analysis of various PN languages, described as a bibliographer’s nightmare \citep[181]{koch_r_2008}, conforms to his developed framework with regularity. \citet[183]{koch_r_2008} describes the formulaic nature of Mathews' description of phonology: “Each article gives essentially the same information, except that the examples are always taken from the language under description”. The same is true of his morpho-syntactic analyses. Earlier evaluations of Mathews” work were similarly lacklustre. \citet[133]{elkin_nature_1937} appraised Mathews' linguistic work as “very superficial”, and Ray commented:

\begin{quote}
    Mathews published short grammatical notes of a great many languages but singularly failed to appreciate the necessity of a detailed account of the suffixes … He gives usually only the number, gender and some cases of the noun, a brief account of the adjective, some forms of the pronouns (the interrogative and demonstrative often being without details) with the principal tenses of the verb and a few adverbs and prepositions. \citep[2]{ray_aboriginal_1925}
\end{quote}

And C. Strehlow’s German editor, M. von Leonhardi (\citeyear{leonhardi_letter_1908-2}; \sectref{sec:key:9.2.2})  described Mathews' ethnographic observations as “practically totally useless”.

Mathews' grammars are predominantly of languages from the southeast of the continent, where he initially encountered Aboriginal groups in the 1870s while working as a surveyor and later magistrate in district courts (\citealt{Thomas_2011}). Without underemphasising the importance of the record left by Mathews – a large number of Pama-Nyungan languages would otherwise be vastly less well understood without his Latinate tables of pronominal paradigms, nominal declensions and verb conjugations – his grammars portray PN languages as differing more in form than in structure. They are formulaic to the point of providing a single analysis for multiple languages. Mathews' grammatical work on Gumbaynggir (\citeyear{mathews_aboriginal_1902}; \citeyear{mathews_notes_1910}), for example, is described by \citet[256]{eades_gumbaynggir_1979} as “very similar in organisation to his articles on many other N. S. W. languages”. Mathews' lack of engagement with basic nuance of Gumbaynggir is indicated by the fact that “no mention is made of the split case system for nouns and pronouns … processes of nominalisation and verbalisation … the class of irregular verbs … or derived adverbs … There is no indication of complex sentence structures” \citep[257]{eades_gumbaynggir_1979}.

Stockigt (\citeyear{Stockigt_2017}; §4.4.5.1) compares these early missionary grammars with Mathews' later grammars of Wiradjuri (\citeyear{mathews_wiradyuri_1904}) and Gamilarray (\citeyear{mathews_languages_1903}). Differences between Mathews' analysis of Wiradjuri and the early missionary sources are found to be mirrored by the differences between Mathews' analysis of Gamilaraay and the earlier missionary analysis of that language \citep{ridley_kamilaroi_1875}, indicating Mathews' regularisation of grammatical material previously documented by \citet[256]{eades_gumbaynggir_1979}. Further evidence of Mathews' tendency to fit linguistic structure to his own rigid framework is shown through a comparison of description of the inclusive/exclusive distinction in first-person non-singular pronouns in early grammars of languages spoken in New South Wales, Awabakal, Wiradjuri and Gamilaraay, and in Arrernte (\citealt{Stockigt_2017}; §4.5).

\section{W. Günther’s analysis of Wiradjuri}
\label{sec:key:4.4}
\label{sec:4.4}

Günther’s grammatical descriptions of Wiradjuri (\citeyear{gunther_native_1838}; \citeyear{gunther_lecture_1840}), made towards the end of missionary engagement with Wiradjuri at Wellington Valley mission, were written with an air of descriptive confidence. The grammars are organised around a reduced inventory of the classical parts of speech. \citet[13]{gunther_native_1838} explained in his opening passage that the article was not relevant to the description, after which he wrote, “Postpositions are neither to be met with in this dialect, postfixes must serve for the purpose”, and explained that the functions carried by the article and by some conjunctions in familiar languages were conveyed by “postfixes” in Wiradjuri. Günther did not discuss morphology under the word-class heading “pre/post-position” or under ``postfixes''. 

While other early grammarians (e.g., Meyer \citeyear{meyer_vocabulary_1843}; \sectref{sec:key:6.1.2.1}) drew a parallel between the function of the word-class “pre/post-positions” and ``case terminations'', Günther’s analysis is radical in abandoning the word-class “pre/post-po\-si\-tions” altogether. This particular remodelling of the traditional descriptive framework to better accommodate PN structure is atypical of the corpus grammars, but was later employed in grammars of languages spoken in New South Wales, by Ridley (\citeyear{ridley_kamilaroi_1875}; \sectref{sec:key:4.5.4}) and subsequently by Livingstone (\citeyear{livingstone_grammar_1892}; \sectref{sec:key:4.6.1}). While Livingstone’s choice was well considered, the absence of this part of speech from Ridley’s grammars probably reflects the general sparseness of Ridley’s descriptions.

In other instances, Günther presented traditional schemata but stated that the category was not required to account for the structure of Wiradjuri. After providing a case paradigm for the ``relative pronoun'', for example, Günther explained that the forms functioned only as interrogatives: 
\begin{quote}
    Relatives seem to be the same thing with the following Interrogatives. It does indeed not appear that Relatives are used except in as much as they are interrogatives. \citep[32]{gunther_native_1838}
\end{quote}

Further, under the traditionally prescribed heading “comparison of adjectives'', Günther, like Threlkeld, did not attempt to force an analysis of the morphological marking of comparative and superlative degrees, but stated: 
\begin{quote}
    Comparisons are formed in a very imperfect manner, or, rather, expressed very indistinctly, for there is strictly speaking no form of comparatives [illegible section crossed out] Thus to say, this is better than that, \textbf{Ngin\-na marong, wirai nginna}; ``This is good, not this''. \textbf{Nginna marrombang na\-naing}; ``This very good, that also, or this is as good as that''. \citep[10]{threlkeld_australian_1834}
\end{quote}

Regarding the passive, Günther’s analysis is particularly astute. Initially he observed that the European passive is conveyed: 
\begin{quote}
    by putting the object [in] accusative and using the active form [of the verb], but the agent or instrument with whom or which the action suffered originates is not named. \citep[366]{gunther_lecture_1840}
\end{quote}

That a transitive clause with an elided agent was functionally equivalent to the SAE passive construction was commonly proposed in the early sources (\sectref{sec:key:2.3.3}). However, Günther went on to state that:

\begin{quote}
    The form referred to is not in reality a Passive, but an Active sentence; only for the sake of laying more emphasis on the Verb or action done, the Noun or agent is … omitted. \citep[366]{gunther_lecture_1840}
\end{quote}

The firmness with which Günther shows that the supposed passive forms are in fact not passive suggests that his statement was designed to correct an earlier analysis, either Watson’s or Handt’s grammar of Wiradjuri (no date) or Threlkeld’s grammar of Awabakal (\citeyear[28]{threlkeld_australian_1834}).

\subsection{Description of number}
\label{sec:key:4.4.1}

Günther (\citeyear[57]{gunther_native_1838}; \citeyear[354]{gunther_lecture_1840}) was the first PN grammarian to observe that nouns that are unmarked for number have no specific or default number reference, and number is instead determined through context \citep[77]{dixon_australian_2002}. He wrote: “The plural … appears to be very rarely made use of, the singular form being often taken for plural signification”. Later grammarians who similarly observed that “the singular is used … where the context shows that the plural number is obviously required” \citep{strehlow_aranda_1944} include \citet[16]{roth_structure_1901}, \citet[6]{livingstone_grammar_1892}, \citet{schoknecht_grammar_1947}, \citet[3]{reuther_dieri_1894} and \citet[11]{hey_elementary_1903}. 

Günther described inflection on nouns for plural number (\citeyear[57]{gunther_native_1838}; \citeyear[354]{gunther_lecture_1840}) but did not provide a paradigm. Nor did he state a syntagmatic rule regarding the relative ordering of inflections for plural number and case on nouns. 

Neither Günther nor \citet[485]{hale_languages_1846} described the morphological marking of dual number on nouns. Nor had Threlkeld in his earlier grammar of Awabakal. In this way, their analyses resemble Livingstone’s description of Minjangbal (\citeyear{livingstone_grammar_1892}; \sectref{sec:key:4.6.1}), also spoken in New South Wales.

\subsection{Morphophonology}
\label{sec:key:4.4.2}

Günther’s description of morphophonemic processes is sophisticated in comparison to that of other corpus grammarians. \citet[4]{ray_aboriginal_1925} described Günther’s grammar as among only a few works which note “the phonetic changes when particles are affixed”. Ray’s assessment is presumably based on Günther in Fraser \citep[59]{gunther_grammar_1892}, which is heavily edited and reads more succinctly than the original \citep[348]{gunther_lecture_1840}. Nevertheless, Günther’s original description is in this regard remarkable. 

\citet[11]{threlkeld_australian_1834} had earlier provided a detailed account of how the shape of stem-final phonological segments affects the shape of the case suffix. He did so by presenting a series of rules accounting for variation in the ergative suffix (\figref{fig:3:44}). Günther accounted for the variant forms of suffixes marking the same cases on nouns (\citeyear[348]{gunther_lecture_1840}) and the same tenses on verbs (\citeyear[349]{gunther_lecture_1840}) in terms of phonological process. He wrote: 
\begin{quote}
    [T]he letter “r” is changed into its relative liquid ``l''[,] and “n” for the sake of euphony into “m” (vide Assimilation). Thus euphony also demands that “a” terminating the root be modified into the dipth(th)ong (sic) ``ai''. \citep[349]{gunther_lecture_1840}
\end{quote}

Attention to morphophonemic processes is not part of the older tradition of the description of classical European languages. It was not until the late eighteenth century, largely through the writings of M. Kruszewski (1851--1887) that linguistic theory developed to take account of morphophonology. In a history of morphophonemics, Kilbury writes:

\begin{quote}
    When Western scholars finally turned their attention to problems in the area of morphophonemics and morphonology, it was largely though the influence of these [i.e., Semitic and Sanskrit] non-western grammatical traditions. \citep[13]{kilbury_development_1976}
\end{quote}

Günther’s pre-theoretical discussion used terms such as ``assimilation'', “euphony” and ``liquid'', in a way that is atypical of the corpus grammars. Of verbs Günther wrote: 

\begin{quote}
    Indeed there is only one original or fundamental conjugation which undergoes a little alteration according to the termination of the verb in the present tense [. … I]t depends principally on the penultimate vowel or syllable, perhaps more properly speaking it depends on the termination, the last letter of the radical part of the word. \citep[65]{gunther_native_1838}
\end{quote}

Of nouns he wrote: 

\begin{quote}
    Properly speaking there is only one original or fundamental declension, but the assimilation of letters to which the language has a strong tendency causes in a few cases slight variation arguably to the last letter of the noun. \citep[344]{gunther_lecture_1840}
\end{quote}

\begin{figure}[b]
\includegraphics[width=.8\textwidth]{figures/Fig61.jpg}
\caption{Günther’s five conjugations of the Wiradjuri verb (\citeyear{gunther_native_1838}: 75)}
\label{fig:4:61}
\end{figure}


Günther nevertheless established five conjugation classes of verbs (\citeyear[75]{gunther_native_1838}; \figref{fig:4:61}). He presented seven declension classes of nouns in 1838 (pp. 50--56) and five in 1840 (pp. 349--353), from which \citet[58]{fraser_australian_1892} tabulated eight classes (\figref{fig:4:62}). Günther presented these verbal conjugation and nominal declension classes despite recognising, and being able to explain, that they were motivated by morphophonemic variation. In this way, Günther accounted for morphophonology twice, once using terminology and explanations not found in classical description, and second by conveying the variation within the traditional schema. This type of double representation using both an innovative descriptive technique and a traditionally prescribed schema to account for the foreign structure is also evident in some grammarians' presentation of cases marking functions outside the Latin inventory as both case suffixes and as prepositions (\sectref{sec:key:3.3.4}).


\begin{figure}[t]
\includegraphics[width=.8\textwidth]{figures/Fig62.png}
\caption{Fraser’s presentation of eight declension classes in Wiradjuri (\citealt{gunther_grammar_1892}: 58)}
\label{fig:4:62}
\end{figure}


Mathews' later description of Wiradjuri (\citeyear{mathews_wiradyuri_1904}) gave only a single declension class of nouns and a single conjugation class of verbs. While acknowledging the existence of morphophonemic processes, Mathews' description (ibid.: 287) fails to provide any detail further than stating “the agent suffix has euphonic changes according to the termination of the word it is attached to. This may be said of the suffixes in all the cases”. He does not mention morphophonemic variation of verb morphology. In this regard Günther’s grammar is a more informative source. 

Günther’s and Threlkeld’s motivation to supply nominal classes of declension, which they made while recognising that in most instances these were induced by phonological alteration at the juncture of the “root” and the ``termination'', may have been motivated by another, non-linguistic factor. The provision of multiple declension classes may have been favoured in order to demonstrate the language’s sophistication. Related to the superiority of the Indo-European “flectional” languages was the Humboldtian notion that the particular ways in which “sound-forms” are shaped within words to express meanings demonstrated and contributed to the mental development of these languages' speakers (see \citealt{losonsky_humboldt_1999}). T. G. H. \citet{strehlow_aranda_1944} for example, held the prejudice when stating that the “single-type declension” of the noun in Arrernte highlights the “primitive character of the native language”.





\largerpage
By way of comparison, Ridley’s (\citeyear{ridley_kamilaroi_1855-1}; \citeyear{ridley_gurre_1856}) grammars of Gamilaraay do not attempt to convey case allomorphy (see \citealt[62]{austin_reference_1993}) by establishing classes of nominal declension or by stating morphophonemic rules. 

According to \citet[63]{austin_reference_1993}, although Ridley “failed to notice that there are four allomorphs of the ergative affix …. [he]… used the other allomorphs of this case affix correctly” when translating the Bible. By contrast, Giacon (\citeyear[5]{giacon_2014}) describes the style of language used in the Gamilaraay primer (\citeyear{ridley_kamilaroi_1856}) as “extremely simplified … without ergative forms.” Although Giacon (\citeyear[23--24]{giacon_2014}) concludes that the lack of ergative marking in Ridley’s translations results from morphosyntactic simplification, it is important to note that \citet{ridley_kamilaroi_1875} explicitly describes ergative optionality in Gamilaraay, stating: “Often, however, the agent suffix is omitted, even before an active verb”. 

 Although Mathews did not describe allomorphy in Wiradjuri, he did in Gamilaraay \citep[262--262]{mathews_languages_1903}, referring to the modification of case suffixes attaching to words with different endings as occurring “for the sake of euphony”. \citet[64--65]{austin_reference_1993}, however, suggests his description of dative allomorphy ``perhaps attempt[ed] to overgeneralise along the lines of allomorphy for the ergative''. 
 
\subsection{Case paradigms}
\label{sec:key:4.4.3}

\tabref{tab:chap4:names} summarises the labels given to nominal case markers in the early grammars of Wiradjuri, and the English prepositional phrase by which forms were translated and shows the case-labels assigned to forms in Grant \& Rudder’s reclaimed grammar. 

\begin{sidewaystable}
{\footnotesize
    \begin{tabularx}{\textwidth}{QQQQQQQQ}
        \lsptoprule
        {Forms (first declension, \citealt[58]{fraser_introduction_1892}*)} & \citet{Günther1838}  & \citet{gunther_lecture_1840} & \citet{fraser_introduction_1892} & {Form translated as:} & \citet{hale_languages_1846} & {Form translated as: } & \citet{GrantRudder2001,GrantRudder2014}\\
        \midrule 
        Unmarked & {nominative declarative} & {nominative declarative} &	{nominative}	& {``X''} &	{simple nominative} & {``X''} &	{nominative}\\
        {-du \textit{-dhu}} & {nominative agentive} & {nominative agentive} &	{nominative agent}	& {``X does''} &	{active nominative} & {``X does will''} &	{ergative}\\
        {-gu \textit{-gu}} & {genitive} & {dative} &	{genitive}	& {``belonging to X''} &	{genitive -guna -guba} & {``of X''} &	{possessive, purposive, allative}\\
        {-gu} & {dative} & dative  &	{dative}	& {``to or for X''} &	{dative 1} & {``for X''} &{possessive, purposive, allative}\\
        Unmarked & {accusative} & {accusative} &	{accusative}	& {``X''} & – & {``X''} &	{accusative}\\
        {-di \textit{-dhi}} & {ablative 1} & {locomotive} &	{locomotive}	& {``from X''} & {ablative} & {``from by, about, concerning X''} &	{ablative}\\
        {-durai \textit{-dhuray}} & {ablative 2} & {conjunctive} &	{conjunctive}	& {``with X''} & – & – &	{comitative}\\
        {-da \textit{-dha}} & {ablative 3} & {locative} &	{locative}	& {``in, on X''} & {dative 2} & {``to, towards X''} &	{locative}\\
        {-durada** \textit{-dhura-dhu}} & {ablative 4} & {instrumental} &	{instrumental}	& {``being with X''} & – & – &	{instrumental}\\
        {\textit{-birrangga†}} & {ablative 5} & – &	–	& {``at X's place''} & – & – &–\\
        \lspbottomrule
    \end{tabularx}
    }
	\legendbox[1]{$^\ast$Shows inflectional allomorph occurring on a stem ending in “a” or “n” \citep[22--24]{grant_wiradjuri_2001}.
	$^{\ast\ast}$The fourth ablative case, later called “instrumentative” and translated as “being with” occurs only in some of Günther’s paradigms. 
	$^\dagger$In the earliest grammar \citet[55]{gunther_native_1838} gave a distinct declension for the proper name “Badarai’.}
    \caption{Names given to case suffixes on nouns in different Wiradjuri sources}
    \label{tab:chap4:names}
\end{sidewaystable}
				 			


Günther extended the classical case paradigm to accommodate the larger PN case inventory, generally giving nine cases. It is likely that in doing so, he was influenced by Threlkeld (although note that an alternative but unsubstantiated lineage would place Handt as the originator of the practice in both languages; see \sectref{sec:key:3.3}, \sectref{sec:key:4.1}). Günther wrote:

\begin{quote}
    The number of cases cannot easily be fixed, since almost every relation in which a noun may be placed, on account of the entire absence of prepositions, is signified by some postfix or other, hence cases must of necessity be numerous. \citep[345]{gunther_lecture_1840}
\end{quote} 


\begin{figure}[b]
\includegraphics[height=.6\textheight]{figures/Fig64.jpg}
\caption{Günther’s first Wiradjuri case paradigm \citep[54]{gunther_native_1838}}
\label{fig:4:64}
\end{figure}

Günther followed Threlkeld’s method of naming additional cases in his earlier grammar \citep[55]{gunther_native_1838}. Like Threlkeld’s case paradigms, Günther’s case paradigms of nouns \citeyearpar[49--56]{gunther_native_1838} show up to four numbered ablative cases (\figref{fig:4:64}). He did not, however, show numbered dative cases. Günther’s choice to engage this numbering system was well considered. He stated: 


\begin{quote}
    If we do not invent new names for the cases peculiar to this language but confine ourselves to the cases known to Latin or Greek scholars we must speak of a 1. 2nd \&\& Ablative terms locative, instrumentative or [illegible] might be accepted as appellations for these new or uncommon cases. \citep[48]{gunther_native_1838}
\end{quote}


Günther’s more complete pronominal case paradigms also show multiple ablative cases, which are differentiated by letters \citeyearpar[15--17]{gunther_native_1838}.

\begin{figure}[b]
\includegraphics[height=.6\textheight]{figures/Fig65.jpg}
\caption{Günther’s second Wiradjuri case paradigm (\citeyear{gunther_lecture_1840}: 351)}
\label{fig:4:65}
\end{figure}


In his later grammar, Günther abandoned Threlkeld’s numbering system, and his own lettering system for pronouns, and instead provided names for the extra cases (\figref{fig:4:65}). He wrote: 

\begin{quote}
    To give a distinct significance to the uncommon or peculiar cases of the noun and to avoid speaking of 1\textsuperscript{st} 2\textsuperscript{nd} [illegible] Ablative , new apelations [sic] have been adopted as far as practicable such as Locative (in, on, at) Locomotive (from whence) Conjunctive (with accompanying, conjointly) Instrumentative (with, through, by means of). \citep[346--347]{gunther_lecture_1840}
\end{quote}


\largerpage
The case termed “ablative 1” in 1838 is termed “locomotive” in 1840. The “ablative 2” is termed ``conjunctive''. The “ablative 3” is termed “locative” and the “ablative 4’, “instrumentative” (\tabref{tab:chap4:names}). A minority of later corpus grammarians assigned names to cases that are extraneous to the Latin inventory (\citealt{meyer_vocabulary_1843,taplin_vocabulary_1867}; \citealt{taplin_notes_1872}; \citealt{taplin_grammar_1878}; \citealt{hagenauer_language_1878}; \citealt{bulmer_language_1878} and \citealt{strehlow_einige_1908,strehlow_notitle_1910}). None of these, with the possible exception of C. Strehlow (\sectref{sec:key:9.3.4.1}) show an influence from Günther or Günther in Fraser. Günther’s innovative presentation of Wiradjuri case had little, if any, further influence.

\subsection{Ergativity}
\label{sec:key:4.4.4}

Günther’s descriptions of ergativity are very similar to the ones by Threlkeld. Günther placed the ergative, ``nominative active'', case in second position at the top of the paradigm after the nominative “nominative declarative” case (\figref{fig:4:66}). He is the only corpus grammarian to label the nominative case the ``nominative declarative''. Fraser later omitted this term from the 1892 Wiradjuri grammar based on Günther.

Note that Günther’s hand-written abbreviation of “nominative declarative” to “n.d.” in paradigms is difficult to differentiate from the abbreviation for “nominative active”, ``n.a.''.

\begin{figure}
\includegraphics[width=.5\textwidth]{figures/Fig66a.jpg}%
\includegraphics[width=.5\textwidth]{figures/Fig66b.png}
\caption{Günther’s presentations of the ergative case termed “nominative active” (\citeyear{gunther_native_1838}: 49; \citeyear{gunther_lecture_1840}: 353)}
\label{fig:4:66}
\end{figure}  


Other corpus grammarians who, like Günther and Threlkeld, placed ergative forms in second paradigmatic position were \citet[6]{ridley_kamilaroi_1875}, Symmons (\citeyear[xiii]{symmons_grammatical_1841}; \citeyear[52]{symmons_grammar_1892}) for pronouns but not for nouns, and \citet[2--3]{moorhouse_vocabulary_1846} for nouns but not for pronouns. Later corpus grammarians, commencing with \citet{teichelmann_outlines_1840}, presented ergative forms differently (\sectref{sec:key:5.4.2}).

\citegen[47]{gunther_native_1838} clarification of ergative function was also similar to Threlkeld’s posing of questions that the ergative and nominative forms would be given in answer to (\sectref{sec:key:3.3.7.1}; \figref{fig:3:43}, \figref{fig:3:46}).  Günther wrote:

\begin{quote}
    Particularly strange appears the peculiarity that there are two Nominatives, the simple nominative or the nominative declarative corresponding to the question, “Who is it” and the active nominative [used] when the person or thing is considered as an agent, answering to the question, “Who does it?” \end{quote}

Again, it appears that in these regards Günther was strongly guided by Threlkeld’s practices, although it remains unknown how Handt’s earlier Wiradjuri analysis was formulated, and indeed whether Handt had originally steered the presentation of Threlkeld’s Awabakal material (\sectref{sec:key:3.3}, \sectref{sec:key:4.1}).

\subsection{H. Hale’s analysis of case}
\label{sec:key:4.4.5}

Threlkeld’s 1834 presentation of the case paradigm showing ten cases with its naming of cases not included in the Latin inventory as separately numbered “ablatives” were practices followed by \citet{hale_languages_1846} when describing Awabakal. Importantly, these practices were \textit{not} employed by Hale when describing Wiradjuri. Hale’s presentation of Wiradjuri, which was based on Watson’s (no date) analysis, shows seven cases and includes a \textit{single} ablative case and a second dative case (\figref{fig:4:67}). This difference shows that Watson’s presentation of Wiradjuri, upon which Hale’s grammar was based, was not the same as Günther’s. 

In support of the supposition that differences between Hale’s and Günther’s analyses result from Hale’s replication of Watson’s Wiradjuri paradigms, rather than from his own rearrangement of the missionaries' grammars, is the similarity between Hale’s “Kāmilarai” (Awabakal) paradigm (\citeyear[486]{hale_languages_1846}) and \citegen{threlkeld_australian_1834}. Hale’s Awabakal paradigm follows Threlkeld (\figref{fig:4:68}) in providing multiple ``ablatives''. Suffixes marking cases included in Günther’s paradigms (\tabref{tab:chap4:names}) are shown as prepositions by \citet[492]{hale_languages_1846}. While Hale’s Awabakal analysis follows Threlkeld’s closely, his Wiradjuri analysis does not resemble Günther’s. 

Thus, Günther’s presentation of Wiradjuri case differs from that given by earlier missionaries at Wellington Valley Mission.

\begin{figure}
\caption{Hale's  case paradigms}
\subfigure[Wiradjuri (\citealt{hale_languages_1846}: 487)]{\label{fig:4:67}
\includegraphics[width=.4\textwidth]{figures/Fig67.jpg}
}
\subfigure[Awabakal (\citealt{hale_languages_1846}: 486)]{\label{fig:4:68}
\includegraphics[width=.4\textwidth]{figures/Fig68.jpg}
}
\end{figure}
	
\subsection{Bound pronouns}
\label{sec:key:4.4.6}

Günther (\citeyear{gunther_native_1838}; \citeyear{gunther_lecture_1840}) made only a single \textit{explicit} reference to the existence of bound pronouns in Wiradjuri (\figref{fig:4:69}). Günther’s sound knowledge of the language allowed him to elucidate how the marking of a possessive NP for clausal case occurred in one of two ways, depending on whether the NP included a free or bound pronoun. He wrote:

\begin{quote}
    If the Possessive Pronoun is put before the noun, it [i.e., the noun] accepts of its termination or in other words is declined. But the more common practice is to put the pronoun behind as a postfix in an abbreviated form. \citep[355]{gunther_lecture_1840}
\end{quote}

\begin{figure}
\includegraphics[width=.8\textwidth]{figures/Fig69.jpg}
\caption{Günther’s illustration of an “abbreviated” pronoun used as a “postfix” (\citeyear{gunther_lecture_1840}: 355)}
\label{fig:4:69}
\end{figure}

The forms given to illustrate the “postfix” being placed “behind” the noun are of the structure:

\ea\label{ex:key:3}
buraugundi
\glt `to my boy' \\
\citep[355]{gunther_lecture_1840} \\
\gll Burai-gu-N-dhi \\
boy-\stockcaps{DAT}-?-1\textsc{sg}.\stockcaps{POSS} \\
\z

\ea\label{ex:key:4}
Buraigunu
\glt `to your boy' \\
\citep[355]{gunther_lecture_1840}\\
\gll Burai-gu-nhu \\
boy-\stockcaps{DAT}-2\textsc{sg}.\stockcaps{POSS} \\
\z

\ea\label{ex:key:5}
Buraigugula
`to his boy' \\
\citep[355]{gunther_lecture_1840} \\
\gll Burai-gu-gula \\
boy-\stockcaps{DAT}-3\textsc{sg}.\stockcaps{POSS} \\
\z

Like other early grammarians (\citealt{teichelmann_outlines_1840}; \citealt{schurmann_vocabulary_1844}; \citealt{Spieseke_1878}), Günther also illustrated the forms when conjugating the verb. His earliest paradigm (\citeyear[77]{gunther_native_1838}) showed bound \textit{and} free-form 2\textsc{sg} pronouns, while the later paradigm (\citeyear[362]{teichelmann_outlines_1840}) did not show bound pronouns. Günther was clearly aware that pronouns might be “abbreviated” and occur as ``postfixes'', yet the forms are not anywhere presented in a systematic way.

Wiradjuri is reclaimed \citep[345]{dixon_australian_2002} as being highly unusual in having 1\textsc{sg} and 2\textsc{sg} bound pronouns that mark a similar range of peripheral functions as the free-form pronouns: nominative/ergative, accusative, dative, genitive, locative and ablative. Bound pronouns typically mark only the core cases and the dative case. The scenario has been reclaimed based on \citet[488]{hale_languages_1846}, who provided “full forms” and “contractions” or “adjunct pronouns” side by side in first and second-person singular pronominal paradigms.

The certainty of existence of 1\textsc{sg} and 2\textsc{sg} bound pronouns marking an extended range of peripheral functions \citep[345]{dixon_australian_2002} is brought into question by comparative reading of the early sources. The reclamation of these forms is based on the assumption that \citet[488--489]{hale_languages_1846} did not engage in ``paradigm filling''. Note that with regard to phonology, \citet[493]{hale_languages_1846} mentioned that the linguistic material he collected during the two weeks he spent working with Watson at Wellington Valley Mission was reviewed and reassessed from notes \textit{after} leaving Australia. Further, the certainty that bound pronouns existed \textit{only} in 1\textsc{sg} and 2\textsc{sg} assumes that Hale did not, for the sake of space, simply choose not to give the “contracted” pronominal forms in the later third-person paradigms. He did after all state: ``\textit{All} the pronouns when postfixed to other words undergo contractions'' (\citealt[483]{hale_languages_1846}; emphasis added). \citet[28--38]{grant_wiradjuri_2001} reclaim a situation in which 3\textsc{sg} bound pronouns mark a full range of case functions.

\subsection{Concluding remarks}
\label{sec:key:4.4.7}

Günther’s somewhat jumbled MS grammars of Wiradjuri (\citeyear{gunther_native_1838}; \citeyear{gunther_lecture_1840}) were probably the culmination of previous missionary investigation of Wiradjuri made at Wellington Valley Mission. The analysis rests on a sound knowledge of the language. Günther’s discussion of morphophonemic alternation in terms of process is exceptional within the corpus, and only Threlkeld’s analysis (\citeyear{threlkeld_australian_1834}) comes close to matching Günther’s account of case allomorphy. 

Shelving speculation that J. S. C. Handt may have played a larger role in the development of PN descriptive practices than the current record evinces (\sectref{sec:key:3.3}; \sectref{sec:key:4.1}), Günther’s presentation of case and his deliberation about whether to number or to name additional “ablative” cases show unequivocal influence from Threlkeld. But like later early grammarians, Günther’s work is also descriptively innovative. Unlike Threlkeld, Günther did not provide interlinear-style glosses. His paradigms demonstrate that he was prepared to develop new approaches independently. While opting for his own innovative nomenclature in the later grammar (\citeyear{gunther_lecture_1840}), his continued use of extended case paradigms, which are atypical of later grammars, followed Threlkeld’s plan.

Comparison of Günther’s MS grammars (\citeyear{gunther_native_1838}; \citeyear{gunther_lecture_1840}) with Hale’s grammar (\citeyear{hale_languages_1846}) shows that Günther’s analysis differed from Watson’s now lost Wiradjuri grammar, which preceded it. That these earliest grammars of Wiradjuri, written shortly after Threlkeld’s analysis, differed not only from Threlkeld, but also from each other, further highlights the tendency shown by many corpus grammarians to produce their own novel descriptive responses to PN structure, rather than to reproduce schemata used by their predecessors. 

No other early grammar of an Australian language approaches the sophistication of Günther’s account of processes of morphophonemic variation. This suggests that the Basel Mission Institute, which trained missionaries to describe the structure of languages spoken at Protestant colonies around the world, was preparing missionaries for encounters with languages showing allomorphic variation before these structures were tackled by European linguistic theory. The terminology Günther employed to name cases in 1840 similarly suggests that the training he received at Basel was steeped in a theory of linguistic description that differed from that to which other early missionaries were exposed.

The potential trajectory of influence that grammatical analyses emanating from the Wellington Valley Mission had on later descriptions of PN languages was hindered by the fact that these works remained unpublished in Australia until 1892. The eventual publication of Günther’s material in \citet{gunther_grammar_1892} may, however, have influenced C. Strehlow’s later Arrernte case paradigm (\citeyear{strehlow_einige_1908}; \sectref{sec:key:9.2.3.2}), which broke a long history of Lutheran descriptive practice. C. Strehlow’s German editor, M. von Leonhardi, was conversant with Günther’s descriptions (\sectref{sec:key:9.3.4.1}, \sectref{sec:key:9.2.2}).

\section[W. Ridley’s grammars of Gamilaraay and of Turrubul]{W. Ridley’s grammars of Gamilaraay (\citeyear{ridley_kamilaroi_1875}; \citeyear{ridley_kamilaroi_1855-1}; \citeyear{ridley_kamilaroi_1856}) and of Turrubul (\citeyear{ridley_kamilaroi_1866})}
\label{sec:key:4.5}
\label{sec:4.5}

After the closure of Wellington Valley mission in 1842, the grammatical structure of languages spoken in New South Wales received little further attention until Rev. W. Ridley (1819--1878) began describing Gamilaraay in 1852. Ridley learnt Gamilaraay while working as a missionary travelling throughout New England, west of the Dividing Range in the north central region of New South Wales. He was never funded to establish a mission.

Educated at Kings College, University of London (B.A., 1842), Ridley developed an interest in missionary work after commencing studies in Law. Having been rejected by the London Missionary Society “because he had once held Plymouth Brethren beliefs” \citep{gunson_australian_2016-1}, Ridley was recruited to Australia by the influential Presbyterian clergyman Dr J. D. Lang (1799--1878) and was subsequently appointed Professor of Greek, Latin and Hebrew at the Australian College \citep[40]{austin_gamilaraay_2008}, a short-lived institution established by Lang. After ordination by the Presbyterian synod, Ridley ministered at Balmain in Sydney, and then in 1851 at Dungog, close to Newcastle, where friendship with an Aboriginal man named “Harry of Bungulgully” rekindled his interest in missionary work (\citealt{gunson_australian_2016-1}; \sectref{sec:key:1.1.3}).

Of all the grammarians who resided in Australia that are considered in this study, Rev. William Ridley is the most strongly connected to the European intelligentsia. His earliest published grammatical analyses of Gamilaraay were written as letters to influential British academics and were subsequently read to learned societies and published in their journals. The only other Australian corpus grammarian known to hold a relationship with an epicentre of European philological thought similar to Ridley’s connections is Carl Strehlow (\sectref{sec:key:9.2.2}).

Ridley fulfilled his evangelistic, ethnographic and linguistic passions while engaged as an itinerant minister in New England (1850--1852), and later (1853--1854) when conducting exploratory surveys in areas of New South Wales of his choice \citep[229--230]{harris_one_1994}. He travelled throughout the Liverpool Plains and the Darling Downs, investigating Aboriginal languages and customs, especially Gamilaraay. 

Ridley commented that the “fragmentary character of this contribution to the Philology of Australia” was due to “the shortness of the time spent in the research” \citeyearpar[v]{ridley_kamilaroi_1866} and described (ibid.: vi) his knowledge of some of the languages included in the publication as “limited”. The sparseness of Ridley’s analyses is partly due to the shorter time he spent immersed in the language compared to many other grammarians considered in this study, who lived among Aboriginal people at missions for years and sometimes decades. 

Ridley’s early linguistic material was produced during a period of diminishing descriptive activity and waning interest in Aboriginal Australians. Ridley’s materials (\citeyear{ridley_kamilaroi_1855}; \citeyear{ridley_kamilaroi_1855-1}; \citeyear{ridley_gurre_1856}), along with Teichelmann’s MSS (\citeyear{teichelmann_dictionary_1857}; \citeyear{teichelmann_annotations_1858}; \sectref{sec:key:5.2.1}), are the only grammatical works during the 1850s, and his lavishly produced grammar from 1866 is the only grammar published in Australia in the decades between Moorhouse (\citeyear{moorhouse_vocabulary_1846}; \sectref{sec:key:6.4}) and Taplin (\citet{taplin_narrinyeri_1874}; \sectref{sec:key:7.3.1}). The only other works produced in the 1860s are MS grammars of languages spoken in South Australia: Taplin’s grammar of Ngarrindjeri (\citeyear{taplin_vocabulary_1867}; \sectref{sec:key:7.3}) and Koch’s grammar of Diyari (\citeyear{koch_untitled_1868}; \sectref{sec:key:8.3.4}). The linguistic inactivity that characterises the decades between 1850 and 1870 stands in contrast to the preceding and later decades.

Yet Ridley published extensively, in both Australia and in Britain, on the languages, manners and customs of the Gamilaraay and other Aboriginal groups from north-central New South Wales and the region around Brisbane (\citeyear{ridley_kamilaroi_1855-1}; \citeyear{ridley_kamilaroi_1856}; \citeyear{ridley_kamilaroi_1866}; \citeyear{ridley_kamilaroi_1875}). Ridley’s grammars (\citeyear{ridley_kamilaroi_1855}; \citeyear{ridley_kamilaroi_1855-1}; \citeyear{ridley_kamilaroi_1866}; \citeyear{ridley_kamilaroi_1875}), along with Taplin’s grammars (\citeyear{taplin_vocabulary_1867}; \citeyear{taplin_notes_1872}; \citeyear{taplin_grammar_1878}; \sectref{sec:key:7.3}) occur within the earliest “survey-era of linguistics” in Australia \citep{mcgregor_encountering_2008}, in which attempts were made to systematically survey Australian languages (\citealt{taplin_1879a_nodate}; \citealt{smyth_aborigines_1878}; \citealt{curr_australian_1886}).

Missionary-grammarians in direct contact with Aboriginal people in the later decades of the nineteenth century held knowledge about Aboriginal languages and culture that was of increasing interest to Australian and international scholars. The ethnographic enquiries made by missionaries Ridley and Taplin contributed to international academic scholarship about Australian “primitive” people. 

Ridley and Taplin’s materials were disseminated broadly, and both men were regarded as experts on anthropological and linguistic matters by European intelligentsia. The Oxford linguist Friedrich Max Müller (1823--1900) approached Ridley directly for information about Australian languages \citep[109]{gardner_southern_2015}. \citet[96]{bleek1871} took Ridley’s Gamilaraay marriage “castes” as evidence that Australian languages were degenerate members of his “sex-denoting” class of languages. Ridley’s work was read closely by F. Müller, who specifically referred to Ridley in his earliest linguistic work \citeyearpar[8]{muller_reise_1867}. In 1872, L. Fison (1832--1907) circulated L. H. Morgan’s (1818--1881) expanded philological lists of kinship terms in questionnaires in the Australian press. The information supplied to Morgan by Ridley, as well as by Taplin, was utilised within the rapidly evolving discipline of anthropology \citep[105--108]{gardner_southern_2015}.

Like Taplin, Ridley republished the same material in multiple locations. \tabref{tab:chap4:ridley} summarises the dates of production and publication of Ridley’s linguistic materials. He made two different analyses of Gamilaraay morphosyntactic structure, the first \citeyearpar{ridley_kamilaroi_1855-1}, and the second \citeyearpar{ridley_kamilaroi_1855}, later published in 1866 and 1875. \citet[10]{austin_reference_1993} refers to grammatical notes given in the back of a copy of \textit{Gurre Kamilaroi}. These have not been cited in this current study. 

\begin{table}
    \begin{tabularx}{\textwidth}{p{1.75cm}p{1.75cm}p{2.25cm}Q}
        \lsptoprule
        Year \newline produced: & Year \newline published:  & Referred \newline to as: & Work \\
        \midrule 
        1853 & 1856 & 1856b & “Kamilaroi Tribe of Australians and Their Dialect”: Letter to T. Hodgkin in \textit{Journal of the Ethnological Society of London}\\
        \tablevspace
        1854 & 1855 & 1855b & “On the Kamilaroi language of Australia: Letter to T. H. Key in \textit{Transactions of the Philological Society}\\
        \tablevspace
        1855 & \stockcaps{MS} & 1855a \newline [republished in 1866, 1875] & “Kamilaroi Grammar and Vocabulary” Mitchell Library\\
        \tablevspace
        unknown & 1856 & 1856a & \textit{Gurre Kamilaroi} Primer\\
        \lspbottomrule
    \end{tabularx}
    \caption{Ridley's descriptions of Gamilaraay }
    \label{tab:chap4:ridley}
\end{table}

\noindent
Ridley’s earliest \textit{produced} description of Gamilaraay society and language (\citeyear{ridley_kamilaroi_1856}) was written in 1853 as a letter to Dr T. Hodgkin (1788--1866), member of the Royal College of Physicians, curator of the anatomical Museum at Guy’s hospital in London \citep{martin_hodgkin_2004}, and founding member of the Ethnological Society of London. Hodgkin was an early British proponent of the notion that language was a racial trait and that philological data provided information on the history of mankind. This work does not contain a grammar, but it does contain Ridley’s earliest account of ergativity (\sectref{sec:key:4.5.4.1}).

Ridley’s earliest \textit{published} Gamilaraay description (\citeyear{ridley_kamilaroi_1855-1}) was written in 1854 as a letter to Professor T. H. Key (1799--1875) of University College London, who was a founding member of the Society for Philological Enquiries – a precursor to the Philological Society of London, of which he was later president. The letter was read to the society and then published in the \textit{Transactions of the Philological Society}. While not a complete grammar, this work does provide case paradigms. Four of the work’s twelve pages discuss linguistic structure and are followed by an extended vocabulary. Note that this and Ridley’s later vocabularies, which follow his grammars, are arranged alphabetically by English entry. In this, Ridley’s work differs from most other vocabularies appended to corpus grammars, which arrange items in the Aboriginal language alphabetically, although Taplin also presented words in alphabetical order by English translation.

In 1855 Ridley produced a more substantial MS grammar (\citeyear{ridley_kamilaroi_1855}), which was later published with some modifications in 1866 (on pp. 3--16 of the text) and again in 1875 with some additional remarks in the introduction and in the description of the phonology and spelling system. 

Although Ridley’s grammatical analysis of Gamilaraay is most commonly referred to as 1875 (\citealt{austin_gamilaraay_2008}; \citealt{giacon_2014}), his analysis of Gamilaraay was finalised by 1855, during his early intense period of missionary engagement with Aboriginal people. His analysis was not developed further over the coming decades. Ridley’s knowledge of Australian languages did, however, broaden in the course of his enquiries. This most substantial of Ridley’s grammars is referred to here as ``\citet{ridley_kamilaroi_1875}'', although the MS grammar \citep{ridley_kamilaroi_1855} labels cases slightly differently from the published works.

Note that although Ridley uses “ng” to represent the velar nasals in the earliest publication (\citeyear{ridley_kamilaroi_1855-1}; \citeyear{ridley_kamilaroi_1856}) and engma in the 1866 and 1875 publications – represented by an inverted “G” – his early MS (\citeyear{ridley_kamilaroi_1855}) shows a handwritten engma. Thus, the absence of the symbol in the early publications resulted from type-setting limitations, rather than from an alteration in the orthography of choice. The only other earlier Australian grammar to employ the symbol is \citet{hale_languages_1846}. There is no indication that Ridley was familiar with Hale’s work.

Ridley’s works and R. H. Mathews” grammar (\citeyear{mathews_languages_1903}) are the most valuable of the older Gamilaraay sources. The lack of example clauses in these analytically sparse descriptions causes difficulty for the reconstruction of classical Gamilaraay structure \citep[5]{Giacon2014}, although Mathews (\citeyear{mathews_languages_1903}) provides a greater number of illustrative clauses than does Ridley.

\subsection{Gamilaraay primer \citeyearpar{ridley_kamilaroi_1856}}
\label{sec:key:4.5.1}

In 1856, Ridley published an extensively illustrated primer titled, \textit{Gurre Kamilaroi}, or ``Kamilaroi Sayings'' (\citeyear{ridley_kamilaroi_1856}). Portions of the primer were republished in \citet[31--33]{ridley_kamilaroi_1866} and Fraser later republished the entire primer as “Sentences in the Kamalarai Dialect” (\citeyear[Appendix F: 127--131]{ridley_gurre_1892}). 

This work, the second of its kind in Australia – following \citet{threlkeld_australian_1836} – was designed to teach school children to read Christian texts in their own language. 

\subsection{Ridley \citeyearpar{ridley_kamilaroi_1866}}
\label{sec:key:4.5.2}

Ridley’s 1866 publication \textit{Kamilaroi, Dippil and Turrubul languages spoken by Australian Aborigines} is the only work by Ridley devoted almost exclusively to language. In addition to the inclusion of his Gamilaraay MS finalised over a decade earlier, Ridley included lexical material for neighbouring Maric languages and Dippil (Waka-Kabic) and a brief grammar of Turrubul (Durubalic) (ibid.: 61--64). 

By 1866, Ridley’s comparative assessment of Australian languages was more developed than in his earlier publications. He observed: “The pronouns of the first and second-person are nearly the same all over Australia” (\citeyear[43]{ridley_kamilaroi_1866}), drawing upon the South Australian data of Grey and Teichelmann, referred to as ``Taihleman''. He also drew a parallel between the religious belief of the Turrubul and the Ngarrindjeri from “Point Macleay” (sic) in South Australia (\citeyear[65]{ridley_kamilaroi_1866}). Taplin (\sectref{sec:key:7.3}) had not at this time published any material from the South Australian mission. Ridley is likely to have been informed about Taplin’s material by Fison, with whom both Taplin and Ridley corresponded (\citealt[4--7, 9--12]{elkin_r_1975}). The work contains two appendices. Appendix A, “Family names, classifications, and marriage and law” (\citeyear[35--38]{ridley_kamilaroi_1866}), established Ridley’s reputation as an authority on Aboriginal social organisation and marriage. It was published six years before his first meeting with Fison. Appendix B, “Specimens of Languages Bordering on Kamilaroi” (ibid.: 39--44), provided a four-language comparative vocabulary of some forty lexical items, mainly body parts, fauna and pronouns. The languages are named “Kingki” and ``Paiamba'', both Maric languages spoken on the Darling Downs. There is additional lexical material for Bigambul, the northern most Central New South Wales language and for a language named ``Kogai'', now referred to as Mandandanji or Gunggari, another Maric language spoken on the Cogoon and Maranoa rivers in a more arid and linguistically under-sampled region \citep[825]{bowern_computational_2012} west of the Darling Downs towards the neighbouring Karnic languages. In this section Ridley shows that “the languages of neighbouring tribes differ very much, and are yet connected” (ibid.: 42) and notes the great similarity of pronouns (ibid.: 43) and that “bular or budela appears for `two' almost all over the country” (ibid.: 44).

For each of the languages given substantial treatment in this publication – Gamilaraay, Dippil and Turrubul – Ridley acknowledged the linguistic assistance of a colonist who, “during many years' residence among that people, had learned to converse with them in their own tongue” (\citeyear[v]{ridley_kamilaroi_1866}). For Gamilaraay, he thanked Rev. C. Greenway (1818--1905), who leased land on the Barwon River from 1848. Greenway’s analysis of Gamilaraay (1878; 1910--1912), some of which was published posthumously, has been characterised by \citet[41]{austin_gamilaraay_2008} as an unacknowledged reproduction of Ridley’s work. \citet[5]{Giacon2014}, however, believes the provenance is less certain. The situation remains unclear.

Ridley thanked Thomas Petrie (1831--1910), who had learnt Turrubul in Brisbane as a child (\sectref{sec:key:1.1.3}), for assistance with the Turrubul material. Thomas was the son of Andrew Petrie (1798--1872), a Scottish mechanic, who like Ridley had been brought to Australia by Lang \citep{hall_petrie_1974}. Ridley visited Andrew Petrie in Moreton Bay in 1855.

Ridley’s Turrubul grammar is substantially shorter than the Gamilaraay grammar. Without elaboration, \citet[63]{ridley_kamilaroi_1866} stressed the complexity of verb morphology, an area of the languages' grammar of which many early grammarians were aware their understanding was inadequate: “The voices, active, reciprocal, causative, permissive, \&c., are numerous; and the tenses are adapted to express various slight modifications of past and future tenses”. The extent to which Ridley’s analysis of Turrubul is based on material gathered from Aboriginal people or was made through Ridley’s comparison of Petrie’s translation of the Bible with his own Gamilaraay translations is not clear. \citet[vi]{ridley_kamilaroi_1866} stated: “Before and after receiving this help [from the colonists], the author communicated with the Aborigines in the districts where these three languages are spoken; and verified and extended, by his own observations, the information thus supplied”. The description of Turrubul concludes with a short ten-entry section headed ``Dialogue'', and a section headed “Paraphrases” containing translations of Genesis chapters 1- 3 and Luke 7--8. 

While working as an itinerant missionary, Ridley formed an association with J. G. Hausmann (1811--1901) and the pair sought to reopen the Zion Hill mission at Moreton Bay – present-day Brisbane – as an Anglican institution \citep[233--234]{harris_one_1994}. Hausmann, who had been trained by Gossner in Berlin, had been brought to Australia by Lang in order to establish Zion Hill (Nundah, ``German Mission'') in 1837. While the plan to reopen the Moreton Bay mission did not eventuate, it is possible that Ridley’s grammar of Turrubul (\citeyear{ridley_kamilaroi_1866}), spoken in the vicinity of Brisbane, was partly informed by the earlier acquisition of the language made by the Gossner-trained missionaries. In 1841, Hausmann’s colleague C. Eipper (1831--1894; \citeyear{eipper_statement_1841}) published a Turrubul wordlist (\citeyear[11]{eipper_statement_1841}) of less than one hundred words. Eipper’s pronominal forms, \textbf{atta} ``I'', \textbf{inta} ``thou'', \textbf{ariba} ``belonging to me'', and \textbf{enuba} ``belonging to thee'', accord with those recorded by \citet[62]{ridley_kamilaroi_1866}. Note here that missionary Eipper was assisted in learning Turrubul by J. C. S. Handt \citep{ganter_eipper_2016}, a Lutheran minister who had trained at Basel, and worked at Wellington Valley mission (1832--1836) before Günther’s arrival (\sectref{sec:key:4.1}).

For Dippil, Ridley thanks a blacksmith named James Davies [Davis] (1808--1889), who in 1828, three years after arriving in New South Wales as a seventeen-year-old convict, was sent to Moreton Bay gaol, from where he soon escaped. Sustained by Aboriginal people from the region for thirteen years, Davis lived beyond the frontier, learning Aboriginal languages and tribal law before being found by Andrew Petrie in 1842. Ridley was evidently tenacious in collecting linguistic data, securing a session with Davis, who, after successfully re-integrating into colonial society, otherwise refused to discuss his experiences in the bush \citep[140--141]{petrie_tom_1904}. As with Turrubul, Ridley’s initial field method of collecting data from Davis was to read verses of the Bible, which it is said Davis could not understand. Subsequently, Ridley collected some Dippil names of animals “and things like that” (ibid.: 141). Ridley’s Dippil investigations may have been further hampered by his inability to locate Aboriginal speakers of the language. Petrie recalls:

\begin{quote}
    On Mr. Ridley’s return from his trip he told Father that nearly all the blacks he came across understood what he (Father) [T. Petrie] had told him, but on the contrary, he met only two who understood the words from Davis. This was because he had gone too far inland. \citep[141]{petrie_tom_1904}
\end{quote}

That Ridley’s Dippil material (\citeyear[47--57]{ridley_gurre_1856}) does not contain a grammar or a translation of religious text is telling of the limitations of his fieldwork.

\subsection{Ridley \citeyearpar{ridley_kamilaroi_1875}}
\label{sec:key:4.5.3}

\citegen{ridley_kamilaroi_1875} republication of the Gamilaraay grammar is given alongside additional non-grammatical material. Written within the era of survey linguistics, and in the same year that Taplin circulated his questionnaire in order to collect linguistic data from a range of languages (\sectref{sec:key:7.3.2}), the 1875 work includes a “comparative table in twenty languages” (\citeyear[119--134]{ridley_kamilaroi_1875}). Ridley was by this stage acquainted with the contents of the MS grammars of Wiradjuri (\sectref{sec:key:4.1}). In 1871, Ridley, who was about to travel into Gamilaraay country for a period of “a few weeks” \citep[v]{ridley_kamilaroi_1875} was visited by Fison. Fison requested that Ridley further investigate marriage systems during his travels, the official purpose of which was to “make philological investigations requested by the Colonial Secretary of New South Wales on behalf of Professor Max Mueller of Oxford” \citep[30]{langham_building_1981}. The 1875 vocabulary follows the same format as the earlier vocabularies but contains additional entries. The type of lexical material added to the 1875 work is telling of the fields of enquiry Ridley pursued during his 1871 expedition. Notable is the greatly extended section headed “Man: his distinctive and relative names” (\citeyear[17]{ridley_kamilaroi_1866}; 1875: 18). Ridley expands the number of entries given in this section in 1875 by almost 50\%, now including terms for “man” and “boy” at different stages of growth and initiation, and distinct terms for older and younger brother and sister.

\subsection{Ridley’s descriptions of case}
\label{sec:key:4.5.4}

Ridley presented three slightly different Gamilaraay case paradigms and one of Turrubul. \tabref{tab:chap4:terminology} shows the forms that Ridley included in his Gamilaraay paradigms and their current analysis.

\begin{table}
{\footnotesize
    \begin{tabularx}{\textwidth}{p{1.3cm}Q@{~}lQp{1.8cm}@{}}
        \lsptoprule
        \textbf{Form of\newline   suffix   shown by   Ridley} & \textbf{Case label assigned to form by \citet{ridley_kamilaroi_1855-1}}& \textbf{Translated as:} & \textbf{Case label assigned to form by \citet{ridley_kamilaroi_1855,ridley_kamilaroi_1866,ridley_kamilaroi_1875}} & \textbf{\citegen{austin_reference_1993}\newline analysis} \\
        \midrule 
        Ø & nominative & “a X” & nominative & absolutive \\
        \textbf{-du} & nominative 2 & “an X” (agent) & nominative 2 & ergative \\
        \textbf{-ŋu} & genitive & “of (or belonging to) X” & possessive & ** \\
        Ø & accusative \& vocative* & “a X” & objective (1866; 1875) accusative (1855a)  & absolutive \\
        \textbf{-go} & motion to & “to an X” & listed under “accusative” but unnamed & dative/ \newline allative \\
        \textbf{-di} & ablative & “from a X” & \hfill ''\hfill ~   & ablative \\
        \textbf{-da} & 	listed under “ablative” but unnamed & “in a X” & \hfill ''\hfill ~ & locative \\
        \textbf{-kunda} & listed under “ablative” but unnamed & “with a X” (stopping) & \hfill ''\hfill ~  &  \\
        \textbf{-ŋunda†} &  – & with a X (going) & \hfill ''\hfill ~  & personal declension locative\newline \citep{Giacon2014} \\
        \textbf{-kale} & – & with a X (going) &\hfill ''\hfill ~  & ? \\
        \lspbottomrule
    \end{tabularx}
    }
	\legendbox[1]{$^\ast$Ridley writes “like 1st Nom”. $^{\ast\ast}$ \citet[64]{austin_reference_1993} who describes the possessive function as marked with –\textit{gu}, remarks that Ridley recorded the function as marked with –\textit{ngu}. $^\dagger$The suffix –\textit{ngunda} is only given in the 1855 MS (\citeyear{ridley_kamilaroi_1855}).}
    \caption{Terminology assigned to Gamilaraay case suffixes } 
    \label{tab:chap4:terminology} % orig = Fig. 71
\end{table}


Here Ridley follows the practices of previous grammarians in New South Wales. His paradigms are similar to those given in Günther’s Wiradjuri grammars (\citeyear{gunther_native_1838}; \citeyear{gunther_lecture_1840}; \sectref{sec:key:4.4.3}) and in Threlkeld’s Awabakal grammar (\citeyear{threlkeld_australian_1834}; \sectref{sec:key:3.3.3}). Ridley presented enlarged paradigms in order to include forms marking functions not associated with SAE case systems. He presents eight-case paradigms.

\largerpage
Unlike most other early PN grammarians, Ridley did not present the word class ``pre/post-positions''. The absence is characteristic of the sparseness of his analysis generally. 

\begin{figure}[b]
\includegraphics[width=.8\textwidth]{figures/Fig72.png}
\caption{Ridley’s earliest case paradigm (\citeyear{ridley_kamilaroi_1855-1}: 74)}
\label{fig:4:72}
\end{figure}

His earliest published Gamilaraay case paradigm of nouns (\citeyear[74]{ridley_kamilaroi_1855-1}; see \figref{fig:4:72}) and his Turrubul case paradigm (\citeyear[61]{ridley_kamilaroi_1866}; \figref{fig:4:73}) maintain the format used by Günther (\figref{fig:4:58}) and by Threlkeld (\figref{fig:3:34}) in presenting multiple forms under the heading ``ablative'', although Ridley does not assign numbers or letters to the additional ablative cases as his predecessors had done. These large paradigms define the New South Wales descriptive school.



\begin{figure}
\includegraphics[width=.8\textwidth]{figures/Fig73.png}
\caption{Ridley’s Turrubul case paradigm (\citeyear{ridley_kamilaroi_1866}: 61)}
\label{fig:4:73}
\end{figure}

In the MS grammar (\citeyear{ridley_kamilaroi_1855}) and later publications (\citeyear{ridley_kamilaroi_1866}; \citeyear{ridley_kamilaroi_1875}), Ridley follows Threlkeld (\sectref{sec:key:3.3.5}) in choosing a noun meaning “eagle” to illustrate case marking.


Despite the publication of Ridley’s analyses (\citeyear{ridley_kamilaroi_1866}; \citeyear{ridley_kamilaroi_1875}), his inclusion of suffixes marking functions not carried by the case systems of SAE languages and the abandonment of the word-class “pre/post-positions” had no discernible influence on grammars of Diyari (\citealt{koch_untitled_1868}; \citealt{schoknecht_grammar_1947}; \citealt{flierl_christianieli_1880}; \citealt{reuther_ms_1899}) or of Arrernte (\citealt{kempe_galtjintana-pepa_1891}; \citealt{strehlow_aranda-und_1907}; \citealt{strehlow_aranda_1944}), written by Lutheran missionaries in Central Australia in the decades after his works were published. However, there is some evidence that his analysis \textit{may} have influenced \citegen{livingstone_grammar_1892} grammar of Minjangbal (\sectref{sec:key:4.6.1}) and \citegen{roth_ethnological_1897} grammar of Pitta-Pitta, and subsequently later grammars of languages spoken in Queensland (\sectref{sec:key:10.1}).

\subsubsection{Ergativity} 
\label{sec:key:4.5.4.1}

Ridley’s earliest account of ergative function and “the difference between the two nominative cases” given in his earliest-written publication occurs in a section headed “phrases” without an accompanying case paradigm (\citeyear[292]{ridley_kamilaroi_1856}; \figref{fig:4:74}). His explanation that the nominative case was given in answer to the question “what’s that?” follows \citet[14]{threlkeld_australian_1834} and had similarly been employed by Günther (\sectref{sec:key:4.4.4}). A later explanation of the ergative case as “the agent of the act described in the following verb” (\citeyear{ridley_kamilaroi_1855}; \figref{fig:4:75}) is, however, further developed than Threlkeld or Günther’s elucidations.

\begin{figure}
\includegraphics[width=.8\textwidth]{figures/Fig74.png}
\caption{Ridley’s earliest account of ergativity (\citeyear{ridley_kamilaroi_1856}: 292)}
\label{fig:4:74}
\end{figure}


In each of Ridley’s paradigms, he followed Threlkeld and Günther in placing the ergative case in second position after the nominative at the top of the paradigm. He names the case the “2nd nominative”, following one of Threlkeld’s naming practices. 

Ridley placed the term “agent” in brackets after the form for clarification (\figref{fig:4:75}), as had the earliest grammarians in South Australia. 

\begin{figure}
\includegraphics[height=.52\textheight]{figures/Fig75.png}
\caption{Ridley’s later case paradigm (\citeyear{ridley_kamilaroi_1855}: 6)}
\label{fig:4:75}
\end{figure}


\subsubsection{Ridley’s later case paradigms}

The nominal case paradigms that Ridley presented in the MS (\citeyear{ridley_kamilaroi_1855}; \figref{fig:4:75}) are slightly different from those republished in 1866 and 1875 (\figref{fig:4:76}). The paradigms in all works labelled only the three syntactic cases and the possessive. The earlier MS (\citeyear{ridley_kamilaroi_1855}) gave unnamed peripheral case forms under the term ``accusative''. The paradigms in published works (\citeyear{ridley_kamilaroi_1866}; \citeyear{ridley_kamilaroi_1875}) listed multiple “objective” forms, which were translated with English prepositional phrases. This presentation differed from Threlkeld’s and that given by all previous Australian grammarians. Ridley’s conception of multiple “objective” cases presents a unique and innovative paradigm. It is this presentation that may have influenced Roth (\sectref{sec:key:10.1.1}).

The earlier MS paradigms (\citeyear{ridley_kamilaroi_1855}) also gave two additional case forms not shown in the publications. One form is marked with the \textit{–ngunda} reclaimed as the personal locative \citep[36--37]{Giacon2014} and the other is suffixed with \textit{–kale}, which appears not to have been reclaimed.


\begin{figure}
\includegraphics[width=.8\textwidth]{figures/Fig76.png}
\caption{Later publications of Ridley’s MS case paradigm (\citeyear{ridley_kamilaroi_1866}: 5; \citeyear{ridley_kamilaroi_1875}: 6)}
\label{fig:4:76}
\end{figure}


\subsection{Bound pronouns}
\label{sec:key:4.5.5}

Ridley did not describe any bound pronominal forms in Gamilaraay, despite his knowledge of Hebrew, which is described in nineteenth century grammars as having “separate” pronouns and “pronominal suffixes” \citep[105--109]{gesenius_gesenius_1910}. Close reading of Ridley’s grammars does, however, suggest the existence of a set of bound forms. They are always shown as a free word (\ref{ex:key:6}) and Ridley’s earliest record (\citeyear{ridley_kamilaroi_1855-1}) shows both 2\textsc{sg} and 2\textsc{dl} forms as vowel initial (ibid.: 75).

\ea\label{ex:key:6}
\gll Murruba inda \\
Good 2\textsc{sg}.\stockcaps{NOM} \\
\glt `you are good' \\
\citep[39]{ridley_kamilaroi_1875}
\z 	

\hspace*{-2.3pt}Ridley also observed that “the nasal at the beginning [of a pronoun, e.g., \textit{nginda}] is sometimes softened down very much, especially in the second-person, which may be regarded at times as \textit{inda}” (\citeyear[6]{ridley_kamilaroi_1875}). The reclamation of a bound pronominal system in Gamilaraay, which is based on a range of sources \citep[129--130]{Giacon2014}, could not have been reconstructed from Ridley’s materials alone (Giacon, pers. comm. 8/12/2016). This fact brings into question the certainty with which other languages of which there is \textit{only} early source material -- e.g., Ngayawang (\sectref{sec:key:6.4}) -- can be said \textit{not} to have exhibited such systems.

\subsection{Concluding remarks}
\label{sec:key:4.5.6}

In comparison with earlier grammars of PN languages, Ridley’s works contain little analytical comment supplementing the paradigms, and provide few example clauses. His most frequently cited work (\citeyear{ridley_kamilaroi_1875}) is only twelve pages long. With the exception of some of Mathews' grammars of PN languages, Ridley’s analyses of Gamilaraay are less detailed and show less grammatical understanding and insights than other grammars of languages spoken in New South Wales. Ridley’s account of case allomorphy is inferior to that given by Threlkeld in Awabakal (\citeyear{threlkeld_australian_1834}; \sectref{sec:key:3.3.5}) and by Günther in Wiradjuri (\citeyear{gunther_native_1838}; \citeyear{gunther_lecture_1840}; \sectref{sec:key:4.4.2}). Ridley’s descriptions of bound pronouns (\sectref{sec:key:4.5.5}) and of case allomorphy (\sectref{sec:key:4.4.2}) are inferior to that later given by \citet{mathews_languages_1903}, although Mathews may have regularised his material. It is little wonder that \citet{ray_aboriginal_1925} evaluated Ridley’s work as unsatisfactory.

The format of Ridley’s earliest case paradigms and his explanation of ergativity show influence from Threlkeld’s grammar. 

Despite the renown of Ridley’s ethnographic investigations, and despite the publication of his material, his grammatical analyses had limited influence on later grammatical of Australian languages. There is no suggestion that grammarians of Diyari or of Arrernte at either of the two South Australian missions (\sectref{sec:key:8.1}) were aware of Ridley’s publications. His presentation of peripheral case forms in his later published paradigms (\citeyear{ridley_kamilaroi_1866}; \citeyear{ridley_kamilaroi_1875}; \figref{fig:4:76}) may, however, have influenced \citet{roth_ethnological_1897} who adopted a schema showing multiple case forms labelled “objective” in his description of Pitta-Pitta case (\citeyear[4]{roth_ethnological_1897}; \sectref{sec:key:10.1.1}).

\section{Livingstone’s grammar of Minjangbal (\citeyear{livingstone_grammar_1892})}
\label{sec:key:4.6}

The Rev. Hugh Livingstone’s grammar of ``Minyung'', now known as Minjangbal, a middle Clarence dialect of Bandjalang \citep[142]{crowley_middle_1978}, spoken on the northeastern coast of New South Wales, is the only major work in Fraser’s compilation of grammatical material of PN languages (\citeyear{fraser_australian_1892}) that was specially written for that publication. While other works included in Fraser – Günther’s Wiradjuri grammar and Homann’s Diyari pronominal paradigm – had previously only been available in MS, they were not specifically produced for that volume.

Little is known about the Rev. Hugh Livingstone, or about the circumstances in which he collected his material, other than that he was a Presbyterian Minister in Lismore, in Minjangbal country, between 1876 and 1886, and later in Western Victoria, from where he wrote the material for Fraser. 

Museum Victoria holds the MSS of his grammatical analysis of “Minyung” (1876--1866) and vocabularies (no date) in the Spencer and Gillen collection.\footnote{Thanks to Margaret Sharpe for bringing this to my attention.}  It is not known how these materials came into Spencer’s possession. The MSS (1876--1886) have not previously been identified as containing the grammatical analysis that was published in \citet{fraser_australian_1892}. The MS vocabulary (no date) contains at least three times the volume of material published in \citet{fraser_australian_1892}. It appears the MS vocabulary was not viewed by \citet{smythe1949}, \citet{cunningham1969}, \citet{geytenbeek1971}, or \citet{crowley_middle_1978}.

Livingstone’s grammatical materials (\citeyear{livingstone_notitle_1876}; \citeyear{livingstone_notitle_1876-1}) are poorly organised. Originally written partly on the reverse pages of a long essay on Futurism, all pages have now lost their original order. The grammar is spread throughout two files, which are given different record numbers by Museum Victoria. The documents include more than a single attempt to write a complete grammar, and one \citeyearpar{livingstone_notitle_1876} is introduced as a fourth revision. Further, the lucidity of material varies so greatly as to suggest that Livingstone was in different states of mind when compiling it. 

Nevertheless, the documents clearly contain the material that was edited and published by Fraser. Like Livingstone’s MS vocabulary (no date), the MS grammars (\citeyear{livingstone_notitle_1876}; \citeyear{livingstone_notitle_1876-1}) contain additional examples that were not included in \citet{fraser_australian_1892}. Of ``gender'', for example, \citet{livingstone_notitle_1876} wrote: “The distinction between the masculine and the feminine is denoted in Minyung, as in English, either by the use of a different word or by the use of the feminine termination, -\textbf{gun}”. Livingstone in \citet[6]{fraser_australian_1892} stated: “Gender / There are two ways by which the feminine is distinguished from the masculine – either by a different word, or by adding the termination –\textbf{gūn}”. While the MS then listed ten pairs of gendered terms, only five were included by Fraser. The section headed “Table of relationships in MINYUG” \citep[21]{livingstone_grammar_1892} is clearly taken from a longer section in the MS \citeyearpar{livingstone_notitle_1876-1} which shows Livingstone’s method of eliciting his genealogical material, recording the conversation between Livingstone and his informants. For example: “What do you call your mother?”, “Waidyong”, “What do you call your mother’s sister?” “Waidyong, all the same”. The MS also gives a table of marriage classes, which was not included in Fraser, and other anthropological material, including “the marking of men”.

The document containing the vocabulary (no date) must have been written after 1912, because the discussion refers to R. R. Marett’s publication \textit{Anthropology} (\citeyear{marett_anthropology_1912}).

Substantial portions of Livingstone’s analysis in \citet{livingstone_grammar_1892} are also \textit{not} contained in the Museum Victoria MSS. Missing are the important sections describing case: “Suffixes to nouns” (\citeyear[9--11]{livingstone_grammar_1892}) and the description of adjectival agreement with noun classes (\citeyear[4--5]{livingstone_grammar_1892}). The nature of Fraser’s edit to the material is therefore difficult to ascertain. The extent to which the material in Fraser replicates Livingstone’s original analysis can, however, be deduced by comparing these documents with the way in which Fraser is known to have standardised Günther’s description of case in Wiradjuri (\sectref{sec:key:4.4.3}) and Taplin’s Ngarrindjeri material (\sectref{sec:key:7.3}; \figref{fig:key:7-144}). Fraser’s presentation of Livingstone’s material is starkly dissimilar in format to Fraser’s edited publications of these other languages. It therefore seems likely that the unconventional description of Minjangbal case, given by Fraser, is Livingstone’s original analysis.

\subsection{Livingstone’s analysis of Minjangbal (\citeyear{livingstone_grammar_1892})}
\label{sec:key:4.6.1}

Livingstone’s analysis was informed by an existing knowledge of Australian Aboriginal languages. In an introductory passage, Livingstone justifies his abandonment of the traditional descriptive framework on the basis of the agglutinative nature of Australian languages:
\begin{quote}
    It is well known that the Australian dialects are agglutinative, everything in the nature of inflection being obtained by suffixes. To this, Minyuġ is no exception; so that if I give an account of its suffixes, that is nearly equivalent to giving an exposition of its grammar. It will therefore, be convenient to take, first, such suffixes as are used with the noun and its equivalents, and, afterwards, those that may be regarded as verbal suffixes. The words that take what may be called the noun-suffixes are (1) Nouns, (2) Adjectives, and (3) Pronouns. \citep[3]{livingstone_grammar_1892}
\end{quote}

The first of the “suffixes to nouns” listed by Livingstone is the ergative inflection, of which he states (\citeyear[9]{livingstone_grammar_1892}) is “usually said to be the sign of the agent-nominative case but it also denotes an instrumental case”. Here Livingstone distinguishes clearly between two case functions marked by the same case form. 

His comment that the ergative case was “usually” termed “agent-nominative” is odd, since the term had previously only been used \textit{in discussions} of case by Threlkeld, and by Meyer (\citeyear[38]{meyer_vocabulary_1843}; \sectref{sec:key:6.1.2.6}). The term “agent” had only been used in description of case by members of the Adelaide School (\sectref{sec:key:5.1}) and by Ridley. The terms “agent” or “agent nominative” had not been used as a case label in any grammar written in Australia. The term “nominative agent” had, however, been employed by Fr. Müller (\citeyear[7, 20]{muller_grundris_1882}; \figref{fig:key:7-77}) in his ninety-four-page discussion and collation of existing PN grammatical material written in German. Müller replaced the original case labels in the source paradigms and employed the term ``nominative agent'', but only when describing Awabakal and Wiradjuri spoken in New South Wales. The term “agent” (agens) had previously been used in Oihenart’s seventeenth century descriptions of Basque as well as in an 1820 description of Hindi (see \citealt[198]{lindner_komposition_2013}). Livingstone’s choice of terminology suggests that his broad knowledge of Australian languages was informed by F Müller’s most comprehensive treatment of the topic.
\begin{figure}
\includegraphics[width=.8\textwidth]{figures/Fig77.png}
\caption{F. Müller's rearrangement of Threlkeld's Awabakal case paradigm (\citeyear{muller_grundris_1882}: 7)}
\label{fig:4:77}
\end{figure}


By the turn of the century the term “nominative agent” had gained some currency in Australian grammars of languages spoken in New South Wales and Victoria. The term was employed by \citet{mathews_languages_1903}.

When describing the marking of cases on nouns, Livingstone (in \citealt[9--11]{livingstone_grammar_1892}) initially listed suffixes and explained their function. This method of describing case had first been employed by Meyer (\citeyear{meyer_vocabulary_1843}; \sectref{sec:key:6.1.2.1}).

\citet[Part IV, 14]{fraser_australian_1892} then presented a conventionally organised paradigm, at the end of the discussion of nominal morphology. It is probable that this section had not originally been included by Livingstone. Fraser uses the case terminology he adopted from Threlkeld (\citeyear{threlkeld_australian_1834}; \figref{fig:3:34}) and used when editing Taplin’s Ngarrindjeri grammar (\figref{tab:chap3:Awabakal}). He gives, for example, a case termed “dative 1” (dative) but does not present numbered ablative cases.

\subsubsection{Grammatical gender}
\label{sec:key:4.6.1.1}
\label{sec:4.6.1.1}

Livingstone’s grammar evidences an intelligent arrangement of the morpho-syn\-tac\-tic structure. Minjangbal is among a handful of Pama-Nyungan languages that exhibit systems of noun classes in which agreement is marked on a nominal modifier \citep[450--453]{dixon_australian_2002}. Livingstone is alone among early PN grammarians in encountering a language that had noun classes. Livingstone’s “classification” of Minjangbal nouns and adjectives (\citeyear[4--5]{livingstone_grammar_1892}) tabulates the agreement of adjectives with four classes of noun. Current analysis of Minjangbal noun classes \citep[43--45]{crowley_middle_1978} is based entirely, and somewhat tentatively, on \citet{livingstone_grammar_1892}, which is the only source for this Bandjalang variety.

Minjangbal is also among the small group of about a dozen Pama-Nyungan languages that make a two-way gender distinction in third-person pronouns \citep[461]{dixon_australian_2002}. Despite describing a system of grammatical gender on nouns, and a gender distinction in third-person pronouns, Livingstone did not present either of these systems as ``gender''. This is curious, given that the four noun classes are, as Livingstone describes them, semantically determined largely by the masculine/feminine, animate/inanimate oppositions underlying Indo-European gender systems \citep[61]{kurzova_1993}. The term “gender” is used within traditional grammar to describe both the property of nouns with which other word classes show agreement, as well as lexical variation for biological gender. (e.g. \citealt[10--11]{gildersleeve_latin_1895}; \citealt[19--32]{ramshorn_lateinische_1824}). Livingstone reserved the term “gender” for a discussion of lexical gender and the naming of male and female pairs: 

\begin{quote}
    There are two ways the feminine is distinguished from the masculine- either by a different word or by adding the termination –gun … \citep[6]{livingstone_grammar_1892}
\end{quote}

This was the common application of the term “gender” in many early grammars of PN languages. Consequently, the category “gender” is maintained in a body of early Pama-Nyungan grammars of languages with no system of gender. Interestingly, this body does not include the works of German speaking missionaries (\citealt{teichelmann_outlines_1840}; \citealt{meyer_vocabulary_1843}; \citealt{schurmann_letter_1844}; \citealt{kempe_galtjintana-pepa_1891}; \citealt{gunther_grammar_1892}; \citealt{strehlow_einige_1908}; n.d.). The grammars that maintain the category “gender” give lexical pairs which refer to different genders of the same type: husband/wife, daughter/son, male kangaroo/female kangaroo etc. (\citealt[10]{threlkeld_australian_1834}; \citealt[6]{livingstone_grammar_1892}; \citealt[15]{roth_ethnological_1897}; \citealt[324]{mathews_languages_1907}).
T. G. H. Strehlow gives a similar and substantial discussion of lexical gender under the heading ``Absence of Gender'', explaining that the language “has to add” adjectival modifiers denoting male and female to the names of species. He does this despite recognising that “[t]he Aranda nouns know no distinctions of gender” \citep[59]{strehlow_aranda_1944}. R. H. Mathews, similarly fills the prescribed category with a description of adjectival modifiers. The fact that “exponents of nominal Gender were recognisable members of another part of speech was irrelevant” \citep[192]{koch_r_2008}.

\section{Concluding remark}
\label{sec:key:4.7}

Threlkeld’s enlarged case paradigms providing positions for case forms marking function that are not carried by morphological case systems in SAE languages, his placename of ergative case forms at the top of the paradigm alongside nominative forms (\sectref{sec:key:3.3}), and his method of clarifying ergative function by the posing of questions that the ergative and nominative forms would be given in answer to (\sectref{sec:key:3.3.7.1}) are descriptive techniques that were employed by both Ridley and Livingstone.

The following chapters show that grammars of languages belonging to country outside New South Wales convey PN case systems differently from the grammars by \citet{threlkeld_australian_1834}, Günther (\citeyear{gunther_native_1838}; \citeyear{gunther_lecture_1840}), and Ridley (\citeyear{ridley_kamilaroi_1875}; \citeyear{ridley_kamilaroi_1855-1}; \citeyear{ridley_kamilaroi_1856}). The description of case and ergativity across the grammars examined in Chapters 3 and 4 is found to be diagnostic of a nineteenth century school of descriptive practice operating in New South Wales.
