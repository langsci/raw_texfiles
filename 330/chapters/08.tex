\chapter{Grammars of Diyari (1868--1899)}
\label{chap:key:8}

This chapter presents several grammars of Diyari (\citealt{koch_untitled_1868}; \citealt{schoknecht_grammar_1947}; \citealt{flierl_dieri_1880}; \citealt{reuther_dieri_1894}; \citeyear{reuther_ms_1899}) and other Karnic languages (\citealt{flierl_dieri_1880}; \citealt{reuther_ms_1901}; \citealt{reuther1901b}) that were written by Lutheran missionaries at Bethesda, east of Lake Eyre in South Australia, as well as the grammars of Diyari written by European philologists (\citealt{planert_australische_1908}; \citealt{gatti_lingua_1930}), whose work was informed by the missionaries' analyses.

Comparison of the analysis of morphosyntactic features (Section \ref{bkm:Ref74222915}) demonstrates that grammars of Diyari produced at the mission are generally descriptively and analytically homogenous, while grammars made by European philologists are descriptively innovative. The discussion considers the influence of \citegen{teichelmann_outlines_1840} grammar (\chapref{chap:key:5}) on \citegen{koch_untitled_1868} earliest description of case and of ergativity in Diyari. Koch’s grammar, which is presented in this study for the first time, was largely reproduced by later missionaries, although Reuther’s grammar (\citeyear{reuther_dieri_1894}), the last and best-known missionary description of Diyari, shows independent influence from elsewhere within the Lutheran School.

\section{Two inland Lutheran South Australian missions}
\label{sec:key:8.1}\label{bkm:Ref514609803}\label{bkm:Ref456104995}

Grammars of Diyari and of Arrernte were written at the two inland missions established by the Lutheran Church in the arid north of South Australia in the second half of the nineteenth century. These are the Bethesda mission (formerly Hermannsburg), Lake Killalpaninna (\textit{Kirlawirlpanhinha}), Kopperamanna (\textit{Kapa\-rra\-maranha}), and Cooper Creek (1866--1915), situated to the east of Lake Eyre, and Hermannsburg mission (later Finke River; 1877--1982), situated in the Western MacDonnell Ranges (\figref{fig:key:1-3}). These Lutheran missions and the Cape Bedford Lutheran mission in Queensland (1886--1942) (\sectref{bkm:Ref514607997}) endured longer than most other missions at which early grammars were written.

The longevity of the Bethesda and Hermannsburg missions fostered the development of a tradition of grammatical description of Diyari and of Arrernte at each mission respectively. Grammars were written as pedagogical tools for learning the language and for preparing sermons, and newly arrived missionaries copied out existing grammars, sometimes altering the orthography and the analysis. In a pre-academic era of linguistic description, and especially within the missionary sphere, unattributed borrowing from an earlier document was unproblematic. Consequently, establishing the original authorship of an analysis takes careful comparative study of the sources. At both missions later, better-known missionaries have been over-credited with having made grammatical analyses, while the analytical achievements of earlier, lesser-known missionaries, who made the inaugural grammatical descriptions, have been overlooked historically (see for example, \citealt[87]{kenny_arandas_2013}; \citealt[527]{hill_broken_2002}). By examining the differences between grammars of the same language, this and the following chapter assign intellectual provenance to the different analyses of Diyari and Arrernte.

Second-generation missionaries at each mission also described languages belonging to country adjacent to the mission site, but which came within the extended reach of the mission. At Bethesda, Flierl described Wangkangurru \citeyearpar{flierl_dieri_1880} and Reuther described both Wangkangurru \citeyearpar{reuther_ms_1901} and Yandrruwandha \citeyearpar{reuther_ms_1901} as well as collecting vocabulary lists of Arabana, Thirrari, Ngamini and Yawarrawarrka. At Hermannsburg, C. Strehlow described Luritja \citeyearpar{strehlow_notitle_1910}.

The endurance of these missions was due to their remoteness, and to the ongoing support received from members of European Lutheran congregations in the settled southern districts of South Australia, with whom mission staff held close familial and community connections. Although the Dresdner missionaries (Teichelmann, Schürmann and Meyer, \sectref{sec:key:5.1}) had been instructed to “gather German settlers into congregations, which would provide a support base for Aboriginal work” \citep[9]{lockwood_vision_2007}, the fledgling migrant German populations around Adelaide had been unable to offer the early Dresdner missionaries the support they required. But by the second half of the nineteenth century, South Australian Lutheran congregations were sufficiently vigorous and were committed to sustaining the inland missions. Once established, the Bethesda and Hermannsburg missions also benefitted from the loyalty of Diyari and Arrernte Christian ``converts'', whose work at the mission stations was essential to their economic viability.

Twenty years of Lutheran missionary inactivity had followed the initial and short-lived attempts of the Dresdners to Christianise Aboriginal people in South Australia (1838--1848). Missionary interest in Central Australia was rekindled by the first crossing of the continent from south to north by J. M. Stuart (1815--1866) in 1861--62, and by the encroachment of pastoralism into the arid interior. A mission site at Coopers Creek to the east of Lake Eyre was considered suitably remote from the disruptive influence of European populations. The site was also chosen as a reciprocal gesture of Christian values towards the Aboriginal people who had saved King, the only survivor of the Burke and Wills expedition \citep[137]{clark_aboriginal_2013}.

Despite the decades separating the closure of the South Australian Dresdner missions (1848) and the second wave of inland mission activity, continuity of support from the Dresdner grammarians provided an intellectual link between these two phases of mission activity in South Australia. Meyer, who wrote a grammar of Ngarrindjeri (\citeyear{meyer_vocabulary_1843}; \sectref{bkm:Ref456096204}), was president of the Bethany-Lobethal Synod (1848--1861; later the Evangelical Lutheran Synod of Australia, henceforth ELSA) and was especially influential in establishing the alliance with the Hermannsburger Missionsgesellschaft (Hermannsburg Mission Society, henceforth HMS) required to re-commence mission work. He is also known to have corresponded with Rev. W. Löhe, founder of Neuendettelsau Mission Institute, which trained the third wave of Lutheran missionaries in South Australia (J. \citealt[249]{strehlow_tale_2011}). Schürmann, who had earlier written a grammar of Barngarla (\citeyear{schurmann_vocabulary_1844}; \sectref{bkm:Ref74127491}), had been the Church’s first choice of missionary for the first inland Lutheran mission, but at the age of 48 he advised that two “young men of courageous faith” should be sourced from Leipzig or the Hermannsburg Mission Society in Germany \citep[60]{zweck_for_2012}. Teichelmann, who co-authored the earliest South Australian grammar with Schürmann (\citeyear{teichelmann_outlines_1840}; \chapref{chap:key:5}), was present at the 1866 dedication in Tanunda of the HMS missionaries who were to establish the mission to the Diyari (\citealt{proeve_work_1952}: 54). This continuity is relevant when assessing the potential influence of the Dresdner missionaries on the grammatical descriptions of their Lutheran successors.

Theological schisms and tentatively established alliances between factions of the Australian Lutheran synods threatened the continuity of each of the inland missions at different stages. These developments are relevant to this examination of the grammatical descriptions made by Lutheran missionaries, since the administering synod determined from which German seminary missionaries were recruited. Seminaries provided different degrees of linguistic training, with possible consequences for the type of linguistic material produced. Further, staff trained at the same German seminary and administered by the same Australian synod were moved between missions. This may have facilitated the dissemination of grammatical understanding across language boundaries.

\largerpage
In order to establish the earliest inland mission in the Lake Eyre Basin, a union between distinct South Australian synods (1863) was required before L. Harms, the director of the HMS in Germany, would agree to supply missionaries (\citealt{brauer_history_1956}: 223--224; \citealt{proeve_work_1952}: 33--38). In {1874} this union dissolved and the South Australian Lutheran Church splintered into two main factions, contributing to the near abandonment of the mission to the Diyari for a period of four years from 1872. In {1876}, the Evangelical Lutheran Immanuel Synod (henceforth ELIS) assumed control of the mission and sourced missionaries from the Bavarian Neuendettelsauer Missionsgesellschaft (Neuendettelsau Mission Society, henceforth Neuendettelsau) from 1878.

Note that the Bethesda mission was officially named “Hermannsburg” before being administered by ELIS. During this phase, however, it was commonly referred to by its location, which shifted several times towards well-watered country. For the sake of clarity, the mission is consistently referred to here as ``Bethesda''.

After ELIS assumed control of Bethesda, the ELSA faction established the second inland mission, called Hermannsburg, further inland in Central Australia, and continued to recruit missionaries from the HMS. A continuing relationship between the Australian synod and the HMS eventually became untenable \citep[148]{harms_traume_2003} and the HMS ceased to support missions in South Australia. After the departure of the last HMS missionaries from Hermannsburg in 1891, the mission was left in the hands of a few unordained staff, as Bethesda had been in 1872. In 1894 the ELIS acquired the Hermannsburg mission station, after which both inland South Australian missions were run as collaborations between Neuendettelsau and ELIS, Bethesda from 1876 and Hermannsburg from 1894.

Unlike Bethesda at Lake Eyre, which had earlier been called ``Hermannsburg'', the Hermannsburg mission on the Finke River to the Arrernte retained its naming after its founding seminary, even after missionaries were sourced from Neuendettelsau.

Relations between the missionaries, the mission staff and their families working at the two inland South Australian Lutheran missions were always particularly close. Bethesda was a resting place for Hermannsburg missionaries making the arduous journey to and from the mission and the Barossa Valley, facilitating a collegial exchange of ideas between these two mission stations. The movement of Aboriginal people between Hermannsburg and Bethesda also supported the missionaries' aims. The first baptisms at Hermannsburg, made eight years after the mission’s establishment, occurred only after the Arrernte people Tekua (Thomas) and Kalimala (Andrea) met with Diyari Christians at Bethesda when accompanying the missionaries on the journey southwards (\citealt[398]{harris_one_1994}; \citealt[41]{Latz2014}).

Neuendettelsau missionaries were also sent to Aboriginal missions in Queensland: Cape Bedford (1886--1942; \sectref{bkm:Ref514608023}), Bloomfield (1887--1901), and Mari Yamba (1887--1902). There was considerable transfer of Neuendettelsau missionaries between the two South Australian missions and the Queensland mission.

The Neuendettelsau-trained missionary C. Strehlow (\sectref{bkm:Ref456272403}), who arrived at the Bethesda mission in 1892, was well acquainted with the structure of Diyari before he was transferred to Hermannsburg in 1894 and began to learn Arrernte. His ability to communicate in two Aboriginal languages is the most well recognised instance of cross-mission linguistic fertilisation (see \citealt{hebart_united_1938}: 193; \citealt{albrecht_mission_2002}: 7; \citealt{kenny_missionary_2008}: 32). The extent to which the movement of staff facilitated a transfer of linguistic analysis across language boundaries deserves consideration more generally. For instance, did the HMS missionaries' understanding of PN morphosyntactic structure acquired through learning Diyari in any way influence the earliest description of Arrernte made by HMS graduate Kempe? Further, did the Lutherans' analysis of Diyari made at Bethesda have any influence on the earliest descriptions of Guugu-Yimidhirr spoken at Cape Bedford mission in Queensland?

\begin{figure}
 \includegraphics[width=.45\textwidth]{figures/stockigt-img092-compressed.jpg}
\caption{Sketch map of South and Central Australia illustrating the overland route taken by the pioneer Lutheran missionaries from Bethany to Hermannsburg, 1875--1877 \citep{Scherer1971}}
\label{fig:key:sketchmap}
\end{figure}

\section{{Missionary grammars of Diyari}}
\label{sec:key:8.2}\label{bkm:Ref333934780}\label{bkm:Ref456095221}

Four grammars of Diyari are known to have been written at the Bethesda mission. Two were written during the HMS mission era, one by lay missionary and teacher W. \citet{koch_untitled_1868} and one by missionary \citet{schoknecht_grammar_1947}. Two grammars were written by Neuendettelsau missionaries (\citealt{flierl_christianieli_1880}; \citealt{reuther_ms_1899}[1894]).

All of these four Diyari grammars are similar enough to be treated as a single analysis, with points of difference noted. Examine, for example, the Diyari grammarians' paradigms declining possessive pronouns (\figref{fig:key:146}; \figref{fig:key:147}).


\begin{figure}[p]
\includegraphics[width=.8\textwidth]{figures/stockigt-img093r.png}
\caption{Reuther’s declension of dual possessive pronouns \citeyearpar[21]{reuther_dieri_1894}}
\label{fig:key:146}
\end{figure}

\begin{figure}[p]
\includegraphics[width=.8\textwidth]{figures/stockigt-img094r.png}
\caption{Koch’s description of possessive pronouns \citeyearpar{koch_untitled_1868}}
\label{fig:key:147}
\end{figure}

Such paradigms, which account for the additional clausal case marking of a pronoun in possessive case, were introduced into descriptions of South Australian languages by Teichelmann and Schürmann (\citeyear{teichelmann_outlines_1840}; \sectref{sec:key:5.3.3}). \citegen{reuther_ms_1899} grammar of Diyari, which was translated into English by Hercus \& Schwarzschild (\citeyear{Hercus_Schwarzschild1981}), and which is the last, and most well known of the early Karnic sources, provides paradigms declining dual possessive pronouns for each number (\figref{fig:key:146}). \citet[27]{austin_notitle_1981} observes that when translating dual possessive pronouns Reuther “inadvertently associated number with the thing possessed rather than the possessor, i.e., he has translated \textbf{ngaldrani,} \textit{ngaldranha} 1dlINCL. POSS, and \textbf{ngalini}, \textit{ngalirni} 1dlEXCL. POSS, as `my two' instead of as `belonging to us two’\thinspace”. The same problem occurs in all early Diyari grammars. The mistake had first been made in the earliest Diyari description from the mission (\citealt{koch_untitled_1868}; \figref{fig:key:147}), who translated the nominative form of the dual possessed pronoun as \textit{unserer beiden} (`our two’) rather than “belonging to us two’.

Note that Taplin made exactly the same mistake, before correcting himself in his MS grammar of Ngarrindjeri (\citeyear{taplin_vocabulary_1867}; \figref{fig:key:148}). Both Taplin and Koch may have been independently led astray by Teichelmann \& Schürmann’s analysis (\sectref{sec:key:5.3.3}).


\begin{figure}
\includegraphics[width=\textwidth]{figures/stockigt-img095.png}
\caption{Taplin’s description of possessive pronouns in Ngarrindjeri \citeyearpar{taplin_vocabulary_1867}}
\label{bkm:Ref449374077}\label{fig:key:148}
\end{figure}

Structurally, the Diyari grammars are organised under the same headings and subheadings, employ the same numbering of sections, and often give the same Diyari example clauses. The extent to which the later grammars replicate the structure and schemata of the first grammar \citep{koch_untitled_1868} is especially evident when comparing the pages detailing verb morphology. In each work – with the possible exception of Schoknecht (\citeyear{schoknecht_grammar_1947}; \sectref{bkm:Ref456087018}), the original of which has not been sighted – pages are ruled in the same, unconventional manner and present some content in portrait format. Compare, for example, Koch (\citeyear{koch_untitled_1868}; \figref{fig:key:149}) with Flierl (\citeyear{flierl_dieri_1880}; \figref{fig:key:150}). Note that the authors used different illustrative verbs. Koch used the verb \textit{rnandra-} `to hit, to strike', and Flierl \textit{nganka-} `to make'. Note also the absence of the velar nasal in Flierl’s representation \textbf{anka-.}

\begin{figure}
\includegraphics[width=.45\textwidth]{figures/stockigt-img096r.png}
\includegraphics[width=.45\textwidth]{figures/stockigt-img097r.png}
\caption{Koch’s template of verbal morphology (1868: no pag.)}
\label{bkm:Ref449374106}\label{fig:key:149}
\end{figure}

\begin{figure}
\includegraphics[width=\textwidth]{figures/stockigt-img098r.png}
\caption{Flierl’s template of verbal morphology \citeyearpar[48]{flierl_christianieli_1880}}
\label{bkm:Ref449374123}\label{fig:key:150}
\end{figure}

\subsection{Neglect of Diyari grammars in histories of Australian linguistic research}
\label{sec:key:8.2.1}\label{bkm:Ref456104302}

No missionary grammars of Diyari were published or translated into English during the time of the mission. The only Diyari grammatical materials emanating from the mission that were published \textit{in Australia} before \citet{austin_switch_1981} are E. Homann’s pronominal paradigms in \citet[86]{taplin_1879a_nodate} and in \citet[43--44]{fraser_australian_1892} and three Karnic wordlists presented in \citet{curr_australian_1886} attributed to “F. E. Jacobs” (1886a; 1886b; 1886c), where the author’s initials and surname are given incorrectly. The wordlists were supplied by laymissionary J. E. Jacob, a German-born South Australian wagoner, who helped establish the mission and became integral to its longevity. Jacob’s wordlists are of unnamed languages identified by the location: “North-West of Lake Eyre'', “The North Shore of Lake Eyre'', and Kopperamana. The languages are Wangkatyaka, Ngamini and Diyari respectively \citep[19]{hercus_grammar_1994}.\footnote{The vocabularies supplied by Jacob, who is said to have been a fluent Diyari speaker \citep[11]{proeve_three_1946}, present the most bizarre orthographic convention attested in this corpus. Jacob uses the trigraph “sth” to represent all word-initial nasals other than the bilabial. The spelling was so unconventional as to receive the following explanation in Curr’s introduction (\citeyear{curr_australian_1886} vol.ii: 12): ``I have thought it necessary to call attention to the nationality of my correspondent, as in many cases the spelling of the words of his vocabulary, taken from an English point of view, represents sounds which it would be impossible for an Australian black to utter. The \textit{sth} which occurs so frequently, I take to represent the nasal sound which is generally expressed by \textit{ng}.''}

The Diyari grammars remained obscure unpublished German MSS well into the twentieth century, and some remain so. The only missionary analysis published in the pre-contemporary era is a grammar by \citet{planert_australische_1908}, published in Germany, based on missionary Wettengel’s documents (\citeyear{planert_australische_1908}; \sectref{bkm:Ref340216645}). In 1874 S. Gason (?1845--1897), the mounted constable at Lake Hope (1865--1871), published \textit{The Dieyerie tribe of Australian Aborigines,} which contained vocabulary, example sentences and some poorly spelled verb forms. The material was republished in \citet{Woods1879} and, in part, in \citet{curr_australian_1886} and in \citet{fraser_australian_1892}.

This situation contrasts with the prompt publication of grammars by the earlier Dresden Mission Institute missionaries (\citealt{teichelmann_outlines_1840}; \citealt{meyer_vocabulary_1843}; \citealt{schurmann_letter_1844}) and with the publication of grammars by Taplin (\citeyear{taplin_notes_1872}; \citeyear{taplin_native_1874}; \citeyear{taplin_1879a_nodate}; \citeyear{taplin_grammar_1880}; \sectref{bkm:Ref514608058}) and by Ridley (\citeyear{ridley_kamilaroi_1866}; \citeyear{ridley_kamilaroi_1875}; \sectref{sec:key:4.5}), which appeared close to the time of the Lutherans' first MS descriptions of Diyari. While missionary Günther’s manuscript analyses of Wiradjuri (\citeyear{gunther_native_1838}; \citeyear{gunther_lecture_1840}) were brought to a greater public through publication in \citet{fraser_australian_1892}, the Diyari grammars are not included in Fraser’s volume. The only Diyari material that is included in Fraser are a pronominal paradigm supplied by HMS missionary E. \citet[43--44]{homann_diyeri_1892} and Gason’s material (\citeyear[44--45]{gason_dieyerie_1874}).

The unpublished Lutheran grammars of Diyari did not inform later non-Lu\-ther\-an descriptions of PN languages examined in this study. W. E. Roth appears unaware of, or was at least unable to access, these German MSS when preparing his 1897 grammar of the related Karnic language Pitta-Pitta. The Lutheran Diyari material does not appear in Roth’s substantial bibliography. F. \citegen{muller_grundris_1882} otherwise fairly thorough collation of Australian grammatical material does not note the existence of the Diyari grammars.

The grammars have also been overlooked in histories of Australian language description. They are not referred to by Ray in an otherwise comprehensive description of “existing material available for the study of Aboriginal languages” (\citeyear{ray_aboriginal_1925}: 2). \citet[9]{elkin_nature_1937}  refers only to \citegen{planert_australische_1908}  Diyari grammar, published in Germany. There is no chapter discussing the early description of Diyari in McGregor’s seminal volume on the historiography of Australian languages \citeyearpar{mcgregor_introduction_2008}. The overview of research and documentation of South Australian languages given by Simpson \textit{et al} (\citeyear[120]{simpson_i_2008}) provides no detail of the missionaries' morphosyntactic analysis of Diyari, and incorrectly states that \citet{austin_switch_1981} was the first published grammar of Diyari, overlooking \citet{eylmann_eingeborenen_1908}, \citet{planert_australische_1908}, \citet{gatti_lingua_1930}, and \citet{schoknecht_grammar_1947}. That the earliest two of these Diyari grammars were written in German, the third in Italian, and the last produced as a self-published pamphlet has certainly contributed to this oversight.

The oversight is also due to the fact that the role that the missionaries' analyses may have played in language reclamation programmes, which began in South Australia in the 1980s, has been supplanted by comprehensive analysis informed by speakers in the modern descriptive linguistic era \citep{austin_switch_1981}. The early grammatical descriptions of Diyari do not have the contemporary relevance of earlier Dresdner grammars upon which language reclamation has so entirely depended.

\citet{austin_grammar_1978}, however, gives a good overview of most of the early Diyari linguistic material. This section was excluded from his 1981 published grammar due to space considerations, but now appears in the second edition which is available online as \citet[241--245]{austin_grammar_2013}.

\largerpage
\hspace*{-3.3pt}In comparison with the recent historical attention focussing on Neuendettelsau-trained missionaries (e.g., \citealt{Hercus_Schwarzschild1981}; \citealt{jones_art_1986}; \citealt{nobbs_bush_2005}), the HMS missionaries at Bethesda have been under-investigated and their grammatical contribution, upon which the Neuendettelsau missionaries relied, has been under-recognised. \citet[24]{kneebone_teichelmann_2005} aims to “highlight the sophistication and reliability of the early documentation of the Dieri language by Hermannsburg missionaries”, thus redressing the historical oversight of the earliest HMS grammarians. It appears, however, that Kneebone was unaware of \citegen{austin_grammar_1978} review of earlier research. Despite the studies made by \citet{austin_grammar_1978} and \citet{kneebone_teichelmann_2005}, the initial HMS missionaries' analysis of Diyari, that occurred prior to the arrival of Neuendettelsau men, continues to be overlooked in histories of the mission (see for example, \citealt{kenny_arandas_2013}: 87).

\section{{The Hermannsburg Mission Society missionaries at Bethesda}}
\label{sec:key:8.3}\label{bkm:Ref456005923}

This section surveys the linguistic work made during the first HMI period of missionary activity at Bethesda (\tabref{tab:key:151}).

\begin{table}
\begin{tabularx}{\textwidth}{QQQQQ}
\lsptoprule
 & Occupation at Bethesda & Dates at Bethesda & Produced grammar in:  & Status \\
 \midrule
\textbf{W. Koch} \textbf{(1848--1869)}  & Teacher & 1868--1869 & 1868 & Unpublished German \stockcaps{MS}\\
\tablevspace
\textbf{E. Homann} \textbf{(1838--1915)} & \stockcaps{HMS}-trained missionary & 1867--1871 & \multicolumn{2}{p{4cm}}{Pronominal paradigm in \citealt{Taplin1879a}: 86}\\
\tablevspace
\textbf{C. H. Schoknecht} \textbf{(1841--1905)} & \stockcaps{HMS}-trained missionary & 1872--1873 & 1872 & Unpublished German \stockcaps{MS}, published in English translation 1947\\
\tablevspace
\textbf{H. H. Vogelsang (1832–1913)} & \stockcaps{HMS}-trained mission assistant & 1867--1913 & \multicolumn{2}{p{4cm}}{Copy of Diyari grammar sent to Howitt in 1878}\\
\lspbottomrule
\end{tabularx}
\caption{The grammatical descriptions of Diyari produced during the HMS phase of mission activity at Bethesda}
\label{bkm:Ref339527967}\label{tab:key:151}
\end{table}

The HMS was founded in 1849 by L. Harms (1808--1865). The mission seminary was philosophically similar to Gossner’s of the Berlin Mission Society, established in 1836, which trained some of the missionaries who established the Zion Hill mission at Moreton Bay (present-day Brisbane). \citet{ganter_hermannsburg_2016} writes that these two mission societies:


\begin{quote}
formed break-away institutions from already established mission societies with the express purpose to accept candidates who were being turned away by the other training colleges, and to prepare them for “heathen mission''. They commenced with very basic curricula but eventually succumbed to external pressures and integrated more demanding standards, and the Bible languages, into their programme in order to achieve ordination for their candidates.
\end{quote}

\hspace*{-1.4pt}That the Bible used at the Hermannsburg seminary was written in Plattdeutsch, a “low” German dialect, is telling of the mission philosophy. The early HMS candidates received only a basic academic education without instruction in any classical languages. The training received by the HMS Lutheran missionaries lacked the classical orientation and academic rigor of that received by the earlier Dresdner, the later Neuendettelsau Lutheran missionaries, and those trained at Basel Mission Institute (\sectref{sec:key:4.1}).

\subsection{The earliest phase of the HMS mission at Bethesda}
\label{sec:key:8.3.1}\label{bkm:Ref74134584}

\begin{figure}
\includegraphics[height=.9\textheight]{figures/stockigt-img099-compressed.png}
\caption{“Süd Australien zur Reise der Missionare Walder, Kramer und Meißel. 1866”, produced by Moravian missionaries, with permission from Unitätsarchiv (Bd.30.10)}
\label{bkm:Ref76478694}
\end{figure}

\begin{figure}
\includegraphics[width=\textwidth]{figures/stockigt-img100-compressed.png}
\caption{“Süd Ost Australien 1867”, produced by Moravian missionaries, with permission from Unitätsarchiv (Bd.30.12).}
\label{bkm:Ref76478713}
\end{figure}

\begin{figure}
\includegraphics[width=.8\textwidth]{figures/stockigt-img101-compressed.png}
\caption{“Hermansburg, lutherische Mission am Kilalpanina Süd Australien'', watercolour signed “G. Meißel 1868'', with permission from Unitätsarchiv (Bd.30.17.b)}
\label{bkm:Ref76480787}\label{fig:key:152}
\end{figure}


\begin{figure}
\includegraphics[width=.8\textwidth]{figures/stockigt-img102-compressed.png}
\caption{“Kopperamana No. 1'', watercolour signed “Meißel 28 May 1868'', with permission from Unitätsarchiv (Bd.30.18.a)}
\label{bkm:Ref76480833}\label{fig:key:153}
\end{figure}


The HMS missionaries sent to Australia to establish the first inland mission – J. F. Gößling (1838–1917), E. Homann (1838–1915), and missionary assistant H. H. Vogelsang (1832–1913) – arrived with G. A. Heidenreich (1828–1910) and C. G. Hellmuth (1827–1895). All of these men played important roles in Australian Lutheran mission.\footnote{Hellmuth founded the German-Scandinavian Lutheran synod in Queensland, which supported the Mari Yamba mission, where HMS graduate A. C. Claussen (1846–1897) was the first missionary.}  The HMS administered the Lutheran mission at Lake Eyre during a period of instability and uncertainty. Continuity of staff was poor.

Concurrent with the Lutherans' decision to establish a mission among South Australian Diyari populations, the Society for Promoting Moravian Missions to the Aborigines of Victoria was similarly motivated to establish a mission in the area \citep[209]{edwards_moravian_2007}. Unbeknownst to each other, a Moravian party consisting of G. Meißel, H. Walder and C. C. W. Kramer left Adelaide only ten weeks before the Lutheran party, consisting of missionaries Homann, Gößling, Vogelsang, and the Australian-born wagoner J. E. Jacob, left Langmeil in the Barossa Valley. Both parties arrived at Lake Hope in December (1866; \citealt[215--218]{edwards_moravian_2007}). The location was chosen partly because it was hoped evangelisation would be more successful among remote populations unaffected by perceived European corruption, and partly as an act of goodwill towards the Diyari people, who had sustained John King, the sole survivor of the Bourke and Wills expedition to the Gulf of Carpentaria. The Lutherans maintained a mutually supportive relationship with the Moravian missionaries, who in October (1867) had settled at Lake Kopperamanna, sixteen kilometres from Killalpaninna (see \figref{bkm:Ref76478694} \& \figref{bkm:Ref76478713}). Meißel produced watercolours of both mission stations (\figref{fig:key:152}; \figref{fig:key:153}).

The arrival of both parties, however, contributed to existing tension in the region, resulting from the recent expansion of pastoralism and the ensuing pressure on scarce water supplies. In early 1867, only ten weeks after first establishing themselves at Killalpaninna, the Lutheran party retreated south to the Barossa Valley, and at the same time the Moravians retreated to Boolcaltaninna (\textit{Bucaltaninna}). While the Lutherans re-established their settlement at Killalpaninna in 1868, the Moravian mission was closed the following year, due to lack of support from its directing board in Saxony and from the Moravian committee in Melbourne, although Moravian missionary Walder remained until May 1869. The Moravians had made some progress in learning Diyari, establishing a school in which missionary Kramer taught (\citealt{proeve_work_1952}: 67). Missionary Meißel later contributed a Diyari wordlist from Lake Kopperamanna to G. Taplin’s “Comparative Table of Languages of the Australian Aborigines” (\citeyear{taplin_notes_1872}[1870]).

There are different accounts of the Lutheran missionaries' progress in learning Diyari during this earliest phase of mission activity. Theodor Harms, the then HMS director, noted the existence of a wordlist collated before the missionaries” 1867 retreat:

\begin{quote}
    The natives assist them in their work, and see how the missionaries snatch one word after the other from their lips. So far they have succeeded in collecting a few hundred words. (Harms 1867, quoted in \citealt{proeve_work_1952}: 68)
\end{quote}


In addition to this evidence that the missionaries elicited their own material before leaving the mission early in 1867 is a letter from Gößling (quoted in \citealt{brauer_history_1956}: 229) that was written during the first short ten-week stint at Killalpaninna March 1867, in which he admits difficulty in learning the language, but states that they had “learned several hundred native words from Mr. Gosse [\textit{sic}: Gason] at Lake Hope”.\footnote{This is probably a transcription error. W. Gosse (1842--1881) led an expedition from Alice Springs to Perth in 1873 on which he discovered and named Ayres Rock. He is not known to have travelled to the Lake Eyre Basin.} Kneebone’s investigation also shows that Gason had dictated “a few hundred words'', which provided Homann and Gößling with the ``first working vocabulary in Dieri [\textit{sic}]” (\citeyear{kneebone_teichelmann_2005}: 80).

Note that Homann and Koch had initially referred to the people at the mission as \textit{Körni} and \textit{Karna} (\citeyear{kneebone_teichelmann_2005}: 259), meaning “Aboriginal person” in Diyari – the origin of the modern name for the language family “Karnic” – rather than Dieri, Diari or Diaeri. Compare this with the naming of the Adelaide language and people as “Kaurna'', which also means “men'', but in a neighbouring language (\sectref{sec:key:5.1.1}).



\subsection{Koch: the second phase of HMS at Bethesda}
\label{sec:key:8.3.2}

In January 1868, after being assured of adequate police protection, the Lutherans returned and established a settlement at Lake Killalpaninna. The party returned without HMS missionary Gößling, but with women who had married the lay missionaries – D. Vogelsang (née Heistermann) and M. E. Jacob (née Auricht)\footnote{After the death of his first wife, H. Vogelsang married Jabob’s wife’s sister, A. M. Vogelsang (née Auricht). These women were the daughters of the influential Pastor J. C. Auricht (1832--1907), who established Auricht’s Printing Press in Tanunda, which printed German newspapers. Another daughter, Luise Auricht, married J. Flierl and also lived at Kilallpaninna before travelling with her husband to New Guinea.} and a young German teacher named Wilhelm Koch (1848--1869).

Teacher W. Koch initially met Gößling, Homann and Vogelsang on the sea voyage to Australia. Fleeing financial and moral dishonour \citep[89--90]{kneebone_teichelmann_2005}, Koch decided to join the missionaries and to travel with them to Lake Eyre, but his initial application to Hermannsburg was rejected. Koch later became reacquainted with the missionaries while working as a teacher in the Barossa Valley during the missionaries' 1867 retreat south. Koch reapplied to join the mission, and was this time accepted.

Koch -- who had studied, although had not completed, four years of \textit{Gymnasium} education, including Latin and Greek -- is described as “a gifted young man” \citep[187]{hebart_united_1938} who “brought to the mission a classical education and considerable natural talent for language” \citep[10]{kneebone_teichelmann_2005}. At the time of their acquaintance, Koch is said to have “helped [the missionaries] to explore the Dieri language about which the missionaries knew nothing except for a wordlist of 300 Dieri words” \citep{ganter_hermannsburg_2016}.

On arrival at Bethesda, Koch was given responsibility for teaching in the mission school and used Diyari as the language of instruction \citep[51]{harms_traume_2003}. Substantial linguistic progress was made in 1868. Homann was assisted in improving his Diyari sermons by Pikally, the first Diyari person to be baptised. In 1869 Homann was preaching to more than fifty people, and he perceived that the Diyari people were beginning to understand his message \citep[52]{harms_traume_2003}.

\largerpage
The first Diyari primer (\citealt{koch_nujanujarajinkiniexa_1870}; \figref{fig:key:154}) was produced for publication during the initial HMS phase of mission work at Bethesda (1866--1873). Assigning authorship to the translations made in the primer is not straightforward. Two of the seven Diyari hymns are shown as having been translated by \citet[26]{koch_nujanujarajinkiniexa_1870}. Although the translator of the remaining five hymns, the commandments, psalms and catechism is not given, the \textit{Report} \textit{of the Lutheran Mission} (\citeyear{koch_untitled_1868}) stated that Koch “translated the Creed, the Ten Commandments, and the Lord’s Prayer” (ibid.: 3). Yet the work has been attributed to Homann \& Gößling \citep[100]{graetz_open_1988}, and elsewhere to Homann alone \citep[59]{harms_traume_2003}. The 1870 primer is here attributed to Homann \& Koch.


\begin{figure}
% \includegraphics[width=\textwidth]{figures/stockigt-img103.emf}
% \includegraphics[width=.8\textwidth]{figures/img103-compressed.png}
\includegraphics[width=.8\textwidth]{figures/stockigt-img103r.png}
\caption{The first Diyari primer \citep{koch_nujanujarajinkiniexa_1870}}
\label{bkm:Ref449374245}\label{fig:key:154}
\end{figure}

This first productive period of missionary activity ended abruptly. The engagement of a large number of Diyari people with missionary activity had coincided with a drought. Lack of fresh food and the increasing salinity of the water at the mission caused illness \citep[56]{harms_traume_2003}. W. Koch died from typhoid in April 1869 at the age of 21. After summer rain fell at the end of 1869, school attendance decreased as Aboriginal families returned to the filled watercourses and lakes away from the mission \citep[57]{harms_traume_2003}.

Koch’s death doubled Homann’s workload, who was now too busy to pursue further linguistic work \citep[56]{harms_traume_2003}. Homann left disheartened and embittered in 1871, after having been accused by the mission committee of being “of little faith'' \citep{ganter_neuendettelsau_2016}.\footnote{H. A. E. Meyer had similarly been forced to defend the decision to close the mission at Encounter Bay against the accusation that the Dresdner missionaries had “lacked the patience and the persistence, and above all the watchword `Love'\thinspace” (J. M. Torbitzki, 03/04/1862, quoted in \citealt[50]{zweck_for_2012}).}

In the months before the arrival of a replacement for Homann, the mission was sustained by the Vogelsang and Jacob families. The Vogelsangs maintained a longstanding commitment to the Diyari community. H. H. Vogelsang remained at the mission until his death in 1913. His son H. Vogelsang, one of eight Vogelsang children born at the mission, was literate in Diyari and was the last teacher in the school before the closure of the mission in 1915. Another son, E. T. Vogelsang, who was also a capable literate user of Diyari (see \figref{fig:key:155}), worked at the South Australian Museum in the 1920s as an attendant, where he attempted to make a translation into English of Reuther’s extensive MS (\sectref{bkm:Ref456262601}). He later worked with H. K. Fry on his publication of Diyari legends (\citeyear{fry_dieri_1937}; \citeyear{fry_dieri_1937-1}) and co-authored anthropological articles about the Diyari with Ronald Berndt (\citealt{Berndt1939}; \citealt{berndt_initiation_1941}). A grandson, Colin Jericho, continues to uphold the connection, maintining contact with Diyari descendents of people born at the mission, and placing commemorative plaques on the graves of Diyari people.


\begin{figure}
\includegraphics[width=.8\textwidth]{figures/stockigt-img104.jpg}
\caption{E. T. Vogelsang’s annotated copy of \citet{flierl_christianieli_1880}}
\label{bkm:Ref459735372}\label{fig:key:155}
\end{figure}

\subsection{Schoknecht: the final phase of HMS missionary work at Bethesda}
\label{sec:key:8.3.3}\label{bkm:Ref456087018}

In January 1872, Homann was replaced by C. H. Schoknecht (1841--1905), the last HMS-trained missionary to work at Bethesda. Mission activity moved between Mundowdna, Cooranina, to Tankimarina and then Bulcaltaninna due to lack of water, and the future of the mission was less than certain \citep[63--67]{Harms2003}. After less than two years' service, Schoknecht followed Gößling and Homann to the more hospitable southern regions and did not return. Thus concluded the involvement of HMS-trained missionaries among the Diyari, and other groups of people at Lake Eyre. Again, the station was managed by the tenacious Vogelsang and Jacob families, this time for a period of four years, until they were partially relieved by the arrival of lay missionary C. A. Meyer, an “elder of the Langmeil congregation” in South Australia \citep[189]{hebart_united_1938}, and in 1878 by the first Neuendettelsau missionary, J. Flierl.

\subsection{Provenance of the early analysis of Diyari}
\label{sec:key:8.3.4}\label{bkm:Ref326948006}

All later missionary grammars of Diyari (\citealt{Schoknecht1872}; \citealt{flierl_dieri_1880}; \citealt{reuther_dieri_1894}; \citeyear{reuther_ms_1899}) replicate almost the entirety of the \textit{existing parts} of Koch’s MS grammar (\citeyear{koch_untitled_1868}). All were written in German and contain little additional analysis. A few passages are, however, given in this earliest Diyari grammar that are not reproduced in later works.

 Schoknecht’s MS grammar of Diyari, written in German in 1872, was published in English translation by Schoknecht’s descendants (\citeyear{schoknecht_grammar_1947}) and is reproduced in \citet{kneebone_language_2005}. The original German MS remains with the family and has not been sighted for this study.

The existence of a grammar predating Schoknecht’s analysis (\citeyear{Schoknecht1872}) written by W. Koch was first established in 2005 \citep[13--14]{kneebone_teichelmann_2005}. In correspondence with HMS mission director Harms, Koch refers to his grammar of Diyari as \textit{Grundzüge der Grammatik} [Basics of a Grammar], a copy of which is known to have been sent by the missionaries in Australia to Harms in Germany \citep[297]{harms_traume_2003}. Kneebone refers to it as the “lost handwritten manuscript by Wilhelm Koch” (\citeyear{kneebone_teichelmann_2005}: 14).

However, another copy of the earliest grammatical description of Diyari produced before Homann’s departure (1871), remained in Australia. After the final withdrawal of the last HMS missionaries, lay missionary C. A. Meyer prepared to travel to Bethesda in order to relieve Vogelsang and Jacob. In the days before his departure from the Barossa Valley, it appears that Meyer wrote to Homann requesting that Homann supply him with the missionaries' analysis of Diyari. Homann’s response communication shows that a grammar of Diyari predating Schoknecht’s analysis remained in Australia:

\begin{quote}
I am quite happy to allow my work regarding the Dieri language to be copied, but I would not like to let it out of my hands, the works are too dear to me for that. In case you want to copy them here, I am happy to be at your disposal to impart important information about them regarding learning the language and if necessary to make them more complete. \citep{homann_letter_1875}
\end{quote}

In 2013, when researching this current study, a copy of Koch’s original MS grammar was located in a box of papers acquired by the University of Adelaide’s Barr Smith Library Rare Books and Special Collections (henceforth BSL) from Professor J. A. FitzHerbert, who was the Hughes Professor of Classics at the University of Adelaide between 1928 and 1957. Comparison of handwriting samples shows that the BSL MS grammar is written in the uneducated hand of H. H. Vogelsang, who is said to have had “a good, basic primary school education as judged by the quality of his written German'' (J. \citealt{strehlow_tale_2011}: 265; \citealt{Stockigt2017} §8.4).

The “Diari” notebooks are likely to have come into the possession of Professor FitzHerbert during the 1920s, when H. H. Vogelsang’s son E. T. Vogelsang was employed at the South Australian Museum as an attendant and was attempting to translate the Reuther MS (1899--1908) into English for publication. FitzHerbert, at the University of Adelaide, situated adjacent to the South Australian Museum, was at the same time investigating Australian languages, having himself translated Vol. V of Reuther’s MS.\footnote{FitzHerbert’s translation is held by the BSL and is presumably the grammar and vocabulary which Austin (2013: 247[1981a]) refers to as being “compiled by the late Professor FitzHerbert apparently based upon \citet{reuther_ms_1899}. I have searched for this material in Adelaide but was unable to find it.”}

FitzHerbert supervised the first grammars of Australian languages written for academic award: J. R. B. Love’s (\citeyear{love_outline_1938}[1933]) grammar of Worora, and the grammar of Western Arrernte written by T. G. H. Strehlow (\citeyear{strehlow_aranda_1944}[1938]) in which \citet[1]{elkin_introduction_1944} described FitzHerbert’s role as that of “the helpful godfather”. In 1930 FitzHerbert formed “a small language committee at the University of Adelaide” \citep[261]{tindale_legend_1935} with South Australian Museum ethnologist N. B. Tindale (1900–1993) and South Australian pastoralist and geologist C. Chewings.

While the BSL MS was written by Vogelsang, its content is not Vogelsang’s original analysis. Passages of the BSL grammar which are \textit{not} contained in later corpus Diyari grammars confirm that the author was highly educated. Examine, for example, the following remark:

\begin{quote}
It is easily explainable why we have not found any other tenses apart from the main forms. In a language which moves only within the sensual and the coarsely material, as does the Diari language, it follows that there is no mention of any historical tenses. \citep[no pag.]{koch_untitled_1868}\footnote{Es ist leicht erklärlich, warum wir keine weiteren Tempora, als die Hauptformen gefunden haben. In einer Sprache, die sich nur im sinnlichen u. grob materiellen bewegt, wie die Diari Sprache, ist deshalb auch nicht von historischen Tempora die Rede.}
\end{quote}


Observe also the use of Latin terminology, and the etymological remark about the term \textit{Zeitwort}, literally “time word'', which appear in the following introductory section to “the verb’', but which are absent from other missionary grammars of Diyari:

\begin{quote}
About the Zeitwort (verb).



The verb expresses an activity in both or one voice and, at the same time, defines the time it is happening, which is also where it has its name from.



For each verb, one has to note:



1. the gender  or  (genus)



2. the mood  “  (modus)



3. the tense  “  (tempus)



4. the number  “  (numerus)



5. the person  “  (persona)



6. the conjugation  “  (conjugation)

\citep{koch_untitled_1868}\footnote{Vom Zeitwort (Verbum) Das Zeitwort drückt eine Thätigkeit, in beiden oder einem Zustand aus, u. bestimmt zugleich die Zeit wann es geschieht, woher es auch seinen Namen hat. Bei jedem Verbum hat man zu merken: 1. die Gattung oder (Genus) 2. die Art [oder] (Modus) 3. die Zeit [oder] (Tempus) 4. die Zahl [oder] (Numerus)  5. die Person [oder] (Persona) 6. die Abänderung [oder] (Conjugation)}
\end{quote}

The BSL Diyari grammar is established as Vogelsang’s copy of Koch’s \textit{Grund\-zü\-ge der Grammatik} [Basics of a Grammar] \citeyearpar{koch_untitled_1868} on internal orthographic evidence. The Hermannsburg missionaries' atypical orthographic representation of the lamino-palatal stop using the Greek letter “x” is diagnostic. This usage follows Lepsius’s universal alphabet (\citeyear{lepsius_allgemeine_1855}; \citeyear{lepsius_standard_1863}), in which the symbol was used to represent the uvular and velar fricatives, or, with an acute accent, palatal fricatives (\citealt[323--324]{whitney_lepsiuss_1861}; \citealt{Stockigt2017} §8.4.1). The lamino-palatal stop was standardly represented in early orthographies using the digraphs “ty'', “tj'', or “ch''. See for example the use of the letter “x” in the title of the first Diyari primer (\citealt{koch_nujanujarajinkiniexa_1870}; \figref{fig:key:154}) in the representation of the suffix -\textit{yitye} “habitual association” (\citealt{austin_grammar_2013}: 45[1981a]). This practice was done away with by the Neuendettelsau-trained missionary Flierl (\sectref{bkm:Ref514690075}).

Koch’s authorship is confirmed through comparative examination of the analysis given in the early Diyari sources. The differences between the sources are important when establishing that the BSL manuscript grammar of Diyari is a copy of Koch’s \textit{Grundzüge der Grammatik} [Basics of a Grammar] \citeyearpar{koch_untitled_1868}. Although no MS vocabulary survives from the earliest HMS phase at Bethesda (1866--1868), it is probable that the vocabulary appended to Schoknecht’s grammar (\citeyear{schoknecht_grammar_1947}), which gives over 1,000 Diyari-German entries, followed by German-Diyari, was largely compiled by Homann, Gößling and Koch, and perhaps incorporated the lexical material collected by Gason (\sectref{bkm:Ref74134584}) and by Moravian missionary Meißel (in \citealt{taplin_notes_1872}[1870]).

\subsubsection{Concluding remark}
\label{sec:key:8.3.4.1}

Significant grammatical description of Diyari was produced during HMS's involvement at Bethesda (1868--1872). The HMS missionaries laid the groundwork for the later and more enduring Neuendettelsau phase of the mission at Bethesda. They secured the co-operation of members of the local community and forged relationships which continued to benefit the Neuendettelsau missionaries in acquiring the language.

\section{Neuendettelsau Missionaries}
\label{sec:key:8.4}

The second group of Lutheran missionaries at the Bethesda mission were trained at the Neuendettelsau Mission Society. The society had been established in 1841 in Bavaria by Rev. W. Löhe in response to the need of German emigrants in America for German clergy. Before forming an alliance with the South Australian ELIS in 1875, the society did not train candidates for mission work, termed \textit{äussere} \textit{mission} (outer-mission), but concentrated on producing pastors for \textit{innere} \textit{mission} (inner-mission), who would serve existing Lutheran and German-Protestant congregations. The preparation of graduates for outer mission in Australia, and in the German colonies in New Guinea and East Africa, commenced in the last quarter of the nineteenth century. Unlike the training received at the HMS, but like that at the Dresden Mission Institute, Neuendettelsau prepared men for ordination as Lutheran pastors as well as missionaries. The young candidates usually came from grammar schools and received either three or four years of training. Regarding the training missionaries received at Neuendettelsau, Ganter writes:

\begin{quote}
From its humble beginnings, the curriculum was steadily built up. In {1859} the teaching timetable started at 8am and ended at 7pm, including half a day on Saturdays, with five hours of formal instruction, interspersed with two hours of independent study. About half of the teaching was dedicated to instruction in English, Hebrew, Latin and Greek. Next to theological training, candidates were also instructed in piano, violin and singing, and writing and oratory. It was hardly possible to extend the daily routine, so more years were added to the training. The 1861 graduates had spent three years at the college. From 1892 this increased to four years, and by 1913 the training had grown to six years. \citep{ganter_neuendettelsau_2016}
\end{quote}


Missionary training became increasingly rigorous under the directorship of Johannes Deinzer (1842--1897), who trained all of the Neuendettelsau missionaries who wrote grammars of PN languages, except for J. Riedel. These were:
\citet{flierl_christianieli_1880,reuther_dieri_1894,reuther_ms_1899,reuther_ms_1901a,reuther_ms_1901b}, C. Strehlow (c. \citeyear{strehlow_aranda-und_1907}; \citeyear{strehlow_einige_1908}; \citeyear{strehlow_notitle_1910}), and \citet{schwarz_koko_1900}, as well as Siebert (\sectref{bkm:Ref456091577}) and Wettengel (\sectref{bkm:Ref326948465}; \tabref{tab:key:157}).

\begin{table}
\small
\begin{tabularx}{\textwidth}{p{2.1cm}QQp{3.1cm}}
\lsptoprule
Missionary & Mission & Grammar & Notes \\
\midrule
J. Flierl\newline (1858--1947): & Bethesda 1878--1885 Cape {Bedford 1886} & Diyari 1880  & Based on \citet{koch_untitled_1868}\\&  & \mbox{Wangkangurru 1880} & \\
\tablevspace
G. J. Reuther\newline (1861--1914) & {Bethesda 1888}-1906 & Diyari 1899 & Based on \citet{koch_untitled_1868}\\
&  & \mbox{Wangkangurru 1901} & \\
%\hhline%%replace by cmidrule{~~--}
 &  & \mbox{Yandrruwandha 1901} & \\
\tablevspace
C. Strehlow\newline (1871--1922) & {Bethesda 1892}-1894; {Hermannsburg  1894}-1922 & Arrernte c.1907, 1908, 1910 & \\
&  & Luritja 1910 & \\
\tablevspace
O. Siebert\newline 1871--1957 & {Bethesda 1896}-1902 &  & Possible author of the Wangkangurru and Yandrruwandha  material in \citet{reuther_three_1981}\\
\tablevspace
N. Wettengel\newline (1869--1923) & {Bethesda 1896}-1902; Hermannsburg 1902-1906 &  & Informed Planert (1907; 1908)\\
\tablevspace
G. H. Schwarz\newline (1868--1959) & Elim/Cape Bedford 1887--1942 &  Guugu-Yimidhirr 1900 & Informed \citet{roth_structure_1901} although the template of the 1900 description is taken from \citet{roth_ethnological_1897}\\
\tablevspace
\mbox{W. G. F. Poland} (1866--1955) & Elim/Cape Bedford 1887--1909 &  & \\
%\hhline%%replace by cmidrule{--~~}
\lspbottomrule
\end{tabularx}
\caption{Neuendettelsau missionary-grammarians trained by Deinzer}
\label{bkm:Ref449375740}\label{tab:key:157}
\end{table}

Equipped with this training, the PN grammars written by the Neuendettelsau missionaries might be expected to be of a higher quality and more analytically insightful than grammars written by missionaries trained at the HMS seminary, who had received a less demanding training.

\subsection{J. Flierl}
\label{sec:key:8.4.1}\label{bkm:Ref514690075}

J. Flierl, the first Neuendettelsau-trained missionary in Australia, arrived at Bethesda in 1878 at the age of twenty. Flierl had completed three years' training under Deinzer, who described him as “one of my most gifted pupils” (quoted in J. \citealt{strehlow_tale_2011}: 250).

Flierl engaged deeply in the linguistic aspects of mission work, making revisions and additions to the first Diyari primer (\citealt{koch_nujanujarajinkiniexa_1870}; \figref{fig:key:154}), which was published in 1880 as: \textit{Christianeli ngujangujara-pepa Dieri jaurani \& Papaia buru kulnolu}. On the basis of this work, Austin (\citeyear{austin_grammar_2013}: 242[1978]) states: “Flierl had a good command of the language”. Flierl produced a Diyari reader, Bible history and hymnal which was published in 1883 titled \textit{First Reading-book in the Dieri language, Wonini-Pepa Dieri-Jaurani Worapala.} Both of these booklets continued to be used in the mission school over the coming decades and were not updated until W. \citet{riedel_ngujangujara_1914}. Flierl also improved and updated the orthography used by missionaries to represent Diyari.

\subsubsection{Flierl’s comparative grammar of Diyari and Wangkangurru \citeyearpar{flierl_christianieli_1880}}
\label{sec:key:8.4.1.1}\label{bkm:Ref335827551}

Flierl’s major grammatical contribution to further analysis of Karnic structure was his brief grammar of Wangkangurru. Maintaining contact with Aboriginal people at Lake Eyre was a continuing challenge during this stage of mission work, and Flierl travelled to the northeast of the mission in order to contact surrounding Aboriginal populations. Lay missionary Jacob, who contributed vocabularies of other Karnic languages to Curr \citep{Jacobs1886a,Jacobs1886b,Jacobs1886c}, also contacted speakers of surrounding languages who did not reside at the mission. Earlier mission references to this neighbouring Karnic language had been made by Schoknecht \citep[19]{hercus_grammar_1994} and by Homann (\citealt{homann1868}), who wrote that he taught children from five tribes, “Diaeri, Wonkanurro, Terrari, Aumeni und Wonkarappanna [Diyari, Wangkangurru, Thirrari, Ngamini, Arabana]” (quoted in \citealt{harms_traume_2003}: 259; see \figref{fig:key:158}).


\begin{figure}
\includegraphics[width=\textwidth]{figures/stockigt-img105.jpg}
\caption{Languages of the Lake Eyre Basin \citep{Anggarrgon2008}.}
\legendbox{Thirrari was spoken to the west of Diyari on the shore of Lake Eyre (\citealt{Austin2013}: 14[1981a])}
\label{bkm:Ref337128069}\label{fig:key:158}
\end{figure}


Flierl’s Wangkangurru grammatical material is given in a comparative grammar of Diyari Wangkangurru \citeyearpar{flierl_dieri_1880}. The Diyari material appears on the left-hand side of a double page headed “D” and the equivalent, but less detailed, Wangkangurru material on the right, headed ``W'' (\figref{fig:key:159}). Flierl also compiled a comparative Diyari Wangkangurru vocabulary, which is set out in exactly the same fashion (no date a). This work is held in the Basedow papers in the Mitchell library, while its sister document, the grammar, is at the Lutheran Archives in Adelaide. Flierl also compiled a comparative vocabulary which tabulates pronouns in four languages: Diyari, Ngamini, Wangkangurru and Arabana \citeyearpar{flierl_comparative_nodate}. This work was sent by Flierl to Deinzer and is held in Neuendettelsau (\figref{fig:key:160}).

Flierl left Bethesda in 1885 in order to establish a mission in German New Guinea (1884--1919). En route, while delayed in Cooktown, Queensland, he founded the Lutheran Elim mission (Cape Bedford, Hopevale) on an Aboriginal reserve north of Cooktown at Cape Bedford (\sectref{bkm:Ref514608157}).


\begin{figure}
\includegraphics[width=\textwidth]{figures/stockigt-img106.png}
\caption{Flierl’s comparative Diyari and Wangkangurru grammar \citeyearpar{flierl_dieri_1880}}
\label{bkm:Ref449375790}\label{fig:key:159}
\end{figure}

%%%Page showing verb paradigm%%%


\begin{figure}
\includegraphics[width=\textwidth]{figures/stockigt-img107r.jpg}
\caption{Flierl's comparative vocabulary of four languages spoken close to the mission \citeyearpar{flierl_christianieli_1880}}
\label{bkm:Ref449375817}\label{fig:key:160}
\end{figure}

%%%Page showing nominal and pronominal paradigms.%%%

\subsection{J. G. Reuther}
\label{sec:key:8.4.2}

Flierl was replaced by another Neuendettelsau graduate, J. G. Reuther (1861--1914), who had received only two years' missionary training. Reuther, who was later described as being “lame at languages” (\sectref{bkm:Ref456091577}; see \citealt{kenny_arandas_2013}: 88), was regarded by his teacher, Deinzer as linguistically \textit{untrained}. Having struggled with classical languages at Neuendettelsau, Reuther dropped the study of Latin, but persevered with Classical Greek, since the Lutheran tradition uses Classical Greek – more correctly Koine Greek – as the source language for translation of the Holy Scriptures (J. \citealt{strehlow_tale_2011}: 332).

Reuther arrived at Bethesda in 1888 and remained until 1906, during which time the Christian instruction and vernacular literacy booklets prepared by Flierl (\citeyear{flierl_dieri_1880}; \citeyear{flierl_wonini-pepa_1883}) continued to be used in the mission school. No new linguistic materials were produced by Reuther prior to the arrival of fellow Neuendettelsau graduate C. Strehlow in 1892. Strehlow worked with Reuther at Bethesda for two years before being transferred to Hermannsburg, where he made grammatical description of Arrernte and Luritja (\sectref{sec:key:9.2}).

Within a year of C. Strehlow’s arrival, Reuther and Strehlow began the enormous task of translating the entire New Testament into Diyari (J. \citealt{strehlow_tale_2011}: 326). \textit{Testamenta Marra} was published in 1897 at Auricht’s Printing Office in Tanunda. It is the first whole translation of the New Testament into any Australian language.

\subsubsection{Reuther’s grammars}
\label{sec:key:8.4.2.1}\label{bkm:Ref456262601}

There is a gap of fourteen years between Flierl’s comparative grammar (\citeyear{flierl_dieri_1880}) and Reuther’s earliest known Diyari grammar (\citeyear{reuther_dieri_1894}). This MS grammar, which is held at the Lutheran Archives in Adelaide, is almost identical to a later grammar completed by Reuther in 1899. The later Reuther work is held within Volume 5 of Reuther’s substantial unpublished German manuscript of 13 volumes, which records Diyari culture, language and belief, and is held at the South Australian Museum. Reuther’s 1899 grammar is the most well-known missionary analysis of Diyari, having been translated into English by the eminent Australian linguist Luise Hercus and her mother, Theodora Schwarzschild, in 1981. Volume 5 also contains separate and less extensive grammatical descriptions of two other Karnic languages, Wangkangurru and a dialect of Yandrruwandha \citep[73--74]{austin_notitle_1981}, which were also translated into English in 1981. The Yandrruwandha grammar is the least detailed of the three works, and the grammar of Diyari the most detailed. \citet{austin_notitle_1981} annotates the translation of the original three analyses.

In addition to the four languages given in an earlier comparative vocabulary by Flierl \citeyearpar{flierl_comparative_nodate} – Diyari, Ngamini, Wangkangurru and Arabana – \citet{reuther_ms_1899} supplied data for Thirrari, Yawarrawarrka, Yandrruwandha, and the northern Thura-Yura language Kuyani (\figref{fig:key:158}).

Reuther’s grammatical record of both Diyari and Wangkangurru better accord with modern analyses of the languages. His record of the distinctive marking in Diyari of case on singular and non-singular nouns and pronouns, shown in his nominal case paradigms (\sectref{bkm:Ref326948719}), substantially improves on the record left by \citet{flierl_christianieli_1880}, Schoknecht (\citeyear{schoknecht_grammar_1947}), and \citet{koch_untitled_1868}.

\hspace*{-2.3pt}Parts of Reuther’s grammars are overtly comparative, observing points of grammatical difference between the three languages. For example, Reuther wrote:

\begin{quote}
In Diari there are three declensions: two of common nouns and one of proper nouns but in Wangkangurru there are only two: one of common nouns and one of proper nouns. In Jandruwanta proper nouns are identical in declension with common nouns, kinship terms, however, have their own declensions: Jandruwanta thus has one declension of ordinary nouns and two declensions of kinship terms. (Reuther, Jandruwantha grammar 1901, translated by Hercus \& Schwarzschild \citeyear[56]{Hercus_Schwarzschild1981})
\end{quote}


Reuther (\citeyear[38]{Hercus_Schwarzschild1981}) also observed that in contrast to Diyari, which has a gender distinction on third-person singular pronouns, “Wonkanguru makes no difference of gender whatsoever: masculine, feminine and neuter are all rendered by the same third-person pronouns and demonstratives.”

\subsubsection{Comparison of Reuther’s and Flierl’s grammars of Wangkangurru (\citeyear{reuther_repr_1899}; \citeyear{flierl_dieri_1880})}
\label{sec:key:8.4.2.2}\label{bkm:Ref466972983}

Reuther’s Wangkangurru grammar \citeyearpar{reuther_repr_1899} provides considerably more analytical detail and example clauses than Flierl’s grammar of the same language \citeyearpar{flierl_dieri_1880}.

The \citeyear{reuther_repr_1899} grammar, for example, gives paradigms for inclusive and exclusive non-singular first-person pronouns \citep[36]{a_three_1981}, while Flierl’s earlier work (\citeyear{flierl_christianieli_1880}) had provided only the inclusive terms. Interestingly, Reuther includes a description of the allative case suffix before the tables of nominal declension: “the ending \textbf{-ruku} is used as a post-position: it occurs when one wishes to imply movement towards the object in question” (\citealt{a_three_1981}: 31). While also described here as a ``post-position'', the description of the form and function of the allative case suffix \textit{within the description of nominal declension} does not occur in any other missionary grammars written at Bethesda (\sectref{bkm:Ref326948895}).

The description of ordinal numbers in the 1901 Wangkangurru grammar is of considerable interest, differing from that given in all Diyari grammars. Reuther’s Diyari grammar follows previous Diyari grammars (e.g., \citealt{schoknecht_grammar_1947}: 9 [1872]) in stating that there are no ordinal numbers. Under the heading “Ordinal numbers” the Wangkangurru grammar (\citeyear{a_three_1981}: 35 [1901]), however, lists forms translated as `the first' … `the fourth', `the last but one' and `the last of all'. As \citet[54]{austin_notitle_1981} observes, “these words can scarcely be regarded as ordinal numbers. They are derivatives formed with –\textit{nganha} (Reuther’s \textbf{ngana}), an elative adjectival suffix meaning `originating from', which is common to Wangkangurru”. The elative suffix is described by \citet[108]{hercus_grammar_1994} as rarely marking case on pronouns and “usually refer[ing] to locations and times.'' “Ordinal numbers” in Wangkangurru had not been described by \citet{flierl_dieri_1880}.

As noted by \citet[54]{austin_notitle_1981}, there are problems with Reuther’s listing of ergative and nominative case forms in the Wangkangurru and Yandrruwandha grammars. In Wangkangurru, the first-person plural pronouns are shown as tripartite (A/S/O), whereas Austin records them as marked with an accusative system (AS/O). This mistake was imported from Flierl’s analysis (\figref{fig:key:168}), which appears to have extrapolated the tripartite marking of 1pl pronouns in Diyari into Wangkangurru. Further, in Yandrruwandha there is confusion concerning the nominative and ergative forms of demonstratives (\citeyear{austin_notitle_1981}: 75).

\subsubsection{Otto Siebert}
\label{sec:key:8.4.2.3}\label{bkm:Ref456091577}

Neuendettelsau-trained missionary Otto Siebert arrived at Bethesda in 1896. He was known as ``the bush missionary'' (\citealt{hercus_otto_2004}: 36) and his task was to “work among the aborigines living in camps in the surrounding districts” (\citealt{hebart_united_1938}: 190). A tradition of travelling away from the mission site in order to preach to Aboriginal populations had been established at Bethesda at least since Flierl’s time. The advantages of evangelising away from the mission site had been similarly perceived by missionary Watson at the Wellington Valley mission in New South Wales (\sectref{sec:key:4.1}), although here the proposal did not eventuate \citep[339]{bridges_church_1978}. Equipped with a tent, Siebert frequently visited Diyari and Wangkangurru camps far from the mission, which Reuther as mission-manager was unable to do.

Like Vogelsang, Siebert corresponded with Howitt, and contributed ethnographic Wangkangurru and Diyari material to Howitt’s \textit{The Native Tribes of South-East Australia} \citeyearpar{howitt_native_1904}. As is evident from the title of \citegen{siebert_sagen_1910} publication, \textit{Sagen und Sitten der Dieri und Nachbarstämme in Zentral-Australien} [Legends and customs of the Dieri and neighbouring tribes in Central Australia], his investigations included the study of Lake Eyre Basin cultural practices and belief systems.

Siebert was also an astute linguist. He observed (quoted in \citealt{kneebone_teichelmann_2005}: 372--373) that the style of language spoken at the mission and used in translation, which he termed \textit{Küchen-Dieri} (kitchen Diyari) differed from \textit{gemeinverständlichen Dieri} (commonly understood Diyari) spoken away from the mission. He translated a collection of eighty psalms while at Bethesda \citep[42]{nobbs_bush_2005}. They remain unpublished and are not held in the public domain. Preliminary investigation suggests that are all written in Diyari. When writing to Howitt in 1899 (see \citealt{nobbs_bush_2005}: 32), Siebert confidently commented that the widely acclaimed publication by B. Spencer (1860--1929) and F. J. Gillen (1855--1912) \textit{The Native tribes of Central Australia} (1899) was linguistically ill-informed. It is possible that Siebert contributed to the grammatical analysis of Karnic languages made during the later Neuendettelsau stage of mission activity.


The strongest suggestion that Siebert made his own independent analysis of Diyari language comes from his own recollections of his interaction with Reuther. Late in life, when in Germany, Siebert corresponded with E. T. Vogelsang, son of H. H. Vogelsang, who was attempting to translate the Reuther MS (1899--1908) while working at the South Australian Museum. \citet{siebert_letter_1935} recalls some academic jealously between himself and Reuther. When N. B. Tindale travelled to Germany in the 1930s and interviewed Siebert as part of an ultimately unsuccessful attempt by the South Australian Museum to have the Reuther MS published in English, Siebert described Reuther as “lame at languages” \citep{tindale_report_1937}. Siebert also stated that the grammatical material in \citet{eylmann_eingeborenen_1908} was not Reuther’s but his own analysis. \citet[93]{eylmann_eingeborenen_1908}, however, acknowledged Reuther as supplying the Diyari grammatical material.

\subsection{Planert’s grammar of Diyari \citeyearpar{planert_australische_1908}}
\label{sec:key:8.4.3}\label{bkm:Ref74225865}\label{bkm:Ref514690025}\label{bkm:Ref340578477}\label{bkm:Ref340216645}\label{bkm:Ref456334295}\label{bkm:Ref326948465}
The Berlin-based philologist W. Planert {(\sectref{sec:key:2.6.1}) had his attention} drawn to PN languages by the return to Germany from Central Australia of the Neuendettelsau-trained missionary N. Wettengel (1869–1923). Having been dismissed by the Immanuel Synod Mission Committee in 1906 (\sectref{bkm:Ref326948956}), Wettengel contacted the Berlin museum in order to sell ethnographic artefacts (J. \citealt{strehlow_tale_2011}: 960--962). Like other missionaries of the time in Central Australia, Wettengel attempted to supplement his income by supplying museums in Germany with desirable, and difficult to procure, ethnographic objects.\footnote{The Wettengel collection remains at the Ethnological Museum in Berlin.}  Planert utilised the linguistic knowledge of Diyari and Arrernte that Wettengel had acquired while working at the Bethesda mission (1889–1901) with fellow Neuendettelsau graduates, Reuther, Siebert and Strehlow, and at the Hermannsburg mission (1901–1906) with Strehlow.

It is clear that Planert’s grammar of Diyari was primarily based on a written copy of the missionaries' analyses, rather than Wettengel’s knowledge of the language, or on Wettengel’s memory of examples given in the written sources. In the grammar of Arrernte, \citet[551]{planert_australische_1907} states that Wettengel possessed valuable manuscript vocabularies of Diyari and of Arrernte, which he intended to publish.\footnote{“There are also two valuable manuscripts belonging to Mr Wettengel which should be published, namely an Aranda and a Dieri dictionary.” (Auch sind zwei wertvolle Manuskripte des HrN. Wettengel, nämlich ein Aranda- und Dieri-Wörterbuch, zu publizieren; \citealt[551]{planert_australische_1907})} Planert’s grammars commence with a long initial section headed \textit{Wortbildung} (Word formation), which is atypical of the corpus, and which provides richer exemplification of derivational processes than do the missionary sources. This section is assessed by Austin (\citeyear{austin_grammar_2013}: 245[1978]) as containing “very useful information although there are one or two errors”. It is likely that Planert drew the illustrated derivational morphology from Wettengel’s vocabularies.

Planert must have had access to the missionaries' MS grammars in addition to the vocabularies. In his Diyari grammar, when describing \textit{Postpositionen} (\citeyear{planert_australische_1908}: 692), Planert replicates the missionaries' presentation (\figref{fig:key:8-183}), giving suffixes in identical order, and with translations using the same German prepositional phrases. Planert’s case paradigms (\figref{fig:2:25}; \sectref{sec:key:2.6}) also indicate that the missionaries' grammars were used as a template, although his presentation of the syntactic cases differs vastly from the missionaries’.

The more complex areas of morphosyntax, including processes of clause subordination, exemplified by the missionaries (\sectref{bkm:Ref73520500}) are not, however, included in Planert’s grammar. Perhaps these were left out because Wettengel was unable to provide clarification. When C. Strehlow became aware that the publication of an Arrernte grammar informed by Wettengel was imminent, he \citeyearpar{strehlow_aranda-und_1907} wrote to his German editor Leonhardi stating: “I am not at all afraid of missionary Wettengel’s research, but rather surprised that he dares to write an Aranda Grammar at all, if one keeps in mind his weak grammatical knowledge”. Further suggestion that Wettengel’s linguistic skills were questionable is given by Siebert (quoted in \citealt{leonhardi_letter_1909}) in a letter to Leonhardi: “The Wettengel’ian Dieri grammar is too good for W[ettengel]. He used the material before him and it was a good thing that he did so”.

Unlike most other early PN grammars, both of Planert’s works conclude with a section headed \textit{Texte}. In the Diyari grammar these are Biblical texts, and in the Arrernte grammar these are legends. Planert supplied both interlinear-style and free translation. In this way Planert’s works resemble F. \citegen{muller_grundris_1882} grammars, also written in German, which included final sections headed \textit{Sprachproben}. The inclusion of these sections perhaps reflects a Humboldtian descriptive tradition, which in America came to be known as the “Boasian trilogy” (\citealt[8--9]{darnell_indo-european_1999}; \sectref{sec:key:3.3}) in which investigation of \textit{Volk} and language were branches of integrated investigation.

\subsection{Gatti’s grammar of Diyari \citeyearpar{gatti_lingua_1930}}
\label{sec:key:8.4.4}\label{bkm:Ref73350344}\label{bkm:Ref516424775}\label{bkm:Ref335988141}
In 1930, Giovanni Gatti (dates unknown) published a grammar of Diyari in Italian \textit{La lingua Dieri: contributo alla conoscenza delle lingue Australiane} (The Dieri language: The Australian contribution to the knowledge of languages). As the title suggests, Gatti presented the Diyari material within a broader discussion of Australian linguistic structure and classification, referencing an impressive range of available primary and secondary material.

Gatti’s Diyari material is, however, based on a limited range of sources. Like \citet[2]{ray_aboriginal_1925} and Elkin (\citeyear[9]{elkin_nature_1937}; \sectref{bkm:Ref456104302}), Gatti did not have access to the missionaries' MSS. More unexpectedly, Gatti did not refer to Homann in \citet{fraser_australian_1892} or to \citet{planert_australische_1908}, which were then by far the most extensive available resources. He stated that his work was informed by Gason in \citet{curr_australian_1886}, by Jacob in \citet{curr_australian_1886}, and largely by \citegen{reuther_testamenta_1897} translation of the New Testament into Diyari \citep[47]{gatti_lingua_1930}. The work is in this way similar to \citegen{ray_study_1893} grammar of Western Torres Strait, which was based on the structure of the language used in the translations of the Gospel of St Mark (\sectref{bkm:Ref514608187}). Unsurprisingly, Austin (\citeyear{austin_grammar_2013}: 245[1978]) describes Gatti’s grammar as “suffer[ing] from the deficiencies of the primary sources, especially in phonology”.

It is not known whether the work was produced as a doctoral dissertation, or in what other context it was written. Further investigation might show the extent to which the grammar describes the mission idiom as printed in the New Testament and how this differs from the variety described by Austin (\citeyear{austin_grammar_1981}, \citeyear{austin_grammar_2013}).

A later paper (\citeyear{gatti_britannnia_1934}) prepared by Gatti discussed the use of different creoles and pidgins in the Pacific region. In this work Gatti listed Diyari terms for newly introduced items, which were collated from the New Testament translation. He also referred to material supplied by the outback pioneer and self-made ethnographer G. Aiston (1879--1943). The work contains a map of subgroups of Australian languages (\figref{bkm:Ref76045165gatti}) which is similar, but not identical to, to the map produced by \citet{schmidt_gliederung_1919} in German (\figref{bkm:Ref76045165}).


\begin{figure}
\includegraphics[width=\textwidth]{figures/stockigt-img108.png}
\caption{Gatti's map of Australian languages \citeyearpar{gatti_britannnia_1934}}
\label{bkm:Ref76045165gatti}
\end{figure}

\begin{figure}
% \includegraphics[width=\textwidth]{figures/stockigt-img109.tif}
\includegraphics[width=\textwidth]{figures/stockigt-img109.png}
\caption{Karte der Eingebornen-Sprachen von Australien \citep{schmidt_gliederung_1919}}
\label{bkm:Ref76045165}
\end{figure}

\section{Early analyses of Diyari and other Karnic languages}
\label{sec:key:8.5}\label{bkm:Ref74222915}\label{bkm:Ref339553314}\label{bkm:Ref336762486}\label{bkm:Ref456253749}
Having introduced the large body of Diyari grammatical material produced in the pre-contemporary descriptive era, the remainder of this chapter assesses and compares the analyses given in different sources.

The missionary-grammarians' descriptions of Diyari morphology and syntax adhere closely to the structure and terminology inherent in the traditional grammatical framework. Each is a succinct and relatively short document compared with other early grammars, notably Threlkeld’s inaugural PN grammar (\citeyear{threlkeld_australian_1834}; \chapref{chap:key:3}), Meyer’s grammar of Ramindjeri (\citeyear{meyer_vocabulary_1843}; \sectref{bkm:Ref74068587}), Kempe’s grammar of Arrernte (\citeyear{kempe_grammar_1891}; \sectref{bkm:Ref74837211}), and Roth’s grammar of Pitta-Pitta (\citeyear{roth_ethnological_1897}; \sectref{bkm:Ref74727849}), all of which made initial descriptions of previously undescribed languages.

\newpage
Each missionary grammar of Diyari is organised around the nine classical parts of speech and the traditional subheadings thereof. Compare \figref{fig:key:161} and \figref{fig:key:162}, given by \citet{flierl_christianieli_1880} and \citet{reuther_dieri_1894}, which not only name the same \textit{Redeteile} (parts of speech) in the same order using Latin terminology, but give the same illustrative example for each. In each grammar the Diyari missionary-grammarians list the subtypes of pronoun that are shown in Kühner’s grammar of Classical Greek (\citeyear[xxiii]{raphael_ausfuhrliche_1834}; \figref{fig:key:2-10}): personal, personal possessive, demonstrative, interrogative, indefinite, reflexive/reciprocal/relative, and correlative. The collapse of the three traditionally distinct pronominal subheadings – reflexive, reciprocal and relative – into a single unit is discussed in §\ref{bkm:Ref339715229}.


\begin{figure}[t]
\includegraphics[width=.8\textwidth]{figures/stockigt-img110r.png}
\caption{Flierl’s Redeteile “parts of speech” \citeyearpar[6]{flierl_christianieli_1880}}
\label{bkm:Ref449376051}\label{fig:key:161}
\end{figure}


\begin{figure}[t]
\includegraphics[width=.8\textwidth]{figures/stockigt-img111r.png}
\caption{Reuther’s \textit{Redeteile} (parts of speech; \citeyear[2]{reuther_dieri_1894})}
\label{bkm:Ref449376087}\label{fig:key:162}
\end{figure}


Within this classical descriptive structure, the grammars of Diyari adequately conveyed a number of PN structures that are not found in classical European languages: ergativity, a split ergative system of marking syntactic case, the difference in the marking of alienably and inalienably possessed constructions, an inclusive/exclusive distinction in non-singular first-person pronouns, and processes of marking clausal dependency within the morphology of the verb. The grammarians tended to appropriate the European classical tradition in order to accommodate foreign structures for which the traditional grammatical framework was descriptively powerless. Koch’s initial description (\citeyear{koch_untitled_1868}) of some of these PN ``peculiarities'', written after a relatively short encounter with the language, is neat and concise relative to the PN corpus.

\subsection{Case paradigms}
\label{sec:key:8.5.1}\label{bkm:Ref326948838}\label{bkm:Ref326948895}
{{Throughout more than three decades of successive descriptions of Diyari, the Lutheran missionaries strictly maintained the Latin six-case paradigm, presenting cases termed “nominative'', “genitive” (dative), “dative” (allative/locative), “accusative'', “vocative” and “ablative” (ergative/instrumental).} }

The Diyari grammars followed \citet{teichelmann_outlines_1840} in only placing case suffixes marking functions associated with the case systems of European languages in the paradigm. The practice differs from that employed by \citet{threlkeld_australian_1834} and by Günther (\citeyear{gunther_native_1838}; \citeyear{gunther_lecture_1840}) in description of languages spoken in New South Wales. Like \citet{teichelmann_outlines_1840}, \citet{symmons_grammatical_1841}, \citet{kempe_grammar_1891}, C. Strehlow (\citeyear{strehlow_untitled_1931}[c.1907]), and T. G. H. \citet{strehlow_aranda_1938}, each of the Diyari sources considered here, with the exception of \citet[58, 64--65]{gatti_lingua_1930}, accounted for the remainder of Diyari case morphology {{under the heading} {\textit{Postpositionen}}}.

Gatti presented a ten “case” Diyari paradigm (\figref{fig:key:163}), which was largely informed by Reuther and Strehlow’s New Testament translation \citeyearpar{reuther_testamenta_1897}. He included suffixes marking the ablative case \textit{ablativo}, and the dative on female nouns \citep[55]{austin_grammar_2013} \textit{dativo di comado}. He also included a case form termed \textit{comitativo}, which shows the proprietive suffix –\textit{nthu}, which is analysed by \citet[48]{austin_grammar_2013} as derivational. The function of the suffix Gatti calls \textit{limitativo} consists of the dative plus the post-inflectional suffix \textit{-rlu} `STILL', so that \textit{Maria-ngulu} would mean “as far as Maria” (Austin pers. comm.).\footnote{Gatti gives the form of the ergative suffix, termed \textit{nominativo agente}, on this female personal name incorrectly. No early Diyari grammarian recorded the different case marking on female person names (\sectref{bkm:Ref339541914}). The ergative form in Gatti’s paradigm should be Mariyandru (\citealt{austin_grammar_2013}: 53[1981a]).}


\begin{figure}
\includegraphics[width=.8\textwidth]{figures/stockigt-img112.jpg}
\caption{Gatti’s case paradigm \citeyearpar[58]{gatti_lingua_1930}}
\label{bkm:Ref449376202}\label{fig:key:163}
\end{figure}

\subsection{Description of ergativity}
\label{sec:key:8.5.2}\label{bkm:Ref456873578}
Grammarians of Diyari followed Schürmann (\citeyear{schurmann_vocabulary_1844}; \sectref{bkm:Ref336283270}) in accounting for the difference in the form and function of ergative and nominative nouns and pronouns within the discussion of the verb. It is here that the Diyari grammarians clarified the relationship of verb transitivity to syntactic argument case marking. Koch and Reuther (\citeyear{reuther_three_1981}: 3[1899]), however, also provided an additional account of ergative function within the description of nominal morphology. \citet[no pag.]{koch_untitled_1868} wrote: “The active form is one specific to this language which expresses the subject in an active manner.''\footnote{``Die Activform ist eine der Sprache eigenthümliche, welche das Subject activ oder thätig ausdrückt.''}

When clarifying ergative function within a discussion of the verb, Koch explained:

\begin{quote}
For the transitive verbs, one always uses the active form as the subject (both for the pronoun and for the substantive). For the intransitives one uses the nominative, likewise also for the reflexive and reciprocal verbs. \citep[no pag.]{koch_untitled_1868}\footnote{``Regel Bei den transitiven Verben gebraucht man als Subject stets die Act. form (sowohl beim pronomen als beim Subst.). Bei den intransitiven gebraucht man den Nom., gleicherweise auch bei den Reflecciven u. Reciproken Verben.''}
\end{quote}

\citet[no pag.]{koch_untitled_1868} also distinguished \textit{Verba Activa} (active verbs) from \textit{Verba Reflecciva} (reflexive verbs) and \textit{Verba Reziproka} (reciprocal verbs). Active verbs were defined as “verbs describing an activity which is performed by the subject themselves”. These were divided into two classes named \textit{transitiva} (Examples \ref{bkm:Ref329630469} \& \ref{bkm:Ref329630485}) and \textit{intransitiva} (Examples \ref{bkm:Ref329630508} \& \ref{bkm:Ref329630522}) and exemplified.\footnote{Note that Koch’s schema is problematic in first assigning reflexive and reciprocal verbs to a non-“active” class, before observing that they behave like the “intransitive” category of his “active” class of verbs.} The terms “transitive” and “intransitive” had previously only been employed with the same modern reference by \citet[20]{moorhouse_vocabulary_1846}.

\ea\label{bkm:Ref329630469}
	kuba natu nandrai \\
	\glt `ich schlage den Knaben'\\
	(\citealt{koch_untitled_1868}: no pag.) \\
	\gll kupa              ngathu     nandra-yi\\
	boy-[\stockcaps{ACC}]  1\textsc{sg}.\stockcaps{ERG}   hit-\stockcaps{PRES}\\
	\glt `I hit the boy'
\z

\ea\label{bkm:Ref329630485}
	Kintella kuballi nandrai \\
	\glt `Der Knabe schlägt den Hund' \\
	(\citealt{koch_untitled_1868}: no pag.) \\
	\gll Kinthala       kupa-li    nandra-yi\\
	dog-[\stockcaps{ACC}]   child-\stockcaps{ERG}    hit-\stockcaps{PRES}\\
	\glt `The child hits the dog'
\z

\ea\label{bkm:Ref329630508}
	nanni wappai \\
	\glt `Ich gehe' \\
	(\citealt{koch_untitled_1868}: no pag.) \\
	\gll nganhi wapa-yi\\
	1\textsc{sg}.\stockcaps{NOM} go-\stockcaps{PRES}\\
	\glt `I go'
\z

\ea\label{bkm:Ref329630522}
	Kuba tikai \\
	\glt `Der Knabe kehrt zurück' \\
	(\citealt{koch_untitled_1868}: no pag.) \\
	\gll kupa thika-yi\\
	boy-[\stockcaps{NOM}] return-\stockcaps{PRES}\\
	\glt `The child returns'
\z

Each subsequent Diyari grammar gave similar, although not identical, clauses to illustrate ergative marking in introductions to the verb. Observe, for example, Flierl’s explanation (\figref{fig:key:164}).


\begin{figure}
\includegraphics[width=.8\textwidth]{figures/stockigt-img113.png}
\caption{Flierl’s explanation of different marking of arguments of transitive and intransitive verbs \citeyearpar[32]{flierl_christianieli_1880}}
\label{bkm:Ref449368374}\label{fig:key:164}
\end{figure}

While it is possible that the Diyari grammarians were independently motivated to clarify ergative form and function in this way, and had not been influenced by Teichelmann \& Schürmann, other aspects of the grammars show unequivocal influence from Teichelmann \& Schürmann (\citeyear{teichelmann_outlines_1840}; \sectref{bkm:Ref326949499}; \sectref{bkm:Ref457382604}; \sectref{bkm:Ref457382655}; \sectref{bkm:Ref73520500}), lending weight to the hypothesis that in this regard the Diyari grammarians were guided by their Lutheran predecessors.

\subsubsection{Placement of the ergative case in the paradigm}
\label{sec:key:8.5.2.1}\label{bkm:Ref326949499}
Following \citet{koch_untitled_1868}, ergative and instrumental case forms were placed at the bottom of case paradigms and labelled “active” and sometimes “ablative'', in each missionary grammar of Diyari and of other Karnic languages (\figref{fig:key:165}; \figref{fig:key:166}).

Teichelmann and Schürmann had been motivated to place the ergative case at the bottom of the paradigm by the syncretism of \textit{most} nominal types in Kaurna for ergative and instrumental function (\sectref{sec:key:5.4.2}). Their exemplification of formally distinct ergative and instrumental interrogatives in Kaurna showed that they used the term “active” to label ergative function and “ablative” to label instrumental function (\figref{fig:5:88}). In Diyari, however, there is no nominal class for which ergative and instrumental function have distinct forms.

In \textit{nominal} paradigms, each Diyari missionary-grammarian, with the exception of Reuther (\sectref{bkm:Ref326948719}), assigned \textit{both} labels “active” and “ablative” to the er\-ga\-tive-instrumental form at the bottom of the paradigm. In this, they followed Teichelmann and Schürmann. Note that Koch uses three German prepositions: \textit{mit, von} and \textit{durch} – roughly translatable into English as ``with'', ``from'' and ``by'' – to translate the multi-functional form labelled both “active” and “ablative''.


\begin{figure}
\includegraphics[width=.5\textwidth]{figures/stockigt-img114.png}
\caption{Koch’s case paradigm of dual nouns \citeyearpar{koch_untitled_1868}}
\label{bkm:Ref449368478}\label{fig:key:165}
\end{figure}



\begin{figure}
\includegraphics[width=\textwidth]{figures/stockigt-img115.png}
\caption{Flierl’s case paradigm of singular nouns \citeyearpar[10--11]{flierl_christianieli_1880}}
\label{bkm:Ref449368538}\label{fig:key:166}
\end{figure}

Like their Lutheran predecessors, the Diyari grammarians, again with the exception of Reuther (\sectref{bkm:Ref326948719}), used the term “active” in order to capture ergative function and the term “ablative” to capture instrumental function. This is evident from the pronominal paradigms (\figref{fig:key:167}; \figref{fig:key:168}) in which the term “ablative” is not used to name either the ablative or ergative cases. Recognising that pronouns were unlikely to act in the role of instrument, the early grammarians did not assign the label “ablative” to the ergative/instrumental pronominal forms. Pronouns in ergative case are called “active''. In doing so, these Diyari grammarians, with the exception of Reuther, followed Teichelmann and Schürmann (\figref{fig:5:85}).


\begin{figure}
\includegraphics[width=.5\textwidth]{figures/stockigt-img116.png}
\caption{Koch’s pronominal paradigm \citeyearpar{koch_untitled_1868}}
\label{bkm:Ref449368589}\label{fig:key:167}
\end{figure}

\begin{figure}
\includegraphics[width=\textwidth]{figures/stockigt-img117.png}
\caption{Flierl’s pronominal paradigm \citeyearpar[18--19]{flierl_christianieli_1880}}
\label{bkm:Ref449368610}\label{fig:key:168}
\end{figure}

\subsubsection{Teichelmann \& Schürmann’s legacy}
\label{sec:key:8.5.2.2}

Teichelmann \& Schürmann’s placement of ergative case forms in final paradigmatic position, which they instigated in their first grammar of a language spoken in South Australia \citeyearpar{teichelmann_outlines_1840}, was carried forward through the missionaries' unpublished descriptions of Diyari, into the grammars of the language published in German in the first decade of the twentieth century.

Planert’s placement of the ergative case last in the paradigm was influenced either by Wettengel’s memory of the missionaries' paradigms, or by a MS grammar Wettengel had in his possession (\sectref{bkm:Ref74225865}). Planert’s Arrernte grammar (\citeyear*{planert_australische_1907}; \figref{fig:2:28}), by contrast, follows existing grammars of that language (\citealt{kempe_grammar_1891}; C. \citealt{strehlow_untitled_1931}~[c.1907]) in placing ergative forms in second position next to the nominative.

Like Planert, \citet[94]{eylmann_eingeborenen_1908} placed Diyari ergative forms last in the paradigm. \citet[58--65]{gatti_lingua_1930}, however, broke from tradition and labelled the Diyari ergative case \textit{nominativo agente} and placed it in second paradigmatic position (\figref{fig:key:163}).

\subsubsection{Reuther’s case paradigms}
\label{sec:key:8.5.2.3}\label{bkm:Ref326948719}

Reuther’s case paradigms (\citeyear{reuther_dieri_1894}; \citeyear{reuther_reuther_1899}) differ from earlier grammars in the way they label the ergative case. In contrast to the earlier Diyari missionaries who use either one or both of the labels “active” and “ablative” to label nouns marked with a suffix that marks ergative and instrumental functions, Reuther’s grammars (\citeyear{reuther_dieri_1894}; \citeyear{reuther_three_1981} [\citeyear{reuther_reuther_1899}; \citeyear{reuther_repr_1899}; \citeyear{reuther_ms_1901}]) which are the last and best-known missionary descriptions of Karnic languages, only use the term “ablative” (\figref{fig:key:169}). The term “active” is absent from all of Reuther’s case paradigms.


\begin{figure}
\includegraphics[width=.5\textwidth]{figures/stockigt-img118.png}
\caption{Reuther's Diyari case paradigm (\citeyear{reuther_dieri_1894}: no pag.) showing the ergative case at the bottom of the paradigm, labelled “ablative''}
\label{bkm:Ref449368800}\label{fig:key:169}
\end{figure}

In this regard, Reuther’s work is similar to Meyer’s, in which the term “ablative” was used to name ergative case function. Note that in contrast to Koch (\figref{fig:key:165}), who translated nouns marked in ergative and instrumental function using three prepositions – \textit{mit, von} and \textit{durch} – Reuther used only \textit{von.}

Providing English equivalents for German prepositions is fraught. The preposition \textit{von} translates into English as either “by’, “of” and “from''.  Hercus \& Schwarzschild (\citeyear[4]{a_three_1981}) translate Reuther’s German rendition of Diyari ``ablative'' nominal forms `by X'. Their translation of the Wangkangurru form labelled ``ablative'' \citeyearpar[31]{reuther_repr_1899} is, however, translated as `from X'. This difference is motivated by the syncretism of ergative and ablative functions on nouns in Wangkangurru \citep[66]{hercus_grammar_1994}. Given the absence of ablative pronominal forms, which do not show syncretism with the ergative case \citep[110]{hercus_grammar_1994} from all of Reuther’s Wangkangurru paradigms, it is more likely that the translation `von X' was used to invoke ergative function, as in the Diyari paradigms, rather than ablative. A translation of Reuther’s “\textit{von} X” into English as “by X” better captures Reuther’s intended meaning of the term “ablative''.

\citet[53]{austin_notitle_1981} assumes that, when choosing the term “ablative” to name ergative/instrumental case, Reuther was “following the classical tradition [and] was aware of the Latin passive construction and therefore called this agentive case `the ablative’\thinspace”. This had been Meyer’s reasoning for choosing the label “ablative''. The “ablative of personal agent” was described by \citet[38]{meyer_vocabulary_1843} as having “the force of Latin ablatives”. However, it is possible that, when dropping one of the two case labels previously used by all Diyari grammarians, Reuther also realised that two case labels were unnecessary when naming case forms which never formally distinguish between ergative and instrumental functions. He dropped the term “active” and maintained the term “ablative''.

\subsubsection{“Ablative” interrogative pronouns}
\label{sec:key:8.5.2.4}\label{bkm:Ref457382604}

The paradigms of Diyari interrogatives from the second half of the nineteenth century show the extent of influence that Teichelmann and Schürmann’s earliest grammar of a South Australian language had on these later Lutheran grammars. Like Teichelmann and Schürmann’s paradigm of Kaurna interrogatives (\figref{fig:5:88}), the descriptions of Diyari also show different “active” and “ablative” forms in the interrogative paradigm. The assumption that Australian languages were \textit{structurally} similar (\sectref{bkm:Ref340310942}) led the early Diyari grammarians to import structures across linguistic boundaries, and to impose Kaurna structures onto Diyari.

The interrogative that was shown in the Kaurna paradigm (\figref{fig:5:88}) as “active'', \textbf{ngando}, was illustrated in ergative function and the interrogative \textbf{ngannarlo} labelled “ablative'' was illustrated with instrumental function (Examples \ref{ex:5:13} and \ref{ex:5:14} on p. \pageref{ex:5:14}). Teichelmann and Schürmann provided these examples to show that these Kaurna interrogatives marked the “active” and ``ablative” cases distinctly. Interrogatives were the only nominal type which were not described as showing syncretism for ergative and instrumental case functions.

Influenced by Teichelmann \& Schürmann’s description, later Lutheran missionaries also provided distinct “active” and “ablative” forms of “personal interrogatives” in Diyari (\figref{fig:key:8-170}) as well as in Wangkangurru (\figref{fig:key:8-171}). While Teichelmann and Schürmann had carefully exemplified the functional difference of the ergative and instrumental interrogatives in Kaurna (Examples \ref{bkm:Ref329626419}  \& \ref{bkm:Ref329626478}), the distinction is not motivated in Diyari, in which all classes of nominals mark ergative and instrumental function identically (\citeyear{austin_grammar_2013}: 53[1981a]).


\begin{figure}
\includegraphics[width=.8\textwidth]{figures/stockigt-img119.png}
\caption{Koch’s case paradigm of interrogatives (\citeyear{koch_untitled_1868}: no pag.)}
\label{bkm:Ref449368926}\label{fig:key:8-170}
\end{figure}

\begin{figure}
\includegraphics[width=.8\textwidth]{figures/stockigt-img120.png}
\caption{Flierl’s case paradigm of interrogatives \citeyearpar[22--23]{flierl_christianieli_1880}\\
showing Diyari forms on the left and Wangkangurru forms on the right}
\label{bkm:Ref449368941}\label{fig:key:8-171}
\end{figure}


The Diyari interrogative \textbf{wale} \textit{warli,}\footnote{There appears to be some confusion in case forms of interrogatives given by the missionaries. While Austin (\citeyear{austin_grammar_2013}: 67~[1981a]) describes the marking of syntactic case on interrogatives as showing ergative alignment (A/SO) the missionaries all show distinct nominative and accusative forms.} which is labelled “active” in the Diyari interrogative paradigms, is described by Austin (\citeyear{austin_grammar_2013}: 67~[1981a]) as the ergative interrogative form. The form that is shown as “ablative” in each of the missionaries' interrogative paradigms is not, however, a regular interrogative case form (see \citealt{austin_grammar_2013}: 67~[1981a]). The form that the missionaries slotted into the paradigmatic position is a pronoun marking possessive function, which is then marked for ergative case.

Interrogative pronouns with possessive function, translated into English as `whose?’, are labelled “dative” by Austin (\citeyear{austin_grammar_2013}: 66[1981a]) and were labelled “genitive” by the missionaries. When in possessive function, these forms may be further inflected for the clausal case in which the possessed NP stands. In Diyari, case is usually marked on the final constituent of a continuous NP, but each constituent can receive case marking if the NP is discontinuous, or “where there is special emphasis or contrast intended” (\citealt{austin_grammar_2013}: 97[1981a]), as in \figref{fig:key:8-172}. In this example the possessive 3sgnfDAT pronoun \textit{nhungkarni} qualifies the head of the possessed NP \textit{kinthala} dog. The NP “his dog” stands in ergative case, and ergative case marking occurs on both constituents. Note that the ergative suffix is “realised as \textit{{}-li} after a and \textit{{}-yali} elsewhere” (\citealt{austin_grammar_2013}: 35[1981a]), except on female proper names which show \textit{{}-ndru} (ibid.:53).


\begin{figure}
% \includegraphics[width=\textwidth]{figures/stockigt-img121.emf}
\includegraphics[width=.8\textwidth]{figures/stockigt-img121r.png}
\caption{Ergative case marking on both constituents of a continuous NP in Diyari (\citealt{austin_grammar_2013}: 97[1981a])}
\label{bkm:Ref466897668}\label{fig:key:8-172}
\end{figure}

Thus, it is likely that a possessive interrogative might receive clausal case marking in Diyari. Indeed, Austin (\citeyear{austin_grammar_2013} [1981: 67]) states: “It may be possible for interrogative-indefinites in dative case where there is an understood head noun to be followed by noun case suffixes … but there are no examples in the corpus”. It is, however, these forms that are used to fill in the vacant position in the paradigm of interrogative pronouns, notwithstanding the discrepancies that exist between interrogative forms in early sources and modern sources. The possessive interrogative \textbf{warnini} marked with the ergative suffix \textbf{-li},  \textbf{warnini-li}, is the form shown as the Diyayi “ablative” interrogative pronoun. This has not previously been understood. The missionaries' “ablative” interrogatives might occur in a phrase such as “Whose dog bit the child?''. Flierl’s Wangkangurru “ablative” interrogative (\figref{fig:key:8-171}) is of the same structure (see \citealt{hercus_grammar_1994}: 127):

\ea
worakunaru\\
\gll waRakunha-ru\\
\stockcaps{INTER}.\stockcaps{POSS}-\stockcaps{ERG}\\
\z

Understanding the missionaries' rationale for inserting these forms in their paradigms requires comparison of this body of material produced by HMS missionaries at Bethesda with earlier grammars of Australian languages written by the Dresdner missionaries, whose works guided later descriptive practices.

\subsection{The ablative case}
\label{sec:key:8.6.1}\label{bkm:Ref456253920}

\citet{teichelmann_outlines_1840}, \citet{meyer_vocabulary_1843}, and the Diyari grammarians' use of the term “ablative” differed from other corpus grammarians, who used the term “ablative” to name the nominal suffix that marked the primary function of the Latin ablative, `motion away from' (\tabref{fig:5:89}). A consequence of Teichelmann \& Schürmann’s and the Diyari grammarians' practice of naming of the case with ergative and instrumental functions either “active” or “ablative'', or simply “ablative'', is that nominals standing in ablative case marking the “location from which motion proceeds and indirect cause” (\citeyear{austin_grammar_2013}: 55[1981a]) were not described within the traditionalist Latin case paradigm. The term had already been “used up” as a descriptor of syntactic case.

Unlike Meyer (\sectref{bkm:Ref74730717}), who also labelled the ergative case “ablative'', Teichelmann \& Schürmann and the Diyari grammarians did not introduce an additional place into the paradigm in order to accommodate the primary function of the Latin ablative. Other grammars -- including later Lutheran grammars of Arrernte (\citealt{kempe_grammar_1891}; C. \citealt{strehlow_untitled_1931} [c.1907]), which utilised a five-case Latin paradigm, but which conceived of the ergative as a second nominative case (\sectref{bkm:Ref339538382}) -- adequately account for PN ablative functions under the heading “ablative''. The absence of ablative forms from nominal paradigms and the relegation of the discussion of ablative function to other sections of the grammar, most notably the section headed “postpositions'', are peculiar to the early Kaurna and Diyari grammars. The Diyari grammarians also exemplified ablative marking and function under the heading “correlative pronouns” (\sectref{sec:key:2.3.3}; Example \ref{ex:key:2-8} on p. \pageref{ex:key:2-8}).

C. Strehlow’s earliest grammar of Arrernte (\citeyear{strehlow_untitled_1931} [c.1907]; \sectref{bkm:Ref456272021}) presented comparative case paradigms of nominals in Arrernte, Diyari and Ramindjeri. The ergative case forms were placed in second paradigmatic position and labelled ``nominative transitive''. Consequently, the ablative forms, labelled “ablative”, \textit{are} included in the paradigm (\figref{fig:key:8-173}). In this way C. Strehlow’s presentation of case in Diyari differs from all other missionary grammars of Diyari.

\begin{figure}
\includegraphics[width=\textwidth]{figures/stockigt-img122r.png}
\caption{Kramer’s copy of C. Strehlow's comparative table of case forms (\citeyear{strehlow_untitled_1931}: 48--49 [c.1907])}
\label{bkm:Ref456019058}\label{fig:key:8-173}
\end{figure}

\subsection{Split system of marking syntactic case and other case syncretism}
\label{sec:key:8.6.2}

The complexity of the ergative split on different nominal types in Diyari was more complicated than the splits encountered by most earlier grammarians, with the exception of Meyer and Taplin, who described Ngarrindjeri, and Threlkeld, who described Awabakal.

In Diyari the alignment of the marking of the syntactic cases – nominative, ergative and accusative – on different nominal types is sensitive to number and gender (\tabref{fig:key:8-174}). Similarly, peripheral cases -- dative, allative, and locative -- show allomorphic variation according to number, gender, and nominal-subclass (\tabref{fig:key:8-175}).

The record of the split in the marking of the syntactic cases, and the sensitivity to number and gender of nouns in peripheral cases, differs across the early sources. It took the missionaries at Bethesda some time to properly convey the marking of different nominal types in nominative, accusative and ergative (``ablative'') cases, and the different marking of allative/locative and dative cases on different types of nouns.

The following sections assess the missionary-grammarians' record of case marking on different nominal types in Diyari. Attention is given to the obstacles that compromised the accuracy of the historical record. \tabref{fig:key:8-174} and \tabref{fig:key:8-175} are referred to throughout the discussion.


\begin{table}
\small
\begin{tabularx}{\textwidth}{QQQ}
\lsptoprule
{ {A/{SO}}}   {Ergative alignment} & {  {AS/O}}  {Accusative alignment} & {  {A/S/O}}  {Tripartite marking}\\
\midrule
Male personal names\newline
 Singular common nouns\newline
 Interrogatives & First and second-person non-singular pronouns & First and second-person singular\newline
 and third-person pronouns\newline
 Non-singular common nouns\newline
 Female person names\\
\lspbottomrule
\end{tabularx}
\caption{The syntactic alignment of different nominal types in Diyari (from \citealt{austin_grammar_2013}: 52)}
\label{bkm:Ref466955740}\label{fig:key:8-174}
\end{table}

\begin{sidewaystable}
\small
\begin{tabularx}{\textwidth}{QQQQQQ}
\lsptoprule
{Case label assigned by Austin}  & {Case label assigned by the missionaries}  & {Singular common nouns}
 & {Male personal names}
 & {Non-singular common nouns}
 & { {Female personal}}  {names}\\
\midrule
Nominative & Ablative/Active  & -Ø & -nha & -Ø & {\shadecell -ni}\\
{Accusative}  & {Nominative}  &  &  & \multicolumn{2}{c}{{}-nha}\\
{Ergative/\newline Instrumental}  & {Accusative}  & \multicolumn{3}{c}{{}-li/-yali}  & {}-ndru\\
{Ablative}  & {{}-}  & {}-ndru & \multicolumn{3}{c}{{}-ngundru}\\
Dative/Possessive  & Genitive  & -ya & \multicolumn{2}{c}{\shadecell -rni} & -nhangka\\
{Allative}  & {Dative}  &  & \multicolumn{2}{c}{{}-ngu}  & {}-nhangu\\
Locative & Dative & {\shadecell -nhi} &  &  & \\
\lspbottomrule
\end{tabularx}
\legendbox{This table does not show the marking of case on spatial locational nominals, which are a small closed class, and which inflect only for ablative and allative cases (\citealt{austin_grammar_2013}: 42[1981a]). \textit{Shaded} cells indicate suffixes undifferentiated in the missionaries' orthographies. Each was shown as -ni.}
\caption{Syncretism of Diyari cases on different types of noun (content informed by \citeyear{austin_grammar_2013}: 53--55)}
\label{bkm:Ref466930554}
\label{fig:key:8-175}
\end{sidewaystable}

\subsection{Sensitivity of case marking to number and gender}
\label{sec:key:8.6.3}

Adherence to the word and paradigm descriptive model was an especially effective method of conveying the sensitivity to number. The application of the model in the initial grammar (\citealt{koch_untitled_1868}; \figref{fig:key:8-176}) successfully conveyed the marking of each case included in the paradigms \textit{on pronouns} in different numbers.

Note that the style of presentation of pronouns in the extant copy of Koch’s analysis differs from later sources, which gave discrete paradigms for person, and placed number on the horizontal axis. Yet, the following discussion shows that it took the Diyari grammarians decades to properly record the complexity of the systems paradigmatically. Some aspects of the complexity, recorded by Austin, were never captured in the missionaries' analyses.


\begin{figure}
\includegraphics[width=.49\textwidth]{figures/stockigt-img123.png}
\includegraphics[width=.49\textwidth]{figures/stockigt-img124.png}
\legendbox{Third-person singular feminine pronouns are shown under the masculine.}
\caption{Koch’s WP presentation of Diyari pronouns (1868).}
\label{bkm:Ref449369144}\label{fig:key:8-176}
\end{figure}

Koch’s initial paradigms for \textit{common nouns} (\citeyear{koch_untitled_1868}; \figref{fig:key:8-177}) failed to show the sensitivity of case marking to number. This earliest case paradigm of a Diyari noun shows the singular \textit{kinthala} `dog', and the dual \textit{kinthalawurlu} `two dogs'. Koch did not provide a paradigm for plural nouns. See that the singular accusative and dual accusative forms are each unmarked. Following Koch, no early grammarian before \citet{reuther_dieri_1894} recorded that non-singular common nouns show tripartite marking (\stockcaps{A/S/O}) and are marked with the suffix \textit{-nha} in accusative case (see \tabref{fig:key:8-174} and \tabref{fig:key:8-175}). Instead, non-singular nouns were shown as having the same ergative alignment (\stockcaps{A/SO}) as singular nouns.


\begin{figure}
\includegraphics[width=.5\textwidth]{figures/stockigt-img125.png}
\caption{Koch’s noun case paradigm (\citeyear{koch_untitled_1868}: no pag.)}
\label{bkm:Ref449369264}\label{fig:key:8-177}
\end{figure}

The dual accusative form shown by Koch (\citeyear[no pag.]{koch_untitled_1868}; \figref{fig:key:8-177}) is unmarked:

\ea
\glll {kintella ulo} \\
*kinthala-wurlu \\
dog-\textsc{dl}-[\stockcaps{ACC}]\\
\glt “two dogs” \z

Neither Schoknecht (\citeyear{schoknecht_grammar_1947}) nor \citet[10, 12]{flierl_dieri_1880} declined dual and plural nominals for case, but rather gave syntagmatic rules stating that case inflection follows inflection for number. Presumably they also thought the situation was no more complicated than Koch had conveyed, and assumed that non-singular nouns were marked in the same way as singular nouns (A/SO).

Reuther’s record of the distinctive marking of case on singular and non-singu\-lar nouns and pronouns shown in his nominal case paradigms (\sectref{bkm:Ref326948719}) substantially improves on the record left by \citet{flierl_dieri_1880}, Schoknecht (\citeyear{schoknecht_grammar_1947}), and \citet{koch_untitled_1868}.

It took grammarians of Diyari over two decades to accurately describe this complexity, with \citet{reuther_dieri_1894} being the first grammarian to properly display the sensitivity to number of the marking of the syntactic cases on nouns. His paradigms show ergative alignment (A/SO) for singular number and tripartite marking on dual and plural common nouns (A/S/O; \figref{fig:key:8-178}).


\begin{figure}
% \includegraphics[width=\textwidth]{figures/stockigt-img126.emf}
% \includegraphics[width=\textwidth]{figures/stockigt-img127.emf}
\includegraphics[width=.45\textwidth]{figures/stockigt-img126.png}
\includegraphics[width=.45\textwidth]{figures/stockigt-img127r.png}
\caption{Reuther’s noun case paradigm \citeyearpar[5--6]{reuther_dieri_1894}}
\label{bkm:Ref449369326}\label{fig:key:8-178}
\end{figure}

The conservative case paradigms provided by the Diyari grammarians, which -- unlike those given by Threlkeld (\citeyear{threlkeld_australian_1834}; \sectref{sec:key:3.3.3}) and by Günther (\citeyear{gunther_native_1838}; \citeyear{gunther_lecture_1840}; \sectref{sec:key:4.4.3}) -- only included forms marked by case suffixes carrying functions that are marked morphologically in SAE languages. As such, they were powerless to convey the asymmetry in the syncretism of allative, dative and locative cases on different nominal types. The function “motion towards X” is not marked distinctly from other case functions on any nominal type (\tabref{fig:key:8-175}). The suffix marking the allative case on singular common nouns \textit{-ya} was, however, exemplified as a ``mono-syllabic affix'', and allative function was also exemplified in a discussion of “correlative pronouns” (\sectref{sec:key:2.3.3}; Example \ref{ex:key:2-9} on p. \pageref{ex:key:2-9}). The missionaries' discussion and exemplification of the marking of allative function falls short of showing the different marking of this function on different nominal types or on the same nominal type in different numbers. The different marking of the case function on male names and on singular and non-singular nouns may have been adequately conveyed if the Diyari missionary-grammarians had, like Threlkeld and Günther, included suffixes marking cases extraneous to the Latin inventory within their paradigms.

Similarly, the different marking of singular and non-singular nouns in the ablative case was unlikely to have been successfully recorded in the early sources because the ablative inflection was treated as a postposition, and postpositions were not declined for number (\sectref{bkm:Ref456253920}).

But differences in the marking of cases on different nominal types where cases were presented paradigmatically were also inadequately conveyed. The earliest Diyari sources failed to show that non-singular common nouns are marked differently for dative (“genitive'') case than are singular common nouns, and that the allative and locative cases, termed “dative'', are marked differently on singular and non-singular nouns (\citealt{Stockigt2017}; §8.6.6). This is in part due to inadequate analysis and failure to recognise the potential complexity of splits, but also to the missionaries' inability to recognise and/or to orthographically distinguish nasal phonemes at interdental, retroflex and alveolar positions. Consequently, \textit{-nhi,} marking the locative case on common singular nouns, \textit{-rni}, marking dative case on non-singular common nouns and male personal names, and \textit{-ni} marking the nominative case on female personal names (\tabref{fig:key:8-175}) were identically represented as \textit{–ni}.

\subsubsection{Nominal declension classes}
\label{sec:key:8.6.3.1}\label{bkm:Ref339541914}
\largerpage
In order to convey morphophonemic variation and the sensitivity of case marking to gender and animacy, the missionaries presented nominal declension classes.

\tabref{fig:key:8-175} summarises a situation in which male proper names and singular common nouns show ergative alignment (A/SO). Male names are overtly marked with the suffix –\textit{nha} in nominative and accusative cases. Singular common nouns are unmarked in these cases. Female proper names show tripartite marking, taking different nominative and ergative suffixes from male proper names. These three nominal types also display different syncretism for the peripheral cases.


\begin{figure}[b]
\includegraphics[width=.8\textwidth]{figures/stockigt-img128r.png}
\caption{Reuther’s case paradigm of a female personal noun \citeyearpar[11]{reuther_dieri_1894}}
\label{bkm:Ref449372220}\label{fig:key:8-179}
\end{figure}


All early Diyari sources fail to account for the unique marking for case on female names in the early sources (\tabref{fig:key:8-175}). That female names are overtly marked in accusative case (A/S/O) while male personal names are not (A/SO) was undescribed. The different marking on female personal nouns in the missionaries' “genitive” (dative) and “dative” (locative/allative) cases was also not recorded in any early source. Reuther (\citeyear{reuther_dieri_1894}: 11; \citeyear{reuther_three_1981}: 7[1899]) first provided paradigms for a female proper noun. He declined \textbf{Parubukana} \textit{Parru-puka-nha}, literally “fish-bread-NOM” (\figref{fig:key:8-179}). The case forms given for this female name are, however, not the same as those recorded by Austin, including the nominative citation form, which in the variety recorded by Austin is marked for nominative case with \textit{-ni.} While the difference may result from language shift, or dialect difference, it is more likely that the absence in the record results from oversight on the part of the missionaries. Note that Gatti (\sectref{bkm:Ref335988141}; \figref{fig:key:163}) also presented an incorrect paradigm of the female name “Maria'', a fact that is unsurprising given that his grammar was informed by the missionaries' translations.


The different case marking on male personal nouns was, however, conveyed in the early grammars through the presentation of nominal declension classes. Each early Diyari source presented three declension classes. While membership of each class was differently defined in each early source, the third declension class consistently showed case marking on male proper nouns which accords with Austin’s modern record (\citeyear{austin_grammar_2013}~[1981a]). The third declension class was motivated by the different marking of case on common singular nouns and male proper nouns (\figref{fig:key:8-180}). The distinctive marking of male proper nouns was properly recorded through provision of a distinct declension class.


\begin{figure}
\includegraphics[width=.8\textwidth]{figures/stockigt-img129.png}
\caption{Koch's Deklination der Eigenamen (Declension of proper nouns; \citeyear{koch_untitled_1868}: no pag.).\\
Shown as the third declension class in later sources.
\label{bkm:Ref449371888}\label{fig:key:8-180}
}
\end{figure}

\largerpage
The first two of the missionary-grammarians' three declension classes were phonologically motivated by morphophonemic alternation, the complexity of which was never properly understood in the early descriptions of Diyari. Each author presented a slightly different range of nominals in the first two declension classes (see, e.g., \figref{fig:key:8-181}) in attempting to accommodate the morphophonemic processes which were later described by Austin (\citeyear{austin_grammar_2013}: 28~[1981a]). These are:

\begin{quote}
\begin{enumerate}
	\item[] The neutralisation of final vowels of trisyllabic common nouns ending in i or u. \\
	\item[] The ergative case suffix –yali $\to$ -li  / a, u \_
\end{enumerate}
\end{quote}

\begin{figure}
\includegraphics[width=\textwidth]{figures/stockigt-img130.png}
\caption{Planert's first two classes of nominal declension in Diyari \citeyearpar[690]{planert_australische_1908}}
\label{bkm:Ref449372014}\label{fig:key:8-181}
\end{figure}

None of the early grammarians of Diyari attempted to account for morphophonemic variation by describing process, as Günther (\citeyear{gunther_native_1838}; \citeyear{gunther_lecture_1840}) had done (see \sectref{sec:key:4.4.2}). The missionary-grammarians were unable to define membership of their claimed noun classes. Schoknecht wrote:

\begin{quote}
The Dieri language had three declensions, which we differentiate as first and second declension and the declension of the nomina propria … The endings of the words do not enable one to decide according to which declension they are to be flexed. Here practical use must come to one’s aid. \citep[2]{schoknecht_grammar_1947}
\end{quote}

The examples provided by each grammarian of Diyari in the tables of declension show a good range of bi-syllabic and tri-syllabic common nouns ending in different vowels, indicating that each grammarian was able to assign individual items to different classes. Only Reuther (\citeyear{reuther_three_1981}: 8[1899]) explained that the number of syllables was a controlling factor.

\subsection{{Postpositions}}
\label{sec:key:8.7.1}\label{bkm:Ref456339501}\label{bkm:Ref457382655}
\largerpage[-1]
Following Teichelmann \& Schürmann (\citeyear{TeichelmannSchürmann1840}; \sectref{sec:key:5.3.2}), the Diyari missionary-gram\-mar\-ians distinguished two types of postpositions. Reuther (\citeyear{reuther_three_1981}: 25[1899]) wrote: “Many locational relationships can be expressed by postpositions, but there are no prepositions. Postpositions are of two kinds, the first kind are strictly speaking not postpositions but simply monosyllabic affixes. The second type are true postpositions, they are free and independent words” (see also \citealt{flierl_christianieli_1880}: 50; \citealt{schoknecht_grammar_1947}: 13). Reuther’s grammars of other Karnic languages (\citeyear{reuther_ms_1901}) do not make this distinction. This section is absent from Koch’s incomplete grammar (\citeyear{koch_untitled_1868}).

The Diyari missionaries' \textit{einsilbige Affixe} (monosyllabic affixes) class corresponds to Teichelmann \& Schürmann’s “postfixa’. Their “independent words” corresponds to Teichelmann \& Schürmann’s ``postpositions''. The second class of “independent words” includes mostly locational words – translated as “in front of'', “behind” etc. – as well as the following three anomalies (\tabref{fig:key:8-182}).


\begin{table}
\begin{tabularx}{\textwidth}{llQQ}
\lsptoprule
\textbf{``Postposition''}
& Translated as: & Exemplifiaction:  & Recoded by \citet{Austin2013} \\
\midrule
\textbf{mara}
 & with & \textbf{noa mara} with the (lawful) husband & {}-\textit{mara} kinship proprietive (kin dyad) (p.48) \\
\textbf{pani} & without & \textbf{kalti pani} without a spear & \textit{pani} Adj: “none’, “no” (pp. 49--50) \\
\textbf{pota} & with & \textbf{Turupota} with fire & \textit{thurru putha} Fire ash A generic nominal followed by a specific nominal (p. 44) \\
\lspbottomrule
\end{tabularx}
\caption{Reuther's class of “postpositions'', which are ``independent words''}
\label{bkm:Ref449372358}
\label{fig:key:8-182}
\end{table}

Like Teichelmann and Schürmann’s “postfixa'', the Diyari grammarians' list of monosyllabic affixes (\figref{fig:key:8-183}) given at the end of the grammar under the heading \textit{Postpositionen} presents the balance of Diyari case inflections that are not included at the front of the grammar under the heading \textit{Substantivum} (1880: 52) because they mark case functions that are not marked morphologically in SAE languages. The “monosyllabic affixes” also includes derivational morphology, as well as post-inflectional morphology.


\begin{figure}
% \includegraphics[width=\textwidth]{figures/stockigt-img131.emf}
\includegraphics[width=.49\textwidth]{figures/stockigt-img131.png}
\includegraphics[width=.49\textwidth]{figures/stockigt-img132.png}
\caption{Flierl’s (\citeyear{Flierl1880}: 52) list of \textit{einsilbige Affixe} (monosyllabic affixes)}
\label{bkm:Ref449372478}\label{fig:key:8-183}
\end{figure}

Flierl’s \textit{einsilbige Affixe} that do not mark case include the post-inflectional suffixes \textbf{ru} –\textit{rlu} glossed `STILL' (\citealt[181]{austin_grammar_2013}; Example \ref{bkm:Ref329626776}) and \textbf{ldra} \textit{{}-lda} marking additional information (\citealt{austin_grammar_2013}: 185~[1981a]; Example \ref{bkm:Ref329626810}).

\newpage
\ea\label{bkm:Ref329626776}
	kalkauralu, \\
	\glt `until the evening' \\
	\citep[52]{flierl_dieri_1880} \\
	\gll Kalkawarra-ya-rlu\\
	evening-\stockcaps{ALL}-\stockcaps{STILL}\\
\z

\ea\label{bkm:Ref329626810}
	jidni ko ja nga ni koldra \\
	\glt `You are ignorant and so am I' \\
	\citep[52]{flierl_dieri_1880} \\
	\gll Yini          kuwu         nganha          kuwu-lda\\
	2\textsc{sg}.\stockcaps{NOM}    ignorant    1\textsc{sg}.\stockcaps{NOM}        ignorant-\textsc{addinf}\\
\z

\largerpage
Also included as “mono-syllabic affixes” are two derivational suffixes, proprietive suffix \textbf{-nto} \textit{-nthu} (\citealt{austin_grammar_2013}: 48~[1981a]; Example \ref{bkm:Ref329626419}) and \textbf{-la} \textit{-lha} “characteristic (\citealt{austin_grammar_2013}: 40~[1981a]; Example \ref{bkm:Ref329626478}), the last of which is absent from Reuther’s grammar (\citeyear{reuther_ms_1899}).

\ea\label{bkm:Ref329626419}
	kupanto
	\glt `with a child' \\
	\citep[52]{flierl_dieri_1880} \\
	\gll {Kupa-nthu}\\
	child-\stockcaps{PROP}\\
\z

\ea\label{bkm:Ref329626478}
	waru warula
	\glt of old. Formerly \\
	\citep[52]{flierl_dieri_1880} \\
	\gll waru-waru-lha \\
	{long ago-\stockcaps{REDUP}-\stockcaps{CHAR}} \\
	\glt of old
\z

The “monosyllabic affixes” that \textit{are} markers of case include those marking locative, dative, allative and ergative/instrumental cases. Here the Diyari grammarians exemplified both the spatial (Example \ref{bkm:Ref329627527}) and causal (Example \ref{bkm:Ref329627545}) functions of the ablative case:

\ea\label{bkm:Ref329627527}
mitandru 
\glt {out of the Earth}\\
\gll  {mitha-ndru}\\
\citep[52]{flierl_dieri_1880} \\
\glt ground-\stockcaps{ABL}
\z

\ea\label{bkm:Ref329627545}
widlandru 
\glt {on account of the woman}\\
\gll  {wilha-ndru}\\
\citep[52]{flierl_dieri_1880}\\
\glt woman-\stockcaps{ABL}
\z

Inflection for locative case on singular common nouns was presented as a “mono-syllabic affix” (Example \ref{bkm:Ref329627573}):

\ea\label{bkm:Ref329627573}
marani {in or on the hand}\\
\gll   {mara-nhi}\\
\citep[52]{flierl_dieri_1880} \\
\glt hand-\stockcaps{LOC}\\
\z

as was the marking of allative case function on singular common nouns (Example \ref{bkm:Ref329627599}):

\ea\label{bkm:Ref329627599}
\gll nghapaia \\
{towards water} \\
\glt \citep[52]{flierl_dieri_1880} \\
\gll  {ngapa-ya}\\
water-\stockcaps{ALL}\\
\z

As with other grammarians examined in this study, suffixes marking case functions not associated with the case systems of European languages that \textit{were} included in the early PN case paradigms were accounted for and exemplified \textit{again} by the Diyari grammarians under the heading ``postposition''. The Diyari grammarians list the ergative/instrumental inflection as a postposition (Example \ref{bkm:Ref329627843}), although it had already been accounted for within the case paradigms:

\ea
\label{bkm:Ref329627843}
\gll pitali \\
{with the stick} \\
\glt \citep[52]{flierl_dieri_1880} \\
\gll  {pirta-li}\\
stick-\stockcaps{ERG}/\stockcaps{INST}\\
\z


\subsection{Inalienably possessed NPs}
\label{sec:key:8.7.2}\label{bkm:Ref456086200}

In most PN languages inalienable possession is distinguished from alienable possession. Inalienable possessive constructions, or “part whole” constructions \citep[59]{dixon_preface_2002}, are often marked through juxtaposition, as in the following Pitta-Pitta example (Example \ref{bkm:Ref329628202}) in which the “possessor” \textbf{mochoomba} `kangaroo' is morphologically unmarked:

\ea\label{bkm:Ref329628202}
\gll Mochoomba wapa\\
{a kangaroo’s} pup \\
\glt \citep[8]{Roth1897} \\
\gll Matyumpa warrpa\\
Kangaroo-[\stockcaps{NOM}]    pup-[\stockcaps{NOM}]\\ 
\z

\citet{koch_untitled_1868} is the earliest PN grammarian to describe the absence of morphological marking of an inalienable possessed NP. Koch notes the differences in the marking of inalienably and alienably possessed NPs twice, once in a description of nouns (\figref{fig:key:8-184}) and again in a description of pronouns (\figref{fig:key:8-185}).


\begin{figure}
\includegraphics[width=.8\textwidth]{figures/stockigt-img133.png}
\caption{Koch’s description of the unmarked inalienably possessed NP, where the possessor is a noun \citeyearpar[no pag.]{koch_untitled_1868}}
\label{bkm:Ref449373839}\label{fig:key:8-184}
\end{figure}

\begin{figure}
\includegraphics[width=.8\textwidth]{figures/stockigt-img134.png}
\caption{The unmarked inalienably possessed NP where the possessor is pronominal \citep[no pag.]{koch_untitled_1868}}
\label{bkm:Ref449373862}\label{fig:key:8-185}
\end{figure}

In the discussion of nouns, Koch (\citeyear{koch_untitled_1868}: no pag.) wrote: “Regarding the use of the genitive, also note that it is only used when the relevant object is not part of the subject itself” (\figref{fig:key:8-184}).\footnote{``Über den Gebrauch des Genetives merke noch, daß er nur dann angewandt wird, wenn das betreffende Object kein Theil des Gegenstandes selbst ist.''} In order to exemplify the difference, Koch supplied examples (Examples \ref{bkm:Ref329628748} \& \ref{bkm:Ref329628762}) to contrast the lack of marking on the “possessor” in an inalienably possessed NP with the suffixation of \textit{-ya}, written by the missionaries as –\textbf{ia}, to the possessor in an alienably possessed NP.

\ea\label{bkm:Ref329628748}
{Teraia         kalti{} {\textit{nicht}}   teri kalti,} \\
\glt {(\citealt{koch_untitled_1868}: no pag.)}\\
\gll    \textit{tari-ya}        \textit{kalthi} not  \textit{tari}   \textit{kalthi}\\
boy-\stockcaps{POSS}  spear       \textit{not}    boy-[\stockcaps{NOM}] spear\\
\z

\ea   \label{bkm:Ref329628762}
{kintella     milki   {\textit{nicht}}{} Kintellaia milki} \\
\glt {(\citealt{koch_untitled_1868}: no pag.)}\\
\gll kinthala      milki not  \textit{kinthala-ya}   \textit{milki}\\
dog-[\stockcaps{NOM}]   eye   \textit{not} dog-\stockcaps{POSS}   eye\\
\z

The juxtaposition of the constituents of the inalienably possessed NP was also shown by \citet[3]{schoknecht_grammar_1947} and \citet[12]{flierl_dieri_1880} using an example which better clarified the difference:

\newpage
\ea          mataraia kalti, {\textit{der mannes speer,}} {\textit{aber nicht}} mataraia milki, {\textit{der mannes augen, sondern}}{} matari milki\\
\citep[12]{flierl_dieri_1880}\\
 \textit{mathari-ya kalthi} the man’s spear, but not      \textit{mathari-ya milki} the man’s eye,      rather       \textit{mathari  milki} \\
man-POSS spear, `the man’s spear'; but not:    man-POSS eye, `the man’s eye', rather     man-[NOM] eye
\z

Koch (\citeyear{koch_untitled_1868}: no pag.) also exemplified the construction in instances in which the possessor is pronominal:

\begin{quote}
Note. 1. On occasion, the nom. of the personal pronoun will also be used where we in German would use the nom. of the possessive pronoun. This occurs where one talks about parts of the body. For example, \textit{mara nani} my hand, not \textit{mara nakani}. (\citealt[no pag.]{koch_untitled_1868}; \figref{fig:key:8-185})\footnote{``Der Nom. Der Personalfürwörter wird auch zuweilen gesagt, wo wir im Deutschen den Nom. des Possessivpronomens setzen. Dies geschieht wenn von Körpertheilen die Rede ist. \textit{z.B.} \textbf{mara nani} \textit{mein Hand nicht} \textbf{mara nakani}.''}
\end{quote}



Strangely, description of the construction is altogether absent from Reuther’s grammars (\citeyear{reuther_dieri_1894}; \citeyear{reuther_reuther_1899}; \citeyear{reuther_ms_1901a}; \citeyear{reuther_ms_1901b}).

Despite the occurrence of this astute observation in the earlier Diyari grammars, the earliest printed translations of liturgical texts into Diyari (\citealt{koch_nujanujarajinkiniexa_1870}) employed a possessive suffix on the “possessor” of body parts (see \citealt{kneebone_teichelmann_2005}: 142, 159). That Reuther did not describe this construction in his grammar, when earlier missionaries had, may suggest that the target language of his description was the variety of Diyari used within the mission domain, rather than fluent native speaker usage.

The earlier missionaries at Bethesda made the first description of the difference between these two constructions when analysing Diyari, and also explained the distinction more accurately than middle-era overviews of Australian structure. That inalienable possessive constructions are marked through juxtaposition was still unrecognised as a feature common to Australian languages in the 1930s. The structure was not mentioned by \citet{elkin_nature_1937}, \citet{capell_structure_1937}, or by \citet{ray_aboriginal_1925}, and was not described by T. G. H. Strehlow (\citeyear{strehlow_aranda_1944}[1938]) in his grammar of Arrernte.

The use of juxtaposition to convey other syntactical relations in Australian languages was, however, being observed at the time. Juxtaposition of phrases was discussed in the comparison of adjective (\citealt{capell_structure_1937}: 55; \citet{elkin_nature_1937}: 41) as was the juxtaposition of clauses, rather than processes of subordination \citep[40]{elkin_nature_1937}.

\subsection{The syntax of complex clauses}
\label{sec:key:8.7.3}\label{bkm:Ref73520500}
Despite there being a good record of complex sentences in the early Diyari sour\-ces, relative to other works considered in this study (e.g., \citealt{ridley_kamilaroi_1875}~[1855a] and \citealt{schurmann_letter_1844}), the subordinating processes that are exemplified in the early Diyari grammars constitute less than half of the seven possible processes recorded in the modern era (\citealt{austin_grammar_2013}: 86[1981a]). After thirty years of immersion in Diyari, the missionaries recorded only three of the seven verbal inflections marking clausal dependency: apprehensional constructions marked with \textit{{}-yathi} (\sectref{bkm:Ref339789035}), DS imperfective clauses (marking relative clauses), and DS perfective clauses (marking sequential clauses) marked with \textit{-rnanhi} and \textit{-ni} respectively (\sectref{bkm:Ref74048540}; \sectref{bkm:Ref339715229}).

\subsubsection{The \textit{denunciativ} mood in Diyari}
\label{sec:key:8.7.3.1}\label{bkm:Ref339789035}

Apprehensional constructions (\sectref{sec:key:5.6.1}) in Diyari (termed “lest” by \citealt{austin_grammar_2013} [1981a]) are marked by the morpheme \textit{-yathi}, which is the only dependent clause verbal inflection in Diyari which does not also mark switch reference, i.e., it does not also indicate whether the subject of the dependent clause is either the same or different subject as the main clause (\citealt{austin_grammar_2013}: 229~[1981a]).

Following Teichelmann \& Schürmann, the apprehensional construction was established as a mood of the verb in the earliest grammar of Diyari by \citet[no pag.]{koch_untitled_1868}, who named it the \textit{denunciativ} (denunciative) and described the mood as “Modus der Drohung oder Ankündigung” (mood of threat or notification). The exemplification, which is given by each later missionary-grammarian, is missing from the extant copy of Koch’s original analysis. The section occurs within the pages of the MS that are left blank (\sectref{bkm:Ref326948006}). Schoknecht’s description also discusses the “denunciative” in terms of a “threat'':

\begin{quote}
This is a mood that is not employed in any language, either ancient or modern of which we have any knowledge. It expresses an announcement [notification] or a threat, and always states, in the first part of the sentence, the means by which the threatened consequence may be averted. \citep[10]{schoknecht_grammar_1947}
\end{quote}

Schoknecht’s description and exemplification of apprehensional constructions (Examples \ref{bkm:Ref339715937}\& \ref{bkm:Ref339715961}) can be assumed to have been taken from a non-extant section of Koch’s analysis (1868):

\newpage
\ea\label{bkm:Ref339715937}
Ninkidani wapamai, nato nandraiati!\\
\glt `Come here, otherwise I strike!' \\
(\citealt{schoknecht_grammar_1947}: 10) \\
\gll nhingki-rda-nhi wapa-mayi, ngathu nandra-yathi \\
here-vicin-\stockcaps{ALL} go.\stockcaps{IMP-EMPH}, 1sg\stockcaps{ERG} hit-\stockcaps{AVERS}\\
\z

Note that in example (Example \ref{bkm:Ref339715961}) the name “\textit{Jesu}” has not been not grammatically integrated. Diyari masculine names are normally marked with \textit{-nha} in accusative case (\tabref{fig:key:8-174}).

\ea
\label{bkm:Ref339715961}Jesu antjanimai, jura paliati!\\
\glt `Love Jesus, or you will die!' \\
(\citealt{schoknecht_grammar_1947}: 10) \\
\gll Yesu ngantya-ni-mayi, yura pali-yathi \\
Jesus-[\stockcaps{ACC}] love.\stockcaps{IMP}-\stockcaps{NM}-\stockcaps{EMPH} 2pl\stockcaps{NOM} die-\stockcaps{AVERS}\\
\glt `love Jesus, lest you die!'
\z


\subsubsection{The conditional mood of the verb}
\label{sec:key:8.7.3.2}\label{bkm:Ref74048540}

The Diyari grammarians presented forms marked with inflection \textbf{-nani} \textit{-rnanhi} in a discussion of the mood of the verb termed \textit{Modus Conditionalis}. The form \textbf{-nan(n)i} \textit{-rnanhi} is now analysed (\citealt{austin_grammar_2013}: 226--229~[1981a]) as a subordinate marker carrying both aspectual and switch reference functions, i.e., marking a dependent clause as imperfective and as having a different subject to that of the main clause (Example \ref{bkm:Ref339719815}).

\ea\label{bkm:Ref339719815}
\gll pantha-ma-mayi kilthi ngakarni ngathu wayi-rna wara-rnanhi \\
 smell-\textsc{tr}.\stockcaps{IMP}-\stockcaps{EMPH} stew-[\stockcaps{ACC}] 1\textsc{sg}.\stockcaps{POSS} 1\textsc{sg}.\stockcaps{ERG} cook-\textsc{ptcp} \textsc{aux}-\stockcaps{IMPERF}.\stockcaps{DS} \\
\glt `smell my stew, that I cooked'
(\citealt{austin_grammar_2013}: 214~[1981a])
\z

Under the heading \textit{Modus Conditionalis}, Schoknecht (\citeyear{schoknecht_grammar_1947}: 12), \citet{flierl_dieri_1880}, and \citet{reuther_dieri_1894} discussed the semantic function of the form in marking the verb as conditional without attributing a syntactic dependency marking function to the morpheme. They each provided an example (Example \ref{bkm:Ref339728750}) in which there are two clauses with \textbf{-nani} \textit{-rnanhi} marking the verb as conditional in each. There appears to be no clausal dependency.

\newpage
\ea\label{bkm:Ref339728750}
judla ~   ~taji-nani,             ~judla ~              pali-nani\\
\glt    `If you eat, you will die' \\
(\citealt{schoknecht_grammar_1947}: 12[1872]) \\
\gll yula           thayi-rnanhi,   yula                 pali-rnanhi\\
2\textsc{dl}.\stockcaps{NOM} eat-\stockcaps{COND},     2\textsc{dl}.\stockcaps{NOM}          die-\stockcaps{COND}\\
\z


Note that the dual form of the second-person pronoun – given in the missionaries' orthography with a pre-stopped lateral (see \citet{austin_grammar_2013}: 27~[1981a]) – falls out only in Reuther’s translation: “If you two eat, you will die” (\citeyear{reuther_dieri_1894}: 37; \citeyear{reuther_three_1981}: 12~[1899]).

Such constructions do not, however, occur in the variety recorded by Austin (pers. comm. 13/08/2010), in which the verb in the second clause would be marked for tense (see \citealt{austin_grammar_2013}: 94~[1981a]; Example \ref{bkm:Ref340429075}), as in the example shown in \figref{fig:key:8-186}.

\ea
\gll \label{bkm:Ref340429075}pali-lha    ngana-yi\\
die-\stockcaps{FUT}  \textsc{aux}-\stockcaps{PRES}\\
\glt `will die' \z

\begin{figure}
\includegraphics[width=.8\textwidth]{figures/stockigt-img135.png}
\caption{\citet{austin_grammar_2013}: 221~[1981a]}
\label{bkm:Ref340429028}\label{fig:key:8-186}
\end{figure}

There is an intriguing structural similarity between this abridged translation of Genesis 2:17 and 3:3 given by the Diyari grammarians to illustrate the conditional mood of the verb, and the construction given by Teichelmann \& Schürmann (\citeyear[V 67]{teichelmann_outlines_1840}; see \ref{ex:5:29} on p. \pageref{ex:5:29}).

Although Teichelmann \& Schürmann’s construction is negative and the Diyari construction is affirmative, the symmetry between the constructions is unlikely to be coincidental. While Teichelmann and Schürmann gave solid evidence that the Kaurna inflection \textit{-ma} did not have dependency-marking function (\sectref{sec:key:5.6.2}), a parallel finite function of the Diyari inflection –\textit{rnanhi} is not substantiated by Austin and was not demonstrated by the missionaries.

The Diyari morpheme \textit{-rnanhi} was used frequently in the missionaries' translations to mark the verb as hypothetical, but not as dependent. This does not accord with the structures recorded by Austin. Examine the following translation of John 11:21:

\ea   Kaparajai, ~ ~ jidni ~ ~      ninkida ~ ~     ngana-nani,       neji ~ ~      ngakani ~  ~ wata ~ ~    pali-nani\\
\glt `Lord if thou had been here, my brother would not have died' \\
(\citealt{reuther_testamenta_1897}) \\
\gll Kaparra-yayi      yini           nhingkirda  ngana{}-rnanhi,   nhiyi      ngakarni       wata       pali-rnanhi\\
boss-\stockcaps{EMPH}  2\textsc{sg}.\stockcaps{NOM}   here     be-\stockcaps{COND}       brother   1\textsc{sg}.\stockcaps{DAT}    \stockcaps{NEG}     die-\stockcaps{COND}\\
\z


Although the English translation shows a subordinate clause as the first component of this counterfactual construction, both Diyari clauses are marked with \textit{-rnanhi}, which in the variety Austin recorded is a different subject subordinator.

The translation into Arrernte of this same counterfactual construction, from John 11: 21, was later discussed by \citet[23]{kempe_grammar_1891} and by T. G. H. Strehlow (\citeyear{strehlow_aranda_1944}: 108[1938]), who in 1938 stated that the passage was “impossible to translate…into good Aranda”.\footnote{\citet[108]{strehlow_aranda_1944} later qualified this statement writing: “[I]t is possible to translate this sentence literally into Aranda”.}

It seems probable that the use of \textit{–rnanhi} in this clause and in “If you eat, you will die” (Example \ref{bkm:Ref339728750}) was a syntactic feature of “mission Diyari” and did not reflect native speaker usage. This structure is likely to have been developed by the missionaries, who had poor control of the language, as a translational solution for conditional and counterfactual constructions required for developing Diyari liturgical material. The Diyari grammarians appear to have calqued this construction from Teichelmann \& Schürmann’s exemplification of the Kaurna suffix \textit{-ma} (Example \ref{sec:key:5.6.2}) and showed two juxtaposed finite “conditional” clauses, which they translated as a conditional complex clause construction.

\subsubsection{“Relative pronouns''}
\label{sec:key:8.7.3.3}\label{bkm:Ref339715229}\label{bkm:Ref339742580}

Like their Lutheran predecessors, the Lutheran Diyari grammarians illustrated the syntax of complex clauses not only in a discussion of the conditional mood, \textit{Modus Conditionalis}, but also under the heading relative pronouns, \textit{Relativa Pro\-nom\-ina}.

\newpage
Following Koch (\citeyear{koch_untitled_1868}: no pag.), each subsequent missionary-grammarian of Diyari treated what are three discrete categories of pronouns in the traditional framework – reflexive, reciprocal and relative – under a single subheading, \textit{Reflexive, Reciproka, und Relativa Pronomina}. Here it is explained that “the forms do not appear in the Diari language” (\citealt{schoknecht_grammar_1947}: 7). These three categories, which in the traditional framework are treated separately, were collapsed into a single unit in the early Diyari grammars, presumably because the function of each of these SAE pronominal categories is not carried pronominally. In keeping with the tendency of the corpus grammarians to identify as structures in PN features that they perceived to be functionally equivalent to the SAE structure traditionally described under that heading in the received grammatical framework (\sectref{sec:key:2.3.3}), the missionary-grammarians explained how the function of SAE reciprocal, reflexive and relative pronouns were conveyed in Diyari, under this single heading. They provided one example clause for each. Koch and following him Schoknecht and Flierl all stated:


\begin{quote}
…in order to express these pronouns, one will need the assistance of the reflexive and reciprocal verbs for our reflexive and reciprocal pronouns and the \textit{present participle for our relative pronouns}. (\citealt{koch_untitled_1868}: no pag.; emphasis added)\footnote{“…um diese pronomina auszudrücken, muß man die Verba reflecciva und reciproka für unsere reflecciven u. reciproken pronomina, und das präsen partizipi für unsere relativen pronomina zu Hülfe nehmen.''}
\end{quote}

Reflexive and reciprocal constructions are illustrated as marked within the morphology of the verb, by \textit{-tharri-} (\citealt{austin_grammar_2013}: 81~[1981a]) and \textit{{}-mali-} (\citealt{austin_grammar_2013}: 80~[1981a]) respectively.

The missionaries then presented two types of “present participles” that they believed “expressed the relative pronoun''. The forms \textbf{-ni} and \textbf{-nan(n)i} are currently analysed (\citealt{austin_grammar_2013}: 194--209, 226--229~[1981a]) as marking the different subject of perfective (sequential; Example \ref{bkm:Ref339729473}) and imperfective dependent (Example \ref{bkm:Ref340480092}) clauses respectively. They mark the verb to which they attach as dependent and indicate that the subject of the dependent clause is differernt from the subject of the main clause. While there is no suggestion that any of the early missionary-grammarians appreciated a difference between a perfective (sequential) and an imperfective dependent clause, it is interesting that they recorded the forms that mark each.

\ea\label{bkm:Ref339729473}
\gll nganhi        wakara-rna     wara-yi,     yundru     matya    nganthi            wayi-rna      wara-ni\\
1\textsc{sg}.\stockcaps{NOM}   come-\textsc{ptcp}     \textsc{aux}-\textsc{pres}   2\textsc{sg}.\stockcaps{ERG}   already   meat-[\stockcaps{ACC}]  cook-\textsc{ptcp} \textsc{aux}-\textsc{seq}.\stockcaps{DS}\\
\glt  “I came after you had already cooked the meat” \\
(\citealt{austin_grammar_2013}: 227~[1981a])
\z

\ea
\label{bkm:Ref340480092}
\gll pantha-ma-mayi          kilthi               ngakarni     ngathu     wayi-rna    wara-rnanhi\\
smell-\textsc{tr}.\stockcaps{IMP}-\stockcaps{EMPH}   stew-[\stockcaps{ACC}]  1\textsc{sg}.\stockcaps{POSS}    1\textsc{sg}.\stockcaps{ERG}    cook-\textsc{ptcp}    \textsc{aux}-\stockcaps{IMPERF}.\stockcaps{DS}\\
\glt `smell my stew, that I cooked' \\
(\citealt{austin_grammar_2013}: 214~[1981a])
\z

A pair of complex constructions (Examples \ref{bkm:Ref339729609} \& \ref{bkm:Ref339729621}) were given by Koch and Schoknecht to show how their “present participle” of the verb carried the function of the European relative pronouns. Each complex structure consists of a declarative clause followed by a dependent clause with a subject that is different from the subject of the main clause.

% Double check

\ea
   \label{bkm:Ref339729609}Ninnaia     anxale     wappaia,     nanna   anxanalli\\
    Den liebe ich, welcher mich liebt,\\

\citep{koch_untitled_1868}

 \gll nhinha-ya              ngantya-lha\footnotemark{} wapa-ya,        nganha       ngantya-?\\
    3\textsc{sg}.\stockcaps{ACC}-near        love-\textsc{ptcp}/\stockcaps{FUT}  \textsc{aux}-\stockcaps{PAST}      1\textsc{sg}.\stockcaps{ACC}        love-?\\
\glt    `I love the one who loves me'
\footnotetext{Austin (\citeyear{austin_grammar_2013}: 94[1981a]) notes that the missionaries' record of the participle \textit{-lha} marking the main verb followed by the past auxiliary \textit{wapa-ya} differs from his own data. The form of the verb in these clauses would in the variety Austin described be \textit{ngantya-rna} and \textbf{kaLakaLari}\textit{{}-rna.} The verbal suffix \textit{-lha} marks future tense according to Austin.}
\z

\newpage
\ea
\gll    nunkangu   nani       kalakalaila\footnotemark{} wappaia,          nakangu       kalakalai      nanni\label{bkm:Ref339729621}\footnotemark{}\\
      und den hasse ich welcher mich haßt.\\
\addtocounter{footnote}{-1}
\footnotetext{This verb meaning `to hate' is shown here as having a S, LOC argument frame. It is an intransitive verb which takes a locative complement. The verb, which is listed in Schoknecht’s vocabulary (1847: 23), is not given in Austin (2013~[1981a]). See \citet[130]{austin_grammar_2013} for discussion of a small class of intransitive verbs with complements standing in locative case, most of which convey cognitive states, for instance \textit{tyampa-} `to be fond of'. Austin (2013: 133~[1981a]) describes the locative case as indicating “the non-controlled cause of a more or less temporary physiological or mental state”.}
\stepcounter{footnote}
\footnotetext{It is possible that the missionaries were here attempting to reproduce a construction in which the abstract noun “love” is marked for ergative case and occurs with the copular verb \textit{ngana-} “to be” (see \citealt{austin_grammar_2013}: 124--125~[1981a]). There are also problems with this analysis. The form \textbf{nanna-} given in the example clause is, for instance, not inflected for tense.}

\citep{koch_untitled_1868}

\gll nhungkangu   nganhi \textbf{kaLakaLari}{}-lha    wapa-ya,          ngakangu    kaLakaLari-rnanhi\\
 3\textsc{sg}.\textsc{f}.\stockcaps{LOC}        1\textsc{sg}.\stockcaps{NOM}      hate-\textsc{ptcp}/\stockcaps{FUT}   \textsc{aux}-\stockcaps{PAST},    1\textsc{sg}.\stockcaps{LOC}      hate-\stockcaps{IMPERF}.\stockcaps{DS}\\
\glt `I hated the one who hates me'
\z

% ---

The second complex construction (Example \ref{bkm:Ref339729621}), ``I hated the one who hates me'', appears to show a declarative clause, ``I hate him'', which is marked for tense, followed by dependent clause, ``hates me'', in which the verb is inflected with the morpheme \textit{{}-rnanhi} marking its subject as different from the subject of the main clause, and as having imperfective aspect. It is not clear what inflection the missionaries showed attached to the dependent verb in the first matrix clause (Example \ref{bkm:Ref339729609}) ``I love the one who loves me''. Austin (pers. comm. 02/09/2016) suggests that the verb form is incorrectly recorded, since the imperfective different subject marker -\textit{rnanhi} is the only inflection that could occur here.

\citet[26]{flierl_dieri_1880} gave a \textit{different} construction with the \textit{same} German translation, suggesting that he had identified problems with the example provided by the earlier missionary-grammarians. In Flierl’s example, the dependent verb in the first matrix clause is marked with \textit{{}-ni,} marking the subject of the dependent perfective sequential clause as different from the main clause (\citealt{austin_grammar_2013}: [1981a]). While the morphology attached to the second dependent verb remains slightly unclear, again, the form is likely to be –\textit{rnanhi.}

\newpage
\ea
\label{bkm:Ref516470384}
Ninaia                   ngato         antjai,            ngana        antjani,\\
Den liebe ich, welcher mich liebt,\\
\citep[26]{flierl_dieri_1880}\\
\gll nhinha-ya            ngathu       nganta-yi,     nganha     nganta-ni\\
3\textsc{sg}.\stockcaps{ACC}-near      1\textsc{sg}.\stockcaps{ERG}     love-\stockcaps{PRES},  1\textsc{sg}.\stockcaps{ACC}    love-\textsc{seq}.\stockcaps{DS}\\
\glt `I love the one who loves me' \\\medskip
{nunkangu}      {ngani}            {kalakalariai,}           {ngakangu}     {kalakala}  {rina}\\
und den hasse ich welcher mich haßt.\\
\citep[26]{flierl_dieri_1880}\\
\gll nhungkangu   nganhi         kaLakaLari-yi,          ngakangu      kaLakaLa\\
    3\textsc{sg}.\textsc{f}\stockcaps{LOC}        1\textsc{sg}.\stockcaps{NOM}       hate-\stockcaps{PRES},              1\textsc{sg}.\stockcaps{LOC}        hate-?\\
\glt `I hate the one who hates me'
\z

While there are problems with the missionaries' command of the language, these clauses given under the heading “relative pronoun” show that after only two years' encounter with Diyari, the missionary-grammarians were aware that the construction of complex clauses only involved verb inflection, rather than the use of relative pronouns.

\subsubsection{Reuther’s description of relative pronouns in Diyari \citeyearpar{reuther_dieri_1894}}
\label{sec:key:8.7.3.4}\label{bkm:Ref74838296}\label{bkm:Ref339792843}\label{bkm:Ref339785105}

Remarkably, under the heading “relative pronouns”, Reuther (\citeyear[29]{reuther_dieri_1894}; \citeyear[17--18]{reuther_repr_1899}), in the last missionary grammar of Diyari, reported and exemplified an utterly different process of clause subordination than had any of the earlier three grammarians. Reuther did not suggest that clause subordination occurred through the “participle of the verb” or any other verb morphology. Rather than explaining how clause dependency was marked within the verb, \citet[29]{reuther_dieri_1894} explained, “Personal pronouns are used in lieu of relative pronouns”, and supplied the following construction with a translation in German that differed slightly from Koch’s (Example \ref{bkm:Ref339729621}) and Flierl’s (Example \ref{bkm:Ref516470384}):

\ea        Ngato     ninaia                  ngantjai,      nulia                   ngakangu   ngumu          nganka-na  warai\\
\glt `Ich liebe den, der mir gut gewesen ist' \\
\citep[29]{reuther_dieri_1894}\\
\gll ngathu    nhinha-ya           nganta-yi,   nhulu-ya              ngakangu  ngumu         ngank-nha  wara-yi\\
1sg\stockcaps{ERG}      3sg\stockcaps{ACC}-near           love-\stockcaps{PRES},      3sgnf\stockcaps{ERG}-near         1sg\stockcaps{LOC}         good                make-\stockcaps{PART} \stockcaps{AUX}-\stockcaps{PRES}\\
\glt `I love him who has been good to me'
\z

The simplest interpretation of this Diyari example, in which verbs in both clauses are marked for tense, is that it is composed of two separate clauses, “I love him'', “He has been good to me'', or as juxtaposed co-ordinated clauses, “I love him and he has been good to me'', where co-referentiality between the pronominal object of the first clause and the subject of the second clause is signalled by suffixing both the pronouns with the same relative distance suffix \textit{-ya} glossed here as `near' (see \citealt{austin_grammar_2013}~[1981a]).

Reuther, however, believed that the second clause `who has been good to me' was marked as subordinate to `I love him' by the third-person personal pronoun \textit{nhulu-ya,} standing in the position of the SAE relative pronoun, which acted “in lieu of relative pronouns” to relativise the construction. His analysis resembles that given by \citegen[33]{meyer_vocabulary_1843} analysis of Ramindjeri, in which that the third-person pronoun was said to sometimes “perform the office of a relative” (\sectref{bkm:Ref73381962}).

It is curious that Reuther’s description of clause subordination is so different from preceding analyses of Diyari, when the rest of his grammar is reasonably similar to the previous works. This new process of clause subordination that Reuther presented is currently not described in Diyari (\citealt{austin_grammar_2013}~[1981a]), while the earlier hypothesis that verbal morphology marked clause dependency \textit{is} described in Diyari, and is a common feature of PN languages.

Just as the absence in Reuther’s grammar of an account of the unmarked inalienably possessed NP (\sectref{bkm:Ref456086200}) suggests that his target language was the language used within the mission domain, rather than fluent native speaker usage, the appearance of this process of clause subordination in Reuther’s grammar indicates that he recorded a standardised variety used by missionaries, and perhaps also by Diyari Christians, for religious and mission purposes.

It is noteworthy that Reuther proposed this analysis in his initial grammar \citeyearpar{reuther_dieri_1894}, three years after HMS-trained Kempe published a substantial grammar of Arrernte (\sectref{bkm:Ref74837211}). Reuther’s view that personal pronouns acted to relativise clauses in Diyari, which diverged from the analyses given in previous missionary descriptions of the language, is likely to have been influenced by a similar analysis of clause dependency in Arrernte presented by Kempe (\sectref{bkm:Ref74837096}; Example \ref{bkm:Ref339739442}, p. \pageref{bkm:Ref339739442}). C. Strehlow’s 1894 transfer from Bethesda to Hermannsburg (\sectref{bkm:Ref456881682}) may have triggered the cross-fertilisation of linguistic analyses between the two missions.

Understanding this alteration to the description of marking clausal dependency in missionaries' records of Diyari requires consideration of the influence of the description of such processes in other works within the South Australian Lutheran School.

\subsection{Concluding remarks}
\label{sec:key:8.7.4}

The missionary grammars produced at the Bethesda mission, written by men trained at different Lutheran seminaries, and over three decades, present remarkably homogenous analyses of Diyari. They bear little resemblance to Threlkeld’s inaugural description of an Australian language (\citeyear{threlkeld_australian_1834}). In addition to traits discussed in this chapter, the Diyari missionary-grammarians' work differs from Threkeld’s grammar in not using hyphens to mark syllable boundaries or meaningful sub-word units, and in not providing interlinear-style glosses. While Threlkeld may have “set a standard for other missionary work in the field which followed soon after” \citep[269]{carey_lancelot_2004}, the Diyari grammarians' analysis of a PN language made in the second half of the nineteenth century is clearly influenced by a later school of descriptive practice.

The prominent influence of their Lutheran predecessors' grammars of South Australian languages, made within the Adelaide School \citep[410]{simpson_notes_1992}, most particularly \citegen{teichelmann_outlines_1840} grammar of Kaurna, is shown in the presentation of small Latinate case paradigms (\sectref{bkm:Ref326948838}), the explanation of ergative function in a discussion of verbs (\sectref{bkm:Ref456873578}), the placement of ergative case forms at the bottom case paradigms in the position of the Latin ablative case (\sectref{bkm:Ref326949499}), the declension of possessive pronouns (\sectref{sec:key:8.2}), the division of “postpositions” into two classes (\sectref{bkm:Ref456339501}), and descriptions of the syntax of complex clauses (\sectref{bkm:Ref74048540}). These similarities show that the earliest HMS-trained missionaries at Bethesda were equipped with copies of the published grammars made by the previous generation of Lutheran missionaries from the south of the state, a fact that is not surprising given the continuity of personal involvement in the 1840s and 1860s South Australian Lutheran missions.

% A remarkably
An astute summary of unmarked inalienably possessed noun-phrases (\sectref{bkm:Ref456086200}) was given by W. Koch, who died at the mission sixteen months after arriving, in a grammar (\citeyear{koch_untitled_1868}) that is relatively short and succinct. Koch’s discussion of the role of verb participles in clause subordination (\sectref{bkm:Ref74048540}) is equally perceptive. Koch’s analysis was reiterated by later missionary-grammarians with minor and historiographically important alterations.

Comparison of the sources shows that Reuther’s grammars (\citeyear{reuther_dieri_1894}; \citeyear{reuther_ms_1899}) show substantial improvement to the missionaries' record of the idiosyncratic and unpredictable marking of case on nominals of different numbers.

That it took the Lutherans almost thirty years to come closer to describing the complexity of the system of case marking on different nominal types is of consequence to the certainty upon which some other early analyses can be relied. That after thirty years of intense engagement the different marking of cases on female personal names appears to have remained undocumented at Bethesda should alert the linguist engaged in the reclamation of other languages from older sources written with much less exposure to the structure of a language to the type of material that is likely to have never been recorded.

On a related point, it would not have been possible to reconstruct the case system of Diyari based on the early sources alone. Case inflections marking distinct functions with the forms \textit{-ni}, \textit{{}-rni}, -\textit{nhi} were orthographically undifferentiated by the missionaries and consequently the suffix \textit{-ni} was shown to mark nominative, dative, allative and locative cases in the early grammars of Diyari (\tabref{fig:key:8-175}). Analysis of the Diyari case system is dependent upon the recorder hearing nasals at alveolar, retroflex and interdental positions and developing an orthography that consistently distinguishes these nasal phonemes. It is important to note this situation when considering the limitations to the reconstruction of a language that was lost before modern analysis.

While the last missionary grammar of Diyari written by Reuther (\citeyear{reuther_dieri_1894}; \citeyear{reuther_ms_1899}) records the sensitivity of case marking to number more accurately than did earlier grammars,  other aspects of his description, including the absence of reference to the unmarked inalienably possessed noun-phrase (\sectref{bkm:Ref456086200}), and his account of processes of clause subordination (\sectref{bkm:Ref339785105}) suggests that Reuther recorded a linguistic variety developed at the mission during decades of mission activity, rather than fluent native speaker usage. Further philological study is required to establish whether features of the language Reuther described also occur in liturgical translation.

The grammars of Diyari written by European philologists (\citealt{planert_australische_1908}; \citealt{gatti_lingua_1930}) who had never heard Diyari, and whose grammars were informed by the missionaries' written records, produce strikingly different depictions of the Diyari case system. The conception of syntactic case presented in Planert’s grammar (\citeyear{planert_australische_1908}), which followed his Arrernte grammar (\citeyear{planert_australische_1907}) produced a year earlier, presents a four-case analysis of split syntactic case systems (\citealt[132]{dixon_preface_2002}; \sectref{sec:key:2.6.1}).


