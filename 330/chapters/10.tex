\chapter{The description of Queensland languages}
\label{chap:key:10}

{This chapter examines} the early description of languages belonging to countries in Queensland. With exception of Ridley’s brief grammar of Turrbul (\citeyear{ridley_kamilaroi_1866}; \sectref{sec:key:4.5.2}), from country in what was then New South Wales, but not far over the border into Queensland established in 1859, Queensland languages were not described until the last decade of the nineteenth century. At this time, two distinct schools of descriptive practice were well established in Australia: that instigated by Threlkeld in the earliest grammar of an Australian language (\citeyear{threlkeld_australian_1834}; \chapref{chap:key:3}) and utilised in grammars of languages from New South wales, and that instigated by Teichelmann \& Schürmann (\citeyear{teichelmann_outlines_1840}; \chapref{chap:key:5}) and utilised by grammarians describing languages from South Australia. Yet the grammars of Queensland languuages were written using an independent template, innovated by the medical practitioner W. E. Roth. These works are examined section \sectref{bkm:Ref516423289} of this chapter.

The following section of this chapter (\sectref{bkm:Ref516423313}) provides an overview of the grammars by S. Ray and A. H. Haddon, which were published in Great Britain and were also written without recourse to earlier grammars of Australian Aboriginal languages.

\section{W. E. Roth}
\label{sec:key:10.1}\label{bkm:Ref74336027}\label{bkm:Ref516423289}\label{bkm:Ref516157254}\label{bkm:Ref514608531}\label{bkm:Ref514606393}
W. E. Roth (1861--1933) published two grammars of distantly related PN languages belonging to distant regions of Queensland. Roth’s first grammar was of Pitta-Pitta (\citeyear{roth_ethnological_1897}; \sectref{bkm:Ref516423738}), a Karnic language from the southwest of Queensland, and related to the languages described by Lutheran missionaries at Bethesda (Chapter 9). Roth collected the material while appointed as a medical officer at the Cloncurry and Boulia hospitals (1894--1897), a position that he described as having “afforded unrivalled opportunities for making enquiry into the language” (\citealt{roth_ethnological_1897}: v).

Material for Roth’s second grammar of Guugu-Yimidhirr (\citeyear{roth_structure_1901}; \sectref{bkm:Ref516423826}), from the east coast of Cape York Peninsula at the Lutheran Cape Bedford (Hopevale) mission, was collected after Roth was appointed as the first Protector of Aborigines for the Northern District of Queensland. Roth then revised and edited a grammar of Nggerrikwidhi (\citeyear{hey_elementary_1903}; \sectref{bkm:Ref516423848}) recorded at Mapoon mission on the western coast of Cape York Peninsula, written by the Moravian missionary N. Hey (1862–1951).

One of seven sons who were educated in France, Germany, and at the University College School in London, four of whom became doctors and three of whom pursued ethnology \citep{Reynolds2008}, Roth, a fellow of the Royal College of Surgeons, practiced both professions among Queensland Aboriginal people.

This body of work is of a different descriptive style and format to other early grammars of PN languages examined in this study, and together forms a discrete school of description. The grammatical component of \citegen{hey_elementary_1903} Nggerrikwidhi publication (twelve pages) is far less detailed than Roth’s previous grammars of Pitta-Pitta (thirty pages) and of Guugu-Yimidhirr (seventeen pages).

The inclusion in Roth’s Guugu-Yimidhirr grammar (\citeyear{roth_structure_1901}: 32--35) of a section headed “composition” is analogous to F. Müller’s \textit{Sprachproben} (\citeyear{muller_grundris_1882}) and Planert’s \textit{Texte} (\citeyear{planert_australische_1907}; \citeyear{planert_australische_1908}; \sectref{bkm:Ref514690025}). These arrangements sit within a descriptive tradition known as the “Boasian trilogy'', which is most recognised in the work of Sapir and Boas \citep[8--9]{darnell_indo-european_1999}, yet it is clearly a feature of nineteenth-century Australian description, particularly that made within a German tradition, to which Roth’s work might be seen to also belong. While \citegen[105--131]{threlkeld_australian_1834} “Illustrations”, \citegen[249]{gunther_native_1838} “sentences or phrases'', and \citegen{teichelmann_outlines_1840} “phraseology” are also part of a tripartite arrangement, the illustrative sections given in these earlier grammars contain much shorter samples of text.

\subsection{Roth’s grammar of Pitta-Pitta (1897)}
\label{sec:key:10.1.1}\label{bkm:Ref74727849}\label{bkm:Ref516423738}\label{bkm:Ref516157801}\label{bkm:Ref516157715}\label{bkm:Ref514607076}\label{bkm:Ref514606874}\label{bkm:Ref514606224}
Roth’s thirty-page grammar of Pitta-Pitta appeared as the first chapter of a much larger ethnographic work, \textit{Ethnological Studies among North-West Central Queensland Aborigines} (\citeyear{roth_ethnological_1897}), which had immediate impact within Australian anthropological circles and was reviewed at length and appreciatively by Spencer \citep[110--112]{mulvaney_oxford_2008}.

The grammar is exceptionally rich in exemplification. Roth supplied up to ten clauses as illustration of a single grammatical point followed by their translations into English, which were given without interlinear translation. It is among the most grammatically insightful, detailed and better-informed early PN grammars. Roth’s grammatical analysis of Pitta-Pitta has been assessed positively:

\begin{quote}
[Roth’s] analysis of the morphology is good … and only on a few points can the present authors dispute his conclusions. However, a number of grammatical forms which Roth did not find can now be described. On the other hand, some of Roth’s statements cannot be confirmed because of the lack of knowledge of present day informants. (\citealt{blake_pitta-pitta_1971}: 2--3)
\end{quote}

\citet[114]{mulvaney_oxford_2008} suggests Roth’s investigations may have been “unpremeditated before he realised his unprecedented observational opportunities as an itinerant outback doctor”. Roth is known to have read \citet{fison_kamilaroi_1880} \textit{after} the 1897 publication of Pitta-Pitta \citep[114]{mulvaney_oxford_2008}. Scrutiny of Roth’s Pitta-Pitta grammar (\citeyear{roth_ethnological_1897}) similarly suggests that he was largely unaware of all previous descriptions of Australian languages. The unique style of his grammatical description suggests that his grammar was written “cold” and was uninfluenced by earlier analyses. Roth makes no mention of previous publications. The work evinces distinctive attempts to convey the structure of the language that set it apart stylistically from earlier works in the corpus.

For example, the ordering of word-classes is unconventional (\figref{fig:key:201}). The corpus grammars are usually arranged according to approximately ten parts of speech and their sub-headings, which are given in roughly the same order (\figref{fig:key:2-12}). Roth’s presentation, by contrast, jumps from discussions concerning “pronouns” to “auxiliary verbs” to “articles” to “possessive pronouns” etc.


\begin{figure}
\includegraphics[width=\textwidth]{figures/stockigt-img148.png}
\caption{Table of contents in Roth’s grammar of Pitta-Pitta \citeyearpar[vi]{roth_ethnological_1897}}
\label{bkm:Ref323755944}
\label{fig:key:201}
\end{figure}

The logic informing the arrangement is underscored by the function of the described grammatical categories, rather than their form. Under the heading “auxiliary verbs'', given in the midst of discussions about nominal morphology, Roth in fact introduced the comitative suffix \textit{-maru} (“concomitant”; \citealt{blake_pitta_1979}: 199), which attaches to nouns but is translated into English using the auxiliary verb “to have''.

Like other corpus grammarians, Roth presented case morphology marking case functions that are not marked by the morphological case system of SAE languages under the word-class heading “preposition''. Roth’s sub-categorisation of types of prepositions into four categories differs from other early PN grammarians, and is a feature of the Queensland School. The categories are “motion'', “rest'', “purpose, reason and means” and “time”.

Like some previous PN grammarians (\citealt{meyer_vocabulary_1843}; \citealt{schurmann_letter_1844}; \citealt{livingstone_grammar_1892}), Roth opted not to provide case paradigms for nouns, but listed nouns in “nominative'', “possessive” and “objective” cases. This and his discussion of pronominal case forms as either “nominative'', “direct object” or “indirect object'' were used in later grammars of languages spoken in Queensland and establish a distinctive sub-school of early PN description.

Roth’s presentation of pronominal case resembles Ridley’s paradigms of case on nouns (\citeyear{ridley_kamilaroi_1866}: 5; \citeyear{ridley_kamilaroi_1875}: 6; \figref{fig:4:75}), marking an important, if tenuous, link between the description of Gamilaraay spoken in New South Wales and the description of Queensland languages. Both Ridley and Roth present multiple forms as cases termed “objective” or “indirect object” in order to account for cases outside the Latin inventory in the same way Ridley had presented Gamilaraay nouns.

Roth’s presentation of pronominal case in Pitta-Pitta was descriptively unique. Roth further sub-categorised his “objective–indirect object” pronouns into four classes, to which he assigned letters “a'', marked “motion towards'', “b” marked “rest with'', “c” marked “from whom something is obtained'', and “d” marked “for whose benefit, use or advantage, something is done” (\tabref{fig:key:202}). Roth observed that the suffixes that attach to the pronominal stem to mark pronouns in different cases (\citealt{blake_pitta-pitta_1971}: 76) were formally similar to the “prepositions” that marked the same function on nouns. He wrote: “traces of … preposition inflexions can be recognised in the … series of personal pronouns objective”.


\begin{table}
\small
\begin{tabularx}{\textwidth}{p{5.5cm}lQ}
\lsptoprule
 {\raggedright Subtypes of Roth’s “pronouns objective – indirect object''}
 &  {Form}
 & {  {Current label of form}} {{(\citealt{blake_pitta-pitta_1971}: 82--83;}}  {\citealt{blake_australian_1979}: 195)}\\
\midrule
a) “motion towards” & \textit{{}-inu}
 & allative\\
b) “rest with” & \textit{{}-ina}
 & locative\\
c) “from whom something is obtained” & {}-\textit{inja} * & ablative\\
d) “for whose benefit, use, or advantage, something is done” & \textit{{}-nga} * & dative, benefactive\\
\lspbottomrule
\end{tabularx}
\caption{Roth’s nomenclature of pronouns in peripheral cases \citeyearpar[4]{roth_ethnological_1897} (Pitta-Pitta)}
\label{bkm:Ref323829014}
\legendbox{ Roth showed type “c” and type “d” as further inflected with either \textit{-nha}  (non-future) or \textit{{}-ku} (future) marking a tense distinction (see \citealt{blake_pitta-pitta_1971}: 90, 98).}
\label{fig:key:202}
\end{table}

Each of Roth’s six pronominal case forms was shown in each person and number. \figref{fig:key:203} shows declension for type “a'', allative case, and type “b'', locative case. The only other corpus grammar to provide a pronominal paradigm of a pronoun marking a case function that is not marked morphologically in SAE languages was \citet[8]{teichelmann_outlines_1840} in a supplementary paradigm showing comitative case (\citealt{Stockigt2017}; \sectref{sec:key:5.3.2}). In this regard, Roth’s description of Pitta-Pitta is the most comprehensive in the corpus.


\begin{figure}
\includegraphics[width=\textwidth]{figures/stockigt-img149.png}
\caption{Roth’s paradigm of pronouns in allative and locative cases (\citeyear[4]{roth_ethnological_1897}; Pitta-Pitta)}
\label{bkm:Ref323761032}
\label{fig:key:203}
\end{figure}

Like Teichelmann \& Schürmann (\sectref{sec:key:5.3.3}), \citet[7]{roth_ethnological_1897} innovated a descriptive solution to account for double case marking. Roth’s description of “secondary possessives” formed from “personal pronouns possessive” (\figref{fig:key:204}) was, like other aspects of his description of Pitta-Pitta, innovated independently, and differs from all earlier corpus grammars.


\begin{figure}
\includegraphics[width=\textwidth]{figures/stockigt-img150.png}
\caption{Roth’s account of clausal case marking on a possessive NP (\citeyear[7]{roth_ethnological_1897}; Pitta-Pitta)}
\label{bkm:Ref77770821}
\label{fig:key:204}
\end{figure}

\subsection{Poland \& Schwarz’s grammar of Guugu-Yimidhirr \citeyearpar{schwarz_koko_1900}}
\label{sec:key:10.1.2}\label{bkm:Ref516423826}\label{bkm:Ref516160134}\label{bkm:Ref514608157}\label{bkm:Ref514608023}\label{bkm:Ref514607997}
The Cape Bedford mission (Elim/Hope Valley/Hopevale) was established by J. Flierl in 1886. Flierl was recruited by the Neuendettelsau Mission Society to establish a mission in New Guinea while he was working at the Bethesda mission in South Australia (\sectref{bkm:Ref514690075}). Flierl’s passage to New Guinea was delayed for a year, and while he was waylaid at Cooktown, he established a mission among speakers of Guugu-Yimidhirr, and many other languages. The mission was administered by the Evangelical Lutheran Immanuel Synod from South Australia and later missionaries were recruited from Neuendettelsau. Stability of mission staff was achieved by the arrival of Neuendettelsau graduates W. G. F. Poland (1866--1955) and G. H. Schwarz (1868--1959) in 1887 and 1888 respectively. Schwarz, who departed in 1944 is still remembered as \textit{Muni} `black’, the Guugu-Yimidhirr translation of his surname.

Upon Flierl’s departure for New Guinea, he was replaced by the lay missionary C. A. Meyer,\footnote{Note that this is \textit{not} H. A. E. Meyer (1813--1862; Chapter 6).} who also travelled from Bethesda to Cape Bedford in 1886 with Johannes Pingilina, a Diyari evangelist who had been introduced to Christianity at the Bethesda mission (\sectref{sec:key:1.1.3}).

When introducing his published grammar of Guugu-Yimidhirr, Roth acknowledged Neuendettelsau-trained missionaries Poland and Schwarz, as having assisted him with the grammar:

\begin{quote}
I purposely wish to give publicity to the assistance which has been invariably rendered [to] me by the Revs. G. H. Schwarz and W. Poland…Especially without the former’s help I should never have discovered the various compounds derived from their simpler roots, nor the meanings of many inflections assumed by words, nor the why and wherefore of many a point which at first seemed inexplicable to me. \citep[preface]{roth_structure_1901}
\end{quote}

Although he named Schwarz as his main informant, it is likely that Poland was responsible for a substantial part of the analysis. The earliest known grammatical MS of Guugu-Yimidhirr was apparently included in a letter written by Poland to Neuendettelsau close to two years after his arrival (\citealt{haviland_how_1980}: 133),\footnote{The letter is dated 16/08/1889 (\citealt{haviland_how_1980}: 133).} but the MS grammar, held at the Neuendettelsau archives, has thus far been unable to be sourced.

\subsection{Hey’s grammar of Nggerrikwidhi \citeyearpar{hey_elementary_1903}}
\label{sec:key:10.1.3}\label{bkm:Ref76120332}\label{bkm:Ref74336064}\label{bkm:Ref516423848}\label{bkm:Ref514607540}
The Victorian-based Presbyterian Foreign Mission Committee of Australia (henceforth PFMC) established the mission in far north Queensland in 1891. Originally called Batavia River Mission, but later Mapoon (1892), the mission was operated as a joint concern by the PFMC and the Moravian mission board in Saxony.

In response to a request from the Victorian mission board to Herrnhut, the Moravian seminary in Saxony, Hey was selected on account of his building skills and knowledge of stock and agriculture (Hey, quoted in \citealt{edwards_moravian_2007}: 378). Hey left school at the age of thirteen, the year after his father’s death, after which he worked on the family farm. Upon his mother’s death, he applied unsuccessfully to the Basel Mission Society, for which he was deemed too old \citep{ganter_killalpaninna_2016}. At the age of twenty-six, he applied to Moravian mission college at Niesky, where he received two years training before being called to Australia in 1891. Ganter writes:

\begin{quote}
As a Moravian, Hey is a somewhat unusual figure, because he did not come from a Moravian community, but rather asked for admission to the Unity of Brethren when he was already past his youth. Without the benefit of a fine Moravian school education, he was thrust into a position of responsibility when the leader of the mission died after a short period at Mapoon. \citep{ganter_killalpaninna_2016}
\end{quote}

The second missionary sent to Australia to establish the first Moravian mission in Queensland was Rev. J. G. Ward. Hey and Ward arrived at Cullen Point in 1891 “accompanied by four carpenters … a police officer and two native troopers” \citep[234]{edwards_moravian_2007}. They began their mission work in an environment of hostility from surrounding Europeans and suspicion from Aboriginal populations. The mission was established after the first wave of frontier violence had disrupted populations. Particular to the frontier history of the Cape York Peninsula is the exploitation of Aboriginal labour in the \textit{bêche-de-mer} industry. The Aboriginal populations among whom the Moravians established themselves at Mapoon had experienced a high degree of social disruption prior to their interaction with missionaries. This situation contrasts most starkly with the Hermannsburg missionaries' arrival on the Finke River (\sectref{bkm:Ref514690108}), where Aboriginal people’s first contact with Europeans was with the missionaries.

Initially the missionaries perceived a great variety of different languages spoken by diminishingly small bands of people around the mission. In {1892}, soon after the mission was established, Ward (quoted in \citealt{edwards_moravian_2007}: 267) wrote: “[T]he tribe is small and dying out, and … their dialect differs widely from that of their neighbouring tribes”.

Nevertheless, Ward argued that “it is essential to acquire some of the idiomatic phrases and words in order to present the essential truth of the gospel” (ibid.: 267). The missionaries made progress with the languages. Three years after the mission’s establishment Ward “had translated the Christmas story into the local dialect [that he] insisted on sharing in Christmas services” \citep[242]{edwards_moravian_2007} shortly before his death. After Ward’s death in 1895, the mission was temporarily abandoned while Hey regained strength, before becoming mission superintendent until his retirement in 1919. Hey is known to have been preaching in the vernacular by 1896 and to have translated hymns, though no record of these translations appears to have survived. It is unfortunate that Hey recalls nothing about the local languages, or about his interaction with Roth in his autobiography, printed in translation in \citet[367--383]{edwards_moravian_2007}.

\subsubsection{Difficulty in assessing Hey’s description of Nggerrikwidhi}
\label{sec:key:10.1.3.1}

At the time of Roth’s appointment as the Protector of Aborigines for the Northern District, Hey was appointed “Superintendent to the Blacks” by the Queensland government, which provided the two men professional contact.

Without a contemporary description of Nggerrikwidhi, it is difficult to assess Hey’s grammar (\sectref{sec:key:2.1}). In the introduction to his work Hey qualified the comprehensiveness of his grammar:

\begin{quote}
Although I have studied Nggerikudi for the last ten years, I must confess that there are still many points of the language which are quite inexplicable to me … I am only justified in publishing the Grammar on the grounds that the aboriginals are fast disappearing, and that the continent is still a \textit{terra incognita} to the philologist. \citep[2]{hey_elementary_1903}
\end{quote}

{Although Hey referred to the name of the language as Nggerikudi, \textit{Nggerri-kwidhi}, probably `sandbank-ASSOC' \citep[137]{breen_we_2008}, is now said to refer to the people who spoke a language called Yopo-timi, possibly \textit{Yupu-thimri}, 1sgNOM-COM (\citealt[137]{breen_we_2008}; \citealt{crowley_mpakwithi_1981}: 149). Within months of establishing Mapoon mission, Ward wrote that the language spoken by people at the mission differed “wildly from their neighbouring tribes” (quoted in \citealt{edwards_moravian_2007}: 267). It is not known whether the linguistic variety that was first encountered at the mission in 1891 was that recorded by Hey in 1903.}

There are two short modern grammars of languages from the region written in the modern descriptive era: Hale’s grammar of Linngithigh (\citeyear{hale_linngithigh_1966}) and Crowley’s grammar of Mpakwithi (\citeyear{crowley_mpakwithi_1981}). Available comparative data suggests that Hey’s Nggerrikwidhi should be viewed as a different language from both Mpakwithi and Linngithigh. \tabref{fig:key:205} shows that each of the described varieties from the region has a distinctive set of free-form first-person nominative/ergative pronouns.


\begin{table}
\begin{tabularx}{\textwidth}{lQQQ@{}}

\lsptoprule
%\hhline%%replace by cmidrule{~---}

 & { {Nggerrikwidhi}}

{ {\citep[12]{hey_elementary_1903}}}
 & { {Mpakwithi}}

\mbox{\citep[170]{crowley_mpakwithi_1981}}
 & { {Linngithigh}}

{ {\citep[181]{hale_linngithigh_1966}*}}\\
\midrule
{\textbf{1.sg.S/A}}
 & {\textbf{yube}}
 & {\textit{a}}\textit{ŋu}
 & \textit{ayoŋ}\\
{\textbf{1.dl.S/A.incl}}
 & {\textbf{liba}}
 & \textit{lægi}
 & {\textit{li}}\textit{ŋgay}\\
{\textbf{1.dl.S/A.excl}}
 & {\textbf{naba}}
 & {\textit{nini}}
 & {\textit{nan}}\\
{\textbf{1.pl.S/A.incl}}
 & {\textbf{ambo}}
 & {\textit{bwi}}
 & {\textit{puy}}\\
{\textbf{1.pl.S/A.excl}}
 & \textbf{nambo}
 & \textit{ɲaŋa}
 & {\textit{nan}}\\
\lspbottomrule
\end{tabularx}
\caption{First-person pronouns in languages spoken in the vicinity of Mapoon mission}
\label{bkm:Ref323845543}
\label{fig:key:205}
\end{table}

%%%(Orthographies not standardised)

\subsection{Provenance of \citegen{schwarz_koko_1900} and \citegen{roth_structure_1901} Guugu-Yimidhirr analyses}
\label{sec:key:10.1.4}\label{bkm:Ref456087169}

The extent to which Roth’s 1901 Guugu-Yimidhirr publication was based on an existing analysis of the language made by missionaries at Cape Bedford mission \citep{schwarz_koko_1900} or his own previous published grammar of Pitta-Pitta \citep{Roth1897} has not been well-understood. Breen writes:

\begin{quote}
It is not clear how much of the work is actually Roth’s. Probably the fieldwork was done by the missionaries … Roth likely turned their unpublished efforts … into a grammar, which followed the pattern of his earlier grammar of Pitta-Pitta. \citep[136]{breen_we_2008}
\end{quote}

What has not previously been recognised about the intellectual reciprocity between Roth and the Lutheran missionaries is that Poland and Schwarz’s Guugu-Yimidhirr grammar \citeyearpar{schwarz_koko_1900} shows a strong influence from the structure of Roth’s Pitta-Pitta grammar. While the missionaries may have provided the content, Roth provided the framework in which they described the language. Comparison of missionaries' documentation of Guugu-Yimidhirr \citep{schwarz_koko_1900} with Roth’s grammar of Pitta-Pitta \citeyearpar{roth_ethnological_1897} also shows that Roth’s influence extended as far as the nature of clauses given as illustration. Compare, for example, Examples (\ref{bkm:Ref465086262}) and (\ref{bkm:Ref465086269}):

\ea
\label{bkm:Ref465086262}Kana-lo nopo-na nungkarea ; kooyungo-ngooro\\
\glt `The man takes care of his wife; (he is) a good fellow'

(\citealt[24]{roth_ethnological_1897}; Pitta-Pitta)
\z

\ea
       \label{bkm:Ref465086269}bama diral nangu(go) na-(mal)-ma : nulu bodan\\
\glt `The man who looks after his wife is a good fellow' \\

(\citealt[no pag]{schwarz_koko_1900}; Guugu-Yimidhirr)
\z

The clause was also replicated in Roth’s Guugu-Yimidhirr publication:

\ea
        \label{bkm:Ref339469326}bama diral nangu-gobantchen-chil: nulu bodan\\
\glt `The man who nurses his own wife is a good fellow' \\
(\citealt[18]{roth_structure_1901}; Guugu-Yimidhirr)
\z

and in Hey’s description of Nggerrikwidhi:

\ea
\gll     ma  endranana-nu   nguno-ma    yi    sea  : ma tanko\\
    man woman-to   his-very-own food gives : man good\\
\glt `The man who gives food to his own wife is good man' \\
(\citealt[13]{hey_elementary_1903}; Nggerrikwidhi)
\z

Poland \& Schwarz’s Guugu-Yimidhirr MS grammar follows Roth’s descriptive template, employing his unconventional ordering of word-class headings (c.1900: no pag.) and his sub-categorisation of prepositions into the four classes of “motion'', “rest'', “purpose, reason and means” and “time” (\figref{fig:key:201}).

Poland \& Schwarz’s presentation of peripheral case forms of Guugu-Yimidhirr pronouns in the MS grammar held at AIATSIS (\figref{fig:key:206}) replicates the presentation innovated by Roth in his 1897 grammar of Pitta-Pitta employing the same lettering system (\tabref{fig:key:202}). Roth subsequently repeated the missionaries' presentation in his published Guugu-Yimidhirr grammar (\citeyear{roth_structure_1901}; \figref{fig:key:207}). The pronominal case forms given left to right in the early sources mark cases termed by \citet[66]{haviland_guugu_1979} abesive, adessive, purposive and dative.


\begin{figure}
\includegraphics[width=\textwidth]{figures/stockigt-img151.jpg}
\caption{Poland \& Schwarz’s pronominal case paradigm (\citeyear[no pag.]{schwarz_koko_1900}; Guugu-Yimidhirr)}
\label{bkm:Ref323828418}
\label{fig:key:206}
\end{figure}

\begin{figure}
% \includegraphics[width=\textwidth]{figures/stockigt-img152.emf}
\includegraphics[width=\textwidth]{figures/stockigt-img152.png}
\caption{Roth's pronominal case paradigm (\citeyear[18]{roth_structure_1901}; Guugu-Yimidhirr)}
\label{bkm:Ref465432856}
\label{fig:key:207}
\end{figure}

Hey described his grammar as having been “drawn up … section by section on the lines followed by Dr Roth” (\citeyear{roth_structure_1901}: 1). It seems likely that upon Roth’s request, Schwarz similarly “drew-up” a grammar of Guugu-Yimidhirr using the missionaries' existing grammatical understanding acquired over the previous twelve years but using the template of the Pitta-Pitta publication \citeyearpar{roth_ethnological_1897}.

While Hey commenced his presentation of pronominal case forms as Roth had done in both Pitta-Pitta (\citeyear{roth_ethnological_1897}) and in Guugu-Yimidhirr (\citeyear{roth_structure_1901}), he was unable to give multiple ``indirect object'' forms, that is, locative, allative and purposive. Rather, under the heading ``objective, indirect object'', Hey wrote: “there are other forms of the above pronouns, meaning: `for me',  `with me', `from me' etc.” (\citeyear{hey_elementary_1903}: 13).

\subsection{Ergativity}
\label{sec:key:10.1.5}

This section examines the description of ergativity in Pitta-Pitta \citep{roth_ethnological_1897}, Guugu-Yimidhirr (\citealt{schwarz_koko_1900}; \citealt{roth_structure_1901}), and Nggerrikwidhi \citep{hey_elementary_1903}. The discussion accounts for the fact that, although the template Roth established in his initial grammar of Pitta-Pitta (\citeyear{roth_ethnological_1897}) was employed throughout this later body of works, ergative morphology was not described in these latter grammars. Consequently, S. Ray’s comparative sketch grammar of the Guugu-Yimidhirr and Nggerrikwidhi (\citeyear[267--270]{ray_linguistics_1907}; \sectref{bkm:Ref339464046}), based on \citet{roth_structure_1901} and \citet{hey_elementary_1903}, would also fail to describe ergative morphology.

\subsubsection{Description of ergativity in Pitta-Pitta}
\label{sec:key:10.1.5.1}

In learning Pitta-Pitta, Roth was presented with a peculiar morphological complexity that is not known to have existed in any other PN languages. Nouns and pronouns show tripartite marking in the non-future but have accusative alignment (\stockcaps{AS/O}) in the future (\tabref{fig:key:208}).


\begin{table}
 \begin{tabularx}{\textwidth}{XXXl}
\lsptoprule
 & \textbf{A}
 & \textbf{S}
 & \textbf{O}\\
 \midrule
 \textbf{Non-future}
 & \textit{{}-lu}
 & \textit{{}-Ø}
 & \textit{{}-nha}\\
 \textbf{Future}
 & \textit{{}-ngu}
 & \textit{{}-ngu}
 & \textit{{}-ku}\\
\lspbottomrule
\end{tabularx}
\caption{Syntactic case forms on Pitta-Pitta nouns}
\label{bkm:Ref465081188}
\label{fig:key:208}
\end{table}

%%%\citep[193]{blake_australian_1979}

Roth made an exceptionally astute analysis of this unusual sensitivity of ergative morphology to verb tense in Pitta-Pitta on nouns (\figref{fig:key:10-153}) and on pronouns (\figref{fig:key:8-210}; see \citealt{blake_pitta-pitta_1971}: 84--90). That Roth described the sensitivity of ergative marking to tense in Pitta Pitta has implication for a similar scenario described and illustrated by \citet{symmons_grammatical_1841} in Nyungar (\sectref{bkm:Ref74322244}), which remains to be properly reclaimed.


\begin{figure}
\includegraphics[width=\textwidth]{figures/stockigt-img153.png}
\caption{Roth's description of ergative marking on Pitta-Pitta nouns \citeyearpar[7]{roth_ethnological_1897}}
\label{bkm:Ref465081043}
\label{fig:key:10-153}
\label{fig:key:209}
\end{figure}

Under the discussion of nominative case, he stated:

\begin{quote}
If the subject governs a transitive verb in present or past time, it takes the suffix –lō …: with an intransitive verb, under similar conditions, no addition is made … In the future time, with both transitive and intransitive verbs, the subject take the suffix –ng-ō. \citep[7]{roth_ethnological_1897}
\end{quote}

\ea
Machoomba-lo   wapa-lo          pooriti-na          pokara-na      tichea\\
\glt `the kangaroo’s pup is eating all the grass'

                  \citep[12]{roth_ethnological_1897}

\gll matyumpa-lu       warrpa-lu       ?-nha       pukarra-nha   thatyi-ya\\
kangaroo-\stockcaps{ERG}    young-\stockcaps{ERG}     ?all-\stockcaps{ACC}        grass-\stockcaps{ACC}      eat-\stockcaps{PRES}\\
\z

Of nominative and ergative pronouns, Roth (\citeyear{roth_ethnological_1897}: 2; 10) stated: “Like other personal pronouns, these are inflexed according as they refer to present and past or future time”, and he carefully tabulated the forms (\figref{fig:key:208}).


\begin{figure}
% \includegraphics[width=\textwidth]{figures/stockigt-img154.emf}
\includegraphics[width=\textwidth]{figures/stockigt-img154.png}
\caption{Roth's presentation of nominative and ergative pronouns in future and non-future tenses in Pitta-Pitta \citeyearpar[10]{roth_ethnological_1897}}
\label{bkm:Ref465081051}
\label{fig:key:8-210}
\end{figure}

\hspace*{-4.4pt}Like other grammarians (\sectref{sec:key:3.3.4}), Roth accounted for ergative morphology twice, once in a discussion of case, and a second time under the heading ``prepositions''. \citet[16]{roth_ethnological_1897} accounted for the function of the ergative suffix on nouns in Pitta-Pitta\textbf{ }in the discussion of the prepositional sub-class “Prepositions of purpose, reason and means''.

Nominals translated with the English prepositions `with’, `by' and `through' were presented as a type “f” of this prepositional class, headed “With, by, through, the physical agency of” (\figref{fig:key:10-155}).


\begin{figure}
\includegraphics[height=.92\textheight]{figures/stockigt-img155.png}
\caption{Roth’s presentation of ergative morphology in Pitta-Pitta as a “preposition'' (1897: 15--16)}
\label{bkm:Ref338507335}
\label{fig:key:10-155}
\end{figure}

Type “f” showed the non-future ergative suffix \textit{-lu} and the future ergative suffix \textit{-ngu.} Each of the examples Roth supplied shows both an ergative NP and an instrumental NP, which show syncretism \citep[193]{blake_pitta_1979}. For example:

\ea
wungata\label{ex:key:10:6:roth}\footnote{The form \textbf{wungata} `woman' is not recorded by \citet{blake_pitta_1979}}{}-ngo    kanari-ko      moorra-ngo           pite\\
\glt `The woman will hit the girl with the stick'

(\citealt{roth_ethnological_1897}: 16, Pitta-Pitta)

\gll wangatha-ngu                  kiniyari-ku    murra-ngu             pithi\\
woman-\stockcaps{FUT}/\stockcaps{ERG}      girl-\stockcaps{DAT}       stick-\stockcaps{FUT}/\stockcaps{INST}   hit-[\stockcaps{FUT}]\\
\z

\subsubsection{Description of ergativity in Guugu-Yimidhirr}
\label{sec:key:10.1.5.2}

In contrast to Roth’s insightful description of ergative marking in Pitta-Pitta, his grammar of Guugu-Yimidhirr \citeyearpar{roth_structure_1901} did not record ergative marking of nouns either in a discussion of case (\figref{fig:key:10-156}) or as a sub-class of preposition (\figref{fig:key:10-157}). Since Guugu-Yimidhirr pronouns are accusatively aligned (\stockcaps{AS/O}), the forms of ergative pronouns were described simply as “nominative” (ibid.: 17). Similarly, Poland \& Schwarz’s grammar of Guugu-Yimidhirr \citeyearpar{schwarz_koko_1900} did not describe ergative function.

\begin{figure}
\includegraphics[width=\textwidth]{figures/stockigt-img156.png}
\caption{The absence of description of ergative morphology on nouns in Roth (\citeyear[16]{roth_structure_1901}; Guugu-Yimidhirr)}
\label{bkm:Ref465081955}\label{fig:key:10-156}
\end{figure}

\begin{figure}
\includegraphics[width=\textwidth]{figures/stockigt-img157r.png}
\caption{The absence of description of ergative morphology on pronouns in Roth (\citeyear[30]{roth_structure_1901}; Guugu-Yimidhirr)}
\label{bkm:Ref465081962}
\label{fig:key:10-157}
\label{fig:key:10-213}
\label{fig:10:213}
\label{fig:key:213}
\end{figure}

Under the prepositional sub-type labelled ``prepositions of purpose, reason and means'' and translated with the English prepositions `with', `by' and `through', where Roth had accounted for the form of Pitta-Pitta nouns in ergative and instrumental functions (\figref{fig:key:8-210}), Roth’s grammar of Guugu-Yimidhirr gave type ``vi'': `with, by though, agency of'. Ergative and instrumental cases show syncretism in Guugu-Yimidhirr \citep[47]{haviland_guugu_1979}. Of the nine suffixes Roth gave as marking this functional range, he stated: “Various suffixes are used to denote this, but the why or wherefore of their use I have not been able to discover” (\citealt{roth_structure_1901}: 30).

\hspace*{-5pt}Allowing for Roth’s orthographic under-differentiation of consonant phonemes, and the fact that he did not analyse processes of morphophonemic alteration to the vowel length of the preceding stem – indicated by \citet[48]{haviland_guugu_1979} as a colon or a dollar sign preceding the suffix – Roth’s list (\figref{fig:key:10-157}) accounted for each of the seven main ergative/instrumental allomorphs given by Haviland (ibid.: 47): \textit{{}-ngun, nda, -: nh, -\$inh, -\$il, -: il,} \textit{and -: ,} except for the last, which shows just lengthening of the stem-final vowel, and which Roth would probably have perceived as an unmarked stem.

Yet all the examples Roth provided of NPs marked with any of these ergative/instrumental allomorphs are in instrumental function. That none of the examples show ergative/instrumental marking on the agent of a transitive clause is explained by the fact that Roth described the variety, which had been established at the mission, in which nouns in ergative case remained unmarked (\sectref{sec:key:2.5.1}). In \textit{all} example sentences supplied in the early Guugu-Yimidhirr grammars (see Example \ref{ex:2:13} on p. \pageref{ex:2:13} and \ref{ex:2:14} on p. \pageref{ex:2:14}) a noun acting in the role of agent is \textit{always} followed by a 3\textsuperscript{rd} person pronoun, which shows accusative alignment (\stockcaps{AS/O}). Pronominal-final NPs is not a feature of the variety described by \citet{haviland_guugu_1979}. Example \ref{bkm:Ref77508300}, given by \citet[30]{roth_structure_1901} to exemplify this subclass of preposition, similarly shows a pronominal-final ergative NP. The plural ergative NP \textit{ngaanndhu -ngay} `woman' receives no ergative morphology but is followed by a 3plERG pronoun. However, the NP \textit{dyuugaar} `sand' in instrumental function is marked with the ergative/instrumental allomorph \textit{{}-\$inh,} which shortens the second syllable of a disyllabic stem.

\ea
Ngando-ngai         dana            ngundar \footnotemark{} dogar-en               baitchare\label{bkm:Ref77508300}n\\
the-woman             they            the-plums           with-sand            covered

(\citealt{roth_structure_1901}: 30, Guugu-Yimidhirr)

\gll ngaandhu-ngay       dhana              ?                      dyuugaar-\$inh        baydya-rrin\\
     woman-\stockcaps{PL} 3\textsc{pl}.\stockcaps{ERG} plum-[\stockcaps{ACC}] sand-\stockcaps{INST} cover-\stockcaps{PAST}\\
\footnotetext{The form \textbf{ngundar} appears not to have been recorded by \citet{haviland_guugu_1979}, who does however (ibid.: 176) give \textit{wunha,} which is translated as `wild ``nanda'' fruit'.}
\z

The oddity that Roth recorded morphology that marked the ergative/instrumental functional range in Guugu-Yimidhirr, but did not describe ergativity, and only exemplified instrumental function, is explained by the fact that the variety he recorded was not fluent native speaker usage, but was a variety used by the missionaries.

Guugu-Yimidhirr presented Poland, Schwarz and Roth with a range of ergative allomorphy larger than that encountered by other missionary-grammarians and which made accounting for ergative forms more difficult. The language also exhibits what \citet[154]{haviland_guugu_1979} terms “ergative hopping''. Transitive complement clauses optionally cause ergative marking on the S argument of an intransitive verb \citep[154--156]{haviland_guugu_1979}. So that in a construction translated as: `the boy doesn’t want to chop the tree down' (ibid.: 154) the NP “the boy” can be marked as either nominative or ergative. This factor would have thwarted the early missionary-grammarians' ability to understand ergative function. Motivated by a complexity in Guugu-Yimidhirr, that other early grammarians appear not to have encountered, the missionaries invented a strategy of placing the correct case form of the pronoun in NP-final position, which avoided the need to mark the noun as ergative.

\subsubsection{Description of ergativity in Nggerrikwidhi}
\label{sec:key:10.1.5.3}

Hey’s \citeyear{hey_elementary_1903} grammar of Nggerrikwidhi, revised and edited by Roth and written using Roth’s descriptive template, does not describe ergative morphology in the discussion of nominal case. Under the prepositional subheading: ``purpose, reason, means'' type ``vi'', where ergative/instrumental morphology was described in Roth’s previous grammars (\citeyear{roth_ethnological_1897}; \citeyear{roth_structure_1901}), \citet[21]{hey_elementary_1903} stated: “with, by, through, is in most cases translated with the suffix –be”. Like in Roth’s Guugu-Yimidhirr grammar (\citeyear{roth_structure_1901}; \figref{fig:10:213}), the suffix \textbf{-be }is only shown attached to nominals in instrumental function. The NPs in ergative function are unmarked, but, unlike the Guugu-Yimidhirr ergative NPs, they do not occur with pronouns.

The three clauses given by \citet[21]{hey_elementary_1903} to illustrate this prepositional sub-class, i.e., instrumental morphology, are Nggerrikwidhi translations of the almost identical English translations of Guugu-Yimidhirr clauses that were given to illustrate the same point in Roth’s grammar (\citeyear{roth_structure_1901}; \figref{fig:key:213}). Compare the following example with Example (\ref{bkm:Ref77508300}).

\ea \label{bkm:Ref77508359}
\gll   Lante    yi     agoi-be       aentchina\\
girl      food     sand-with   cover{ }up\\
\citep[21]{hey_elementary_1903}
\z

The other two clauses: “man dog stick-with strikes”, “woman cold-through the effects of sick”, similarly show how closely Hey’s Nggerrikwidhi \citeyearpar{hey_elementary_1903} grammar mirrors \citegen{roth_structure_1901} Guugu-Yimidhirr grammar. Hey’s 1903 description of Nggerrikwidhi gives no information about the marking of the agent of a transitive verb.

\section{Ray \& A. C. Haddon}
\label{sec:key:10.2}\label{bkm:Ref516423313}\label{bkm:Ref514607856}

S. Ray (1858--1939) had no formal linguistic training beyond the standard preparation for his profession as an elementary school teacher in east London. Having developed an interest in Oceanic languages as an “armchair philologist'', he corresponded with missionaries and amassed “as comprehensive a language data base as possible on one of the last groups of languages left to be investigated” \citep[183]{shnukal_at_1998} before conducting his own fieldwork in the region. He published two grammars (\citeyear{ray_study_1893}; \citeyear{ray_linguistics_1907}) of Kalaw Lagaw Ya, a Pama-Nyungan language spoken in the western Torres Strait, and sometimes referred to as “Western Torres Straits” (henceforth WTS). He also published short grammars of three PN languages spoken in the far north of Queensland (\citeyear{ray_linguistics_1907}; \tabref{fig:key:214}).


\begin{table}
\begin{tabularx}{\textwidth}{lp{3cm}Ql}
\lsptoprule
 \textbf{Year}
 & \textbf{Language}
 & \textbf{Informed by}
 & \textbf{Length}\\
 \midrule
\textbf{1893}
 & WTS & Existing wordlists including those collected by Haddon, (1888--9) MacGillivray’s vocabulary of Gudang \citeyearpar{gordon_remarks_1852} and the Gospel of St Mark \citep{scott_evangelia_1879}
 & 46 pages\\
\textbf{1907}
 & \stockcaps{WTS} & Ray’s own data collected in 1898 and possibly also missionary translations & 42 pages\\
\textbf{1907}
 & Guugu-Yimidhirr and Nggerrikwidhi & \citet{roth_structure_1901}; \citet{hey_elementary_1903}
 & 4 pages\\
\textbf{1907}
 & Yadhaykenu & Thursday Island Police Officer named Oikantu, (Jimmy Matauri) and family & 4 pages\\
\lspbottomrule
\end{tabularx}
\caption{Ray’s grammars of PN languages}
\label{bkm:Ref323815775}
\label{fig:key:214}
\end{table}

\subsection{Ray \& A. C. Haddon 1893}
\label{sec:key:10.2.1}\label{bkm:Ref514608187}

Ray’s first grammatical description of WTS was published in 1893, before he ever visited the Torres Straits. This grammar and associated linguistic material from the region were made in conjunction with A. C. Haddon (1855--1940), who was then a biologist at the University of Cambridge and later became an ethnologist \citep{Mullens1996}. Haddon had sought Ray’s assistance in preparing for publication the vocabularies he himself had collected when visiting the Torres Straits to study the area’s marine biology in 1888--9.

Like most other PN grammatical material produced outside Australia, the primary purpose of the vocabularies, phonological comparisons and grammatical analyses appearing in the large 1893 publication was classificatory. The work attempts to make internal classification of Torres Straits languages and to establish their relatedness to the languages spoken on the Papuan mainland to the north and the Australian mainland to the south. An east-west linguistic divide in the Torres Strait had been observed prior to Ray’s engagement in the field, notably by the English philologist R. G. Latham (1812--1888). Informed by data gathered on the 1846--1850 \textit{Voyage of the Rattlesnake}, \citet{gordon_remarks_1852} classified the Eastern Torres Strait language, Miriam, as a Papuan dialect. By comparing the WTS dialect, Kala Lagaw Ya (Kowrarega, Kaiwaligau Ya), with Gudang and numerous other Australian languages from a wide variety of sources, he argued that WTS was related to the languages of mainland Australia. \citegen[509]{ray_linguistics_1907} investigations agreed with Latham’s findings, as does current thought \citep{alpher_western_2008}, although the relatedness of WTS to the Paman languages spoken on Queensland’s Cape York Peninsula has proven controversial (\citealt{dixon_languages_1980}: 234; \citeyear{dixon_preface_2002}: 608, 681; \citealt{hunter_reappraising_2011}).

Ray stressed the inadequacy of the material upon which his earliest grammar of WTS was based (\citeyear{ray_study_1893}: 119, 279). The work was informed by Haddon’s vocabularies and those of others, most notably MacGillivray’s vocabulary of the non-WTS language Gudang (1852) as well as what Ray describes as: “the only text available for the elucidation of the Saibai grammatical forms” \citep[119]{ray_study_1893}, a missionary’s translation of the Gospel of St Mark. The tenuous practice of drawing up of a grammar based on the structure of language used by missionaries in Bible translation was utilised also by \citet{gatti_lingua_1930} when writing a grammar of Diyari (\sectref{bkm:Ref516424775}).

Ray’s linguistic work is generally held in high esteem. His linguistic expertise was utilised by other members of the interdisciplinary expedition, notably by W. H. R. Rivers, the expedition’s psychologist, who recorded local genealogies and developed a method that came to underpin social anthropology. Ray is known to have been a tireless and meticulous recorder who worked his informants hard (\citealt{shnukal_at_1998}: 190). \citet[181]{shnukal_at_1998} observes that Ray made “sound grammatical description of the languages and the most comprehensive vocabulary lists published thus far … the linguistic researcher can be assured that Ray’s observations are reliable”.

\subsection{S. Ray \citeyearpar{ray_linguistics_1907}}
\label{sec:key:10.2.2}\label{bkm:Ref339464046}

Ray and Haddon’s 1893 linguistic publication appeared as Haddon was garnering support for a return expedition to the area. Haddon envisaged an expedition that was to simultaneously establish his career as an anthropologist and secure the status of anthropology as an academic discipline within universities in the twentieth century.

After the success of the 1893 publication, Ray, the self-taught linguist, was chosen as the multi-disciplinary expedition’s linguist. His linguistic findings were published as Vol.I, Pt. III of the expedition Reports \citeyearpar{ray_linguistics_1907}, of which he was largely the sole author. In the quest to establish the linguistic relatedness, and the origins of the people speaking the region’s diverse languages, the 1907 publication cast a wide grammatical net. Divided into four sections, this work not only examined Torres Strait and Papuan languages – sections I and III respectively – but also devoted section II to the study of “languages of the Cape York Peninsula, North Queensland” and section IV to “The linguistic position of the languages of Torres Straits, Australia and British New Guinea”.

In addition to numerous vocabularies and the presentation of missionaries' translations of liturgical texts with inserted interlinear gloss, Section I provides grammatical sketches of Meriam and WTS. \citet[5]{ray_linguistics_1907} describes the 1907 forty-two-page WTS description as “superseding all that was formerly written on the structure … of the language”. The 1907 work is more extensive than the earlier 1893 WTS grammar and contains additional description of derivational processes, revised case terminology, lexical examples of dialect variation, and example clauses taken from his own data rather than from missionaries' translations.

Section II presents the languages of Cape York Peninsula. It is the shortest section of the work. While a relationship between WTS and languages spoken at the northern extremity of the Peninsula had been recognised in 1893, this work sets out to “determine how far the particular languages in contact with those of the Straits represent those of Queensland more generally” \citep[264]{ray_linguistics_1907}

Ray notes the meagre quantity and insufficient quality of lexical resources available, stating that grammatical material only became available in the years after the expedition but prior to the publication of its reports. He refers to Roth’s grammar of Guugu-Yimidhirr (\citeyear{roth_structure_1901}; \sectref{bkm:Ref74336027}), spoken relatively far south on the Peninsula, without acknowledging the contribution of the Lutheran missionaries, and to N. Hey’s description of Nggerrikwidhi (\citealt{hey_elementary_1903}; \sectref{bkm:Ref74336064}), spoken slightly south-west of the Peninsula’s northern extreme. Informed by these two sources, Ray presented a four page “abbreviated” grammar of both languages which Ray described as having “practically an identity of structure” (\citeyear{ray_linguistics_1907}: 267). The missionaries' orthography was altered in order to conform to other material presented in the Reports.

Following this grammar is a very short grammatical sketch of another Cape York language, Yadhaykenu, which Ray describes as having been obtained


\begin{quote}
from a Yayaikana native, named Oikantu, generally known at Thursday Island as Jimmy Matauri, a member of the native police force. He spoke English fairly well, and had to some extent forgotten his native language.
\end{quote}

Ray noted a resemblance between this material and MacGillivray’s vocabulary of Gudang (1852). Yadhaykenu is currently described as a dialect of Uradhi, a language closely related to its northern Paman neighbour, Gudang, both once spoken at the very tip of the Peninsula.

\citet[181]{shnukal_at_1998} writes that, “although Ray is known to few contemporary scholars outside the field of linguistics, within this field his reputation is secure”. That may be so, but Ray’s contribution to the developing understanding of Australian linguistic typology, which heralded in the second descriptive era (1930–1960), has been under-recognised. His entry titled “Australian languages” in the Australian Encyclopaedia (\citealt[2--15]{ray_aboriginal_1925}) is the earliest of three overviews written on the cusp of the second descriptive era collating what had been discovered about Australian languages in the pre-contemporary era. McGregor’s “overview [of] existing histories of research on Australian Aboriginal languages” (\citeyear{mcgregor_missionary_2008}: 2), for instance, makes no mention of Ray’s entry, published over a decade before \citet{capell_structure_1937} and \citet{elkin_nature_1937}.
