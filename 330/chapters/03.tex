\chapter{Lancelot Threlkeld’s earliest analyses of an Australian language}
\label{chap:key:3}

This chapter discusses L. E. Threlkeld’s (1788--1859) grammar of Awabakal (\citeyear{threlkeld_australian_1834}), which gave the earliest account of both nominal and verbal morphology in an Australian language. After providing some historical context in which this inaugural Australian grammar was written, Threlkeld’s presentation of the case system and his account of ergativity are investigated in detail, thus establishing a baseline to which later corpus grammars can be compared. Section 3.3.6 unravels Threlkeld’s description of bound pronouns and his description of compound pronouns, which were termed by the author ``the conjoined dual case'', and which have both resulted in different contemporary reclamations of the systems that Threlkeld described.

\section{Historical overview}
\label{sec:key:3.1}

Threlkeld, the son of a London brush-maker, pursued a range of diverse occupations -- tradesman, circus performer, actor and businessman -- before commencing missionary training \citep{gunson_australian_2016-1}. He trained at Gosport, a Congregationalist Missionary Seminary in Hampshire, which had been established by Rev. D. Bogue (1750--1925), founding member of the London Missionary Society \citep[291]{Champion1939}.

\citet[165]{newton_more_1987} details Threlkeld’s grammatical influences. He names R. Lowth (1710--1787) – whose influential and early pedagogical grammar of English (\citealt{lowth_short_1762}) first claimed that “preposition stranding” was improper in English – the lexicographer S. Johnson (1709--1784), and Threlkeld’s tutor at the Gosport seminary, M. Wilks, “who taught him English grammar and provided a foundation for the study of other grammars”. 

Although seemingly lacking the rigorous classical training of some of the later missionary-grammarians, Threlkeld demonstrated a broad linguistic knowledge, describing (\citeyear[x]{threlkeld_australian_1834}) Awabakal as having “much of the Hebrew form in the conjugation, the dual of the Greek and the deponent of Latin”. He observed (\citeyear[vi]{threlkeld_australian_1834}) that Australian “tongues” were similar to Polynesian languages in marking dual number but were “more definite in the use of tenses”. In his reminiscences he recalled that the Australian language presented greater challenges than Polynesian languages, noting particularly the “conjoined dual case” in Awabakal pronouns (\sectref{sec:key:3.3.6.2}), and the number of declensions and conjugations of the verb \citep[42]{threlkeld_notitle_1974}.

Threlkeld arrived in Sydney in 1824 on what was supposed to be a return voyage to England. By 1826, nearly four decades after the colonisation of New South Wales, he had established the colony’s first mission, on the eastern side of Lake Macquarie, north of Sydney, at a site called Reid’s Mistake or Bahtabah. After a dispute over mission expenditure in 1828, Threlkeld was dismissed by the London Mission Society, but pursued his linguistic studies while working as protector and court interpreter in collaboration with his Aboriginal friend and main linguistic informant Biraban (\sectref{sec:key:1.1.3}). Mission work resumed the following year after land was granted on the western side of the lake at a site named ``Ebenezer'' \citep{gunson_australian_2016-1}.\footnote{Missions of the same name, “Ebenezer”, were later established, one by the Lutheran missionary Teichelmann close to Adelaide in 1843, and another by the Moravian missionaries Hagenauer and Spieseke in western Victoria in 1858.}

Prior to making successive descriptions of Awabakal, Threlkeld had spent six years at London Mission Society missions in Polynesia. \citet[vi]{threlkeld_australian_1834} was explicit about his choice to use the orthography which he was familiar with from working in Polynesia (\figref{fig:3:29}).

\begin{figure}
\includegraphics[width=.8\textwidth]{figures/Figure 29.png}
\caption{Threlkeld's deliberation about the orthographic system \citeyearpar[vi]{threlkeld_australian_1834}}
\label{fig:3:29}
\end{figure} 

\largerpage
It is highly likely that Threlkeld had read the grammar of Tahitian written by fellow London Mission Society missionary J. \citet{davies_grammar_1823}, whose work “had the greatest prestige and influence of older Polynesian grammars and was the one to form the framework for most of the following missionary grammars” (\citealt[109]{hovdhaugen_beginning_1993}). It is also likely that Threlkeld, and early missionaries at Wellington Valley, had read, or were exposed to W. P Crook’s 1799 grammar and dictionary of the languages of the Marquesas (\sectref{sec:key:4.1}). Threlkeld was thus perhaps better prepared for an encounter with languages bearing little resemblance to SAE than were most early grammarians in Australia.

Threlkeld produced three works describing Awabakal morpho-syntactic structure -- Threlkeld (\citeyear{threlkeld_specimens_1927}; \sectref{sec:key:3.2}), Threlkeld (\citeyear{threlkeld_australian_1834}; \sectref{sec:key:3.3}), and Threlkeld (\citeyear{threlkeld_key_1850}) -- of which the 1834 work is the most comprehensive, and is treated in detail here. While the 1850 publication, \textit{A key to the structure of an Australian language}, contains valuable “illustrative sentences”, it does not present additional analysis of the language, other than elaborating on the phono-semantic theory, which was outlined in 1827 (\sectref{sec:key:3.2.1}). Written when \citet[3]{threlkeld_key_1850} described the language as “all but \textit{extinct}”, there is something wistful about the 1850 publications, which includes (ibid.: 5--7) “reminiscences of Birabān”.

In 1836 Threlkeld produced \textit{An Australian spelling book in the language as spoken by the Aborigines…}. It is the first of four nineteenth-century primers printed in an Aboriginal language in order that an Aboriginal Christian congregation would be literate and able to access Christian teachings in their first language. The others are Ridley’s Gamilaraay primer (\citeyear{ridley_gurre_1856}), Homann \& Koch’s Diyari primer (\citeyear{koch_nujanujarajinkiniexa_1870}), and Kempe’s primer in Arrernte (\citeyear{kempe_intalinja_1880}). A Wiradjuri primer, which remained unpublished, was also drafted at Wellington Valley by missionary Watson in 1835 (\citealt[413]{bridges_church_1978}). The tradition of vernacular literacy at missions in New South Wales, instigated by the Presbyterian missionaries Threlkeld and Ridley, precedes the earliest publication of similar materials by Lutherans in South Australia (\sectref{sec:key:8.3.2}, \sectref{sec:key:9.1}).

In 1873 Rev. W. \citet[275]{ridley_report_1873} wrote that Threlkeld’s grammars, which were printed in a modest print-run (\citealt[169]{newton_more_1987}) were “now not to be purchased”. Like other PN grammars considered in the corpus, particularly \citet{gunther_lecture_1840} and \citet{symmons_grammatical_1841}, Threlkeld’s linguistic works were rescued from obscurity through republication in Symmons (\citeyear{symmons_grammar_1892}; \sectref{sec:key:1.1.4}). Fraser presented a slightly edited version of Threlkeld’s 1834 work and a heavily edited and abbreviated version of the 1850 work. Threlkeld’s translations of Christian texts (see \citealt[170]{newton_more_1987}) were also first published in \citet{symmons_grammar_1892}.

\section{Threlkeld 1827}
\label{sec:key:3.2}

Threlkeld’s earliest work, \textit{Specimens of a dialect of the Aborigines of New South Wales, Being the First Attempt to Form Their Speech Into a Written Language} (\citealt{threlkeld_specimens_1927}), was published within two years of the establishment of the mission at Bahtabah. Although Threlkeld stated in his introduction that “no speculative arrangement of grammar [wa]s attempted” (\citealt[iii]{threlkeld_specimens_1927}), the work, which does not contain a vocabulary, is grammatical in nature. This thirty-page publication shows progress towards the grammatical analysis of Awabakal \citep{threlkeld_australian_1834}. 

The 1827 publication includes a table of nominative pronouns in singular, plural and then dual number \citep[4--8]{threlkeld_specimens_1927}, providing up to thirteen example clauses for each. Following are hundreds of simple clauses selected “from upwards of fifteen hundred Sentences” (ibid.: iii), grouped either as “interrogative sentences” or ``imperative sentences'', and thoughtfully arranged in order to illustrate specific constructions. Examine, for instance, clauses showing the function of the interrogative \textit{minyaring} `what?' inflected with \textit{–pirang}, marking ablative case `from what?', with \textit{–tin} marking causal case `on what account?', and with \textit{-ku} marking instrumental case `with what?' (\citealt[51]{lissarrague_salvage_2006}; \figref{fig:3:30}). They are given with a free and an interlinear-style translation, described by Threlkeld as “word for word, without regard to English arrangement or grammar, in order to shew the idiom of the aboriginal tongue” (\citealt[iii]{lissarrague_salvage_2006}).

\begin{figure}
\includegraphics[width=.8\textwidth]{figures/Figure 30.png}
\caption{Threlkeld’s presentation of clauses illustrating case marking (\citeyear[10]{threlkeld_specimens_1927}; Awabakal)}
\label{fig:3:30}
\end{figure}

Threlkeld’s relationship with Biraban was formed during this earliest phase of mission at Bahtabah, as shown by the inclusion of his informant’s name in the 1827 publication, at this stage spelled “Berehbahn”(\figref{fig:3:31})
\begin{figure}
\includegraphics[width=.8\textwidth]{figures/Figure 31.png}
\caption{\label{fig:3:31} Threlkeld’s reference to “Berehbahn” (\citeyear[13]{threlkeld_specimens_1927}; Awabakal) }
\end{figure}


The work concludes with a two-page discussion of verbal morphology (ibid.: 26--27). Here Threlkeld presented nineteen morphemes, called ``signs'', which he showed unattached, and in isolation from the verb stem (\sectref{sec:key:2.4.2}). Each is then illustrated in a number of clauses (\figref{fig:3:32}).
\begin{figure}
\includegraphics[width=.8\textwidth]{figures/Figure 32.png}
\caption{\label{fig:3:32} Threlkeld’s presentation of tense inflections as “signs” (\citeyear[26]{threlkeld_specimens_1927}; Awabakal) }
\end{figure}


\section{Phono-semantic theory}
\label{sec:key:3.2.1}

Threlkeld delineated word-internal constituents using hyphens. His perception of the word-internal units differs from the few later corpus grammarians who also used hyphens. When introducing the initial section of his 1834 grammar dealing with phonology and orthography, \citet[1]{threlkeld_australian_1834} explained that “Words are composed of Syllables, and Syllables of Letters”. Like later grammarians, Threlkeld differentiates between a sound and an arbitrarily chosen symbol used to represent that sound, and attempted to employ a standardised orthography. In \citeyear{threlkeld_specimens_1927} Threlkeld quotes Johnson: 

\begin{quote}
    The Orthography of a new Language formed by a synod of Grammarians upon the principles of Science would be to proportion the number of letters to that of sounds, that every sound may have its own character, and every character a single sound. \citep[1]{threlkeld_specimens_1927}
\end{quote}

Here \citet[1]{threlkeld_specimens_1927} also explained that “Dr Lowth’s rule hath been attended to in syllabication, namely `divide the syllables in spelling as they are naturally divided in the right pronunciation'”. Words were divided into syllables so that ``syllabication shall be the picture of actual pronunciation''. 

While Threlkeld’s primary motivation for dividing words into syllables using hyphens may have been to guide pronunciation, he also perceived a one-to-one correspondence between syllables and minimal units of meaning. Threlkeld’s interlinear-style glosses tend to represent syllables delineated with hyphens as meaningful sub-word units. In 1827 discrete function tends to be overly assigned to individual syllables through an interlinear-style translation (\ref{ex:3:3-17} \& \ref{ex:3:3-18}). The following phrase, for example, was given in answer to the question: What is fish for?

\ea\label{ex:3:3-17}
\gll Tah-ke-le-ko \\                                 
eat-be-to-for \\
\glt `for to be eat' \\
\citep[11]{threlkeld_specimens_1927} \\
\gll Tjaki-li-ku \\
eat-\stockcaps{NMSR}-\textsc{purp} \\
\glt  `for eating' \\
\z
     
\ea\label{ex:3:3-18}
\gll Won-tah ko lahng bulah \\
Whither for do ye two \\
\glt `Whither are ye two going?' \\
\citep[8]{threlkeld_specimens_1927} \\
\gll wantja-kulang pula \\
\stockcaps{INTER}-\stockcaps{ALL} 2\textsc{dl}.\stockcaps{NOM}  \\
\glt 
\z

Bi-syllabic mono-morphemic sub-word units tend not to be glossed in Threlkeld’s interlinear-style translations.

Large passages from the 1850 publication \citep[19--30, 38--43]{threlkeld_key_1850} pursue a spurious phono-semantic theory, which Threlkeld had introduced in 1827 (\citealt[1--3]{threlkeld_specimens_1927}) in which “every sound forms a root” (\citeyear[90]{threlkeld_key_1850}). Threlkeld set out to “demonstrate the correctness of the supposition” that: 

\begin{quote}
    every character which represents those sounds, become likewise a visible root, so that every letter which forms the Alphabet of the Language, is in reality a root, conveying an abstract idea of certain governing powers which are essential. \citep[9]{threlkeld_key_1850}
\end{quote}

It is difficult to reconcile the overall strength of Threlkeld’s analysis with this aspect of his work, and these sections were omitted in Fraser’s 1892 republication of \citet{threlkeld_key_1850}. The passages have been politely disregarded by modern linguists, who have used Threlkeld’s analyses to reconstruct the structure of the language (\citealt[46]{oppliger_phonology_1984}; \citealt{lissarrague_salvage_2006}).

The use of hyphens to mark syllables and word internal sub-units was not widely adopted by later grammarians. Those who segregated words into hyphenated sections did so either solely as a pronunciation guide (e.g., \citealt{symmons_grammatical_1841}) or in recognition of meaningful word-internal units that were not necessarily monosyllabic and which approach a modern analysis of morpheme boundaries (e.g., \citealt{meyer_vocabulary_1843}) or for both reasons (\citealt{roth_ethnological_1897}). 

\section{Threlkeld (1834)}
\label{sec:key:3.3}

Threlkeld’s major grammatical study was published in 1834. Its title, \textit{Australian Grammar Comprehending the Principles and Natural Rules of the Language …}, is telling of the author’s perception of his achievement in not constraining the description within the existing classical paradigm. He later reiterated that the work was “formed on the natural principles of the language, and not constrained to accord with any known grammar of the dead or living languages. The peculiarities of its structure being such, as to totally prevent the adaptation of any one as a model” \citep[x]{threlkeld_australian_1834}. 

This major grammatical work was a completed MS by 1832, when Archdeacon Broughton obtained a grant to have it and Threlkeld’s translation of the Gospel of St Luke published (\citealt[277]{bridges_church_1978}). Only the grammar was published. The 1834 grammar is arranged very differently from Threlkeld’s earliest grammatical analysis (\citealt{threlkeld_specimens_1927}). It is worth contemplating that Threlkeld’s 1834 analysis may have been guided by the Basle-trained Lutheran minister and Church Missionary Society missionary J. S. C. Handt, who was waylaid in Sydney between June 1831 and August 1832, before establishing the Wellington Valley Mission (\sectref{sec:key:4.1}). But in the absence of a shred of evidence beyond inference (\sectref{sec:key:4.1}), the grammar is here assumed to be Threlkeld’s own. The only reference Threlkeld makes to previous Australian linguistic material is to the wordlist by Isaac Scott Nind (1797–1868), medical surgeon at the King George Sound military garrison in Western Australia \citep{nind_description_1831}.

Threlkeld’s 1834 analysis has been highly regarded in overviews of early Australian linguistic description (\citealt[2]{ray_aboriginal_1925}; \citealt[664]{capell_history_1970}; \citealt[14]{wurm_languages_1972}; \citealt[165]{newton_more_1987}). Threlkeld’s descriptive accomplishment prompted \citet[264]{capell_history_1970} to describe Threlkeld’s work as having “reached an unusually high standard, especially in comparison with the deplorable work done in Australia for almost the next hundred years” (\citeyear[664]{capell_history_1970}). \citet[253]{carey_lancelot_2004} goes as far as to state that Threlkeld’s work was “the \textit{most accomplished} linguistic investigation of any of the 250 Aboriginal languages of Australia undertaken prior to the twentieth century” (emphasis added). Without systematic comparative reading of the sources, the statement remains impressionistic. This study shows that inaugural investigations of some other PN languages, for example, Meyer’s 1843 description of Ngarrindjeri, Kempe’s 1891 description of Arrernte, and Roth’s 1897 description of Pitta-Pitta are of comparable length and descriptive insight, and are as richly exemplified. They are no less accomplished than Threlkeld’s works. 

In comparison with most later corpus works, Threlkeld’s 1834 grammar (Chapter 1, part I: \citeyear[1--78]{threlkeld_australian_1834}) is relatively long and analytically comprehensive. The grammar is followed by a twenty-five page Awabakal to English vocabulary arranged alphabetically under subheadings, some of which are semantic domains and others parts of speech (Chapter 1, Part II: \citeyear[79--104]{threlkeld_australian_1834}). The final section of the publication, headed “Illustrations”(Chapter 2: 105--131) presents a vast array of data grouped in order to illustrate grammatical principles. This tripartite arrangement of a grammar, a vocabulary and a concluding section providing samples of text occurs frequently within the corpus (\citealt{gunther_native_1838}; \citealt{teichelmann_outlines_1840}; \citealt{muller_grundris_1882}; \citealt{roth_structure_1901}; \citealt{planert_australische_1907}). 

\subsection{Threlkeld’s influence on later PN grammarians} 
\label{sec:key:3.3.1}

Threlkeld was himself aware that his inaugural grammar of an Australian language would aid later grammarians in Australia. In \citet[vii]{threlkeld_australian_1834}, he stated: “[W]hen one dialect becomes known, it will assist materially in obtaining a speedier knowledge of any other that may be attempted, than had no such assistance been rendered”. In a letter to C. Schürmann, who had sent him a copy of the vocabulary he collected in Adelaide, \citet{threlkeld_letter_1842} wrote: 
\begin{quote}
    It is vexing that …[my linguistic work] should now be of no use excepting as a help to others engaged in similar pursuits, indeed that was one of the principle objects I had in view when composing it.
\end{quote}

\citet[169]{newton_more_1987} describes how Threlkeld had a flair for self-promotion, and “selectively directed copies of his printed works into the private libraries of royalty and public institutions in Britain and elsewhere”. The potential influence of Threlkeld’s work was enabled by this dissemination, and his works were well known to the earliest corpus grammarians. W. Watson and J. S. C. Handt, who established the Wellington Valley Mission in 1832 (\chapref{chap:key:4}) were given a preliminary copy of Threlkeld’s 1827 work while they were still in London, before it had even reached publication (\citealt[410]{bridges_church_1978}; \citealt[12]{graetz_open_1988}). Similarly, the earliest South Australian missionaries, Teichelmann and Schürmann, were acquainted with Threlkeld’s grammar \textit{prior} to leaving Europe in 1838 (\sectref{sec:key:5.2.2}). C. Strehlow’s German editor was conversant with Threlkeld’s grammar in the early twentieth century (\sectref{sec:key:9.3.4.1}). Ridley (\citeyear[293]{ridley_gurre_1856}; \sectref{sec:key:4.4}), who states that he had recently spent time with Threlkeld, was acquainted with Threlkeld’s grammar \citeyearpar{threlkeld_australian_1834}. Ridley also pointed out that it was through his suggestion that Threlkeld undertook to send a copy of his 1850 publication to Dr. Hodgkin (1788--1866), to whom Ridley had sent his earliest produced description of Gamilaraay society and language \citep{ridley_kamilaroi_1856}.

Threlkeld’s grammar (\citeyear{threlkeld_australian_1834}) was referred to more frequently by other corpus grammarians than was any other early grammar of an Australian language. 

Later corpus grammarians noted that they followed Threlkeld’s “method of spelling words”(\citealt[338]{gunther_lecture_1840}; \citealt[v]{teichelmann_outlines_1840}; \citealt[290]{ridley_kamilaroi_1856}), which they recognised as having also been used “by other missionaries experienced in the Polynesian languages” (\citealt[v]{teichelmann_outlines_1840}). \citet{ridley_kamilaroi_1866} later explained that his system of spelling was used ``in accordance with the practice of those who have reduced to writing the Polynesian languages''. \citet[vii]{moorhouse_vocabulary_1846} indirectly referred to Threlkeld’s spelling system in the same way, stating that he used the “orthography … recommended by the Royal Geographical Society, and in which most of the Polynesian and New Holland languages are recorded”. 

Threlkeld’s grammatical analysis was also referred to by later grammarians. \citet[vi]{moorhouse_vocabulary_1846} referred to Threlkeld’s grammar in a comparison of Australian pronominal forms. \citet[76]{ridley_kamilaroi_1855-1}, after confessing that he was uncertain about verbal morphology in Gamilaraay, presented Threlkeld’s analysis of the ``Newcastle and Lake Macquarie dialect'' as illustration of the potential complexity of the Australian verb. \citet[40--41]{meyer_vocabulary_1843} referred particularly to parts of “Mr Threlkeld’s” analysis of the verb (\citeyear[68, 127]{threlkeld_australian_1834}; \sectref{sec:key:6.1.2.8}) and Threlkeld’s examples given by Meyer were subsequently republished by H. C. \citet[489]{gabelentz_uber_1861}.

A number of remarks made by Threlkeld in the introduction to his grammar (\citeyear[v--xii]{threlkeld_australian_1834}) are echoed in later works, for example, the hindrance that Aboriginal peoples' “wandering habits” posed to missionary activity and language acquisition (\citealt[xi]{threlkeld_australian_1834}; \citealt[v]{meyer_vocabulary_1843}; \citealt[1]{kempe_grammar_1891}), the description of the linguistic structure as “peculiar” (\citealt[ix]{threlkeld_australian_1834}; \citealt[lxii]{thomas_succinct_1878}; \citealt[1]{kempe_grammar_1891}), and the self-effacing remark qualifying the completeness of the work (\citealt[ix]{threlkeld_australian_1834}; \citealt[v]{meyer_vocabulary_1843}). While it is possible that these reiterations were independently motivated by shared frustrations, they also suggest the degree of primacy held by the work. 

In assessing the extent to which Threlkeld’s grammar was “essential in establishing a framework for the study of his fellow-missionary linguists” (\citealt[269]{carey_lancelot_2004}), later chapters of this study compare later descriptions of case with that given by Threlkeld. They consider Australian structures that were not described by Threlkeld, but which were accounted for in later grammars. The study notes instances in which Threlkeld’s descriptive innovations fell into disuse. 

\hspace*{-3.1pt}Threlkeld’s phono-semantic theory (\sectref{sec:key:3.2.1}) was not pursued by subsequent grammarians.

\subsection{Subversion of the traditional framework }
\label{sec:key:3.3.2}

Threlkeld subverted the traditional descriptive framework in order to convey the “peculiarities” of Awabakal. Although content is given under the ten classical parts of speech and their subheadings, Threlkeld adapted the familiar structure for his own purpose. Rather than the heading ``the article'', Threlkeld gave “of the \textit{substitute} for the article'' (emphasis added) (\citeyear[9]{threlkeld_australian_1834}), without in fact describing any “substitute” method of marking definiteness, but instead describing that a pronoun might be marked for the case and number of the noun.

Similarly, of comparatives and superlatives (\citeyear[17]{threlkeld_australian_1834}), Threlkeld wrote: “The following are the methods used in comparison, there being no particles to express equality”. Unlike later corpus grammarians (\citealt{schwarz_koko_1900}; \citealt{roth_structure_1901}; \citealt{strehlow_aranda_1944}), Threlkeld did not intend to show that the comparative and superlative were morphologically marked. He gave a construction that he recognised as serving the same semantic purpose of the SAE inflectional marking of comparative degree.

\ea\label{ex:3:19}
Ke-kul ko-ri-en un-ni yan-ti un-noa ki-lo-a \\
\glt This is not so sweet as that \\
\citep[17]{threlkeld_australian_1834} \\
\gll kaykal-kuriyaN aNi ya\stockcaps{NT}i aNuwa-kiluwa \\
sweet-\stockcaps{PRIV}-[\stockcaps{NOM}] this-[\stockcaps{NOM}] thus that-\stockcaps{SEMB} \\
\glt \citep[62]{lissarrague_salvage_2006} 
\z

\subsection{Third-person “neuter” pronouns}
\label{sec:key:3.3.2.1}

By subverting the category of neuter pronouns in the traditional framework, Threlkeld accounted for the inflection for case of demonstrative pronouns in Awabakal. Awabakal is among the minority of PN languages that make a two-way gender distinction in third-person singular pronouns. Like other early grammarians describing PN languages that morphologically mark gender in third-person pronouns – Diyari, Pitta-Pitta, Minjangbal and Kalaw Lagaw Ya – Threlkeld easily described the distinction, since third-person pronouns in SAE language also mark gender. Threlkeld, however, also presented \textit{neuter} third-person pronouns, although no PN languages make a three-way third-person pronominal gender distinction.

As with many PN languages, Awabakal has demonstratives that inflect for case \citep[335]{dixon_australian_2002}. Threlkeld first presented a paradigm of “neuter” third-person pronouns showing case-inflected forms of \textit{ngali} translated as `this, present'. Following is a paradigm showing \textit{ngala} `that, at hand', and a paradigm showing \textit{ngalawa}, `that, beside' (\figref{fig:3:33}). Note that Threlkeld’s “nominative 2” forms in each paradigm have a distinct deictic stem. The vacant schema of an anticipated third gender of pronouns provided a useful slot in which to describe three sets of Awabakal demonstratives (\citealt[34--36]{lissarrague_salvage_2006}).

\begin{figure}
\includegraphics[width=.8\textwidth]{figures/Figure 33.png}
\caption{\label{fig:3:33} Threlkeld’s presentation of “neuter” third-person pronouns (\citeyear[22]{threlkeld_australian_1834})}
\end{figure}


Unlike in many PN languages, however, Awabakal demonstratives act as noun modifiers, and do not function as the head of a NP. \citet[34]{lissarrague_salvage_2006} explains that they “do not stand in the place of a singular third-person pronoun but may be used with that pronoun to emphasise who has done the action”. 

\hspace*{-2.9pt}Threlkeld perceived that his “neuter” 3sg pronouns were not functionally equivalent to SAE neuter pronouns, and he understood that their grammatical number was indicated by a suffixing pronoun: “These pronouns are singular and plural according to the pronoun attached to them” (\citeyear[22]{threlkeld_australian_1834}). He perceived (ibid.: 21) that the forms were “inexpressible in English in consequence of the locality of the person being included in the word used as a pronoun”, because he could not translate these bi-morphemic phrasal forms (\ref{ex:3:3-20}) with an equivalent English word-for-word translation. Instead, he gave the phrasal compounds a relative function, “this is he who”:

\ea\label{ex:3:3-20}
Ngali-noa \\
\glt This is he who \\
\citep[22]{threlkeld_australian_1834} \\
\gll \textit{ngali-nyuwa} \\
this-[\textsc{erg}]-3\textsc{sg}.\stockcaps{ERG} \\
\glt `this one' \\
\z

% \footnote{Where \textit{ngali} and \textit{ngali-ku} can mark ergative case \citep[34]{lissarrague_salvage_2006}} : Overflow ?
         
His accompanying explanation that the forms were “so compound in their significance as to include the demonstrative and the relative” (\citealt[21]{threlkeld_australian_1834}) results only from the mapping of the syntax of the English translation onto the described structure. 

\subsection{Case paradigms}
\label{sec:key:3.3.3}

The large case systems of Australian languages (\citealt{iggesen_number_2013}) challenged early grammarians, who were equipped with a descriptive framework which accommodated the Latin case system with five morphologically marked cases (nominative, accusative, genitive, dative and ablative), or the Greek and German systems with four (nominative, accusative, genitive, dative).

\citet{threlkeld_australian_1834} extended the SAE case paradigm to include suffixes marking functions not carried by the SAE morphological case systems (\figref{fig:3:34}). He gave between nine and eleven cases paradigmatically, depending on the class of nominal. Awabakal is currently described as inflecting for ten cases \citep[26]{lissarrague_salvage_2006}.

In contrast to Threlkeld’s enlarged paradigms, many other early grammarians maintained classically conservative five-case Latinate paradigms (\sectref{sec:key:5.3.1}).

\begin{figure}
\includegraphics[width=.8\textwidth]{figures/Figure 34.png}
\caption{\label{fig:3:34} Threlkeld’s ten-case paradigm (\citeyear[14]{threlkeld_australian_1834})}
\end{figure}

Threlkeld altered the basic schema of the traditional framework by employing a numbering system to name the additional cases. He gave two dative cases and four ablative cases. 

\begin{figure}
\includegraphics[width=.8\textwidth]{figures/Figure 35.png}
\caption{Hierarchy of inflectional cases (\citealt[159]{blake_1994})}
\label{fig:3:35}
\end{figure}

According to Blake (\citeyear[157--162]{blake_1994}) the cases that are present in the case system of a language adhere to a hierarchy of inclusion (\figref{fig:3:35}). Cases sitting higher on the hierarchy, i.e., to the left, are included in the case system before cases to the right. In small case systems the lowest ranked case on the hierarchy, i.e. the case furthest to the right, assumes oblique or “elsewhere” function.



In languages with a four-case system it is the dative case that assumes the “elsewhere” function. In the five-case system of Latin, the ablative has the broadest oblique function. Threlkeld and other early PN grammarians thus anticipated that heterogeneous case function might be marked by suffixes labelled dative and ablative. By providing numbered “ablative” and “dative” cases, Threlkeld recognised that some of the functions traditionally termed “dative” and “ablative” were marked by different case forms in Awabakal. \tabref{tab:chap3:Awabakal} summarises the labels given to Awabakal case suffixes in different sources.

Threlkeld’s earliest nominal paradigms of a PN language are solid examples of a description responding to the data presented by the language rather than the demands of a prescribed descriptive framework. 

\begin{table}
\small
    \begin{tabularx}{\textwidth}{p{2.3cm}p{2.2cm}p{2.5cm}Qp{1.7cm}}
        \lsptoprule
        Form as shown by Threlkeld & Reconstruction \citet[26]{Lissarrague2006} & \mbox{\citet[14]{Threlkeld1834}}\newline \citet{fraser_introduction_1892} &   \citet[7]{Müller1882} & \citet{Lissarrague2006}  \\
        \midrule
        \textbf{Ko-re} “Man”& \textit{Kuri-Ø} & nominative 1 & \textit{nominativ,\newline subjektiv} & nominative \\
        \textbf{Ko-re-ko} &	\textit{Kuri-ku}	& nominative 2 & \textit{nominativ\newline agentiv} &	ergative/\newline instrumental\\
        \textbf{Ko-re-ko-ba}	& \textit{Kuri -kupa}	& genitive	& \textit{gentiv} &	genitive\\
        \textbf{Ko-re-ko} &	\textit{Kuri -ku} &	dative 1 &	\textit{dativ} &	dative \\
        \textbf{Ko-re-ká-ko}	& \textit{Kuri -kaku} &	dative 2 &	\textit{adessiv}	& allative 1*\\
        \textbf{Ko-re} &	\textit{Kuri -Ø} &	accusative	& \textit{accusative} &	accusative \\
        \textbf{Ko-re-tin} &	\textit{Kuri -tin} &	ablative 1 &	\textit{ablativ}	& causal\\
        \mbox{\textbf{Ko-re-ká-bi-rung}} &	\textit{Kuri -kapirang} &	ablative 2	& \textit{abessiv} &	ablative\\
        \textbf{Ko-re-ko-a}	& \textit{Kuri -kuwa} &	ablative 3 &	\textit{social} &	perlative\\
        \textbf{Ko-re-ka-ba}	& \textit{Kuri -kapa} &	ablative 4	& \textit{commorativ}	& locative\\
        \lspbottomrule
    \end{tabularx}
    \caption{The labels given to Awabakal case suffixes in different sources. The suffix marking the “allative 2” case \citep[30]{lissarrague_salvage_2006} \textit{–ko-láng} \textit{–kulang}, expressing “motion towards” was described by Threlkeld only as a preposition. See  \figref{tab:chap3:morphemes}}
    \label{tab:chap3:Awabakal}
\end{table}

F. \citet[7--8]{muller_grundris_1882} rearranged the order, and altered Threlkeld’s case terminology, when re-publishing Threlkeld’s paradigms (\figref{fig:3:37}). Much of the terminology he used: ``social'', ``adessiv'', ``abessiv'', and ``commorativ'', is not found in other early grammars of PN language written in Australia, suggesting the theoretical divide separating Australian practice and developing European linguistic thought.

\begin{figure}
\includegraphics[width=.8\textwidth]{figures/Figure 37.png}
\caption{\label{fig:3:37} F. Müller’s Awabakal case paradigms (\citeyear[7]{muller_grundris_1882})}
\end{figure}


\subsection{The structure of pronouns in peripheral cases}
\label{sec:key:3.3.3.1}

Threlkeld declined singular, dual and plural Awabakal pronouns for each person in nine cases. He supplemented these paradigmatic presentations with syntagmatic statements. After providing a full paradigm for first-person inclusive dual pronouns (\citeyear[23]{threlkeld_australian_1834}; \figref{fig:3:38}) he wrote (ibid.: 23): “It will be perceived that the particles form the accusative into the other cases”. Here he described how the peripheral case forms, i.e., his numbered “ablative” cases, were built through attachment of case suffixes – termed “particles” – to the accusative pronominal stem. This type of statement is not characteristically found in the description of the fusional structures of SAE language (\sectref{sec:key:2.4}) and represents an innovative descriptive response to agglutinative morphology. Threlkeld’s explanation that it was the accusative pronominal stem to which inflections for peripheral cases attached was followed by \citet[8]{teichelmann_outlines_1840}, who also attempted to account for similar pronominal structures in Kaurna.

\begin{figure}
\includegraphics[width=.8\textwidth]{figures/Figure 38.png}
\caption{\label{fig:3:38} Threlkeld’s presentation of 1dl pronouns (\citeyear[23]{threlkeld_australian_1834}; Awabakal)}
\end{figure}


Threlkeld’s ability to conceive that suffixes marking peripheral case functions on pronouns were part of the morphological case system of Awabakal is sophisticated in comparison to some later grammarians. \citet[71]{capell_structure_1937}, for example, followed most earlier accounts of pronouns in Arrernte (\sectref{sec:key:9.3.3}) in conceiving of suffixes marking functions that are not carried by the morphological case systems of SAE languages as ``pre/post-positions''. He wrote: “In some languages they [postpositions] may themselves require a case other than the nominative, just as in the Classical languages”. He gave this example in order to “show how a post-position may be used with a special case” \citep[71]{capell_structure_1937}:

\protectedex{
\begin{enumerate}
	\item[] \hphantom{"}\textbf{era}, he;
	\item[] \hphantom{"}\textbf{ekura}, of him;
	\item[] \hphantom{"}\textbf{ekura gata}, with him
\end{enumerate}
}

Capell’s \textbf{ekura gata} \textit{ikwerekerte} is the 3SG pronoun in proprietive case, with comitative function. In Arrernte the dative forms of the pronoun, here 3SGDAT \textit{ikwere}, is the stem to which the peripheral case suffix \textit{-kerte} attaches. Teichelmann \& Schürmann’s (\citeyear[8]{teichelmann_outlines_1840}) presentation of Kaurna pronouns in comitative case as single words, for example, 3sgCOM \textbf{padlaityangga} better represents the forms than does Capell’s presentation of pronouns with similar function in Arrernte, made a century later. 

\subsection{Prepositions}
\label{sec:key:3.3.4}

Under the word-class heading “preposition’, \citet[77]{threlkeld_australian_1834} listed case suffixes carrying functions that are outside the Latin case inventory and not associated with the case systems of SAE languages \textit{again} (\figref{fig:3:39}), despite already including \textit{most} case suffixes in the nominal case paradigms (\citeyear[13--16]{threlkeld_australian_1834}; \figref{fig:3:34}). The only suffixes which are reclaimed as case markers, but which are not included paradigmatically by Threlkeld and are described by Threlkeld only as “prepositions” are the comitative form \textit{–kaTuwa} and the suffix \textit{–kulang} marking `motion towards', currently termed allative 2 \citep[30]{lissarrague_salvage_2006}.

\begin{figure}
\includegraphics[width=.8\textwidth]{figures/Figure 39.png}
\caption{\label{fig:3:39} Threlkeld’s listing of prepositions \citeyearpar[77]{threlkeld_australian_1834} }
\end{figure}


Observe here briefly the note below the allomorphs of the ergative and instrumental cases (\figref{fig:3:39}), which according to Threlkeld are marked with the same forms on all nominal types, as occurs in many Australian languages \citep[135]{dixon_australian_2002}. Threlkeld (\citeyear[77]{threlkeld_australian_1834}) wrote: “Expressed in English only when instrumental by the particles with, by, for”. The comment indicates that Threlkeld was aware that these forms marked two distinct case functions. When marking instrumental function, he translated the form with an English prepositional phrase, while noting that such a translation was not appropriate for a nominal with the same shape when functioning as the agent of a transitive verb. The comment shows considerable insight, and a penetrating description of ergative and instrumental syncretism, in comparison with some later grammarians, notably Reuther (\sectref{sec:key:8.5.2.3}). The understanding was not shared by all later grammarians, notably H. A. E. Meyer (\sectref{sec:key:6.1.2}), who translated nominals in ergative case as “by X”.

The discussion of inflectional case morphology under the word-class “preposition” resulted from \textit{functional} rather than \textit{structural analogy} with SAE prepositions. PN case forms carrying functions that were carried by a prepositional phrase in SAE languages were commonly described as “prepositions” in the corpus grammars. That some PN case morphology was discussed under the heading “case” and other case morphology was discussed as “prepositions” reflected the particular way in which SAE case systems are in part synthetic and are in part analytic. 


\begin{table}
\small
    \begin{tabularx}{\textwidth}{QQQQ}
        \lsptoprule
        Prepositions given by Threlkeld (1834: 77) & Translated by Threlkeld as: & Reclaimed form (Lissarrague 2006) & Function (Lissarrague 2006)  \\
        \midrule
        \textbf{ba}	& of (pronouns)	& \textit{-pa} & Possessive suffix on pronouns \\
        \textbf{ko-ba} &	of (nouns) & \textit{-kupa} &	Possessive suffix on nouns \\
        \textbf{ku}	& part of &	\textit{?} &	? \\
        \textbf{bi-rang} &	of, out of, from &	\textit{-pirang}	 & Ablative case suffix \\
        \textbf{ko-láng} &	to, towards	& \textit{-kulang} &	Allative 2 “Motion towards” \\
        \textbf{tin} &	from, on account of, for, because of, in consequence of (nouns) &	\textit{-Tin} &	Causative case suffix on nouns \\
        \textbf{kai} &	from, on account of, for, because of, in consequence of (pronouns) &	\textit{-kay} &	Causative case suffix on pronouns \\
        \textbf{ko, lo, o, ro, to} &	particles, denoting agency or instrumentality &	\textit{-ku, -lu, -u, -ru, -Tu} &	Ergative and Instrumental case suffix \\
        \textbf{ka-to-a} &	with &	\textit{-kaTuwa}	& Comitative case suffix \\
        \textbf{ka}	& in (denotes time) & \textit{-ka}	 & Locative case suffix (Variant) \\
        \textbf{ka ba} &	in, on, at & \textit{-kapa} & Locative case suffix \\
        \textbf{mur-ra-ring} &	into &	\textit{?} &	? \\
        \textbf{mur-rung} &	within &	\textit{marrang}	& Locational word, “within” \\
        \textbf{war-rai} &	outside	& \textit{?}	& ? \\
        \lspbottomrule
    \end{tabularx}
    \caption{The reclaimed function of morphemes listed as prepositions by Threlkeld}
    \label{tab:chap3:morphemes}
	\label{fig:3:40}
\end{table}

Like other subsequent grammarians (\citealt[492]{hale_languages_1846}, \citealt[27]{roth_structure_1901}, \citealt[19]{hey_elementary_1903}, \citealt[8]{taplin_grammar_1880}, \citealt[13--17]{meyer_vocabulary_1843}), Threlkeld maintained the term ``preposition'', as opposed to “postposition”, to describe PN suffixing case morphology. \tabref{tab:chap3:morphemes} shows the reclaimed function of the morphemes Threlkeld lists as ``prepositions''. Of the fifteen forms Threlkeld presented as ``prepositions'', ten mark case (\figref{fig:3:31}). Some forms are described as occurring only on pronouns and some only on nouns. The last three forms are translated as “into”, “within” and “outside”. The presentation of such “locational qualifiers” (\citealt[282]{dixon_languages_1980}) as “prepositions” is common to the corpus, occurring in descriptions of Kaurna (\sectref{sec:key:5.3.2}), Diyari (\sectref{sec:key:8.7.1}), and Arrernte (\sectref{sec:key:9.3.3.1},) and is also observed in Mathews' grammars (\citealt[204]{koch_r_2008}).

Different corpus grammarians classified morphology included under the heading “pre/post-positions” in ways that are diagnostic of the descriptive school to which they belonged. Threlkeld’s treatment of case suffixes and locational qualifiers as “pre/post-positions” differs from practices later initiated by Teichelmann \& Schürmann (\citeyear{teichelmann_outlines_1840}; \sectref{sec:key:5.3.2}) and by Roth (\sectref{sec:key:10.1.1}).


\largerpage
\subsection{Declension classes}
\label{sec:key:3.3.5}

Like many PN languages, Awabakal case suffixes undergo morphophonemic alternation. The recorded historical forms have been used by \citet[26]{lissarrague_salvage_2006} in her reclamation efforts; see \tabref{tab:chap3:MorphophonemicAwabakal}.

\begin{table}
    \begin{tabularx}{\textwidth}{Qllllll}
        \lsptoprule
         & Erg./Inst. & Perl. & Loc. & All. & Abl. & Caus.  \\
        \midrule
        Following a stem-final vowel or velar nasal	& \textit{-ku} &	\textit{-kuwa} &	\textit{-kapa (-ka)}	& \textit{-kaku}	& \textit{-kapirang}	& \textit{-Tin} \\
        \tablevspace
        Following a stem-final liquid &	\textit{-u}	& \textit{-uwa} & \textit{-apa \newline (-a)} & \textit{-aku} & \textit{-apirang} & \textit{-iN} \\
        \tablevspace
        Following a stem-final palatal nasal or semi- vowel	& \textit{-tju} & \textit{-tjuwa}	& \textit{-tjapa (-tja)}	& \textit{-tjaku} & \textit{-tjapirng} & \textit{-TjiN} \\
        \tablevspace
        Following a stem-final alveolar nasal &	\textit{-tu}	& \textit{-tuwa} & \textit{-tapa (-ta)} &	\textit{-taku} &	\textit{-tapirang} &	\textit{-tiN} \\
        \lspbottomrule
    \end{tabularx}
    \caption{Morphophonemic alternation to case suffixes in Awabakal (from \citealt[26]{lissarrague_salvage_2006})}
    \label{tab:chap3:MorphophonemicAwabakal}
\end{table}

Awabakal case marking is sensitive to animacy distinctions for the syntactic cases marking A, S and O, as well as for some of the peripheral cases. Compare, for example, the different form of the genitive suffix on proper and common nouns in the following clauses. In Example \ref{ex:3:3-21} the interrogative pronoun referring to a person is marked with the possessive suffix \textit{–ampa}. In the second clause the interrogative pronoun referring to a non-animate entity is differently marked for the same function, with \textit{–kupa}.

\ea\label{ex:3:3-21}
\glll Ngan-umba       unni      	      wonnai?             Biriban-umba       unni   	               wonnai \\
ngan-ampa       aNi         	      waNay               Pirapan-ampa     aNi      	waNay \\
\stockcaps{INTER}-\stockcaps{POSS}   this-[\stockcaps{NOM}]    child-[\stockcaps{NOM}]     Pirapan-\stockcaps{POSS}      this-[\stockcaps{NOM}]       child-[\stockcaps{NOM}] \\
\glt Whose child is this? Biriban’s, this child. \\
(Transcription \citealt[42]{lissarrague_salvage_2006}) 
\z

\ea\label{ex:3:3-22}
\glll Minnaring kopa	 unni?  \\
minyaring -kupa 	 aNi  \\
\stockcaps{INTER}-\stockcaps{POSS}	 this-[\stockcaps{NOM}] \\
\glt What does this belong to? \\
(Transcription \citealt[51]{lissarrague_salvage_2006})
\z

Lissarrague reconstitutes a situation in which “pronouns, proper nouns and some nouns which refer to people use the nominative/accusative pattern and common nominals use the ergative/absolutive pattern. The personal interrogative/indefinite \textit{ngaN} `who' uses ``tripartite marking'' (\citealt[26]{lissarrague_salvage_2006}; \tabref{tab:3:42}). However, \citet[39]{lissarrague_salvage_2006} explains that there is “some discrepancy in the data with human nouns”. Elsewhere (ibid.: 42), Lissarrague describes how “proper nouns and human nouns are also marked for accusative case” with \textit{–Nang}, which also marks the tripartite personal interrogative as accusative.

\begin{table}
	\centering
	\begin{tabular}{lll}
		\lsptoprule
		Ergative & Accusative & Tripartite  \\
		alignment & alignment & marking \\
		\textsc{a/so} & \textsc{as/o}  & \textsc{a/s/o} \\
		\midrule
		Common nouns & Pronouns & Personal \\
									 & 					 & interrogative \\
									 & \multicolumn{2}{c}{? Proper nouns and some} \\
									 & \multicolumn{2}{c}{nouns referring to people ?}  \\
		\lspbottomrule
	\end{tabular}
	\caption{The syntactic alignment of nominal types in Awabakal (following \citealt[26]{lissarrague_salvage_2006})}
	\label{tab:3:42}
\end{table}


Threlkeld adapted the traditional descriptive framework to accommodate both morphophonemic alternation and sensitivity to animacy. 

Latin nominal morphology is conceived of as having five noun classes, traditionally called declensions, which each mark case differently. Threlkeld presented Awabakal as having “Seven Declensions of Nouns … declined according to their use \textit{and} termination” (emphasis added; \citeyear[10]{threlkeld_australian_1834}). Two declension classes were motivated by different case marking on types of nominals and five by morphophonemic alternation. Threlkeld explained: 
\begin{quote}
    When used for the name of an individual person, they are declined in the 1st declension, whatever may be the termination of the word; but when used as the name of a place they are declined in the 7th Declension. \citep[10]{threlkeld_australian_1834}
\end{quote}

Threlkeld’s presentation is especially skilful. He exemplified the sensitivity of Awabakal case marking to animacy using ``Biraban'', the name of his Aboriginal friend and main linguistic informant. He explained that since his friend’s name also “means the bird called Eagle-hawk … it must be declined in the second declension” (\citealt[13]{threlkeld_australian_1834}). Threlkeld juxtaposed the case marking for the noun \textit{biraban}, used as proper noun (declension 1) and as a common noun (declension 2; \figref{fig:3:43}). This presentation of nominal declensions allowed Threlkeld to show that the accusative case takes zero marking on common nouns and is marked with the suffix \textit{–(N)ang} on proper nouns (\citealt[10]{threlkeld_australian_1834}; \citealt[42]{Lissarrague2006}) and that possessive function is marked differently on common and proper nouns in Awabakal.

Threlkeld’s ability to convey the different marking of lower and higher animate nouns in Awabakal through the presentation of declension classes was effective compared with the extent to which many later grammarians dealt with the different markings for the same case function on different types of nouns (see, e.g., \sectref{sec:key:8.6.3}). Threlkeld’s adeptness in this regard recurs only in Günther (\citeyear{gunther_native_1838}; \citeyear{gunther_lecture_1840}). The marking of case on different noun-types tended to confuse early grammarians.

\begin{figure}
\includegraphics[width=.8\textwidth]{figures/Figure 43.png}
\caption{Threlkeld’s first two declension classes (\citeyear[13]{threlkeld_australian_1834})}
\label{fig:3:43}
\label{fig:3.11}
\end{figure}

After first explaining that the inflection for ergative case altered “perhaps merely to coalesce readily in pronunciation” \citep[6]{threlkeld_australian_1834}), Threlkeld justified the remaining five declension classes by stating rules governing the form of ``the particle of agency'', i.e. ergative inflection (ibid.: 11; \figref{fig:3:44}). Rules 4 and 5 discuss how allomorphy was determined by word size, a feature which is common to PN languages \citep[152]{baker_word_2014}, but otherwise not recognised in the corpus grammars.

Not all of the detail regarding complex allomorphy, to which Threlkeld alludes, has been properly understood. The distinct case marking of place names, allomorphic sensitivity to word size and the exact nature of the animacy divide have not been reclaimed.

\begin{figure}
\includegraphics[width=.8\textwidth]{figures/Figure 44.png}
\caption{\label{fig:3:44}\label{fig:3.12} Threlkeld’s rules governing ergative allomorphy  (\citeyear[11]{threlkeld_australian_1834})}
\end{figure}

\subsection{Pronouns}
\label{sec:key:3.3.6}

When describing Awabakal pronouns, Threlkeld appears to have encountered a wealth of complexity, the likes of which were not presented to other early grammarians of Australian languages. In deconstructing Threlkeld’s analysis of bound pronouns (\sectref{sec:key:3.3.6.1}) and of phrasal A + O pronominal sequences (\sectref{sec:key:3.3.6.2}), the following sections present analyses of the systems that Threlkeld strove to describe, which differ from other modern analyses of the available data.

\subsubsection{Bound pronouns}
\label{sec:key:3.3.6.1}

\citet{threlkeld_australian_1834} was the earliest of a group of corpus grammarians (\citealt{gunther_native_1838}; \citealt{gunther_lecture_1840}; \citealt{teichelmann_outlines_1840}; \citealt{meyer_vocabulary_1843}; \citealt{schurmann_vocabulary_1844}; \citealt{hale_languages_1846}; \citealt{ridley_kamilaroi_1866}; \citealt{ridley_kamilaroi_1875}) who were presented with the additional descriptive challenge of having to account for bound pronouns. These grammarians described languages covering a continuous bloc of the southeast of the continent that exhibit sets of bound personal pronouns \citep[337--401]{dixon_australian_2002}. Note that no distinction is made here between suffixing and clitic forms \citep[353]{dixon_australian_2002}. Pronouns of either type are here referred to as ``bound pronouns''.

Bound pronouns generally mark the core syntactic arguments A, S and O, and less frequently the dative case. Systems of bound pronouns differ between languages in terms of both function and syntactic constraints. It is probable that no two systems encountered by the early grammarians were identical, although with limited data it is hard to tell how similar the Gamilaraay system was to that of the related Central New South Wales language Wiradjuri, for example, or the Barngarla system was to the system in the related Thura-Yura language Kaurna. The cases for which bound pronouns are marked, their positioning within the clause, the parts of speech to which they can attach, their status as either obligatory or optional, their ordering relative to one another, and the way in which the systems interrelate with the marking of other grammatical categories are language-specific. Different types of systems resulted in different types of treatment in the corpus grammars, not because grammarians were aware of all the linguistic parameters, but because of the way particular systems drew their attention. 

Usually, the bound form is a truncation of the free-form pronoun having lost initial C(V). In Kaurna, for example, the 1sgERG free-form is \textit{ngathu}, and its bound form counterpart is \textit{–athu}. Awabakal is the only corpus language exhibiting bound pronouns in which most bound forms are \textit{not} transparently related to that of a free-form pronoun.

Some Awabakal free and bound pronouns are closely related to the free-form pronouns in the neighbouring language Gathang (\citealt[61--63]{holmer_attempt_1966}; \citealt[62--72]{lissarrague_grammar_2010}; \tabref{tab:chap3:AwabakalGathang}). Gathang has no bound pronouns in core arguments (\citealt[356]{dixon_australian_2002}) and free pronouns are accusatively aligned (\stockcaps{AS/O}). Some bound pronouns and some free-form pronouns in Awabakal resemble Gathang pronouns. \citet[356--357]{dixon_australian_2002} writes that “at some time in the past there must have been borrowing of pronominal forms and possible merging of paradigms [between the two languages]”.

Although Threlkeld represented the forms, which are reclaimed as bound pronouns, as free-standing words in pronominal paradigms, he did represent them as attached morphemes in the imperative mood of the verb \citep[51]{threlkeld_australian_1834} in a schema of the traditional framework in which bound pronouns were commonly illustrated in the corpus grammars.

\begin{table}
    \begin{tabularx}{\textwidth}{llll}
        \lsptoprule
        & \multicolumn{2}{c}{Awabakal} & Gathang \\
        & \multicolumn{2}{c}{\citep{lissarrague_salvage_2006}}	& \citep{lissarrague_grammar_2010} \\
        \midrule
        & Free Pronouns &	Bound Pronouns	& Free pronouns \\
        \midrule
        1sg\stockcaps{AS} &	ngatjuwa &	-pang &	ngathuwa, ngatha \\
        1sgO &	amuwang & -tja	& barrangang \\
        2sg\stockcaps{AS} &	ngiNtuwa &	-pi	& biyay \\
        2sgO &	Ngiruwang &	-piN &	biinang, biyangang \\
        3sg\stockcaps{AS}. M &	Nyuawa &	-nyuwa &	nyuwa \\
        3sgO. M &	Ngikuwang &	-puN &	nyuguwang, nyuguwangang \\
        3sg\stockcaps{AS}. F	& puwa\stockcaps{NT}uwa &	\stockcaps{NO} \stockcaps{FORM}	& nyunda, nyunduwa \\
        3sgO. F	& puwaNuwaN	& -NuwaN &	nyun-gung \\
        \lspbottomrule
    \end{tabularx}
    \caption{Awabakal and Gathang pronouns (\citealt[62--72]{lissarrague_grammar_2010}) }
    \label{tab:chap3:AwabakalGathang}
\end{table}

It is not clear whether all Awabakal bound pronouns followed the same accusatively aligned system of the free-form pronouns (AS/O) or that exactly the same ranges of case functions were carried by free and bound form pronouns. Awabakal is the only language considered in this study in which bound forms \textit{may} not all have followed the system of marking syntactic case exhibited by free-form pronouns.

There are different views put forward by analysts as to what Threlkeld’s representation tells us about the syntactic alignment of bound pronouns in Awabakal (\citealt{capell_structure_1937}; \citealt{oppliger_phonology_1984}; \citealt{lissarrague_salvage_2006}; \citealt{dixon_australian_2002}). It is possible that Threlkeld’s confusing representations of pronouns reflects a now irretrievable asymmetrical marking of the case function on free and bound pronouns in different number, person and gender.

All sources agree that there were no non-singular bound pronouns, and no 3sgF bound form.

Threlkeld is the only corpus grammarian who placed free and what are currently analysed as bound pronouns in different positions of the same pronominal paradigm, i.e., some cases in the paradigm are shown to be carried by a free-form pronouns and others only by a bound pronoun (\figref{fig:3:46}).
 
\begin{figure}
\includegraphics[width=.8\textwidth]{figures/Figure 46a.png}
\includegraphics[width=.8\textwidth]{figures/Figure 46b.png}
\caption{\label{fig:3:46} Threlkeld’s pronominal case paradigm \citeyearpar[19--20]{threlkeld_australian_1834} }
\end{figure}


The form \textit{-tja}, which is reclaimed as the \textit{bound} 1sgACC pronoun, is shown as the sole 1sg accusative form. The form \textit{amuwang}, which Lissarrague reclaims as the \textit{free} 1sgACC pronoun (\citeyear[46]{lissarrague_salvage_2006}), is shown by Threlkeld as the 1sg pronoun standing in the case called `dative 1', which was translated `for X', and which is reclaimed as marking the dative case on nouns (\tabref{tab:chap3:Awabakal}). Compare Lissarrague’s analysis of the forms \textit{amuwang}  and \textit{–tja} in the following example (\ref{ex:3:23}) with the labels assigned to the same forms in Threlkeld’s paradigm (\figref{fig:3:46}):

\ea\label{ex:3:23}
\glll {Ammoung be }	{weah-lah.}	{Weah-lah be teah} \\
Amuwang-pi 	wiya-la.          	wiya-la-pi-tja \\
1\textsc{sg}.\stockcaps{ACC}-2\textsc{sg}.\stockcaps{NOM} 	speak-\stockcaps{IMP} 	speak-\stockcaps{IMP}-2\textsc{sg}.\stockcaps{NOM}-1\textsc{sg}.\stockcaps{ACC} \\
\glt `Speak to me. Do tell me' \\
(Transcription \citealt[46]{lissarrague_salvage_2006}) 
\z

See also that the forms Lissarrague reclaims as 1sgA/S free pronoun \textit{ngatjuwa} and 1sgA/S bound pronoun \textit{-pang} (\tabref{tab:chap3:AwabakalGathang}) are shown in Threlkeld’s paradigm (\figref{fig:3:46}) as marking different cases. \textit{Ngatjuwa} is labelled `nominative 1' (nominative) and is placed in the first position, and \textit{pang} is labelled `nominative 2' (ergative) and is placed in the second position.

It is not clear whether Threlkeld’s presentation of bound and free-forms in the same paradigm resulted from his failure to understand that there were two pronominal systems of marking case – which may have been differently aligned from one another – or whether it reflected a situation in which free and bound forms were part of a single pronominal case system. 



\largerpage
Threlkeld described the two variants of the 1sgNOM pronoun \textit{ngatjuwa} and \textit{pang} partly in terms of a distinction between bound and free-forms. He invented new terminology to describe this unfamiliar phenomenon. In his introduction to pronominal paradigms he described the free and bound “nominative” forms, which are placed in the first two positions in the paradigm (\figref{fig:3:47}).
\begin{figure}
\includegraphics[width=.7\textwidth]{figures/Figure 47.png}
\caption{\label{fig:3:47} Threlkeld’s account of the “personal nominative” and the “verbal nominative” (\citeyear[18]{threlkeld_australian_1834})}
\end{figure}


Free-forms were labelled ``personal nominative'' – abbreviated ``P. N.'' and described as the ``primitive'' forms. They were said to be ``used by themselves, in answer to the Interrogative or emphatically with the Verb''. Bound pronouns were labelled ``verbal nominative'' – abbreviated ``V. N.'' and were ``never by themselves nor in answer to the Interrogatives''. Lissarrague’s interpretation (\citeyear[39--46]{lissarrague_salvage_2006}) (\figref{tab:chap3:AwabakalGathang}) that both free and bound forms were ergatively aligned assumes that Threlkeld’s distinction between ``P. N.'' and ``V. N.'' related only to free and bound form distinction and not to a case distinction.

Threlkeld provided further explanation of the functions of the forms labelled P. N. and V. N. in notes included within the paradigm (\figref{fig:3:46}), explaining that the form labelled ``P. N.'', i.e., a free-form pronoun, is given “in answer to the interrogative of personal agency”, and that the form labelled V. N., i.e., the bound form, is given “in answer to an interrogative of the act”. Here his explanation \textit{might} be seen to be distinguishing the forms on the basis of case, the free-form standing in ergative case and the bound form standing in nominative case. This is Oppliger’s interpretation (\citeyear[64--69]{oppliger_phonology_1984}). She describes a situation in which free-forms were ergatively aligned (A/SO) and bound forms accusatively aligned (AS/O).

In another section of the grammar, Threlkeld used the term V. N. explicitly to refer to a case distinction. Of his “neuter” 3\textsc{sg} pronouns, i.e., demonstrative forms (\sectref{sec:key:3.3.2}), he wrote: ``[T]hey govern the verbal nominative pronoun and not the Nominative 1 [i.e., nominative case]'' (\citeyear[21]{threlkeld_australian_1834}). This juxtaposition of the term “V. N.” with the “nominative 1” (nominative) case suggests that the term implied overt marking for ergative case. Threlkeld appears to be explaining that these demonstrative forms showed ergative or tripartite alignment, a situation recounted by Hale (\citeyear[490--491]{hale_languages_1846}; see also \citealt[74--75]{oppliger_phonology_1984}).

Observe also that the “V. N.” forms are placed in the second paradigmatic position (\figref{fig:3:46}), where Threlkeld placed his “nominative 2” (ergative) forms, and the “P. N.” forms are placed in the first position, where Threlkeld placed nominative case forms (\sectref{sec:key:3.3.7}), further suggesting that the forms were distinguished on the basis of case.
Interestingly, the earliest readings of Threlkeld (\citealt[488]{hale_languages_1846}; \citealt[251]{muller_reise_1867}) interpreted the difference between Threlkeld’s “personal nominative” and “verbal nominative” as relating to both a case distinction and a bound/free distinction. \citet[251]{muller_reise_1867} described the Awabakal ``nominativ subjectiv'' (ergative) pronoun (\figref{fig:3:48}) as only occurring ``with verbs for example, \textit{tatan-paň} `I eat' ''.
\begin{figure}
\includegraphics[width=.8\textwidth]{figures/Figure 48.png}
\caption{\label{fig:3:48} F. Müller’s Awabakal pronominal case paradigm (\citeyear[251]{muller_reise_1867})}
\end{figure}

\newpage
\citet[488]{hale_languages_1846} and \citet[351,395]{dixon_australian_2002} assert that the language had no free-form singular accusative pronouns (\tabref{tab:chap3:Dixon}).

\begin{table}
    \begin{tabularx}{\textwidth}{lll}
        \lsptoprule
        & Singular Pronouns & Non-Singular Pronouns:  \\
        \midrule
        S \& A & \stockcaps{FREE} \& \stockcaps{BOUND} \stockcaps{FORMS}	& \stockcaps{ONLY} \stockcaps{FREE}-\stockcaps{FORMS} \\
        O &	\stockcaps{ONLY} \stockcaps{BOUND} \stockcaps{FORMS} &	\stockcaps{ONLY} \stockcaps{FREE}-\stockcaps{FORMS} \\
        \lspbottomrule
    \end{tabularx}
    \caption{Dixon's analysis of Awabakal free and bound form pronouns (\citeyear[351]{dixon_australian_2002})}
    \label{tab:chap3:Dixon}
\end{table}

This conclusion rests upon the assumption that accusative function could only be marked by the bound accusative forms \textit{tia} (1sgO) \textit{bin} (2sgO) and \textit{bon} (3sgmO) (AS/O) described by Threlkeld as “the object of transitive verbs” (\citeyear[20]{threlkeld_australian_1834}) and called ``accusative''. Dixon’s analysis implies that the free-form pronouns, which are \textit{reclaimed} as marking the accusative case (\citealt{oppliger_phonology_1984}; \citealt{lissarrague_salvage_2006}) – \textit{amuwang} (1sgO) \textit{ngiruwang} (2sgO) and \textit{ngikuwang} (3sgmO) – mark a different case. Threlkeld assigned the label “dative 1” (1834: 19--22) to these forms. 

The scenario proposed by Dixon is, however, unlikely. In Example \ref{ex:3:23}, it is possible to treat the 1sg free-form \textit{amuwang} as dative, marking the addressee of the verb “to speak” rather than as accusative (see, e.g., \citealt[179]{wilkins_mparntwe_1989}). But in the following clause, the free-form \textit{amuwang} and the bound form \textit{tja} appear to be marking the same accusative argument.

\ea\label{ex:3:24}
\glll {Karai tia}      		 nguwa   	 emmoung    	 {takilli ko} \\
Karay-tja  		 ngu-wa  	 amuwang             tjaki-li-ku \\
Flesh-[\stockcaps{ACC}]-1\textsc{sg}.\stockcaps{ACC}	 give-\stockcaps{IMP} 	1\textsc{sg}.\stockcaps{ACC}		 eat-\stockcaps{NMSR}-\stockcaps{PURP} \\
\glt `Give me flesh to eat' \\
(Gloss and transcription from \citealt[42]{lissarrague_salvage_2006})
\z

An accusative argument frequently marks the second argument of a ditransitive verb “to give” in PN languages (\citealt[35--36]{blake_case_1977}; \citealt[2]{schebeck_adnjamathanha_1973}; \citealt[169]{wilkins_mparntwe_1989}; \citealt[294]{henderson_topics_2013}; \citealt[75]{hercus_grammar_1999}). \citet[42]{lissarrague_salvage_2006} explains: “In a sentence with two objects, a pronoun with dative function is in accusative case”. This example suggests that Threlkeld failed to see that there were two formally distinct ways of marking the same case function. 

\citet[56]{capell_structure_1937} was of the view that Threlkeld’s paradigm represented a single pronominal system, rather than an inadvertent conflation of a bound system and a free system into a single paradigm. Writing in the early middle descriptive era, when understanding of systems of bound pronouns in PN languages remained largely undeveloped (\sectref{sec:key:6.3}), he took Threlkeld’s paradigms at face value, believing that the case forms of Awabakal pronouns were highly irregular, and described Awabakal as unusual among Australian languages in “subject[ing] the pronoun to a real declension” (see \sectref{sec:key:2.4.1.1}).

That Threlkeld’s discussion of bound pronouns in Awabakal is opaque and has confused contemporary reclamations is not surprising given the complexity this earliest Australian grammarian tackled. The system of bound pronouns that Threlkeld described in Awabakal appears to have interacted with the marking of the syntactic cases in a way that is atypical of what is known about systems in other languages described by corpus grammarians. Threlkeld’s convoluted account of what was a complicated system consequently provided little assistance to later corpus grammarians, and his terminology “personal nominative” and “verbal nominative” fell into disuse. 

\subsubsection{The ``conjoined dual case''}
\label{sec:key:3.3.6.2}

Threlkeld presented a partial paradigm of A + O pronominal sequences in Awabakal, which he termed the “conjoined dual case” (\figref{fig:3:50}). The forms were given after presentation of the dual pronouns and before the plural.
\begin{figure}
\includegraphics[width=.8\textwidth]{figures/Figure 50.png}
\caption{\label{fig:3:50} Threlkeld’s “conjoined dual case” (\citeyear[24]{threlkeld_australian_1834})}
\end{figure}


The forms Threlkeld presented are compound pronouns, which he described as “governed by the active transitive verbs” \citep[24]{threlkeld_australian_1834}. Each compound pronoun comprises a constituent acting as the agent and a constituent acting as the object of the same clause, or as \citet[152]{elkin_nature_1937} put it: “a combination of pronoun in the nominative and accusative cases”. For example: 

\ea\label{ex:3:25}
\glll Minnung 	bunnun 	        ngaiya 	biloa?  \\
minyang 	wupa-NaN    ngaya 	piN-luwa \\
\textsc{interr:abs}	do-\textsc{fut} 	      then 	2\textsc{sg}.\textsc{acc}-3.\textsc{sg}.\textsc{nom}.\textsc{m} \\
\glt `What will he do to you?' \\
(Gloss and transcription from \citealt[42]{lissarrague_salvage_2006})
\z

The ordering of constituents relative to one another is determined by pronominal person, regardless of case. The order is 1 > 2 > 3. All constituents are singular. Lissarague writes:

\begin{quote}
    Compound pronouns follow verbs, conjunctions, negations, interrogative particles, interrogatives and even interjections. It is not understood when a conjoined form is used instead of the usual bound pronoun forms. It is not known if compound pronouns stand alone, or if they are phonetically attached to the word they follow … Threlkeld represents them as both a phonetic part of the preceding word and as elements that stand alone. Only the compounds which appear in the above paradigm [reproduced in \figref{fig:3:51}] are permitted. \citep[47--48]{lissarrague_salvage_2006}
\end{quote}

Lissarrague’s analysis of the forms (\figref{fig:3:51}) updates Threlkeld’s phonological representation and shows that the first constituent of the compound form is related to the Awabakal bound pronouns and the second constituent to the Awabakal free pronouns (\tabref{tab:chap3:AwabakalGathang}).

However, it is possible to speculate that the first constituent takes the form of the corresponding bound pronoun. In this scenario (\tabref{tab:chap3:ReclamationThrelkeld}), the placement of the morpheme boundary differs from that suggested by Threlkeld and reproduced by \citet[47]{lissarrague_salvage_2006}. Here the first constituent is an entire bound pronoun (column 4, \tabref{tab:chap3:ReclamationThrelkeld}) rather than a form that “resembles the first syllable of bound forms” (\citeyear[47]{lissarrague_salvage_2006}). The first constituent is either a 1sgA, a 2sgA or a 2sgO bound form.

The second constituent is the final phonological segment or segments of the free-form pronouns, 2sgO, 3sgfO, 3sgmO, 3sgfA or 3sgmA (column 6). Only one of these segments, 3sgfO \textit{–NuwaN}, is the same form as the equivalent bound pronoun that occurs freely in other environments. Each other second-constituent bound form does not occur elsewhere.

Allowing for the reduction of clusters of nasals with unknown places of articulation, represented as ``N'', across the boundary between the two constituents in the “I her” form – which may result from a recording error – as well as for the rounding of the final vowel in the second 2sgO constituent of the “I thee” form, this analysis works reasonably well. Note, however, that the forms \textit{pi-\textbf{N}ang} “thou him” and \textit{pi-\textbf{L}awa} “he thee” show phonological variation at the morpheme boundary (column 5) and cannot be fully accounted for by this alternative analysis.

\begin{figure}[t]
\includegraphics[width=.8\textwidth]{figures/Figure 51.png}
\caption{Lissarrague's (\citeyear[47]{lissarrague_salvage_2006}) analysis of compound pronouns}
\label{fig:3:51} 
\end{figure}

	 
\begin{table}
\footnotesize
	\begin{tabularx}{\textwidth}{QQQQQQ}
		\lsptoprule
		\citet[24]{Threlkeld1834} & \citet[47]{Lissarrague2006} & Alternative structure & 1St constituent Bound pronoun & Phonological changes required & 2nd constituent Last phonological segment(s) of free pronoun:  \\
		\midrule
		Ba-núng “I, thee”	& Pa-Nung	& Pang-wang	& pang- 1sgA	& Rounding of second vowel	& Ngiru-wang 2sgO \\
		\tablevspace
		Bá-nó-un “I her”	& Pa-NuwaN	& Pang-NuwaN	& pang 1sgA	& Reduction of nasal + nasal cluster	& puwa-NuwaN 3sgfO \\
		\tablevspace
		Bi-núng “Thou, him”	& Pi-nyung	& Pi-Nang & pi- 2sgA	& Unexplained additional nasal	& Ngikuw-ang 3sgmO \\
		\tablevspace
		Bi-nó-un “Thou her”	& Pi-Nuwan	& Pi-Nuwan	& pi- 2sgA	& None	& Puwa-NuwaN 3sg\stockcaps{FO} \\
		\tablevspace
		Bi-ló-a “he, thee”	& piN-luwa	& Pi-lawa	& Pi(N)- 2sgO& 	Unexplained lost nasal and additional lateral	& Nguw-awa 3sgmA \\
		\tablevspace
		Bín-tó-a “She, thee” & piN-tuwa	& piN-Tuwa	& piN- 2sgO	& None	& PuwaN-Tuwa 3sgfA \\
		\lspbottomrule
	\end{tabularx}
	\legendbox{The analysis offers an alternative to that given by \citet{lissarrague_salvage_2006}. See also \citet{koch_george_2011} for an historical reconstruction of compound, free and bound pronouns in Awabakal and related languages. Note that the pronouns that are shown as either nominative or ergative in \tabref{tab:chap3:AwabakalGathang} are shown here as agents because these compounds always constitute both arguments of a transitive verb. Note that the form of compound pronouns proposed here (column 3) follows \citegen[40--41]{lissarrague_salvage_2006} reclamation of the vowels in free-form pronouns.}
	\caption{Reclamation of compound A and O pronominal sequences described by Threlkeld as the “conjoined dual case’}
	\label{tab:chap3:ReclamationThrelkeld}
\end{table}

Contrary to \citet[48]{lissarrague_salvage_2006}, Threlkeld did not describe the forms in which the first constituent is a second-person accusative bound pronoun “in the wrong order”. The translation into English of the two forms of which the first constituent is in accusative case, “He, thee” and “She, thee” (\tabref{tab:chap3:ReclamationThrelkeld}), is correct. Threlkeld’s free translations \textit{do not} use English word order to convey the case frame of the compound. These are not interlinear translations. That Threlkeld was aware that the first constituent of each form was an accusative second-person pronoun is evident by his translation using “thee” rather than “thou’.

Similar compound pronouns with “prefixing” singular bound-form pronouns are reclaimed for Wiradjuri (\citealt[51--54]{grant_grammar_2014}), although these were not explicitly described in the early sources (\citealt{gunther_native_1838}, \citeyear{gunther_lecture_1840}; \citealt{hale_languages_1846}). While it is unfortunately not clear on what grounds the structure has been said to have existed in Wiradjuri, it is presumably an examination of early Wiradjuri texts, or the speech of the last fluent speakers which had given rise to the assertion.

\subsection{Description of ergativity }
\label{sec:key:3.3.7}

This final section detailing Threlkeld’s analysis of Awabakal discusses his presentation and understanding of ergative morphology and function. This aspect of his analysis, like his presentations of the large case system, is especially important in establishing his influence on the schools of descriptive practice that subsequently developed in Australia. 

Subsequent to Threlkeld’s earliest account of PN ergative morphology, the ergative case is well described by later PN grammarians, who would certainly have been better prepared to describe this “peculiar” pattern of case-marking having read Threlkeld’s work. It is, however, difficult to know with certainty whether ergative systems of marking syntactic case would have been adequately described by later PN grammarians without the guidance provided by Threlkeld.

The ergative case is the only case outside the Latin inventory that is consistently included and named in early PN case paradigms and descriptions of case. The only exceptions are the early descriptions of Guugu-Yimidhirr (\citealt{schwarz_koko_1900}; \citealt{roth_structure_1901}) and Nggerrikwidhi (\citealt{hey_elementary_1903}; \sectref{sec:key:2.5.1}, \sectref{sec:key:10.1.5.3}), neither of which give ergative forms in case paradigms. The important role that the ergative case plays in disambiguating the arguments predicated by the verb apparently forced the inclusion of ergative case forms in even the most conservative case paradigms.

\subsubsection{Terminology and explanation of ergative function}
\label{sec:key:3.3.7.1}

By attempting to describe this new morphological phenomenon, Threlkeld experimented with a variety of terminological and descriptive techniques. Some were followed by later grammarians, some were not. \tabref{tab:chap3:ergativepm} presents the terminology used to describe the ergative case in the corpus grammars, and in some later descriptions of Australian languages. It shows which of Threlkeld’s terms were, and were not, employed by later grammarians, as well as the terms commonly used by the other corpus grammarians, but which Threlkeld had not employed. Works that were written outside Australia are placed in bold. The table is referred to throughout the following chapters.

\begin{table}
	\small
    \begin{tabularx}{\textwidth}{Qp{9cm}}
        \lsptoprule
        Name given to the ergative case & Author, date, language \textit{(variation)} \newline [additional information] \\
        \midrule
        \textbf{Active \newline nominative} &	Threlkeld 1834: 7 Awabakal 
                            \newline Günther 1838: 43, 1840: 35 Wiradjuri (\textit{nominative active})
                            \newline Fraser – based on Günther –, 1892: 57 Wiradjuri
                            \newline Schürmann 1844a: 4 Barngarla
                            \newline Moorhouse 1846: 3--4 Ngayawang
                            \newline \textbf{Bleek 1858: 2 Australian languages} \\
\tablevspace
        \textbf{Active} &	Teichelmann \& Schürmann 1840: 5--9 Kaurna
                            \newline Meyer 1843: 38 Ngarrindjeri [in discussion of anti-passive]
                            \newline Koch 1868: no pag. Diyari (\textit{Activ})
                            \newline Homann 1879: 86 Diyari (\textit{Activus})*
                            \newline Schoknecht 1947: 2[1872] Diyari
                            \newline Flierl 1880: 10 Diyari (\textit{Activ}) \\
\tablevspace
        \textbf{Subjective nominative} &	F. Müller 1867: 247--250 Australian languages but predominantly Awabakal \\
\tablevspace
        \textbf{Nominative 2} &	\textit{Threlkeld 1834: 12 Awabakal}
                                \newline Ridley 1855b: 74 Gamilaraay (\textit{2nd Nominative})
                                \newline Ridley 1866: 5, (\textit{2nd Nominative})
                                \newline Ridley 1866: 61 Turrubul
                                \newline Fraser – based on Threlkeld –1892: 16 HRLM
                                \newline Fraser – based on Homann – 1892: 43 Diyari
                                \newline T. G. H. Strehlow 1944: 74 Arrernte \\
\tablevspace
        \textbf{Agent \newline (nominative)} &	\textit{Threlkeld 1834: 6 Awabakal}
                                        \newline Ridley 1855b: 74 Gamilaraay (agent- given in brackets after \textit{2nd Nominative})
                                        \newline Livingston, in Fraser 1892: 9 Minjangbal
                                        \newline Mathews 1903b: 261 Gamilaraay (\textit{Nominative Agent})
                                        \newline \textbf{F. Müller 1882: 7, 20 NEW SOUTH WALES languages (\textit{Agens})
                                        \newline Gatti 1930: 21, 58 Diyari (\textit{Nominativo agente})}
                                        \newline Tindale 1937a Wanjiwalku (\textit{Agentive})
                                        \newline Capell 1937: 50 Australian languages (\textit{Agent})
                                        \newline Holmer 1966 Kattang, Thangatti (\textit{Nominative agentive}) \\
	\lspbottomrule
    \end{tabularx}
	\caption{Case labels assigned to the ergative case in early grammars of Pama-Nyungan languages (Works in bold are those written outside Australia)}
	\label{tab:chap3:ergativepm}
\end{table}

\begin{table}
	\small
	\begin{tabular}{p{2cm}p{9cm}}
		\lsptoprule
		\textbf{Nominative transitive}	& C. Strehlow 1931a [c.1907]: 28 Arrernte 
		\newline J. Riedel 1931 [c.1923]: 104 Arrernte 
		\newline C. Strehlow 1910: 1 Arrernte and Luritja \\
		\textbf{Nominative}	& Symmons 1841: xiii Nyungar [for pronouns]
		\newline Fraser – based on Symmons –1892: 52 Nyungar [for pronouns]
		\newline Kempe 1891 : 3 Arrernte
		\newline Roth 1897 : 7 Pitta-Pitta
		\newline Ray 1907: 272 Yadhaykenu \\
		\textbf{Instrumental} &	\textbf{F. Müller 1882: 50, 66, 75 South Australian languages
			\newline Ray 1897: 127, 136 \stockcaps{WTS}} \\
		\textbf{Active instrumental} &	\textbf{Ray 1907: 19 WTS
			\newline G. von der Gabelentz 1891: 151 Australian languages (\textit{Activo-instrumentalis})} \\
		\textbf{Agent causative} &	\textit{Threlkeld 1834: 10 Awabakal} \\
		\textbf{Causative} &	Taplin 1867: no pag., 1880: 9, 11 Ngarrindjeri [for nouns and pronouns]
		\newline Taplin 1975[1870]: 124--126, 1879b: 124 Ngarrindjeri [pronouns only]
		\newline R. H. Mathews 1907: 324 Arrernte and several languages
		\newline J. M. Black 1920: 83, 85, 91 Kaurna, Narungga, Kukarta respectively
		\newline Ablative	Symmons 1841: ix [for nouns]
		\newline Fraser – based on Symmons – 1892: 49 Nyungar [for nouns]
		\newline Meyer 1843: 12. Ramindjeri [for nouns and pronouns]
		\newline Moorhouse 1846: 10--18 Ngayawang [for pronouns]
		\newline Taplin 1975[1870]: 123, 1879b: 123--124 Ngarrindjeri 
		\newline Taplin 1880: 13 Ngarrindjeri [for demonstrative pronouns]
		\newline Fraser 1892: 46 – based on Moorhouse – Ngayawang (\textit{ablative 6})
		\newline Fraser 1892: 29 – based on Taplin, 1879b –. Ngarrindjeri (\textit{ablative 1})
		\newline Hagenauer 1878: 43 Wergaya
		\newline Bulmer 1878: 31 Kunai
		\newline Reuther 1894: 5 (\textit{Ablativ}) 1981[1899]: 3 [Diyari]
		\newline Reuther 1981[1901]: 31 Wangkangurru 
		\newline Reuther 1981 [c.1901]: 57 Yandrruwandha \\
		\lspbottomrule
	\end{tabular}
\end{table}

\begin{table}
	\small
	\begin{tabular}{p{2cm}p{9cm}}
	\lsptoprule
	\textbf{Ergative} &	\textbf{Schmidt 1902 WTS
		\newline Planert 1907a: 555 Arrernte
		\newline 1908: 689 Diyari (also Ngarrindjeri : 693)
		\newline C. Strehlow 1908: 699 Arrernte
		\newline Schmidt 1919b: 43 Ngarrindjeri 
		\newline Gatti 1930: 58 Diyari (\textit{Ergativo})
		\newline Holmer 1963: 59 Australian languages}
	\newline Holmer 1966: 47 Kattang, Thangatti
	\newline Holmer 1971: 7--8 Australian languages \\
	\textbf{Operative}	& Smythe 1975[1949]: 275 Bandjalang
	\newline Capell 1956: 63--64 Australian languages
	\newline Hercus 1969: 46, 128 Victorian languages 
	\newline Blake \& Breen 1971: 48 Pitta-Pitta \\
	\lspbottomrule
	\end{tabular}
\end{table}

Threlkeld’s inaugural account of Australian ergativity (\citeyear{threlkeld_australian_1834}) occurs reasonably early within European linguistic encounter with the twenty-five percent of the world’s languages exhibiting ergative structures. Scrutiny of the early presentation of PN ergative structures affords the opportunity to examine the nature of the relationship between linguistic theory and descriptive methodology in Australia and in Europe. The range of terminology Threlkeld experimented with suggests that he \textit{may} have been acquainted with some existing descriptions of some ergative languages. Of the terms he employed, “active” had previously been used to describe ergativity by A. Oihenart (\citeyear{oihenart_notitia_1638}; \citealt{oihenart_1656}) and following him by W. Humboldt in descriptions of Basque (\citeyear{humboldt_anmerkung_1801}; \citeyear{humboldt_berichtigungen_1817}; see \citealt[186, 198]{lindner_komposition_2013}). The term “agent”(\textit{agens}) had been used in Oihenart’s seventeenth-century descriptions of Basque as well as in an 1820 description of Hindi (see \citealt[198]{lindner_komposition_2013}).

Other terms which had previously been employed to describe the ergative case outside Australia, but which were not employed by Threlkeld, include O. Fabricius' (\citeyear{fabricius_1791}; \citeyear{fabricius_forsog_1801}) use of “nominativus transitivus” in Kalaallisut (see \citealt[186, 198]{lindner_komposition_2013}) and the term “instrumental” in description of Marathi (1805) and Hindi (1827; see \citealt[198]{lindner_komposition_2013}).

Threlkeld’s grammars show that he had a sound understanding of ergative function and forms. Threlkeld conceived of the ergative case as a special type of nominative. In discussion under the word-class ``substantives'', he differentiated “the two nominative cases” (\citeyear[7]{threlkeld_australian_1834}) by explaining that “the first nominative is simply declarative wherein the subject is inactive; as, this is a bird … The second nominative is when the subject is an agent causative of action: as … in the bird eats” (\citeyear[10]{threlkeld_australian_1834}). Elsewhere in the grammar he differentiated the two cases by explaining that the ergative has “the power of operating” (\citeyear[6]{threlkeld_australian_1834}), while the nominative “merely declares the person, or thing or the quality” (ibid.: 5). 


Reflecting this conception of the ergative case as a special sort of nominative case, Threlkeld placed ergative case forms in second position after the nominative forms at the top of his case paradigms.

Of the many terms Threlkeld chose to name the case, all include the term “nominative": ``agent nominative'', “nominative 2” and “active nominative". The term “agent nominative” was only used in the introduction to case and not as a case label. In some nominal declensions, Threlkeld translated nominative forms as “a X'', and ergative forms as “the X is the agent who …'', or “The X is the agent spoken of”. The “nominative 2” was used in most case paradigms (\figref{fig:3:43}, \figref{fig:3:46}) where it contrasts with the “nominative 1” (nominative case). But when presenting interrogative pronouns, the ergative case was labelled the “active nominative” and opposed to the “simple nominative” (\citeyear[7]{threlkeld_australian_1834}; \figref{fig:3:54}).
 
\begin{figure}
\includegraphics[width=.8\textwidth]{figures/Figure 54a.png}
\includegraphics[width=.8\textwidth]{figures/Figure 54b.png}
\caption{Threlkeld’s interrogative case paradigm (\citeyear[7–8]{threlkeld_australian_1834})}
\label{fig:3:54}
\end{figure}


All of these terms were subsequently employed in the primary corpus, except the term ``nominative 2''. It was, however, re-employed by Fraser in his edition of Threlkeld’s work (\citeyear{fraser_australian_1892}) and in Fraser’s presentation of missionary Homann’s Diyari pronominal paradigm (\citeyear{fraser_australian_1892}). The practice was resurrected by T. G. H. \citet{strehlow_aranda_1944}. Note that D. \citet[207]{trudinger_grammar_1943} and, following him, N. \citet[3]{tindale_pitjantjatjara_1963} reversed the order of “nominative” cases, employing “nominative I” for the ergative and “nominative II” for the nominative case.

In the first case paradigm of a noun (\figref{fig:3:43}) and of a pronoun (\figref{fig:3:46}), Threlkeld provided further clarification of the different functions of the unmarked nominal in nominative case and that marked with the ergative suffix. He did this by stipulating the type of question that the different forms might be given in answer to. This particular way of clarifying the functional difference between ergative and nominative cases, in terms of the type of interrogative each is given in answer to, became a characteristic feature of later grammars of languages in NSW, being utilised by \citet{gunther_native_1838} in description of Wiradjuri (\sectref{sec:4.4}) and by Ridley in his earliest description of Gamilaraay (\citeyear{ridley_kamilaroi_1856}; \sectref{sec:4.5}).

\subsubsection{Split systems}
\label{sec:key:3.3.7.2}

Although Threlkeld adequately accounted for ergative functions and forms, it is important to observe that the ergative case is not conferred equivalent status to that of the accusative case, which is so entrenched in the classical paradigm. Threlkeld perceived that the overt marking of an agent brought about a split in the marking of the subject, but attributed no significance to the different marking of a subject and an object. Ergative case forms are, for example, omitted from paradigms of nominals showing accusative alignment (\stockcaps{AS/O}) in the dual pronouns (\figref{fig:3:38}). By contrast, Threlkeld’s case paradigms of nouns showing ergative alignment (\stockcaps{A/SO}) do include the accusative case forms even when they are formally identical to the nominative. Examine, for example, the second declension (\figref{fig:3:43}) and the third declension (\figref{fig:3:34}). Rather than having theoretically assimilated the ergative case into the case system, Threlkeld sees it as a marked form of the nominative subject. Only nominals that are \textit{overtly} marked for ergative case, taking a different shape from the nominative case forms, are described as ergative and included in Threlkeld’s case paradigms.

\subsection{The syntax of complex clauses}
\label{sec:key:3.3.8}

Threlkeld accounted for the subordinating function of a range of verbal morphology in Awabakal in lengthy tables of verb conjugations that constitute more than half his grammar \citep[33--77]{threlkeld_australian_1834}. Under the heading ``subjunctive'', verbs marked with dependency-marking suffixes are described as being “in \textit{regimen}”, i.e., ruled by something else. For example, verbs inflected with the suffix \textit{–wil}, were described by \citet[49]{threlkeld_australian_1834} as being “in regimen denoting the purpose of the subject”. The suffix is described by \citet[77]{lissarrague_salvage_2006} as the desiderative or purposive suffix, which “occurs in a subordinate clause and indicates subsequent action resulting from the main clause”. Threlkeld’s description of subordinating morphemes marking forms as being in \textit{regimen} was not used again in the description of an Australian language.

A range of morphology with subordinating function was accounted for by \citet[70]{threlkeld_australian_1834}, including the subordinating clitic \textit{–pa} (\citealt[93]{lissarrague_salvage_2006}) marking the subjunctive mood of the verb. Clauses marked with the clitic are translated as temporal relative clauses “while X”, given without the main clause (\figref{fig:3:55}; \ref{ex:3:26}). In a detailed investigation of Threlkeld’s description and use of this subordinating clitic \textit{–pa}, \citet[126--132]{wafer_waiting_2011}, however, state that the clearest reference Threlkeld made to \textit{–pa} occurred under the heading “adverbs of time” (\citeyear[75--76]{threlkeld_australian_1834}), concluding that his idiomatic usage of the clitic was better than his ability to describe its function. That Threlkeld’s account of the subordinating clitic \textit{–pa} within a discussion of verbal mood is overlooked by Wafer and Carey signals the potential opaqueness of structures in early grammars even to the most astute observer.

\begin{figure}[t]
\includegraphics[width=.8\textwidth]{figures/Figure 55.png}
\caption{Threlkeld’s presentation of \textit{–pa} in a discussion of the subjunctive mood of the verb (\citeyear[70]{threlkeld_australian_1834})}
\label{fig:3:55} 
\end{figure}


\ea\label{ex:3:26}
Wi-yán    no-a   ba \\
\glt `While he speaks' \\
(\citealt[70]{threlkeld_australian_1834})\\
\gll Wiya-N		nyuwa-pa \\
speak-\stockcaps{PRES}	3\textsc{sg}.\textsc{m}.\stockcaps{NOM}-\stockcaps{SUB} \\
\z

\section{Concluding remarks}
\label{sec:key:3.4}

Threlkeld’s inaugural description of a PN language accounts for a remarkable range of foreign morpho-syntactic complexity, or as \citet[482]{hale_languages_1846} put it, a “mass of information which is entirely new”: ergative morphology, compound pronouns, bound pronouns, and a morphological case system with a large case inventory. The work is substantially more detailed than many early PN grammars, with the exception of Meyer (\citeyear{meyer_vocabulary_1843}; \sectref{sec:key:6.1}), Kempe (\citeyear{kempe_grammar_1891}; \sectref{sec:key:9.1.2}), and Roth (\citeyear{roth_ethnological_1897}; \sectref{sec:key:10.1.1}).

Threlkeld tackled the description of new grammatical categories boldly. He invented terminology -- e.g., “verbal nominatives” and “personal nominatives'' -- and introduced new schemata, such as the “conjoined dual case''. More than many other later grammarians he was prepared to abandon the traditional framework, presenting, for example, extended case paradigms, and \textit{not} attempting to describe the morphological marking of comparative and superlative degrees of the adjective. As Hale \citep[482]{roth_ethnological_1897} pointed out, however, the “strangeness of the principles on which the structure of the language was found to rest … rendered a clear arrangement, at first a matter of difficulty.” Aspects of the analysis -- e.g., the description of bound pronouns -- are difficult to decipher, and the grammar is not without error. That Threlkeld did not shy away from attempting to account for morphosyntactic complexities probably curbed the potential influence of his work. That he was not tempted to produce a more easily digestible, if regularised, description reflects the respect he had for the intelligence of Aboriginal people, and his desire to provide what he later described as: 
\begin{quote}
    [A] testimony against the contemptible notion entertained by too many, who flatter themselves that they are of a higher order of created beings than the aborigines of this land, whom they represent as ``mere baboons, having no language but that in common with the brute!'' \citep[4]{threlkeld_key_1850}
\end{quote}
