\chapter{Conclusion}
\label{chap:key:11}

This history of ideas about the grammatical structures of Australian Aboriginal languages has identified three schools of descriptive practice that developed in Australia prior to the involvement of academic institutions. One of these was instigated by L. E. \citet{threlkeld_australian_1834} in the earliest attempt to comprehensively describe the grammatical structure of an Australian Aboriginal language.

While Threlkeld’s grammar is one of the more detailed and intelligently compiled works considered in this study, to characterise it as “the most accomplished linguistic investigation of any of the 250 Aboriginal languages of Australia undertaken prior to the twentieth century” \citep[253]{carey_lancelot_2004} devalues other remarkable early descriptions of Australian morphosyntax. The inaugural and detailed grammars of Ramindjeri \citep{meyer_vocabulary_1843}, Arrernte \citep{kempe_grammar_1891}, and Pitta-Pitta \citep{roth_ethnological_1897} each gave outstandingly nuanced descriptions of morphosyntactic idiosyncrasies in those languages, which Threlkeld had not encountered in Awabakal, or at least not perceived. \citet{meyer_vocabulary_1843} conveyed pronominal kin suffixes, bound pronouns, and the antipassive construction with descriptive clarity, and \citet{roth_ethnological_1897} described the different marking of syntactic cases in future and non-future tenses. Other early grammars of Australian languages that are shorter than Threlkeld’s work, but which were written after substantially less exposure to the language than the eight years it took Threlkeld to produce his major grammar (1834), also evince considerable insight. \citegen{livingstone_grammar_1892} grammar of Minjangbal is remarkable in describing a system of grammatical gender, which no other early grammarian appears to have encountered. \citeapo{symmons_grammatical_1841} grammar of Nyungar, which presented pronouns referring to dyadic kin relations, appears to have recorded an as yet unreclaimed system of optional ergativity that interacts with tense. This work has received nowhere near the recognition it deserves. While Threlkeld’s work certainly had its own strengths – his  account of the sensitivity of case marking to animacy through the presentation of classes of nominal declension is better than that given in many later grammars – his work is better judged as among the best early grammars, rather than “the most accomplished''.

Nor can Threlkeld’s grammar be seen to have been “essential in establishing a framework” \citep[269]{carey_lancelot_2004} for the later description of Australian languages. Early Australian grammarians were probably guided by Threlkeld’s choice of orthography more than they were by any aspect of his representation of morphology and syntax, although the ways in which Australian languages might have been differently represented orthographically if Threlkeld’s work had not been produced is itself open to debate. Some grammarians (\citealt{gunther_lecture_1840}: 338; \citealt{teichelmann_outlines_1840}: v; \citealt{ridley_kamilaroi_1856}: 290) acknowledged complying with the “system of letters'', which Threlkeld had imported from missionary descriptions of Polynesian languages, not because Australian and Polynesian languages were considered to be phonologically similar, but because they were geographically proximate (\citealt{threlkeld_australian_1834}: vi). Only \citet{gunther_native_1838} mentioned implementing a feature of Threlkeld’s grammatical framework. That he did so when deliberating about abandoning Threlkeld’s descriptive method in favour of his own innovation, i.e., providing labels to case forms rather assigning numbers to “ablative” cases, is characteristic of the tendency observable throughout much of the corpus to invent new descriptive methods, rather than to follow existing practices.

Threlkeld’s inaugural account of Australian ergativity \citeyearpar{threlkeld_australian_1834} occurs reasonably early within European linguistic encounter with the twenty-five percent of the world’s languages exhibiting ergative structures. It will remain a point of conjecture as to whether later grammarians would adequately have described ergative forms and function had they not been enlightened by Threlkeld’s well-cir\-culated description. But subsequent pre-contemporary grammarians developed techniques for explaining ergative function that Threlkeld had not employed, and much of the array of terminological innovations used to name ergative case forms in Australian languages had not been used by Threlkeld.

Threlkeld’s influence on the later description of Australian morphology and syntax is largely confined to grammars of languages spoken in New South Wales. His method of clarifying ergative function, by the posing of questions that the ergative and nominative forms would be given in answer to (\sectref{sec:key:3.3.7.1}) was followed by W. \citet{gunther_native_1838} when describing Wiradjuri, by W. Ridley when describing Gamilaraay, but only in his earliest publication \citeyearpar{ridley_kamilaroi_1856}, and by H. Livingstone when describing Minjanbal (1892 [1876--1886]). This trait is not found in grammars of languages spoken outside New South Wales. Threlkeld’s enlarged case paradigms, which provided positions for case forms marking function that are not carried by morphological case systems in SAE languages, were also appropriated by Günther and by Ridley in description of languages spoken in New South Wales, as was Threlkeld’s technique of placing ergative case forms at the top of the paradigm alongside nominative forms.

Sometimes missionary-grammarians wrongly imported structures exhibited and described in one language into the description of another language. Based on the nineteenth century understanding that Australian languages were phonologically and grammatically similar, and that linguistic diversity was more a matter of lexicon (see \citealt{moorhouse_vocabulary_1846}: v-vi; \citealt{fraser_australian_1892}), missionary-grammarians tended to underestimate how grammatically different Australian languages might be. \citet{flierl_dieri_1880} incorrectly described tripartite marking of 1pl pronouns in Wangkangurru (\sectref{bkm:Ref466972983}) by analogy with the system in Diyari. Moorhouse’s description of undifferentiated marking on non-singular nominals in Ngayawang (\citeyear{moorhouse_vocabulary_1846}) (\sectref{bkm:Ref340295413}) appears to have extrapolated data from Teichelmann \& Schürmann’s Kaurna grammar (\citeyear{teichelmann_outlines_1840}), as does Meyer’s attempt to show personal pronouns acting to relativise clauses in Ramindjeri (\sectref{bkm:Ref73381962}).

The second school of early grammatical practice that developed in the pre-academic era of Australian linguistic description was instigated by Lutheran missionaries describing languages spoken near Adelaide in the early 1840s, particularly \citet{teichelmann_outlines_1840} grammar of Kaurna, the second published grammar of an Australian language. Different aspects of Teichelmann \& Schürmann’s descriptive template, which had not been inspired by \citet{Threlkeld1834} influenced a large body of South Australian grammatical description.

This South Australian School of grammatical description (1840--1938) of which the Adelaide School (\citealt[410]{simpson_notes_1992}; 1840--1858) is the earliest component, is the most strongly attested of the three identified schools. It is defined by a greater number of shared descriptive practices, which are found in a larger body of work, and it endured for a longer period. Descriptive practices innovated by the Lutheran Dresden-trained missionary-grammarians -- \citet{teichelmann_outlines_1840}, \citet{meyer_vocabulary_1843}, and \citet{schurmann_letter_1844} -- were employed in later grammars of South Australian languages, including Diyari and Arrernte spoken at the inland Lutheran missions. The descriptive features that demarcate this body of work include: the implementation of a conservative Latinate paradigm (\sectref{sec:key:5.3.1}), the naming of the ergative case and placement of ergative forms in case paradigms (\sectref{bkm:Ref336283270}, \sectref{sec:key:5.4.2}), the method of clarifying ergative function within a discussion of the verb (\sectref{sec:key:6.2.1.6}), the division of “postpositions” into two functionally motivated categories (\sectref{sec:key:5.3.2}), and the inclusion of paradigms of “possessive pronouns” to account for double case marking (\sectref{sec:key:5.3.3}).

Yet the descriptive practices engaged by the Adelaide School grammarians are remarkably heterogeneous. Schürmann’s abandonment of “substantive” case paradigms in his grammar of Barngarla (\citealt{schurmann_letter_1844}), which alters the presentation given in the previous grammar of Kaurna he had co-authored with Teichelmann \citep{teichelmann_outlines_1840}, displays a reasoned descriptive response to his developed understanding of Australian case systems. Aspects of Meyer’s grammar of Ngarrindjeri (\citealt{meyer_vocabulary_1843}) -- his use of hyphens to mark meaningful sub-word units, the provision of interlinear-style glosses, and the presentation of an enlarged pronominal case paradigm including forms carrying functions not marked morphologically in SAE languages -- suggest a direct influence from \citet{threlkeld_australian_1834} that is not evident in other grammars of the Adelaide School. Meyer’s case paradigms generated their own trajectory of influence that was independent of his Dresden-trained colleagues. Taplin’s 1870 presentation of case \citep[85]{taplin_notes_1872} in the closely related language Ngarrindjeri, which was reproduced by Hagenauer and Bulmer in \citet{thomas_succinct_1878}, had been influenced by Meyer. These paradigms show the earliest use of the term “ergative'', although not to describe the ergative case (\sectref{sec:2.6}).

The longitudinal study of the description of Diyari and Arrernte (Chapters 8 \& 9) shows that, beyond the limitations imposed by pre-phonemic orthographies, there are serious limitations to the type of material that can be reclaimed from historical sources alone. It is instructive to observe the processes of clause subordination that could not be retrieved from the early grammars of Diyari and Arrernte without guidance from Austin’s (\citeyear{austin_grammar_2013}[1981a]) and \citeapo{wilkins_mparntwe_1989} modern grammars of the languages, thus pinpointing the structures that are likely to remain unreclaimed in languages of which there is no modern description.

The third and final school of early Australian grammatical description was instigated by a medical practitioner, W.E. Roth (\sectref{bkm:Ref514608531}), in an insightful and well-exemplified grammar of Pitta-Pitta (\citeyear{roth_ethnological_1897}), spoken in southwest Queensland. The features of Roth’s template, which were subsequently utilised in later grammars of Guugu-Yimidhirr (\citealt{schwarz_koko_1900}; \citealt{roth_structure_1901}) and of Nggerrikwidhi \citep{hey_elementary_1903} and which define the Queensland School of description that Roth singlehandedly spawned, relate to the description of nouns and pronouns in peripheral cases. That such a descriptively innovative work was written relatively late in the pre-academic era of grammatical description in Australia, with little or no recourse to previous analyses, reveals much about the development of linguistic ideas in the country.

Australian languages were described without the benefit of coordinated academic effort, or the leadership of any home-grown or imported luminaries. The beginnings of institutionalised linguistics in Australia, which commenced within the discipline of anthropology at the University of Sydney in 1926, and with the formation of the University of Adelaide language committee in 1930--1931, occurred at least half a century after the founding of parallel institutions in North America (\citealt{campbell_american_1997}: 35--37, 57). Missionary-grammarians were often unaware of works written in other remote areas of the country. Once stationed in Australia, missionary-grammarians often worked in intellectual isolation from fellow grammarians who were posted across far-flung regions of the country.

The three schools of descriptive practice identified in this study, a New South Wales School, a South Australian School, and a Queensland School, are each delineated by constitutionally independent Australian colonies. The regional pattern of the development of ideas about the best way to describe Australian languages is due, in part, to the fact that missionary-grammarians operated within different Christian denominations, which were ethnically and linguistically distinct, and which had headquarters in the different colonial capitals.

The \textit{ad hoc} nature of the development of linguistics in Australia before the 1930 meant that pioneering descriptive responses to newly encountered structures by missionary-grammarians in Australia were not reliably integrated into a central body of emerging thought. Although improvements to analyses of Diyari and Arrernte occurred at the Bethesda and Hermannsburg missions, where successive generations of missionaries described the same language, this study has found that little overall improvement in the analyses of PN languages occurred in the pre-academic era of description.

In this way, nineteenth and early twentieth century linguistics in Australia differs significantly from American Indian linguistics, described as generally being “up to date with and [having] benefitted directly from contemporary linguistic thinking” \citep[28]{campbell_american_1997}. With exception of C. Strehlow, the corpus grammarians worked without sustained connection with European linguistic intelligentsia. While synchronic grammatical descriptions of PN languages produced in Australia before 1930 (\tabref{tab:1:1}) informed grammatical material produced outside the country (\tabref{tab:1:2}), the study of the early descriptions of PN ergativity shows that a movement of ideas between Europe and Australia was largely unidirectional. Grammatical material produced in Europe presenting a new conception of the marking of syntactic cases (\citealt{muller_grundris_1882}; \citealt{planert_australische_1907}; \citealt{planert_australische_1908}) failed to fuel subsequent linguistic theory or methodology in Australia. Similarly, presentations of both consonants and vowels based on articulatory parameters, which were employed in the pre-academic era of the description of Australian languages only by German philologists, appear not to have been read by grammarians in Australia, and if they were, not understood or assimilated into Australian practice.

When observing the relative influence that individual grammars came to have on later works in the corpus and establishing the existence of “traditions” of descriptive practice it is important to recognise that grammarians who came to be particularly influential did not do so because their work was perceived to be more descriptively concise, or insightful. In the absence of any emerging centralised body in which the study of Australian linguistic structure could be fostered, some grammars simply had a greater potential trajectory of influence than did others. Symmons' description of Nyungar (\citeyear{symmons_grammatical_1841}), the third published grammar of an Australian language, published in the Swan River Colony (Perth) over two and a half thousand kilometres west of Adelaide and accessible at the time only by ship, appears to have been read by no later corpus grammarian. The greater impact of \citet{teichelmann_outlines_1840} analysis than of \citegen{threlkeld_australian_1834} or Günther’s (\citeyear{gunther_native_1838}; \citeyear{gunther_lecture_1840}) had less to do with the relative merit of their analysis than it did with the circumstances optimising the South Australian Lutheran missionaries' influence. While Threlkeld’s large case paradigms which placed ergative case forms in second position next to the nominative clearly influenced Günther, any ongoing influence emanating from the Wellington Valley Mission was stifled by the fact that Günther’s MSS remained unpublished until 1892. The Wiradjuri grammars were not written within a continuing tradition of intra-denominational mission activity that might have facilitated the dissemination of linguistic material, as were the Lutherans' descriptions of South Australian languages.

While the strength of the South Australian Lutheran school of description might be called a “tradition” of Lutheran practice, it should also be kept in mind that the features shared by this body of work do not result from the type of training the Lutherans received, or from anything particular about their “Germanness''. Their descriptive homogeneity does not ensue from a shared intellectual heritage, in the same way that European philologists' presentation of vowel triangles and tables of consonant do. The one possible exception is the Lutherans' description of possessive pronouns. German possessive pronouns agree with the case and gender of the noun they qualify. The forms are irregular and are tabulated in grammars of German (see, e.g., \citealt{bauer_grundzuge_1871}: 40). It is, however, unlikely that this shared understanding would independently have motivated the paradigms in different corpus grammars, and Teichelmann \& Schürmann’s presentation can be assumed to have motivated the recurrence of the feature.

Descriptive breakthroughs made during the first era of Australian grammatical description, including the description of ergativity, the unmarked inalienably possessed NP, pronominal sensitivity to kinship, inclusive and exclusive pronominal distinctions, the juxtaposition of dependent clauses, and four case analysis of split syntactic case systems, have been largely unnoticed by modern Australian grammarians, who have tended to describe the same grammatical categories without acknowledgment that the inaugural description occurred in the pre-academic era. The oversight is symptomatic of a discontinuity in the tradition describing Aboriginal languages, which cast aside the pre-academic works as unworthy of serious consideration.

The ways in which the early grammars of Australian languages informed and inspired European linguistic theory remains under-investigated. Meyer’s description of ergativity in Ngarrindjeri (\citeyear{meyer_vocabulary_1843}), which is atypical of the corpus in equating the ergative argument with the function of the Latin ablative that marks the agent of a passive construction, was an original catalyst for passive interpretations of ergative structures (H.C. \citealt{gabelentz_uber_1861}). The ways in which the corpus grammarians' pre-theoretical syntagmatic analyses of agglutinative morphology, representation of word-internal units, and accounts of ergativity, which were read by European philologists, who were eager for more frustratingly rare Australian data, deserves further investigation.

The grammars examined in this study are important primary documents of colonial history in Australia and deserve closer interdisciplinary attention. That the early grammatical records of Australian Aboriginal languages have received little scholarly attention outside the discipline of linguistics might be explained by a scholarly aversion to anything grammatical. Impressionistically, and perhaps due to Australians' tendency to monolingualism, undergraduate-level linguistic principles are perceived to be too technical to inform historical investigations of the colonial Australian frontier. In the absence of such study, it would be simplistic to assume a correlation between the degree of linguistic training and the quality of a grammar produced by missionary-grammarians.

The popular assumption that a rigorously trained grammarian who had studied a greater number of classical languages would make better analyses of Australian linguistic structures than grammarians with lesser training is indeed not verified by this investigation. A causal link is firstly difficult to draw because what is known about the type of training provided at different mission institutions varies. Nevertheless, this study suggests that the relative rigor of the linguistic training received by missionary-grammarians who described Australian languages had little, if any, bearing on the quality of grammatical description they produced. Highly-trained grammarians did not make better descriptions of Australian languages than did grammarians with only ubiquitous school-boy Latin. The detail given in Threlkeld’s inaugural PN grammar (\citeyear{threlkeld_australian_1834}), or in Koch’s inaugural grammar of Diyari \citep{koch_untitled_1868}, for instance, are in no way accounted for by what is known about the authors' formal training. Conversely, W. Ridley’s grammars of Gamilaraay (\citeyear{ridley_kamilaroi_1875} [\citeyear{ridley_kamilaroi_1866}; \citeyear{ridley_gurre_1856}]; \citeyear{ridley_kamilaroi_1855}; \citeyear{ridley_kamilaroi_1856}) and of Turrubul (\citeyear{ridley_kamilaroi_1866}; \sectref{sec:key:4.5}) are among the least detailed descriptions considered in this study. Yet Ridley was educated at Kings College, University of London (B.A., 1842), and was later Professor of Greek, Latin and Hebrew at the Australian College. The sparsity of Ridley’s descriptions might be excused given the short time he engaged with the language. But consider the first grammar of Diyari \citep{koch_untitled_1868} written by W. Koch, who may have been bright, but had not completed a \textit{Gymnasium} secondary education in Germany. This remarkable work was written by a man who died less than two years after first hearing the language.

Despite the better training received by Neuendettelsau missionaries in comparison to the HMI-trained missionaries, the grammars of Diyari (\citealt{flierl_dieri_1880}; \citealt{reuther_dieri_1894}; \citeyear{reuther_ms_1899}) and grammars of Arrernte (C. \citet{strehlow_untitled_1931}~[c.1907]; \citeyear{strehlow_einige_1908}; \citeyear{strehlow_notitle_1910}) written by the Neuendettelsau men are not of a noticeably different quality than earlier grammars of the same languages by HMI-trained missionaries. C. Strehlow’s grammars of Arrernte are all less detailed than \citegen{kempe_grammar_1891} first grammar. That Strehlow replicated entire passages from Kempe’s work shows that he was to a substantial degree satisfied with the HMI missionary’s analysis. Even the most rigorously trained classical grammarian remained descriptively and theoretically ill equipped to describe PN languages. The strength of individual descriptions has less to do with training than with the inherent intelligence and aptitude of the author and with the length of time and the type of exposure he had with the language.

The study of the circumstances surrounding the production of the early grammatical documents considered in this study shows that missionaries documented Australian Aboriginal languages for a variety of reasons, in addition to their primary motivation to convert people to Christianity. The missionaries' earliest linguistic investigations of Australian languages were sometimes as much intellectual and political endeavours as they were evangelistic.

Threlkeld, for example, continued to document the language spoken at Lake Macquarie, after the closure of the Ebenezer mission in 1841, and at a time when he perceived (\citeyear{threlkeld_key_1850}: 3) that the language was extinct. Similarly, Teichelmann completed his later Kaurna analyses (\citeyear{teichelmann_dictionary_1857}; \citeyear{teichelmann_verb_1858}) when he perceived that the language was no longer spoken. In correspondence to Grey, Teichelmann wrote:

\begin{quote}
	Sir, – According to your wish, I have copied into English, my collection of words and grammatical remarks on the language of the Aborigines who once inhabited the district round about Adelaide; for they have disappeared to a very few … Also, I do not entirely approve of the orthography of the native language, as we have spelt it, but it is now useless to alter anything in it after the Tribe has ceased to be. (Teichelmann, quoted in \citealt{bleek_library_1858}: 40)
\end{quote}

The very earliest grammatical description of Australian languages was ultimately made within a climate of salvage linguistics, where a record was being taken for posterity. In their engagement with the language, which outlived the last generation of fluent speakers, Threlkeld and Teichelmann are early precursors of the more theoretically and less theologically motivated recording of Australian languages that characterises the later Neuendettelsau missionaries who wrote linguistic and ethnographic descriptions of Arrernte and of Diyari between 1880 and 1920 (G. J. Reuther, C. Strehlow [1907--1920], O. Siebert [1910]).

A response letter written by the Australian anthropologist A. W. Howitt’s to lay missionary Vogelsang from 1878 (\sectref{sec:key:8.3.4}) specifies the motivation that might drive a missionary to describe and document an Aboriginal language:

\begin{quote}
As to the grammar ... I have been looking out for the best way to make use of it \& fear that there would not be any chance of selling sufficient copies of your Dieri grammar to pay for bringing it out. … I think the best plan with your grammar would be to ask one of the learned societies either in Adelaide, Melbourne, London, Paris or Berlin to publish it … It is usual in such cases for the author to receive some 20 copies for himself and he of course \textit{becomes known to the scientific men all over the world}. (\citealt{howitt_vogelsang_nodate}; emphasis added)
\end{quote}

An undercurrent of thought that is counter to social Darwinism runs in many of the missionary sources. Missionary-grammarians sought to elevate the status of the “primitive” languages by drawing attention to grammatical complexity. By showing that the Aboriginal language was capable of being construed using the same terminology and framework as Classical Greek and Latin, early missionary-grammarians in Australia effectively afforded the languages' speakers credibility as intelligent and equal humans. Schürmann put forward a particularly crafty argument in a grammar of Barngarla, spoken in South Australia:


\begin{quote}
[I]t has been thought unlikely that so rude and ignorant a nation, as the natives of this continent are admitted to be, should possess a regular grammatical system.\footnote{Here Schürmann refers the reader to the descriptive vocabulary of the colonial settler G. F. Moore (1798--1886), who had described grammatical structure of the Western Australian language Nyungar as “simple and rudimentary and not very copious” \citeyearpar[74]{moore_descriptive_1842}.} This argument is however evidently untenable, for else it would follow on the other hand, that the most civilized nation or the most cultivated language must have the most artificial and complex grammar...The English language for instance, highly cultivated as it is in all its branches of literature, has the simplest grammar imaginable, so that one might infer with as much reason... that the more a language is cultivated, the more its grammar will be simplified. \citep[v]{schurmann_letter_1844}
\end{quote}

The restitution of linguistic material from the early linguistic records examined in this study necessitates the generation of a range of linguistic materials with different purposes and written for different audiences. Language reclamation programmes made by descendant speakers of the recorded varieties may not, at least initially, depend on an exact knowledge of the intricacies of the case systems, where language is being used predominantly in naming, texting, or for “common conversation tropes such as enquiring about health, and commenting on the weather” \citep{blake_nineteenth-century_2016}. That a reclaimed language will necessarily be a phonological and morphosyntactic variant of the original variety at which the corpus grammars were aimed is seen as inevitable, and perhaps desirable, within “revivalistic” contexts if tangible outcomes are to be achieved. Philological scholars are nevertheless obliged to pursue the best possible reclamation of morphosyntactic and phonological structures from early linguistic records regardless of their immediate objectives.

The reclamation of material from antique grammars of Australian language is enhanced when individual analyses are considered in relation to the entire body of early documentation. The type of description that was generated when Pama-Nyungan grammatical structure was mapped onto the descriptive framework developed to describe Standard Average European languages is to a ceratin degree predictable. By defining the looking glass through which Australian morphosyntax were observed, this study has refined a method of extracting precious morphosyntactic data from the early recordings of Australian morphology and syntax.
