\chapter{Later grammars of the Adelaide School}
\label{chap:key:6}

This chapter presents the three grammars of the Adelaide descriptive school which followed \citegen{teichelmann_outlines_1840} grammar of Kaurna. Each was published in English and describes a language belonging to country in the earliest settled southern and coastal regions of South Australia.

Lutheran missionary H. A. E. \citegen{meyer_vocabulary_1843} long and detailed grammar of Ramindjeri is presented first (\sectref{bkm:Ref73553344}). Despite being acquainted with Teichelmann and Schürmann during his training at the Evangelisch-Lutherische Missions-Gesellschaft zu Dresden (DMI) in Germany, and working collaboratively with Schürmann in Australia, Meyer’s description of case given in this second published grammar of a South Australian language is very different than that given by his Lutheran colleagues \citet{TeichelmannSchürmann1840}. That said, Meyer’s placement of the ergative case in the case paradigm, which subsequently influenced Taplin’s grammars (\sectref{bkm:Ref514606622}) of another Ngarrindjeri variety, was guided by Teichelmann \& Schürmann’s innovative design. Meyer’s account of ergativity, which is atypical of the corpus, was read widely by European philologists and informed the earliest typological accounts of ergativity in Europe (\sectref{bkm:Ref73965175}).

The third published grammar of a language belonging to country in the colony of South Australia was written by C. Schürmann (\citeyear{Schürmann1844}; \sectref{bkm:Ref73471921}), who had earlier published the 1840 Kaurna work with Teichelmann.

The discussion diverges to examine what middle-era linguists in Australia were able to discover about bound pronouns in Australian languages based on the corpus grammars examined thus far in the study (\sectref{bkm:Ref73553541}). Other later grammars in the corpus are of languages that are not known to have exhibited bound or enclitic pronouns.

The chapter concludes by examining the final South Australia grammar of the Adelaide School, written by M. Moorhouse (\citeyear{moorhouse_vocabulary_1846}; \sectref{bkm:Ref73379277}), who was not a missionary, but the Protector of Aborigines in the colony.

\section{Meyer’s grammar of Ramindjeri \citeyearpar{meyer_vocabulary_1843}}
\label{sec:key:6.1}

\label{bkm:Ref73553344}\label{bkm:Ref74068587}\label{bkm:Ref73712330}
After being trained for three years at the Jänicke Mission Institute (1833--1836), Lutheran missionary H. A. E. Meyer (1813--1862) commenced training at the DMI in 1837, a year later than Teichelmann \& Schürmann. Unlike Teichelmann, Schürmann and Klose,\footnote{Klose appears to have been less linguistically capable and was presumably trained at the DMI as a missionary school-teacher, rather than as a preacher (see \citealt{Lockwood2014}: 4). Klose taught at the Pirltawardli school from 1840 until its closure in 1845.}  Meyer was partially trained at the University of Erlangen, where he learnt Tamil in preparation for missionary work in India. At Erlangen, Meyer was taught by F. Rückert (1788--1866), Professor of Oriental languages and a renowned poet \citep[74]{Lockwood2014}.\footnote{This is \textit{not} Pastor Johann Wilhelm Rückert, who had trained the missionaries at the Jänicke Mission Institute.}

Meyer travelled to Australia, rather than to India, in order to work with Schürmann, with whom he had formed a close friendship. The pair had expected to work collaboratively at a new mission station at Encounter Bay, and Schürmann had commenced learning Ramindjeri from Tammuruwe Nankanere, “Encounter Bay Bob” (\sectref{sec:key:1.1.3}), compiling a comparative “Adelaide / Encounter Bay” wordlist \citep[68]{Amery2016}. Schürmann was instead appointed by Governor Gawler as deputy Protector of Aborigines at Port Lincoln. He did, however, accompany Meyer on his first journey to Encounter Bay, south of Adelaide, in 1840, and visited Meyer on at least three subsequent occasions \citep[65]{Gale2011}.

Meyer’s initial introduction to Ramindjeri, a dialect of Ngarrindjeri, was thus supplied by Schürmann’s comparative wordlist. Meyer quickly concluded that the languages were structurally dissimilar \citep[65--66]{Gale2011}. Indeed, although geographically proximate, the Thura-Yura languages (\citealt{SimpsonHercus2004}), described by \citet{TeichelmannSchürmann1840} and \citet{schurmann_vocabulary_1844}, are separated from the Lower Murray language Ngarrindjeri \citep{Horgen2004} by a major cultural and linguistic divide that runs along the watershed of the Mount Lofty Ranges \citep[670]{Dixon2002}.

In 1840, Meyer settled at Encounter Bay and commenced ministering to Aboriginal people in the Lower Murray region. He was an astute linguistic observer. In {1841}, he described having communicated with a member of the Milmajerar people, whose territory was southeast of the Murray mouth, and about whom little was known. Governor Gawler was at the time keen to establish peaceful relations with these people after the reprisal hanging of two Aboriginal men in 1840 in retribution for the murder of survivors of the shipwreck Maria. Meyer (10/03/1841, quoted \citealt[68]{Gale2011}) realised that cross-linguistic comprehension did not necessarily indicate linguistic similarity:

\begin{quote}
	The most peculiar thing is that I was understood by the Milmajerar man and that he answered me comprehensively. We were not able to discover whether he understands two languages or whether they share the one and the same language.
\end{quote}

By 1841, Meyer reported that he had completed the outlines of a grammar \citep[68]{Gale2011} and had translated the commandments and hymns into Ngarrindjeri. The work was completed and ready for publication early in 1843, two years and four months after Meyer first heard the language.

In the year that the grammar was published, Governor Grey withdrew financial support for the mission at Encounter Bay, after which Meyer himself leased government land nearby at the mouth of the Inman River and attempted to establish an ultimately unsuccessful agricultural settlement \citep[15]{Lockwood2007}. During 1847 Schürmann worked with Meyer at Encounter Bay. In {1848}, the mission settlement, school and farm were abandoned, and Meyer left to work as a Pastor at Bethany and Hoffnungsthal in the Barossa Valley.

\hspace*{-2.4pt}Unlike Kaurna and Barngarla, described by Meyer’s fellow Dresdner colleagues, a variety of the language described by Meyer continued to be grammatically described after the closure of the Encounter Bay mission. The ways in which G. Taplin’s analyses of Ngarrindjeri (\citeyear{taplin_vocabulary_1867}; \citeyear{taplin_notes_1872}; \citeyear{taplin_grammar_1880}) were guided by Meyer’s initial analysis and by other grammars of the Adelaide School are examined in \sectref{bkm:Ref336429071}.

\subsection{The naming of Ngarrindjeri}
\label{sec:key:6.1.1}

M. \citet[11]{McDonald2002}, in a study of Ngarrindjeri phonetics and phonology, explains that she adopts the term “Ngarrindjeri” as ``a convenient cover term applying to a language and people of a group of related tribes of Aborigines''. She expands:

\begin{quote}
	In 1870, when the original tribal and dialect groupings had been obscured by white settlement, the Reverend George Taplin termed the group at his Point McLeay Mission the Narrinyeri – from the word for ``mankind''. Subsequent repetition in scholarly studies, government reports and religious materials has established the name firmly in the minds of people, who know themselves now as Ngarrinjeri. \citep[11]{McDonald2002}
\end{quote}

The term “Narrinyeri” was in fact first used by Taplin in his diaries as early as 1861 \citep[65]{Lane1997} and appears in his 1867 MS grammar.

Norman Tindale (1900--1993), anthropologist at the South Australian Museum, records:

\begin{quote}
	Narinjeri: generalised term selected by Taplin for the aborigines along the Lower Murray River and Lake Alexandrina – it is not a tribal name but a term meaning “indigenes” as distinct from strangers and white man. Name of the peoples of Lower Murray River as selected by Taplin for a non-native “nation” concept. (Tindale n.d.)
\end{quote}

\citet[241]{Dixon1980} terms the same entity ``Yaralde'', as do \citet{McDonald2002} and Horgen (\citeyear{Horgen2004}; albeit as ``Yaraldi''), who worked with him. \citet[227]{Radcliffe-Brown1918} had used the term “Yaralde” to denote one of five groups, including ``Encounter Bay'', who ``spoke closely related languages''. Dixon here describes the process by which he adopted the term:

\begin{quote}
	The name of one tribal language has been chosen fairly arbitrarily, as the name for the complete language. \citep[241]{Dixon1980}
\end{quote}

Similar redeployment in scholarly literature of a name classically referring to a small social entity or linguistic variety to a larger, higher-level social and/or linguistic group , which may not have been a meaningful or named entity in pre-contact Aboriginal society – the “non-native `nation' concept'', as Tindale put it in the 1930s – has occurred elsewhere in Australia, notably in the evolution of the term “Arrernte” (\citealt{Stockigt2021a}), and in Radcliffe-Brown’s (\citeyear{Radcliffe1030}: 223) later use “Yarlade-type” to classify Australian kinship systems.

The suffix \textit{{}-indjeri} (and the variant \textit{{}-inyeri}; \citealt{tindale_notes_1937}: 107), which is present in the group names Ngarrindjeri (Narrinyeri), Ramindjeri, Milmendjeri, and others, is the associative suffix meaning “belong to” or “pertaining to” \citep[63]{meyer_vocabulary_1843}. The following glossed example with free translation given by Meyer as illustration contains his only usage of the term \textit{Raminyeri}, the language being referred to as ``The Encounter Bay Language'' in early sources. A locative suffix \textit{-ong} appears on the place named \textit{Ram} in its citation form.

\ea\label{ex:key:36}
	\gll Ram- ... inyeri -ap     porl \\
	Ramong  {}   of       I       child \\
	\glt  `I am a native of Ramong' \\
	\citep[63]{meyer_vocabulary_1843} \\
	\gll Ram-inyeri-ap        po:rli\\
	Ram-\stockcaps{ASSOC}-1\stockcaps{SG.NOM}       child \\
\z

Regarding the etymology of the term Ngarrindjeri, \citet[84]{meyer_vocabulary_1843} gave \textit{Narr-inyeri} `Australian native; mankind'. \citet[34]{taplin_1879a_nodate} suggests that the term \textit{Narrinyeri} is a contraction of \textit{Kornarrinyeri}, literally `belonging to men'. Here the initial syllable of \textit{korna}, meaning `men' – after which the Kaurna are named (\sectref{sec:key:5.1.1}) – has been lost, and the second segment -\textit{injeri} is the associative suffix, `pertaining to'. Taplin also suggests an alternative etymology in which the first segment \textit{narr-} means `plain, intelligible (referring to language)'. The second etymology is in my opinion more plausible since initial syllable loss is not otherwise reported for the language.

\subsection{Meyer’s analysis of Ramindjeri (Ngarrindjeri) (\citeyear{meyer_vocabulary_1843})}
\label{sec:key:6.1.2}\label{bkm:Ref456096204}

Meyer’s grammar of Ramindjeri, a dialect of a language which Taplin later termed ``Ngarrindjeri'', is the longest (42 pages) and most detailed of the works produced by the Adelaide School, and appears with a vocabulary of some 1750 entries \citep[64]{Gale2011}. The work contains rich ethnographic content. Like other works in the corpus, Meyer’s grammar now provides rare insight into Aboriginal experience of early colonial contact (see Example \ref{ex:1:3} on p. \pageref{ex:1:3}).

As well as being referred to more frequently by nineteenth-century philologists in Europe than any other grammar of an Australian language (\sectref{sec:key:6.1.2.7}), later Lutheran grammarians in Australia read and referred to Meyer’s work. Meyer’s analysis was reiterated by Lutheran missionary C. \citet{strehlow_untitled_1931} in a comparative grammatical study of three languages (\sectref{sec:key:9.2.3.1}; \sectref{sec:key:9.3.5.2}). Strehlow referred to the language Meyer described as “die Encounter Bay Sprache” (the Encounter Bay language). It is clear that Lutheran missionary Kempe (\sectref{sec:key:9.1.2}) had also read Meyer’s work, although \citet{Kempe1891} did not refer specifically to Meyer. Compare the similarity of this comment made by Meyer in his introduction to his Ramindjeri grammar:

\begin{quote} 
I submit these sheets … with the hope that … they will be interesting to the philosopher and philologist, as exhibiting the peculiar structure of a language spoken by a people very generally considered the lowest in the scale of civilisation. \citep[vii]{meyer_vocabulary_1843}
\end{quote}

with this statement given by Kempe in first grammar of Arrernte:

\begin{quote}
The pages are submitted in the hope that they will prove interesting to the philologist, as exhibiting the peculiar structure of the language spoken by a people generally considered among the lowest in the scale of mankind, and will contribute a little towards perpetuating the knowledge of a language of one of the Australian tribes of natives before their probable entire extinction at a not very remote period. \citep[1]{Kempe1891}
\end{quote}

\subsubsection{Non-paradigmatic descriptions of case system}
\label{sec:key:6.1.2.1}\label{bkm:Ref456091394}

Meyer did not present an initial formal case paradigm for nouns. After the customary introduction to “substantives” (nouns; \citeyear[10--11]{meyer_vocabulary_1843}), which dealt first with gender: “there is no difference of form on account of gender”, and then with number: “the dual is formed by adding the termination `\textbf{engk}' … the plural … by adding `\textbf{ar}' and rejecting the terminating vowel …”, Meyer treated case, the third category prescribed for substantives, under the traditional framework. He wrote: “The relations expressed by the Latin and Greek cases are in this language expressed by particles added to the root in the following manner”. There follows an eight-page listing of thirty-three “particles” (ibid.: 10--17).

Although Meyer did ultimately present an informal nominal paradigm (\figref{fig:key:95}), he pursued a different descriptive tradition that was subsequently used by Schürmann (\citeyear{schurmann_vocabulary_1844}; \sectref{bkm:Ref456091421}), Livingstone (\citeyear{Livingstone1892}; \sectref{sec:key:4.6.1}) and Roth (\citeyear{Roth1897}; \sectref{bkm:Ref514606874}). These grammarians broke from the tradition of presenting the morphology of nouns in paradigms. Instead, they chose what they presumably perceived was a preferable method of presenting case. Livingstone, whose analysis of Minjangbal benefited from the decades of research subsequent to the Adelaide School, justified his abandonment of the traditional descriptive framework in terms of the agglutinative nature of the morphology:

\begin{quote}
It is well known that the Australian dialects are agglutinative, everything in the nature of inflection being obtained by suffixes … so that if I give an account of its suffixes, that is nearly equivalent to giving an exposition of its grammar. \citep[3]{Livingstone1892}
\end{quote}

Unlike \citet{TeichelmannSchürmann1840} and \citet{Threlkeld1834}, but like Günther (\citeyear{gunther_native_1838}; \citeyear{gunther_lecture_1840}), Meyer did not provide the word-class heading “pre/postposition” towards the end of the grammar. Instead, he discussed the full range of nominal inflectional morphology, called ``particles'', under the heading: “Of substantives and their cases, and of prepositions” \citeyearpar[10]{meyer_vocabulary_1843}. This break from the traditional grammatical framework allowed Meyer to avoid having to decide whether a nominal suffix should, or should not, be described as a marker of case. The practice was later employed by Schürmann (\citeyear{schurmann_vocabulary_1844}; \sectref{bkm:Ref456091421}) and by Livingstone (\citeyear[9--11]{Livingstone1892}; \sectref{sec:key:4.6.1}). Formal criteria for identifying whether or not a nominal suffix marks case – like the ability for the inflected nominal to relativise (see \citealt{wilkins_mparntwe_1989}: 157--159) – were developed in a descriptive era much later than that considered here.

Meyer listed his “particles” \citeyearpar[10--17]{meyer_vocabulary_1843} first under headings naming the classical European cases, and second as European prepositions. While the division still reflected whether the case form translated into a SAE language as either a case inflected word, or a prepositional phrase, this presentation better conveyed that these “particles” were part of a single system.

This section of Meyer’s grammar suggests a direct influence from Threlkeld that is not present in Teichelmann \& Schürmann’s work (\citeyear{teichelmann_outlines_1840}). In a preamble to the main section of the grammar headed “etymology’, \citet[5--7]{Threlkeld1834} had discussed the importance of ``particles''. While the term had been used by Teichelmann and Schürmann, they predominantly discussed “terminations” or “affixes” (\citeyear[3]{teichelmann_outlines_1840}). The similarity between Meyer’s and Threlkeld’s discussions relates not only to the choice of terminology, but also to the approach to the description of case. The inclusion of this initial section headed “etymology” in Threlkeld’s grammar is curious. He presented material here that is also presented paradigmatically in the main, much longer body of the grammar. It is as if Threlkeld experimented with an alternative, less formal presentation of \textit{describing} the function of ``particles'', before resorting to the traditional and familiar style of presentation.

Such an arrangement additionally afforded the flexibility of showing that a single function might be differently marked on different types of nominals. An emphasis on the description of \textit{function} rather than \textit{form} allowed Threlkeld, for example, to explain that higher-animate nouns in accusative case were overtly marked, while common nouns took the same unmarked form as the nominative, a pattern commonly found in PN languages \citep[14]{Blake1977} and accounted for by \citet{Silverstein1976} in terms of an animacy hierarchy. Threlkeld stated:

\begin{quote}
names of persons have the terminating particle –\textbf{nung … Threkeld-nung} \textit{…} other common substantives … are placed before the active verb without change from the simple nominative, nor can any error arise, because when used as an agent, the sign of that case would be attached. \citep[6--7]{threlkeld_australian_1834}
\end{quote}

By contrast, later grammarians of Western Arrernte (\citealt{kempe_grammar_1891}; \citealt{mathews_arranda_1907}), which has a similar split in the marking of syntactic case of higher-animate nouns (\citealt{Stockigt2017}; \sectref{sec:key:9.3.5}) failed to appreciate this different marking of the same function on different nominal types. They presented an account of the case system only through traditional paradigms.

\hspace*{-4.4pt}Meyer listed English prepositions and then discussed how the same function was marked on different nominal types in Ramindjeri. The preposition \textit{to} \citep[13]{meyer_vocabulary_1843}, for example, was exemplified with nouns marked for allative case with the forms \textbf{-ungai} and \textbf{-angk}: “I will go to the river”. Meyer explained that the two forms “may not be used for one another, but no rule can at present be given for their correct application, except that \textbf{-ungai} may never be used with pronouns.” The preposition \textit{to} was then listed again in order to exemplify allative marking on place names (\figref{fig:key:94}).

\begin{figure}
\includegraphics[width=.8\textwidth]{figures/stockigt-img034.png}
\caption{Meyer’s translation into Ramindjeri of the English preposition “to”  \citeyearpar[13]{meyer_vocabulary_1843}}
\label{fig:key:94}
\end{figure}

\hspace*{-4.4pt}Meyer clarified the meaning of many “particles” by giving both English and Latin prepositions. The use of both Latin and English reduced the ambiguity of assigning function to nominal inflections with translation of prepositional phrases. For example, Meyer translated the “particles” \textit{-ambe} and \textit{angk} as the preposition `for'. The first is additionally equated to the Latin \textit{pro} indicating `instead of', the second is described as “indicating the end or motive of an action” (\citeyear[17]{meyer_vocabulary_1843}).

\hspace*{-4.4pt}At the conclusion of this long discussion Meyer presented an informal paradigm that captured the detail of the discussion (\figref{fig:key:95}). The absence of case-labels, except ``nominative'', ``accusative'' and ``vocative'', and the inclusion of multiple functions for some suffixes, is very different from earlier grammarians' presentation of case. Note that although other suffixes are not assigned case labels, their ordering roughly follows the traditional paradigm, genitive > dative > ablative. The ergative suffix \textit{-il,} translated with the preposition `by', is placed in the position of the Latin ablative.

\begin{figure}
% \includegraphics[width=.8\textwidth]{figures/stockigt-img035.emf}
\includegraphics[width=.8\textwidth]{figures/stockigt-img036.png}
\caption{Meyer’s informal case paradigm of nouns  \citeyearpar[17–18]{meyer_vocabulary_1843}}
\label{fig:key:95}
\end{figure}

\subsubsection{Reclaiming the Ngarrindjeri case system}
\label{sec:key:6.1.2.2}

When a language is reclaimed for revitalisation purposes from limited or conflicting source data, it is sometimes necessary to make informed but arbitrary decisions so that the language might again be viable in some capacity. Those decisions are not always well documented. The degree to which authors of reclaimed grammars, or learners' guides, detail their decision-making processes depends in part on the purpose of the publication, and the audience for whom the work is written. \citet{Clendon2015} reconstructed grammar of Barngarla, based on \citet{schurmann_vocabulary_1844}, attends in detail to the processes of inference from the early source and from surrounding languages. Authors of material written primarily for a linguistically uneducated audience – for instance, \citegen{gale_dorothy_2010} \textit{Ngarrindjeri learners' guide} – tend not to document decisions as transparently.

The certainty with which a case suffix can be said to mark a particular range of case functions in a reclaimed grammar depends not only on the skill of the early grammarian, but also on the level of complexity he encountered. Meyer’s grammar suggests a system of marking case on different types of nominals that may have been more complicated than that encountered by other corpus grammarians, and which is now not easily reclaimed. The reclamation of the “Ngarrindjeri” case system is further complicated by conflicting data from different varieties, recorded at different intervals after contact (\tabref{tab:2.1}). The way in which different nominal types formally delineated the marking of several non-syntactic cases is not clear. One comprehensive attempt \citep[95]{Horgen2004} does not account for the different marking of the dative/allative or the associative/ablative/causal functional continuums on different nominal types. Horgen writes:

\begin{quote}
	A number of forms are given for the Yaraldi locative, allative and ablative. In \citet[12--18]{meyer_vocabulary_1843} and \citet[8]{Taplin1880} a number of “prepositions” are given, in addition to case markers. These have not been incorporated. \citep[95]{horgen_languages_2004}
\end{quote}

\hspace*{-1.2pt}When assessing the potential to reclaim languages solely from archival sources, it is important to here observe that the Diyari case system, recorded from speakers in the modern era \citep{Austin1981}, could not have been retrieved from the record left by nineteenth century missionaries (\sectref{sec:key:8.5}). The isomorphic patterning of case function on different nominal types in Diyari would have been lost if that language had not survived into the modern descriptive era.

\subsubsection{Pronominal case paradigms}
\label{sec:key:6.1.2.3}\label{bkm:Ref74730717}

Although Meyer and his colleague Schürmann did not present formal case paradigms for nouns, they did so for pronouns. Meyer initially presented paradigms of the syntactic cases for first, second and third-person on three numbers. Ergative case forms were placed in third paradigmatic position after the nominative and accusative cases. Ergative forms were labelled “ablative” (\figref{fig:key:96}).

\begin{figure}
\includegraphics[width=.8\textwidth]{figures/stockigt-img037.png}
\includegraphics[width=.8\textwidth]{figures/stockigt-img038.png}
\caption{Meyer’s pronominal case paradigm showing only the syntactic cases (and vocative; \citeyear[22--23]{meyer_vocabulary_1843}.}
\label{fig:key:96}
\end{figure}

Meyer’s choice of the term “ablative” as the descriptor of the ergative case was \textit{not} conventional practice at the time (\textit{contra} \citealt{lindner_ergative_2014}: 190). In choosing the label “ablative” to name ergative pronouns, Meyer may have intentionally invoked the ``ablative-of-personal-agent'', rather than the “ablative-of-means-of-instrument” invoked by Teichelmann \& Schürmann to name instrumental function (\tabref{tab:chap5:mapping}). Alternatively, Meyer may have, at least initially, been confused about how the earlier Dresdner missionaries had used the terms “active” to name ergative function and “ablative” to name instrumental function, and the type of nominals in Kaurna and in Ramindjeri that marked this functional range uniformly or distinctly.

Meyer was probably motivated to provide a pronominal paradigm for the syntactic cases because he needed to record the unpredictable marking of pronouns in different numbers in ergative and nominative cases. The traditional Word and Paradigm representation best recorded the sensitivity to number of the marking of syntactic case on pronouns.

The Ngarrindjeri split on nominals (\tabref{tab:key:97}) has been reclaimed by \citet{bannister_longitudinal_2004} from Meyer’s record of Ramindjeri (\citeyear{meyer_vocabulary_1843}), from Taplin’s analyses of the language spoken at the mission (1872 and 1880), and from twentieth-century recordings.

\begin{table}
	\begin{tabularx}{\textwidth}{QQQ}
		\lsptoprule
		Ergative alignment \stockcaps{A/SO} & Accusative alignment \stockcaps{AS/O} & Tripartite marking \stockcaps{A/S/O} \\
		\midrule
		Nouns, \newline proper nouns & All 2\textsuperscript{nd} person pronouns, 1dl, 1pl \& 3pl pronouns & 1sg, 3sg, 3dl pronouns, demonstrative, interrogative pronouns \\
		\lspbottomrule
	\end{tabularx}
	\caption{Reclaimed split in the marking of syntactic case in Ngarrindjeri}
	\label{tab:key:97}
\end{table}

Meyer then presented a full case paradigm for first-person pronouns (\figref{fig:key:98}). Unlike \citet{TeichelmannSchürmann1840}, he extended the case paradigm beyond the classical five-case paradigm. \citet{Threlkeld1834} and \citet{Günther1838} had previously presented enlarged paradigms. There is no evidence that Meyer had seen Günther’s manuscripts. Unlike Threlkeld, Meyer assigned case labels to the additional forms.


\begin{figure}
	\includegraphics[height=.5\textheight]{figures/stockigt-img039.png}
	\caption{Meyer’s extended pronominal paradigm for first-person pronouns \citeyearpar[24]{meyer_vocabulary_1843}}
	\label{fig:key:98}
\end{figure}

The syntactic cases are again placed in the first three positions, labelled ``nominative'', “accusative” and ``ablative''. This repositioning of the “ablative” (ergative) and accusative cases out of their traditional fourth and last position in the traditional paradigm (nom > gen > dat > acc > abl) as well as Meyer’s initial presentation of these three cases alone (\figref{fig:key:96}) suggests that Meyer conceived that these cases were functionally distinct from the peripheral cases.

Since the term “ablative” had been “used up” to describe the ergative case, Meyer invented a new label to name the suffix marking the ``ablative of separation'', which in many other early grammars is termed “ablative” (\tabref{tab:chap5:mapping}) and after which the case is named, from the Latin \textit{ablātīv-us}, `to carry away'. Meyer instead named the pronoun marking this in an orthodox way as the “ablative” function, using the English preposition ``From'' and translated the form as ``from X''.

He named the pronoun marked with the suffix \textit{-angk} ``dative'', and translated it as ``to X''. The suffix is reclaimed as marking the allative and locative cases on pronouns and common nouns \citep[22; 53]{gale_dorothy_2010} but dative on kin terms (ibid.: 24).

The pronoun marked with the suffix \textit{-ambe} was named ``For'', and translated as ``for X''. The range of dative functions that this suffix marked on different nominal types is uncertain.

\subsubsection{Bound Pronouns}
\label{sec:key:6.1.2.4}\label{bkm:Ref73463170}

In 1840, just after arriving in South Australia, and three years before his grammar of Ngarrindjeri was published, Meyer wrote:

\begin{modquote}
	The difference [between Kaurna and Ngarrindjeri] consists not only in words but also in the formation of the same. In Adelaide, to the best of my knowledge, the pronoun is placed before the substantive, here by comparison one makes use of suffixes … although they also make use of the pronoun for several things, \textbf{nanauwe, namauwe} and \textbf{lomauwe.} \citep{meyer_correspondence_1839}
\end{modquote}

Teichelmann and Schürmann’s grammar of Kaurna was complete at the time Meyer wrote this. Although the forms Meyer provided as exemplification – ``\textbf{Gela\-no}: brother, \textbf{Gelanowe:} my brother, \textbf{Gelau\-we:} your brother, \textbf{Gelau\-walle}: his brother and so forth in several cases''  – were possessed kinterms (\sectref{bkm:Ref330037209}), Ngarrindjeri bound pronouns were immediately recognised by Meyer in his earliest encounter with the language. That the Kaurna forms remained unknown to him suggests that the systems of bound pronouns operated differently in the two languages.

\citet{meyer_vocabulary_1843} gave the earliest representation of case forms of bound pronouns paradigmatically, alongside the free-forms under the word-class heading ``Pronouns''. He tabulated nominative, accusative and ergative “ablative forms'' (\figref{fig:key:99}).


\begin{figure}
\includegraphics[width=.65\textwidth]{figures/stockigt-img040.png}
\caption{Meyer’s paradigm of bound pronouns in nominative, accusative and ergative cases (and vocative; \citeyear[23]{meyer_vocabulary_1843}})
\label{fig:key:99}
\end{figure}

Meyer does, however, appear to have recognised that bound pronouns also inflected for other cases, as is evident from the note attached to the asterisked form of the 1sgO form in the paradigm. He (\citeyear[23]{meyer_vocabulary_1843}) states: “\textbf{an} occurs sometimes as a prefix, as, \textbf{Yarn-ir an-angg-itye} he spoke to me” (Example \ref{ex:key:37}). Here the form \textbf{an-angg} is shown marking the second argument of the verb “to speak”.

\ea \label{ex:key:37}
\glll Yarn-ir-an-angg-itye \\
yarn-ir-anangk-itye \\
speak-\stockcaps{PAST}-3\textsc{sg}.\stockcaps{DAT}-3\textsc{sg}.\stockcaps{NOM}\footnotemark \\
\footnotetext{Note that the 1sg pronominal suffix –\textit{anangk}, which is here said to mark the dative case, is reclaimed by Gale \textit{et al} (2010: 56) to maintain Meyer’s bi-morphemic analysis. Gale proposes that the form comprises the 1sg accusative form \textit{-an} suffixed with -\textit{angk,} which is reclaimed as marking the indirect object on pronouns (ibid.: 56) or the allative case and locative cases on nouns (ibid.: 51).}
\z

\subsubsection{Kin possession}
\label{sec:key:6.1.2.5}\label{bkm:Ref330037209}

Meyer also provided case paradigms for a nominal subclass, termed “pronominal substantives” (\figref{fig:key:100}) and which he described as:

\begin{quote}
probably contracted forms of compound words, but for which the etymology cannot, for the most part, be at present traced. They are all, as far as hitherto known, words expressing relationship of consanguinity. \citep[34]{meyer_vocabulary_1843}
\end{quote}

Here he described how possessive marking on kin relationship terms differed from that on other noun types, a phenomenon found in some other Australian languages \citep[396]{Dixon2002}.

Importantly, Meyer (ibid.: 34) recognised that case marking differed on this nominal subclass, stating: “They differ from substantives in the formation of the cases”. He supplied separate declension tables. He also listed additional terms (ibid.: 36) which he did not decline: his/her father, my brother, his/her brother, my sister, thy sister, his/her sister.

\begin{figure}
\includegraphics[width=.8\textwidth]{figures/stockigt-img041.png}
\caption{Meyer’s case paradigm of terms denoting kin possession \citeyearpar[23]{meyer_vocabulary_1843}.}
\label{fig:key:100}
\end{figure}

Note that in these paradigms of ``pronominal substantives'', the ergative forms termed “ablative” are placed towards the bottom of the paradigm in the traditional position of the Latin ablative. That this occurs here and in the paradigm of substantives (\figref{fig:key:95}) but not in the paradigm of pronouns (\figref{fig:key:96}) may reflect a syncretism of the ergative case with other functions of the Latin ablative, which did not occur on pronouns.

More than half a century later, C. Strehlow (\citeyear{strehlow_untitled_1931}: 50--51 [composed c. 1907]; \sectref{bkm:Ref456272021}), the last early Lutheran missionary-grammarian, recognised that Meyer’s “pronominal substantives” had the same function as a parallel set in Arrernte and as a set described by Reuther in Diyari. Strehlow presented the terms meaning “my father” in the three languages and used Meyer’s terminology “pronominal-substantive” to name the nominal subclass (\figref{fig:key:101}).

\begin{figure}
\includegraphics[width=.8\textwidth]{figures/stockigt-img042.png}\medskip

\legendbox[.8]{\textit{Sowohl in der Aranda als in der Dieri \& Enc. B. Sprache gibt es eine besondere Art von Nomina, die man als Pronomina-Substantive bezeichnen kann, da sie aus Substantiven bestehen, die mit Pronomina poss. oder dem} \textit{unzertrennl. verbunden sind, welch letztere im Deutschen am besten mit – “Eigener” eigen – wiedergegeben warden. Z. B. katiltja} \textit{- (mein eigener Vater)} (C. \citealt{Strehlow1931a}: 50--51 [c. 1907]).}\vspace{.3cm}
\legendbox[.8]{In the Aranda as well as in the Dieri and Encounter Bay language there is a particular type of noun, which one can define as a pronominal substantive, since they consist of substantives, which are inseparably combined with pronouns, poss[essive] or dem[onstrative], with the latter best being rendered in German as - `one’s own'. For example, \textbf{katiltja} (my own father).}
\caption{C. Strehlow’s presentation of terms denoting kin possession in three languages (\citeyear{strehlow_untitled_1931}: 50--51 [c. 1907])}
\label{fig:key:101}
\end{figure}

The inseparable possessive pronouns Strehlow described in Arrernte are currently referred to as “pronominal kin suffixes” (\citealt{Wilkins1989}: 133--135; see also \citealt{Henderson2013}: 260). The Diyari forms Strehlow referred to were first tabulated by \citet[10]{Reuther1894} but are not recorded by \citet[56]{austin_grammar_2013} in the modern descriptive era.

By analogy with circumstances surrounding Reuther’s description of the function of relative pronouns in clause subordination (\sectref{bkm:Ref339792843}), Reuther’s record of the distinct case marking on kinship terms (1894: 12), which had \textit{not} been recorded by earlier Diyari grammarians, nor reported by Austin (2013: 56[1981a]) but \textit{had} been described by \citet[34]{meyer_vocabulary_1843} as ``Pronominal substantives'', and reiterated by C. Strehlow (\citealt[50--51]{strehlow_untitled_1931}; \figref{fig:key:101}) should perhaps be treated with a degree of circumspection. It is, however, also possible that an awareness of these kinship terms in Ramindjeri (\textit{Die Encounter Bay Sprache}) motivated Reuther to elicit the Diyari forms, which may have fallen into disuse by the time of Austin’s recording.

\subsubsection{Meyer’s description of ergativity}
\label{sec:key:6.1.2.6}
\label{bkm:Ref335757427}
\label{sec:6.1.2.6}
\largerpage
Meyer’s description of the ergative case differs from that made by all other early grammarians, except Symmons (\citeyear{Symmons1841}; \sectref{bkm:Ref336779457}). It is possible that Symmons’ work influenced Meyer’s, although there is no other evidence suggesting that Meyer, or any other grammarian, was aware of Symmons' 1841 description of Nyungar. It is likely that \citet{meyer_vocabulary_1843} and \citet{Symmons1841} were independently prompted to make similar analyses, perhaps both having been influenced by \citegen{teichelmann_outlines_1840} presentation of case and of ergativity.

\citet[38]{meyer_vocabulary_1843} described ergative forms as having “the force of Latin ablatives”, and the ergative suffix (ibid.: 61) as “corresponding to the ablative case in Latin, by”. Here Meyer implied the ``ablative of personal agent'', marking the agent performing the action of a passive verb. Consequently, Meyer glosses an overtly marked NP in ergative case with the English preposition “by” in translations of forms supplied in paradigms, in interlinear translation of clauses, and sometimes in free translation of clauses (Example \ref{ex:key:38}). His use of the preposition “by” to translate the ergative NP in a transitive clause, which is rendered in English passive voice, is atypical of the corpus grammars, being otherwise employed only by Symmons (\citeyear{Symmons1841}; \sectref{bkm:Ref73466178}).

\ea\label{ex:key:38}
	Ngate nakk-ir korne
	\glt `by me seeing has been a man' \\
	\citep[33]{meyer_vocabulary_1843} \\
	\gll ngati nak-ir ko:rni \\
	1\textsc{sg}.\stockcaps{ERG}   see-\stockcaps{PAST}   man-[\stockcaps{ACC}] \\
	\glt `I have seen the man'
\z

\ea\label{ex:key:39}
	Ngand-im memp-ing 
	\glt `by whom have you been beaten' \\
	\citep[33]{meyer_vocabulary_1843} \\
	\gll  ngand-im memp-ing\\
	\stockcaps{inter}.\stockcaps{ERG}-2\textsc{sg}.\stockcaps{ACC} hit-\stockcaps{PAST} \\
\z

Meyer’s translation of the overtly marked ergative noun-phrase using a prepositional phrase “by X” renders an English passive construction, in which the agent is demoted to an oblique argument, and the verb is forced into a participle form. Like other early grammarians of Australian languages, Meyer was, nevertheless, capable of imagining that a noun phrase with overt ergative case marking could be translated using a noun in unmarked nominative case, rather than the prepositional phrase ``by X''. Thus, he gave a free translation, ``I have seen the man'', next to his passive interlinear translation ``by me seeing has been a man''. Here it appears that Meyer did not intend to convey that Australian ergative constructions were passive.

\subsubsection{Meyer’s ``Duplex form of the verb'': The antipassive}
\label{sec:key:6.1.2.7}\label{bkm:Ref73965175}
\largerpage
The convoluted syntax of \citegen{meyer_vocabulary_1843} interlinear and free translations of AOV clauses (Example \ref{ex:key:38}) probably arose from the need to account for the existence in Ngarrindjeri of a valency decreasing syntactic process, which occurs in some but not all PN languages \citep[206--207]{dixon_australian_2002}, and which has been termed “antipassive” \citep{Silverstein1972} by analogy with European passive constructions. Under the heading the “Duplex form of the Verb” (\citeyear{meyer_vocabulary_1843}: 38--42), Meyer described an antipassive process in Ramindjeri (\citealt{Terrill1997}; \citealt[206]{Dixon2002}; \citealt[146--152]{Dixon2004}), which is thought not to have been subsequently described in another Australian language for over a century.

The antipassive process involves alteration to verb transitivity through a detransitivising morpheme, occurring before the tense suffix, which places the agent into the unmarked nominative case and the object into peripheral case function. In Ngarrindjeri the object is demoted to a peripheral case that shows syncretism with the ergative. The antipassive counterpart of:

\ea\label{ex:key:40}
\glll Korn-il lakk-in māme \\
Ko:rn-il lak-un  ma:mi \\
man-\stockcaps{ERG} spear-\stockcaps{PRES} fish-[\stockcaps{ACC}] \\
\glt  `The man spears the fish.'
\z

is

\ea\label{ex:key:41}
\glll Korne laggel-in mām-il \\
Ko:rni lak-el-in ma:m-il \\
man-[\stockcaps{NOM}] spear-\stockcaps{ANTIP}-\stockcaps{PRES} fish-\stockcaps{INST}\footnotemark \\
\glt `The man spears the fish.'
\footnotetext{The function of the suffix \textit{-il} marking the oblique argument in these antipassive constructions is not clear. It is glossed here as instrumental, in part by analogy with the neighbouring Lower Murray language Ngayawang in which both instrumental and ergative functions are marked with the seemingly related suffix \textit{-al} (\citealt{Horgen2004}:95, 209). The marking of instrumental function in Ngarrindjeri is not well understood. Based on \citet{meyer_vocabulary_1843} \textit{and} \citet{Taplin1880} instrumental function has been reclaimed by \citet[105--106]{Horgen2004} as being marked with the forms \textit{{}-angk} or -\textit{ung(g)-ay} on singular nouns, which show syncretism with the allative case, but with the form \textit{{}-ung-engg-ul} on dual nouns, which shows syncretism with the locative case. The marking of instrumental function on plural nouns has not been reclaimed.}
\z

In this section of the grammar the interlinear gloss of the underived AVO structure, `man by spearing fish' was said to be equivalent to the English: \textit{by the man is spearing the fish} and \textit{there is a spearing the fish by the man} (ibid.: 39).

Meyer’s unusual account of ergativity in an Australian language and his convoluted translations of underived simple AOV clauses, given under the heading the “Duplex form of the Verb” \citeyearpar[38--42]{meyer_vocabulary_1843} captured the attention of linguists in Germany. Meyer’s grammar of Ramindjeri is consequently referred to and is quoted in the secondary German literature more than other corpus grammars. His clauses were later rendered in German in the earliest philological literature theorising about ergativity (H. C. \citealt{gabelentz_uber_1861}: 489--490; F. \citealt{muller_grundris_1882}: 2). Although Meyer did not intend to convey that Ngarrindjeri had a passive voice, these subsequent interpretations of his depiction of ergativity, which differs from the analysis given in other corpus grammars, resulted in the finding that some Australian languages had a passive voice \citep{StocktingForthcomingPascaleJeanMichel}. Further, such passive interpretations of Meyer’s account of ergative morphology in Ramindjeri were used in the German literature to support claims about the relative evolutionary status of Ramindjeri in comparison with other Australian languages \citep[55]{Stockigtinpress}. The absence of passive voice was seen to be especially indicative of more lowly-ranked languages. H. C. von der Gabelentz, for example, described languages “which do not have a passive at all … [as] allocated to the lowest rank of formation'' \citeyearpar[464]{gabelentz_uber_1861},\footnote{``D]es Passivum in den einzelnen Sprachen selbst über, so begegnen wir zunächst einer ganzen Reihe deselben Welche das Passivum überhaupt nicht besitzen und welche also nach dieser Richtung auf der niedrigsten Stufe der Ausbildung stehen'' \citep[464]{gabelentz_uber_1861}.}\textsuperscript{} and T. G. H. \citet[112]{strehlow_aranda_1944} described this absence as one of the “hallmarks of a primitive language''.

\subsubsection{Meyer’s reference to \citet{Threlkeld1834}}
\label{sec:key:6.1.2.8}

Meyer believed that his view of the antipassivisation process in Ramindjeri was supported by Threlkeld’s illustrative Awabakal material. He stated:

\begin{quote}
Whether our explanation be, or be not, regarded as theoretically correct, these distinctions will at least serve the useful purpose of showing when it is proper to use the forms \textbf{ngape} and \textbf{korne}, and when \textbf{ngate} and \textbf{korn-il} … our view of the case is supported by the analogy with the New South Wales dialect [Awabakal], and the opinion of Mr Threlkeld. \citep[4]{meyer_vocabulary_1843}
\end{quote}

\begin{figure}
\includegraphics[width=.7\textwidth]{figures/stockigt-img043.png}
\caption{Meyer’s reference to \citegen{threlkeld_australian_1834} grammar \citeyearpar[40]{meyer_vocabulary_1843}}
\label{fig:key:102}
\end{figure}

Such presentation of another grammarian’s material in order to substantiate an analysis is atypical of the corpus. The only other grammarians who reproduce material given in another corpus grammar are \citet[76]{ridley_kamilaroi_1855-1}, who, struggling to analyse the Gamilaraay verb, presented Threlkeld’s analysis, and C. Strehlow (\citeyear{strehlow_untitled_1931} [c. 1907]).

Meyer notes that the “participle” of the verb and the alteration to the case of the interrogative in Ramindjeri is similar to the following examples given by Threlkeld.

\ea\label{ex:key:42}
	Nganto wiyan? 
	\glt `Who speaks?' \\
	\citep[127]{Threlkeld1834} \\
	\gll NgaN-Tu wiya-n \\
	\stockcaps{INTER}-\stockcaps{ERG}   speak-pres \\
\z

\ea\label{ex:key:43}
	Ngan                  unnung                wiyellin                       yong 
	\glt Who there talking out there \\
	\citep[127]{Threlkeld1834} \\
	\gll NgaN                   aNang       wiya-li-N          yung \\
	\textsc{inter}.\stockcaps{NOM}      that\stockcaps{NOM}        speak-\stockcaps{DTR}-\stockcaps{PRES}    there\footnotemark\\
	\footnotetext{The gloss of the morpheme \textit{-li \-} as detransitivising differs from the assessment made by \citet[79]{lissarrague_salvage_2006}, who shows the morpheme in the same construction as marking continuous aspect. \citet[117]{oppliger_phonology_1984} names the morpheme “continuative derivational”. It is possible that the language had two formally distinct morphemes, one marking continuative aspect and the other with derivational application, which remain phonemically undifferentiated in the modern interpretation (see, e.g., \citealt{threlkeld_australian_1834}:48).}
\z

Contrary to \citegen[41]{meyer_vocabulary_1843} suggestion that Threlkeld had “not noticed” these features, it is probable that Threlkeld \textit{did} notice that the first of these interrogative pronouns stood in the “active nominative” (ergative) case and the second in the “simple nominative” (nominative), and that the alteration was associated with the shape of the verb. Threlkeld’s case paradigm of interrogative pronouns (ibid.: 7--8) shows \textit{ngan-to} as the active nominative and translates \textit{ngan} as ``who?''.

\largerpage
Following Meyer, these same Awabakal clauses were again republished by H. C. \citet[489]{gabelentz_uber_1861} in a discussion of ergative function.

\subsubsection{``Relative pronouns''}
\label{sec:key:6.1.2.9}\label{bkm:Ref73381962}

Although Meyer’s innovative approach to the description of case was forged independently of the analysis given by Teichelmann and Schürmann (\citeyear{teichelmann_outlines_1840}; \sectref{bkm:Ref456096204}), his description of processes of clause relativisation imports his Lutheran predecessors' analysis across language boundaries. Meyer assumes Teichelmann and Schürmann’s treatment of Kaurna and imposes it directly on Ramindjeri.

\citet[33]{meyer_vocabulary_1843} supplied Example \ref{ex:key:44} to illustrate his assertion that sometimes personal pronouns could also at times act to relativise a clause. He stated, somewhat hesitantly, “[T]he personal pronoun \textbf{kitye}, he, \textit{seems} also sometimes to perform the office of a relative” (emphasis added=.

\ea\label{ex:key:44} % gloss needs re-alignment, confusing in orig
	\gll Ngāte {} nakk- ir korne, yarn-… ir an-ang-itye watañgrau \\
	{By me} (a) seeing {has been} man, speaking was {me to he} yesterday \\
	\glt `I have seen the man who spoke to me yesterday' \\
	\citep[33]{meyer_vocabulary_1843} \\
	\gll Ngati nak-ir ko:rni, yarn-ir-anang-itye watangrau \\
	1\textsc{sg}.\stockcaps{erg} see-\stockcaps{past} man-[\stockcaps{acc}]   speak-\stockcaps{past}-1\textsc{sg}.\stockcaps{all}/\stockcaps{dat}-3\textsc{sg}.\stockcaps{nom}       yesterday \\
\z

Unlike in Teichelmann and Schürmann’s examples (\ref{ex:5:31} on p. \pageref{ex:5:31} \& \ref{ex:5:34} on p. \pageref{ex:5:34}), there is no morphology indicating dependency. Meyer shows two independent clauses: “I saw the man. He spoke to me yesterday”. The bound pronominal compound –\textit{anang-itye,} which Meyer represented as a free-form and into which he inserted two morpheme boundaries, is comprised of the first-person dative bound form pronoun \textit{anang} (see note at Example \ref{ex:key:37}), to which is attached the third-person nominative bound pronoun –\textit{itye} (\figref{fig:key:99}), of which the free-form is \textit{kitye}. It is this last element –\textit{itye} which Meyer analysed as “fulfilling the function of the relative”.

\citet[32]{meyer_vocabulary_1843} also stated, again hesitantly, “[T]he demonstrative pronoun nāiye, that, \textit{appears} also to be sometimes used as a relative, as in the following examples” (emphasis added; \figref{fig:key:103}). Again, there is no morphosyntactic motivation for analysing any of the clauses he supplied (Examples \ref{ex:key:45} \& \ref{ex:key:46}) as subordinate.

\begin{figure}
\includegraphics[width=.8\textwidth]{figures/stockigt-img044.png}
\caption{Meyer’s illustration of demonstrative pronouns acting to form subordinate clauses \citeyearpar[32]{meyer_vocabulary_1843}. The notes to which Meyer refers explain processes of morphophonemic alternation.}
\label{fig:key:103}
\end{figure}

\ea\label{ex:key:45}
	\gll Nāiye lēw-…in mant-angg-an \\
	That {living  is} {house at my} \\
	\glt `He who lives at my house' \\
	\citep[32]{meyer_vocabulary_1843} \\
	\gll Naiyi le:w-in mant-angk-an \\
	DEM sit-\stockcaps{PRES} house-\stockcaps{LOC}-1\stockcaps{POSS} \\
	\glt `that one lives at my house'
\z

Example \ref{ex:key:46} has an interrogative rather than a demonstrative pronoun:

\ea\label{ex:key:46} % unclear
	\gll  Ngande-….m-..angg-…..engg-ul {}  ram-……ing? \\
    {By whom you to them two}  {by (a)} {speaking was}? \\
	\glt `Who are the two who told you?' \\
	\citep[32]{meyer_vocabulary_1843} \\
	\gll ngandi-mangk-engkul ram-ing \\
	\stockcaps{inter}.\stockcaps{erg}-2\stockcaps{dat}-3\textsc{dl}.\stockcaps{erg} speak-\stockcaps{past}\\
	\glt `which two spoke to you?'\footnote{The number of the 2DAT -\textit{mangk} bound pronoun is unspecified. See \sectref{sec:key:7.3.2}.}
\z

Example \ref{ex:key:45} can only be translated as ``that one lives at my house'', and Example \ref{ex:key:46} as ``which two spoke to you?''. Despite Meyer’s claim, there is \textit{no} evidence that pronouns form relative or subordinated clauses in this Ramindjeri data. It seems that Meyer has misapplied the conclusions drawn from Teichelmann \& Schürmann’s Kaurna data onto his own.

Meyer imported Teichelmann \& Schürmann’s claim that demonstrative pronouns could act to relativise clauses in Kaurna, and sought to find evidence that personal pronouns in Ramindjeri could similarly act to subordinate clauses. His misappropriation of the Kaurna data served to feed a developing doctrine in the Lutheran sub-corpus that pronouns act to relativise clauses in Australian languages (\sectref{bkm:Ref339742580}, \sectref{bkm:Ref74838296}, \sectref{sec:key:9.3.5.1}).

\subsection{Concluding remarks}
\label{sec:key:6.1.3}

This second published description of a South Australian language is descriptively innovative. Rather than implementing the traditional paradigmatic presentation used by his fellow Dresden-trained colleagues \citet{teichelmann_outlines_1840}, Meyer’s presentation of case (\sectref{bkm:Ref456091394}) and his conception of the syntactic case frame of transitive verbs (\sectref{bkm:Ref335757427}) differ from that of his Lutheran predecessors. Neither of these aspects of Meyer’s grammar were subsequently as influential on the Lutheran sub-corpus as were the schemata employed by Teichelmann \& Schürmann, but Meyer’s paradigm influenced G. Taplin (\sectref{sec:key:7.3}) who commenced description of Ngarrindjeri two decades after the closure of Meyer’s Encounter Bay mission. Meyer’s influence on Taplin’s paradigms resulted in the genesis of the term ``ergative'', although \textit{not} referring to the syntactic case (\sectref{sec:key:2.6}).

Meyer’s listing of ``particles'', and his description of their function, may show a direct influence from Threlkeld that is not present in \citet{teichelmann_outlines_1840}. Other aspects of Meyer’s description that resemble Threlkeld’s work, but which are not shared with Teichelmann \& Schürmann, include the use of hyphens to mark the meaningful word-internal units, and the inclusion of interlinear-style translations.

Although the format of Meyer’s grammar differs from Teichelmann \& Schürmann’s earlier work \citeyearpar{teichelmann_outlines_1840}, Meyer’s analysis of certain PN structures was made with the benefit of the accumulated understanding gained within the Adelaide School. Meyer’s description of kin-possession (\sectref{bkm:Ref330037209}) is especially sophisticated. The clarity with which Meyer presents bound pronouns (\sectref{bkm:Ref73463170}) shows a marked improvement on Teichelmann \& Schürmann’s burying of the forms in traditional schema that were seen to be functionally equivalent (\sectref{sec:key:2.5}).

Teichelmann and Schürmann’s grammar nevertheless held sway over Meyer’s analysis. While Meyer’s pronominal case paradigms differed from all preceding PN representations of case, the placement of the ergative case forms towards the bottom of the paradigm shows Teichelmann \& Schürmann’s influence, and Meyer’s discussion of processes of clause subordination (\sectref{bkm:Ref73381962}) is unmistakably guided by his Lutheran predecessors.

\section{Schürmann’s grammar of Barngarla \citeyearpar{schurmann_letter_1844}}
\label{sec:key:6.2}\label{bkm:Ref73471921}\label{bkm:Ref74127491}\label{bkm:Ref73712365}

This and the following section investigate the last two grammars that comprise the ``Adelaide School'': Lutheran missionary \citegen{schurmann_vocabulary_1844} grammar of Barngarla and \citegen{moorhouse_vocabulary_1846} grammar of Ngayawang. Comparison of these works with the earlier grammars of the school, \citet{teichelmann_outlines_1840} and \citet{meyer_vocabulary_1843}, shows that while these grammarians learned from one another, these works employ a range of diverse descriptive practices.

In 1840 Schürmann was offered the position of Deputy Protector of Aborigines at Port Lincoln, which had been settled the previous year (\citealt{Lockwood2014}: 93, 121). Schürmann was reluctant to relocate to such an isolated outpost, as he had anticipated working at Encounter Bay with Meyer. His choice was, however, made for him, as he himself described: “His Excellency [i.e., the Governor of South Australia] … refuted all my objections, saying that I had been sent to S. A. generally and not to any particular portion of it” (\citealt{schurmann_vocabulary_1844}, reprinted in \citealt{schurmann_id_1987}: 109).

Thus, Schürmann ceased to work in close connection with his Dresdner brothers Teichelmann and Klose among the Kaurna in Adelaide, and Meyer among the Ramindjeri at Encounter Bay, and commenced working as Protector at the remote settlement of Port Lincoln, an area which at the time covered most of the Eyre Peninsula.

Schürmann was retained in the area in some missionary capacity until early 1846, although his ability to act as a two-way interpreter between Aboriginal people and the European justice system kept him in high demand, and in 1842 he was recalled to Adelaide to act as court interpreter. It was on his return to Port Lincoln in 1843 that Schürmann commenced working on a MS dictionary of Barngarla, which was sent to Grey that year. Schürmann perceived this work, which contained 2000 entries, as incomplete \citep[77]{rathjen_difficult_1998}. In 1844 he published \textit{A vocabulary of the Parnkalla, spoken by the natives inhabiting the western shores of Spencers' Gulf. To which is attached a collection of grammatical rules.} It is not clear whether the attached grammar had also been sent to Grey. Schürmann described that the work was written

\begin{quote}
[t]o assist those settlers in Port Lincoln who may feel disposed to acquire the native language. In more thickly populated districts around Adelaide, the colonists have less occasion to learn the language of the Aborigines, since the latter can speak English intelligibly. But in Port Lincoln where the white population bears so small a proportion to that of the black, quite the reverse obtains, so that to establish and maintain intercourse with the natives there, it seems almost necessary that the Settlers should endeavour to make themselves acquainted with the language. \citep[iii]{schurmann_vocabulary_1844}
\end{quote}

Schürmann’s stated pedagogical motivation for producing the Barngarla grammar contrasts with the impetus Meyer described in the introduction to his Ramindjeri work \citeyearpar[vii]{meyer_vocabulary_1843}, hoping it would prove “interesting to the philosopher and philologist”. \citet{Simpson2021} points out that the different reasons described by Meyer and Schürmann for publishing their grammars are likely to have affected the style in which each was written, and the schemata engaged. While sections of Meyer’s work are almost impenetrable -- “The duplex form of the verb” (\sectref{bkm:Ref73965175}), for instance -- Schürmann’s work is targeted towards a broader audience. Like Meyer, Schürmann does, however, abandon the traditional and easily understood framework when describing case marking on nouns (\sectref{bkm:Ref456091421}).

In the year that his ethnographic description was published (\citeyear{schurmann_aboriginal_1846}), Schürmann left the Eyre Peninsula to work with Meyer at Encounter Bay, and in 1848, after witnessing the final stage of the Encounter Bay mission and after the closure of Lutheran missions in South Australia, he returned to Port Lincoln as court interpreter (1848--1853). At the request of H. Young (1803--1870), Governor of South Australia (1848--1854), he opened a school for Aboriginal children just to the north of Port Lincoln in 1849. The school operated until 1852, when students were moved to the nearby, better-funded Anglican mission school at Poonindie, which had opened in 1850. Aboriginal people with whom the Dresdners had worked in Adelaide were also relocated to Poonindie near Port Lincoln. Finally, Schürmann moved to Western Victoria in 1853, where, like Meyer and Teichelmann, he worked as a pastor in European Lutheran congregations until his death. No further grammatical analysis of Barngarla or any other Aboriginal language was made by the Anglicans at Poonindie.

Like \citet{kempe_grammar_1891}, who made an inaugural description of Arrernte close to first contact (\sectref{bkm:Ref74837211}), and like missionaries Hey and Ward at the Mapoon mission in far north Queensland (\sectref{bkm:Ref514607540}), \citet[249]{schurmann_aboriginal_1846} observed a multitude of regional dialects: “The principal mark of distinction between the tribes is the difference of language or dialect”. He observed that of the several “tribes” inhabiting the Eyre Peninsula, two were in daily contact with Europeans, the Barngarla who inhabited the eastern coast of the peninsula, and the Nauo from ``the southern and western parts of the district''. Schürmann collected Barngarla data at Port Lincoln from people who had already shifted away from traditional territory towards European settlements and replaced the previous Nauo populations (\citealt{hercus_tragedy_2001}).

\subsection{Schürmann’s analysis of Barngarla \citeyearpar{schurmann_vocabulary_1844}}
\label{sec:key:6.2.1}\label{bkm:Ref73715161}

When naming his twenty-two page grammatical analysis (\citeyear{schurmann_vocabulary_1844}) a ``collection of grammatical rules … `prefixed' to a vocabulary'', Schürmann reveals his own view of the work. While the dictionary contains some 3000 entries \citep[83]{rathjen_difficult_1998}, approximately the length of the Kaurna vocabulary (\citeyear*{TeichelmannSchürmann1840}), the grammar contains far less exemplification than do the earlier Dresdner grammars (\citealt{teichelmann_outlines_1840}; \citealt{meyer_vocabulary_1843}). There is no accompanying “phraseology” section.

\citet[iv]{schurmann_vocabulary_1844} noted the lexical similarities between Kaurna and Barngarla, while also observing that “[i]n forming an opinion on the affinity of languages or dialects, one has to look not only to the number of similar words, but still more to the grammatical structure and idiom”. Schürmann’s lexical comparison is sophisticated in comparison with others of the same era (\citealt{grey_vocabulary_1839}; \citealt{moorhouse_letter_1840}). He shows that while some verb stems are similar in “Adelaide” and “Parnkalla” (\figref{fig:key:104}) the final ``syllable'', i.e., inflection for present tense, differed.


\begin{figure}
\includegraphics[width=.8\textwidth]{figures/stockigt-img045.png}
\caption{Schürmann’s comparative Barngarla and Kaurna vocabulary \citeyearpar[iv]{schurmann_vocabulary_1844}}
\label{fig:key:104}
\end{figure}

Schürmann also observed that the initial consonants of lexical items in Kaurna were commonly dropped or lenited in Barngarla (\citealt{simpson_thura-yura_2004}: 189). Note that there is no evidence that missionaries working at the Hermannsburg mission between 1877 and 1920, who were learning two languages, Arrernte and Luritja, at the mission observed initial consonant deletion in Arrernte (see \citealt{koch_methodological_2004}: 135--136) on words that are cognate in these two languages. This observation was first made by W. Schmidt (\citeyear[50]{schmidt_gliederung_1919}; see further \citealt[24]{koch_arandic_2004}).


Having previously prepared his analysis of Kaurna for publication with \citet{teichelmann_outlines_1840}, and probably having assisted Meyer, or at least partially overseen Meyer’s analysis of Ramindjeri, Schürmann approached the description of Barngarla with confidence. The work casts aside some of the traditional schemata engaged in the earlier Kaurna grammar and provides instead concise description of structures in their own terms. Of the adjective he (\citeyear{schurmann_letter_1844}: 9) observed: “The adjective differs with regard to form in no wise from the substantive, it being susceptible not only of a dual and plural numbers, but also of all the suffixes”.

\citet[9--10]{schurmann_vocabulary_1844} also provided clear descriptions of the marking for case and number on a noun-phrase (\figref{fig:key:105}). Schürmann’s simple statement and demonstration efficiently conveys that number (and case) are marked only on the final constituent of the Barngarla noun phrase. Schürmann’s examples do not, however, clarify the relative ordering of inflection for number and case.


\begin{figure}
\includegraphics[width=.8\textwidth]{figures/stockigt-img046.png}
\includegraphics[width=.8\textwidth]{figures/stockigt-img047.png}
\caption{Schürmann’s explanation and exemplification of plural marking (and case) on the NP \citeyearpar[9--10]{schurmann_vocabulary_1844}}
\label{fig:key:105}
\end{figure}


\ea % needs re-formatting
	Mangalla yurarri innamatta
	\glt `friendly men (are) these' \\
	\citep[10]{Schürmann1844} \\
	\gll Mangarla yurha-rri inha-madla \\
	Friendly man-\stockcaps{hum}.\stockcaps{PL}    this-\stockcaps{pl}\\
	\glt (Gloss and transcription, \citealt[161]{Clendon2015})
\z

In discussing Barngarla structure, Schürmann references an existing body of knowledge about Australian morphology and syntax. He listed \citeyearpar[v]{schurmann_vocabulary_1844} all the previously published Australian material. This placement of the work within a broader body of literature is not characteristic of later Lutheran description in South Australia, or of early Australian grammars more broadly. When discussing the inflection for ergative case on nouns, for example, he wrote: “\textit{nga} forms what has been \textit{termed by other writers upon the idiom} of Australia, the `active nominative case' ” (emphasis added).

Concerning gender, \citet[3]{schurmann_vocabulary_1844} wrote: “\textit{In accordance with the other Australian dialects} no distinctions on account of gender have been discovered in the Parnkalla language” (emphasis added). Note that, despite this statement, Threlkeld had clearly described the Awabakal gender distinction on 3\textsuperscript{rd} person pronouns (\sectref{sec:key:3.3.2.1}). Similarly, Livingstone (\citeyear{Livingstone1892}; \sectref{sec:4.6.1.1}) did not describe the third-person pronominal gender distinction in Minjangbal as ``gender''. Pronominal distinction in third-person pronouns was not seen as constituting the category “gender” by early grammarians.

In his ethnographic description, \citet[29--30]{schurmann_aboriginal_1846} summarised the most “striking peculiarities” of the language. The material was also published in the Dresden \textit{Missionsblatt} \citep{rathjen_difficult_1998}. Schürmann listed nine “peculiarities” common to the structure of the languages he knew. His list is similar to that given by Moorhouse (\citeyear{moorhouse_vocabulary_1846}; \sectref{bkm:Ref340310942}) and to that captured in Taplin’s 1875 circular (\sectref{bkm:Ref514607617}). Most of Schürmann’s discussion concerns features that the languages were perceived as lacking: fricatives, articles, grammatical gender, numerals greater than three, reflexive pronouns, passive voice and prepositions. He also observed that the languages shared: morphological marking of dual number, pronominal sensitivity to kinship, suffixation of nouns with case inflection, and complex verb morphology (\citeyear{moorhouse_vocabulary_1846}: 250--251).

The impression emerging in the 1840s was that Australian languages belonged to a single family \citep{grey_languages_1845} by nature of their shared grammatical structure, as well as shared lexicon. The idea was supported by empirical grammatical evidence in three publications made in the same year: H. \citet[479]{hale_languages_1846}, \citet[vi]{moorhouse_vocabulary_1846}, and \citet[29--30]{schurmann_aboriginal_1846}.

\subsubsection{Case suffixes or prepositions}
\label{sec:key:6.2.1.1}\label{bkm:Ref456091421}

Like Meyer, Schürmann did not present case paradigms of nouns, but instead listed and described the function of nominal suffixes. Following Meyer’s earlier innovation, he listed twenty-four nominal suffixes (\citeyear{schurmann_letter_1844}: 4--8) in place of a case paradigm. The choice of presentation allowed Schürmann to show that the suffix marked different nominal types with different function. Examine, for instance, \figref{fig:key:106}, in which Schürmann describes the function of suffixes marking locative and allative cases on proper nouns and pronouns.


\begin{figure}
\includegraphics[width=.8\textwidth]{figures/stockigt-img048.png}
\caption{Schürmann’s non-paradigmatic representation of locative and allative case on proper nouns \citeyearpar[6]{schurmann_vocabulary_1844}}
\label{fig:key:106}
\end{figure}

Schürmann’s grammar of Barngarla \citeyearpar{schurmann_vocabulary_1844} has previously been assessed by \citet[83--88]{rathjen_difficult_1998} who focuses on Schürmann’s method of description, concluding that the 1844 grammar:

\begin{quote}
reveals a significant progression in linguistic thought from the conventionally documented Adelaide grammar. Schürmann is attempting to record the language outside of the classical grammatical framework in which he had been trained, a task which required him to develop new linguistic models and ways of thinking. \citep[88]{rathjen_difficult_1998}
\end{quote}

Rathjen’s assessment of \citet{schurmann_letter_1844} is made in comparison with \citet{teichelmann_outlines_1840} as well as with Schoknecht’s 1872 description of Diyari (\citealt{schoknecht1872}; \sectref{bkm:Ref456087018}). Rathjen’s finding that Schürmann innovated new descriptive schemata in order to frame Barngarla structure rests largely on his description of case suffixes. She writes:

\begin{quote}
	In the Parnkalla grammar, there is no attempt to provide noun paradigms (declensions), although there is a rudimentary attempt to do this in the Adelaide grammar, Schürmann is now aware of the suffixing nature of the language where a “variety of terminations supply the place of cases and prepositions, conjunctions and adverbs” … Whereas Schoknecht’s \textit{Grammar of the Language of the Dieri Aborigines’} some 30 years later provides classical paradigms … Schürmann is already aware of the inappropriateness of reducing the native language to such a structure. \citep[86]{rathjen_difficult_1998}
\end{quote}

Rathjen’s assessment fails to recognise that Schürmann’s practice followed Meyer’s, original innovation.

Schürmann put to good use this method of listing and discussing the function of sub-word units in order to demonstrate the multiple functions of a single morpheme. He showed that the ergative “active nominative” suffix –\textit{nga} marked instrumental “ablative” function, the locative case spatially and temporally, and had a causal function (\citeyear{schurmann_letter_1844}: 4--5; \figref{fig:key:107}; Example \ref{ex:key:48})


\begin{figure}
\includegraphics[width=.5\textwidth]{figures/stockigt-img049.png}
\includegraphics[width=.5\textwidth]{figures/stockigt-img050r.png}
\caption{Schürmann’s non-paradigmatic presentation of the function of the suffix marking ergative, instrumental, locative and causal case functions \citeyearpar[4–5]{schurmann_vocabulary_1844}}
\label{fig:key:107}
\end{figure}

\ea\label{ex:key:48}
Marrályinga ngai píttanarrù kányanga
\glt `The boy me did hit with a stone'\\
\citep[5]{schurmann_vocabulary_1844} \\
\gll Marralyi-nga ngayi birda-nga-aru ganya-nga \\
Boy-\stockcaps{ERG} 1\textsc{sg}.\stockcaps{ACC}  hit/pelt-\stockcaps{PAST}-3\textsc{sg}.\stockcaps{ERG} stone-\stockcaps{INST} \\
\z

Schürmann’s application of Meyer’s descriptive innovation also drew Elkin’s attention, who in 1937 commended Schürmann’s choice of presentation:

\begin{quote}
	In some languages there are suffixes to indicate what seems to be every conceivable condition of the noun. Schuermann gives twenty-four for Parnkalla …Schuermann rightly described these particles by their function and did not try to invent terms to describe them, like exative (=from) ergative (=with) ablative (=by) which were used by the Rev. J. Bulmer. \citep[150]{elkin_nature_1937}
\end{quote}

 The alternative practice, which Elkin judges so unfavourably, was in fact instigated by Taplin in his description of Ngarrindjeri (\citeyear{taplin_notes_1872}; \citeyear{taplin_grammar_1878}) and was followed by both \citet{hagenauer_language_1878} and \citet{bulmer_language_1878}, who replicated Taplin’s paradigm in descriptions of Wergaya and Ganai published in Brough Smyth (\sectref{sec:key:7.2}).

Schürmann’s choice not to provide case paradigms for nouns may additionally have been motivated by his perception that nominal inflection for case in Barngarla did not constitute declension, since the marking of case was not fused with the marking of number (\sectref{sec:key:2.4.1.1}). He wrote: “there is no declension of substantives in the common sense of the word” \citep[4]{schurmann_vocabulary_1844}.

\subsubsection{Pronominal case paradigms}
\label{sec:key:6.2.1.2}

Like \citet{meyer_vocabulary_1843}, Schürmann did, however, “reduce” Barngarla pronouns to declension (\citealt{schurmann1840}: 10--14). Unlike Meyer, and the earlier grammar Schürmann co-authored with \citet{teichelmann_outlines_1840}, Schürmann placed the ergative case, here termed ``active nominative'', in second position next to the nominative form (\figref{fig:key:108}). There is no association of the ergative case with the label ``ablative'', or with the position of the ablative case in Latin paradigms. Unlike previous presentations of ergativity given in South Australia (\citealt{teichelmann_outlines_1840}; \citealt{meyer_vocabulary_1843}), \citet{schurmann_letter_1844} conceived of the ergative case as a second type of nominative.


\begin{figure}
%\includegraphics[width=.8\textwidth]{figures/stockigt-img051.emf}
\includegraphics[width=.5\textwidth]{figures/stockigt-img052.png}
\caption{Schürmann’s singular and plural pronominal case paradigm \citeyearpar[11]{schurmann_vocabulary_1844}}
\label{fig:key:108}
\end{figure}

\subsubsection{Bound pronouns}
\label{sec:key:6.2.1.3}\label{bkm:Ref73561555}\label{bkm:Ref73561601}

Schürmann’s description of Barngarla, made the year following Meyer’s publication, continued to present bound pronouns within a discussion of “verbs” rather than presenting the forms as an additional pronominal paradigm, as Meyer had done. Rather than showing the forms as marking a mood of the verb, he shows them as person and number markers. His choice probably does \textit{not} indicate that bound pronouns were largely restricted to attachment to verbs. There is evidence in both languages that they could also attach to interrogative and demonstrative pronouns (Examples \ref{ex:5:34} \& \ref{ex:5:35} on p. \pageref{ex:5:34}).

Regarding his exemplification of the forms of pronouns used with transitive and intransitive verbs (\sectref{sec:key:6.2.1.6}; \figref{fig:key:112}; \figref{fig:key:113}), Schürmann stated:

\begin{quote}
	In the above paradigms the pronouns have been placed before the verb to show the full form of both the verbs and the pronouns; but the natives very commonly pronounce the pronoun after the verb and more or less contract the two into one word. \citep[22]{schurmann_vocabulary_1844}
\end{quote}

He provided an additional paradigm showing nominative bound pronouns attached to the intransitive verb \textit{nguka-ta}, go-PRES/FUT, and ergative bound pronouns attached to the transitive verb \textit{witti-ta,} spear-PRES/FUT for each person in singular, dual and plural number (\figref{fig:key:109}). Note that the anticipated 3dl bound pronoun \textit{-alanbi} is not given, indicating that Schürmann did not “fill-in” anticipated forms in this paradigm.


\begin{figure}
\includegraphics[width=.65\textwidth]{figures/stockigt-img053.png}
\caption{Schürmann’s presentation of bound pronouns \citeyearpar[22]{schurmann_vocabulary_1844}}
\label{fig:key:109}
\end{figure}

Although Schürmann did not tabulate bound forms of pronouns in accusative function, there is evidence that an object could be marked by a bound pronoun. Examine the following example in which the form \textit{-adli,} a reduction of the 1dlACC pronoun \textit{ngadli} (\figref{fig:key:110}), stands in accusative case.

\ea\label{ex:key:49} % formatting
\gll {Karpanga   {iridningutu adli}} \\
{`the house/room separates us'} \\
\glt (\citealt[V:8]{schurmann_vocabulary_1844}) \\
\gll garrba-nga   iridni-ngu-dhu-w-adli \\
house-\stockcaps{ERG}   separate-\stockcaps{CAUS}-\stockcaps{PRES}-\stockcaps{EP}-1\textsc{dl}.\stockcaps{ACC}\\
\glt (Gloss and transcription from \citealt[44]{Clendon2015})
\z

In contrast to this analysis of bound pronouns, \citet[345]{dixon_preface_2002} concludes that the system of bound pronouns in Barngarla is unusual among Australian languages in not marking object function. Dixon’s error results from taking the early sources at face value without unpicking the grammarians' rationale for describing the language in ways that the traditional paradigm could accommodate. Bound pronouns in accusative function were not as easily illustrated within traditional schemata, because verbs in SAE languages do not agree with pronouns in accusative case. \citet[203--204]{koch_r_2008} points out that Mathews attempted, but struggled, to account for the agreement of verbs with their object, i.e., the attachment of accusative bound pronouns to verbs.

\subsubsection{Pronouns specifying kinship relations}
\label{sec:key:6.2.1.4}\label{bkm:Ref456255508}

Schürmann’s choice to present paradigms of pronouns may have been motivated by his wish to show in an organised manner his analysis of pronouns specifying kin relationship. The forms Schürmann described were probably part of what was a much larger system of pronouns specifying the kin relationship that may have resembled the complexity recorded by \citet{schebeck_adnjamathanha_1973} in Adnyamathanha, a northern member of the Thura-Yura family (see also \citealt{hercus_perception_1973}).

Numbers placed inside Schürmann’s dual paradigms (\figref{fig:key:110}) were marked to indicate that the form showed a kinship reference between the pair referred to by the dual pronoun \citep[11--13]{schurmann_vocabulary_1844}. These pronouns that refer to dyadic kin relations \citep{evans_interesting_2003} denote kin pairs and are thus categorically different from the possessive kinterms Meyer had described in Ngarrindjeri (\sectref{sec:key:6.1.2.5}).

The number ``3'' placed next to a form in Schürmann’s paradigm (\figref{fig:key:110}) is footnoted to indicate that marked 1dl and 2dl forms specified that the referents were “certain relatives, as a mother and her children, uncle and nephew”. The number ``4'' next to other forms in the same paradigms marked that the referents were “a father and one of his children only”. The number ``5'' placed next to the 3dl form indicated the referents were spouses “husband and wife”.


\begin{figure}[t]
\includegraphics[width=.8\textwidth]{figures/stockigt-img054.png}
\includegraphics[width=.8\textwidth]{figures/stockigt-img055.png}
\caption{Schürmann’s dual pronominal case paradigm \citeyearpar[11–12]{schurmann_vocabulary_1844}}
\label{fig:key:110}
\end{figure}

Schürmann placed the number “2” next to an alternative 2sg form \textit{nuro}, which was declined alongside the unmarked 2sg form \textit{ninna}. The form \textit{nuro} was described as being used “by a father and his children addressing each other”.

Schürmann was not, however, the first Australian grammarian to record such forms. \citet[xiv–xv]{symmons_grammatical_1841} had earlier recorded three sets of dual pronouns in each number in Nyungar (\figref{fig:key:111}). The first specifying same generation ``brother and sisters and friends'', the second different generation ``uncle and nephew, parent and child'' and the third spousal ``husband and wife''. Schürmann is not known to have been acquainted with Symmons' Western Australian grammar.


\begin{figure}
\includegraphics[width=.8\textwidth]{figures/stockigt-img056.png}
\includegraphics[width=.8\textwidth]{figures/stockigt-img057r.png}
\caption{Symmons' presentation of pronouns specifying kinship \citeyearpar[xiv-xv]{symmons_grammatical_1841}}
\label{fig:key:111}
\end{figure}

That some Australian languages had sets of pronouns expressing kinship relations was observed as a typological feature by \citet[5]{ray_aboriginal_1925}, who gave Schürmann’s “Parnkalla” forms as illustration. Ray (ibid.: 5) also observed that “[i]n Western Australia similar terms are very numerous”. It is not known whether \citet{symmons_grammatical_1841} was Ray’s Western Australian source.

\subsubsection{The naming of the ergative case}
\label{sec:key:6.2.1.5}

Schürmann used the term “active nominative” to describe the ergative case. The term had, as Schürmann himself observed, been used by previous grammarians in Australia: \citet{threlkeld_australian_1834} and Günther (\citeyear{gunther_native_1838}, \citeyear{gunther_lecture_1840}). The term had not, however, been used by Schürmann in the grammar of Kaurna written with \citet{teichelmann_outlines_1840}, or by \citet{symmons_grammatical_1841}, or by \citet{meyer_vocabulary_1843}, except in the section of his grammar describing the “duplex form of the verb” (\sectref{sec:key:6.1.2.7}). Following Schürmann the term “active nominative” was subsequently used only by \citet{moorhouse_vocabulary_1846}, W. \citet{bleek_library_1858}, and in Fraser’s republication of Günther (\citeyear{gunther_grammar_1892}; \tabref{tab:chap3:ergativepm}).

\subsubsection{Clarification of ergativity under the heading ``The Verb''}
\label{sec:key:6.2.1.6}\label{bkm:Ref336283270}\label{bkm:Ref336350800}
\largerpage

Starting with \citet{Threlkeld1834}, early Australian grammarians classified verbs according to the arguments that they selected. In what is a confusing discussion, Threlkeld subclassified verbs into two overarching categories, an “active” class (transitive), which attributed “an act to an agent” and a “neuter” (intransitive) class, which attributed ``a state of being to a subject''. Each of Threlkeld’s two classes was subject to fifteen listed “accidents” (\citeyear{threlkeld_australian_1834}: 28), which accounted for forms derived through reciprocal, reflexive, and other valency altering processes. Threlkeld’s terminological opposition, “active” vs “neuter”, had been previously used in accounts of Tibetan ergativity from the 1830s \citep[130]{vollmann_descriptions_2008}.

Teichelmann and Schürmann’s grammar (\citeyear{teichelmann_outlines_1840}: 14--15) introduced the terms “transitive” and “intransitive” into the description of the Australian verb. Their source for these terms is not clear. They gave six genera of the verb, the first of which was termed ``active, or transitive'', and the second ``neuter or intransitive''. The division of verbs into these two classes was motivated by the rigidity with which Kaurna verbs select either a single S argument or an A and an O argument. The tendency is shared with most Australian languages, in which very few verbs are ambitransitive \citep[176--178]{dixon_preface_2002}. The division became a common feature of the Adelaide School grammars (see \citealt{simpson_i_2008}: 123--124) and grammars of Diyari. It was in a discussion of these subtypes of verb that Schürmann, and the later grammarians who followed him, clarified the different function of nominals in ergative and nominative cases.

\begin{figure}[b]
\includegraphics[width=.5\textwidth]{figures/stockigt-img058.png}
\caption{Schürmann’s “neuter” verb showing pronouns in one type of “nominative” case (\citeyear[17]{schurmann_vocabulary_1844})}
\label{fig:key:112}
\end{figure}

Schürmann showed “the application of the two nominative cases of the Pronouns [by] select[ing] one Neuter and one Active Verb” (\citeyear[16]{schurmann_vocabulary_1844}). Each tense and mood of the verb was exemplified twice, first using the neuter intransitive verb \textit{nguka-} `to go' (\figref{fig:key:112}) with pronouns in nominative case, and then using the active transitive verb \textit{witti}{}- `to spear' (\figref{fig:key:113}) with pronouns in ergative case. Schürmann thus successfully conveyed the role of verb transitivity in argument predication.




\begin{figure}
\includegraphics[width=.5\textwidth]{figures/stockigt-img059r.png}
\caption{Schürmann’s “active” verb showing pronouns in another type of “nominative” case (\citeyear[20]{schurmann_vocabulary_1844})}
\label{fig:key:113}
\end{figure}

Following Schürmann, a group of grammarians explained ergative function not in a discussion of nominal morphology, but rather under the word-class heading ``the verb''. The grammarians who elucidate the function of an ergatively marked nominal through demonstrating the role of verb transitivity in determining syntactic case frames include \citet[19]{Moorhouse1846}, \citet[14]{taplin_grammar_1880} and grammarians of Diyari (\citealt{koch_untitled_1868}: no pag.; \citealt{schoknecht_grammar_1947}: 9; \citealt{flierl_christianieli_1880}: 32; \citealt{reuther_three_1981}: 43--449 [1899]).

A remark explaining the relationship of verb transitivity to argument predication within a discussion of the verb classes became a feature of the Lutheran descriptive template. As missionaries refined their craft to suit Australian case systems, the description of ergativity under the heading “the verb” became a feature of a developing template of case description. Presenting the role of the ergative case in this section of the grammar was a strategy which Taplin also adopted, but only in his final (\citeyear{taplin_grammar_1878}) grammar of Ngarrindjeri (\sectref{sec:key:7.3.4}).

\subsection{Concluding remark}
\label{sec:key:6.2.2}
\largerpage
Schürmann’s presentation of case (see \sectref{sec:key:6.2.1.1}) is found to have followed Meyer (see \sectref{sec:key:6.1.2.1}), although Schürmann has previously been credited for having developed the innovative descriptive practices. Schürmann’s method of explaining ergative function under the heading “the verb” (\sectref{sec:key:6.2.1.6}) had not previously been employed by grammarians of the Adelaide School, and exerted an influence on later Lutheran PN description (\sectref{sec:key:8.5.2}) that was independent of the earlier Dresdners' materials.

Schürmann’s grammar of Barngarla is the shortest of the Dresdner grammars. Yet it gives a succinct and clear analysis of case (\sectref{sec:key:6.2.1.1}) and includes an insightful description of pronouns that refer to dyadic kin relations (\sectref{sec:key:6.2.1.4}). The clarity of the description might explain why Schürmann’s short work is referred to more frequently by \citet{ray_aboriginal_1925} and by \citet{elkin_nature_1937} than are other grammars of the Adelaide School.

\section{Middle-era understanding of bound pronouns}
\label{sec:key:6.3}\label{bkm:Ref73553541}
\largerpage

While the Dresdner missionaries may have been alerted to the possible existence of a class of pronouns that are “never used by themselves” \citep[18]{threlkeld_australian_1834}, they could not have found Threlkeld’s analysis of an atypical system (\sectref{sec:key:3.3.6.1}) overly helpful. Threlkeld’s terminology, “personal nominative” and “verbal nominative” fell into obscurity.

Bound pronouns were described as “affixes” in Kaurna (\citealt{TeichelmannSchürmann1840}: 23; \sectref{sec:key:5.5}), as “contractions” in Barngarla (\citealt[22]{schurmann_vocabulary_1844}; \sectref{sec:key:6.1.2.4}), as “inseparable” in Ramindjeri (\citealt[23]{meyer_vocabulary_1843}; \sectref{sec:key:6.1.2.4}), and as “euphonised” by Taplin (\citeyear[no pag.]{taplin_vocabulary_1867}; \sectref{sec:key:2.3.2}). Bound pronouns are not found in Diyari or in Arrernte, described by later Lutheran missionaries.\footnote{Arrernte has a specialised set of possessive bound pronouns that attach only to kin terms \citep[133--135]{wilkins_mparntwe_1989}. A similar set was recorded in Diyari by \citet{reuther_dieri_1894}, although \citet[56--57]{austin_grammar_2013} could not find the forms in late twentieth century speech.}  Meyer gave the most transparent description and Teichelmann and Schürmann the most opaque. The extent to which the forms were obscured in the early grammars was probably related to the nature of the system and its functional load in comparison with the free-form pronouns.

Despite these numerous, but sometimes opaque, descriptions of bound pronouns in the early grammars of PN languages, the recognition of these systems in middle-era overviews of Australian languages (\citealt{ray_aboriginal_1925}; \citealt{capell_structure_1937}; \citealt{elkin_nature_1937}) are confined to material gleaned from Mathews' grammars, and relate largely to the marking of possession in languages from the southeast of the country.

\citet{ray_aboriginal_1925}, who had \textit{not} encountered bound pronouns in his own study of Western Torres Strait (\citeyear{ray_study_1893}; \citeyear{ray_linguistics_1907}) or Paman languages (\citeyear{ray_linguistics_1907}; \sectref{sec:key:10.2}), refers to their function only in marking the possessed constituent of a possessive NP, which he described as confined to “[t]he Victorian languages, some of the eastern languages and a few others [which] denote possession by means of a suffix” \citep[5]{ray_aboriginal_1925}. His findings were informed by Mathews' grammars of Bunganditj (\citeyear{mathews_language_1903}), Thaguwurru (Daungwurrang; \citeyear{mathews_aboriginal_1902}), Thurrawal (Dharawal; \citeyear{mathews_gundungurra_1901}) and Tyatyalla (Djadjala; \citeyear{mathews_aboriginal_1902}). Similarly, \citegen[148--149]{elkin_nature_1937} awareness of bound pronouns is limited to bound pronouns in possessive case and is informed by Mathews' description of Djadjala (\citealt{mathews_aboriginal_1902}), spoken in Victoria.

\citet[188--189]{koch_r_2008} shows that in the languages from New South Wales and Victoria described by Mathews possession is marked not only by the possessive case suffix attached to the dependent possessor element of the possessed NP, as usually occurs in PN languages. In these languages the possessed constituent, which is the head of the NP, is also marked with an enclitic possessive pronoun. The English translational equivalent of a possessive phrase might be: ``man’s boomerang-his''. Mathews conveyed this structure in the standard fashion in his grammars, presenting paradigms of nouns inflected with a portmanteau morpheme marking possessive case and number and person (\figref{fig:key:114}). As \citet[194]{koch_r_2008} observes, Mathews was aware that these suffixes were pronominal.

\begin{figure}
\includegraphics[width=.8\textwidth]{figures/stockigt-img060.png}
\caption{Mathews' exemplification of Djadjala bound pronouns in possessive case \citeyearpar[78]{mathews_aboriginal_1902}}
\label{fig:key:114}
\end{figure}

Mathews' (\citeyear{mathews_aboriginal_1902}:78) statement, “Anything over which possession can be exercised is subject to inflection for person and number”, was quoted by \citet[58]{capell_structure_1937}, who also presented Mathews' Djadjala paradigm to show that “suffixed pronouns” marked possession in some languages. \citet[55]{capell_structure_1937} described how in the southeast “the genitive relationship is doubly indicated”.


\begin{figure}[b]
\includegraphics[width=.8\textwidth]{figures/stockigt-img061r.png}
\caption{Capell's description of bound pronouns in ergative and accusative cases based on Mathews' \citeyearpar{mathews_gundungurra_1901} grammar \citeyearpar[68]{capell_structure_1937}}
\label{fig:key:115}
\end{figure}

\largerpage
Unlike \citet{ray_aboriginal_1925} and \citet{elkin_nature_1937}, \citet[68--69]{capell_structure_1937} did describe bound pronouns in other cases. Capell had conducted field-work in “incorporating languages” – non-Pama-Nyungan prefixing languages – in which verbs mark agreement for the number and case of the predicated arguments. \citet[68]{capell_structure_1937} used Mathews' description of Dharawal (\citealt{mathews_gundungurra_1901}; \figref{fig:key:115}) to substantiate the hypothesis that some of the non-prefixing languages of New South Wales “show development in the same direction” as the “incorporating languages” of the Kimberley, but noted, “the system has never been investigated in full”.

That \citet{ray_aboriginal_1925} and \citet{elkin_nature_1937} did not retrieve the existence of bound pronouns in other cases as a feature of some PN languages is due to the opaque nature of their description in early grammars. Elkin refers to both \citegen{teichelmann_outlines_1840} grammar and to \citegen{schurmann_vocabulary_1844} work, in which bound pronouns were illustrated.

The oversight might be also due to the curious and unexplained absence in Elkin’s overview of reference to \citegen{meyer_vocabulary_1843} grammar, which gave such a succinct analysis of bound pronouns (\sectref{bkm:Ref73463170}). Elkin (\citeyear[133]{elkin_nature_1937}) did, however, refer to \citet{taplin_1879a_nodate}, which presented a paradigm of Ngarrindjeri bound pronouns (\figref{fig:key:7-134}), but without any exemplification. While \citet[2]{ray_aboriginal_1925} does list Meyer’s work, it is not clear that the “Narrinyeri” material to which he refers did not also come from Taplin’s grammars. Note that \citet{capell_structure_1937}, too, does not refer to Meyer.

\citet{blake_nineteenth-century_2016} describes how the early MS and published descriptions of Australian languages “lay in libraries for decades largely forgotten”. Many early grammars became collector’s items that were priced beyond the reach of scholars and sometimes libraries. That the print-run of some early grammatical material on Australian languages was so small, and the works were generally unavailable to later researchers, may also have hindered the dissemination of the valuable description of bound pronouns contained in \citet{meyer_vocabulary_1843}.

\section{Moorhouse’s grammar of Ngayawang (\citeyear{moorhouse_vocabulary_1846})}
\label{sec:key:6.4}\label{bkm:Ref514605398}\label{bkm:Ref73379277}\label{bkm:Ref456104612}\label{bkm:Ref456103534}

M. Moorhouse (1813--1876), a trained medical practitioner, arrived in South Australia in 1839 to take up a position as the first permanent Protector of Aborigines in the colony. His responsibility for the welfare of Aboriginal people extended to linguistic and ethnographic description, as well as to Christian ``civilisation''. In these capacities he had dealings with the Dresdner missionaries.

Moorhouse assumed appointment as protector amidst some controversy that previous protectors -- G. Stevenson (1837), W. Bromley (1837), and W. Wyatt (1837--1840) -- had not done enough to understand the local Aboriginal populations during their brief appointments. According to \citet[5]{foster_two_1990}, “[t]he pressure on Moorhouse, and the missionaries with whom he worked, to produce detailed descriptions of Aboriginal culture was intense.”

Moorhouse gave detailed ethnographic and linguistic descriptions in the official reports required of him as protector. His twenty-six-page report from 1841, which he co-authored with Teichelmann, was included in the catalogue of Sir G. Grey’s library (\citealt[5]{bleek_library_1858}; \figref{fig:key:116}). It is at least as informative as the ethnological publications made by the Dresdners (\citealt{teichelmann_aboriginals_1841}; \citealt{schurmann_letter_1844}; \citealt{meyer_manners_1846}). The section headed “language” (Moorhouse \& Teichelmann, in \citet{foster_two_1990}: 49--53 [1841]) contains a five-page description of Kaurna, which presents a case paradigm of pronouns (\figref{fig:5:87}), systematic exemplification of the formation of adjectives and inchoative verbs from nouns, a long listing of verbal ``modifications'', and a short listing of “postpositions” and ``postfixa''. The work makes use of some of the schemata developed by Teichelmann and Schürmann but is also descriptively innovative.


\begin{figure}
\includegraphics[width=\textwidth]{figures/stockigt-img062.png}
\caption{\citegen[5]{bleek_library_1858} summary of Moorhouse’s report (\citeyear{moorhouse1842})}
\label{fig:key:116}
\end{figure}

In his role as protector, Moorhouse travelled to the edges of the frontier around Adelaide reporting on the Aboriginal people living in different regions. One of the main purposes of his early journeys, on which he was sometimes accompanied by the Dresdner missionaries, was to establish which languages were spoken in what regions and how similar they were to one another.

In a letter written to Schürmann in 1841, Moorhouse described a journey upstream along the Murray, stating that he had a boy with him from Mt Barker “who sp[oke] the Adelaide and Pitta languages and could communicate with all the Natives we saw along the Murray, Rapid and Rufus rivers. In this respect the journey was an important one”. The term “Pitta” is a Kaurna word meaning “native goose” and referred to people from the Murray River (\citealt{teichelmann_outlines_1840}V: 40).

Teichelmann (diary 9/12/1840) described the purpose of a journey he made with Moorhouse in December 1840, via the Adelaide Hills to Lakes Alexandrina and Albert, as being “to research how far to the east our language was understood and spoken … [and] how far Brother Meyer’s sphere of activity could extend”. Describing the linguistic research he and Moorhouse carried out on the journey, Teichelmann stated that they “made as much progress in a few hours in this language as we had previously only been able to make in 6 months among our own natives.”

Moorhouse collated this linguistic material in a letter to the Private Secretary (\citeyear{moorhouse_letter_1840}), which included a comparative pronominal paradigm in four languages termed “Adelaide'', ``Encounter Bay'', “Pomunda” and “west of the lake” (\tabref{tab:key:117}) and a comparative list of numerals and the marking of number on nouns.\footnote{{On this trip Moorhouse attempted to make contact with the people from the Adelaide Hills, writing in December (\citeyear{moorhouse_letter_1840}), “We left Adelaide on the 4th and reached the Angas River in the evening. We expected to meet 70 Natives located on that river, but they had left a few days before we arrived. On the following day we rode over the two Special Surveys in that district, one taken by J. Morphett \& the other by G. H. Davenport Esquires, but could meet with no Natives at their usual place of encampment”. The Davenport survey was located in the upper catchment of the Angas River around Macclesfield. Thus, Moorhouse just missed out on recording the language spoken by the people living on the upper reaches of the Angas River before they disappeared. It is now unclear whether the territory of the Ngarrindjeri groups extended from the lakes to the top of the catchment. Virtually nothing is known of the ``Peramangk'', the language from the Adelaide Hills, and it remains unclassified \citep[170]{simpson_early_1996}}} The forms shown in this paradigm evince the linguistic diversity of the resource-rich region. Each set of pronouns differs from that given in his 1846 grammar of “Murray River” (ibid.: vi-v, 10--12; \figref{fig:key:119}).

Moorhouse’s investigations into the forms of pronouns in languages converging near the lakes at the Murray estuary (\citeyear{moorhouse_letter_1840}, \citeyear{moorhouse_vocabulary_1846}: vi [sent to Grey in 1845]) informed \citegen[365]{grey_languages_1845} precise location of the “five principal dialects” on his map (\figref{fig:map2}).

\begin{table}
\begin{tabular}{p{1cm}p{2cm}p{2cm}p{2cm}p{2cm}}
\lsptoprule
& “Adelaide” & “Encounter Bay” & “Pomunda” & “West of \newline the Lake” \\
\midrule
\textbf{1sg} & Ngaii & Ngapuna & Ngap & Ngapo\\
\textbf{1dl} & Ngadli & Ngal & Nganal, Ngel & Ngeli\\
\textbf{1pl} & Ngadlu & Ngunu & Nangan & Nangano\\
\textbf{2sg} & Ninna & Nginti & Ngint & Ngint\\
\textbf{2dl} & Niwa & Ngul & Ngul & Ngulo\\
\textbf{2pl} & Na & Nommu & Ngunu & Ngun\\
\textbf{3sg} & Pa - Padlo & Ka & Kiyika & Tikai\\
\textbf{3dl} & Purla & Kangge & Kukuka & Ukukuk\\
\textbf{3pl} & Parna & Kar & Kukuki & Kukuku\\
\lspbottomrule
\end{tabular}
\caption{Moorhouse’s comparative pronominal paradigm \citeyearpar{moorhouse_letter_1840}}
\label{tab:key:117}
\end{table}

Moorhouse’s “Annual Report of the Aborigines Department” in 1843 provided information that supplemented that given in the 1841 report, and was organised in a similar outline (\figref{fig:key:116}). Under the heading ``language'', Moorhouse (in \citet{foster_two_1990}: 60[1843]) wrote: “This branch has not been neglected in the past year, and a vocabulary of the Murray dialect, accompanied by some sketches of the grammar should now have been forwarded, had not those who are able to instruct in the language been absent in the bush”. The work to which Moorhouse refers presumably became part of his grammar and vocabulary of Ngayawang titled: \textit{A Vocabulary and Outline of the Grammatical Structure of the River Murray Language spoken by the natives of South Australia from Wellington on the Murray, as far as the Rufus} forwarded to Grey in 1845 (\citealt{moorhouse_vocabulary_1846}: v) and published in 1846.

Like each of the earlier published grammars of a PN language, except \citet{schurmann_letter_1844}, Moorhouse identified the variety he described in the title of the grammars in terms of the geographical region in which it was spoken: “The Murray River Language spoken by the natives of South Australia from Wellington on the Murray, as far as the Rufus”. The described area is curiously large, or long and narrow along the River, especially considering the linguistic diversity Moorhouse had described to the south (\citeyear{moorhouse_letter_1840}). Moorhouse (in \citealt{taplin_1879a_nodate}: 30) gave the name of the tribe as ``Meru'', `man'. The language was also referred to as “Pitta” in the 1840s. The earliest record of the term “Ngayawang” was given by E. J. \citet{Eyre1845}, who gave the name as “Aiawong” (ibid.: 396, 399) or “Moorunde” (ibid.: 396), ``Moorunde'' being a place name. \citet[396,399]{Eyre1845} described “Aiawong” as a chain of very similar dialects running along the same length of the Murray River that Moorhouse had identified and observed that the languages spoken just either side of the river were mutually unintelligible. Ewens (in \citealt{taplin_1879a_nodate}: 30) named the language ``Niawoo'', and described it as spoken by the ``Moorundee Tribe''.

\subsection{Moorhouse’s analysis of Ngayawang \citeyearpar{moorhouse_vocabulary_1846}}
\label{sec:key:6.4.1}\label{bkm:Ref340310942}\label{bkm:Ref466032933}

Moorhouse introduced his grammar of Ngayawang with an overview of Australian linguistic structures (\citeyear[v--vii]{moorhouse_vocabulary_1846}; \figref{fig:key:118}), which drew on the existing analyses of PN languages. It is one of three such descriptions -- the other two being \citet{Schürmann1846} and \citet{hale_languages_1846} -- published in 1846 providing empirical \textit{grammatical} evidence that Australian languages belonged to a single family.


\begin{figure}
\includegraphics[width=.5\textwidth]{figures/stockigt-img063.png}
\caption{Moorhouse’s typological summary of Australian languages (\citeyear[vi]{moorhouse_vocabulary_1846})}
\label{fig:key:118}
\end{figure}

Eight out of ten of the features Moorhouse listed as evidence that Australian languages “had their origin from one common source” (\citeyear{moorhouse_vocabulary_1846}: vi) related to grammar. Moorhouse observed that there were no articles, auxiliary verbs, relative pronouns, prepositions, gender, or distinct forms of the passive voice. Each of these categories receives no further mention in the description of Ngayawang, other than some explanation of “passive” constructions (\sectref{sec:key:7.1.1.1}). Regarding point nine, “gender”, Moorhouse noted that Threlkeld had described gender in third-person pronouns (\sectref{sec:key:3.3.2.1}), a feature that \citet[3]{schurmann_aboriginal_1846} had not described as gender. Few other early grammarians positioned their grammatical description as being of a language belonging to a larger family for which shared grammatical features had been identified.

Moorhouse also offered the “striking similarity in the pronouns” \citep[vi]{schurmann_aboriginal_1846} as evidence of shared heritage and provided a comparative paradigm in six languages, including ``Swan River'', informed by Grey (\citeyear{grey_vocabulary_1839}; \citeyear{grey_journals_1841}) and New South Wales, informed by Threlkeld (\citeyear{threlkeld_australian_1834}; \figref{fig:key:119}). While this approach and conclusion followed Grey (\citeyear{grey_journals_1841}; \figref{fig:key:120}), Moorhouse additionally observed the relative dissimilarity of third-person forms in comparison with other pronouns. This had not previously been noted.

On the grounds that first and second-person pronouns and the form of the dual suffix on nouns were the same in many languages, \citet[vii]{moorhouse_vocabulary_1846} made the fanciful claim that Aboriginal people had “separated in pairs, and these words, being in daily use were retained … as children were born … the terms for the third-person had to be invented."


\begin{figure}
\includegraphics[width=.8\textwidth]{figures/stockigt-img064.png}
\includegraphics[width=.8\textwidth]{figures/stockigt-img065.png}
\caption{Moorhouse’s comparative pronominal paradigm (\citeyear[vi--vii]{moorhouse_vocabulary_1846})}
\label{fig:key:119}
\end{figure}

\begin{figure}
\includegraphics[width=.8\textwidth]{figures/stockigt-img066.jpg}
\caption{Grey’s comparative pronominal paradigm (\citeyear{Grey1840})}
\label{fig:key:120}
\end{figure}

While emphasising that Australian languages belonged to one family on grammatical and phonological grounds, Moorhouse also stressed:

\begin{quote}
	The term “dialect” is scarcely applicable to the languages of New Holland. They differ in root more than English, French and German … yet there is evidence sufficient to satisfy any one they belong to one family. \citep[v--vi]{moorhouse_vocabulary_1846}
\end{quote}

It is important to observe that Moorhouse established the \textit{difference} in languages by referring to their lexicon but established their \textit{similarity} by examining their phonological and grammatical structure. \citet[293]{ridley_kamilaroi_1856} similarly pointed out that the relatedness of Australian languages was evident through their grammatical structure, in spite of their lexical diversity, writing: “Though not one \textit{word} in a thousand in Kamilaroi resembles that dialect [`Lake Macquarie language’, Awabakal] I already perceive important points of resemblance in \textit{grammar}”. These views are likely to have contributed to a prevailing assumption in the later part of the nineteenth century that Australian languages were of the same structure and that their diversity and mutual unintelligibility was more a matter of lexicon. That the grammatical homogeneity of Aboriginal languages was over-estimated resulted in later Lutheran missionary-grammarians borrowing from the works of the Adelaide School more than was warranted (\sectref{sec:key:8.7.4}).

Moorhouse’s grammar (\citeyear{moorhouse_vocabulary_1846}) is sparsely exemplified but contains more example clauses than Schürmann’s grammar of Barngarla \citeyearpar{schurmann_letter_1844}. At least one example clause is given for each of his “particles''. There are no clauses given in the section headed “verbs'', other than a single reflexive construction \citep[20]{moorhouse_vocabulary_1846}. The vocabulary gives close to 1000 entries. Most are given with a single definition and none with an illustrative clause.

The relative sparseness of Moorhouse’s work might explain its curious omission from F. Müller’s otherwise fairly comprehensive republication of available Australian grammatical material (\citeyear{muller_grundris_1882}). Like all other work of the Adelaide School, Moorhouse’s grammar had been catalogued by Bleek (\citeyear{bleek_library_1858}, Vol. II, Pt I: 15) and was accessible to Müller.

Much of Moorhouse’s description of Ngayawang \citeyearpar{moorhouse_vocabulary_1846} was reproduced in \citet{taplin_1879a_nodate}. The responses Taplin had received to the linguistic questions included in his 1875 circular from Police Trooper Ewens \citep[30]{taplin_1879a} at Blanche\-town were sparse. Taplin consequently supplemented information from that area using Moorhouse’s work, as well as his own knowledge (\citeyear[32]{taplin_1879a_nodate}). Taplin wrote: “no tribe in South Australia has died with such rapidity as this” \citep[31]{taplin_1879a_nodate}.

Moorhouse’s account of case (\sectref{sec:key:6.4.1.1}) and of ergativity (\sectref{sec:key:6.4.1.5}) employed some of the schemata and terminology innovated by the Dresdner missionaries. Following \citet{teichelmann_outlines_1840}, for example, Moorhouse also used the term “preventative or negative optative” mood to describe the apprehensional construction (\sectref{sec:key:5.6.1}). When illustrating instrumental case function, grammarians of the Adelaide School presented clauses with the English translation: “I shall build a house with stone'', where the term for “stone” or “brick” is marked in instrumental case function. Compare Example \ref{ex:5:15} on p. \pageref{ex:5:15}, given by Teichelmann \& Schürmann, with the clauses given by \citet[15]{meyer_vocabulary_1843} and Moorhouse (\citeyear[6]{moorhouse_vocabulary_1846}; \figref{fig:key:121}), each illustrating what Moorhouse termed the “instrumental particle'', \citet[15]{meyer_vocabulary_1843} “the expression of the preposition `with'\thinspace'', and \citet[24]{teichelmann_outlines_1840} “the ablative case''.

\begin{figure}
\includegraphics[width=.8\textwidth]{figures/stockigt-img067.png}
\includegraphics[width=.8\textwidth]{figures/stockigt-img068.png}
\caption{Similar clauses given in \citet[15]{meyer_vocabulary_1843} and in Taplin’s reproduction of Moorhouse (\citeyear{taplin_1879a_nodate}: 33 [\citeyear{moorhouse_vocabulary_1846}: 6])}
\label{fig:key:121}
\end{figure}

Such provision of a translational equivalent of a clause previously presented in a grammar of a different language occurs elsewhere in the corpus. W. Koch’s reproduced Teichelmann \& Schürmann’s Kaurna examples in Diyari (\sectref{bkm:Ref74048540}) and \citet{schwarz_koko_1900} and \citet{hey_elementary_1903} reproduced \citegen{roth_ethnological_1897} Pitta-Pitta examples, in Guugu-Yimidhirr and Nggerrikwidhi respectively (\sectref{bkm:Ref456087169}).

But beyond Moorhouse’s use of the Dresdners' schemata, terminology, and even example clauses, there is evidence that the structures Moorhouse described were imported into the description of Ngayawang from Teichelmann \& Schürmann’s description, on the assumption that the languages were structurally identical. Moorhouse’s description of case suggests that this description of Ngayawang was mapped onto the system previously employed by the Dresdners.

\subsubsection{Case systems}
\label{sec:key:6.4.1.1}\label{bkm:Ref333854849}

Under the heading “parts of speech”, \citet[2]{moorhouse_vocabulary_1846} abandoned the class “preposition” stating: “relation, expressed in English by prepositions, is expressed in this language by cases or particles, used as terminal affixes”. Moorhouse presented the heading “particle” straight after “substantive'', where he listed and described the function of cases common to PN languages that are not shown in the case paradigms, i.e., Teichelmann \& Schürmann’s “postfixa'', as well as the range of locational words, and Teichelmann \& Schürmann’s “postpositions''. The placement of this content directly after the discussion of case, rather than at the end of the grammar under the heading “postposition'', was informed by the awareness that some of the forms listed as “particles” were functionally equivalent to the suffixes that were included in the case paradigms. The same awareness was earlier reflected in the schema engaged by Threlkeld (\citeyear{threlkeld_australian_1834}) and Meyer (\citeyear{meyer_vocabulary_1843}; \sectref{sec:key:6.1.2.1}), and Schürmann (\citeyear{schurmann_letter_1844}; \sectref{sec:key:6.2.1.1}). Later Lutheran presentation of case (\sectref{sec:key:8.5.1}) followed the traditional presentation inspired by European case systems which was employed by Teichelmann \& Schürmann (\citeyear{teichelmann_outlines_1840}; \sectref{sec:key:5.3.1}).

\subsubsection{System of marking syntactic case}
\label{sec:key:6.4.1.2}\label{bkm:Ref340295413}

It is likely that Moorhouse’s description of case in Ngayawang was guided as much by what his knowledge of previously described South Australian languages led him to anticipate, as it was by information gathered from speakers of the language.

The missionaries had described Kaurna (\citealt{teichelmann_outlines_1840}) and Barngarla \citep{schurmann_letter_1844} as having identical systems of marking syntactic cases, showing an ergative system ({A/SO}) on all singular nouns, and undifferentiated marking ({ASO}) on non-singular nouns (\tabref{tab:key:122}). These Thura-Yura languages, Kaurna and Barngarla (\tabref{tab:chap1:map}), belong to the higher PN subgroup “Central” (\citealt{bowern_computational_2012}). The system Moorhouse described for the distantly related language Ngayawang, a Lower Murray language (\tabref{tab:chap1:map}) belonging to the higher level subgroup “South-eastern” (\citealt{bowern_computational_2012}), was identical.

On the basis of key morphological paradigms, Ngayawang is thought to be more closely related to Ngarrindjeri languages than to Kaurna \citep[170]{simpson_early_1996}. Phonologically, it shares phonotactic features with languages spoken up and down-stream along the river that are atypical of PN languages (see \citealt{horgen_languages_2004}: 62--74). The language also shares phonotactic features with Kaurna that are not shared with Ngarrindjeri, for example, the “tp” cluster, and pre-stopped nasals and laterals.

While it is possible that Ngayawang, spoken upstream on the Murray River from Ngarrindjeri, shared diffused grammatical features with Thura-Yura varieties spoken over the west of the Mt Lofty Ranges, the similarity in the presentations of syntactic case (\tabref{tab:key:122}) should be treated with circumspection, given what is known about Moorhouse’s method of data collection. Moorhouse’s record is, however, the most extensive source from which the language can be reclaimed.


\begin{table}
\begin{tabularx}{.8\textwidth}{QQ}
\lsptoprule
\multicolumn{2}{c}{Kaurna, Barngarla, Ngayawang} \\
\midrule
Ergative alignment \stockcaps{A/SO} & Undifferentiated \stockcaps{ASO}\\
\midrule
All singular nouns \& \newline all singular pronouns, & All non-singular nouns \& non-singular pronouns\\
\lspbottomrule
\end{tabularx}
\caption{The alignment of syntactic case marking on nominals in Kaurna, Barngarla and Ngayawang, as described in the early sources.}
\label{tab:key:122}
\end{table}

After giving case paradigms (\figref{fig:key:124}; \figref{fig:key:126}), \citet[6]{moorhouse_vocabulary_1846} mentioned that “[a]nother form of the plural is sometimes used, which might be termed the active nominative, as it appears to be used only as causative of action”. Given the substantiating clause (\figref{fig:key:123}), it is hard to understand why the form \textbf{merinnamara} was not entered as the “active nominative” in the plural paradigm. The precise structure of the form, based on \textit{mera} `man' and the ergative marker \textit{-ana}, is not clear. The way in which number and case marking may have interacted in Ngayawang has not been reclaimed \citep[94--96]{Horgan2004}. What is important for the purpose of this historiographical investigation is that Moorhouse seemingly overlooked the form initially, assuming that the grammatical structure of the language he described must necessarily have mirrored that described by \citet{teichelmann_outlines_1840} and \citet{schurmann_letter_1844}.


\begin{figure}
\includegraphics[width=.8\textwidth]{figures/stockigt-img069.png}
\caption{Moorhouse’s demonstration of the ergative marking on plural nouns \citeyearpar[6]{moorhouse_vocabulary_1846}}
\label{fig:key:123}
\end{figure}

Ngarrindjeri, which was preserved into the twentieth century at the Point McLeay mission, is the best documented variety of the Lower Murray Areal group. The complexity of the split in marking Ngarrindjeri syntactic case stands in contrast to Moorhouse’s record. If the same complexity of split existed in Ngayawang as was documented by Meyer in Ngarrindjeri, Moorhouse’s field method would have failed to elicit the data.

With a limited number of clauses, and given that there is evidence that Moorhouse’s data was collected via translation of specific words and phrases from what was probably a simplified form of Kaurna used in European interface, the authenticity of Moorhouse’s grammar as a representation of Ngayawang structure is questionable. His description of the alignment of the syntactic cases suggests that he did not only borrow aspects of the Dresdners' descriptive template, but that morpho-syntactic structures were also transferred across language boundaries.

The assumption at the time that Aboriginal languages were of similar grammatical structure and that linguistic divergence related more to lexicon led Meyer to import structures described in Kaurna (\citealt{teichelmann_outlines_1840}) into his description of processes of clause subordination in Ramindjeri (\sectref{sec:key:6.1.2.9}). The belief that Nyayawang is said to belong within an island of languages extending further upstream along the Murray River which do not exhibit pronominal enclitics \citep[340]{dixon_australian_2002} when neighbouring languages Kaurna and Ngarrindjeri have bound pronominal systems should consequently be treated with a degree of caution. Given what is known about Moorhouse’s method of fieldwork and the scantiness of the grammar, it is possible that the absence of a bound pronominal system in Moorhouse’s grammar is a descriptive oversight. Teichelmann \& Schürmann’s inexplicit description of bound pronouns in Kaurna (\sectref{sec:key:5.5}) is unlikely to have alerted Moorhouse to the possibility of their existence in Ngayawang.

\subsubsection{Case paradigms}
\label{sec:key:6.4.1.3}

Despite presenting a language in which all nominals were ergatively aligned or were undifferentiated -- i.e., where there was no overt accusative marking -- Moorhouse did not opt to re-employ the realignment of pronominal case forms that he had engaged with Teichelmann when presenting Kaurna in 1841 (in \citealt{foster_two_1990}: 49; \figref{fig:5:87}).

 Unlike \citet{meyer_vocabulary_1843} and \citet{schurmann_letter_1844}, Moorhouse presented paradigms declining nouns for case (1846: 3--5), although with some reservation that the forms did not “strictly form declensions” (ibid.: 5; \sectref{sec:key:2.4.1.1}). Ngayawang case inflection was seen, after deliberation, to constitute declension by virtue of phonological alteration at the stem and inflection boundary.

Like other grammarians of the Adelaide School, and indeed of the larger Lutheran school of South Australian description, Moorhouse’s case paradigms are conservative (\figref{fig:key:124}; \figref{fig:key:126}) in comparison with those previously presented of languages spoken in New South Wales (\citealt{threlkeld_australian_1834}; \citealt{gunther_native_1838}; \citeyear{gunther_lecture_1840}). The paradigms do, however, include two functionally and formally distinct “ablative” cases. The first “ablative” case inflected with \textit{-anno} and translated “at, remaining with X” is the locative \citep[95]{horgen_languages_2004} and possibly also marked comitative function. Although the second “ablative” case is in fact unnamed, it is likely that Moorhouse had another “ablative” form in mind, since the form is translated with the prototypical ablative function, “from, away from X”. The nominal inflections included in Moorhouse’s paradigms are shown in \tabref{tab:key:125}.


\begin{figure}[t]
\includegraphics[width=.8\textwidth]{figures/stockigt-img070.png}
\includegraphics[width=.8\textwidth]{figures/stockigt-img071.png}
\caption{Moorhouse’s case paradigm on \textit{meru} `man' \citeyearpar[3--4]{moorhouse_vocabulary_1846}}
\label{fig:key:124}
\end{figure}

\begin{figure}[t]
\includegraphics[width=.8\textwidth]{figures/stockigt-img072.png}
\caption{Moorhouse’s case paradigm on \textit{nguilpo} `child' \citeyearpar[4]{moorhouse_vocabulary_1846}}
\label{fig:key:126}
\end{figure}



\begin{table}
\fittable{
\begin{tabular}{llll}
\lsptoprule
Form & Translation & Case label & \citealt{Horgen2004}: 95\\
\midrule
{}- ø & `(a/the) X' & Nominative & Absolutive\\
{}-anna & `a X, the agent' & Active nominative & Ergative\\
{}-ning, -ngo, -ung & `of, belonging to X' & Genitive & Genitive \\
{}-anno & `to, locally X' & Dative & Locative\\
{}-allarnu* & `to, giving to X' & Dative (unnamed) & Not reclaimed\\
{}- ø & `(the) X' & Accusative & Absolutive\\
{}-anno & `at, remaining with X' & Ablative & Locative\\
{}-nnainmudl & `from, away from X' & Ablative (unnamed) & Ablative\\
\lspbottomrule
\end{tabular}
}
\legendbox{$\ast$ Only in singular paradigm of “child” (see below)}
\caption{Case forms and functions given by Moorhouse on nouns}
\label{tab:key:125}
\end{table}

Moorhouse presented two paradigms of case on nouns, first for \textit{meru} `man' and then for \textit{nguilpo} `child', explaining, “it will be seen that the modes of inflection differ; so much so, as to lead one to suspect the existence of several declensions. I have not yet been able to discover how or in what class of words these variations occur” (\citeyear{moorhouse_vocabulary_1846}: 5). That Moorhouse was unable to describe the type of nouns incurring particular inflections, or to describe morphophonemic processes, as Threlkeld (\citeyear{threlkeld_australian_1834}; \figref{fig:3:44}) and Günther (\citeyear{gunther_native_1838}; \citeyear{gunther_lecture_1840}; \sectref{sec:key:4.4.2}) had done, is not surprising given his comparatively short engagement with the language.


The singular paradigm for \textit{nguilpo} `child' gives two functionally and formally distinct “dative” cases. All other paradigms show a single “dative” case. The form in the singular paradigm that is not included in the non-singular paradigms terminates in \textit{-allarnu}, and is translated `to, giving to X'. Again, although the case is unnamed, it is assumed here to be a “dative 2” because of its translation as the iconic dative. It is possible that the suffix \textit{-allarnu}, although unreclaimed, is marking the accusative case. It is shown in the paradigm attached to the indirect object of the verb ``to give'', which would immediately have attracted the label ``dative'', but which is often marked by the accusative suffix in PN languages (\citealt{blake_case_1977}: 35--36; \citealt{schebeck_adnjamathanha_1973}: 2; \citealt{wilkins_mparntwe_1989}: 169; \citealt{henderson_topics_2013}: 294; \citealt{hercus_grammar_1999}: 75).

\subsubsection{Moorhouse in Taplin \citeyearpar{taplin_1879a_nodate}}
\label{sec:key:6.4.1.4}

In reproducing Moorhouse’s work, \citet[31]{taplin_1879a_nodate} renamed the cases and adjusted the paradigms, applying another variation on the numerous conventions he experimented with when representing Ngarrindjeri case (\sectref{sec:key:7.3.4}). Moorhouse’s “active nominative” case is now termed “causative” in the singular paradigm (\stockcaps{A/S/O}). Taplin deleted Moorhouse’s comment “the agent” next to the ergative form and inserted instead “by X” (\figref{fig:key:127}).

Taplin’s representation of some other case forms given by Moorhouse is peculiar. He introduced his own invented term “exative” to name the spatial function of the ablative case carried by dual and plural pronouns ending in –\textit{mudl} and translated as it `from x'. But singular pronouns marked with the same form and given the same translation were termed “ablative''. The forms termed “ablative” in the dual and the plural are marked by \textit{-anno.} This form was left out of the singular paradigm, presumably because it was the same as the “causative''.

\begin{figure}[t]
\includegraphics[width=.8\textwidth]{figures/stockigt-img073.png}
\caption{Taplin’s rearrangement of Moorhouse’s Ngayawang case paradigm (\citeyear{taplin_1879a_nodate}: 31 [1846])}
\label{fig:key:127}
\end{figure}

\subsubsection{Description of ergativity}
\label{sec:key:6.4.1.5}

Moorhouse termed nouns in ergative case “active nominative” and placed them in second paradigmatic position (\figref{fig:key:124}; \figref{fig:key:126}). In doing so, he followed Threlkeld’s (\tabref{tab:3:42}) and Schürmann’s representation of pronouns (\figref{fig:key:108}). This practice differs from Teichelmann and Schürmann’s and Meyer’s placement of ergative case in last position, which was to become influential on later Lutheran PN description. \citet[3]{moorhouse_vocabulary_1846} provided additional clarity within the tables of nominal declension by translating ergative forms as `X did, does or will act upon a subject'.

Moorhouse placed ergative pronouns in different paradigmatic position from ergative nouns. Ergative pronouns were placed last in the paradigm and were called both “active nominative” and “ablative” (\figref{fig:key:128}). The use of both “active nominative” and “ablative” to name the ergative pronouns was influenced by \citet{teichelmann_outlines_1840}. Next to the “active nominative/ablative” form, Moorhouse placed the note “the agent or by me”. While previous grammarians had used the term “agent” to describe ergative case forms, Moorhouse also used the term “patient” (\citeyear{moorhouse_vocabulary_1846}: 4) to describe accusative forms, a usage that is unique in the corpus.

\begin{figure}[t]
\includegraphics[width=.8\textwidth]{figures/stockigt-img074.png}
\caption{Moorhouse’s 1sg pronominal case paradigm \citeyearpar[10]{taplin_1879a_nodate}}
\label{fig:key:128}
\end{figure}

Moorhouse adopted Schürmann’s presentation of ergative function through exemplification of the case frame of arguments predicated by transitive and intransitive verbs (\figref{fig:key:112}; \figref{fig:key:113}). He used the neuter verb \textit{terri-} `to stand' followed by the active verb \textit{parldke-} `to strike' \citeyearpar[19--23]{moorhouse_vocabulary_1846} and clearly described the arguments predicated by each:

\begin{quote}
1\textsuperscript{st} – Neuter or intransitive, or those which describe the state or condition of a subject; or an action which has no effect upon an object …

2\textsuperscript{nd} – Active or transitive, or those which describe an act which passes from an agent to some external object \citep[20]{moorhouse_vocabulary_1846}
\end{quote}

\subsection{Concluding remarks}
\label{sec:key:6.4.2}

Moorhouse’s understanding of PN structure was developed in conjunction with the Dresdner missionaries. Teichelmann’s description of Kaurna (\citeyear{teichelmann_outlines_1840}), which appeared in a report co-authored with Moorhouse, as well as the comparative pronominal paradigms of Lower Murray languages (\citeyear{teichelmann_outlines_1840}) that Moorhouse collected on a tour of the area with Teichelmann, probably primed Moorhouse to prepare his own grammar. His grammar of Ngayawang (\citeyear{moorhouse_vocabulary_1846}) is unlikely to have been produced if not for the practice of grammatical description instigated by the Dresdners. In writing the last of the flurry of grammatical work emanating from the appointment of the Dresdner missionaries to South Australia, Moorhouse utilised a mixture of descriptive techniques developed by Dresdner missionary-grammarians as well as some of his own.

While Moorhouse’s grammar of Ngayawang is well informed about Australian grammatical typology, his lack of long-term engagement with speakers of the language and his mapping of the structure onto a template supplied by the Dresdeners produced a grammar that failed to provide a nuanced account of the structure of the language.

\section{Conclusion: The Adelaide School}
\label{sec:key:6.5}

The Adelaide School missionary-grammarians  honed their descriptive skills in response to their increased understanding of PN structures. As \citet[410]{simpson_notes_1992} has observed, their “grammars and vocabularies … show[] how … [the grammarians] learned from one another”. That said, the descriptive frameworks employed by the Adelaide School grammarians are far from homogeneous. Such proclivity for descriptive independence is evident in other early grammars. Watson’s lost Wiradjuri grammar differed from Günther’s manuscripts (1838--1840), for instance. A notable exception to this tendency is, however, found in the Lutherans' descriptions of Diyari (\chapref{chap:key:8}).

Various aspects of the analyses innovated by grammarians of the Adelaide School influenced different groups of later grammarians. Teichelmann \& Schürmann’s case paradigms came to be particularly influential on grammars of Diyari (\chapref{chap:key:8}). Schürmann’s neat exposition of the different marking of ergative and nominative pronouns with transitive and intransitive verbs (\sectref{bkm:Ref336283270}) followed by Moorhouse (\sectref{sec:key:6.4.1.5}) was employed by \citet{taplin_grammar_1878}, but only in his last grammar of Ngarrindjeri (\sectref{bkm:Ref514607800}) and in grammars of Diyari (\sectref{bkm:Ref456873578}), and Meyer’s conception of the ergative case was to have repercussions for Taplin’s later presentation of case. Meyer employed a framework that is \textit{less} similar to \citegen{teichelmann_outlines_1840} than are the later Lutheran grammarians' descriptions of Diyari (\chapref{chap:key:8}).

Both \citegen{meyer_vocabulary_1843} and \citegen{schurmann_letter_1844} diminished reliance on the traditional paradigm to convey the function of case forms, in comparison with \citet{teichelmann_outlines_1840}, shows an increasing awareness within the Adelaide School that the complexity and size of PN case systems could not adequately be conveyed through simply assigning traditional case labels to case forms. This realisation was not carried forward into later Lutheran descriptions of South Australian languages.

After the closure of the Dresdner missions, the retiring missionaries were quickly engaged within the growing German Lutheran communities in Australia, for which there was a shortage of pastors. Despite the following decades of missionary inactivity in South Australia, the Dresdner missionary-grammarians supported the establishment of the later inland South Australian Lutheran missions (\sectref{bkm:Ref456104995}). This continuity provided an environment in which the earliest morpho-syntactic analyses of South Australian PN languages were preserved and passed down to later generations of Lutheran missionaries trained at the Hermannsburg Mission Society and the Neuendettelsau Mission Society.

