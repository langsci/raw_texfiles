\title{Advances in formal Slavic linguistics 2018}
% \subtitle{Add subtitle here if it exists}
\BackBody{\textit{Advances in formal Slavic linguistics 2018} offers a selection of articles that were prepared on the basis of talks presented at the conference Formal Description of Slavic Languages (FDSL 13) or at the parallel Workshop on the Semantics of Noun Phrases, which were held on December 5–7, 2018, at the University of Göttingen. The volume covers a wide array of topics, such as situation relativization with adverbial clauses (causation, concession, counterfactuality, condition, and purpose), clause-embedding by means of a correlate, agreeing vs. transitive ‘need’ constructions, clitic doubling, affixation and aspect, evidentiality and mirativity, pragmatics coming with the particle li, uniqueness, definiteness, maximal interpretation (exhaustivity), kinds and subkinds, bare nominals, multiple determination, quantification, demonstratives, possessives, complex measure nouns, and the NP/DP parameter. The set of object languages comprises Russian, Czech, Polish, Bulgarian, Macedonian, Serbo-Croatian, and Torlak Serbian. The numerous topics addressed demonstrate the importance of Slavic linguistics. The original analyses prove that substantial progress has been made in major fields of research.}
\author{Andreas Blümel  and Jovana Gajić  and Ljudmila Geist  and Uwe Junghanns  and Hagen Pitsch}
\renewcommand{\lsSeries}{osl}% use series acronym in lower case
\renewcommand{\lsSeriesNumber}{4}
\renewcommand{\lsISBNdigital}{978-3-96110-322-5}
\renewcommand{\lsISBNhardcover}{978-3-98554-018-1}
\BookDOI{10.5281/zenodo.5155544}
\typesetter{Radek Šimík}
\proofreader{Andreas Blümel, Jovana Gajić, Ljudmila Geist, Nicole Hockmann, Uwe Junghanns, Hagen Pitsch, Freya Schumann}
\renewcommand{\lsID}{280}
% \lsCoverTitleSizes{51.5pt}{17pt}% Font setting for the title page
