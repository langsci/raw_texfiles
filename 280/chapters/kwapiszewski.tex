\documentclass[output=paper]{langscibook}
\ChapterDOI{10.5281/zenodo.5483108}

\author{Arkadiusz Kwapiszewski\orcid{0000-0002-2957-0862}\affiliation{University of Oxford} and Kim Fuellenbach\orcid{0000-0003-4836-1617}\affiliation{University of Oxford}}
\title[Reference to kinds and subkinds in Polish]
      {Reference to kinds and subkinds in Polish}

\abstract{This paper investigates the syntax and semantics of direct kind reference in Polish. Taking \citet{Borik.Espinal2012,Borik.Espinal2015} as our point of departure, we argue that kind-referring nominals in Polish have the same properties as their counterparts in English, Spanish, and Russian. Specifically, they are definite and numberless. Even though Polish does not realize definiteness overtly, we present evidence from pronominal co-reference and object topicalization to show that Polish kind nominals are definite. We then point to a previously unaddressed contradiction regarding modified kinds. \citeauthor{Borik.Espinal2012}'s assumption that bare nouns denote singleton sets of kinds is incompatible with the intersective approach to kind modification \citep{McNally.Boleda2004,Wagiel2014}. To circumvent this issue, we introduce a subkind operator \cnst{sk} into the semantics, linking it to the projection of a subkind phrase in the syntax. This allows us to account for some novel data involving kind modifiers (e.g. \textit{Bengal}) and kind classifiers (e.g. \textit{kind of}). Tentatively, we suggest that the subkind head is a type of a more general classifier head \citep{Borer2005, Picallo2006, Kratzer2007}.

\keywords{genericity, kind reference, kind modification, subkinds, nominals, number, definiteness, Polish}
}


\begin{document}
\SetupAffiliations{mark style=none}
\maketitle

% Section 1 - Introduction

\section{Introduction}

Ever since \citeposst{Carlson1977} seminal dissertation, semantic ontology has been assumed to contain at least two sorts of individuals: objects (spatiotemporal instantiations of individuals) and kinds (abstract types of individuals). Unsurprisingly, we call kind-referring a nominal which refers to a kind-level individual (see \citealt{Krifka1995}). A typical example is the English definite \textit{the dodo} in \REF{ex:dodo}. Since the property \textit{be extinct} cannot be predicated of concrete individuals, the subject DP must refer to the kind `dodo' directly.

\ea The dodo is extinct. \label{ex:dodo}
\z

\noindent
Though most studies of kind reference focus on English, some researchers have investigated this phenomenon from a cross-linguistic perspective, seeking to establish generalizations about the structure of kind-referring nominals across languages (see especially \citealt{Chierchia1998} and \citealt{Dayal2004}). More recently, \citeauthor{Borik.Espinal2012} have developed a syntactic and semantic account of kind reference which falls squarely within this tradition. In a series of papers, \citet{Borik.Espinal2012, Borik.Espinal2015, Borik.Espinal2016, Borik.Espinal2018} draw on evidence from English, \ili{Russian}, and \ili{Spanish} to argue that kind-referring DPs are definite and numberless (i.e. lacking the projection of number).\largerpage
%\footnote{All abbreviations used in this paper are listed at the end.}

In the first half of this paper, we investigate whether \citeauthor{Borik.Espinal2012}'s hypothesis holds for \ili{Polish}. We hypothesize that kind nominals in \ili{Polish} have the same structure as their counterparts in \ili{Romance} and \ili{Germanic} languages, which means that they are both definite and numberless. \sectref{sec:2-nominals-def} discusses the role of definiteness in deriving reference to kinds. Unlike English and \ili{Spanish}, \ili{Polish} does not realize definiteness overtly, which makes it difficult to diagnose the presence of definiteness in kind-referring DPs. Taking on this challenge, we present new evidence from object topicalization which supports the hypothesis that \ili{Polish} kind nominals are definite.

\sectref{sec:3-nominals_numberless} addresses the role of number in licensing kind, subkind, and object readings. The presence of number is shown to block direct reference to kinds, admitting only reference to subkinds or objects instead. From this, we conclude that \ili{Polish} kind-referring DPs are numberless, thus extending the empirical coverage of \citeauthor{Borik.Espinal2012}'s theory to a new language.

The second half of the paper turns to the derivation of modified kinds (e.g. \textit{the Bengal tiger}). We start \sectref{sec:4-subkinds}  by pointing out a contradiction between \citeauthor{Borik.Espinal2012}'s theory of \textsc{definite numberless kinds} and the intersective approach to \textsc{kind modification} advocated by \citet{McNally.Boleda2004}, \citet{Wagiel2014}, and \citet{Borik.Espinal2015}. While \citeauthor{Borik.Espinal2012} presuppose that NP denotations are atomic (i.e.\ \sib{tiger} is a singleton set of kinds), \citet{McNally.Boleda2004} assume taxonomic NP denotations (i.e. \sib{tiger} includes the kind `tiger' and all of its subkinds). We suggest a way of integrating the two approaches by introducing a subkind operator \cnst{sk} into the semantics and linking it to the projection of a \textsc{subkind phrase} in the syntax. This allows us to maintain that NPs have atomic rather than taxonomic denotations, while still deriving the correct interpretations for modified kinds.

Finally, \sectref{sec:5-conclusions} summarizes our main findings concerning reference to kinds and subkinds in \ili{Polish}, and make explicit the denotations and structures for the proposed operators and DP projections.

% Section 2

\section{Reference to kinds is definite}\label{sec:2-nominals-def}

The goal of this section is to lay out our assumptions about the relation between \textsc{definiteness} and the availability of direct reference to kinds. To begin with, \sectref{sec:sem_def} provides a brief overview of the syntax and semantics of kind-referring DPs in \ili{Romance} and \ili{Germanic} languages, which have an overt definite article in their inventory of functional morphemes. It will be suggested that definiteness, understood as the uniqueness-presupposing $\iota$ operator in the sense of \citet{Partee1987}, is a necessary component of kind reference in those languages.

In \sectref{sec:def_kinds_pol}, we extend the analysis to \ili{Polish}, a language without a morphological exponent of definiteness. After discussing our theoretical assumptions concerning the syntax-semantics interface, particularly our rejection of semantic type-shifting and the universal character of the DP $\leftrightarrow$ individual mapping, we present new evidence from object topicalization to show that \ili{Polish} kind-referring DPs are definite.

\subsection{The semantics of definiteness}
\label{sec:sem_def}

Let us start with a few examples of kind-referring DPs taken from English \REF{ex:kind-ref-en}, \ili{German} \REF{ex:kind-ref-ger}, \ili{Spanish} \REF{ex:kind-ref-spa}, and \ili{French} \REF{ex:kind-ref-fr}. The first thing we observe is that a morphologically singular count noun requires the definite article to achieve kind reference. The variants without the article are all ungrammatical.\footnote{Note that this generalization does not extend to mass kinds. Kind-referring DPs derived from mass nouns exhibit mixed behaviour with respect to the obligatoriness of the definite article: they require the definite article in \ili{French}, reject it in English, and take one optionally in \ili{German}. We do not discuss mass kinds in this paper, leaving them for future research.}

\ea \label{ex:kind-ref-crossling}
\ea *(The) dodo is extinct.\label{ex:kind-ref-en} \hfill (English)
\ex \gll \minsp{*(} Der) Dodo ist ausgestorben.\label{ex:kind-ref-ger}\\
{} the dodo is out.died \\ \hfill (\ili{German})
\ex  \gll \minsp{*(} El) dodo está extinto.\label{ex:kind-ref-spa}\\
       {} the dodo is extinct\\ \hfill (\ili{Spanish})
\ex \gll \minsp{*(} Le) dodo est éteint.\label{ex:kind-ref-fr}\\
       {} the dodo is extinct\\ \hfill (\ili{French})
%        \glt `The dodo has gone extinct.'
\z\z

\noindent
This leads us to ask about the function of the definite article in \REF{ex:kind-ref-crossling}. According to \citet{Krifka1995}, the presence of the article is necessary for syntactic well-formedness, but it has no effect on the semantic computation \REF{ex:deriv_krifka}. In his view, bare count NPs refer to kinds directly, whereas the article is merely ``ornamental'', inserted to satisfy structural constraints that are orthogonal to the semantics. This entails that the definite article is two-way ambiguous, denoting the identity function on the kind reading and the $\iota$ operator on the object reading.

\ea \label{ex:deriv_krifka}
\ea \sib{dodo} $=$ \textsc{dodo}
\ex \sib{the} $=$ $\lambda x.x$
\ex \sib{the dodo} $=$ \textsc{dodo}
\z \z

\noindent
\citet{Dayal2004} takes a different approach, arguing that the denotation of the definite article is constant across kind-referring and object-referring contexts. Specifically, the definite article always translates as the $\iota$ operator, which maps a predicate \textit{P} onto the unique element satisfying that predicate (see \citealt{Partee1987}). Furthermore, \citet{Dayal2004} assumes that NP denotations are ambiguous between properties of kinds and properties of objects. In \REF{ex:deriv_dayal_1}, the type variable \textit{t} ranges over the values \textit{k} (for `kind') and \textit{o} (for `object'), depending on the context of its occurrence. Reference to kinds emerges when the NP is contextually ``calibrated'' to denote a property of kinds, with the kind `dodo' selected by the uniqueness-presupposing $\iota$ operator, as illustrated in \REF{ex:deriv_dayal_3} below.

\ea \label{ex:deriv_dayal}
\ea \sib{dodo} $=$ $\lambda x^t.\textsc{dodo}(x^t)$ \label{ex:deriv_dayal_1}
\ex \sib{the} $=$ $\lambda P.\iota x[P(x)]$
\ex \sib{the dodo} $=$ $\iota x^k[\textsc{dodo}(x^k)]$ \label{ex:deriv_dayal_3}
\z \z


%\ea \label{ex:deriv_dayal}
%\ea \sib{dodo} $=$ $\lambda x$\textsuperscript{t}.$\textsc{dodo}(x$\textsuperscript{t}) \label{ex:deriv_dayal_1}
%\ex \sib{the} $=$ $\lambda P.\iota x[P(x)]$
%\ex \sib{the dodo} $=$ $\iota x$\textsuperscript{k}$[\textsc{dodo}(x$\textsuperscript{k})] \label{ex:deriv_dayal_3}
%\z \z

\noindent
To recapitulate, \citet{Dayal2004} dispenses with \citeposst{Krifka1995} assumption that the definite article is ambiguous, but admits a two-way ambiguity between object- and kind-level denotations for bare NPs.

In many respects, the proposal of \citet{Borik.Espinal2012,Borik.Espinal2015} can be seen as another step towards ambiguity reduction in the semantics, and a closer correspondence between syntactic structure and semantic interpretation. For \citeauthor{Borik.Espinal2012}, just like for \citet{Dayal2004}, the definite article in \ili{Romance} and \ili{Germanic} kinds translates as the $\iota$ operator. Their main innovation is the hypothesis that bare NPs unambiguously denote properties of kinds, while object denotations are derived via the Carlsonian realization relation \cnst{r} in the presence of number (see \citealt{Carlson1977}). We defer the discussion of the relation between number and kind reference until \sectref{sec:3-nominals_numberless}. For now, the important point is that the only difference between \citeauthor{Borik.Espinal2012}'s and \citeposst{Dayal2004} approach to the derivation of definite kinds concerns the representation of bare NPs: while \citet{Dayal2004} assumes that they are ambiguous \REF{ex:deriv_dayal_1}, \citeauthor{Borik.Espinal2012} postulate that they are properties of kinds \REF{ex:deriv_borik_1}.

\ea \label{ex:deriv_borik}
\ea \sib{dodo} $=$ $\lambda x^k.\textsc{dodo}(x^k)$ \label{ex:deriv_borik_1}
\ex \sib{the} $=$ $\lambda P.\iota x[P(x)]$
\ex \sib{the dodo} $=$ $\iota x^k[\textsc{dodo}(x^k)]$ \label{ex:deriv_borik_3}
\z \z

\noindent
Given the crucial role played by definiteness in converting properties of kinds to kind individuals in \REF{ex:deriv_dayal} and \REF{ex:deriv_borik}, \ili{Slavic} languages constitute an important litmus test for the theories of kind reference outlined above. Since \ili{Polish} lacks a determiner system, the presence of the definite feature carried on the syntactic D head does not have an observable morphological exponent. And yet, the existence of the DP projection in \ili{Slavic} has been defended by \citet{Pereltsvaig2007} for \ili{Russian} and by \citet{Willim2000}, \citet{Migdalski2001} and \citet{Rutkowski2007} for \ili{Polish}, based on evidence from demonstrative pronouns and prenominal possessives, among others. In the next section, we build on the results of this work to argue that \ili{Polish} kind-referring nominals are definite DPs.\footnote{\label{ftn:dp-hypothesis}We acknowledge that there is a more nuanced, ongoing debate about the status of the DP in \ili{Slavic} languages. There are some arguments against a DP and in favor of an NP-analysis. Most prominently, \citet{Boskovic2005} and \citet{Boskovic2007} focus on the mutual exclusivity of adjectival left-branch extraction and the presence of a DP. In a similar vein, \citet{Ceglowski2017} builds on various types of left-branch extractions and provides experimental data in support of this hypothesis. This said, we think that the empirical and theoretical arguments in favor of the DP hypothesis outweigh the arguments against it.}


\subsection{Definite kinds in Polish} \label{sec:def_kinds_pol}

We have considered English, \ili{German}, \ili{Spanish}, and \ili{French} DPs, all of which require the presence of a definite determiner in kind-referring contexts. In this section, we turn to parallel examples in \ili{Polish}, building on the discussion of \ili{Russian} in \citet{Borik.Espinal2012, Borik.Espinal2016}. By arguing for covert definiteness in \ili{Polish} kind-referring DPs, we extend the empirical coverage of \citeposst{Dayal2004} and \citeauthor{Borik.Espinal2012}'s theories to another language. We also discuss new evidence from object topicalization, which strengthens the case for definiteness in \ili{Polish} kind-referring DPs.

We begin this section with a simple but important argument in support of the DP status of kind-referring nominals. As discussed in \sectref{sec:sem_def}, the existence of the DP projection in \ili{Slavic} is relatively well-established (see \citealt{Willim2000} for \ili{Polish} and \citealt{Pereltsvaig2007} for \ili{Russian}, but see also footnote \ref{ftn:dp-hypothesis} for an important qualification). When present, the \textsc{determiner} projection is responsible for the computation of reference, with the result that DP $\rightarrow$ \textsc{individual} in the semantics.

\begin{sloppypar}
Here, we follow \citet{Borer2005} in adopting an even stronger assumption. Namely, we assume that the D head is the only source of referentiality, and that predicative NPs (type $\stb{e,t}$) cannot be type-shifted to individuals (type \textit{e}) in the semantics. This amounts to an isomorphic mapping between syntax and semantics, which we can represent schematically as DP $\leftrightarrow$ individual. From this perspective, any nominal which introduces a referent into the discourse should bear the syntactic hallmarks and distribution of a DP.\end{sloppypar}
% % % \largerpage %to have the whole (7a) on page vi

With this in mind, consider the two-sentence discourse in \REF{ex:whale_ref}. On its most salient reading, the kind-referring subject \textit{wieloryb} `the whale' is co-referential with the pronoun \textit{niego}. Since \textit{wieloryb} licenses pronominal reference, it is, by hypothesis, a DP. Crucially, not all bare nouns in \ili{Polish} are referential. Witness the inability of the bare plural \textit{książki} `books' to co-refer with the pronoun \textit{je} in \REF{ex:bookshelf_1}. This is due to the PP \textit{na książki} being part of a kind compound, with the modified NP corresponding to the English nominal compound \textit{bookshelf}. Given that the inclusion of the demonstrative determiner in \REF{ex:bookshelf_2} renders the DP obligatorily referential, we find further support for the DP $\leftrightarrow$ \textsc{referentiality} connection.\footnote{From here, if not indicated otherwise, all examples are from \ili{Polish}.}\largerpage

\ea \gll
Wieloryb$_i$ jest na skraju wymarcia. Mimo to w niektórych krajach ciągle się na niego$_i$ poluje.\\
whale.\textsc{nom.m} is on verge extinction.\textsc{gen} despite this in some countries still \textsc{refl} for him hunt\\
\glt `[The whale]$_i$ is on the verge of extinction. Despite this, people still hunt it$_i$ in some countries.' \label{ex:whale_ref}
\ex \ea[\#] {\gll
 Robert zbudował półkę na książki$_j$. Kupił je$_j$ wczoraj w księgarni.\\
 Robert.\textsc{nom} built.\textsc{pfv} shelf.\textsc{acc} for books.\textsc{acc.f} bought.\textsc{pfv} them.\textsc{f} yesterday in bookshop.\textsc{loc}\\
\glt Intended: `Robert built a [book]$_j$shelf. He bought it$_j$ / them$_j$ yesterday in a bookshop.'}
\label{ex:bookshelf_1}

\ex[] {\gll
Robert zbudował półkę na \minsp{[} te książki]$_j$. Kupił je$_j$ wczoraj w księgarni.\\
Robert.\textsc{nom} built.\textsc{pfv} shelf.\textsc{acc} for {} these books.\textsc{acc}.\textsc{f} bought.\textsc{pfv} them.\textsc{f} yesterday in bookshop.\textsc{loc}\\
\glt `Robert built a shelf for [these books]$_j$. He bought them$_j$ yesterday in a bookshop.'} \label{ex:bookshelf_2}
\z \z

\noindent
Despite its relative merits, the argument based on reference can get us only so far. Even if our assumptions about the universal mapping from DP to individual are correct, we have only shown that kind-referring nominals are DPs, not that they are definite DPs. We still need to demonstrate that the relevant D head bears the feature definite, as opposed to being indefinite or simply unspecified for definiteness.\footnote{Crucially, indefinites are also referential DPs in the sense that they introduce variables which license pronominal co-reference \citep{Heim1982, Kamp.Reyle1993}.} This is what we aim to show in the remaining part of this section, drawing on novel evidence from object topicalization.

Consider the minimal pair in \REF{ex:party}. The contrast between \REF{ex:party_1} and \REF{ex:party_2} relates to the cardinality of the set of girls introduced in the first sentence: while \REF{ex:party_1} mentions a single girl, \REF{ex:party_2} mentions several. The second sentence is identical in both examples, with the accusative object \textit{dziewczynę} `girl' appearing in the sentence-initial position and the nominative subject \textit{przystojny mężczyzna} `handsome man' coming last. The resulting OVS word order is informationally marked, as it deviates from the canonical \ili{Polish} SVO. In the normal case, the fronted object is interpreted as the topic (\textsc{top}) of the sentence.\footnote{An anonymous reviewer points out that \REF{ex:party_1} sounds best when the fronted object is accompanied by a demonstrative determiner. While we agree with this judgment, a bare DP is also acceptable in this context. Since the focus of this section is on the definite\slash indefinite opposition, we leave demonstratives out of the subsequent discussion.}
% % % \largerpage[-1] %to shift (8) on the next page

As it turns out, the topicalized object is acceptable when the context set is singular \REF{ex:party_1} but it is ruled out when the context set is plural \REF{ex:party_2}. From this, we conclude that topicalized objects impose a uniqueness presupposition on their referents, and hence that such objects are definite. This is in line with our intuitive conception of the topic as the informational anchor of a sentence, characterized by such properties as identifiability, familiarity and contextual uniqueness. What this means for our purposes, however, is that we can use object topicalization as a diagnostic of definiteness in kind-referring DPs.\largerpage[-1]

\ea \label{ex:party}
\ea \gll
Na przyjęciu była jedna dziewczyna. Dziewczynę\textsubscript{\textsc{top}} poprosił do tańca przystojny mężczyzna.\\
at party.\textsc{loc} was one girl.\textsc{nom} girl.\textsc{acc} asked to dance handsome.\textsc{nom} man.\textsc{nom}\\

\glt `There was one girl at the party. A handsome man asked the girl to a dance.' \label{ex:party_1}
\pagebreak
\ex \gll
Na przyjęciu było kilka dziewczyn. \minsp{\#} Dziewczynę\textsubscript{\textsc{top}} poprosił do tańca przystojny mężczyzna.\\
at party.\textsc{loc} were several girls.\textsc{nom} {} girl.\textsc{acc} asked to dance handsome.\textsc{nom} man.\textsc{nom}\\

\glt `There were several girls at the party. A handsome man asked the girl to a dance.' \label{ex:party_2}
\z \z

\noindent Before extending this analysis to the domain of kinds, let us examine one more example from the domain of objects. In \REF{ex:cactus}, the first sentence either does \REF{ex:cactus_1} or does not \REF{ex:cactus_2} involve topicalization of the object \textit{kaktus} `cactus'. The follow-up sentence refers to another entity of the same kind, i.e. to a second cactus. If topicalized objects are definite, then \REF{ex:cactus_1} is expected to presuppose the existence of a unique cactus, giving rise to a contradiction with subsequent material.\footnote{Recent work has shed some doubt on the presuppositional effect of topicalization \citep{chapters/seres, Simik.Demian2020}. We leave it as a future task to determine how these proposals affect our argumentation in the main text (if at all).} This is indeed the case.\footnote{This effect is relatively subtle, since the uniqueness presupposition can be pragmatically accommodated without giving rise to a contradiction. For example, one of the cacti might stand out by virtue of being exceptionally large or noteworthy or particularly dear to Mary's heart. In that case, it would be possible to refer to it with a definite description, and the English translation of \REF{ex:cactus_1} produces the same sort of ``defeasible'' infelicity. This qualification notwithstanding, the contrast between \REF{ex:cactus_1} and \REF{ex:cactus_2} is sufficiently robust to warrant the conclusions in the main text. For more on uniqueness and presupposition accommodation, see \citet{Frazier2006}, \citet{VonFintel2008} and references therein.} As for the non-topicalized variant \REF{ex:cactus_2}, it seems that the object can be either definite or indefinite, with the latter interpretation strongly favored by the subsequent context.\footnote{Note that the interaction of definiteness with topicalization, scrambling, intonation, and, to an extent, genericity has been observed previously, e.g. \citet{Szwedek1974}.}

\ea \label{ex:cactus}
\ea \gll
Kaktusa\textsubscript{\textsc{top}} podlała Maria. \minsp{\#} Drugi kaktus nie potrzebował jeszcze wody.\\
cactus.\textsc{acc} watered Mary.\textsc{nom} {} second cactus.\textsc{nom} not needed yet water\\

\glt `Mary watered the cactus. The other cactus did not need water yet.' \label{ex:cactus_1}

\ex \gll
Maria podlała kaktusa. Drugi kaktus nie potrzebował jeszcze wody.\\
Mary.\textsc{nom} watered cactus.\textsc{acc} second cactus.\textsc{nom} not needed yet water\\

\glt `Mary watered a / the cactus. The other cactus did not need water yet.'\\ \label{ex:cactus_2}

\z\z

\noindent
Having demonstrated that object topicalization correlates with definiteness, we can now carry our observations over from the object to the kind domain.

Recall that, according to \citet{Borik.Espinal2012, Borik.Espinal2015}, definiteness is necessary for the emergence of direct reference to kinds. While the English definite \textit{the lightbulb} refers to the maximal kind `lightbulb', the indefinite \textit{a lightbulb} refers only to its subkinds, including `halogen', `fluorescent' and `LED'. The choice between definite and indefinite gives rise to different semantic entailments. Consider an (idealized) scenario in which a successful patent application extends automatically from kinds to all of their subkinds. In that case, \REF{ex:lightbulb_eng_1} grants the evil corporation a patent on all lightbulbs, whether `incadescent', `fluorescent' or any other type. The meaning of \REF{ex:lightbulb_eng_2} is much weaker, since it gives the patentee intellectual rights to only one kind of lightbulb, e.g. `LED' lights.

\ea \label{ex:lightbulb_eng}
\ea The evil corporation patented the lightbulb. \label{ex:lightbulb_eng_1}
\ex The evil corporation patented a lightbulb.\label{ex:lightbulb_eng_2}
\z \z

\noindent
With this in mind, consider the \ili{Polish} examples below. According to conventional wisdom, Thomas Edison is the inventor of the kind `lightbulb'. This fact strongly biases the discourse in \REF{ex:lightbulb_kind} towards the maximal kind reading of the object \textit{żarówka} `lightbulb'. In contrast, the context in \REF{ex:lightbulb_subkind}, which explicitly mentions several subkinds of lighbulbs, is compatible only with the \textsc{subkind} reading of the bare nominal object. What makes this context necessary is that most \ili{Polish} speakers interpret \textit{kind predicate}\,+\,\textit{bare object} constructions as referring to maximal kinds in out-of-the-blue situations.\footnote{We thank an anonymous reviewer for raising this important issue.} This default preference is especially strong when the ambiguous nominal is accompanied by a predicate like \textit{wynaleźć} `invent', which is more often applied to basic kinds (e.g. \textit{the wheel}, \textit{the computer}, \textit{the alphabet}) than to their subkinds. However, when presented with a sufficiently rich context and a more balanced predicate, our informants readily accept that \ili{Polish} bare nominals are ambiguous between definite kind reference and indefinite subkind reference.\largerpage


\ea \label{ex:lightbulb_kind}
\gll
Przełomowe wynalazki są od dawna chronione prawem patentowym.\\
ground-breaking inventions are since long protected law.\textsc{inst} patent.\textsc{adj.inst}\\
\glt `Ground-breaking inventions have long been protected by the patent law.'

\ea \label{ex:lightbulb_kind_1}
\gll
Żarówkę\textsubscript{\textsc{top}} opatentował w 1879 roku Tomasz Edison.\\
lightbulb.\textsc{acc} patented in 1879 year.\textsc{loc} Thomas.\textsc{nom} Edison.\textsc{nom}\\
\glt `Thomas Edison patented the lightbulb in 1879.'

\ex \label{ex:lightbulb_kind_2}
\gll
Tomasz Edison opatentował żarówkę już w 1879 roku.\\
Thomas.\textsc{nom} Edison.\textsc{nom} patented lightbulb.\textsc{acc} already in 1879 year.\textsc{loc}\\
\glt `Thomas Edison patented the lightbulb already in 1879.'
\z \ex \label{ex:lightbulb_subkind}
\gll
W 2019 roku firmy amarykańskie opatentowały cztery rodzaje baterii i trzy rodzaje żarówek.\\
in 2019 year.\textsc{loc} companies.\textsc{nom} american.\textsc{nom} patented four kinds.\textsc{acc} batteries.\textsc{gen} and three kinds.\textsc{acc} lightbulbs.\textsc{gen}\\
\glt `In 2019, American companies patented four kinds of batteries and three kinds of lightbulbs.'

\ea[\#]  { \label{ex:lightbulb_subkind_1}
\gll Żarówkę\textsubscript{\textsc{top}} opatentowała firma mojej żony.\\
 lightbulb.\textsc{acc} patented company.\textsc{nom} my.\textsc{gen} wife.\textsc{gen}\\
\glt `My wife's company patented the lightbulb.'}

\ex[] {\label{ex:lightbulb_subkind_2}
\gll
Firma mojej żony opatentowała żarówkę.\\
company.\textsc{nom} my.\textsc{gen} wife.\textsc{gen} patented lightbulb.\textsc{acc}\\
\glt `My wife's company patented a / the lightbulb.'}
\z \z

\noindent
With these caveats in place, let us return to the examples at hand. Given that topicalized objects are definite and that \REF{ex:lightbulb_kind_1} and \REF{ex:lightbulb_subkind_1} involve object topicalization, we expect \textit{żarówka} `lightbulb' to exhibit the same range of readings as the English definite \textit{the lightbulb}. Specifically, \textit{żarówka} should admit definite kind reference and disallow indefinite subkind reference. In keeping with this prediction, \REF{ex:lightbulb_kind_1} is judged to be true while \REF{ex:lightbulb_subkind_1} is deemed unacceptable. However, indefinite subkind reference becomes available when the object occupies its canonical postverbal position, as in \REF{ex:lightbulb_subkind_2}. Importantly, the availability of a subkind reading in \REF{ex:lightbulb_subkind_2} parallels the availability of an indefinite reading in \REF{ex:cactus_2}.


To summarize our main findings in this section, we have argued that topicalized objects are definite \REF{ex:party_1}, \REF{ex:party_2}, \REF{ex:cactus_1}, and that they must refer to maximal kinds (\ref{ex:lightbulb_kind_1}), (\ref{ex:lightbulb_subkind_1}). As for postverbal objects, they  can be indefinite (\ref{ex:cactus_2}), which makes it possible for them to denote subkinds (\ref{ex:lightbulb_subkind_2}).\footnote{Note that proper names and mass kind nominals can also undergo object topicalization. Does that mean that they are all definite DPs, like the corresponding nominals in some \ili{Romance} languages? The answer depends at least partially on our assumptions about the syntax-semantics mapping (see our discussion at the beginning of this section). If syntax and semantics are isomorphic, then proper names and mass kinds are indeed expected to project full DP structure. For two influential syntactic approaches to reference and proper names, see \citet{Longobardi1994,Longobardi2001,Longobardi2005}, and \citet{Borer2005}.}

Overall, our results strongly suggest that \ili{Polish} kind-referring DPs are definite, just like the corresponding DPs in \ili{Romance} and \ili{Germanic} languages. In \sectref{sec:3-nominals_numberless}, we turn to the other component of \citeposst{Borik.Espinal2012} theory: the role of number in the derivation of kind, subkind and object readings.

% Section 3

\section{Reference to kinds is numberless} \label{sec:3-nominals_numberless}

According to \citet{Borik.Espinal2012, Borik.Espinal2015}, kind-referring DPs are numberless. Since these nominals do not include a number projection, the traditional term ``definite singular kinds'' turns out to be a misnomer.

We start by briefly outlining \citeauthor{Borik.Espinal2012}'s theory in \sectref{sec:sem_num}. This provides the background for our treatment of \ili{Polish} kind-referring DPs in \sectref{sec:numberless_kinds_pol}. By arguing that \ili{Polish} nominals, in their kind-referring uses, are also numberless, we take them to be parallel to other cases treated in the literature, in terms of their underlying semantic and syntactic representation.

\subsection{The semantics of number}
\label{sec:sem_num}

Traditionally, number is assumed to take one of a small set of values. In the context of European languages, and English in particular, nominals are typically assumed to be either singular or plural. In line with \citet{Borik.Espinal2012, Borik.Espinal2015}, we depart from this traditional view and argue that nominals may additionally be numberless, i.e. they may lack the number projection altogether. We thus distinguish three possibilities for the valuation of number: \textsc{singular}, \textsc{plural}, and \textsc{numberless} (corresponding to indefinite singular, bare plural and definite kinds, respectively).

Definite kinds are argued to be numberless rather than singular because they resist number-marking and do not permit the insertion of kind classifiers such as \textit{kind of}, \textit{species of}, and \textit{type of} without the addition of number. Support comes, among others, from \ili{Spanish}, where kind-referring subjects are grammatical only in the absence of any overt expression of number; see \REF{ex:neveradefsg} vs. (\ref{ex:neveradefpl}--\ref{ex:neverasubkind}).\footnote{Although the definite subject takes a singular determiner in \REF{ex:neveradefsg}, we follow \citeauthor{Borik.Espinal2012} in assuming that this is simply a default morphophonological realization and that the feature singular is neither syntactically nor semantically present in this DP.}
Direct reference to the kind `fridge' is blocked not only by plural inflection and overt numerals \REF{ex:neveradefpl}, but also by kind classifiers \REF{ex:neverasubkind}, which require number to project.

\ea \label{ex:nevera}
\ea[] {\gll La nevera se inventó en el siglo XVIII.\\
the.\textsc{sg} fridge \textsc{cl} invented in the century XVIII.\\
\glt `The fridge was invented in the 19\textsuperscript{th} century.'}
\label{ex:neveradefsg}

\ex[*] {\gll  Las (dos) neveras se inventaron en el siglo XVIII.\\
 the.\textsc{pl} (two) fridges \textsc{cl} invented in the century XVIII.\\
\glt Intended: `The (two) fridges were invented in the 19\textsuperscript{th} century.'
\label{ex:neveradefpl}}

\ex[*]{ \gll La clase de nevera se inventó en el siglo XVIII.\\
 the.\textsc{sg} class of fridge \textsc{cl} invented in the century XVIII.\\
\glt Intended: `The type of fridge was invented in the 19\textsuperscript{th} century.'\\\hfill (\citealt{Borik.Espinal2012}; \ili{Spanish})
\label{ex:neverasubkind}}
\z \z

\noindent
In \citeauthor{Borik.Espinal2012}'s theory, the number projection is responsible for introducing the Carlsonian realization operator \cnst{r}, which relates kinds to their spatiotemporal instantiations (see \REF{def:realization_operator}; see also \citealt{Carlson1977}). This explains why direct kind reference is incompatible with number: the latter shifts NP denotations from the domain of kinds to the domain of objects. The formal denotation given to a singular number head in \citet{Borik.Espinal2015} is reproduced below. According to \REF{ex:number}, number turns the property of kinds supplied by the bare NP into a property of objects. This shift is effected by the realization operator \cnst{r}.

\ea \textsc{the realization operator}\label{def:realization_operator}\\
    $\cnst{r}(x^k, y^o) \Leftrightarrow y^o$ instantiates $x^k$

%\ea \textsc{the realization operator}\label{def:realization_operator}\\
 %   $\cnst{r}(x$\textsuperscript{k}, $ y$\textsuperscript{o}) $\Leftrightarrow y$\textsuperscript{o} instantiates $x$\textsuperscript{k}
%\z
%Where do we need \textsubscript insetad of mathmode?



\ex \sib{\textsc{number$^{\textsc{-pl}}$}} $=$ $\lambda P_{\stb{e^k,t}}\lambda y^o.\exists x^k [P(x^k) \wedge \cnst{r}(x^k, y^o) \wedge \cnst{atom}(y^o)]$ \label{ex:number}
\z


\noindent
Even though number is linked to the object domain, it still allows for subkind readings, as evidenced by the English examples below. While the definite subject in \REF{ex:tiger-kind} refers directly to the kind `tiger', and so cannot be used contrastively, its counterparts involving demonstrative determiners \REF{ex:tiger-dem} and numerals \REF{ex:tiger-num} are acceptable in the same context. Similarly, quantification over subkinds is also possible, as in \REF{ex:tiger-quant}. \citeauthor{Borik.Espinal2012} assume that demonstratives, numerals and quantifiers all require the projection of number. Accordingly, they conclude that reference to subkinds is mediated by number, and that subkind denotations are derived from object denotations either via coercion \citep{Borik.Espinal2012} or via type-shifting \citep{Borik.Espinal2015}.\largerpage[-1]

\ea
\ea The tiger is on the verge of extinction (*but that one is not). \label{ex:tiger-kind}
\ex {This / That} tiger is on the verge of extinction (but that one is not). \label{ex:tiger-dem}
\ex One tiger is on the verge of extinction (but six are not). \label{ex:tiger-num}
\ex {No / Some / Every} tiger is on the verge of extinction.
\label{ex:tiger-quant}
\z \z

\noindent
In sum, \citeauthor{Borik.Espinal2012} propose that direct reference to kinds is possible only in the absence of number. Since number encodes the \cnst{r} operator, its projection shifts NP denotations from the kind domain to the object domain. As for subkind readings, they are derived from object readings in the presence of number.


\subsection{Numberless kinds in Polish} \label{sec:numberless_kinds_pol}

By considering data from \ili{Spanish} and English regarding the status of number in kind- vs. object-referring DPs, we have established that the projection of number blocks direct reference to kinds. Instead, only reference to objects or subkinds is licensed.
We now apply the same logic to \ili{Polish}.

First, the overt presence of number clearly blocks direct kind reference in \ili{Polish}. Number can be realized overtly by demonstratives \REF{ex:tiger_dem}, numerals \REF{ex:tiger_num}, and quantifiers \REF{ex:tiger_quant}. A nominal expression incorporating any of these elements may range over objects or subkinds, but crucially it may not refer to the kind `tiger' directly.

\ea \label{ex:tiger}
\ea \gll
\minsp{\{} Ten / Tamten\} tygrys wymarł w XX wieku.\\
{} this {} that tiger.\textsc{nom} {went extinct} in 20\textsuperscript{th} century\\
\glt `\{This / That\} (kind of) tiger went extinct in the 20\textsuperscript{th} century.'
\label{ex:tiger_dem}

\ex \gll
Jeden tygrys jest na skraju wymarcia.\\
one tiger.\textsc{nom} is on verge.\textsc{loc} extinction.\textsc{gen}\\
\glt `One (kind of) tiger is on the verge of extinction.'
\label{ex:tiger_num}

\ex \gll
\minsp{\{} Jakiś / Każdy\} tygrys jest zagrożony wymarciem.\\
{} some {} every tiger is threatened extinction.\textsc{inst}\\
\glt `\{Some / Every\} tiger is under threat of extinction.'
\label{ex:tiger_quant}
\z \z

\noindent
Further, the insertion of kind classifiers in \REF{ex:tiger_class} is similarly incompatible with direct kind reference. The only reading available involves existential quantification over subkinds, as suggested by the use of the indefinite article in the English translation. Crucially, recall that English and \ili{Spanish} do not permit the definite article to co-occur with kind classifiers either (although cf. \REF{ex:attributive_modifier} for a possible qualification of this claim).

\ea \gll
\minsp{\{} Rodzaj / Gatunek / Typ\} tygrysa jest zagrożony wymarciem.\\
{} kind {} species {} type tiger.\textsc{gen} is threatened extinction.\textsc{inst}\\
\glt `A \{kind / species / type\} tiger is under threat of extinction.'
\label{ex:tiger_class}
\z

\noindent
Thus, \ili{Polish} behaves like English and \ili{Spanish} in that it has three possible values for number: plural, singular, and numberless, with the latter two realized as the singular morphological form. Overall, the properties of \ili{Polish} kind-referring DPs are in line with those of \ili{Romance} and \ili{Germanic} kind nominals.
In the next section, we build on the results of \sectref{sec:2-nominals-def} and \sectref{sec:3-nominals_numberless} to address the issue of kind modification.

% Section 4

\section{Kind modification}\label{sec:4-subkinds}

\subsection{Introduction}

Having argued that the denotation of kinds in \ili{Polish} is underlyingly the same as in other languages, we now turn to the question of how to represent subkinds.

There are two main semantic routes leading from properties of kinds to properties of subkinds. The first route was illustrated in \sectref{sec:3-nominals_numberless} in connection with the examples in \REF{ex:tiger}, with \REF{ex:tiger_num} repeated as \REF{ex:subkind_1} below. According to \citeauthor{Borik.Espinal2012}, the presence of morphosyntactic number shifts NP denotations from properties of kinds to properties of objects, which can then be coerced or type-shifted into subkind denotations in the appropriate context. Crucially, this way of referring to subkinds relies on the presence of number in the syntax and semantics.

\ea \label{ex:subkind_1} \gll
Jeden tygrys jest na skraju wymarcia.\\
one tiger.\textsc{nom} is on verge extinction.\textsc{gen}\\
\glt `One (kind of) tiger is on the verge of extinction.'
\z

\noindent
The second route from kinds to subkinds is by way of kind modifiers. The NP \textit{Bengal tiger} is a typical example, with the kind modifier \textit{Bengal} selecting a specific subkind (or set of subkinds) from the denotation of \textit{tiger}. The corresponding example in \ili{Polish}, featuring the classifying adjective \textit{bengalski}, is presented directly below.

\ea \label{ex:subkind_2} \gll
Tygrys bengalski jest na skraju wymarcia.\\
tiger.\textsc{nom} Bengal.\textsc{m} is on verge extinction.\textsc{gen}\\
\glt `The Bengal tiger is on the verge of extinction.'
\z

\noindent
In recent years, our understanding of kind modification has significantly improved thanks to the work of \citet{McNally.Boleda2004} on relational nouns in \ili{Catalan}, as well as to \citet{Wagiel2014} on classifying adjectives in \ili{Polish} and \citet{Borik.Espinal2015} on kind modifiers in \ili{Spanish}. On their approach, the composition of nouns and their modifiers is intersective, proceeding via the composition rule of \textsc{predicate modification} (see \citealt{Heim.Kratzer1998}). In the case of \textit{Bengal tiger}, the set of kinds denoted by \sib{tiger} $=$ \{\textsc{bengal tiger}, \textsc{siberian tiger}, {\dots} \} intersects with the set of kinds denoted by \sib{Bengal}  $=$ \{\textsc{bengal tiger}, \textsc{bengal cat}, {\dots}\}, yielding the correct denotation for the modified NP.

In \sectref{sec:problem-intersec}, we point out that the intersective approach to kind modification is incompatible with the theory of definite numberless kinds proposed by \citet{Borik.Espinal2012, Borik.Espinal2015}. This tension is due to their differing assumptions about the denotation of bare nouns like \textit{tiger}. While \citet{McNally.Boleda2004} assume that nouns denote the maximal kind and all of its subkinds, \citet{Borik.Espinal2012} presuppose that nouns denote singleton sets of kinds. \sectref{sec:towards-solution} elaborates on this problem and lays the groundwork for a solution. Finally, in \sectref{sec:structural-approach}, we integrate the two theories by introducing a subkind operator into the semantics and linking it to a functional head in the syntax. This operator derives properties of subkinds from properties of kinds, thus allowing for intersective kind modification.


\subsection{Incompatibility with intersective kind modification}
\label{sec:problem-intersec}

The simplest way to bring out the tension between intersective kind modification and definite numberless kinds is to go through a pair of step-by-step derivations. We start by deriving direct kind reference in \REF{ex:def_deriv}, with the $\iota$ operator applying to the kind predicate denoted by \textit{tiger}.

\ea \label{ex:def_deriv}

\ea \sib{ $[_\text{NP}$ tygrys ] } $= \lambda x^k.\textsc{tiger}(x^k)$
\label{ex:def_deriv_1}

\ex \sib{ $[_\text{DP}$ \textsc{def} $[_\text{NP}$ tygrys ] ] } $= \iota x^k.\textsc{tiger}(x^k)$
\label{ex:def_deriv_2}
\z \z

\begin{sloppypar}
\noindent
The derivation in \REF{ex:inter_deriv} is slightly more complex, as it involves modification by the classifying adjective \textit{bengalski} `Bengal'. It begins with the definitions of \sib{tygrys} and \sib{bengalski}, both of which denote simple properties of kinds (\ref{ex:inter_deriv_1}--\ref{ex:inter_deriv_2}). These properties are subsequently conjoined in \REF{ex:inter_deriv_3} and bound by the $\iota$ operator in \REF{ex:inter_deriv_4}. The result, a kind-level individual, has the appropriate semantic type to combine with the kind-level predicate \textit{być na skraju wymarcia} `to be on the verge of extinction' in \REF{ex:subkind_2} above.
\end{sloppypar}

\ea \label{ex:inter_deriv}

\ea \sib{ $[_\text{NP}$ tygrys ] } $=$  $\lambda x^k.\textsc{tiger}(x^k)$
\label{ex:inter_deriv_1}

\ex \sib{ $[_\text{AP}$ bengalski ] } $=$  $\lambda x^k.\textsc{bengal}(x^k)$
\label{ex:inter_deriv_2}

\ex \sib{ $[_\text{NP}$ tygrys $[_\text{AP}$ bengalski ] ] } $=$ $\lambda x^k.\textsc{tiger}(x^k) \wedge \textsc{bengal}(x^k)$
\label{ex:inter_deriv_3}

\ex \sib{ $[_\text{DP}$ \textsc{def} $[_\text{NP}$ tygrys $[_\text{AP}$ bengalski ] ] ] } $=$ $\iota x^k.\textsc{tiger}(x^k) \wedge \textsc{bengal}(x^k)$
\label{ex:inter_deriv_4}
\z \z

\noindent The problem with the derivations in \REF{ex:def_deriv} and \REF{ex:inter_deriv} is that they make distinct assumptions about the membership of the set of kinds corresponding to \sib{tiger}. Beginning with definite kind reference, the fact that the $\iota$ operator can apply to \sib{tiger} in \REF{ex:def_deriv_2} entails that \sib{tiger} is a singleton set containing only the maximal kind `tiger'. In other words, this derivation assumes that NP denotations are atomic, as illustrated in \figref{fig:atomic} below, where the outlined area corresponds to the denotation of the NP.{\interfootnotelinepenalty=10000\footnote{\label{ftn:invent}One might wonder if the assumption of atomic NP denotations is a necessary conclusion from \REF{ex:def_deriv}. A possible alternative would be to replace the $\iota$ operator with a maximality operator \textsc{max} defined over sets of pluralities. On its kind referring reading, \textit{the tiger} would then receive a similar analysis to \textit{the boys} in the object domain, picking out the maximal individual in the denotation of a cumulative NP. The problem with this line of thinking is that the domain of kinds is not organized in a semi-lattice structure à la \citet{Link1983}. In addition, this theory makes some incorrect empirical predictions. If definite kinds are underlyingly maximal plurals, we expect the sentence \textit{Charles Babbage invented the computer} to be roughly synonymous with \textit{Charles Babbage invented every kind of computer}. Needless to say, this prediction is not borne out. (For further discussion of the entailments licensed by the predicates \textit{invent} and \textit{be extinct}, see \citealt{MuellerReichau2013}).}


\begin{figure}[H]
\centering
    \begin{forest}
%    for tree={s sep=1cm, inner sep=0, 1=0}
    [\textsc{mammal}
        [\textsc{tiger}, draw
            [\textsc{bali}
            ]
            [\textsc{bengal}
            ]
            [\textsc{siberian}
            ]
            [ {\dots}
            ]
        ]
        [\textsc{lion}
            [\textsc{berber}
            ]
            [...
            ]
        ]
        [\textsc{dog}
            [\textsc{poodle}
            ]
            [ {\dots}
            ]
        ]
    ]
    \end{forest}
\caption{Atomic NP denotations} \label{fig:atomic}
\end{figure}

\noindent
Turning now to modified kind reference in \REF{ex:inter_deriv}, it is incompatible with \sib{tiger} being a singleton set, since \sib{tiger} must be able to intersect with the set \sib{Bengal} in a non-trivial manner. This suggests that the subkind `bengal tiger' is also a member of \sib{tiger}. In that case, however, we are no longer dealing with atomic NP denotations.\largerpage

Rather, for the derivation to work, NPs must have taxonomic denotations, corresponding to the contents of the rectangle in \figref{fig:taxonomic}.\footnote{Perhaps the most influential study to assume taxonomic NP denotations is \citet{Dayal2004}. However, since Dayal derives kind reference via the $\iota$ operator, as already discussed in \sectref{sec:sem_def}, she still needs a mechanism for restricting NP denotations to atomic kinds; otherwise, composition with the $\iota$ operator would violate uniqueness. The question, then, is whether atomic denotations are to be derived from taxonomic ones or the other way around. To the extent that taxonomic denotations are structurally more complex, involving the projection of number or the insertion of kind modifiers, we agree with \citet{Borik.Espinal2012, Borik.Espinal2015} that atomic denotations are more basic.}}

%\vspace*{10pt}

\begin{figure}[h]
\centering
    \begin{forest}
%    for tree={s sep=1cm, inner sep=0, 1=0}
    [\textsc{mammal}
        [\textsc{tiger}, tikz={\node [draw, fit to=tree]{};}
            [\textsc{bali}
            ]
            [\textsc{bengal}
            ]
            [\textsc{siberian}
            ]
            [ {\dots}
            ]
        ]
        [\textsc{lion}
            [\textsc{berber}
            ]
            [ {\dots}
            ]
        ]
        [\textsc{dog}
            [\textsc{poodle}
            ]
            [ {\dots}
            ]
        ]
    ]
    \end{forest}

\caption{Taxonomic NP denotations} \label{fig:taxonomic}
\end{figure}

\subsection{Towards a solution}\label{sec:towards-solution}\largerpage

The incompatibility between atomic and taxonomic NP denotations leaves us with three options. We can (i) abandon \citeposst{Borik.Espinal2012} theory of definite numberless kinds, (ii) abandon \citeposst{McNally.Boleda2004} theory of intersective kind modification, or (iii) find a way of reconciling the two, thus preserving their individual insights and contributions.

Let us begin by considering option (i). Recall that atomic NP denotations follow from the assumption that definiteness translates into \citeposst{Partee1987} $\iota$ operator, which presupposes \textsc{uniqueness}. However, other approaches to the semantics of definiteness have been proposed in the literature. The as-of-yet unresolved debate around the underlying nature of definiteness has focused on aspects thereof that are not directly related to kind and subkind reference. For instance, \citeauthor{Schwarz2009} (\citeyear{Schwarz2009} and subsequent work in \citeyear{Schwarz2013}) breaks down definite determiners into the morphosyntactically identifiable components of \textsc{familiarity} and uniqueness. \citet{Coppock.Beaver2014,Coppock.Beaver2015} elaborate on the notion of definiteness as uniqueness. They argue that \textsc{determinacy} and definiteness are distinct by providing examples of definites which have an indeterminate interpretation, and therefore do not presuppose existence. Ultimately, however, these alternative proposals agree that uniqueness is a crucial component of definiteness. As such, they are not incompatible with the hypothesis that NPs denote singleton sets of kinds.

An alternative approach would be to adopt \citeposst{Lobner1985} idea of definiteness as ``unequivocal identifiability''.\footnote{We would like to thank an anonymous reviewer for bringing up this possibility.} This conception of definiteness can be reconciled with taxonomic NP denotations if we assume that maximal kinds are unequivocally identifiable in \citeauthor{Lobner1985}'s sense due to their position at the top of the taxonomic hierarchy. This is an intriguing hypothesis, but it remains to be seen whether it can be formalized in precise terms, and what kind of taxonomic structure it requires. We thus leave this possibility for future work and retain the assumption of atomic NP denotations for the rest of this paper.

What about the second option, i.e. abandoning our commitment to intersective kind modification? Indeed, \citeauthor{Borik.Espinal2012} seem to have tacitly adopted this solution in their more recent work (see \citealt{Borik.Espinal2018, Borik.Espinal2016}). In their representation of the \ili{Russian} modified kind nominal \textit{slon afrikanskij} `African elephant', the adjective has the semantic type $\stb{\stb{e^k,t},\stb{e^k,t}}$, which makes it a function from properties of kinds to properties of kinds. \citeauthor{Borik.Espinal2012}'s revised syntax and semantics for modified kinds is reproduced in \REF{ex:noninter_deriv} below.

\ea
\sib{ $[_\text{DP}$ \textsc{def} $[_\text{NP}$ slon $[_\text{AP}$ afrikanskij ] ] ] } $=$ $\lambda x^k.$(\sib{afrikanskij}(\sib{slon}))$(x^k)$
\label{ex:noninter_deriv}
\z

\noindent
As it stands, \REF{ex:noninter_deriv} leaves a number of questions unanswered. Most importantly, it does not specify how the adjectival function affects the denotation of the noun. What is the precise relationship between $\lambda x^k.$\sib{slon}$(x^k)$, on the one hand, and $\lambda x^k.($\sib{afrikanskij}(\sib{slon}))$(x^k)$, on the other? Without this information, it is impossible to verify whether \REF{ex:noninter_deriv} derives the correct truth conditions for \textit{slon afrikanskij}.
% % % \largerpage[2] %to get (24) on one page and eventually fn 16 on one page

One simple possibility is that \sib{afrikanskij} takes the property of kinds denoted by \sib{slon} and conjoins it with the predicate of African kinds, yielding the result in \REF{ex:naive_deriv_3}. The composition process no longer relies on predicate modification, proceeding exclusively via \textsc{function application} instead. Still, \REF{ex:naive_deriv} fails for the same reason as the derivation in \REF{ex:inter_deriv}: if \sib{slon} denotes a singleton set of kinds, then its intersection with the set of African kinds is an empty set.

\ea \label{ex:naive_deriv}
\ea \sib{slon} $=$ $\lambda x^k.\textsc{elephant}(x^k)$ \label{ex:explicit_deriv_1}
\ex \sib{afrikanskij} $=$ $\lambda P_{\stb{e^k, t}}\lambda x^k.[P(x) \wedge \textsc{african}(x)]$ \label{ex:naive_deriv_2}
\ex \sib{slon afrikanskij} $=$ $\lambda x^k.[\textsc{elephant}(x) \wedge \textsc{african}(x)]$ \label{ex:naive_deriv_3}
\z \z

\noindent
Let us see, then, if we can improve on the idea in \REF{ex:naive_deriv}. Our starting assumption is that \sib{slon} denotes a singleton set of kinds \REF{ex:explicit_deriv_1b} and that \sib{afrikanskij} maps properties of kinds onto other properties of kinds by means of some yet-to-be-specified function \cnst{func}:

\ea \label{ex:explicit_deriv}
\ea \sib{slon} $=$ $\lambda x^k.\textsc{elephant}(x^k)$ \label{ex:explicit_deriv_1b}
\ex \sib{afrikanskij} $=$ $\lambda P_{\stb{e^k, t}}\lambda y^k. \cnst{func}(P)(y^k)$ \label{ex:explicit_deriv_2b}
\ex \sib{slon afrikanskij} $=$ $\lambda y^k. \cnst{func}(\lambda x^k.\textsc{elephant}(x^k))(y^k)$ \label{ex:explicit_deriv_3b}
\z \z


\noindent What are the minimal requirements for the content of \cnst{func}? Since \cnst{func} can take the singleton set of kinds \{\textsc{elephant}\} as an input and return the set \{\textsc{african elephant}\} as an output, it must necessarily incorporate some sort of a subkind operator in its definition. The subkind operator, defined in \REF{def:subkind_operator} below, is a dyadic relation between kinds and their subkinds (which are also in the kind domain).\footnote{Other suggestions for operators relating kinds to subkinds have been made, most notably by \citet[77]{Krifka.etal1995}. \citeposst{Krifka.etal1995} taxonomic subkind relation \cnst{t} relates a subkind $x$ to a (basic level) kind $y$ in an asymmetric and transitive manner: $\cnst{t}(x,y)$. However, this account makes no explicit assumptions about the relationship between kinds and subkinds, and in particular, it does not comment on the mechanism of kind-modification. Rather, \citet{Krifka.etal1995} focus on the distinction between the domain of kinds and the domain of objects.}
In effect, \cnst{func} is now able to derive a set of subkinds \{\textsc{african elephant}, \textsc{asian elephant}, \textsc{indian elephant}, ... \} from the input set \{\textsc{elephant}\}.

\ea \textsc{the subkind operator}\label{def:subkind_operator}\\
    $\cnst{sk}(x^k, y^k) \Leftrightarrow y^k$ is a subkind of $x^k$
\z

\noindent
What remains is for \cnst{func} to select the appropriate subkind from this set. This can be plausibly achieved by intersecting this set with the set of African kinds \textsc{african} $=$ \{\textsc{african elephant}, \textsc{african giraffe}, \textsc{african language}, \textsc{african music}, {\dots} \}, much in the spirit of \citeposst{McNally.Boleda2004}. Without postulating such a set of African kinds, the systematic contribution of the adjective \sib{afrikanskij} to the meaning of \sib{\textsc{n} afrikanskij} (roughly, `specific to Africa') cannot be captured.\largerpage

In light of the above, we propose the following definition of the kind-modify\-ing function \cnst{func}. In our view, the classifying adjective \sib{afrikanskij} takes a property of kinds \textit{P} as an input, derives from it a property of \textit{P}-subkinds by means of the \cnst{sk} operator, and finally conjoins that property with the kind predicate \textsc{african} \REF{ex:final_deriv_2}. The result of applying \REF{ex:final_deriv_2} to \REF{ex:final_deriv_1} is a predicate of \textsc{african} kinds that stand in a subkind relation to the kind `elephant', i.e. a description of the kind `african elephant' \REF{ex:final_deriv_3}.

\ea \label{ex:final_deriv}
\ea \sib{slon} $=$ $\lambda x^k.\textsc{elephant}(x^k)$ \label{ex:final_deriv_1}

\ex \sib{afrikanskij} $=$ $\lambda P_{\stb{e^k, t}}\lambda y^k. \exists x^k [P(x^k) \wedge \cnst{sk}(x^k, y^k)$ $\wedge$ $\textsc{african}(y^k)]$ \label{ex:final_deriv_2}

\ex \sib{slon afrikanskij} $=$ $\lambda y^k. \exists x^k [\textsc{elephant}(x^k)$ $\wedge$ $\cnst{sk}(x^k, y^k)$ $\wedge$ $ \textsc{african}(y^k)]$
\label{ex:final_deriv_3}
\z \z

\noindent
We are now in the position to verify whether a non-intersective analysis of kind modification, with the adjective \textit{afrikanskij} denoting a complex function from properties of kinds to properties of kinds, allows us to avoid the contradiction identified in \sectref{sec:problem-intersec}. The short answer is yes. By hard-wiring the \cnst{sk} operator into the denotation of kind modifiers, we can maintain our assumption that NPs denote sets of atomic kinds and still derive modified kinds along the lines of \citeauthor{Borik.Espinal2012}'s (\citeyear{Borik.Espinal2012,Borik.Espinal2015}) theory.
% % \largerpage %to get the whole fn 16 on p. xx

\sloppy However, the assignment of the complex type $\stb{\stb{e^k,t}, \stb{e^k,t}}$ to classifying adjectives comes at a certain cost.
Barring the possibility of type-shifting, kind modifiers are now locked to the attributive position, contrary to empirical fact \REF{ex:predicative_modifier}.

\ea \label{ex:predicative_modifier}
\ea \gll
Ten \minsp{\{} rodzaj / gatunek / typ \} słonia jest afrykański, a tamten jest azjatycki.\\
this.\textsc{nom} {} kind.\textsc{nom} { } species.\textsc{nom} { } type.\textsc{nom} { } elephant.\textsc{gen} is African.\textsc{nom} and that.\textsc{nom} is Asian.\textsc{nom}\\
\glt `This \{kind/species/type\} of elephant is African and that one is Asian.'
\ex \gll
Ten rodzaj szczoteczki jest elektryczny.\\
this kind.\textsc{nom} toothbrush.\textsc{gen} is electric.\textsc{nom}\\
\glt `This kind of toothbrush is electric.'
\z \z

\noindent Furthermore, if the lexical entries of classifying adjectives encode their own \cnst{sk} operators, then the DP \textit{afrykański rodzaj słonia} `African kind of elephant' should range exclusively over subkinds of subkinds of the kind `elephant', including such specialized kinds as `African forest elephant' and `African bush elephant'. This is because the classifying adjective \textit{afrykański} and the kind classifier \textit{rodzaj} would each introduce an instance of the \cnst{sk} operator into the semantic derivation. Contrary to this prediction, definite subkind reference to `African elephant' is possible in \REF{ex:attributive_modifier_1}, derived via a single application of the \cnst{sk} operator.\footnote{Indefinite subkind reference is also available for the subject of \REF{ex:attributive_modifier_1}, but we assume that it involves the projection of number, analogously to the variant with the overt cardinal below:
\ea
\gll
Jeden afrykański rodzaj słonia jest na granicy wymarcia.\\
one African.\textsc{nom} kind.\textsc{nom} elephant.\textsc{gen} is on verge.\textsc{loc} extinction.\textsc{gen}\\
\glt `One African kind of elephant is on the verge of extinction.'
\z}

\ea \label{ex:attributive_modifier}
\ea \gll
Afrykański \minsp{\{} rodzaj / gatunek / typ \} słonia jest na granicy wymarcia.\\
African.\textsc{nom} {} kind.\textsc{nom} { } species.\textsc{nom} { } type.\textsc{nom} { } elephant.\textsc{gen} is on verge.\textsc{loc} extinction.\textsc{gen}\\
\glt `\{The / An\} African \{ kind / species / type \} of elephant is on the verge of extinction.'
\label{ex:attributive_modifier_1}

\ex \gll
\minsp{\{} Rodzaj / Gatunek / Typ \} słonia jest na granicy wymarcia.\\
{} kind.\textsc{nom} { } species.\textsc{nom} { } type.\textsc{nom} { } elephant.\textsc{gen} is on verge.\textsc{loc} extinction.\textsc{gen}\\
\glt `A \{ kind / species / type \} of elephant is on the verge of extinction.'
\label{ex:attributive_modifier_2}
\z \z

\noindent
In light of this result, consider the final contrast between \REF{ex:attributive_modifier_1} and \REF{ex:attributive_modifier_2}, with and without the classifying adjective. The former licenses definite reference to `African elephant', while the latter admits only indefinite subkind reference, similarly to their English translations. This asymmetry can be explained if \sib{rodzaj słonia} in \REF{ex:attributive_modifier_2} corresponds to a plural set of subkinds, which does not satisfy the uniqueness presupposition on the $\iota$ operator, thereby excluding the definite reading. But if we first conjoin \sib{rodzaj słonia} with \sib{afrykański}, the result might well be a singleton set, rendering definite subkind reference licit in \REF{ex:attributive_modifier_1}.

In sum, while abandoning intersective kind modification removes the contradiction pointed out in \sectref{sec:problem-intersec}, the hypothesis that classifying adjectives have the semantic type $\stb{\stb{e^k,t},\stb{e^k,t}}$ and that they lexicalize the \cnst{sk} operator runs afoul of the empirical facts in (\ref{ex:predicative_modifier}--\ref{ex:attributive_modifier}). For this reason, we hold on to \citeposst{McNally.Boleda2004} and \citeposst{Wagiel2014} assumption that kind modifiers are simple properties of kinds (contra \citealt{Borik.Espinal2018}). In the next section, we show how to reconcile this assumption with the theory of definite numberless kinds. Tightening the link between syntactic structure and interpretation, our proposal links the appearance of the \cnst{sk} operator to the projection of a \textsc{SubkindP(hrase)} in the syntax.

\subsection{A structural approach to kind modification}
\label{sec:structural-approach}

We assume the following structure for \textit{słoń afrykański} on its subkind reading:

\ea $[_\text{DP}$ \textsc{def} $[_\text{SubkindP}$ $[_\text{AP}$ afrykański $]$ $[_\text{Subkind'}$  Subkind $[_\text{NP}$ słoń ] ] ] ]
\label{ex:structure_1}
\z

\noindent
This structure incorporates a syntactic projection labelled SubkindP. This projection is the structural locus of the \cnst{sk} operator. The NP is in the complement of SubkindP, while the AP occupies the specifier position. In this way, the Subkind head mediates the semantic composition of the noun and the adjective. A step-by-step translation of this structure is presented below:

\ea
\ea \sib{słoń} $=$ $\lambda x^k.\textsc{elephant}(x^k)$\\
\ex \sib{\textsc{subkind}} $=$ $\lambda P_{\stb{e^k, t}}\lambda y^k.\exists x^k [P(x^k) \wedge \cnst{sk}(x^k, y^k)]$
\ex \sib{\textsc{subkind} słoń} $=$ $\lambda y^k.\exists x^k [\textsc{elephant}(x^k) \wedge \cnst{sk}(x^k, y^k)]$\\
    \hfill via \textsc{function application}
\ex \sib{afrykański} $=$ $\lambda x^k.\textsc{african}(x^k)$\\
\ex \sib{afrykański (c)} $=$ $\lambda y^k.\exists x^k [\textsc{elephant}(x^k) \wedge \cnst{sk}(x^k, y^k) \wedge \textsc{african}(y^k)]$\\
    \hfill via \textsc{predicate modification}
\ex \sib{\textsc{def} (e)} $=$ $\iota y^k.\exists x^k [\textsc{elephant}(x^k) \wedge \cnst{sk}(x^k, y^k) \wedge \textsc{african}(y^k)]$\\
\z \z

\noindent
By postulating the syntactic Subkind head, which translates as the semantic \cnst{sk} operator, we have achieved several things. Firstly, we have resolved the contradiction inherent in the derivations in (\ref{ex:def_deriv}--\ref{ex:inter_deriv}) above. Furthermore, we have done so while maintaining a simple intersective semantics for kind modifiers à la \citet{McNally.Boleda2004}.

An outstanding question concerns the prenominal vs. postnominal status of \ili{Polish} adjectives. Classifying (kind-level) adjectives tend to follow the noun in \ili{Polish}, but they can also precede it, e.g. \textit{słoń afrykański} vs.\ \textit{?afrykański słoń} `African elephant'. This contrasts with modifying (object-level) adjectives, which obligatorily precede the noun, e.g. \textit{czerwony robot} vs. \textit{*robot czerwony} `red robot'. Given the structure in \REF{ex:structure_1}, we must find a way of linearizing the noun to the left of the classifying adjective. One way of achieving this result is via head movement. For an approach postulating head movement of N to some functional projection above SubkindP, see \citet{Rutkowski.Progovac2005} and \citet{Rutkowski2012}.

Alternatively, we could assume a more flexible approach to syntactic structure along the lines of \citet{Cinque2005, Cinque2010}, with linear order derived by means of phrasal movement. In order to arrive at the (AP) > NOM > (AP) word order for modified kinds in \ili{Polish}, where the brackets indicate optionality, we only need to assume that SubkindP optionally attracts the NP to its specifier. (A related possibility is that there is an agreement projection above SubkindP and that this AgrP optionally attracts the NP.)

While we do not intend to adjudicate between the head-movement and phrasal-movement approaches to adjectival ordering, we note the significance of the word-order data for our analysis. Specifically, the fact that classifying adjectives exhibit different word-order properties from modifying ones supports the structural approach to kind modification, according to which classifying adjectives are associated with a dedicated subkind projection in the syntax.\footnote{For further discussion of the nominal syntax in \ili{Polish}, see \citet{Ceglowski2017}, \citet{Witkos.etal2018}, and \citet{Witkos.Dziubala.Szrejbrowska2018}.}

Having touched upon the issue of linearization, we now turn to the empirical consequences of our proposal. One advantage of positing a syntactic SubkindP is that it enables us to model the definite subkind reading of \textit{afrykański rodzaj słonia} in \REF{ex:attributive_modifier_1}, repeated as \REF{ex:kind_head} below. We assign this DP the syntactic structure in \REF{ex:structure_2}. Our claim is that the kind classifier \textit{rodzaj} `kind' is an overt realization of the subkind head. This move not only  captures the semantics of kind classifiers, which license the \cnst{sk} operator, but it also accounts for their co-occurrence with classifying adjectives.

\ea \gll
Afrykański \{ rodzaj / gatunek / typ \} słonia jest na granicy wymarcia.\\
African.\textsc{nom} { } kind.\textsc{nom} { } species.\textsc{nom} { } type.\textsc{nom} { } elephant.\textsc{gen} is on verge.\textsc{loc} extinction.\textsc{gen}\\
\glt `\{The / An\} African \{ kind / species / type \} of elephant is on the verge of extinction.'
\label{ex:kind_head}
\z

\ea $[_\text{DP}$ \textsc{d} $[_\text{SubkindP}$ $[_\text{AP}$ afrykański $]$ $[_\text{Subkind'}$ rodzaj $[_\text{NP}$ slonia ] ] ] ]
\label{ex:structure_2}
\z

\noindent
Furthermore, if the structural approach is on the right track, it appears that we must allow the Subkind head to be recursive. A recursive application of the \cnst{sk} operator is clearly necessary to derive such examples as \REF{ex:recursive_1}, \REF{ex:recursive_2}, and \REF{ex:recursive_3}, all of which refer to subkinds of subkinds.\footnote{We thank an anonymous reviewer for raising the issue of recursive subkind derivation, and for asking us to discuss examples \REF{ex:recursive_2} and \REF{ex:recursive_3} specifically.} At a sufficiently abstract level of representation, the examples in (\ref{ex:recursive_1}--\ref{ex:recursive_3}) share the same underlying structure, with two Subkind projections inserted between the NP and the DP layers.\largerpage

\ea \label{ex:recursive_1}
\ea \gll
polska literatura współczesna\\
\ili{Polish}.\textsc{adj} literature.\textsc{nom} contemporary.\textsc{adj}\\
\glt `contemporary \ili{Polish} literature'

\ex
{}[\textsubscript{DP} \textsc{d} [\textsubscript{Subkind2P} polska [\textsubscript{Subkind2'} Subkind\textsubscript{2} [\textsubscript{Subkind1P} współczesna [\textsubscript{Subkind1'} Subkind\textsubscript{1} [\textsubscript{NP} literatura ] ] ] ] ] ]
\z \z

\ea \label{ex:recursive_2}
\ea \gll
ten słoń afrykański\\
this elephant.\textsc{nom} African.\textsc{adj}\\
\glt `this (kind of) African elephant'

\ex
{}[\textsubscript{DP} ten [\textsubscript{Subkind2P} Subkind\textsubscript{2} [\textsubscript{Subkind1P} afrykański [\textsubscript{Subkind1'} Subkind\textsubscript{1} [\textsubscript{NP} słoń ] ] ] ] ]
\z \ex \label{ex:recursive_3}
\ea \gll
\{ rodzaj / gatunek / typ \} słonia afrykańskiego\\
{ } kind.\textsc{nom} { } species.\textsc{nom} { } type.\textsc{nom} { } elephant.\textsc{gen} African.\textsc{adj}.\textsc{gen}\\
\glt `a \{kind / species / type\} of African elephant'

\ex
{}[\textsubscript{DP} \textsc{d} [\textsubscript{Subkind2P} rodzaj [\textsubscript{Subkind1P} afrykańskiego [\textsubscript{Subkind1'} Subkind\textsubscript{1} [\textsubscript{NP} słonia ] ] ] ] ]
\z \ex \label{ex:recursive_4}
\ea \gll
afrykański \{ rodzaj / gatunek / typ \} słonia\\
African.\textsc{nom} { } kind.\textsc{nom} { } species.\textsc{nom} { } type.\textsc{nom} { } elephant.\textsc{gen}\\
\glt `\{the / an\} African \{kind / species / type\} of elephant'

\ex $[_\text{DP}$ \textsc{d} $[_\text{SubkindP}$ $[_\text{AP}$ afrykański $]$ $[_\text{Subkind'}$ rodzaj $[_\text{NP}$ slonia ] ] ] ]
\z \z

\noindent
To derive the modified nominal in \REF{ex:recursive_1}, which refers to a subkind of contemporary literature, all we need to assume is that the adjectives \textit{polska}{ }`\ili{Polish}' and \textit{współczesna}{ }`contemporary' occupy the specifier positions of Subkind$_2$P and Subkind$_1$P, respectively. As for the subkind-of-a-subkind reading of \REF{ex:recursive_2}, the AP \textit{afrykański} `African' occupies the lower SpecSubkind$_1$P, while Subkind$_2$P projects covertly to provide focus alternatives for the demonstrative determiner (i.e. \textit{this} subkind of African elephant, but not \textit{that} one). Example \REF{ex:recursive_3} is very similar to \REF{ex:recursive_2}, with the main difference that the higher Subkind$_2$ head is realized overtly by one of the kind classifiers \textit{rodzaj}\slash\textit{gatunek}\slash\textit{typ}.

In closing, consider the contrast between \textit{rodzaj}.\textsc{nom} \textit{słonia}.\textsc{gen} \textit{afrykańskiego}.\textsc{gen} \REF{ex:recursive_3} and \textit{afrykański}.\textsc{nom} \textit{rodzaj}.\textsc{nom} \textit{słonia}.\textsc{gen} \REF{ex:recursive_4} (the latter repeated from \REF{ex:kind_head} above). Although these examples are similar on the surface, their interpretation differs in a way directly predicted by our account. In \REF{ex:recursive_3},  the classyfing adjective \textit{afrykański} and the kind classifier \textit{rodzaj} occupy distinct Subkind projections, yielding the recursive subkind-of-a-subkind reading. The existence of two Subkind projections in \REF{ex:recursive_3} is supported by the following considerations: (i) the adjective \textit{afrykański} agrees with the lexical noun \textit{słoń} rather than with the kind classifier \textit{rodzaj}, and (ii) the adjective and the kind classifier are not linearly adjacent.

In contrast, the adjective in \REF{ex:recursive_4} agrees with the kind classifier in gender, number and case. It also immediately precedes the kind classifier in the linear order. This suggests that they originate in one and the same SubkindP, as argued already at the end of \sectref{sec:towards-solution} (see example \REF{ex:attributive_modifier_1} and the surrounding discussion). As expected, while the nominal in \REF{ex:recursive_3} ranges exclusively over subkinds of subkinds, \REF{ex:recursive_4} may refer directly to the subkind `African elephant'. The structural approach to kind modification, together with the assumption that the subkind head may be recursive, successfully captures this subtle semantic contrast.


\subsection{Possible extensions}\largerpage

One outstanding question concerns the relationship between the subkind operator \cnst{sk} and the realization operator \cnst{r} (introduced in SubkindP and NumberP, respectively). As has been amply demonstrated, subkind readings are normally available in the presence of number (see especially \sectref{sec:numberless_kinds_pol}). Indeed, it was this observation which motivated \citet{Borik.Espinal2012,Borik.Espinal2015} to hypothesize that subkind denotations are built on number. According to their analysis, subkind readings are derived from object readings by means of coercion or type-shifting.

However, since we have explicitly denied the existence of type-shifting in \sectref{sec:def_kinds_pol}, we must find an alternative explanation for the co-occurrence of number and subkind interpretation. Below, we outline a possible solution to this problem.

Our proposal assumes the existence of a Classifier phrase in the nominal extended projection. This functional head is ordered between NumberP and NP (see \citealt{Borer2005} and \citealt{Picallo2006}, among others).

For concreteness, we adopt the particular proposal of \citet{Kratzer2007}, according to which ClassifierP derives a set of singular atoms from the kind property supplied by the NP. This means that [$-$plural] is the default value of number (as per \citeauthor{Borik.Espinal2012}'s assumptions). Plural denotations are derived at the [$+$plural] head via the operation of sum closure. As a result, the internal structure of a DP looks as in \figref{fig:nom_proj}.

\begin{figure}[H]
\centering
    \begin{forest}
    for tree={s sep=1cm, inner sep=0, l=0}
    [DP
        [Determiner
        ]
        [NumberP
            [\textsc{[$\pm$plural]}
           ]
            [ClassifierP
             [Classifier
                 ]
                 [NP
                   ]
                ]
         ]
    ]
    \end{forest}
    \caption{The extended projection of N}
    \label{fig:nom_proj}
\end{figure}

\noindent
Tentatively, we propose that SubkindP is simply a type or `flavor' of ClassifierP rather than an independent piece of functional structure. If this is on the right track, then its co-occurrence with NumberP is fully expected. We further assume that ClassifierP is the locus of the realization operator \cnst{r} (contra \citealt{Borik.Espinal2012, Borik.Espinal2015}, who attribute \cnst{r} to number). Thus, depending on its particular value, Classifier can introduce either the \cnst{sk} or the \cnst{r} operator into the semantic derivation. When \cnst{sk} is present, \sib{ClassifierP} denotes a set of atomic subkinds. When \cnst{r} appears, \sib{ClassifierP} translates as a set of atoms from the object domain. The presence of [$+$plural] renders both of these sets cumulative.

Given our discussion of recursive subkinds at the end of \sectref{sec:structural-approach}, we must allow for the presence of multiple classifier heads in the syntactic structure. But does this mean that \textsc{classifier}[\cnst{sk}] and \textsc{classifier}[\cnst{r}] may alternate and interleave in a completely unrestricted manner? Not if we let semantics constrain the output of syntactic derivations. We propose that the iteration of classifier heads is constrained by the semantic restrictions on the application of the \cnst{sk} and \cnst{r} operators. On the one hand, we expect \textsc{classifier}[\cnst{sk}] to iterate freely. This is because its input (a set of kinds) is of the same type as its output (another set of kinds), which is a necessary condition for recursion. On the other hand, \textsc{classifier}{[\cnst{r}]} shifts nominal denotations from the domain of kinds to the domain of objects. As such, it can apply at most once following all applications of \cnst{sk}.

Finally, we must explain why the projection of number is incompatible with direct reference to kinds, admitting only object or subkind reference (see \sectref{sec:3-nominals_numberless} for the relevant discussion). To account for this observation, it is enough to assume that the projection of number entails the projection of Classifier, and hence the appearance of \cnst{r} or \cnst{sk} in the semantics. This a natural conclusion to draw, especially if Classifier is responsible for determining the unit of counting, as is commonly assumed. In fact, the claim that NumberP can project if and only if ClassifierP projects is made explicitly in \citet{Picallo2006}.

In sum, by adopting the classifier projection and identifying it as the locus of the \cnst{sk} and \cnst{r} operators, we have been able to account for all the data covered by \citeauthor{Borik.Espinal2012}'s original theory. What is more, we have done so without resorting to type-shifting or coercion as the source of subkind interpretations. According to our analysis, all subkind readings, whether triggered by number, kind modifiers, or kind classifiers, are derived in a uniform manner: they involve the projection of ClassifierP/SubkindP, which introduces the \cnst{sk} operator into their semantics.

% Section 5

\section{Conclusion}\label{sec:5-conclusions}

In this paper, we have argued that \ili{Polish} kind-referring nominals have the same syntax and semantics as their counterparts in \ili{Romance} and \ili{Germanic} languages. Specifically, we have shown that \ili{Polish} kind nominals are definite, as supported by the evidence from object topicalization. We have also shown that they are numberless, extending the conclusions of \citet{Borik.Espinal2012, Borik.Espinal2015} drawn on the basis of English, \ili{Spanish}, and \ili{Russian} data.

The main argument pursued in this paper concerns the incompatibility between \citeauthor{Borik.Espinal2012}'s theory of definite numberless kinds and \citeposst{McNally.Boleda2004} idea of intersective kind modification. While the former presupposes atomic NP denotations, the latter assumes that NPs denote entire taxonomies. We have shown that atomic NPs can combine with kind modifiers only through the mediation of the subkind operator \cnst{sk}. By linking this operator to a SubkindP in the syntax, we have been able to account for some new data involving the co-occurrence of kind modifiers and kind classifiers.

In addition to that, we have made the tentative suggestion that SubkindP is a type of a more general Classifier projection, the latter assumed already in \citet{Borer2005}, \citet{Picallo2006}, and \citet{Kratzer2007}. By transferring the Carlsonian realization operator \cnst{r} from the number to the Classifier head, we did away with the need for type-shifting in the semantics. Instead, we have provided a uniform structure for all cases of reference to subkinds, whether achieved through number, classifying adjectives and/or kind classifiers: all of these constructions involve the projection of a Classifier[\textsc{sk}] on top of the NP.

We summarize the whole system directly below. In (\ref{def:definiteness_repeat}--\ref{def:np_repeat}), we list the semantic denotations of all the elements which enter into our analysis.

\ea \textsc{definiteness} \label{def:definiteness_repeat}\\
     \sib{\textsc{d[$+$def]}} $=$ $\lambda P.\iota x[P(x)]$
\ex \textsc{number} \label{def:number_repeat}\\
\ea    \sib{\textsc{num[$+$pl]}} $=$ $\lambda P\lambda X.$*$P(x)$
\ex    \sib{\textsc{num[$-$pl]}} $=$ $\lambda P\lambda x.P(x)$
\z \ex \textsc{the realization operator} \label{def:realization_operator_repeat}\\
\ea  \cnst{r}($x^k, y^o) \Leftrightarrow y^o$ instantiates $x^k$
\ex  \sib{Classifier\textsc{[r]}} $=$ $\lambda P_{\stb{e^k, t}}\lambda y^o.\exists x^k [ P(x^k) \wedge \cnst{r}(x^k, y^o) ]$
\z \ex \textsc{the subkind operator}\label{def:subkind_operator_repeat}\\
\ea  $\cnst{sk}(x^k, y^k) \Leftrightarrow y^k$ is a subkind of $x^k$
\ex  \sib{Classifier\textsc{[sk]}} $=$ $\lambda P_{\stb{e^k, t}}\lambda y^k.\exists x^k [ P(x^k) \wedge \cnst{sk}(x^k, y^k) ]$
\z \ex \textsc{atomic np denotations}\label{def:np_repeat}\\
    \sib{NP} $=$ $\lambda x^k.P_{\textsc{noun}}(x^k) \wedge |P_{\textsc{noun}}| = 1$
\z

\noindent
The final structures assigned to kind-, subkind- and object-denoting definite DPs are presented in Figures \ref{fig:direct_kind_structure}, \ref{fig:subkind_structure}, and \ref{fig:individual_structure}, respectively. Finally, \figref{fig:number_structure} shows that NumberP projects only in the presence of ClassifierP. By introducing one of the operators \cnst{r} or \cnst{sk}, the Classifier head blocks direct reference to kinds and triggers reference to objects or subkinds instead. This derives \citeauthor{Borik.Espinal2012}'s central observation that definite kind-referring DPs are necessarily numberless.\largerpage

\begin{figure}[b]
\centering
    \begin{forest}
    for tree={s sep=1cm, inner sep=0, l=0}
    [DP
        [\textsc{[$+$def]}]
                 [NP]
         ]
    ]
    \end{forest}
    \caption{The structure of a definite kind nominal}
    \label{fig:direct_kind_structure}
\end{figure}

\begin{figure}
\centering
    \begin{forest}
    for tree={s sep=1cm, inner sep=0, l=0}
    [DP
        [\textsc{[$+$def]}]
         [NumberP
          [\textsc{[$-$plural]}]
          [ClassifierP
            [ (classifying AP) ]
            [ Classifier$'$
             [\textsc{[sk]}]
                 [NP]
            ]
           ]
         ]
    ]
    \end{forest}
    \caption{The structure of a definite (modified) subkind nominal}
    \label{fig:subkind_structure}
\end{figure}

\begin{figure}
\centering
    \begin{forest}
    for tree={s sep=1cm, inner sep=0, l=0}
    [DP
        [\textsc{[$+$def]}]
         [NumberP
          [\textsc{[$-$plural]}]
          [ClassifierP
             [\textsc{[r]}
                 ]
                 [NP]
          ]
         ]
    ]
    \end{forest}
    \caption{The structure of a definite object-level nominal}
    \label{fig:individual_structure}
\end{figure}

\begin{figure}
\centering
    \begin{forest}
    for tree={s sep=1cm, inner sep=0, l=0}
    [DP
        [\textsc{[$\pm$def]}]
         [NumberP
          [\textsc{[$\pm$plural]}]
          [ClassifierP
             [\textsc{[r/sk]}
                 ]
                 [NP]
          ]
         ]
    ]
    \end{forest}
    \caption{NumberP requires the projection of ClassifierP}
    \label{fig:number_structure}
\end{figure}


If our analysis is on the right track, the mapping between syntactic structure and semantic interpretation is very nearly isomorphic. In this way, our work extends the line of research starting with \citet{Krifka1995} and continued in \citet{Dayal2004} and \citet{Borik.Espinal2012, Borik.Espinal2015}, which seeks to explicitly relate the syntax and semantics of kind-, subkind- and object-referring DPs.


%%%%%%%%%%%%%%%%%%%%%%%%%%%%%%%%%%%%%%%%%%%%
% % \documentclass[]{diss}
% \begin{document}

\chapter{Introduction}
% German locative phrases
Spatial language is a vast topic. This book focusses on locative phrases, 
which are phrases that single out objects in the physical environment with the 
\textsc{communicative intention} 
to draw attention to these objects. The following shows an example
of a locative phrase from German.
\ea
\label{e:der-block-rechts-der-kiste-von-dir-aus}
\gll der Block rechts der Kiste von dir aus \\
the.{\NOM} block.{\NOM} right.{\PREP} the.{\GEN} box.{\GEN} from.{\PREP} your.{\DAT} perspective \\
\glt `The block to the right of the box from your perspective'\\
\z

% highly developed tool
Phrases like this can be seen as highly complex tools that help dialog 
partners to establish spatial reference. The utterance conveys to the hearer 
a number of instructions such as (1) apply the spatial relation {\footnotesize\tt right},
(2) use a particular landmark and (3) take the perspective of the interlocutor.
These instructions, when applied properly, allow an interlocutor 
to identify the object in question. The syntactic structure, i.e. the words and the 
grammatical relations of the utterance, encode
which concepts and categories should be used and how the instructions work together. 
For instance, the fact that the hearer's perspective on the scene should be taken
is conveyed by the phrase \textit{von ... aus} (`from ... your perspective').

% cross-linguistic variation
Languages vary widely in how they solve the problem of spatial reference --
including both how they conceptualize space and how they
talk about it \citep{levinson2006grammars,levinson2003space}\oldindex{Levinson, S. C.}\oldindex{Wilkins, D.}. 
Spatial position of objects can be expressed using 
a variety of syntactic means including case, 
adpositions, particles, and verbs. But, maybe more importantly, 
there is a breathtaking variety in how people conceptualize space,
which spatial relations they know, what counts as a landmark, how
perspective is used, etc. Just to give a few simple examples, Spanish has
three basic proximal distinctions, while German has two.
In Barcelona people make active use of the topology of
the surrounding landscape, referring regularly to the
seaside and mountainside when giving navigation instructions. 
In other languages `uphill' and `downhill' 
are used to refer to proximal objects \citep{levinson2003space}\oldindex{Levinson, S. C.}.

%% highly specialized tool 
These examples show that spatial language is a highly developed tool for 
establishing reference in a spatial environment. How did spatial language 
become this way? There is an emerging view now that the most plausible answer 
to this question is that spatial language is a \textsc{complex adaptive system}\is{complex adaptive system}
(see \citealt{steels2000language}\oldindex{Steels, L.} for the general idea of language as a complex adaptive system), 
that is constructed and changed by its users for the same purpose it is used for today, 
namely to describe spatial scenes,
establish reference to objects in the environment, give instructions for navigation, etc.
This process is, of course, not the same process of construction that a group
of engineers use when they are building a bridge. In such classic engineering
problems, a team of people with a more or less complete view of the problem
designs a top-down solution. By contrast, nobody has a global view on the state 
of a language. Rather, language lives in the individuals of 
the language community. Every individual has 
its own views on the state of the language, i.e. what words and 
grammatical relations are available.

% complexity -> look to biology
When we combine the evidence from the complexity of particular spatial languages,
such as German locative phrases, and the variation that can be seen across 
languages, it seems reasonable to consider results from a science 
that routinely deals with complexity and variation -- biology. Biological species are highly 
complex solutions to particular environmental and social challenges. The solutions
found by each species exhibit a high degree of variation. This simple observation has
forced biology to come up with precise models and predictions to explain the 
origins of species. It comes as no surprise, then, that theories of language, particularly 
language evolution and language change, have adopted concepts 
from biology related to variation, complexity and the emergence of order 
in biological systems.  

% selectionist theory of language evolution
This book defends the \textsc{selectionist theory of language evolution},
which exploits biological concepts to explain how language is shaped 
by the communicative needs and environmental conditions that a 
community or population faces. 
The theory hypothesizes that agents create variation within their language 
and select working solutions based on how successful they are 
in communication (\textsc{communicative success}), 
how complex they are in processing (\textsc{cognitive effort}) and 
other factors. 

% whole systems approach
Studying language change from the perspective of communicative
intentions requires a great deal of insight into how humans or artificial 
systems can realize their specific communicative intentions 
in social interactions in the physical world. Such holistic explanations 
necessitate a \textsc{whole systems approach}\is{whole systems approach}
\citep{steels2001language}\oldindex{Steels, L.},
in which great care is taken to ensure that perception, conceptualization 
and linguistic processing systems are integrated to an extent that 
interaction between agents is possible. Only when all of this machinery is 
in place can one attempt to examine questions of language change.

% operational models
In particular, a whole systems approach requires an operational theory of 
language. How are utterances processed? How is space conceptualized?
How is linguistic knowledge represented? How does language
interact with the perception of the physical reality? A whole systems approach 
requires concrete answers to each of these important questions. The resulting 
burden placed on operational models is of course far greater than for high-level 
explanations or logical reasoning about these processes. But concrete, 
mechanistic accounts allow much greater insights into the phenomena studied.
In the best case, a successful model of language evolution in a whole systems 
approach validates many aspects of the theory of language and language 
change at the same time.

% contributions
This book contributes to the understanding of spatial language in two ways. 
First, it provides a detailed operational reconstruction of German locative phrases
using a whole systems approach. Second, it explores the evolution of spatial
language within the same computational framework. The two parts  
together argue for (1) the validity of the approach to language, and (2) the validity
and explanatory power of the selectionist theory of language evolution.


%%%%%%%%%%%%%%%%%%%%%%%%%%%%%%%%%%%%
\section{Locative spatial language}
\label{s:intro-spatial-language}
If one wants to make an interesting claim about how language evolves, one needs a
solid idea what language actually is, how linguistic knowledge is represented, and
how to organize linguistic processing. These questions are best answered
by reconstructing a complex natural language phenomenon such as German locative 
phrases. Such phrases are used for establishing reference to 
static objects and identifying them by denoting their spatial position \citep{miller1976language}\oldindex{Johnson-Laird, P. N.}\oldindex{Miller, G. A.}.
They can be distinguished from other parts of spatial language
that are dealing with motion or navigation \citep{eschenbach2004functional}\oldindex{Eschenbach, C.}. 

German locative phrases can be analyzed in terms of components 
or systems which together form a locative phrase. \REF{e:der-block-rechts-der-kiste-von-dir-aus} consists of three parts: 
a \textsc{spatial relation}, which is combined with a \textsc{landmark} and 
a \textsc{perspective}.

\begin{description}
\item[Spatial Relations] The defining quality of locative spatial phrases
are that they contain locative spatial relations such as 
\textit{rechts} (`right'), \textit{vorne} (`front'), \textit{nah} (`near'), 
\textit{n\"ordlich} (`north on' and so forth).
These relations are called locative because they encode
static spatial relationships and do not refer to change of position in time. 
In \REF{e:der-block-rechts-der-kiste-von-dir-aus}, \textit{rechts} (`right') is the locative spatial relation.
In this book we study three classes of spatial relations. 
\emph{Proximal} relations are based on distance estimations. 
Examples of proximal relations in German are \textit{nah} (`near') and \textit{fern} (`far').
The second class is called \emph{projective}
relations and includes direction-based spatial relations such as \textit{links} (`left') and \textit{vor} (`front').
The last class considered are \emph{absolute} relations such as \textit{n\"ordlich}
(`north') and \textit{\"ostlich} (`east'). These are also direction-based, but the direction
is related to a geocentric reference system such as the magnetic poles of the
earth.
\item[Landmarks] A spatial relation is at least a binary and always
relates to something. This something is typically called \textsc{landmark}.
In \REF{e:der-block-rechts-der-kiste-von-dir-aus}, the landmark 
is expressed in the determined noun phrase 
\textit{der Kiste} (`the box') immediately following the spatial relation.
\item[Perspective] For certain spatial relations perspective is important.
\REF{e:der-block-rechts-der-kiste-von-dir-aus} features a 
perspective that is marked via the phrase \textit{von ... aus} (`from ... viewpoint').
The marker expresses that the viewpoint on the scene is the hearer. 
\end{description}

%%%%%%%%%%%%%%%%%%%%%%%%%%%%%%%%%%%%
\section{A theory of language evolution}
\label{s:intro-evolutionary-linguistics}
Theories of language evolution have to explain the evolution of language
by defining the role and contribution of four different factors on language:
biology, cognition, social cognition, and culture 
\citep{steels2009cognition,steels2011self-organization}. 

\begin{description}
\item[Biology] To study language evolution from the biological
perspective is to ask questions about the relationship of biology,
in particular genetics and ecology, with linguistic behavior.
The question can be roughly split into two parts.
First, what is the biological influence on the general capacity for language
in the human population? Second, one can ask for the influence of biology on 
the particular language spoken by individuals. The first is a general question for 
the processing capabilities that need to be present for language.
This includes that humans require sufficient memory and powerful neural 
circuitry for processing language, but also production organs for speech 
and auditory capacities. The second question is how much the biological basis 
determines the particular language individuals speak. 
In other words, how much the lexicon and/or the grammar
of a language are influenced by genetic conditions. 

\item[Cognition] Biology has provided us with neural 
circuitry that enables distinct cognitive capabilities.
The cognitive perspective on language asks: what are the basic cognitive 
processing mechanisms underlying production and parsing of language, 
interpretation, conceptualization, but also categorization, perception
etc.? Language depends on a number of capabilities that may or may 
not be prior to language, such as temporal clustering 
of events, spatial navigation, perception-action systems 
\citep{rizzolatti1998language,arbib2002mirror,steels2012mirror,steels2008mirror}\oldindex{Arbib, M. A.}\oldindex{Rizzolatti, G.},
memory and so on and so forth. For instance, some have linked the evolution 
of language to an increase in capacity for storing cognitive categories and 
their interrelations \citep{schoenemann1999syntax}\oldindex{Schoenemann, P.}. Another 
strand of cognitive influences on language evolution are general 
cognitive operators such as analogy and
learning operators, for instance sequential learning 
\citep{christiansen2001sequential}\oldindex{Dale, R.}\oldindex{Christiansen, M. H.}\oldindex{Ellefson, M. R.}\oldindex{Conway, C. M.}.

\item[Social Cognition] Inevitably, language is a social phenomenon that
occurs when humans interact. Social cognition researchers, for instance, 
are interested in the social mechanisms that are needed for children to acquire 
language, but also in the social mechanisms that are prerequisite for the emergence of language.
Proposals include things such as ``theory of mind'' \citep{dunbar1998theory}\oldindex{Dunbar, R.} which
is the capacity to understand another individual's state of mind, 
``joint attention''\is{joint attention} \citep{carpenter1998social}\oldindex{Tomasello, M.}\oldindex{Nagell, K.}\oldindex{Carpenter, M.}\oldindex{Butterworth, G.}\oldindex{Moore, C.}
which is the ability to track interlocutor gaze and mutual 
attentiveness to the same object, ``social learning skills''\is{social learning skills} such as imitation learning 
\citep{tomasello1992social}\oldindex{Tomasello, M.} and the ability and the urge to ``share intentions''\is{shared intentions} 
\citep{tomasello2005understanding}\oldindex{Call, J.}\oldindex{Carpenter, M.}\oldindex{Moll, H.}\oldindex{Tomasello, M.}\oldindex{Behne, T.}. Many of these mechanisms are deeply rooted in biology.
For instance, \cite{dunbar2003social}\oldindex{Dunbar, R.} and \cite{worden1998evolution}\oldindex{Worden, R.} argue that
theory of mind is a necessary preadaptation for language and that it has 
evolved via natural selection\is{selection}.

\item[Culture] Language is a cultural phenomenon that is undergoing 
steady change on the cultural level. New words, speech sounds, morphemes,
semantic and syntactic structures arise all the time in language \citep{steels2011self-organization}.
This manifests in the incredible amount of cross-cultural variation
on all levels of linguistic processing \citep{evans2009universals}\oldindex{Levinson, S. C.}\oldindex{Evans, N.}, for example, phonemes 
\citep{maddieson1984patterns,oudeyer2005self}\oldindex{Oudeyer, P. Y.}\oldindex{Maddieson, I.}, spatial semantics \citep{levinson2003space}\oldindex{Levinson, S. C.},
and syntax \citep{levinson2006grammars}\oldindex{Levinson, S. C.}\oldindex{Wilkins, D.}. 
This evidence points to strong cultural negotiation processes in which continuous invention 
is channeled to produce complex useful communication systems.
Many of such processes orchestrating change and diversification have been identified.
Grammaticalization, for instance, tries to explain the shift from
lexical items to grammatical items \citep{hopper2003grammaticalization}\oldindex{Hopper, P. J.}\oldindex{Traugott, E. C.}.
Others have pointed to generational change as the trigger for 
development in language \citep{smith2003iterated}\oldindex{Brighton, H.}\oldindex{Smith, K.}\oldindex{Kirby, S.}.
The question from the perspective of cultural evolution is what are the mechanisms 
that bring about change in language and what are the principles with
which agents conventionalize language up to the point that interlocutors 
have a chance of understanding each other.
\end{description}


I emphasize the cultural point of view in this book. That is, my primary concern is
with change in language on the cultural level independent of changes in the human 
biology. Language change occurs on a smaller time scale than, for instance, the 
adaptation of a new biological organ, let alone a new species. There is absolutely no 
doubt that languages evolve fast. One just has to look through a text by 
Shakespeare or Goethe to see that a few hundred years can have impact
on vocabulary and grammatical structure. It took Vulgar Latin a mere 1500 years to 
evolve into about a dozen different languages such as French, Italian, 
Portuguese or Catalan (e.g., see \citealt{pope1952latin}\oldindex{Pope, M. K.} for French). 
If we observe languages today, we can easily see 
that new words are invented all the time.
In academic and technological contexts, for instance, new concepts arise all the time.
Roughly 30 years ago vocabulary such as \textit{email} or \textit{website} did not even exist.
What drives change in language, in what circumstances does it take place and 
what are necessary requirements for language change to occur? These are questions 
that cultural evolution theories of language have to address.

\subsection{Language systems and language strategies}
Cultural theories of language evolution have to take a close look at individual 
trajectories of language change \citep{steels2011self-organization}. For instance, 
how did the Russian aspectual system emerge or why does English have a
system of determiners and Russian not? How do spatial language systems 
develop over time? In other words, cultural theories of language
evolution must provide models for the emergence and evolution of concrete
\textsc{language systems} \citep{steels2011self-organization}. \is{language system}\is{language strategy}
Language systems package a particular 
\textsc{semantic system} (e.g. a set of spatial categories) and a 
particular way of expressing these distinctions (e.g. a corresponding set of 
lexical items). The absolute German system, for instance, consists of 
four absolute spatial categories and the corresponding strings, e.g. 
\textit{n\"ordlich} (`north'), \textit{s\"udlich} (`south'). These spatial
categories are the basic building blocks of absolute spatial conceptualization 
in German. They can be compositionally combined with landmarks to
build complex spatial phrases. Interestingly, the 
German locative systems effectively consist of different conceptualization
strategies that have distinct but converging evolutionary trajectories. For instance,
the absolute system is connected to the invention of the compass,
whereas projective systems often at least in part can be traced back to 
body parts \citep{traugott1991grammaticalization}\oldindex{Heine, B.}\oldindex{Traugott, E. C.}.
Nevertheless, many locative spatial relations are used in the same syntactic 
context.

Spatial language systems such as the proximal or projective system 
are characterized by a degree of cohesion and systematicity 
that points to an underlying principle that organizes acquisition, emergence 
and coordination. We call the mechanisms organizing a particular language
system the \textsc{language strategy} \citep{steels2011self-organization}.
Language strategies have a \emph{linguistic} and a \emph{conceptual} 
part. For example, on the conceptual 
side absolute spatial categories share that they
are part of the same conceptualization strategy which uses absolute directions
to the magnetic poles of the earth. Syntactically all spatial relations share
that they are expressed in a similar way namely lexically and that they can
be expressed as adjective, adverb and preposition. 


\subsection{Selectionist theory of language evolution}
In this book I follow the \emph{selectionist} theory of language evolution
\citep{steels2011self-organization}, which applies the dominant theoretical construct 
in biology \textsc{natural selection} and uses it to explain language change\is{selection}
on the level of language systems and language strategies.
Additionally, the concepts of \textsc{self-organization}\is{self-organization},
\textsc{recruitment}\is{recruitment} and \textsc{co-evolution}\is{co-evolution} of syntax and semantics are
used as theoretical pillars.

\begin{figure}
\begin{center}
\includegraphics[width=0.7\columnwidth]{figs/select-strat}
\end{center}
\caption[Selective pressures on language systems and strategies]{The fitness of utterances for communication affects both the language
system and the language strategy. The effect of the success of a single utterance 
on the language strategy is smaller which leads to slower change on the level 
of the strategy. (Figure adapted from \citealt{steels2011self-organization})}
\label{f:strategy-system-selection}
\end{figure}

\begin{description}
\item[Selection]\is{selection}
Selectionism rests on two principles: \emph{generation} of
possible variants and \emph{selection} of variation based on fitness. 
The most important factor in determining the fitness of a 
particular language strategy, but also
of a particular language system, is communicative success.
A communicative interaction between two interlocutors is successful if the
communicative intention of the speaker is reached. For instance, if the speaker
wanted to draw attention to some object, the communication is successful if the
hearer pays attention to that object. Communicative success drives selection\is{selection}
on the levels of the language system, but also on the level of language
strategies (see \figref{f:strategy-system-selection}). 

Variation occurs in the systems for two reasons. 
First, agents are actively trying to solve 
problems in communication \citep{steels2000language}\oldindex{Steels, L.}. 
Agents introduce new categories, 
new words and grammar when they detect problems that they 
cannot solve using the current
language they know. Second, language is an inferential 
communication system \citep{sperber1986relevance}\oldindex{Wilson, D.}\oldindex{Sperber, D.} which means 
that the information provided in an utterance is often 
incomplete and ambiguous. Interpreting phrases is an active 
process in which the hearer
is fusing information from the context, from the dialogue 
and his knowledge about the language 
to arrive at the best possible interpretation. In this process 
of course hearers might interpret
the utterance differently then intended. This is the 
second source of variation.

\item[Self-organization]
\cite{steels2011self-organization}  assumes that selection\is{selection} is not enough to explain 
language change and proposes another driving force in the evolution of language: 
self-organization\is{self-organization} -- a concept used to account for complex 
phenomena in physical and biological systems. In short, self-organization is a way to explain 
how global structure arises out of local interaction of subunits \citep{camazine2003self}\oldindex{Camazine, S.}\oldindex{Franks, N. R.}\oldindex{Bonabeau, E.}\oldindex{Sneyd, J.}\oldindex{Deneubourg, J. L.}\oldindex{Theraula, G.}. 
An example from biology for self-organization is swarm behavior in a school of fish. 
Each individual fish locally controls its behavior based on the estimation of the 
position and direction of its immediate neighbors. On the global level this leads to 
consistent swarm behavior. Self-organization is typically seen as a complementary 
mechanism to selection\is{selection}, although there is some discussion 
on how to reconcile the two mechanisms. 
\cite{kauffman1993origins}\oldindex{Kauffman, S. A.}, for instance, proposes the following idea. 
Local components and the interaction rules are determined by selection\is{selection}, whereas the global emergent behavior is explained using self-organization. 
Applied to the swarm behavior this means that the anatomy of fish as 
well as the perceptual feedback loop are a product of natural selection. 
The global emergent swarm behavior is the product of self-organization. \is{selection}

Similar to swarm behavior, agents in a population evolving a language have
to achieve global coherence in the language they use. Each agent has its own 
private representations of the language that they speak and they can 
adjust their own representations based on local interactions with peers. 
How, from local interactions, agents can agree
on a globally shared communication system is the problem of alignment\is{alignment}.
Psychologists have found that interlocutors align on all levels of linguistic processing
even over the course of a few interactions, i.e., dialogue
\citep{garrod1994conversation,pickering2004toward}\oldindex{Garrod, S.}\oldindex{Doherty, G.}\oldindex{Pickering, M. J.}.
Similar mechanisms applied over a long time span are required for driving populations 
to self-organize a sufficiently shared communication system 
\citep{steels2002bootstrapping}\oldindex{Steels, L.}\oldindex{Kaplan, F.}. 

\item[Recruitment]\is{recruitment}
The last problem for an account of how languages change in the 
selectionist theory of language evolution is the problem of language strategy
generation. The hypothesis is that language strategies are recruited by assembling
basic cognitive operations\is{cognitive operation} \citep{steels2007recruitment}\oldindex{Steels, L.}. For instance, an 
absolute spatial conceptualization strategy\is{conceptualization strategy}
involving distinctions such as ``north'' and ``south''  consists of basic categorization
mechanisms and the ability to track ones own direction. The two abilities are assembled
into the strategy which encompasses the different absolute spatial distinctions.
The process is called \textsc{recruitment}\is{recruitment} because the cognitive mechanisms which
are assembled could, in principle, have evolved or could be learned independently 
from language. 

\item[Co-evolution] \is{co-evolution}
One of the tenants of the theory of linguistic selection is that syntax and semantics co-evolve. 
The idea is that recruitment\is{recruitment} of conceptualization strategies and the invention of new
semantic distinctions and spatial relations trigger evolution of the syntax of a language
\citep{steels1997distinctions,steels1998synthesizing}\oldindex{Steels, L.}.  
For instance, presumably when the absolute system in German emerged based
on a new way of construing reality, this at the same time triggered 
the invention of new words. 
\end{description}

\subsection{Evolutionary explanations}
In every science one has to define what counts as an explanation. 
This book is guided by what counts as an evolutionary explanation
in biology, ethology and psychology \citep{tinbergen1963aims,dunbar1998theory}\oldindex{Tinbergen, N.}\oldindex{Dunbar, R.}. 
In order to explain a complex trait from the evolutionary perspective 
one has to provide explanations on four different levels: \textsc{function}, 
\textsc{mechanism}, \textsc{ontogeny} and \textsc{phylogeny}.

\is{function}\is{ontogeny}\is{phyologeny}\is{mechanism}
\begin{description} 
\item[Function] An explanation for a particular behavior has to show what
the behavior is good for, i.e. what is its purpose. For Darwinian biology, 
the function of a behavior has to be explained in terms of its impact 
on survival or, more precisely, on the production of offspring. For evolutionary
linguistics this turns into the question of how a particular language system or a particular strategy
helps an agent to be more successful in communication. For example,
one can explain particular spatial language systems with respect to their ability
to help agents solve communicative problems in spatial navigation and 
spatial reference.

\item[Mechanism] Besides function, one has to identify the mechanisms that give rise 
to the behavior. 
This is actually called ``causation'' by \cite{tinbergen1963aims}\oldindex{Tinbergen, N.} 
and it refers to the cause and effect relations that generate a particular behavior. 
For instance, one can explain how aggressive behavior is generated by looking
at changes in hormone levels in an organism, e.g., testosteron causes aggressive behavior.
For spatial language this entails a detailed operational model of the 
production and parsing of spatial language.

\item[Ontogeny] The next question is how a particular behavior is acquired. To
answer this question one has to identify the developmental steps that
the behavior undergoes, but also what is the ontogenetic basis 
of the behavior. What is learned and what is instinct? For spatial language this 
requires insights into how spatial language is learned.

\item[Phylogeny] A fourth part of every evolutionary explanation has to
identify the evolutionary history of a behavior. What are
the sequential stages of evolution of a behavior? What are the prerequisites
of a behavior? How do evolutionary older behaviors influence 
the behavior under question? These questions have to be answered
with respect to the function of the behavior. In other words, one needs
explanations of how the behavior evolved to fulfill its current function. 
For language evolution scholars have to identify how a particular strategy 
evolved over time. Was it adapted from an older strategy? 
How did syntax and semantics of the language system under consideration co-evolve over time?
\end{description}

\section{Main hypothesis}
\label{s:intro-main-hypothesis}
This book provides experimental evidence for the theory of linguistic
evolution. The hypothesis is that \emph{spatial language syntax and 
spatial semantics co-evolve through a cultural process based on selection\is{selection}, 
self-organization and recruitment}\is{self-organization}\is{recruitment}. This book explores the hypothesis for 
the different components of spatial language: spatial relations, landmarks 
and perspective. Computational experiments
show the emergence of spatial relations, the negotiation of the use of landmarks
and perspective. I also explore different strategies for expressing
spatial conceptualizations: lexical and grammatical strategies.

\section{Contributions}
\label{s:intro-objectives}
This book provides detailed accounts of 
the function, mechanisms, ontogeny and phylogeny of spatial language. 
\begin{enumerate}
\item The first contribution is an explanation of the mechanisms behind spatial language
for German locative phrases in a complete reconstruction including
perception, semantic and syntactic processing. Once the mechanisms are in place,
we test the function and impact of components of spatial language in experiments
by removing the component in question and examining the effect the removal has
on communication.
\item The second contribution is to explain steps in the co-evolution\is{co-evolution} of spatial 
syntax and spatial semantics through computational models.
\end{enumerate}

This book follows a whole systems approach which allows us to define external 
criteria for the progress in each of the objectives. The defining moment for the 
underlying conception of language is communication. A communication system 
is \textsc{successful} if it allows robotic agents to achieve 
their communicative goals such as drawing the attention to an object in the environment. 

% steps in the evolution of spatial language
\subsection{Evolutionary stages}
\label{s:stages}
One way of understanding evolutionary processes is to try to identify
evolutionary stages. Over the years, different steps in the evolution
of language have been identified involving varying degrees of 
specificity \citep{bickerton1999language,jackendoff1999possible,steels2005emergence}\oldindex{Steels, L.}\oldindex{Jackendoff, R.}\oldindex{Bickerton, D.}. 
All of these proposals, while differing in detail and the exact number of stages,
agree that language evolution starts at some pre-grammatical stage and 
increases in complexity to the form of language, in particular, grammar 
that we see today. Obviously any evolutionary account of language has to show how 
the current state of complexity of language can be traced back to earlier simple stages. 

This book orients itself alongside \cite{steels2005emergence}\oldindex{Steels, L.}, who proposes
a number of stages of complexity which are relevant for this book:
\textsc{single-word utterances}, \textsc{multi-word utterances}, and \textsc{grammatical utterances}
(see \figref{f:co-evolution-complexity}). 

\begin{figure}
	\includegraphics[width=.7\columnwidth]{figs/co-evolution-complexity}
	\caption[Co-evolution of syntactic and semantic complexity.]{Co-evolution\is{co-evolution} of syntactic and semantic complexity.}
	\label{f:co-evolution-complexity}
\end{figure}
 
\begin{description}
\item[Single-word utterances] In this stage agents utter single words that pertain to a particular concept
or category used for discriminating objects. Examples for spatial language include utterances that 
directly refer to spatial regions such as \textit{links} (`left') or \textit{n\"ordlich} (`north').
When agents can only express themselves using a single word, this single word necessarily encodes
the complete conceptualization strategy. For instance, which landmark or perspective is used for conceptualization is holistically coded in the single word. Since there is no additional information 
about which conceptualization strategy the term is referring to, agents have to implicitly agree on 
the precise spatial construal the term is referring to. This limits the re-use of spatial categories 
in different spatial conceptualization strategies because agents have no way of 
disambiguating the use of the same spatial relation in different strategies.
\item[Multi-word utterances] Single-word communication systems are not very flexible. 
There is no compositionality and particularly there is no re-use. In German, for instance,
projective relations can be used with different landmarks. In the multi-word utterance stage 
agents can express different constituents by using a number of lexical items. 
Besides expressing the spatial category used, agents can also 
mark landmarks. An example utterance is \textit{links Kiste} (`left box') which is used 
to signal that the region left of the box is meant.
\item[Grammatical utterances] When we look at natural language, we can see 
that the \emph{same} constituents can be part of different conceptualization strategies. 
Imagine an utterance like \textit{Kiste link} (`box left') without the grammatical information, in particular, 
without word order and lexical class information. In that case a hearer does not know whether 
\textit{link} is an adjective or an adverb. This syntactic underdetermination has consequences for 
the semantic interpretation. If the phrase is interpreted as an adjective noun phrase as in 
\textit{linke Kiste} (`left box'), the spatial category acts as a modifier on the set of boxes. If the 
spatial relation is interpreted as an adverb, then box might be a landmark and the whole phrase 
denotes a region next to the landmark as in \textit{links der Kiste} (`to the left of the box').
Grammar signals the difference in these two semantic interpretations and
disambiguates the conceptualization strategies. Consequently, agents equipped with grammatical
strategies can disambiguate even more strategies and consequently, they can be more expressive.
\end{description}

The goal of this book is to identify, implement and test the mechanisms that drive 
the evolution of language on \emph{each} of these stages. The mechanisms 
we are interested in are not descriptions of the phenomena but mechanistic explanations 
which identify the computational and cognitive components
that enable robotic agents to self-organize communication systems.
The procedure to find and validate mechanistic explanations is to
\begin{enumerate} 
\item hypothesize \textsc{invention}, \textsc{adoption} and \textsc{alignment}\is{alignment} \textsc{operators} for the syntax and semantics
according to each stage of complexity,
\item equip agents with these operators,
\item test the evolutionary dynamics in populations of equipped agents,
\item measure the communicative success, adaptivity and expressivity.
\end{enumerate}
\is{invention}\is{adoption}
Invention, adoption and alignment\is{alignment} operators are the backbone of the evolutionary models 
of this book. For instantiating the theory of linguistic selection\is{selection} one has to identify 
agent-level mechanisms that orchestrate the global behavior of the population. The mechanisms 
can be classified into the following three classes.
\begin{description}
\item[Invention operators] Invention is the process of introducing variation 
into the system by inventing a new spatial relation or a word or even grammar
in order to solve a problem in communication. A speaker, for instance,
who is unable to discriminate an object might introduce a new spatial
category to be able to identify the object. Subsequently, he might invent
a new word to be able to express the new spatial category.
Invention operators introduce variation and novelty into the system.
\item[Adoption operators] Adoption is the process by which an
agent acquires a new word, a new spatial relation or a new piece of grammar.
Acquisition is carried out by hearers in interactions when they observe
new items that they are unable to process.
Adoption is another source of novelty and variation. An agent that
picks up a new word might have a different idea of what that word
means than the speaker actually intended. 
\item[Alignment operators] Invention is local to an interaction. When two
agents communicate and one of them invents a new word, this word
might be acquired by the interlocutor, but the knowledge about this 
word is still local. Alignment\is{alignment} operators orchestrate the self-organization\is{self-organization}
of the system and the global alignment of language.
\end{description}


\subsection{Co-evolution of syntactic and semantic complexity}
In each stage, syntactic complexity co-evolves\is{co-evolution}
with semantic complexity (see Figure \ref{f:co-evolution-complexity}). 
Syntactic complexity rises because the number of words
per utterance increases (from the single-word stage to the multi-word stage) 
and because syntactic categorizations such as word order, morphology, 
agreement become important (from multi-word to grammar).

The notion of semantic complexity is harder to define. Obviously
German spatial language is complex. But why does this seem obvious? 
What are the properties that make it a complex semantic system? For this book, 
complex semantics is defined with respect to spatial language as: the language supports a large 
number of conceptualizations of a spatial scene. There are two factors influencing 
the complexity of the space of possible conceptualizations of a spatial scene.

\begin{description}
\item[Number of relations] A first level of semantic complexity
is related to the number of spatial categories. For the part
of German locative phrases considered in this book, we already have 
12 spatial relations. But there are, of course, many more relations not considered
in this book such as  dynamic relations. 
For some scholars this is the only definition of semantic complexity (compare 
\citealt{schoenemann1999syntax}\oldindex{Schoenemann, P.}).
\item[Number of conceptualization strategies]
A second notion of semantic complexity is the number of
conceptualization strategies a language supports. German, for instance, supports
many different categorization systems: projective, e.g. \textit{links} (`left') or \textit{rechts} (`right'), 
proximal, e.g. \textit{nah} (`near') and \textit{fern} (`far'), and absolute, e.g. \textit{n\"ordlich} (`north') 
and \textit{s\"udlich} (`south'). This is one aspect. The other aspect is that
these systems are part of different conceptualization strategies. 
Examples of this re-use were already given earlier with respect to adjectives and adverbs.
\end{description}

\section{Structure of the book}
\label{s:intro-structure}
This book is structured into three main parts besides this introduction 
and the conclusion. Part \ref{p:embodied-language-games} explains the interaction
model and the technical systems needed for studying spatial language.
Part \ref{p:german-locative-phrases} deals with objective number one and details the reconstruction
efforts for the German locative system. In Part \ref{p:spatial-language-evolution}, I detail how spatial
language evolves based on the model of evolutionary stages.


\subsection{Part I: Spatial language games and technical background}
\is{spatial language game}
\subsubsection{Spatial language games}
Spatial language occurs mainly in interactions of individuals in spatial scenes.
To research spatial language in such a communication-based approach
to language a number of things need to be in place. We need a model of 
interactions in spatial scenes. This is the topic of Chapter~\ref{s:spatial-language-games} which 
introduces spatial language games which are routinized interactions
consisting of defined roles for interlocutors -- speaker and hearer. 
The chapter explains the basic interaction scheme and the 
linguistic and non-linguistic behaviors that define a spatial language
game. 

\subsubsection{Embodied cognitive semantics with IRL}
In order to achieve the objectives of this book, we need computational 
formalisms that support the reconstruction and evolution investigations. 
One of such formalisms in part developed for this book is the 
Incremental Recruitment Language (IRL). IRL is (a) a formalism
for representing semantics, (b) a set of planning algorithms for automatic
conceptualization and interpretation, and (c) a set of tools that make
semantics an open-ended adaptive system. \chapref{s:irl} 
introduces the formalism and the technology behind it.

\subsubsection{Construction grammar with FCG}
Another important backbone of the investigations is Fluid Construction 
Grammar (FCG). FCG is a formalism for representing and processing linguistic 
knowledge. \chapref{s:fcg} details how mappings from semantics to syntax 
are implemented using FCG and gives an example of processing a simple phrase.\is{Fluid Construction Grammar}\enlargethispage{2\baselineskip}

\subsection{Part II: Reconstructing German locative phrases}
To ground the modeling efforts in sufficient knowledge of a real spatial language system, 
I decided to reconstruct a part of German spatial language -- German locative 
phrases. The second part of this book reconstructs the syntax and semantics 
of German locative phrases. The part starts out with an in-depth look at
German locative spatial language as a natural language phenomenon. Chapter
\ref{s:german-spatial-language-introduction} gives more examples of the syntactic variety and the connection to the
space of conceptualization strategies supported in German locative phrases.
This sets the scope for the reconstruction effort, but also identifies a number
of processing issues that the reconstruction has to deal with in order to be successful.

\subsubsection{Spatial semantics}
The following chapter details the operationalization of spatial semantics. 
\chapref{s:german-space-semantics} the basic semantic building blocks of German locative phrases
and discusses how they work together to make up the complex 
semantics of spatial scenes.  

\subsubsection{Syntactic processing}
A close look at German locative phrases reveals a number of interesting phenomena.
Most importantly it uncovers the tight relationship between spatial 
syntax and spatial semantics. \chapref{s:german-locative-phrases-syntax} 
explains how FCG can be used 
to model the tight connection between the words and grammatical 
relations observed in German locative phrases and the world of spatial semantics.
These mappings are interesting because they pose particular challenges to
the organization of linguistic processing. The re-use of the same spatial categories
in different strategies for conceptualizing reality and their syntactic expression
requires sophisticated mechanisms for dealing with many-to-many mappings 
in language. Another important issue is how to deal with the case
system of German. All of these aspects of linguistic processing are
discussed in \chapref{s:german-locative-phrases-syntax}.

\subsubsection{Conceptualization of spatial scenes}
Spatial scenes do not come a priori labeled, categorized and construed.
Agents have to autonomously conceptualize reality given the
particular communicative goal they have. 
\chapref{s:german-locative-phrases-semantic-processing} deals with the problem
of conceptualization which is the problem of how to construct semantic
structure that is helpful in reaching communicative intentions. The chapter
gives an overview of different factors influencing the conceptualization
of spatial scenes and compares different implementations of 
spatial conceptualization.

\subsubsection{Integrating syntactic and semantic processing}
The last chapter of this part reports on the integration of syntax, semantics 
and conceptualization. One of the issues that can be studied in an approach
like mine is \textsc{semantic ambiguity}\is{semantic ambiguity} which refers to the fact that natural
language is often ambiguous with respect to the precise interpretation
of a phrase. But humans are very strong in communicating even though 
language only encodes hints at how to conceptualize reality. The key
is that humans integrate the sparse information communicated in utterances 
with knowledge about the current context of the interaction. 
\chapref{s:german-locative-phrases-syntax-semantics-integration}
explains how one can operationalize this process of disambiguation through
the context using the conglomerate of systems for linguistic and semantic
processing as well as perception.



\subsection{Part III: Spatial language evolution}
Finally the book turns to evolution in the third part. The organization of this part
orients itself along the stages of complexity introduced earlier. There are two
parts on single-word utterance systems, followed by a chapter on multi-word utterance
systems. The part closes with a chapter on the evolution of grammatical structure.

\subsubsection{Acquisition and formation of basic spatial category systems} 
The first chapter in this part explains how the basic building blocks of 
spatial language -- spatial relationships and corresponding words --  
become shared in populations of agents.
This corresponds to complexity stage one -- single words. The goal of the chapter is 
to define the language strategies necessary for forming single-word spatial language systems. 

Single-word spatial language systems are built by a particular strategy of 
conceptualizing reality which includes a priori commitments to certain reference 
objects, frames of reference and perspectives on the scene. The chapter shows how a 
language strategy which is a combination of a particular strategy for conceptualizing reality 
plus the necessary invention operators for basic spatial categories build the language systems 
that allow agents to communicate successfully. Language strategies are tested in two 
scenarios -- \textsc{acquisition} and \textsc{formation}. In acquisition a learner agent has to pick 
up the spatial language system spoken by a tutor. In formation all agents start from scratch 
and progressively develop categories and lexical items.

The most important influence on what kind of language system emerges is the 
language strategy. The chapter details different language strategies 
necessary for building proximal, projective and absolute systems which encompass
dedicated invention, adoption and alignment operators as well as the different 
conceptualization strategies. The success of the learning operators and the 
conceptualization strategy is tested in experiments where populations are fitted with a 
particular strategy. The resulting languages spoken by individual agents are analyzed with 
respect to communicative success and how similar they are to each other. 

Another important factor influencing the emerging 
language system are environmental conditions. The chapter studies the impact of 
environmental conditions systematically by manipulating environmental features 
such as global landmarks or the statistical distribution of objects. 

Obviously, natural languages support many conceptualization strategies at the same time.
German, for instance, simultaneously has a proximal, a projective and an absolute system. 
So one can ask what happens when agents are simultaneously operating 
different strategies. I hypothesize that agents need additional cognitive mechanisms 
for choosing between different strategies and that choosing a strategy can be realized
using the discriminative power of each strategy in a particular context. 
When an agent has to invent a new category they use the strategy that is most
discriminating using a new category. Experiments show that this principle
allows agents to build multiple language systems at the same time.
Lastly, the chapter also studies the impact of different environmental 
layouts on formation of language systems for interacting strategies.

\subsubsection{Origins and alignment of spatial conceptualization strategies}
\chapref{s:strategies} deals with the emergence and 
alignment of conceptualization strategies.
When one compares different languages of the world it becomes clear that many languages 
differ in the kinds of conceptualization strategies they support. Some languages 
solely use an absolute system, others can use intrinsic and relative systems and so on
and so forth. Consequently, the evolution of spatial language is intricately connected to the 
origins and evolution of spatial conceptualization strategies. The chapter shows that
conceptualization strategies are organized in a process of recruitment, selection\is{selection} and 
self-organization\is{self-organization}\is{recruitment}. 

% competition
To explain conceptualization strategies from the viewpoint of the theory of linguistic
selection\is{selection} is to explain (a) how different conceptualization strategies are created
and (b) how they are selected for in communication. Competition is an important aspect
of selection. Obviously environmental conditions and communicative success 
are main influences on which strategies are selected for because they are more successful. 
The chapter proposes alignment operations that update and track the score
of conceptualization strategies so that agents can locally align in their
interactions. I show that these operators lead to global convergence
of the population on using single conceptualization strategies.
The chapter studies competition of different strategies for landmarks and 
frames of reference and shows that with the right alignment strategy
agents can agree on using a particular conceptualization strategy while
co-evolving a lexicon and ontology of spatial relations at the same time.

Besides selection\is{selection} the theory has to explain how conceptualization
strategies are created. This is were the idea of recruitment\is{recruitment} comes into play.
Conceptualization strategies are assemblies of cognitive operations\is{cognitive operation}. For instance,
an absolute strategy consists of a particular way of applying spatial categories
plus the computation of a global landmark. Recruitment\is{recruitment} is the process
of drawing from the pool of cognitive operations\is{cognitive operation} and assembling and packaging
them so that the complete structure for conceptualization can be scored and 
the score updated and tracked. In a second set of experiments creation
and competition of strategies are studied together.

\subsubsection{Multi-word lexical systems for expressing landmarks}
Single-word utterance systems are limited in how much information can be conveyed
in them. Upon hearing a single term it is hard to decide what conceptualization
strategy was it part of. Which landmark is used? Which perspective
did the speaker have in mind? These are questions that cannot be decided
by just looking at a single word, unless of course the word is known and always
refers to the same landmark and the same conceptualization strategy.
When we look at human language we see a lot of re-use of spatial relations.
Absolute, projective and proximal relations in German can be used with
different landmark objects. \chapref{s:multi-word} examines what mechanisms
are needed for agents to mark landmark objects using lexical items 
while at the same time co-evolving a lexicon and ontology of spatial relations. 
Once these mechanisms are in place success of such extended lexical systems
can be studied and compared to systems which only support a single
conceptualization strategy.

\subsubsection{Grammar as a tool for disambiguating spatial phrases}
The part on language evolution of this book is concluded by \chapref{s:grammar} that examines
the role and evolution of grammatical language. 

Lexical systems which are all systems studied up to this point in the book, 
have considerable shortcomings. One can study the effect grammar has by removing
grammatical knowledge from the German locative grammar implemented for this
book. The results presented in \chapref{s:grammar} show that agents operating a 
German locative system without grammar have significantly lower communicative 
success. I show that environmental conditions and diverging perspective on the scene can increase 
the drop in communicative success. The lack of grammar increases semantic
ambiguity of phrases which means that the number of possible interpretations
of a phrase escalates. As a consequence, the number of wrongly interpreted topics enlarges
as well.

Given such a clear communicative advantage for having grammar, one can 
study the necessary operators that enable agents to develop a grammar for 
disambiguating spatial phrases. This is the topic of the second part of 
\chapref{s:grammar} which reports on the precise implementation of these operators.
I test the operators in multi-agent experiments which prove that the 
hypothesized invention, learning and alignment operators allow agents 
to become increasingly more successful in communication because 
they develop an effective grammatical communication system.

%
% \bibliographystyle{diss}
% \bibliography{papers,space}
% \end{document}
% \input{2-nominals-definite.tex}
% \input{3-nominals-numberless.tex}
% \input{4-subkinds.tex}
% \input{5-conclusions.tex}
%For a start: Do not forget to give your Overleaf project (this paper) a recognizable name. This one could be called, for instance, Simik et al: OSL template. You can change the name of the project by hovering over the gray title at the top of this page and clicking on the pencil icon.

\section{Introduction}\label{sim:sec:intro}

Language Science Press is a project run for linguists, but also by linguists. You are part of that and we rely on your collaboration to get at the desired result. Publishing with LangSci Press might mean a bit more work for the author (and for the volume editor), esp. for the less experienced ones, but it also gives you much more control of the process and it is rewarding to see the quality result.

Please follow the instructions below closely, it will save the volume editors, the series editors, and you alike a lot of time.

\sloppy This stylesheet is a further specification of three more general sources: (i) the Leipzig glossing rules \citep{leipzig-glossing-rules}, (ii) the generic style rules for linguistics (\url{https://www.eva.mpg.de/fileadmin/content_files/staff/haspelmt/pdf/GenericStyleRules.pdf}), and (iii) the Language Science Press guidelines \citep{Nordhoff.Muller2021}.\footnote{Notice the way in-text numbered lists should be written -- using small Roman numbers enclosed in brackets.} It is advisable to go through these before you start writing. Most of the general rules are not repeated here.\footnote{Do not worry about the colors of references and links. They are there to make the editorial process easier and will disappear prior to official publication.}

Please spend some time reading through these and the more general instructions. Your 30 minutes on this is likely to save you and us hours of additional work. Do not hesitate to contact the editors if you have any questions.

\section{Illustrating OSL commands and conventions}\label{sim:sec:osl-comm}

Below I illustrate the use of a number of commands defined in langsci-osl.tex (see the styles folder).

\subsection{Typesetting semantics}\label{sim:sec:sem}

See below for some examples of how to typeset semantic formulas. The examples also show the use of the sib-command (= ``semantic interpretation brackets''). Notice also the the use of the dummy curly brackets in \REF{sim:ex:quant}. They ensure that the spacing around the equation symbol is correct. 

\ea \ea \sib{dog}$^g=\textsc{dog}=\lambda x[\textsc{dog}(x)]$\label{sim:ex:dog}
\ex \sib{Some dog bit every boy}${}=\exists x[\textsc{dog}(x)\wedge\forall y[\textsc{boy}(y)\rightarrow \textsc{bit}(x,y)]]$\label{sim:ex:quant}
\z\z

\noindent Use noindent after example environments (but not after floats like tables or figures).

And here's a macro for semantic type brackets: The expression \textit{dog} is of type $\stb{e,t}$. Don't forget to place the whole type formula into a math-environment. An example of a more complex type, such as the one of \textit{every}: $\stb{s,\stb{\stb{e,t},\stb{e,t}}}$. You can of course also use the type in a subscript: dog$_{\stb{e,t}}$

We distinguish between metalinguistic constants that are translations of object language, which are typeset using small caps, see \REF{sim:ex:dog}, and logical constants. See the contrast in \REF{sim:ex:speaker}, where \textsc{speaker} (= serif) in \REF{sim:ex:speaker-a} is the denotation of the word \textit{speaker}, and \cnst{speaker} (= sans-serif) in \REF{sim:ex:speaker-b} is the function that maps the context $c$ to the speaker in that context.\footnote{Notice that both types of small caps are automatically turned into text-style, even if used in a math-environment. This enables you to use math throughout.}$^,$\footnote{Notice also that the notation entails the ``direct translation'' system from natural language to metalanguage, as entertained e.g. in \citet{Heim.Kratzer1998}. Feel free to devise your own notation when relying on the ``indirect translation'' system (see, e.g., \citealt{Coppock.Champollion2022}).}

\ea\label{sim:ex:speaker}
\ea \sib{The speaker is drunk}$^{g,c}=\textsc{drunk}\big(\iota x\,\textsc{speaker}(x)\big)$\label{sim:ex:speaker-a}
\ex \sib{I am drunk}$^{g,c}=\textsc{drunk}\big(\cnst{speaker}(c)\big)$\label{sim:ex:speaker-b}
\z\z

\noindent Notice that with more complex formulas, you can use bigger brackets indicating scope, cf. $($ vs. $\big($, as used in \REF{sim:ex:speaker}. Also notice the use of backslash plus comma, which produces additional space in math-environment.

\subsection{Examples and the minsp command}

Try to keep examples simple. But if you need to pack more information into an example or include more alternatives, you can resort to various brackets or slashes. For that, you will find the minsp-command useful. It works as follows:

\ea\label{sim:ex:german-verbs}\gll Hans \minsp{\{} schläft / schlief / \minsp{*} schlafen\}.\\
Hans {} sleeps {} slept {} {} sleep.\textsc{inf}\\
\glt `Hans \{sleeps / slept\}.'
\z

\noindent If you use the command, glosses will be aligned with the corresponding object language elements correctly. Notice also that brackets etc. do not receive their own gloss. Simply use closed curly brackets as the placeholder.

The minsp-command is not needed for grammaticality judgments used for the whole sentence. For that, use the native langsci-gb4e method instead, as illustrated below:

\ea[*]{\gll Das sein ungrammatisch.\\
that be.\textsc{inf} ungrammatical\\
\glt Intended: `This is ungrammatical.'\hfill (German)\label{sim:ex:ungram}}
\z

\noindent Also notice that translations should never be ungrammatical. If the original is ungrammatical, provide the intended interpretation in idiomatic English.

If you want to indicate the language and/or the source of the example, place this on the right margin of the translation line. Schematic information about relevant linguistic properties of the examples should be placed on the line of the example, as indicated below.

\ea\label{sim:ex:bailyn}\gll \minsp{[} Ėtu knigu] čitaet Ivan \minsp{(} často).\\
{} this book.{\ACC} read.{\PRS.3\SG} Ivan.{\NOM} {} often\\\hfill O-V-S-Adv
\glt `Ivan reads this book (often).'\hfill (Russian; \citealt[4]{Bailyn2004})
\z

\noindent Finally, notice that you can use the gloss macros for typing grammatical glosses, defined in langsci-lgr.sty. Place curly brackets around them.

\subsection{Citation commands and macros}

You can make your life easier if you use the following citation commands and macros (see code):

\begin{itemize}
    \item \citealt{Bailyn2004}: no brackets
    \item \citet{Bailyn2004}: year in brackets
    \item \citep{Bailyn2004}: everything in brackets
    \item \citepossalt{Bailyn2004}: possessive
    \item \citeposst{Bailyn2004}: possessive with year in brackets
\end{itemize}

\section{Trees}\label{s:tree}

Use the forest package for trees and place trees in a figure environment. \figref{sim:fig:CP} shows a simple example.\footnote{See \citet{VandenWyngaerd2017} for a simple and useful quickstart guide for the forest package.} Notice that figure (and table) environments are so-called floating environments. {\LaTeX} determines the position of the figure/table on the page, so it can appear elsewhere than where it appears in the code. This is not a bug, it is a property. Also for this reason, do not refer to figures/tables by using phrases like ``the table below''. Always use tabref/figref. If your terminal nodes represent object language, then these should essentially correspond to glosses, not to the original. For this reason, we recommend including an explicit example which corresponds to the tree, in this particular case \REF{sim:ex:czech-for-tree}.

\ea\label{sim:ex:czech-for-tree}\gll Co se řidič snažil dělat?\\
what {\REFL} driver try.{\PTCP.\SG.\MASC} do.{\INF}\\
\glt `What did the driver try to do?'
\z

\begin{figure}[ht]
% the [ht] option means that you prefer the placement of the figure HERE (=h) and if HERE is not possible, you prefer the TOP (=t) of a page
% \centering
    \begin{forest}
    for tree={s sep=1cm, inner sep=0, l=0}
    [CP
        [DP
            [what, roof, name=what]
        ]
        [C$'$
            [C
                [\textsc{refl}]
            ]
            [TP
                [DP
                    [driver, roof]
                ]
                [T$'$
                    [T [{[past]}]]
                    [VP
                        [V
                            [tried]
                        ]
                        [VP, s sep=2.2cm
                            [V
                                [do.\textsc{inf}]
                            ]
                            [t\textsubscript{what}, name=trace-what]
                        ]
                    ]
                ]
            ]
        ]
    ]
    \draw[->,overlay] (trace-what) to[out=south west, in=south, looseness=1.1] (what);
    % the overlay option avoids making the bounding box of the tree too large
    % the looseness option defines the looseness of the arrow (default = 1)
    \end{forest}
    \vspace{3ex} % extra vspace is added here because the arrow goes too deep to the caption; avoid such manual tweaking as much as possible; here it's necessary
    \caption{Proposed syntactic representation of \REF{sim:ex:czech-for-tree}}
    \label{sim:fig:CP}
\end{figure}

Do not use noindent after figures or tables (as you do after examples). Cases like these (where the noindent ends up missing) will be handled by the editors prior to publication.

\section{Italics, boldface, small caps, underlining, quotes}

See \citet{Nordhoff.Muller2021} for that. In short:

\begin{itemize}
    \item No boldface anywhere.
    \item No underlining anywhere (unless for very specific and well-defined technical notation; consult with editors).
    \item Small caps used for (i) introducing terms that are important for the paper (small-cap the term just ones, at a place where it is characterized/defined); (ii) metalinguistic translations of object-language expressions in semantic formulas (see \sectref{sim:sec:sem}); (iii) selected technical notions.
    \item Italics for object-language within text; exceptionally for emphasis/contrast.
    \item Single quotes: for translations/interpretations
    \item Double quotes: scare quotes; quotations of chunks of text.
\end{itemize}

\section{Cross-referencing}

Label examples, sections, tables, figures, possibly footnotes (by using the label macro). The name of the label is up to you, but it is good practice to follow this template: article-code:reference-type:unique-label. E.g. sim:ex:german would be a proper name for a reference within this paper (sim = short for the author(s); ex = example reference; german = unique name of that example).

\section{Syntactic notation}

Syntactic categories (N, D, V, etc.) are written with initial capital letters. This also holds for categories named with multiple letters, e.g. Foc, Top, Num, etc. Stick to this convention also when coming up with ad hoc categories, e.g. Cl (for clitic or classifier).

An exception from this rule are ``little'' categories, which are written with italics: \textit{v}, \textit{n}, \textit{v}P, etc.

Bar-levels must be typeset with bars/primes, not with an apostrophe. An easy way to do that is to use mathmode for the bar: C$'$, Foc$'$, etc.

Specifiers should be written this way: SpecCP, Spec\textit{v}P.

Features should be surrounded by square brackets, e.g., [past]. If you use plus and minus, be sure that these actually are plus and minus, and not e.g. a hyphen. Mathmode can help with that: [$+$sg], [$-$sg], [$\pm$sg]. See \sectref{sim:sec:hyphens-etc} for related information.

\section{Footnotes}

Absolutely avoid long footnotes. A footnote should not be longer than, say, {20\%} of the page. If you feel like you need a long footnote, make an explicit digression in the main body of the text.

Footnotes should always be placed at the end of whole sentences. Formulate the footnote in such a way that this is possible. Footnotes should always go after punctuation marks, never before. Do not place footnotes after individual words. Do not place footnotes in examples, tables, etc. If you have an urge to do that, place the footnote to the text that explains the example, table, etc.

Footnotes should always be formulated as full, self-standing sentences.

\section{Tables}

For your tables use the table plus tabularx environments. The tabularx environment lets you (and requires you in fact) to specify the width of the table and defines the X column (left-alignment) and the Y column (right-alignment). All X/Y columns will have the same width and together they will fill out the width of the rest of the table -- counting out all non-X/Y columns.

Always include a meaningful caption. The caption is designed to appear on top of the table, no matter where you place it in the code. Do not try to tweak with this. Tables are delimited with lsptoprule at the top and lspbottomrule at the bottom. The header is delimited from the rest with midrule. Vertical lines in tables are banned. An example is provided in \tabref{sim:tab:frequencies}. See \citet{Nordhoff.Muller2021} for more information. If you are typesetting a very complex table or your table is too large to fit the page, do not hesitate to ask the editors for help.

\begin{table}
\caption{Frequencies of word classes}
\label{sim:tab:frequencies}
 \begin{tabularx}{.77\textwidth}{lYYYY} %.77 indicates that the table will take up 77% of the textwidth
  \lsptoprule
            & nouns & verbs  & adjectives & adverbs\\
  \midrule
  absolute  &   12  &    34  &    23      & 13\\
  relative  &   3.1 &   8.9  &    5.7     & 3.2\\
  \lspbottomrule
 \end{tabularx}
\end{table}

\section{Figures}

Figures must have a good quality. If you use pictorial figures, consult the editors early on to see if the quality and format of your figure is sufficient. If you use simple barplots, you can use the barplot environment (defined in langsci-osl.sty). See \figref{sim:fig:barplot} for an example. The barplot environment has 5 arguments: 1. x-axis desription, 2. y-axis description, 3. width (relative to textwidth), 4. x-tick descriptions, 5. x-ticks plus y-values.

\begin{figure}
    \centering
    \barplot{Type of meal}{Times selected}{0.6}{Bread,Soup,Pizza}%
    {
    (Bread,61)
    (Soup,12)
    (Pizza,8)
    }
    \caption{A barplot example}
    \label{sim:fig:barplot}
\end{figure}

The barplot environment builds on the tikzpicture plus axis environments of the pgfplots package. It can be customized in various ways. \figref{sim:fig:complex-barplot} shows a more complex example.

\begin{figure}
  \begin{tikzpicture}
    \begin{axis}[
	xlabel={Level of \textsc{uniq/max}},  
	ylabel={Proportion of $\textsf{subj}\prec\textsf{pred}$}, 
	axis lines*=left, 
        width  = .6\textwidth,
	height = 5cm,
    	nodes near coords, 
    % 	nodes near coords style={text=black},
    	every node near coord/.append style={font=\tiny},
        nodes near coords align={vertical},
	ymin=0,
	ymax=1,
	ytick distance=.2,
	xtick=data,
	ylabel near ticks,
	x tick label style={font=\sffamily},
	ybar=5pt,
	legend pos=outer north east,
	enlarge x limits=0.3,
	symbolic x coords={+u/m, \textminus u/m},
	]
	\addplot[fill=red!30,draw=none] coordinates {
	    (+u/m,0.91)
        (\textminus u/m,0.84)
	};
	\addplot[fill=red,draw=none] coordinates {
	    (+u/m,0.80)
        (\textminus u/m,0.87)
	};
	\legend{\textsf{sg}, \textsf{pl}}
    \end{axis} 
  \end{tikzpicture} 
    \caption{Results divided by \textsc{number}}
    \label{sim:fig:complex-barplot}
\end{figure}

\section{Hyphens, dashes, minuses, math/logical operators}\label{sim:sec:hyphens-etc}

Be careful to distinguish between hyphens (-), dashes (--), and the minus sign ($-$). For in-text appositions, use only en-dashes -- as done here -- with spaces around. Do not use em-dashes (---). Using mathmode is a reliable way of getting the minus sign.

All equations (and typically also semantic formulas, see \sectref{sim:sec:sem}) should be typeset using mathmode. Notice that mathmode not only gets the math signs ``right'', but also has a dedicated spacing. For that reason, never write things like p$<$0.05, p $<$ 0.05, or p$<0.05$, but rather $p<0.05$. In case you need a two-place math or logical operator (like $\wedge$) but for some reason do not have one of the arguments represented overtly, you can use a ``dummy'' argument (curly brackets) to simulate the presence of the other one. Notice the difference between $\wedge p$ and ${}\wedge p$.

In case you need to use normal text within mathmode, use the text command. Here is an example: $\text{frequency}=.8$. This way, you get the math spacing right.

\section{Abbreviations}

The final abbreviations section should include all glosses. It should not include other ad hoc abbreviations (those should be defined upon first use) and also not standard abbreviations like NP, VP, etc.


\section{Bibliography}

Place your bibliography into localbibliography.bib. Important: Only place there the entries which you actually cite! You can make use of our OSL bibliography, which we keep clean and tidy and update it after the publication of each new volume. Contact the editors of your volume if you do not have the bib file yet. If you find the entry you need, just copy-paste it in your localbibliography.bib. The bibliography also shows many good examples of what a good bibliographic entry should look like.

See \citet{Nordhoff.Muller2021} for general information on bibliography. Some important things to keep in mind:

\begin{itemize}
    \item Journals should be cited as they are officially called (notice the difference between and, \&, capitalization, etc.).
    \item Journal publications should always include the volume number, the issue number (field ``number''), and DOI or stable URL (see below on that).
    \item Papers in collections or proceedings must include the editors of the volume (field ``editor''), the place of publication (field ``address'') and publisher.
    \item The proceedings number is part of the title of the proceedings. Do not place it into the ``volume'' field. The ``volume'' field with book/proceedings publications is reserved for the volume of that single book (e.g. NELS 40 proceedings might have vol. 1 and vol. 2).
    \item Avoid citing manuscripts as much as possible. If you need to cite them, try to provide a stable URL.
    \item Avoid citing presentations or talks. If you absolutely must cite them, be careful not to refer the reader to them by using ``see...''. The reader can't see them.
    \item If you cite a manuscript, presentation, or some other hard-to-define source, use the either the ``misc'' or ``unpublished'' entry type. The former is appropriate if the text cited corresponds to a book (the title will be printed in italics); the latter is appropriate if the text cited corresponds to an article or presentation (the title will be printed normally). Within both entries, use the ``howpublished'' field for any relevant information (such as ``Manuscript, University of \dots''). And the ``url'' field for the URL.
\end{itemize}

We require the authors to provide DOIs or URLs wherever possible, though not without limitations. The following rules apply:

\begin{itemize}
    \item If the publication has a DOI, use that. Use the ``doi'' field and write just the DOI, not the whole URL.
    \item If the publication has no DOI, but it has a stable URL (as e.g. JSTOR, but possibly also lingbuzz), use that. Place it in the ``url'' field, using the full address (https: etc.).
    \item Never use DOI and URL at the same time.
    \item If the official publication has no official DOI or stable URL (related to the official publication), do not replace these with other links. Do not refer to published works with lingbuzz links, for instance, as these typically lead to the unpublished (preprint) version. (There are exceptions where lingbuzz or semanticsarchive are the official publication venue, in which case these can of course be used.) Never use URLs leading to personal websites.
    \item If a paper has no DOI/URL, but the book does, do not use the book URL. Just use nothing.
\end{itemize}

\section*{Abbreviations}

\begin{tabularx}{.5\textwidth}{@{}lX@{}}
%\textsc{1}&first person\\
\textsc{acc}&{accusative case}\\
\textsc{cl}&{classifier}\\
\textsc{cop}&{copula}\\
\textsc{f}&{feminine gender}\\
\textsc{gen}&{genitive case}\\
\textsc{inst}&{instrumental case}\\
\textsc{loc}&{locative case}\\
\textsc{m}&{masculine gender}\\
\end{tabularx}%
\begin{tabularx}{.5\textwidth}{@{}lX@{}}
\textsc{nom}&{nominative case}\\
\textsc{\textsc{pf}}&{phonological form}\\
\textsc{pfv}&{perfective aspect}\\
\textsc{pl}&{plural number}\\
\textsc{pres}&{present tense}\\
\textsc{sg}&{singular number}\\
\textsc{top}&{topic}\\
\textsc{refl}&{reflexive}\\
\end{tabularx}

\section*{Acknowledgments}
This research has been funded by the Grand Union DTP. We would like to thank members of the audience at the Workshop \emph{Semantics of Noun Phrases} for their comments, the organizers of FDSL\,13 in Göttingen, Germany, December 2018, as well as two anonymous reviewers of the conference submission and three anonymous reviewers of the proceedings article. Special thanks are due to E. Matthew Husband for stimulating discussion and feedback on a previous draft of this paper. All remaining errors are ours.

{\sloppy\printbibliography[heading=subbibliography,notkeyword=this]}

\end{document}
