\addchap{Notational conventions}\label{notational} 
This section will briefly introduce the notational conventions used in this book. As is common in sign language linguistics, manual signs are glossed using small capitals (e.g., \textsc{sign}). All glosses are in English (irrespective of the sign language). If the English translation consists of several words, but only a single sign is used, this is indicated by hyphens. The sign \textsc{perform-magic}, for example, is a single sign, but English requires a multi-word expression. Compounds are indicated by hash signs (e.g., \textsc{police\#person} `police man'). Pointings used as pronouns or to localize absent referents are glossed \textsc{index}. Subscripts indicate the direction of indices in signing space: 1 $=$ towards the signer's chest, 2 $=$ towards the addressee, 3 $=$ towards some other point in signing space. To distinguish between different points in space, lower case letters are used (\textsc{index}\textsubscript{3a}, for example, could be one point in space to be differentiated from \textsc{index}\textsubscript{3b}). Possessive pronouns are glossed \textsc{poss}, again using indices (e.g., \textsc{poss}\textsubscript{1} means `my', \textsc{poss}\textsubscript{2} means `your'). Similar indices are used when referring to verb signs moving from one location in space to another. Thus, the gloss \textsubscript{1}\textsc{give}\textsubscript{2} is to be interpreted as the sign meaning `to give' moving from the signer’s location to the location of the addressee. 

Addition symbols are used for indicating reduplications\is{reduplication}. Thus \textsc{person}++ means that the sign for `person' is produced twice and \textsc{person}+++ means that the sign is produced three times. Other modifications of the movement path of signs are indicated using subscripts. An example would be \textsc{go}\textsubscript{durative} meaning that the sign for `to go' is modified for durative aspect. The exact form the modification takes will be described in the main text if necessary.

There are several manual signs with special names. Among them is the sign \textsc{pam} (`person agreement marker') which is usually analyzed as a type of auxiliary verb expressing agreement. An example sentence is \textsc{paul angry pam maria} `Paul is angry at Maria'. Another sign with a name of its own is \textsc{bem} which is analyzed as a benefactive\is{benefactive} marker. It can be translated as `for' in most cases. An example sentence is \textsc{paul bem maria cake bake} `Paul bakes a cake for Maria'. The last sign with its own name I want to briefly mention is \textsc{p-ug},\is{palm-up gesture} an abbreviation standing for `palm-up gesture'\is{palm-up gesture} produced one- or two handedly with the palms facing upwards. This sign, often appearing clause-finally in questions, has an unclear status between a sign and a gesture and will be discussed in Sections \ref{polargeneralsectionlabel} and \ref{constint}. 

Non-manual markers, i.\,e., markers which are not produced with the hands, but simultaneously with manual material, for example, with the face or the shoulders, are glossed using lines marking their on- and offsets. An example is given in (\ref{pugdgsnotationalconv}).\is{palm-up gesture}

%\item \label{ex34}
%\slg{ix}\textsubscript{3a} \slg{light there}\slg[hn,es]{*peter at-home must} \\
%`The light is on. Peter must be at home.'\hfill (\cite{bross}, p.194)

\begin{exe}
\ex\label{pugdgsnotationalconv}
\attop{\slg[wh]{maria angry pam who p-ug}
\glt `At whom is Maria angry?' \label{ex:notation:pugdgsa1}}
\end{exe}



%\begin{exe}
%\ex\label{pugdgsnotationalconv}
%{\hspace{148pt}wh}  \\
%{$\overline{\textrm{\textsc{maria angry pam who p-ug}}}$}
%\glt `At whom is Maria angry?' \label{ex:notation:pugdgsa} 
%\end{exe}

\noindent The line above the manual signs indicates that the whole clause is accompanied by a non-manual marking glossed `wh'. The exact articulation of the non-manuals will be described in the main text. In cases in which I wanted to stress that a non-manual marker has its intensity peak at the beginning of a clause the glosses above the lines are left-aligned. An example is given in (\ref{pugdgsnotationalconvcsacas}).

\begin{exe}
\ex\label{pugdgsnotationalconvcsacas}
\attop{\slgl[mirative]{maria angry pam paul}
\glt `Surprinsingly, Maria is angry at Paul.'\label{ex:notation:pugdgsa2}}
\end{exe}
 
