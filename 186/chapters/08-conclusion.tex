\chapter{Concluding remarks}\label{chap:conclusion}

\section{The path forward}

Although all fundamentals of the relation between analogy and formal grammar were covered, some relevant related topics still need to be considered. These would require discussions of their own. In this section I will briefly delineate them.

\subsection{The limits of analogy}

The approach to analogical classifiers presented in this book does not only apply to complex systems.
They also apply to systems such as the Korean nominative marker, where nouns ending in a consonant take \textit{-i} and nouns ending in a vowel take \textit{-ka}.
Since these simple cases can be modelled without the need for inheritance hierarchies or complex analogical systems, they raise the question of where the limits of analogy lie.
Once analogical classifiers are in place in a grammar, it becomes easy to analyze these alternations as inflection classes.
This, however, does not mean that analogical classifiers are necessarily always the right answer.

This is a topic that needs further work.
It requires a good theoretical footing and techniques that would allow us to evaluate what kind of approach is better suited for a given case.
Analogical models can be compared in terms of their accuracy and coverage, but it is hard to compare analogical models to their alternatives in these terms.

\subsection{Analogical classifiers or proportional analogies}

A similar and related question which would deserve a detailed treatment is the comparison of analogical classifiers and models of proportional analogies.
As I discussed in Chapter~\ref{chap:problems}, both analogical classifiers and proportional analogy models share some core assumptions but also diverge in some key properties. While analogical classifiers require an abstraction step which links lexemes to classes and classes to forms, proportional analogy models can directly link forms to forms. This makes proportional analogy a conceptually simpler system, but it is not completely clear that it can correctly handle all cases that analogical classifiers can deal with. A thorough comparison of both approaches with relation to complex and typologically diverse phenomena is still needed.

\subsection{The features of analogy}

Probably the most intriguing question left unanswered, is the one about the relation between the nature of the morphological process and the position of the analogical relations (Chapter~\ref{chap:structural}).
From a purely theoretical perspective, there is no reason why analogy should care more about the final or initial segments of a word than the mid segments.
Analogical models could be stronger in starting from the second phoneme or only take into account a subset of phonemes.
Or, even more basic, it is unclear why analogy does not seem to take into account complete stems but only focuses on portions of stems.
I have suggested that this is likely related to learning and usage.
Tracking similarities in complete stems requires more effort than tracking similarities for only the edges of stems, which means that speakers might only track as much as they need and no more.
But this is only a conjecture, and proper theoretical, computational and experimental work needs to address this question.

\subsection{Coverage}

Finally, a question I mostly ignored is that of coverage.
None of the models reached 100\% accuracy, but speakers of the languages studied show very little uncertainty regarding the choices they have to make.
It is not often the case that a Spanish speaker is uncertain about the conjugation of some verb (although from personal experience it does happen), or that a Russian speaker does not know what the diminutive of a noun should be \autocite{Gouskova.2015}.
What this means is that analogical classifiers are much more precise than what we saw in this book.
One of the reasons for the low accuracy in many cases was that the models had much less information than what an actual speaker would have.
Spanish speakers do not just encounter verb stems but rather whole inflected forms, and they often see more than one of the stems of any verb.
An important question still missing an answer is how accurate the analogical classifiers of speakers actually are, and how much information about inflection class is really contained in stems and in fully inflected forms.
Similarly, we do not know how much speakers actively rely on analogical relations found in the system, and how much of it are just leftovers from historical processes.

\section{Final considerations}

The main proposal of this book is that the analogical relations responsible for class assignment operate on the hierarchies that define those same classes.

I have shown that analogy as predictor of  class membership is not solely restricted to one domain, or to just one language family, but can be found in gender assignment, number and case inflection, as well as verb conjugation classes, and derivational affix competition.
I have looked at Romance, Germanic, Slavic, Oto-Manguean, Chadic, and Bantu languages. I have shown that the analogical approach I propose here generalizes well to a wide range of phenomena and languages.

Chapter~\ref{chap:gender-assignment} presented two cases of interactions between gender and inflection class taken from Latin and Romanian.
I proposed that using cross-classification in the hierarchy between gender and inflection class could easily capture these interactions, and showed that the analogical models closely reflected these hierarchies.

In Chapter~\ref{chap:hybrid}, I explored overabundance and derivational doubletisms.
In these cases, there are two mutually exclusive markers/suffixes which express the same meaning. A set of lexemes can only combine with one of the two, while a second set of lexemes can combine with several.
The Croatian example illustrated this with the markers for the instrumental singular, which can be \textit{-em} or \textit{-om}.
In Russian, I explored the alternation between the three diminutive suffixes \textit{-ik}, \textit{-chik}, and \textit{-ok}.

Chapter~\ref{chap:structural} looked at a different aspect of analogical models. In this chapter I presented evidence for the claim that the nature of the morphological processes at play has an impact on the kinds of features that the analogical relations take into account.
Swahili and Otomi use prefixes to mark inflection of nouns and verbs, respectively.
In both cases, the initial segments of the stem were more important for the analogical model than the final segments.
In Hausa, plural formation includes broken plurals which keep the last consonant of the singular form of the noun but change the penultimate and final vowels.
In this case, the analogical model found the vowels of the singular were the most relevant predictors.

Finally, Chapter~\ref{chap:complex} explored two cases where inflection of verbs (Spanish) and nouns (Kasem) comprise several independent levels.
In Spanish, verbs can belong to three main inflection classes but also undergo several different stem changing processes.
In Kasem, nouns can belong to one of many different inflection classes (understood as the combination of singular and plural markers), and also undergo lengthening and diphthongization.
To capture these different dimensions of inflection, I proposed hierarchies where individual processes are captured by independent subtrees but come together to form the complete inflection classes.
The analogical models fitted to these cases showed a strong correlation with the proposed hierarchies and also showed a certain degree of organization along the different subtrees.

In this book I have presented a way of understanding analogy as a type constraint (\textsc{atc}).
This model consists of two basic building blocks: a type hierarchy and individual analogical constraints.
The type hierarchy captures all common properties between inflection or derivation classes, and organizes the individual lexemes according to their morphological behavior.
The analogical constraints operate on a type by type basis, specifying the phonological and semantic properties lexemes that belong to a certain type must fulfill.
The innovative key aspect of this model is that analogical constraints work on a binary basis, and that all types, both concrete and abstract, can impose analogical constraints.

Given a hierarchy of classes for some inflectional or derivational system, for every class in the hierarchy, a series of analogical constraints determine what phonological and semantic features items belonging to that class must satisfy.
This model allows for a straightforward integration of analogy into the grammar while keeping them distinct and modular.
The \textsc{atc} model makes the prediction that analogical relations will show reflexes of the hierarchy. In Part II, I presented evidence from six case studies that support this claim.
These case studies showed that the structure of the hierarchy clearly has reflexes on the analogical relations.

%%% Local Variables:
%%% mode: latex
%%% TeX-master: "../main"
%%% End:
