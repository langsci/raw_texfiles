\chapter{Theoretische Grundlagen}
\label{ch:theorie}

Kongruenz ist ein syntaktisches Phänomen, das sich in morphologischer
Markierung ausdrückt. \citet{corbett2006} richtet sich in seiner
Arbeitsdefinition nach \citet[610]{steele1978}, die \term{Kongruenz} als eine
irgendwie geartete systematische \isi{Kovarianz} zwischen einer semantischen
oder formalen Eigenschaft eines Elements und einer formalen Eigenschaft eines
anderen definiert.%
%
	\footnote{\blockcquote[610]{steele1978}{The term \emph{agreement} commonly
		refers to some systematic covariance between a semantic or formal
		property of one element and a formal property of another}.%
	}
%
Hervorzuheben ist dabei der Begriff der \term{Kovarianz}: Informationen über
den \isi{Kopf} einer Phrase zeigen sich an einem anderen Wort, das sich auf
diesen Kopf bezieht. Dies kann im Fall eines Substantivs\is{Substantiv} ein
Modifizierer des Substantivs sein, der mit diesem in seinen grammatischen
Merkmalen\is{Merkmale!grammatische} übereinstimmt, zum Beispiel ein
Adjektiv\is{Adjektiv!attributiv} oder ein \isi{Artikel}, oder auch ein anderer
Kopf, der anaphorisch\is{Anapher} von diesem Substantiv abhängt, wie ein
Pronomen. Kongruenz zeigt sich daneben auch in der Übereinstimmung des Verbs in
Person und Numerus\is{Personenmerkmal} mit seinem Subjekt.
\citet[20]{corbett2006} fasst die \isi{Kongruenzrelation} lakonisch als
\textquote{essentially a matter of \q{displaced} information} zusammen. Die
unterschiedlichen Grammatiktheorien haben verschiedene Auffassungen davon, wie
Kongruenz zustande kommt, beziehungsweise davon, wie sie formal zu modellieren
ist. \citeauthor{corbett2006} beschreibt das Phänomen aus morphologischer und
typologischer Perspektive.

%%%%%%%%%%%%%%%%%%%%%%%%%%%%%%%%%%%%%%%%%%%%%%%%%%%%%%%%%%%%%%%%%%%%%%%%%%%%%%%

\section{Controller, Target und Domäne}
\label{sec:ctrltarg}

\is{Controller|(}
\is{Target|(}

Im Hinblick auf Kongruenz nach \citet{corbett2006} spielen die Begriffe
\term{Controller} (Kongruenzauslöser) und \term{Target} (Kongruenzziel) eine
zentrale Rolle.%
%
	\footnote{In der deutschsprachigen Fachliteratur finden sich verschiedene
	Terminologien. \citet{fleischerschallert2011} sprechen beispielsweise von
	\q{Kongruenzträger} und \q{-ziel}, \citet{panther2009} von
	\q{Kontrolleur} und \q{Ziel}.}
%
Dieses Begriffspaar bezeichnet die beiden zuvor genannten, in bestimmten
grammatischen Merkmalen\is{Merkmale!grammatische} übereinstimmenden Instanzen,
die in einer \isi{Kongruenzrelation} zueinander stehen. Die Kongruenzrelation
geht vom Controller aus, während das Target die Information zu dessen
grammatischen Merkmalen sozusagen \q{empfängt}.

\begin{figure}
\centering
	\begin{tabular}[t]{l @{} l c}
		\itshape{grün}
		& \itshape{-er}
		& \itshape{Baum}
		\\

		%
		& \colorbox{gray}{\textcolor{white}{-\textsc{nom.m.sg}}}
		& \textcolor{gray}{\textsc{nom.m.sg}}
		\\

		\cmidrule(lr){1-2}
		\cmidrule(lr){3-3}

		\mc{2}{c}{\textsc{target}}
		& \mc{1}{c}{\textsc{controller}}
		\\

		\mc{2}{c}{\tikzmark{ctrltarg_targ}}
		& \mc{1}{c}{\tikzmark{ctrltarg_ctrl}}
		\\
	\end{tabular}
\begin{tikzpicture}[remember picture, overlay]
\draw [myarrow]
	([yshift=2.5ex]{pic cs:ctrltarg_ctrl})
	|- ++ (south:2ex) -|
	([yshift=2.5ex]{pic cs:ctrltarg_targ})
	node [near start, below] {\smaller[1]\scshape kongruenz};
\end{tikzpicture}
\caption{\q{Verschobene} Information in einer Kongruenzrelation}
\label{fig:ctrltarg}
\end{figure}

Das Beispiel in \figref{fig:ctrltarg} enthält ein \isi{Substantiv}, \fw{Baum}.
Von diesem ist bekannt, dass es die folgenden grammatischen
Merkmale\is{Merkmale!grammatische} trägt: \feat{maskulin}, \feat{Singular}.
Ferner sei angenommen, dass es im \feat{Nominativ}\is{Kasus} steht. Die
Merkmale \feat{maskulin} und \feat{Singular} sind dem
Lexikoneintrag\is{Lexikon} des Wortes inhärent; sie werden nicht morphologisch
an ihm markiert. Die grammatischen Informationen zeigen sich jedoch in einem
Portmanteau-Morphem als Suffix \fw{-er} am Adjektiv\is{Kongruenz!attributive}
\fw{grün}, das selbst keine Genus-\is{Genusmarkierung} und Numerusmerkmale oder
Informationen zum Kasus der \isi{Nominalphrase} (NP) enthält. Genus-, Numerus-
und Kasusmerkmale des Substantivs werden durch die Flexion am Adjektiv
widergespiegelt und erscheinen daher \q{verschoben}. Das
Adjektiv\is{Adjektiv!attributiv} ist damit das Target der
Kongruenz\-relation\is{Kongruenzrelation}, während das Substantiv den
Controller darstellt.

Bei Kongruenz geht es im Grunde also um die Vererbung beziehungsweise das Teilen
grammatischer Merkmale\is{Merkmale!grammatische} (\fw{grammatical features};
\cite{corbett2012}). Bei Personenbezeichnungen sind
Informationen zum \isi{Genus} und \isi{Numerus} tendenziell an deren
semantische Eigenschaften\is{Merkmale!semantische} gebunden; bei Dingen und
Abstrakta ergeben sie sich anhand formaler Kriterien
\autocites[vgl.][2--4, 125--132]{corbett2006}{koepckezubin2017}. So bezeichnet
\fw{Mutter} konzeptuell\is{Denotation} eine \textquote{Frau, die ein oder
mehrere Kinder geboren hat}, oder allgemeiner eine
\blockcquote[s.\,v.~\fw{Mutter}]{duden-online}{Frau, die in der Rolle einer
Mutter ein oder mehrere Kinder versorgt, erzieht}. \fw{Mutter} hat damit
\feat{feminin} als Wert des grammatischen Merkmals \feat{Genus}. Doch dass etwa
\fw{Baum} maskulin und \fw{Liebe} feminin ist, hat kein Korrelat in der
außersprachlichen Realität, sondern ist eine
\isi{Konvention} des Deutschen\il{Neuhochdeutsch}.%
%
	\footnote{Nichtsdestoweniger ist die \isi{Genuszuweisung} bei
		Inanimata\is{Inanimata} nicht ganz willkürlich.
		\citeauthor{koepckezubin2017} haben seit den 1980er Jahren gezeigt
		\autocites[z.\,B.][]%
			{koepcke1982}%
			{koepckezubin1996}%
			{koepckezubin2009}%
			{koepckezubin2017},
		dass nicht nur Derivationssuffixe wie
			\fw{-ling} (\textsc{m}),
			\fw{-schaft} (\textsc{f}) oder
			\fw{-chen} (\textsc{n}),
		sondern auch die phonologische Struktur von Wortstämmen und die
		Zugehörigkeit von Substantiven\is{Substantiv} zu bestimmten
		semantischen Feldern einen starken Einfluss auf die
		\isi{Genuszuweisung} haben. Zum Beispiel tendieren konsonantenreiche
		Einsilber zum Maskulinum (%
			\fw{der} /braɪ̯/,
			/ɛrnst/,
			\smash{/ʃtrʊmp͡f/}%
			% , aber
			% \fw{die} /ʃprɔʏ̯/,
			% \fw{das} /fɔlk/,
			% \fw{die} /frɪst/
			;
		\cite[vgl.][475--479]{koepckezubin1996}), während Zweisilber, die auf
		Schwa enden, häufig dem Femininum zugeordnet werden (%
			\fw{die} /liːbə/,
			\smash{/ʃprɪt͡sə/,}
			/taʃə/%
			% , aber
			% \fw{der} /kɛːzə/,
			% \fw{das} /aʊ̯ɡə/
			;
		\cite[vgl.][207--209]{koepckezubin2017}). Was semantische Felder
		betrifft, sind zum Beispiel Sprachbezeichnungen gewöhnlich Neutra (%
			\fw{das Deutsche},
			\fw{Hindi},
			\fw{Nahuatl};
		\cites[siehe]%
			[480]{koepckezubin1996}%
			[137--139]{koepckezubin2009}%
			[210--214]{koepckezubin2017}
		für weitere Beispiele aus anderen Feldern).%
		\label{fn:koepckezubin}
	}

Wie darüber hinaus aus dem Schema in \figref{fig:termini} deutlich wird,
unterscheiden \citet{wechslerzlatic2003} nicht nur zwischen grammatischer und
pragmatischer\is{Pragmatik} Kongruenz, sondern innerhalb der grammatischen
Kongruenz noch solche mit den Merkmalskategorien\is{Merkmale!grammatische}
\feat{Concord}\is{Concord} und \feat{Index} (\sectref{sec:indexconcord};
vergleiche \cite[8--17]{wechslerzlatic2003}). Die theoretische Grundlage für
diese Unterscheidung liefert ihnen die \isi{Head-driven Phrase Structure
Grammar} (HPSG; \cite{pollardsag1994}). Im Gegensatz zu der Anmerkung von
\citet[164]{fleischer2012}, dass \emph{formal} und \emph{semantisch} in der
Forschung synonym zu \emph{grammatisch} und \emph{pragmatisch} benutzt werden,
unterscheiden \citet{wechslerzlatic2003} also alle vier Termini. Pragmatische
und semantische Kongruenz überschneiden sich mit dem Begriff \fw{constructio ad
sensum}, da sie sich beide aus der Semantik speisen.

\begin{figure}
\centering
\begin{tikzpicture}[baseline=(grm.base), shorten >= 4pt, shorten <= 4pt]
	\node [draw, rectangle, align=center] (frm) at (0,0) {
		formal\\ \footnotesize (\fw{ad formam})
	};
	\node [draw,rectangle, align=center] (sem) at (3,0) {
		semantisch\\ \footnotesize (\fw{ad sensum})
	};
	\node [draw, rectangle] (grm) at (0,2) {grammatisch};
	\node [draw, rectangle] (prg) at (3,2) {pragmatisch};
	\draw [-latex] (grm) -- (frm);
	\draw [-latex] (grm) -- (sem);
	\draw [-latex] (prg) -- (sem);

	\node [draw, rectangle, align=center, gray] (con) at (0,-2) {
		\textsc{concord} \\
		\footnotesize (Modifikatoren)
	};
	\node [draw, rectangle, align=center, gray] (idx) at (3,-2) {
		\textsc{index} \\
		\footnotesize (anaphorische\\
		\footnotesize Bindung)
	};
	\draw [-latex, gray] (con) -- (frm);
	\draw [-latex, gray] (idx) -- (frm);
	\draw [-latex, gray] (idx) -- (sem);
\end{tikzpicture}
\captionsetup{width=.725\textwidth}
\caption%
	{Einteilung der Kongruenztypen und beteiligte
	Merkmale nach \citet{wechslerzlatic2003}}
\label{fig:termini}
\end{figure}

\phantomsection
\label{phsec:domain}
\is{Domäne|(}

Ein weiterer wichtiger Faktor in der Charakterisierung von
Kongruenzbeziehungen\is{Kongruenzrelation} nach \citet{corbett2006} ist die
\term{Domäne}. Dieser Begriff wird von \citet{corbett2006} nicht formal
definiert. Aus seinen Ausführungen ist aber zu entnehmen, dass damit der
Abstand zwischen Controller und Target im Sinne der Konstituenz von Sätzen
gemeint ist. So gibt es nach \citet[54]{corbett2006} vier Domänen mit
wachsendem Umfang\is{Lokalität} und abnehmender
Kanonizität:\is{Kongruenzrelation!nicht-kanonische}\largerpage

\begin{enumerate}
	\item innerhalb der NP;
	\item außerhalb der NP aber innerhalb des Teilsatzes;
	\item außerhalb des Teilsatzes aber innerhalb des Satzes;
	\item außerhalb des Satzes.
\end{enumerate}

In (\ref{ex:beidedomains_1}--\ref{ex:beidedomains_4}) wird jeweils ein Beispiel
pro Domäne gegeben. In \REF{ex:beidedomains_1} stellt \lit{bede} \wdef{beide}
als Target einen \isi{Modifikator} seines Controllers \lit{wingarten}
\wdef{Weingarten} in derselben NP\is{Nominalphrase} beziehungsweise in
derselben \isi{Nominalgruppe} dar. Das Target steht attributiv\is{Attribut} zum
Controller.

\begin{exe}
\ex \label{ex:beidedomains_1}
	\gll die bede wingarten \\
		die beide-\textsc{acc.pl.m.st} Weingarten-\textsc{acc.pl.m} \\
	\trans \wdef{die beiden Weingärten}
		\parencites%
			(Nr.~1221, Zürich, 1290)%
			[484,9]{cao2}
\end{exe}

Der Controller \lit{zil} \wdef{Ziele, Fristen} in \REF{ex:beidedomains_2}
bildet den \isi{Kopf} der in sich abgeschlossenen Genitiv-NP\is{Nominalphrase}
\lit{der vor genanten zil} \wdef{der vorgenannten Ziele}.
\lit{\textins{B}eidiv} \wdef{beide} als Modifizierer und damit Target des
Controllers \lit{zil} \wdef{Ziele, Fristen} steht nachgestellt nicht in
derselben NP wie sein Controller \lit{zil}, aber dennoch im gleichen Satzteil,
da die komplexe NP \lit{der vor genanten zil / ainez / oder beidiv} \wdef{eines
oder beide der vorgenannten Ziele} das Akkusativobjekt zu \lit{verſitzzet}
\wdef{versäumt} bildet. Damit ist \lit{beidiv} auch hier ein \isi{Attribut}.

\begin{exe}
\ex \label{ex:beidedomains_2}
	\gll ſwenne man der {vor genanten} zil / ainez / oder
		beidiv \textelp{} verſitzzet \\
		so=wenn man der vorgenannten Ziel[\textsc{gen.pl.n}] {} eines {}
		oder beide-\textsc{acc.pl.n.st} {} versäumt \\
	\trans \wdef{falls man eines oder beide der vorgenannten Ziele \textelp{}
		versäumt}
		\parencites%
			(Nr.~619, Augsburg, 1283)%
			[47,31]{cao2}
\end{exe}

In \REF{ex:beidedomains_3} ist das Target \lit{beideu} \wdef{beide} als
\isi{Attribut} auf den Controller \lit{rihtær} \wdef{Richter} bezogen, befindet
sich formal aber nicht im gleichen Satzteil wie dieser, dennoch aber im
gleichen (Teil-)Satz, insofern beide NPs vom gleichen Verb \lit{ſprachen}
\wdef{sprachen} abhängen.

\begin{exe}
\ex \label{ex:beidedomains_3}
	\gll Die rihtær ſprachen beideu {dar zuͦ} \\
		die Richter[\textsc{nom.pl.m}] sprachen beide-\textsc{nom.pl.n.st}
			dazu \\
	\trans \wdef{die Richter äußerten sich beide dazu}
		(%
			B1: 28ra,8;
			vgl.~\KC: V.~10090; \cite[267]{schroeder1895}%
		)
\end{exe}

Im letzten Schritt bezieht sich das Target \lit{bede} \wdef{beide} in
\REF{ex:beidedomains_4} zwar auf \lit{herren} \wdef{Herren}, doch bildet
\lit{herren} das Subjekt zum Verb \lit{lident} \wdef{leiden}, während
\lit{bede} zusammen mit dem \isi{Personalpronomen} \lit{Si} \wdef{sie} das
Subjekt von \lit{ligent} \wdef{liegen} bildet. Damit stehen der
(Erst-)Controller \lit{herren} und das Target \lit{bede} in unterschiedlichen
Sätzen. Das \norm{bėide}-Target bezieht sich anaphorisch\is{Anapher} auf seinen
Controller.

\begin{exe}
	\ex \label{ex:beidedomains_4}
		\gll Min herren lident ovch groze not. \\
			mein Herr-\textsc{nom.pl.m} leiden auch große Not \\
	\sn \gll Si ligent bede fvͤr tot \\
			\textsc{3pl.nom} liegen beide-\textsc{nom.pl.m.st} für tot \\
	\trans \wdef{Meine Herren leiden auch große Not. Sie liegen beide tot
		\textins{darnieder}.}
		(%
			VB: 86ra,3--4;
			vgl.~\KC: V.~12033--12034; \cite[301]{schroeder1895}%
		)		
\end{exe}

\is{Domäne|)}
\is{Target|)}
\is{Controller|)}

%%%%%%%%%%%%%%%%%%%%%%%%%%%%%%%%%%%%%%%%%%%%%%%%%%%%%%%%%%%%%%%%%%%%%%%%%%%%%%%

\section{Index und Concord}
\label{sec:indexconcord}
\is{Merkmale!grammatische|(}

\phantomsection
\label{phsec:index}
\is{Index|(}

Das Merkmal \feat{Index} ist in der HPSG\is{Head-driven Phrase Structure
Grammar} seinerseits Teil des Merkmals \feat{Content} eines lexikalischen
Zeichens, das sich aus der Semantik speisende Informationen über dessen Denotat
enthält, konkret also grammatikalisierte\is{Grammatikalisierung}
Personen\-merkmale\is{Personenmerkmal} wie \feat{Person},
\feat{Numerus}\is{Numerus} und \feat{Genus}\is{Genus}
\autocite[15--17]{wechslerzlatic2003}. Auch in der mit der HPSG\is{Head-driven
Phrase Structure Grammar} verwandten Lexical-Functional Grammar (LFG;
\cites{kaplanbresnan1982}{bresnan2001}{bresnanetal2016}{dalrymple2023})
existiert ein solches Merkmal \autocite[189--190]{bresnanetal2016}. Das Merkmal
\feat{Index} bildet die Basis für anaphorische\is{Anapher} Referenz, indem es
ein Individuum oder ein Ding als Instanz des Bezeichneten im Diskurs
verankert\is{Diskursanker} \autocite[10--11]{wechslerzlatic2003}. Während eine
große Nähe zwischen Index und Semantik konstatiert wird, ist diese nicht
absolut, da auch das generische Pronomen\is{generischer Gebrauch} \fw{man} oder
das expletive Subjektspronomen \fw{es} einen Index besitzen, obwohl sie sich
nicht auf eine bestimmte semantische Größe beziehen
\autocite[11--13]{wechslerzlatic2003}. Dies wird bei der Konjugation von Verben
und der Bindung\is{Bindung} von Reflexivpronomen\is{Reflexivpronomen} sichtbar
\REF{ex:explvbkonj}.

\begin{exe}
\ex \label{ex:explvbkonj}
	\begin{xlist}
	\ex Es\tsub{i} regnet\tsub{i}.
	\ex Man\tsub{i} kann\tsub{i} es sich\tsub{i} vorstellen.
	\end{xlist}
\end{exe}

\citet{kingdalrymple2004} argumentieren des Weiteren, dass \feat{Index} in
koordinierten\is{Koordination} nominalen Strukturen ein nicht-distributives
Merkmal darstellt, indem die Kombination zweier Substantive\is{Substantiv}
einen neuen Index erhält, mit dem kongruiert wird
\autocite[74--76]{kingdalrymple2004}. \fw{Jan} und \fw{Markus} in
\REF{ex:coordidx} haben zwar jeder für sich einen Singular-Index, die Gruppe
\fw{Jan und Markus} hat aber einen Plural-Index, was sich in der Kongruenz
zwischen Subjekt und Verb zeigt.

\begin{exe}
\ex\label{ex:coordidx}
\begin{xlist}
	\ex[]{[Jan\tsub{i} und Markus\tsub{j}]\tsub{k} spielen\tsub{k} Fußball.}
	\ex[*]{Jan\tsub{i} und Markus\tsub{j} spielt\tsub{{i/j}} Fußball.}
\end{xlist}
\end{exe}

\is{Index|)}

\phantomsection
\label{phsec:concord}
\is{Concord|(}

Das Gegenstück zum Index bildet das Merkmal \feat{Concord}, das in der
HPSG\is{Head-driven Phrase Structure Grammar} Teil des Kopfmerkmals\is{Kopf}
ist \autocite[17]{wechslerzlatic2003}. Das Merkmal \feat{Concord} existiert
ebenso in der LFG\is{Lexical-Functional Grammar}
\autocite[189--192]{bresnanetal2016} und enthält in beiden Theoriesystemen
syntaktisch relevante Informationen über den nominalen Phrasenkopf, die im
Deutschen\il{Neuhochdeutsch} für einzelne, unkoordinierte\is{Koordination} NPs
weitgehend deckungsgleich mit denen des
\isi{Index} sind:%
%
	\footnote{\citet{wechslerzlatic2003} untersuchen den Fall des
	Bosnisch-Kroatisch-Montenegrinisch-Serbischen (\ili{BKMS}), in dem es
	Substantive\is{Substantiv} gibt, für die dies nicht der Fall ist.}
%
Zu den Merkmalen \feat{Numerus}\is{Numerus} und \feat{Genus}\is{Genus} tritt
das strukturelle Merkmal \feat{Kasus}\is{Kasus}, dafür spielt \feat{Person}
hier keine Rolle. Kongruenz über das Merkmal \feat{Concord} herrscht
typischerweise zwischen einem nominalen \isi{Kopf} und seinen Modifizierern,
was in der HPSG\is{Head-driven Phrase Structure Grammar} mit dem \term{head
feature principle}\is{head feature principle@\emph{head feature principle}} begründet wird. Dieses besagt im
Grunde, dass ein Phrasenkopf seine grammatischen Informationen mit seinen
Töchtern teilt \autocites[vgl.][34]{pollardsag1994}[22]{wechslerzlatic2003}.
Daher ist anders das Merkmal \feat{Concord} im Grunde auf die
NP\is{Nominalphrase} beschränkt, während das Merkmal \feat{Index}\is{Index}
überall dort vorkommt, wo anaphorische\is{Anapher} Bindung\is{Bindung} eine
Rolle spielt \parencites[14--16, 22]{wechslerzlatic2003}[189]{bresnanetal2016}.

In Bezug auf koordinierte\is{Koordination} Substantive\is{Substantiv} wie in
\REF{ex:coordidx} sei angemerkt, dass \citet[76--78]{kingdalrymple2004}
\feat{Concord} im Gegensatz zu \feat{Index} als distributives Merkmal
analysieren. Dies bedeutet, dass Modifizierer von koordinierten Substantiven
mit jedem Konjunkt einzeln in Concord-Merkmalen übereinstimmen müssen, um einen
akzeptablen Ausdruck zu produzieren. Das Beispiel in \REF{ex:engartdiscong}
illustriert dies, insofern \fw{these} \wdef{diese} nur dann verwendet werden
kann, wenn beide von ihm determinierten Konjunkte jeweils im Plural stehen
\REF{ex:engartdiscong_1} oder wenn es lediglich ein einzelnes Substantiv
determiniert und stattdessen die \isi{Konjunktion} auf höherer syntaktischer
Ebene stattfindet \REF{ex:engartdiscong_4}.

\begin{exe}
\ex\label{ex:engartdiscong}
	\langinfo%
		{Englisch}
		{}
		{\cite[nach][70]{kingdalrymple2004}}
	\begin{xlist}
	\ex[]{these{\ob}\textsc{pl}{\cb} boys{\ob}\textsc{pl}{\cb}
		and girls{\ob}\textsc{pl}{\cb}}
		\label{ex:engartdiscong_1}
	\ex[*]{these{\ob}\textsc{pl}{\cb} boys{\ob}\textsc{pl}{\cb}
		and girl{\ob}\textsc{sg}{\cb}}
		\label{ex:engartdiscong_2}
	\ex[*]{this{\ob}\textsc{sg}{\cb} boy{\ob}\textsc{sg}{\cb}
		and girls{\ob}\textsc{pl}{\cb}}
		\label{ex:engartdiscong_3}
	\ex[]{these{\ob}\textsc{pl}{\cb} boys{\ob}\textsc{pl}{\cb}
		and this{\ob}\textsc{sg}{\cb} girl{\ob}\textsc{sg}{\cb}}
		\label{ex:engartdiscong_4}
	\end{xlist}
\end{exe}

\is{Concord|)}
\is{Merkmale!grammatische|)}

%%%%%%%%%%%%%%%%%%%%%%%%%%%%%%%%%%%%%%%%%%%%%%%%%%%%%%%%%%%%%%%%%%%%%%%%%%%%%%%

\section{Genus, Sexus und Belebtheit}
\label{sec:gendsex}

\is{Genus|(}
\is{Sexus|(}
\is{Belebtheit|(}

Wenn im Folgenden von \term{Genus} die Rede ist, bezeichnet der Begriff das
grammatische Geschlecht eines Substantivs\is{Substantiv}. Da es sich beim
Mittelhochdeutschen\il{Mittelhochdeutsch} um eine flektierende Sprache handelt,
ist die häufig zitierte Definition von \citet[231]{hockett1958} relevant, die
Genera relational begreift als Klassen der Substantive, die sich im Verhalten
von darauf bezogenen Wörtern zeigen.%
%
	\footnote{\foreigntextcquote{english}[231]{hockett1958}{Genders are classes
		of nouns reflected in the behavior of associated words}.}
%
Genus wird also als grammatische Kategorie der Klassifizierung von
Substantiven\is{Substantiv} aufgefasst, die an der Kongruenzform von darauf
bezogenen Targets\is{Target} erkennbar ist. Im Sinne
\posscite[62--63]{corbett1991} bezieht sich diese Definition auf koverte
Genusmarkierung (\fw{covert gender}),\is{Genusmarkierung} bei dem die
Klassenzugehörigkeit eines Substantivs nicht aus seiner Form erschlossen werden
kann. Wie zum Beispiel von \citet{koepckezubin2017} gezeigt, besitzt das
Deutsche\il{Neuhochdeutsch} allerdings durchaus Aspekte overter
\isi{Genuszuweisung}, sei es durch die Semantik des Bezeichneten, die
Zugehörigkeit zu einer bestimmten Klasse von Dingen oder die phonologische
Struktur des Lexems.

Von den 257 von \citet{corbett2013b} untersuchten, typologisch diversen
Sprachen besitzen 84 (32,7\,\%) ein sexusbasiertes System als semantische
Grundlage der Klasseneinteilung\is{Klassenbildung}; in den Daten von
\tit{Grambank} \autocite{skirgardetal2023} fällt dieser Wert auf lediglich 410
von 2.206 Varietäten (18,6\,\%; \cite[siehe][]{haynie:gb051}). Gerade der
Aspekt der Sexusbasiertheit führt zu einiger Komplexität beim Zusammenspiel von
Geschlecht im biologischen\is{Geschlecht!biologisches} und soziokulturellen
Sinn sowie der sekundären Nutzung von Geschlecht als grammatischer
Kategorie\is{Grammatikalisierung} \autocites[dazu
ausführlich][]{kotthoffnuebling2018}{steriopolosteriopolo2022}. Die
Doppeldeutigkeit von \fw{gender} im Englischen\il{Englisch} als Bezeichnung
einerseits für das soziale Geschlecht\is{Gender} und die damit verbundene
Geschlechter\-rollen\-praxis\is{Geschlechterrolle} und andererseits für die
grammatische Kategorie verkompliziert tendenziell den Diskurs, indem zwei
miteinander interagierende Ebenen terminologisch vermischt werden.

\is{Gender|(}

Der Zusammenhang von sozialem Geschlecht und Sprache führt immer wieder zu
stark emotional geführten, sprachästhetisch bis sozialkritisch motivierten
Debatten insbesondere über die Notwendigkeit der expliziten sprachlichen
Sichtbarmachung von Frauen üblicherweise durch Movierung der maskulinen Form
eines \isi{Nomen Agentis} sowie von Personen, die sich dem queeren Spektrum
zugehörig fühlen, durch weitere typografische Mittel \autocite[dazu kritisch
resümierend][]{kasper2022}. Mit \citet[61--89]{kotthoffnuebling2018} ist
anzumerken, dass der Laiendiskurs um das sogenannte \q{Gendern} häufig sehr
oberflächlich bleibt. Ein explizit andro\-zentrischer Bias durchdringt die
gesamte Nominalmorphologie des Deutschen\il{Neuhochdeutsch} sehr viel tiefer.

Aus sprachhistorischer\is{Sprachgeschichte} Sicht gründet sich diese kritische
Beobachtung darin, dass die indogermanische Ursprache\il{Indogermanisch} vor
der Abspaltung des anatolischen Zweigs wahrscheinlich lediglich eine
Distinktion\is{Belebtheit} \feat{[±\,belebt]} oder \feat{[±\,human]} besaß,
die sowohl im Zusammenhang mit \isi{Individualisierbarkeit} als auch in
syntaktischer Hinsicht mit Agensfähigkeit stand. Das spätere
Maskulinum\is{Grammatikalisierung} setzt das belebte\is{Animata} Genus fort,
das Neutrum das unbelebte\is{Inanimata}. Zur expliziten Kennzeichnung von
belebten Feminina hat sich sekundär das Suffix \fw{*-h₂} etabliert, das darüber
hinaus zur Derivation von Kollektiva und Abstrakta diente, wie in
\tabref{tab:pie_h2} gezeigt \autocites%
	[73--74, 77]{ringe2017}%
	[195--197, 205--207]{fritzmeierbruegger2021}%
	[167--172]{klein2022}%
. Mit \citet[313]{corbett1991} ist anzumerken, dass die Umnutzung und
Rekombination von vorhandenem morphologischen Material bei der
Ausdifferenzierung von Genera keine Seltenheit darstellt.

\begin{table}
\centering
\caption{\fw{*h₂}-Derivate im Urindogermanischen\il{Indogermanisch}}
\begin{tabular}[t]{
	l @{} l @{} l @{~} l
	c
	l @{} l @{} l @{~} l
	l
}

\lsptoprule

\fw{*wĺ̥kʷ}
	& \fw{-o}
	& \fw{-s}
	& \wdef{Wolf}
& $\to$
& \fw{*wl̥kʷ}
	& \fw{-í}
	& \fw{-h₂}
	& \wdef{Wölfin}
& \parencite[102, 132]{ringe2017} % Kapitel 3.2.2.i, 3.2.4.iii
\\

\fw{*kʷékʷl}
	& \fw{-o}
	& \fw{-s}
	& \wdef{Rad}
& $\to$
& \fw{*kʷekʷl}
	& \fw{-é}
	& \fw{-h₂}
	& \wdef{Rädersatz}
& \parencite[59]{ringe2017} % Kapitel 2.3.4.ii
\\

\fw{*bʰewg-}
	& %
	& %
	& \wdef{fliehen}
& $\to$
& \fw{*bʰug}
	& \fw{-á}
	& \fw{-h₂}
	& \wdef{Flucht}
& \parencite[74]{ringe2017} % Kapitel 2.4.2.i
\\

\lspbottomrule
\end{tabular}
\label{tab:pie_h2}
\end{table}

Bezüglich der eingangs referierten Definition von Genus als relationaler Größe
hebt \citet[42]{koepcke1982} hervor, dass \textquote{\textins*{d}er
kommunikative Wert von Genuszuweisungen\is{Genuszuweisung} \textelp{} in erster
Linie darin \textins{liegt}, daß sie dem Sprachbenutzer im kommunikativen
Zusammenhang\is{Genusgebrauch} anaphorische Referenzierungen\is{Anapher}
erleichtern}, zum Beispiel, indem häufig im gleichen Kontext\is{Pragmatik}
genannte Dinge unterschiedliche Genera besitzen \autocite[dazu
auch][320--323]{corbett1991}. Darüber hinaus vermag Genus auf der Ebene der
Pragmatik die Einstellung einer Sprecherin oder eines Sprechers zu
signalisieren, was Respekt oder Verachtung, Zu- oder Abneigung gegenüber einer
Person durch die Verwendung des \q{richtigen} oder \q{falschen}
Genus\is{Genusgebrauch} in Bezug auf deren Identität betrifft
\autocites[322--323]{corbett1991}[52--62]{aikhenvald2016}.

\citet{steriopolosteriopolo2022} verorten diese affektive Funktion des
Genus\-gebrauchs\is{Genusgebrauch} im sozialen Geschlecht (\fw{social gender})
als Bindeglied zwischen biologischem\is{Geschlecht!biologisches} und
grammatischem Geschlecht\is{Genus}. Sie attestieren dieser Ebene, die den
sozialen oder ontologischen Status eines Individuums anspricht, eine starke
emotionale Komponente, wenn durch abweichenden\is{Ausnahme}
Genus\-gebrauch\is{Genusgebrauch} zum Beispiel despektiertlich kommuniziert
wird, dass sich eine Person nicht in Einklang mit der ihr zugemessenen
\isi{Geschlechterrolle} verhält. Abgesehen von lexikalisierten Fällen von
abweichendem Genus wie mittelhochdeutsch\il{Mittelhochdeutsch} \norm{wīp}
\wdef{Frau} (Neutrum mit Bezug auf eine weibliche Person) oder \norm{kindelīn}
\wdef{Kindlein} (Neutrum mit Bezug auf junge Menschen) spielt situativ
abweichender Genusgebrauch bei Personen im Belegmaterial keine Rolle.

Im Zusammenhang mit \norm{bėide} \wdef{beide} kommen im Belegmaterial natürlich
nicht nur Personen vor, sondern häufig auch Sachen, wie etwa ein \norm{garte}
\wdef{Garten} und ein \norm{acker} \wdef{Acker}, deren Verkauf die Urkunde
Nr.~3249 \autocites(Freiburg i.\,Br., 1299)[][417,2--14]{cao4} behandelt.
Belebtheit macht sich nicht nur bei der \isi{Genuszuweisung}, sondern
auch unter morphologischen Gesichtspunkten bemerkbar, obwohl sie keine
eigentliche grammatische Kategorie des
Mittelhochdeutschen\il{Mittelhochdeutsch} darstellt. In den Worten
\posscite[99]{dahl1999} ist \term{Belebtheit}\is{Belebtheit}, also der
Unterschied zwischen belebten und unbelebten Entitäten, in den Grammatiken
menschlicher Sprachen so allgegenwärtig, dass sie tendenziell als
selbstverständlich hingenommen und damit unsichtbar wird.%
%
	\footnote{\foreigntextcquote{english}[99]{dahl1999}{Animacy\is{Belebtheit},
		or the distinction between animate and inanimate entities, is so
		pervasive in the grammars of human languages that it tends to be taken
		for granted and become \textins{sic} invisible}.}

\begin{figure}
\centering
\includegraphics[
	width=\linewidth,
	keepaspectratio,
]{./figures/belebtheitshierarchie.pdf}
% one-off increase caption width for page no. not to slip to 2nd line all alone
\captionsetup{width=.85\textwidth}
\caption{\isi{Belebtheitshierarchie} \autocite[nach][72]{kotthoffnuebling2018}}
\label{fig:animhier}
\end{figure}

\isi{Belebtheit} wird als Kontinuum konzipiert; die
\term{Belebtheitshierarchie}\is{Belebtheitshierarchie} reicht von Menschen als
belebtester Kategorie über Tiere zu Gegenständen und Abstrakta als am wenigsten
belebt (siehe \figref{fig:animhier}). Die Sprecherinstanz steht dabei
gewöhnlich an der Spitze der Skala, da sie den Lokus der Wahrnehmung bildet
\autocites%
	{silverstein1976}%
	[185--200]{comrie1989}%
	[203]{bossong1998}%
	[40--46]{siewierska2004}%
	[439--441]{bickel2011}%
	[63--79]{kotthoffnuebling2018}. %
Belebtheit ist \citet[101--102, 110--112]{dahl1999} zufolge auch ein wichtiger
Faktor bei der \isi{Genuszuweisung}, da ein hoher Grad von Belebtheit mit
semantischen Eigenschaften des Bezeichneten~-- und hier gerade Sexus
beziehungsweise sexuelle Differenzierbarkeit~-- als Quelle von Genus
korrelliert (\fw{referential gender}), während ein hoher Grad von
Unbelebtheit\is{Inanimata} mit formalen Kriterien der Genuszuweisung einhergeht
(\fw{lexical gender}). Wie aus \figref{fig:animhier} deutlich wird, gibt es der
Natur von Kontinua entsprechend keine scharfe Grenze zwischen den Polen
\feat{belebt} und \feat{un\-belebt} und damit viele Übergangs- und
Zweifelsfälle.\is{Ambiguität}

Da die Personen, die in den hier untersuchten Texten benannt werden, seit
Jahrhunderten tot und im Fall der \tit{Kaiserchronik} (\KC) zusätzlich
literarisch überformt sind, lässt sich nicht nachvollziehen, wie sich das
Verhältnis von sozialem und biologischem\is{Geschlecht!biologisches} Geschlecht
in jedem einzelnen Fall genau verhält, zumal es anachronistisch wäre, moderne
Konzepte von sexueller Identität\is{sexuelle Identität} auf das Hochmittelalter
beziehungsweise die Antike im Spiegel eines hochmittelalterlichen Texts zu
übertragen \autocite[siehe z.\,B.][]{klinger2002}. Weil Kontexte, in denen cis-
und heterosexuelle Normen\is{Heteronormativität} explizit unterlaufen oder
infrage gestellt werden, im Belegmaterial nicht
vorkommen,\is{Komplexitätsreduktion} gehe ich bei der Klassifizierung meiner
Daten von cis- und heteronormativen Gegebenheiten aus.

Wenn im Weiteren vereinfachend\is{Komplexitätsreduktion} von \term{Sexus} als
semantischer Basis von Genus die Rede ist, bezieht sich der Begriff nicht
primär auf eine biologische\is{Geschlecht!biologisches} \isi{Lesart}. Vielmehr
ist damit diejenige Geschlechter\-rolle\is{Geschlechterrolle} gemeint, die in
der Wortbedeutung einer Personenbezeichnung angelegt ist und daher stereotyp
erwartet wird oder die vom Textzusammenhang etwa durch Namennennung implizierte
Geschlechterrolle. Gerade in der älteren Literatur ist in diesem Kontext häufig
vom \q{natürlichen Geschlecht} die Rede; \citet[67]{panther2009} spricht
diesbezüglich vom \q{konzeptuellen Genus}\is{Denotation}. Um auf das zuvor
zitierte Beispiel zurückzukommen, wird im Folgenden die Bezeichnung \fw{Mutter}
entsprechend der in der Wortbedeutung angelegten weiblichen Geschlechterrolle
als eine weibliche Person denotierend aufgefasst
\autocite[vgl.][s.\,v.~\fw{Mutter}]{duden-online}.

\is{Gender|)}

Bezüglich komplexer Kombinationen von Genus, Sexus und der Kongruenz darauf
bezogener Targets\is{Target} diskutiert \citet[183--184]{corbett1991}
sogenannte \term{Hybridnomina}\is{Hybridnomen} (\fw{hybrid nouns}), also
Substantive\is{Substantiv}, deren Genus und Sexus nicht übereinstimmen und bei
denen in der pronominalen Referenz typischerweise Variation zwischen
formaler\is{Kongruenz!formale} und semantischer
Kongruenz\is{Kongruenz!semantische} herrscht, also zum Beispiel, wenn \fw{das
Mädchen} im weiteren Verlauf formal mit \fw{es} oder semantisch mit \fw{sie}
pronominal referenziert wird. \citet{klein2022} unterscheidet darüber hinaus
zwischen Hybridnomina im engeren Sinn und Epikoina\is{Epikoina}.

Bei Hybridnomina\is{Hybridnomen} im engeren Sinn liegt ein Konflikt
zwischen Genus und Sexus auf der lexikalischen Ebene vor
\autocite[145]{klein2022}. In diesem Zusammenhang ist das
historisch\is{Diachronie} langlebige Wort für \fw{Frau} als formales Neutrum
mit weiblicher \isi{Denotation} hervorzuheben: Im Alt- und
Mittelhochdeutschen\il{Mittelhochdeutsch} lautet es \norm{wīb} beziehungsweise
\norm{wīp}. Auch im modernen Deutschen\il{Neuhochdeutsch} findet sich \fw{Weib}
noch mindestens bis ins 20.~Jahrhundert in nicht-pejorativer Verwendung
\autocite[166]{fleischer2012}. Im Fall von \fw{Mädchen} wird
konventionell\is{Konvention} eine junge weibliche Person bezeichnet, das
neutrale Genus wird dem Lexem durch das Diminutivsuffix \norm{-chen} formal
zugewiesen. Auch bei Bezeichnungen wie \fw{die Wache} oder, pejorativ, \fw{die
Type} und \fw{die Tunte}, kommt es zu Diskrepanzen in der Lexik, weil diese
formal zu den Feminina zählen, jedoch in ihrer Semantik gewöhnlich mit Männern
assoziiert werden \autocite[vgl.~auch][67--68]{panther2009}.

\phantomsection
\label{phsec:epikoina}
\is{Epikoina|(}

Eine andere Spielart von Hybridnomina liegt bei \term{Epikoina} vor, also im
Grunde sexus\-indifferenten\is{generischer Gebrauch} Lexemen wie \fw{Person},
das zwar formal feminin ist, sich aber der Wortbedeutung nach auf einen
Menschen unabhängig von dessen sozialem oder
biologischem\is{Geschlecht!biologisches} Geschlecht bezieht, oder \fw{Mensch},
das zwar formal maskulin ist, für das aber prinzipiell dasselbe gilt.%
%
	\footnote{Siehe aber die Ergebnisse der Fragebogenstudie zu
		\fw{Mensch} und \fw{Person} von \citet[174--183]{klein2022}, bei denen
		die Probandinnen und Probanden in Einklang mit dem Genus des jeweiligen
		Lexems \norm{Mensch} eher mit Männern und \fw{Person} eher mit Frauen
		assoziiert haben.}
%
In diesen Fällen liegt in grammatisch spezifischen Kontexten ein Konflikt auf
der referenziellen Ebene vor, weil das jeweils geltende \isi{Sexusmerkmal} von
der bezeichneten Person abhängt und damit vom konkreten
Gebrauchskontext\is{Pragmatik} bestimmt wird, also pragmatisch
\autocite[142--144]{klein2022}.

Ein im Deutschen\il{Neuhochdeutsch} langlebiges Epikoinon ist \fw{Kind},
althochdeutsch\il{Althochdeutsch} \norm{kind} und
mittelhochdeutsch\il{Mittelhochdeutsch} \norm{kint}, das formal neutral ist,
sich im konkreten Fall aber sowohl auf Mädchen als auch auf Jungen beziehen
kann, genereller auf nicht\babelhyphen{hard}erwachsene Menschen.%
%
	\footnote{Darüber hinaus kann sich \fw{Kind} im übertragenen Sinn auch auf
		erwachsene Menschen beziehen, zum Beispiel, wenn damit die
		Abhängigkeitsbeziehung zu den Eltern oder metaphorisch zu Gott betont
		wird \autocite[s.\,v.~\textit{kint}]{lexer:mhdhwb}. In manchen Fällen
		kann \fw{Kind} auch besonders auf Mädchen oder junge Frauen bezogen
		sein \autocites[808--816]{drw7}[s.\,v.~\textit{Kind}]{duden-online}.}
%
Insbesondere \fw{Kind} verdient Aufmerksamkeit, da die Variation zwischen
\feat{Neutrum} in Einklang mit dem formalen \isi{Genusmerkmal} sowie
\feat{maskulin} oder \feat{feminin} in Übereinstimmung mit dem semantischen
\isi{Sexusmerkmal}\is{Merkmale!semantische} der bezeichneten Person auch von
deren Alter abhängt, insofern Kinder häufig als weniger agentiv und damit
weniger belebt als Erwachsene konzeptualisiert werden
\autocites[196]{comrie1989}[258--259]{birkenesfleischer2022}[151]{klein2022}.

In dieser Hinsicht stellt \citet[172--174]{klein2022} eine Hierarchie innerhalb
der Epikoina auf, da Abstufungen in der Belebtheit und damit einhergehend der
von Menschen wahrgenommenen Ausgeprägtheit sexueller Unterschiede zwischen
Individuen der bezeichneten Gruppe einen Einfluss darauf haben, ob ein
Epikoinon beim pronominalen Bezug eher mit formaler\is{Kongruenz!formale} oder
semantischer Kongruenz\is{Kongruenz!semantische} auftritt
\autocite[vgl.~auch][74--83]{kotthoffnuebling2018}. Sogenannte generische
Maskulina\is{generischer Gebrauch}, also (vermeintlich)
geschlechtsneutrale\is{Genusindifferenz} Bezeichnungen wie \fw{Bürger} und
\fw{Gast}, mittelhochdeutsch\il{Mittelhochdeutsch} \norm{burgǟre} und
\norm{gast}, bewegen sich darüber hinaus im Spannungsfeld zwischen einer
spezifischen, referenziellen \isi{Lesart} mit Bezug auf einen
bestimmten\is{Definitheit} Mann einerseits und einer unspezifischen,
nicht-referenziellen Lesart mit Bezug auf irgendein Individuum aus der
benannten Gruppe und die damit assoziierten stereotypen
Geschlechterrollen\is{Geschlechterrolle} andererseits
\autocites[91--122]{kotthoffnuebling2018}[159--160, 179--180]{klein2022}. Wo
zum Beispiel \norm{burgǟre} anders als in \REF{ex:nongenmasc} keine
spezifische Referenz besitzt, gehe ich im Folgenden
vereinfachend\is{Komplexitätsreduktion} davon aus, dass damit allgemein die
Stadt bewohnende Menschen gemeint sind.

\begin{exe}
\ex \label{ex:nongenmasc}
	\gll Hermann der Marſpurrer \textelp{} vnd Cvͦnrat der Turſte
			\textelp{} die {vor genemten} burgare bede von V́berlingen \\
		Hermann[\textsc{nom.sg.m}] der Marspurrer {} und
			Konrad[\textsc{nom.sg.m}] der Turste {} die vorgenannten
			Bürger[\textsc{nom.pl.m}] beid-\textsc{nom.pl.m} von Überlingen \\
	\trans \wdef{Hermann der Marspurrer \textelp{} und Konrad der Turste
		\textelp{}, die vorgenannten beiden Bürger von Überlingen} \\
		\parencites%
			(Nr.~N~288, Überlingen, Bodenseekr., 1285)%
			[223,19--21]{cao5}
\end{exe}

\is{Epikoina|)}
\is{Körperteil|(}

Körperteile stellen eine weitere Übergangskategorie dar, indem sie zwar
Ding\-charakter haben, aber Teil eines Organismus sind.\is{Ambiguität} Im
speziellen Kontext mit \norm{bėide} liegen in der Belegsammlung zu dieser
Arbeit keine Fälle von kombiniertem pronominalen Bezug auf Körperteile vor,
sondern lediglich der Fall in \REF{ex:bodyparts_attr}, bei dem \lit{baide}
\wdef{beide} attributiv\is{Attribut} \lit{hênde} \wdef{Hände} quantifiziert,
sowie mehrere Fälle mit \norm{bėide} als \isi{Konjunktion}, von denen einer
exemplarisch in \REF{ex:bodyparts_conj} zitiert wird. In diesem Beispiel wird
die \isi{Koordination} von \lit{ader} \wdef{Ader} und \lit{lit} \wdef{Glied}
mit \lit{baidiv} \wdef{beide} eingeführt, jedoch im weiteren Verlauf kein
kombinierter Bezug auf beide Konjunkte durch Pronomina hergestellt.

\begin{exe}
\ex \begin{xlist}
	\ex \label{ex:bodyparts_attr}
		\gll Si wand ír baide hênde \\
			sie wand ihr beid-\textsc{acc.pl.f.st} Hand-\textsc{acc.pl.f} \\
		\trans \wdef{Sie wand ihre beiden Hände.}
			(%
				K:~6rb,19; vgl. abweichend
				A1:~4va,26--27;
				H:~5va,4;
				M:~7vb,8;
				B1:~4vb,49;
				P:~9ra,5;
				\KC:~V.~913; \cite[98]{schroeder1895}% setno 1065
			)

	\ex \label{ex:bodyparts_conj}
		\gll baidiv ader unt lit. \\
			beide Ader[\textsc{acc.sg.f}] und Glied[\textsc{acc.sg.m/n}] \\
		\trans \wdef{sowohl Ader als auch Glied}
			(%
				A1:~32rb,31; vgl.
				H:~44va,15;
				M:~56vb,21;
				C1:~39rb,3;
				K:~45ra,32;
				Z:~147va,20;
				\KC:~V.~7468; \cite[218]{schroeder1895}% setno 2018
			)
\end{xlist}
\end{exe}

Der Belebtheitsstatus\is{Belebtheit} von Körperteilen ist zumindest im modernen
Standarddeutschen beachtenswert, da sich deren Possessivsyntax von anderen,
weniger belebten Dingen abhebt, wie die Beispiele in
\REF{ex:stdgeralien} zeigen. Dieses Verhalten lässt sich als Unterschied in der
\isi{Alienabilität} fassen \autocites{nichols1988}[17--18]{heine1997}. Eine
Suche im \REM{}\il{Mittelhochdeutsch} nach dem inalienablen Typ mit
\textsc{pronomen}\is{Personalpronomen}~(\textsc{dat})~--
\textsc{artikel}\is{Artikel}~(\textsc{acc})~--
\textsc{substantiv}\is{Substantiv} hat keine Ergebnisse geliefert.

	\begin{exe}
	\protectedex{%
	\ex \begin{xlist}
		\ex[]{Ich wasche mein Auto.
				\jambox{\hphantom{*\,}$\feat{[+ alienabel]}$}}
		\ex[]{Ich wasche mir die Hände.
				\jambox{\hphantom{*\,}$\feat{[- alienabel]}$}}
		\ex[*]{Ich wasche mir das Auto.
				\jambox{*\,$\feat{[- alienabel]}$}}
		\end{xlist}
		\label{ex:stdgeralien}%
	}
	\end{exe}

Neben formalen\is{Merkmale!grammatische} und semantischen
Merkmalen\is{Merkmale!semantische} kann auch die \isi{Distanz} zwischen einem
\isi{Controller} und seinem \isi{Target} einen Einfluss auf die Wahl der
Kongruenzform haben, wenn mehrere Möglichkeiten bestehen. Unter dieser Prämisse
untersuchen \citet{panther2009} und \citet{binanzeretal2022} semantische
Kongruenz für die moderne Standardsprache des Deutschen\il{Neuhochdeutsch},
\citet{fleischer2012} geht diesem Aspekt in diachroner Perspektive nach. Diesen
Studien ist die Erkenntnis gemein, dass Kongruenz \fw{ad
sensum}\is{Kongruenz!semantische} zum einen mit wachsender linearer
Distanz\is{Distanz!lineare} zwischen Controller und Target tendenziell zunimmt,
zum anderen verschiedene Arten von Pronomina und anaphorischen
Ausdrücken\is{Anapher} eine unterschiedlich hohe Affinität zur semantischen
Kongruenz\is{Kongruenz!semantische} aufweisen, sich also auch die syntaktische
\isi{Domäne} von Controller und Target auf die Wahl der Kongruenzform
auswirkt \autocites%
	(Abschnitte~\ref{sec:ctrltarg} und~\ref{sec:kongrhier})%
	[84--85]{panther2009}[197--199]{fleischer2012}%
.

\is{Annotation|(}
Da insgesamt also damit zu rechnen ist, dass die Semantik und der
pragmatische\is{Pragmatik} Kontext einer Personenbezeichnung einen starken
Einfluss auf die \isi{Genuskongruenz} von anaphorischen\is{Anapher}
Targets\is{Target} haben, möchte ich im Folgenden
formale\is{Merkmale!grammatische} und semantische
Geschlechtsmerkmale\is{Sexusmerkmal}\is{Merkmale!semantische} in der Annotation
von Beispielen konsequent abbilden. Zu diesem Zweck erweitere ich die
herkömmliche Annotation des Genus um einen Index, der den Sexus kodiert, wie in
\tabref{tab:gendsex} angegeben.

\begin{table}[h]
\centering
\caption{grammatisches und semantisches Geschlecht}
\begin{tabular}{l l l l} % @{\hspace{4\tabcolsep}}
\lsptoprule
\mc{2}{c}{Genus} & \mc{2}{c}{Sexus} \\ % \smallskip

\cmidrule(r){1-2}
\cmidrule(l){3-4}

\textsc{m} & maskulin & \SM & männlich     \\
\textsc{f} & feminin  & \SF & weiblich     \\
\textsc{n} & neutral  & \SI & unbelebt     \\
           &          & \SA & unspezifisch \\
           &          & \SX & unbekannt    \\
\lspbottomrule
\end{tabular}
\label{tab:gendsex}
\end{table}

Dabei wird für die Bezeichnung der formalen Kategorie Genus die lateinische
Terminologie verwendet, für die Bezeichnung der semantischen Kategorie Sexus
die deutsche. Zum Beispiel wird \norm{wīp} \wdef{Frau} als \NeutF\ annotiert,
\norm{kint} \wdef{Kind} je nachdem, ob das Geschlecht des Kindes im
Kontext\is{Pragmatik} bekannt ist, als \NeutM, \NeutF\ oder \NeutX. Wenn sich
im Kontext eines Beispiels \norm{ėrbe} \wdef{Erbe} oder \norm{burgǟre}
\wdef{Bürger} nicht auf eine bestimmte\is{Definitheit} Person beziehen lassen,
also unspezifische Referenz vorliegt, werden sie mit \MascA\ annotiert. Im Fall
von \norm{schuech} \wdef{Schuh} als unbelebtem Substantiv\is{Substantiv} steht
\MascI, bei \norm{minne} \wdef{Liebe} als Abstraktum \FemI. Da es sich bei
Körperteilen nicht um Menschen handelt, wurden auch diese unter \isi{Vorbehalt}
als Inanimata\is{Inanimata} gewertet. So wird \norm{hant} \wdef{Hand} mit
\FemI{} annotiert, es sei denn, damit ist metonymisch ein Diener oder eine
Dienerin gemeint, in welchem Fall die Annotation entsprechend \FemM{} oder
\FemF{} lautet. Semantisch komplexe Bezeichnungen wie \norm{drīvaltichėit}
\wdef{Dreifaltigkeit} kommen im ausgewerteten Material nicht vor. \norm{Got}
\wdef{Gott} wurde gemäß seiner Bezeichnung als \norm{truhtīn} \wdef{Herrscher}
und \norm{hērre} \wdef{Herr} (z.\,B.\ \KC:~V.~8314, 13525; \cite[234,
323]{schroeder1895}) als \MascM\ gewertet.

Im modernen Standarddeutschen liegen darüber hinaus einzelne Wörter wie
\fw{Schild} oder \fw{Korpus} vor, die je nach Bedeutungskontext\is{Pragmatik}
unterschiedliche Genera besitzen: \fw{das Schild} (\NeutI) für die
Hinweistafel, \fw{der Schild} (\MascI) für den Schläge und Hiebe abwehrenden
Schirm; \fw{der Korpus} (\MascI) als Schallkörper eines Saiteninstruments,
\fw{das Korpus} (\NeutI) als strukturierte Sammlung von Texten oder
Belegstellen. Ferner ist neben standardsprachlichem \fw{die Butter} (\FemI)
regional auch \fw{der Butter} (\MascI) verbreitet
\autocite[s.\,v.~\textit{der/die Butter}]{elspassmoeller2003}. Ein
mittelhochdeutsches\il{Mittelhochdeutsch} Beispiel für
Substantive\is{Substantiv} mit bedeutungs\-unterscheidendem Genus ist \norm{dęr
tėil} (\MascI) \wdef{Anteil, Zugeteiltes, Eigentum} gegenüber \norm{daȥ tėil}
(\NeutI) \wdef{Teil von einem Ganzen, Stück, Seite, Abteilung}
\autocite[s.\,v.~\textit{teil}]{lexer:mhdhwb}.

Des Weiteren liegt in der mittelhochdeutschen\il{Mittelhochdeutsch} Periode
noch eine größere Zahl an Substantiven\is{Substantiv} vor, deren Genus bei
gleicher Bedeutung variabel belegt ist oder die ihr Genus im Lauf der Zeit
gewechselt haben \autocite[157--166]{ksw2}, zum Beispiel \norm{die} (\FemI)
oder \norm{daȥ jārƶīt} (\NeutI) \wdef{Jahrestag} sowie
mittel\-hoch\-deutsch\il{Mittelhochdeutsch} \fw{die wiƶƶe} (\FemI) \wdef{Wissen,
Verstand, Klugheit} \autocite[vgl.][s.\,v.~\textit{witze}]{lexer:mhdhwb}
gegenüber neu\-hoch\-deutsch\il{Neuhochdeutsch} \fw{der Witz} (\MascI). In
diesen Fällen wurde im Kontext der jeweiligen Text\-stelle nach Hinweisen
gesucht, mit welchem Genus das jeweilige Lexem verwendet wird, soweit dies
möglich war.

\is{Körperteil|)}
\is{Annotation|)}
\is{Belebtheit|)}
\is{Sexus|)}

%%%%%%%%%%%%%%%%%%%%%%%%%%%%%%%%%%%%%%%%%%%%%%%%%%%%%%%%%%%%%%%%%%%%%%%%%%%%%%%

\section{Genusresolution}
\label{sec:gendres}
\is{Genusresolution|(}

Ein Problem für Kongruenz entsteht dann, wenn zum Beispiel durch
\isi{Koordination} von zwei Nominalen (Substantiven\is{Substantiv} oder
\isi{Personalpronomen}) unterschiedliche grammatische
Merkmale\is{Merkmale!grammatische} derselben Kategorie (Genus\is{Genusmerkmal},
\isi{Numerus}) pro \isi{Controller} vorliegen. Die Frage in diesem Fall ist,
wie ein \isi{Target}, das für die jeweilige Kategorie flektiert, mit diesen
divergierenden Merkmalen umgeht. Eine Möglichkeit der Konfliktlösung besteht
darin, lediglich mit dem nächsten Konjunkt zu kongruieren
\autocites[\fw{closest conjunct agreement}\is{Kongruenz!partielle};
vgl.][179--180]{corbett1983}[168--170]{corbett2006}, wie in dem Schema in
\figref{fig:ccagraphic} gezeigt.

\begin{figure}
\centering
\begin{tikzpicture}[
		baseline=(clbl.base),
		box/.style={
			draw,
			minimum height=2.5em,
		% 	font=\itshape,
		},
		wordbox/.style={
			draw,
			minimum height=1.75em,
		% 	font=\itshape,
		},
		lbl/.style={
			minimum height=1.5em,
			font={\smaller\scshape}
		},
		every node/.style={anchor=base}
	]

	\node [wordbox,                                ] (1) {A};
	\node [wordbox, base right=1ex of 1, draw=white] (2) {und};
	\node [wordbox, base right=1ex of 2            ] (3) {B};
	\node [         base right=3ex of 3            ] (4) {\dots};
	\node [wordbox, base right=3ex of 4, draw=white] (5) {C\tsub{B}};

	\node (C) [box, rectangle, fit={(1) (2) (3)}, thick] {};
	\node (T) [box, rectangle, fit=(5)                 ] {};

	\node (clbl) [lbl, above=.5ex of C] {controller};
	\node (tlbl) [lbl, above=.5ex of T] {target};

	\draw [-, draw=white, double=black, ultra thick] (3) -- ++(south:2em) -| (T);
	\node [lbl, below=1.25em of 4.south] {kongruenz};
\end{tikzpicture}

\begin{tikzpicture}[
		baseline=(clbl.base),
		box/.style={
			draw,
			minimum height=2.5em,
		% 	font=\itshape,
		},
		wordbox/.style={
			draw,
			minimum height=1.75em,
		% 	font=\itshape,
		},
		lbl/.style={
			minimum height=1.5em,
			font={\smaller\scshape}
		},
		every node/.style={anchor=base}
	]

	\node [wordbox,                                ] (1) {B};
	\node [wordbox, base right=1ex of 1, draw=white] (2) {und};
	\node [wordbox, base right=1ex of 2            ] (3) {A};
	\node [         base right=3ex of 3            ] (4) {\dots};
	\node [wordbox, base right=3ex of 4, draw=white] (5) {C\tsub{A}};

	\node (C) [box, rectangle, fit={(1) (2) (3)}, thick] {};
	\node (T) [box, rectangle, fit=(5)                 ] {};

	\node (clbl) [lbl, above=.5ex of C] {controller};
	\node (tlbl) [lbl, above=.5ex of T] {target};

	\draw [-, draw=white, double=black, ultra thick] (3) -- ++(south:2em) -| (T);
	\node [lbl, below=1.25em of 4.south] {kongruenz};
\end{tikzpicture}
\caption{Partielle Kongruenz mit dem nächsten Konjunkt}
\label{fig:ccagraphic}
\end{figure}

\citet[169]{corbett2006} gibt im Rahmen von \isi{Genuskongruenz} das Beispiel
in \REF{ex:cca} aus dem Swahili\il{Swahili}. Dort treten \fw{kiti} \wdef{Stuhl}
und \fw{mguu wa meza} \wdef{Tischbein} auf, die jeweils einer unterschiedlichen
Nominalklasse angehören: \fw{kiti} \wdef{Stuhl} gehört zu den Klassen 7 (Sg.)
und 8 (Pl.), \fw{mguu} \wdef{Bein} zu den Klassen 3 (Sg.) und 4 (Pl.). Das
gemeinsame \isi{Target} \fw{u-/kimevunjika} \wdef{zerbrochen} zeigt Kongruenz
in der Nominalklasse nur mit demjenigen Konjunkt, das ihm am nächsten steht (3
bzw.~7).

\begin{exe}
\ex \label{ex:cca}
	\langinfo%
		{Swahili}%
		{}%
		{\cite[45]{bokamba1985} in \cite[169]{corbett2006}}
	\begin{xlist}
	\ex \label{ex:cca_1}
		\gll ki-ti na m-guu wa meza u-me-vunjika \\
			\textsc{cl7}-chair and \textsc{cl3}-leg of table
			\textsc{cl3}-\textsc{prf}-broken \\
		\trans \wdef{the chair and the leg of the table are broken}

	\ex \label{ex:cca_2}
		\gll m-guu wa meza na ki-ti ki-me-vunjika \\
			\textsc{cl3}-leg of table and \textsc{cl7}-chair
			\textsc{cl7}-\textsc{prf}-broken \\
		\trans \wdef{the leg of the table and the chair are broken}
	\end{xlist}
\end{exe}

In der \isi{Genuskongruenz} der mittelhochdeutschen\il{Mittelhochdeutsch}
Sprachperiode wird in den oberdeutschen\il{Oberdeutsch}
Schreibdialekten\is{Schreibdialekt} in solchen Fällen dagegen die
\isi{Kombinationsstrategie} angewandt, wie in \figref{fig:combgraphic}
verdeutlicht
\autocites[vgl.][312]{grimm1890}[329]{grimm1898}[39--41]{behaghel1928}[187--189]{dal2014}:
Wenn für das \isi{Target} keine Form der Deklinationsendung vorliegt, die im
Plural genusindifferent\is{Genusindifferenz} ist, muss der Unterschied zwischen
den einzelnen Controllern\is{Controller} durch die Kombination ihrer
Personenmerkmale\is{Personenmerkmal} aufgelöst werden, damit das Target
regelgemäß kongruieren kann
\autocites[vgl.][182--193]{corbett1983}[269--306]{corbett1991}[243--263]{corbett2006}.

\begin{figure}
\centering
\begin{tikzpicture}[
		baseline=(clbl.base),
		box/.style={
			draw,
			minimum height=2.5em,
		},
		wordbox/.style={
			draw,
			minimum height=1.75em,
		},
		lbl/.style={
			minimum height=1.5em,
			mynodefont,
		},
		every node/.style={anchor=base}
	]

	\node [wordbox,                                ] (1) {A};
	\node [wordbox, base right=1ex of 1, draw=white] (2) {und};
	\node [wordbox, base right=1ex of 2            ] (3) {B};
	\node [         base right=3ex of 3            ] (4) {\dots};
	\node [wordbox, base right=3ex of 4, draw=white] (5) {C\tsub{{A+B}}};

	\node (C) [box, rectangle, fit={(1) (2) (3)}, thick] {};
	\node (T) [box, rectangle, fit=(5)                 ] {};

	\node (clbl) [lbl, above=.5ex of C] {controller};
	\node (tlbl) [lbl, above=.5ex of T] {target};

	\draw [-, thick] (C) -- ++(south:2em) -| (T);
	\node [lbl, below=1em of $(2.south)!0.5!(5.south)$] {kongruenz};
\end{tikzpicture}
\caption{Kongruenz mit beiden Konjunkten}
\label{fig:combgraphic}
\end{figure}

Dies ist in den oberdeutschen\il{Oberdeutsch} Dialekten der Fall im Nom./Akk.\
Pl.\ der starken Adjektiv\-deklination\is{Adjektivdeklination}
\autocite[182]{ksw2}. Im Großteil der ausgewerteten Belege steht daher aufgrund
der Kombination der semantischen
Personenmerkmale\is{Personenmerkmal}\is{Merkmale!semantische} die neutrale Form
\norm{bėidiu} \wdef{beide (\NeutMF)}, wie in \REF{ex:gendres} in Bezug auf
\lit{Rvͦdiger} (\MascM) und seine \lit{hovſfrowe} \wdef{Ehefrau
(\FemF)} illustriert, auch wenn auf formaler Ebene \norm{bėide} \wdef{beide
(\textsc{m+f})} ansonsten sowohl für Maskulina als für auch Feminina gilt.

\begin{exe}
\ex \label{ex:gendres}
	\gll swenne aber her Rvͦdiger vnd ſin
			hovſfrowe bediv niht enſint\\
		so=wenn aber Herr Rüdiger[\textsc{nom.sg.\MascM}] und sein
			Ehefrau[\textsc{nom.sg.\FemF}] beide-\textsc{nom.pl.\NeutMF.st}
			nicht \textsc{neg}=sind \\
	\trans \wdef{Wenn aber Herr Rüdiger und seine Ehefrau beide
		nicht \textins{mehr} sind}
		\parencites%
			(Nr.~3262, Regensburg, 1299)%
			[425,13--14]{cao4}
\end{exe}

Eine direkte Entsprechung von \REF{ex:gendres} mit umgekehrter
Reihenfolge\is{Abfolge} der grammatischen Merkmale der Konjunkte ist im
ausgewerteten Material nicht vorhanden, sondern lediglich der Fall mit
\norm{bėide} als \isi{Modifikator} eines
Personalpronomens\is{Personalpronomen}, das sich seinerseits auf zwei
\isi{Controller} in dieser Abfolge bezieht. \citet[96, 145]{askedal1973} gibt
die in \REF{ex:askfmbeidiu} zitierten Stellen an, die sich mit den leicht
anders formulierten Fällen im ausgewerteten Material decken\is{Validierung},
sodass davon auszugehen ist, dass diese keine Sonderfälle darstellen. Da bei
beiden Abfolgen von \MascM\ und \FemF\ regelmäßig dieselbe Form \norm{bėidiu}
erscheint, ist nicht davon auszugehen, dass in \REF{ex:gendres} und
\REF{ex:askfmbeidiu} partielle Kongruenz\is{Kongruenz!partielle}
wie in \REF{ex:cca} auftritt und es sich bei \norm{bėidiu} um eine Form des
Nom.~Sg.~F.\ handelt, die ebenfalls auf \norm{-iu} endet.

\begin{exe}
	\ex \label{ex:askfmbeidiu}
		\begin{xlist}
		\ex \gll vnſer moͮter iwer frîundin. \\
				unser Mutter[\textsc{nom.sg.\FemF}] euer Freundin \\
		\sn \gll unde vnſer vater ſint beidiv tot. \\
				und unser Vater[\textsc{nom.sg.\MascM}] sind
				beide-\textsc{nom.pl.\NeutMF.st} tot \\
			\trans \wdef{Unsere Mutter, eure Freundin, und unser Vater sind
				beide tot.}
				(%
					\iai{Gottfried von Straßburg}, \tit{Tristan}: V.~18644--18645
					nach München, Bayerische Staatsbibl., Cgm~51: 96rb,23--24;
					% [\cite[1286]{hsc}];
					vgl.~\cite[259]{maroldschroeder1969}%
					% (= S. 259 → in ⁵2004: S. 313)
				)
			\label{ex:askfmbeidiu_1}
	
		\ex \gll {Condwir amvrſ} daz wip din \\
				Condwiramurs[\textsc{nom.sg.\FemF}] das Frau dein \\
		\sn \gll vn̄ din ſvn Loherangrin \\
				und dein Sohn[\textsc{nom.sg.\MascM}] Loherangrin \\
		\sn \gll sint beidiv mit dir dar benant \\
				sind beide-\textsc{nom.pl.\NeutMF} mit dir da benannt \\
		\trans \wdef{Condwiramurs, deine Frau, und dein Sohn Loherangrin
			sind beide mit dir dahin berufen.}
			(%
				\iai{Wolfram von Eschenbach}, \tit{Parzival}: 781,17--19
				nach St.~Gallen, Stiftsbibl., Cod.~Sang.~857: 275a,32--34;
				% [\cite[1211]{hsc}];
				vgl.~\cite[785]{knechtschirok2003}% (= S. 785)
			)
			\label{ex:askfmbeidiu_2}
	\end{xlist}
\end{exe}

\is{Genus|)}

%%%%%%%%%%%%%%%%%%%%%%%%%%%%%%%%%%%%%%%%%%%%%%%%%%%%%%%%%%%%%%%%%%%%%%%%%%%%%%%

\section{Kongruenz in der Lexical-Functional Grammar}
\label{sec:lfgkongr}
\is{Lexical-Functional Grammar|(}

Die Lexical-Functional Grammar (LFG;
\cites{kaplanbresnan1982,bresnan2001,bresnanetal2016,dalrymple2023}; \cites[zur
Einführung z.\,B.][]{buttking2015}[223--246]{mueller2023}) ist eine
beschränkungsbasierte\is{Beschränkung}, lexikalisch orientierte,
nicht-trans\allowbreak{}formationale Grammatiktheorie und basiert auf der
Intuition, dass trotz aller Variation in der Syntax und Morphologie die
unterliegende \isi{funktionale Struktur} (F-Struktur,
\fw{f-structure}) verschiedener Sprachen weitgehend gleich ist
\autocite[42]{bresnanetal2016}.
% %
% 	\footnote{\foreignblockcquote{english}[42]{bresnanetal2016}{The principle
% 		of universality states that \emph{internal structures are largely
% 		invariant across languages.} The formal model of internal structure in
% 		\textsc{lfg} is the f-structure, which stands for \q{functional
% 		structure}}.%
% 	}
% %
Die sprachliche Grundlage, auf der die LFG entwickelt wurde und
weiterentwickelt wird, ist typologisch sehr breit aufgestellt.
\citet[223--224]{mueller2023} gibt einen exemplarischen Überblick über die
Sprachen, für die mehr oder weniger ausführliche Beschreibungen in diesem
Theorieframework vorliegen \autocites[zum modernen Standarddeutschen
vgl.][]{berman2003}. Die Wahl der theoretischen Anbindung ist der
Praktikabilität geschuldet. Die LFG operiert vornehmlich mit grammatischen
Merkmalen\is{Merkmale!grammatische} im Sinne \posscite{corbett2012} im
Zusammenspiel mit der \isi{Konstituentenstruktur} und ist damit bestens
geeignet, um Kongruenz als morphosyntaktisches Phänomen zu analysieren.
% ; der Annotationsformalismus und die
% Darstellung von Merkmalsstrukturen sind vergleichsweise unkompliziert.

Hauptbestandteil des Formalismus der LFG ist die \isi{funktionale Struktur},
die eine der als parallel gedachten Repräsentationsebenen darstellt, auf denen
Sprache operiert \autocite[840--844]{buttking2015}.%
%
	\footnote{\citet[862--865]{buttking2015} nennen darüber hinaus zum Beispiel
		noch die
		A-Struktur (Argumente),\is{Argumentstruktur}
		C-Struktur (Konstituenten),
		I-Struktur (Information),
		M-Struktur (Morphologie),
		P-Struktur (Prosodie)
		und die
		S-Struktur (Semantik).
	}
%
Die F-Struktur\is{funktionale Struktur} wird in
Attribut-Wert-Matrizen\is{Attribut-Wert-Matrix} dargestellt
(\fw{attribute-value matrices};
\cites[vgl.][44--45]{bresnanetal2016}[207--221]{mueller2023}), die
Informationen strukturiert präsentieren, vergleiche \figref{fig:avm}. Als
Attribute in der linken Spalte dienen Funktionen oder
Merkmale\is{Merkmale!grammatische} wie \feat{Subjekt} (\feat{subj}),
\feat{Objekt} (\feat{obj}), \feat{Prädikator} (\feat{pred}) oder \feat{Numerus}
(\feat{num}). Zugehörige Werte in der rechten Spalte können grammatische
Eigenschaften wie \feat{Plural} (\feat{pl}), Funktionsmorpheme wie
\textit{und}, Wortformen wie \wdef{Baum}, A-Struk\-turen\is{Argumentstruktur}
wie \astruct{schreiben}{\ups{subj}, \ups{obj}} sowie F-Strukturen oder Mengen
von F-Struk\-turen sein. Jedem Attribut sind dabei ein oder mehrere eindeutige
Werte zugewiesen; Funktionen (\feat{subj}, \feat{obj}, etc.) können nur einfach
instanziiert werden \autocite[vgl.][44--58]{bresnanetal2016}.

\begin{figure}
\centering
	{% \avmsetup{attributes=\scshape}
	\avm{[
		vorname		& \wdef{Max} \\
		nachname	& \wdef{Meier} \\
		geburtstag	& 10.10.1985 \\
		vater		& [
			vorname		& \wdef{Peter} \\
			nachname	& \wdef{Meier} \\
			geburtstag	& 10.5.1960 \\
			vater		& \dots \\
			mutter		& \dots \\
		] \\
		mutter		& \dots \\
	]}}
\caption{Attribut-Wert-Matrix \autocite[nach][209]{mueller2023}}
\label{fig:avm}
\end{figure}

Darüber hinaus wird eine \isi{Konstituentenstruktur} (C-Struktur,
\fw{c-struc\-ture}) angesetzt, die parallel zur F-Struktur\is{funktionale
Struktur} den Aufbau von syntaktischen Konstituenten abbildet und eine Variante
der \xbar{X}-Theorie\is{Xʹ-Theorie} \autocites{chomsky1970,jackendoff1977}
darstellt. Projektionen innerhalb des Baumes können dabei funktional annotiert
sein; nicht-verzweigende \xbar{X}-Kategorien werden als überflüssig angesehen
und in der Regel weggelassen. Dies wird in \figref{fig:cfstruct} illustriert,
wo kein struktureller Unterschied zwischen \isi{Spezifikator}, \isi{Adjunkt}
und \isi{Komplement} gemacht wird, weil die \isi{Annotation} disambiguiert.

\begin{figure}
\begin{forest} align text
[CP\mysn{cfstruct_CP}
	[{\anno[\pass{subj}]{NP\mysn{cfstruct_NP}}}
	 	[\anno{\xhead{N}\mysn{cfstruct_N}}
	 		[Lola]
	 	]
	]
	[\anno{\xhead{C}\mysn{cfstruct_C}}
		[rennt]
	]
]
\node [avmcontainer] {\avm{%
	\tikzmark{cfstruct_f}$f$: [
		pred	& \astruct{rennen}{\ups{subj}} \\
		tense	& \textsc{prs} \\
		%
		subj	&	\tikzmark{cfstruct_subj}$g$: [
			pred	& \wdef{Lola} \\
			pers	& 3 \\
			num		& \textsc{sg} \\
			case	& \textsc{nom} \\
		]
	]%
}}; 
\end{forest}
\begin{tikzpicture}[remember picture, overlay]
	\draw [myarrow]
		([yshift=.5ex]{pic cs:cfstruct_CP})
		to [out=east, in=west]
		([yshift=.5ex]{pic cs:cfstruct_f});
	\draw [myarrow]
		([yshift=.5ex]{pic cs:cfstruct_C})
		to [out=east, in=west]
		([yshift=.5ex]{pic cs:cfstruct_f});
	\draw [myarrow]
		({pic cs:cfstruct_NP})
		to [out=south east, in=240]
		({pic cs:cfstruct_subj});
	\draw [myarrow]
		([yshift=.5ex]{pic cs:cfstruct_N})
		to [out=east, in=240]
		({pic cs:cfstruct_subj});
\end{tikzpicture}
\caption{Analyse des Satzes \fw{Lola rennt}}
\label{fig:cfstruct}
\end{figure}

In \figref{fig:cfstruct} wird die Korrespondenz zwischen der
C-Struktur\is{Konstituentenstruktur} und der F-Struktur\is{funktionale
Struktur} des Satzes \fw{Lola rennt} dargestellt. Der Satz betsteht aus einer
NP\is{Nominalphrase} mit dem \isi{Kopf} \fw{Lola}, die das Subjekt
(\feat{subj}) des Satzes darstellt, und einer \isi{Komplementiererphrase}
(\fw{complementizer phrase}, CP), die das Subjekt dominiert und das
(finite) Verb \fw{rennt} zum Kopf hat. Ein Pfeil nach unten (↓) bezeichnet
in der Annotation\is{Annotation} den jeweiligen Knoten selbst, ein Pfeil nach
oben (↑) den darüber liegenden. \q{\pass{subj}} ist daher so zu verstehen, dass
die Informationen in dem damit annotierten Knoten die Eigenschaften der
Subjekt\-funk\-tion des darüberliegenden Knotens darstellen. Die optionalen
Pfeile zwischen C- und F-Struktur\is{funktionale Struktur} verdeutlichen,
welcher Knoten seine Informationen in welcher F-Struktur (hier: $f$ oder $g$)
ablegt. Die F-Struktur wird vom Verb \fw{rennen} prädiziert (\feat{pred}) und
nimmt als Argument ein Subjekt \ups{subj}. Dessen grammatische Eigenschaften
bilden den Wert des Attributs \feat{Subjekt} (\feat{subj}).

Dabei wird ein wichtiges Merkmal der LFG sichtbar: Informationen des jeweiligen
Knotens werden mit dem nächsthöheren Knoten vereinigt. Das bedeutet, dass
Attributen stets eindeutige und miteinander kompatible Werte zugewiesen sind,
um einen kohärenten\is{Kohärenz} Ausdruck zu erzeugen
\autocite[vgl.][43--54]{bresnanetal2016}. Anders als im eingangs vorgestellten
morphologischen Ansatz entsteht Kongruenz hier durch den Abgleich von
grammatischen Merkmalen\is{Merkmale!grammatische}.
\citet[7]{wechslerzlatic2003} zufolge wird Kongruenz also nicht als gerichteter
Prozess verstanden, bei dem Merkmalsbündel kopiert oder verschoben werden,
sondern als das Zusammenspiel zweier Elemente, die partielle Informationen über
dasselbe linguistische Objekt spezifizieren. Kongruenz resultiert daraus, dass
die von den beiden Quellen gebotenen Informationen kompatibel miteinander sein
müssen.
% %
% 	\footnote{\foreignblockcquote{english}[7]{wechslerzlatic2003}{%
% 		\textins*{A}greement is not viewed as a directional process of copying
% 		or moving feature bundles, but rather as two elements specifying
% 		partial information about a single linguistic object. Agreement results
% 		from the fact that this information coming from two sources must be
% 		compatible}.
% 	}
% %
Dies wird in \figref{fig:lolamorphlex} verdeutlicht, wo die
Lexikoneinträge\is{Lexikon} für \fw{Lola} und \fw{rennt} vereinfacht
wiedergegeben werden.

\begin{figure}
\begin{tabular}[t]{@{} l @{\hspace{2em}} c @{\hspace{2em}} l}
	$Lola$
	&	N
	&	\begin{tabular}[t]{l l l}
			\ups{pred}	& =	& \wdef{Lola} \\
			\ups{pers}	& =	& 3 \\
			\ups{num}	& =	& \textsc{sg} \\
		\end{tabular}
	\medskip \\

	$rennt$
	&	C
	&	\begin{tabular}[t]{l l l}
			\ups{pred}		& = 	& \astruct{rennen}{\ups{subj}} \\
			\ups{subj pers}	& \req	& 3 \\
			\ups{subj num}	& \req	& \textsc{sg} \\
		\end{tabular}
\end{tabular}
\caption{Morpholexikalische Definition von \fw{Lola} und \fw{rennt}}
\label{fig:lolamorphlex}
\end{figure}

\fw{Lola} bildet als Substantiv\is{Substantiv} einen \isi{Diskursanker} und
definiert (=) die angegebenen Merk\-male: \feat{3.~Person} und \feat{Singular}.
Die finite Verbform \fw{rennt} bedingt (\req) ihrem Lexikoneintrag\is{Lexikon}
gemäß eine Subjekt-NP\is{Nominalphrase}, die die
Personenmerkmale\is{Personenmerkmal} \feat{3.~Person} und \feat{Singular}
aufweist. Da die Personenmerkmale von \fw{Lola} diese
Bedingung\is{Beschränkung} (\fw{constraint}) erfüllen, kommt es zur Kongruenz
zwischen Subjekt und Verb \autocite[vgl.][59]{bresnanetal2016}.

In Anlehnung an die Beispiele \REF{ex:coordidx} und \REF{ex:engartdiscong}
werden in den Abbildungen~\ref{fig:lfgcoord_1} und~\ref{fig:lfgcoord_2} noch
einmal zur Illustration die jeweils grammatisch akzeptablen Varianten vor dem
Hintergrund des gewählten Theorieframeworks gezeigt. Die
Annotation\is{Annotation} der koordinierten NPs\is{Koordination} folgt
\citet{peterson2004}. In \figref{fig:lfgcoord_1} erfordert das Verb
\fw{spielen} ein Subjekt mit den Personenmerkmalen\is{Personenmerkmal}
\feat{3.~Person} und \feat{Plural} in Übereinstimmung mit dem kombinierten
\isi{Index} der Subjekt-NP\is{Nominalphrase} in $g$; zwei Singulare ergeben
zusammen einen Plural.%
%
	\footnote{Mit \q{g-und} wird das gruppenbildende \fw{und} bezeichnet, bei
		dem sich die Konjunkte auf jeweils unterschiedliche Indizes $i, j$
		beziehen \autocite[382--383]{dalrymple2001}.}

\begin{figure}
\begin{forest} shorter edges, narrower nodes, align text
[CP\mysn{lfgcoord1_CP}
	[{\anno[\pass{subj}]{NP\mysn{lfgcoord1_NP1}}}
		[{\anno[\updownelem]{\xhead{N}}}
			[Jan]
		]
		[Conj
			[und]
		]
		[{\anno[\updownelem]{\xhead{N}}}
			[Markus]
		]
	]
	[\anno{\xbar{C}}
		[\anno{\xhead{C}}
			[spielen, name=spielen, minimum width=4em]
		]
		[\anno{VP\mysn{lfgcoord1_VP}}
			[{\anno[\pass{obj}]{NP\mysn{lfgcoord1_NP2}}}
				[\anno{\xhead{N}}
					[Fußball]
				]
			]
		]
	]
]
%
% Annotation des Knotens zu breit, als dass die AVM noch hinpasst
\node at (spielen) [below=1ex] {
	\smaller[2]\upshape\tabcolsep=.5ex%
	\begin{tabular}[t]{@{} l l l @{}}
		\ups{subj pers}	& \req & 3 \\
		\ups{subj num}	& \req & \textsc{pl} \\
	\end{tabular}%
};
%
\node [avmcontainer=1em, font=\footnotesize] {
	\avm{%
	\tikzmark{lfgcoord1_f}$f$: [
		pred	& \astruct{spielen}{\ups{subj}, \ups{obj}} \\
		tense	& \textsc{prs} \\
		%
		subj	& \tikzmark{lfgcoord1_g}$g$: [
			conj & \textit{g-und} \\
			%	
			\mc{2}{l}{%
				\{
					[
						pred	&	\wdef{Jan} \\
						index	&	[
							pers	& 3 \\
							gend	& \textsc{m} \\
							num		& \textsc{sg} % \\
						] \\
						case	& \textsc{nom} % \\
					]\smallskip\\
					%
					[
						pred	&	\wdef{Markus} \\
						index	&	[
							pers	& 3 \\
							gend	& \textsc{m} \\
							num		& \textsc{sg} % \\
						] \\
						case	& \textsc{nom} % \\
					]
				\}%
			} \\
			%
			index	&	[
				pers	& 3 \\
				num		& \textsc{pl} \\
			] \\
		] \\
		%
		obj	& \tikzmark{lfgcoord1_h}$h$: [
			pred	&	\wdef{Fußball} \\
			case	&	\textsc{acc} \\
		] \\
	]}
};
\end{forest}
\begin{tikzpicture}[remember picture, overlay]
	\draw [myarrow]
		([yshift=.5ex]{pic cs:lfgcoord1_CP})
		to [out=east, in=112]
		([yshift=.5ex]{pic cs:lfgcoord1_f});
	\draw [myarrow]
		([yshift=.5ex]{pic cs:lfgcoord1_VP})
		to [out=south east, in=west, min distance=0.5cm]
		([yshift=.5ex]{pic cs:lfgcoord1_f});
	\draw [myarrow]
		([yshift=.5ex]{pic cs:lfgcoord1_NP1})
		to [out=-12, in=north west]
		([yshift=.5ex]{pic cs:lfgcoord1_g});
	\draw [myarrow]
		([yshift=.5ex]{pic cs:lfgcoord1_NP2})
		to [out=east, in=north west]
		([yshift=.5ex]{pic cs:lfgcoord1_h});
\end{tikzpicture}
\caption{Analyse des Satzes \fw{Jan und Markus spielen Fußball}}
\label{fig:lfgcoord_1}
\end{figure}

\largerpage[-1]
Demgegenüber ist es innerhalb der in \figref{fig:lfgcoord_2} gezeigten
NP\is{Nominalphrase} notwendig, dass der De\-termi\-nierer \fw{diese} im
\isi{Numerus} mit dem \isi{Concord}-Merkmal jedes einzelnen der beiden
Konjunkte übereinstimmt, damit Kongruenz hergestellt werden kann und ein
kohärenter\is{Kohärenz}, grammatisch wohlgeformter Ausdruck entsteht.%
%
	\footnote{Im Deutschen\il{Neuhochdeutsch} kommen noch Kasus und Genus
	hinzu, die hier vereinfachend\is{Komplexitätsreduktion} weggelassen wurden.
	\citet{dalrymple2001} benutzt \feat{Spezifikator}\is{Spezifikator}
	(\feat{spec}) als grammatische Funktion bei der \isi{Annotation} des
	Determinierers\is{Determinierer}, \citet{bresnanetal2016} hingegen
	behandeln \feat{spec} als Merkmal. Beide folgen in ihren Beispielen
	der DP-Hypothese\is{Determiniererphrase} \autocite{chomsky1986}, für die
	sich u.\,a.\ auch \citet[9--26]{demske2001} ausspricht, insofern sie sie
	aus diachroner Sicht als sinnvoll für die Analyse von NPs im
	Deutschen\il{Neuhochdeutsch} erachtet.}
%
\citet[91--94]{kingdalrymple2004} zufolge stellen Modifikatoren\is{Modifikator}
im Deutschen\il{Neuhochdeutsch} zusätzlich Bedingungen\is{Beschränkung} an
\isi{Index}-Merkmale.

\begin{figure}
	\begin{forest} narrower nodes, align text
	[{\anno[\pass{subj}]{DP\mysn{lfgcoord2_DP}}}
		[\anno{\xhead{D}}
			[diese, name=diese, minimum width=4em]
		]
		[{\anno{NP\mysn{lfgcoord2_NP}}}
		% 	[\anno{\xbar{N}}
				[{\anno[\updownelem]{\xhead{N}}}
					[Jungen]
				]
				[Conj
					[und]
				]
				[{\anno[\updownelem]{\xhead{N}}}
					[Mädchen]
				]
		% 	]
		]
	]
	%
	% Annotation des Knotens zu breit, als dass die AVM noch hinpasst
	\node at (diese) [below=1ex] {
		\smaller[2]\upshape\tabcolsep=.5ex%
		\begin{tabular}[t]{@{} l l l @{}}
			\ups{concord num}	& \req & \textsc{pl} \\
			\ups{index num}		& \req & \textsc{pl} \\
		\end{tabular}%
	};
	%
	\node [avmcontainer, font=\footnotesize] {
		\avm{%
		\tikzmark{lfgcoord2_f}$f$: [
			spec	& [
				pred	& \wdef{diese}
			] \\
			%
			conj	& \textit{g-und} \\
			%
			\mc{2}{l}{%
				\{
					[
						pred	& \wdef{Jungen} \\
						%
						concord	& [
							case	& \textsc{nom} \\
							num		& \textsc{pl} \\
							gend	& \textsc{m}
						] \\
						%
						index	& [
							pers	& 3 \\
							num		& \textsc{pl} \\
							gend	& \textsc{m}
						]
					]\smallskip\\
					%
					[
						pred	& \wdef{Mädchen} \\
						%
						concord	& [
							case	& \textsc{nom} \\
							num		& \textsc{pl} \\
							gend	& \textsc{f}
						] \\
						%
						index	& [
							pers	& 3 \\
							num		& \textsc{pl} \\
							gend	& \textsc{f}
						]
					]
				\}%
			} \smallskip \\
			%
			index	& [
				pers	& 3 \\
				num		& \textsc{pl} \\
			]
		]}
	};
	\end{forest}
	\begin{tikzpicture}[remember picture, overlay]
		\draw [myarrow]
			([yshift=.5ex]{pic cs:lfgcoord2_NP})
			to [out=east, in=west]
			([yshift=.5ex]{pic cs:lfgcoord2_f});
		\draw [myarrow]
			([yshift=.5ex]{pic cs:lfgcoord2_DP})
			-- ++(right:1em)
			to [out=east, in=west]
			([yshift=.5ex]{pic cs:lfgcoord2_f});
		% \draw [myarrow]
		% 	([yshift=.5ex]{pic cs:lfgcoord2_DP})
		% 	to [out=east, in=west]
		% 	([yshift=.5ex]{pic cs:lfgcoord2_g});
	\end{tikzpicture}
\caption{Analyse des Satzfragments \fw{diese Jungen und Mädchen}}
\label{fig:lfgcoord_2}
\end{figure}

Da der Plural des Demonstrativ\-pronomens im modernen
Deutschen\il{Neuhochdeutsch} keine Genusdifferenzierung aufweist, ist der
Ausdruck im Beispiel möglich. Wenn die Konjunkte im Singular stehen, können die
unterschiedlichen Genusmerkmale\is{Genusmerkmal} jedoch nicht vereinigt werden:
*\fw{dieser/s Junge und Mädchen}.

\is{Distanz!syntaktische|(}
Die unterschiedlichen syntaktischen Domänen\is{Domäne} nach
\citet[54]{corbett2006}, die in \sectref{sec:ctrltarg} vorgestellt wurden,
lassen sich in der Terminologie der LFG als Abhängigkeiten zwischen immer
weniger lokalen\is{Lokalität} F-Strukturen\is{funktionale Struktur} fassen,
insofern (lexikalische) C-Strukturköpfe\is{Konstituentenstruktur} (\xhead{X})
mit F-Strukturköpfen (Prädikatoren, \textsc{pred}) korrespondieren
\autocite[117]{bresnanetal2016}. Dies wird anhand von
\figref{fig:fstructdomains} verdeutlicht.

Dort ist \fw{braune} als adjektivisches
Kongruenztarget\is{Kongruenz!attributive}\is{Target} ein \isi{Adjunkt} von
\fw{Katze} und steht damit in derselben F-Struktur\is{funktionale Struktur} $g$
wie sein \isi{Controller} (gleiche NP\is{Nominalphrase}). Auf \fw{Katze} ist
ebenfalls der Possessor \fw{Janas} in $h$ bezogen. \fw{Janas} und \fw{Katze}
befinden sich zwar in verschiedenen F-Strukturen ($g$ und $h$), allerdings ist
der Possessor in $h$ Teil der Subjektfunktion $g$ (gleicher Satzteil). Das
Reflexivpronomen\is{Reflexivpronomen} \fw{sich} in $j$ bildet das direkte
Objekt von \fw{putzen}. Da es sich bei dem Objekt um ein Reflexivpronomen
handelt, ist es mit dem Subjekt in $g$ koindiziert\is{Koindizierung} und stellt
ebenfalls dessen Kongruenztarget dar, allerdings steht \fw{sich} als Objekt in
einer anderen F-Struktur als das Subjekt ($j$ gegenüber $g$). Da aber Subjekt
und Objekt vom gleichen Verb \fw{putzen} abhängen, befinden sich beide dennoch
innerhalb derselben F-Struktur $f$ (gleicher Satz).

\begin{figure}
	\begin{forest} shorter edges, narrower nodes, align text
	[CP\mysn{fsdom_CP}
		[{\anno[\pass{subj}]{DP\mysn{fsdom_DPs}}}
			[{\anno[\pass{poss}]{DP\mysn{fsdom_DPp}}}
				% [\anno{\xhead{D}}
					[Janas, roof]
				% ]
			]
			[\anno{NP\mysn{fsdom_NP}}
				[{\anno[\elem{adj}]{AP\mysn{fsdom_AP}}}
					[\anno{\xhead{A}}
						[braune]
					]
				]
				[\anno{\xhead{N}}
					[Katze]
				]
			]
		]
		[\anno{\xbar{C}}
			[\anno{\xhead{C}}
				[putzt]
			]
			[\anno{VP\mysn{fsdom_VP}}
				[{\anno[\pass{obj}]{DP\mysn{fsdom_DPr}}}
					[\anno{\xhead{D}}
						[sich]
					]
				]
			]
		]
	]
	%
	\node [avmcontainer=3ex, font=\footnotesize] {
		\avm{%
			\tikzmark{fsdom_f}$f$: [
				subj	& \tikzmark{fsdom_subj}$g$: [
					poss	& \tikzmark{fsdom_poss}$h$: [
						pred	& \wdef{Jana} \\
						case	& \textsc{gen} \\
					] \\
					%
					adj		& \tikzmark{fsdom_adj}\{
						[
							pred	& \wdef{braun}
						]
					\} \\
					%
					pred	& \astruct{Katze}{\ups{poss}} \\
					case	& \textsc{nom} \\
					pers	& 3 \\
					gend	& \textsc{f} \\
					num		& \textsc{sg} \\
				]~$i$\smallskip \\
				%
				pred	& \astruct{putzen}{\ups{subj}, \ups{obj}} \smallskip \\
				%
				obj	& \tikzmark{fsdom_obj}$j$: [
					prontype	& \textit{refl} \\
					pred		& \textit{pro} \\
					case		& \textsc{acc} \\
				]~$i$ \\
			]}
	};
	\end{forest}
	\begin{tikzpicture}[remember picture, overlay]
		\draw [myarrow]
			([yshift=.5ex]{pic cs:fsdom_CP})
				to [out=east, in=north west]
			([yshift=.5ex]{pic cs:fsdom_f});
		\draw [myarrow]
			([yshift=.5ex]{pic cs:fsdom_VP})
			to [out=12, in=south west]
			([yshift=.5ex]{pic cs:fsdom_f});
		\draw [myarrow, out looseness=1.6]
			({pic cs:fsdom_DPs})
			to [out=-18, in=north west]
			([yshift=.5ex]{pic cs:fsdom_subj});
		\draw [myarrow, in looseness=1.25]
			({pic cs:fsdom_DPp})
			to [out=-24, in=230]
			({pic cs:fsdom_poss});
		\draw [myarrow]
			({pic cs:fsdom_AP})
			to [out=south east, in=south west]
			({pic cs:fsdom_adj});
		\draw [myarrow]
			({pic cs:fsdom_DPr})
			to [out=south east, in=south west]
			({pic cs:fsdom_obj});
	\end{tikzpicture}
	\caption{Analyse des Satzes \fw{Janas braune Katze putzt sich}}
	\label{fig:fstructdomains}
\end{figure}

\is{Distanz!syntaktische|)}
\is{Genusresolution|)}
\is{Lexical-Functional Grammar|)}

%%%%%%%%%%%%%%%%%%%%%%%%%%%%%%%%%%%%%%%%%%%%%%%%%%%%%%%%%%%%%%%%%%%%%%%%%%%%%%%

\section{Gefloatete Quantoren}
\label{sec:floatquant}
\is{gefloateter Quantor|(}
\is{Quantifizierung|(}

Wenn im Rahmen dieser Arbeit der Terminus \term{Quantor} verwendet wird, dann
werden darunter zunächst solche \isi{Determinierer} eines nominalen Ausdrucks
verstanden, die etwas über seine Anzahl oder Menge aussagen, wie zum Beispiel
\fw{alle, manche, einige} oder auch \fw{beide}. Dabei ist zu beachten, dass
sich verschiedene Quantoren syntaktisch unterschiedlich verhalten
\autocites[27--28]{pittner1995}[11--12]{haspelmath1997}. Das in dieser Arbeit
insbesondere behandelte \fw{beide} bezeichnet eine Menge von genau
zwei\is{Dual} im Kontext eindeutig identifizierbaren Elementen einer Gruppe und
ist damit definit\is{Definitheit}
\autocites[vgl.][265--268]{lyons1999}[307]{keenan2006}.

\phantomsection
\label{phsec:quantdist}
Als Modifikatoren\is{Modifikator} von Substantiven\is{Substantiv} (und in
manchen Fällen auch Pronomina, zum Beispiel \fw{sie beide}, \fw{alle diese})
unterliegen Quantoren wie \norm{bėide} im
Mittelhochdeutschen\il{Mittelhochdeutsch} wie auch \fw{beide} im
Neuhochdeutschen\il{Neuhochdeutsch} der \isi{Adjektivdeklination}. Das
bedeutet, sie kongruieren mit ihrem nominalen \isi{Controller}
formal\is{Kongruenz!formale} in den Kategorien \isi{Kasus} (\feat{case}),
\isi{Genus} (\feat{gend}) und \isi{Numerus} (\feat{num}) über ein
fusionales Suffix, das diese Eigenschaften kombiniert kodiert
\autocites(vgl.~auch die Abschnitte~\ref{sec:ctrltarg} und
\ref{sec:lfgkongr})[181--184]{ksw2}[772]{woellstein2022}. Daneben ist bei
\fw{beide} im Deutschen\il{Neuhochdeutsch} zu beachten, dass es zusammen mit
einem Definit-\is{Definitheit} oder Possessivartikel\is{Artikel} auftreten kann
\REF{ex:beidedet_2}, anders als Definit- und Possessivartikel selbst, die
gewöhnlich in komplementärer Distribution\is{Distribution!komplementäre} zu
einander stehen (*\fw{das mein Buch}). Im Unterschied zu
Adjektiven\is{Adjektiv!attributiv} ist es andererseits aber auch nicht möglich,
in der syntaktischen\is{Distribution!syntaktische} Position von \fw{beide}
weitere Wörter vom gleichen Typ unterzubringen, weder durch Koordination
\REF{ex:beidedet_3} noch durch Reihung \REF{ex:beidedet_4}, wie es etwa bei
Adjektiven trotz Restriktionen bezüglich ihrer \isi{Abfolge} grundsätzlich
möglich ist. Im Unterschied zu Adjektiven können Quantoren vom Typ
\fw{alle} und \fw{beide} im Deutschen\il{Neuhochdeutsch} außerdem nicht
prädikativ gebraucht werden, wie aus \REF{ex:beidepred_2} deutlich wird
(vgl.~auch \cite[181, Fußnote~1]{merchant1996}).

\begin{exe}
\label{ex:beidedet}
\ex \begin{xlist}
	\ex[]{beide/wenige Bücher}
		\label{ex:beidedet_1}
	\ex[]{die beiden/wenigen Bücher}
		\label{ex:beidedet_2}
	\ex[*]{die beiden und wenigen Bücher}
		\label{ex:beidedet_3}
	\ex[*]{die beiden, wenigen Bücher}
		\label{ex:beidedet_4}
\end{xlist}

\ex \begin{xlist}
	\ex glückliche/beide/viele Hunde
		\label{ex:beidepred_1}
	\ex Die Hunde sind glücklich/*beide/\tsup{?}viele.
		\label{ex:beidepred_2}
\end{xlist}
\end{exe}

Aufgrund dieser Unterschiede sowohl zu \q{typischen}
Determinierern\is{Determinierer} als auch zu Adjektiven\is{Adjektiv!attributiv}
wird in Einklang mit der im folgenden zu besprechenden Literatur angenommen,
dass Quantoren wie \fw{beide} eine gesonderte funktionale Kategorie \xhead{Q}
nominalen Typs repräsentieren \autocite[vgl.~auch][]{dipper2005}. Anders als
zum Beispiel \fw{all} und das englische\il{Englisch} \fw{both} \wdef{beide},
die ganz links in der \isi{Nominalgruppe} stehen (\fw{all die \dots}, \fw{both
the \dots}), reiht sich die \isi{Quantorenphrase} (QP) mit \fw{beide} im
Standarddeutschen gewöhnlich zwischen DP\is{Determiniererphrase} und
NP\is{Nominalphrase} ein, wie in \figref{fig:nomstack} dargestellt (vgl.~auch
\cite[44--45 mit Fußnote 30]{lyons1999}). Als funktionale nominale Kategorie
ist \xhead{Q} wie \xhead{D} aus Sicht der LFG\is{Lexical-Functional Grammar}
ein \term{Ko-Kopf}\is{Kopf} von \xhead{N} (\fw{cohead};
\cite[124]{bresnanetal2016}). Dies bedeutet, dass die Information in DP und QP
mit NP in derselben F-Struktur\is{funktionale Struktur} vereinigt wird.
Zusätzlich wird hier angenommen, dass der Quantor zumindest bei seiner
Verwendung innerhalb der DP ein Merkmal\is{Merkmale!grammatische}
\feat{quant} definiert, da er aufgrund des distributiven Unterschieds zu
Adjektiven nicht als Teil der Menge der Adjunkte\is{Adjunkt} (\feat{adj})
gezählt werden sollte.

\begin{figure}
	\begin{forest} narrower nodes, align text
	[DP\mysn{nomstack_DP}
		[\anno{\xhead{D}}
			[die]
		]
		[\anno{QP\mysn{nomstack_QP}}
			[\anno{\xhead{Q}}
				[beiden]
			]
			[\anno{NP\mysn{nomstack_NP}}
				[{\anno[\elem{adj}]{AP}}
					[roten]
				]
				[\anno{\xhead{N}}
					[Bücher]
				]
			]
		]
	]
	%
	\node [avmcontainer] {
		\avm{%
		\tikzmark{nomstack_f}[
			def	& $+$ \\
			%
			quant	& 	[
				pred	& \wdef{beide} \\
			] \\
			%
			adj	& \{
				[
					pred	& \wdef{rot}
				]
			\} \\
			%
			pred	& \wdef{Buch} \\
			case	& \textsc{nom} \\
			pers	& 3 \\
			gend	& \textsc{n} \\
			num		& \textsc{pl} \\
		]}
	};
	\end{forest}
	\begin{tikzpicture}[remember picture, overlay]
		\draw [myarrow]
			([yshift=.5ex]{pic cs:nomstack_DP})
			to [out=east, in=west]
			([yshift=.5ex]{pic cs:nomstack_f});
		\draw [myarrow]
			([yshift=.5ex]{pic cs:nomstack_QP})
			to [out=east, in=west]
			([yshift=.5ex]{pic cs:nomstack_f});
		\draw [myarrow]
			([yshift=.5ex]{pic cs:nomstack_NP})
			to [out=east, in=west]
			([yshift=.5ex]{pic cs:nomstack_f});
	\end{tikzpicture}
	\caption{Analyse des Satzfragments \fw{die beiden roten Bücher}}
	\label{fig:nomstack}
\end{figure}

\citet[790]{schwartz2000} hebt darüber hinaus hervor, dass Quantoren wie
\fw{jeder}, \fw{manche} und \fw{alle} auch eine pronominale Funktion ausüben
können \autocite[vgl.~auch][11--12]{haspelmath1997}. In diesen Fällen steht der
Quantor nicht attributiv\is{Attribut} zu einem Substantiv wie in
\figref{fig:nomstack}, sondern fungiert als eigenständiger \isi{Kopf} eines
Satzglieds \REF{ex:quantpron}.

\begin{exe}
\ex \label{ex:quantpron}
\begin{xlist}
	\ex \parbox[t]{.55\linewidth}{%
			Anna\tsub{i} liest gerne und Paul\tsub{j} auch.\\
			Beiden\tsub{{i+j}} habe ich ein Buch geschenkt.%
		}\jambox{(indirektes Objekt)}
		\label{ex:quantpron_1}
	\ex \parbox[t]{.55\linewidth}{%
			Ich brauche einen neuen Mixer\tsub{i}.\\
			Er\tsub{i}/Dieser\tsub{i}/Meiner\tsub{i} ist kaputt.%
		}\jambox{(Subjekt)}
		\label{ex:quantpron_2}
\end{xlist}
\end{exe}

In \REF{ex:quantpron_1} bezieht sich \fw{beiden} auf \fw{Anna} und \fw{Paul},
steht jedoch in einem anderen Satz. Während \norm{Anna und Paul} das Subjekt
von \norm{lesen} bilden, stellt \fw{beiden} das indirekte Objekt von
\fw{geschenkt} dar. Dabei ist zu beachten, dass \fw{beiden} einen
bestimmten\is{Definitheit} Bezugskontext\is{Pragmatik} benötigt. Das Paar, das
der Quantor referenziert, muss also durch den sprachlichen Kontext eindeutig
identifizierbar sein
\autocites[vgl.~z.\,B.][274]{lyons1999}[788]{schwartz2000}[983]{janssen2004}.
Genauso verhalten sich die Pronomina \fw{er} (\isi{Personalpronomen}),
\fw{dieser} (\isi{Demonstrativpronomen}) und \fw{meiner}
(\isi{Possessivpronomen}) in \REF{ex:quantpron_2}, insofern auch sie
definit\is{Definitheit} sind und voraussetzen, dass ein eindeutiger Bezug
hergestellt werden kann \autocites[vgl.][145--148]{lyons1999}. Dieser wird im
Beispiel durch den \fw{Mixer} im vorangehenden Satz gegeben.

Ein weiteres Charakteristikum von Quantoren wie \fw{alle},
\fw{beide}, \fw{einige} oder \fw{viele} ist, dass sie innerhalb desselben
Satzes getrennt von ihrer \isi{Nominalgruppe} stehen können, wie in
\REF{ex:floatsubj} mit Bezug auf das Subjekt gezeigt. Die Versionen mit
\isi{Distanzstellung} in (b) und (c) sind jeweils markiert\is{Markiertheit},
insofern das Augenmerk auf der Menge liegt. Allerdings müssen auch hier
mehrere, oberflächlich ähnliche Konstruktionen unterschieden werden
\autocites[27--28]{pittner1995}[65--67]{fanselowcavar2002}.

\begin{exe}
\ex \label{ex:floatsubj}
\begin{xlist}
	\ex \label{ex:floatsubj_1}
		{\ob}\tsub{QP}~Alle {\ob}\tsub{NP}~Kinder{\cb}{\cb} mögen Schokolade.
	\ex \label{ex:floatsubj_2}
		{\ob}\tsub{DP}~Die Kinder{\cb}\tsub{i} mögen {\ob}\tsub{QP}~alle{\cb}\tsub{i}
		Schokolade.
	\ex \label{ex:floatsubj_3}
		{\ob}\tsub{QP}~Alle{\cb}\tsub{i} mögen {\ob}\tsub{DP}~sie/*die
		Kinder{\cb}\tsub{i} Schokolade.
\end{xlist}
\end{exe}

\citet{sportiche1988} beschäftigt sich mit der Syntax des Allquantors \fw{tous}
\wdef{alle} im Französischen\il{Französisch}. Er argumentiert, dass
\term{gefloatete Quantoren}~-- Quantoren, die getrennt von ihrer Bezugs-NP
rechts vom Verb stehen~-- keine Adverbien\is{Adverb} sind, wie zuvor
vorgeschlagen wurde, sondern in einem anaphorischen Verhältnis\is{Anapher} zu
der NP stehen, die sie modifizieren \autocite[428--433]{sportiche1988}. Ferner
werde nicht der Quantor nach rechts verschoben (\q{gefloatet}), sondern die
Subjekt-NP aus ihrer Position hinter dem Quantor aus der \isi{Verbphrase} (VP)
extrahiert, also nach vorn gezogen. Dabei werde eine \isi{Spur} (\fw{trace}; t)
zurückgelassen, was die anaphorische Relation zwischen Quantor und
Quantifiziertem erkläre \autocite[432--433]{sportiche1988}. Eigentlich handelt
es sich also um \term{Stranding}\is{Stranding}~-- das Zurücklassen von
syntaktischem Material an der Stelle seiner ursprünglichen Generierung.

\citet{shlonsky1991} untersucht dies am modernen Hebräischen (\ili{Ivrit}) und
bestätigt\is{Validierung} weitgehend \citeauthor{sportiche1988}s Erkenntnisse.
Aufgrund der Daten aus dem \ili{Ivrit} modifiziert er \posscite{sportiche1988}
Hypothese über die Konstituenz der Konstruktion dahingehend, dass Quantor und
Quantifiziertes eine \isi{Quantorenphrase} mit dem Quantor als \isi{Kopf} und
einer Determiniererphrase\is{Determiniererphrase} (DP) als \isi{Komplement}
bilden. \citet{merchant1996} überträgt die Erkenntnisse von
\citet{sportiche1988} und \citet{shlonsky1991} auf das
Deutsche\il{Neuhochdeutsch} und nimmt weitere Verfeinerungen ihrer Hypothese
bezüglich möglicher Positionen des gestrandeten\is{Stranding} Quantors vor. Für
das Englische\il{Englisch} \fw{all} \wdef{alle} mit Bezug auf das Subjekt gibt
\citet{merchant1996} grob die \isi{Konstituentenstruktur} in
\figref{fig:qfgg} an (die Pfeile wurden zur Verdeutlichung der postulierten
Transformation\is{Government and Binding} hinzugefügt).

\begin{figure}
	\begin{forest} shorter edges
	[IP
		[DP\tsub{i}
			[{the boys},roof,name=DP]
		]
		[\xbar{I}
			[\xhead{I}
				[have]
			]
			[VP
				[QP
					[t′\tsub{i},name=t2]
					[\xbar{Q}
						[\xhead{Q}
							[all]
						]
						[t\tsub{i},name=t1]
					]
				]
				[\xbar{V}
					[\xhead{V}
						[seen]
					]
					[DP
						[{the film},roof]
					]
				]
			]
		]
	]
	\draw [-latex, gray, looseness=1.5]
		(t1) to [out=south west, in=south] (t2);
	\draw [-latex, gray]
		(t2) to [out=west, in=south] (DP);
	\end{forest}
	\caption{Analyse des Satzes \fw{The boys have all seen the film}
	\autocite[nach][180]{merchant1996}}
	\label{fig:qfgg}
\end{figure}

\citet{sportiche1988}, \citet{shlonsky1991} und \citet{merchant1996}
beschränken sich sämtlich auf den Quantor \fw{alle}, nur \citet{pittner1995}
geht darüber hinaus und untersucht auch das ähnlich funktionierende \fw{beide};
nur diese beiden fasst sie im Rahmen ähnlicher Konstruktionen als \q{echte}
gefloatete Quantoren auf. Was darüber hinaus \citeauthor{shlonsky1991},
\citeauthor{pittner1995} und \citeauthor{merchant1996} eint, ist, dass sie alle
den Unterschied in der \isi{Lesart} zwischen der Version mit
\isi{Kontaktstellung} und der mit \isi{Distanzstellung} behandeln. Der
Unterschied wird in den Abbildungen~\ref{fig:siebeidefhrd_1}
und~\ref{fig:siebeidefhrd_2} den Beispielen bei \citet[30--31]{pittner1995}
folgend illustriert.

\begin{figure}
\begin{tikzpicture}[baseline=(A.base)]
	\node [draw, rectangle] (A) at (0,1) {A};
	\node [draw, rectangle] (B) at (2,1) {B};
	\node [draw, rectangle] (fhrd1) at (0,0) {Fahrrad};
	\node [draw, rectangle] (fhrd2) at (2,0) {Fahrrad};
	\draw [-latex] (A) -- (fhrd1);
	\draw [-latex] (B) -- (fhrd2);
\end{tikzpicture}
\\
\caption{Sie haben beide ein Fahrrad}
\label{fig:siebeidefhrd_1}
\end{figure}

\begin{figure}
\begin{tikzpicture}[baseline=(A.base)]
	\node [draw, rectangle] (A) at (0,1) {A};
	\node [draw, rectangle] (B) at (4,1) {B};
	\node [draw, rectangle, gray] (fhrd1) at (0,0) {Fahrrad};
	\node [draw, rectangle      ] (fhrd)  at (2,0) {Fahrrad};
	\node [draw, rectangle, gray] (fhrd2) at (4,0) {Fahrrad};
	\draw [-latex] (A) to [out=east, in=north] (fhrd);
	\draw [-latex] (B) to [out=west, in=north] (fhrd);
	\draw [-latex, gray] (A) -- (fhrd1);
	\draw [-latex, gray] (B) -- (fhrd2);
\end{tikzpicture}
\caption{Sie beide haben ein Fahrrad}
\label{fig:siebeidefhrd_2}
\end{figure}

Während der Satz in \figref{fig:siebeidefhrd_1} mit gefloatetem Quantor
ausschließlich die \isi{Lesart} zulässt, bei der beide Parteien jeweils ein
Fahrrad besitzen (distributive Lesart), lässt das Beispiel in
\figref{fig:siebeidefhrd_2} zusätzlich die Interpretation zu, dass beide
Parteien gemeinsam ein Fahrrad besitzen (kollektive Lesart). Zusätzliche
Betonung des Quantors kann in diesem Fall zwar zur Disambiguierung beitragen
(\emph{Sie \textsc{beide} haben ein Fahrrad} = distributiv), ist aber bei der
rein textlichen Wiedergabe von Sprache ohne spezielle Markierung nicht
nachvollziehbar.

\phantomsection
\label{phsec:hebrqf}
Die bisher referierten Aufsätze diskutieren gefloatete Quantoren im Rahmen von
Government and Binding\is{Government and Binding} \autocite{chomsky1981}. Wie
könnte eine Adaption in die LFG\is{Lexical-Functional Grammar} aussehen? Diesen
Versuch unternimmt \citet{spector2009} in Rückgriff auf \citet{shlonsky1991}
für das \ili{Ivrit}, siehe \REF{ex:hebrqf} und \figref{fig:hebrqf}. Anders als
im Deutschen\il{Neuhochdeutsch} oder im Französischen\il{Französisch}
kongruiert der gefloatete Quantor (\fw{kol} \wdef{alle}) dort nicht bloß,
sondern beinhaltet ein klitisches Pronomen, das mit dem Quantifizierten
koindiziert\is{Koindizierung} ist.%
%
	\footnote{Im Sinne der
		Grammatikalisierungs\-theorie\is{Grammatikalisierung} nach
		\citet[44]{lehmann2015} stellt Klitisierung eine Vorstufe von Kongruenz
		dar.%
	}

\begin{exe}
\ex \label{ex:hebrqf}
	\langinfo%
		{Ivrit}
		{}
		{\cite[nach][522, 537]{spector2009}}\\
	\gll ha-yeladim halxu kulam la-yam \\
		\textsc{def}=children[\textsc{3pl.m}]\tsub{i} went all=\textsc{3pl.m}\tsub{i}
			to=sea.\textsc{def} \\
	\trans \wdef{The children went all to the sea.}
\end{exe}

\begin{figure}
	\begin{forest} narrower nodes, align text
	[IP\mysn{hebrqf_IP}
		[NP\mysn{hebrqf_NP}
			[\xhead{N}
				[ha-yeladim]
			]
		]
		[\xbar{I}
			[\xhead{I}
				[halxu]
			]
			[S%\mysn{hebrqf_S}
				[QP\mysn{hebrqf_QP}
					% [\xbar{Q}
						[\xhead{Q}
							[kulam]
						]
					% ]
				]
				[VP%\mysn{hebrqf_VP}
					[PP\mysn{hebrqf_PP}
						[la-yam, roof]
					]
				]
			]
		]
	]
	%
	\node [avmcontainer, font=\footnotesize] {
		\avm{\tikzmark{hebrqf_f}[
			top	& \tikzmark{hebrqf_top}[
				pred	& \wdef{child} \\
				def		& $+$ \\
				num		& \textsc{pl} \\
			]~$i$
			\smallskip \\
			%
			subj	& \tikzmark{hebrqf_subj}[
				pred	& \astruct{all}{\ups{obj}} \\
				obj	& [
					pred	& \textit{pro} \\
					pers	& 3 \\
					gend	& \textsc{m} \\
					num		& \textsc{pl} \\
				]~$i$ \\
			]
			\smallskip \\
			%
			pred	& \astruct{went}{\ups{subj}, \ups{\Oblq{dir}}}
			\smallskip \\
			%
			\Oblq{dir}	& \tikzmark{hebrqf_obl}[
				pred	& \astruct{to}{\ups{obj}} \\
				obj	& [
					pred	& \wdef{sea} \\
					def		& $+$ \\
					num		& \textsc{sg} \\
				] \\
			] \\
		]}
	};
	\end{forest}
	\begin{tikzpicture}[remember picture, overlay]
		\draw [myarrow]
			([yshift=.5ex]{pic cs:hebrqf_IP})
				to [out=east, in=west]
			([yshift=.5ex]{pic cs:hebrqf_f});
		% \draw [myarrow]
		% 	([yshift=.5ex]{pic cs:hebrqf_S})
		% 		to [out=east, in=west]
		% 	([yshift=.5ex]{pic cs:hebrqf_f});
		% \draw [myarrow]
		% 	([yshift=.5ex]{pic cs:hebrqf_VP})
		% 		to [out=east, in=west]
		% 	([yshift=.5ex]{pic cs:hebrqf_f});
		%
		\draw [myarrow]
			([yshift=.5ex]{pic cs:hebrqf_NP})
				to [out=-6, in=south west]
			([yshift=.5ex]{pic cs:hebrqf_top});
		\draw [myarrow]
			({pic cs:hebrqf_QP})
				to [out=-45, in=south west]
			([yshift=.5ex]{pic cs:hebrqf_subj});
		%
		\draw [myarrow]
			([yshift=.5ex]{pic cs:hebrqf_PP})
				to [out=east, in=south west]
			([yshift=.5ex]{pic cs:hebrqf_obl});
	\end{tikzpicture}
	\caption{Analyse des Satzes \fw{Ha-yeladim halxu kulam la-yam} \wdef{Die
		Kinder fuhren alle ans Meer}}
		% \foreignlanguage{hebrew}{הילדים הלכו כולם לים}
	\label{fig:hebrqf}
\end{figure}

\citet[533--534]{spector2009} kommt ähnlich wie \citet[29]{pittner1995} zu dem
Schluss, dass der gefloatete Quantor eine eigenständige NP\is{Nominalphrase}
darstellt. Die Topikalisierung\is{Topik} des Quantifizierten werde durch die
\isi{Distanzstellung} des Quantors angezeigt; die Konstruktion sei dadurch
besonders markiert\is{Markiertheit}. Beide NPs, Topik (\feat{top}) und
Subjekt (\feat{subj}), sind in \citeauthor{spector2009}s Analyse durch
\isi{Koindizierung} verbunden, werden aber nicht in der
F-Struktur\is{funktionale Struktur} miteinander vereinigt
\autocite[vgl.][99]{bresnanetal2016}.

Die Analysen von \citet{shlonsky1991} und \citet{spector2009} für das
\ili{Ivrit} könnten insofern für die Analyse des Deutschen\il{Neuhochdeutsch}
relevant sein, als \citet[179]{merchant1996} eine Ähnlichkeit zwischen dem
hebräischen\il{Ivrit} \fw{kol} \wdef{alle} und dem deutschen\il{Neuhochdeutsch}
\fw{alle} bezüglich des Auftretens von Flexion in Abhängigkeit der
Stellungsvariante\is{Abfolge} vermerkt: Die Kongruenzendung fehlt, wenn der
Quantor einer definiten\is{Definitheit} NP\is{Nominalphrase} vorangeht: \fw{all
die X} aber \fw{alle X}. Der Fall von \fw{die beiden X} und \fw{beide X} würde
eine separate Diskussion\is{Desiderat} benötigen. Nachfolgend möchte ich einige
vorläufige Überlegungen zur Modellierung der Situation im
Mittelhochdeutschen\il{Mittelhochdeutsch} anstellen.

Während im modernen Standarddeutschen \isi{Determinierer} und
Adjektive\is{Adjektiv!attributiv} normalerweise links vom \isi{Substantiv}
stehen, können Adjektive im Mittelhochdeutschen\il{Mittelhochdeutsch} auch
rechts davon auftreten \autocite[185--186, 237--243]{ksw2}, selten auch
Determinierer wie Possessivartikel\is{Artikel}, \norm{dehėin} \wdef{kein} und
\norm{bėide} \wdef{beide} \autocite[515--517, 551--552, 623--624]{ksw2}. Im
hier ausgewerteten Material liegen immerhin zwanzig Fälle im \tit{Corpus der
altdeutschen Originalurkunden} (\CAO) und drei in der \tit{Kaiserchronik} (\KC)
mit nachgestelltem \norm{bėide} hinter einem einfachen Plural-Substantiv vor
\REF{ex:beidepost_2}.

\begin{exe}
	\ex \label{ex:beidepost_2}
		\gll ſo man die iargezit beidú begat \\
			so man die Jahrestag[\textsc{acc.pl.\FemI}]
			beide-\textsc{acc.pl.\NeutI.st} begeht \\
		\trans \wdef{wenn man die Jahrestage beide begeht}
			\parencites%
				(Nr.~3331, Straßburg und Colmar, 1299)%
				[468,21--22]{cao4}
\end{exe}

Da die LFG\is{Lexical-Functional Grammar} ohne Transformationen auskommt (also
keine Verschiebungen von syntaktischen Einheiten auf einer abstrakten Ebene wie
in \figref{fig:qfgg} angenommen werden), bietet sich für die \isi{Lesart} mit
\isi{Kontaktstellung} in \figref{fig:beidepost_2cont} an, \norm{die}
(\xhead{D}) und \norm{bėide} (\xhead{Q}) als funktionale Ko-Köpfe von
\norm{jārƶīt} \wdef{Jahrestag} (\xhead{N}) zu behandeln.%
%
	\footnote{\isi{Artikel} und Quantor sind im ursprünglichen Beispiel nicht
		kongruent, insofern \norm{jārƶīt} entweder feminin oder neutral belegt
		ist \autocite[s.\,v.~\fw{jârzît}]{lexer:mhdhwb}. Der \isi{Artikel}
		\norm{die} flektiert maskulin-feminin, der Quantor \norm{bėidiu}
		dagegen neutral.}

\begin{figure}
	\begin{forest} shorter edges, narrower nodes, align text
	[VP
		[{\anno[\pass{obj}]{DP\mysn{beidepost2c_DP}}}
			[\anno{\xhead{D}}
				[die]
			]
			[\anno{QP\mysn{beidepost2c_QP}}
				[\anno{NP\mysn{beidepost2c_NP}}
					[\anno{\xhead{N}}
						[jārƶīt]
					]
				]
				[\anno{\xhead{Q}}
					[bėide]
				]
			]
		]
		[\anno{\xhead{V}}
			[begāt]
		]
	]
	%
	\node [avmcontainer] {
		\avm{%
		\tikzmark{beidepost2c_f}$f$: [
			obj	& \tikzmark{beidepost2c_obj}[
				def	& $+$ \\
				pred	& \wdef{Jahrestag} \\
				case	& \textsc{acc} \\
				num		& \textsc{pl} \\
				gend	& \textsc{f} \\
				anim	& $-$ \\
				quant	& [
					pred	& \wdef{beide} \\
				] \\
			] \\
			%
			pred	& \astruct{begehen}{\ups{subj}, \ups{obj}} \\
		]}
	};
	\end{forest}
	\begin{tikzpicture}[remember picture, overlay]
	\draw [myarrow]
		({pic cs:beidepost2c_NP})
		to [out=south east, in=south west]
		([yshift=.5ex]{pic cs:beidepost2c_obj});
	\draw [myarrow]
		([yshift=.5ex]{pic cs:beidepost2c_QP})
		to [out=east, in=south west]
		([yshift=.5ex]{pic cs:beidepost2c_obj});
	\draw [myarrow]
		([yshift=.5ex]{pic cs:beidepost2c_DP})
		to [out=east, in=south west]
		([yshift=.5ex]{pic cs:beidepost2c_obj});
	\end{tikzpicture}
	\captionsetup{width=.75\textwidth}
	\caption{Analyse des Satzfragments \norm{die jārƶīt bėide begāt}
		\wdef{die beiden Jahrestage begeht}}
	\label{fig:beidepost_2cont}
\end{figure}

Basierend auf der Analyse von \citet{spector2009} für das \ili{Ivrit} dient das
Schema in \figref{fig:beidepost_2dist} dagegen als Arbeits\-hypothese für die
Lesart mit gefloatetem Quantor. Parallel zu \feat{Subjekt} (\feat{subj}) und
\feat{Topik}\is{Topik} (\feat{top}) werden hier die Rollen \feat{Objekt}
(\feat{obj}) und \feat{Fokus}\is{Fokus} (\feat{foc}) für \norm{bėidiu} und die
dazugehörige DP\is{Determiniererphrase} angenommen. Die aufeinander bezogenen
Teile ($g$ und $h$) sind koindiziert\is{Koindizierung} ($i$), um die
anaphorische\is{Anapher} Funktion von \norm{bėide} zu erfassen.%
%
	\footnote{Die Annotationsform\is{Annotation}
		\q{\uncertain{$x$}{$a$}~\req~$v$} in \figref{fig:beidepost_2dist}
		bedeutet, dass entsprechend der Flexion der annotierten Wortform
		außerhalb des lokalen Funktionskerns (hier $h$) eine grammatische
		Funktion $x$ als \isi{Controller} existiert, die ein Attribut $a$
		besitzt, das den Wert $v$ für sein \isi{Target} voraussetzt
		(\fw{inside-out functional uncertainty};
		\cite[66--70]{bresnanetal2016}).}
%
Eine eingehendere Diskussion der Konstruktion müsste\is{Desiderat} auf die
Bindungsrelation\is{Bindung}, etwa im Vergleich zu
Reflexivpronomina\is{Reflexivpronomen}, und gegebenenfalls funktionale
Präzedenz\is{Abfolge} eingehen \autocite[vgl.][213,
254--285]{bresnanetal2016}. Auch für den in \REF{ex:floatsubj_3} illustrierten
Unterschied in der Akzeptabilität von \textit{Alle mögen sie/*die Kinder
Schokolade} könnte dies eine Rolle spielen.

\begin{figure}
	\begin{forest}
		shorter edges,
		narrower nodes,
		align text
	[VP
		[\anno{VP}
			[{\anno[\pass{foc}]{DP\mysn{beidepost2d_DP}}}
				[\anno{\xhead{D}}
					[die]
				]
				[\anno{NP\mysn{beidepost2d_NP}}
					[\anno{\xhead{N}}
						[jārƶīt]
					]
				]
			]
			[\anno{VP}
				[{\anno[\pass{obj}]{QP\mysn{beidepost2d_QP}}}
					[\anno{\xhead{Q}}
						[bėidiu, name=beidiu, minimum width=5em]
					]
				]
			]
		]
		[\anno{\xhead{V}}
			[begāt]
		]
	]
	%
	% Annotation des Knotens zu breit, als dass die AVM noch hinpasst
	\node at (beidiu) [below=1ex] {
		\smaller[2]\upshape\tabcolsep=.5ex%
		\begin{tabular}[t]{@{} l l l @{}}
			\uncertain{$x$}{case}	& \req & \textsc{acc} \\
			\uncertain{$x$}{num}	& \req & \textsc{pl} \\
			\uncertain{$x$}{gend}	& \req & \textsc{n} \\
			$\lor$ \uncertain{$x$}{anim} & \req & $-$ \\
		\end{tabular}%
	};
	%
	\node [avmcontainer] {
		\avm{%
		\tikzmark{beidepost2d_f}$f$: [
			foc	& \tikzmark{beidepost2d_foc}$g$: [
				def		& $+$ \\
				pred	& \wdef{Jahrestag} \\
				case	& \textsc{acc} \\
				num		& \textsc{pl} \\
				gend	& \textsc{f} \\
				anim	& $-$ \\
			]~$i$ \smallskip \\
			%
			obj	& \tikzmark{beidepost2d_obj}$h$: [
				pred	& \wdef{beide} \\
			]~$i$ \smallskip\\
			%
			pred	& \astruct{begehen}{\ups{subj}, \ups{obj}} \\
		]}
	};
	\end{forest}
	\begin{tikzpicture}[remember picture, overlay]
	\draw [myarrow]
		({pic cs:beidepost2d_NP})
		to [out=-30, in=south west]
		([yshift=.5ex]{pic cs:beidepost2d_foc});
	\draw [myarrow, in looseness=.7]
		([yshift=.5ex]{pic cs:beidepost2d_QP})
		to [out=east, in=215]
		([yshift=1ex]{pic cs:beidepost2d_obj});
	\draw [myarrow]
		({pic cs:beidepost2d_DP})
		to [out=-24, in=south west]
		([yshift=.5ex]{pic cs:beidepost2d_foc});
	\end{tikzpicture}
	\captionsetup{width=.79\textwidth}
	\caption{Analyse des Satzfragments \norm{die jārƶīt bėidiu begāt}
		 \wdef{die Jahrestage beide begeht}}
	\label{fig:beidepost_2dist}
\end{figure}

\phantomsection
\label{phsec:constambig}
Die Schemata in den Abbildungen~\ref{fig:beidepost_2cont}
und~\ref{fig:beidepost_2dist} zeigen außerdem, dass die Nachstellung des
Quantors tendenziell zu syntaktischer \isi{Ambiguität} führt. In den hier
gesammelten Daten tritt das Problem vor allem bei den zahlreichen Belegen für
\norm{si bėide} \wdef{sie beide} auf, wenn \norm{si} \wdef{sie} im
\isi{Mittelfeld} steht (vgl.~\figref{fig:sibeideambig}). \isi{Kontaktstellung}
ist also nicht gleich Kontaktstellung, da auch dann ein gefloateter Quantor
vorliegen kann. \citet[623--624]{ksw2} benennen dieses Problem ebenfalls und
zählen \blockquote{solche Fälle immer als attributiv\is{Attribut}
nachgestellt}, also wie in \figref{fig:sibeideambig} links dargestellt.

\begin{figure}
\begin{forest}
	shorter edges,
	align text,
	[\anno{VP}
		[{\anno[\pass{gf}]{QP}}
			[\anno{DP}
				[\anno{\xhead{D}}
					[si]
				]
			]
			[\anno{\xhead{Q}}
				[bėide]
			]
		]
		[\anno{\xbar{V}}
			[\dots]
		]
	]
\end{forest}
\hspace{2em}
\begin{forest}
	shorter edges,
	align text,
	[\anno{VP}
		[{\anno[\pass{df}]{DP}}
			[\anno{\xhead{D}}
				[si]
			]
		]
		[\anno{VP}
			[{\anno[\pass{gf}]{QP}}
				[\anno{\xhead{Q}}
					[bėide]
				]
			]
			[\anno{\xbar{V}}
				[\dots]
			]
		]
	]
\end{forest}
\caption{Zwei Varianten von oberflächlicher Kontaktstellung}
\label{fig:sibeideambig}
\end{figure}

\is{Quantifizierung|)}
\is{gefloateter Quantor|)}

%%%%%%%%%%%%%%%%%%%%%%%%%%%%%%%%%%%%%%%%%%%%%%%%%%%%%%%%%%%%%%%%%%%%%%%%%%%%%%%

\section{Die Kongruenzhierarchie}
\label{sec:kongrhier}
\is{Kongruenzhierarchie|(}

Die von \citet{corbett1979} formulierte \term{Kongruenzhierarchie}
(\fw{agreement hierarchy}) basiert auf dem typologischen Vergleich einer Reihe
(indo-)europäischer Sprachen, die sowohl formale als auch semantische Kongruenz
zulassen.%
%
	\footnote{Dies sind: 
		Deutsch, % \il{Neuhochdeutsch},
		Englisch, % \il{Englisch},
		Französisch, % \il{Französisch},
		Lateinisch, % \il{Lateinisch},
		Russisch und % \il{Russisch},
		Spanisch % \il{Spanisch}
	\autocite[214--215]{corbett1979}. Die \isi{Stichprobe} ist durch die
	Beschränkung\is{Einschränkung} auf große europäische Sprachen typologisch
	nicht repräsentativ. \citet[218]{corbett2006} listet weitere Sprachen
	auf~-- auch nicht-indogermanische~--, für die seitdem Untersuchungen
	durchgeführt wurden.}
%
Das Ergebnis der Studie ist, dass mit wachsender \q{syntaktischer
Distanz}\is{Distanz!syntaktische} zwischen \isi{Controller} und \isi{Target}
die Wahrscheinlichkeit für das Auftreten von semantischer statt formaler
Kongruenz steigt \autocite[218--223]{corbett1979}. \citet[204]{corbett1979}
formuliert die Kongruenzhierarchie folgendermaßen:

\begin{exe}
\ex attributive > predicate > relative pronoun > personal pronoun
	\is{Attribut}
	\is{Prädikat}
	\is{Relativpronomen}
	\is{Personalpronomen}
\end{exe}

Je weiter links ein \isi{Target} einzuordnen ist, desto höher ist die
Wahrscheinlichkeit, dass formale Kongruenz\is{Kongruenz!formale} auftritt; je
weiter rechts, desto höher die Wahrscheinlichkeit für semantische
Kongruenz\is{Kongruenz!semantische}. Die Abfolge\is{Abfolge} ist monoton: Wenn
bei einem Targettyp semantische Kongruenz möglich ist, ist dies auch bei allen
Targettypen rechts von ihm möglich. Da für das Deutsche\il{Neuhochdeutsch}
manche von \citeauthor{corbett1979}s Kategorien für den von ihm untersuchten
Genusausgleich relevanter als andere sind (\isi{Genus} ist kein Merkmal der
Verbalflexion; Possessivpronomina\is{Possessivpronomen} kongruieren doppelt),
schlägt \citet[193]{fleischer2012} mit einem Blick speziell auf die
Genuskongruenz folgende Konkretisierung vor:

\begin{exe}
\ex \isi{Artikel}, attributives \isi{Demonstrativpronomen} \\
	\hspace*{1em} > andere attributive\is{Attribut} Relationen \\
		\is{Prädikat}
	\hspace*{2em} > \isi{Relativpronomen} \\
	\hspace*{3em} > anaphorisches \isi{Demonstrativpronomen},
		\isi{Personalpronomen} \\
	\hspace*{4em} > anaphorisches\is{Anapher} \isi{Possessivpronomen}
\end{exe}

Er betont den Sonderstatus des Relativpronomens\is{Relativpronomen}, das sowohl
eine anaphorische\is{Anapher} als auch eine attributive\is{Attribut} Funktion
hat, und merkt weiterhin an, dass das
\blockcquote[194]{fleischer2012}{attributiv verwendete
Adjektiv\is{Adjektiv!attributiv} und Partizip \textelp{} im
Spätalthochdeutschen\il{Althochdeutsch} und
Mittelhochdeutschen\il{Mittelhochdeutsch} auch mit semantischer
Kongruenz\is{Kongruenz!semantische} auf\textins{tritt}, umgekehrt zeigt das
\isi{Personalpronomen} in älteren althochdeutschen\il{Althochdeutsch} und vor
allem in den jüngsten neuhochdeutschen\il{Neuhochdeutsch} Texten auch formale
Kongruenz\is{Kongruenz!formale}}.

Bezogen auf die syntaktische Motivation der empirisch erarbeiteten Erkenntnisse
schreibt \citet[216]{corbett1979}, dass attributive
Kongruenz\is{Kongruenz!attributive} auf die einfache Phrase beschränkt sei,
prädikative Kongruenz\is{Kongruenz!prädikative} die Phrase überschreite, doch
auf den Teilsatz beschränkt bleibe, während \isi{Personalpronomen} nicht auf
den Satz der Instanz beschränkt sei, die die
Kongruenzbeziehung\is{Kongruenzrelation} kontrolliert.
% %
% 	\footnote{\foreignblockcquote{english}[216]{corbett1979}{%
% 		\textins*{A}ttributive agreement represents agreement within the simple
% 		phrase, predicative agreement goes beyond the phrase but is restricted
% 		to the clause, the agreement of the relative pronoun goes beyond the
% 		clause but is restricted to the sentence, while the personal pronoun is
% 		not restricted to the sentence of the item controlling agreement}.}

Die Kernthese von \citet{wechslerzlatic2003} ist daher, dass formale
Kongruenz\is{Kongruenz!formale} bei Attributen\is{Attribut} zu erwarten ist, da
innerhalb der NP\is{Nominalphrase} \feat{Concord}\is{Concord} das entscheidende
Kongruenzmerkmal darstellt. Hingegen ist semantische
Kongruenz\is{Kongruenz!semantische} bei \isi{Personalpronomen} außerhalb der
bindenden NP deshalb mit höherer Wahrscheinlichkeit zu erwarten, weil Pronomen
über den referenziellen \isi{Index} an ihren Referenten gebunden sind
\autocite[89--91]{wechslerzlatic2003}. Die Autoren argumentieren also für eine
Korrelation zwischen der Distribution\is{Distribution!syntaktische} von
Kongruenz zeigenden Wortarten, deren syntaktischer \isi{Lokalität} sowie den
anwendbaren Kongruenz\-merk\-malen.

\begin{sloppypar}
\citet[89]{wechslerzlatic2003} erklären weiter, dass \isi{Personalpronomen}
auch deiktisch oder exophorisch verwendet werden können, also mit Verweis auf
außersprachliche Referenten, ohne dass diese zunächst innersprachlich
verankert\is{Diskursanker} wurden. In diesen Fällen komme entweder grammatische
oder pragmatische\is{Pragmatik} Kongruenz zum Zuge, letztere speist sich aus
der Semantik des Bezeichneten. Relativpronomina seien dagegen immer
anaphorisch\is{Anapher} gebunden~-- wobei sich die Autoren hier vermutlich auf
den innersprachlichen Kontext\is{Pragmatik} beziehen~-- und kongruieren in
\isi{Index}-Merkmalen, möglicherweise auch in \isi{Concord}-Merkmalen, da sie
NP-intern\is{Nominalphrase} auftreten. Aufgrund induktiver Kriterien ergibt
sich das Schema in \figref{fig:theoagrdist}.
\end{sloppypar}

\begin{figure}
\setlength{\tabcolsep}{4pt}
\begin{tabularx}{\linewidth}{>{\itshape}c l C c C c C}
\lsptoprule
%
	& %
	& \textit{formal}
	& %
	& $\to$
	& %
	& \textit{semantisch}
	\\

\cmidrule{3-7}

%
	& %
	& \mc{3}{c}{grammatisch}
	& %
	& \multirow{2}{*}[-.5ex]{\makecell{prag-\\ matisch}}
	\\

\cmidrule{3-5}

%
	& %
	& \textsc{concord}
	& %
	& \textsc{index}
	& %
	& %
	\\

\midrule

\mr{5}{*}[1.5ex]{\rotatebox{-90}{lokal $\to$ distal}}
	& Attribute von \xhead{N}
	& \chk
	& %
	& %
	& %
	& %
	\\

%
	& sekundäres Prädikat
	& \chk
	& %
	& %
	& %
	& %
	\\

%
	& Relativpronomen
	& \chk
	& und
	& \chk
	& %
	& %
	\\

%
	& primäres Prädikat
	& %
	& %
	& \chk
	& %
	& %
	\\

%
	& Personalpronomen
	& %
	& %
	& \chk
	& oder
	& \chk
	\\

\lspbottomrule
\end{tabularx}
\caption%
	{Interaktion von Kongruenztyp und Wortartensyntax in Erweiterung von
	\citet[84]{wechslerzlatic2003}}
\label{fig:theoagrdist}
\end{figure}

Es sei darauf hingewiesen, dass dieses Schema keinen Anspruch hat,
allgemeingültige Aussagen zu treffen, sondern lediglich dazu dient, theoretisch
begründete Voraussagen zur Ausprägung von Kongruenz in den jeweiligen
syntaktischen Konstellationen zu machen. Dies ersetzt keinesfalls die deduktive
Analyse von Daten, wie sie in den Kapiteln~\ref{ch:caoanalyse} und
\ref{ch:kcanalyse} vorgenommen wird.

\is{Kongruenzhierarchie|)}

%%%%%%%%%%%%%%%%%%%%%%%%%%%%%%%%%%%%%%%%%%%%%%%%%%%%%%%%%%%%%%%%%%%%%%%%%%%%%%%

\section{Erweiterter Konjunktionsbegriff}
\label{sec:erwkonjbegr}
\is{Konjunktion|(}

Neben Fällen, in denen sich \fw{beide} direkt oder indirekt auf zwei mit
\fw{und} verbundene Konjunkte bezieht (zum Beispiel \textsc{substantiv}
\fw{und} \textsc{substantiv})\is{Substantiv}, gibt es vereinzelt Kontexte, in
denen die kombinierten \isi{Controller} nach syntaktischen Kriterien keine
Konjunkte im formalen Sinn darstellen
\autocite[vgl.~auch][247--248]{askedal1973}. Nichtsdestoweniger können zwei
formal getrennte Controller $i$ und $j$ im Diskurs existieren, auf die sich
\norm{bėide} \wdef{beide} als \isi{Target} gleichermaßen bezieht. Dies ist in
der Belegstelle in \REF{ex:disjointctrl_1} der Fall, in der es um ein Geschäft
geht, das \fw{Agneſ} und \fw{Lukas} gemeinsam getätigt haben. Sie sagen
\lit{bedi} \wdef{beide (\textsc{\NeutMF.pl})} aus, die Zahlung erhalten zu
haben.%
%
	\footnote{In der ausgewerteten \isi{Stichprobe} tritt \lit{be(i)di} wie an
		der in \REF{ex:disjointctrl_1} zitierten Stelle auch an folgenden
		weiteren Stellen ausschließlich bei der Kombination von männlichen und
		weiblichen Controllern\is{Controller} auf:
		%
		\citet[124,23; Nr.~81, Kl.~Tennenbach, Kr.~Emmendingen, 1264]{cao1};
		\citet[205,38--39; Nr.~190, Basel, 1273]{cao1} und
		\citet[175,14; Nr.~N~230, Straßburg, 1283]{cao5}.
		%
		Bei der Belegannotation\is{Annotation} wurde davon ausgegangen, dass
		\lit{be(i)di} eine entrundete Variante der neutralen Form \norm{bėidiu}
		darstellt, die in diesen Kontexten vorherrscht \autocites(siehe auch
		Anhang~\ref{sec:caoalemschwa})[vgl.][41]{paul2007}. Die Grafie der
		betreffenden Urkunden\is{Urkunde} fällt nicht dadurch auf, dass dort
		regelmäßig \lit{-i} für einen unbetonten Nebensilbenvokal geschrieben
		wird.\label{fn:caoalemschwa}}

\begin{exe}
\ex\label{ex:disjointctrl_1}
	\setlength{\glossglue}{6pt plus 2pt minus 2pt}
	\gll % das
			vur Agneſ\tsub{i} mit hern Lukas\tsub{j} hant \textelp{}
			% het gegeben ze coͧffenne \textelp{}
			vn̄ hant bedi\tsub{{i+j}} veriehen, 
			% daſ ſie gewert ſint dis ſilberſ \\
			\textelp{} \\
		% dass
			Frau Agnes[\textsc{nom.sg.\FemF}] mit Herrn
			Lukas[\textsc{dat.sg.\MascM}] Hand {}
			% hat gegeben zu kaufen {}
			und haben beide-\textsc{nom.pl.n\subMF.st} ausgesagt
			% dass sie gewährt sind dieses Silbers \\
			\textelp{} \\
		\trans \wdef{%
			% dass
			Frau Agnes mit Herrn Lukas' Hand \textelp{}
			% hat zum Kauf gegeben \textelp{}
			und haben beide ausgesagt, 
			% dass ihnen das Silber gezahlt wurde.}
			\textelp{}}
			\parencites%
				(Nr.~N~202, Straßburg, 1281)%
				[156,11--16]{cao5}
\end{exe}

Auf formaler Ebene werden \fw{Agneſ} und \fw{Lukas} aber nicht koordiniert.
Vielmehr liegt mit \lit{mit hern Lukas hant} \wdef{mit Herrn Lukas Hand} eine
adverbiale\is{Adverbial} \isi{Präpositionalphrase} vor, in der \lit{Lukas}
seinerseits ein Genitivattribut\is{Attribut} zu \lit{hant} \wdef{Hand}
darstellt. Trotz allem sind beide Personen damit im Diskurs
verankert\is{Diskursanker}, sodass im weiteren Verlauf Verben wie \lit{hant}
\wdef{haben (\textsc{3pl.ind.prs})}, Pronomina wie \norm{ſie} \wdef{sie
(\textsc{pl})} und damit auch \lit{bedi} \wdef{beide} auf sie gemeinsam Bezug
nehmen können.

Während in \REF{ex:disjointctrl_1} die beiden \isi{Controller} in
unterschiedlichen Satzgliedern desselben Satzes stehen, ist die Verbindung der
Controller in \REF{ex:disjointctrl_2} unter formalen Kriterien noch lockerer.
So bezieht sich \lit{ſi beidv̍} \wdef{sie beide (\textsc{\NeutMF})} dort
gleichermaßen auf Frau Mechthild und ihren Mann Heinrich. Letzterer wird
syntaktisch getrennt in einem eingeschobenen, unabhängigen Temporalsatz
genannt.

\begin{exe}
\ex\label{ex:disjointctrl_2}
	\gll dv̓ vorgenante fro Mehthilt\tsub{i} / vor e ſi\tsub{i} den
			vorgenanten Heinrichen\tsub{j} neme \textelp{} vn̄ ſo
			ſi\tsub{{i+j}} beidv̍\tsub{{i+j}} ſterbent \textelp{}
			% ſo iſt daſ vorgenante guͦt \textelp{} {dem ſelben} ſpittal vn̄
			% den dv̓rftigen lidig
			\\
		die vorgenannt Frau Mechthild[\textsc{nom.sg.\FemF}] {} vor ehe
			\textsc{3sg.\FemF.nom} den vorgenannten
			Heinrich-\textsc{acc.sg.\MascM} nahm {} und so
			\textsc{3pl\subMF.nom} beide-\textsc{nom.pl.\NeutMF.st} sterben {}	
			% so ist das vorgenannte Gut {} demselben Spital und den
			% Bedürftigen ledig
			\\
		\trans \wdef{Die vorgenannte Frau Mechthild, bevor sie den
			vorgenannten Heinrich \textins{zum Mann} nahm \textelp{} Und wenn
			sie beide sterben \textelp{}%
			% , fällt das vorgenannte Gut an dasselbe Spital und die
			% Bedürftigen zurück.
			}
			\parencites%
				(Nr.~2008, Freiburg i.\,Br., 1294)%
				[253,30--39]{cao3}
\end{exe}

Innerhalb des Temporalsatzes wird Mechthild zwar durch das Pronomen \lit{ſi}
\wdef{sie} vertreten, doch auch hier liegt keine Konjunktion vor, da \lit{ſi}
das Subjektpronomen und Heinrich das Objekt von \lit{neme} \wdef{nahm}
darstellt. Beide \isi{Controller} werden zwar syntaktisch unabhängig
voneinander im Diskurs etabliert, stehen aber für den kombinierten Bezug durch
Pronomina wie \norm{si} \wdef{sie} und damit auch für \norm{bėide} zur
Verfügung.

Im Sinne eines erweiterten Konjunktionsbegriffs werden im Folgenden auch
\norm{bėide}-Targets\is{Target} wie die in \REF{ex:disjointctrl_1} und
\REF{ex:disjointctrl_2} berücksichtigt, obwohl ihre \isi{Controller} formal
nicht mit einer Konjunktion verbunden sind. Was in diesen Fällen zählt, ist,
dass sich das jeweilige Target auf semantischer Ebene gleichzeitig auf zwei
Controller bezieht, indem die Controller im Kontext\is{Pragmatik}
verankert\is{Diskursanker}~-- indiziert~-- sind.

\is{Konjunktion|)}
