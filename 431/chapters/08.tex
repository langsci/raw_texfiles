\chapter{Auswertung zur \tit{Kaiserchronik}}
\label{ch:kcanalyse}

Wie in \chapref{ch:caoanalyse} zum \tit{Corpus der altdeutschen Originalurkunden} (\CAO)
werden im Folgenden zunächst die gesammelten Belege aus der \tit{Kaiserchronik}
(\KC{}) dialektgeografisch eingeordnet. Danach folgt die Untersuchung der
Verteilung der Formen des Quantors \norm{bėide} \wdef{beide} in Bezug auf die
unterschiedlichen morphosyntaktischen Kontexte, in denen er belegt ist.
Zunächst sollen diejenigen Kontexte beleuchtet werden, in denen \norm{bėide}
von Substantiven sowie Pronomina abhängt und deren Personenmerkmale durch
Kongruenz reflektiert. In einem zweiten Teil wird nach möglichen Effekten der
Distanz zwischen Controller und Target gefragt. Der dritte Abschnitt schließt
dieses Kapitel mit einer Untersuchung von \norm{bėide} als Konjunktion in
Anlehnung an die Untersuchungen von \citet{askedal1974} und
\citet{gjelsten1980} ab. Da die \KC{} in einer Vielzahl von Handschriften
überliefert ist, werden dabei, wenn möglich, relevante Parallelbelege
hinzugezogen. Wie zuvor wird jeweils ein tabellarischer Überblick über die
Belegverteilung gegeben. Einzelfälle, Ausnahmen und Zweifelsfälle werden
exemplarisch diskutiert.

\section{Verteilung der gesammelten Belege in Zeit und Raum}
\label{subsec:beiddispmap}

\is{Dialektgeografie|(}
\is{Distribution!geografische|(}

Die Karte in \figref{fig:kartebelegzahl} zeigt die Menge der pro Handschrift
exzerpierten Belege für mittelhochdeutsch \norm{bėide} pro Ort beziehungsweise
Gebiet. Dabei sticht besonders das bairische\il{Bairisch} Sprachgebiet hervor.
Dies ist dem Umstand geschuldet, dass die \KC{} vor allem im
bairisch-österreichischen\il{Bairisch} Raum überliefert ist
\autocite{klein1988}. Dezidiert alemannische\il{Alemannisch}
Textzeugen, die sich unter geografischen Gesichtspunkten\is{Dialektgeografie}
mehr oder weniger direkt mit dem Großteil der Urkundenbelege vergleichen
ließen, sind abgesehen von K (mittelalemannisch\il{Alemannisch}) nur unter den
Fragmenten zu finden, die bei dieser Arbeit nicht berücksichtigt wurden (vgl.
\sectref{sec:miningkc}).%
%
	\footnote{Die Handschriftensiglen richten sich nach \citetitle{kcdigital},
		siehe unter \citeurl{kcdigital}.%
	}

\begin{figure}
\centering
\includegraphics[
	trim=0 10mm 75mm 0, clip,
	width=\linewidth,
	keepaspectratio
]{./figures/belege_gebiet.pdf}
\caption%
	{Anzahl der exzerpierten Belege für mittelhochdeutsch \norm{bėide} pro
	Sprachlandschaft und Handschrift. Maßgeblich ist die Annotation der Punkte,
	nicht ihre grobe Platzierung auf der Karte
	(Hintergrund nach \cite[Karte~47.4]{wiesinger1983})}
\label{fig:kartebelegzahl}
\end{figure}

Im Regelfall ist bei mittelalterlichen Handschriften keine genaue Einordnung in
Zeit und Raum möglich, da diese~-- im Unterschied zu Urkunden~-- keine
derartigen Selbstauskünfte\is{Vorlagenproblematik} bieten
\autocites[1309--1310]{wegera2000}[117--121]{bein2011}. Insofern sind bei den
Handschriften C1, H, M, P und VB nur weitläufige Regionen als
(dialekt-)geografische Bezugsgröße angegeben. Die Platzierung der Handschriften
auf der Karte folgt weitestgehend den Angaben des \citetitle{hsc}
\nosh\autocite{hsc} sowie den größtenteils deckungsgleichen Angaben in
\citet{kcdigital} und \citet{wolf:kckat}. Aufgrund der fehlenden Möglichkeit
einer genauen Ortszuweisung in den meisten Fällen haben die markierten Punkte
also lediglich Näherungscharakter und beanspruchen keinesfalls eine geografisch
exakte Festlegung auf den jeweiligen Kartenpunkt.

Darüber hinaus merkt \citet{klein1988} zur Handschrift H an, diese möge
\textquote{zwar in Hessen entstanden sein}, aber
\blockcquote[118]{klein1988}{im
thüringisch\il{Thüringisch}-hessischen\il{Hessisch} \isi{Schreibdialekt}
ge\-schrie\-ben und zeugt somit nicht für eine rheinische, sondern für eine
thüringisch\il{Thüringisch}-hessische\il{Hessisch}
\q*{Kaiserchronik}\nocite{schroeder1895}-Rezeption}.%
%
	\footnote{Der Terminus \emph{Hessen} ist aufgrund der Geschichte des
	Bundeslandes ungenau. Dem Textzusammenhang nach wird wohl die historische
	Landschaft gemeint sein, die nicht deckungsgleich mit dem Territorium des
	modernen Bundeslandes ist \autocite[vgl.~z.\,B.][853]{wiesinger1983}.}
%
\citet{kcdigital} sowie \citet[23]{wolf:kckat} geben mit
\citet[237--238]{millerzimmermann2007} vorsichtig \q{Hessen (Mainz?)} als
Entstehungsort an.

\phantomsection%
\label{phsec:vbherkunft}%
Ferner beobachtet \citet[226]{schneider1987} abweichend von den Angaben im
\citetitle{hsc} und \citet{kcdigital}, dass der Text der Handschrift VB
regelmäßig die eher für das Mitteldeutsche\il{Mitteldeutsch} typische Kennform
\norm{quam} \wdef{kam} neben bairischem\il{Bairisch} \norm{chom} enthält.
Obwohl sie dem Schreiber Bemühungen zur Vermeidung von Dialektismen attestiert,
erwägt sie als Entstehungsort den südböhmischen\il{Böhmisch} oder
ost\-bairisch-ober\-pfälzischen\il{Bairisch} Raum.

Neben der räumlichen Dimension spielt bei sprachhistorischen Untersuchungen
auch der zeitliche Bezug\is{Distribution!zeitliche} eine Rolle. Textzeugen der
\KC{} finden sich vom letzten Viertel des 12.~Jahrhunderts (A1) bis ins späte
16.~Jahrhundert (T), wobei das Gros ins 13./14.~Jahrhundert fällt. Die in der
vorliegenden Untersuchung berücksichtigten Handschriften der \KC{} entstanden
zwischen dem letzten Viertel des 12.~Jahrhunderts und der Mitte des
15.~Jahrhunderts, wie in \figref{fig:zeitstrahl} gezeigt. Die für die
Auswertung relevanten Textzeugen verteilen sich~-- neben A1 aus dem
12.~Jahrhundert~-- auf die beiden Jahrhunderthälften um 1300 und stehen damit
zeitlich den Urkunden\is{Urkunde} des \CAO{} relativ nahe.

\begin{figure}[p]
\centering
\includegraphics[
	height=.75\textheight,
]{./figures/ueberlieferungszeitraeume.pdf}
\caption{Zeitliche Verteilung der untersuchten Handschriften und Urkunden}
\label{fig:zeitstrahl}
\end{figure}

\is{Distribution!geografische|)}
\is{Dialektgeografie|)}

\section{Targets nach Personenmerkmalen des Controllers}
\label{sec:kctargpers}
\is{Target|(}
\is{Personenmerkmal|(}
\is{Controller|(}

\subsection{Nominale Controller}

Wie bei der Belegsammlung zum \CAO{} fällt die Belegmenge für den direkten
Bezug von \norm{bėide} \wdef{beide} auf zwei nominale Controller im
ausgewerteten \KC{}-Material gering aus. Für den hier untersuchten
syntaktischen Kontext liegen zwei Belege vor; zusammen mit der Kombination von
Substantiv und Pronomen sind es vier. Bei den zum Vergleich gesammelten Belegen
zur direkten Abhängigkeit von einzelnen Controllern im Plural finden sich
dagegen 19 Beispiele.

\subsubsection{Kombinierte nominale Controller}
\label{subsubsec:conomctrlpers}

Das Beispiel in \REF{ex:beid2subst} und das Schema in \figref{fig:beid2subst}
verdeutlichen den syntaktischen Kontext, der im Folgenden zu untersuchen sein
wird. Der Quantor \norm{bėide} bezieht sich als Target direkt auf zwei
Controller, \lit{Willehalm} und \lit{Dietreich}, ohne dass eine Pronominalform
dazwischen steht.

\begin{exe}
\protectedex{%
\ex \label{ex:beid2subst}
	\gll Willehalm vnd Dietreich. \\
		Willehalm[\textsc{nom.sg.\MascM}] und Dietrich[\textsc{nom.sg.\MascM}] \\
\sn \gll wurden baíde da erſlagen. \\
		wurden beide-\textsc{nom.pl.\MascM.st} da erschlagen \\
	\trans \wdef{Willehalm und Dietrich wurden beide dort erschlagen.}
			(%
				C1:~83vb,36--37%
			)%
}
\end{exe}

\begin{figure}
\begin{tikzpicture}[baseline=(1a_lb.base)]
	\node at (0,2)  (1a)    {\lit{Willehalm}};
	\node           (1a_box)[draw,rectangle,fit=(1a)] {};
	\node           (1a_lb) [above=.5ex of 1a_box, mynodefont]
	                        {controller 1};

	\node at (0,0)  (1b)    {\lit{Dietreich}};
	\node           (1b_box)[draw,rectangle,fit=(1b)] {};
	\node           (1b_lb) [above=.5ex of 1b_box, mynodefont]
	                        {controller 2};

	\node at (3,1) (2)      {\lit{baíde}};
	\draw (2) node (2_box)  [draw,rectangle,fit=(2)] {};
	\node (2_lb)   [above=.5ex of 2_box, mynodefont] {target};

	\draw [-latex] (1a_box) to [out=east, in=west] (2_box);
	\draw [-latex] (1b_box) to [out=east, in=west] (2_box);
\end{tikzpicture}
\caption{Direkter Bezug von \norm{bėide} auf zwei Controller}
\label{fig:beid2subst}
\end{figure}

Die erwähnten vier Belege mit einem \norm{bėide}-Target, das sich direkt auf
zwei Substantive bezieht, verteilen sich auf nur drei verschiedene
Parallelstellen\is{Paralleltext}. Die Menge an Kombinationen von
Personenmerkmalen wird dadurch stark reduziert. \tabref{tab:koordnomctrl} gibt
eine Übersicht über die Zahl der Belege für den jeweiligen Flexionstyp und die
zugehörige Kombination der Personenmerkmale der Controller im hier besprochenen
syntaktischen Kontext.

\begin{table}
\captionsetup{width=.75\textwidth}
\caption{Flexion nach Personenmerkmalen der kombinierten nominalen Controller}
\begin{tabular}{
	>{\scshape}l >{\scshape}l
	r r
	r
}
\lsptoprule
\normalfont Controller 1
	& \normalfont Controller 2
	& \norm{bėide}
	& \norm{bėidiu}
	& Summe
	\\

\midrule

3sg.\MascM & 3sg.\MascM &  2 &  1  &  3 \\
3sg.\FemF  & 2sg\subM   &    &  1  &  1 \\

\midrule

\mc{2}{l}{Summe}          &  2 &  2  &  4 \\

\lspbottomrule
\end{tabular}
\label{tab:koordnomctrl}
\end{table}

Von den drei Belegen zur Kombination zweier maskulin-männ\-licher Referenten
sind zwei derselben Parallelstelle\is{Paralleltext} zugehörig: zum einen, der
in \REF{ex:beid2subst} zitierte Beleg sowie seine Parallelstelle in
\REF{ex:dietwill_3}. Wie erwartet, zeigt der Quantor für diese
Merkmalskombination die Form \norm{bėide} in allen Fällen. Zur dritten Stelle
mit \norm{bėidiu} siehe \REF{ex:babstimbaideu}.

\begin{exe}
% \ex \label{ex:dietwill} % 203
% 	\begin{xlist}
% 	\ex \label{ex:dietwill_2}
% 		\gll Willehalm vnd Dietreich. \\
% 			Willehalm[\textsc{nom.sg.\MascM}] und Dietrich[\textsc{nom.sg.\MascM}] \\
% 	\sn \gll wurden baíde da erſlagen. \\
% 			wurden beide-\textsc{nom.pl.\MascM.st} da erschlagen \\
% 		\trans \wdef{Willehalm und Dietrich wurden beide dort erschlagen.}
% 			(%
% 				C1:~83vb,36--37%
% 			)

	\ex \label{ex:dietwill_3}
		\gll Wilhalm vnd dietrich \\
			Willehalm[\textsc{nom.sg.\MascM}] und Dietrich[\textsc{nom.sg.\MascM}] \\
	\sn \gll Wurden baide do erſlagen \\
			wurden beide-\textsc{nom.pl.\MascM.st} da erschlagen \\
		\trans \wdef{Willehalm und Dietrich wurden beide dort erschlagen.}
			(%
				K:~95vb,12--13%
			)
% 	\end{xlist}
\end{exe}

Beispiele für die Kombination von maskulin-männlichen und feminin-weib\-lichen
Substantiven ($\MascM+\FemF$, $\FemF+\MascM$) liegen im hier untersuchten
Kontext zumindest formal keine vor. Es gibt allerdings einen Einzelbeleg für
die Kombination von weiblicher und männlicher Referenz bei Substantiv und
Personal\-pronomen\is{Personalpronomen}, der in \REF{ex:mutterdu} wiedergegeben
wird.

\begin{exe}
\ex\label{ex:mutterdu}
	\gll Zvͦ dem chûnig ſprach er ſan \\
		zu dem König[\textsc{dat.sg.\MascM}] sprach er sodann \\
	\textelp{}
\sn \gll Dein muͦter vnd dv \\
		dein Mutter[\textsc{nom.sg.\FemF}] und \textsc{2sg\subM.nom} \\
\sn \gll Schv̂ln beideu chv̂men {dar zvͦ} \\
		sollen beide-\textsc{nom.pl.\NeutMF.st} kommen dahin \\
	\trans \wdef{Zum König sprach er sodann: \enquote{\textelp{} Deine
		Mutter und du sollt beide dahin kommen.}}
		(%
			B1:~23rc,5--14%
		)
\end{exe}

Hier verbirgt sich hinter dem \lit{dv} \wdef{du} trotz fehlender
\isi{Genusmarkierung} beim Pronomen der 2.\ Pers.\ Sg.\ ein männlicher
Referent, nämlich der \lit{chûnig} \wdef{König}, der direkt angesprochen wird.
Der Beleg passt damit in das Bild, das schon die Auswertung der Urkunden
ergeben hat. Auch bei Pro\-nomina ohne Genusmarkierung tritt aufgrund der
Personenmerkmale ihrer Referenz bei kombiniertem Bezug die neutrale Form auf.

\phantomsection
\label{phsec:babstimbaideu}
Der Beleg in \REF{ex:babstimbaideu} mit \norm{bėidiu} in Bezug auf zwei
männliche Referenten wurde bereits erwähnt. Die Passage wurde hier so
interpretiert, dass sich \lit{ím} \wdef{ihm} auf Karl den Großen bezieht, also
nicht auf \lit{wideme} \wdef{Dotierungen, Stiftungen}
\autocite[vgl. zur  Definition][s.\,v.~\fw{wideme}]{lexer:mhdhwb} und
\lit{zehende} \wdef{Zehnten}. Letzteres Wortpaar steht im Gen.~Pl.
\autocite[vgl.][341]{paul2007}, sodass die erwartete Kongruenzform des
Quantors regelmäßig \norm{bėider(e)} \wdef{beider} lauten müsste. Andere Belege
mit \norm{bėidiu} als Genitivform wurden weder für diese Handschrift noch für
die anderen exzerpiert.%
%
	\footnote{In Bezug auf den Text der Edition von
	\citet{schroeder1895} übersetzt \citet[249]{mayer1874}:
	\blockquote{Als König Karl dann zu Gericht saß, trat der Pabst vor ihn hin
	und klagte, daß die Rechte, welche seinen Vorfahren seien verliehen worden,
	ihm von den Römern entrissen wurden, so seien ihm namentlich Zehenten und
	Widdume genommen}; vgl. auch \citet[83]{weis2022}.}

\begin{exe}
\ex\label{ex:babstimbaideu}
	\gll Karle an daz gerichte ſaz \\
		Karl[\textsc{nom.sg.\MascM}] an das Gericht saß \\
\sn \gll Der babſt klegt ím daz \\
		der Papst[\textsc{nom.sg.\MascM}] klagte \textsc{3sg.\MascM.dat} dass \\
\sn \gll Der wideme vnd der zehende gar \\
		der Dotierungen und der Zehnten gar \\
\sn \gll Waͤren baidu̍ worden bar \\
		wären beide-\textsc{nom.pl.\NeutM.st} geworden ledig \\
\sn \gll Von ſínen vorvarn \\
		von seinen Vorfahren \\
	\trans \wdef{Karl setzte sich zu Gericht. Der Papst klagte ihm, dass
		\textins{sie} beide an Dotierungen und gar an Zehnten ledig geworden
		wären durch seine Vorfahren.}
		(%
			K:~85vb,22--24; vgl. abweichend
			\KC:~V.~14383--14385;
			\cite[341]{schroeder1895}%
		)
\end{exe}

Die Form \lit{beider} \wdef{beider} in K ist die für den Gen.\ Pl.\ regelmäßig
belegte, wie das Beispiel in \REF{ex:k_beider} exemplarisch zeigt, sodass davon
ausgegangen werden kann, dass sich \lit{baidiu} \wdef{beide} an der zuvor
zitierten Stelle tatsächlich auf Karl und den Papst bezieht.

\begin{exe}
\ex \label{ex:k_beider}
	% \begin{xlist}
	% \ex \label{ex:k_beider_1}
	% 	% Versumbruch durch "/" ersetzt
	% 	\gll V̍nſer baider kínt {/} \\
	% 		unser beide-\textsc{gen.pl.st} Kind \\
	% 	\gll Bevilch ich allen die hie ſínt \\
	% 		befehle ich allen die hier sind \\
	% 	\trans \wdef{unser beider Kind \textins{=~Clemens} befehle ich allen an,
	% 		die hier sind.} (%
	% 			K:~9rb,25--26; vgl.
	% 			C1:~8vb,27--28;
	% 			% VC:~6va,13--14;
	% 			Z:~31va,3--4; abweichend
	% 			A1:~7va,3--4;
	% 			H:~9va,42--43;
	% 			P:~15va,14--15;
	% 			\KC:~V.~1638--1639;
	% 			\cite[111]{schroeder1895}%
	% 		)

	% \ex \label{ex:k_beider_2}
		% Versumbruch durch "/" ersetzt
		\gll In Rome bi ír baider zít {/} \\
			in Rom bei ihr beide-\textsc{gen.pl.st} Zeit \\
		\gll Huͦb ſich vrlu̍g vnd ſtrit \\
			hob sich Krieg und Kampf \\
		\trans \wdef{Zu ihrer beider Zeit erhoben sich in Rom Kampf und Krieg.}
			(%
				K:~28vb,36--37; vgl.
				% VC:~17va,31--32;
				Z:~93va,26--94ra,1; abweichend
				B1:~14va,49--50;
				VB:~24ra,30--31;
				P:~42ra,17--18;
				A1:~20vb,9--10;
				M:~36rb,6--7;
				H:~28rb,39--40;
				\KC:~V.~4837--4838;
				\cite[170]{schroeder1895}%
			)

	% \ex \label{ex:k_beider_3}
	% 	\gll Daz waz ír baider vngewín \\
	% 		Das war ihr beide-\textsc{gen.pl.st} Verlust \\
	% 	\trans \wdef{Das war ihr beider Verlust.}
	% 		(%
	% 			K:~103vb,27; vgl.
	% 			C1:~91va,10;
	% 			% VC:~60va,34;
	% 			Z:~352ra,6;
	% 			\KC:~V.~208;
	% 			\cite[83]{schroeder1895}%
	% 		)
	% \end{xlist}
\end{exe}

\subsubsection{Einfache nominale Plural-Controller}
\label{subsubsec:nomctrlpers}

In diesem Abschnitt werden zum Vergleich Belege wie der in
\REF{ex:beidplsubst} angeführte und in \figref{fig:beidplsubst} illustrierte
diskutiert. Das Target \lit{bêde} \wdef{beide} ist hier ebenfalls unmittelbar
auf seinen Controller bezogen. Im Vergleich zum vorigen Abschnitt handelt es
sich beim Controller jedoch nicht um die Kombination von Substantiven, sondern
nur um ein einzelnes Substantiv, das im Plural steht.

\begin{exe}
\ex \label{ex:beidplsubst}
	\gll die {gotes boten} bêde \\
		 die Gottesboten[\textsc{nom.pl.\MascM}] beide-\textsc{nom.pl.\MascM.st} \\
	\trans \wdef{die beiden Gottesboten}
		(%
			\KC:~V.~7845;
			\cite[225]{schroeder1895}%
		)
\end{exe}

\begin{figure}
\begin{tikzpicture}[baseline=(1_lb.base)]
	\node (1)      [align=center]
	               {\lit{gotes boten}};
	\node (1_box)  [draw,rectangle,fit=(1)] {};
	\node (1_lb)   [above=.5ex of 1_box, mynodefont]{controller};

	\node (2)      [right=4em of 1_box, align=center]
	               {\lit{bêde}};
	\draw (2) node (2_box1) [draw,rectangle,fit=(2)] {};
	\node (2_lb1)  [above=.5ex of 2_box1, mynodefont] {target};

	\draw [-latex] (1_box) to (2_box1);
\end{tikzpicture}
\captionsetup{width=.75\textwidth}
\caption{Direkter Bezug eines Targets auf einen einzelnen Controller}
\label{fig:beidplsubst}
\end{figure}

Kontexte, in denen \norm{bėide} in einer direkten
Kongruenzbeziehung\is{Kongruenzrelation} mit einem einzelnen Substantiv im
Plural steht, sind auch in der \KC{} im Vergleich zu Kontexten mit kombinierter
Referenz zahlreicher vorhanden. \tabref{tab:simpnomctrla} zeigt ihre Verteilung
nach Personenmerkmalen und Flexionstyp.

\begin{table}
\captionsetup{width=.75\textwidth}
\caption{Flexion nach Personenmerkmalen der einfachen nominalen Controller}
\begin{tabular}{>{\scshape}l r r r}
\lsptoprule
\normalfont Controller
	& \norm{bėid(e)}
	& \norm{bėidiu}
	& Summe
	\\

\midrule

\MascM  & 11 &  2 & 13 \\
\NeutM  &    &  1 &  1 \\
\NeutA  &    &  1 &  1 \\

\midrule

\FemI   &  1 &    &  1 \\

\midrule

\normalfont Summe & 12 &  4 & 16 \\

\lspbottomrule
\end{tabular}
\label{tab:simpnomctrla}
\end{table}

Mit elf Belegen zu fünf Stellen verteilen sich die meisten
\norm{bėid(e)}-Formen mit maskulin-männlichem Bezug wie nach formalen Kriterien
erwartet \autocites(siehe auch \tabref{tab:adjparadgmstr_pr})[vgl.][182]{ksw2}.
Interessant sind in diesem Kontext die Abweichungen\is{Ausnahme}, das heißt,
die zwei Belege für \norm{bėidiu} mit maskulin-männlichem Bezug. Diese werden
in \REF{ex:richtherriu} wiedergegeben. Starke Adjektive\is{Adjektivdeklination}
zeigen in B1 ansonsten \norm{-iu} im Plural regel\-mäßig nur bei den Neutra;
Maskulina und Feminina sind dagegen stets endungslos\is{Endungslosigkeit}
(vgl.~\tabref{tab:kcadjdeclovw}).

\begin{exe}
\ex \label{ex:richtherriu}
	\begin{xlist}
	\ex \gll Die rihtær ſprachen beideu {dar zuͦ} \\
			die Richter[\textsc{nom.pl.\MascM}] sprachen beide-\textsc{nom.pl.\NeutM.st}
			dazu \\
		\trans \wdef{Die Richter äußerten sich beide dazu}
			(%
				B1:~28ra,8; vgl.~abweichend
				\KC:~V.~10090;
				\cite[267]{schroeder1895}% 1140 mit Parallelstelle in H
			)
		\label{ex:richtherriu_1}

	\ex \gll Die herren baten ir ſa \\
			Die Herren[\textsc{nom.pl.\MascM}] baten ihr alsbald \\
	\sn \gll Beideu beſvnder \\
			beide-\textsc{nom.pl.\NeutM.st} einzeln \\
		\trans \wdef{Die Herren hielten alsbald jeweils beide um ihre Hand an.}
			(%
				B1:~31va,48--49; vgl.
				\KC:~V.~11385--11386;
				\cite[289]{schroeder1895}% 1112x
			)
		\label{ex:richtherriu_2}
	\end{xlist}
\end{exe}

\phantomsection
\label{phsec:baideuwarn}
Der Beleg zu \norm{bėidiu} bei einem Neutrum mit männlichem Bezug in
\REF{ex:baideuwarn}, ein Parallelbeleg\is{Paralleltext} zu dem in
\REF{ex:babstimbaideu} zitierten, enthält formale
Kongruenz\is{Kongruenz!formale}. Das Lexem
\lit{warn} \wdef{Kinder} zu mittelhochdeutsch \norm{barn} \wdef{Kind, Sohn;
Jüngling, Held (?)}
\autocites[s.\,v.~\fw{barn}]{mwb1}[vgl.~auch][53]{kroonen2013} bezieht sich
hier metaphorisch auf Karl den Großen und Papst Leo~III., die vom
Kaiserchronisten als Brüder dargestellt werden (V.~14370;
\cites[341]{schroeder1895}[vgl.][83]{weis2022}). Obwohl sich \lit{warn} also
auf zwei erwachsene Männer bezieht, was die Wahrscheinlichkeit für semantische
Kongruenz erhöht (\sectref{sec:gendsex}), zeigt der Quantor in formaler
Übereinstimmung mit seinem Controller die neutrale Form.

\begin{exe}
\protectedex{%
\ex \label{ex:baideuwarn}
	\gll der wídem vnd der zehent gar. \\
		der Dotierungen und der Zehnten gar \\
\sn \gll wærn baidev warn bar. \\ % 1123
		wären beide-\textsc{nom.pl.\NeutM.st} Kinder[\textsc{nom.pl.\NeutM}] ledig \\
	\trans \wdef{an Dotierungen und Zehnten wären beide Kinder ledig}
		(%
			C1:~75rb,3--4; vgl.~abweichend
			K:~85vb,24;
			\KC:~V.~14384--14385;
			\cite[341]{schroeder1895}%
		)%
}
\end{exe}

Auch der Quantor in \REF{ex:beideuher} dekliniert nach dem neutralen
\isi{Genus} gemäß formaler Kongruenz\is{Kongruenz!formale} innerhalb der
Nominalphrase\is{Nominalphrase} (NP). In \tabref{tab:simpnomctrla} wurde dieser
Beleg mit \textsc{n\subA} gekennzeichnet, da \lit{her} \wdef{Heer} bei der
Annotation als \term{Committee Noun} \autocite[211--213]{corbett2006} aufgefasst
wurde: Der Begriff, obwohl formal im Singular, bezieht sich in seiner Semantik
auf eine Gruppe von Menschen. In jedem Fall zeigt sich nicht das Fehlen von
overter Flexion, die ansonsten in der Stichprobe zu B1 für den starken
Nom./Akk.~Pl.~M./F. belegt ist (vgl.~\tabref{tab:kcadjdeclovw}).

\begin{exe}
\ex \label{ex:beideuher}
	\gll Der wær herre ûber beideu her \\
		der wäre Herr über beide-\textsc{acc.pl.\NeutA.st} Heer[\textsc{acc.pl.\NeutA}] \\
	\trans \wdef{Der wäre Herr über beide Heere}
		(%
			B1:~31rc,3; zu
			\KC:~V.~11272\,ff.;
			\cite[287]{schroeder1895}% 1110
		)
\end{exe}

Der Beleg für ein unbelebtes\is{Inanimata} Femininum hat die Form \lit{baide},
insofern auch hier der Quantor innerhalb der NP\is{Nominalphrase} formale
Kongruenz zeigt \REF{ex:uozehende_2}, obwohl es sich um
Körperteile\is{Körperteil} handelt und damit um etwas, von dem auszugehen ist,
dass es eine mittlere Position zwischen den Polen \feat{belebt} und
\feat{unbelebt}\is{Belebtheit} einnimmt (vgl.~\sectref{sec:gendsex} zur
Annotation von \isi{Genus} bei Inanimata).

\begin{exe}
	\ex \gll Si wand ír baide hênde \\
			sie wand ihr beide-\textsc{acc.pl.\FemI.st} Hand-\textsc{acc.pl.\FemI} \\
		\trans \wdef{Sie wand ihre beiden Hände.}
			(%
				K:~6rb,19; vgl.
				\KC:~V.~913;
				\cite[98]{schroeder1895}%
			)
		\label{ex:uozehende_2}
\end{exe}

\subsubsection{Zusammenfassung}

Die Belegstellen zum kombinierten direkten Bezug von \mbox{\norm{bėide}} auf
zwei Substantive fallen nicht aus dem Rahmen bisheriger Ergebnisse, wenn es
auch die geringe Belegzahl unmöglich macht, generelle Aussagen zu treffen. In
beiden Fällen zeigte sich die Form \norm{bėide} mit Bezug auf maskuline und
feminine Referenten. Im Fall der Kongruenz eines Quantors mit einem einzelnen
Substantiv im Plural wiesen die Targets innerhalb der NP\is{Nominalphrase}
ebenfalls formale Kongruenz\is{Kongruenz!formale} auf, doch liegen zwei Belege
für neutrales \norm{bėidiu} bei eindeutig maskulin-männlichem Bezug vor.

\subsection{Anaphorische Controller}
\is{Anapher|(}

Mit 19 Stellen liegt der größere Teil des Belegmaterials zur \KC{} für die
Kongruenzrelation zwischen \norm{bėide} in indirekter Abhängigkeit von zwei
nominalen Controllern vor. Die Kombination von \isi{Substantiv} und einem
Pronomen\is{Personalpronomen} der ersten oder zweiten Person als
Diskurs\-anker\is{Diskursanker} wird hier mitgezählt. Die Zahl der Textstellen
mit indirektem Bezug zwischen Quantor und einzelnem Substantiv im Plural
beläuft sich auf neun.

\subsubsection{Indirekter Bezug auf kombinierte nominale Controller}
\label{subsubssec:iconomctrlpers}

Hier soll zunächst der Kongruenzbezug zwischen kombinierten Controllern und
Quantor mit einem Pronomen als Verbindungsglied anhand der gesammelten Belege
untersucht werden. Das Verhältnis zwischen Kongruenzcontrollern und Target wird
in \REF{ex:beidanactrl} und \figref{fig:beidanactrl} illustriert.%
%
	\footnote{Auch für die \KC{} gilt, dass das \isi{Personalpronomen} der
		3.~Pers.\ Pl.\ Nom./Akk. in der Regel zu \norm{si} ausgeglichen
		erscheint,
		% also keine Differenzierung zwischen maskulin-femininem
		% \norm{sie} und neutralem \norm{siu} nachzuvollziehen ist, 
		vgl.~\citet[213--214]{paul2007} und \citet[369, 390--397]{ksw2} sowie
		die Teiluntersuchung zur Form des Pronomens in
		\sectref{subsubsec:monoflexionkc}.}

\begin{exe}
\ex\label{ex:beidanactrl}
	\gll Si nâmen di muoter mit dem sun, \\
		sie nahmen die Mutter[\textsc{acc.sg.\FemF}] mit dem
			Sohn[\textsc{dat.sg.\MascM}] \\
% \sn \gll si viengen si bî dem hâre, \\
% 		sie fingen \textsc{3pl\subMF.acc} an dem Haare \\
	\textelp{}
\sn \gll si vuorten si baide zewâre \\
		sie führten \textsc{3pl\subMF.acc} beide-\textsc{acc.pl.m+f\subMF.st}
			wirklich \\
\sn \gll vur die burch an daz velt \\
		vor die Stadt an das Ebene \\
	\trans \wdef{Sie nahmen die Mutter mit dem Sohn.
		% Sie fingen sie an den Haaren.
		\textelp{}
		Ja, sie führten sie beide vor die Stadt zu der Ebene.}
		(%
			KC:~V.~14269--14272;
			\cite[339]{schroeder1895}%
		)
\end{exe}

\begin{figure}
\begin{tikzpicture}[baseline=(1a_lb.base)]
	\node at (0,2)  (1a)    [gray]
	                        {\lit{muoter}};
	\node           (1a_box)[draw,gray,rectangle,fit=(1a)] {};
	\node           (1a_lb) [above=.5ex of 1a_box, gray, mynodefont]
	                        {controller 1};

	\node at (0,0)  (1b)    [gray]
	                        {\lit{sun}};
	\node           (1b_box)[draw,gray,rectangle,fit=(1b)] {};
	\node           (1b_lb) [above=.5ex of 1b_box, gray, mynodefont]
	                        {controller 2};    

	\node at (3,1) (2)      {\lit{si}};
	\draw (2) node (2_box1) [
	                    draw,
	                    gray,
	                    minimum height=3em,
	                    minimum width=3em,
	                    xshift=-.5ex,
	                    yshift=+.5ex,
	                    rectangle
	                ] {};
	\draw (2) node (2_box2) [
	                    draw,
	                    minimum height=3em,
	                    minimum width=3em,
	                    xshift=+.5ex,
	                    yshift=-.5ex,
	                    rectangle
	                ] {};
	\node           (2_lb1) [above=.5ex of 2_box1, gray, mynodefont]
	                        {target};
	\node           (2_lb2) [below=.5ex of 2_box2, mynodefont]
	                        {controller};

	\node at (6,1)  (3)      {\lit{baide}};
	\node           (3_box)  [draw,rectangle,fit=(3)] {};
	\node           (3_lb)   [above=.5ex of 3_box, mynodefont]
	                        {target};

	\draw [-latex,gray] (1a_box) to [out=east, in=west] (2_box1);
	\draw [-latex,gray] (1b_box) to [out=east, in=west] (2_box1);
	\draw [latex-]      (3_box)  to [yshift=1.5ex]      (2_box2);
\end{tikzpicture}
\caption{Indirekter Bezug eines Targets auf zwei Erstcontroller über ein
Personalpronomen}
\label{fig:beidanactrl}
\end{figure}

Auch wenn es sich bei \lit{muoter mit dem sun} \wdef{Mutter mit dem Sohn} nicht
strikt um eine syntaktische \isi{Koordination} vom Typ \emph{X und Y} handelt
(vgl.~\sectref{sec:erwkonjbegr} zur \q{erweiterten} Koordination), werden
\lit{muoter} \wdef{Mutter} und \lit{sun} \wdef{Sohn} durch \lit{si} \wdef{sie}
kombiniert zusammengefasst. Der Quantor \lit{baide} modifiziert dieses \lit{si}
und kongruiert mit ihm. Der Quantor kongruiert damit direkt mit dem
\isi{Personalpronomen} \lit{si} und indirekt mit \lit{muoter} und \lit{sun}.

\tabref{tab:kcsimprefctrl} gibt die
Belegverteilung in der \KC{} für die in \REF{ex:beidanactrl} exemplarisch
ausgeführte Kongruenzrelation wieder, wobei auch hier nur Handschriften
berück\-sichtigt wurden, die beide Kongruenzformen aufweisen (effektiv: B1 und
VB; vgl.~auch \sectref{sec:adjdeclkc} zur Adjektivdeklination in
\KC{}-Handschriften). In der Tabelle entfällt textbedingt der größte Anteil auf
die Kombination zweier masku\-lin-\allowbreak{}männ\-licher Referenten
($\textsc{3sg.\MascM + 3sg.\MascM}$). Insgesamt weisen 14 von 16 Belegen (zu
zwölf Textstellen) die zu erwartende Flexionsform auf. Bei den zwei übrigen
Belegen handelt es sich um einen Beleg vom Typ \norm{bėide} mit Bezug auf zwei
Neutra sowie einen Beleg für \norm{bėidiu} bei kombinierten Maskulina.

\begin{table}
\centering
\caption{Flexion nach Personenmerkmalen der anaphorischen Controller
(kombinierter Bezug)}
\begin{tabular}{
	>{\scshape}l @{$~+~$} >{\scshape}l
    r r
    r
}
\lsptoprule
\mc{2}{c}{Controller}
    & \norm{bėid(e)}
    & \norm{bėidiu}
    & Summe
    \\

\midrule

% Controller              | e  | iu | Σ
3sg.\MascM & 3sg.\MascM & 11 &  1 & 12 \\

\midrule

1sg\subF & 2sg\subX     &  1 &  1 &  2 \\
2sg\subM & 1sg\subF     &    &  1 &  1 \\
2sg\subM & 3sg.\FemF    &    &  1 &  1 \\

\midrule

\mc{2}{l}{Summe}          & 12 &  4 & 16 \\

\lspbottomrule
\end{tabular}
\label{tab:kcsimprefctrl}
\end{table}

Für diese Studie am interessantesten ist das Belegpaar zur 1.\ Pers.\ Sg.
(weiblich) in Kombination mit einer 2.\ Pers.\ Sg.: An der betreffenden Stelle
spricht ein \norm{wīp} \wdef{Frau} zu seinem \norm{kindelīn} \wdef{Kindlein}
(V.~910--932; \cite[98]{schroeder1895}). Das Geschlecht des Kindes ist
unbekannt; es muss sich dem Kontext nach um einen Säugling handeln (vgl.~auch
\sectref{phsec:epikoina}).\is{Epikoina} Die beiden Belegstellen (vgl.
abweichend \KC:~V.~931--932; \cite[98]{schroeder1895}) werden in
\REF{ex:wipkindelin} wiedergegeben.

\begin{exe}
\ex \label{ex:wipkindelin}
	\begin{xlist}
	\ex \label{ex:wipkindelin_1}
		\gll Sit wír nv mvͤzzen verderben \\
			da \textsc{1pl\tsub{\SF/\SX}.nom} nun müssen zugrunde.gehen \\
	\sn \gll Vnd beide von den heiden ſterben \\
			und beide-\textsc{nom.pl.m+f\tsub{\SF/\SX}.st} von den Heiden
				sterben \\
		\trans \wdef{Da wir (jetzt) wohl zugrunde gehen werden und beide durch
			die Heiden sterben.}
			(%
				VB:~5rb,33--34%
				% ; vgl.~abweichend
				% \KC:~V.~931--932;
				% \cite[98]{schroeder1895}%
			)
		
	\ex \label{ex:wipkindelin_2}
		\gll Seit wir muͤzzen verderben. \\
			da \textsc{1pl\tsub{\SF/\SX}.nom} müssen zugrunde.gehen \\
	\sn \gll und beideu von den haiden ſterben \\
			und beide-\textsc{nom.pl.n\tsub{\SF/\SX}.st} von den Heiden
				sterben \\
		\trans \wdef{Da wir (jetzt) wohl zugrunde gehen werden und beide durch
			die Heiden sterben.}
			(%
				B1:~4vb,57--58%
				% ; vgl.~abweichend
				% \KC:~V.~931--932;
				% \cite[98]{schroeder1895}%
			)
	\end{xlist}
\end{exe}

In der Klassifikation in \tabref{tab:kcadjdeclovw} gehört VB zur Gruppe 3 mit
leichter Variation zwischen \norm{-iu} und \norm{-e} im Plural Neutrum in der
Stichprobe zur Adjektiv\-flexion\is{Adjektivdeklination}, wobei
\norm{-iu} überwiegt. Um einen solchen Fall mag es sich auch hier handeln.
Daneben besteht die Möglichkeit, dass in \REF{ex:wipkindelin_1} der generellen
\isi{Belebtheit} der Controller wegen die Form \lit{beide} auftritt. Die Form
\lit{beideu} \wdef{beide} in \REF{ex:wipkindelin_2} passt zum einen zur
Kombination von zwei formalen Neutra, zum anderen als Resolutionsform, insofern
\feat{weiblich} und \feat{unbekannt} keine Schnittmenge besitzen
(\sectref{subsubsec:x+x_dir_anim}).

Der andere oben genannte Beleg mit \norm{bėidiu} statt regelhaftem \norm{bėide}
bei der Kombination zweier Maskulina wird in \REF{ex:papstkoenig} angeführt.
Die \norm{iu}-Form des Quantors bei kombiniertem männlichen Bezug ist
irregulär\is{Ausnahme}. % (vgl.~\sectref{subsec:m+m_anim_beidiu}).

\begin{exe}
\protectedex{%
% Gerade dieses Beispiel bricht nie gut um. Zeilenumbrüche entfernt, "/"
% zwischen Versen eingefügt.
\ex\label{ex:papstkoenig} % 224
	\gll Der papſt vnd der chv̂nich {/} \\
		der Papst[\textsc{nom.sg.\MascM}] und der
		König[\textsc{nom.sg.\MascM}] \\
	\gll Si warn zegot biderb vnd frumic {/} \\
		\textsc{3pl\subM.nom} waren {zu=Gott} brav und tüchtig {} \\
	\gll Zegot ſtuͦnt allr ir geſín {/} \\
		{zu=Gott} stand aller ihr Sinnen \\
	\gll Beideu ſchatz vnd gewín {/} \\
		beide Schatz und Gewinn \\
	\gll Liezzen ſi beideu gelich {/} \\
		ließen \textsc{3pl\subM.acc} beide-\textsc{acc.pl.\NeutM.st} gleich \\
	\trans \wdef{Der Papst und der König, sie waren Gott gegenüber brav und
		tüchtig. Auf Gott war all ihr Sinnen gerichtet. Sowohl Schatz als auch
		Gewinn war ihnen beiden gleich.}
		(%
			B1:~17vb,30--34; vgl.~abweichend
			\KC:~V.~6110--6113;
			\cite[202]{schroeder1895}%
		)%
}
\end{exe}

Sowohl bei \lit{ſchatz} \wdef{Schatz} als auch bei \lit{gewín} \wdef{Gewinn}
handelt es sich um unbelebte Maskulina. Es scheint im Kontext der Stelle
sinnvoller, das neutrale \lit{beideu} in Zeile~34 nicht darauf, sondern auf
\lit{ſi} \wdef{sie}~-- den \lit{papſt} \wdef{Papst} und den \lit{chv̂nich}
\wdef{König}~-- zu beziehen. Der Blick in die Parallelstellen\is{Paralleltext}
in VB und A1 stützt diese Interpretation, insofern es hier trotz abweichendem
Wortlaut eindeutig um das gottgefällige Handeln von Kaiser Philippus und Papst
Sixtus geht
\REF{ex:papstkoenig2}.

\begin{exe}
\ex\label{ex:papstkoenig2}
\begin{xlist}
	\ex \label{ex:papstkoenig2_1}
		\gll Beidiv ſchatz vnd gewín \\
			beide Schatz[\textsc{nom.sg.\MascI}] und
				Gewinn[\textsc{nom.sg.\MascI}] \\
	\sn \gll Liezzen ſie gelíche. \\
			ließen \textsc{3pl\subM.acc} gleich \\
		\trans \wdef{Sowohl Schatz als auch Gewinn war ihnen gleich.}
			(%
				VB:~29vb,38%
			)

	\ex \label{ex:papstkoenig2_2}
		\gll baidiv ſcaz unde gewin. \\
			beide Schatz[\textsc{acc.sg.\MascI}] und Gewinn[\textsc{acc.sg.\MascI}] \\
	\sn \gll liezen ſi in beſlifen. \\
			ließen \textsc{3pl\subM.nom} \textsc{refl.dat.pl\subM} entgehen \\
		\trans \wdef{Sowohl Schatz als auch Gewinn ließen sie sich entgehen}
			(%
				A1:~26rb,40--41; vgl.
				H:~36ra,40--41;
				\KC:~V.~6112--6113;
				\cite[202]{schroeder1895}%
			)
\end{xlist}
\end{exe}

Für den kombinierten Bezug auf Personen von eindeutig unterschiedlichem
Geschlecht liegt zumindest ein einziger Beleg vor. Die Kombination von 2.\
Pers. Sg.\ (männlich) und 1.\ Pers.\ Sg.\ (weiblich) wird durch
\norm{bėidiu} aufgenommen, womit \isi{Genusresolution} vorliegt: \lit{so wærn
wir baidev verloͤrn} \wdef{dann wären wir beide verloren} (C1:~60rb,44--60va,3).

\subsubsection{Indirekter Bezug auf unkombinierte Plural-Controller}
\label{subsubsec:beid2p2snglnkc}

Beim indirekten Bezug zwischen einem einzelnen Substantiv im Plural und
\norm{bėide} als Target wie in \REF{ex:nplsibeide} und \figref{fig:nplsibeide}
liegt in der Stichprobe zur \KC{} keine Variation vor. In allen neun
Fällen sind ausschließlich Formen von \norm{bėid(e)} belegt, davon besitzen
alle eine männlich-maskuline Referenz.

\begin{exe}
\ex \label{ex:nplsibeide}
	\gll do die hêrren pranten \\
			als die Herr-\textsc{nom.pl.\MascM} brannten \\
	\textelp{}
\sn \gll da ze Kastel er si baide begraif \\
			dort zu Kastel er \textsc{3pl\subM.acc}
			beide-\textsc{acc.pl.\MascM.st} ergriff \\
		\trans \wdef{Als die Herren brandschatzten \textelp{} Dort in Kastel
			ergriff er sie beide}
			(%
				\KC:~V.~16104--16111;
				\cite[372]{schroeder1895}%
			)
\end{exe}

\begin{figure}
\begin{tikzpicture}[baseline=(2_lb1.base)]
    \node at (0,0)  (1)     [gray]
                            {\lit{hêrren}};
    \node           (1_box) [draw,gray,rectangle,fit=(1)] {};
    \node           (1_lb)  [above=.5ex of 1_box, gray, mynodefont]
                            {controller};

	\node at (3,0) (2)      {\lit{si}};
    \draw (2) node (2_box1) [
                        draw,
                        gray,
                        minimum height=3em,
                        minimum width=3em,
                        xshift=-.5ex,
                        yshift=+.5ex,
                        rectangle
                    ] {};
    \draw (2) node (2_box2) [
                        draw,
                        minimum height=3em,
                        minimum width=3em,
                        xshift=+.5ex,
                        yshift=-.5ex,
                        rectangle
                    ] {};
    \node           (2_lb1) [above=.5ex of 2_box1, gray, mynodefont]
                            {target};
    \node           (2_lb2) [below=.5ex of 2_box2, mynodefont]
                            {controller};

    \node at (6,0)  (3)      {\lit{baide}};
    \node           (3_box)  [draw,rectangle,fit=(3)] {};
    \node           (3_lb)   [above=.5ex of 3_box, mynodefont]
                            {target};

    \draw [-latex,gray] (1_box)  to [yshift=-1.5ex]     (2_box1);
    \draw [latex-]      (3_box)  to [yshift=1.5ex]      (2_box2);
\end{tikzpicture}
\caption{Indirekter Bezug eines Targets auf einen einzelnen
	Erstcontroller über ein Personalpronomen}
\label{fig:nplsibeide}
\end{figure}

\subsubsection{Zu \posscite{askedal1973} Hypothese der Monoflexion}
\label{subsubsec:monoflexionkc}

\is{Monoflexion|(}
Bezüglich \posscite{askedal1973} Hypothese zur Monoflexion
(vgl.~\sectref{subsubsec:monoflexioncao} zur Situation im \CAO{}) ist in den
\KC{}-Handschriften, die für diese Analyse ausgewertet wurden, nur sehr wenig
Variation zwischen \norm{bėide} und \norm{bėidiu} anzutreffen. Die Kombination
vom Typ \norm{si bėide} ist in allen Handschriften, in denen sowohl
\norm{bėide} als auch \norm{bėidiu} als Quantor auftritt (B1, C1, K und VB;
vgl.~\tabref{tab:beidevar}) mit Ausnahme einer Stelle in B1 die einzig belegte.
Belege mit Relativpronomen\is{Relativpronomen} als Controller liegen lediglich
drei vor, die sich auf zwei Stellen beziehen.

\begin{table}
\centering
\caption{Kombinationen von \norm{si/sie/siu} und \norm{di/die/diu} mit \norm{bėide/-iu} in der \tit{Kaiserchronik}}
\begin{tabular}{
	l
	@{\hspace{4\tabcolsep}}
	r
	r
	@{\hspace{4\tabcolsep}}
	r
	@{\hspace{4\tabcolsep}}
	r
}
\lsptoprule

Controller
	& \norm{bėide}
	& \norm{bėid}
	& \norm{bėidiu}
	& Summe
	\\

\midrule

\norm{si}  & 12 &  6 &  1 & 19 \\

\midrule

\norm{di}  &  1 &  1 &    &  2 \\
\norm{die} &  1 &    &    &  1 \\

\midrule

Summe      & 14 &  7 &  1 & 22 \\

\lspbottomrule
\end{tabular}
\label{tab:siebeidevar}
\end{table}

\is{Ausnahme|(}
Weshalb in dem Beleg in \REF{ex:papstkoenig3}, der auch an dieser Stelle
relevant ist, \norm{bėidiu} auftritt, ist nicht eindeutig\is{Ambiguität}
nachvollziehbar, da es an dieser Stelle um zwei Männer geht, nämlich den Papst
und den König. In \sectref{phsec:babstimbaideu} wurde zur Merkmalskombination
im vorliegenden syntaktischen Kontext argumentiert, dass der Bezug von
\lit{beideu} \wdef{beide} auf das Figurenpaar dem Kontext nach wahrscheinlicher
ist als auf \lit{ſchatz vnd gewín} \wdef{Schatz und Gewinn}, auch wenn
letzteres nicht hundertprozentig ausgeschlossen werden kann. Es lässt sich
spekulieren, ob \lit{beideu} mit Bezug auf die zwei Männer möglicherweise unter
Einfluss des \lit{beideu} im vorhergehenden Vers gesetzt wurde.

\begin{exe}
\protectedex{%
% "/" eingefügt und Zeilenumbrüche entfernt, weil es sonst nie gut umbricht.
\ex\label{ex:papstkoenig3} % 224
	\gll Der papſt vnd der chv̂nich {/} \\
		der Papst[\textsc{nom.sg.\MascM}] und der König[\textsc{nom.sg.\MascM}] \\
	\gll Si warn zegot biderb vnd frumic {/} \\
		\textsc{3pl\subM.nom} waren {zu=Gott} brav und tüchtig \\
	\gll Zegot ſtuͦnt allr ir geſín {/} \\
		{zu=Gott} stand aller ihr Sinnen \\
	\gll Beideu ſchatz vnd gewín {/} \\
		beide Schatz und Gewinn \\
	\gll Liezzen ſi beideu gelich {/} \\
		ließen \textsc{3pl\subM.acc} beide-\textsc{acc.pl.\NeutM.st} gleich \\
	\trans \wdef{Der Papst und der König, sie waren Gott gegenüber brav und
		tüchtig. Auf Gott war all ihr Sinnen gerichtet. Sowohl Schatz als auch
		Gewinn war ihnen beiden gleich.}
		(%
			B1:~17vb,30--34; vgl.~abweichend
			\KC:~V.~6110--6113;
			\cite[194]{schroeder1895}%
		)%
}
\end{exe}

Zwei Belege für \norm{sie bėide} liegen in VB (21ra,25--31 und
22ra,34--22rb,6) vor; dazu V.~4255--4261 und abweichend V.~4455--4470
\autocite[159, 163]{schroeder1895}. Jedoch steht \norm{bėide} in beiden dieser
Fälle im Reim\is{Versende}, sodass hier ohnehin nicht mit Variation bei der
Flexion des Quantors zu rechnen ist und die Belege daher nicht in die
bereinigte Stichprobe eingeflossen sind. In den Handschriften A1 und M, die bei
dieser Detailauswertung ansonsten nicht berücksichtigt wurden, liegt zu einer
Stelle jeweils die Kombination \norm{siu bėide} vor \REF{ex:siubedea1m}. Hier
bezieht sich der Quantor indirekt auf ein eindeutig maskulin-männliches
Substantiv im Plural: \norm{jungere} \wdef{Jünger}. Die Belege für \lit{bede}
\wdef{beide} stehen dabei jedoch ebenfalls im Reim\is{Versende} und damit in
einer Neutralisierungsposition
\autocites[vgl.][662--663]{grimm1870}[89]{askedal1973}.

\begin{exe}
\ex \label{ex:siubedea1m}
	\begin{xlist}
	\ex \label{ex:siubedea1m_1}
		\gll er nam ſiner iungere zuene. \\
				er nahm seiner Jünger[\textsc{gen.pl.\MascM}] zwei[\MascM] \\
	\sn \gll er ſante ſiv dar bede. \\
			er sandte \textsc{3pl\subM.acc} fort
				beide-\textsc{acc.pl.\MascM.st} \\
		\trans \wdef{Er nahm zwei seiner Jünger. Er sandte sie beide fort.}
			(%
				A1:~16vb,28--29; vgl.
				\KC:~V.~3937--3938;
				\cite[153]{schroeder1895}%
			)
	
	\ex \label{ex:siubedea1m_2}
		\gll Er nam ſíner ivnger zwene. \\
				er nahm seiner Jünger[\textsc{gen.pl.\MascM}] zwei[\MascM] \\
	\sn \gll Er ſant ſiv dar bede. \\
			er sandte \textsc{3pl\subM.acc} fort
				beide-\textsc{acc.pl.\MascM.st} \\
		\trans \wdef{Er nahm zwei seiner Jünger. Er sandte sie beide fort.}
			(%
				M:~29va,18--19; vgl.
				\KC:~V.~3937--3938;
				\cite[153]{schroeder1895}%
			)
\end{xlist}
\end{exe}

Wie bei der Diskussion der \CAO{}-Belege in demselben Kontext erörtert, ist die
Form \norm{siu} mit persönlicher Referenz ein Merkmal vor allem des
Alemannischen\il{Alemannisch} \autocite[vgl.][395]{ksw2}. Auch
\lit{bede} \wdef{beide} kommt hauptsächlich in Straßburger Urkunden sowie im
hier ausgeklammerten mitteldeutschen\il{Mitteldeutsch}
Sprachraum\is{Dialektgeografie} vor. Irritierend ist, dass A1 und M dem
bairischen\il{Bairisch} Sprachraum zugeordnet werden
\autocites{wolf:kckat}[266--276]{fleischer2019}.%
%
	\footnote{Auch wenn \citet[40--41]{schneider1987} insgesamt zu dem Schluss
		kommt, dass der \KC{}-Text in A1 bairische\il{Bairisch} Merkmale
		aufweist, bemerkt sie, dass keine eindeutige sprachliche
		Einordnung der Handschrift als solche möglich sei, da
		\blockquote[{\cite[40]{schneider1987},
		vgl.~\nosh\cites[519]{gaertner1999}[1638]{2vl11}}]{die einzelnen
		Dichtungen bzw.\ Textgruppen ganz unterschiedliche
		sprachliche\is{Distribution!geografische} Kriterien
		auf\textins{weisen}, \textelp{} auch verschiedene
		zeitliche\is{Distribution!zeitliche} Sprachstufen
		wieder\textins{spiegeln} \textins{sic}}.%
	}
%
Die Frage ist also, inwiefern diese Formen in einen
bairischen\il{Bairisch} Schreib\-kontext passen.

\citet[396--397]{ksw2} führen \norm{seu} (mit neuhochdeutscher
Diphthongierung\is{neuhochdeutsche Diphthongierung}) als ver\-all\-gemeinerte
Form des Nom./Akk.\ Pl.\ im Bairischen\il{Bairisch} an und mutmaßen, dass diese
Form in der Umgebung des Habsburgers Albrecht~I.\ (1255--1308) aus dem
Alemannischen\il{Alemannisch} entlehnt worden sein könnte. Indes wird A1 dem
12.~Jahrhundert zugeordnet \autocite{kcdigital,wolf:kckat}. Die Form \lit{bede}
\wdef{beide} kommt in A1 neben \lit{bediu}, \lit{bæde} und \lit{bædiu} vor~--
anders als in Straßburger Urkunden ist der Quantor also nicht starr.
\citet[103]{wiesinger2001} interpretiert die Schreibung von
mittelhochdeutsch\il{Mittelhochdeutsch} \norm{ėi} als \lit{e} als
\blockquote{häufige mono\-graphische Variante \textelp{}
im 12.\ und 13.~Jahrhundert}.

Die Belegverteilung von \norm{bėide} und \norm{bėidiu} weist in beiden
Handschriften, A1 und M, eine Trennung in \norm{bėide} als Quantor und
\norm{bėidiu} als \isi{Konjunktion} auf, was allerdings auch daran liegen
könnte, dass für den Quantor nahezu ausschließlich männliche beziehungsweise
maskuline Referenten vorliegen. Anzumerken ist, dass die Form \norm{bed-} in A1
nur bis Bl.~33vb belegt ist, also nur für die erste Hälfte des Texts. In M
kommt \norm{bed-} dagegen nur als Quantor vor, und in diesen Fällen nur mit
\norm{-e} oder -Ø.

\is{Ausnahme|)}

Zuvor wurde festgestellt, dass im Belegmaterial zur \KC{} nur
\norm{si} in Kombination mit \norm{bėide} auftritt. Die Frage drängt sich
auf: Wie sieht es im Vergleich dazu mit regulären Vorkommen von \norm{si}
\wdef{sie} aus? Da die \KC{}-Hand\-schrif\-ten in der vorliegenden Form nicht
morphologisch annotiert sind, ist es nicht möglich, automatisiert nach Formen
des Personal\-pronomens\is{Personalpronomen} der 3.~Pers.\ Pl.\ Nom./Akk.\ zu
suchen, ohne dabei potenzielle Falschpositive\is{Falschpositiv} zur 3.~Pers.\
Sg.~F.\ Nom./Akk.\ oder zu \norm{sī} \wdef{sei} als 1./3.~Pers.\ Sg.\ Konj.\
Präs.\ von \norm{sīn} \wdef{sein} zu erhalten. Um die Menge der Belege eines so
frequenten Lexems handhabbar zu machen, wurden eintausend zufällig ausgewählte
Zeilen der einzelnen in \tabref{tab:sieprn} aufgeführten Textzeugen ausgezählt.
Dieses Vorgehen gewährleistet, dass die Belege der \isi{Stichprobe} aus
thematisch unterschiedlichen Episoden stammen, über den gesamten Text gestreut
und damit repräsentativ sind. Vorkommen von \norm{si} als Basis für ein
Enklitikum (\norm{siȥ/-n} < \norm{si eȥ/in} \wdef{sie es/ihn}, \norm{sine} <
\norm{si ne} \wdef{sie \textsc{neg}};
vgl.~\cites[384--387]{ksw2}[31]{paul2007}) wurden dabei nicht gezählt.

\begin{table}
\centering
\captionsetup{width=.82\textwidth}
\caption{Varianten des Personalpronomens der 3.~Pers.~Pl.\ Nom./Akk.\ in je 1.000 Zeilen}
\begin{tabular}{l
	r
	@{\hspace{4\tabcolsep}}
	r r r
	@{\hspace{4\tabcolsep}}
	r
}
\lsptoprule
Hs.
	& \norm{si}
	& \norm{sie}
	& \norm{siu}
	& \norm{sei}
	& Summe
	\\

\midrule

A1
	& 123
	& 2
	& 1
	& % --
	& 126
	\\

M
	& 86
	& % --
	& 6
 	& 2
	& 94
	\\

\midrule

B1
	& 83
	& 1
	& % --
	& 2
	& 86
	\\

VB
	& 37
	& 38
	& % --
	& % --
	& 75
	\\

\midrule

C1
	& 97
	& % --
	& % --
	& % --
	& 97
	\\

K
	& 50
	& % --
	& % --
	& % --
	& 50
	\\

\midrule

Summe
	& 476
	&  41
	&   7
	&   4
	& 528
	\\

\lspbottomrule
\end{tabular}
\label{tab:sieprn}
\end{table}

In \tabref{tab:sieprn} fällt die Häufung von \norm{si} bei allen Handschriften
außer VB auf~-- hier kommen \norm{si} und \norm{sie} nahezu gleich häufig vor
und vor allem völlig parallel zueinander \REF{ex:vbsisie}. Auffällig ist auch,
dass \lit{bi} und \lit{bie} für mittelhochdeutsche \norm{bī} \wdef{bei}
nebeneinander auftreten. Allerdings finden sich im \CAO{} die Grafien \lit{ie}
und \lit{ıͤ} für mittelhochdeutsch \norm{ī} im Bairischen\il{Bairisch} allgemein
\autocite[2910--2911]{reiffenstein2003}, zum Beispiel in Salzburg und Brixen,
weit abseits des mitteldeutschen Sprachraums\is{Dialektgeografie}
(\cites(Nrn.~2906, 3137)[vgl.~z.\,B.][208,35--36; 349,40 und 350,4]{cao4};
weitere Beispiele in \cite[24--25]{becker2013}). In C1 und K tritt das Pronomen
dagegen allein in der Form \norm{si} auf. Aufgrund der
alemannischen\il{Alemannisch} Schreibsprache von K wäre unbelegtes *\lit{ſu̍}
\wdef{sie} neben belegtem \lit{du̍} \wdef{die} als Vertreter von
genusindifferentem\is{Genusindifferenz} \norm{siu} \wdef{sie} zumindest
denkbar.

\begin{exe}
\ex \label{ex:vbsisie}
 	\begin{xlist}
 	\ex \gll Von got ſi den gewalt haten \\
		     von Gott \textsc{3pl\subM.nom} den Macht hatten \\
 	\sn \gll Daz ſie chvcten die toten \\
		     dass \textsc{3pl\subM.nom} schauten die Toten \\
		\trans \wdef{Von Gott hatten sie \textins{=~die Herren} die Macht, die
			Toten zu schauen.}
			(%
				VB:~42ra,38--39; vgl.
				\KC:~V.~8664--8665;
				\cite[241]{schroeder1895}%
			)
 		\label{ex:vbsisie_1}

	\ex \gll Er ſande ſi Romiſhen frowen \\
		     er sandte \textsc{3pl\subI.acc} römischen Frauen \\
	\sn \gll Vnd hiez ſie manen aller trıͮwen \\
		     und hieß \textsc{3pl\subF.acc} gemahnen aller Gelübde \\
		\trans \wdef{Er sandte sie \textins{=~Briefe} den römischen Frauen und
			hieß sie, sich aller Gelübde zu erinnern}
			(%
				VB:~50va,5--6; vgl.
				\KC:~V.~10467--10468;
				\cite[273]{schroeder1895}%
			)
		\label{ex:vbsisie_2}
	\end{xlist}%
\end{exe}

Bezüglich \posscite{askedal1973} Hypothese lässt sich also feststellen, dass
\norm{si bėide} in allen unter\-suchten Handschriften eher auf die hohe
\isi{Frequenz} von \norm{si} zurückzuführen ist, als dass sich Pro\-nomen und
Quantor in der morphologischen Realisation ihrer Personenmerkmale ergänzen
würden, zumal \norm{si bėidiu} nur ein einziges Mal als Ausnahme\is{Ausnahme}
beobachtet wurde. Die zwei Belege für \norm{siu bėide} in A1 und M sind durch
Variation zu erklären, insofern \norm{siu} in diesen Handschriften eine
mög\-liche Variante des Pronomens im Akkusativ darstellt, während \norm{bėide}
als Quantor immer diese Form hat. Dazu ist allerdings zu beachten, dass fast
ausschließlich Belege mit maskulinem Bezug in diesem Kontext vorliegen, deren
reguläre Form \norm{bėide} lautet. Ein besonderes morphologisches Zusammenspiel
der beiden Wortformen als Konstruktion ist also auch hier nicht gegeben.

\is{Monoflexion|)}

\subsubsection{Zusammenfassung und Vergleich}
\label{subsubsec:persfeatsmry}

\tabref{tab:kc_e_iu_coord} fasst die Tabellen~\ref{tab:koordnomctrl}
und~\ref{tab:kcsimprefctrl} zusammen. Die Tabelle zeigt einerseits die
Verteilung der ausgewerteten Belege grob nach \isi{Belebtheit}, Kombination des
Geschlechts der Controller sowie der Form des Quantors in direkter Abhängigkeit
von kombinierten nominalen Elementen (N\tsub{i}~+~N\tsub{j}). Andererseits
weist sie die Belegverteilung bei direkter Abhängigkeit von einem Pronomen aus,
das sich auf die Kombination zweier nominaler Elemente bezieht (D\tsub{i+j}).
Es fanden sich in keinem dieser syntaktischen Kontexte Belege für kombinierten
unbelebten Bezug.\is{Desiderat}

\begin{table}
\centering
\caption{Form nach syntaktischem Kontext (kombinierter Bezug)}
\setlength{\tabcolsep}{4pt}
\begin{tabular}{
	l l
	c
	r r
	c
	r r
	c
	r
}
\lsptoprule
\mr{2}{*}[-.5ex]{Belebtheit}
	& \mr{2}{*}[-.5ex]{Geschlecht}
	& %
	& \mc{2}{c}{N\tsub{i}~+~N\tsub{j}}
	& %
	& \mc{2}{c}{D\tsub{i+j}}
	& %
	& \mr{2}{*}[-.5ex]{Summe}
	\\

\cmidrule{4-5}
\cmidrule{7-8}

%
	& %
	& %
	& \norm{bėid(e)}
	& \norm{bėidiu}
	& %
	& \norm{bėid(e)}
	& \norm{bėidiu}
	& %
	& %
	\\

\midrule

belebt
	& gleich
	& %
	&  2
	&  1
	& %
	& 11
	&  1
	& %
	& 15
	\\

%
	& verschieden
	& %
	& 
	&  1
	& %
	& 
	&  2
	& %
	&  3
	\\

\midrule

\mc{2}{l}{Summe}
	& %
	&  2
	&  2
	& %
	& 11
	&  3
	& %
	& 18
	\\

\lspbottomrule
\end{tabular}
\label{tab:kc_e_iu_coord}
\end{table}

Zwar sind für den gemischtgeschlechtlichen Bezug nur jeweils einzelne Belege
vorhanden. Die Belegverteilung passt allerdings zu derjenigen, die für die
Urkunden des \CAO{} beobachtet werden konnte (\tabref{tab:cao_e_iu_coord}). Bei
der Kombination von Controllern gleichen Geschlechts tritt hauptsächlich die
Form \norm{bėide} auf, während bei der Kombination von Controllern
unterschiedlichen Geschlechts regelmäßig die Form \norm{bėidiu} neben
\norm{bėide} belegt ist. Als Ausnahme\is{Ausnahme} zeigte sich bei den Belegen
mit kombinierten Erstcontrollern\is{Erstcontroller} \norm{bėidiu} mit Bezug auf
zwei eindeutig männliche Erstcontroller in B1. Daneben lagen zwei
parallele Belege zur Kombination der beiden belebten\is{Animata} Neutra
\norm{wīp} \wdef{Frau} und \norm{kindelīn} \wdef{Kindlein} vor, in denen der
Beleg aus B1 ebenfalls den Typ \norm{bėidiu} enthält, VB dagegen den Typ
\norm{bėide}.

\tabref{tab:kc_e_iu_simp} bezieht sich auf \tabref{tab:simpnomctrla} und die
neun Belege in \sectref{subsubsec:beid2p2snglnkc}. Der syntaktische Kontext ist
der direkte (N\tsub{i}) und indirekte Bezug (D\tsub{i}) von \norm{bėide} auf
einzelne Controller im Plural. Hier liegt zumindest ein einzelner
\norm{bėide}-Beleg mit unbelebtem\is{Inanimata} Bezug vor. Wie im \CAO{}
flektiert der Quantor in diesem Kontext weitestgehend entsprechend der starken
adjektivischen Deklination\is{Adjektivdeklination} nach formalen
Personenmerkmalen\is{Merkmale!grammatische}: \norm{bėide} bei maskulinem und
femininem Bezug, \norm{bėidiu} bei neutralem. In Abweichung\is{Ausnahme} von
der Regel liegt auch in diesem Kontext bei zwei Belegen mit maskulinem Bezug
die neutrale Form \norm{bėidiu} vor.

\begin{table}
\centering
\caption{Form nach syntaktischem Kontext (einfacher Bezug)}
\begin{tabular}{
	l l
	c
	r r
	c
	r r
	c
	r
}
\lsptoprule
\mr{2}{*}[-.5ex]{Belebtheit}
	& \mr{2}{*}[-.5ex]{Genus}
	& %
	& \mc{2}{c}{N\tsub{i}}
	& %
	& \mc{2}{c}{D\tsub{i}}
	& %
	& \mr{2}{*}[-.5ex]{Summe}
	\\

\cmidrule{4-5}
\cmidrule{7-8}

%
	& %
	& %
	& \norm{bėid(e)}
	& \norm{bėidiu}
	& %
	& \norm{bėid(e)}
	& \norm{bėidiu}
	& %
	& %
	\\

\midrule

belebt
	& maskulin
	& %
	& 11
	&  2
	& %
	&  9
	& 
	& %
	& 22
	\\

%
	& neutral
	& %
	& 
	&  2
	& %
	& 
	& 
	& %
	&  2
	\\

\midrule

unbelebt

%
	& feminin
	& %
	&  1
	& 
	& %
	& 
	& 
	& %
	&  1
	\\

\midrule

\mc{2}{l}{Summe}
	& %
	& 12
	&  4
	& %
	&  9
	& 
	& %
	& 25
	\\

\lspbottomrule
\end{tabular}
\label{tab:kc_e_iu_simp}
\end{table}

\posscite{askedal1973} Hypothese der Monoflexion lässt sich anhand der
\KC{} nicht erhärten. \norm{Si bėide} ergibt sich wie im
\CAO{} durch die Prävalenz von \norm{si} als Pronomen der 3.\ Pers.\
Pl., wobei \norm{si bėidiu} nur ein einziges Mal im Belegmaterial vorkommt. Wie
von \citet[89]{askedal1973} im Anschluss an \citet[662--663]{grimm1870}
bemerkt, kann auch in der \KC{} beobachtet werden, dass zumindest im
exzerptierten Material die Form mit \norm{-iu} nie im Reim\is{Versende}
auftritt.

\is{Controller|)}
\is{Anapher|)}
\is{Target|)}
\is{Personenmerkmal|)}

%%%%%%%%%%%%%%%%%%%%%%%%%%%%%%%%%%%%%%%%%%%%%%%%%%%%%%%%%%%%%%%%%%%%%%%%%%%%%%%

\section{Targets nach Distanz zum Controller}
\label{sec:kctargdist}
\is{Target|(}
\is{Controller|(}
\is{Distanz|(}

Bei der Untersuchung des \CAO{} hat sich herausgestellt, dass weder die
syntaktische Domäne \isi{Domäne} der Kongruenzrelation\is{Distanz!syntaktische}
(\sectref{phsec:domain}) noch der Abstand in Wortformen zwischen Contoller und
Target einen auffälligen Einfluss auf die Wahl der Kongruenzform des Quantors
haben. Nichtsdestoweniger besteht \citet{corbett1979} zufolge zumindest die
Möglichkeit dafür, die es wert ist, für die \KC{} gesondert untersucht zu
werden. Zwar sind insbesondere für die direkte Abhängigkeit zwischen
kombinierten nominalen Controllern und \norm{bėide}-Targets nur wenige Belege
verfügbar; es lässt sich trotzdem untersuchen, inwiefern sich die \KC{}-Belege
mit denen aus dem Urkundenkorpus zu einem Bild zusammen\-fügen.

\subsection{Nominale Controller}

Da der Wortformenabstand\is{Distanz!lineare} und der syntaktische Abstand
indirekt voneinander abhängen, erscheint es am sinnvollsten, die wenigen
vorhandenen Belege für \norm{bėide} \wdef{beide} in direkter Abhängigkeit von
zwei Controllern nicht nach Art des Abstands getrennt zu diskutieren, sondern
beide Kategorien wie zuvor zusammenzufassen. Die Verteilung der Belege wird in
\tabref{tab:codistp} aufgeführt.

\begin{table}
\setlength{\tabcolsep}{4pt}
\caption{Form nach Distanz von kombinierten Controllern}
\begin{tabular}{
	l
	c >{\scshape}l >{\scshape}l
	r
	r
	r
}
\lsptoprule

Domäne
	& Wortdist.
	& \normalfont Controller 1
	& \normalfont Controller 2
	& \norm{bėide}
	& \norm{bėidiu}
	& Summe
	\\

\midrule

gl. Teilsatz
	& 3 / 1
	& 3sg.\FemF
	& 2sg\subM
	& 
	& 1
	& 1
	\\

%
    & %
    & 3sg.\MascM
    & 3sg.\MascM
    & 2
    &
    & 2
    \\

\midrule

anderer Satz
	& 10 / 8
	& 3sg.\MascM 
	& 3sg.\MascM
	&
	& 1
	& 1
	\\

\midrule

\mc{4}{l}{Summe}
	& 2
	& 2
	& 4
	\\

\lspbottomrule
\end{tabular}
\label{tab:codistp}
\end{table}

Auch hier ist zu beachten, dass die Belege für zwei kombinierte 3.\ Pers.\ M.\
(männl.) im gleichen Teilsatz zu demselben Set von
Parallelstellen\is{Paralleltext} gehören: \norm{bėide} bezieht sich an dieser
Stelle auf \norm{Dietrīch} und \norm{Willehalm} \REF{ex:dietwill2}. Hier steht
in beiden Fällen in Einklang mit \isi{Genus} und \isi{Sexus} \norm{bėide}. Bei
gleicher Entfernung zu den jeweiligen Controllern steht in \REF{ex:mutterdu2}
bei der Kombination von feminin-weiblichem \lit{muͦter} \wdef{Mutter} mit
pronominalem \lit{dv} \wdef{du} die formal neutrale Form \lit{beideu}
\wdef{beide} in Übereinstimmung mit dem gemischten Geschlecht der Controller.

\begin{exe}
\ex \label{ex:dietwill2} % 203
	\begin{xlist}
	\ex \label{ex:dietwill2_2}
		\gll Willehalm vnd Dietreich. \\
			Willehalm[\textsc{nom.sg.\MascM}] und
				Dietrich[\textsc{nom.sg.\MascM}] \\
	\sn \gll wurden baíde da erſlagen. \\
			wurden beide-\textsc{nom.pl.\MascM.st} da erschlagen \\
		\trans \wdef{Willehalm und Dietrich wurden beide dort erschlagen.}
			(%
				C1:~83vb,36--37;
				3/1 Wortformen, gleicher Teilsatz%
			)

	\ex \label{ex:dietwill2_3}
		\gll Wilhalm vnd dietrich \\
			Willehalm[\textsc{nom.sg.\MascM}] und
				Dietrich[\textsc{nom.sg.\MascM}] \\
	\sn \gll Wurden baide do erſlagen \\
			wurden beide-\textsc{nom.pl.\MascM.st} da erschlagen \\
		\trans \wdef{Willehalm und Dietrich wurden beide dort erschlagen.}
			(%
				K:~95vb,12--13;
				3/1 Wortformen, gleicher Teilsatz%
			)
	\end{xlist}

\ex \label{ex:mutterdu2}
	\gll Dein muͦter vnd dv \\
		dein Mutter[\textsc{nom.sg.\FemF}] und \textsc{2sg\subM.nom} \\
\sn \gll Schv̂ln beideu chv̂men {dar zvͦ} \\
		sollen beide-\textsc{nom.pl.\NeutMF.st} kommen dahin \\
	\trans \wdef{Deine Mutter und du \textins{=~der König} sollt beide dorthin
		kommen.}
		(%
			B1:~23rc,13--14;
			3/1~Wortformen, gleicher Teilsatz%
		)
\end{exe}

\phantomsection
\label{phsec:baideuwarn3}
Ein von der Regel abweichender\is{Ausnahme} Beleg (eine
Parallelstelle\is{Paralleltext} zu C1:~75rb,3--4, siehe auch die
Abschnitte~\ref{phsec:babstimbaideu} und~\ref{phsec:baideuwarn}) mit einer
formal neutralen Form in Bezug auf zwei maskulin-männ\-liche Referenten im
vorhergehenden (Teil-)Satz liegt in K vor
\REF{ex:baideuwarn3}. Die Form \lit{baidu̍} \wdef{beide} im Nebensatz bezieht
sich hier wohl auf \lit{babſt} \wdef{Papst} und \lit{ím} \wdef{ihm}
(=~\lit{Karle}) im Hauptsatz und damit auf zwei Männer, zeigt aber die formal
neutrale Form (vgl.~\sectref{phsec:babstimbaideu} zur Kombination von
Personenmerkmalen in diesem syntaktischen Kontext).

\begin{exe}
% "/" eingefügt, Zeilenumbrüche entfernt, damit nachfolgend langes Beispiel
% hoffentlich besser umbricht
\ex \label{ex:baideuwarn3}
	\gll Karle an daz gerichte ſaz {/} \\
	    Karl[\textsc{nom.sg.\MascM}] an das Gericht saß \\
	\gll Der babſt klegt ím daz {/} \\
		der Papst[\textsc{nom.sg.\MascM}] klagte \textsc{3sg.\MascM.dat} dass \\
	\gll Der wideme vnd der zehende gar {/} \\
		der Stiftungen und der Zehnten gar \\
	\gll Waͤren baidu̍ worden bar {/} \\
		wären beide-\textsc{nom.pl.\NeutM.st} geworden bloß \\
	\gll Von ſínen vorvarn \\
		von seinen Vorfahren \\
	\trans \wdef{Karl (der Große) saß zu Gericht. Der Papst klagte ihm, dass
		beide durch seine Vorgänger an Stiftungen und gar an Zehnten ledig
		geworden wären.}
		(%
			K:~85vb,21--25; vgl.~abweichend
			\KC:~V.~14382--14386;
			\cite[341]{schroeder1895}%
		)
\end{exe}

Aufgrund der geringen Zahl der Belege lässt sich keine Aussage dazu machen, ob
die Form des Quantors durch die vergleichsweise hohe syntaktische Entfernung zu
seinen Controllern begünstigt wird. Im Urkundenmaterial liegt ein einziger
Beleg zu kombinierten männlichen Controllern vor~-- ein \isi{Personalpronomen}
der 1.~Person im gleichen Teilsatz mit neun und vier Wortformen
Abstand~--,\is{Distanz!lineare} deren Quantor eine Form des Typs \norm{bėide}
zeigt (\ref{ex:1sg1sgbeide}; siehe auch \tabref{tab:caocodistp}). Dabei steht
\lit{baide} streng genommen in einer Neutralisierungsposition
\autocites[vgl.][90--91]{askedal1973}[191]{gjelsten1980}.

\begin{exe}
\ex\label{ex:1sg1sgbeide}
	\gll wir · krafte von hohenloch · vn̄ · wir
		ludewic von durne · geloben baide vf vnſern eit \\
		\textsc{1sg\subM.hon.nom} {} Kraft von Hohenlohe {} und {}
		\textsc{1sg\subM.hon.nom} Ludwig von Durne {} geloben
		beide-\textsc{nom.pl.\MascM.st} auf unseren Eid \\
	\trans \wdef{Wir, Kraft von Hohenlohe, und wir, Ludwig von Durne,
		versprechen beide unter Eid}
		\parencites(Nr.~2529, Burg Hohlach, Kr.~Neustadt an der Aisch-Bad Windsheim, 1296)[563,5--6]{cao3}%
\end{exe}

Für die als Vergleich gesammelten Belege zu \norm{bėide} mit direktem Bezug auf
einzelne Plural-Substantive liegen insgesamt 17 Belege zu zwölf
Textstellen vor (siehe \tabref{tab:pldistp}). Alle bis auf vier Belege
enthalten eine Form vom Typ \norm{bėid(e)}. Der Quantor steht in keinem Fall
weiter als im gleichen Teilsatz von seinem Controller entfernt. Anhand der
Tabelle lässt sich beobachten, dass auch hier \norm{bėide} beziehungsweise
\norm{bėidiu} im gleichen Satzglied wie sein Controller regelmäßig formale
Kongruenz aufweist, was in diesem syntaktischen Kontext zu erwarten ist
(\isi{Concord} innerhalb der NP\is{Nominalphrase}). Die vier maskulinen Belege
sowie der eine feminine flektieren mit \norm{-e}, die beiden neutralen mit
\norm{-iu}. Beachtenswert\is{Ausnahme} erscheinen die zwei Belege mit
\norm{bėidiu} in Bezug auf ein Maskulinum im gleichen Teilsatz, die neutrales
\norm{-iu} entgegen der Form und Semantik aufweisen.

\begin{table}
\centering
\caption{Form nach Distanz vom einfachen nominalen Controller}
\begin{tabular}{
	l
	c >{\scshape}l
	r r
	r
}
\lsptoprule

Domäne
	& Wortdist.
	& \normalfont Controller
	& \norm{bėid(e)}
	& \norm{bėidiu}
	& Summe
	\\

\midrule

gl. Satzglied
	& 0
	& \MascM
	& 10 % war: 4 (s. Kommentar zu 'unsicher' unten)
	&
	& 10 % war: 4 (s. Kommentar zu 'unsicher' unten)
	\\

%
	& %
	& \NeutM
	& 
	& 1
	& 1
	\\

%
	& %
	& \NeutA
	& 
	& 1
	& 1
	\\

%
	& %
	& \FemI
	& 1
	&
	& 1
	\\

\midrule

gl. Teilsatz
	& 1--3
	& 3pl.\MascM
	& 2
	& 2
	& 4
	\\

\midrule

\mc{3}{l}{Summe}
	& 13
	&  4
	& 17
	\\

\lspbottomrule
\end{tabular}
\label{tab:pldistp}
\end{table}

\phantomsection
\label{phsec:richtherriu2}
Diese zwei Belege \REF{ex:richtherriu2} sind bereits zuvor als
Unregelmäßigkeit\is{Ausnahme} aufgefallen (vgl.~\sectref{ex:richtherriu}). In
beiden Fällen stehen semantisch männliche Controller~-- \lit{rihtær}
\wdef{Richter} und \lit{herren} \wdef{Herren}~-- mit einer Form des Typs
\norm{bėidiu}, die im Rahmen der Handschrift B1 im Plural ansonsten formale
Neutra eindeutig markiert. Vergleichbare Fälle im gleichen syntaktischen
Kontext (N\tsub{i} $\to$ \norm{bėide}) sind bei ähnlicher Beleglage im
\CAO{}-Material nicht vorhanden. Bei gleichem Geschlecht tritt dort
ausschließlich \norm{bėid(e)} auf (\sectref{subsec:caodistnomctrl}).

\begin{exe}
\ex \label{ex:richtherriu2}
	\begin{xlist}
	\ex \label{ex:richtherriu2_1}
		\gll Die rihtær ſprachen beideu {dar zuͦ} \\
			die Richter[\textsc{nom.pl.\MascM}] sprachen beide-\textsc{nom.pl.\NeutM.st}
			dazu \\
		\trans \wdef{Die Richter äußerten sich beide dazu}
			(%
				B1:~28ra,8; vgl.~abweichend
				\KC:~V.~10090;
				\cite[267]{schroeder1895}% 1140 mit Parallelstelle in H
			)

	\ex \label{ex:richtherriu2_2}
		\gll Die herren baten ir ſa \\
			Die Herren[\textsc{nom.pl.\MascM}] baten ihr alsbald \\
	\sn \gll Beideu beſvnder \\
			beide-\textsc{nom.pl.\NeutM.st} einzeln \\
		\trans \wdef{Die Herren hielten alsbald jeweils beide um ihre Hand an.}
			(%
				B1:~31va,48--49; vgl.
				\KC:~V.~11385--11386
				\cite[289]{schroeder1895}% 1112x
			)
	\end{xlist}
\end{exe}

Zusammenfassend lässt sich feststellen, dass sich die wenigen Belege für
\norm{bėide} mit direktem Bezug auf zwei kombinierte Controller im
\KC{}-Material weitgehend regelmäßig verhalten: Bei gleichem
kombinierten Geschlecht steht die \norm{e}-Form, bei verschiedenem die
\norm{iu}-Form, auch wenn sich das Kongruenztarget in \isi{Distanzstellung} zu
seinem Controller befindet. Allerdings sind längst nicht alle möglichen
Kombinationen von Personenmerkmalen in allen syntaktischen Kontexten im
\KC{}-Material belegt.\is{Desiderat} Beispielsweise liegen keine
Vergleichs\-möglich\-keiten vor, um die interessanteren Fälle kontextualisieren
zu können~-- dies betrifft neutrales \norm{bėidiu} mit kombiniertem Bezug auf
\norm{bābest} \wdef{Papst} und \norm{Karl} (gleicher Teilsatz). Darüber hinaus
liegen keine Belege mit unbelebtem Bezug vor. Ein Vergleich mit der
Belegsammlung zum \CAO{} diesbezüglich ist also nicht möglich.

In Hinblick auf \norm{bėide}-Targets in direkter Abhängigkeit von einem
einzelnen Controller im Plural ist insgesamt festzustellen, dass sie sich
ebenfalls großteils regelmäßig verhalten. In der \isi{Domäne} \emph{gleiches
Satzglied} zeigt der einzige Beleg für ein männliches Neutrum die neutrale Form
\norm{bėidiu}, genauso auch ein Neutrum mit unspezifiziertem Geschlecht. Es
spricht zumindest nichts gegen die Annahme, dass innerhalb der
NP\is{Nominalphrase} das \isi{Genus} des Controllers ausschlaggebend ist. In
der Domäne \emph{gleicher Teilsatz} zeigt sich leichte Variation, insofern
jeweils einmal \norm{bėidiu} mit Bezug auf \norm{rihtǟre} \wdef{Richter} und
\norm{hērren} \wdef{Herren} auftritt. Auch hier liegen zu wenige Belege vor,
als dass ein Vergleich möglich wäre. Daneben stehen sechs weitere Belege für
\norm{bėide}, deren Kongruenzdomäne nicht eindeutig\is{Ambiguität} zugeordnet
werden kann und die deshalb gemäß \citet[623]{ksw2} der Domäne \emph{gleiches
Satzglied} zugeschlagen wurden (vgl.~\sectref{sec:floatquant}).

\subsection{Anaphorische Controller}
\is{Anapher|(}

\tabref{tab:kcanadist} enthält die Aufstellung nach syntaktischer und absoluter
Distanz\is{Distanz!lineare} derjenigen Belege, bei denen der Quantor
\norm{bėide} mittelbar über ein Pronomen von zwei nominalen Controllern abhängt
\REF{ex:beidanactrl}. Bei der Kombination von zwei gleichen Genera\is{Genus}
oder \isi{Sexus} steht in den untersuchten \KC{}-Handschriften regelmäßig eine
Form des Typs \norm{bėid(e)}. Die formal neutrale Form \norm{bėidiu} ist,
abgesehen von einer Ausnahme, nur bei Erstcontrollern\is{Erstcontroller} mit
ungleichem Sexus oder Genus belegt.

\begin{table}
\setlength{\tabcolsep}{4pt}
\captionsetup{width=.75\textwidth}
\caption{Form nach Distanz vom anaphorischen Controller (kombinierter Bezug)}
\begin{tabular}{
	l
	c
	r r c
	r r c
	r
}

\lsptoprule

\mr{3}{*}[-1ex]{Domäne}
	& \mr{3}{*}[-1ex]{Wortdist.}
	& \mc{5}{c}{belebt}
	& \mr{3}{*}[-1ex]{Summe}
	\\

\cmidrule{3-7}

%
	& %
	& \mc{2}{c}{gleich}
	& %
	& \mc{2}{c}{verschieden}
	& %
	\\

\cmidrule{3-4}
\cmidrule{6-7}

%
	& %
	& \mc{1}{c}{\norm{bėid(e)}}
	& \mc{1}{c}{\norm{bėidiu}}
	& %
	& \mc{1}{c}{\norm{bėid(e)}}
	& \mc{1}{c}{\norm{bėidiu}}
	& %
	\\

\midrule

% Auch hier gemäß Klein et al. (2018: § P 367) die 'unsicheren' Formen als
% 'gleiches Satzglied' gezählt.
gleiches Satzglied
	& 0
	& 2 % war: 1 % belebt gleich beide
	& 1 % war: 0 % belebt gleich beidiu
	& %--
	& % belebt verschieden beide
	& 1 % war: 0 % belebt verschieden beidiu
	& 4 % Summe
	\\

\midrule

gleicher Teilsatz
	& 1--2
	& 12 % belebt gleich beide
	& % belebt gleich beidiu
	& %--
	& % belebt verschieden beide
	& 1 % belebt verschieden beidiu
	& 13 % Summe
	\\

%
	& 3--4
	& % belebt gleich beide
	& % belebt gleich beidiu
	& %--
	& 1 % belebt verschieden beide
	& 1 % belebt verschieden beidiu
	& 2 % Summe
	\\

\midrule

\mc{2}{l}{Summe}
	& 14 % belebt gleich beide
	&  1 % belebt gleich beidiu
	& %--
	&  1 % belebt verschieden beide
	&  3 % belebt verschieden beidiu
	& 19 % Summe
	\\

\lspbottomrule
\end{tabular}
\label{tab:kcanadist}
\end{table}

Aus der Tabelle geht lediglich hervor, dass die meisten Targets sehr kurz
hinter ihrem anaphorischen Controller stehen, wie auch schon für das
\CAO{} festgestellt (\sectref{subsec:caodistanactrl}). Ob \isi{Genus} oder
\isi{Sexus} die entscheidende Kategorie ist, lässt sich aufgrund der
Belegsituation für die \isi{Domäne} \emph{gleiches Satzglied} nicht
einschätzen. Die Targets mit \isi{Distanzstellung} in der Domäne \emph{gleicher
Teilsatz}\is{Distanz!syntaktische} verteilen sich insgesamt ähnlich wie im
\CAO{} (vgl.~\tabref{tab:caoanadist}), insofern ist zu vermuten, dass in
Kontexten mit kombiniertem Bezug auch in der \KC{} semantische
Kongruenz\is{Kongruenz!semantische} die wichtigere Rolle spielt. Wie zuvor
sollen im Folgenden kurz diejenigen Belege besprochen werden, die sich entgegen
dem regulären Muster verhalten.

Die Belege in \REF{ex:wipkindelin2} und \REF{ex:papstkoenig4} seien noch einmal
wiederholt, um sie im Kontext des vorliegenden Unter\-suchungsaspekts zu
thematisieren. Als \isi{Diskursanker} wurde in \REF{ex:wipkindelin2} das
\isi{Personalpronomen} \norm{wir} \wdef{wir} angenommen, mit dem das \norm{wīp}
\wdef{Frau} in direkter Rede auf sich und sein \norm{kindelīn} \wdef{Kindlein}
verweist (V.~910--932; \cite[98]{schroeder1895}). Das Kind wird in der Episode
nicht beim Namen genannt, daher ist nicht bekannt, ob es sich um einen Jungen
oder ein Mädchen handelt.

\begin{exe}
\ex \label{ex:wipkindelin2}
	\begin{xlist}
	\ex \label{ex:wipkindelin2_1}
		\gll Sit wír nv mvͤzzen verderben \\
			da \textsc{1pl\tsub{\SF/\SX}.nom} nun müssen zugrunde.gehen \\
	\sn \gll Vnd beide von den heiden ſterben \\
			und beide-\textsc{nom.pl.m+f\tsub{\SF/\SX}.st} von den Heiden
				sterben \\
		\trans \wdef{Da wir (jetzt) wohl zugrunde gehen werden und beide durch
			die Heiden sterben.}
			(%
				VB:~5rb,33--34; vgl.~abweichend
				\KC:~V.~931--932;
				\cite[98]{schroeder1895}%
			)
		
	\ex \label{ex:wipkindelin2_2}
		\gll Seit wir muͤzzen verderben. \\
			da \textsc{1pl\tsub{\SF/\SX}.nom} müssen zugrunde.gehen \\
	\sn \gll und beideu von den haiden ſterben \\
			und beide-\textsc{nom.pl.n\tsub{\SF/\SX}.st} von den Heiden
				sterben \\
		\trans \wdef{Da wir (jetzt) wohl zugrunde gehen werden und beide durch
			die Heiden sterben.}
			(%
				B1:~4vb,57--58; vgl.~abweichend
				\KC:~V.~931--932;
				\cite[98]{schroeder1895}%
			)
	\end{xlist}%
\end{exe}

Innerhalb der \KC{}-Stichprobe ist aufgrund der geringen Belegmenge kein
Vergleich möglich. Die Handschrift B1 verhält sich aber zumindest zu den
Urkunden des \CAO{} kongruent, insofern ihre Version der Stelle die
formal\is{Kongruenz!formale} neutrale Form \lit{beideu} enthält
\REF{ex:wipkindelin2_2}, die hier am wahrscheinlichsten als Resolutionsform in
Ermangelung klarer semantischer Hinweise auf die vorliegende
Geschlechterkombination aufgefasst werden kann.

Im Urkundenmaterial gar nicht vertreten ist der in \REF{ex:papstkoenig4}
dargestellte Fall von \norm{bėidiu} in Bezug auf zwei maskulin-männliche
\isi{Erstcontroller}~-- bei gleichem Geschlecht weisen die Urkunden des
\CAO{} regelmäßig den Typ \norm{bėid(e)} auf. Da dies auch der einzige Fall
dieser Art innerhalb des \KC{}-Belegmaterials ist, ist hier weder ein Vergleich
innerhalb einzelner Textzeugen der \KC{} oder zwischen ihnen, noch ein
Vergleich mit dem \CAO{} möglich.

\begin{exe}
\protectedex{%
\ex\label{ex:papstkoenig4} % 224
% "/" eingefügt und Zeilenumbrüche entfernt, weil es sonst nie gut umbricht
% Weil das Beispiel schon mehrfach vorkam, Mittelteil entfernt.
	\gll Der papſt vnd der chv̂nich {/} \\
		der Papst[\textsc{nom.sg.\MascM}] und der König[\textsc{nom.sg.\MascM}] \\
% 	\gll Si warn zegot biderb vnd frumic \\
% 		sie waren {zu=Gott} brav und tüchtig \\
% 	\gll Zegot ſtuͦnt allr ir geſín \\
% 		{zu=Gott} stand aller ihr Sinnen \\
	\textelp{}
	\gll Beideu ſchatz vnd gewín {/} \\
		beide Schatz und Gewinn \\
	\gll Liezzen ſi beideu gelich \\
		ließen \textsc{3pl\subM.acc} beide-\textsc{acc.pl.\NeutM.st} gleich \\
	\trans \wdef{Der Papst und der König
		% , sie waren Gott gegenüber brav und tüchtig. Auf Gott war all ihr
		% Sinnen gerichtet.
		\textelp{}
		Sowohl Schatz als auch Gewinn war ihnen beiden gleich.}
		(%
			B1:~17vb,30--34; vgl.~abweichend
			\KC:~V.~6110--6113;
			\cite[194]{schroeder1895}%
		)%
}
\end{exe}

Festzuhalten ist, dass die \isi{Domäne} in diesem Fall nicht
eindeutig\is{Ambiguität} bestimmbar ist (gleiches Satzglied oder gleicher
Teilsatz?), da Pronomen und Quantor direkt hintereinander stehen und auch der
Kontext keinen Hinweis darauf gibt, ob eine kollektive oder eine distributive
\isi{Lesart} wahrscheinlicher ist (vgl.~\sectref{sec:floatquant} zur Semantik
von gefloateten Quantoren\is{gefloateter Quantor}). \citet[623]{ksw2} folgend
wurden diese unsicheren Fälle als attributiv\is{Attribut} nachgestellt und
damit als sich im gleichen Satzglied wie ihr Controller befindlich gezählt.

Bei der Abhängigkeit des Quantors von einem \isi{Personalpronomen}, das sich
auf ein einzelnes Plural-Substantiv bezieht (indirekte Abhängigkeit vom
\isi{Erstcontroller}), konnte keine Variation zwischen \norm{-e} und \norm{-iu}
festgestellt werden. Hier steht in allen relevanten Handschriften (B1, C1, K
und VB) sechsmal \norm{bėide} und einmal \norm{bėid} (B1:~31va,30--31; zu
insgesamt vier Stellen).

\is{Anapher|)}

\subsection{Wortformenabstand zu kombinierten Erstcontrollern}
\is{Distanz!lineare|(}
\is{Erstcontroller|(}

\tabref{tab:kccaodist} weist die Belegzahlen für die durchschnittlichen
Distanzen der kombinierten Erstcontroller zu ihrem \norm{bėide}-Target aus. Wie
zuvor wurde dazu das arithmetische Mittel der Wortformdistanzen der beiden
Erstcontroller zu ihrem \norm{bėide}-Target gebildet und eine
Klasseneinteilung\is{Klassenbildung} vorgenommen. Es fällt auf, dass die
Ketten\is{Verweiskette} hier wesentlich kürzer sind als bei den
Urkunden\is{Urkunde}, daher liegen auch weniger Distanzklassen vor; Klassen
ohne Belege wurden in der Darstellung übersprungen. Im Durchschnitt beträgt die
Distanz rund 19 Wortformen zwischen Erstcontroller und \norm{bėide}-Target
gegenüber rund 42 Wortformen bei der Urkundenstichprobe.

\begin{table}
\centering
\caption{Form nach Distanz von kombinierten Erstcontrollern}
%
% {1,2,3,4} × 5 ^ {1,2,3}
%
\begin{tabular}{
	c
	l
	r r c
	r r c
	r
}

\lsptoprule

\mr{3}{*}[-1ex]{Klasse}
	& \mr{3}{*}[-1ex]{Wortdist.}
	& \mc{5}{c}{belebt}
	& \mr{3}{*}[-1ex]{Summe}
	\\

\cmidrule{3-7}

%
	& %
	& \mc{2}{c}{gleich}
	& %
	& \mc{2}{c}{verschieden}
	& %
	\\

\cmidrule{3-4}
\cmidrule{6-7}

%
	& %
	& \mc{1}{c}{\norm{bėid(e)}}
	& \mc{1}{c}{\norm{bėidiu}}
	& %
	& \mc{1}{c}{\norm{bėid(e)}}
	& \mc{1}{c}{\norm{bėidiu}}
	& %
	\\

\midrule

1
	& ≤ 5
	& 6 % belebt gleich beide
	& % belebt gleich beidiu
	& %--
	& % belebt verschieden beide
	& 1 % belebt verschieden beidiu
	& 7 % Summe
	\\

2
	& ≤ 10
	& 1 % belebt gleich beide
	& % belebt gleich beidiu
	& %--
	& % beide belebt verschieden
	& % belebt verschieden beidiu
	& 1 % Summe
	\\

3
	& ≤ 15
	& 4 % belebt gleich beide
	& 1 % belebt gleich beidiu
	& %--
	& % belebt verschieden beide
	& 2 % belebt verschieden beidiu
	& 7 % Summe
	\\

4
	& ≤ 20
	& 2 % belebt gleich beide
	& 1 % belebt gleich beidiu
	& %--
	& % belebt verschieden beide
	& % belebt verschieden beidiu
	& 3 % Summe
	\\

8
	& ≤ 100
	& 3 % belebt gleich beide
	& % belebt gleich beidiu
	& %--
	& % belebt verschieden beide
	& % belebt verschieden beidiu
	& 3 % Summe
	\\

\midrule

\mc{2}{l}{Summe}
	& 16 % belebt gleich beide
	&  2 % belebt gleich beidiu
	& %--
	& % belebt verschieden beide
	&  3 % belebt verschieden beidiu
	& 21 % Summe
	\\

\lspbottomrule
\end{tabular}
\label{tab:kccaodist}
\end{table}

Auch wenn in \tabref{tab:kccaodist} zur \KC{} mit 21 gegenüber 119 Belegen
sehr viel weniger als in \tabref{tab:caodist} zum \CAO{} vorliegen,
treten keine auffälligen Muster zutage. Formen vom Typ \norm{bėide} stehen
regelmäßig in Abhängigkeit von Controllern mit gleichem Geschlecht;
\norm{bėidiu} erscheint bei den wenigen Stellen mit gemischtgeschlechtlichen
Controllern wie dem in \REF{ex:combgenddist1}, unabhängig von der Entfernung der
Erstcontroller zum Target.

\begin{exe}
\ex \label{ex:combgenddist1}
	% \begin{xlist}
	% \ex \label{ex:combgenddist_1} % 225
	% 	\gll Dein muͦter vnd dv \\
	% 		dein Mutter[\textsc{nom.sg.\FemF}] und \textsc{2sg\subM.nom} \\
	% \sn	\gll Schv̂ln beideu chv̂men {dar zvͦ} \\
	% 		sollen beide-\textsc{nom.pl.\NeutMF.st} kommen dahin \\
	% 	\trans \wdef{Deine Mutter und du sollt beide dorthin kommen.}
	% 		(%
	% 			B1:~23rc,13--14;
	% 			3/1~Wortformen, gleicher Teilsatz%
	% 		)

	% \ex \label{ex:combgenddist_2} % 236
		\gll nv waiz du wol das ich bín. \\
		     nun weißt \textsc{2sg\subM.nom} wohl dass \textsc{1sg\subF.nom} bin \\
	\sn \gll deines bruͦder weip. \\
	         deines Bruders Frau[\NeutF] \\
	\sn	\gll vraiſchet das {dhaín leip.} \\
		     erführe das niemand \\
	\sn \gll so wærn wir baidev verloͤrn. \\
		     so wären \textsc{1pl\subMF.nom} beide-\textsc{nom.pl.\NeutMF.st}
		     verloren \\
		\trans \wdef{Nun weißt du wohl, dass ich die Frau deines Bruders bin.
			Auch wenn niemand davon erführe, wären wir beide verloren.}
			(%
				C1:~60rb,45--60va,3;
				14/11~Wortformen, anderer Satz%
			)

	% \ex \label{ex:combgenddist_3} % 272
	% 	\gll Svn dv ſolt ſie líp han \\
	% 	     Sohn \textsc{2sg\subM.nom} sollst \textsc{3pl.\FemF.acc} lieb haben \\
	% \sn \gll Sie iſt dín mvter vnd díne frowe \\
	% 	     sie ist dein Mutter und deine Herrin \\
	% \sn \gll Daz ivch got beídív ſchowe \\
	% 	     dass \textsc{2pl\subMF.acc} Gott beide-\textsc{acc.pl.\NeutMF.st} schaue \\
	% 	\trans \wdef{Sohn, du sollst sie lieb haben. Sie ist deine Mutter und
	% 		deine Herrin. Möge Gott seinen Blick auf euch beide richten.}
	% 		(%
	% 			VB:~14rb,40--42;
	% 			14/12~Wortformen, anderer Satz%
	% 		)
	% \end{xlist}
\end{exe}

Die beiden Belege mit neutralem \norm{bėidiu} mit Bezug auf \lit{papſt}
\wdef{Papst} und \lit{chv̂nich} \wdef{König} beziehungsweise \lit{babſt}
\wdef{Papst} und \lit{im} \wdef{ihm} in \REF{ex:combgenddist2} treten erst in
der Distanzklasse 3 auf (Abstand von 11 bis 15 Wortformen). Im \CAO{} erscheint
umgekehrt die Form \norm{bėide} bei belebten\is{Animata} Targets mit
verschieden\-geschlechtlichen Controllern ab einer Distanz von 6 bis 10
Wortformen, während bei Kombinationen mit gleichem Geschlecht ausschließlich
\norm{bėide} belegt ist.

\begin{exe}
\ex \label{ex:combgenddist2}
	\begin{xlist}
	\ex\label{ex:papstkoenig6} % 224
		\gll Der papſt vnd der chv̂nich \\
			der Papst[\textsc{nom.sg.\MascM}] und der
			König[\textsc{nom.sg.\MascM}] \\
	\sn \gll Si warn zegot biderb vnd frumic \\
			\textsc{3pl\subM.nom} waren {zu=Gott} brav und tüchtig \\
	\sn \gll Zegot ſtuͦnt allr ir geſín \\
			{zu=Gott} stand aller ihr Sinnen \\
	\sn \gll Beideu ſchatz vnd gewín \\
			beide Schatz und Gewinn \\
	\sn \gll Liezzen ſi beideu gelich \\
			ließen \textsc{3pl\subM.acc} beide-\textsc{acc.pl.\NeutM.st}
			gleich \\
		\trans \wdef{Der Papst und der König, sie waren Gott gegenüber brav und
			tüchtig. Auf Gott war all ihr Sinnen gerichtet. Sowohl Schatz als
			auch Gewinn war ihnen beiden gleich.}
			(%
				B1:~17vb,30--34; vgl.~abweichend
				\KC:~V.~6110--6113;
				\cite[194]{schroeder1895};
				20/16~Wortformen, anderer Satz%
			)

	\protectedex{
	\ex\label{ex:babstimbaideu2}
		\gll Karle an daz gerichte ſaz \\
			Karl[\textsc{nom.sg.\MascM}] an das Gericht saß \\
	\sn \gll Der babſt klegt ím daz \\
			der Papst[\textsc{nom.sg.\MascM}] klagte \textsc{3sg.\MascM.dat} dass \\
	\sn \gll Der wideme vnd der zehende gar \\
			der Dotierungen und der Zehnten gar \\
	\sn \gll Waͤren baidu̍ worden bar \\
			wären beide-\textsc{nom.pl.\NeutM.st} geworden ledig \\
	\sn \gll Von ſínen vorvarn \\
			von seinen Vorfahren \\
		\trans \wdef{Karl setzte sich zu Gericht. Der Papst klagte ihm, dass
			beide an Dotierungen und gar an Zehnten ledig geworden wären durch
			seine Vorfahren.}
			(%
				K:~85vb,22--24; vgl. abweichend
				\KC:~V.~14383--14385;
				\cite[339]{schroeder1895};
				16/10~Wortformen, anderer Satz%
			)
	}
	\end{xlist}
\end{exe}

Die verfügbaren Belege verhalten sich unabhängig von der Distanz zwischen
(Erst-)Controller und Target regelmäßig in Einklang mit den Beobachtungen zu
den Personenmerkmalen (\sectref{subsubsec:persfeatsmry}). Eine auffällige Zu-
oder Abnahme sowohl von \norm{bėide} als auch \norm{bėidiu} mit wachsendem
Abstand ist nicht zu beobachten. Die Urkunden\is{Urkunde} des \CAO{} weisen
tendenziell mehr Variation bei gemischtgeschlechtlichen Controllern auf (vgl.\
\tabref{tab:caodist}), wobei in den Urkunden insgesamt auch wesentlich mehr
Paare aus Mann und Frau auftreten als in der \KC{}.

\is{Erstcontroller|)}
\is{Distanz!lineare|)}
\is{Distanz|)}
\is{Controller|)}
\is{Target|)}

%%%%%%%%%%%%%%%%%%%%%%%%%%%%%%%%%%%%%%%%%%%%%%%%%%%%%%%%%%%%%%%%%%%%%%%%%%%%%%%

\section{\norm{Bėide} als Konjunktion}
\label{sec:kckonjunktion}
\is{Konjunktion|(}

Wesentlich häufiger als die Verwendung von \norm{bėide} \wdef{beide} als
Quantor ist die Verwendung als Konjunktion anzutreffen. Insgesamt wurden 307
Belege aus dem \KC{}-Material extrahiert, wobei auch hier nicht alle
Handschriften relevant sind (vgl.\ \tabref{tab:kcbeidefuncvar}), sodass
schließlich 79 Belege in die Analyse eingeflossen sind. Um einer der von
\citet{gjelsten1980} benannten Schwächen in \posscite{askedal1974} Untersuchung
aus dem Weg zu gehen, wurden Belege, in denen der Quantor vor einem Vokal steht
und seine Flexion sich somit in einer potenziellen Neutralisierungsposition
befindet, nicht in die Auszählung einbezogen \autocite[vgl.][191--193,
201]{gjelsten1980}. Belege mit \norm{bėide} im Reim\is{Versende}
\autocites[vgl.][662--663]{grimm1870}[89]{askedal1973} wurden im Kontext der
Konjunktion bei der Belegsammlung keine vermerkt.

Unabhängig von der Frage, ob sich bei der Konjunktion \norm{bėide \dots\ unde}
\wdef{sowohl \dots\ als auch} Personenmerkmale\is{Personenmerkmal} der
Konjunkte auf die Form der Konjunktion auswirken, ist die Verteilung der Formen
bei manchen der Handschriften in \tabref{tab:kcbeidefuncvar} auffällig. In den
untersuchten bairischen\il{Bairisch} Handschriften A1, B1, C1, M und VB sowie
in der mittelalemannischen\il{Alemannisch} Handschrift K deutet sich trotz
unterschiedlich großer Belegzahlen für den Quantor gegenüber der Konjunktion
ein mehr oder weniger großer Unterschied in der Verwendung zwischen der Form
des Quantors und der Form der Konjunktion an.%
%
	\footnote{In der Tabelle steht \q{--} für \emph{kein Unterschied},
	\q{(\chk)} für \emph{eher unterschieden} und \q{\chk} für
	\emph{Unterschied}.}
%
Um der \isi{Diachronie} Rechnung zu tragen, sind in \tabref{tab:kcbeidefuncvar}
die untersuchten Handschriften nach der Reihenfolge ihrer ungefähren
Entstehungszeit aufgelistet. Die Auflistung macht deutlich, dass vor dem
14.~Jahrhundert in den bairischen\il{Bairisch} Handschriften bei der
Konjunktion noch beide Formen vorkommen, ab dem 14.~Jahrhundert dann bei der
Konjunktion bis auf wenige Ausnahmen regelmäßig \norm{bėidiu} erscheint.

\begin{sidewaystable}
\centering
\caption%
{Flexionsform von \norm{bėide} in den exzerpierten Textzeugen nach Funktion}
\begin{tabular}[t]{
	l c
	l
	r r
	r r
	c
}
\lsptoprule

\mr[c]{2}{*}[-.5ex]{Hs.}
	& \mr[c]{2}{*}[-.5ex]{Rez.}
	& \mr[c]{2}{*}[-.5ex]{Entstehungszeit}
	& \mc{2}{c}{Quantor}
	& \mc{2}{c}{Konjunktion}
	& \mr[c]{2}{*}[-.5ex]{\makecell[c]{unterschiedl. \\ Verhalten}}
	\\

\cmidrule(rl){4-5}
\cmidrule(rl){6-7}

%
	& %
	& %
	& \norm{bėid(e)}
	& \norm{bėidiu}
	& \norm{bėid(e)}
	& \norm{bėidiu}
	\\

\midrule

A1
	& A
	& 4.~Viertel 12.~Jh.
	& 11
	& 
	& 15
	& 19
	& (\chk)
	\\

\midrule

P
	& B
	& 2.~Viertel 13.~Jh.
	& 6
	& 
	& 15
	& 
	& ---
	\\

H
	& A
	& um 1250
	& 11
	& 
	& 33
	& 
	& ---
	\\

VB
	& B
	& um 1290/1300
	& 9
	& 1
	& 22
	& 23
	& (\chk)
	\\

\midrule

C1
	& C
	& Anfang 14.~Jh. \mkbibparens{um 1320}
	& 7
	& 2
	& 
	& 32
	& \chk
	\\

B1
	& B
	& 2.~Viertel 14.~Jh.
	& 11
	& 6
	& 2
	& 27
	& (\chk)
	\\

M
	& A
	& 1330/40er Jahre
	& 7
	& 
	& 
	& 38
	& \chk
	\\

K
	& C
	& 2.~Hälfte 14.~Jh. \mkbibparens{vor 1378}
	& 9
	& 1
	& 1
	& 30
	& \chk
	\\

\midrule

Z
	& C
	& Mitte 15.~Jh. \mkbibparens{vor 1467}
	& 9
	& 
	& 29
	& 
	& ---
	\\

\lspbottomrule
\end{tabular}
\label{tab:kcbeidefuncvar}
\end{sidewaystable}

Wie zuvor beobachtet, liegt in den mitteldeutschen\il{Mitteldeutsch}
Handschriften H und P kein Unterschied in der Form der Konjunktion vor
\autocite[vgl.][181]{ksw2}. Auch in der späten schwäbischen\il{Schwäbisch}
Handschrift Z existieren weder beim Quantor noch bei der Konjunktion Formen vom
Typ \norm{bėidiu}. Dies dürfte mit dem Abbau\is{Genusdistinktion, Abbau der}
von \norm{-iu} zugunsten von \norm{-e} im oberdeutschen\il{Oberdeutsch}
Sprachraum\is{Dialektgeografie} des Frühneuhochdeutschen\il{Frühneuhochdeutsch}
zu erklären sein, der sich im Schwäbischen\il{Schwäbisch} spätestens ab der
zweiten Hälfte des 15.~Jahrhunderts bemerkbar macht (Dominanz von \norm{-e} ab
dem 16.~Jahrhundert; \cites[vgl.][210]{moserstopp1978}[120]{solmswegera1991}).

Unter den übrigen Handschriften liegt in A1 zwar keine Variation beim Quantor
vor, dafür aber bei der Konjunktion. In B1 und VB treten in beiden Kontexten
beide Formen auf, wobei der Quantor in VB nahezu ausschließlich die Form
\norm{bėide} hat, während bei der Konjunktion die Zahl der Belege nahezu
ausgeglichen ist. In K steht der einzige Beleg für \norm{bėidiu}
unregelmäßig\is{Ausnahme} im Bezug auf zwei Maskulina. In B1 variiert der
Quantor, während die Konjunktion in nahezu allen Fällen eine Form des Typs
\norm{bėidiu} aufweist. In den Handschriften C1, K und M liegt dagegen ein mehr
oder weniger deutlicher Unterschied in der Verwendung von \norm{bėide} als
Quantor einerseits und \norm{bėidiu} als Konjunktion andererseits vor. Aufgrund
der Tatsache, dass die \isi{Stichprobe} zu A1, K, M und VB inhaltsbedingt fast
ausschließlich maskuline \isi{Controller} enthält, ist nicht davon auszugehen,
dass der Quantor dort systematisch stets \norm{bėide} lautet.

\tabref{tab:beidefuncpos} listet \norm{bėid(e)} gegenüber \norm{bėidiu} nach
Personenmerkmalhaltigkeit\is{Personenmerkmal} der verschiedenen im
\KC{}-Material belegten koordinierten Wortarten und Phrasentypen auf und
sortiert die einzelnen Handschriften nach Variation in der Form der Konjunktion
und Entstehungszeit. Die größte Varia\-tion herrscht demnach in den
Handschriften A1 und VB, wobei in A1 keine Belege für \norm{bėide \dots\ unde}
in rein syntaktischen Kontexten auftreten, das heißt, mit Konjunkten ohne
Personen\-merkmale. Hieraus ergibt sich die Frage nach dem Grad und der Zeit
der \isi{Grammatikalisierung} der Konstruktion (\sectref{sec:beideconj}). Auch
in den verhältnismäßig frühen mitteldeutschen\il{Mitteldeutsch} Handschriften H
(zu Rezension A) und P (zu Rezension B) fehlen entsprechende Belege oder es
sind kaum welche vorhanden.

\begin{sidewaystable}
\centering
\caption{Form nach Wortart/Phrasentyp der Konjunkte}
\begin{tabular}[t]{
	l c
	l
	r r
	r r
	r r
}
\lsptoprule

\mr[c]{2}{*}[-.5ex]{Hs.}
	& \mr[c]{2}{*}[-.5ex]{Rez.}
	& \mr[c]{2}{*}[-.5ex]{Entstehungszeit}
	& \mc{2}{c}{N}
	& \mc{2}{c}{Adj, V}
	& \mc{2}{c}{Adv, PP}
	\\

\cmidrule(rl){4-5}
\cmidrule(rl){6-7}
\cmidrule(rl){8-9}

%
	& %
	& %
	& \textit{bėid(e)}
	& \textit{bėidiu}
	& \textit{bėid(e)}
	& \textit{bėidiu}
	& \textit{bėid(e)}
	& \textit{bėidiu}
	\\

\midrule

A1
	& A
	& 4.~Viertel 12.~Jh.
	& 13
	& 16
	& 2
	& 3
	& 
	& 
	\\

VB
	& B
	& um 1290/1300
	& 17
	& 18
	& 2
	& 3
	& 3
	& 2
	\\

\midrule

C1
	& C
	& Anfang 14.~Jh. \mkbibparens{um 1320}
	& 
	& 27
	& 
	& 
	& 
	& 5
	\\

B1
	& B
	& 2.~Viertel 14.~Jh.
	& 
	& 21
	& 1
	& 2
	& 1
	& 4
	\\


M
	& A
	& 1330/40er Jahre
	& 
	& 31
	& 
	& 5
	& 
	& 2
	\\

K
	& C
	& 2.~Hälfte 14.~Jh. \mkbibparens{vor 1378}
	& 
	& 25
	& 
	& 
	& 
	& 5
	\\

\midrule

P
	& B
	& 2.~Viertel 13.~Jh.
	& 13
	& 
	& 1
	& 
	& 1
	& 
	\\

H
	& A
	& um 1250
	& 28
	& 
	& 5
	& 
	& 
	& 
	\\

Z
	& C
	& Mitte 15.~Jh. \mkbibparens{vor 1467}
	& 24
	& 
	& 
	& 
	& 5
	& 
	\\

\lspbottomrule
\end{tabular}
\label{tab:beidefuncpos}
\end{sidewaystable}

Obwohl sich \citet{gjelsten1980} und \citet{ksw2} für das
Mittelhochdeutsche\il{Mittelhochdeutsch} ausdrücklich gegen Variation der
Konjunktion \norm{bėide} aufgrund von Kongruenz aussprechen~--
\citet{gjelsten1980} gibt für die von ihr untersuchten (aber unbezeichneten)
Texte \isi{freie Variation} an~--, sollen im Folgenden die Verhältnisse für die
in dieser Untersuchung verwendete Auswahl an \KC{}-Handschriften anhand der
Personen\-merkmale\is{Personenmerkmal} der Konjunkte nachvollzogen werden.

\subsection{Mit zwei Controllern}
\label{subsec:konj2ctrl}
\is{Controller|(}

Wie aus \tabref{tab:beidefuncpos} deutlich wird, weisen die Handschriften M und
B1, sämtliche Handschriften der C-Rezension (C1, K, Z) sowie die
mitteldeutschen\il{Mitteldeutsch} Handschriften H und P keine Variation in der
Form der Konjunktion auf. In allen relevanten Kontexten steht entweder eine
Form des Typs \norm{bėide} (H, P, Z) oder des Typs \norm{bėidiu} (B1, C1, K,
M). Bei der folgenden Untersuchung werden daher nur die Handschriften A1 und VB
berücksichtigt, die im syntaktischen Kontext \norm{bėide} \textsc{substantiv}
\norm{unde} \textsc{substantiv}\is{Substantiv} Variation aufweisen.

Bisweilen erscheint problematisch, dass die Konjunkte generisch\is{generischer
Gebrauch} verwendet werden \REF{ex:beideundegnrc}. Bei Lexemen wie \norm{man}
\wdef{Mann} und \norm{wīp} \wdef{Frau} ist in solchen Fällen im Zusammen\-hang
nicht immer eindeutig,\is{Ambiguität} ob das jeweilige Konjunkt im Singular
oder im Plural steht, da die Form im Singular und Plural die gleiche ist: Bei
\norm{man} handelt es sich historisch\is{Diachronie} um ein Wurzelnomen, bei
\norm{wīp} um ein starkes Neutrum der \fw{a}-Deklination \autocites[353--354,
584]{kroonen2013}[254, 305]{braune2023}. Darüber hinaus sind gerade
Kombinationen von maskulin-männlichen mit feminin-weiblichen Controllern
($\MascM+\FemF$) nicht belegt, sondern nur solche mit neutral-weiblichen
($\MascM+\NeutF$).\is{Desiderat}

\begin{exe}
\ex \label{ex:beideundegnrc}
	\gll daz allez ſin hibiſc bekert wart. \\
		dass alles sein Geschlecht bekehrt wurde \\
\sn \gll baide wip unde man. \\
		beide Frau[\textsc{nom.sg+pl.\NeutF}] und Mann[\textsc{nom.sg+pl.\MascM}] \\
	\trans \wdef{Dass seine ganze Sippe bekehrt wurde, Frau(en) wie Mann (Männer).}
		(%
			A1:~27va,9--11; vgl.
			\KC:~V.~6360--6361;
			\cite[198]{schroeder1895}%
		)
\end{exe}

Insgesamt verteilen sich die Belege aus A1 und VB in
\tabref{tab:konjnomperskc} für belebte\is{Animata} Konjunkte nahezu gleichmäßig
auf \norm{bėide} und \norm{bėidiu}. Bei der gewählten Einteilung der Tabelle
nach Konjunkten mit gleichem und unterschiedlichem semantischen oder
grammatischen Geschlecht fällt auf, dass zumindest tendenziell die Variation
zwischen \norm{e} und \norm{iu} in Fällen von unterschiedlichem Geschlecht
größer ist als bei gleichem, wobei keine Belege mit feminin-weiblichen
Konjunkten vorliegen, sodass diesem Sachverhalt nicht weiter nachgegangen
werden kann.\is{Desiderat} Trotz allem liegen in der Gruppe der belebten
Konjunkte mit gemischtem \isi{Genus} beziehungsweise \isi{Sexus} zehn Belege
mit \norm{bėidiu} gegenüber sieben Belegen mit \norm{bėide} vor.

\begin{sidewaystable}
\centering
\caption{Form nach Personenmerkmalen nominaler Konjunkte}
\begin{tabular}{
	l
	r r c r r
	c
	r r c r r
	r
}

\lsptoprule

\mr{3}{*}[-1ex]{Hs.}
	& \mc{5}{c}{belebt}
	& %
	& \mc{5}{c}{unbelebt}
	& \mr{3}{*}[-1ex]{Summe}
	\\

\cmidrule{2-6}
\cmidrule{8-12}

%
	& \mc{2}{c}{gleich}
	& %
	& \mc{2}{c}{verschieden}
	& %
	& \mc{2}{c}{gleich}
	& %
	& \mc{2}{c}{verschieden}
	& %
	\\

\cmidrule{2-3}
\cmidrule{5-6}
\cmidrule{8-9}
\cmidrule{11-12}

%
	& \norm{bėid(e)}
	& \norm{bėidiu}
	& %
	& \norm{bėid(e)}
	& \norm{bėidiu}
	& %
	& \norm{bėid(e)}
	& \norm{bėidiu}
	& %
	& \norm{bėid(e)}
	& \norm{bėidiu}
	& %
	\\

\midrule

A1
	& % belebt gleich beide
	& % belebt gleich beidiu
	& %
	&  4 % belebt verschieden beide
	&  6 % belebt verschieden beidiu
	& %
	&  2 % unbelebt gleich beide
	&  3 % unbelebt gleich beidiu
	& %
	&  4 % unbelebt verschieden beide
	&  4 % unbelebt verschieden beidiu
	& 23 % Summe
	\\

VB
	&  6 % belebt gleich beide
	&  1 % belebt gleich beidiu
	& %
	&  3 % belebt verschieden beide
	&  4 % belebt verschieden beidiu
	& %
	&  4 % unbelebt gleich beide
	&  4 % unbelebt gleich beidiu
	& %
	&  4 % unbelebt verschieden beide
	&  8 % unbelebt verschieden beidiu
	& 34 % Summe
	\\

\midrule

Summe
	&  6 % belebt gleich beide
	&  1 % belebt gleich beidiu
	& %
	&  7 % belebt verschieden beide
	& 10 % belebt verschieden beidiu
	& %
	&  6 % unbelebt gleich beide
	&  7 % unbelebt gleich beidiu
	& %
	&  8 % unbelebt verschieden beide
	& 12 % unbelebt verschieden beidiu
	& 57 % Summe
	\\

\lspbottomrule
\end{tabular}
\label{tab:konjnomperskc}
\end{sidewaystable}

Geht man angesichts dieser Beleglage trotz der Einwände von
\citet{gjelsten1980} und \citet{ksw2} davon aus, dass
Personenmerkmale\is{Personenmerkmal} in der \KC{} (noch) einen Einfluss auf die
Form der Konjunktion haben könnten, oder aber die konjunktionale Verwendung von
\norm{bėide} nicht klar von der pronominal-kataphorischen\is{Katapher}
unterschieden werden kann,\is{Ambiguität} erscheint die neutrale Form der
Konjunktion in \REF{ex:rittergebure} mit \lit{ritter} \wdef{Ritter} und
\lit{gebovre} \wdef{Bauern} als Kombination zweier eindeutiger Maskulina als
unregelmäßig\is{Ausnahme}, wenn sie überhaupt dekliniert ist. In
\sectref{sec:adjdeclkc} wurde zwar berichtet, dass in VB in der
Adjektivstichprobe teilweise ein Nebeneinander von \norm{-e} und \norm{-iu}
oder \lit{-eu} im Plural Neutrum vorliegt.\is{Adjektivdeklination} Formen mit
\norm{-iu} kommen umgekehrt aber nicht im Plural Maskulinum und Femininum vor.

\begin{exe}
\ex\label{ex:rittergebure}
	\gll Beidiv ritter vnd gebovre \\
		beide Ritter[\textsc{acc.pl.\MascM}] und Bauer-\textsc{acc.pl.\MascM} \\
	\trans \wdef{sowohl Ritter als auch Bauern}
		(%
			VB:~109va,29%
		)
\end{exe}

Auch die Belege in \REF{ex:konjmixbeide1} und \REF{ex:konjmixbeide2} erscheinen
nach den bisherigen Beobachtungen zur Flexion von \norm{bėide} in Abhängigkeit
von semantischen Personenmerkmalen\is{Personenmerkmal}\is{Merkmale!semantische}
als unregelmäßig\is{Ausnahme}. In beiden Fällen werden in
\REF{ex:konjmixbeide1} \wdef{Mann} und \wdef{Frau} kombiniert. In diesem
Kontext wäre bei pronominaler Verwendung von \norm{bėide} mit
\isi{Genusresolution} zu rechnen, doch steht trotz allem eine Form vom Typ \norm{bėide}
statt \norm{bėidiu}.

\begin{exe}
\ex \label{ex:konjmixbeide1}
	\begin{xlist}
	\ex \gll Daz beide man vnd wíp \\
			dass beide Mann[\textsc{nom.sg.\MascM}] und Frau[\textsc{nom.sg.\NeutF}] \\
	\sn \gll Sín alſ eíne~\textins{sic} líp \\
			seien wie ein Leib \\
		\trans \wdef{dass sowohl Mann als auch Frau wie ein Leib sein sollen}
			(%
				VB:~45vb,12--13; vgl.
				\KC:~V.~9536--9537;
				\cite[257]{schroeder1895}%
			)
		\label{ex:konjmixbeide1_1}

	\ex \label{ex:konjmixbeide1_2}
		\gll Eineſ tageſ der fromden wart erſlagen \\
			Eines Tages der Fremden wurde erschlagen \\
		\textelp{}
	\sn \gll Beide manne vnd wibe \\
			beide Mann-\textsc{nom.pl.\MascM} und Frau-\textsc{nom.pl.\NeutF} \\
		\trans \wdef{Eines Tages wurden von den Fremden erschlagen \textelp{} 
			sowohl Männer als auch Frauen.}
			(%
				VB:~3vb,11--14; vgl.
				\KC:~V.~625--628;
				\cite[93]{schroeder1895}%
			)
	\end{xlist}
\end{exe}

Bei umgekehrter Reihung der Konjunkte zeigt sich, wie in \REF{ex:konjmixbeide2}
exemplarisch illustriert, kein Unterschied in der Form der Konjunktion, sodass
auch Kongruenz mit dem nächsten Konjunkt\is{Kongruenz!partielle}
(\sectref{sec:gendres}) ausgeschlossen werden kann. In allen Fällen der
Kombination von \lit{wip} \wdef{Frau (\NeutF)} und \lit{man} \wdef{Mann
(\MascM)} steht die Form \lit{baide}.

\begin{exe}
\ex \label{ex:konjmixbeide2}
	% \begin{xlist}
	% \ex \label{ex:konjmixbeide2_1}
		\gll baide wip und man. \\
			beide Frau[\textsc{nom.pl.\NeutF}] und Mann[\textsc{nom.pl.\MascM}] \\ % \medskip
	\sn \gll ſciden alle weíninde dann. \\
			schieden alle weinend dannen \\
		\trans \wdef{Sowohl Frauen als auch Männer verabschiedeten sich alle
			weinend.}
			(%
				A1:~7ra,38--39; vgl.
				\KC:~V.~1564--1565;
				\cite[110]{schroeder1895}%
			)

	% \ex \label{ex:konjmixbeide2_2}
	% 	\gll baide wip unde man.~/ \\
	% 		beide Frau[\textsc{nom.pl.\NeutF}] und Mann[\textsc{nom.pl.\MascM}] \\ % \medskip
	% \sn \gll allenthalben zuſan. \\
	% 		allenthalben zusahen \\
	% 	\trans \wdef{Sowohl Frauen als auch Männer sahen allenthalben zu.}
	% 		(%
	% 			A1:~58va,44--45; vgl.
	% 			\KC:~V.~13573--13574;
	% 			\cite[327]{schroeder1895}%
	% 		)

	% \ex \label{ex:konjmixbeide2_3}
	% 	\gll daz allez ſin hibiſc bekert wart. \\
	% 		dass alles sein Geschlecht bekehrt wurde \\
	% \sn \gll baide wip unde man. \\
	% 		beide Frau[\textsc{nom.pl.\NeutF}] und Mann[\textsc{nom.pl.\MascM}] \\
	% 	\trans \wdef{Dass seine ganze Familie bekehrt wurde, Frau(en) wie Mann (Männer).}
	% 		(%
	% 			A1:~27va,9--11;
	% 			\KC:~V.~6360--6361;
	% 			\cite[198]{schroeder1895}%
	% 		)

	% \ex \label{ex:konjmixbeide2_4}
	% 	\gll Sie wolden die ſtat gewínnen \\
	% 		Sie wollten die Stadt einnehmen \\
	% \sn \gll Ez enmvͤſe gar verbrínnen \\
	% 		es \textsc{neg}=müsse gar verbrennen \\
	% \sn \gll Beide wip vnd man \\
	% 		beide Frau[\textsc{nom.pl.\NeutF}] und Mann[\textsc{nom.pl.\MascM}] \\
	% 	\trans \wdef{Sie wollten die Stadt einnehmen, selbst wenn
	% 	sowohl Frauen als auch Männer \textins{dabei} verbrennen müssten.}
	% 		(%
	% 			VB:~21vb,7--8; vgl.~abweichend
	% 			\KC:~V.~4379--4381;
	% 			\cite[161]{schroeder1895}%
	% 		)
	% \end{xlist}
\end{exe}

In den Tabellenspalten mit unbelebten\is{Inanimata} Konjunkten in
\tabref{tab:konjnomperskc} ist das Verhältnis insgesamt nahezu ausgeglichen,
mit leichter Tendenz zu \norm{bėidiu}. Auffällig ist hier vor allem der leichte
Überhang von \norm{bėidiu} gegenüber \norm{bėide} bei der Kombination von
Nomina mit unterschiedlichem \isi{Genus}, verglichen mit Kombinationen von
Nomina mit gleichem Genus. Bei gleichem Genus stehen unabhängig vom Numerus
insgesamt sechs Belege mit \norm{-e} sieben Belegen mit
\norm{-iu} gegenüber; bei verschiedenem Genus beläuft sich die Zahl auf
insgesamt acht Belege mit \norm{-e} gegenüber insgesamt zwölf Belegen mit
\norm{-iu}.

Bei Handschrift VB verteilen sich die Belege für \norm{bėide} und
\norm{bėidiu} bei übereinstimmendem unbelebten\is{Inanimata} \isi{Genus}
gleichmäßig. Im Fall von Konjunkten mit verschiedenem Genus liegt in VB
eine Präferenz für \norm{bėidiu} vor; bei A1 verteilen sich die Belege wieder
gleichmäßig auf beide Formen. Ausgewählte Beispiele zu \q{Irregularitäten}
finden sich in \REF{ex:konjbeidirreginan}.

\begin{exe}
\ex \label{ex:konjbeidirreginan}
	\begin{xlist}
	\ex \gll Beidiv ſchatz vnd gewín \\
			beide Schatz[\textsc{nom.sg.\MascI}] und Gewinn[\textsc{nom.sg.\MascI}] \\
		\trans \wdef{sowohl Schatz als auch Gewinn}
			(%
				VB:~29vb,41; vgl.
				\KC:~V.~6112;
				\cite[194]{schroeder1895}%
			)
		\label{ex:konjbeidirreginan_1}

	\ex \gll baide ſilber unt golt \\
			beide Silber[\textsc{acc.sg.\NeutI}] und Gold[\textsc{acc.sg.\NeutI}] \\
		\trans \wdef{sowohl Silber als auch Gold}
			(%
				A1:~68rb,43; vgl.
				\KC:~V.~16012;
				\cite[370]{schroeder1895}%
			)
		\label{ex:konjbeidirreginan_2}

	\ex \gll beide geuoͮre vnt ere \\
			beide Nutzen[\textsc{acc.sg.\NeutI}] und Ansehen[\textsc{acc.sg.\FemI}] \\
		\trans \wdef{sowohl Nutzen als auch Ansehen}
			(%
				A1:~4ra,16--17; vgl.
				\KC:~V.~772;
				\cite[95]{schroeder1895}%
			)
		\label{ex:konjbeidirreginan_3}
	\end{xlist}
\end{exe}

Bei der Kombination von \lit{ſchatz} \wdef{Schatz} und \lit{gewín}
\wdef{Gewinn} in \REF{ex:konjbeidirreginan_1} zeigt die Konjunktion die sonst
eindeutig neutrale Form entgegen dem übereinstimmend maskulinen \isi{Genus} der
beiden Konjunkte. In \REF{ex:konjbeidirreginan_2} liegen mit \lit{ſilber}
\wdef{Silber} und \lit{golt} \wdef{Gold} ebenfalls Konjunkte mit
übereinstimmendem Genus der Konjunkte vor, insofern es sich jeweils um Neutra
handelt. Die Konjunktion tritt dabei mit dem \norm{e}-Suffix auf, das in dieser
Handschrift regelmäßig für Maskulina und Feminina Plural bezeugt ist. Nicht
ersichtlich scheint die Form \norm{bėide} auch in \REF{ex:konjbeidirreginan_3},
wo das Neutrum \lit{geuoͮre} \wdef{Nutzen} und das Femininum \lit{ere}
\wdef{Ansehen, Ruhm} miteinander kombiniert werden.

Bei der Betrachtung der obigen Belege aus A1 und VB nach
\isi{Kasus} in \tabref{tab:kckoordnomctrlcase} verteilen sich die
konjunktionalen \norm{bėide}-Belege ähnlich gleichmäßig auf die zwei Formen wie
in der äquivalenten Aufstellung für das \CAO{} in
\tabref{tab:caokoordnomctrlcase}. Für den Dativ wurde ein einziger Beleg
gefunden \REF{ex:datregul}. Darüber hinaus liegt für die Kombination mit einem
geschlechtlich uneindeutigen\is{Ambiguität} belebten\is{Animata} Konjunkt noch
mindestens ein weiterer Beleg für \norm{bėidiu} vor \REF{ex:datextra}.
% Weiterhin findet sich mit \REF{ex:datextra_2} noch ein Beleg für
% \norm{bėide}, der dem Kontext nach im Dativ stehen dürfte, allerdings ist
% auch der Sexus seiner Konjunkte nicht eindeutig.

\begin{exe}
\ex \label{ex:datregul}
	\gll mit ſiden unbehangen \\
		mit Seide umhangen \\
\sn \gll beidiv golt unt wat. \\
		beide Gold[\textsc{dat.sg.\NeutI}] und Gewandstoff[\textsc{dat.sg.\FemI}] \\
\sn \gll vnt ander ſlahte rat. \\
		und ander Art Zeug \\
	\trans \wdef{mit Seide umhangen, sowohl Gold als auch Gewandstoff und
		anderlei Zeug}
		(%
			A1:~7ra,23--25; vgl.
			\KC:~V.~1543--1545;
			\cite[109]{schroeder1895}%
		)

\ex \label{ex:datextra}
	% \begin{xlist}
	% \ex \label{ex:datextra_1}
		\gll er hiz ſcriben allen diſen umberinch. \\
			er hieß schreiben all-\textsc{dat.pl.st} dies-\textsc{dat.pl.st} ringsum \\
	\sn \gll baidiv muter unt kint \\
			beide Mutter[\textsc{dat.sg.\FemF}] und Kind[\textsc{dat.sg.\NeutA}] \\
		\trans \wdef{Er wies an, allen diesen ringsum zu schreiben: sowohl
			Mutter als auch Kind}
			(%
				A1:~3rb,46--3va,1; vgl.
				\KC:~V.~617--618;
				\cite[92]{schroeder1895}%
			)

	% \ex \label{ex:datextra_2}
	% 	\gll Beid arme vnd riche \\
	% 		beide Arme[\textsc{dat.pl.\MascA}] und
	% 			Reiche[\textsc{dat.pl.\MascA}] \\
	% \sn \gll Er bevalch in flízzicliche \\
	% 		er anbefehlen \textsc{3pl.dat} eifrig \\
	% \sn \gll Sin ſvn alle viere \\
	% 		sein Sohn[\textsc{acc.pl.\MascM}] alle vier \\
	% 	\trans \wdef{Sowohl Arme\textins{n} als auch Reiche\textins{n}: Er vertraute
	% 		ihnen eifrig seine Söhne alle vier an.}
	% 		(%
	% 			VB:~94rb,23--26; zu
	% 			\KC:~V.~13854--13855;
	% 			\cite[332]{schroeder1895}%
	% 		)
	% \end{xlist}
\end{exe}

Belege mit unspezifischem\is{Ambiguität} Geschlecht wie in \REF{ex:datextra}
wurden auch hier ausgelassen. Darüber hinaus kommt \norm{bėidiu naht unde tac}
\wdef{sowohl Nacht als auch Tag} mehrfach als temporales
Adverbial\is{Adverbial} vor, bei dem die Konjunkte als Akk.\ Sg.\ gewertet
wurden. Dies ist auch der Grund für die hohe Belegzahl für \norm{bėidiu} bei
unbelebten\is{Inanimata} Akkusativen mit verschiedenem \isi{Genus}. Im
Gegensatz dazu korrespondiert die äußerst geringe Anzahl von Dativ-Belegen mit
dem auffälligen Fehlen von Formulierungen wie \lit{beidiv in vnd den
langenmænteln} \wdef{sowohl ihnen als auch den Langenmänteln}
\autocites(Nr.~249, Augsburg, 1280)[382,8]{cao1}, also von Konjunkten, die
Determiniererphrasen (DPs) enthalten, nicht nur im Dativ. Möglicherweise liegt
der Grund dafür in der Versstruktur der \KC{}, obwohl gerade die A-Rezension
sehr lose mit Metrum und Reim\is{Vers} umgeht
\autocite[92]{wolf2008}.

\begin{sidewaystable}
\centering
\caption{Form nach dem Kasus nominaler Konjunkte}
\begin{tabular}{
	l
	r r c r r
	c
	r r c r r
	r
}
\lsptoprule
\mr{3}{*}[-1ex]{\normalfont Kasus}
	& \mc{5}{c}{belebt}
	& % --
	& \mc{5}{c}{unbelebt}
	& \mr{3}{*}[-1ex]{Summe}
	\\

\cmidrule{2-6}
\cmidrule{8-12}

%
	& \mc{2}{c}{gleich}
	& % --
	& \mc{2}{c}{verschieden}
	& % --
	& \mc{2}{c}{gleich}
	& % --
	& \mc{2}{c}{verschieden}
	& % --
	\\

\cmidrule{2-3}
\cmidrule{5-6}
\cmidrule{8-9}
\cmidrule{11-12}

%
	& \norm{bėid(e)}
	& \norm{bėidiu}
	& % --
	& \norm{bėid(e)}
	& \norm{bėidiu}
	& % --
	& \norm{bėid(e)}
	& \norm{bėidiu}
	& % --
	& \norm{bėid(e)}
	& \norm{bėidiu}
	& % --
	\\

\midrule

\textsc{nom}
	& 4	% M+M: 4
	& %
	& % --
	& 6 % M+F: 2, F+M: 4
	& 5 % M+F: 2, F+M: 3
	& % --
	& 1 % F+F: 1
	& 2 % F+F: 1, N+N: 1
	& % --
	& %
	& %
	& 18
	\\

\textsc{acc}
	& %
	& 1 % M+M: 1
	& % --
	& %
	& %
	& % --
	& 5 % M+M: 2, F+F: 2, N+N: 1
	& 5 % M+M: 3, F+F: 1, N+N: 1
	& % --
	& 7  % M+F: 2, M+N: 1, F+N: 1, N+M: 1, N+F: 2
	& 13 % M+F: 2, M+N: 3, F+M: 5, F+N: 2, N+M: 1
	& 31
	\\

\midrule

\textsc{dat}
	& %
	& %
	& % --
	& %
	& %
	& % --
	& %
	& %
	& % --
	& %
	& 1 % N+F: 1
	& 1
	\\

\midrule

Summe
	& 4
	& 1
	& % --
	& 6
	& 5
	& % --
	& 6
	& 7
	& % --
	& 7
	& 14
	& 50
	\\

\lspbottomrule
\end{tabular}
\label{tab:kckoordnomctrlcase}
\end{sidewaystable}

Belege mit eindeutig flektiertem \norm{bėide} in Verbindung mit
\norm{unde} treten nur in Z auf \REF{ex:kczbeidenunde_1}. Dort steht
\norm{bėiden} im Dat.\ Pl.\ subsumierend mit Bezug auf die nachstehenden
Konjunkte. Die Parallelstellen\is{Paralleltext} enthalten dagegen unabhängig
vom \isi{Kasus} \norm{bėidiu} bis auf die mitteldeutschen\il{Mitteldeutsch}
Handschriften H und P mit entsprechendem \norm{bėide}. Der Text von K zu
\REF{ex:kczbeidenunde_1} weicht hier ab und formuliert \lit{\textins{da}rzuͦ man
vn̄ wip} \wdef{dazu Männern und Frauen} (K:~3va,22).
% Die in \REF{ex:kczbeidenunde_2} zitierte Stelle besitzt keine direkte
% Entsprechung in den A- und B-Rezensionen, da der C-Kaiserchronist hier
% umformuliert.

\begin{exe}
\protectedex{
\ex \label{ex:kczbeidenunde_1}
	% \begin{xlist}
	% \ex \label{ex:kczbeidenunde1_1}
		\gll Baiden man vnd wib \\
			beide-\textsc{dat.pl\subMF.st} Mann[\textsc{dat.sg.\MascM}] und
				Frau[\textsc{dat.sg.\NeutF}] \\
	\sn \gll Gebott er allen an den lyb \\
			gebot er all-\textsc{dat.pl\subMF.st} an den Leib \\
		\trans \wdef{Beiden, Mann und Frau, gebot er bei ihrem Leben}
			(%
				Z:~10va,9--10; vgl.
				C1:~3va,19--20; % beidev man vnd weip.
				K:~3va,22--23; abweichend % [da]rzuͦ man vnd(e) wip
				A1:~3va,1--2; % bediu wip unt man.
				M:~5va,16--17; % Baidív wip vnd man.
				H:~3vb,10--12; % Bede und(e) man.
				B1:~3vc,51--53; % Baideu weip vnde man
				VB:~3vb,5--7; % Beidiv wip vnd man
				P:~6ra,8--10; % beide wip vnd(e) man.
				\KC:~V.~619--621;
				\cite[92]{schroeder1895}%
			)

	% \ex \label{ex:kczbeidenunde_2}
	% 	\gll Baiden man vnd wÿb~/ \\
	% 		beide-\textsc{dat.pl\subMF.st} Mann[\textsc{dat.sg.\MascM}] und
	% 			Frau[\textsc{dat.sg.\NeutF}] \\
	% \sn \gll Sol es gan an den lÿb \\
	% 		soll es gehen an den Leib \\
	% 	\trans \wdef{Beiden, Mann und Frau, soll es ans Leben gehen.}
	% 		(%
	% 			Z:~15ra,18--19; vgl.
	% 			C1:~4vb,2; % baidev man vnd weip.
	% 			K:~4vb,33--34; zu % Baidu̍ man vnd(e) wip
	% 			\KC:~V.~846--849;
	% 			\cite[97]{schroeder1895}%
	% 		)
	% \end{xlist}
}
\end{exe}

\subsection{Mit zwei Targets}
\label{subsec:beidkoordtarg}
\is{Target|(}

Wie eingangs festgestellt, kommen Personenmerkmale\is{Personenmerkmal} auch in
solchen Kontexten zum Tragen, in denen zwei Kongruenztargets kombiniert werden.
Dies ist im ausgewerteten Belegmaterial zur \KC{} der Fall bei Adjektiven~--
sowohl attributiven \REF{ex:konjattradj}\is{Adjektiv!attributiv} als auch
koprädikativen \REF{ex:konjcopredadj}\is{Adjektiv!prädikativ}~-- sowie bei
Verben \REF{ex:konjvb}.

\largerpage % Damit die letzte Zeile der Stellenangabe nicht auf die nächste
			% Seite rutscht 😕
\begin{exe}
\ex \begin{xlist}
	\ex\label{ex:konjattradj}
		\gll Vnd maníge ſtolze froͮwen \\
			und viele herrliche Frau-\textsc{nom.pl.\FemF} \\%
	\sn \gll Beidív hoͤbſh vnd gemeit \\
			beide höfisch[\textsc{nom.pl.\FemF}] und lieblich[\textsc{nom.pl.\FemF}] \\
		\trans \wdef{und viele herrliche Edelfrauen, \textins{die waren} sowohl
			höfisch als auch lieblich} (%
				VB:~21va,21--22; zu
				\KC:~V.~4351--4352;
				\cite[161]{schroeder1895}%
			)

	\ex \label{ex:konjcopredadj}
		% Versumbruch durch "/" ersetzt wegen Seitenumbruch
		\gll oͮch gelagen dem chunige {/} \\
			auch lagen dem König \\
		\gll ſine tuerlichen helde. {/} \\
			seine herrlichen Helden-\textsc{nom.pl.\MascM} \\
		\gll baide woͮnt unt tot. \\
			beide verwundet[\textsc{nom.pl.\MascM}] und tot[\textsc{nom.pl.\MascM}] \\
		\trans \wdef{Auch lagen dem König seine herrlichen Helden sowohl
			verwundet als auch tot \textins{darnieder}.}
			(%
				A1:~67vb,34--36; vgl.
				\KC:~V.~15880--15883;
				\cite[368]{schroeder1895}%
			)

	\ex \label{ex:konjvb}
		\gll daz im der chaiſer willicliche. \\
			dass ihm der Kaiser[\textsc{nom.sg.\MascM}] willig \\
	\sn \gll baidiv leh unt gab. \\
			beide verleihen[\textsc{3sg\subM.ind.pst}] und
			geben[\textsc{3sg\subM.ind.pst}] \\
	\sn \gll des richen herzogen craft. \\
			des mächtigen Herzogs Gewalt \\
		\trans \wdef{dass ihm der Kaiser willig die Gewalt des mächtigen
			Herzogs sowohl verlieh als auch schenkte.}
			(%
				A1:~73ra,19--21; vgl.
				\KC:~V.~17108--17110;
				\cite[389]{schroeder1895}%
			)
	\end{xlist}
\end{exe}

\norm{Bėide} dient hier eher einem syntaktischen Zweck, insofern zum
Beispiel \norm{bėide man} \wdef{beide Männer} als NP\is{Nominalphrase} allein
stehen kann, \norm{*bėide grōȥ} \wdef{*beide groß} oder \norm{*bėide lēch}
\wdef{*beide lieh} aber nicht. Nichtsdestoweniger
koindizieren\is{Koindizierung} Adjektive\is{Adjektiv!attributiv} und Verben
diejenigen Substantive\is{Substantiv} oder Pronomina, auf die sie sich
beziehen, sodass \norm{bėide} zumindest theoretisch indirekt über die Konjunkte
eine Affinität zu den Personenmerkmalen\is{Personenmerkmal} der Controller der
Konjunkte zeigen könnte, wie anhand von \REF{ex:recquery} in
\figref{fig:recquery} schematisch dargestellt.

\begin{exe}
\ex\label{ex:recquery}
	\gll ſi macheten ir antwerc. \\
		sie machten ihr Belagerungsgerät[\textsc{acc.sg.\NeutI}] \\
\sn \gll bediv groz unde ſtarc. \\
		beide groß[\textsc{acc.sg.\NeutI}] und stark[\textsc{acc.sg.\NeutI}] \\
	\trans \wdef{Sie machten ihr Belagerungsgerät sowohl groß als auch stark.}
			(%
				A1:~32vb,34--35; vgl.
				\KC:~V.~7586--7587;
				\cite[220]{schroeder1895}%
			)
\end{exe}

\begin{figure}
\begin{tikzpicture}[baseline=(2_lb1.base)]
    \node (1)      {\lit{antwerc}};
    \node (1_box)  [draw,rectangle,fit=(1)] {};
    \node (1_lb)   [above=.5ex of 1_box, mynodefont]{controller};

    \node (2)      [right=4em of 1_box, align=center]
                   {\lit{groz}\\ \lit{ſtarc}};
    \draw (2) node (2_box1) [
                        draw,
                        minimum height=5em,
                        minimum width=5em,
                        xshift=-1ex,
                        yshift=+1ex,
                        rectangle
                    ] {};
    \draw (2) node (2_box2) [
                        draw,
                        dashed,
                        minimum height=5em,
                        minimum width=5em,
                        xshift=+1ex,
                        yshift=-1ex,
                        rectangle
                    ] {};
    \node (2_lb1)  [above=.5ex of 2_box1, mynodefont] {target};
    \node (2_lb2)  [below=.5ex of 2_box2, mynodefont] {controller};

    \node (3)      [right=4.5em of 2] {\lit{$^?$bediv}};
    \node (3_box)  [draw,dashed,rectangle,fit=(3)] {};
    \node (3_lb)   [above=.5ex of 3_box, mynodefont] {target};

    \draw [-latex] (1_box)  to [yshift=-2ex] (2_box1);
    \draw [latex-, dashed] (3_box) to [yshift=1.5ex] (2_box2);
\end{tikzpicture}
\caption{Hypothetischer indirekter Bezug eines Targets auf einen einzelnen
	Erstcontroller über ein prädikatives Adjektiv}
\label{fig:recquery}
\end{figure}

Auch wenn \lit{groz} \wdef{groß} und \lit{ſtarc} \wdef{kräftig, stark} in
\REF{ex:recquery} keine Flexionsendungen aufweisen,\is{Kongruenz!prädikative}
da es sich um koprädikative Adjektive\is{Adjektiv!prädikativ} handelt
\autocite[188]{ksw2}, verweisen beide Adjektive dennoch auf ihr Bezugswort
\lit{antwerc} \wdef{Belagerungsgerät}. Wenn die Konjunktion \lit{bediv}
\wdef{beide} Kongruenz aufweisen soll, müsste sie ihrerseits mit den beiden
Adjektiven in den von ihnen referenzierten
Personenmerkmalen\is{Personenmerkmal} übereinstimmen.

\begin{table}
\centering
\captionsetup{width=.74\textwidth}
\caption{Form nach Personenmerkmalen adjektivischer und verbaler Konjunkte}
\begin{tabularx}{.8\textwidth}{
	X
	r Y
	Y
}
\lsptoprule

\normalfont Controller
	& \norm{bėid(e)}
	& \norm{bėidiu}
	& Summe
	\\

\midrule

% Ctrl 1   | e  | iu | Σ
\textsc{sg}.\MascM &    &  2 &  2 \\

\midrule

\textsc{pl}.\MascM &  5 &  3 &  8 \\
\textsc{pl}.\FemF  &    &  1 &  1 \\

\midrule

\textsc{sg}.\NeutI &    &  1 &  1 \\

\midrule

Summe      &  5 &  7 & 12 \\
\lspbottomrule
\end{tabularx}
\label{tab:konjadjvv}
\end{table}

Bei dieser Untersuchung sind insgesamt zwölf Belege aus den drei Handschriften
A1, B1 und VB verfügbar, siehe \tabref{tab:beidefuncpos}. Die Belege sind ihren
Personenmerkmalen\is{Personenmerkmal} nach in \tabref{tab:konjadjvv}
aufgeführt. Es wurden nur Belege im Nominativ und Akkusativ berücksichtigt. In
der Tabelle stehen fünf Belegen für \norm{bėid(e)} sieben für \norm{bėidiu}
gegenüber. Ob sich unbelebte\is{Inanimata} Targets tendenziell anders verhalten
als belebte\is{Animata}, kann nicht ausgemacht werden\is{Desiderat}, da für
unbelebte \norm{bėide}-Targets nur ein einziger Beleg vorhanden ist. Dass mit
belebtem Bezug \norm{bėidiu} in sechs Fällen vorkommt, verwundert, wenn
Kongruenz angenommen wird. Beispiel \REF{ex:konjadjvvbeidiu} führt exemplarisch
drei Stellen an.

\begin{exe}
\ex \label{ex:konjadjvvbeidiu}
	\begin{xlist}
	% \ex \label{ex:konjadjvvbeidiu_1} % kumulierte Wortentfernung: 3/7
	% 	\gll daz im der chaiſer willicliche. \\
	% 		dass ihm der Kaiser[\textsc{nom.sg.\MascM}] willig \\
	% \sn \gll baidiv leh unt gab. \\
	% 		beide verleihen[\textsc{3sg\subM.ind.pst}] und
	% 		geben[\textsc{3sg\subM.ind.pst}] \\
	% \sn \gll des richen herzogen craft. \\
	% 		des mächtigen Herzogs Gewalt \\
	% 	\trans \wdef{dass ihm der Kaiser willig die Gewalt des mächtigen
	% 		Herzogs sowohl verlieh als auch schenkte.}
	% 		(%
	% 			A1:~73ra,19--21; vgl.
	% 			\KC:~V.~17108--17110;
	% 			\cite[389]{schroeder1895}%
	% 		)

	\ex \label{ex:konjadjvvbeidiu_2} % kumulierte Wortentfernung: 4/8
		\gll Daz der kvnich zelande fvͦre \\
			dass der König[\textsc{nom.sg.\MascM}] zu=Lande
			fahren[\textsc{3sg\subM.sbjv.pst}] \\
	\sn \gll Beidiv lobte vnd ſwuͤre \\
			beide geloben-\textsc{3sg\subM.sbjv.pst} und
			schwören[\textsc{3sg\subM.sbjv.pst}] \\
	\sn \gll Daz er daz riche rehte rihte \\
			dass er das Reich recht regiere \\
		\trans \wdef{dass der König heimfahren \textins{und} sowohl geloben als
			auch schwören \textins{solle}, dass er das Reich recht regiere.}
			(%
				VB:~102ra,43; zu
				\KC:~V.~15366--15368;
				\cite[359]{schroeder1895}%
			)

	% \protectedex{%
	% \ex \label{ex:konjadjvvbeidiu_3} % kumulierte Wortentfernung: 3/7
	% % "/" eingefügt und manuelle Zeilenwechsel entfernt
	% 	\gll ouch gelagen dem chvnig {/} \\
	% 		auch lagen dem König \\
	% 	\gll Mænig held frûmig {/} \\
	% 		viel Held[\textsc{nom.pl.\MascM}] tapfer \\
	% 	\gll Beidíu wund vnd tot \\
	% 		beide verwundet[\textsc{nom.pl.\MascM}] und tot[\textsc{nom.pl.\MascM}] \\
	% 	\trans \wdef{Außerdem lagen dem König viele tapfere Helfer
	% 		sowohl verwundet als auch tot \textins{darnieder}.}
	% 		(%
	% 			B1:~42vb,42--44; vgl.
	% 			\KC:~V.~15880--15882;
	% 			\cite[363]{schroeder1895}%
	% 		)%
	% }

	\ex \label{ex:konjadjvvbeidiu_4} % kumulierte Wortentfernung: 3/7
	% "/" eingefügt und manuelle Zeilenwechsel entfernt
		\gll Ovch gelagen dem kvͤnige {/} \\
			auch lagen dem König \\
		\gll Manige hende frvmige. {/} \\
			viele Hand[\textsc{nom.pl.f\subM}] tapfere \\
		\gll Beidiv wnt vnd tot \\
			beide verwundet[\textsc{nom.pl.f\subM}] und
			tot[\textsc{nom.pl.f\subM}] \\
		\trans \wdef{Außerdem lagen dem König viele tapfere Helfer
			sowohl verwundet als auch tot \textins{darnieder}.}
			(%
				VB:~150ra,20--22; vgl.
				\KC:~V.~15880--15882;
				\cite[363]{schroeder1895}%
			)

	% \ex \label{ex:konjadjvvbeidiu_5} % kumulierte Wortentfernung: 2/6
	% % "/" eingefügt und manuelle Zeilenwechsel entfernt
	% 	\gll Ja gelag er an zweível {/} \\
	% 		ja lag ihr ohne Zweifel \\
	% 	\gll Mer dann ſûbentauſent man {/} \\
	% 		mehr denn siebentausend Mann[\textsc{nom.pl.\MascM}] \\
	% 	\gll Beidív wunt vnd erſlagen ſan \\
	% 		beide verwundet[\textsc{nom.pl.\MascM}] und
	% 		erschlagen[\textsc{nom.pl.\MascM}] sodann \\
	% 	\trans \wdef{Ja, es lagen da ohne Zweifel mehr als siebentausend Mann
	% 	von ihnen sowohl verwundet als auch erschlagen \textins{darnieder}.}
	% 		(%
	% 			B1:~45va,46--48; vgl.
	% 			\KC:~V.~16891--16893;
	% 			\cite[368]{schroeder1895}%
	% 		)

	\ex \label{ex:konjadjvvbeidiu_6} % kumulierte Wortentfernung: 2/6
		\gll Vnd maníge ſtolze froͮwen \\
			und viele herrliche Edelfrau[\textsc{nom.pl.\FemF}] \\
	\sn \gll Beidív hoͤbſh vnd gemeit \\
			beide höfisch[\textsc{nom.pl.\FemF}] und lieblich[\textsc{nom.pl.\FemF}] \\
		% Wohl keine Parallelstelle bei Schröder (1895) bzw. in KC A1
		\trans \wdef{und viele herrliche Edelfrauen, \textins{die waren} sowohl
			höfisch als auch lieblich}
			(%
				VB:~21va,21--22; zu
				\KC:~V.~4351--4352;
				\cite[161]{schroeder1895}%
			)
	\end{xlist}
\end{exe}

% Mit \lit{chaiſer} \wdef{Kaiser} \REF{ex:konjadjvvbeidiu_1} und \lit{kvnich}
% \wdef{König} \REF{ex:konjadjvvbeidiu_2} liegen eindeutig männlich
% denotierte Maskulina vor. Da es sich bei den Kongruenztargets in beiden Fällen
% um Verben handelt, steht die Konjunktion im gleichen (Teil-)Satz wie die
% Verben. In (\ref{ex:konjadjvvbeidiu_3}--e) stehen
% koprädikative Adjektive, die sich jeweils auf das Objekt des Satzes beziehen.
% Auch in diesen Fällen ist der Bezug eindeutig maskulin-männlich. Diese
% Adjektive stehen in einem anderen Satzglied als ihr Erstcontroller. Zuletzt
% sind die Adjektive in \REF{ex:konjadjvvbeidiu_6} als Apposition zu ihrem
% Controller aufzufassen. Syntaktisch befinden sie sich im gleichen Satzglied.

Mit \lit{kvnich} \wdef{König} in \REF{ex:konjadjvvbeidiu_2} liegt ein eindeutig
männlich denotiertes\is{Denotation} Maskulinum vor. Da es sich bei dessen
Kongruenztargets um Verben handelt, steht die Konjunktion im gleichen
(Teil-)Satz wie die Verben. In \REF{ex:konjadjvvbeidiu_4} steht ein
koprädikatives Adjektiv\is{Adjektiv!prädikativ}, das sich auf das Objekt des
Satzes bezieht. Auch in diesem Fall ist der Bezug dem Kontext nach
maskulin-männlich, insofern es sich bei den \lit{hende\textins{n}}
\wdef{Händen} um Soldaten handelt. Das Adjektiv steht in einem anderen
Satzglied als sein \isi{Erstcontroller}. Zuletzt sind die Adjektive in
\REF{ex:konjadjvvbeidiu_6} als Apposition\is{Apposition} zu ihrem Controller
aufzufassen. Syntaktisch befinden sie sich im gleichen Satzglied.

Die fünf Belege für \norm{bėid(e)} in \tabref{tab:konjadjvv} beziehen sich auf
koprädikative Adjektive\is{Adjektiv!prädikativ} und verteilen sich auf drei
Parallelstellen\is{Paralleltext}%
% ; sie sind
% in \REF{ex:konjadjvvbeide} aufgelistet. Die Stelle in
% \REF{ex:konjadjvvbeide_1} ist darüber hinaus eine Parallele zu
% \REF{ex:konjadjvvbeidiu_3} und \REF{ex:konjadjvvbeidiu_4}; ebenso gehören
% \REF{ex:konjadjvvbeide_2} und \REF{ex:konjadjvvbeide_3} zu der Stelle in
% \REF{ex:konjadjvvbeidiu_5}.
% Die Eigenschaften dieser Stellen wurden bereits diskutiert
. Einer der Belege wird in \REF{ex:konjadjvvbeide} zitiert.

\begin{exe}
\ex \label{ex:konjadjvvbeide}
    % \begin{xlist}
	% \ex \label{ex:konjadjvvbeide_1} % kumulierte Wortentfernung: 3/5
	% % "/" eingefügt und manuelle Zeilenwechsel entfernt
	% 	\gll oͮch gelagen dem chunige {/} \\
	% 		auch lagen dem König \\
	% 	\gll ſine tuerlichen helde. {/} \\
	% 		seine herrlichen Helden[\textsc{nom.pl.\MascM}] \\
	% 	\gll baide woͮnt unt tot \\
	% 		beide wund[\textsc{nom.pl.\MascM}] und tot[\textsc{nom.pl.\MascM}] \\
	% 	\trans \wdef{Außerdem lagen dem König seine herrlichen Helden
	% 			sowohl verwundet als auch tot \textins{darnieder}.}
	% 		(%
	% 			A1:~67vb,35--36; vgl.
	% 			\KC:~V.~15880--15882;
	% 			\cite[368]{schroeder1895}%
	% 		)
	
	% \ex \label{ex:konjadjvvbeide_2} % kumulierte Wortentfernung: 2/6
		% "/" eingefügt und manuelle Zeilenwechsel entfernt
		\gll ia gelac ir anzwiuel. {/} \\
			ja lag ihr ohne=Zweifel \\
		\gll mer denne {ſiben tuſent} man. {/} \\
			mehr denn siebentausend Mann[\textsc{nom.pl.\MascM}] \\
		\gll baide woͮnt. unt erſlagen. \\
			beide verwundet[\textsc{nom.pl.\MascM}] und erschlagen[\textsc{nom.pl.\MascM}] \\
		\trans \wdef{Ja, es lagen da zweifellos mehr als siebentausend Mann
			von ihnen sowohl verwundet als auch erschlagen \textins{darnieder}.}
			(%
				A1:~72ra,32--34; vgl.
				\KC:~V.~16891--16893;
				\cite[386]{schroeder1895}%
			)
	
	% \ex \label{ex:konjadjvvbeide_3} % kumulierte Wortentfernung: 2/6
	% % "/" eingefügt und manuelle Zeilenwechsel entfernt
	% 	\gll Ja belac da ane zwiuel {/} \\
	% 		ja lag da ohne Zweifel \\
	% 	\gll Mere danne {ſiben tvſent} man {/} \\
	% 		mehr denn siebentausend Mann[\textsc{nom.pl.\MascM}] \\
	% 	\gll Beid wnt vnd erſlagen ſan. \\
	% 		beide verwundet[\textsc{nom.pl.\MascM}] und erschlagen[\textsc{nom.pl.\MascM}]
	% 		sodann \\
	% 	\trans \wdef{Ja, es lagen da zweifellos mehr als siebentausend Mann
	% 		von ihnen sowohl verwundet als auch erschlagen \textins{darnieder}.}
	% 		(%
	% 			VB:~110va,9--11; vgl.
	% 			\KC:~V.~16891--16893;
	% 			\cite[386]{schroeder1895}%
	% 		)

	% \ex \label{ex:konjadjvvbeide_4} % kumulierte Wortentfernung: 4/10
	% 	\gll Daz dv dich an die goͤt laſt \\
	% 		dass du dich ohne die Gott[\textsc{nom.pl.\MascA}] lässt \\
	% \sn \gll Die beid ſtvmmen ſint vnd blint \\
	% 		die beide stumm[\textsc{nom.pl.\MascA}] sind und
	% 			blind[\textsc{nom.pl.\MascA}] \\
	% 	\trans \wdef{dass du der Götter entsagst, die sowohl stumm sind als
	% 		auch blind}
	% 		(%
	% 			B1:~30rb,60--30rc,1; zu
	% 			\KC:~V.~10900\,ff.;
	% 			\cite[281]{schroeder1895}%
	% 		)
	
	% \ex \label{ex:konjadjvvbeide_5} % kumulierte Wortentfernung: 4/8
	% 	\gll Daz dv dich ane die got laſt \\
	% 		dass du dich ohne die Gott[\textsc{nom.pl.\MascA}] lässt \\
	% \sn \gll Die beide blint vnd ſtvmmen ſínt \\
	% 		die beide blind[\textsc{nom.pl.\MascA}] und
	% 			stumm[\textsc{nom.pl.\MascA}] sind \\
	% 	\trans \wdef{dass du der Götter entsagst, die sowohl stumm sind als
	% 		auch blind}
	% 		(%
	% 			VB:~51va,29--30; zu
	% 			\KC:~V.~10900\,ff.;
	% 			\cite[281]{schroeder1895}%
	% 		)
	% \end{xlist}
\end{exe}

Da \norm{bėide} und \norm{bėidiu} in \REF{ex:konjadjvvbeidiu} und
(\ref{ex:konjadjvvbeide}) nebeneinander mit Personenbezeichnungen stehen, deren
\isi{Genus} und \isi{Sexus} sowohl formal als auch semantisch eindeutig sind,
und die Form der Konjunktion zwischen Parallelstellen\is{Paralleltext} im
gleichen Kontext variiert, ist nicht davon auszugehen, dass Genus oder
Sexus die Ursache für die Variation sind. Darüber hinaus ist die Größe des
kumulierten Wortformenabstands\is{Distanz!lineare} in allen Fällen ähnlich.
Auch hinsichtlich der syntaktischen Distanz\is{Distanz!syntaktische} zeigt sich
keine Affinität einer der beiden Formen für einen bestimmten Kontext. Insgesamt
lässt sich daher auch im hier behandelten syntaktischen Kontext,
\norm{bėide} \textsc{adjektiv/verb} \norm{unde}
\textsc{adjektiv/verb},\is{Adjektiv} kein formales oder semantisches Kriterium
für die Variation der Konjunktion zwischen \norm{bėide} und \norm{bėidiu}
festmachen.

\is{Controller|)}
\is{Target|)}

\subsection{Rein syntaktischer Kontext}
\label{subsec:kcbeidquantsyncont}

In Kontexten wie \REF{ex:syntintvar1} mit Adverbien\is{Adverb} und
\REF{ex:syntintvar3} mit Adjektiven,\is{Adjektiv} in denen \norm{bėide \dots\
unde} mit Wortarten oder Phrasen auftritt, die keine
Personenmerkmale\is{Personenmerkmal} beinhalten, ist Kongruenz von \norm{bėide}
mit seinen Konjunkten an sich ausgeschlossen. Die Konjunkte definieren in
diesen Fällen weder ihre eigenen Personen\-merkmale, noch spiegeln sie als
Kongruenztargets die Personenmerkmale eines Controllers\is{Controller} wider.
Die Konstruktion ist hier also rein zur Betonung der Zweiheit der Optionen
aufzufassen. Im Belegmaterial der \KC{} umfasst dies \norm{bėide \dots\ unde}
mit Adverbien und Präpositionalphrasen\is{Präpositionalphrase} (PPs), wie in
(\ref{ex:syntintvar1}--\ref{ex:syntintvar4}) dargestellt. Die zitierten Stellen
umfassen alle exzerptierten Belege für diesen Kontext.

\begin{exe}
\ex \label{ex:syntintvar1}
	\begin{xlist}
	\ex \gll Beideu vruͦ vnd ſpæt \\
		beide früh und spät \\
		\trans \wdef{sowohl früh als auch spät}
		(%
			B1:~18rb,5; vgl.~abweichend
			\KC:~V.~6279;
			\cite[196]{schroeder1895}%
		)

	\ex \gll Beideu ſpæt vnde vruͦ \\
		beide spät und früh \\
		\trans \wdef{sowohl spät als auch früh}
		(%
			B1:~19va,15; zu
			\KC:~V.~6834--6839;
			\cite[206]{schroeder1895}%
		)

	\ex \label{ex:syntintvar1_3}
	\gll Beide ſpat vnd frv \\
		beide spät und früh \\
		\trans \wdef{sowohl spät als auch früh}
			(%
				VB:~33ra,36; zu
				\KC:~V.~6834--6839;
				\cite[206]{schroeder1895}%
			)
\end{xlist}

\ex \label{ex:syntintvar3}
	\begin{xlist}
	\ex \gll Beideu ſtill vnd ûber laut \\
		beide still und über laut \\
		\trans \wdef{sowohl still als auch mit Getöse}
			(%
				B1:~22va,1; zu
				\KC:~8014--8020;
				\cite[228]{schroeder1895}%
			)

	\ex \label{ex:syntintvar3_2}
		\gll Beídív ſtille vnd vber lovt \\
		beide still und über laut \\
		\trans \wdef{sowohl still als auch mit Getöse}
			(%
				VB:~38vb,24; zu
				\KC:~V.~8014--8020;
				\cite[228]{schroeder1895}%
			)
\end{xlist}
\end{exe}

Die Stellen in \REF{ex:syntintvar1} zeigen Variation zwischen B1
und VB, wobei keine dieser Handschriften eine besondere Affinität zu
\norm{bėide} hat~-- der Großteil der Belege für \norm{bėide} als
Konjunktion in diesem syntaktischen Kontext lautet \norm{bėidiu}. In
\REF{ex:syntintvar3} zeigen zwar beide Belege eine Form des Typs
\norm{bėidiu}, doch liegen keine offensichtlichen Gründe dafür vor, dass in
\REF{ex:syntintvar1_3} \lit{ſpat} \wdef{spät} und \lit{frv} \wdef{früh}
mit \lit{Beide} stehen, dagegen in der\-selben Handschrift nur ein paar Blätter
weiter mit \lit{ſtille} \wdef{still} und \lit{lovt} \wdef{laut} die Form
\norm{Beídív} auftritt. Anzumerken ist auch hier, dass für VB eine
Tendenz zur Abschwächung von \norm{-iu} zu \norm{-e} festgestellt werden
konnte.

Variation zwischen verschiedenen Textstellen mit identischem Wortlaut innerhalb
der\-selben Handschrift liegt in \REF{ex:syntintvar2} und \REF{ex:syntintvar5}
vor; die Parallelstellen\is{Paralleltext} sind jeweils zusammengefasst.
Beispiel \REF{ex:syntintvar4} zeigt eine weitere Stelle mit
Adverbial\is{Adverbial}/PP-Konjunkten\is{Präpositionalphrase}. Während bei der
Stichprobe zur \isi{Adjektivdeklination} in B1 die Verteilung der Suffixe klar
zwischen \norm{-e} und \norm{-iu} getrennt ist, kommen in
\REF{ex:syntintvar2_1} und \REF{ex:syntintvar2_2} beide Formen der Konjunktion,
\norm{bėide} und \norm{bėidiu}, nebeneinander mit denselben Konjunkten vor. In
\REF{ex:syntintvar2_3} und \REF{ex:syntintvar2_4} stehen ebenfalls bei der
ansonsten identischen Formulierung unterschiedliche Formen der Konjunktion. Die
Annahme, dass die jeweilige Form der Konjunktion in Abhängigkeit von den
Konjunkten auftritt, ist auch hier unplausibel.

\begin{exe}
\ex \label{ex:syntintvar2}
\begin{xlist}
	\ex \label{ex:syntintvar2_1}
		\gll Beid zeberg vnd zetal \\
			beide zu=Berg[\textsc{dat.sg.\MascI}] und zu=Tal[\textsc{dat.sg.\NeutI}] \\
		\trans \wdef{sowohl zu Berg als auch zu Tal}
			(%
				B1:~21rc,30; zu
				\KC:~V.~7572--7583;
				\cite[220]{schroeder1895}%
			)

	\ex \label{ex:syntintvar2_3}
		\gll Beidív zeberge vnd zetal \\
			beide zu=Berg[\textsc{dat.sg.\MascI}] und zu=Tal[\textsc{dat.sg.\NeutI}] \\
		\trans \wdef{sowohl zu Berg als auch zu Tal}
			(%
				VB:~36vb,13; zu
				\KC:~V.~7572--7583;
				\cite[220]{schroeder1895}%
			)
\end{xlist}

\ex \label{ex:syntintvar5}
\begin{xlist}
	\ex \label{ex:syntintvar2_2}
		\gll Beideu zeberg vnd zetal \\
			beide zu=Berg[\textsc{dat.sg.\MascI}] und zu=Tal[\textsc{dat.sg.\NeutI}] \\
		\trans \wdef{sowohl zu Berg als auch zu Tal}
			(%
				B1:~32rb,35; zu
				\KC:~V.~11610\,ff.;
				\cite[293]{schroeder1895}%
			)

	\ex \label{ex:syntintvar2_4}
		\gll Beide zeberge vnd zetal \\
			beide zu=Berg[\textsc{dat.sg.\MascI}] und zu=Tal[\textsc{dat.sg.\NeutI}] \\
		\trans \wdef{sowohl zu Berg als auch zu Tal}
			(%
				VB:~83vb,3; zu
				\KC:~V.~11610\,ff.;
				\cite[293]{schroeder1895}%
			)
\end{xlist}

\ex \label{ex:syntintvar4}
	\gll Beide zehemde vnd zebruch \\
		beide zu=Hemd[\textsc{dat.sg.\NeutI}] und
			zu=Hose[\textsc{dat.sg.\FemI}] \\
	\trans \wdef{sowohl für beim Hemd als auch bei der Hose}
		(%
			VB:~99rb,21; vgl.~abweichend
			\KC:~V.~14799;
			\cite[349]{schroeder1895}%
		)
\end{exe}

Wie einleitend zu diesem Abschnitt erwähnt, spielen im vorliegenden
syntaktischen Kontext Personenmerkmale keine Rolle. Da in beiden hier
untersuchten Handschriften, B1 und VB, beide Formen im gleichen syntaktischen
Kontext nebeneinander vorkommen, ergibt sich die Frage, ob stattdessen die Art
des Konjunkts einen Einfluss auf die Form der Konjunktion haben könnte. Weil
aber auch unter diesem Aspekt Variation herrscht, selbst innerhalb derselben
Handschrift, ist diese Annahme angesichts der geringen Belegzahl vorsichtig zu
verneinen. Es zeigt sich eher eine Tendenz, dass \norm{bėidiu} die bevorzugte
Form darstellt, falls noch keine sporadische Abschwächung von \norm{-iu} zu
\norm{-e} wie in VB vorliegt.

\is{Konjunktion|)}

\subsection{Zusammenfassung}

In Hinblick auf \norm{bėide \dots\ unde} \wdef{sowohl \dots\ als auch} als
korrelative \isi{Konjunktion} zeigten sich keine eindeutigen\is{Ambiguität}
Hinweise auf \isi{Kovarianz} mit den Personenmerkmalen\is{Personenmerkmal} der
Konjunkte, so vorhanden. Auch wenn man annimmt, dass
pronominal-kataphorisches\is{Katapher} \norm{bėide} in den Belegen zu
koordinierten Nominalen\is{Koordination} enthalten ist, fallen diese nicht
durch eine Konzentration auf \norm{bėide} bei gleichem und \norm{bėidiu} bei
verschiedenem belebten\is{Animata} Geschlecht beziehungsweise bei
unbelebtem\is{Inanimata} Bezug auf.

Sowohl in A1 als auch in VB können prinzipiell beide Formen sowohl mit
belebten\is{Animata} als auch unbelebten\is{Inanimata} Konjunkten mit
über\-einstim\-mendem oder verschiedenem \isi{Genus} oder \isi{Sexus}
auftreten. Bei VB können die Belege zwar so interpretiert werden, dass bei
belebten Konjunkten eine leichte Tendenz zu \norm{bėide} und bei unbelebten zu
\norm{bėidiu} vorliegt. Dies deckt sich\is{Validierung} jedoch nicht mit der
wesentlich älteren Handschrift A1, in der keine derartige Tendenz auszumachen
ist~-- hier scheint generell \norm{bėidiu} bevorzugt zu werden.

Auch mit Konjunkten, die ihrerseits Kongruenztargets\is{Target} darstellen,
macht sich kein Zusammen\-hang zwischen Personen\-merkmalen\is{Personenmerkmal}
und angenommener Flexion der \isi{Konjunktion} bemerkbar. Für diese
Teilauswertung wurden Belege aus den Handschriften A1, B1 und VB verwendet. In
vergleichbaren Kontexten herrscht dort ein Nebeneinander von \norm{bėide} und
\norm{bėidiu}. Neben den Personenmerkmalen der Konjunkte wurde eine mögliche
Abhängigkeit der Variation von der \isi{Distanz} zwischen den kombinierten
Controllern\is{Controller} und ihrem Target erwogen. Doch auch hier zeigte sich
Entfernung nicht als bestimmender Faktor, sodass auch für diesen syntaktischen
Kontext kein formales oder semantisches Kriterium gefunden wurde, das für die
Variation verantwortlich gemacht werden kann.

Zu guter Letzt wurden Fälle aus B1 und VB untersucht, in denen einzelne
Wortformen oder komplexe syntaktische Phrasen koordiniert\is{Koordination}
werden, die keine Personenmerkmale\is{Personenmerkmal} aufweisen, das heißt,
Adverbien\is{Adverb} und Adverbiale\is{Adverbial}. Mit Kongruenz dürfte in
diesem Kontext also nicht zu rechnen sein. Auch in diesem Fall herrscht
Variation zwischen \norm{bėide} und \norm{bėidiu}, ohne dass dafür ein
entscheidender, durch die Grammatik bedingter Faktor vorzuliegen scheint.
Gleiche Formulierungen variieren teilweise innerhalb derselben Handschrift im
Abstand weniger Seiten. Variation in VB kann auch hier vornehmlich mit dem
Abbau von \norm{-iu} als Flexion zugunsten von \norm{-e} erklärt werden.
