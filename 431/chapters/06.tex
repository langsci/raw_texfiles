\chapter{\norm{-e} und \norm{-iu} im Paradigma der starken Adjektivflexion}
\label{ch:adjflex}
\is{Dialektgeografie|(}
\is{Adjektivdeklination|(}
\is{Paradigma|(}

In diesem Kapitel wird aufgrund einer Stichprobe aus dem Quellenmaterial
das Paradigma der starken Adjektivflexion mit Fokus auf den Nom./Akk.\ Pl.\ für
die untersuchten Texte rekonstruiert, um sicherzustellen, dass die gesammelten
Belege aus Urkunden des \tit{Corpus der altdeutschen Originalurkunden} (\CAO)
und der \tit{Kaiserchronik}-Handschriften (\KC) auch bezüglich ihrer
dialektgeografischen\is{Dialektgeografie} Dimension korrekt interpretiert
werden. Indem vorab sondiert wird, ob an bestimmten wichtigen
oberdeutschen\il{Oberdeutsch} Ausstellungs\-orten beziehungsweise in den
jeweiligen Handschriften aufgrund schreibsprachlicher Gegebenheiten überhaupt
eine Opposition zwischen \norm{e}- und \norm{iu}-Formen vorliegt
\autocite[vgl.][182]{ksw2}, können Handschriften sowie Urkundenbelege aus
Regionen, in denen dies nicht der Fall ist, von der weitergehenden Untersuchung
ausgeschlossen oder mit entsprechender Vorsicht behandelt werden. Das Paradigma
der starken Adjektivflexion, wie es allgemein für das Oberdeutsche zu
mittelhochdeutscher Zeit angesetzt wird, ist in \tabref{tab:adjparadgmstr_pr}
und \ref{tab:adjparadgmstr_nm} mit seinen Synkretismen aufgeführt.

\afterpage{%
\begin{table}
% one-off increase caption width for page no. not to slip to 2nd line all alone
\captionsetup{width=.85\textwidth}
\caption{Pronominal-starke Adjektivflexion in oberdeutschen\il{Oberdeutsch}
	Schreib\-dialekten des Mittelhochdeutschen\il{Mittelhochdeutsch}
	\autocite[182, Abbildung~A~3]{ksw2}}
\begin{tabular}{| l c c c c c c}
\hline

%
	& \mc{3}{|c}{\textsc{sg}}
	%
	& \mc{3}{|c|}{\textsc{pl}}
	\\

\cline{2-7}

%
	& \mc{1}{|>{\centering}m{2em}}{\textsc{n}}
	& \mc{1}{|>{\centering}m{2em}}{\textsc{m}}
	& \mc{1}{|>{\centering}m{2em}}{\textsc{f}}
	%
	& \mc{1}{|>{\centering}m{2em}}{\textsc{n}}
	& \mc{1}{|>{\centering}m{2em}}{\textsc{m}}
	& \mc{1}{|>{\centering}m{2em}|}{\textsc{f}}
	\\

\hline
\hline

\textsc{nom}
	& \mc{1}{|c}{\mr{2}{*}{-eȥ}}
	& \mc{1}{|c}{-er}
	& \mc{2}{|c}{-iu}
	%
	& \mc{2}{|c|}{\mr{2}{*}{-e}}
	\\

\cline{1-1}
\cline{3-4}

\textsc{acc}
	& \mc{1}{|c}{}
	& \mc{1}{|c}{-en}
	& \mc{1}{|c}{-e}
	%
	& \mc{1}{|c}{}
	& \mc{2}{|c|}{}
	\\

\hline
\hline

\textsc{dat}
	& \mc{2}{|c}{-em(e)}
	& \mc{1}{|c}{}
	%
	& \mc{3}{|c|}{-en}
	\\

\cline{1-3}
\cline{5-7}

\textsc{gen}
	& \mc{2}{|c}{-es}
	& \mc{4}{|c|}{-er(e)}
	\\

\hline
\end{tabular}
\label{tab:adjparadgmstr_pr}
\end{table}

\begin{table}
% one-off increase caption width for page no. not to slip to 2nd line all alone
\captionsetup{width=.85\textwidth}
\caption{Nominal-starke Adjektivflexion in oberdeutschen\il{Oberdeutsch}
	Schreib\-dialekten des Mittelhochdeutschen\il{Mittelhochdeutsch}
	\autocite[182, Abbildung~A~3]{ksw2}}
\begin{tabular}{| l c c c}
\hline

%
	& \mc{3}{|c|}{\textsc{sg}}
	\\

\cline{2-4}

%
	& \mc{1}{|>{\centering}m{2em}}{\textsc{n}}
	& \mc{1}{|>{\centering}m{2em}}{\textsc{m}}
	& \mc{1}{|>{\centering}m{2em}|}{\textsc{f}}
	\\

\hline
\hline

\textsc{nom}
	& \mc{3}{|c|}{-Ø}
	\\

\cline{1-1}
\cline{3-4}

\textsc{acc}
	& \mc{1}{|c|}{}
	& \mc{2}{|c}{}
	\\

\cline{1-2}

\end{tabular}
\label{tab:adjparadgmstr_nm}
\end{table}%
}

%%%%%%%%%%%%%%%%%%%%%%%%%%%%%%%%%%%%%%%%%%%%%%%%%%%%%%%%%%%%%%%%%%%%%%%%%%%%%%%

\section{Adjektivdeklination im \tit{Corpus der altdeutschen
Originalurkunden}}
\label{sec:adjdeclcao}

\phantomsection
\label{phsec:formelhaftigkeit}
\is{Formelhaftigkeit|(}

Eine automatische Lemmatisierung und Formenbestimmung für alle Token des
\CAO{} auf Basis des \REM{} unter Verwendung des Taggers\is{Tagging} von
\citet{schmid2019} ist verfügbar
\autocites[vgl.][207]{beckerschallert2021}[155--158]{beckerschallert2022b}. Bei
der Erstellung der Stichprobe für die Rekonstruktion der starken
Adjektivflexion hat sich allerdings gezeigt, dass der Tagger Schwierigkeiten
hatte, die Flexionsendungen korrekt zuzuordnen, insbesondere bei der schwachen
Deklination mit ihren einzigen Flexiven \norm{-e} und \norm{-en}.

\posscite{schmid2019} Tagger\is{Tagging} hat insgesamt 60.022 Wortformen im
\CAO{} als attributive Adjektive\is{Adjektiv!attributiv} kategorisiert, darunter
auch adjektivisch verwendete Partizipien. Bei der automatischen
Annotation\is{Annotation} wurde die editorische Abkürzung \fw{S.} für
\emph{Siegel} 3.321-mal fälschlich als Adjektivform gekennzeichnet. Von den
verbleibenden 56.701 Wortformen wurden 31.805 mit 95-prozentiger Sicherheit als
Adjektiv gewertet. Der Anteil an attributiven Adjektiven und Partizipien im
ganzen \isi{Korpus} beträgt damit ungefähr 1,5 bis 3\,\%; sie finden in den
Urkunden\is{Urkunde} also eher selten Verwendung. Auffällig häufig kommen diese
Wortformen als Attribute von Präpositionalobjekten in Rechtsformeln vor
\autocites[\ref{ex:adjprepobj}; vgl.][30]{becker2016}.

\begin{exe}
\ex \label{ex:adjprepobj}
	\begin{xlist}
	\ex \norm{mit gemėinem rāte}
		\wdef{mit einvernehmlichem Rat}
	\ex \norm{mit gesamenter hant}
		\wdef{mit vereinter Hand}
	\ex \norm{mit verdāhtem muete}
		\wdef{mit besonnener Absicht}
	\ex \norm{von ēhafter nōt}
		\wdef{durch einen rechtsgültigen Hinderungsgrund}
	\ex \norm{ƶe ėinem offenen urkunde}
		\wdef{zum öffentlichen Zeugnis}
	\ex \norm{ƶe ręhtem schirme}
		\wdef{zum rechtmäßigen Beistand}
	\end{xlist}
\end{exe}

Eine zusätzliche Herausforderung besteht darin, dass die Urkunden\is{Urkunde}
für sich genommen in der Regel zu kurz sind, um alle Formen des Paradigmas zu
enthalten. Für die Hälfte (50,4\,\%) der identifizierbaren Ausstellungs\-orte
liegt nur eine einzige Urkunde vor, die laut Ortsverzeichnis dem jeweiligen
Ausstellungs\-ort als einzigem zugeordnet ist. Selbst wenn sämtliche Urkunden
ausgewertet würden, wäre es nicht möglich, für sämtliche Ausstellungs\-orte das
vollständige Deklinationsparadigma zu rekonstruieren. In
\sectref{subsec:cao_adjflex_disc} werden daher auch nur einzelne illustrative
Beispiele diskutiert und keine kompletten Paradigmen wie in den
Tabellen~\ref{tab:adjparadgmstr_pr} und~\ref{tab:adjparadgmstr_nm} angegeben.

\is{Formelhaftigkeit|)}

\subsection{Anlage der Stichprobe}
\label{subsec:cao_sample}
\is{Stichprobe|(}

Um eine Stichprobe zu erstellen, die den geschilderten Einschränkungen
weitestmöglich entgegenkommt, wurden zunächst diejenigen Ausstellungs\-orte
ermittelt, die eine möglichst große Menge an einzig ihnen zugewiesenen
Urkunden\is{Distribution!geografische} aufweisen und dabei die verschiedenen
oberdeutschen Dialekt\-räume\is{Dialektgeografie} abdecken, auch in Hinsicht
auf die auszuwertenden \KC{}-Handschriften (vgl.~\figref{fig:adjstpr_orte}).

\begin{table}
\centering
\caption{Ausstellungsorte der Stichprobe}
\begin{tabular}{r l r}
\lsptoprule
Rang
	& Ausstellungs\-ort
	& Urkunden
	\\

\midrule

 1 & Straßburg  & 219 \\
 2 & Zürich     & 143 \\
 4 & Basel      & 117 \\
 5 & Augsburg   &  95 \\
 6 & Salzburg   &  75 \\
 7 & Regensburg &  53 \\
 9 & Wien       &  48 \\
10 & Konstanz	&  46 \\
13 & Nürnberg   &  26 \\
15 & München    &  22 \\
22 & Ulm        &  16 \\

\lspbottomrule
\end{tabular}
\label{tab:adjstpr_orte}
\end{table}

\begin{figure}
\centering
\includegraphics[
	trim=0 10mm 75mm 0, clip,
	width=\linewidth,
	keepaspectratio
]{./figures/adjektivstichprobe_cao.pdf}
\caption{Belegmenge pro Ausstellungs\-ort in der Stichprobe zur
	Adjektivdeklination
	(Hintergrund nach \cite[Karte~47.4]{wiesinger1983})}
\label{fig:adjstpr_orte}
\end{figure}

Zusätzlich wurden für alle in \tabref{tab:adjstpr_orte} aufgelisteten
Ausstellungs\-orte noch alle weiteren Orte im Umkreis von etwa vierzig
Kilometern (0,25°) hinzugenommen, zumal gerade für Nürnberg, München und Ulm
jeweils nur wenige Urkunden zur Verfügung stehen. Mögliche schreibsprachliche
Unterschiede zwischen Stadt und Umland sowie die Lage Augsburgs im
Übergangsgebiet vom Schwäbischen zum Bairischen wurden dabei außer Acht
gelassen, zumal hier nicht phonologische, sondern morphosyntaktische Variation
angesprochen wird (vgl.~\sectref{phsec:solms} und \cite[132]{solms2014}). Bei
der Durchsicht und Klassifizierung der Belege erwies\is{Validierung} sich
dieser \isi{Vorbehalt} als unproblematisch.

Für die Anfertigung der Stichprobe ist es nur bedingt nützlich, automatisiert
alle jeweiligen Formen für einen bestimmten Ort ausgeben zu lassen, da die
automatische Wortartzuordnung des Taggers\is{Tagging} bei Adjektiven nicht
zuverlässig funktioniert und die Wortformenbestimmung teils Lücken aufweist
oder ebenso ungenau ist. Daher wurde zur Auswahl häufiger Lemmata eine Liste
aller im \WMU{} verzeichneten Lemmata mit der dort angegebenen
Häufigkeit\is{Frequenz} herangezogen.%
%
	\footnote{Die Liste (mit Stand 2009) wurde der AG~Sprachgeschichte der
	Philipps-Universität Marburg 2013 im Rahmen eigener Studien zum
	\CAO{} von \iai{Ursula Schulze} (†) zur Verfügung gestellt.}
	%
	% U.S. an Jürg Fleischer am 13.02.2013 15:11 Uhr
	% J.F. an Oliver Schallert am 14.02.2013 08:54 Uhr
	% O.S. an mich (Beckerc2@students) am 16.02.13 18:59 Uhr
%
Die Auswahl fiel dabei auf die in \tabref{tab:adjsmpwords} aufgelisteten
Lemmata. Um Irrläufer\is{Falschpositiv} zu vermeiden, wurden nur diejenigen
Belege für die einzelnen Lemmata aus der \isi{Datenbank} extrahiert, die der
Tagger\is{Tagging} von \citet{schmid2019} mit mindestens 95-prozentiger
Sicherheit dem jeweiligen Lemma zuordnen konnte.

\begin{table}[h]
\centering
\caption{Lemmata der Stichprobe}
\begin{tabular}{l l r l @{\citereset}}
\lsptoprule

Lemma
	& Übersetzung
	& Häufigkeit
	& Quelle
	\\

\midrule
\norm{ēhaft}
	& \wdef{rechtsgültig}
	& 105
	& \cite[419--420]{wmu1}
	\\
\norm{ēlich}
	& \wdef{rechtmäßig}
	& 350
	& \cite[448--449]{wmu1}
	\\
\norm{ganƶ}
	& \wdef{ganz}
	& 631
	& \cite[549--550]{wmu1}
	\\
\norm{grōȥ}
	& \wdef{groß}
	& 270
	& \cite[761--762]{wmu1}
	\\
\norm{guet}
	& \wdef{gut}
	& 1.643
	& \cite[770--772]{wmu1}
	\\
\norm{klėine}
	& \wdef{klein}
	& 116
	& \cite[1011--1012]{wmu2}
	\\

\norm{niuwe}
	& \wdef{neu}
	& 355
	& \cite[1322--1324]{wmu2}
	\\

\lspbottomrule

\end{tabular}
\label{tab:adjsmpwords}
\end{table}

Die häufigsten Adjektive, die oft in festen Fügungen wie den eingangs zitierten
auftreten, wurden bei der Auswahl der Lemmata vermieden. Possessivbegleiter~--
hauptsächlich \norm{mīn} \wdef{mein} (11.250 Belege;
\cite[1231--1232]{wmu2})~-- wurden aufgrund ihrer Häufigkeit\is{Frequenz}
verwendet, um Lücken aufzufüllen, abgesehen vom Nom.\ Sg. aller Genera sowie
dem Akk.\ Sg.\ N., wo häufig auch eine Form ohne Flexion steht
\autocites(vgl.~\tabref{tab:adjparadgmstr_nm})[216]{paul2007}[507,
510--511]{ksw2}. Allerdings ist anzumerken, dass \norm{mīn} \wdef{mein}
bisweilen auch im Nom./Akk.\ Pl.\ M./F. unflektiert auftritt. \citet[510]{ksw2}
fassen dies \blockquote{im Wesentlichen \textins{als} Resultat der vom Oobd.
ausgehenden Schwa-\isi{Apokope}} auf. Sie weisen aber auch darauf hin, dass
daneben flektiertes \norm{alle} (mit Schwa) vor unflektiertem
\norm{mīn}/\norm{dīn}/\norm{sīn} \wdef{mein}/\wdef{dein}/\wdef{sein} keine
Seltenheit in ihrem Korpusmaterial darstellt. Entsprechende Belege wurden
deshalb hier ausgeklammert.

Für das Alemannische\il{Alemannisch} weisen \citet[271, Abbildung~A~47]{ksw2}
für die zweite Hälfte des 13.~Jahrhunderts 10\,\% flexionslose Formen bei
vorangestellten Adjektiven in Positionen mit regulärem \norm{e}-Flexiv aus,
34\,\% für die erste Hälfte des 14.~Jahrhunderts; Schwäbisch wird nicht
gesondert verzeichnet. Bei den Pluralbelegen in der Stichprobe aus Urkunden des
\CAO{} fällt auf, dass unflektiertes \norm{mīn} \wdef{mein} in allen Fällen vor
\norm{ėrben} \wdef{Erben} steht, wobei das Fehlen einer overten Flexionsendung
hier auch auf Hiatusvermeidung\is{Hiatus} zurückgeführt werden kann.

Noch bestehende Lücken wurden durch manuelle Durchsicht der Urkundentexte zum
jeweiligen Ort gefüllt, wobei darauf geachtet wurde, nach Möglichkeit
wenigstens zwei Belegstellen zu finden und dabei Variation zu berücksichtigen.
Die Annotation\is{Annotation} der Belege geschah in jedem Fall manuell. Belege,
bei denen das Lemma bei der automatischen Annotation falsch zugeordnet wurde,
wurden übergangen. Die Tabellen im Anhang~\ref{sec:caoadjquanttab} listen die
ausgewerteten Belegmengen pro Bezugsort mit den hinzugezogenen
Ausstellungs\-orten in dessen Umkreis auf. Im Folgenden sollen kurz die Belege
zum Nom./Akk.\ Sg.\ F.\ und Pl.\ pro untersuchtem Ort charakterisiert werden.

\is{Stichprobe|)}

\subsection{Diskussion}
\label{subsec:cao_adjflex_disc}

\il{Alemannisch|(}
\subsubsection{Straßburg}
\label{par:adjstrassburg}
In Straßburger Urkunden wird regelmäßig kein Unterschied zwischen \norm{e}- und
\norm{iu}-Formen gemacht \REF{ex:adjstrbgregel}. In der Stichprobe sind
außerdem zwei Belege zu \norm{ērbǟre} \wdef{ehrenhaft, edel} mit \lit{-i} für
den Nom.\ Sg.\ F.\ enthalten \REF{ex:adjstrbgi_1}. Auch für den Nom./Akk.\ Pl.\
N.\ ist einmal \lit{-i} in der Stichprobe enthalten \REF{ex:adjstrbgi_3}, sowie
viermal \lit{-u/-û} neben regelmäßigem \lit{-e} \REF{ex:adjstrbgu}, davon ein
Doppelbeleg für \norm{kristenlich} \wdef{christlich} mit \lit{-u} und
\lit{-e} \REF{ex:adjstrbgu_1}. Die betreffenden Belege stammen alle aus
Straßburg selbst.

\begin{exe}
\ex \label{ex:adjstrbgregel}
	\begin{xlist}
	\ex \label{ex:adjstrbgregel_1}
		\gll ir erſame botten \\
			ihr ehrbar-\textsc{nom.pl.\MascM.st} Bevollmächtigten \\
		\trans \wdef{ihre ehrsamen Bevollmächtigten}
			\parencites(Nr.~N~7, Straßburg, 1262)[7,39]{cao5}

	\ex \label{ex:adjstrbgregel_2}
		\gll ander cleine vorderunge \\
			andere klein-\textsc{nom.pl.\FemI.st} Forderungen \\
		\trans \wdef{andere kleine Forderungen}
			\parencites(Nr.~N~14, Straßburg, 1262?)[14,3]{cao5}

	\ex \label{ex:adjstrbgregel_3}
		\gll vnsere ingeſigele \\
			unser-\textsc{nom.pl.\NeutI.st} \\
		\trans \wdef{unsere Siegel}
			\parencites(Nr.~N~227, Straßburg, 1283)[174,4]{cao5}
	\end{xlist}

\ex \label{ex:adjstrbgi}
	\begin{xlist}
	\ex{\label{ex:adjstrbgi_1}
		\let\eachwordthree\eachwordtwo
		\let\eachwordtwo\eachwordone
		\glll A: dîe êrberi frowe \\
			B: dú erberi frowe \\
			{} \textsc{def.nom.sg.\FemF} ehrhaft-\textsc{nom.sg.\FemF.st} Edelfrau \\
		\trans \wdef{die ehrhafte Edelfrau}
			\parencites(Nrn.~N~109~AB, Straßburg, 1272)[79,21]{cao5}}

	\ex \label{ex:adjstrbgi_3}
		\gll Vnſeri kint \\
			unser-\textsc{nom.pl.\NeutX.st} Kind \\
		\trans \wdef{unsere Kinder}
			\parencites(Nr.~N~142, Straßburg, 1276)[100,18]{cao5}
	\end{xlist}
\end{exe}

\begin{exe}
\ex \label{ex:adjstrbgu}
	\begin{xlist}
	\ex{\label{ex:adjstrbgu_1}
		\let\eachwordthree\eachwordtwo
		\let\eachwordtwo\eachwordone
		\glll A: alle criſtenliche dinc \\
			B: allv criſtenlichu dinc \\
			{} alle-\textsc{acc.pl.\NeutI} christlich-\textsc{acc.pl.\NeutI.st} Ding \\
		\trans \wdef{alle christlichen Dinge}
			\parencites(Nrn.~N~10~AB, Straßburg, 1262)[9,20--21]{cao5}}

	\ex \label{ex:adjstrbgu_3}
		\gll vnſerû kint \\
			unser-\textsc{acc.pl.\NeutX} Kind \\
		\trans \wdef{unsere Kinder}
			\parencites(Nr.~2663, Straßburg, 1297)[62,35]{cao4}

	\ex \label{ex:adjstrbgu_4}
		\gll vnſerû jngeſigele \\
			unser-\textsc{acc.pl.\NeutI} Siegel \\
		\trans \wdef{unsere Siegel}
			\parencites(Nr.~2663, Straßburg, 1297)[63,25]{cao4}
	\end{xlist}
\end{exe}

Da \lit{-i} und \lit{-u} in der Regel an Stellen erscheinen, an denen nach dem
oberdeutschen\il{Oberdeutsch} Flexionsparadigma mit \norm{-iu} zu
rechnen ist, kann davon ausgegangen werden, dass diese beiden Grafien nicht
\norm{-e} zuzuordnen sind, obwohl \lit{-i} für \norm{-e} noch in der zweiten
Hälfte des 13.~Jahrhunderts im Alemannischen\il{Alemannisch} belegt ist
\autocites[41]{paul2007}[305]{ksw2}[vgl.~auch][466--467]{schirmunski1962}. Bei
der Teiluntersuchuchung im Anhang~\ref{sec:caoalemschwa} zur Grafie von
mittelhochdeutsch unbetontem \norm{e} [ə] traten bei den gewählten
nicht-adjektivischen Lexemen in Straßburg keine \lit{i}-Schreibungen auf.

\subsubsection{Basel}
\label{par:adjbasel}
Für Basel haben sich auch bei manueller Durchsicht keine Belege für Formen des
Nom./Akk.\ Sg./Pl.\ F.\ finden lassen. Dennoch zeigt sich hier zumindest für
Maskulina und Neutra klarer als in Straßburg die für das
Oberdeutsche\il{Oberdeutsch} typische Unterscheidung zwischen \norm{e}- und
\norm{iu}-Typen, die in der Regel durch \lit{-e} und \lit{-u/-ú} vertreten sind
\REF{ex:adjbaselregel}. Für den Akk.\ Pl.\ N.\ ist jeweils ein Beleg für
\norm{-iu} und einer für \norm{-e} in der Stichprobe enthalten
\REF{ex:adjbaselu}.

\begin{exe}
\ex \label{ex:adjbaselregel}
	\begin{xlist}
	\ex \label{ex:adjbaselregel_1}
		\gll anden vorgenanten hern peter vn̄ ſine erben \\
			an=den vorgenannten Herrn Peter und sein-\textsc{acc.pl.\MascM.st}
				Erben \\
		\trans \wdef{an den vorgenannten Herrn Peter und seine Erben}
			\parencites(Nr.~1682, Basel, 1293)[16,15--16]{cao3}

	\ex \label{ex:adjbaselregel_2}
		\gll minu kint dú vorgenanten \\
			mein-\textsc{nom.pl.\NeutMF} Kind die vorgenannten \\
		\trans \wdef{meine Kinder, die vorgenannten}
			\parencites(Nr.~3184, Rheinfelden, Kt.~Aargau, 1299)[376,13]{cao4}
	\end{xlist}

\ex \label{ex:adjbaselu}
	\begin{xlist}
	\ex \label{ex:adjbaselu_1}
		\gll fur Minv kint Walthern vn̄ hennrichen vn̄ Rvͦdolfen \\
			für mein-\textsc{acc.pl.\NeutM.st} Kind Walther-\textsc{obl} und
			Heinrich-\textsc{obl} und Rudolf-\textsc{obl} \\
		\trans \wdef{für meine Kinder Walther und Heinrich und Rudolf}
			\parencites(Nr.~1108, Basel, 1289)[402,36--37]{cao2}

	\ex \label{ex:adjbaselu_2}
		\gll in mine vaſ \\
			in mein-\textsc{acc.pl.\NeutI.st} Fass \\
		\trans \wdef{in meine Fässer}
			\parencites(Nr.~N~483, Mulhouse, Dépt.~Haut-Rhin, 1291)[351,23]{cao5}
	\end{xlist}
\end{exe}

\subsubsection{Zürich}
\label{par:adjzuerich}
Auch in Zürich zeigt sich zumindest im Plural die typische
oberdeutsche\il{Oberdeutsch} Verteilung von \norm{e}- und \norm{iu}-Typen, die
hier in der Regel durch \lit{-e} und \lit{-ú} repräsentiert sind
\REF{ex:adjzuerregel}. Für den Nom.\ Sg.\ F.\ ist in der Stichprobe sowohl ein
Beleg für \norm{-iu} als auch für \norm{-e} enthalten \REF{ex:adjzueru}. Diese
stammen allerdings nicht aus Zürich selbst, sondern aus dem ca.\ 40~km in
südöstlicher Richtung liegenden Hohenrain (Kt.~Luzern).

\begin{exe}
\ex \label{ex:adjzuerregel}
	\begin{xlist}
	\ex \label{ex:adjzuerregel_1}
		\gll zewene geliche brieue \\
			zwei[\MascI] gleich-\textsc{acc.pl.\MascI.st}
				Urkunde-\textsc{acc.pl.\MascI} \\
		\trans \wdef{zwei gleiche Urkunden}
			\parencites(Nr.~2209, Zürich, 1295)[364,33]{cao3}

	\ex \label{ex:adjzuerregel_2}
		\gll ich herre walther vn̄ minu kint \\
			ich Herr Walther und mein-\textsc{nom.pl.\NeutX.st} Kind \\
		\trans \wdef{ich, Herr Walther, und meine Kinder}
			\parencites(Nr.~456, Hohenrain, Kt.~Luzern, 1281)[396,33--34]{cao1}
	\end{xlist}

\ex \label{ex:adjzueru}
	\begin{xlist}
	\ex \label{ex:adjzueru_1}
		\gll min elichú huſvrowͮe \\
			mein rechtmäßig-\textsc{nom.sg.\FemF.st} Ehefrau \\
		\trans \wdef{meine rechtmäßige Ehefrau}
			\parencites(Nr.~260, Hohenrain, Kt.~Luzern, 1276)[271,9]{cao1}

	\ex \label{ex:adjzueru_2}
		\gll mín êliche wirten \\
			mein rechtmäßig-\textsc{nom.sg.\FemF.st} Ehefrau \\
		\trans \wdef{meine rechtmäßige Ehefrau}
			\parencites(Nr.~1888, Hohenrain, Kt.~Luzern, 1294)[173,11]{cao3}
	\end{xlist}
\end{exe}

Neben zwei Belegen mit \lit{rechte} \wdef{rechtmäßige} im Akk.~Sg.~F.\
entsprechend dem in \REF{ex:adjzuere_1} finden sich auch zwei mit \lit{rechtú}
\wdef{rechte} für denselben syntaktischen Kontext wie in \REF{ex:adjzuere_3};
vergleiche dazu \citet[270--271]{ksw2}. Bei den letzteren zwei Urkunden handelt
es sich um eine Urkunde und ihre Bestätigung \autocite[375]{caor}. Für beide
ist Zürich als Ausstellungs\-ort angegeben.

\begin{exe}
\ex \label{ex:adjzuere}
	\begin{xlist}
	\ex \label{ex:adjzuere_1}
		\gll in rechte giſelſchaft \\
			in recht-\textsc{acc.sg.\FemI.st} Geiselschaft \\
		\trans \wdef{in rechtsgemäße Geiselschaft}
			\parencites(Nr.~35, Zürich, 1256 und Nr.~188, ebd., 1272)[66,31; 204,31]{cao1}

	\ex \label{ex:adjzuere_3}
		\gll das es rechtú not tête \\
			dass es recht-\textsc{acc.sg.\FemI.st} Not täte \\
		\trans \wdef{dass es von Rechts wegen erforderlich wäre}
			\parencites(Nr.~1591, Zürich, 1292 und Nr.~1756, ebd., 1292/93)[731,3]{cao2}[67,26--27]{cao3}
	\end{xlist}
\end{exe}

Daneben ist die Häufung von Belegen ohne overte Markierung\is{Genusmarkierung}
im Nom./Akk.\ Pl.\ M.\ auffällig: drei Belegen mit \norm{-e} stehen neun ohne
Flexionsendung gegenüber. Bei diesen unflektierten Belegen handelt es sich
sämtlich um Vorkommen des Possessivbegleiters \norm{mīn} \wdef{mein} wie in
(\ref{ex:adjzuer0_1}--b), die flektiertem \norm{mine} \wdef{meine}
\REF{ex:adjzuer0_3} und \norm{geliche} \wdef{gleiche} gegenüberstehen. Ein Beleg
für \norm{mīn} und der für \norm{mīne} stammen nicht aus Zürich selbst, sondern
aus Eschenbach (Kt.~Luzern), etwa 45~km südwestlich von Zürich und 10~km
nordöstlich von Luzern. Selbst wenn diese zwei Belege nicht gezählt werden,
ändert sich das Bild nicht grundlegend.

\begin{exe}
\ex \label{ex:adjzuer0}
	\begin{xlist}
	\ex \label{ex:adjzuer0_1}
		\gll ſo ſvln min erbin · diz {vol fvͦrin.} \\
			so sollen mein[\textsc{nom.pl.\MascA}] Erben {} dies ausführen \\
		\trans \wdef{so sollen meine Erben dies ausführen}
			\parencites(Nr.~19, Zürich, 1251)[28,20]{cao1}

	\ex \label{ex:adjzuer0_2}
		\gll fúr mich / vnd min nachkomen \\
			für mich {} und mein[\textsc{acc.pl.\MascA}] Nachkommen \\
		\trans \wdef{für mich und meine Nachkommen}
			\parencites(Nr.~2051~A, Zürich, 1294)[278,25--26]{cao3}

	\ex \label{ex:adjzuer0_3}
		\gll fv̓r alle mine nahkomen alde mine erben \\
			für alle mein-\textsc{acc.pl.\MascA.st} Nachkommen oder
				mein-\textsc{acc.pl.\MascA.st} Erben \\
		\trans \wdef{für alle meine Nachkommen oder meine Erben}
			\parencites(Nr.~1982, Eschenbach, Kt.~Luzern, 1294)[239,20--21]{cao3}
	\end{xlist}
\end{exe}

Generell erscheinen im Raum Zürich Formen des Nom./Akk.\ Pl.\ M.\ anders
markiert als solche des Nom./Akk.\ Pl.\ N.

\subsubsection{Konstanz}
\label{par:adjkonst}
Die Belegverteilung in Urkunden aus der Bodenseeregion um Konstanz weist die
für das oberdeutsche\il{Oberdeutsch} typische Differenzierung zwischen
\norm{e}- und \norm{iu}-Typen auf \REF{ex:adjkonstregel}, letztere treten auch
grafisch als \lit{iu} in Erscheinung (in St.~Galler Urkunden auch als
\lit{vͥ}). Daneben sind wenige flexionslose Belege in Positionen des Paradigmas
enthalten, in denen ansonsten regelmäßig \norm{e} steht, nämlich einmal im
Akk.\ Sg.\ F.\ \REF{ex:adjkonstesg} sowie viermal im Nom.\ Pl.\ M.\
\REF{ex:adjkonstepl}.

\begin{exe}
\ex \label{ex:adjkonstregel}
	\begin{xlist}
	\ex \label{ex:adjkonstregel_1}
		\gll Huc · Cvͦnrade · vͦlrich · vn̄ Johanſ mine ſv̓ne \\
			Hugo {} Konrad {} Ulrich {} und Johannes
				mein-\textsc{nom.pl.\MascM.st} Söhne \\
		\trans \wdef{Hugo, Konrad, Ulrich und Johannes, meine Söhne}
			\parencites(Nr.~2654, Konstanz, 1297)[57,33]{cao4}

	\ex \label{ex:adjkonstregel_2}
		\gll wir die aigeniv Jngeſigel hant \\
			wir die eigen-\textsc{acc.pl.\NeutI.st} Siegel haben \\
		\trans \wdef{wir, die eigene Siegel haben}
			\parencites(Nrn.~530~AB, Konstanz, 1282)[468,3--4]{cao1}
	\end{xlist}

\ex \label{ex:adjkonstesg}
	\begin{xlist}
	\ex \label{ex:adjkonstesg_1}
		\gll mine tohter \\
			mein-\textsc{acc.sg.\FemF.st} Tochter \\
		\trans \wdef{meine Tochter}
			\parencites(Nr.~2226, Überlingen, Bodenseekr., 1295)[373,42]{cao3}

	\ex \label{ex:adjkonstesg_2}
		\gll fur min muͦter \\
			für mein[\textsc{acc.sg.\FemF}] Mutter \\
		\trans \wdef{für meine Mutter}
			\parencites(Nr.~530~B, Konstanz, 1282)[464,12]{cao1}
	\end{xlist}
\end{exe}

\begin{exe}
\ex \label{ex:adjkonstepl}
	\begin{xlist}
	\ex \label{ex:adjkonstepl_1}
		\gll mine ſv̓ne \\
			mein-\textsc{nom.pl.\MascM.st} Söhne \\
		\trans \wdef{meine Söhne}
			\parencites(Nr.~2654, Konstanz, 1297)[57,44]{cao4}

	\ex \label{ex:adjkonstepl_2}
		\gll {des ſelben} huſes reht erben \\
			desselben Hauses recht[\textsc{nom.pl.\MascA}] Erben \\
		\trans \wdef{desselben Hauses rechtmäßige Erben}
			\parencites(Nr.~2675, Konstanz, 1297)[71,34]{cao4}
	\end{xlist}
\end{exe}
\il{Alemannisch|)}

\il{Schwäbisch|(}
\subsubsection{Ulm}
\label{par:adjulm}
Die Stichprobe zu Ulm verhält sich vollkommen unauffällig: Es wird klar
zwischen \norm{-e} und \norm{-iu} geschieden, beide Typen sind auch in der
Grafie regelmäßig so realisiert \REF{ex:adjulm}.

\begin{exe}
\ex \label{ex:adjulm}
	\begin{xlist}
	\ex \label{ex:adjulm_1}
		\gll ander biderbe lute \\
			ander brav-\textsc{nom.pl.\MascX.st} Leute \\
		\trans \wdef{andere brave Leute}
			\parencites(Nr.~3308, Ulm, 1299)[455,44]{cao4}

	\ex \label{ex:adjulm_2}
		\gll mine vorginanten froͤwen von Sovelingen \\
			mein-\textsc{nom.pl.\FemF.st} vorgenannten Frauen von Söflingen \\
		\trans \wdef{meine vorgenannten Frauen von Söflingen}
			\parencites(Nr.~2467, Ulm, 1296)[526,36]{cao3}

	\ex \label{ex:adjulm_3}
		\gll vnſeriv Jnſigel \\
			unser-\textsc{acc.pl.\NeutI.st} Siegel \\
		\trans \wdef{unsere Siegel}
			\parencites(Nr.~1662, Ulm, 1293)[3,12--13]{cao3}
	\end{xlist}
\end{exe}

\subsubsection{Augsburg}
\label{par:adjaugsburg}
Auch in Augsburg wird bei der starken Adjektivdeklination klar zwischen
\norm{e}- und \norm{iu}-Typen unterschieden \REF{ex:adjaugs}. Für den letzteren
ist für Augsburg selbst hauptsächlich \lit{iuͤ} neben einfachem \lit{iu}
bezeugt; im ca. 23~km in nordöstlicher Richtung entfernten Aichach findet sich
regelmäßig \lit{u} in \lit{minu} \wdef{meine}. Auffällige Unterschiede im
Gebrauch zwischen Stadt und Umland machen sich auch hier nicht bemerkbar.

\begin{exe}
\ex \label{ex:adjaugs}
	\begin{xlist}
	\ex \label{ex:adjaugs_1}
		\gll zwen erbere man \\
			zwei[\MascM] ehrenhaft-\textsc{nom.pl.\MascM.st} Mann \\
		\trans \wdef{zwei ehrenhafte Männer}
			\parencites(Nr.~1270, Augsburg, 1290)[508,40]{cao2}

	\ex \label{ex:adjaugs_2}
		\gll gvͤte Hantveſte \\
			gut-\textsc{acc.pl.\FemI.st} Urkunde \\
		\trans \wdef{gültige Urkunden}
			\parencites(Nr.~3471, Augsburg, 1299)[557,21]{cao4}

	\ex \label{ex:adjaugs_3}
		\gll Ane minivͤ Manlehen \\
			ohne mein-\textsc{acc.pl.\NeutI.st} Mannslehen \\
		\trans \wdef{ohne meine Mannslehen}
			\parencites(Nr.~1363, Augsburg, 1291)[586,9]{cao2}
	\end{xlist}
\end{exe}
\il{Schwäbisch|)}

\il{Ostfränkisch|(}
\subsubsection{Nürnberg}
\label{par:adjnuernberg}
Die Belege der Stichprobe zu Nürnberg zeigen ebenfalls keine Auffälligkeiten in
der Verteilung. Allein die relativ geringe Belegmenge führt dazu, dass für den
Akk.\ Pl.\ M. sowie für den Nom./Akk.\ Pl.\ F.\ keine Belege gefunden werden
konnten. Da \norm{-e} im Akk.\ Sg.\ F.\ und im Nom.\ Pl.\ M. ansonsten aber
erwartungsgemäß \norm{-iu} (geschrieben ebenfalls \lit{iu}) im Nom.\ Sg.\ F.\
und Nom./Akk.\ Pl.\ N.\ gegenübersteht, kann mit etwas Vorsicht davon
ausgegangen werden, dass der Regelfall auch hier gilt. Beispiele werden in
\REF{ex:adjnuern} gegeben.

\begin{exe}
\ex \label{ex:adjnuern}
	\begin{xlist}
	\ex \label{ex:adjnuern_1}
		\gll ich vn̄ mine ſvne \\
			ich und mein-\textsc{nom.pl.\MascM.st} Söhne \\
		\trans \wdef{ich und meine Söhne}
			\parencites(Nr.~3428, Kl.~Seligenporten, Kr.~Neumarkt in der Oberpfalz, 1299)[525,15]{cao4}

	\ex \label{ex:adjnuern_2}
		\gll daz vnſ ehaftiv not irre \\
			dass uns rechtsgültig-\textsc{nom.sg.\FemI.st} Not hindere \\
		\trans \wdef{dass wir aus einem rechtsgültigen Grund verhindert sind}
			\parencites(Nr.~949, Nürnberg, 1287)[301,12--13]{cao2}

	\ex \label{ex:adjnuern_3}
		\gll vnſeriv chint \\
			unser-\textsc{nom.pl.\NeutX.st} Kind \\
		\trans \wdef{unsere Kinder}
			\parencites(Nrn.~1972~AB, Nürnberg, 1294)[225,17--18]{cao3}
	\end{xlist}
\end{exe}

\subsubsection{Regensburg}
\label{par:adjregensburg}
Die exzerpierten Stellen aus Regensburger Urkunden verhalten sich regelmäßig.
Auch hier besteht die Schwierigkeit, dass sich keine Belege für den Nom./Akk.\
Pl.\ F.\ haben finden lassen. Im Singular sowie zwischen Plural Maskulinum und
Neutrum zeigt sich jedoch klar der Formenunterschied; \norm{iu}~/yː/ ist
diphthongiert als \lit{eu} repräsentiert. Die Schwa-\isi{Apokope}
\autocites{lindgren1953}[109--111]{paul2007} macht sich hier bemerkbar,
insofern an Stellen des Paradigmas, die sonst regelmäßig \norm{-e} aufweisen,
die Endung fehlen kann \REF{ex:adjregbge}.%

\begin{exe}
\ex \label{ex:adjregbge}
	\begin{xlist}
	\ex \label{ex:adjregbge_3}
		\gll an vbel liſte \\
			ohne schlecht[\textsc{acc.pl.\MascI}] Absichten \\
		\trans \wdef{ohne schlechte Absichten}
			\parencites(Nr.~1970, Laaber, Kr.~Regensburg, 1294)[224,26]{cao3}

	\ex \label{ex:adjregbge_1}
		\gll ander erſame levte \\
			ander ehrsam-\textsc{nom.pl.\MascA.st} Leute \\
		\trans \wdef{andere ehrsame Leute}
			\parencites(Nr.~3404, Regensburg, 1299)[507,43]{cao4}

	\ex \label{ex:adjregbge_2}
		\gll an ein new ſetze \\
			ohne ein neu[\textsc{acc.sg.\FemI}] Rebpflanzung\footnotemark{} \\
		\trans \wdef{ausgenommen eine neue Rebpflanzung}
			\parencites(Nr.~N~447, Regensburg, 1290)[328,38]{cao5}
			\footnotetext{Siehe \citet[\nopp 102]{caor}; \citet[s.\,v.~\fw{setze}]{%
				lexer:mhdhwb} definiert \norm{sėƶƶe} in diesem Zusammenhang als
				\textquote{ein mit reben besetztes grundstück von bestimmter
				grösse} und bestimmt es als starkes Femininum.}
			%
	\end{xlist}
\end{exe}
\il{Ostfränkisch|)}

\il{Bairisch|(}
\subsubsection{München}
\label{par:adjmuenchen}
Mehr noch als in Regensburg zeigt sich die Schwa-\isi{Apokope} in der
Stichprobe zu München, insofern an allen Stellen, an denen sonst regelgemäß
\norm{-e} steht, die Endung häufig fehlt. Der Flexionstyp \norm{-iu} weist
auch hier die Form \lit{-iu} auf, neben der bairischen Form \lit{-eu}. Beide
Flexionstypen verteilen sich regelgemäß: -Ø steht systematisch also einer
Grafie vom Typ \norm{-iu} gegenüber \REF{ex:adjmuench}.

\begin{exe}
\ex \label{ex:adjmuench}
	\begin{xlist}
	\ex \label{ex:adjmuench_1}
		\gll zwo erbwær iunchfrawen \\
			zwei[\FemF] ehrenhaft[\textsc{acc.pl.\FemF}] Jungfrauen \\
		\trans \wdef{zwei ehrenhafte junge Frauen}
			\parencites(Nr.~1024, München, 1288)[348,21]{cao2}

	\ex \label{ex:adjmuench_2}
		\gll ſiniv kint \\
			sein-\textsc{nom.pl.\NeutA.st} Kind \\
		\trans \wdef{seine Kinder}
			\parencites(Nr.~2371, München, 1296)[473,9]{cao3}
	\end{xlist}
\end{exe}

\subsubsection{Salzburg}
\label{par:adjsalzburg}
Die Belege aus Salzburger Urkunden verhalten sich ähnlich wie die Münchner
Belege. Auch hier wird regelmäßig zwischen den Typen \norm{-e} und
\norm{-iu} unterschieden, die als \lit{-e}/-Ø beziehungsweise \lit{-eu}
grafischen Ausdruck finden \REF{ex:adjsalzbg}.

\begin{exe}
\ex \label{ex:adjsalzbg}
	\begin{xlist}
	\ex \label{ex:adjsalzbg_1}
		\gll ærwer / vnt ehaft ſache \\
			ehrenhaft[\textsc{nom.sg.\FemI}] {} und rechtsgültig[\textsc{nom.sg.\FemI}]
				Sache \\
		\trans \wdef{ehrenhafter und rechtsgültiger Grund}
			\parencites(Nr.~818, Bad Reichenhall, Kr.~Berchtesgadener Land, 1286)[177,1--2]{cao2}

	\ex \label{ex:adjsalzbg_2}
		\gll vmb ander niwe veſt \\
			um andere neu-\textsc{acc.pl.\FemI.st} Burg \\
		\trans \wdef{bezüglich anderer neuer Burgen}
			\parencites(Nr.~695, Salzburg, um 1285)[104,19]{cao2}

	\ex \label{ex:adjsalzbg_3}
		\gll dev vorgenantev gvͦt \\
			die vorgenannt-\textsc{acc.pl.\FemI.st} Gut \\
		\trans \wdef{die vorgenannten Güter}
			\parencites(Nr.~2446, Salzburg, 1296)[514,11--12]{cao3}
	\end{xlist}
\end{exe}

Darüber hinaus tritt einmal \lit{Newev} \wdef{neue} mit einer \norm{iu}-Form
statt der gewöhnlichen \norm{e}-Form im Akk.\ Sg.\ F.\ auf
\REF{ex:adjsalzbgirr_1}. Mit \citet[270]{ksw2} lässt sich vermuten, dass es sich
dabei um \textquote{\mbox{eine} Angleichung an den Nom.~Sg.~fem.~stark}
handelt, die sie als möglicherweise schwerpunktmäßig bairisch betrachten.

\begin{exe}
	\ex \label{ex:adjsalzbgirr_1}
		\gll der ein Newev veſte gebown hat \\
			der ein neu-\textsc{acc.sg.\FemI.st} Burg gebaut hat \\
		\trans \wdef{der eine neue Burg gebaut hat}
			\parencites(Nr.~695, Salzburg, um 1285)[103,11]{cao2}
\end{exe}

\subsubsection{Wien}
\label{par:adjwien}
Auch für Wien lässt sich ein klarer Unterschied zwischen \norm{e}- und
\norm{iu}-Typen in der Flexion ausmachen, der sich regelmäßig in den Grafien
\lit{-e/-Ø} und \lit{-eu/-iu} manifestiert. Für den Plural hat es sich
allerdings als schwierig herausgestellt, eindeutige Belege für starke Maskulina
und Feminina zu finden. Diese treten jeweils mit \isi{Apokope}
auf \REF{ex:adjwienapo}. Für den Akk.\ Sg.\ F.\ fanden sich die Formen in
\REF{ex:adjwienakksgf}.

\begin{exe}
\ex \label{ex:adjwienapo}
	\begin{xlist}
	\ex \label{ex:adjwienregel_1}
		\gll wider mein gêrben \\
			gegen mein[\textsc{nom.sg.\MascM}] Erben \\
		\trans \wdef{gegen meinen Erben}
			\parencites(Nr.~N~263, Wien, 1284)[210,24]{cao5}

	\ex \label{ex:adjwienregel_2}
		\gll brief vnd grozz hantveſt \\
			Urkunde und groß[\textsc{acc.pl.\FemI}] Urkunde \\
		\trans \wdef{Briefe und große Urkunden}
			\parencites(Nr.~67, Stift~Heiligenkreuz, Bz.~Baden, 1263)[103,21]{cao1}
	\end{xlist}

\ex \label{ex:adjwienakksgf}
	\begin{xlist}
	\ex \label{ex:adjwienakksgf_1}
		\gll dvrch rehte notdvrft \\
			durch recht-\textsc{acc.sg.\FemI.st} Notlage \\
		\trans \wdef{aufgrund der rechtsgültigen Notlage}
			\parencites(Nr.~2966, Wien, 1298)[243,3]{cao4}

	\ex \label{ex:adjwienakksgf_2}
		\gll ain halbe wiſe \\
			ein halb-\textsc{acc.sg.\FemI.st} Wiese \\
		\trans \wdef{eine halbe Wiese}
			\parencites(Nr.~N~718, Wien, 1295)[518,5]{cao5}
	
	\ex \label{ex:adjwienakksgf_3}
		\gll fver min tohter \\
			für mein[\textsc{acc.sg.\FemF}] Tochter \\
		\trans \wdef{für meine Tochter}
			\parencites(Nr.~1578, Wien, 1292)[724,6]{cao2}
	\end{xlist}
\end{exe}

Wie auch in Salzburg liegt ein Beleg mit einer Form vom Typ \norm{-iu} im
Akk.~Sg.~F. vor, der nicht in das regelmäßige oberdeutsche\il{Oberdeutsch}
Paradigma passt \REF{ex:adjwienakksgf_4}. Auch hier muss wohl eine Ausweitung
der Neutrumform ins Femininum angenommen werden, sodass in dieser Position
Variation zwischen \norm{-e} und \norm{-iu} vorliegt.

\begin{exe}
\ex \label{ex:adjwienakksgf_4}
	\gll rehtev	fuerziht \\
		rechtsgemäß-\textsc{acc.sg.\FemI.st} Verzicht\footnotemark{} \\
	\trans \wdef{rechtsgemäßer Verzicht}
		\parencites(Nr.~2424, Wien, 1296)[500,7]{cao3}
		\footnotetext{Der Eintrag für \norm{verƶiht} in \citet{lexer:mhdhwb}
		weist das Wort~-- anders als im Neuhochdeutschen\il{Neuhochdeutsch}~--
		als starkes Femininum mit der Bedeutung \wdef{Entsagung,
		Verzichtleistung} aus. Aus dem Text der Urkunde wird deutlich, dass es
		um die Zusage des rechtsgemäßen Verzichts des vormaligen Eigentümers
		auf die Sache nach dem Verkauf von Erbbesitz geht \autocite[vgl.~%
		auch][506]{caor}.}
\end{exe}

\il{Bairisch|)}

\subsection{Zusammenfassung}

Insgesamt verhalten sich die Belege zur starken Adjektivflexion bezüglich des
Nom./\allowbreak{}Akk.\ Sg.~F.\ und Nom./Akk.\ Pl.\ aller gewählten
Ausstellungs\-orte größtenteils entsprechend dem oberdeutschen\il{Oberdeutsch}
Deklinationsparadigma, wie in \tabref{tab:adjparadgmstr_pr} angegeben. Eine
Ausnahme\is{Ausnahme} bildet Straßburg, wo in Positionen mit regelmäßiger
\norm{iu}-Form in vergleichbarer Menge \norm{e}-Formen belegt sind. Durchgängige
Setzung von \norm{-e} im Nom./Akk.\ Sg.~F.\ sowie im Nom./Akk.\ Pl.\ ist in der
mittelhochdeutschen\il{Mittelhochdeutsch} Zeit an sich ein Merkmal des
Mitteldeutschen\il{Mitteldeutsch}
\autocites[181]{ksw2}[vgl.~auch][832]{wiesinger1983}. Die
\tabref{tab:adjcaoovw} fasst die Ergebnisse der Auswertung noch einmal
zusammen. Da \norm{mīn}-Belege im Nom.\ Sg.~F.\ nicht aussagekräftig sind,
wurden sie für diese Stelle des Paradigmas nicht gezählt.

Die belegten Grafietypen pro Ort und Stelle im Paradigma sind in
\tabref{tab:adjcaoovw} nach Häufigkeit\is{Frequenz} ihres Auftretens in der
Stichprobe sortiert. Dabei ist zu beachten, dass im Nom.\ Sg.\ aller drei
Genera\is{Genus} auch regulär eine nominal-starke Form ohne Flexionsendung
stehen kann, wie in \tabref{tab:adjparadgmstr_nm} dargestellt. Andernfalls ist
im Nom.\ Sg.~F.\ und Nom./Akk.\ Pl.~N.\ im oberdeutschen\il{Oberdeutsch}
Paradigma regelmäßig mit einer Form vom Typ \norm{-iu} zu rechnen. Die letzte
Spalte der Tabelle zeigt eine Bewertung zur Existenz einer systematischen
Unterscheidung zwischen \norm{e}- und \norm{iu}-haltigen Formen am jeweiligen
Bezugsort an. Ein Haken (\chk) bedeutet, dass diese vorliegt, ein
eingeklammerter Haken, dass die Unterscheidung sich nicht immer deutlich
äußert. Von Belang ist sie besonders im Plural.

\begin{table}
\centering
\caption{Belegte adjektivische Flexive in relevanten Paradigmen\-feldern im
\tit{Corpus der altdeutschen Originalurkunden}}

\begin{tabular}{
	| l |
	  c r | c r |
	  c r | c r | c r |
	  c |
}
\hline
\mr{2}{*}{Region}
	& \mc{2}{ c|}{\mr{2}{*}{\textsc{nom.sg.f}}}
	& \mc{2}{ c|}{\mr{2}{*}{\textsc{acc.sg.f}}}
	& \mc{6}{ c|}{\textsc{nom+acc.pl}}
	& \mr{2}{*}{\norm{e : iu}}
	\\

\cline{6-11}

%
	& \mc{2}{ c|}{}
	& \mc{2}{ c|}{}
	& \mc{2}{ c|}{\textsc{m}}
	& \mc{2}{ c|}{\textsc{f}}
	& \mc{2}{ c|}{\textsc{n}}
	& \mc{1}{ c|}{}
	\\

\hline
\hline

\mr{3}{*}{Straßburg}
	& -iu	& 2
	& -e	& 3
	& -e	& 15
	& -e	& 1
	& -e	& 6
	& \mr{3}{*}{(\chk)}
	\\

%
	& -Ø	& 2
	&   	& %
	& -Ø	& 2
	&   	& %
	& -iu	& 5
	& \mc{1}{ c|}{}
	\\

%
	& -e	& 2
	&   	& %
	&   	& %
	&   	& %
	&   	& %
	& \mc{1}{ c|}{}
	\\

\hline

\mr{2}{*}{Basel}
	& -Ø	& 5
	& ?		& %
	& -e	& 6
	& ?		& %
	& -iu	& 2
	& \mr{2}{*}{\chk}
	\\

%
	& -iu	& 3
	& 		& %
	& -Ø 	& 4
	& 		& %
	& -e	& 1
	& \mc{1}{ c|}{}
	\\

\hline

\mr{3}{*}{Zürich}
	& -iu	& 1
	& -e	& 4
	& -Ø	& 9
	& ?		& %
	& -iu   & 2
	& \mr{3}{*}{\chk}
	\\

%
 	& -e	& 1
	& -Ø	& 2
	& -e	& 3
	& 		& %
	& 		& %
	& \mc{1}{ c|}{}
	\\

%
	& 		& %
	& -iu	& 2
	& 		& %
	& 		& %
	& 		& %
	& \mc{1}{ c|}{}
	\\

\hline

\mr{2}{*}{Konstanz}
	& -iu	& 3
	& -e	& 1
	& -e	& 5
	& -e	& 1
	& -iu	& 6
	& \mr{2}{*}{\chk}
	\\

%
	& -Ø	& 2
	& -Ø	& 1
	& -Ø	& 4
	& 		& %
	&   	& %
	& \mc{1}{ c|}{}
	\\

\hline

\mr{2}{*}{Ulm}
	& -iu	& 1
	& -e	& 1
	& -e	& 1
	& -e	& 2
	& -iu	& 1
	& \mr{2}{*}{\chk}
	\\

%
	& 		& % einziger Ø-Beleg aus mîn
	&   	& %
	& -Ø	& 1
	& -Ø	& 1
	&   	& %
	& \mc{1}{ c|}{}
	\\

\hline

\mr{2}{*}{Augsburg}
	& -iu	& 4
	& -e	& 4
	& -e	& 15
	& -e	& 2
	& -iu	& 10
	& \mr{2}{*}{\chk}
	\\

%
	& -Ø	& 1
	& -Ø	& 1
	& -Ø	& 4
	&   	& %
	& -Ø	& 1
	& \mc{1}{ c|}{}
	\\

\hline

\mr{2}{*}{Nürnberg}
	& -iu	& 1
	& -Ø	& 1
	& -e	& 1
	& ?		& %
	& -iu	& 4
	& \mr{2}{*}{\chk}
	\\

%
	& 		& %
	& 		& %
	& -Ø	& 1
	& 		& %
	&   	& %
	& \mc{1}{ c|}{}
	\\

\hline

\mr{2}{*}{Regensburg}
	& -iu	& 2
	& -Ø	& 3
	& -Ø	& 6
	& ?		& %
	& -iu	& 5
	& \mr{2}{*}{\chk}
	\\

%
	& 		& %
	& -e	& 1
	& -e	& 1
	& 		& %
	&   	& %
	& \mc{1}{ c|}{}
	\\

\hline

\mr{2}{*}{München}
	& -iu	& 1
	& -Ø	& 3
	& -Ø	& 5
	& -Ø	& 2
	& -iu	& 3
	& \mr{2}{*}{\chk}
	\\

%
	& -Ø	& 1
	&   	& %
	&   	& %
	&   	& %
	& 		& %
	& \mc{1}{ c|}{}
	\\

\hline

\mr{2}{*}{Salzburg}
	& -iu	& 1
	& -Ø	& 4
	& -Ø	& 4
	& -e	& 1
	& -iu	& 7
	& \mr{2}{*}{\chk}
	\\

%
	& -Ø	& 1
	& -iu	& 1
	& 		& %
	& 		& %
	&   	& %
	& \mc{1}{ c|}{}
	\\

\hline

\mr{3}{*}{Wien}
	& -iu	& 2
	& -e	& 2
	& -Ø	& 5
	& -Ø	& 1
	& -iu	& 5
	& \mr{3}{*}{\chk}
	\\

%
	& 		& %
	& -Ø	& 2
	&   	& %
	& 		& %
	& 		& %
	& \mc{1}{ c|}{}
	\\

%
	& 		& %
	& -iu	& 1
	& 		& %
	& 		& %
	& 		& %
	& \mc{1}{ c|}{}
	\\

\hline
\end{tabular}
\label{tab:adjcaoovw}
\end{table}

%%%%%%%%%%%%%%%%%%%%%%%%%%%%%%%%%%%%%%%%%%%%%%%%%%%%%%%%%%%%%%%%%%%%%%%%%%%%%%%

\section{Adjektivdeklination in der \tit{Kaiserchronik}}
\label{sec:adjdeclkc}

Zur Überprüfung der Validität\is{Validierung} des Paradigmas für die \KC{}
wurde eine Stichprobe zu jeder im Rahmen dieser Teiluntersuchung ausgewerteten
Handschrift angefertigt. Da die Texte der \KC{} bisher nicht morphologisch
annotiert sind, konnten nicht alle neun Textzeugen zielgerichtet exhaustiv
ausgewertet werden. Zur Erstellung der Stichprobe wurde daher die
Tokenfrequenz\is{Frequenz} jeder durch Leerzeichen abgetrennten Zeichenkette in
allen vorliegenden \KC{}-Handschriften ermittelt. Die Liste wurde dann auf
Adjektive im oberen Häufigkeitsbereich hin durchgesehen.

Bei der Auswahl wurde neben der Häufigkeit\is{Frequenz} vor allem darauf
geachtet, Lemmata zu wählen, deren Wortformen sich grafisch mit möglichst
wenigen anderen Lemmata überschneiden. Die Lemmata \norm{guet} \wdef{gut}
(ca.~3.149 Belege) und \norm{grōȥ} \wdef{groß} (ca.~2.432 Belege) stellten
geeignete Kandidaten für die Analyse der starken Adjektivdeklination dar.
Zusätzlich wurden die Adjektive \norm{hėilic} \wdef{heilig} (ca.~1.816 Belege),
\norm{alt} \wdef{alt} (ca.~1.201 Belege) und \norm{arm} \wdef{arm} (ca.~640
Belege) ausgewertet. Diese erscheinen in der \KC{} fast ausschließlich mit
Bezug auf Männer und in definiten\is{Definitheit}, also schwach deklinierenden
Kontexten. Nicht alle Stellen des Paradigmas sind für die vorliegende
Auswertung relevant, daher werde ich mich auch hier bei der Charakterisierung
der einzelnen \KC{}-Handschriften nur auf die starken Formen des Nom./Akk.\
Pl.\ aller drei Genera konzentrieren.

\subsection{Anlage der Stichprobe}
\is{Stichprobe|(}

Um alle zu den jeweiligen Lemmata gehörigen Adjektivformen zu ermitteln, wurde
der Gesamttext aller \KC{}-Textzeugen zunächst mit Hilfe eines grob
formulierten regulären Ausdrucks\is{regulärer Ausdruck} für jedes dieser
Lemmata durchsucht. Die Ergebnisliste wurde nach grafischer Variation
systematisiert. Darauf basierend wurde ein regulärer Ausdruck zusammengebaut.
Auf dieser Grundlage wurden alle neun zu untersuchenden Textzeugen der
\KC{} durchsucht und relevante Textstellen exzerpiert.

Da sich die Menge der relevanten Textstellen pro Handschrift in der Regel im
mittleren bis oberen dreistelligen Bereich belief, wurde exemplarisch etwa
jeder vierte Beleg ausgewertet. Hierzu wurden \isi{Kasus}, \isi{Genus} und
\isi{Numerus} sowie der Deklinationstyp (stark/schwach) und die
Verwendung (attributiv voran-/nach\-ge\-stellt, pronominal) annotiert. Falls
für eine der besonders relevanten Stellen des Paradigmas (Nom./Akk.\ Pl.\
M./F./N.\ st.) auf diese Weise keine Belege gefunden werden konnten, wurde die
Liste noch einmal nach entsprechenden Beispielen durchsucht. Trotz allem
ergaben sich immer wieder Lücken. Nominalisierungen von Adjektiven wie
\norm{junc unde alt} \wdef{Jung und Alt}, \norm{diu hėiligen} \wdef{die
Heiligen} oder appositiv\is{Apposition} nachgestellte Attribute wie in
\norm{Sylvester der guete} \wdef{Sylvester, der gute} wurden nicht in die
Analyse einbezogen.

\is{Stichprobe|)}

\subsection{Diskussion}
\label{subsec:kc_adjflex_disc}

Im Folgenden werden die einzelnen Textzeugen bezüglich des Unterschieds in der
Plural\-flexion zwischen maskulin-femininem \norm{-e} und neutralem
\norm{-iu} kurz anhand von Beispielen charakterisiert.

\subsubsection{A1}
In A1 zeigt sich ein ausgeprägter Unterschied zwischen
\norm{-e} beim Maskulinum und Femininum gegenüber \norm{-iu} beim Neutrum
\REF{ex:kca1regel}. Der einzige Beleg für \norm{-e} im Nom./Akk.\ Pl.\ N.\ bei
\norm{grōȥ} \wdef{groß} wurde als Ausnahme\is{Ausnahme} gewertet
\REF{ex:kca1akkplne}.

\begin{exe}
\ex \label{ex:kca1regel}
	\begin{xlist}
	\ex \label{ex:kca1regel_1}
		\gll und ſuln din guͦte fríunt ſin. \\
			und werden dein gut-\textsc{nom.pl.\MascA.st} Freund sein \\
		\trans \wdef{und werden dir gute Freunde sein}
			(%
				A1:~13rb,3; vgl.~%
				\KC:~V.~3089;
				\cite[137]{schroeder1895}%
			)

	\ex \label{ex:kca1regel_2}
		\gll deſ chom ſi ſit ingroze note. \\
			dessen kam sie später in=groß-\textsc{acc.pl.\FemI.st} Nöte \\
		\trans \wdef{dadurch kam sie später in große Nöte}
			(%
				A1:~49vb,13; vgl.~%
				\KC:~11413;
				\cites[290]{schroeder1895}%
			)

	\ex \label{ex:kca1regel_3}
		\gll ſi ſageten groziv nivmære. \\
			sie sagten groß-\textsc{acc.pl.\NeutI.st} Neuigkeiten \\
		\trans \wdef{Sie berichteten große Neuigkeiten}
			(%
				A1:~33rb,33; vgl.~%
				\KC:~V.~7710;
				\cites[222]{schroeder1895}%
			)
	\end{xlist}

\ex\label{ex:kca1akkplne}
	\gll der durch ſine lute. \\
		der durch sein-\textsc{acc.pl.\MascA.st} Leute \\
\sn \gll ſo groze zaichen tæte. \\
		so groß-\textsc{acc.pl.\NeutI} Zeichen täte \\
	\trans \wdef{der durch seine Leute so große Wunder wirkte}
		(%
			A1:~45rb,10--11; vgl.~%
			\KC:~V.~10331--10332;
			\cite[271]{schroeder1895}%
		)
\end{exe}

\subsubsection{M}
In M erscheint \norm{-e} regelmäßig apokopiert\is{Apokope} oder das Adjektiv ist
in den betreffenden Kontexten unflektiert. Im Plural Neutrum ist dagegen klar
\norm{-iu} bezeugt. Belege für den Plural Femininum liegen in der Stichprobe
keine vor.

\begin{exe}
\ex \label{ex:kcmpl}
	\begin{xlist}
	\ex \label{ex:kcmpl_1}
		\gll Piternâr warẽ auch gut chnehte. \\
			Viterber waren auch gut[\textsc{nom.pl.\MascM}] Knechte \\
		\trans \wdef{Die Viterber waren auch gute Krieger.}
			(%
				M:~32vb,34; vgl.~%
				\KC:~V.~4383;
				\cite[161]{schroeder1895}%
			)

	\ex \label{ex:kcmpl_2}
		\gll Wir weln iv grozzív wunder ſan. \\
			Wir wollen euch groß-\textsc{acc.pl.\NeutI.st} Wunder sagen \\
		\trans \wdef{Wir wollen euch große Wunder berichten.}
			(%
				M:~14va,10; vgl.~%
				\KC:~V.~1839;
				\cite[115]{schroeder1895}%
			)
	\end{xlist}
\end{exe}

Im Nom.\ Sg.\ steht im Femininum hauptsächlich \norm{-iu} \REF{ex:kcmsg_1}; im
Akkusativ tritt neben der Nullendung\is{Endungslosigkeit} und \norm{-e} in
wenigen Fällen das Flexiv \norm{-iu} auf (\ref{ex:kcmsg_2}--c):
\citet[191--192]{reichmannwegera1993} zufolge kann im
Ostoberdeutschen\il{Bairisch} bis in die zweite Hälfte des 15.~Jahrhunderts
\norm{-iu} auch im Akk.\ Sg.\ F.\ sowie \norm{-iu} neben
\norm{-e} im Nom./Akk.\ Pl.\ M./F. stehen. Welche Form in M also für den Plural
Femininum gilt, lässt sich ohne Belege nicht eindeutig\is{Ambiguität}
feststellen. Aufgrund der Mengenverhältnisse ist jedoch mit \norm{-e}/-Ø zu
rechnen.

\begin{exe}
\ex \label{ex:kcmsg}
	\begin{xlist}
	\ex \label{ex:kcmsg_1}
		\gll Heiligiv magt nv erlôs {vns ſich}. \\
			heilig-\textsc{nom.sg.\FemF} Jungfrau nun erlöse uns\footnotemark \\
		\trans \wdef{Heilige Jungfrau, jetzt erlöse uns doch!}
			(%
				M:~84ra,21; vgl.~%
				\KC:~V.~11027;
				\cite[283]{schroeder1895}%
			)
		\footnotetext{Das \isi{Personalpronomen} der 1.~Pers.~Pl.~Akk. lautete
			im Althochdeutschen \norm{unsih} \autocite[344--346]{braune2023}.
			Laut \citet[362]{ksw2} muss sich \textquote{\textins*{d}er Ersatz
			der Form \fw{unsich} durch die Dativform \fw{uns} \textelp{} im
			12.~Jh.\ sehr zügig vollzogen haben}. Diese alte Form ist hier
			anscheinend nicht mehr erkannt worden.}

	\ex \label{ex:kcmsg_2}
		% Versumbruch durch "/" ersetzt wegen Seitenumbruch
		\gll Si het grozz wûnne. {/} \\
			sie hatten groß[\textsc{acc.sg.\FemI}] Freude \\
		\gll Mit ir peider leibe. \\
			mit ihr beider Körper \\
		\trans \wdef{Sie hatten große Freude an ihren beiden Körpern.}
			(%
				M:~10ra,20--21; vgl.~%
				% H:~7rb,18--19;
				% B1:~5va,32--33;
				% VB:~6vb,30--31;
				% C1:~6vb,11--12;
				% K:~8ra,12--13;
				\KC:~V.~1230--1231;
				\cite[104]{schroeder1895}%
			)

	\ex \label{ex:kcmsg_3}
		\gll Si chomen all ín grozzev not. \\
			Sie kamen alle in groß-\textsc{acc.sg.\FemI.st} Not \\
		\trans \wdef{Sie kamen alle in große Not.}
			(%
				M:~40va,2; vgl.~%
				\KC:~V.~5384;
				\cite[180]{schroeder1895}%
			)
	\end{xlist}
\end{exe}

\subsubsection{H}
In H zeigt sich entsprechend dem mitteldeutschen\il{Mitteldeutsch}
\isi{Schreibdialekt} der Handschrift durch\-gängig \norm{-e} im
Nom./Akk.\ Pl.\ wie auch im Singular Femininum \autocite[vgl.][181--184]{ksw2}.
Der Genus\-unterschied im Plural ist also aufgehoben, wie in \REF{ex:kchregel}
illustriert. Belege für den Plural Femininum sind in der Stichprobe keine
vorhanden.

\begin{exe}
\ex \label{ex:kchregel}
	\begin{xlist}
	\ex \label{ex:kchregel_1}
		\gll Owi wie guͦte knechte ſie waren. \\
			{Oh weh} wie gut-\textsc{nom.pl.st} Knechte[\MascM] sie waren \\
		\trans \wdef{Oh weh, was für gute Krieger sie waren!}
			(%
				H:~2va,37; vgl.~%
				\KC:~V.~311;
				\cite[85]{schroeder1895}%
			)

	\ex \label{ex:kchregel_2}
		\gll Vil guͦte werc er worchte. \\
			vil gut-\textsc{acc.pl.st} Werk[\NeutI] er wirkte \\
		\trans \wdef{Er wirkte viele gute Werke.}
			(%
				H:~79va,6; vgl.~%
				\KC:~V.~13072;
				\cite[318]{schroeder1895}%
			)
	\end{xlist}
\end{exe}

\subsubsection{B1}
Wie in M fehlt auch in B1 im Plural Maskulinum und Femininum häufig
das \norm{-e} der Flexionsendung (\ref{ex:kcb1regel_1}--b),
wenn man davon ausgeht, dass sich die drei exzerpierten Belege regelmäßig
verhalten. Im Neutrum liegt regelmäßig eine Form vom Typ \norm{-iu} vor
\REF{ex:kcb1regel_3}.

\begin{exe}
\ex \label{ex:kcb1regel}
	\begin{xlist}
	\ex \label{ex:kcb1regel_1}
		\gll vnd erwelte fûnfhundert alt heren \\
			und erwählte fünfhundert alt[\textsc{nom.pl.\MascM}] Herren \\
		\trans \wdef{und erwählte fünfhundert alte Herren/Älteste}
			% \footnote{Vgl.~\citet[s.\,v.~\textit{althêrre}]{mwb1}.}
			(%
				B1:~23vb,23; vgl.~%
				\KC:~V.~8477;
				\cite[237]{schroeder1895}%
			)

	\ex \label{ex:kcb1regel_2}
		\gll Dez chom ſi ſeit in groz note \\
			des kam sie später in groß[\textsc{acc.pl.\FemI}] Nöte \\
		\trans \wdef{dadurch kam sie später in große Nöte}
			(%
				B1:~31vb,15; vgl.~%
				\KC:~V.~11413;
				\cite[290]{schroeder1895}%
			)

	\ex \label{ex:kcb1regel_3}
		\gll zwai grozeu her ſínt ſament chomen \\
			zwei[\NeutI] groß-\textsc{nom.pl.\NeutI.st} Heer sind zusammen
				gekommen \\
		\trans \wdef{zwei große Heere sind zusammengekommen}
			(%
				B1:~11rc,2; zu
				\KC:~V.~3535;
				\cite[146]{schroeder1895}%
			)
	\end{xlist}
\end{exe}

\subsubsection{VB}
\label{par:adjvb}
Die Handschrift VB weist im Plural einen relativ klaren Unterschied
zwischen \norm{-e} beziehungsweise -Ø für das Maskulinum und Femininum
einerseits und \norm{-iu} für das Neutrum andererseits auf \REF{ex:kcvbregel},
wenn auch nur ein einziger Beleg für das Maskulinum vorliegt.

\begin{exe}
\ex \label{ex:kcvbregel}
	\begin{xlist}
	\ex \label{ex:kcvbregel_1}
		\gll Daz ſich nie geſamte ein her ſo frvm \\
			dass sich nie sammelte ein Heer so trefflich \\
	\sn \gll An den cheiſer Jvlium \\
			an den Kaiser Julius \\
	\sn \gll Noch alſo groze magen \\
			noch genauso groß-\textsc{nom.pl.\MascM.st} Verwandte \\
		\trans \wdef{dass sich nie ein trefflicheres Heer unter Kaiser Julius 
			versammelte, noch genauso große Verbündete.}
			(%
				VB:~95rb,5--7; vgl.~%
				\KC:~V.~14035--14036;
				\cite[335]{schroeder1895}%
			)

	\ex \label{ex:kcvbregel_2}
		\gll Ich han vil groze ſorgen \\
			Ich habe viel groß-\textsc{acc.pl.\FemI.st} Sorgen \\
		\trans \wdef{Ich habe sehr große Sorgen.}
			(%
				VB:~92va,22; zu
				\KC:~V.~13514;
				\cite[309]{schroeder1895}%
			)

	\ex \label{ex:kcvbregel_3}
		\gll Ir ſvlt gvtiv mezzer tragen. \\
			Ihr solt gut-\textsc{acc.pl.\NeutI.st} Messer tragen \\
		\trans \wdef{Ihr sollt gute Messer tragen.}
			(%
				VB:~24va,23; vgl.~%
				\KC:~V.~4944;
				\cite[172]{schroeder1895}%
			)
	\end{xlist}
\end{exe}

Daneben liegt ein Beleg mit \norm{-e} in \lit{gvte} \wdef{gute} vor, der als
Akk.\ Pl.\ N.\ aufgefasst wurde \REF{ex:kcvbaccplne_2}. Es ist denkbar, dass
dieser Beleg im Zusammenhang mit der zuvor festgestellten Nähe zum
Mitteldeutschen\il{Mitteldeutsch} steht \autocites(siehe auch
\sectref{phsec:vbherkunft})[vgl.][181--184]{ksw2}.

\begin{exe}
\ex \label{ex:kcvbaccplne}
	\begin{xlist}
	\ex \gll Karl waſ goteſ wigant. \\
			Karl war Gottes Streiter \\
	\sn \gll Manech gvte reht er benant. \\
			viel gut-\textsc{acc.pl.\NeutI.st} Recht er benannte \\
	\sn \gll Vil heiden er becherte \\
			viel Heiden er bekehrte \\
		\trans \wdef{Karl war Gottes Streiter. Er stellte viele gute
			Gesetze auf. Er bekehrte viele Heiden.}
			(%
				VB:~100vb,13--15; vgl.~%
				B1:~40va,10--12%
			)
		\label{ex:kcvbaccplne_2}

	\ex \gll Karl was ain wârer gotes wigant, \\
			Karl war ein wahrer Gottes Streiter \\
	\sn \gll die haiden er ze der cristenhaite getwanc. \\
			die Heiden er zu der Christentum zwang \\
		\trans \wdef{Karl war ein wahrer Gottesstreiter. Er zwang die Heiden
			zum Christentum.}
			(%
				\KC:~V.~15073--15074;
				\cite[354]{schroeder1895}; vgl.~%
				A1:~64va,37--38;
				M:~115va,13--14;
				H:~92ra,1--2;
				C1:~78rb,34;
				K:~89rb,26--27;
				Z:~304va,4--5%
			)
		\label{ex:kcvbaccplne_4}
		\\
	\end{xlist}
\end{exe}

\citet[585]{ksw2} zufolge wird \norm{manic} \blockquote{in aller Regel
stark-pronominal flektiert oder bleibt flexivlos}. Letzteres ist auch hier der
Fall und macht \norm{ręht} \wdef{Recht(e)} ambig\is{Ambiguität} bezüglich
seines Numerus. Da das Adjektiv an dieser Stelle \mbox{\lit{gvte}} \wdef{gute}
lautet und nicht \lit{gvteſ} \wdef{gutes}, wurde auf Plural geschlossen. Die
zugehörige Stelle in der Edition nach A1 weicht im Wortlaut komplett ab
\REF{ex:kcvbaccplne_4}; die angegebenen A- und C-Handschriften überliefern
ansonsten mehr oder weniger denselben Wortlaut.

\subsubsection{P}
Die Handschrift P weist keinen Unterschied zwischen \norm{e}- und
\norm{iu}-Formen des starken Adjektivs auf, was auf ihren
mitteldeutschen\il{Mitteldeutsch} \isi{Schreibdialekt} zurückzuführen ist. Es
ließen sich keine Belege für den Plural Femininum finden, obwohl aufgrund der
vergleichsweise geringeren Belegmenge jeder zweite Beleg ausgewertet wurde. Je
ein Beispiel für die anderen beiden Pluralformen wird in \REF{ex:kcpregel}
gegeben.

\begin{exe}
\ex \label{ex:kcpregel}
	\begin{xlist}
	\ex \label{ex:kcpregel_1}
		\gll ir ſit vil guͦte knechte. \\
			ihr seid viel gut-\textsc{nom.pl.st} Diener[\MascM] \\
		\trans \wdef{Ihr seid sehr gute Diener.}
			(%
				P:~45ra,26; vgl.~%
				B1:~16rb,18;
				VB:~26vb,35%
			)

	\ex \label{ex:kcpregel_2}
		\gll dv fordereſ zvͦ mir groze dinc. \\
			du forderst zu mir groß-\textsc{acc.pl.st} Ding[\NeutI] \\
		\trans \wdef{du stellst große Anforderungen an mich}
			(%
				P:~19ra,10; vgl.~%
				\KC:~V.~1991;
				\cite[118]{schroeder1895}%
			)
	\end{xlist}
\end{exe}

\subsubsection{C1}
Für die Handschrift C1 lässt sich im Plural ein Unterschied zwischen
\norm{e}- und \norm{iu}-Formen feststellen \REF{ex:kcc1regel}. An allen diesen
Stellen im Paradigma kann die Endung allerdings auch fehlen. Darüber hinaus ist
\norm{-iu} auch im Akk.\ Sg.\ F.\ anzutreffen. Für den Plural ist \norm{-iu}
nur im Neutrum belegt, sodass sich die Form \norm{bėidiu} \wdef{beide} in der
nachfolgenden Untersuchung klar zuordnen lässt.

\begin{exe}
\ex \label{ex:kcc1regel}
	\begin{xlist}
	\ex \label{ex:kcc1regel_1}
		\gll Biterner warn guͤt chnechte. \\
			Viterber waren gut[\textsc{nom.pl.\MascM}] Knechte \\
		\trans \wdef{Die Viterber waren gute Krieger.}
			(%
				C1: 23vb,33; vgl.~%
				\KC:~V.~4383;
				\cite[161]{schroeder1895}%
			)

	\ex \label{ex:kcc1regel_2}
		\gll er begîe ſo grôzze vnmazzen. \\
			er beging so groß-\textsc{acc.pl.\FemI.st} Maßlosigkeiten \\
		\trans \wdef{er beging so große Maßlosigkeiten}
			(%
				C1: 7ra,18; vgl.~%
				\KC:~V.~1286;
				\cite[105]{schroeder1895}%
			)

	\ex \label{ex:kcc1regel_3}
		\gll gvͤtev werch er worcht. \\
			gut-\textsc{acc.pl.\NeutI.st} Werk er wirkte \\
		\trans \wdef{Gute Werke wirkte er.}
			(%
				C1: 68vb,2; vgl.~%
				\KC:~V.~13072;
				\cite[318]{schroeder1895}%
			)
	\end{xlist}
\end{exe}

\begin{sloppypar}
Da nur ein Beleg mit unflektierter Adjektivform im Plural Neutrum gegenüber
acht flektierten vorliegt \REF{ex:kcc1akkpln0_1}, kann \norm{bėid} für diese
Handschrift mit etwas Vorsicht dem Maskulinum-Femininum zugeordnet werden.
Soweit Parallel\-stellen\is{Paralleltext} zu finden waren, ist die Formulierung
mit unflektiertem \norm{grōȥ} \wdef{groß} den C-Handschriften gemein. Die A- und
B-Handschriften enthalten das Synonym \mbox{\norm{michel}} \wdef{groß} ebenfalls
unflektiert \REF{ex:kcc1akkpln0_2}, mit Ausnahme von VB, wo das Adjektiv
ausfällt.
\end{sloppypar}

\begin{exe}
\ex \label{ex:kcc1akkpln0}
	\begin{xlist}
	\ex \label{ex:kcc1akkpln0_1}
		\gll got hat groͤz wunder durch dich getan. \\
			Gott hat groß[\textsc{acc.pl.\NeutI}] Wunder durch dich getan \\
		\trans \wdef{Gott hat große Wunder durch dich getan.}
			(%
				C1:~72ra,38; vgl.~%
				K:~82rb,35;
				Z:~279ra,17;
				\KC:~V.~13778;
				\cite[330]{schroeder1895}%
			)

	\ex \label{ex:kcc1akkpln0_2}
		\gll got hat michel woͮnder durch dich getan. \\
			 got hat groß[\textsc{acc.pl.\NeutI}] Wunder durch dich getan \\
		\trans \wdef{Gott hat große Wunder durch dich getan.}
			(%
				A1:~59va,14; vgl.~%
				M:~105rb,21;
				H:~83vb,28;
				B1:~36avc,24;
				VB:~93vb,32;
				\KC:~V.~13778;
				\cite[330]{schroeder1895}%
			)
	\end{xlist}
\end{exe}

\subsubsection{K}
\il{Alemannisch|(}
Der Fall der recht späten alemannischen Handschrift K ist ähnlich gelagert wie
der von C1, da auch hier an allen relevanten Stellen des Paradigmas die
Flexionsendung fehlen kann. Die Belege in \REF{ex:kckregel} geben je ein
Beispiel für das Maskulinum und das Femininum. Der maskuline Beleg in
\REF{ex:kckregel_1} ist eine Parallelstelle\is{Paralleltext} zu dem in
\REF{ex:kcb1regel_1} präsentierten. Mit \lit{alte} \wdef{alte} liegt hier
eindeutig ein Adjektiv vor, kein Kompositum \norm{althērren} \wdef{Älteste}
\autocite[vgl.][s.\,v.~\textit{althêrre}]{mwb1}. Bei dem femininen Beleg in
\REF{ex:kckregel_2} fehlt eine overte Flexionsendung. Da jedoch das Auftreten
von \norm{-iu} in der Stichprobe zu dieser Handschrift eindeutig auf den
Nom.~Sg.~F.\ und den Nom./Akk.~Pl.~N.\ beschränkt ist, ist davon auszugehen,
dass für den Plural Femininum \norm{-e}/-Ø gilt.

\begin{exe}
\ex \label{ex:kckregel}
	\begin{xlist}
	\ex \label{ex:kckregel_1}
		\gll Si erwêlte fu̍nfhvndert \\
			Sie erwählte fünfhundert \\
	\sn \gll Die alte herren waͤren \\
			die alt-\textsc{nom.pl.\MascM.st} Herren waren \\
		\trans \wdef{Sie erwählte fünfhundert, die alte Herren waren}
			(%
				K:~51ra,37; vgl.~%
				\KC:~V.~8477;
				\cite[237]{schroeder1895}%
			)

	\ex \label{ex:kckregel_2}
		\gll Er begie ſo groz unmâzzen \\
			Er beging so groß[\textsc{acc.pl.\FemI}] Maßlosigkeiten \\
		\trans \wdef{er beging so große Maßlosigkeiten}
			(%
				K:~8rb,25; vgl.~%
				\KC:~V.~1286;
				\cite[105]{schroeder1895}%
			)
	\end{xlist}
\end{exe}

Neben dem Beleg in \REF{ex:kckregel_3} und einem weiteren für \norm{-iu} im
Plural Neutrum liegen zwei Belege ohne overte Flexion wie der in
\REF{ex:kckregel_4} vor. \norm{\textins{W}under} \wdef{Wunder} ist im Numerus
nicht eindeutig\is{Ambiguität} und der Akk.\ Sg.\ N.\ kann nominal-stark
endungslos\is{Endungslosigkeit} sein (vgl.~\tabref{tab:adjparadgmstr_nm}),
sodass nur der Vergleich mit Parallelstellen\is{Paralleltext} einen Hinweis
geben kann.

\begin{exe}
\ex	\begin{xlist}
	\ex \label{ex:kckregel_3}
		\gll Dín macht do {volle brachte} \\
			dein Macht da vollbrachte \\
	\sn \gll Groͤzu̍ wunder {maͤnig ualt} \\
			groß-\textsc{acc.pl.\NeutI.st} Wunder vielerlei \\
		\trans \wdef{Deine Macht vollbrachte dort vielerlei große Wunder.}
			(%
				K:~1ra,20; vgl.~%
				C1:~1ra,18--19%
				Z:~1ra,21--22%
			)

	\ex \label{ex:kckregel_4}
		% Zeilenumbruch durch "/" ersetzt
		\gll Der durch ſín lu̍te {/} \\
			der durch sein Leute \\
		\gll So groz wunder tu̍te \\
			so groß[\textsc{acc.pl.\NeutI}] Wunder täte \\
		\trans \wdef{der durch seine Leute so große Wunder wirkte}
			(%
				K:~62rb,21--22; vgl.~%
				C1:~54va,13--14;
				Z:~206va,19--20;
				\KC:~V.~10331--10332;
				\cite[271]{schroeder1895}%
			)
	\end{xlist}
\end{exe}

Hierbei ergibt sich, dass die Stelle in den angegebenen A- und B-Handschriften
dem Plural zuzuordnen ist, auch wenn dort stattdessen synonym \norm{ƶėichen}
\wdef{Wunder} steht. Exemplarisch wird der Text von A1 und B1 in
\REF{ex:groziuwunder} zitiert. Der Wortlaut der für die jeweilige Rezension
angegebenen Stellen in anderen Handschriften entspricht hinlänglich dem der
zitierten Stellen.

\begin{exe}
\ex \label{ex:groziuwunder}
	\begin{xlist}
	\ex \label{ex:groziuwunder_1}
		% Zeilenumbruch durch "/" ersetzt
		\gll der durch ſine lute. {/} \\
			der durch seine Leute \\
		\gll ſo groze zaichen tæte. \\
			so groß-\textsc{acc.pl.\NeutI.st} Wunder täte \\
		\trans \wdef{der durch seine Leute so große Wunder wirkte}
			(%
				A1:~45rb,10--11; vgl.~%
				H:~62rb,29--30;
				\KC:~V.~10331--10332;
				\cite[271]{schroeder1895}%
			)

	\ex \label{ex:groziuwunder_2}
		% Zeilenumbruch durch "/" ersetzt
		\gll Der durch ſeiner leut ræte {/} \\
			der durch seiner Leute Rat \\
		\gll So groͤziv zaichen tæt \\
			so groß-\textsc{acc.pl.\NeutI.st} Wunder täte \\
		\trans \wdef{der durch den Rat seiner Leute so große Wunder wirkte}
			(%
				B1:~28vb,17--18; vgl.~%
				VB:~49vb,17--18;
				\cite[271]{schroeder1895}%
			)
	\end{xlist}
\end{exe}
\il{Alemannisch|)}

\subsubsection{Z}
\il{Schwäbisch|(}
Wie auch bei H und P ist die Genusopposition im Plural bei Z getilgt, da
durchweg nur \norm{-e} und -Ø belegt sind \REF{ex:kczregel}. Für den Plural
Femininum liegen keine Belege vor. Da diese schwäbische\il{Schwäbisch}
Handschrift auf die Mitte des 15.~Jahrhunderts datiert wird
\autocite[32]{wolf:kckat}, ist davon auszugehen, dass der Grund dafür der
vollständige Vollzug der \isi{Nebensilbenabschwächung} ist. Diese setzt sich im
Alemannischen\il{Alemannisch} ab der ersten Hälfte des 14.~Jahrhunderts durch
\autocites(Schwäbisch wird nicht gesondert ausgewiesen)[vgl.][267, Abbildung
A~69]{ksw2}.

\begin{exe}
\ex \label{ex:kczregel}
	\begin{xlist}
	\ex \label{ex:kczregel_1}
		\gll Des iſt nu große zÿt \\
			des ist nun groß-\textsc{nom.sg.\FemI.st} Zeit[\FemI] \\
		\trans \wdef{Dafür ist es jetzt höchste Zeit.}
			(%
				Z:~233ra,17; zu
				\KC:~V.~11620;
				\cite[293]{schroeder1895}%
			)

	\ex \label{ex:kczregel_2}
		\gll Vnd ſöllen gute freunde ſin \\
			und werden gut-\textsc{nom.pl.st} Freunde[\MascA] sein \\
		\trans \wdef{und werden gute Freunde sein}
			(%
				Z:~61va,1; vgl.~%
				\KC:~V.~3089;
				\cite[137]{schroeder1895}%
			)

	\ex \label{ex:kczregel_3}
		\gll Ich gedenck an alte ding ferre \\
			ich denke an alt-\textsc{acc.pl.st} Ding[\NeutI] fern \\
		\trans \wdef{Ich denke an alte, ferne Dinge}
			(%
				Z:~135ra,9; vgl.~%
				\KC:~V.~6849;
				\cite[206]{schroeder1895}%
			)
	\end{xlist}
\end{exe}
\il{Schwäbisch|)}

\subsection{Zusammenfassung}

\tabref{tab:kcadjdeclovw} schematisiert die zuvor beschriebenen Ergebnisse der
Stichprobe zur Adjektivflexion und teilt die Handschriften in vier Gruppen ein:

\begin{labeling}{Gruppe 1:}
	\item[Gruppe 1:] Eine systematische Opposition im Plural ist vorhanden
	(A1, B1).

	\item[Gruppe 2:] Eine systematische Opposition ist einigermaßen deutlich,
	wird aber durch das Fehlen einer Flexionsendung mitunter verwischt
	(C1).

	\item[Gruppe 3:] Eine systematische Opposition ist wohl vorhanden, aber
	durch das Fehlen von Belegen, häufiges Fehlen einer Flexionsform oder das
	Nebeneinander von \norm{-e} und \norm{-iu} nicht zweifelsfrei feststellbar
	(K, M, VB).\is{Ambiguität}

	\item[Gruppe 4:] Es ist keine systematische Opposition vorhanden
	(H, P, Z).
\end{labeling}

\begin{table}
\centering
\caption{Belegte adjektivische Flexive in relevanten Paradigmen\-feldern in der
\tit{Kaiserchronik}}
\begin{tabular}{
	| c | c |
	  c r | c r |
	  c r | c r | c r |
	  c |
}
\hline

% ÜBERSCHRIFTEN %%%%%%%%%%%%%%%%%%%%%%%%%%%%%%%%%%%%%%%%%%%%%%%%%%%%%%%%%%%%%%%

\mr{2}{*}{Gruppe}
	& \mr{2}{*}{Hs.}
	& \mc{2}{ c|}{\mr{2}{*}{\textsc{nom.sg.f}}}
	& \mc{2}{ c|}{\mr{2}{*}{\textsc{acc.sg.f}}}
	& \mc{6}{ c|}{\textsc{nom+acc.pl}}
	& \mr{2}{*}{\norm{e : iu}}
	\\

\cline{7-12}

%
	& %
	& \mc{2}{ c|}{}
	& \mc{2}{ c|}{}
	& \mc{2}{ c|}{\textsc{m}}
	& \mc{2}{ c|}{\textsc{f}}
	& \mc{2}{ c|}{\textsc{n}}
	& \mc{1}{ c|}{}
	\\

\hline
\hline

% GRUPPE 1: A1, B1 %%%%%%%%%%%%%%%%%%%%%%%%%%%%%%%%%%%%%%%%%%%%%%%%%%%%%%%%%%%%

\mr{4}{*}{1}
	& \mr{2}{*}{A1}
	& -iu	& 2
	& -e	& 3
	& -e	& 2
	& -e	& 1
	& -iu	& 4
	& \mr{4}{*}{\chk}
	\\

%
	& %
	& -Ø	& 1
	& -Ø	& 1
	& 		& %
	& 		& %
	& -e	& 1
	& \mc{1}{ c|}{}
	\\

\cline{2-12} %%%%%%%%%%%%%%%%%%%%%%%%%%%%%%%%%%%%%%%%%%%%%%%%%%%%%%%%%%%%%%%%%%

%
	& \mr{2}{*}{B1}
	& -iu	& 5
	& -Ø	& 13
	& -Ø	& 1
	& -Ø	& 2
	& -iu	& 12
	& \mc{1}{ c|}{}
	\\

%
	& %
	& -Ø	& 2
	& 		& %
	& 		& %
	& 		& %
	& 		& %
	& \mc{1}{ c|}{}
	\\

\hline

% GRUPPE 2: C1 %%%%%%%%%%%%%%%%%%%%%%%%%%%%%%%%%%%%%%%%%%%%%%%%%%%%%%%%%%%%%%%%

\mr{2}{*}{2}
	& \mr{2}{*}{C1}
	& -iu	& 10
	& -Ø	& 8
	& -e	& 1
	& -e	& 1
	& -iu	& 8
	& \mr{2}{*}{(\chk)}
	\\

%
	& %
	& -Ø	& 3
	& -iu	& 5
	& -Ø	& 1
	& -Ø	& 1
	& -Ø	& 1
	& \mc{1}{ c|}{}
	\\

\hline

% GRUPPE 3: VB, M, K %%%%%%%%%%%%%%%%%%%%%%%%%%%%%%%%%%%%%%%%%%%%%%%%%%%%%%%%%%

\mr{8}{*}{3}
	& \mr{3}{*}{VB}
	& -iu	& 5
	& -e	& 11
	& -e	& 1
	& -e	& 2
	& -iu	& 5
	& \mr{8}{*}{?}
	\\

%
	& %
	& -e	& 3
	& -Ø	& 2
	& 		& %
	& -Ø	& 1
	& -e	& 1
	& \mc{1}{ c|}{}
	\\

%
	& %
	& -Ø	& 1
	& 		& %
	& 		& %
	& 		& %
	& 		& %
	& \mc{1}{ c|}{}
	\\

\cline{2-12} %%%%%%%%%%%%%%%%%%%%%%%%%%%%%%%%%%%%%%%%%%%%%%%%%%%%%%%%%%%%%%%%%%

%
	& \mr{3}{*}{M}
	& -iu	& 5
	& -Ø	& 7
	& -Ø	& 2
	& ?		& %
	& -iu	& 14
	& \mc{1}{ c|}{}
	\\

%
	& %
	& -Ø	& 1
	& -iu	& 2
	& 		& %
	& 		& %
	& 		& %
	& \mc{1}{ c|}{}
	\\

%
	& %
	& 		& %
	& -e	& 1
	& 		& %
	& 		& %
	& 		& %
	& \mc{1}{ c|}{}
	\\

\cline{2-12} %%%%%%%%%%%%%%%%%%%%%%%%%%%%%%%%%%%%%%%%%%%%%%%%%%%%%%%%%%%%%%%%%%

%
	& \mr{2}{*}{K}
	& -iu	& 8
	& -Ø	& 5
	& -e	& 2
	& -Ø	& 1
	& -Ø	& 2
	& \mc{1}{ c|}{}
	\\

%
	& %
	& -Ø	& 2
	& -e	& 3
	& -Ø	& 1
	& 		& 
	& -iu	& 2
	& \mc{1}{ c|}{}
	\\

\hline

% GRUPPE 4: H, P, Z %%%%%%%%%%%%%%%%%%%%%%%%%%%%%%%%%%%%%%%%%%%%%%%%%%%%%%%%%%%

\mr{5}{*}{4}
	& \mr{2}{*}{H}
	& -Ø	& 2
	& -e	& 5
	& -e	& 9
	& ?		& %
	& -e	& 5
	& \mr{5}{*}{\crs}
	\\

%
	& %
	&	 	& %
	& -Ø	& 3
	& 		& %
	& 		& %
	& -Ø	& 1
	& \mc{1}{ c|}{}
	\\

\cline{2-12} %%%%%%%%%%%%%%%%%%%%%%%%%%%%%%%%%%%%%%%%%%%%%%%%%%%%%%%%%%%%%%%%%%

%
	& \mr{1}{*}{P}
	& -e	& 2
	& -e	& 9
	& -e	& 2
	& ?		& %
	& -e	& 5
	& \mc{1}{ c|}{}
	\\

\cline{2-12} %%%%%%%%%%%%%%%%%%%%%%%%%%%%%%%%%%%%%%%%%%%%%%%%%%%%%%%%%%%%%%%%%%

%
	& \mr{2}{*}{Z}
	& -e	& 3
	& -e	& 4
	& -e	& 2
	& ?		& %
	& -e	& 2
	& \mc{1}{ c|}{}
	\\

%
	& %
	& 		& %
	& -Ø	& 1
	& -Ø	& 1
	& 		& %
	& 		& %
	& \mc{1}{ c|}{}
	\\

\hline
\end{tabular}
\label{tab:kcadjdeclovw}
\end{table}

Am relevantesten für die vorliegende Untersuchung sind die Gruppen~1 und~2. In
Gruppe~3 sollten sich Belege aus VB im Plural tendenziell zuordnen lassen.
Belege aus Handschriften der Gruppe~4 sind aufgrund der fehlenden Opposition
nicht von Belang. Nimmt man \tabref{tab:beidevar} hinzu, fallen bei der Analyse
des Quantors die Handschriften A1 und M zusätzlich aus dem Raster, da bei
diesem im Gegensatz zu regulären attributiven
Adjektiven\is{Adjektiv!attributiv} nur die \norm{e}-Form auftritt. Damit steht
A1 als ältester vollständiger und editionshistorisch\is{Editionsphilologie}
bedeutendster Textzeuge der \KC{} hier nicht zur Verfügung.

\is{Paradigma|)}
\is{Adjektivdeklination|)}
\is{Dialektgeografie|)}
