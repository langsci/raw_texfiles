\chapter{Auswertung zum \tit{Corpus der altdeutschen Originalurkunden}}
\label{ch:caoanalyse}

Im Folgenden werden zunächst die aus dem \tit{Corpus der altdeutschen
Originalurkunden} (\CAO) exzerpierten Belege zu \norm{bėide} \wdef{beide} näher
untersucht. Zunächst erfolgt eine kurze geografische Einordnung der Belege, um
sicherzustellen, dass der Großteil davon in das schreib\-dialektal relevante
Gebiet fällt. Durch die Kartierung wird außerdem deutlich, welche Regionen
gegebenenfalls über- oder unter\-repräsentiert sind. Danach erfolgt die
Untersuchung in Hinsicht auf das Auftreten der Wortformen \norm{bėide} und
\norm{bėidiu} als determinierender Quantor in Abhängigkeit von
Personenmerkmalen in unterschiedlichen syntaktischen Kontexten. Daran schließt
sich eine Untersuchung nach Distanz zwischen Controller und Target an. Am
Schluss des Kapitels steht der Vergleich zwischen \norm{bėide} als
Konjunk\-tion in der Konstruktion \norm{bėide \dots\ unde} \wdef{sowohl \dots\
als auch}.

\section{Geografische Verteilung der gesammelten Belege}
\is{Dialektgeografie|(}
\is{Distribution!geografische|(}

Die Karte in \figref{fig:beidemapcao} zeigt die Abdeckung des gesammelten
Belegmaterials in Bezug auf die Anzahl der Belege für \norm{bėide} pro
Ausstellungsort für Quantor und \isi{Konjunktion}; die Liste der ausgewerteten
Urkunden befindet sich im Anhang (\sectref{subsec:ausgewurk}). Der
alemannische\il{Alemannisch} Raum ist breit abgedeckt. Besonders Straßburg
(22~Belege), Zürich (15), Basel (15) und Freiburg i.\,Br.\ (12) stechen dabei
heraus. Im Schwäbischen\il{Schwäbisch} dominiert Augsburg (29); im
Bairischen\il{Bairisch} tut sich vor allem das Mittelbairische\il{Bairisch} mit
den großen Städten Wien (23) und Salzburg (14) hervor. Ebenfalls dem
Belegmaterial geschuldet ist die Tatsache, dass die Belegdichte nördlich des
Mains gegen null tendiert, abgesehen von Köln (25). Da davon auszugehen ist,
dass das untersuchte Phänomen ohnehin nur das Oberdeutsche\il{Oberdeutsch}
betrifft \autocite[181--184]{ksw2}, stellt die Verteilung der Belege kein
größeres Problem für die Auswertung dar.

\begin{figure}
\centering
\includegraphics[
	trim=0 10mm 75mm 0, clip,
	width=\linewidth,
	height=.75\textheight,
	keepaspectratio
]{./figures/cao_beide_iu-per-place-quant+conj.pdf}
\caption{Anzahl der Belege für mittelhochdeutsch \norm{bėide} pro
Ausstellungs\-ort im \tit{Corpus der altdeutschen Originalurkunden}
(Hintergrund nach \cite[Karte~47.4]{wiesinger1983})}
\label{fig:beidemapcao}
\end{figure}

Die 35 westmitteldeutschen\il{Westmitteldeutsch} Belege aus Köln (25~Belege),
Altleiningen (Kr.~Bad Dürkheim; 3), Sayn (Kr.~Mayen-Koblenz; 2), Worms (2),
Rüdigheim (Kr. Marburg-Biedenkopf; 1), Schloss Naumburg (Main-Kinzig-Kreis; 1)
und Veldenz (Kr.~Bern\-kastel-Wittlich; 1) zeigen den Angaben der Grammatik
gemäß\is{Validierung} keine Variation zwischen \norm{bėide}- und
\norm{bėidiu}-Formen. Diese Belege werden im weiteren Verlauf von der Analyse
ausgeschlossen; die betroffenen Urkunden werden im Anhang
(\sectref{subsec:ausgesurk}) aufgeführt. Bei der Besprechung individueller
Belege finden sich im Alemannischen\il{Alemannisch} noch weitere Fälle, in
denen am Ort nach kursorischer Durchsicht der Belege für \norm{e}- und
\norm{iu}-Formen in Urkundentexten des jeweiligen Ortes im \CAO{} keine
Variation vorliegt (vgl. auch \sectref{sec:adjdeclcao} zu Straßburg). Da
nachfolgend aufgrund der Anzahl der Belege nicht sämtliche von ihnen einzeln
besprochen werden können, ist mit weiteren solchen Fällen gewissermaßen als
\q{Grundrauschen} zu rechnen.

\is{Distribution!geografische|)}
\is{Dialektgeografie|)}

\section{Targets nach Personenmerkmalen des Controllers}
\label{sec:caotargpers}
\is{Target|(}
\is{Personenmerkmal|(}

\subsection{Nominale Controller}
\subsubsection{Kombinierte nominale Controller}
\label{subsubsec:perscombsgnp}

Zunächst werden \norm{bėide} und \norm{bėidiu} \wdef{beide} in direkter
Abhängigkeit von koordinierten\is{Koordination} nominalen Ausdrücken
(Substantive, Pronomina) wie zum Beispiel \norm{her Rvͦdiger vnd ſin
hovſfrowe} \wdef{Herr Rüdiger und seine Ehefrau} in \REF{ex:beid2coordncao1}
und dem Schema in \figref{fig:beid2coordncao1_1} näher betrachtet. Dabei werden
Kombinationen von \isi{Personalpronomen} und Eigenname wie \norm{ich unde
Ulrich} \wdef{ich und Ulrich} an dieser Stelle mitgezählt, da auch
Personalpronomina\is{Personalpronomen} eine nominale Kategorie darstellen,
insofern sie sich im Kontext dieser Untersuchung regelmäßig auf ein
substantivisches Antezedens beziehen und dessen Personenmerkmale reflektieren.

\begin{exe}
\ex \label{ex:beid2coordncao1}
		\gll swenne aber her Rvͦdiger vnd ſin
			hovſfrowe bediv niht enſint\\
			so=wenn aber Herr Rüdiger[\textsc{nom.sg.\MascM}] und sein
			Ehefrau[\textsc{nom.sg.\FemF}] beide-\textsc{nom.pl.\NeutMF.st} nicht
			\textsc{neg}=sind\\
			\trans \wdef{Wenn aber Herr Rüdiger und seine Ehefrau
				beide nicht \textins{mehr} sind}
				\parencites(Nr.~3262, Regensburg, 1299)[425,13--14]{cao4}
\end{exe}

\begin{figure}
\begin{tikzpicture}[baseline=(1a_lb.base)]
			\node at (0,2)  (1a)    {\norm{her Rvͦdiger}};
			\node           (1a_box)[draw,rectangle,fit=(1a)] {};
			\node           (1a_lb) [above=.5ex of 1a_box, mynodefont]
		                            {controller 1};

			\node at (0,0)  (1b)    {\norm{ſin hovſfrowe}};
			\node           (1b_box)[draw,rectangle,fit=(1b)] {};
			\node           (1b_lb) [above=.5ex of 1b_box, mynodefont]
		                            {controller 2};

			\node at (3,1) (2)      {\norm{bediv}};
			\draw (2) node (2_box)  [draw,rectangle,fit=(2)] {};
			\node (2_lb)   [above=.5ex of 2_box, mynodefont] {target};

			\draw [-latex] (1a_box) to [out=east, in=west] (2_box);
			\draw [-latex] (1b_box) to [out=east, in=west] (2_box);
		\end{tikzpicture}
\caption{Direkter Bezug von \norm{bėide} auf zwei Controller}
\label{fig:beid2coordncao1_1}
\end{figure}

Daneben liegen noch drei Belege wie der in \REF{ex:beid2coordncao2} vor, die
zunächst getrennt von der regulären Auswertung behandelt wurden, da bei ihnen
die Flexion des Targets im \isi{Hiatus}, also vor einem Wort steht, das mit
einem Vokal beginnt \autocites[vgl.][90--91]{askedal1973}[191--193,
201]{gjelsten1980}. Dies hat bei den Prosatexten\is{Prosa} des \CAO{} jedoch
keine große Auswirkung (vgl.~\sectref{phsec:caohiatus}), sodass diese Belege
aufgrund der geringen Anzahl zur Analyse hinzugezogen werden können, zumal sie
ansonsten einschlägig sind. In \tabref{tab:combnomctrl} erscheinen sie grau
gedruckt.

\begin{exe}
\ex \label{ex:beid2coordncao2}
	% \begin{xlist}
	% \ex \label{ex:beid2coordncao2_1}
	% 	\gll wir · krafte von hohenloch · vn̄ · wir ludewic
	% 		von durne · geloben baide vf vnſern eit \\
	% 		\textsc{1sg\subM.hon.nom} {} Kraft von Hohenlohe {} und {}
	% 		\textsc{1sg\subM.hon.nom} Ludwig von Durne {} geloben
	% 		beide-\textsc{nom.pl.\MascM.st} auf unseren Eid \\
	% 	\trans \wdef{Wir, Kraft von Hohenlohe, und Wir, Ludwig von Durne,
	% 		geloben beide Kraft unseres Eides}
	% 		\parencites(Nr.~2529, Burg Hohlach, Kr.~Neustadt an der Aisch-Bad Windsheim)[563,5--6]{cao3}

	% \ex \label{ex:beid2coordncao2_2}
		\gll Jch seibot von freihaim / vnd fraͮwe Peters mein
			Havſfraͮwe / veriehen beidev vnd tuͦn chvnt \\
			\textsc{1sg\subM.nom} Seibot von Freiheim {} und Frau
			Peters[\textsc{nom.sg.\FemF}] mein Ehefrau {} bekennen
			beide-\textsc{nom.pl.\NeutMF.st} und tun kund \\
		\trans \wdef{Ich, Seibot von Freiheim, und Frau Peters, meine Ehefrau,
			bekennen beide und machen bekannt}
			\parencites(Nr.~3248, München, 1299)[416,23]{cao4}

	% \ex \label{ex:beid2coordncao2_3}
	% 	\gll minen hof \textelp{} verkaufft han mit dem
	% 		zehenden \textelp{}, beidev vnuerſchaidenlichen
	% 		fvͤr reht aigen \\
	% 		meinen Hof[\textsc{acc.sg.\MascI}] {} verkauft habe mit dem
	% 		Zehnt-\textsc{dat.sg.\MascI} {} beide-\textsc{acc.pl.\NeutI.st}
	% 		gleichermaßen für rechtmäßig Eigentum \\
	% 	\trans \wdef{meinen Hof verkauft habe mit dem Zehnten \textelp{}, beide
	% 		gleichermaßen zum rechtmäßigen Eigentum}
	% 		\parencites(Nr.~N~241, Augsburg, 1283)[195,37--39]{cao5}
	% \end{xlist}
\end{exe}

\tabref{tab:combnomctrl} fasst die Personenmerkmale der in
\REF{ex:beid2coordncao1} und \REF{ex:beid2coordncao2} zitierten Beispiele
zusammen und gibt die Häufigkeiten\is{Frequenz} der jeweiligen Kongruenzformen
des Quantors pro Merkmalskombination an. Zur Angabe der belegten Flexionstypen
ist anzumerken, dass die Spalte \norm{bėidiu} gemäß den Ergebnissen in
\sectref{sec:adjdeclcao} zur regionalen Ausprägung von \norm{-e} und \norm{-iu}
in der starken \isi{Adjektivdeklination} im \CAO{} die verschiedenen als
äquivalent angesehenen Grafien der Flexionsendung zusammenfasst. Dies sind:
\lit{iu/iv},
\lit{îv},
\lit{ivͤ},
\lit{ıͤv},
\lit{uͥ/vͥ},
\lit{ú/v́},
\lit{v̓},
\lit{u/v},
\lit{eiw},
\lit{eu/ev}
sowie
\lit{i}.

\begin{table}
\captionsetup{width=.75\textwidth}
\caption{Flexion nach Personenmerkmalen der kombinierten nominalen Controller}
\begin{tabular}{>{\scshape}l >{\scshape}l r r}
\lsptoprule
\normalfont Controller 1
	& \normalfont Controller 2
	& \norm{bėide}
	& \norm{bėidiu}
	\\
\midrule
3sg.\MascM      & 3sg.\MascM       &        & 1        \\
3sg.\MascM      & 3sg.\FemF        &        & 1        \\
\midrule
\mc{2}{l}{Summe}                   &        & 2        \\
\midrule
\gr{1sg\subM}   & \gr{1sg\subM}    & \gr{1} &          \\
\gr{1sg\subM}   & \gr{3sg.\FemF}   &        & \gr{1}   \\
\gr{3sg.\MascI} & \gr{3sg.\MascI}  &        & \gr{1}   \\
\midrule
\mc{2}{l}{\gr{Summe}}                & \gr{1} & \gr{2}   \\
\lspbottomrule
\end{tabular}
\label{tab:combnomctrl}
\end{table}

Die niedrige Belegzahl in \tabref{tab:combnomctrl} erweckt den Eindruck, dass
die direkte Modifikation zweier Konjunkte durch \norm{bėide} an sich sehr
selten vorkommt. Unter den regulär gewerteten Belegen aus
\REF{ex:beid2coordncao1} liegt jeweils ein Fall mit gleichem Geschlecht der
Controller und einer mit gemischtem vor; in beiden Fällen steht eine Form vom
Typ \norm{bėidiu}. Nimmt man noch die grau gedruckten Belege entsprechend dem
aus \REF{ex:beid2coordncao2} hinzu, bei denen die Flexionsendung im
\isi{Hiatus} steht, kommt bei belebten\is{Animata} Controllern bei gleichem
Geschlecht ein Beleg für \norm{-e}, bei verschiedenem Geschlecht ein Beleg für
\norm{-iu} und bei unbelebten\is{Inanimata} Controllern mit gleichem
\isi{Genus} ebenfalls ein Beleg für \norm{-iu} hinzu.

Auch wenn die Belegzahl hier gering ist, fügen sich die vorhandenen Belege in
die Erkenntnisse der vorliegenden Untersuchung ein:
\citet[39--40]{behaghel1928} und \citet[118]{dal2014} stellen fest, dass
besonders im Zusammenhang mit zwei belebten\is{Animata} Nominalphrasen (NPs)
mit unterschiedlichem \isi{Genus} die morphologisch neutrale Kongruenzform mit
\norm{-iu} auftritt.

\phantomsection
\label{phsec:jungherren}
Daneben merkt \citet[384]{paul2007} an, dass die Möglichkeit besteht, dass auch
zwei Substantive\is{Substantiv} mit gleichem \isi{Genus} durch \norm{bėidiu}
aufgenommen werden können. Dies ist der Fall bei dem in
\REF{ex:beid2coordncao1_2} zitierten Beleg, der hier vorsichtig so
interpretiert wird, dass sich \lit{bedu} \wdef{beide} in pronominaler Verwendung
gleichzeitig auf Rüdiger und Ulrich bezieht. Die Urkunde Nr.~2915
\autocite[213,14--42]{cao4}, die den Beleg enthält, stammt aus dem
alemannischen\il{Alemannisch} Sprachraum\is{Dialektgeografie} (Kloster
St.~Urban, Kt.~Luzern; hochalemannisch\il{Alemannisch}). Mit
\norm{iu}-Markierung tritt neben dem in \REF{ex:beid2coordncao1_2} zitierten
Beispiel nur noch \lit{endru guͦt} \wdef{andere Güter (\textsc{nom.pl.\NeutI})}
\autocites(Nr.~2915)[213.27]{cao4} auf. Auch wenn \lit{bedu Jungherren}
\wdef{beide Junker} eine eigene Einheit mit Bezug auf zwei nicht namentlich
genannte Personen in der Zeugenliste bilden sollte, wäre die neutrale Form
\lit{bedu} in Abhängigkeit von maskulin-männlichem \lit{Jungherren} immer noch
auffällig.

\begin{exe}
\ex \label{ex:beid2coordncao1_2}
	\gll Diſ dingeſ gezuga ſint · Ruͦd · von der palma
			min Oͤhen · vͦlrich von Grvͤnenberch min Oͤhen bedu Jungherren
			\textelp{} \\
		Dies Verhandlung-\textsc{gen.sg.\NeutI} Zeugen sind {}
			Ruͦd[\textsc{nom.sg.\MascM}] {} von der Palme mein Oheim {}
			Ulrich[\textsc{nom.sg.\MascM}] von Grünenberg mein Oheim
			beide-\textsc{nom.pl.\NeutM.st}
			Jungherren[\textsc{nom.pl.\MascM}] {} \\
	\trans \wdef{Zeugen dieser Verhandlung sind: Ruͦd von der Palme,
		mein Onkel, Ulrich von Grünenberg, mein Onkel~-- beide Junker~--
		\textelp{}}
			\parencites(Nr.~2915, Kl.~St.~Urban, Kt.~Luzern, 1298)[213,33--35]{cao4}
\end{exe}

\subsubsection{Einfache nominale Plural-Controller}
\label{subsubsec:persplnp}

Als Vergleichsmaterial zu den Belegen in \sectref{subsubsec:perscombsgnp} mit
unmittelbarem Bezug auf koordinierte Controller wurden hier solche aufgenommen,
in denen sich \norm{bėide} direkt auf ein einzelnes Substantiv im Plural
bezieht. Für diesen Kontext gibt es weit mehr Belege, als dass jeder einzelne
zitiert werden könnte. Das Beispiel in \REF{ex:beid2snglncao} und das
dazugehörige Schema in \figref{fig:beid2snglncao} dienen zur Illustration.
\tabref{tab:simpnomctrl} bietet eine Übersicht der Verteilung der verschiedenen
Kongruenz\-formen von \norm{bėide} in Abhängigkeit von den Personenmerkmalen
des jeweiligen Controllers.

\begin{exe}
\ex \label{ex:beid2snglncao}
	\gll ſo mag er wol / die bede wingarten \textelp{}
			verkoͮfen \\
		so kann er wohl {} die beide-\textsc{acc.pl.\MascI.st}
			Weingarten-\textsc{acc.pl.\MascI} {} verkaufen \\
	\trans \wdef{so kann er wohl die beiden Weingärten \textelp{} verkaufen}
		\parencites(Nr.~1221, Zürich, 1290)[484,9]{cao2}
\end{exe}

\begin{figure}
\begin{tikzpicture}[baseline=(1_lb.base)]
	\node at (0,0) (1)    {\lit{wingarten}};
	\node          (1_box)[draw,rectangle,fit=(1)] {};
	\node          (1_lb) [above=.5ex of 1_box, mynodefont] {controller};
	\node at (3,0) (2)      {\lit{bede}};
	\draw (2) node (2_box)  [draw,rectangle,fit=(2)] {};
	\node (2_lb)   [above=.5ex of 2_box, mynodefont] {target};
	\draw [-latex] (1_box) to [out=east, in=west] (2_box);
\end{tikzpicture}
\captionsetup{width=.75\textwidth}
\caption{Direkter Bezug eines Targets auf einen einzelnen Controller}
\label{fig:beid2snglncao}
\end{figure}

\begin{table}
\captionsetup{width=.75\textwidth}
\caption{Flexion nach Personenmerkmalen der einfachen nominalen Controller}
\begin{tabular}{l r r r}
\lsptoprule
Controller
	& \norm{bėid(e)}
	& \norm{bėidiu}
	& Summe
	\\
\midrule
\MascM  & 21 &    & 21 \\
\midrule
\MascI  &  7 &    &  7 \\
\NeutI  &  1 &  5 &  6 \\
\midrule
Summe   & 29 &  5 & 34 \\
\lspbottomrule
\end{tabular}
\label{tab:simpnomctrl}
\end{table}

Der einzelne Beleg für \norm{bėide} in Bezug auf ein Neutrum wird in
\REF{ex:1584_gut} zitiert. Nach der Zuordnung im Ortsverzeichnis
\autocite{cao-online} stammt die Urkunde aus Herrenchiemsee. Der nächstgelegene
Vergleichsort in der Adjektivstichprobe ist München
(\sectref{par:adjmuenchen}). In anderen Urkunden aus Herrenchiemsee und
Umgebung (vornehmlich aus Kloster Altenhohenau, ebenfalls Kr.~Rosenheim) ist
\lit{-iu/eu} als Flexionsform für den Nom./Akk.\ Pl.\ N.\ zumindest bei
Determinierern\is{Determinierer} wie \norm{alliu} \wdef{alle (\textsc{pl.n})},
\norm{disiu} \wdef{diese (\textsc{pl.n})}, oder \norm{unseriu} \wdef{unsere
(\textsc{pl.n})} etabliert.

\begin{exe}
\protectedex{
\ex\label{ex:1584_gut}
	\gll ſo ſint baide gvt / dev wiſe / vn̄ freioltzmoſen / ledichleichen
			% / dez gotzhauſ datze phaffenwerd
			\\
		so sind beide-\textsc{nom.pl(.n\subI?).st}
			Gut[\textsc{nom.pl.\NeutI}] {} die Wiese {} und Freiolzmosen {}
			frei
			% {} des Gotteshaus da=zu Pfaffenwerde
			\\
	\trans \wdef{so stehen beide Güter, die Wiese und Freiolzmosen,
		% dem Gotteshaus in Pfaffenwerde
		\textelp{} zur freien Verfügung}
		\parencites(Nr.~1584, Kl.~Herrenchiemsee, Kr.~Rosenheim, 1292)[727,26--27]{cao2}
}
\end{exe}

\subsubsection{Zusammenfassung}

Insgesamt ist festzustellen, dass \norm{bėide} mit direktem Bezug auf nominale
Controller im ausgewerteten Urkundenmaterial sehr selten vorkommt. Es stellt
sich die Frage, was der Grund dafür ist. Allgemeingültige Aussagen lassen sich
in jedem Fall angesichts der prekären Beleglage keine machen. Die Belege passen
allerdings zur Beobachtung der einschlägigen Grammatikwerke, dass in
Abhängigkeit von zwei belebten\is{Animata} Controllern mit unterschiedlichem
Geschlecht regelmäßig die neutrale Form \norm{bėidiu} auftritt. Daneben tritt
\norm{bėidiu} einmal mit Bezug auf zwei Männer auf, was von
\citet[384]{paul2007} als Möglichkeit zwar angeführt, aber nicht weiter
thematisiert wird.

Bei der Auswertung von \norm{bėide} mit direktem Bezug auf ein
Plural-Substantiv als Controller kommen bei den belebten\is{Animata}
Controllern nur männliche Referenten vor; in allen Fällen steht \norm{bėide}.
Bei unbelebten\is{Inanimata} Controllern korrespondiert maskulines \isi{Genus}
regulär mit \norm{bėide}, beim Neutrum liegt neben regulärem \norm{bėidiu}
ebenfalls ein Beleg mit \norm{bėide} vor.

\subsection{Anaphorische Controller}
\label{subsec:refctrl}
\is{Anapher|(}

Im letzten Abschnitt zeigte sich, dass im \CAO{}-Belegmaterial
Substantive\is{Substantiv} sehr selten direkte Controller von \norm{bėide}
bilden. Der Großteil der direkten Controller von \norm{bėide} besteht aus
verschiedenen pronominalen Wortarten. Dies stimmt mit den Beobachtungen von
\citet[624--625]{ksw2} überein.

\subsubsection{Indirekter Bezug auf kombinierte nominale Controller}
\label{subsubsec:beid2p2coordncao}

Pronominale Controller wie \norm{wir} \wdef{wir} oder \norm{si} \wdef{sie}
beziehen sich bereits auf eine Kombination von Referenten mit ihren jeweiligen
Personenmerkmalen. Das Pronomen ist mit seinem Bezug über den referenziellen
\isi{Index} verbunden; \norm{bėide} selbst kongruiert mit dem Pronomen. Das
Beispiel in \REF{ex:beid2p2coordncao} verdeutlicht den hier untersuchten
syntaktischen Zusammenhang. Während für das \q{klassische} Mittelhochdeutsche
von einem Unterschied zwischen Maskulinum-Femininum \norm{sie} und Neutrum
\norm{siu} ausgegangen wird \autocites[vgl.][213--214]{paul2007}[369,
390--397]{ksw2}, fehlt sowohl im \CAO{} als auch in der \KC{} eindeutige
Evidenz für diesen Genusunterschied, siehe die Teiluntersuchungen zur Form des
Pronomens in \sectref{subsubsec:monoflexioncao}.

\begin{exe}
\ex\label{ex:beid2p2coordncao}
	\gll daz vlreich \textelp{} vor vns hat genomen ze einer houſfrowen vrowen
			Elzbeten \textelp{} vnd ſi beidev \textelp{} verzigen habent
			\textelp{} alles des rehtes \\
		dass Ulrich[\textsc{nom.sg.\MascM}] {} vor uns hat genommen zu einer
			Ehefrau Frau Elisabeth-\textsc{acc.sg.\FemF} {} und
			\textsc{3pl\subMF{}.nom} beide-\textsc{nom.pl.\NeutMF.st} {}
			verzichtet haben {} alles des Rechts \\
		\trans \wdef{dass Ulrich \textelp{} vor uns zur Ehefrau
			genommen hat Frau Elisabeth \textelp{} und sie beide \textelp{}
			verzichtet haben auf alles Anrecht}
				\parencites(Nr.~2843, Salzburg, 1297)[175,22--25]{cao4}
\end{exe}

In \tabref{tab:caosimprefctrl} werden in den Spalten unter \emph{Controller}
jeweils die grammatischen Merkmale des direkten Controllers von \norm{bėide}
angegeben: \textsc{1pl} entspricht \fw{wir/uns/unser}, \textsc{3pl} entspricht
\fw{sie/die/sich}. Die kombinierten Angaben danach stehen für die
Personen\-merkmale der beiden \term{Erstcontroller}, das heißt, der nominalen
Controller, auf die sich \norm{bėide} gemäß dem Schema in
\figref{fig:beid2p2coordncao} indirekt bezieht.

\begin{figure}
\begin{tikzpicture}[baseline=(1a_lb.base)]
	\node at (0,2)  (1a)    [gray]
                            {\lit{vlreich}};
	\node           (1a_box)[draw,gray,rectangle,fit=(1a)] {};
	\node           (1a_lb) [above=.5ex of 1a_box, gray, mynodefont]
                            {controller 1};

	\node at (0,0)  (1b)    [gray]
	                        {\lit{Elzbeten}};
	\node           (1b_box)[draw,gray,rectangle,fit=(1b)] {};
	\node           (1b_lb) [above=.5ex of 1b_box, gray, mynodefont]
                            {controller 2};    

	\node at (3,1) (2)      {\lit{ſi}};
	\draw (2) node (2_box1) [
	                    draw,
	                    gray,
	                    minimum height=3em,
	                    minimum width=3em,
	                    xshift=-.5ex,
	                    yshift=+.5ex,
	                    rectangle
	                ] {};
	\draw (2) node (2_box2) [
	                    draw,
	                    minimum height=3em,
	                    minimum width=3em,
	                    xshift=+.5ex,
	                    yshift=-.5ex,
	                    rectangle
	                ] {};
	\node           (2_lb1) [above=.5ex of 2_box1, gray, mynodefont]
	                        {target};
	\node           (2_lb2) [below=.5ex of 2_box2, mynodefont]
	                        {controller};

	\node at (6,1)  (3)      {\lit{beidev}};
	\node           (3_box)  [draw,rectangle,fit=(3)] {};
	\node           (3_lb)   [above=.5ex of 3_box, mynodefont]
	                        {target};

	\draw [-latex,gray] (1a_box) to [out=east, in=west] (2_box1);
	\draw [-latex,gray] (1b_box) to [out=east, in=west] (2_box1);
	\draw [latex-]      (3_box)  to [yshift=1.5ex]      (2_box2);
\end{tikzpicture}
\caption{Indirekter Bezug eines Targets auf zwei Erstcontroller über ein Personalpronomen}
\label{fig:beid2p2coordncao}
\end{figure}

\begin{table}
\centering
\caption{Flexion nach Personenmerkmalen der anaphorischen Controller
(kombinierter Bezug)}
\begin{tabular}{
>{\scshape}l
	>{\scshape}l @{$~+~$} >{\scshape}l
    r
    @{\hspace{4\tabcolsep}}
    r
    @{\hspace{4\tabcolsep}}
    r
}
\lsptoprule
\mc{3}{c}{Controller}
    & \norm{bėid(e)}
    & \norm{bėidiu}
    & Summe
    \\
\midrule
% Controller                  | e  | iu | Σ
1pl & 1sg\subM   & 1sg\subM   &  4 &    &   4 \\
    & 1sg\subM   & 3sg.\MascM &  1 &    &   1 \\

\cmidrule{2-6}

    & 1sg\subM   & 1sg\subF   &  1 &  2 &   3 \\
    & 1sg\subM   & 3sg.\FemF  &  4 & 12 &  16 \\
    & 1sg\subF   & 1sg\subM   &  2 &    &   2 \\
    & 1sg\subF   & 3sg.\MascM &  1 &  1 &   2 \\

\midrule

3pl & 3sg.\MascM & 3sg.\MascM &  6 &    &   6 \\
    & 3sg.\FemF  & 3sg.\FemF  &  5 &    &   5 \\

\cmidrule{2-6}

    & 3sg.\MascM & 3sg.\FemF  &  9 & 28 &  37 \\
    & 3sg.\FemF  & 3sg.\MascM &    &  4 &   4 \\

\cmidrule{2-6}

    & 3sg.\MascI & 3sg.\MascI &  2 &  4 &   6 \\
    & 3sg.\NeutI & 3sg.\NeutI &    & 23 &  23 \\

\cmidrule{2-6}

    & 3sg.\MascI & 3sg.\FemI  &    &  4 &   4 \\
    & 3sg.\MascI & 3sg.\NeutI &    &  1 &   1 \\
    & 3sg.\NeutI & 3sg.\MascI &    &  1 &   1 \\
    & 3sg.\NeutI & 3sg.\FemI  &    &  1 &   1 \\

\midrule

\mc{3}{l}{Summe}                 & 35 & 81 & 116 \\

\midrule

\gr{3pl}  & \gr{3sg.\MascM} & \gr{3sg.\MascM} & \gr{6} &        &  \gr{6} \\
          & \gr{3sg.\FemF}  & \gr{3sg.\FemF}  & \gr{2} &        &  \gr{2} \\

\cmidrule{2-6}

          & \gr{3sg.\MascM} & \gr{3sg.\FemF}  & \gr{2} & \gr{3} &  \gr{5} \\
          & \gr{3sg.\FemF}  & \gr{3sg.\MascM} &        & \gr{2} &  \gr{2} \\

\cmidrule{2-6}

          & \gr{3sg.\NeutI} & \gr{3sg.\NeutI} &        & \gr{1} &  \gr{1} \\

\cmidrule{2-6}

          & \gr{3sg.\NeutI} & \gr{3sg.\MascI} &        & \gr{1} &  \gr{1} \\
          & \gr{3sg.\NeutI} & \gr{3pl.\MascI} &        & \gr{1} &  \gr{1} \\

\midrule

\mc{3}{l}{\gr{Summe}}                         & \gr{10} & \gr{8} & \gr{18} \\

\lspbottomrule
\end{tabular}
\label{tab:caosimprefctrl}
\end{table}

\subsubsubsection{Belebt, gleiches Geschlecht}

Die Belege in \tabref{tab:caosimprefctrl} für \norm{bėide} mit direktem Bezug
auf Pronomina, die eine Kombination zweier Menschen vom gleichen Geschlecht
repräsentieren, verhalten sich ausgesprochen regelmäßig. In allen Fällen liegt
ein Beleg vom Typ \norm{bėide} vor, in einem Fall in der apokopierten Form
\norm{bėid}, siehe die Beispiele in \REF{ex:cao_samegend_beide}. Belege für
\norm{bėidiu} mit indirektem Bezug entweder auf zwei Männer oder auf zwei
Frauen sind nicht bezeugt, anders als bei der Beleggruppe mit direktem Bezug
(\sectref{subsubsec:perscombsgnp}).

\begin{exe}
\ex \label{ex:cao_samegend_beide}
\begin{xlist}
	\ex \label{ex:cao_samegend_beide_2}
		\gll Otten vnd albrehten von walhen \textelp{} wand
			ſiz ped ziehent an di lande gwizzen \\
			Otto-\textsc{acc.sg.\MascM} und Albrecht-\textsc{acc.sg.\MascM} von Walchen
			{} da \textsc{3pl\subM.nom}=es beide[\textsc{nom.pl.\MascM}] ziehen an die
			Länder gewiss \\
		\trans \wdef{Otto und Albrecht von Walchen \textelp{} da sie es beide
			gewiss an Land~(?) ziehen}
			\parencites(Nr. 491, Salzburg, 1281)[431,41 und 432,38]{cao1}

	\ex \label{ex:cao_samegend_beide_3}
		% \setlength{\glossglue}{0pt plus 1pt minus 0pt}
		\gll sweſter Elſbêth vn sweſter Mechthilt \textelp{} ſo
			ſiv beide tôt ſint \\
			Schwester Elisabeth[\textsc{nom.sg.\FemF}] und Schwester
			Mechthild[\textsc{nom.sg.\FemF}] {} so \textsc{3pl\subF.nom}
			beide-\textsc{nom.pl.\FemF.st} tot sind \\
		\trans \wdef{Schwester Elisabeth und Schwester Mechthild \textelp{}
			Wenn sie beide tot sind}
			\parencites(Nr.~1504, Zürich, 1291)[679,12--13]{cao2}
\end{xlist}
\end{exe}

Zu \REF{ex:cao_samegend_beide_3} ist anzumerken, dass der Beleg aus Zürich und
damit aus dem hochalemannischen\il{Alemannisch} Gebiet stammt. Elisabeth und
Mechthild werden für das Alemannische\il{Alemannisch} typisch mit \norm{siu}
\wdef{sie} bezeichnet. Diese Form ist nicht als Neutrum aufzufassen, sondern als
genusindifferent\is{Genusindifferenz} aufgrund der \blockcquote[395]{ksw2}{im
Alem.\ wirksamen Tendenz, die neutrale Form \norm{siu} des Nom./\,Akk.Pl.\
\textelp{} auch auf das Mask.\ und Fem.\ auszudehnen}. Trotzdem zeigt sich in
der Stichprobe zu Zürich (\sectref{par:adjzuerich}) ein Unterschied zwischen
\norm{e}- und \norm{iu}-Formen in der Deklination von
Adjektiven\is{Adjektivdeklination}, sodass davon auszugehen ist, dass es sich
hier um einen veritablen Beleg für maskulin-feminines \norm{bėide} in Bezug auf
zwei Frauen handelt.

\subsubsubsection{Belebt, verschiedenes Geschlecht}

Im Gegensatz zu den Belegen im vorigen Absatz liegt bei Targets, die indirekt
von Controllern mit unterschiedlichem \isi{Genus} beziehungsweise \isi{Sexus}
abhängen, Variation zwischen \norm{bėide} und \norm{bėidiu} vor. Die Beispiele
in \REF{ex:cao_diffgend_33_beide} stammen % \citet{cao-online} zufolge
beide aus Wien (\sectref{par:adjwien}).

\begin{exe}
\ex \label{ex:cao_diffgend_33_beide}
	\begin{xlist}
	\ex \label{ex:cao_diffgend_33_beide_1}
		\gll Her Ernſt
				% · vnſer burger /
				\textelp{}
				vnd ver Gerdroͤvt
				% ſein hovsvrowe /
				\textelp{}
				da ſi baide lebten \\
			Herr Ernst[\textsc{nom.sg.\MascM}]
				% {} unser Bürger {}
				{}
				und Frau Gertrud[\textsc{nom.sg.\FemF}]
				% sein Ehefrau {}
				{}
				als \textsc{3pl\subMF.nom} beide-\textsc{nom.pl.m+f\subMF.st}
				lebten \\
		\trans \wdef{Herr Ernst
				% , unser Bürger,
				\textelp{}
				und Frau Gertrud
				%, seine Ehefrau,
				\textelp{}
				als sie beide am Leben waren}
			\parencites(Nr.~1073, Wien, 1289)[374,40--41]{cao2}

	\ex \label{ex:cao_diffgend_33_beide_2}
		\gll Jch Heinrich der Swab vnd min houſvrowe Chvnegunt
				\textelp{} {dar vber} geb wir paideu diſen prief \\
			\textsc{1sg\subM.nom} Heinrich der Schwab und mein Ehefrau
				Kunigunde[\textsc{nom.sg.\FemF}] {} darüber geben
				\textsc{1pl\subMF.nom} beide-\textsc{nom.pl.n\subMF.st} diesen
				Urkunde \\
		\trans \wdef{Ich, Heinrich der Schwab, und meine Ehefrau, Kunigunde
			\textelp{} dazu geben wir beide diese Urkunde auf}
				\parencites(Nr.~N~475, Wien, 1291)[342,19 und 28]{cao5}
	\end{xlist}
\end{exe}

An der Verteilung in \tabref{tab:caosimprefctrl} ist zu erkennen, dass 80\,\%
der Belege mit belebter\is{Animata} Referenz auf die Kombination von Mann und
Frau als \isi{Erstcontroller} entfallen. Dies ist als Merkmal der Textsammlung
zu werten. Unabhängig von Person und Numerus stehen 17 Belegen für \norm{bėide}
47 Belege für \norm{bėidiu} gegenüber. Gerade beim indirekten Bezug auf eine
Kombination von dritten Personen unterschiedlichen Geschlechts
($\textsc{3sg.\MascM} + \textsc{3sg.\FemF}$, $\textsc{3sg.\FemF} +
\textsc{3sg.\MascM}$; vgl.~Beispiel \ref{ex:cao_diffgend_33_beide}) schlägt die
Belegzahl mit 32 zu 9 sehr stark zugunsten von \norm{bėidiu} aus.

Ähnlich verhalten sich die Belegzahlen für den indirekten Bezug auf die
Kombination von erster und dritter Person mit verschiedenem Geschlecht
($\textsc{1sg\subM} + \textsc{3sg.\FemF}$, $\textsc{1sg\subF} +
\textsc{3sg.\MascM}$). Auch in diesem Kontext lässt sich Variation zwischen
\norm{bėide} und \norm{bėidiu} beobachten, wobei das Verhältnis mit 13 zu 5
ebenfalls zugunsten der neutralen Form ausfällt. Das Beispiel für diesen
Kontext in \REF{ex:cao_diffgend_13_beide} stammt aus dem
niederalemannischen\il{Alemannisch} Dialektgebiet.

\begin{exe}
\ex \label{ex:cao_diffgend_13_beide}
	% (Zu viele Beispiele, als dass der Umbruch klappt :-( )
	%
	% \begin{xlist}
	% \ex \label{ex:cao_diffgend_13_beide_1}
		\gll Jch Goͮta \textelp{} do kunt \textelp{} daſ ich mit mineſ
			wirteſ hant \textelp{} Sunderliche heiſſe wir
			beide / den birſelere~\scalebox{.9}{\textelp{}} \\
			\textsc{1sg\subF.nom} Guta {} tue kund {} dass ich mit meines
			Ehemann[\textsc{gen.sg.\MascM}] Hand {} insbesondere heißen
			\textsc{1pl\subMF.nom} beide-\textsc{nom.pl.m+f\subMF.st} {} den
			Birseler~{} \\
		\trans \wdef{Ich, Guta, \textelp{} mache bekannt \textelp{}, dass ich
			mit der Hand meines Ehemannes \textelp{} Insbesondere heißen wir
			beide den Birseler \textelp{}}
				\parencites(Nr.~199, Basel, 1273)[210,21--28]{cao1}

	% \ex \label{ex:cao_diffgend_13_beide_2}
	% 	% \setlength{\glossglue}{0pt plus 2pt minus 1pt}
	% 	\gll Jch Rvͦdolf \textelp{} vnde Adelheit min wirtin / tvͤn
	% 			kvnt \textelp{} daſ wir beidú \textelp{} haben
	% 			g̍eg̍eben \textelp{} \\
	% 		\textsc{1sg\subM.nom} Rudolf {} und Adelheid[\textsc{nom.sg.\FemF}]
	% 			mein Ehefrau {} tun kund {} dass \textsc{1pl\subMF.nom}
	% 			beide-\textsc{nom.pl.\NeutMF.st} {} haben gegeben \\
	% 	\trans \wdef{Ich, Rudolf \textelp{}, und Adelheid, meine Ehefrau,
	% 		machen bekannt \textelp{}, dass wir beide \textelp{} gegeben haben}
	% 			\parencites(Nr.~1154, Guebwiller, Dépt.~Haut-Rhin, 1289)[432,5--11]{cao2}
	% \end{xlist}
\end{exe}

Beim indirekten Bezug auf die Kombination von zwei ersten Personen
unterschiedlichen Geschlechts ($\textsc{1sg\subM} + \textsc{1sg\subF}$,
$\textsc{1sg\subF} + \textsc{1sg\subM}$) sind die Werte mit drei zu zwei für
\norm{bėide} nahezu ausgeglichen, jedoch ist die Belegzahl zu gering, um dies
als eindeutige Tendenz zu werten.\is{Ambiguität} Zwei Belege mit \lit{bede}
\wdef{beide (\textsc{m+f}?)} und einer mit \norm{bedi} \wdef{beide
(\textsc{n}?)} stammen ferner aus Straßburg und sind mit Vorsicht zu werten
(\sectref{par:adjstrassburg}). Das Beispiel in \REF{ex:cao_diffgend_11_beide}
illustriert den Kontext.

\begin{exe}
\ex \label{ex:cao_diffgend_11_beide}
	% \begin{xlist}
	% \ex \label{ex:cao_diffgend_11_beide_1}
		\gll wîr dú vorgenanten herre Otto / vn fro Berchte
				\textelp{} wîr heîn och beîdú \textelp{}
				gelobet ſtete zehanne \\
			\textsc{1pl\subMF.nom} die vorgenannten Herr
				Otto[\textsc{nom.sg.\MascM}] {} und Frau
				Berta[\textsc{nom.sg.\FemF}] {} \textsc{1pl\subMF.nom} haben
				auch beide-\textsc{nom.pl.\NeutMF.st} {} versprochen beständig
				zu=haben \\
		\trans \wdef{Wir, die vorgenannten, Herr Otto und Frau Berta,
		\textelp{} Wir haben auch beide \textelp{} versprochen, beständig zu
			halten}
			\parencites(Nr.~2931, Basel, 1298)[223,1--6]{cao4}

	% \ex \label{ex:cao_diffgend_11_beide_2}
	% 	\gll Wir loben ovch paide, ich Herman von
	% 			Hipleinſtorf vnd ich Agnes ſein hauſvrowe \textelp{} \\
	% 		\textsc{1pl\subMF.nom} versprechen auch
	% 			beide-\textsc{nom.pl.m+f\subMF.st} \textsc{1sg\subM.nom}
	% 			Hermann von Hippersdorf und \textsc{1sg\subF.nom} Agnes sein
	% 			Ehefrau {} \\
	% 	\trans \wdef{Wir versprechen auch beide, ich, Hermann von
	% 		Hippersdorf, und ich, Agnes, seine Ehefrau, \textelp{}}
	% 		\parencites(Nr.~N~701, Wien, 1295)[506,32--33]{cao5}
	% \end{xlist}
\end{exe}

\phantomsection
\label{phsec:vbctrl}
In einigen Fällen wurden Nullsubjekte\is{Nullsubjekt} als Controller
aufgenommen, da im Teilsatz kein overtes Subjektpronomen vorhanden ist.
\norm{Bėide} steht dennoch an derselben Stelle, die es in der
\isi{Distanzstellung} als gefloateter Quantor\is{gefloateter Quantor}
(\sectref{sec:floatquant}) einnimmt. Es wird angenommen, dass \norm{bėide} mit
dem Nullsubjekt kongruiert \autocites[siehe
auch][419]{dalrymple2001}[210]{bresnanetal2016}. Ein Beispiel wird in
\REF{ex:vvfinctrl} gegeben.

\begin{exe}
\ex \label{ex:vvfinctrl}
	% \begin{xlist}
	% (Zu viele Beispiele für den Umbruch :-( )
	%
	% \ex \label{ex:vvfinctrl_1}
	% 	\gll Einen garten vnde einen aker {} ligent beidú bi \textelp{} \\	
	% 		einen Garten-\textsc{acc.sg.\MascI} und einen
	% 			Acker[\textsc{acc.sg.\MascI}] \textsc{rel\subI} liegen
	% 			beide-\textsc{nom.pl.\NeutI.st} bei {} \\
	% 	\trans \wdef{einen Garten und einen Acker, \textins{die} beide bei
	% 		\textelp{} liegen}
	% 		\parencites(Nr.~3249, Freiburg i.\,Br., 1299)[417,4--5]{cao4}

	% \ex \label{ex:vvfinctrl_2}
	% 	\gll wir leben \textelp{} vn̄ {} ſuln daſ ſelbe guͦt beidú nieſſen \\
	% 		\textsc{1pl\subMF.nom} leben {} und Ø\subMF{} sollen das selbe Gut
	% 			beide-\textsc{nom.pl.\NeutMF.st} nutzen \\
	% 	\trans \wdef{wir leben \textelp{} und \textins{wir} sollen
	% 		dasselbe Gut beide nutzen}
	% 		\parencites(Nr.~3376, Neuenburg am Rhein, Kr.~Breisgau-Hochschwarzwald, 1299)[493,21--22]{cao4}

	% \ex \label{ex:vvfinctrl_3}
		\gll Jch Berhtholt vn̄ Liebeſte \textelp{} vn̄ {} hant das offenliche
				bediv veriehen miteinander \\
			\textsc{1sg\subM.nom} Berthold und Liebste[\textsc{nom.sg.\FemF}]
				{} und Ø\subMF{} haben das öffentlich
				beide-\textsc{nom.pl.\NeutMF.st} bezeugt miteinander \\
		\trans \wdef{Ich, Berthold, und Liebste \textelp{} und \textins{wir}
			haben das öffentlich beide bezeugt miteinander}
			\parencites(Nr.~N~150, Kl.~Niedermünster, Dépt.~Bas-Rhin, 1277)[108,31--32]{cao5}

	% \ex \label{ex:vvfinctrl_4}
	% 	\gll daſ ſie gewert ſint \textelp{} vn̄ {} hant bedi veriehen \\
	% 		dass \textsc{3pl\subMF.nom} gewährt sind {} und Ø\subMF{} haben
	% 			beide-\textsc{nom.pl.\NeutMF.st} bezeugt \\
	% 	\trans \wdef{dass sie bezahlt wurden \textelp{} und \textins{sie} haben
	% 		beide bezeugt}
	% 		\parencites(Nr.~N~202, Straßburg, 1281)[156,16]{cao5}
	% \end{xlist}
\end{exe}

% Daneben stehen die zwei Belege in \REF{ex:vvfinctrl2}, die zunächst aus
% formalen Gründen getrennt von den regulären Belegen gezählt wurden, dennoch
% das Phänomen illustrieren. Auch hier steht \norm{bėide} vor einem Wort, das
% mit Vokal beginnt.

% \begin{exe}
% \ex \label{ex:vvfinctrl2}
% 	\begin{xlist}
% 	\ex \label{ex:vvfinctrl2_1}
% 		\gll Wir Otto von Ohſſinſtein / vn̄ Otte der Lantvogt \textelp{} vn̄ {}
% 				henkent {dar vmbe} bêide vnſer Jngeſigel an diſen selben
% 				brief \\
% 			\textsc{3pl\subM.nom} Otto von Ochsenstein {} und Otto der Landvogt
% 				{} und Ø\subM{} hängen darum beide-\textsc{nom.pl.\MascM.st}
% 				unser Siegel an diesen selben Urkunde \\
% 		\trans \wdef{Wir, Otto von Ochsenstein und Otto der Landvogt \textelp{}
% 			und \textins{wir} hängen darum beide unser Siegel an diese Urkunde}
% 			\parencites(Nr.~1145, Burg Ochsenstein, Dépt.~Bas-Rhin, 1289)[427,5--6]{cao2}

% 	\ex \label{ex:vvfinctrl2_2}
% 		\gll B˜chart vn̄ Cvͦnrat von lv̓ſtenowe \textelp{} vn̄ {} hant dc beide
% 				/ ain dem adern vor mir {vf gegeben} \\
% 			Burkhard[\textsc{nom.sg.\MascM}] und Konrad[\textsc{nom.sg.\MascM}]
% 				von Lustnau {} und Ø\subM{} haben das
% 				beide-\textsc{nom.pl.\MascM.st} {} ein dem andern vor mir
% 				aufgegeben \\
% 		\trans \wdef{Burkhard und Konrad von Lustnau \textelp{} und
% 			\textins{sie} haben das beide, einer dem anderen, vor mir
% 			aufgegeben}
% 			\parencites(Nr.~2607, Tübingen, 1297)[32,41--33,1]{cao4}
% 	\end{xlist}
% \end{exe}

\phantomsection
\label{phsec:beidegen}
Während nach \citet[623]{ksw2} regulär nur im Nom./Akk.~Pl.\ mit \norm{bėide}
oder \norm{bėidiu} zu rechnen ist, liegt darüber hinaus ein Beleg vor, in dem
\norm{bėide} ausnahmsweise im Genitiv auftritt \REF{ex:1843_kinde}. In Urkunde
Nr.~1843 \autocite[146,11--47]{cao3} bezeugt Ulrich von Wichtrach, das
beurkundete Rechtsgeschäft in Einvernehmen mit seiner Ehefrau Clementine und
ihrer beiden Kinder abgewickelt zu haben.

\begin{exe}
\ex\label{ex:1843_kinde}
	% https://www.query.sta.be.ch/detail.aspx?ID=62091 (08.01.24)
	% Bern, Staatsarchiv des Kantons Bern, Abt. C I a, Stift 1293.11.30
	\gll Jch Vͦlrich \textelp{} kvnde \textelp{} daz ich mit guͦtem rathe~/
	mit miner wirtin \textelp{} vnd vnſer beide kinde hant
			vnd willen \\
		\textsc{1sg\subM.nom} Ulrich {} verkünde {} dass \textsc{1sg\subM.nom}
			mit gutem Rat-\textsc{dat.sg}~{} mit
			\textsc{1sg\subM.gen-dat.sg.\FemF.st}
			Ehefrau[\textsc{dat.sg.\FemF}] {} und \textsc{1pl\subMF.gen}
			beide-\textsc{gen.pl.?} Kind-\textsc{gen.pl.\NeutA} Hand und
			Willen \\
	\trans \wdef{Ich, Ulrich, \textelp{} verkünde \textelp{}, dass ich mit
		gutem Rat \textins{sowie} mit Hand und Willen meiner Ehefrau \textelp{}
		und unser beiden Kinder}
		\parencites(Nr.~1843, Thun, Kt.~Bern, 1293)[146,11--13]{cao3}
\end{exe}

Aus dem Text der Urkunde gehen die Namen der Kinder nicht hervor. Die Form
\lit{beide} könnte sich daher entweder auf \lit{vnſer} \wdef{unser} oder auf
\lit{kinde} \wdef{Kinder} beziehen. In beiden Fällen steht
unregelmäßig\is{Ausnahme} \norm{bėide} im Gen.\ Pl. Die Korrektheit der
\isi{Transkription} ließ sich anhand eines Fotos\is{Digitalisat} der
Original\-urkunde bestätigen (\figref{fig:1843}). Ein \norm{er}-Haken oder ein
Nasalstrich, der die Flexion zu \norm{bėider} beziehungsweise \norm{bėiden}
ändern würde, sind auch im Original nicht vorhanden. Regelmäßiger
\fw{n}-Schwund am Wortende \autocite[vgl.][171--172]{weinhold1863} lässt sich
in dieser Urkunde nicht beobachten.

\begin{figure}[h]
\centering
\includegraphics[
	width=\linewidth,
]{./figures/CAO_01843_ausschnitt.jpg}
\caption%
	{Ausschnitt aus Bern, Staatsarchiv, StABE~C~I~a, Stift 1293.11.30
		\autocites(Nr. 1843; Foto: Staatsarchiv Bern)[146,12--13]{cao3}}
\label{fig:1843}
\end{figure}

Der einzige weitere Beleg in einem ähnlichen syntaktischen Kontext wird in
\REF{ex:682_insigel} zitiert. Hier urkunden Abt Volland und Prior Berthold
gemeinsam mit der Klostergemeinschaft. Die Form \lit{bediu} lässt auf Kongruenz
mit \lit{inſigel} \wdef{Siegel} schließen.

\begin{exe}
\protectedex{
\ex\label{ex:682_insigel}
	% https://www.wubonline.de/?wub=4267 (08.01.24)
	% Stuttgart, Hauptstaatsarchiv, A 514 U 50
	\gll % Tougen wir abt vollant / vnd Bertholt der prior / vnd der Conuent von
			% Hirſowe kunt \textelp{}
			{dar umbe} haben % \footnotemark{}
			wir an dizen breif vnſer bediu inſigel
			% / vnd Grauen Aberetheſ von · hohenberc
			\\
		% tun wir Abt Volland[\textsc{nom.sg.\MascM}] {} und
		% 	Berthold[\textsc{nom.sg.\MascM}] der Prior {} und der
		% 	Konvent[\textsc{nom.sg.\MascM}] von Hirsau kund {}
			Darum heben
			\textsc{1pl\subM.nom} an diesen Urkunde \textsc{1pl\subM.gen}
			beide-\textsc{acc.pl.\NeutI.st} Siegel[\textsc{acc.pl.\NeutI}]
			% {} und Graf-\textsc{gen.sg} Albrecht-\textsc{gen} von {} Hohenberg
			\\
		\trans \wdef{% machen wir, Abt Volland und Berthold der Prior und der
			% Konvent von Hirsau bekannt \textelp{}
			darum hängen wir an diese Urkunde unsere beiden Siegel%
			% und Graf Albrechts von Hohenberg
			}
			\parencites(Nr.~682, Kl.~Hirsau, Kr.~Calw, 1284)
			%[96,3+11--12]{cao2}
			[96,11]{cao2}
		%
		% \footnotetext{Das \citet[780]{wmu1} vermerkt eine
		% 	\textquote{\textins*{g}elegentl.\ Vermengung der Formen von
		% 	\emph{haben} und \emph{heben}}, die auch an dieser Stelle
		% 	anzunehmen ist.}
}
\end{exe}

Im \tit{Referenzkorpus Mittelhochdeutsch} (\REM; \nosh\cite{rem}) finden sich
zum Vergleich noch zwei Belege mit \norm{bėide} aus dem
mitteldeutschen\il{Mitteldeutsch} Sprachraum\is{Dialektgeografie}
\REF{ex:remgenbeide}, der in dieser Studie ansonsten ausgeklammert wurde.
Beispiele für \norm{bėidiu} im Gen.\ Pl.\ konnten nicht gefunden werden, das
heißt, \lit{-í} in \lit{beidí} ist in \REF{ex:remgenbeide_1} nicht als Variante
von \norm{-iu} zu werten, sondern als ostmitteldeutsche\il{Ostmitteldeutsch}
Schreibweise für \norm{-e} \autocites[52--53]{paul2007}[305]{ksw2}.

\begin{exe}
\ex \label{ex:remgenbeide}
\begin{xlist}
	\ex \label{ex:remgenbeide_1}
		\gll % von den luitín
			die vrí beidí gívoren ſien \\
			% von den Leuten
			die ihr-\textsc{gen.pl} beide-\textsc{gen.pl}
				Nachbarn-\textsc{nom.pl.\MascA}\footnotemark{} sind \\
		\trans \wdef{%
			%von den Leuten,
			die ihr beider Nachbarn sind} (%
				\tit{Mühlhäuser Rechtsbuch};
				Nordhausen, Stadtarchiv, Ms. II, Na 6;
				% \cite[1379]{hsc};
				\cite[M320: 17v,21--22]{rem}%
			)
		%
		\footnotetext{Mhd.~\norm{gevurche} \wdef{Furchgenosse, Nachbar}
			\autocite[vgl.][s.\,v.~\fw{gevurche}]{mwb2}. Danke an Jonas Richter
			(Göttingen) für diesen Hinweis.}

	\ex \label{ex:remgenbeide_2}
		\gll van dode der kindˢe beide of irre eyn \\
			von Tod-\textsc{dat.sg} der Kind-\textsc{gen.pl}
				beide-\textsc{gen.pl.st} oder ihrer ein \\
		\trans \wdef{durch den Tod beider Kinder oder eines von ihnen}
			(%
				Köln, Hist.~Archiv der Stadt, Best.~210
				\mkbibparens{Domstift}, U~3/759;
				\cite[M350: 5,11]{rem}%
			% Digi: https://historischesarchivkoeln.de/document/Vz_E9493491-3A61-4625-8DB0-09465C24BCC8 (08.01.24)
			)
\end{xlist}
\end{exe}

\subsubsubsection{Unbelebt, gleiches Genus}
\is{Genus|(}
\is{Inanimata|(}

Bei Belegen für \norm{bėide} mit indirektem Bezug auf unbelebte kombinierte
\isi{Erstcontroller} mit gleichem Genus ergibt sich in
\tabref{tab:caosimprefctrl} ein Spiegelbild zum belebten Gegenstück. Neben 23
Fällen von \norm{bėidiu} mit neutralem Bezug ist die neutrale Form auch in vier
von sechs Fällen mit kombiniertem maskulinen Bezug zu finden, wie in
\REF{ex:cao_samegend_inan_mm_beidiu}. Belege für die Kombination zweier
unbelebter Feminina sind im ausgewerteten Material keine vorhanden.

\begin{exe}
\ex \label{ex:cao_samegend_inan_mm_beidiu}
	% \begin{xlist}
	% \ex \label{ex:cao_samegend_inan_mm_beidiu_1}
		\gll vnſerne zehenden zeandeluingen vnde ainen Garten \textelp{} div
				wier baidiv fvr reht aigen her haigen~\textins{sic} braht \\
			unseren Zehnt-\textsc{acc.sg}.\MascI{} zu=Andelfingen und einen
				Garten-\textsc{acc.sg.\MascI} {} \textsc{rel.acc.pl.\NeutI} wir
				beide-\textsc{acc.pl.\NeutI.st} für rechtmäßig Eigentum her
				haben gebracht \\
		\trans \wdef{unseren Zehnten zu Andelfingen und einen Garten \textelp{},
			die wir beide als recht\-mäßiges Eigentum hergebracht haben}
			\parencites(Nrn.~1201~AB, Kl.~Heiligkreuztal, Kr.~Biberach,
				1290)[472,10--18]{cao2}

	% (Zu viele Belege für den Umbruch :-( )
	%
	% \ex \label{ex:cao_samegend_inan_mm_beidiu_2}
	% 	\gll Einen garten vnde einen aker {}
	% 			ligent beidú bi \textelp{} Mit allem dem rehte alſ er
	% 			{-- --} ſú bedú von mir hatte \\
	% 		%
	% 		einen Garten-\textsc{acc.sg.\MascI} und einen
	% 			Acker[\textsc{acc.sg.\MascI}] \textsc{rel\subI} liegen
	% 			beide-\textsc{nom.pl.\NeutI.st} bei {} mit allem dem Recht als
	% 			er {} \textsc{3pl\subI.acc} beide-\textsc{acc.pl.\NeutI.st} von
	% 			mir hatte \\
	% 	\trans \wdef{einen Garten und einen Acker, \textins{die} liegen beide
	% 		bei \textelp{}, mit all dem Recht, wie er sie beide von mir hatte}
	% 		\parencites(Nr.~3249, Freiburg i.\,Br., 1299)[417,4--6]{cao4}%
	% \end{xlist}
\end{exe}

Der Beleg in \REF{ex:cao_samegend_inan_mm_beidiu}~-- bis auf kleine
Unterschiede in der Großschreibung sind beide Fassungen identisch~-- scheint
auf den ersten Blick ambig\is{Ambiguität} bezüglich des Antezedens von
\lit{baidiv} zu sein. Allerdings kann die \isi{Lesart} mit \lit{wier baidiv}
\wdef{wir beide} als Einheit im Kontext der Urkunde ausgeschlossen werden, da
drei Aussteller genannt werden: die Brüder \lit{wezel vn̄ hainrich wezel vnde
Cvͦnrat der Bodemer} \autocites(Nrn.~1201~AB)[472,7]{cao2}.

Während sich im Material zum \CAO{} keine kombinierten unbelebten Feminina
finden ließen, ergab eine kurze Recherche im \REM{} zwei Belege, von denen
einer exemplarisch in \REF{ex:beid2p2combrem} zitiert wird. Beide Stellen
enthalten die neutrale Form \norm{bėidiu} mit Bezug auf kombinierte unbelebte
Feminina, allerdings stammt der zitierte aus einem gereimten\is{Vers} Text.%
%
	\footnote{Bei der anderen Stellen handelt es sich um den
		\tit{Baumgarten geistlicher Herzen} in
		% München, Bayerische Staatsbibl.,
		Cgm~6247: 108v,19--21 (\cite[vgl.][M401]{rem}; dazu
		% München, Bayerische Staatsbibl.,
		Cgm 183: 4v,5, \cite[vgl.][M405Y]{rem}).}

\begin{exe}
\ex \label{ex:beid2p2combrem}
	% \begin{xlist}
	% \ex \label{ex:beid2p2combrem_1}

	% (Zu viele Beispiele für den Umbruch :-( )
	%
	% \ex \label{ex:beid2p2combrem_2}
		\gll daz triwe vn̄ ere werde. \\
			dass Treue[\textsc{nom.sg.\FemI}] und Ruhm[\textsc{nom.sg.\FemI}] werde \\
	\sn \gll begraben in die erde. \\
			begraben in die Erde \\
	\sn \gll ſo ligent ſi beidiv hie begraben. \\
			so liegen \textsc{3pl\subI.nom} beide-\textsc{nom.pl.\NeutI.st} hier
			begraben \\
		\trans \wdef{dass Treue und Ruhm werde begraben in der Erde. So liegen sie
			beide hier begraben.}
			(%
				\iai{Gottfried von Straßburg}, \tit{Tristan}:
				V.~18661--18663 nach
				München, Bayerische Staatsbibl., Cgm~51: 96rb,38--39;
				% \cite[1286]{hsc}
				vgl.~\cites[259]{maroldschroeder1969}[M342]{rem}%
			)%
	
	% \ex \label{ex:beid2p2combrem_3}
	% 	\gll der iſt genant ſineſ vater tugent vn̄ wiſheit {da von} ſvln ſínív
	% 			ſchulcheit\upshape\footnotemark{} diſiv beidiv von ím
	% 			lernen. \\
	% 		der ist genannt seines Vater Tugend[\textsc{nom.sg.\FemI}] und
	% 			Weisheit[\textsc{nom.sg.\FemI}] davon sollen seine Schulkind
	% 			diese-\textsc{acc.pl.\NeutI.st} beide-\textsc{acc.pl.\NeutI.st}
	% 			von ihm lernen \\
	% 	\trans \wdef{der wird genannt Tugend und Weisheit seines Vaters.
	% 		Deshalb sollen seine Schulkinder diese beide von ihm lernen.}
	% 		(%
	% 			\tit{Baumgarten geistlicher Herzen} nach
	% 			München, Bayerische Staatsbibl., Cgm~6247: 108v,19--21;
	% 			% \cite[1450]{hsc};
	% 			vgl.~\cite[M401]{rem}%
	% 		)
	% \end{xlist}
\end{exe}
% %
% 	\footnotetext{Die Schreibung wurde am Digitalisat der Handschrift
% %	(\href{https://mdz-nbn-resolving.de/urn:nbn:de:bvb:12-bsb00104263-7}{urn:nbn:de:bvb:12-bsb00104263-7}; 08.01.24)
% 	verifiziert, die Glossierung aus dem \REM{} übernommen. Dieselbe
% 	Textstelle in M405Y (München, Bayerische Staatsbibl., Cgm 183: 4v,5%
% 	% ; \cite[9715]{hsc};
% % 	\href{https://mdz-nbn-resolving.de/urn:nbn:de:bvb:12-bsb00006141-6}{urn:nbn:de:bvb:12-bsb00006141-6}; 08.01.24
% 	) enthält dafür \lit{ſchvͦl chint} \wdef{Schulkinder} und ebenso \lit{diſiv
% 	beidiv} \wdef{diese beide (\textsc{pl.n\subI})}.}

\subsubsubsection{Unbelebt, verschiedenes Genus}

Die Belege in \tabref{tab:caosimprefctrl} für \norm{bėide}-Targets in
Abhängigkeit von unbelebten Erstcontrollern\is{Erstcontroller} mit
unterschiedlichem Genus verhalten sich noch regelmäßiger als ihr belebtes
Gegenstück. Das Beispiel in \REF{ex:cao_diffgend_inan} illustriert diesen
Kongruenzkontext.

\begin{exe}
\ex \label{ex:cao_diffgend_inan}
	% \begin{xlist}
	% \ex \label{ex:cao_diffgend_inan_1}
		\gll daz ich auz minem hauz vnd auz miner hofſtat
				div bediv min recht eigen ſint \textelp{} \\
			dass ich aus meinem Haus[\textsc{dat.sg.\NeutI}] und aus meiner
				Grundstück[\textsc{dat.sg.\FemI}] \textsc{rel.nom.pl.\NeutI}
				beide-\textsc{nom.pl.\NeutI.st} mein rechtmäßig Eigentum sind \\
		\trans \wdef{dass ich aus meinem Haus und aus meinem Grundstück, die 
			beide mein rechtmäßiges Eigentum sind \textelp{}}
			\parencites(Nr.~1282, Regensburg, 1290)[526,37--38]{cao2}

	% \ex \label{ex:cao_diffgend_inan_2}
	% 	\gll an dem hofe da ce Holtzhvſen \textelp{} vnde an der
	% 			holtzmark die ich da han, div baidiv min
	% 			reht aigen waren \\
	% 		an dem Hof-\textsc{dat.sg.\MascI} da zu Holzhausen {} und an der
	% 			Waldstück[\textsc{dat.sg.\FemI}] \textsc{rel.nom.sg.\FemI} ich
	% 			da habe \textsc{rel.nom.pl.\NeutI}
	% 			beide-\textsc{nom.pl.\NeutI.st} mein rechtmäßig Eigentum
	% 			waren \\
	% 	\trans \wdef{an dem Hof in Holzhausen \textelp{} und an dem Waldstück,
	% 		das ich da habe, die beide mein rechtmäßiges Eigentum waren}
	% 		\parencites(Nr.~N~272, Augsburg, 1285)[215,30--31]{cao5}
	% \end{xlist}
\end{exe}

In allen sieben Fällen steht eine Form vom Typ \norm{bėidiu}. Sieben Belege
sind nicht genug, um daraus eine Regel abzuleiten, aber die vorhandenen Belege
fügen sich mit denen zum belebten Kontext zu einem Bild zusammen. Insgesamt ist
auffällig, dass nahezu alle Targets mit unbelebter kombinierter Referenz die
neutrale Kongruenzform \norm{bėidiu} aufweisen, unabhängig vom Genus ihrer
\isi{Erstcontroller}.

\is{Inanimata|)}
\is{Genus|)}

\subsubsection{Indirekter Bezug auf unkombinierte Plural-Controller}
\label{subsubsec:beid2p2snglncao}

Zuletzt bleiben noch Belege für Targets zu diskutieren, die sich auf einen
pronominalen Controller beziehen, der sich seinerseits auf ein einzelnes
Substantiv im Plural bezieht. Ein Beispiel für diesen syntaktischen
Kontext geben \REF{ex:beid2p2snglncao} und die Illustration der
Kongruenzbeziehung\is{Kongruenzrelation} in
\figref{fig:beid2p2snglncao}.

\begin{exe}
\ex \label{ex:beid2p2snglncao}
	\gll vnd ſol ez den zwein wroͮwen gen biz an ir beider tot / ſo
			ſi beide nit enſint \textelp{} \\
		und soll es den zwei Frau-\textsc{dat.pl.\FemF} geben bis an ihr
			beider Tod {} wenn \textsc{3pl\subF.nom}
			beide-\textsc{nom.pl.m+f\subF.st} nicht \textsc{neg}=sind {} \\
	\trans \wdef{und soll es den zwei Frauen geben bis an ihr beider Tod.
		Wenn sie beide nicht \textins{mehr} sind, \textelp{}}
		\parencites(Nr.~2568, Sirnau, Kr.~Esslingen, 1297)[3,31]{cao4}
\end{exe}

\begin{figure}
\begin{tikzpicture}[baseline=(2_lb1.base)]
    \node at (0,0)  (1)     [gray]
                            {\lit{wroͮwen}};
    \node           (1_box) [draw,gray,rectangle,fit=(1)] {};
    \node           (1_lb)  [above=.5ex of 1_box, gray, mynodefont]
                            {controller};

	\node at (3,0) (2)      {\lit{ſi}};
    \draw (2) node (2_box1) [
                        draw,
                        gray,
                        minimum height=3em,
                        minimum width=3em,
                        xshift=-.5ex,
                        yshift=+.5ex,
                        rectangle
                    ] {};
    \draw (2) node (2_box2) [
                        draw,
                        minimum height=3em,
                        minimum width=3em,
                        xshift=+.5ex,
                        yshift=-.5ex,
                        rectangle
                    ] {};
    \node           (2_lb1) [above=.5ex of 2_box1, gray, mynodefont]
                            {target};
    \node           (2_lb2) [below=.5ex of 2_box2, mynodefont]
                            {controller};

    \node at (6,0)  (3)      {\lit{beide}};
    \node           (3_box)  [draw,rectangle,fit=(3)] {};
    \node           (3_lb)   [above=.5ex of 3_box, mynodefont]
                            {target};

    \draw [-latex,gray] (1_box)  to [yshift=-1.5ex]     (2_box1);
    \draw [latex-]      (3_box)  to [yshift=1.5ex]      (2_box2);
\end{tikzpicture}
\caption{Indirekter Bezug eines Targets auf einen einzelnen Erstcontroller über
ein Personalpronomen}
\label{fig:beid2p2snglncao}
\end{figure}

Wie zuvor dienen diese Belege zum Vergleich mit dem Bezug auf kombinierte
\isi{Erstcontroller}. Ihre Verteilung nach den Personenmerkmalen des
Erstcontrollers gibt \tabref{tab:caosimprefctrl2} an. Wie auch in
\tabref{tab:simpnomctrl}, dort mit direktem Bezug auf ein Substantiv im Plural,
lösen maskuline und feminine Referenz beim indirekten Bezug über ein Pronomen
die Kongruenzform \norm{bėide} aus, auf neutrale Referenz folgt dagegen eine
Form vom Typ \norm{bėidiu}. \isi{Belebtheit} scheint auch hier keine Rolle zu
spielen, wenn sich die drei Belege für unbelebte Feminina regelmäßig verhalten.

\begin{table}
\centering
\caption{Flexion nach Personenmerkmalen der anaphorischen Controller (einfacher
Bezug)}
\begin{tabular}{
>{\scshape}l
%	@{\hspace{4\tabcolsep}}
	>{\scshape}l
    r
    r
    r
}
\lsptoprule
\mc{2}{c}{Controller}
    & \norm{bėid(e)}
    & \norm{bėidiu}
    & Summe
    \\
\midrule
% Controller     | e  | iu | Σ
3pl & \MascM    &  2 &    &  2 \\
     & \FemF     &  3 &    &  3 \\
     & \NeutF    &    &  4 &  4 \\
     & \NeutX    &    &  2 &  2 \\

\cmidrule{2-5}

     & \FemI     &  3 &    &  3 \\

\midrule

\mc{2}{l}{Summe} &  8 &  6 & 14 \\

\lspbottomrule
\end{tabular}
\label{tab:caosimprefctrl2}
\end{table}

Im Fall der Formen mit \norm{-iu} unterscheidet \tabref{tab:caosimprefctrl2}
zwischen dem Bezug auf Neutra mit weiblichem Bezug und solchen mit unbekanntem
Bezug. Auch wenn die resultierende Kongruenz\-form in beiden Fällen die gleiche
ist, möchte ich die Belege kurz anhand des Beispiels in
\REF{ex:cao_beidiu_neutfem} charakterisieren.

\begin{exe}
\ex \label{ex:cao_beidiu_neutfem}
	% \begin{xlist}
	% \ex \label{ex:cao_beidiu_neutfem_1}
	% 	\gll engiltrvt ir tohtir Vn̄ annvn ir thohtir thohtir \textelp{}
	% 			ſtirpth der {vor genandon} kint eiz \textelp{} ſterbinz {beidiv
	% 			\textelp{}} \\
	% 		Engeltraut ihr Tochter und Anna-\textsc{obl} ihr Tochter Tochter {}
	% 			stirbt der vorgenannten Kind[\textsc{gen.pl.\NeutF}] eines {}
	% 			sterben=\textsc{3pl\subF.nom} beide-\textsc{nom.pl.\NeutF.st}
	% 			\\
	% 	\trans \wdef{Engeltraut, ihre Tochter, und Anna, ihrer Tochter Tochter
	% 		\textelp{} Stirbt der vorgenannten Kinder eines \textelp{}
	% 		Sterben sie beide \textelp{}}
	% 		\parencites(Nr. 629, St.~Gallen, 1284)[57,24--25]{cao2}

	% \protectedex{%
	% \ex \label{ex:cao_beidiu_neutfem_2}
	% 	\gll Gerhauſe vnde Diemvde \textelp{} vnde ſwenne {der ſelben} chinde
	% 			einz ſtirbet \textelp{} die wile ſi beidiv lebent \textelp{}
	% 			vnde daz der chinde einz dannoch lebt oder ſi beidiv \\
	% 		Gerhaus-\textsc{dat.sg.\FemF} und Diemut-\textsc{dat.sg.\FemF} {}
	% 			und so=wenn derselben Kind-\textsc{gen.pl.\NeutF} eines stirbt
	% 			{} die Weile \textsc{3pl\subF.nom}
	% 			beide-\textsc{nom.pl.\NeutF.st} leben {} und dass der
	% 			Kind-\textsc{gen.pl.\NeutF} eines {dann noch} lebt oder
	% 			\textsc{3pl\subF.nom} beide-\textsc{nom.pl.\NeutF.st} \\
	% 	\trans \wdef{Gerhaus \textins{sic} und Diemut \textelp{} und wenn
	% 		irgendwann derselben Kinder eines stirbt \textelp{} Während sie
	% 		beide am Leben sind \textelp{} und dass der Kinder eines dann
	% 		noch lebt oder sie beide \textelp{}}
	% 		\parencites(Nr.~2719, Nürnberg, 1297)[96,43--97,9]{cao4}%
	% }

	% \ex \label{ex:cao_beidiu_neutfem_3}
		\gll ſweſter Gerdrauden vnd ſweſter Diemvden
				% hern wernhereſ chinden
				\textelp{} vnd ſwenne der vorbenannten chinde einez ſtirbet
				\textelp{} Di weil ſi peidev lebent
				\\
			Schwester Gertraut[\textsc{dat.sg.\FemF}] und Schwester
				Diemut[\textsc{dat.sg.\FemF}]
				% Herrn Wernhers Kinder-\textsc{dat.pl.\NeutF}
				{} und so=wenn der vorbenannten Kind-\textsc{gen.pl.\NeutF}
				eines stirbt {} 
				die Weile \textsc{3pl\subF.nom}
				beide-\textsc{nom.pl.\NeutF.st} leben
				\\
		\trans \wdef{Schwester Gertraut und Schwester Diemut,
			% Herrn Wernhers Kindern
			\textelp{} Und wenn der vorgenannten Kinder eines stirbt
			\textelp{} Während sie beide am Leben sind}
			\parencites(Nr.~2960, Engelthal, Kr.~Nürnberger Land, 1298)[240,31--38]{cao4}
	% \end{xlist}
\end{exe}

Bei allen vier \norm{bėidiu}-Targets geht es um \norm{kint}, also wörtlich
\wdef{Kinder}, wobei dieser Begriff gemäß
\citet[s.\,v.~\fw{kint}]{lexer:mhdhwb} hier treffender in der Bedeutung
\wdef{Tochter} % (\ref{ex:cao_beidiu_neutfem_1}--c)
oder besonders im zitierten Beispiel vielleicht spezifischer als
\wdef{Klosterangehörige} % (\ref{ex:cao_beidiu_neutfem_2}--c)
aufzufassen ist. Die \norm{kint} sind dabei im Kontext der Rechtssprache nicht
zwangsläufig minderjährig, insofern Erwachsene unabhängig von ihrem Alter
Kinder ihrer Eltern sind \autocites[vgl.][1736]{schwab2012}[siehe
auch][258--259]{birkenesfleischer2022}. Um welche Personen es sich bei den
\norm{kint} jeweils handelt, geht aus dem Kontext der jeweiligen Urkunden
hervor.

% Bei \REF{ex:cao_beidiu_neutfem_2} und \REF{ex:cao_beidiu_neutfem_3}
Im zitierten Beispiel handelt es sich um
% die gleichen Personen,
die Schwestern Gertraut und Diemut, Töchter des Wernher vom Stein und
Mitglieder des Konvents von Engelthal \autocite[Kr.~Nürnberger Land;
vgl.][619]{caor}. Die Kongruenz zwischen \lit{ſi} \wdef{sie} und \lit{chinde}
\wdef{Kinder} wurde hier jeweils aufgrund der Nähe zwischen pronominalem Target
und nominalem \isi{Controller} angenommen. Im ganzen unter\-suchten Material
ließen sich keine Belege für direkte Kongruenz zwischen \norm{bėidiu} und zwei
weiblichen Controllern finden. Eine Recherche diesbezüglich im \REM{} lieferte
ebenfalls keine Ergeb\-nisse.

Dem Beispiel in \REF{ex:cao_beidiu_neutfem} steht der Beleg in
\REF{ex:cao_beidiu_neutunkn} gegenüber. In dieser Urkunde werden die Kinder
nicht beim Namen genannt, allerdings geht der Umstand aus dem Text hervor, dass
zumindest eines von ihnen noch nicht \lit{ze ſinen tagen} \wdef{zu seinen Tagen}
(\cites(Nr.~214)[218,18--19]{cao1}; \cite[vgl.][26]{caor}) gekommen, also
minderjährig ist. Auch hier wurde Kongruenz nach der Form zwischen
\lit{kinden} \wdef{Kindern} und \lit{ſvͥ beidvͥ} \wdef{sie beide} angenommen.

\begin{exe}
\protectedex{%
\ex \label{ex:cao_beidiu_neutunkn}
	\gll mit zewain kinden \textelp{} daz ſvn ſvͥ beidvͥ han vnze an ir tôt
			\textelp{} vn̄ ſwen ſvͥ beidvͥ {en ſîn} \textelp{} \\
		mit zwei Kind-\textsc{dat.pl.\NeutX} {} das sollen
			\textsc{3pl\subX.nom} beide-\textsc{nom.pl.\NeutX.st} haben bis an
			ihr Tod {} und so=wenn \textsc{3pl\subX.nom}
			beide-\textsc{nom.pl.\NeutX.st} \textsc{neg}=sind {} \\
	\trans \wdef{mit zwei Kindern \textelp{} das sollen sie beide besitzen
		bis zu ihrem Tod \textelp{} Und wenn sie beide nicht \textins{mehr}
		sind \textelp{}}
		\parencites(Nr.~214, Rottweil, 1274)[218.17--24]{cao1}%
}
\end{exe}

\subsubsection{Zu \posscite{askedal1973} Hypothese der Monoflexion}
\label{subsubsec:monoflexioncao}

\is{Monoflexion|(}

\citet[99]{askedal1973} stellt bezüglich \emph{sie beide} eine Hypothese zur
\textsc{Monoflexion} auf. Diese besagt, dass nur eines der
beiden Glieder neutral flektiert würde. Er schließt dies aus der
Häufigkeit\is{Frequenz} der Belege für \norm{die/diu bėide} beziehungsweise
\norm{si bėide/bėidiu} in seiner \isi{Stichprobe} aus der
\citetitle{maroldschroeder1969}-Edition von \citet{maroldschroeder1969} und der
Edition des \citetitle{lachmannhartl1952} von \citet{lachmannhartl1952}.
\tabref{tab:asksiebeidekombis} gibt die Ergebnisse der Auswertung von
\citet{askedal1973} wieder. Bei den Zahlen in Klammern
\blockcquote[99]{askedal1973}{sind die wahrscheinlichen Elisionsformen mit
einbegriffen}, also Kontexte, in denen der Vokal des Flexionssuffixes im
\isi{Hiatus} oder am \isi{Versende} steht.

\begin{table}
\centering
\caption{Kombinationen von \norm{si} und \norm{die/diu} mit
	\norm{bėide/bėidiu}\\ bei \citet[99]{askedal1973}}
\begin{tabular}{
	l
	r r
	r
	@{\hspace{4\tabcolsep}}
	r r
}
\lsptoprule
Controller
	& \mc{2}{c}{\norm{bėide}}
	& \norm{bėidiu}
	& \mc{2}{c}{Summe}
	\\

\midrule

\norm{si} & 7 & (14)      &  6 & 13 & (20) \\

\midrule

\norm{die} & 0 & (1)      &    &  0 &  (1) \\
\norm{diu} & \mc{2}{c}{1} &    & \mc{2}{c}{1} \\

\midrule

Summe      & 8 & (16)     &  6 & 14 & (22) \\
\lspbottomrule
\end{tabular}
\label{tab:asksiebeidekombis}
\end{table}

In einer Teilauswertung des \CAO{} konnte festgestellt werden, dass in der Tat
\norm{si} die häufigste Form des Pronomens der 3.\ Pers.\ Nom./Akk.\ Pl.\
darstellt. Dies dürfte jedoch weniger der Vermeidung von Redundanz in der
Markierung\is{Genusmarkierung} als vielmehr der Tatsache geschuldet sein, dass
sich Ende des 13.~Jahrhunderts der Typ \norm{si} ohne
Genusdistinktion\is{Genusdistinktion, Abbau der} als die geläufige
oberdeutsche\il{Oberdeutsch} Form dieses Pronomens etabliert hat. \citet[392,
Abbildung P~26]{ksw2} verzeichnen in der zweiten Hälfte des 13.\ Jahrhunderts
für das Bairische\il{Bairisch}, den alemannisch-bairischen
Übergangsbereich\il{Schwäbisch} und das Alemannische\il{Alemannisch} Werte
zwischen 96 und 100\,\% für diesen Typ. Die \tabref{tab:caosiebeidekombis}
zeigt die im \CAO{}-Material belegten Kombinationen und ihre
Häufigkeit\is{Frequenz}.

\begin{table}
\centering
\caption{Kombinationen von \norm{si/sie/siu} und \norm{di/die/diu} mit
	\norm{bėide/bėidiu} im \tit{Corpus der altdeutschen Originalurkunden}}
\begin{tabular}{
	l
	r r
	r
}
\lsptoprule
Controller
	& \norm{bėid(e)}
	& \norm{bėidiu}
	& Summe
	\\

\midrule

%          | e  | iu | Σ  |
\norm{si}  &  8 & 19 & 27 \\
\norm{sie} &  3 &  3 &  6 \\
\norm{siu} &  5 &  9 & 14 \\

\midrule

\norm{di}  &  1 &    &  1 \\
\norm{die} &  3 &    &  3 \\
\norm{diu} &    &  4 &  4 \\

\midrule

Summe      & 20 & 35 & 55 \\
\lspbottomrule
\end{tabular}
\label{tab:caosiebeidekombis}
\end{table}

Eine klare Präferenz für bestimmte Kombinationen ist aus den Daten in
\tabref{tab:caosiebeidekombis} nicht herauszulesen. Es bleibt höchstens darauf
hinzuweisen, dass sowohl \norm{sie bėidiu} als auch \norm{siu bėide} mehrmals
belegt sind. Bei \norm{sie bėidiu} stammen zwei Belege aus dem
bairischen\il{Bairisch} Sprachraum\is{Dialektgeografie} (Rothenburg ob der
Tauber, Kloster Stams) und einer aus dem schwäbischen\il{Schwäbisch} (Kloster
Heiligkreuztal); bei \norm{siu bėide} fanden sich alle fünf Belege in
alemannischen\il{Alemannisch} Urkunden (Kloster Kirchberg, Überlingen, Zürich).
Da in keiner der Urkunden, die diese Belege enthalten, ein systematischer
Unterschied zwischen \norm{si}, \norm{sie} und \norm{siu} festgestellt werden
konnte, müssen sie als zum genusneutralen\is{Genusindifferenz} Typ \norm{si}
gehörig gewertet werden.

Die Belege für \norm{di(e) bėide} und \norm{diu bėidiu} verhalten sich wie
erwartet. Im ersten Fall beziehen sie sich auf Paare aus zwei Männern oder zwei
Frauen, im zweiten auf Paare aus Mann und Frau oder unbelebte Dinge
(\sectref{subsubsec:beid2p2coordncao}).

\subsubsection{Zusammenfassung und Vergleich}
\label{subsubsec:anaperssum}

Der weitaus größere Teil der Urkundenbelege für \norm{bėide} entfällt auf
Targets mit pro\-nomi\-nalem \isi{Controller} und damit auf eine bloß indirekte
Abhängigkeit von kombinierten Controllern. Dies bedeutet nicht, dass weitere
Untersuchungen für diesen syntaktischen Zusammenhang irrelevant wären, da auch
auf \isi{Distanz} Variation in der \isi{Genusmarkierung} von \norm{bėide} als
indirektes Target zweier NPs beobachtet werden kann. Es konnte nicht
festgestellt werden, dass Pronomen, die nicht nach \isi{Genus} flektieren,
einen neutralisierenden Effekt auf die Form von \norm{bėide} haben.

\tabref{tab:cao_e_iu_coord} fasst die Verteilung der Belege für \norm{bėide}
aus den Tabellen~\ref{tab:combnomctrl} und~\ref{tab:caosimprefctrl} noch einmal
vereinfacht zusammen. Insgesamt scheint bei der
Kongruenzbeziehung\is{Kongruenzrelation} zwischen zwei
Controllern\is{Controller} und \norm{bėide} (N\tsub{i}~+~N\tsub{j}) die
\isi{Belebtheit} der Substantive\is{Substantiv} eine wichtigere Rolle als deren
\isi{Genus} zu spielen. Aufgrund der geringen Zahl der Belege für den direkten
Bezug auf kombinierte Controller wurden hier die in \tabref{tab:combnomctrl}
grau gedruckten Belege mitgezählt, da sie sich nicht anders verhalten als ihre
regulär gezählten Gegenstücke. Aussagen über das Verhalten des Targets beim
direkten Bezug von \norm{bėide} auf zwei unbelebte\is{Inanimata} Substantive
können keine gemacht werden, da dieser Kontext in der Stichprobe nicht
belegt ist.

\begin{table}
\centering
\caption{Form nach syntaktischem Kontext (kombinierter
	Bezug)}
\setlength{\tabcolsep}{4pt}
\begin{tabular}{
	l l
	c
	r r
	c
	r r
	c
	r
}
\lsptoprule
\mr{2}{*}[-.5ex]{Belebtheit}
	& \mr{2}{*}[-.5ex]{Sexus}
	& %
	& \mc{2}{c}{N\tsub{i}~+~N\tsub{j}}
	& %
	& \mc{2}{c}{D\tsub{i+j}}
	& %
	& \mr{2}{*}[-.5ex]{Summe}
	\\

\cmidrule{4-5}
\cmidrule{7-8}

%
	& %
	& %
	& \norm{bėid(e)}
	& \norm{bėidiu}
	& %
	& \norm{bėid(e)}
	& \norm{bėidiu}
	& %
	& %
	\\

\midrule

belebt
	& gleich
	& %
	&   1
	&   1
	& %
	&  16
	& 
	& %
	&  18
	\\

%
	& verschieden
	& %
	& 
	&   2
	& %
	&  17
	&  47
	& %
	&  66
	\\

\midrule

unbelebt
	& gleich
	& %
	& 
	&   1
	& %
	&   2
	&  27
	& %
	&  30
	\\

%
	& verschieden
	& %
	& 
	& 
	& %
	& 
	&   7
	& %
	&   7
	\\

\midrule

\mc{2}{l}{Summe}
	& %
	&   1
	&   4
	& %
	&  35
	&  81
	& %
	& 121
	\\

\lspbottomrule
\end{tabular}
\label{tab:cao_e_iu_coord}
\end{table}

Beim direkten Bezug auf zwei belebte\is{Animata} \isi{Controller} mit
verschiedenem \isi{Sexus} ist nur \norm{bėidiu} belegt, allerdings mit zwei
Belegen nicht sehr häufig. Wesentlich besser belegt ist der indirekte Bezug von
\norm{bėide} auf zwei \isi{Erstcontroller} mittels eines Pronomens
(D\tsub{i+j}). Hier liegen für den indirekten Bezug auf zwei belebte
Erstcontroller vom gleichen Geschlecht ausschließlich Belege für
\norm{bėide} vor. Beim indirekten Bezug auf zwei belebte
Substantive mit unterschiedlichem Geschlecht tritt dagegen Variation zwischen
\norm{bėide} und \norm{bėidiu} auf, wobei die neutrale Form
\norm{bėidiu} mehr als doppelt so häufig vertreten ist.

Beim indirekten Bezug auf kombinierte unbelebte\is{Inanimata}
\isi{Erstcontroller} vom gleichen \isi{Genus} entfällt der Großteil der Belege
auf \norm{bėidiu} und lediglich zwei auf \norm{bėide}. Im Fall des indirekten
Bezugs auf zwei unbelebte Substantive mit unterschiedlichem Geschlecht ist nur
die Form \norm{bėidiu} belegt. Insgesamt scheint in diesem Kontext also die
Form \norm{bėidiu} unabhängig vom \is{Genus} bevorzugt zu werden.

Die Beleglage für den direkten Bezug auf einzelne Plural-\isi{Controller} wird
in \tabref{tab:cao_e_iu_simp} vereinfacht dargestellt, als Quelle dienen die
Daten aus den Tabellen~\ref{tab:simpnomctrl} und~\ref{tab:caosimprefctrl2}. Der
Vergleichbarkeit mit \tabref{tab:cao_e_iu_coord} halber wurde auch hier
zunächst nach \isi{Belebtheit} unterschieden. Die Genera\is{Genus} der
einzelnen Controller werden jeweils einzeln aufgeführt, da hier keine
Genus\-kombinationen vorliegen. Bei weitem nicht alle Felder können mit
den Belegen aus dem Material des \CAO{} gefüllt werden.\is{Desiderat} Aufgrund
der Lücken lassen sich mit dieser Tabelle allein keine generellen Aussagen zur
Verteilung der \norm{bėide}-Formen mit direktem und indirekten Bezug auf
einzelne Plural-Controller (N\tsub{i}) machen. Die ausgefüllten Stellen stimmen
aber mit dem überein, was nach \sectref{sec:adjdeclcao} regelmäßig zu erwarten
ist: Formen vom Typ \norm{bėide} beschränken sich hauptsächlich auf Kon\-texte
mit Bezug auf Maskulina und Feminina; \norm{bėidiu} ist auf den neutralen Bezug
beschränkt. Dies scheint unabhängig von der Belebtheit und vom syntaktischen
Kontext zu gelten.

\begin{table}
\centering
\caption{Form nach syntaktischem Kontext (einfacher
	Bezug)}
\begin{tabular}{
	l l
	c
	r r
	c
	r r
	c
	r
}
\lsptoprule
\mr{2}{*}[-.5ex]{Belebtheit}
	& \mr{2}{*}{Genus}
	& %
	& \mc{2}{c}{N\tsub{i}}
	& %
	& \mc{2}{c}{D\tsub{i}}
	& %
	& \mr{2}{*}[-.5ex]{Summe}
	\\

\cmidrule{4-5}
\cmidrule{7-8}

%
	& %
	& %
	& \norm{bėid(e)}
	& \norm{bėidiu}
	& %
	& \norm{bėid(e)}
	& \norm{bėidiu}
	& %
	& %
	\\

\midrule

belebt
	& maskulin
	& %
	& 21
	& 
	& %
	&  2
	& 
	& %
	& 23
	\\

%
	& feminin
	& %
	& 
	& 
	& %
	&  3
	& 
	& %
	&  3
	\\

%
	& neutral
	& %
	& 
	& 
	& %
	& 
	&  6
	& %
	&  6
	\\

\midrule

unbelebt
	& maskulin
	& %
	&  7
	& 
	& %
	& 
	& 
	& %
	&  7
	\\

%
	& feminin
	& %
	& 
	& 
	& %
	&  3
	& 
	& %
	&  3
	\\

%
	& neutral
	& %
	&  1
	&  5
	& %
	& 
	& 
	& %
	&  6
	\\

\midrule

\mc{2}{l}{Summe}
	& %
	& 29
	&  5
	& %
	&  8
	&  6
	& %
	& 48
	\\

\lspbottomrule
\end{tabular}
\label{tab:cao_e_iu_simp}
\end{table}

Beim lexikalischen Blick auf die Kombination \norm{si bėide} hat sich ergeben,
dass~-- wie von \citet{askedal1973} für die jeweils verwendeten kritischen
Ausgaben des \tit{Parzival} und des \tit{Tristan} beobachtet~-- auch im
\CAO{} mit Abstand die häufigste Variante \norm{si bėide/-iu} ist.
\citeauthor{askedal1973} nimmt an, dass das Fehlen der morphologischen
Markierung von Genus\is{Genusmarkierung} an einem der beiden Glieder
grammatikalisiert\is{Grammatikalisierung} ist, macht dies aber höchst
fragwürdig an Einzelbelegen fest. Das umfangreichere Material des \CAO{} weist
darüber hinaus mehr Variation auf als die sehr kleine Stichprobe von
\citeauthor{askedal1973}. Die Häufigkeit\is{Frequenz} von \norm{si} \wdef{sie}
ist zumindest im \CAO{} der Vereinfachung des pronominalen
Paradigmas\is{Paradigma} hin zu einer genusübergreifenden\is{Genusindifferenz}
Pluralform zuzuschreiben\is{Genusdistinktion, Abbau der}, die nach
\citet[391--392]{ksw2} im Oberdeutschen\il{Oberdeutsch} der zweiten Hälfte des
13.~Jahrhunderts den Normalfall darstellt \autocite[vgl.\
auch][37--39]{sparmann1961}.

\is{Monoflexion|)}
\is{Personenmerkmal|)}
\is{Target|)}

%%%%%%%%%%%%%%%%%%%%%%%%%%%%%%%%%%%%%%%%%%%%%%%%%%%%%%%%%%%%%%%%%%%%%%%%%%%%%%%

\section{Targets nach Distanz zum Controller}
\label{sec:caotargdist}

\is{Target|(}
\is{Controller|(}
\is{Distanz|(}

Die Untersuchungen von \citet{corbett1979} zeigen, dass mit zunehmender Distanz
zwischen Controller und Target die Wahrscheinlichkeit für das Auftreten von
Kongruenz \fw{ad sensum} steigt. Die Distanz lässt sich auf zweierlei Arten
messen: zum einen als linearer Abstand in Wortformen\is{Distanz!lineare} (im
Sinne von in sich geschlossenen morphosyntaktischen Einheiten,
vgl.~\cite[252--253]{bauer2000}), zum anderen als syntaktischer
Abstand\is{Distanz!syntaktische} hinsichtlich der \term{Domäne}\is{Domäne} von
Target und Controller (gleiches Satzglied, gleicher Teilsatz, anderer
(Teil-)Satz; vgl.~\sectref{sec:ctrltarg}). Die Schemata in den
Abbildungen~\ref{fig:caodirdist_words} und~\ref{fig:caodirdist_syn}
verdeutlichen die jeweilige Zählweise.

\begin{figure}
\begin{tikzpicture}[
	word/.style={
		draw,
		minimum height=1.5em,
		minimum width=4em,
		font=\itshape
	},
	lbl/.style={
		minimum height=1.5em,
		mynodefont
	},
	every node/.style={
		anchor=base,
		text height=1.5ex,
		text depth=.5ex
	},
	every edge/.append style={
		min distance=2.25em
	}
]

\node [word]                      (1) {Ich};
\node [word, base right=1ex of 1] (2) {unde};
\node [word, base right=1ex of 2] (3) {mīn};
\node [word, base right=1ex of 3] (4) {hūsvrouwe};
\node [word, base right=1ex of 4] (5) {verjęhen};
\node [word, base right=1ex of 5] (6) {bėidiu};

\path[-]      (1) edge [out=north, in=north] node [lbl, above] {0} (2);
\path[-]      (2) edge [out=north, in=north] node [lbl, above] {1} (3);
\path[-]      (3) edge [out=north, in=north] node [lbl, above] {2} (4);
\path[-]      (4) edge [out=north, in=north] node [lbl, above] {3} (5);
\path[-latex] (5) edge [out=north, in=north] node [lbl, above] {4} (6);
\path[-]      (4) edge [out=south, in=south] node [lbl, below] {0} (5);
\path[-latex] (5) edge [out=south, in=south] node [lbl, below] {1} (6);
\end{tikzpicture}
\caption{Abstandsberechnung nach Wortformen}
\label{fig:caodirdist_words}
\end{figure}

\begin{figure}
	\begin{tikzpicture}[
	baseline=(vflbl.base),
	box/.style={
		draw,
		minimum height=2.5em,
		font=\itshape
	},
	wordbox/.style={
		draw,
		minimum height=1.75em,
		font=\itshape
	},
	lbl/.style={
		minimum height=1.5em,
		mynodefont
	},
	every node/.style={
		anchor=base,
		text height=1.5ex,
		text depth=.5ex
	}
]

\node [wordbox,                    ] (1) {Ich};
\node [wordbox, base right=1ex of 1] (2) {unde};
\node [wordbox, base right=1ex of 2] (3) {mīn};
\node [wordbox, base right=1ex of 3] (4) {hūsvrouwe};
\node [wordbox, base right=3ex of 4] (5) {verjęhen};
\node [wordbox, base right=3ex of 5, thick] (6) {bėidiu};
\node [         base right=3ex of 6] (7) {\phantom{abcdef}};

\node (VF)  [box, rectangle, fit={(1) (2) (3) (4)}, thick] {};
\node (LSK) [box, rectangle, fit=(5)] {};
\node (MF)  [box, rectangle, fit=(6)] {};
\node (RSK) [box, rectangle, fit=(7)] {};

\node (vflbl)  [lbl, above=.5ex of VF]  {VF};
\node (lsklbl) [lbl, above=.5ex of LSK] {LSK};
\node (mflbl)  [lbl, above=.5ex of MF]  {MF};
\node (rsklbl) [lbl, above=.5ex of RSK] {RSK};

\draw [-, draw=white, double=black, ultra thick] (VF) -- ++(south:2em) -| (6);
\end{tikzpicture}
\vspace{.5\baselineskip}
\caption{Abstandsberechnung nach der syntaktischen Domäne}
\label{fig:caodirdist_syn}
\end{figure}

In \figref{fig:caodirdist_words} wird der Abstand\is{Distanz!lineare} zwischen
den beiden Controllern \norm{ich} \wdef{ich} und \norm{(mīn) hūsvrouwe}
\wdef{(meine) Ehefrau} zu ihrem Target \norm{bėidiu} \wdef{beide} nach
Wortformen gezählt. Der erste Sprung zur jeweils nächsten Wortform wird mit
null angesetzt, da zwischen \norm{ich} und \norm{unde} \wdef{und} sowie
zwischen \norm{hūsvrouwe} \wdef{Ehefrau} und \norm{verjęhen} \wdef{aussagen,
bekennen} jeweils kein Abstand besteht. Auf diese Weise ergibt sich für das
Beispiel insgesamt ein kombinierter Wortformenabstand von vier und eins
zwischen Controller und Target.

Bei den Untersuchungen zum Belegmaterial des \CAO{} hat sich die
Entfernungsbestimmung nach Satzgliedern und Sätzen (im Folgenden vereinfachend:
syntaktische Distanz\is{Distanz!syntaktische}) allerdings als sinnvoller
erwiesen. In \figref{fig:caodirdist_syn} wird das Beispiel der Anschaulichkeit
halber vereinfacht im Rahmen des Feldermodells\is{Feldermodell} des
deutschen Satzes wiederholt \autocites%
	{drach1963}%
	{reis1980}%
	{hoehle1986}%
	{hoehle2019a}%
	{hoehle2019b}%
	[vgl.~auch][]{woellstein2010}[44--49]{mueller2023}%
. Die Controller \norm{ich} und \norm{hūs\-vrouwe} stehen im
\isi{Vorfeld} (\textsc{vf}), während das Target \norm{bėidiu} nach der linken
\isi{Satzklammer} (\textsc{lsk}) im \isi{Mittelfeld} (\textsc{mf}) steht. Damit
steht das Target in einem anderen Satzteil als seine Controller, dennoch aber
im gleichen Teilsatz, in diesem Fall dem Hauptsatz (\norm{verjęhen}
\wdef{aussagen, bekennen} nimmt in der Regel einen Nebensatz als
\isi{Komplement}). Da die linke Satzklammer mit einem Vollverb besetzt ist,
bleibt die rechte Satzklammer (\textsc{rsk}) leer. Der
Wortformenabstand\is{Distanz!lineare} steigt tendenziell mit zunehmendem
Abstand zwischen Satzgliedern und (Teil-)Sätzen.

\is{Anapher|)}

\subsection{Nominale Controller}
\label{subsec:caodistnomctrl}

Für die syntaktische Distanz\is{Distanz!syntaktische} liegt die in
\tabref{tab:caocodistp} gezeigte Belegverteilung vor. Hierbei ist keine
Variation in der \isi{Lokalität} zwischen beiden Controllern zu beobachten. Es
liegen also keine Fälle vor, in denen beide Controller in unterschiedlichen
(Teil-)Sätzen stehen. Aufgrund der geringen Belegmenge sind auch hier keine
starken Konzentrationen von Belegen zu beobachten. Die unter \isi{Vorbehalt}
aufgenommenen Belege sind grau gedruckt; bei ihnen handelt es sich in allen
Fällen zufällig um Targets in \isi{Distanzstellung} zu ihrem Controller
(vgl.~\sectref{sec:floatquant} zu gefloateten Quantoren\is{gefloateter Quantor}).
In allen Fällen folgt das Target seinen Controllern.

\begin{table}
\setlength{\tabcolsep}{4pt}
\caption{Form nach Distanz von kombinierten Controllern}
\begin{tabular}{
	l
	c >{\scshape}l >{\scshape}l
	r
	r
	r
}
\lsptoprule

Domäne
	& Wortdist.
	& \normalfont Controller 1
	& \normalfont Controller 2
	& \norm{bėide}
	& \norm{bėidiu}
	& Summe
	\\

\midrule

gl.~Satzteil
	& 3 / 0
	& 3sg.\MascM
	& 3sg.\FemF
	& %
	& 1
	& 1
	\\

%
	& 7 / 2
	& 3sg.\MascM
	& 3sg.\MascM
	& %
	& 1
	& 1
	\\

\midrule

\mc{4}{l}{Summe}
	& 
	& 2
	& 2
	\\

\midrule

\gr{gl. Teilsatz}
	& \gr{9 / 3}
	& \gr{1sg\subM}
	& \gr{3sg.\FemF}
	& %
	& \gr{1}
	& \gr{1}
	\\

%
	& \gr{9 / 4}
	& \gr{1sg\subM}
	& \gr{1sg\subM}
	& \gr{1}
	& %
	& \gr{1}
	\\

\midrule

\gr{and. (Teil-)Satz}
	& \gr{20 / 3}
	& \gr{3sg.\MascI}
	& \gr{3sg.\MascI}
	& %
	& \gr{1}
	& \gr{1}
	\\

\midrule

\mc{4}{l}{\gr{Summe}}
	& \gr{1}
	& \gr{2}
	& \gr{3}
	\\

\lspbottomrule
\end{tabular}
\label{tab:caocodistp}
\end{table}

Trotz der geringen Belegmenge lässt die Verteilung vermuten, dass
Personenmerkmale\is{Personenmerkmal} eine wichtigere Rolle spielen als die
syntaktische Distanz\is{Distanz!syntaktische} oder die
Wortformen\-distanz\is{Distanz!lineare}. Einzig der Beleg in
\REF{ex:n241_hofzehnt} für zwei unbelebte\is{Inanimata} Maskulina mit einem
Abstand von zwanzig und drei Wortformen fällt durch die unerwartete
\norm{bėidiu}-Form auf. In \sectref{subsubsec:beid2p2coordncao} bezüglich der
indirekten Referenz auf kombinierte Controller konnte jedoch beobachtet werden,
dass bei unbelebtem Bezug häufig auch unab\-hängig vom \isi{Genus} der
Referenten \norm{bėidiu} steht.

\begin{exe}
\ex\label{ex:n241_hofzehnt}
	\setlength{\glossglue}{5pt plus 2pt minus 1pt}
	\gll daz ich minen frowen von ſande Katerinen minen hof, den ich ze
			Jnningen hete, da etwenne der alte Stevdler ovf ſaz, verkaufft han
			mit dem zehenden, der drovz gat, beidev vnuerſchaidenlichen fvͤr
			reht aigen \\
		dass ich meinen Frauen von Sankt Katharinen meinen
			Hof[\textsc{acc.sg.\MascI}] den ich zu Inningen hatte wo vormals
			der alte Steudler auf saß verkauft habe mit dem
			Zehnt-\textsc{dat.sg.\MascI} der daraus geht
			beide-\textsc{acc.pl.\NeutI.st} gleichermaßen für rechtmäßig
			Eigentum \\
	\trans \wdef{dass ich meinen Frauen von St.~Katharinen meinen Hof, den ich
		in Inningen hatte, den vormals der alte Steudler inne hatte, verkauft
		habe \textins{zusammen} mit dem Zehnten, den er einbringt, beide
		gleichermaßen zum rechtmäßigen Eigentum}
		\parencites(Nr.~N~241, Mainau und Konstanz, 1275)[195,37--38]{cao5}
\end{exe}

Wie auch zuvor bleiben noch diejenigen \norm{bėide}-Belege zu diskutieren, die
unmittelbar von einer einfachen NP im Plural abhängen. In allen Fällen
verteilen sich die Formen von \norm{bėide} in \tabref{tab:caopldistpct} wie
erwartet. Bei maskulinem Bezug steht \norm{bėide}, bei neutralem in der Regel
\norm{bėidiu}. Ob das Target dem Controller vorangeht oder nachfolgt, hat keine
Auswirkung auf die Beleg\-verteilung.

\begin{table}
\centering
\caption{Form nach Distanz vom einfachen nominalen Controller}
\begin{tabular}{
	l
	c >{\scshape}l
	@{\hspace{4\tabcolsep}}
	r
	r
	@{\hspace{4\tabcolsep}}
	r
}
\lsptoprule

Domäne
	& Wortdist.
	& \normalfont Controller
	& \norm{bėid(e)}
	& \norm{bėidiu}
	& Summe
	\\

\midrule

gl. Satzglied
	& 0--1
	& \MascM
	& 21
	& %
	& 21
	\\

%
	& %
	& \MascI
	& 7
	& %
	& 7
	\\

%
	& %
	& \NeutI
	& 1
	& 3
	& 4
	\\

\midrule

gl. Teilsatz
	& 0--2
	& 3pl.\NeutI
	& %
	& 2
	& 2
	\\

\midrule

\mc{3}{l}{Summe}
	& 29
	& 5
	& 34
	\\

\lspbottomrule
\end{tabular}
\label{tab:caopldistpct}
\end{table}

\subsection{Anaphorische Controller}
\label{subsec:caodistanactrl}
\is{Anapher|(}

Die Frage, wie sich \norm{bėide}-Targets in Bezug auf die Distanz zu
pronominalen Controllern verhalten, lässt sich zweiteilen. Zum einen kann die
Distanz der direkten Abhängigkeit zwischen anaphorischem Controller und Target
untersucht werden,\is{Distanz!syntaktische} zum anderen kann die Distanz des
Pronomens zu seinem Referenten miteinbezogen werden, also die indirekte
Abhängigkeit zwischen den Erstcontrollern\is{Erstcontroller} und dem
\norm{bėide}-Target. Zunächst erfolgt in \tabref{tab:caoanadist} ein Blick auf
die direkte Abhängigkeit zwischen anaphorischem Controller und Target.

\tabref{tab:caoanadist} weist die Belegverteilung nach syntaktischem
Abstand\is{Distanz!syntaktische} der Kongruenzbeziehung\is{Kongruenzrelation}
aus. Es ist deutlich sichtbar, dass die allermeisten Targets im gleichen
Satzglied wie ihr Controller und größtenteils in unmittelbarer Nähe zueinander
stehen \autocite[vgl.][526--527]{ksw2}. Da aus \sectref{sec:caotargpers} zur
Untersuchung der Form des Quantors \norm{bėide} anhand der referenzierten
Personenmerkmale\is{Personenmerkmal} bereits deutlich wurde, dass die
ausschlaggebenden Merkmale für die Variation zwischen \norm{bėide} und
\norm{bėidiu} \isi{Belebtheit} und gleiches beziehungsweise verschiedenes
Geschlecht\is{Sexus} beziehungsweise \isi{Genus} sind, werden in der Tabelle
die einzelnen Targets nicht mehr nach ihren Personenmerkmale aufgelistet,
sondern nach Belebtheit und gleichem oder verschiedenem Geschlecht
beziehungsweise Genus.

\begin{sidewaystable}
\captionsetup{width=.8\textwidth}
\caption{Form nach Distanz vom anaphorischen Controller (kombinierter Bezug)}
\setlength{\tabcolsep}{4pt}
\begin{tabular}{
	l
	c
	r r c
	r r c
	r r c
	r r
	r
}

\lsptoprule

\mr{3}{*}[-1ex]{Domäne}
	& \mr{3}{*}[-1ex]{Wortdist.}
	& \mc{5}{c}{belebt}
	& %
	& \mc{5}{c}{unbelebt}
	& \mr{3}{*}[-1ex]{Summe}
	\\

\cmidrule{3-7}
\cmidrule{9-13}

%
	& %
	& \mc{2}{c}{gleich}
	& %
	& \mc{2}{c}{verschieden}
	& %
	& \mc{2}{c}{gleich}
	& %
	& \mc{2}{c}{verschieden}
	& %
	\\

\cmidrule{3-4}
\cmidrule{6-7}
\cmidrule{9-10}
\cmidrule{12-13}

%
	& %
	& \mc{1}{c}{\norm{bėid(e)}}
	& \mc{1}{c}{\norm{bėidiu}}
	& %
	& \mc{1}{c}{\norm{bėid(e)}}
	& \mc{1}{c}{\norm{bėidiu}}
	& %
	& \mc{1}{c}{\norm{bėid(e)}}
	& \mc{1}{c}{\norm{bėidiu}}
	& %
	& \mc{1}{c}{\norm{bėid(e)}}
	& \mc{1}{c}{\norm{bėidiu}}
	& %
	\\

\midrule

gleiches Satzglied
	& 0--1
	& 10 % belebt gleich beide
	& % belebt gleich beidiu
	& %--
	& 15 % belebt verschieden beide
	& 38 % belebt verschieden beidiu
	& %==
	& % unbelebt gleich beide
	& 1 % unbelebt gleich beidiu
	& %--
	& % unbelebt verschieden beide
	& 2 % unbelebt verschieden beidiu
	& 66 % Summe
	\\

\midrule

gleicher Teilsatz
	& 0--1
	&  6 % belebt gleich beide
	& % belebt gleich beidiu
	& %--
	& % belebt verschieden beide
	&  4 % belebt verschieden beidiu
	& %==
	&  2 % unbelebt gleich beide
	& 26 % unbelebt gleich beidiu
	& %--
	& % unbelebt verschieden beide
	&  5 % unbelebt verschieden beidiu
	& 41 % Summe
	\\

%
	& 2--3
	& % belebt gleich beide
	& % belebt gleich beidiu
	& %--
	&  1 % belebt verschieden beide
	&  4 % belebt verschieden beidiu
	& %==
	& % unbelebt gleich beide
	& % unbelebt gleich beidiu
	& %--
	& % unbelebt verschieden beide
	& % unbelebt verschieden beidiu
	&  5 % Summe
	\\

%
	& 13
	& % belebt gleich beide
	& % belebt gleich beidiu
	& %--
	& % beide belebt verschieden
	&  1 % belebt verschieden beidiu
	& %==
	& % unbelebt gleich beide
	& % unbelebt gleich beidiu
	& %--
	& % unbelebt verschieden beide
	& % unbelebt verschieden beidiu
	&  1 % Summe
	\\

\midrule

\mc{2}{l}{Summe}
	& 16 % belebt gleich beide
	& % belebt gleich beidiu
	& %--
	& 16 % belebt verschieden beide
	& 47 % belebt verschieden beidiu
	& %==
	& 2 % unbelebt gleich beide
	& 27 % unbelebt gleich beidiu
	& %--
	& % unbelebt verschieden beide
	& 7 % unbelebt verschieden beidiu
	& 115 % Summe
	\\

\lspbottomrule
\end{tabular}
\label{tab:caoanadist}
\end{sidewaystable}

Bei Targets in der \isi{Domäne} \emph{gleicher Teilsatz} konzentrieren sich die
Belege auf \norm{bėidiu}. Hierzu ist anzumerken, dass es sich bei 23 dieser
Targets um Quantoren nach Relativpronomen\is{Relativpronomen} mit kombinierter
unbelebter\is{Inanimata} neutraler Referenz aus Belegen wie dem in
\REF{ex:insigel} handelt. Formulierungen dieser Art sind insbesondere in
Augsburger Urkunden\is{Urkunde} sehr häufig anzutreffen und damit im hier
gesammelten Belegmaterial überrepräsentiert. \citet{haacke1964} hebt
diesbezüglich die herausragende Rolle des Augsburger Stadtschreibers Konrad
(nachweisbar von Mitte der 1270er Jahre bis 1285) für die Schaffung eines
verbindlichen deutschsprachigen Urkundenformulars hervor
\autocite[111--112]{haacke1964}. Als Merkmal einer Konrad-typischen
Corroboratio gibt er als Beispiel: \lit{darvmbe han ich im geben diſen brief /
verſigelt vnde gefeſtent mit minem Jnſigel vnde mit der Stet Jnſigel zu
auſpurch div baidiv dran hangent} \autocites(Nr.~N~272, Augsburg,
1285)[120--121]{haacke1964}[vgl.~dazu][216,1--2]{cao5}. Dies entspricht im
Grunde der Formulierung in \REF{ex:insigel}.

\begin{exe}
\ex\label{ex:insigel}
	\gll Daz deſ niht vergezzen werde {dar vmb} iſt gemachet dirre brief
			verſigelt vn̄ geveſtent / mit der ſtet Jnſigel ze auſpurch / vn̄
			mit vnſerm Jnſigel dıͤv baidıͤv dran hangent \\
		Dass dessen nicht vergessen werde darum ist gemacht dieser Urkunde
			versiegelt und festgemacht {} mit der Stadt
			Siegel[\textsc{dat.sg.\NeutI}] zu Augsburg {} und mit unserem
			Siegel[\textsc{dat.sg.\NeutI}] \textsc{rel.nom.pl.\NeutI}
			beide-\textsc{nom.pl.\NeutI.st} daran hängen \\
	\trans \wdef{Damit dies nicht vergessen werde, darum ist diese Urkunde
		ausgefertigt, versiegelt und bestätigt mit dem Siegel der Stadt
		Augsburg und mit unserem Siegel, die beide daran hängen.}
		\parencites(Nr.~3056, Augsburg, 1298)[304,15--17]{cao4}
\end{exe}

Aus \tabref{tab:caoanadist} wird außerdem deutlich, dass weder der Abstand
zwischen Controller und Target in Wortformen\is{Distanz!lineare} noch die
syntaktische Distanz\is{Distanz!syntaktische} zwischen ihnen einen markanten
Einfluss auf die Belegverteilung hat. Auch unter der Perspektive der Distanz
gelten also die Schlussfolgerungen aus \sectref{subsubsec:beid2p2coordncao}.
Einzig auffällig ist, dass sich bei der \isi{Distanzstellung} von Targets die
belebten\is{Animata} Belege wesentlich regelmäßiger entsprechend semantischen
Kriterien verhalten als in der \isi{Kontaktstellung} (\emph{gleiches
Satzglied}). Und zwar steht bei neun von zehn Belegen mit
gemischtgeschlechtlichem Bezug eine Form vom Typ \norm{bėidiu}, während in
Kontaktstellung nur etwa 70\,\% der Belege auf diese Form entfallen. Beim
unbelebten\is{Inanimata} Bezug dagegen ist kein entsprechender Unterschied zu
beobachten.

Betreffend den indirekten Bezug auf einzelne Substantive im Plural als
\isi{Erstcontroller} gilt dasselbe. Auch hier verteilen sich die fünf Targets
in der \isi{Domäne} \emph{gleicher Teilsatz} nicht anders als diejenigen in der
Domäne \emph{gleiches Satzglied}. An den bisherigen Beobachtungen zum Einfluss
der Personenmerkmale auf die Form von \norm{bėide} in
\sectref{subsubsec:beid2p2snglncao} ändert sich also ebenfalls nichts.

\is{Anapher|)}

\subsection{Wortformenabstand zu kombinierten Erstcontrollern}
\is{Distanz!lineare|(}
\is{Erstcontroller|(}

Sehr viele Belege für \norm{bėide} stehen direkt nach einem Pronomen, daher ist
die Aussagekraft der Auszählungen im vorhergehenden Abschnitt eher gering.
Folgend wird daher untersucht, ob die Entfernung zwischen
\norm{bėide}-Targets und ihren Erstcontrollern einen Effekt auf die Wahl der
Kongruenzform des Targets hat.

Da es für viele Entfernungswerte\is{Verweiskette} nur einen einzigen Beleg mit
diesem Wert gibt, ist es sinnvoll, sie zu Klassen zusammenzufassen. Um dem
Problem aus dem Weg zu gehen, gegebenenfalls mit verschiedenen Klassenwerten
für die jeweiligen Erstcontroller arbeiten zu müssen, wurde zur
\isi{Komplexitätsreduktion} bei der \isi{Klassenbildung} jeweils das
arithmetische Mittel des Wortformenabstands für jedes Controllerpaar gewertet,
vergleiche das fiktive Beispiel in \tabref{tab:worddistarith}.

\begin{table}
\centering
\caption{Berechnung des gemittelten Wortformenabstands}
\begin{tabular}[t]{c r c c c}
	\lsptoprule
	Erstcontroller
		& Wortdist.
		& Controller
		& Wortdist.
		& Target
		\\

	\midrule

	\norm{Konrad}
		& 47
		& \mr{2}{*}{\norm{si}}
		& \mr{2}{*}{0}
		& \mr{2}{*}{\norm{bėidiu}}
		\\

	\norm{Elisabėth}
		& 43
		& %
		& %
		& %
		\\

	\midrule

	\norm{Konrad}/%
	\norm{Elisabėth}
		& ∅~45
		& \norm{si}
		& 0
		& \norm{bėidiu}
		\\

	\lspbottomrule
\end{tabular}
\label{tab:worddistarith}
\end{table}

In diesem Beispiel folgt das Kongruenztarget \norm{bėidiu} seinem direkten
Controller \norm{si} \wdef{sie} unmittelbar, das heißt mit einem Abstand von
null Wortformen. Das Pronomen \norm{si} wiederum geht in letzter Instanz zurück
auf das Erstcontroller-Paar \norm{Konrad} und
\norm{Elisabėth}\is{Verweiskette}. Da zwei Controller involviert sind, ist
deren Abstand zu ihrem pronominalen Target verschieden: Zwischen \norm{Konrad}
und \norm{si} befinden sich 47 Wortformen, zwischen \norm{Elisabėth} und
\norm{si} 43 Wortformen. Um beim Vergleich von Distanzen nicht ständig mit zwei
Werten operieren zu müssen, wird vereinfacht\is{Komplexitätsreduktion} mit dem
gemittelten Abstand zwischen \norm{Konrad} und \norm{Elisabėth} zu \norm{si}
von 45 Wortformen gerechnet. Der Abstand zwischen \norm{bėidiu} und seinen
Erstcontrollern beträgt damit etwa 46 Wortformen.

Die \tabref{tab:caodist} stellt die Verteilung der Belege nach Distanzklasse
mit Angabe der durchschnittlichen Distanz zwischen Erstcontrollern und dem
untersuchten Target dar\is{Verweiskette}, jeweils nach \isi{Belebtheit} und
gleichem oder verschiedenem Geschlecht beziehungsweise \isi{Genus} aufgeteilt.
Die Einteilung nach Vielfachen von Fünferpotenzen mit wachsender
Intervallbreite wurde gewählt, weil sehr wenige Belege in unmittelbarer Nähe
der Erstcontroller vorliegen und viele im zweistelligen Entfernungs\-bereich.
Auch in dieser Darstellung wird deutlich, dass unabhängig von der linearen
Entfernung in Wortformen mit großer Regelmäßigkeit bei den belebten\is{Animata}
Targets bei gleichem Geschlecht der Erstcontroller die Form \norm{bėid(e)}
steht, bei unterschiedlichem Geschlecht \norm{bėidiu}. Ebenso regelmäßig steht
bei den unbelebten\is{Inanimata} Targets unabhängig vom Genus der
Erstcontroller \norm{bėidiu}.

\begin{sidewaystable}
\caption{Form nach Distanz von kombinierten Erstcontrollern}
%
% {1,2,3,4} × 5 ^ {1,2,3}
%
\setlength{\tabcolsep}{4pt}
\begin{tabular}{
	c
	l
	r r c
	r r c
	r r c
	r r
	r
}

\lsptoprule

\mr{3}{*}[-1ex]{Klasse}
	& \mr{3}{*}[-1ex]{Wordist.}
	& \mc{5}{c}{belebt}
	& %
	& \mc{5}{c}{unbelebt}
	& \mr{3}{*}[-1ex]{Summe}
	\\

\cmidrule{3-7}
\cmidrule{9-13}

%
	& %
	& \mc{2}{c}{gleich}
	& %
	& \mc{2}{c}{verschieden}
	& %
	& \mc{2}{c}{gleich}
	& %
	& \mc{2}{c}{verschieden}
	& %
	\\

\cmidrule{3-4}
\cmidrule{6-7}
\cmidrule{9-10}
\cmidrule{12-13}

%
	& %
	& \mc{1}{c}{\norm{bėid(e)}}
	& \mc{1}{c}{\norm{bėidiu}}
	& %
	& \mc{1}{c}{\norm{bėid(e)}}
	& \mc{1}{c}{\norm{bėidiu}}
	& %
	& \mc{1}{c}{\norm{bėid(e)}}
	& \mc{1}{c}{\norm{bėidiu}}
	& %
	& \mc{1}{c}{\norm{bėid(e)}}
	& \mc{1}{c}{\norm{bėidiu}}
	& %
	\\

\midrule

1
	& ≤ 5
	& 5 % belebt gleich beide
	& 1 % belebt gleich beidiu
	& %--
	& % belebt verschieden beide
	& 1 % belebt verschieden beidiu
	& %==
	& 1 % unbelebt gleich beide
	& 18 % unbelebt gleich beidiu
	& %--
	& % unbelebt verschieden beide
	& 3 % unbelebt verschieden beidiu
	& 29
	\\

2
	& ≤ 10
	& % belebt gleich beide
	& % belebt gleich beidiu
	& %--
	& 8 % belebt verschieden beide
	& 3 % belebt verschieden beidiu
	& %==
	& 1 % unbelebt gleich beide
	& 6 % unbelebt gleich beidiu
	& %--
	& % unbelebt verschieden beide
	& % unbelebt verschieden beidiu
	& 18
	\\

3
	& ≤ 15
	& 2 % belebt gleich beide
	& % belebt gleich beidiu
	& %--
	& % belebt verschieden beide
	& 6 % belebt verschieden beidiu
	& %==
	& % unbelebt gleich beide
	& 2 % unbelebt gleich beidiu
	& %--
	& % unbelebt verschieden beide
	& 1 % unbelebt verschieden beidiu
	& 11
	\\

4
	& ≤ 20
	& % belebt gleich beide
	& % belebt gleich beidiu
	& %--
	& 1 % belebt verschieden beide
	& 3 % belebt verschieden beidiu
	& %==
	& % unbelebt gleich beide
	& % unbelebt gleich beidiu
	& %--
	& % unbelebt verschieden beide
	& 1 % unbelebt verschieden beidiu
	& 5
	\\

5
	& ≤ 25
	& 2 % belebt gleich beide
	& % belebt gleich beidiu
	& %--
	& % beide belebt verschieden
	& 2 % belebt verschieden beidiu
	& %==
	& % unbelebt gleich beide
	& 1 % unbelebt gleich beidiu
	& %--
	& % unbelebt verschieden beide
	& % unbelebt verschieden beidiu
	& 5
	\\

6
	& ≤ 50
	& 2 % belebt gleich beide
	& % belebt gleich beidiu
	& %--
	& 2 % beide belebt verschieden
	& 10 % belebt verschieden beidiu
	& %==
	& % unbelebt gleich beide
	& 1 % unbelebt gleich beidiu
	& %--
	& % unbelebt verschieden beide
	& 1 % unbelebt verschieden beidiu
	& 16
	\\

7
	& ≤ 75
	& 1 % belebt gleich beide
	& % belebt gleich beidiu
	& %--
	& % beide belebt verschieden
	& 11 % belebt verschieden beidiu
	& %==
	& % unbelebt gleich beide
	& % unbelebt gleich beidiu
	& %--
	& % unbelebt verschieden beide
	& 1 % unbelebt verschieden beidiu
	& 13
	\\

8
	& ≤ 100
	& 1 % belebt gleich beide
	& % belebt gleich beidiu
	& %--
	& 2 % beide belebt verschieden
	& 4 % belebt verschieden beidiu
	& %==
	& % unbelebt gleich beide
	& % unbelebt gleich beidiu
	& %--
	& % unbelebt verschieden beide
	& % unbelebt verschieden beidiu
	& 7
	\\

9
	& ≤ 125
	& 1 % belebt gleich beide
	& % belebt gleich beidiu
	& %--
	& % beide belebt verschieden
	& 2 % belebt verschieden beidiu
	& %==
	& % unbelebt gleich beide
	& % unbelebt gleich beidiu
	& %--
	& % unbelebt verschieden beide
	& % unbelebt verschieden beidiu
	& 3
	\\

10
	& ≤ 250
	& 2 % belebt gleich beide
	& % belebt gleich beidiu
	& %--
	& 4 % beide belebt verschieden
	& 5 % belebt verschieden beidiu
	& %==
	& % unbelebt gleich beide
	& % unbelebt gleich beidiu
	& %--
	& % unbelebt verschieden beide
	& % unbelebt verschieden beidiu
	& 11
	\\

11
	& ≤ 375
	& % belebt gleich beide
	& % belebt gleich beidiu
	& %--
	& % beide belebt verschieden
	& % belebt verschieden beidiu
	& %==
	& % unbelebt gleich beide
	& % unbelebt gleich beidiu
	& %--
	& % unbelebt verschieden beide
	& % unbelebt verschieden beidiu
	& 0
	\\

12
	& ≤ 500
	& % belebt gleich beide
	& % belebt gleich beidiu
	& %--
	& % beide belebt verschieden
	& 1 % belebt verschieden beidiu
	& %==
	& % unbelebt gleich beide
	& % unbelebt gleich beidiu
	& %--
	& % unbelebt verschieden beide
	& % unbelebt verschieden beidiu
	& 1
	\\

\midrule

\mc{2}{l}{Summe}
	&  16 % belebt gleich beide
	&   1 % belebt gleich beidiu
	& %--
	&  17 % belebt verschieden beide
	&  48 % belebt verschieden beidiu
	& %==
	&   2 % unbelebt gleich beide
	&  28 % unbelebt gleich beidiu
	& %--
	&   0 % unbelebt verschieden beide
	&   7 % unbelebt verschieden beidiu
	& 119 % Summe
	\\

\lspbottomrule
\end{tabular}
\label{tab:caodist}
\end{sidewaystable}

Eine auffällige Abweichung\is{Ausnahme} besteht bei den Targets mit
belebten\is{Animata} Erstcontrollern von verschiedenem Geschlecht in der
Distanzklasse 2 (durchschnittlicher Abstand zwischen sechs und zehn
Wortformen). Nahezu drei Viertel der Belege für diesen Kontext zeigen die Form
\norm{bėide}. Im Vergleich zu demselben Kontext in anderen Distanzklassen liegt
also eine umgekehrte Verteilung der Belege vor. Bei allen diesen Belegen
bezieht sich \norm{bėide} auf ein Paar aus Mann und Frau und steht direkt
hinter oder kurz nach einem \isi{Personalpronomen} in einem anderen Teilsatz
als seine Erstcontroller. Prinzipiell unterscheiden sich die Belege jedoch
nicht von anderen bisher besprochenen. Ein regionaler Cluster ist auch nicht
erkennbar~-- die Belege stammen sowohl aus dem mittelbairischen als auch aus
dem hoch- und niederalemannischen Sprachraum\is{Dialektgeografie}.

Ein Anstieg von \norm{bėide}-Targets mit belebter\is{Animata} Referenz bei
verschiedenem Geschlecht der Erstcontroller ist auch in der Distanzklasse 10
(durchschnitlicher Abstand zwischen 126 und 250 Wortformen) zu beobachten. Die
Form \norm{bėide} macht hier nahezu die Hälfte der Belege für diesen Kontext
aus. Bei niedrigeren Abständen kommen dagegen nur vereinzelt Targets vom Typ
\norm{bėide} vor. Die Verteilung erweckt den Eindruck, dass Variation zwischen
\norm{bėide} und \norm{bėidiu} besonders bei eher nahen Targets mit sechs bis
zehn Wort\-formen Abstand herrscht, wobei \norm{bėidiu} ansonsten durchweg die
Hauptform darstellt. Der plötzliche Anstieg der \norm{bėide}-Form bei Targets
mit verschiedenem belebten\is{Animata} Geschlecht im oberen Abstandsbereich
wird als Zufall gewertet.

\is{Erstcontroller|)}
\is{Controller|)}
\is{Distanz!lineare|)}
\is{Distanz|)}

\subsection{Zusammenfassung und Vergleich}

Die Belege für \norm{bėide} in direkter Abhängigkeit von Pronomina nach
Distanz\is{Distanz!lineare} zeigen, dass der Großteil der Targets in
unmittelbarer Entfernung zu seinem \isi{Controller} steht: direkt dahinter und
damit in der Regel im gleichen Satzglied\is{Distanz!syntaktische}
\autocite[vgl.\ auch][625--626]{ksw2}. Der Abstand zwischen Controller und
Target scheint keine Auswirkungen auf die Form des Targets zu haben; vielmehr
variieren die Targets durchweg in Bezug auf die
Personenmerkmale\is{Personenmerkmal} ihres Controllers beziehungsweise ihrer
Controller. Auch in Hinblick auf den Wortformenabstand zwischen
Erstcontrollern\is{Erstcontroller} und \norm{bėide}-Targets ändert sich daran
prinzipiell nichts. Damit gelten die Schlüsse, die in
\sectref{sec:caotargpers} bezüglich des Einflusses der Kombination von
Personenmerkmalen auf die Form von \norm{bėide} gezogen wurden, auch unter
Berücksichtigung von Wortformenabstand und syntaktischer Distanz.

Auffällig ist allerdings, dass einerseits \norm{bėide} mit Bezug auf die
Kombination zweier belebter\is{Animata} \isi{Erstcontroller} vom gleichen
Geschlecht ($\SM+\SM{}$ und $\SF+\SF$) und andererseits \norm{bėidiu} mit Bezug
auf die Kombination zweier belebter Erstcontroller mit unterschiedlichem
Geschlecht ($\SM+\SF$, $\SF+\SM$) unabhängig von der \isi{Distanz} dominant
ist. Dasselbe gilt für \norm{bėidiu} in Bezug auf die Kombination
unbelebter\is{Inanimata} \isi{Controller} unabhängig von deren
\isi{Genus}. Die Vermutung, dass mit wachsendem Abstand\is{Distanz!lineare} die
spezifische Information \emph{Menschen von unterschiedlichem Geschlecht}, die
\isi{Genusresolution} bei Targets hin zum Neutrum auslöst, von der
unspezifischeren Information \emph{Menschen} verdrängt wird, hat sich im
untersuchten Material des \CAO{} nicht bestätigt\is{Validierung}. Es ließ sich
also nicht beobachten, dass \norm{bėide} \wdef{beide (\textsc{m+f})} mit
wachsender Distanz zwischen Controller und Target gegenüber \norm{bėidiu}
\wdef{beide (\textsc{n})} dominiert.

Es lässt sich trotzdem feststellen, dass im Nahbereich zwischen
durchschnittlich sechs bis zehn Wortformen Abstand\is{Distanz!lineare} zwischen
Erstcontrollern\is{Erstcontroller} und dem jeweils untersuchten Target und
wieder im Fernbereich der Klasse 10 (Abstand zwischen 126 und 250 Wortformen)
die Form \norm{bėide} verstärkt in Bezug auf belebte\is{Animata} Erstcontroller
mit verschiedenem Geschlecht auftritt. In allen diesen Fällen ist \norm{bėide}
indirekt über ein oder mehrere anaphorische Ausdrücke\is{Anapher} (Pronomina
wie \norm{ich, wir, uns}) mit der letzten vollständigen Nennung seiner
\isi{Controller} verbunden.

Auch wenn Urkunden\is{Urkunde} mündlich\is{Mündlichkeit} durchgeführte
Verhandlungen dokumentieren und sich vermuten lässt, dass sie gemäß der Formel
\norm{hȫren lęsen} \wdef{lesen hören} darauf ausgelegt sind, mündlich verlesen
zu werden \autocites[595]{schmidtwiegand1998b}[31]{schulze2011}, könnte man den
Grund dafür in konzeptueller \isi{Schriftlichkeit} suchen.
\citet[588--589]{frenz1998b} zufolge kam es vor, dass vor Abfassung einer
Reinschrift ein Textentwurf den Urkundenparteien zur Zustimmung vorgelegt
wurde. Geübte Schreiberinnen und Schreiber waren sich bei der Abfassung des
Texts vermutlich durchgängig um die Zusammensetzung der urkundenden Partei
bewusst und setzten die Kongruenzformen dementsprechend.

\is{Target|)}

%%%%%%%%%%%%%%%%%%%%%%%%%%%%%%%%%%%%%%%%%%%%%%%%%%%%%%%%%%%%%%%%%%%%%%%%%%%%%%%

\section{\norm{Bėide} als Konjunktion}
\label{sec:caokonjunktion}
\is{Konjunktion|(}

Für die Verwendung von \norm{bėide} \wdef{beide} als Konjunktion ließen sich
157 Belege im \CAO{}-Material finden. Davon schieden dreißig Belege aus, die
westmitteldeutschen\il{Westmitteldeutsch} Urkunden entstammen, da in diesem
Sprachraum\is{Dialektgeografie} allein der Typ \norm{bėide} existiert. Analog
zur Auswertung in \sectref{sec:adjdeclcao} zur \isi{Adjektivdeklination} im
\CAO{} gibt \tabref{tab:caobeidefuncvar} die Belegmengen in der Region
ausgewählter Ausstellungs\-orte nach Funktion von \norm{bėide} getrennt wieder.
Beim Quantor wurden zum Vergleich nur Kontexte mit zwei
Controllern\is{Controller} (direkter und indirekter Bezug) berücksichtigt.

\begin{table}
\centering
\caption{Form von \norm{beide} und \norm{bėidiu} an exemplarischen
Ausstellungsorten nach Funktion}
\begin{tabular}[t]{
	l
	r r
	r r
	r
	c
}
\lsptoprule

\mr[c]{2}{*}[-.5ex]{Region}
	& \mc{2}{c}{Quantor}
	& \mc{2}{c}{Konjunktion}
	& \mr[c]{2}{*}[-.5ex]{Summe}
	& \mr[c]{2}{*}[-.5ex]{\makecell[c]{unterschiedl.\\ Verhalten}}
	\\

\cmidrule(rl){2-3}
\cmidrule(rl){4-5}

%
	& \textit{bėid(e)}
	& \textit{bėidiu}
	& \textit{bėid(e)}
	& \textit{bėidiu}
	& %
	& %
	\\

\midrule

Straßburg
	&  9
	&  2
	& 13
	&  1
	& 25
	& ---
	\\

Basel
	&  4
	&  6
	&  2
	&  3
	& 15
	& ---
	\\

Zürich
	&  8
	&  4
	&  1
	&  2
	& 15
	& ---
	\\

\midrule

Konstanz
	&  5
	&  2
	&  1
	&  7
	& 15
	& \chk
	\\

Ulm
	&  1
	& %
	& %
	& %
	&  1
	& ?
	\\

Augsburg
	& %
	& 25
	& %
	&  3
	& 28
	& ---
	\\

\midrule

Nürnberg
	& %
	&  5
	&  2
	&  4
	& 11
	& ---
	\\

Regensburg
	&  1
	&  5
	& %
	&  3
	&  9
	& ---
	\\

München
	& %
	&  1
	& %
	&  1
	&  2
	& ?
	\\

Salzburg
	&  4
	&  5
	& %
	&  2
	& 11
	& \chk
	\\

Wien
	& 10
	&  6
	&  1
	&  7
	& 24
	& \chk
	\\

\midrule

Summe
	&  42
	&  61
	&  20
	&  33
	& 156
	\\

\lspbottomrule
\end{tabular}
\label{tab:caobeidefuncvar}
\end{table}

Unterschiedliches Verhalten zwischen Quantor und Konjunktion ist nur in den
Regionen um Konstanz, Salzburg und Wien zu beobachten. Hier liegt beim Quantor
Variation zwischen \norm{bėide} und \norm{bėidiu} vor, während bei der
Konjunktion nahezu ausschließlich \norm{bėidiu} belegt ist. Keine Aussagen
können zu Ulm und München gemacht werden, da hier nur vereinzelte Belege
vorliegen. Für Augsburg, Nürnberg und Regensburg darf aus den Tabellendaten
nicht geschlossen werden, dass der Quantor hier generell zu \norm{bėidiu}
erstarrt ist~-- die Werte ergeben sich aus der Tatsache, dass für diese Orte
hauptsächlich Belege für Paare aus Mann und Frau sowie für Neutra
beziehungsweise \isi{Inanimata} vorliegen. Allein für Straßburg ist fast
durchgängig die Form \lit{bede} \wdef{beide} belegt. Der Aufstellung in
\citet[621, Abbildung P~177]{ksw2} ist zu entnehmen, dass diese Form im
Alemannischen\il{Alemannisch} ansonsten einen Sonderfall darstellt~-- sie
verzeichnen dafür in der zweiten Hälfte des 13.~Jahrhunderts 0\,\% und für die
erste Hälfte des 14.~Jahrhunderts lediglich 2\,\% bezogen auf die Gesamtzahl
der Belege für \norm{bēd-} \wdef{beid-} in ihrem \isi{Korpus}.

Zusätzlich zur obigen Tabelle stellen die
Abbildungen~\ref{fig:cao_beide_geofunctypes_quant}
und~\ref{fig:cao_beide_geofunctypes_conj} die geografische
Verteilung\is{Distribution!geografische} der \norm{bėide}- und
\norm{bėidiu}-Typen nach Verwendungskontext grafisch für alle in der
\isi{Stichprobe} vorhandenen Belegorte dar. Auffällig ist, dass besonders im
Nieder\-alemannischen\il{Alemannisch} bei der Konjunktion hauptsächlich
\norm{beide}-Formen gegenüber sowohl \norm{bėide} als auch \norm{bėidiu} beim
Quantor vertreten sind, sieht man von der geringeren Belegzahl insgesamt ab.

\begin{figure}
\includegraphics[
	trim=0 10mm 75mm 0, clip,
	width=\linewidth,
	keepaspectratio,
]{./figures/cao_beide_iu-quant.pdf}
\captionsetup{width=.85\textwidth}
\caption{Geografische Verteilung der Formentypen bei
	\norm{bėide} als Quantor (\norm{X, Y \dots\ (si) bėide};
	(Hintergrund nach \cite[Karte~47.4]{wiesinger1983})}
\label{fig:cao_beide_geofunctypes_quant}
\end{figure}

\begin{figure}
\includegraphics[
	trim=0 10mm 75mm 0, clip,
	width=\linewidth,
	keepaspectratio,
]{./figures/cao_beide_iu-conj.pdf}
\captionsetup{width=.85\textwidth}
\caption{Geografische Verteilung der Formentypen bei
	\norm{bėide} als Konjunktion (\norm{bėide X unde Y};
	(Hintergrund nach \cite[Karte~47.4]{wiesinger1983})}
\label{fig:cao_beide_geofunctypes_conj}
\end{figure}

Darüber hinaus liegt in je einem Fall Variation zwischen einer \isi{Urkunde}
und ihrer Kopie bei ansonsten gleichem Wortlaut \REF{ex:intraurkvar1} sowie
Variation innerhalb derselben Urkunde vor \REF{ex:intraurkvar2}. Beide Urkunden
stammen aus dem ostfränkischen\il{Ostfränkisch} Sprachraum\is{Dialektgeografie}
und illustrieren das dortige Neben\-einander beider Formen von konjunktionalem
\norm{bėide}, wie aus der Karte in \figref{fig:cao_beide_geofunctypes_conj}
deutlich wird.

\begin{exe}
\protectedex{
\ex \label{ex:intraurkvar1}
	\let\eachwordthree\eachwordtwo
	\let\eachwordtwo\eachwordone
	\glll A: daz wir deſte baz beide der ſtat zu Wirceburg vnd deme
			lande den Vride gevurdern mugen \\
		B: daz wir deſte baz beidu der ſtat zu Wirceburg/ Vnd dem lande
			den fride gevurdern mugen \\
		{} dass wir desto besser beide der Stadt zu Würzburg und dem Land
			den Frieden fördern können \\
	\trans \wdef{damit wir den Frieden sowohl in der Stadt Würzburg als
		auch im Land umso besser fördern können}
		\parencites(Nrn.~1126~AB, Würzburg, 1289)[414,36--39]{cao2}
}

\ex \label{ex:intraurkvar2}
	\begin{xlist}
	\ex \label{ex:intraurkvar2_1}
		\gll aller anſprache / immerewiclichen gelazen beidiv Ledic
				vnde vri \\
			aller Anklage {} immer.ewiglich gelassen beide ledig und frei \\
		\trans \wdef{für immer und ewig aller Anklage sowohl ledig als auch frei
			gelassen}
			\parencites(Nr.~2293, Bamberg, 1295)[420,23]{cao3}

	\ex \label{ex:intraurkvar2_2}
		\gll vnſer zinsgelt beide klein vnde groz \\
			unser Zinsgeld beide klein und groß \\
		\trans \wdef{unser Zinsgeld, sowohl kleines als auch großes}
			\parencites(Nr.~2293, Bamberg, 1295)[420,30]{cao3}
	\end{xlist}
\end{exe}

\subsection{Mit zwei Controllern}
\label{subsec:caokonj2ctrl}
\is{Controller|(}

Mit 59 Belegen ist die Koordination von zwei nominalen Elementen der häufigste
Anwendungsfall der Konstruktion \norm{bėide \dots\ unde} \wdef{sowohl \dots\ als
auch} im untersuchten Material. Ein Beispiel für die hier diskutierten
Verwendungsweisen wird in \REF{ex:caokonj2ctrl} gegeben. Als Elemente, die
koordiniert werden, dienen einzelne Substantive\is{Substantiv}
\REF{ex:caokonj2ctrl_1}, Pronomina\is{Personalpronomen} und Namen
\REF{ex:caokonj2ctrl_2}, komplexe Nominalphasen\is{Nominalphrase}
\REF{ex:caokonj2ctrl_3} oder auch freie Relativsätze\is{Relativsatz}
\REF{ex:caokonj2ctrl_4}. Gemeinsam ist den hier untersuchten Konjunkten, dass
sie Controller enthalten.

\begin{exe}
\ex \label{ex:caokonj2ctrl}
	\begin{xlist}
	\ex \label{ex:caokonj2ctrl_1}
		\gll beidv̍ reben vn̄ garten \\
			beide Reben und Garten \\
		\trans \wdef{sowohl Reben als auch Garten}
			\parencites(Nr.~2353, Basel, 1296)[462,28--29]{cao3}

	\ex \label{ex:caokonj2ctrl_2}
		\gll beidiv ich / Vnd hainrich von meringen \\
			beide ich {} und Heinrich von Meringen \\
		\trans \wdef{sowohl ich als auch Heinrich von Meringen}
			\parencites(Nr.~1347, Kl.~Steingaden, Kr.~Weilheim-Schongau, 1291)[578,25]{cao1}

	\ex \label{ex:caokonj2ctrl_3}
		\gll beidv er / vn̄ oͮch alle die / den er ſvͥ gît \\
			beide er {} und auch alle die {} denen er sie gibt \\
		\trans \wdef{sowohl er als auch alle die, denen er sie gibt}
			\parencites(Nr.~1566, Hüfingen, Schwarzwald-Baar-Kr., 1292)[717,18]{cao2}

	\ex \label{ex:caokonj2ctrl_4}
		\gll bediv di nv lebent · vn̄ hernahe chvmftige ſint \\
			beide die nun leben {} und hernach künftig sind \\
		\trans \wdef{sowohl die nun leben als auch hernach künftig sind}
			\parencites(Nr.~1352, Wien, 1291)[580,8]{cao2}
	\end{xlist}
\end{exe}

\tabref{tab:caokoordnomctrl} listet die Belegzahlen für \norm{bėide} und
\norm{bėidiu} in Abhängigkeit von den Personenmerkmalen\is{Personenmerkmal} der
Konjunkte auf. Es ist zwar mit \citet[626]{ksw2} davon auszugehen, dass die
koordinierten Elemente ursprünglich appositiv\is{Apposition} zu einem
pronominal gebrauchten \norm{bėide} verwendet wurden (\norm{bėide, X unde Y}).
Aus der Belegverteilung im Urkundenmaterial ist jedoch nicht per se zu
schließen, dass konjunktional gebrauchtes \norm{bėide} bei der Koordination von
nominalen Elementen noch regelmäßig kongruiert.

\begin{table}[tp]
\centering
\caption{Form nach Personenmerkmalen nominaler Konjunkte}
\begin{tabular}{>{\scshape}l >{\scshape}l r r r}
\lsptoprule
\normalfont Konjunkt 1
	& \normalfont Konjunkt 2
	& \norm{bėid(e)}
	& \norm{bėidiu}
	& Summe
	\\
\midrule

% Gleiches Geschlecht
1sg\subM   & 1pl\subM   &    &  1 &  1 \\
1sg\subM   & 3sg.\MascM &  3 &  1 &  4 \\
3sg.\MascM & 3sg.\MascM &  3 &  3 &  6 \\
3sg.\MascM & 3pl.\MascM &    &  1 &  1 \\
3sg.\FemF  & 3sg.\FemF  &    &  1 &  1 \\
%                          6    7   13

\midrule

% Unterschiedliches Geschlecht
1sg\subM   & 3sg.\FemF  &    &  1 &  1 \\
1sg\subF   & 3sg.\MascM &  1 &    &  1 \\
3sg.\MascM & 3sg.\FemF  &    &  1 &  1 \\
3pl.\FemF  & 3pl.\MascM &    &  2 &  2 \\
%                          1    4    5

\midrule

% Mit gleichem Genus und egalem Geschlecht
1sg\subM   & 3pl.\MascA &  3 &    &  3 \\
3sg.\MascA & 3sg.\MascA &  1 &  1 &  2 \\
3sg.\MascA & 3pl.\MascA &    &  1 &  1 \\
3sg.\MascM & 3pl.\MascA &    &  1 &  1 \\
3pl.\MascA & 3pl.\MascA &  9 &  5 & 14 \\
3pl.\MascM & 3pl.\MascA &  1 &    &  1 \\
3sg.\MascM & 3pl.\NeutA &  1 &    &  1 \\
%                         15    8   23

\midrule

% Mit unterschiedlichem Genus und egalem Geschlecht oder unbelebt
3sg.\NeutA & 3pl.\MascA &    &  1 &  1 \\
3sg.\MascM & 3sg.\NeutI &    &  1 &  1 \\
3pl.\MascA & 3pl.\NeutI &  1 &  5 &  6 \\
%                          1    7    8

\midrule

% Gleiches Genus (unbelebt)
3sg.\FemI  & 3sg.\FemI  &    &  1 &  1 \\
3sg.\NeutI & 3sg.\NeutI &  1 &    &  1 \\
3sg.\MascI & 3pl.\MascI &  1 &    &  1 \\
%                          2    1    3

\midrule

% Unterschiedliches Genus (unbelebt)
3sg.\MascI & 3sg.\NeutI &    &  1 &  1 \\
3sg.\FemI  & 3sg.\NeutI &  1 &  1 &  2 \\
3sg.\NeutI & 3sg.\FemI  &    &  3 &  3 \\
3pl.\MascI & 3pl.\FemI  &  2 &    &  2 \\
3pl.\MascI & 3sg.\NeutI &  1 &    &  1 \\
3pl.\FemI  & 3sg.\MascI &    &  1 &  1 \\
3pl.\FemI  & 3pl.\MascI &    &  1 &  1 \\
%                          4    7   11

\midrule
\mc{2}{l}{Summe}        & 29 & 34 & 63 \\
\lspbottomrule
\end{tabular}
\label{tab:caokoordnomctrl}
\end{table}

Der erste Abschnitt in \tabref{tab:caokoordnomctrl} enthält Konjunkte mit
übereinstimmenden \isi{Genus}-\is{Genusmerkmal} und
Sexus\-merkmalen\is{Sexusmerkmal} ($\MascM{} + \MascM{}$ und $\FemF{} +
\FemF$). Anders als bisher beobachtet, steht in 54\,\% der Fälle die ansonsten
neutrale Form \norm{bėidiu}. Im dritten Abschnitt, der mindestens ein Element
mit unspezifischem \isi{Sexus} enthält, das formal jedoch maskulin ist (\MascA,
zum Beispiel \norm{lǖte} \wdef{Leute}, \norm{burgǟre} \wdef{Bürger}), und
einem weiteren solchen oder einem maskulin-männlichen Element ($\MascA{} +
\MascA{}$ und $\MascM{} + \MascA$) häufen sich dagegen die Belege bei
\norm{bėid(e)}, das 65\,\% der für diesen Kontext belegten Formen ausmacht. Im
letzten Abschnitt der Tabelle, der unbelebte\is{Inanimata} Elemente mit
unterschiedlichem Genus enthält, wird anders als zuvor \norm{bėidiu} nicht klar
favorisiert: Mit 58\,\% für \norm{bėidiu} sind die Verhältnisse nahezu
ausgeglichen. Insgesamt ergibt sich keine klare, von
Personenmerkmalen\is{Personenmerkmal} abhängige Verteilung.

Für den zweiten Abschnitt mit unterschiedlichen
Geschlechtsmerkmalen\is{Sexusmerkmal} belebter\is{Animata} Konjunkte scheint
eine klare Präferenz für \norm{bėidiu} vorzuliegen, die ähnlich eindeutig zu
sein scheint wie bei der Verwendung von \norm{bėide} als Quantor. Dasselbe gilt
für die vierte Gruppe, die verschiedene Kombinationen mit unspezifischem
Geschlecht sowie von belebten mit unbelebten\is{Inanimata} Konjunkten enthält.
Sollte dies als Evidenz für \isi{Genusresolution} dienen?
\citet[187]{gjelsten1980} räumt in ihrer Kritik an \posscite{askedal1974}
Arbeit zu konjunktionalem \norm{bėide} ein, dass bei
substantivischen\is{Substantiv} Konjunkten nicht klar zwischen pronominalem
Gebrauch (\norm{bėide,~X unde~Y}) und konjunktionalem (\norm{bėide~X unde~Y})
unterschieden werden kann. Angesichts der recht homogenen Verteilung der Formen
\norm{bėide} und \norm{bėidiu} ist aber nicht davon auszugehen, dass
ausgerechnet hier plötzlich regelmäßig Kongruenz zwischen \norm{bėide} und den
Konjunkten herrscht.

\tabref{tab:caokoordnomctrl} trennt nicht nach \isi{Kasus}; die folgende
\tabref{tab:caokoordnomctrlcase} stellt die Verteilung zumindest der
Kombinationen von Konjunkten dar, die eine klare Zuordnung des Geschlechts
erlaubt. Belege mit Gruppenbezeichnungen wie \norm{lǖte} \wdef{Leute
(\MascA)}, \norm{arme} \wdef{Arme (\MascA)} und \norm{alle die, den ęr si
gibet} \wdef{alle die (\MascA?), denen er sie gibt} sowie Kombinationen von
belebten\is{Animata} und unbelebten Konjunkten wie in \norm{lǖte unde guet}
\wdef{Leute (\MascA) und Gut (\NeutI)} wurden nicht gewertet. Bei allen diesen
Belegen ist ohnehin nur \norm{bėide} für den Nom., Akk.\ und den Dat.\ belegt;
für den Gen.\ liegen keine Belege vor.

\begin{sidewaystable}
\centering
\caption{Form nach dem Kasus nominaler Konjunkte}
\begin{tabular}{
	>{\scshape}l
	r r c r r
	c
	r r c r r
	r
}
\lsptoprule

%
	& \mc{5}{c}{belebt}
	& % --
	& \mc{5}{c}{unbelebt}
	& \mr{3}{*}[-1ex]{Summe}
	\\

\cmidrule{2-6}
\cmidrule{8-12}

%
	& \mc{2}{c}{gleich}
	& % --
	& \mc{2}{c}{verschieden}
	& % --
	& \mc{2}{c}{gleich}
	& % --
	& \mc{2}{c}{verschieden}
	& % --
	\\

\cmidrule{2-3}
\cmidrule{5-6}
\cmidrule{8-9}
\cmidrule{11-12}

%
	& \norm{bėid(e)}
	& \norm{bėidiu}
	& % --
	& \norm{bėid(e)}
	& \norm{bėidiu}
	& % --
	& \norm{bėid(e)}
	& \norm{bėidiu}
	& % --
	& \norm{bėid(e)}
	& \norm{bėidiu}
	& % --
	\\

\midrule

nom
	& 4
	& 4
	& % --
	& 1
	& 2
	& % --
	& %
	& %
	& % --
	& %
	& 3
	& 14
	\\

acc
	& 2
	& %
	& % --
	& %
	& %
	& % --
	& 3
	& %
	& % --
	& 2
	& 3
	& 10
	\\

\midrule

dat
	& %
	& 2
	& % --
	& %
	& 1
	& % --
	& %
	& %
	& % --
	& 1
	& 1
	& 5
	\\

gen
	& %
	& 1
	& % --
	& %
	& 1
	& % --
	& %
	& %
	& % --
	& 1
	& %
	& 3
	\\

\midrule

\normalfont Summe
	& 6
	& 7
	& % --
	& 1
	& 4
	& % --
	& 3
	& %
	& % --
	& 4
	& 7
	& 32
	\\

\lspbottomrule
\end{tabular}
\label{tab:caokoordnomctrlcase}
\end{sidewaystable}

Bei der Betrachtung nach \isi{Kasus} ergibt sich für den Nom./Akk.\ keine klare
Präferenz bei belebten\is{Animata} Konjunkten mit gleichem Geschlecht, da hier
die Verhältnisse nahezu ausgeglichen sind. Für Kombinationen mit verschiedenem
belebten Geschlecht liegen zu wenige Belege vor, um eine Regel aufzustellen.
Dasselbe gilt für die drei Belege für \norm{bėide} bei gleichem
unbelebten\is{Inanimata} \isi{Genus}. Bei verschiedenem unbelebten Genus
entfällt der Großteil der Belege auf \norm{bėidiu}. Daneben liegen auch Belege
für \norm{bėide} und \norm{bėidiu} im Dativ und Genitiv vor. Wenn \norm{bėide}
entsprechend der formalen Merkmale\is{Kongruenz!formale} der jeweiligen NP
flektieren würde, dürften diese Formen nicht auftreten. Abgesehen von je einem
Beleg für \norm{bėide} bei verschiedenem unbelebten Genus steht in den
vereinzelten belegten Fällen \norm{bėidiu}.

Insgesamt ist keine deutliche Häufung auf \norm{bėide} bei gleichem
belebten\is{Animata} Geschlecht sowie auf \norm{bėidiu} bei
unbelebtem\is{Inanimata} Bezug unabhängig vom \isi{Genus} auszumachen, wie dies
bei der quantifizierenden Verwendung der Fall war. Angesichts der vereinzelten
Belege in den einzelnen Kasusfeldern\is{Kasus} kann wohl auch für belebte
Konjunkte mit verschiedenem Geschlecht keine Regel für das Auftreten von
\norm{bėidiu} abgeleitet werden. Wenn sich Belege mit pronominalem \norm{bėide}
eingeschlichen haben sollten, treten diese nicht klar aus der Auszählung
hervor.

\is{Controller|)}

\subsection{Mit zwei Targets}
\label{subsec:caobeidkoordtarg}
\is{Target|(}

Parallel zur Auswertung in \sectref{subsubsec:beid2p2coordncao} zum Einfluss
der Personenmerkmale\is{Personenmerkmal} auf die Form von \norm{bėide} beim
indirekten Bezug auf kombinierte nominale Controller lassen sich auch für die
Konjunktion \norm{bėide} solche Konjunkte untersuchen, die selbst keine
Personenmerkmale definieren, sondern Kongruenz\-targets darstellen. In der
Stichprobe sind dies konkret verschiedene Arten von Adjektiven, die in
\REF{ex:caobeidkoordtarg} beispielhaft angeführt sind: vorangestellt
attributive\is{Adjektiv!attributiv} \REF{ex:caobeidkoordtarg_1}, nachgestellt
attributive \REF{ex:caobeidkoordtarg_2} und prädikative
\REF{ex:caobeidkoordtarg_3}\is{Adjektiv!prädikativ}, wobei die Zuordung bei
Nachstellung nicht immer eindeutig ist.\is{Ambiguität}

\begin{exe}
\ex \label{ex:caobeidkoordtarg}
	\begin{xlist}
	\ex \label{ex:caobeidkoordtarg_1}
		\gll beide geiſtliches / vn̄ wertliches gerihtes \\
			beide geistlich-\textsc{gen.sg.\NeutI.st} {} und
			weltlich-\textsc{gen.sg.\NeutI.st} Gericht-\textsc{gen.sg.\NeutI} \\
		\trans \wdef{sowohl geistlichen als auch weltlichen Gerichts}
			\parencites(Nr.~1764, Bamberg, 1293)[71,26]{cao3}

	\ex \label{ex:caobeidkoordtarg_2}
		\gll erberge leut peideu gaiſtlich vnd wertlich \\
			ehrenhafte Leute[\textsc{nom.pl.\MascA}] beide geistlich[\textsc{nom.pl.\MascA}]
			und weltlich[\textsc{nom.pl.\MascA}] \\
		\trans \wdef{sowohl geistliche und weltliche ehrenhafte Leute}
			\parencites(Nr.~1153, Engelthal, Kr.~Nürnberger Land, 1289)[431,44]{cao2}

	\ex \label{ex:caobeidkoordtarg_3}
		\gll vnde hat {div ſelben} gvͦt \textelp{} aller anſprache /
			immerewiclichen gelazen beidiv Ledic vnde vri \\
			und hat dieselben Gut[\textsc{acc.pl.\NeutI}] {} aller Anklage {}
			immer.und.ewig gelassen beide ledig[\textsc{acc.pl.\NeutI}] und
			frei[\textsc{acc.pl.\NeutI}] \\
		\trans \wdef{und hat dieselben Güter \textelp{} immer und ewig sowohl
			ledig als auch frei von aller Anklage gelassen}
			\parencites(Nr.~2293, Bamberg, 1295)[420,21--23]{cao3}
	\end{xlist}
\end{exe}

Auch für die Koordination von Adjektiven lässt sich keine regelmäßige
Korrespondenz zwischen der Kombination bestimmter
Personenmerkmale\is{Personenmerkmal} und konjunktional gebrauchtem \norm{bėide}
feststellen, wie \tabref{tab:caokoordtarg} zeigt. Für Konjunkte mit gleichen
Genus-\is{Genusmerkmal} und Sexusmerkmalen\is{Sexusmerkmal} liegt nur ein
einziger Beleg vor; bei der Konjunktion von maskulinen Targets mit beliebigem
\isi{Sexus} sind die Verhältnisse nahezu ausgeglichen, insofern in 57\,\% der
Fälle \norm{bėide} belegt ist. Bei den unbelebten\is{Inanimata} Targets mit
unterschiedlichem \isi{Genus} steht in 64\,\% der Belege \norm{bėidiu}.

\begin{table}
\centering
\caption{Form nach Personenmerkmalen adjektivischer Konjunkte}
\begin{tabular}{>{\scshape}l >{\scshape}l r r r}
\lsptoprule
\normalfont Konjunkt 1
	& \normalfont Konjunkt 2
	& \norm{bėid(e)}
	& \norm{bėidiu}
	& Summe
	\\
\midrule

% Gleiches Geschlecht
sg.\FemF        & sg.\FemF  &  1 &    &  1 \\

% Mit gleichem Genus und egalem Geschlecht
pl.\MascA       & pl.\MascA &  4 &  3 &  7 \\

\midrule

% Gleiches Genus (unbelebt)
sg.\MascI       & sg.\MascI &  1 &  3 &  4 \\
sg.\FemI        & sg.\FemI  &    &  1 &  1 \\
sg.\NeutI       & sg.\NeutI &  4 &  1 &  5 \\
pl.\NeutI       & pl.\NeutI &    &  1 &  1 \\
%                                5    6   11

\midrule
\mc{2}{l}{Summe}              & 10 &  9 & 19 \\
\lspbottomrule
\end{tabular}
\label{tab:caokoordtarg}
\end{table}

\is{Target|)}

\subsection{Rein syntaktischer Kontext}
\label{subsec:caobeidquantsyncont}

Im Urkundenmaterial kommt neben der Konjunktion von Controllern und Targets,
die Personenmerkmale definieren beziehungsweise widerspiegeln, auch die
Konjunktion solcher Elemente vor, die keinerlei Personenmerkmale definieren.
Die Konstruktion \norm{bėide \dots\ unde} \wdef{sowohl
\dots\ als auch} hat hier insofern funktionalen Charakter, als
\norm{bėide} als \isi{Fokuspartikel} dient, die die Zweiheit der Optionen
betont \autocites(siehe auch
\sectref{sec:ovwbeideconj})[425--428]{johannessen2005}. Im exzerpierten
Material kommt \norm{bėide} in diesem Zusammenhang mit
Temporaladverbien\is{Adverb} \REF{ex:caokoordsyn_1}, Lokaladverbien
\REF{ex:caokoordsyn_2} und Präpositionalphrasen\is{Präpositionalphrase}
(\ref{ex:caokoordsyn_2}--c) vor.

\begin{exe}
\ex \label{ex:caokoordsyn}
	\begin{xlist}
	\ex \label{ex:caokoordsyn_1}
		\gll beide vor vn̄ noͤch \\
			beide vor und nach \\
		\trans \wdef{sowohl davor als auch danach}
			\parencites(Nr.~N~689, Straßburg, 1295)[499,25]{cao5}

	\ex \label{ex:caokoordsyn_2}
		\gll beide da vn̄ an allin ſteitin \\
			beide da und an allen Stätten \\
		\trans \wdef{sowohl dort als auch an allen \textins{anderen} Orten}
			\parencites(Nr.~N~321, Rosheim, Dépt.~Bas-Rhin, 1286)[245,24]{cao5}

	\ex \label{ex:caokoordsyn_3}
		\gll baidev zv Dorfe vnd ze velde \\
			beide zu Dorf-\textsc{dat.sg} und zu Feld-\textsc{dat.sg} \\
		\trans \wdef{sowohl im Dorf als auch auf dem Feld}
			\parencites(Nr.~3319, Michelstetten, Bz.~Mistelbach, 1299)[461,28]{cao4}
	\end{xlist}
\end{exe}

In \tabref{tab:caokoordsyn} werden der Vollständigkeit halber die Belegzahlen
pro Kombination der Wortart oder des Phrasentyps der Konjunkte aufgelistet.
Eine auffällige Häufung, die Anlass dazu geben würde, eine Präferenz in
Abhängigkeit von den Konjunkte anzunehmen, ist auch hier nicht gegeben. Mit
Ausnahme von zwei bairisch-österreichischen\il{Bairisch} Belegen aus Fallbach
(Bz.~Mistelbach) und eines aus St.~Paul im Lavanttal (Bz.~Wolfsberg) stammen
die Belege für die Form \norm{bėide} sämtlich aus dem
alemannischen\il{Alemannisch} Sprachraum\is{Dialektgeografie}. Die Belege für
\norm{bėidiu} stammen umgekehrt hauptsächlich aus dem bairischen\il{Bairisch}
Sprachraum.

\begin{table}
\centering
\caption{Form in Abhängigkeit von adverbialen und präpositio\-nalen Konjunkten}
\begin{tabular}{l l r r r}
\lsptoprule
Konjunkt 1
	& Konjunkt 2
	& \norm{bėid(e)}
	& \norm{bėidiu}
	& Summe
	\\
\midrule

AP      & AP      &  2 &  1 &  3 \\

\midrule

PP      & AP      &  2 &    &  2 \\
AP      & PP      &  1 &    &  1 \\
%                    3    0    3

\midrule

PP      & PP      & 10 & 27 & 37 \\

\midrule
\mc{2}{l}{Summe}  & 15 & 28 & 43 \\
\lspbottomrule
\end{tabular}
\label{tab:caokoordsyn}
\end{table}

\is{Konjunktion|)}

\subsection{Zusammenfassung}

Die aus dem \CAO{} exzerpierten Belege zu \norm{bėide} als \isi{Konjunktion}
stützen\is{Validierung} die in \citet[626--627]{ksw2} formulierte
Beobachtung, dass die Form in diesem Kontext nicht mehr nach grammatischen
Kriterien variiert, sondern \isi{freie Variation} zwischen \norm{bėide} und
\norm{bėidiu} vorliegt. Weder mit koordinierten Controllern\is{Controller} noch
mit koordinierten\is{Koordination} Targets\is{Target} wird eine eindeutige
Abhängigkeit der Form der Konjunktion von den
Personenmerkmalen\is{Personenmerkmal} der Konjunkte deutlich. Dass der Aspekt
von \norm{bėide} als \isi{Fokuspartikel} \autocites(siehe auch
\sectref{sec:ovwbeideconj})[425--428]{johannessen2005} in diesem Kontext
vorherrscht, ist auch daran zu erkennen, dass \norm{bėide} regelmäßig mit
solchen Konjunkten auftritt, die keinerlei Personenmerkmale definieren oder
reflektieren, sodass kein Kontext vorliegt, in dem Kongruenz angewendet werden
kann (\sectref{sec:beideconj}).

Auch die geografische Verteilung\is{Distribution!geografische} betreffend
decken sich\is{Validierung} die \CAO{}-Belege mit den Aussagen in
\citet[627--628]{ksw2}. Belege für den Typ \norm{bėide} verteilen sich vor
allem auf das \mbox{(Nieder-)}\allowbreak{}Alemannische\il{Alemannisch} und
Ostfränkische\il{Ostfränkisch}, im bairischen\il{Bairisch} und
schwäbischen\il{Schwäbisch} Sprachraum\is{Dialektgeografie} steht dagegen
hauptsächlich \norm{bėidiu}. Ein Unterschied zwischen \norm{-e} und \norm{-iu}
anhand der syntaktischen Funktion von \norm{bėide} macht sich nur insofern
bemerkbar, als die Form bei konjunktionalem\is{Konjunktion} Gebrauch in den
Belegen zu Straßburg, Konstanz, Salzburg und Wien deutlich erstarrt erscheint,
während beim Quantor zumindest an den drei zuletzt genannten Orten beide
Flexionsformen auftreten.
