\documentclass[output=paper]{langscibook} 
\ChapterDOI{10.5281/zenodo.10497369}

\title{Translanguaging in Bulgarian street signs} 
\author{Emilia Slavova \orcid{0000-0003-3993-4604} \affiliation{St. Kliment Ohridski University of Sofia}}

\abstract{Translanguaging has become a popular concept in sociolinguistic studies. It marks a shift from thinking of languages as bounded and homogeneous systems, closely linked to national identities, towards perceiving them as mobile resources which are in constant contact with other languages and are combined creatively to create new meanings. One area where translanguaging can easily be observed is urban spaces, where people from diverse linguistic and cultural backgrounds meet. Bulgarian street culture is an interesting case in point. After a long period of isolation and repression behind the Iron Curtain, Bulgarian street culture has flourished in recent years. The monolingual paradigm, strictly observed in the past, has been replaced by the multilingual and translanguaging trend. This has given rise to street signs (shop windows, café, and restaurant names, etc.) in multiple languages and combinations of languages, used in creative ways not only to address an international customer base, but also to appeal to local citizens with an open mindset, multilingual repertoires, and a global, transcultural identity. Using a linguistic landscape approach, this chapter explores the street culture and naming practices in central Sofia as the city negotiates between a local and a global identity.}

%TODO: Keywords erlaubt?

\IfFileExists{../localcommands.tex}{
  \addbibresource{../localbibliography.bib}
  % add all extra packages you need to load to this file

\usepackage{tabularx,multicol}
\usepackage{url}
\urlstyle{same}

\usepackage{listings}
\lstset{basicstyle=\ttfamily,tabsize=2,breaklines=true}

\usepackage{langsci-basic}
\usepackage{langsci-optional}
\usepackage{langsci-lgr}
\usepackage{langsci-osl}
% \usepackage{./langsci/styles/langsci-lgr}
% \usepackage{./langsci/styles/langsci-osl}
% \usepackage{langsci-gb4e}

\usepackage{tikz}
\usetikzlibrary{patterns,calc}
\pgfdeclarepatternformonly{south east lines}{\pgfqpoint{-0pt}{-0pt}}{\pgfqpoint{3pt}{3pt}}{\pgfqpoint{3pt}{3pt}}{
    \pgfsetlinewidth{0.6pt}
    \pgfpathmoveto{\pgfqpoint{0pt}{3pt}}
    \pgfpathlineto{\pgfqpoint{3pt}{0pt}}
    \pgfpathmoveto{\pgfqpoint{.2pt}{-.2pt}}
    \pgfpathlineto{\pgfqpoint{-.2pt}{.2pt}}
    \pgfpathmoveto{\pgfqpoint{3.2pt}{2.8pt}}
    \pgfpathlineto{\pgfqpoint{2.8pt}{3.2pt}}
    \pgfusepath{stroke}}
    
\usepackage{stmaryrd}
\usepackage{wasysym}
\usepackage{multirow}
\usepackage{caption}
\usepackage{subcaption}
\usepackage{mathrsfs}
\usepackage{qtree}

\usepackage{linguex}


  %pminos do not split footnotes
% \interfootnotelinepenalty=10000 %Footnote in Laporte chapters has to be split SN


%\DeclareIndexNameFormat{default}{%
%\nameparts{#1}%
%\usebibmacro{index:name}%
%{\index[names]}%
%{\namepartfamily}%
%{\namepartgiveni}%
% {}% L1
% {}% L2
%{\namepartprefix}% generates spurious space L3
%{\namepartsuffix}% generates spurious space L4
%}

%  {\DeclareIndexNameFormat{default}{%
%     \usebibmacro{index:name}{\index[names]}{#1}{#3}{#5}{#7}}}

%\DeclareIndexNameFormat{default}{%
%  \usebibmacro{index:name}{\sindex[nom]}{#1}{#3}{#5}{#7}}

%\DeclareIndexNameFormat{default}{%
%  \usebibmacro{index:name}{\sindex[person]}{#1}{#3}{#5}{#7}}
%\DeclareIndexNameFormat{default}{%
%\nameparts{#1} \usebibmacro{index:name}{\sindex[person]]}{\namepartfamily}{‌​\namepartgiven}{\nam‌​epartprefix}{\namepa‌​rtsuffix}}

%\newcommand{\smiley}{:)}

%\renewbibmacro*{index:name}[5]{%
%\usebibmacro{index:entry}{#1}%
%{\iffieldundef{usera}{}{\thefield{usera}\actualoperator}\mkbibindexname{#2}{#3}{#4}{#5}}}

% \newcommand{\noop}[1]{}

%remove for final
%\overfullrule=1mm

\newcommand{\tobi}[2]}}
\renewcommand{\S}[1]{\tobi{#1}{\textsc{*}}}

% this volume references
% puts: [this volume]
% already defined: \citetv
%\newcommand{\citepv}[1]{(\citeauthor{#1} \citeyear*{#1} [this volume])}
\newcommand{\citealtv}[1]{\citeauthor{#1} \citeyear*{#1} [this volume]}

%parentheses around example number
\newcommand{\pref}[1]{(\ref{#1})}

% in-text examples

\newcommand{\lnex}[1]{\textit{#1}} %target lang word
\newcommand{\lnlit}[1]{(lit.: `#1')} %literal reading
\newcommand{\lnlat}[1]{(#1)} % latinization
\newcommand{\lntrans}[1]{`#1'} %translation
\newcommand{\lnexl}[2]%
{\lnex{#1}{} \lnlat{#2}} % ex with latinization
\newcommand{\lnexlat}[3]{\lnex{#1}{} \lnlat{#2}{} \lntrans{#3}} % ex with latinization and tranl.

%ch01
\newcommand{\co}[1]{\mbox{\textbf{#1}}}

%ch09

\newcommand{\cyrbulg}[1]{\begin{otherlanguage*}{bulgarian}#1\end{otherlanguage*}}


%ch10
\newcommand{\nlp}{{\small NLP}}
\newcommand{\mwe}{{\small MWE}}
\newcommand{\rae}{{\small RAE}}
\newcommand{\lvc}{{\small LVC}}
\newcommand{\pos}{{\small P}o{\small S}}
%\newcommand{\todo}[1]{ \textcolor{red}{#1} }

%\renewcommand{\labelenumi}{\theenumi}
%\ainamefmt{{vv}{ll}{, ff}{, jj}} % fullname

\newcommand{\biberror}[1]{{\color{red}#1}}

\newcommand{\osenovaitem}{--~} 
  %% hyphenation points for line breaks
%% Normally, automatic hyphenation in LaTeX is very good
%% If a word is mis-hyphenated, add it to this file
%%
%% add information to TeX file before \begin{document} with:
%% %% hyphenation points for line breaks
%% Normally, automatic hyphenation in LaTeX is very good
%% If a word is mis-hyphenated, add it to this file
%%
%% add information to TeX file before \begin{document} with:
%% %% hyphenation points for line breaks
%% Normally, automatic hyphenation in LaTeX is very good
%% If a word is mis-hyphenated, add it to this file
%%
%% add information to TeX file before \begin{document} with:
%% \include{localhyphenation}
\hyphenation{
    Beck-man
    Ngu-yen
    back-chan-nel
    back-chan-nels
    mo-not-o-nous
    ste-reo-typ-i-cal
}

\hyphenation{
    Beck-man
    Ngu-yen
    back-chan-nel
    back-chan-nels
    mo-not-o-nous
    ste-reo-typ-i-cal
}

\hyphenation{
    Beck-man
    Ngu-yen
    back-chan-nel
    back-chan-nels
    mo-not-o-nous
    ste-reo-typ-i-cal
}
 
  \togglepaper[1]%%chapternumber
}{}



\begin{document}
\maketitle

\section{Introduction}
In the context of globalization, weakened national borders, and superdiversity \citep{Blommaert.2011}, languages are also losing their strict boundaries, creating new, hybrid forms. These linguistic phenomena clearly challenge the traditional view of languages as bounded, homogeneous systems that are used in isolation from other languages and are closely linked to their speakers’ national identity and sense of belonging. Translanguaging is a fairly recent sociolinguistic term which captures these new ways of meaning-making and identity formation and allows researchers to explore these phenomena from a novel perspective beyond the traditional categories inherited from earlier generations. 

One area where translanguaging is particularly salient in late modernity is urban linguistic spaces. Large, cosmopolitan capital cities are places where people of diverse linguistic and cultural backgrounds congregate. This diversity is encouraged through various semiotic signs, including the languages displayed in public spaces and the way these languages mix to index new identities \citep{Blommaert.2010,BlommaertMaly.2014,Gorter.2013}.  %TODO: Hier fehlte eine Quelle!


Bulgarian street culture is a good example. As a result of the country's opening up to the Western world after the fall of the Iron Curtain, its accession to the EU in 2007, and the overall effects of globalization and increased mobility, there has been considerable transformation of urban spaces in Sofia. Both central (as the capital city) and peripheral (with regard to Western European city culture), the city strives to define itself as cosmopolitan, open, and part of a larger European context. This trend is countered by a growing sense of nationalism and tribalism, often fuelled (or at least benevolently supported) by the media and populist politicians. 

In this context, this chapter explores the dominant linguistic norms and the linguistic landscape in the centre of Sofia; specifically, the naming practices employed by shop, café, and restaurant owners. These are considered against the background of dominant ideologies and naming practices during socialist times. The analysis seeks to establish how a post-communist country processes its difficult history and current social upheavals, negotiates its relationship with the outside world, and redefines its own identity through the linguistic choices in its urban landscapes. 

\section{Translanguaging: Theoretical background}
Mobility has become a central factor in current sociolinguistic studies, rethinking language beyond narrow national confines and looking at it in its wider social context and deep historical embeddedness \citep{Blommaert.2011}. Languages are no longer seen as normative systems that exert their power over the individual, who has to follow rules and observe external standards; instead, they are regarded as resources on which individuals can draw, exploring them creatively in order to express themselves freely. Dominant language norms are shifting from a focus on monolingualism and idealized linguistic purity towards a preference for hybridity and diversity \citep{Jrgensen.2011}. A burgeoning field of research has appeared urging sociolinguistics “to unthink its classic distinctions and biases and to rethink itself as a sociolinguistics of mobile resources” \citep[1]{Blommaert.2010}. A more detailed analysis of these shifts follows.

\subsection{‘Language’: A reconceptualization}
The conceptualization of language as a self-contained, homogeneous system with clear boundaries has been seriously challenged in recent decades. After the emergence of sociolinguistics in the 1960s, ``language'' was no longer regarded in isolation, devoid of any social context. Contemporary approaches view it as a social practice rather than an abstract, rule-governed system. The very concept of a ``language'' as a discrete unit is under scrutiny by critical discourse analysts, and there is a growing recognition that separate ``languages'' are a social construct, ideological creations asserting linguistic boundaries to reinforce political ones \citep{Blommaert.2011,Saraceni.2015,Jrgensen.2011}.

Drawing on recent studies, \citeauthor{Blommaert.2011} argue that “named languages” (such as ``English'', ``German'', and ``Bengali'') are ideological constructions related to the emergence of the nation-state, and that in “differentiating, codifying and linking a ``language'' with a ``people'', linguistic scholarship itself played a major role in the development of the European nation-state” \citep[4]{Blommaert.2011}. While conceding that the traditional idea of a ``language'' still has immense ideological power, they observe that there has been a major shift in fundamental ideas about language, language groups, and communication, which leads to a very different approach in the study of languages.

\begin{quote}
  Rather than working with homogeneity, stability and boundedness as the starting assumptions, mobility, mixing, political dynamics and historical embedding are now central concerns in the study of languages \citep[3]{Blommaert.2011}.

\end{quote}

\noindent
Similarly, \citet[27]{Jrgensen.2011} argue that traditional concepts of different ``languages'' as bounded systems are sociocultural constructs, ill-suited to capturing the reality of real language use in late-modern superdiverse societies. They point out that drawing borders between closely related languages (for example, German and Dutch) using purely linguistic criteria is not possible, since the idea of individual languages is based on ideology rather than real-life language usage.

Languages, Mario Saraceni argues, “don’t just exist alongside each other, but merge, blend, mesh, coalesce into a symbiosis where traditional labels struggle to find a place” (\citeyear[xi]{Saraceni.2015}). In places where monolingualism is the official language policy, there may be strong institutional support for the national language at the price of suppressing neighbouring minority languages, so that a language may exist as a separate entity. This, however, is hardly the case in contemporary societies characterized by mobility, interconnectedness, and great linguistic and cultural diversity. A new theoretical approach is needed.

\subsection{Languaging and translanguaging}
As it has recently been argued, instead of theorizing language as a static, monolithic, rule-governed system, it may be more appropriate to look at actual language usage, which some scholars refer to as languaging \citep{Swain.2006,Lankiewicz.2014}. The concept was popularized by Merrill Swain, who sees languaging as a dynamic process of using language to make meaning and to shape knowledge and experience, of “coming-to-know-while-speaking” (\citeyear[97]{Swain.2006}). 

This definition is extended by \citet[4]{Lankiewicz.2014}, who describe languaging not only as a way of knowing and meaning-making, but of identity formation and the “unbridled, natural way of using language beyond the normative constraints of a language”.
  
Other related terms, with slight differences in meaning, have been put forward to capture this dynamic, constraint-free language usage observed in present-day superdiverse societies: plurilingualism, polylanguaging, poly-lingual languaging, metrolingualism, translingualism, transidiomaticity, translingual writing, trans-lingual practices, multiliteracies, pluriliteracy, fused lects, heterography, and ludic Englishes \citep[9]{Canagarajah.2013}. 

The translingual paradigm, \citet[6]{Canagarajah.2013} contends, highlights two key significant concepts: firstly, communication transcends individual languages; secondly, communication transcends words and involves diverse semiotic resources and ecological affordances. While \textit{multilingual} may suggest an additive relationship between languages, \textit{translingual} moves away from the \textit{mono/multi-dichot\-o\-my} which he sees as reductive \citep[7]{Canagarajah.2013}.
 
In unifying these definitions, translanguaging can be understood as a dynamic process of creative language use, playful meaning-making, and identity formation, in which languages, unconstrained by notions of homogeneity, boundedness, and isolation, mix and mesh freely, transcending language boundaries and involving various semiotic resources.

\subsection{Language norms: From monolingualism to translanguaging}
In spite of these recent theories, the monolingual paradigm is so pervasive that people still perceive languages as monolithic, homogeneous, bounded systems, isolated from other languages \citep{Jrgensen.2011}. Until not long ago, such attitudes were part of the mainstream language ideology. Before the rise of sociolinguistics in the 1960s, code-switching was typically considered deviant behaviour, and bilinguals were thought of as imperfect language users. The monolingualism norm stipulates that only one language should be spoken at a time; persons who command two or more languages should speak each language ``purely'', without mixing it with other languages \citep[33]{Jrgensen.2011}.

This attitude has changed in recent years, and the monolingualism norm has been replaced by the bilingualism/multilingualism norm, according to which people who have a good command of two (or more) languages will use their full linguistic repertoire, switching between languages when necessary. \citet{Jrgensen.2011} contrast this with the polylanguaging norm, which does not require full command of two (or more) languages, but instead allows language users to employ whatever linguistic features are at their disposal, no matter how well they know the languages involved and whether they can make claims to ``possess'' the languages \citep[34]{Jrgensen.2011}.

I would argue that the polylanguaging norm is not a ``‘norm'' per se. While the monolingualism norm has established itself as the unmarked, default norm, explicitly upheld and imposed by institutions such as the media and the educational system, the bilingualism/multilingualism norm has recently become acceptable and widespread, for example, in road signs which have to be bilingual. The polylanguaging norm, however, can hardly claim to be an institutionalized norm. It is rather the absence, or relaxation, of rigid language norms which require linguistic purity and norm-observance. So I would suggest treating it as a trend rather than a norm, and in line with the previous discussion of translanguaging, I would replace ``polylanguaging'' with ``translanguaging'', as the ``trans'' prefix works better than ``multi'' and ``poly'' at capturing the fluidity of languages and the transcendence of linguistic boundaries.

\section{Linguistic landscaping}
The phenomena discussed above could be observed not only in the speech of individual speakers, but also in contexts such as the written texts in shop windows, and the names of cafés and restaurants found at a given location. I decided to explore the streets of central Sofia and observe how language visibility, social attitudes, and linguistic norms have changed in the relatively short time span of a few decades. The method chosen is described as linguistic landscape studies in \citet[1]{BlommaertMaly.2014} – an attempt to create “accurate and detailed inventories of urban multilingualism” through investigating “the presence of publicly visible bits of written language: billboards, road and safety signs, shop signs, graffiti and all sorts of other inscriptions in the public space, both professionally produced and grassroots”. An overview of the field and methodological approaches used in recent years can be found in \citet{Gorter.2013}. 

Linguistic landscape studies in Bulgaria are still rare, but there are exceptions, such as \citegen{Politov.2015} study of multilingualism in Sofia’s urban spaces, \citegen{Krusteva.2017} exploration of the linguistic landscape in the Sofia subway, and \citegen{AtanassovaDivitakova.2017} survey of the linguistic landscape of Veliko Turnovo. 

This study limits its scope to the examination of the naming practices and linguistic choices in shop windows and the names of cafés and restaurants (collectively labelled ``street signs'' or just ``signs'' for the purposes of the study), as observed in the centre of Sofia in the summer of 2021. They have been contextualized and historicized against the social changes which have taken place in recent years (discussed in Section \ref{slavova:sec:4}). An exploration of peripheral neighbourhoods or locations outside the capital would undoubtedly have produced different results, but this could be the subject of a different study.

The signs have been divided into three large groups with three subgroups each, with somewhat fuzzy boundaries between them. They are discussed in Sections \ref{slavova:sec:5}, \ref{slavova:sec:6}, and \ref{slavova:sec:7} (as presented in \autoref{slavova:tab:1}). 

\begin{table}
  \caption{Classification of signs}
  \label{slavova:tab:1}
  \begin{tabularx}{\textwidth}{lQ}\midrule\toprule
    \multirow[t]{3}{*}{5 Monolingual signs} & 5.1 Monolingual signs in Bulgarian \\
    & 5.2 Monolingual signs in English \\
    & 5.3 Monolingual signs in other languages \\\midrule
    \multirow[t]{3}{*}{6 Bilingual signs} & 6.1 Bilingual signs in Bulgarian and English \\
    & 6.2 Bilingual signs in Bulgarian and another language \\
    & 6.3 Bilingual/multilingual signs in English and other languages \\ \midrule
    \multirow[t]{3}{*}{7 Translingual signs} & 7.1 Transliterated words and names \\
    & 7.2 Mixing languages and scripts\\
    & 7.3 Word play and double meaning\\ \bottomrule\midrule
  \end{tabularx}
\end{table}




The data consist of a total of 240 signs, 160 of which were randomly collected in the central streets of Sofia. They provide illustrative examples of the different types of naming practices employed, and the various linguistic strategies of code-mixing, script-mixing, and translanguaging. The other 80 signs were collected systematically in a single street, Ivan Shishman Street, a trendy shopping destination with predominantly small shops, cafés, and restaurants. The systematic approach allows for a quantitative analysis of the naming practices employed (discussed in Section \ref{slavova:sec:8}). 

A separate corpus of 120 images was collected for the sociohistorical analysis in Section \ref{slavova:sec:4}. It was gathered from a vast online collection of images from the socialist period \citep{Danov} and a selection of images borrowed from the same database \citep{HighViewArt.2014}, as well as a recent photographic collection of old street signs surviving to this day \citep{Dnevnik.2011}.

The analysis attempts to answer the following questions:

\begin{enumerate}
  \item What is the overall pattern in terms of linguistic norms, languages, and language-mixing strategies?
  \item What motivates those linguistic choices?
  \item How are these practices culturally and historically embedded?
\end{enumerate}

\noindent
The following sections seek to answer these questions.

\section{Sociolinguistic context} \label{slavova:sec:4}
In order to appreciate the change which has taken place in Sofia’s linguistic landscape in recent decades, a look at the linguistic, social, cultural, and historical context is in order. Three major factors are outlined below, namely the importance of the Cyrillic alphabet in the identity construction of Bulgarians, the Soviet influence after 1944 and Bulgaria’s isolation from the Western world, and the nation’s opening up to the outside world in 1989 after the fall of the Iron Curtain, which coincided with the spread of globalization and English being recognized as a global language.

\subsection{The Cyrillic script and Bulgarian identity}
Even though Bulgarians tend to think of the Cyrillic script as the Bulgarian alphabet, they actually share it with some 250 million other people. Russia accounts for the greatest number of users, and for this reason, it is often wrongly considered to be ``the Russian alphabet'' \citep{Nikolova.2021}. Serbia, North Macedonia, Montenegro, Belarus, and Ukraine also use it, albeit with some modifications (the Latin alphabet is also used in Serbia, Montenegro, and Bosnia). While it is mainly associated with Slavic languages, not all Slavic languages use it. At the same time, a number of Finno-Ugric, Turkic, Iranian, and Caucasian languages have adopted it. Some of them shifted from Cyrillic to Latin after the disintegration of the Soviet Union in 1991, or are currently in the process of doing so \citep{Iliev.2013}.

For foreign visitors who are used to Roman letters, however, Cyrillic can pose a serious challenge. Deciphering street signs and other texts is largely impossible, even when they hide familiar borrowings. Having some familiarity with the Cyrillic letters may help. In \autoref{slavova:tab:2}, they have been divided into three groups: letters which are identical to Latin letters, so-called false friends which look the same but have different values in Cyrillic and Latin, and unique letters which have no counterpart in the Latin script.

\begin{table}
  \begin{tabular}{lll}\midrule\toprule
    Identical letters &	False friends &	Unique letters \\\midrule
    \textit{Аа} – a & \textit{Вв} – v & \textit{Бб} – b \\
    \textit{Ее} – e & \textit{Сc} – s & \textit{Гг} – g \\
    \textit{Кк} – k & \textit{Нн} – n & \textit{Дд} – d \\
    \textit{Мм} – m & \textit{Рp} – r & \textit{Жж} – zh \\
    \textit{Оо} – o & \textit{Ии} – i & \textit{Зз} – z \\
    \textit{Тт} – t & \textit{Хх} – h & \textit{Лл} – l \\
    & \textit{Уу} – u & \textit{Пп} – p \\
    & & \textit{Фф} – f \\
    & & \textit{Цц} – ts \\
    & & \textit{Чч} – ch \\
    & & \textit{Шш} – sh \\
    & & \textit{Щщ} – sht \\
    & & \textit{Ъъ} – $\wedge$ as in cut \\ 
    & & \textit{Юю} – yu \\
    & & \textit{Яя} – ya \\ \bottomrule\midrule
  \end{tabular}
  \caption{The Bulgarian Cyrillic alphabet}
  \label{slavova:tab:2}
\end{table}

\noindent
It is worth noting that the upper-case and lower-case forms of Cyrillic letters may not coincide with the respective Latin letters; for example, the upper-case form of <M> is identical, but the lower-case <m> in Bulgarian is spelled <м>, while the Bulgarian <\textit{т}> corresponds to the cursive form of lower-case <T>. 

Commissioned in the late ninth century by the First Bulgarian Empire during what was considered The Golden Century in Bulgarian history, the Cyrillic alphabet was developed for the purposes of Old Church Slavonic in parallel with the introduction of Christianity. It replaced the official Bulgarian Glagolitic script, created by Constantine the Philosopher (Saint Cyril), but was later (somewhat misleadingly) named after him \citep{Iliev.2013}. 

The episode is ingrained in the memories of countless students through the educational system, which has glorified the Cyrillic alphabet and turned it into an indelible part of Bulgarian national identity. There is even a national holiday celebrating the Day of Bulgarian culture and the Slavonic alphabet, complete with its own hymn. For this reason, any attempt to replace the Cyrillic script with the Latin one, or even to use Latin in informal communications, is met with resistance. Bulgarians feel intimately related to the Cyrillic script, which has developed as a strong marker of national ideology and identity construction \citep{Norman.2019}.

\subsection{The Soviet influence}
Bulgaria was liberated from Ottoman rule with the help of Russia at the end of the nineteenth century and was occupied by the Red Army in 1944. Isolated from the West behind a symbolic Iron Curtain, it lived under a totalitarian regime for 45 years. The Soviet influence was palpable: a centralized economy; a dominant socialist ideology which permeated every aspect of public life; no consumerist culture, no brand names, almost no advertising; long queues for deficit goods; strong political, economic, and cultural ties with the Soviet state and the other countries in the Eastern bloc; and very limited contact with the rest of the world.

Russian functioned as a \textit{de facto} second language in Bulgaria. It was studied from an early age in state schools. A pen-pal exchange system with Soviet children was established for schoolchildren. On Fridays, the national television network broadcast the news in Russian, without translation, straight from the Ostankino Tower in Moscow. The Friday news was followed by the obligatory Soviet film. Russian was used as a \textit{lingua franca} for international travel, conferences, and academic exchanges within the Eastern bloc. 

One of the most well-stocked bookshops was the Russian bookshop named \textit{Lenin} in central Sofia, as found in the sociohistorical corpus:

\begin{exe}
  \ex\label{slavova:ex:1}
  \gll \textit{Съветски} \textit{книги} \textit{КНИЖАРНИЦА} \textit{ЛЕНИН} \\
  savetski knigi knizharnitsa lenin \\
  \glt ‘Soviet books BOOKSHOP LENIN’
\end{exe}

\noindent
Next to it was another landmark location indexing the symbolic closeness to the Soviet capital:

\begin{exe}
  \ex\label{slavova:ex:2}
  \gll \textit{Ресторант} \textit{Москва} \\
  restorant moskva \\
  \glt ‘Restaurant Moscow’
\end{exe}

\noindent
Both signs were written in Bulgarian, but since the Russian alphabet is similar and the two languages are from the same language family, sharing multiple cognate words, the signs were completely understandable to Russian speakers. A large number of other words were imported from Russian during that period. The influence of other foreign languages (mostly French, English, and German), which were popular in the country before the Second World War, was suppressed during the Soviet regime. Even though monolingualism was the dominant norm at the time, imposed by a nationalist state ideology, a rigid school system, and controlled state media, Russian had symbolically permeated the Bulgarian language and street signs through many borrowings and cultural references.

Instead of advertisements of brand names and luxury goods, cityscapes were dominated by ideological slogans, such as the large imperative neon slogan on a wall facing the Central Departmental Store:

\begin{exe}
  \ex\label{slavova:ex:3}
  \gll \textit{Гледайте} \textit{новите} \textit{български} \textit{и} 
  \textit{съветски} \textit{филми}! \\
  gledaite novite balgarski i savetski filmi \\
  \glt ‘Watch the new Bulgarian and Soviet films!’
\end{exe}


The sociohistorical corpus of archival images from Sofia in the 1960s, 1970s, and 1980s reveals many no-name shop signs simply stating what they sell: \textit{Хляб} ‘bread’; \textit{Плодове и зеленчуци} ‘fruit and vegetables’; \textit{Месо и колбаси} ‘meat and meat products’; \textit{Яйца и птици} ‘eggs and birds’; \textit{Бързи закуски} ‘fast food’; \textit{Шапки} ‘hats’; \textit{Обувки} ‘shoes’; \textit{Очила} ‘glasses’; \textit{Парфюмерия} ‘perfumery’; \textit{Платове} ‘fabrics’; \textit{Трикотаж} ‘knitwear’; \textit{Аптека} ‘pharmacy’; \textit{Кафе} ‘café’. 

Most shops sold the same limited range of goods at fixed prices. Only a few had names, for example the women’s clothing store \textit{Валентина} ‘Valentina’ (appropriately, a female name). 

With almost every shop and street sign written in Cyrillic, a traveller from outside the Eastern bloc would have found it difficult to navigate the city unless they knew Cyrillic. Even taxis (\textit{такси}) and hotels (\textit{хотел}) had their signs written in Bulgarian. One of the few shops with a name in Latin script was \textit{Corecom}, a chain of tax-free shops operating between the 1960s and 1990s. Deriving its name from the French \textit{Co}(\textit{mptoir de}) \textit{re(présentation et de) com(merce)}, ‘a direction of representation and commerce’, \textit{Corecom} traded in a foreign currency (US dollars), sold luxury Western goods to foreign visitors and privileged locals, and gave Bulgarians a taste of the coveted Western consumerist culture.

In some cases, bilingual signs did appear. One pharmacy in the corpus had a sign in Bulgarian and a smaller sign in French: \textit{Аптека} | \textit{Pharmacie}. For ideological reasons, French rather than English was the preferred language for addressing foreigners. Air France had an office in central Sofia and a sign in both Bulgarian and English. Bulgarian Airlines also had a sign in both languages. Some hotels followed a similar pattern (\textit{Grand Hotel Balkan} | \textit{Гранд хотел Балкан}). 

Apart from a few exceptions, however, the Cyrillic script was omnipresent. When, by some curious twist of international trade relations with the West, Coca-Cola started producing its iconic beverage in Bulgaria in 1965 (Bulgaria was the first country in the Eastern bloc to produce the drink and to export it to other Eastern bloc countries), it designed its bottles with Cyrillic text on them: \textit{Кока-Кола}, ‘Coca-Cola’. 

The geographical references found in the streets of Sofia provided self-referen\-ces (\textit{Grand Hotel Sofia}, \textit{Grand Hotel Bulgaria}, \textit{Hotel Rodina} [‘motherland’], \textit{Grand Hotel Balkan}), referred to Soviet realia (\textit{Park Hotel Moskva}, surprisingly spelled in Latin script), or represented capitals from the Eastern bloc written in Cyrillic (\textit{Prague} café, \textit{Berlin} ice-cream parlour, \textit{Budapest} Hungarian restaurant). Another popular location in central Sofia was the elegant Russian restaurant called \textit{Crimea}. International influences were limited by the invisible Iron Curtain; Soviet references dominated the cityscape. \textit{Moscow} was not only the name of a central restaurant, but also a hotel and a cinema; \textit{Lenin} was the name of a bookshop, a boulevard, a neighbourhood in the city, and a university.

\subsection{Opening up to the outside world}
The Soviet influence and Bulgaria’s isolation from the Western world came to a rather abrupt end in 1989, following the fall of the Berlin Wall. As the country gradually transitioned to a democratic form of government and a free-market economy, the USSR disintegrated, and multinational companies entered the market, Russian was also replaced as the dominant second language, with English quickly taking its place. A new era had begun, not only in political and economic terms, but also linguistically.

This coincided with a global phenomenon: the spread of American-driven globalization in the 1990s, and the assertion of English as the global language of international communication. The first McDonald’s restaurant opened in Sofia in 1995. Other global Western brands followed suit. The liberation of the market meant that everybody could now start a business and open a shop, and multiple new outlets appeared in the streets. 

English was seen as the language of freedom and democracy, of free trade and global culture; as a result, many people rushed to learn it. Organizations such as the British Council and the US Peace Corps facilitated the process by bringing in language-teaching expertise, training materials, teachers, and teacher trainers. English language teaching schools flourished. English became the main foreign language taught in schools, quickly replacing Russian. Other Western languages (French, German, and Spanish) were also in high demand, though they did not see the explosive success English enjoyed. 

As Bulgaria transitioned politically and economically and opened up to the Western world, it applied for membership of the European Union. The country became a member in 2007, giving Bulgarian citizens new opportunities to travel, study, and work abroad. Many EU citizens also moved to Bulgaria permanently. English became the main foreign language. Bulgarian, on the other hand, became one of the 24 official languages of the European Union, and the Cyrillic script joined the Latin and Greek scripts as one of now three officially recognized scripts in the EU.

However, in spite of the significant cultural shift and Bulgaria’s opening up to the outside world, the country is still split between its pro-Western orientation, liberal-democratic views, and EU membership, on the one hand, and strong nationalistic, xenophobic, anti-EU, anti-democratic undercurrents, often coupled with nostalgia for the Soviet past and its symbols, on the other. How much the two opposing trends have influenced Sofia’s linguistic landscape is the object of study in what follows.

\section{Monolingual signs}\label{slavova:sec:5}
The first group of street signs includes signs written in a single language. The signs have been divided into three groups: signs in Bulgarian, signs in English, and signs in other languages. Contrary to the monolingual norm discussed earlier, the monolingual signs include not only signs in Bulgarian, but also signs in English and signs in other languages, as well as languages that have not been identified. 

\subsection{Monolingual signs in Bulgarian}
Some of the signs in Bulgarian follow the well-known pattern of simply stating what they sell, without any naming and branding, as in Example \xref{slavova:ex:4}, \autoref{slavova:fig:1}. These signs seem to be relics from socialist times. The new shop owners must have taken a conscious decision to preserve the original design and to keep the spirit of the past.

\begin{exe}
  \ex\label{slavova:ex:4}
  \gll Месо \\
  meso \\
  \glt ‘Meat’
\end{exe}

\begin{figure} %TODO: Grafik einfügen
  \includegraphics[width=0.65\textwidth]{Figure_1_Slavova}
  \caption{\textit{Месо} ‘Meat’. Photo: Emilia Slavova.}
  \label{slavova:fig:1}
\end{figure}

\noindent
While Example \xref{slavova:ex:5} seems to follow a similar pattern, it uses very different stylistics. In contrast to the main text that simply states a job description (baker), the accompanying text points to a recent trend for craft foods. It follows a Western pattern by giving the start of the enterprise (“2013”). The contemporary design of the sign clearly indexes a hipster outlet for young, middle-class, outward-looking people.

\begin{exe}
  \ex\label{slavova:ex:5}
  \gll ХЛЕБАР. Занаятчийски хляб София 2013 \\
  hlebar zanayatchiiski hlyab sofia 2013 \\
  \glt ‘Baker. Craft bread Sofia 2013’
\end{exe}


\textit{Магазин №10} (Example \ref{slavova:ex:6}) follows a similar pattern: it simply calls itself “shop”. The sign has a distinctly vintage style and could either be a relic salvaged from socialist times or a newly designed sign in a retro style. The artistic décor and eccentric handmade clothes clearly show that the name is intentionally bland and seemingly dated, while the shop itself is anything but.

\begin{exe}
  \ex\label{slavova:ex:6}
  \gll Магазин № 10 \\
  magazin nomer deset \\
  \glt ‘Shop No. 10’
\end{exe}

\noindent
\textit{Бутик №2} (Example \ref{slavova:ex:7}), a trendy clothes shop, has a similarly bland name with an arbitrary number, but the word choice follows a different logic. It is a transliteration of a European word, ‘boutique’, which entered the Bulgarian language in the 1990s. Most likely borrowed from English (‘trendy fashion shop’), the word is actually French in origin, with cognates in other European languages: \textit{bottega} (Italian); \textit{apotheca} (‘storehouse’ in Latin); and even \textit{ἀποθήκη} (\textit{apothḗkē} in Ancient Greek). In Cyrillic, however, these etymological connections are completely lost. Boutiques entered the linguistic market in the 1990s, alongside a range of small, independent shops, often selling handmade goods, in contrast to the centrally produced socialist garments of previous times.

\begin{exe}
  \ex\label{slavova:ex:7}
  \gll Бутик № 2 \\
  Butik nomer dve \\
  \glt ‘Boutique No. 2’
\end{exe}

\noindent
\textit{Ревю} (Example \ref{slavova:ex:8}) also uses an English word, ‘review’, which in Bulgarian has a narrower meaning and refers mostly to fashion. Likewise, \textit{Бистро} (Example \ref{slavova:ex:9}) is a French borrowing with presumably Russian roots, hardly recognizable in its Cyrillic orthography.

\begin{exe}
  \ex\label{slavova:ex:8}
  \gll Ревю \\
  revyu \\
  \glt ‘Review’
\end{exe}

\begin{exe}
  \ex\label{slavova:ex:9}
  \gll Бистро \\
  bistro \\
  \glt ‘Bistro’
\end{exe}

The same can be said for \textit{Doner Miami} (Example \ref{slavova:ex:10}), selling Turkish doners but evoking a faraway destination in the USA, and \textit{The Master and Margarita} (Example \ref{slavova:ex:11}), a flower shop bearing the name of the Russian literary classic by the dissident (Kyiv-born) writer Mikhail Bulgakov. The latter points to a distinction many Bulgarians make between Russian culture, to which they feel deeply connected, and Soviet ideology, which they despise.

\begin{exe}
  \ex\label{slavova:ex:10}
  \gll Дюнер Маями \\
  dyuner mayami \\
  \glt ‘Doner Miami’ \\
  \ex\label{slavova:ex:11}
  \gll Майстора и Маргарита \\
  maistora i margarita \\
  \glt ‘The Master and Margarita’
\end{exe}


Finally, \textit{Кьоше} (Example \ref{slavova:ex:12}), a trendy souvenir corner shop, has a name which sounds distinctly Bulgarian, but is actually Turkish in origin (\textit{köşe} ‘corner’). Due to the many years of Ottoman rule, Turkish borrowings have often blended with Bulgarian, or are stylistically marked as more intimate, informal, and ``folksy'' compared with their more neutral Bulgarian counterparts.

\begin{exe}
  \ex\label{slavova:ex:12}
  \gll Кьоше \\
  kyoshe \\
  \glt ‘Corner’
\end{exe}

\subsection{Monolingual signs in English}
Some of the monolingual signs in English found in the streets of Sofia have a distinctly global orientation. \textit{Simple: Taste the world} (restaurant) is a case in point. Others have English names without any specific cultural reference, such as \textit{Social Café, Farmers Soups and Sandwiches, The Little Things} (restaurant), \textit{The Gourmet House} (fine china), \textit{LightHouse} (candles), and \textit{Orange. Books Music Stationery} (a multistorey bookshop). The only example of a shop with a distinctly English orientation is \textit{Elephant Bookstore} (Example \ref{slavova:ex:13}).

\begin{exe}
  \ex\label{slavova:ex:13} ELEPHANT Bookstore. Vintage \& English Gifts
\end{exe}

Tourists in Bulgaria no longer need to know Cyrillic in order to find their hotel. Hotel names, such as \textit{Grand Hotel Sofia}, are usually spelled in Latin and are easy to read. Many traditional gift shops (Example \ref{slavova:ex:14}) also have names spelled in Latin. In a trendy tourist shop, \textit{GIFTED} (Example \ref{slavova:ex:15}), \autoref{slavova:fig:2}, you can not only buy souvenirs, but also leave your luggage, book a visit to the \textit{Red Flat} to explore life under socialism, or book a \textit{Free Sofia Tour} trip, all in English.

\begin{exe}
  \ex\label{slavova:ex:14} SOUVENIRS of Bulgaria
  \ex\label{slavova:ex:15} GIFTED. Gallery, Luggage lockers, Urban culture hub, The Red Flat, Free Sofia Tour
\end{exe}

\begin{figure} %TODO: Bildquelle!!
  \includegraphics[width=0.65\textwidth]{Figure_2_Slavova}
  \caption{GIFTED. Photo: Emilia Slavova.}
  \label{slavova:fig:2}
\end{figure}

\subsection{Monolingual signs in other languages}
It would be easy to assume that the global status of English would allow it to dominate the linguistic landscape. But a closer look reveals that many of the signs in Sofia’s streets are in other languages, including German (Example \ref{slavova:ex:16}),
 Spanish (Example \ref{slavova:ex:17}), French (Example \ref{slavova:ex:18}),
  and Italian (Examples \ref{slavova:ex:19},  \ref{slavova:ex:20}, and \ref{slavova:ex:21}).

  \largerpage
\begin{exe}
  \ex\label{slavova:ex:16} BISMARCK. Frischer Fisch und Deutsches Bier seit 2020
  \glt ‘Bismarck. Fresh fish and German beer since 2020’
  \ex\label{slavova:ex:17} La Casa del Habano 
  \glt ‘The Havana House’ 
  \ex\label{slavova:ex:18} Bistrot L’Etranger. Maison fondée en 2001 
  \glt ‘Bistro The Stranger. House founded in 2001’ 
  \ex\label{slavova:ex:19} Gelateria Caffeteria CONFETTI 
  \glt ‘Ice-cream parlour Café Confetti’ 
  \ex\label{slavova:ex:20} Trattoria Neapolitana. Pizza \& Aperitivo Est. 1867 
  \glt ‘Neapolitan eatery. Pizza and Aperitif Established 1867’ 
  \ex\label{slavova:ex:21} La Bottega Due Piani 
  \glt ‘The two floors shop’
\end{exe}


Italian seems to be the most popular language related to cuisine in central Sofia, and even though many Italian words have become sufficiently transparent in Bulgarian (\textit{pizza, caffeteria}), others are not (\textit{trattoria, bottega, piani}). They are used to convey foreignness and – by association with Italian – to promise tasty food, even if the meaning may be opaque. 

This is not the case with \textit{Банкович} (Example \ref{slavova:ex:22}), a popular Serbian restaurant. The sign is spelled in Cyrillic, but the Serbian-sounding family name \textit{Brankovich} and a slight change in the spelling of \textit{restaurant}, which is missing the final <\textit{т}> (\textit{ресторан} versus \textit{ресторант}) signifies that it is in Serbian.

\begin{exe}
  \ex\label{slavova:ex:22}
  \gll Ресторан Банкович \\
  restoran bankovich \\
  \glt ‘Bankovich restaurant’
\end{exe}


Another Balkan cuisine establishment in the corpus is \textit{Beyzade} (Example \ref{slavova:ex:23}), \autoref{slavova:fig:3}. The name, meaning ‘noble, aristocratic, from a good family’ in Turkish, is opaque in Bulgarian, but clearly indexes Oriental foreignness, further supported by the familiar Turkish desserts and pastry \textit{baklava}, \textit{kunefe}, and \textit{borek}.

\begin{exe}
  \ex\label{slavova:ex:23} BEYZADE. Baklava, kunefe \& borek 
\end{exe}

\begin{figure} %TODO: Bildquelle!!
  \includegraphics[width=0.65\textwidth]{Figure_3_Slavova}
  \caption{\textit{Beyzade}. Photo: Emilia Slavova.}
  \label{slavova:fig:3}
\end{figure}

A similar effect is achieved by the distinctly Japanese-sounding name \textit{Tanoshi} (Example \ref{slavova:ex:24}). A Romanized variant of the Japanese word (meaning ‘fun, pleasant, delightful’), the name carries no specific meaning to most Bulgarians, apart from indexing Japanese language and culture, which is its main function.

\begin{exe}
  \ex\label{slavova:ex:24} TANOSHI
\end{exe}
\clearpage

\section{Bilingual signs}\label{slavova:sec:6}
The next group in the corpus consists of signs in two languages: Bulgarian and English, Bulgarian and another language, or English and another language (and, occasionally, more than one other language). While in some instances the bilingual signs have identical meanings in both languages, this is not the case most of the time. The two languages usually complement each other, with part of the information only available in one language. Since the graphic design features of the signs are lost in the examples below, a dividing line | is used to separate the two languages for the sake of clarity.

\subsection{Bilingual signs in Bulgarian and English}
Some outlets choose to have their names spelled both in English and Bulgarian, with one script (English in this case) considerably larger than the other (Example \ref{slavova:ex:25}).

\begin{exe}
  \ex\label{slavova:ex:25}
  \gll {Souvenirs from Bulgaria |} Сувенири от България \\
  { } suveniri ot balgariya \\
  \glt ‘Souvenirs from Bulgaria | Souvenirs from Bulgaria’
\end{exe}

\noindent
In other cases, such as Example \xref{slavova:ex:26}, only one of the words is translated, and the other is not.

\begin{exe}
  \ex\label{slavova:ex:26}
  \gll Книги Хеликон {| Books} \\
  knigi helikon { } \\
  \glt ‘Helikon Books | Books’
\end{exe}

\noindent
Elsewhere, the two languages function together as two parts of a whole, code-mixing between Bulgarian and English, as in Examples \xref{slavova:ex:27}, \xref{slavova:ex:28}, and \xref{slavova:ex:29}. 

\begin{exe}
  \ex\label{slavova:ex:27}
  \gll Магазини {| KINKY} \\
  magazini { } \\
  \glt ‘Kinky Shops’ \\
  \ex\label{slavova:ex:28}
  \gll Химическо чистене {| Clean the world} \\
  himichesko chistene { } \\
  \glt ‘Dry cleaning | Clean the world’ \\
  \ex\label{slavova:ex:29}
  \gll Cozy маркет | Кафе алкохол цигари \\
  { } market { } kafe alkohol tsigari \\
  \glt ’Cozy market | Coffee, alcohol, cigarettes’
\end{exe}

\noindent
A number of outlets choose a mixture of Bulgarian and English for their signs, code-switching between the two languages, while the cultural reference points elsewhere. Examples include \textit{Royal Thai} (Example \ref{slavova:ex:30}), \textit{Lokah} (Example \ref{slavova:ex:31}), where a Sanskrit word is spelled in Latin script, and \textit{JOVAN The Dutch Bakery} (Example \ref{slavova:ex:32}), followed by descriptions in Bulgarian which clarify what they offer: Thai food, Indian goods, and Dutch baked goods.

\begin{exe}
  \ex\label{slavova:ex:30}
  \gll {Royal Thai |} Тайландски ресторант \\
  { } tailandski restorant \\
  \glt ‘Royal Thai | Thai Restaurant’ \\
  \ex\label{slavova:ex:31}
  \gll {Lokah |} Индийски стоки \\
  { } indiiski stoki \\
  \glt ‘Lokah | Indian Goods’ \\
  \ex\label{slavova:ex:32}
  \gll {JOVAN. The Dutch Bakery |} Холандската Фурна \\
  { } holandskata furna \\
  \glt ‘Jovan. The Dutch Bakery | The Dutch Bakery’
\end{exe}


Another example of a mixture of Bulgarian and English, indexing a third culture (Arabian) by means of the fictional character Ali Baba from the popular book \textit{Arabian Nights}, is shown in Example \xref{slavova:ex:33}.

\begin{exe}
  \ex\label{slavova:ex:33}
  \gll Бърза закуска {| Ali Baba Fast food} \\
  barza zakuska { } \\
  \glt ‘Fast snack | Ali Baba Fast food’
\end{exe}


A curious example in this group is Example \xref{slavova:ex:34}, \autoref{slavova:fig:4}, a shop selling a traditional Bulgarian snack: \textit{banitsa} (a kind of pastry, also known as \textit{borek} in Turkish). It is considered typically Bulgarian, even though variants can be found in other Balkan cuisines. The “traditional” in the name clearly evokes nationalistic feelings, also discretely signalled by a thin line at the top representing the Bulgarian national flag (white, green, and red). In spite of this, the shop cannot resist the temptation to add an English word (\textit{handmade}) to its very traditional Bulgarian name, making it much less traditional than it purports to be.

\begin{exe}
  \ex\label{slavova:ex:34}
  \gll {Handmade |} Традиционна БАНИЦА \\
  { } traditsionna banitsa \\
  \glt ‘Handmade | Traditional Pastry’
\end{exe}

\begin{figure} %TODO: Bildquelle!!
  \includegraphics[width=0.65\textwidth]{Figure_4_Slavova}
  \caption{Traditional \textit{banitsa}. Photo: Emilia Slavova.}
  \label{slavova:fig:4}
\end{figure}


The next example also uses Bulgarian and English to refer to a third culture: Russian. The Russian restaurant \textit{Arbat} (Example \ref{slavova:ex:35}) is in the same location as the \textit{Moscow} restaurant from Example \xref{slavova:ex:2} above. Having previously been an Italian restaurant (\textit{Corso}), it is now Russian once again, but under a new name, \textit{Arbat}, after a fashionable street in Moscow. The name of the street is the same in Russian and Bulgarian, but the spelling of \textit{Russian} (single <с> rather than double) and \textit{restaurant} (with a final <\textit{т}>) clearly index that the sign is in Bulgarian rather than Russian (\textit{руски ресторант} versus \textit{русский ресторан}). The presence of Bulgarian and English, but no Russian, in the name of a Russian restaurant in Sofia, is a rather bold decision on the part of the owners.

\begin{exe}
  \ex\label{slavova:ex:35}
  \gll АРБАТ. Руски ресторант {| Russian restaurant} \\
  arbat ruski restorant { } \\
  \glt ‘Arbat. Russian restaurant | Russian restaurant’
\end{exe}

\subsection{Bilingual signs in Bulgarian and another language}
The next examples mix Bulgarian and another language: French (Example \ref{slavova:ex:36}), Italian (Example \ref{slavova:ex:37}),
 and Russian (Example  \ref{slavova:ex:38}), \autoref{slavova:fig:5}. Interestingly, in the final example, the store name is in Russian (\textit{Берёзка}), recognizable by the diacritics over the second ``e'', but the descriptor (‘Russian food store’) is spelled in Bulgarian – \textit{руски}, rather than in Russian, \textit{русский} – even though the inverted word order (the adjective following the noun) follows the Russian syntax rather than the Bulgarian. Another interesting detail concerning the name is that \textit{Берёзка} was the name of the Western goods shops in the USSR, similar to the Bulgarian \textit{Corecom} discussed earlier.

\begin{exe}
  \ex\label{slavova:ex:36}
  \gll {BONJOUR JULIETTE |} Пекарна \\
  { } pekarna \\
  \glt ‘Bonjour Juliette | Bakery’ \\
  \ex\label{slavova:ex:37}
  \gll {Gelato di Natura Sofia |} Натурални фермерски продукти \\
  { } naturalni fermerski produkti \\
  \glt ‘Gelato di Natura Sofia | Natural farmer’s products’ \\
  \ex\label{slavova:ex:38}
  \gll Берёзка | Гастроном руски \\
  biryozka { } gastronom ruski \\
  \glt ‘Birch tree | Russian food store’
\end{exe}

\begin{figure} %TODO: Bildquelle!!
  \includegraphics[width=0.65\textwidth]{Figure_5_Slavova}
  \caption{\textit{Beryozka}. Photo: Emilia Slavova.}
  \label{slavova:fig:5}
\end{figure}


The \textit{Remedium} pharmacy (Example \ref{slavova:ex:39}) has its descriptor (\textit{pharmacy}) written both in Bulgarian and English, and its name in Cyrillic. However, the word \textit{remedium} does not exist in Bulgarian. It is a Cyrillicized form of a Latin word meaning ‘cure’ with multiple cognates in European languages: English: \textit{remedy}; French: \textit{remède}; Italian: \textit{rimedio}; Portuguese: \textit{remédio}; Romanian: \textit{remediu}; Spanish: \textit{remedio}, and so on. Spelled in Cyrillic, the meaning is lost on foreign visitors, while Bulgarians could only understand it through another language they are familiar with.

\begin{exe}
  \ex\label{slavova:ex:39}
  \gll Аптека {| Pharmacy |} Ремедиум 2 \\
  apteka { } remedium 2 \\
  \glt ‘Pharmacy | Pharmacy | Remedium 2’
\end{exe}

\subsection{Bilingual/multilingual signs in English and another language}
The next group of signs mix English and another language, without any recourse to Bulgarian: a combination of Italian and English (Example \ref{slavova:ex:40}); Spanish and English (Example \ref{slavova:ex:41}); French and English (Example \ref{slavova:ex:42}); Turkish and English (Example \ref{slavova:ex:43}); Arabic and English (Example \ref{slavova:ex:44}); and Portuguese and English (Example \ref{slavova:ex:45}). The final example, \textit{Garafa}, is a misspelling of the Portuguese/Spanish word \textit{garrafa}, of Arabic origin (meaning ‘bottle’), but functions as a transliteration of the Bulgarian word \textit{гарафа} (/garafa/), meaning an open-top flask for pouring wine (similar to the English \textit{carafe}).

\begin{exe}
  \ex\label{slavova:ex:40} MAMMA MIA | Restaurant and Pizzeria 
  \ex\label{slavova:ex:41} EL GRADO | Jewellery 
  \ex\label{slavova:ex:42} Bijoux | Trendy 
  \ex\label{slavova:ex:43} Djanam | Duner \& Burger 
  \ex\label{slavova:ex:44} HAMAM | Home of textile 
  \ex\label{slavova:ex:45} Garafa | Wine Shop
\end{exe}

\noindent
\textit{El Shada} (Example \ref{slavova:ex:46}), \autoref{slavova:fig:6}, is rather difficult to analyse. Seemingly in Italian (also supported by the food served), the definite article \textit{El} appears to be in Spanish. The name may be a variation of \textit{El Shaddai} (Hebrew for ‘God Almighty’; there is a Gospel song with the name El-Shada), or an incorrect spelling of \textit{L’sciadà} (‘rolling pin’ in Ladin), or it could have some other etymology. It could also be emulating foreignness without having any particular meaning. The other words in the name, however, are clear enough. \textit{Pizza} is now a well-established Bulgarian word. \textit{Pasta} has also become popular as meaning ‘Italian dish made of dough’, although until the 1990s, it was used with a different meaning in Bulgarian: either ‘cake’, ‘pastry’, or ‘paste’. \textit{Vino} was borrowed so long ago that it is now an authentic Bulgarian word with no alternative.

\begin{exe}
  \ex\label{slavova:ex:46} EL SHADA | Pizza. Pasta. Vino
\end{exe}

\begin{figure} %TODO: Bildquelle!!
  \includegraphics[width=0.65\textwidth]{Figure_6_Slavova}
  \caption{\textit{El Shada}. Photo: Emilia Slavova.}
  \label{slavova:fig:6}
\end{figure}

\noindent
The next Example \xref{slavova:ex:47} mixes three languages. The Italian name \textit{Ottimo}  (‘optimal, excellent’, a superlative form of \textit{buono} ‘good’) is coupled with a Bulgarian and an English reminder to call and pick up the food.

\begin{exe}
  \ex\label{slavova:ex:47} 
  \gll {OTTIMO pizza \& pasta |} Обади се и вземи за вкъщи {| Pick up} \\
  { } obadi se i vzemi za vkashti { } \\
  \glt ‘Ottimo pizza \& pasta | Call and take away | Pick up’
\end{exe}


The final example in this group, \textit{NAMOOΣ} (Example \ref{slavova:ex:48}), also uses more than two languages. \textit{Namoos} (the alternative spelling) is the only word in the corpus with a Greek letter in it (<Σ>). It creates the impression that it is a Greek word, but it is actually Arabic (\textit{nāmūs} ‘law’, ‘custom’, or ‘honour’). The reason it was chosen was probably the connection with the Ancient Greek word \textit{nómos} (\textit{ΝΌΜΟΣ}) ‘law, custom’. The next word refers to the Greek island of Mykonos. The slogan is in English, and the booking details are in Bulgarian.

%\begin{exe}
%  \ex\label{slavova:ex:48}
%  \gll {NAMOOΣ Mykono! | Mediterranean love \& food |} телефон за резервации \\
%  { } telefon za rezervatsii \\
%  \glt ‘Namoos Mykono! | Mediterranean love \& food | phone number for bookings’
%\end{exe}

\begin{exe}
  \ex\label{slavova:ex:48}
 {NAMOOΣ Mykono! | Mediterranean love \& food |} \\
  \gll телефон за резервации \\
   telefon za rezervatsii \\
  \glt ‘Namoos Mykono! | Mediterranean love \& food | \\ phone number for bookings’
\end{exe}


\section{Translingual signs}\label{slavova:sec:7}
The final group in the corpus includes signs which mix languages and scripts in unexpected ways, such as transliterated words and names, code-mixing and script-mixing, and word play and double meaning. Linguistic signs are often mixed with graphic signs to enhance understanding and augment the meaning.

\subsection{Transliterated words and names}
The first example is \textit{Pileto} (Example \ref{slavova:ex:49}), \autoref{slavova:fig:7}. It represents a transliteration of the Bulgarian word for ‘chicken’, with an image of a chicken incorporated into the O sign of the logo, so that the meaning can be derived from the graphic sign, even if the word is unclear. The rest of the sign is in English. The word may be opaque to non-Bulgarian speakers, but it serves as an attention-getter and adds a Bulgarian ``vibe'' to a name that is easy to read, transcending not only the boundary between separate languages, but also linguistic and visual signs.

\begin{exe}
  \ex\label{slavova:ex:49} PILETO | Hand made
  \glt ‘The bird | Handmade’
\end{exe}

\begin{figure}
  \includegraphics[width=0.65\textwidth]{Figure_7_Slavova}
  \caption{\textit{Pileto}. Photo: Emilia Slavova.}
  \label{slavova:fig:7}
\end{figure}

\noindent
\textit{Bilkova}  (Example \ref{slavova:ex:50}), a trendy bar popular with locals and foreigners alike, uses a similar strategy. It has retained its Bulgarian name and identity, but has mixed it with Latin script, a global identity, and a visual symbol: a leaf intertwined with the letter <V>.

\begin{exe}
  \ex\label{slavova:ex:50} BAR BILKOVA est. 1991
  \glt ‘Herbal bar est. 1991’
\end{exe}


Another example of the same strategy is the optician’s shop called \textit{Ochila}
(Example \ref{slavova:ex:51}), the Bulgarian word for ‘glasses’, transliterated in Latin but completely opaque from the perspective of English. The Bulgarian word for ‘optician’ does not contribute to making the meaning any clearer, but a diacritic line connecting the O and C letters of OC\textit{hila}, %TODO: Halbes Wort kursiv?
imitating a graphic image of glasses, disambiguates the transliterated word.

\begin{exe}
  \ex\label{slavova:ex:51}
  \gll ОПТИКА OCHILA \\
  optika ochila \\
  \glt ‘Optician’s glasses’
\end{exe}


The reverse strategy is used in another optician’s shop, \textit{New Vision} (Example \ref{slavova:ex:52}). This time, the English name is written in Cyrillic, requiring not only familiarity with the Cyrillic script from foreigners, but also familiarity with the English language from Bulgarians, who would not be able to make sense of the name otherwise.

\begin{exe}
  \ex\label{slavova:ex:52}
  \gll ОПТИКА Ню Вижън \\
  optika new vision \\
  \glt ‘New Vision Optician’s’
\end{exe}

\subsection{Mixing languages and scripts}
In the next group of examples, we can see how the mixing of languages and visual signs works in practice. In \textit{бебеshore} (Example \ref{slavova:ex:53}), \autoref{slavova:fig:8}, a diminutive, informal, endearing term meaning a baby/toddler, there is a rather interesting case of language- and script-mixing. On the one hand, the word \textit{бебе} in Bulgarian means ‘baby’; \textit{shore} is spelled in English but functions as the Bulgarian diminutive suffix, -\textit{шор}. However, the English meaning of \textit{shore} is then activated through the visual symbol of waves at the end of the word. \textit{Bebster} has no particular meaning in Bulgarian and is a blend between \textit{baby} and \textit{hipster}.

\begin{exe}
  \ex\label{slavova:ex:53}
  \gll Бебеshore {| bebster style} \\
  bebeshor { } \\
  \glt ‘Little baby | baby style’
\end{exe}

\begin{figure}
  \includegraphics[width=0.65\textwidth]{Figure_8_Slavova}
  \caption{\textit{Bebeshore}. Photo: Emilia Slavova.}
  \label{slavova:fig:8}
\end{figure}


In \textit{Zelena Art Gallery} (Example \ref{slavova:ex:54}), there is code- and script-mixing between Bulgarian and English. The main word (\textit{Zelena}) is spelled almost in Cyrillic, with the exception of the first letter, <Z>, which is spelled in Latin script. The green colour of the sign plays a part in visually conveying the message.

\begin{exe}
  \ex\label{slavova:ex:54}
  \gll ZЕЛЕНА {| ART GALLERY} \\
  Zelena { } \\
  \glt ‘Green | Art Gallery’
\end{exe}


Code- and script-mixing are also used in a dairy shop, \textit{МЛЕКАРНИЦА Раково} (Example \ref{slavova:ex:55}). Everything is written in Bulgarian apart from one key word: \textit{milk}, spelled \textit{mилк}. The Bulgarian word is \textit{мляко} (\textit{mlyako}), not \textit{milk}. As the English word is used instead, it is transliterated in Cyrillic, with the exception of the first letter, <m>. While the upper-case letters <M> coincide in both languages, the lower-case ones differ, and <\textit{m}> in Bulgarian should be read as /t/ (\textit{tilk}). However, the graphic association with the popular chocolate brand \textit{milka} makes the code-mixing transparent and ensures that <m> is read as the Latin letter /m/, even though the rest is in Cyrillic.

\begin{exe}
  \ex\label{slavova:ex:55}
  \gll МЛЕКАРНИЦА Раково. Вкусът на млякото | mилк \\
  mlekarnitsa rakovo vkusat na mliakoto { } milk \\
  \glt ‘Dairy Rakovo. The taste of milk | milk’
\end{exe}

\subsection{Word play, double meaning}
The final group of signs use word play and double meaning as an attention-grabbing strategy. Such is the case with \textit{Ave New} (Example \ref{slavova:ex:56}), a pun on the English word of French origin \textit{avenue}, where the word is intentionally misspelled in order to evoke an association with something new – quite appropriate for a fashion shop.

\begin{exe}
  \ex\label{slavova:ex:56} AVE NEW
\end{exe}

\noindent
\textit{Bar Maze} (Example \ref{slavova:ex:57}) and \textit{Mazetto} 
(Example \ref{slavova:ex:58}) may seem to be in English/Ital\-ian, but both evoke a Bulgarian word: \textit{мазе} /maze/, meaning ‘basement’ and ‘the basement’ (Bulgarian adds the definite article at the end of words as a suffix, and –\textit{to} is the suffix for neuter gender). In spite of the maze drawn on the graphic sign of \textit{Bar Maze}, the Bulgarian word easily comes to mind for Bulgarians. The \textit{Mazetto} sign leads to a bar in an actual basement, so it is even easier to interpret it as a Romanized Bulgarian word. Few people would know the Italian word \textit{mazzetto} ‘bunch’, and the word itself was probably selected because of the phonetic similarity with a Bulgarian word, while referencing a (misspelled) Italian word.

\begin{exe}
  \ex\label{slavova:ex:57} BAR MAZE
  \ex\label{slavova:ex:58} MAZETTO | Vintage concept
\end{exe}

\noindent
The next sign, \textit{SupaStar} (Example \ref{slavova:ex:59}), \autoref{slavova:fig:9}, uses a Bulgarian word, \textit{supa}, a borrowing meaning ‘soup’, and makes it a part of an English compound, \textit{superstar}. Knowledge of Bulgarian is needed in order to work out what this establishment offers; otherwise, customers have to rely on the image of a bowl of soup in the middle of the round sign.

\begin{exe}
  \ex\label{slavova:ex:59}
  \gll {SupaStar |} Домашни Супи и Сандвичи \\
  { } domashni supi i sandvichi \\
  \glt ‘Soup Star | Homemade Soups and Sandwiches’
\end{exe}

\begin{figure}
  \includegraphics[width=0.65\textwidth]{Figure_9_Slavova}
  \caption{\textit{SupaStar}. Photo: Emilia Slavova.}
  \label{slavova:fig:9}
\end{figure}


The final example is \textit{Mandzha Stantsia: Zona Mexicana} (Example \ref{slavova:ex:60}), \autoref{slavova:fig:10}. \textit{Mandzha} is a very colloquial word for ‘cooked food’ in Bulgarian. While it feels as intimate and folksy as a Turkish borrowing, it is actually an Italian borrowing: \textit{mangia} is the imperative form of the Italian verb \textit{mangiare} ‘eat’. The combination of a noun modifying another noun is a typical English construction which has entered Bulgarian in recent years. The Bulgarian language requires an adjective rather than a modifying noun in this initial position. \textit{Stantsia} derives from \textit{station} and is a well-established word in Bulgarian, no longer perceived as a borrowing. \textit{Zona} ‘zone’ is another borrowing which is now firmly established and no longer feels foreign. The text is spelled in Cyrillic, but the message is a mixture of languages. The Bulgarian transliteration \textit{мехикана} /mehikana/ reflects the Spanish pronunciation. There is a reversal of noun and adjective which is rather untypical of Bulgarian syntax but is becoming more popular due to interference from other languages.

\begin{exe}
  \ex\label{slavova:ex:60}
  \gll Манджа станция. Зона мехикана \\
  mandzha stantsia zona mehikana \\
  \glt ‘Food station. Mexican zone’
\end{exe}

\noindent
The result of all this is a charming linguistic and cultural mash-up, perhaps serving as a metaphor for everything that has happened in the Bulgarian language after the firm grip of nationalist/communist ideology was shaken off, and the free linguistic market took its place.

\begin{figure}
  \includegraphics[width=0.65\textwidth]{Figure_10_Slavova}
  \caption{\textit{Mandza Station. Zona Mexicana.} Photo: Emilia Slavova.}
  \label{slavova:fig:10}
\end{figure}

\section{Quantitative analysis} \label{slavova:sec:8}
The quantitative analysis of the street signs on one particular street, Ivan Shishman Street (\autoref{slavova:tab:3}), reveals that of the 80 signs observed and analysed, multilingual signs prevail at 46\%. Combined with the bilingual signs in Bulgarian and English, the percentage of bilingual/multilingual signs increases to 54\%. Monolingual signs in Bulgarian make up only 17\% of the signs, closely followed by signs in English, at 15\%. The share of translingual signs is similar, at 14\%.

\begin{table}
  \begin{tabular}{lrr}\midrule\toprule
    Type & Number & Percentage \\\midrule
    Monolingual signs in Bulgarian & 14 & 17\% \\
    Monolingual signs in English & 12 & 15\% \\
    Bilingual signs in Bulgarian and English & 6 & 8\% \\
    Multilingual signs & 37 & 46\% \\
    Translingual signs & 11 & 14\% \\ \bottomrule\midrule
  \end{tabular}
  \caption{Street signs on Ivan Shishman Street}
  \label{slavova:tab:3}
\end{table}


It must be said, however, that the classification is rather arbitrary, as not all the signs can be unambiguously assigned to a single category. 

In many cases, it is only the different script that distinguishes a Bulgarian word from a foreign borrowing. 

The main conclusions that can be drawn from quantifying the signs on a central street in Sofia are that

%TODO: Könnte den Style ändern durch auskommentieren, wäre aber nicht konsistent
\begin{enumerate}%[label=*(\arabic*)]
  \item bilingual and multilingual signs prevail, making up more than half of the signs;
  \item under a fifth of the signs are monolingual signs in Bulgarian;
  \item English monolingual signs almost equal the number of Bulgarian monolingual signs;
  \item English is widely used as an international language but has not displaced many other foreign languages, which are also visible; and
  \item compared to the pre-1989 period, the translingual trend is steadily gaining ground.
\end{enumerate}


However, as discussed earlier, these results are related to the central location and trendiness of the selected street and may differ considerably in more marginal areas of the city or country.

\section{Discussion}
The linguistic landscape study of Sofia’s central streets reflects the rapid transition from a closed, centralized economy and state-controlled, heavily ideologized naming practices to an open, free-market economy, where individual shop owners use a wide range of linguistic resources to reach a wider customer base.

The data show a clear shift from a monolingual norm, dominant under socialism and part of the state’s ideology of national homogeneity, towards multilingual and translingual trends, characteristic of the new openness and global outlook. 

It would be easy to assume that the global status of English would allow it to dominate the linguistic landscape and to suppress all other languages, but this is not the case. The languages identified in the corpus include Italian, French, Spanish, Turkish, Russian, German, and Arabic, as well as some unidentified languages. English is used widely and functions as a neutral language, a lingua franca, indexing other cultures (Greek, Turkish, Indian, Thai, Russian, etc.), without necessarily being connected to either British or American culture. This shows that in spite of the explosive success of English in Bulgaria in recent decades, the country has opened up to many other languages and cultures, and Sofia’s streets are communicating this to locals and visitors alike. 

Borrowings, code-switching, language- and script-mixing, blending, double meaning, and word play are typical strategies used in addressing Bulgarian and international customers and evoking a mixture of local and global identities. In spite of the relatively small share of translingual signs found in the quantitative corpus, the pattern is clear: once languages have been allowed to coexist, mix, and mesh in the streets freely, they do so in different ways and to varying degrees of entanglement. 

The reasons to use the above-mentioned languages and mixing strategies can be summarized as follows:

\begin{enumerate}
  \item Addressing English-speaking travellers and visitors to the city;
  \item Addressing outward-oriented Bulgarian customers;
  \item Attention-grabbing via some unusual choice of language or word;
  \item Signalling a global orientation and a cosmopolitan identity;
  \item Retaining a local Bulgarian character within this global identity.
\end{enumerate}


The analysis of the corpus shows that central Sofia today is a vibrant, multilingual, and cosmopolitan destination like many other cities in Europe; a far cry from the isolated, ideologized, monolingual place it used to be under communism. Most of the Soviet references are gone; with the exception of \textit{Park Hotel Moskva}, all the other establishments mentioned above have changed their names and purpose. \textit{Cinema Moscow} is a shopping centre. The former \textit{Lenin} bookshop was a \textit{Happy Sushi} restaurant until recently; now it is an investment management company called \textit{Expat Capital}. 

In spite of the obvious influence of American-dominated globalization in recent years, there are few explicit references to American culture and multiple references to other cultures through the medium of the English language and the Latin script. The share of monolingual signs in Bulgarian is small, but in most multilingual signs, there is a tendency to index Bulgarian identity within a larger global identity. Cyrillic script is blended with foreign words and vice versa, in an effort to transcend national and linguistic boundaries and play with languages and meanings in creative ways. 

And even though there has been a disturbing nationalistic and anti-globalist wave in recent years, this has not affected the linguistic landscape of central Sofia in any significant way. Instead of the expected overt displays of national identity, there is a clear cosmopolitan identity with a distinctly Bulgarian flavour.

\section{Conclusion}
This study confirms that political and historical upheavals have serious linguistic consequences. This is clearly seen in the linguistic landscape of Sofia during the transition from a totalitarian communist state to a free-market economy. 

Monolingualism is only viable under strict rules and a nation-state ideology which enforces it and suppresses other languages. If left alone, languages will mix and mesh, transform, and become hardly distinguishable from each other. Under a free linguistic economy, the monolingual paradigm is displaced by the multilingual and translanguaging paradigm in a celebration of open borders, free movement, and diversity. 

As Russia invaded Ukraine at the time of revising this chapter, what seemed like a distant past no longer feels so distant. The traumatic history of an unjustified foreign invasion, the memories of the suppression of cultural and linguistic diversity, and the fear of the return of a highly ideologized, repressive political and linguistic regime have all become particularly salient. Sociolinguists have a significant role to play in witnessing and recording the linguistic expressions of social and political upheavals – and a duty to issue a stark warning about the dangers of history repeating itself and totalitarianism reawakening. 



\printbibliography[heading=subbibliography, notkeyword=this]

\end{document}
