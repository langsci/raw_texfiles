\documentclass[output=paper]{langscibook}
\ChapterDOI{10.5281/zenodo.10497385}

\author{Inga Hennecke \orcid{0000-0002-9079-4849} \affiliation{University of Tübingen}}

\title[Systems of pragmatic markers in contact]{Systems of pragmatic markers in contact: Processes and outcomes}

\abstract{
    Pragmatic markers are highly polyfunctional and polysemic lexical units that generally occur in sentence-peripheral positions and do not contribute to the propositional content of an utterance. In situations of language contact, pragmatic markers are particularly susceptible to borrowing and other cross-linguistic influences because of their syntactic and semantic detachability. This paper presents a corpus-based analysis of the influence of language contact with English on the system of pragmatic markers in spoken Manitoban French, a variety of Canadian French spoken in Manitoba. To this aim, three sets of partially equivalent pragmatic markers were chosen for analysis: \textit{comme} and \textit{like}; \textit{alors}, \textit{donc}, and \textit{so}; and \textit{bon}, \textit{ben}, and \textit{well}. The analysis shows vastly different outcomes of long-term language contact on specific markers in one system. Four outcomes are discussed in this paper; namely, the emergence of new discourse-pragmatic functions, the borrowing of a marker from the other language, changes in frequency and productivity of specific markers, and the absence of specific markers in the system.
}

\IfFileExists{../localcommands.tex}{
  \addbibresource{../localbibliography.bib}
  % add all extra packages you need to load to this file

\usepackage{tabularx,multicol}
\usepackage{url}
\urlstyle{same}

\usepackage{listings}
\lstset{basicstyle=\ttfamily,tabsize=2,breaklines=true}

\usepackage{langsci-basic}
\usepackage{langsci-optional}
\usepackage{langsci-lgr}
\usepackage{langsci-osl}
% \usepackage{./langsci/styles/langsci-lgr}
% \usepackage{./langsci/styles/langsci-osl}
% \usepackage{langsci-gb4e}

\usepackage{tikz}
\usetikzlibrary{patterns,calc}
\pgfdeclarepatternformonly{south east lines}{\pgfqpoint{-0pt}{-0pt}}{\pgfqpoint{3pt}{3pt}}{\pgfqpoint{3pt}{3pt}}{
    \pgfsetlinewidth{0.6pt}
    \pgfpathmoveto{\pgfqpoint{0pt}{3pt}}
    \pgfpathlineto{\pgfqpoint{3pt}{0pt}}
    \pgfpathmoveto{\pgfqpoint{.2pt}{-.2pt}}
    \pgfpathlineto{\pgfqpoint{-.2pt}{.2pt}}
    \pgfpathmoveto{\pgfqpoint{3.2pt}{2.8pt}}
    \pgfpathlineto{\pgfqpoint{2.8pt}{3.2pt}}
    \pgfusepath{stroke}}
    
\usepackage{stmaryrd}
\usepackage{wasysym}
\usepackage{multirow}
\usepackage{caption}
\usepackage{subcaption}
\usepackage{mathrsfs}
\usepackage{qtree}

\usepackage{linguex}


  %pminos do not split footnotes
% \interfootnotelinepenalty=10000 %Footnote in Laporte chapters has to be split SN


%\DeclareIndexNameFormat{default}{%
%\nameparts{#1}%
%\usebibmacro{index:name}%
%{\index[names]}%
%{\namepartfamily}%
%{\namepartgiveni}%
% {}% L1
% {}% L2
%{\namepartprefix}% generates spurious space L3
%{\namepartsuffix}% generates spurious space L4
%}

%  {\DeclareIndexNameFormat{default}{%
%     \usebibmacro{index:name}{\index[names]}{#1}{#3}{#5}{#7}}}

%\DeclareIndexNameFormat{default}{%
%  \usebibmacro{index:name}{\sindex[nom]}{#1}{#3}{#5}{#7}}

%\DeclareIndexNameFormat{default}{%
%  \usebibmacro{index:name}{\sindex[person]}{#1}{#3}{#5}{#7}}
%\DeclareIndexNameFormat{default}{%
%\nameparts{#1} \usebibmacro{index:name}{\sindex[person]]}{\namepartfamily}{‌​\namepartgiven}{\nam‌​epartprefix}{\namepa‌​rtsuffix}}

%\newcommand{\smiley}{:)}

%\renewbibmacro*{index:name}[5]{%
%\usebibmacro{index:entry}{#1}%
%{\iffieldundef{usera}{}{\thefield{usera}\actualoperator}\mkbibindexname{#2}{#3}{#4}{#5}}}

% \newcommand{\noop}[1]{}

%remove for final
%\overfullrule=1mm

\newcommand{\tobi}[2]}}
\renewcommand{\S}[1]{\tobi{#1}{\textsc{*}}}

% this volume references
% puts: [this volume]
% already defined: \citetv
%\newcommand{\citepv}[1]{(\citeauthor{#1} \citeyear*{#1} [this volume])}
\newcommand{\citealtv}[1]{\citeauthor{#1} \citeyear*{#1} [this volume]}

%parentheses around example number
\newcommand{\pref}[1]{(\ref{#1})}

% in-text examples

\newcommand{\lnex}[1]{\textit{#1}} %target lang word
\newcommand{\lnlit}[1]{(lit.: `#1')} %literal reading
\newcommand{\lnlat}[1]{(#1)} % latinization
\newcommand{\lntrans}[1]{`#1'} %translation
\newcommand{\lnexl}[2]%
{\lnex{#1}{} \lnlat{#2}} % ex with latinization
\newcommand{\lnexlat}[3]{\lnex{#1}{} \lnlat{#2}{} \lntrans{#3}} % ex with latinization and tranl.

%ch01
\newcommand{\co}[1]{\mbox{\textbf{#1}}}

%ch09

\newcommand{\cyrbulg}[1]{\begin{otherlanguage*}{bulgarian}#1\end{otherlanguage*}}


%ch10
\newcommand{\nlp}{{\small NLP}}
\newcommand{\mwe}{{\small MWE}}
\newcommand{\rae}{{\small RAE}}
\newcommand{\lvc}{{\small LVC}}
\newcommand{\pos}{{\small P}o{\small S}}
%\newcommand{\todo}[1]{ \textcolor{red}{#1} }

%\renewcommand{\labelenumi}{\theenumi}
%\ainamefmt{{vv}{ll}{, ff}{, jj}} % fullname

\newcommand{\biberror}[1]{{\color{red}#1}}

\newcommand{\osenovaitem}{--~} 
  %% hyphenation points for line breaks
%% Normally, automatic hyphenation in LaTeX is very good
%% If a word is mis-hyphenated, add it to this file
%%
%% add information to TeX file before \begin{document} with:
%% %% hyphenation points for line breaks
%% Normally, automatic hyphenation in LaTeX is very good
%% If a word is mis-hyphenated, add it to this file
%%
%% add information to TeX file before \begin{document} with:
%% %% hyphenation points for line breaks
%% Normally, automatic hyphenation in LaTeX is very good
%% If a word is mis-hyphenated, add it to this file
%%
%% add information to TeX file before \begin{document} with:
%% \include{localhyphenation}
\hyphenation{
    Beck-man
    Ngu-yen
    back-chan-nel
    back-chan-nels
    mo-not-o-nous
    ste-reo-typ-i-cal
}

\hyphenation{
    Beck-man
    Ngu-yen
    back-chan-nel
    back-chan-nels
    mo-not-o-nous
    ste-reo-typ-i-cal
}

\hyphenation{
    Beck-man
    Ngu-yen
    back-chan-nel
    back-chan-nels
    mo-not-o-nous
    ste-reo-typ-i-cal
}
 
  \togglepaper[9]%%chapternumber
}{}

\begin{document}
\maketitle

\section{Introduction}
Pragmatic markers remain a controversial topic in scientific discussion and there is still no consensus on their exact classification, delimitation, and definition (for a detailed overview, see \citealt{MosegaardHansen.1998, Andersen.2001,Aijmer.2002, Aijmer.2006}). Pragmatic markers, which are often also referred to as discourse markers or discourse particles, are highly polyfunctional lexical units that may contain a high number of polysemic semantic meaning patterns. They also demonstrate syntactic flexibility and often occur in sentence-peripheral positions. Pragmatic markers generally fulfill discourse-pragmatic functions and do not contribute to the propositional content of the utterance. In that sense, they do not assume a grammatical relation to the other elements of the utterance (see \citealt{Hennecke.2014} for a detailed discussion). Further, some pragmatic markers, such as the English \textit{like}, may also fulfill several types of hedging functions, particularly approximation and attenuation (see \citealt{Kaltenbock.2010}).

The focus of the scientific debate on pragmatic markers has mostly centered around the characteristics and functions of pragmatic markers among monolingual native speakers. However, a number of studies focus on cross-linguistic issues of pragmatic markers, such as the role of pragmatic markers in situations of language contact and the implications of language contact on pragmatic markers (e.g., \citealt{Mougeon.1991}; \citealt{Maschler.2000}; \citealt{Matras.2000}; \citealt{Hlavac.2006}; \citealt{Torres.2008}). This paper aims to investigate three pairs of English and French pragmatic markers in Manitoban French, a variety of Canadian French that is spoken in the Canadian province of Manitoba. Manitoban French has been in an intensive, long-term language-contact situation with English and is therefore an ideal source for the investigation of the evolution of pragmatic markers in language contact (for more detailed information on the language contact situation in Manitoba, see \cite{Hennecke.2014}).


\section{Pragmatic markers in contact}\label{hennecke:sec:2}
Pragmatic markers are considered very susceptible to borrowing and other cross-linguistic influences because of their syntactic and semantic detachability (see Section \ref{hennecke:sec:3.2}). Still, this is not the only reason researchers are interested in bilingual pragmatic markers. It is widely considered that pragmatic markers tend to be difficult to translate from one language to another and have more than one translation equivalent. Furthermore, determining their semantic value and pragmatic functions is often a challenge, and it is not even unequivocally clear if they encode conceptual or procedural meaning. Another interesting factor comes from diachronic analysis of pragmatic markers, in that the moment of the emergence of their discourse-pragmatic functions is generally hard to determine. This is because most pragmatic markers emerged through processes of pragmaticalization from already existing lexical items (see \citealt{Aijmer.1997, Dostie.2004}). But it is also due to the fact that pragmatic markers generally occur in spoken language, and often only in very informal speech. Therefore, it is difficult to retrace their diachronic evolution by means of written corpus data. 

While it is evident that pragmatic markers have specific characteristics in comparison to other lexical units, it is still not clear what happens to them in situations of strong language contact. \citet{Clyne.1972} investigated the borrowing of German pragmatic markers in the English discourse of German-speaking immigrants in Australia. \citet{Mougeon.1991} examined the borrowing of English pragmatic markers in Canadian French discourse. There is a wide range of examples that account for the influence of the pragmatic markers from one language on the other language in situations of intensive and/or long-term language contact. \citet[264]{Torres.2008} attempted to classify the possible outcomes of pragmatic markers in contact as follows:

\begin{enumerate}
    \item The two sets of discourse markers will coexist.
    \item Similar markers from each language will acquire differentiated meanings.
    \item The markers from one language may replace those of the other language.
\end{enumerate}

To support this classification, they cite different examples of studies on pragmatic markers in language contact. As an example of the first case, they mention \citegen{Hill.1986} work on Spanish in contact with Mexicano  and \citet{Brody.1987}, who analyzed Spanish in contact with different indigenous languages, for example Mexicano and Mayan \citep[264]{Torres.2008}. As an example of the second case, they cite \citet{Solomon.1995} and her work on Spanish in contact with Yucatec \citep[265]{Torres.2008}. The third outcome of pragmatic markers in contact was examined by \citet{Goss.2000} in their work on Texas German. In this case, the whole German set of markers was replaced by English markers \citep{Goss.2000}.

The peculiarities of pragmatic markers in language contact are mostly due to their general characteristics. Still, it is unclear if the three outcomes mentioned above are mutually exclusive or if long-term, intensive language contact always results in outcome number three; namely, the complete replacement of one set of pragmatic markers. The opposite option would be that sets of pragmatic markers from two languages might co-occur over an extended period of time without having too strong an influence on each other. This research question will be investigated by means of data from Manitoban French in Section \ref{hennecke:sec:3} of this paper. Previous research on pragmatic markers in contact has mainly focused on individual markers in one contact variety or on one specific outcome of a contact situation. 

The English marker \textit{so} (see also Section \ref{hennecke:sec:3.2}) has been investigated as a potential case of borrowing in different contact situations. \citet{Mougeon.1991} assume that all markers first occur in the respective other language as code switches and then gradually become borrowings. \citet[199]{Mougeon.1991} state that \textit{so} in Ontarian French can be regarded as a core lexical borrowing. They assert that the use of \textit{so} in Canadian French discourse may be due to intensive, long-term language contact because it is particularly prominent in Canadian French varieties that have experienced strong language contact with English \citep[201]{Mougeon.1991}. Although they argue that the “degree of bilingualism is a poor predictor of variation in \textit{so} usage” \citep[201]{Mougeon.1991}, they still find that \textit{so} is mostly used by speakers who have the most contact with English in their everyday lives. This fact leads them to the assumption that “core lexical borrowings like \textit{so} or other sentence connectors may start out as code switches (either as single words or as part of switched sentences) which by dint of repetition become loanwords” \citep[211]{Mougeon.1991}. That is to say, more fluent bilingual speakers introduce the English marker to French discourse and less fluent speakers repeat this linguistic behavior. Still, they consider an explanation proposed in the work of Weinreich \& Haugen under which these kinds of borrowings emerge through the “acculturation of bilingual speakers who experience high levels of contact with a superordinate language” \citep[212]{Mougeon.1991}. The phenomenon of integrating English pragmatic markers in the discourse of another language is by no means restricted to the language pair English–French. Several authors have investigated the language contact of Spanish in the US, in particular the use of \textit{so} and its Spanish equivalent \textit{entonces} in the Spanish discourse of bilingual speakers \citep{SilvaCorvalan.1995,Aaron.2004,Torres.2002,Lipski.2005,Torres.2008}. The same phenomenon can even be proved for other bilingual speakers, such as Croatian–English bilinguals \citep{Hlavac.2006}. \citet{Hlavac.2006} explains the frequent occurrences of \textit{so} in bilingual Croatian discourse by the marker’s polyfunctionality. This characteristic cannot be assigned to its Croatian equivalents \citep[1896]{Hlavac.2006}. 

The present corpus analysis aims to investigate a specific set of pragmatic markers in contact in more detail to identify distinct types of language-contact phenomena. To this aim, this paper focuses on different processes of contact-induced language change in a language-contact situation. According to \citet[2]{Heine.2005}, “contact-induced influence manifests itself in the transfer of linguistic material from one language to another”. When talking about cross-linguistic change, \citeauthor{Heine.2005} assume a model language (also referred to as the source language), providing the pattern for transfer, and a replica language (also referred to as the target or borrowing language), receiving the pattern. This paper follows \citet{Heine.2010} in their terminology for the main types of contact-induced linguistic transfer, which are borrowing and replication (see \citealt[87]{Heine.2010}). In this sense, the cross-linguistic transfer that affects “meanings (including grammatical meanings or functions) or combinations of meanings” \citep[2]{Heine.2005} will be referred to as \textit{replication}, while cross-linguistic transfer that affects “form–meaning units or combinations of form–meaning units” \citep[2]{Heine.2005} will be termed \textit{borrowing}. Both processes will be described in more detail in the following part of this paper.

\section{Pragmatic markers in contact in Manitoban French}\label{hennecke:sec:3}
Due to its sociolinguistic and historical evolution, the French language in Manitoba has been exposed to a strong, long-term influence of English for more than two centuries. French has long been a minority language in the province of Manitoba. At the end of the nineteenth century, all French schools were banned in Manitoba, and for most of the twentieth century, the French community was not allowed to teach their children in French. It was only in 1979 that the French community regained the same rights as the English community and French became an official language de jure (see \citealt{Hennecke.2014} for a detailed sociohistorical description). Still, Manitoban French remains a \textit{de facto} minority language that is only spoken by 3.2 percent of the population of Manitoba (Statistics Canada 2016).\footnote{\url{https://www150.statcan.gc.ca/n1/pub/89-657-x/89-657-x2019014-eng.htm} (accessed on 20.11.2021)}  
Therefore, Manitoban French has been and continues to be strongly influenced by English.


The following analysis is based on a corpus of spoken Manitoban French, the FM Corpus (see \citealt{Hennecke.2014} for a detailed presentation and discussion of the transcriptions and the corpus data). The corpus data consist of recordings of informal everyday conversations. They contain 35,660 tokens, divided into 15 communications from 20 speakers. The corpus data of the FM Corpus were collected in 2010 and 2012 in St. Boniface, the French quarter of Winnipeg, and consist of two-thirds of French utterances, while English utterances only make up one-third of the data. The transcriptions of the corpus data are based on the HIAT convention to transcribe spoken data (\textit{Halbinterpretative Arbeitstranskriptionen} ‘Semi-Interpretative Working Transcriptions', see \cite{Ehlich.1976}).\footnote{See \url{https://exmaralda.org/de/hiat} for a detailed description of the HIAT transcription convention and a transcription manual.} All speakers in the corpus are aged between 17 and 30 and were born and raised in the Franco-Manitoban environment of St. Boniface or its neighboring districts. All speakers indicated French as their mother tongue and can be identified as balanced bilingual speakers of English and French. In the following examples from the corpus data, each speaker is referred to using an anonymous speaker ID (e.g., DM, ZA, GR). Due to the prevailing sociolinguistic circumstances, all speakers were regularly exposed to English and French in everyday life from an early age. Sections \ref{hennecke:sec:3.1} to \ref{hennecke:sec:3.3} will present and discuss selected examples of French and English markers in the FM Corpus; namely, \textit{comme} and \textit{like}; \textit{alors}, \textit{donc}, and \textit{so}; and \textit{bon}, \textit{ben}, and \textit{well}. The three sets of markers were selected because of their occurrence in the FM data (see Section \ref{hennecke:sec:3.4} for further discussion). All examples in Sections \ref{hennecke:sec:3.1} to \ref{hennecke:sec:3.3} are taken from the FM Corpus, unless indicated otherwise. Section \ref{hennecke:sec:3.4} will then discuss the absence of certain pragmatic markers in the FM Corpus data.

\subsection{The markers \textit{comme} and \textit{like}}\label{hennecke:sec:3.1}
In European French, the conjunction and adverb \textit{comme} ‘like’ is a highly multifunctional lexical unit. Even in its diachronic evolution, the lexical unit \textit{comme} has expanded semantic patterns and developed several functions that can also be found in its Portuguese and Spanish counterpart \textit{como} and its Italian counterpart \textit{come} \citep{Mihatsch.2009a}. In Canadian French, the lexical unit \textit{comme} shows some important peculiarities. Recent findings suggest that \textit{comme} has developed functions that are not attested for \textit{comme} in European French. The most salient new functions of \textit{comme} include the extension in its use as a hedge (e.g., a quantitative approximation marker) and its use in quotation. New functions that have been detected in Canadian French include quantitative approximation, or the rounder function according to \citet{Prince.1982}, and the indirect discourse and autocitation functions, which will be defined as \textit{quotative} functions in the following, and the assertion function. In current research on \textit{like}, this function is, from a syntactic perspective, commonly referred to as sentence-final use, or, from a pragmatic perspective, as focus function (see \citealt{Underhill.1988}). Depending on the pragmatic function of \textit{like} in the specific utterance, assertion can also comprehend shield functions, according to the terminology of \citet{Prince.1982}. In European French, \textit{comme} cannot fulfill this set of functions. Therefore, the question arises as to which underlying process of language change can be identified for the new meanings and functions of \textit{comme} in Canadian French. 

In spoken English, \textit{like} functions as a highly polysemous and syntactic flexible lexical unit that can take several discourse functions. According to \citet[49]{Meehan.1991}, \textit{like} has been known in its function as a conjunction since the fourteenth century, and has since developed new functions, such as its use in exemplification and its different discourse functions, such as focus and quotative. Besides these functions, \textit{like} can also appear as a hedge and a hesitation marker. Some of the discourse functions of \textit{like}, such as its use as a focus and quotative marker, only emerged in more recent times and became very frequent in spoken language. This frequency in spoken language, initially restricted to young speakers of American English, rapidly expanded to other sociolinguistic groups and other varieties of English spoken outside the US. This rapid evolution of language change took place in the second half of the twentieth century (e.g., \citealt{Buchstaller.2002,Buchstaller.2009,Vandelanotte.2009}). 

The markers \textit{comme} and \textit{like} both seem to be very frequent in Manitoban French. In the FM Corpus data, \textit{comme} occurs 577 times, whereas \textit{like} appears 255 times. When regarding the occurrences of \textit{like} in the FM Corpus, 237 items out of 255, or 93 percent, appear in the function of a pragmatic marker. The other 7 percent include occurrences of \textit{like} as a verb or as a comparison marker. For \textit{comme}, 554 occurrences of the item, or 96 percent, appear in the function of a pragmatic marker. 

The FM Corpus data show that \textit{like} takes functions that are commonly attested for American and Canadian French \citep{DArcy.2017}. The English marker \textit{like} is highly flexible and can appear in various positions of an utterance and mark a large scope, as in \xref{hennecke:ex:1}.

\begin{exe}
    \ex\label{hennecke:ex:1} DM: \textit{like} what if • all the services and everything was already done in French you know.
\end{exe}

\noindent
The marker \textit{like} in the FM Corpus commonly takes various hedging functions, which are either lexical \xref{hennecke:ex:2} or numeric \xref{hennecke:ex:3} approximation.

\begin{exe}
    \ex\label{hennecke:ex:2} ZA: • I wanna do a \textit{like} solo album pendant l’hiver.
    \glt ZA: • I wanna do a \textit{like} solo album during winter.
    \ex\label{hennecke:ex:3} GR: well he’s got • his eeh wine cellar in the basement he’s got about \textit{like} forty bottles.
\end{exe}

\noindent
\textit{Like} in the FM Corpus also functions as a focus marker, highlighting specific focal information, as in \xref{hennecke:ex:4}.

\begin{exe}
    \ex\label{hennecke:ex:4} ZA: ben c’est un film d’une heure et demi hein? • \textit{comme} it’s a feature length • • • deal • it’s \textit{like} huge • •
    \glt ZA: ‘well, it’s an hour and a half film, right? •  \textit{like} it’s a feature length • • •  deal • it’s \textit{like} huge • •’
\end{exe}

Further, in the FM Corpus, \textit{like} can fulfill different quotative functions that are commonly attested in different varieties of English, for example introducing quoted speech \xref{hennecke:ex:5}, quoted thought \xref{hennecke:ex:6}, or quoted attitude \xref{hennecke:ex:7}.

\begin{exe}
    \ex\label{hennecke:ex:5} DM: I called his eh constituency office the other day and I said I wanted to meet with them • • I was \textit{like} “I wanna meet you” eh • they/ they phoned back and they said that it was all booked up for September I’m \textit{like} “that’s fine because I’m on a trip.”
    \ex\label{hennecke:ex:6} ZA: alors c’est eux qui s’occupent de la distribution c’est/ they do all the work for me and submit it to festivals so I’m \textit{like} “oh good I don’t have to worry about this.”
    \glt ZA: ‘so they’re the ones who take care of the distribution / they do all the work for me and submit it to festivals so I’m \textit{like} “oh good I don’t have to worry about this.”’
    \ex\label{hennecke:ex:7} WIL: and then • all of a sudden • she’s \textit{like} “what a great Francophone scene we have there is in Winnipeg.”
\end{exe}

\noindent
\textit{Comme} in Manitoban French can take a hedging function with a large scope \xref{hennecke:ex:8} and mark numeric approximation \xref{hennecke:ex:9}.

\begin{exe}
    \ex\label{hennecke:ex:8} DM: (\dots) in the meantime euhm je travaillais juste à \textit{comme} • produire le document lui-même.
    \glt DM: ‘(…) in the meantime euhm I was just working on \textit{like} • producing the document itself.’
    \ex\label{hennecke:ex:9} CAR: ça fait \textit{comme} cinq fois qu’ (elle) • te prend avec ça ((0.8s)) un joke
    \glt CAR: ‘it’s \textit{like} five times that she • caught you with it ((0.8s)) a joke.’
\end{exe}

\noindent
The French marker \textit{comme} may also take focus functions in Manitoban French, as in \xref{hennecke:ex:10}.

\begin{exe}
    \ex\label{hennecke:ex:10} CAR: elle est vraiment \textit{comme} la meilleure artiste de nos jours
    \glt CAR: ‘she is really \textit{like} the best artist of our time’
\end{exe}

\noindent
With regard to the FM Corpus, one can observe a striking use of \textit{être comme} ‘be like’ as a quotative. As in different varieties of English, \textit{être comme} can fulfill the functions of introducing quoted speech \xref{hennecke:ex:11}, quoted thought \xref{hennecke:ex:12}, and quoted attitude \xref{hennecke:ex:13}.

\begin{exe}
    \ex\label{hennecke:ex:11} GER: so Joey \textit{était} \textit{comme} • • “how about premier novembre?” ils \textit{sont} \textit{comme} “ok” • • he (did) a writing • he \textit{est} \textit{comme} “perfect”
    \glt GER: ‘so Joey \textit{was like} • • “how about first November?” they \textit{are like} “ok” • • he (did) a writing • he \textit{is like} “perfect”’
    \ex\label{hennecke:ex:12} WIL: • • ça c’est la rumeur qui/ qui passe maintenant puis \textit{j’suis} \textit{comme} “I don’t care I’m getting her out.”
    \glt WIL: • • ‘that’s the rumour that’s going around now and \textit{I’m} \textit{like} “I don’t care I’m getting her out.”’
    \ex\label{hennecke:ex:13} GR: là tout le monde \textit{est} \textit{comme} “oh my god”.
    \glt GR: ‘here everyone \textit{is like} “oh my god”.’
\end{exe}

\noindent
Further, different varieties of English show the use of a quotative form, \textit{go like}, and it is possible to observe the equivalent form, \textit{aller comme}, in the FM Corpus.

\begin{exe}
    \ex\label{hennecke:ex:14} DAN: I don’t know • mais quand tu fais une fau(te) ça \textit{va comme} “cling cling cling” • and then ça ça va venir and then on va être tout frustrés and then on va • • casser les guitars.
    \glt DAN: ‘I don’t know • but when you do a mistake \textit{it goes} like “cling cling cling” • and then it’ll come and then we’ll get all frustrated and then we’ll • • break the guitars.’
\end{exe}


In conclusion, the analysis of the markers \textit{comme} and \textit{like} revealed that the European French equivalent \textit{genre}, a comparably new pragmatic marker in spoken European French, is not present in the corpus data. The marker \textit{comme} has developed new meanings and functions, such as its use as quotative, hedging, and numeric approximation, which, at first sight, appear to be replicated from the English \textit{like}. Following \citet{Heine.2005,Heine.2010}, replication may “in the same way affect morphological, syntactic, and pragmatic structures, the noun phrase, and the verb phrase in the same way as the organization of clauses and clause combining” \citep[261]{Heine.2005}. Furthermore, \citet[89]{Heine.2010} state that grammatical replication contrasts with borrowing in that it does not include the transfer of phonetic material, which is a crucial point in borrowing. Hence it is important to differentiate grammatical replication from polysemy copying, also called calquing or loan translations, in which a meaning is only copied. According to \citeauthor{Heine.2012}, “polysemy copying can be described as an abrupt rather than a gradual change, and it tends to be associated with lexical rather than grammatical replication” (\citeyear[126]{Heine.2012}).

In conclusion, it is not possible to term the process of the extension of pragmatic functions and semantic meaning patterns of \textit{comme} in Manitoban French unequivocally as contact-induced language change or even replication in the sense of \citet{Heine.2005,Heine.2010}. Similar processes have been reported in other Romance languages (e.g., \citealt{Mihatsch.2009a}). Therefore, language contact might also be just one specific factor among others that accelerated the process of language change in Manitoban French compared to European French (for a detailed discussion, see \citealt{Hennecke.2014}).

\subsection{The markers \textit{alors}, \textit{donc}, and \textit{so}}
\label{hennecke:sec:3.2}
In current research on French pragmatic markers, \textit{alors} and \textit{donc} have commonly been treated together. \citeauthor{MosegaardHansen.1998} explains this by the fact that both markers “originate in temporal anaphoric expressions” and that both are “frequently used in argumentational structures, where they mark a result or a conclusion” (\citeyear[321]{MosegaardHansen.1998}). In modern French, \textit{donc} has completely lost its original temporal use but has developed new discourse-pragmatic functions. According to \citet[165]{MosegaardHansen.1997}, \textit{donc} has two main functions: marking a conclusion, a consequence, or a result and marking repetitions such as reformulations, paraphrases, or summaries. Furthermore, \textit{donc} can take emphatic functions that are not restricted to imperative phrases, as shown for Old French \citep[329]{MosegaardHansen.1998}. For \textit{donc} as a marker of discourse structuring, \citet{Bolly.2009} establish a differentiation between its syntactic-semantic functions and its discourse functions. As a syntactic-semantic function, they list the use of \textit{donc} as a conclusion or consequence marker \citep[7]{Bolly.2009}. As a discourse function, they specify the use of \textit{donc} as a repetition marker, as a marker of participative transition, and as a marker of conceptual structuring \citep[12]{Bolly.2009}. Thereby, they distinguish two kinds of repetition markers: a repetition marker of conclusive orientation that includes a recapitulation, and a repetition marker that implies reformulation and explications \citep[12]{Bolly.2009}. In contrast, \textit{donc} as a marker of conceptual structuring marks a reorientation toward a new subject or a subject that has been mentioned earlier in the conversation \citep[9]{Bolly.2009}. 


Regarding the French \textit{alors}, three functions can commonly be distinguished – the temporal, the causal, and the discourse-structuring functions (see \citealt{Degand.2011,LeDraoulec.2007,MosegaardHansen.1997}). The marker \textit{alors} can function as a consequence or result marker in Modern French \citep[9]{Degand.2011}. As a causal marker, \textit{alors} still contributes to the propositional content of an utterance. This is not the case for the metadiscursive use of \textit{alors}, where the marker only modifies the illocutionary force of the utterance and “can be left out without changing the semantic content” \citep[15]{Degand.2011}. As a metadiscursive device, \textit{alors} structures discourse or introduces new topics or topic shifts. 

Apart from the function of marking results or conclusions, \citeauthor{MosegaardHansen.1998} highlights two main functions of \textit{alors}: marking reperspectivization or reorientation (\citeyear[335]{MosegaardHansen.1998}) and marking foregrounding (\citeyear[348]{MosegaardHansen.1998}). By reperspectivization or reorientation, Mosegaard Hansen understands the uses of \textit{alors} as a structuring device, to introduce a new topic, parentheses, or citations. Furthermore, in this function, \textit{alors} can be used as a topic and discourse starter \citep[335]{MosegaardHansen.1998}. When foregrounding, \textit{alors} marks “transitions from more backgrounded to more foregrounded material, especially, but not exclusively in narrative” \citep[348]{MosegaardHansen.1998}. 

From a crosslinguistic perspective, the English marker \textit{so} is generally considered the translation equivalent of the French markers \textit{alors} and \textit{donc}. The marker \textit{so} is among the best-investigated pragmatic markers in the English language. It is well known that \textit{so} is a highly multifunctional particle that can occur in different grammatical and discourse-pragmatic functions. \citet{Schiffrin.1987} detects two separate but not exclusive functions of \textit{so}. Firstly, she points out the function of \textit{so} as a causal marker that connects propositional content or illocutionary acts. Secondly, she focuses on the purely discursive functions of \textit{so} as an interaction marker \citep[218]{Schiffrin.1987}. In addition, in informal speech, “\textit{so} is a turn-transition device which marks a speaker’s readiness to relinquish a turn” \citep[218]{Schiffrin.1987}. \citet{Bolden.2009} clearly distinguishes the inferential use of \textit{so} from its utterance-initial functioning. According to \citeauthor{Bolden.2009}, \textit{so} can be seen as a marker of “emergence of incipiency” and “is a resource for establishing discourse coherence and […] accomplishing understanding” (\citeyear[996]{Bolden.2009}). The utterance-initial \textit{so} is “used in contexts where a particular course of action is oriented to by the interlocutors as having been pending or relevantly missing” and “on turn constructional units that pursue abandoned or interrupted interactional projects” \citep[996]{Bolden.2009}. In conclusion, the marker \textit{so} generally functions as a consequence, result, or conclusion marker, to introduce a recapitulation or reformulation of something said earlier, or to mark reorientation and reperspectivization (e.g., topic starting or topic changing). 

As mentioned in Section \ref{hennecke:sec:2}, \textit{so} has already been analysed as a possible borrowing in different situations of language contact. Here, most authors claim that \textit{so} in the discourse of bilingual speakers is a fully integrated loan or a core borrowing. It can be stated that most researchers regard the transfer of form-meaning units of pragmatic markers as a case of borrowing (e.g., \citealt{Mougeon.1991,SilvaCorvalan.1995,Torres.2002,Torres.2008}). In contrast, \citeauthor{Lipski.2005} argues that the insertion of the English \textit{so} into Spanish discourse is a case of “metalinguistic bracketing” that is “Spanish discourse filtered through the English metasystem” (\citeyear[13]{Lipski.2005}). He further postulates that this is possibly due to the simultaneous activation of the two languages \citep[6]{Lipski.2005}. This means the speakers utter the English marker unconsciously and this is to some degree a sign of the acculturation that the speakers experience. This idea points in the same direction as \citet{Matras.2000}, who claims a cognitive trigger for bilingual discourse markers. According to \citeauthor{Matras.2000}, “this cognitive motivation […] is so strong that it will at times override the social and communicative constraints on the discourse, leading to counterstrategic, accidental, or unintentional choices (i.e., slips)” (\citeyear[514]{Matras.2000}). This idea of a metasystem or cognitive filter cannot fully be adopted in this paper, as it does not sufficiently account for the fact that not all pragmatic markers of a language seem to underlie this filter and that the various markers of a language behave very differently in language contact.

All three markers introduced above are found in the data of the FM Corpus. It is striking that the French markers \textit{donc} and \textit{alors} appear infrequently in the corpus data and only in purely French utterances; that is, they only occur in utterances where the matrix language is French and never proceed or follow parts of English discourse. This is striking because pragmatic markers are generally known for being easy to insert in bilingual discourse, in part due to their semantic and syntactic detachability. Further, it can be detected in the data that \textit{donc} and \textit{alors} are only used by a small number of speakers.

The marker \textit{donc} only has 12 occurrences in the FM Corpus and is used by only four speakers. Still, \textit{donc} can be found in various pragmatic functions, such as a topic changer \xref{hennecke:ex:15}, a topic starter \xref{hennecke:ex:16}, and a marker of a conclusion or consequence \xref{hennecke:ex:17}.

\begin{exe}
    \ex\label{hennecke:ex:15} FLO: • • ehm • • \textit{donc} • Inga est-ce que tu l’as commencé ton chose?
    \glt FLO: • • ‘ehm • • \textit{so} • Inga have you started your thing?’
    \ex\label{hennecke:ex:16} FLO: Word ! • • • ok • • • \textit{donc} est-ce que vous avez besoin du temps • • • on stage • avant de commencer?
    \glt FLO: • • • ‘ok • • • \textit{so} do you need the time • • • on stage • before you start?’
    \ex\label{hennecke:ex:17} DAN: À cause y a/ dans un chanson • • je • je nomme tous les membres du groupe. FLO : cool • ok • \textit{donc} moi j’v/ j’… j’ai pas besoin d’l’ faire (…)
    \glt DAN: ‘Because there is/ in a song • • I • I name all the members of the band.’ FLO: ‘cool • ok • \textit{so} I’m going to… I don’t need to (…)’
\end{exe}

\noindent
Further, \textit{donc} in the FM Corpus occurs as a consequence or result marker, as in \xref{hennecke:ex:18}, or as a reformulation marker, as in \xref{hennecke:ex:19}.

\begin{exe}
    \ex\label{hennecke:ex:18} WIL: (…) en enlevant ça de mon/ de mes/ ma liste de dépenses • • • tu m’obliges de prendre leur quinze pourcent vers des coûts • • d’autres coûts • • • et \textit{donc} je (suis) déficitaire de ce moment là • • •
    \glt WIL: ‘(…) by removing that from  my/ my/ my list of expenses • • • you force me to take their fifteen percent towards costs • • other costs • • • and \textit{so} I (am) in deficit from then on • • •’
    \ex\label{hennecke:ex:19} WIL: (…) c’est elle qui décide combien d’argent est donné • ((…)) … ya c/ mais non • ça c’est pas efficace • • • \textit{donc} ((1.2s)) c’est elle qui écrit la lettre (…)
    \glt WIL: ‘(…) it’s her who decides how much money is given • ((…)) … but no • it’s not effective • • • \textit{so} ((1.2s)) it’s her who writes the letter (…)’
\end{exe}


In the same way, \textit{alors} is used infrequently in the FM data and only by a small number of speakers. Still, some of its discourse-pragmatic functions that are documented in European French are also evident in Franco-Manitoban spoken discourse. \textit{Alors} appears as a marker of consequence or result in \xref{hennecke:ex:20} and as a marker of repetition (explication) in \xref{hennecke:ex:21}.

\begin{exe}
    \ex\label{hennecke:ex:20} JO: puis c’est comme “ya je va me changer” \textit{alors} il se change puis là il sort puis il avait sa casquette là puis ça • j’étais comme “what the…?”
    \glt JO : ‘then it’s like “yah I’m going to change” \textit{so} he changes then he goes out then he had his cap on then it • I was like “what the...?”’
    \ex\label{hennecke:ex:21} JO: elle travaille à • • à temps partiel \textit{alors} elle travaille les après-midis puis c’est une classe d’onzième (…)
    \glt JO: ‘she works • • part-time \textit{so} she works in the afternoons then it’s an eleventh grade class (…)’
\end{exe}

\noindent
Furthermore, in the FM Corpus data, the marker \textit{alors} is used for turn management, for example as a topic changer \xref{hennecke:ex:22} and as a topic starter \xref{hennecke:ex:23}.

\begin{exe}
    \ex\label{hennecke:ex:22} JO: oh il était ici avant? \\
    NI: oui. \\
    JO: oh ya? • • • ha. \\
    NI: \textit{alors} vous prenez un cours ensemble? c’est quoi? (…)
    \glt JO: ‘oh he was here before?’ \\
    NI: ‘yes’ \\
    JO: ‘oh yah ? • • • ha’. \\
    NI: ‘\textit{so} you’re taking a class together? What is it? (…)’
    \ex\label{hennecke:ex:23} NI: (…) t’sais comme on se sert encore de ces choses là (…) mais eh \textit{alors} lui il doit apprendre comme comment on travai/ travaille avec le cuir • puis la fourrure
    \glt NI: ‘(…) you know how we still use these things (...) but \textit{then} he has to learn how to work with leather •  then fur’
\end{exe}

\noindent
In the FM data, \textit{alors} does not appear as a discourse-structuring device, but in Example \xref{hennecke:ex:24} it is used to bridge a moment of discourse planning.

\begin{exe}
    \ex\label{hennecke:ex:24} JO: mais quand même t’sais les personnes l’appellent Macaroni puis il y avait • • des (( )) comme ça • ici puis \textit{alors}…ouais ça • • \textit{alors} • • \textit{alors} quand même j’ pense ça/ ça eu un effet (…)
    \glt JO: ‘but still, you know, people call him Macaroni and then there were • • (( )) like that • here and \textit{then}… yeah that • • \textit{then} • • \textit{then} still I think that had an effect (…)’
\end{exe}

\noindent
Like the marker \textit{donc} in the FM Corpus, \textit{alors} also occurs in a high number of pragmatic functions when compared to its infrequent use. 

The English marker \textit{so} appears in purely French and purely English discourse environments in the FM Corpus and in bilingual discourse. In this context, an utterance is considered bilingual if one language occurs on the left-hand side of the marker \textit{so} and another language on the right-hand side, as in \xref{hennecke:ex:25}.

\begin{exe}
    \ex\label{hennecke:ex:25} PJ: (…) but the most of it is there \textit{so} je pourrais envoyer ça.
    \glt PJ: ‘(…) but the most of it is there \textit{so} I could send that.’
\end{exe}

\noindent
When looking at the distribution of \textit{so} in the FM Corpus, it is striking that it appears more frequently in bilingual (20\%) or French (42\%) sentence environments. \textit{So} is used in purely English discourse in only 38\% of the occurrences. Therefore, for the purpose of this analysis, the focus will be on \textit{so} in French and bilingual contexts. 

When indicating resultant parts of utterances, \textit{so} in the FM Corpus can mark results, as in \xref{hennecke:ex:26}, and consequences, as in \xref{hennecke:ex:27}.

\begin{exe}
    \ex\label{hennecke:ex:26} PJ: lls ont fait une autre comme négative \textit{so} it’s just like totally the wrong pictures and they fucked with them.
    \glt PJ: ‘They did another as negative \textit{so} it’s just like totally the wrong pictures and they fucked with them.’
    \ex\label{hennecke:ex:27} DR: puis Damian va être ici aussi \textit{so} he’s canning the date with you tomorrow…
    \glt DR: ‘Then Damian will be here too \textit{so} he’s canning the date with you tomorrow…’
\end{exe}

\noindent
When marking a conclusion, \textit{so} often includes the pragmatic functioning of introducing an explication, as in \xref{hennecke:ex:28}, or a reasoning (of something said earlier).

\begin{exe}
    \ex\label{hennecke:ex:28} FLO: aah vous avez l’âge à mon petit frère (puis) ma petite soeur ((1.s)) je me sens vieille. \\
    GER: c’est des jumeaux? \\
    FLO: non • • mais l’une (est née) en quatre-vingt-onze puis l’un en quatre-vingt-treize • • \textit{so} you are in the middle so
    \glt FLO: ‘ahh you are the age of my little brother (then) my little sister ((1.s)) I feel old’. \\
    GER: ‘are they twins?’ \\
    FLO: ‘no • •  but one (was born) in ninety-one then one in ninety-three • •  \textit{so} you are in the middle so’
\end{exe}

\noindent
The marker \textit{so} as a repetition marker can introduce a further explication \xref{hennecke:ex:29} or a reformulation \xref{hennecke:ex:30} of something said earlier.

\begin{exe}
    \ex\label{hennecke:ex:29} GR: ça parle de • • • comment que • l/ les français on voulait/ • les francophones on voulait nos droits puis là il y avait un backlash politique • sévère • • \textit{so} il y a des anglophones • on pense • qui ont • • brûlé le bâtiment de la Société Franco-Manitobaine
    \glt GR: ‘it talks about • • • how • the French wanted • the Francophones wanted our rights and then there was a political backlash • severe • • \textit{so} there are Anglophones • we think • who •  burned the Société Franco-Manitobaine building’
    \ex\label{hennecke:ex:30} DM: Ça j’ai écrit en anglais mais je voulais vraiment que ça soit en français aussi• euhm • • so eu-h • c’est c’est bien mais là • puisque ça traite de la culture dakota • euh • t’sais les amérindiens, right? c’est traduit en dakota aussi, \textit{so} c’est trilingue in the end
    \glt DM: ‘I wrote this in English but I really wanted it to be in French as well • euhm • • so eu-h • it’s good but • since it deals with Dakota culture • euh • you know Native Americans, right? It’s translated into Dakota as well, \textit{so} it’s trilingual in the end’
\end{exe}

\noindent
Further, \textit{so} also can fulfill typical discourse-management functions, such as topic starting \xref{hennecke:ex:31} and topic changing \xref{hennecke:ex:32}.

\begin{exe}
    \ex\label{hennecke:ex:31} CAR: hallo. \\
    ME: hallo. \\
    FLO: ya \textit{so} moi je vais juste vous parler un petit peu parce que • il va avoir • du temps… ya il ya avoir du • temps pendant le show après le show comme ben/ après votre set • •
    \glt CAR: ‘hallo’ \\
    ME: ‘hallo’ \\
    FLO: ‘yah \textit{so} I’m just gonna talk to you a little bit because • there’s gonna be • time … yah there’s gonna be • time during the show after the show like after your set • •’
    \ex\label{hennecke:ex:32} WIL: shut up. shut up • \textit{so} ya c’est un projet avec le CJP. C’est un projet originally du CJP.
    \glt WIL: ‘shut up. shut up • \textit{so} yah this is a project with the CJP .$_\smile$ this is an original project of the CJP.’
\end{exe}

\noindent
Unlike \textit{alors} and \textit{donc} in Franco-Manitoban discourse, \textit{so} can occur in utterance-final positions in the FM Corpus data without difficulty \xref{hennecke:ex:33}.

\begin{exe}
    \ex\label{hennecke:ex:33} DM: well they’re pretty lucky they had/ they were pretty close to him \textit{so} eh yeah.
\end{exe}


In conclusion, the corpus-based analysis of the markers \textit{alors}, \textit{donc}, and \textit{so} show that the markers \textit{alors} and \textit{donc} appear on a low-frequency basis, while the marker \textit{so} occurs very frequently, especially in bilingual contexts and monolingual French discourse. Still, \textit{alors} and \textit{donc} have not lost any of their semantic meanings or pragmatic functions that are attested in spoken European French. Despite its increase in frequency and its use in bilingual and French discourse, \textit{so} has not developed new functions or new meaning patterns in Manitoban French. The results from the corpus analysis and previous research on the marker \textit{so} in other contact varieties indicate that \textit{so} is indeed a case of borrowing from English (e.g., \citealt{Mougeon.1991,Torres.2008}).

\subsection{The markers \textit{ben}, \textit{bon}, and \textit{well}}\label{hennecke:sec:3.3}
The pragmatic markers \textit{bon} and \textit{ben} are derived from the adjective \textit{bon} ‘good’ and from the adjective and adverb \textit{bien} ‘good’ \citep[222]{MosegaardHansen.1998}. \citet[91]{Waltereit.2007} traces the reduced form of \textit{bien}, \textit{ben}, as a pragmatic marker back to the eighteenth century and \textit{bon} as a pragmatic marker even further, back to the sixteenth century \citep[92]{Waltereit.2007}. According to \citet[225]{MosegaardHansen.1998}, the marker \textit{bon} has two main functions: its interjective use and its proper discourse-marking use. The former includes the utterance-initial \textit{bon}, which is mostly retroactive and indicates acceptance. In contrast, the latter use includes \textit{bon} in non-utterance-initial positions. Here, \textit{bon} can appear either in turn-final or turn-medial positions or inside a sentential structure \citep[234]{MosegaardHansen.1998}. \citeauthor{Beeching.2011} ascribes the following functions and meanings to \textit{bon}: “positive evaluation, acceptance, \textit{mot de la fin}, provisional acceptance (stage-marking) and concession” (\citeyear[102]{Beeching.2011}). Here, provisional acceptance may mark conflicts of opinions among speakers. Furthermore, she recognizes the function of \textit{bon} as a face-threat mitigator, a hesitation and repair marker, and a pause filler. 

According to \citet[247]{MosegaardHansen.1998}, in contrast to \textit{bon}, \textit{ben} marks the unacceptability and irrelevance of a discourse phenomenon. It can mark inaccuracy, lack of importance, or the obvious and superfluous \citep[247]{MosegaardHansen.1998}. Furthermore, she states that “\textit{ben} always functions on a level of utterance content” \citep[234]{MosegaardHansen.1998}. \citet{Waters.2009} puts the focus on the discourse-structuring functions of \textit{ben}. For her, the main functions of \textit{ben} are as an initial turn-opener and “at the boundary between two intonation units” \citep[15]{Waters.2009}. In these two positions, \textit{ben} then fulfills diverse pragmatic functions, all of which comment on the preceding utterance of the previous speaker or the current speaker themselves \citep[15]{Waters.2009}. 


The marker \textit{well} is among the best-investigated markers of the English language, and a large number of studies examine this marker from a wide range of perspectives (e.g., \citealt{Lakoff.1973,Schiffrin.1987,Jucker.1997,Aijmer.2003,Beeching.2011}). This is mainly due to its frequency in spoken English and to the pragmaticalization pathways it has undergone (see \citealt{Beeching.2011}). 

Most researchers agree on the fact that \textit{well} can, among other things, embody some sort of positive value judgement \citep{Aijmer.2003}, conformity to a norm \citep{Bolinger.1989}, or acceptance \citep{Carlson.1984}. Some authors additionally point out that the pragmatic marker \textit{well} has a wide range of meanings that vary from partial agreement to complete disagreement \citep{Cuenca.2008}. If the speaker aims to express (partial) disagreement, the pragmatic \textit{well} does indeed function as a face-threat mitigator in that it can introduce a dispreferred response or express demur \citep{Beeching.2011}. Here, the speaker can mark concession, flag incoherence \citep{Beeching.2011}, or indicate a discrepancy between propositional attitudes of the speaker and the hearer \citep{Smith.2000}. In this case, the speaker tries to reestablish common ground that was lacking before \citep{Smith.2000}. \textit{Well} as a face-threat mitigator is mostly employed in an utterance-initial position. This syntactic position is also commonly used to flag a conclusion or a partial conclusion. It is striking that \textit{well} is mostly employed in utterance-initial positions, while it occurs rarely in utterance-medial or utterance-final positions. As an item in utterance-final positions, \textit{well} is commonly used as a bracketing and hesitation device. On a discourse-structuring level, \textit{well} may function as a repair and hesitation device, a boundary marker, or a pause filler. 

The FM Corpus data show that the pragmatic marker \textit{bon} is only used twice in the whole corpus data and only by one speaker. In these two occurrences, \textit{bon} fulfills discourse-structuring functions, as in \xref{hennecke:ex:34}.

\begin{exe}
    \ex\label{hennecke:ex:34} WIL: finalement la semaine passée j’appelle parce que j’ai envoyé • • • un courriel à ce temps là ((1.s)) basically disant “\textit{bon} • • • Promo Musique vous allez pas couvrir ces coûts là vous allez couvrir ces coûts là au lieu • ” (…)
    \glt WIL: ‘finally last week, I call because I sent • • • an email at that time ((1.s)) basically saying “\textit{well} • • • Music promotion you’re not going to cover those costs there you're going to cover those costs there instead • ” (...)’
\end{exe}


In contrast, in the FM Corpus, \textit{ben} marks a wide range of polysemous senses that vary from complete agreement \xref{hennecke:ex:35} to partial agreement \xref{hennecke:ex:36}, partial disagreement \xref{hennecke:ex:37}, and complete disagreement \xref{hennecke:ex:38}.

\begin{exe}
    \ex\label{hennecke:ex:35} GER: c’est comme “ooh • • c’était ça le maximum? • • • ha • mettez une liste avec tous les maximums”
    \glt WIL: \textit{ben} oui • • c’est pas difficil • like what
    \glt GER: ‘it’s like “ooh • • was that the maximum? • • • ha • put a list with all the maximums”’
    \glt WIL: ‘\textit{well} yes • •  it’s not difficult • like what’
    \ex\label{hennecke:ex:36} ZA: ya • • t/ ya t’as pas le choix là? mais eeh…I wonder if you can peel off the sponsors ((laughing)) in fact just like/ • I like just…   \\
    PJ: \textit{ben} il faut (quand même) connaître (( )) but
    \glt ZA: ‘yah • • yah you have no choice? But eeh… I wonder if you can peel off the sponsors ((laughing)) in fact just like/ • I like just…’
     \glt PJ: ‘\textit{well} you have to know anyway (( )) but’
    \ex\label{hennecke:ex:37} CAR: c’est comme vingt secondes. \\
    FLO: ah ok. \\
    DAN: \textit{ben} trente secondes.
    \glt CAR: ‘it’s like twenty seconds.’ \\
    FLO: ‘ah ok.’ \\
    DAN: ‘\textit{well} thirty seconds.’
    \ex\label{hennecke:ex:38} CAR: ya peut-être une minute. \\
    DAN: non • on smash pas les guitars pendant (( )) \\
    CAR: \textit{ben} non parce que faut qu’on • faut qu’on smash • • and then toi tu va chercher ta guitare •
    \glt CAR: ‘yah maybe a minute ago.’ \\
    DAN: ‘no • we don't smash the guitars during (( ))’ \\
    CAR:  ‘\textit{well} no because we have to •  we have to smash • • and then you go and get your guitar •'
\end{exe}

\noindent
Further, \textit{ben} can mark that something is obvious. In \xref{hennecke:ex:39}, the speaker signals that the previous utterance was superfluous or not necessary for the current conversation.

\begin{exe}
    \ex\label{hennecke:ex:39} GR: in what sense? ((laughing)) \\
    DM: \textit{ben} • • • in the fullest sense right? \\
    GR: ya. ya. ce serait awesome.
    \glt GR: ‘in what sense?’ ((laughing)) \\
    DM: ‘\textit{well} • • • in the fullest sense right?’ \\
    GR: ‘yah. yah. That would be awesome.’
\end{exe}


When structuring discourse, \textit{ben} can be used not only to introduce new or pending topics, but also to mark the beginning of a subtopic or a bracket \xref{hennecke:ex:40}, introduce reported speech \xref{hennecke:ex:41}, or flag a conclusion \xref{hennecke:ex:42}.

\begin{exe}
    \ex\label{hennecke:ex:40} NI: on arait • on avait une atten/ une entente avec/ • \textit{ben} en plus ils chargent comme 30 dollars de l’heure so
    \glt NI: ‘we would • we had a / an agreement with/ • \textit{well} besides that they charge like 30 dollars an hour so’
    \ex\label{hennecke:ex:41} WIL: puis là dans un courriel elle dit “\textit{ben} chose certaine • les CBL là seront eh seront couverts là ces coûts là seront couverts • • (…)”
    \glt WIL: ‘and then in an email she says “\textit{well}, one thing for sure •  the CBL there will be eh will be covered there these costs there will be covered • •  (...)”’
    \ex\label{hennecke:ex:42} DM: ya il était comme “je suis fou de Winnipeg” e h ya. c’est awesome! c’est comme • il est déjà franco-manitobain là you know (( ))) \textit{ben} c’est ça.
    \glt DM: ‘yah he was like “I’m crazy about Winnipeg” e h yah. It’s awesome! it’s like • he’s already Franco-Manitoban there you know (( ))) \textit{well} that's it.’
\end{exe}


The pragmatic marker \textit{well} is not a particularly frequent item in Franco-Mani\-to\-ban spoken language. Still, a range of varying senses and functions that are attributed to the marker in spoken colloquial English can also be found in the FM Corpus data. \textit{Well} in the FM data can mark complete agreement only when it is combined with another item of positive evaluation, as in \xref{hennecke:ex:43}.

\begin{exe}
    \ex\label{hennecke:ex:43} GR: c’est ça • • you make them look good.
     \glt WIL: ya exactement • •\textit{well} • oui • • • puis I wanna use that argument and I know it’s true I just need the backup. • • •
    \glt GR: ‘that’s it • • you make them look good.’
    \glt WIL: ‘yah exactly • •\textit{well} • yes • • • then I wanna use that argument and I know it’s true I just need the backup. • • •'
\end{exe}


In Example \xref{hennecke:ex:44}, \textit{well} expresses partial agreement with the framework of the preceding speaker.

\begin{exe}
    \ex\label{hennecke:ex:44} ME: so you wanna go to • • to Europe? \\
    NI: eh • y/ yeah ((1s)) yeah • • • \textit{well} like • • I’ve been to France ((1.3s)) and I’ve been to London. I’d go back ‘cause I didn’t spend very much time there • •
\end{exe}


Concerning its pragmatic functions, \textit{well} in the FM Corpus can mark concession \xref{hennecke:ex:45} or can be used as a face-threat mitigator \xref{hennecke:ex:46}.

\begin{exe}
    \ex\label{hennecke:ex:45} ME: he died last year. \\
    DM: yeah. • •\textit{well} they’re pretty lucky they had/ they were pretty close to him so eh. yeah. yeah. yeah.
    \ex\label{hennecke:ex:46} ZA: is any of it good? • • Who are these people? I’ve never heard of these˙ \\
    WIL: eehm… \\
    ZA: hm \textit{well} ((1.2)) do these guys actually get to have careers? comme • est-ce qu’ils font de l’argent?
    \glt ZA: ‘is any of it good? • • Who are these people? I’ve never heard of these˙’ \\
    WIL: ‘eehm…’ \\
    ZA: ‘hm \textit{well} ((1.2)) do these guys actually get to have careers? Like • do they make money?’
\end{exe}


It is striking that in most occurrences in the FM data, \textit{well} acts as a discourse-structuring device, such as for bracketing \xref{hennecke:ex:47} and at the beginning of reported speech \xref{hennecke:ex:48}.

\begin{exe}
    \ex\label{hennecke:ex:47} GR: they were gonna enact a law that made the province bilingual ((1.5s)) but (( )) la crise linguistique \textit{well} there’s a huge backlash from the public within the party. yeah I guess.
    \ex\label{hennecke:ex:48} SA: je commençais à enseigner and like • (après) comme trois ans j’avais comme • • • comme presque cent étudiants like/ like to myself and I was like “ah \textit{well} I guess comme • je devrais peut-être (( )) l’enseignement”.
    \glt SA: ‘I was starting to teach and like • (after) like three years I had like • • • like almost a hundred students like/ like to myself and I was like “ah \textit{well} I guess like • maybe I should (( )) be teaching”.’
\end{exe}


In conclusion, the corpus-based analysis of the markers \textit{bon}, \textit{ben}, and \textit{well} demonstrated that the markers \textit{bon} and \textit{well} only occur infrequently and mostly in monolingual contexts in the FM Corpus data. In contrast, the marker \textit{ben} appears particularly frequently in the data, but has not undergone any other changes in its semantic meaning patterns or its discourse-pragmatic functions. Furthermore, it is striking that \textit{ben} generally occurs in monolingual contexts and cannot be considered a case of borrowing in English discourse. It can be speculated that the marker \textit{ben} is preferred over other partially equivalent markers such as \textit{bon}, \textit{(en)fin}, or \textit{bref} because of its strong semantic overlap with the English \textit{well}. Still, it remains unclear why the markers \textit{ben} and \textit{well} do not coexist to the same degree in the FM  data. More large-scale corpus data is needed to investigate this phenomenon more closely.

\subsection{The absence of pragmatic markers in Manitoban French}\label{hennecke:sec:3.4}
Research on pragmatic markers has shown that co-occurrence of discourse features is very common, not only in French but also in English \citep{Pichler.2010}. For instance, speakers of European French have a wide range of co-occur\-ring markers at their disposal and these markers are used frequently in spoken language. Surprisingly, very frequent markers from European French, such as \textit{enfin/fin} ‘so, well’, \textit{bref} ‘well’, \textit{tu vois} ‘you see’, \textit{genre} ‘like’, and \textit{quoi} ‘what’, do not occur at all in the Manitoban French corpus data. Furthermore, even frequent markers that have emerged in Quebec French, such as (\textit{ça}) \textit{fait que} ‘well’ and \textit{coudon} ‘so’, do not appear at all in the FM Corpus.

In addition, it is striking that all markers analyzed in Sections \ref{hennecke:sec:3.1} to \ref{hennecke:sec:3.3} seem to have undergone changes in frequency and/or productivity. While \textit{comme}, compared to European French, has experienced a huge increase in frequency that goes hand-in-hand with a broadening of productivity, other markers such as \textit{bon} have seen a considerable decrease in frequency, together with a possible, yet not unequivocally provable, decrease in productivity. At this point, it is important to mention that an increase in frequency of a certain phenomenon does not necessarily correlate with an increase in productivity of the same phenomenon and vice versa (see \citealt{Poplack.2001,Poplack.2010}). Using a variationist sociolinguistic approach, \citet{Poplack.2001} proved that while the use of the subjunctive in Quebec French is decreasing, the use of the subjunctive with the verbs \textit{valoir} ‘to be worth’ and \textit{falloir} ‘must’ is increasing \citep{Poplack.2010}. This means that only a small number of verbs are used with the subjunctive in spoken Quebec French, but this small number of verbs is used considerably more often with the subjunctive than before \citep{Poplack.2010}. 

In the use of pragmatic markers in Manitoban French, we see a somewhat similar but not comparable evolution. While a small number of markers, such as \textit{like}, \textit{comme}, \textit{so}, and \textit{ben}, are used very frequently, there is no comparable co-occurrence of markers. There may be two explanations for the lack of variation in the discourse-marking system in Manitoban French. \citet{Sankoff.1997} observed in their corpus data from Anglophone L2 French speakers in Montreal that the speakers used significantly fewer pragmatic markers when speaking in their L2 than in their L1. In this case, the use of pragmatic markers increased in parallel to the L2 language skills \citep[213]{Sankoff.1997}. Therefore, native language skills may be one plausible reason for the size of the discourse-marking system in Manitoban French. Furthermore, \citet[214]{Sankoff.1997} identified a correlation between the use of certain pragmatic markers and sociolinguistic factors such as a speaker’s childhood environment, gender, and social class. An in-depth analysis of sociolinguistic factors relating to the speakers in the FM Corpus is not possible within the framework of this study. Hence, the description of the corpus data in Section \ref{hennecke:sec:3} must suffice to provide an insight into the speakers’ sociolinguistic environment and their language skills (see also \citealt{Hennecke.2014}). All the speakers in the FM Corpus consider French as their L1, despite the strong influence of English in all situations of their everyday lives. All the speakers in the corpus received their education up to their high school diploma exclusively in French. Furthermore, all the speakers live in St. Boniface, the French quarter of Winnipeg, and actively participate in social and cultural activities in their community. These facts do not provide evidence that a lack of language skills may trigger a lack of variation in the discourse-marking system of Manitoban French. A certain influence of specific sociolinguistic factors cannot be ruled out, but these factors do not seem to be the only reason for the evolution of pragmatic markers in Manitoban French. Further, the evolution of spoken Manitoban French and its sociohistorical development may have influenced the system of pragmatic markers. Here, the strong language contact with English and the constant situation of bilingual discourse in the everyday lives of the speakers may play a role in the absence of certain markers. More research on this topic needs to be done to find unequivocal explanations for this specific phenomenon.

\section{Conclusion}
In previous studies on pragmatic markers, it has been stated that these items may undergo different processes in language contact (e.g., \citealt{Mougeon.1991,Torres.2008}). These studies claim that pragmatic markers are well suited to borrowing in language contact but that similar markers from two languages may also acquire different meanings or be replaced by one item from one language. It has even been suggested that two sets of pragmatic markers may coexist or that all markers from one language may replace all markers from the other language (e.g., \citealt{Brody.1987,Goss.2000}). This study aimed to investigate the processes and outcomes of language change in a long-term situation of language contact by means of a self-compiled corpus of bilingual Franco-Manitoban conversations. 

The analysis of pragmatic markers in the FM Corpus shows four different outcomes of pragmatic markers in contact. Firstly, the marker \textit{comme} takes functions from the English equivalent \textit{like} and sees a rise in frequency. Secondly, the English marker \textit{so} is borrowed in French discourse, while its equivalent French markers \textit{alors} and \textit{donc} are used infrequently but keep their functions in French discourse. Thirdly, the French marker \textit{ben} and its equivalent \textit{well} occur frequently and mostly in monolingual discourse. In contrast, the marker \textit{bon} is almost inexistent in the corpus data. Finally, some markers that are frequent in varieties of European or Québec French do not occur at all in Manitoban French 
(e.g., \textit{bref}, \textit{quoi}, (\textit{en})\textit{fin}). 

The analysis also reveals five different outcomes of language contact on the system of pragmatic markers in Manitoban French:

\begin{enumerate}
    \item Contact-induced language change
          \begin{enumerate}
              \item Emergence of new semantic meaning patterns
              \item Emergence of new discourse-pragmatic functions
          \end{enumerate}
    \item Borrowing of a marker from the other language
    \item Frequency change
          \begin{enumerate}
              \item For the benefit of a marker from the same language
              \item For the benefit of a marker from the other language
          \end{enumerate}
    \item Coexistence of two markers
          \begin{enumerate}
              \item Coexistence in their original linguistic system
              \item Coexistence in the same linguistic system
          \end{enumerate}
    \item Absence of pragmatic markers
\end{enumerate}


This analysis showed that the classification provided by \citet{Torres.2008} is too limited and restricted, in that language contact can have complex and diverse impacts on systems of pragmatic markers in contact. Further, the analysis demonstrated that the different types of language change are not mutually exclusive in one system of pragmatic markers in contact. In conclusion, this corpus analysis revealed that pragmatic markers in a contact situation might indeed undergo different processes of language change. The analysis highlighted three processes; namely, the contact-induced change of the marker \textit{comme}, the borrowing of the marker \textit{so}, and a decline in frequency, particularly in the case of the markers \textit{donc} and \textit{bon}.

These findings support previous research results on pragmatic markers in language contact (e.g., \citealt{Torres.2008,Hlavac.2006}). Furthermore, this study highlights the necessity of detailed cross-language analyses of pragmatic markers to determine their contact-induced changes in their pragmatic functioning and meaning patterns more precisely.

\printbibliography[heading=subbibliography, notkeyword=this]

\end{document}
