\documentclass[output=paper,arabicfont]{langscibook}
\ChapterDOI{10.5281/zenodo.10497379}

\author{Nancy Hawker \orcid{0000-0002-0693-4045} \affiliation{Independent researcher}}

\title [Creating a class of elite Palestinian multilinguals in Israel]
{Creating a class of elite Palestinian multilinguals in Israel: Reflections on research in late capitalism}
\abstract{In this chapter, speeches from 2015 and 2022 by Palestinian politicians in Israel using Hebrew and Arabic are analysed for pragmatic functions, with a focus on politeness strategies, translation, humour, and inter-group boundaries. Language ideologies are related to the equation of one nation, one language, which speakers reference while practising multilingualism. The multilingual code is articulated with the emergence of an educated middle class that aspires to consumerism and liberal politics. In the context of late capitalism, it is the commodification of multilingual skills that circumscribes political aspirations in combination with considerations of class and conflict.
}


\IfFileExists{../localcommands.tex}{
   \addbibresource{../localbibliography.bib}
   % add all extra packages you need to load to this file

\usepackage{tabularx,multicol}
\usepackage{url}
\urlstyle{same}

\usepackage{listings}
\lstset{basicstyle=\ttfamily,tabsize=2,breaklines=true}

\usepackage{langsci-basic}
\usepackage{langsci-optional}
\usepackage{langsci-lgr}
\usepackage{langsci-osl}
% \usepackage{./langsci/styles/langsci-lgr}
% \usepackage{./langsci/styles/langsci-osl}
% \usepackage{langsci-gb4e}

\usepackage{tikz}
\usetikzlibrary{patterns,calc}
\pgfdeclarepatternformonly{south east lines}{\pgfqpoint{-0pt}{-0pt}}{\pgfqpoint{3pt}{3pt}}{\pgfqpoint{3pt}{3pt}}{
    \pgfsetlinewidth{0.6pt}
    \pgfpathmoveto{\pgfqpoint{0pt}{3pt}}
    \pgfpathlineto{\pgfqpoint{3pt}{0pt}}
    \pgfpathmoveto{\pgfqpoint{.2pt}{-.2pt}}
    \pgfpathlineto{\pgfqpoint{-.2pt}{.2pt}}
    \pgfpathmoveto{\pgfqpoint{3.2pt}{2.8pt}}
    \pgfpathlineto{\pgfqpoint{2.8pt}{3.2pt}}
    \pgfusepath{stroke}}
    
\usepackage{stmaryrd}
\usepackage{wasysym}
\usepackage{multirow}
\usepackage{caption}
\usepackage{subcaption}
\usepackage{mathrsfs}
\usepackage{qtree}

\usepackage{linguex}


   %pminos do not split footnotes
% \interfootnotelinepenalty=10000 %Footnote in Laporte chapters has to be split SN


%\DeclareIndexNameFormat{default}{%
%\nameparts{#1}%
%\usebibmacro{index:name}%
%{\index[names]}%
%{\namepartfamily}%
%{\namepartgiveni}%
% {}% L1
% {}% L2
%{\namepartprefix}% generates spurious space L3
%{\namepartsuffix}% generates spurious space L4
%}

%  {\DeclareIndexNameFormat{default}{%
%     \usebibmacro{index:name}{\index[names]}{#1}{#3}{#5}{#7}}}

%\DeclareIndexNameFormat{default}{%
%  \usebibmacro{index:name}{\sindex[nom]}{#1}{#3}{#5}{#7}}

%\DeclareIndexNameFormat{default}{%
%  \usebibmacro{index:name}{\sindex[person]}{#1}{#3}{#5}{#7}}
%\DeclareIndexNameFormat{default}{%
%\nameparts{#1} \usebibmacro{index:name}{\sindex[person]]}{\namepartfamily}{‌​\namepartgiven}{\nam‌​epartprefix}{\namepa‌​rtsuffix}}

%\newcommand{\smiley}{:)}

%\renewbibmacro*{index:name}[5]{%
%\usebibmacro{index:entry}{#1}%
%{\iffieldundef{usera}{}{\thefield{usera}\actualoperator}\mkbibindexname{#2}{#3}{#4}{#5}}}

% \newcommand{\noop}[1]{}

%remove for final
%\overfullrule=1mm

\newcommand{\tobi}[2]}}
\renewcommand{\S}[1]{\tobi{#1}{\textsc{*}}}

% this volume references
% puts: [this volume]
% already defined: \citetv
%\newcommand{\citepv}[1]{(\citeauthor{#1} \citeyear*{#1} [this volume])}
\newcommand{\citealtv}[1]{\citeauthor{#1} \citeyear*{#1} [this volume]}

%parentheses around example number
\newcommand{\pref}[1]{(\ref{#1})}

% in-text examples

\newcommand{\lnex}[1]{\textit{#1}} %target lang word
\newcommand{\lnlit}[1]{(lit.: `#1')} %literal reading
\newcommand{\lnlat}[1]{(#1)} % latinization
\newcommand{\lntrans}[1]{`#1'} %translation
\newcommand{\lnexl}[2]%
{\lnex{#1}{} \lnlat{#2}} % ex with latinization
\newcommand{\lnexlat}[3]{\lnex{#1}{} \lnlat{#2}{} \lntrans{#3}} % ex with latinization and tranl.

%ch01
\newcommand{\co}[1]{\mbox{\textbf{#1}}}

%ch09

\newcommand{\cyrbulg}[1]{\begin{otherlanguage*}{bulgarian}#1\end{otherlanguage*}}


%ch10
\newcommand{\nlp}{{\small NLP}}
\newcommand{\mwe}{{\small MWE}}
\newcommand{\rae}{{\small RAE}}
\newcommand{\lvc}{{\small LVC}}
\newcommand{\pos}{{\small P}o{\small S}}
%\newcommand{\todo}[1]{ \textcolor{red}{#1} }

%\renewcommand{\labelenumi}{\theenumi}
%\ainamefmt{{vv}{ll}{, ff}{, jj}} % fullname

\newcommand{\biberror}[1]{{\color{red}#1}}

\newcommand{\osenovaitem}{--~}
   %% hyphenation points for line breaks
%% Normally, automatic hyphenation in LaTeX is very good
%% If a word is mis-hyphenated, add it to this file
%%
%% add information to TeX file before \begin{document} with:
%% %% hyphenation points for line breaks
%% Normally, automatic hyphenation in LaTeX is very good
%% If a word is mis-hyphenated, add it to this file
%%
%% add information to TeX file before \begin{document} with:
%% %% hyphenation points for line breaks
%% Normally, automatic hyphenation in LaTeX is very good
%% If a word is mis-hyphenated, add it to this file
%%
%% add information to TeX file before \begin{document} with:
%% \include{localhyphenation}
\hyphenation{
    Beck-man
    Ngu-yen
    back-chan-nel
    back-chan-nels
    mo-not-o-nous
    ste-reo-typ-i-cal
}

\hyphenation{
    Beck-man
    Ngu-yen
    back-chan-nel
    back-chan-nels
    mo-not-o-nous
    ste-reo-typ-i-cal
}

\hyphenation{
    Beck-man
    Ngu-yen
    back-chan-nel
    back-chan-nels
    mo-not-o-nous
    ste-reo-typ-i-cal
}

   \boolfalse{bookcompile}
   \togglepaper[6]%%chapternumber
}{}



\begin{document}
\maketitle

\section{Introduction} 
Past research on Palestinians who live in Israel, and speak both Hebrew and Arabic, has shown that they are interactively guided in their language choices by certain tacit norms, namely
\begin{enumerate}
\item the principle of Arabic avoidance outside the in-group (‘only speak Arabic with Arabs'); and
\item the principle of valuation of multilingualism (‘display your multilingual assets') \citep{hawker2019a}.
\end{enumerate}

There are also explicit norms that are policed whenever Palestinian speakers speak Arabic on institutional state platforms that contribute, in a daily bureaucratic way, to the identification of Israel with Zionism \citep{handelman2020a}. Here, Zionism is used in a narrow definition, as a specific aspect of an ideology that justifies and promotes particular political actions: a combination of ideas repeatedly put to practice in state-building and state-affirming rituals and symbols constituting Israel as the state of people who claim Jewish nationality, following 19th- and 20th-century European ideas of nation-statehood.\footnote{This definition of Zionism is often qualified as secular political Zionism, distinguished from cultural and spiritual Zionism, and from Jewish religion (Raz-Krakotzkin 2021: 34 in \cite{rouhana2021a}).} These norms are based on Hebrew dominance on Zionist platforms, and include 
\begin{enumerate}
\item minimising the audibility of Arabic in Israeli institutions;
\item announcing and preparing the audience for short switches to Arabic; and
\item translating any Arabic speech immediately into Hebrew.
\end{enumerate}

The task of this chapter is to trace the manifestation, negotiation, and contestation of these norms, and to see whether they are still relevant, using observations and analyses spanning several years. The chapter will argue that the ability to challenge norms increases with the speakers’ relative social power (at intersections of class and other identities). The issue of relative power opens up specific platforms for Palestinian politicians in the Knesset (Israeli parliament), since they are institutionally given the floor to speak for set periods of time and gain confidence in their discursive skills by drawing on their multilingualism. 

I will argue that within this generally elite circle, relative power differentials are sociolinguistically manifest in expressions of politeness, which organize social hierarchy. Class, as an economy-based form of social hierarchy, maps onto understanding the status of the Palestinian minority in Israel either as systematically discriminated against, or economically disadvantaged. Simply put, if the frame is discrimination, the speaker can politically side with the oppressed by using negative politeness. If the frame is disadvantage, the speaker can politically align with the privileged by using positive politeness. And yet all the speakers, regardless of their political politeness strategy, display their multilingual assets on a first order of indexicality, as a correlate of their middle-class education and their aspirations for economic opportunity and political engagement.

\largerpage
The evidence to test this argument is taken from audio-visual recordings and stenographer’s minutes of Knesset debates. The selection of texts from 2015 and 2022 highlights instances where there was extensive metalinguistic commentary on the use of Arabic on this platform, constituted as Zionist (in the narrow, nation-state-building sense). The analysis of this evidence centres on strategies of politeness, translation, and identity-definition, created interactionally. Since these pragmatic functions of speech can only be found in interactions, three long stretches of speech are reproduced as evidence, in my own idiomatic English translation.

\subsection{Class in sociolinguistics}

\largerpage
The idea that, in late capitalism, the service- and knowledge-based economies predispose speakers to repackaging language skills as economic and political resources will be expressed in the proposition that language itself becomes a ‘means of production' \citep{heller2017a}. The emergent Palestinian middle class in Israel are the owners of multilingual discursive means of production and, consequently, are relatively less exposed to precarity in a structural sense than those who do not own those resources. This conceptualization is different from the treatment of class as a habitus that forms language ideologies enabling judgments of taste \citep{bourdieu1982a}. Nevertheless, these ideologies, which bear the scars of past (class) struggles over value \citep[892]{myles1999a}, do affect the success of commodification and commercialization. The marketplace for Palestinians’ and other Arabs’ multilingualism is already skewed towards devaluing Arabic speakers’ Arabic. The linguistic offerings of late capitalism are ambivalent; multilingualism has cultural capital, but in a commodified way, geared towards individual material benefits rather than economic emancipation, freedom, and social justice \citep{davis2008a}.

By the time of the 2015 elections, the second generation of Palestinians and other Arabs socialized under Israeli systems were in their prime, and the third generation had come of political age. The Palestinian and other Arab middle class in Israel has developed as a correlate of the neo-liberalization of the economy since the 1980s. With this limited middle class came the gradual emergence of a new voice: an Arabic and multilingual voice, claiming equality in the common good in the space that is controlled by the Israeli state. They deploy multilingual styles \citep{eckert2004a} appropriate to their educated middle-class in-group, for legal argumentation, humour, critical self-reflection, and sometimes earnest anger to call Israel’s bluff on its self-description as a democracy. These styles also function to stretch claims to equal rights from the strategies  to survive in the territorial limits of Israel (as it was under military rule) and to be protected from precarity (as it was in the years of manual labour) to all aspects of liberal freedoms. These freedoms appeared to be within reach – such are the expectations that the middle class is trained to see as a right \citep{dean2014a}. This voice is skilled in its use of multilingualism and language-contact phenomena for different pragmatic functions, and co-constructions of political meanings.

I have commodified my own multilingualism for precarious employment in late-capitalist knowledge work, and I self-consciously display my skills in this volume, as does every contributing author. I am not of the Palestinian or Israeli middle class, but am perhaps of the cosmopolitan European middle class (\citealt{hartung2017a}; \citealt[21–86]{nowicka2016a}). I currently work for Amnesty International in the Middle East and North Africa Office, which is an organization that campaigns for human rights and takes the inherent equality and dignity of all humans as a premise beyond question. The sociolinguistic analysis of the multilingual political discourse of middle-class Palestinians who claim equality is therefore, through several degrees of separation, also the analysis of my own discourse. Nevertheless, the execution of the analysis might be as scientific as any science \citep[76–78]{bourdieu2004a}.

\subsection{Method of data collection and selection}
\largerpage
The evidence I present in Section 2 of this chapter is selected from recordings and observations at 25 locations across Israel during the 2015 parliamentary election campaigns. The recordings involved 18 Arabic-speaking politicians, 4 Hebrew-speaking politicians, aides, vocal supporters/critics, and members of the public. The locations included a range of situations from house visits to mass events. I also analysed pre-recorded sources, mostly on YouTube or the Knesset Channel, of Knesset debates and TV broadcasts. In total, I coded and analysed 57 hours of recordings, identifying switches between Hebrew and Arabic and their contexts. For a historical perspective, I examined 21 documents from the Knesset archives. These documents were official records of parliamentary and other party political debates, where Arabic was spoken in a Hebrew-dominant context, totalling more than 500 pages \citep{hawker2019a}. I add new evidence to this in Section 3 from political speeches in 2022 – this is important due to contextual changes and changes in the styles of multilingual discourse. I chose the textual examples in Sections 2 and 3 as emblematic illustrations of general patterns that emerged in this research.

\section{Arabic avoidance and Arabic promotion in Hebrew-dominant contexts: Contradictions?}

On 23 December 2015, Ayman Odeh, a Palestinian politician in Israel, made a speech ahead of Christian and Muslim holidays, in which he switched from Hebrew to Arabic for a few sentences. Ayman Odeh leads the Joint List, which is a coalition of parties formed in 2015 that mostly attract the votes of Palestinian and other Arab citizens of Israel. The text analysed below was taken from the official Knesset record from 23 December 2015 \citep[130–131]{knesset2015a}; the translation is my own. The Knesset records are all noted in Hebrew, in line with the Rules of Procedure \citep{knesset2012a}, and occasionally – as seen below – in Arabic transcribed into Hebrew script. As in all examples for this chapter, what was Hebrew in the original (in this case, in the stenographer’s official record) is presented in italics, while Arabic is in roman font. There is no audio record of this event – that I could find – against which to compare the stenographer’s record. Therefore, the difference in the font as a language identifier for the translation is based on my recognition of how the Knesset stenographer would have distinguished the two languages as separate and bounded, providing the stenographer would have followed widespread attitudes in the chronotope of modern Israel/Palestine \citep[xii–xiii]{shohamy2005a}. The stenographer’s metalinguistic comment “speaks in Arabic; henceforth translation” is underlined. Whatever speech has been translated from Arabic for the records is enclosed in parentheses, as if literally bounded. Overlapping and inaudible speech – which connotes interruptions or heckling – is marked in the record with hyphens, reproduced here in the same way. The names of the speakers and their parliamentary parties are in bold and parentheses, as they are in the original; all of the speakers are Jewish citizens of Israel, apart from Ayman Odeh, who is a Palestinian citizen of Israel.\\

\setlength\linenumbersep{-.5cm}
\begin{exe}\ex\label{hawker:ex1}
\textit{Ayman Odeh wishing everyone happy holidays in the Knesset, 23 December 2015.}\\

\exi{}
\begin{linenumbers*}
\textbf{\textit{Ayman Odeh (the Joint List):}} 

\textit{I have a request from the Chair, sir. Would you please\\ allow me to speak in Arabic for a minute, because today is\\ the Day of the Birth of the Prophet Muhammad, peace be\\ upon him, and the day after tomorrow is Christmas - - -}

\textbf{\textit{Rachel Azariah (Kulanu):}}

\textit{But today is my birthday - - -}

\textbf{\textit{Ayman Odeh (the Joint List):}}

\textit{- - - so with your permission, I would like to turn to my\\
 people - - -}

\textbf{\textit{Shelly Yachimovich (Zionist Camp):}}

\textit{- - - You don’t need to ask for permission - - -}

\textbf{\textit{Ayman Odeh (the Joint List):}}

\textit{- - - and tell them - - -}

\textbf{\textit{Shelly Yachimovich (Zionist Camp):}} 

\textit{You don’t need to ask for permission.}

\textbf{\textit{Chair Yehiel Hilik Bar (Zionist Camp):}}

\textit{It is an official language according to the Rules of\\
 Procedure.}

\textbf{\textit{Ayman Odeh (the Joint List):}}

\textit{- - - I turn to my compatriots: (\ul{speaks in Arabic; henceforth}\\
\ul{translation}: I turn to my compatriots, to our people, all our \\
people, on the occasion of the Day of Prophet Muhammad’s \\
Birth and of Christmas. It is said, “A light shone in the \\
eastern sky, shining brightly in the dark night,” and there \\
are some who add, “A light shone in the eastern sky, to \\
guide the wise men to the manger.” Happy holidays to all \\
our people, for the Prophet Muhammad’s Birthday and for \\
Christmas, and to anyone who gives for us to continue \\
struggling for a life in dignity in the land of our ancestors.) \\
In relation to the proposed bill - - -}

\textbf{\textit{Ksenia Svetlova (Zionist Camp): - - - }}

\textbf{\textit{Ayman Odeh (the Joint List):}}

\textit{Thank you very much.} Happy holidays to everyone. \textit{Happy\\
holidays to everyone.}

\textbf{\textit{Shelly Yachimovich (Zionist Camp): - - -}}

\textbf{\textit{Ayman Odeh (the Joint List):}}

Thank you, thank you, Shelly, bless you. […] \textit{With regards\\
to the proposed bill - - -}

\textit{\textbf{Omer Bar-Lev (Zionist Camp):}}

\textit{Tell us what you said.}

\textbf{\textit{Ayman Odeh (the Joint List):}}

\textit{Basically, what I said was, happy holidays to everyone, we \\
shall continue to struggle together for life in dignity in our \\
historical homeland.}

\textbf{\textit{Anat Berko (Likud):}}

\textit{What “historical homeland”!}\footnote{
In the record: \RL{איזה
היסטורית}!
 (literally, “what historical!”), own translation.}
 
\textbf{\textit{Ayman Odeh (the Joint List):}}

\textit{Alright. So, we - - -}

\textbf{\textit{Shouts: - - -}}

\end{linenumbers*}
\end{exe}

Any length of speech in Arabic in the Knesset is a remarkable event, at least since the mid-1960s \citep[36–39]{hawker2019a}.

Some of the Jewish interlocutors in the Knesset appear to be largely comfortable with the liberal tolerance of different religious holidays, multilingually expressed in short formulaic greetings. Ksenia Svetlova and Shelly Yachimovich respond to Ayman Odeh, possibly in Arabic and hence unrecordable to the stenographer  (lines 32 and 36). In an ordinary context such as a short commercial interaction among strangers, such expressive speech acts would have the function of phatic communion and politeness \citep{amer2020a}. In the context of the Knesset, however, greetings in Arabic and across religious dividers are exceptional events. The speakers are enacting a liberal vision where citizens of all religions would be accorded mutual respect. This vision is what Palestinian and Jewish intellectuals have termed a hoped-for future \textit{convivencia} \citep[24, 33]{shohat2017a}, or conviviality. My interpretation of Ayman Odeh’s staging of the performance of multicultural tolerance analyses it as a declarative speech act intended to bring about this liberal transformation. The multilingual exchange, in which Arabic phrases are immediately translated by an authoritative Palestinian speaker into Hebrew, is the discourse chosen for this performance.

However, the speech act presents a fluid, not to say ambiguous, relationship between religious identities and secular aspects of nationhood such as “ancestors” (line 30) and “homeland” (line 45). It becomes clearer, in Ayman Odeh’s repetition and summary in Hebrew of what he had said in Arabic, that he sees religious holidays as elements of inherited traditions that are specific to Palestinians (lines 44–45). Further, this inheritance grows into political rights situated in contemporary “struggles”. The ambiguity allowed some interlocutors to react cooperatively to the religious element, and others confrontationally to the secular national one.\footnote{The context for the linking of religion and politics is the fluid relationship between Judaism and secular national rights to land in Israel/Palestine. This relationship is structured in laws, state and parastate institutions, and symbols \citep[86–101]{yadgar2020a}, and enables discourses of Jewish exclusivity in Israel \citep{rouhana2021a}.}

\hspace*{-3.5pt}While Ayman Odeh is aware that Arabic and Hebrew language policies strongly index one or the other national project (\citealt{rafael1994a}; \citealt[27]{suleiman2019a}), and indeed he relies on this order of indexicality, his practices are in fact multilingual. He is manipulating the indexicality for a more complex political message than the equation of one nation, one language, and the logic of conflict that derives from this equation being applied twice in the same geographical space \citep{suleiman2004a}. My argument is that on a higher indexical order, he is enregistering a multilingual code that he has mastered and co-created as member of an elite Palestinian class in Israel. I will now turn to the detailed analysis of the code-switching involved in this multilingual discourse in Example \xref{hawker:ex1} .

Ayman Odeh’s switch from Hebrew to Arabic is highly conspicuous, and issues of translation are also highlighted. He opens his speech from the speaker’s dais with a request directed at the chair for permission to speak Arabic. He justifies this request on the basis that he is addressing “his” people, the Palestinians who celebrate the Prophet’s Birthday and Christmas. What is significant is that he feels the need to justify his actions. The announcement of the switch to Arabic is interrupted first by an incongruous joke from Rachel Azariah and then by a legalistic framing from Shelly Yachimovich. I will analyse each element: the announcement, the joke, the legalistic framing, and the request for translation. 

Announcing a switch from Hebrew to Arabic, when a Palestinian Arabic speak\-er is in the presence of Jewish Israeli interlocutors, is the norm in all my findings \citep[63–87]{hawker2019a}. Conspicuous announcements, accompanied by markers of negative politeness, occurred in all cases that I recorded and observed where Palestinian speakers spoke Arabic on public platforms where at least part of the audience was Jewish, as in this case, or in small meetings where both Palestinian and Jewish citizens of Israel were present. In fact, in three instances, it was sufficient that only one person in the audience was Jewish Israeli among Palestinian citizens of Israel, all of them with knowledge of both languages, for a switch from Arabic to Hebrew to be announced. The way I have summarized these findings is that there is a norm among Arabic speakers in Israel to avoid speaking Arabic to anyone who does not want to be identified as an Arab in that interaction. It is a norm that has evolved in the habitus of colonized experiences since 1948 \citep{papp2011a}, resulting in the containment of Arabic within safely identified in-groups and the suppression of Arabic in any situation where there might be Jewish Israelis who do not identify as Arab. Violations of that norm – that is, directing any Arabic speech, even as an honest mistake, towards a person who does not identify as Arab – therefore must be justified in conspicuous announcements. Here, the announcement specifies that the addressees are Palestinians identified, and in fact constituted, as Arabs who are indexed by their language. The norm of avoiding Arabic outside the in-group is therefore not violated, despite the immediate Hebrew-speaking context; it is reinforced by the announcement.

Defiance is another aspect of the announcement of the switch to Arabic, and is mitigated by extensive negative politeness formulae, such as “would you please allow me” and “with your permission” (lines 2–3 and 9). Announcing the switch in an apologetic manner acknowledges another challenge to established norms: the speech from the Knesset dais breaks the doxa of Arabic silence on Zionist platforms. Since approximately 2010, Palestinian politicians in Israel have started to claim Arabic audibility in public institutions where Hebrew monolingualism dominates \citep[37–52]{hawker2019a}. These claims have taken the form of short rhetorical declarations in Arabic, immediately translated into Hebrew by the speakers themselves, as in this case. Usually, these declarations are met by metalinguistic comments that protest against Arabic audibility. Ayman Odeh is possibly pre-empting the protests by attending to his Hebrew-speaking interlocutors’ face \citep{eelen2001a}. 

The idea that it is ‘polite' to speak in the dominant language to interlocutors outside the suppressed in-group is reminiscent of the sociolinguistic situation in Catalonia during its emergence from Franco’s dictatorship \citep{woolard2012a}. Therefore, Ayman Odeh is being especially polite about being impolite; his request for permission both accepts the dominance of the Hebrew-speakers (which is ‘polite') and asserts the right to speak Arabic (‘impolite') in the face of the expected objection. The use of negative politeness strategies in particular create the layout of the conversation – negative politeness does not seek common ground between the interlocutors, but respects the distance and hierarchy between them \citep[25–29]{eelen2001a}. The Hebrew-speaking Jewish Israeli chair of the Knesset plenum is not ‘in' the in-group that Ayman Odeh is addressing, and this is reinforced by the announcement and the apologetic request. Both norms – the avoidance of Arabic in mixed company and the silencing of Arabic on Zionist institutional platforms – are equally restated and challenged.

The protest against the switch to Arabic comes in the form of a joke. Rachel Azariah’s joke is a kind of racist joke \citep{weaver2016a}. The implication of saying, in effect, “If Muhammad and Jesus’s birthdays are worth speaking in Arabic, then my birthday is worth speaking in Hebrew,” (line 7) is the depreciation of the value of the Muslim and Christian holidays, because Rachel Azariah’s importance to herself and those around her is not of the same public nature as that of Muhammad and Jesus. This type of humour organizes affiliations – who is the object of the joke and who is the intended audience \citep{meyer2000a}. It is not clear from the official record how well the joke is received; there is no indication of laughter, but there is no objection to it either. This particular joke might not have passed the thresholds of humour or offensiveness \citep{kuipers2016a}.

There was a more effective protest, however, against Ayman Odeh’s politeness strategy: “You don’t need to ask for permission,” Shelly Yachimovich said twice, interrupting him (lines 12 and 16). The chair intervened to validate the legalistic rule that is brought to bear against the tacit norm of silencing any Arabic speech: “It [Arabic] is an official language according to the Rules of Procedure” (lines 18–19). In fact, the letter of the Knesset Rules of Procedure does not mention Arabic as an official language, but rather as a second one. 

\begin{quote}
The sittings are conducted in Hebrew. Members of Knesset have the right to speak in Arabic as well; speeches in Arabic shall be translated into Hebrew. Only when a guest from abroad delivers a speech in the Plenum are speeches in different languages optional therein. \citep{knesset-nd-b}
\end{quote}

The chair had to reiterate the rule allowing Arabic speech because an objection to Arabic was expected. Shelly Yachimovich neutralized that prejudiced objection before it could even materialize. “You don’t need to ask for permission [to speak Arabic]” was an assertion that Israeli politics is governed by rules, not prejudices. Shelly Yachimovich was defending the republican structure of the state institutions against the accusation, implicit in Ayman Odeh’s excessive politeness yet perfectly pragmatically understood, that Hebrew is dominant and Arabic suppressed. At the time of this speech, and until July 2018, Arabic was the second official language of Israel. Its demotion to “language with special status” in 2018’s Jewish Nation-State Law was a legal materialization of the prejudice that transformed the republican structure. 

The republican structure informed the decision that all Knesset records be kept in Hebrew \citep[articles 22 and 38]{knesset2012a}.\footnote{The only speech recorded in Arabic in the Knesset minutes is the speech by Anwar Sadat on 20 November 1977, as he was a foreign guest \citep{knesset1977a}.} The only exceptions are short interjections transcribed into Hebrew script such as “happy holidays to everyone” and “thank you”, which remain in Arabic (lines 34 and 38). Professional interpreting was discontinued in the Knesset plenum in the 1960s \citep[32–37]{hawker2019a}. The vagueness of the passive voice in the Rules of Procedure – “speeches in Arabic shall be translated into Hebrew” – leaves responsibility for the delivery of translation open. The possibility of a need to translate into Arabic, for the benefit of Arabic-speaking citizens of Israel, is not considered. Translation into Hebrew for the stenographer’s minutes happens separately from interpretation in the plenum, and therefore Omer Bar-Lev needed to explicitly ask Ayman Odeh to consecutively interpret his own words (line 41). Ayman Odeh’s translation is a summary which reveals the political message: “Basically, what I said was, happy holidays to everyone, we shall continue to struggle together for life in dignity in our historical homeland.” This revelation triggers unspecified “shouts” or heckling (line 50), which are not retrievable in the record. Shelly Yachimovich and the Chair Yehiel Hilik Bar were using rules and procedures to shore up the republican structure against precisely this type of communication breakdown. The breakdown was triggered by the Jewish Israeli ethnonationalist objection (“What ‘historical homeland’!”; line 47) to a Palestinian claim to life in dignity in their country. The content of the “shouts” can only be inferred from Ayman Odeh’s response to them. Ayman Odeh’s response centred on language, rather than on politics or religion.

After two more turns of heckling, Ayman Odeh said the following, in Hebrew-only speech: 

\begin{quote}
\textit{That Arabs know Hebrew and most Jews don’t know Arabic is an added value to Arabs and not to Jews. Historically, Jews learnt more languages than other peoples. It just so happens that here, here specifically, the Arabs know more languages than Jews do, and that’s an added value in their favour and not in the Jews’ favour. I would like to say that knowing more than one culture enriches a person, it adds to a person, it allows a person to understand the history of another people, and to relate to their pains and longings.} (\citealt[132–133]{knesset2015a}; own translation)
\end{quote}

The shouts and objections to multilingual discourse and its political message are countered by a spontaneous defence of language learning. I argue that this justification reads as a manifesto for the multilingual code of the emerging Palestinian middle class \citep{ghanem2016a}. They are turning their multilingualism, once a burden of a minority contained by the state’s rules, into cultural capital \citep{block2013a}. In turn, this cultural capital is a means of (self-)production in the context of late capitalism – they are producing their own material success as a class. It is a manifesto for a liberal future that the middle class see as their right \citep{dean2014a}, but that sounds radical to the Knesset audience of 2015. The inaudible shouting in the debate was the rumblings of ethnonationalism \citep{peled1992a}, overrunning the republican rules that permitted, within constraints, performances of radical visions of a liberal future.

Addressing Arabic only to Arabs, and displaying multilingualism on Zionist institutional platforms in Israel, was therefore not contradictory. The liberal political messages conveyed by the styles of the multilingual speech were in dynamic relation to two other language-ideological positions. One was ethnonationalism, which demanded the strict adherence to the equation of one nation, one language, with Israel being notionally the preserve of the Jewish Israeli nation, and the other was republicanism, which carefully managed the diversity of a linguistic minority by establishing rules, and heaped the duties of translation and language learning on the minority \citep{gal2012a}.

\section{New evidence from 2022: Palestinians enter the establishment?}

The Knesset record for 23 December 2015 contains 531 pages of stenographer’s minutes. Using the method of identifying Arabic passages thanks to the stenographer’s metalinguistic comments, I was able to count that Arabic speech on that day added up to three of those pages, or less than 1 per cent of the day’s record. Fast-forwarding to 4 January 2022 reveals a different picture. The record for that day amounts to 289 pages, of which 33 are in Arabic, and five more pages constitute a debate about the inadmissible audibility of Arabic in the Knesset and its translation, resulting in 13 per cent of the day’s record. In this chapter, I conduct a time-lapse comparison. In 2015, Ayman Odeh spoke Arabic in a way that previously would have been unthinkable; seven years later, another significant event was recorded, begging the question whether the multilingual liberal future has indeed come about. 

One of my data-collection methods since the 2015 election campaigns has been to follow the YouTube, Twitter, and Facebook accounts of Palestinian members of the Knesset. For the 2020 Israeli elections, I also conducted field visits on electoral meetings \citep{hawker2020a}. When the speakers code-switch, I pay heightened attention. Usually, the multilingual Palestinian members of the Knesset speak Arabic to their Palestinian audiences, and Hebrew to their Jewish Israeli audiences and mixed audiences (such as the Knesset). This is what I would expect, in line with the two norms presented above: (1) the avoidance of Arabic in mixed company, and (2) the silencing of Arabic on Zionist institutional platforms. However, a video on the Facebook page of The Fans of Dr Mansour Abbas caught my eye, as it violated the second norm. 

In the two-minute video, Palestinian members of the Knesset Mansour Abbas and Ahmad Tibi were having a debate on the speaker’s podium in the Knesset, in Arabic, with no heckling or other interruptions, no request for permission, and no self-translation \citep{facebook2022a}. In the Knesset minutes for that day, I added up everything that followed the stenographer’s comment “speaks in Arabic; henceforth translation” in the parentheses denoting Arabic speech in Hebrew translation. I counted Arabic phrases transcribed into Hebrew script, without professional translation for the record. I noted metalinguistic comments by Knesset members, and the total added up to the 38 pages mentioned above. The video of the Abbas-Tibi exchange corresponds to page 58 of the Knesset minutes, in the middle of 27 pages of nearly uninterrupted Arabic (recorded in its official Hebrew translation; \cite[45–72]{knesset2022-a}). Owing to spatial constraints, I cannot reproduce the entirety of the Arabic segment here, as I was able to for Example \xref{hawker:ex1}. This abundance in itself was astonishing.

I selected the data presented in this section based on the themes emerging from the analysis of the 2015 speech. The speeches asking for permission to speak Arabic and the accompanying politeness strategies, the protests against Arabic audibility and requests for translation into Hebrew, references to explicit institutional rules and to tacit norms, and usages of Arabic that define and redefine identities are all presented below. On the basis of this new evidence from 2022, I will argue strongly that the norm that had suppressed Arabic in the Knesset has been broken. The context for this change is that for the first time in Israeli history, a political party for which only Palestinian citizens of Israel vote,\footnote{Though ballots are anonymous, the list of voting preferences by location on the Central Elections Committee website (\url{https://votes24.bechirot.gov.il/}) indicates that most voters for the Islamist party live in Palestinian towns and villages in Israel.} and that claims to represent the interests of that social group in particular, was included in the government coalition formed on 13 June 2021. The Joint List split in two in January 2021, and the Islamist party (named United Arab List) led by Mansour Abbas entered the government coalition. The rest of the Joint List, still led by Ayman Odeh, went down from 15 seats to 6 in 2021 (out of the 120 total in the Knesset) and remained in the opposition. The Arabic debate in the Knesset captured in the video was therefore a debate between a representative of the Israeli government and of opposition to it. That political divide, too, was astonishing.

I will also argue, equally strongly, that more (self-)reflection is needed in the field of language-contact studies. I have been arguing for a number of years that looking at contact between languages that index nationalities that are in apparently zero-sum conflict solely through the prism of that conflict entails the omission of important areas of contact such as economic activity, which cannot be reduced to inter-national conflict (\citealt{hawker2018a}; \citealt{heller2015a}). Here is where the attention to class comes into play. Socioeconomic class intersects with nationality (and other identities) to create more subtle nuances of language indexicalities, in dynamic relation to the central equation of one nation, one language. 

Nevertheless, the new evidence challenges some aspects of my earlier argument about the enregisterment of multilingual discourse as a marker of Palestinian middle-class public positioning, embraced with increasing confidence. The challenge is that the material aspects of class, and the link these aspects have to relative power, may play a greater role in destigmatizing marginal groups’ discourse, and in enregistering new code patterns, than any other identity marker I had previously considered. It is not a liberal utopia come true; it is the late-capitalist materialism trump card on the table. The topic of debate on 4 January 2022 was an amendment to a law regarding municipal planning, concerning approving the connection of marginalized Palestinian and other Arab communities in Israel to the electricity grid. The amendment was referred to as the “Electricity Law” \citep{knesset2022-b}.

\subsection{Permission and politeness}

In Example \xref{hawker:ex1}, negative politeness strategies were a way of announcing a violation of the Arabic silencing in Israeli state institutions, and a way of highlighting distinctions of hierarchy that placed Hebrew and Hebrew speakers above Arabic and Arabic speakers in the Knesset. The announcement of switches to Arabic in Example \xref{hawker:ex2} (all taken from the 4 January 2022 Knesset stenographer’s minutes, own translation) uses positive rather than negative politeness.


\begin{exe}\ex\label{hawker:ex2}
\textit{Announcing switches to Arabic in the Knesset, 4 January 2022.}
 \begin{xlist}
\exi{a.}\label{hawker:ex2a}
 \textit{\textbf{Walid Taha (United Arab List):}} [after one and a half pages \\
of minutes of his speech in Hebrew]

\textit{(\ul{Speaks in Arabic; henceforth translation}: To the Arabs in \\
society I say, the discussions around the Electricity Law have \\
started now, and the Opposition with the exception of the Joint \\
List have left the plenum, and I do not know if they will come \\
back again to vote tomorrow morning or not. […]) I want to \\
thank the members of my committee again. I assume that you \\
understood everything I said in Arabic. […]}

\textbf{\textit{Chair Mansour Abbas (United Arab List):}}

\textit{But certainly the Minister Eli Avidar understood every word.}
\end{xlist}
\end{exe}

Polite attention to the Jewish Israeli interlocutors who might not have “understood everything [that was] said in Arabic” came in Hebrew at the end of the Arabic section, and there was no translation. Instead of negative politeness, there is a strategy of inclusion, or positive politeness. Mentioning “my committee” – Walid Taha was then the head of the Internal Affairs Committee, where most members are Jewish citizens of Israel – and making assumptions about their knowledge, is the opposite strategy to that of Ayman Odeh’s in Example \xref{hawker:ex1}, where he reinforced distance, hierarchy, and respect through negative politeness. In both cases, the effect of the politeness strategy is to mitigate the ‘impoliteness' of using Arabic to address interlocutors who do not wish to be addressed as Arabs, as per the norm of Arabic avoidance in mixed company. Positive politeness strategies say, effectively, “We are in a team together, so you won’t be offended by the way I speak.” Mansour Abbas furthered this strategy by including Cabinet Minister Eli Avidar in the team. Eli Avidar’s Knesset profile mentions his knowledge of Arabic \citep{knesset-nd-a}. The pattern of launching into Arabic without preliminaries is then repeated by several Palestinian speakers in the Knesset \citep[46–51]{knesset2022-a}. The Arabic speeches are interspersed with Hebrew speeches of similar length, from the same multilingual speakers. The pattern appears to be that when the speech is specifically targeted at fellow Palestinian Knesset members or Palestinian citizens of Israel beyond the immediate audience, Arabic is resorted to. Hebrew is for everyone. 

\begin{exe}\exi{(2)} \begin{xlist}
\exi{b.}\label{hawker:ex2b}
 \textit{\textbf{Ayman Odeh (Joint List):}}

\textit{Honourable Chair, my fellow members} [masc.] \textit{and members}\newline[fem.] \textit{of the Knesset, allow me please to speak in the Arabic\\
language, even though I am in favour of minority rights,\\
including collective rights, and the Jews are in the minority\\
here; but nevertheless, allow me this time.}

\textbf{\textit{Minister in the Prime Minister’s Office Eli Avidar:}}

\textbf{- - -}

\textbf{\textit{Ayman Odeh (Joint List):}}

\textit{Listen, you are Jewish, but you are Egyptian, a speaker of \\
Arabic.}

\textit{(\ul{Speaks in Arabic; henceforth translation}: I want to speak in \\
Arabic out of a desire to hide. On the principle of “if you leave \\
it in the heart it will torment, and if you pour it out it will \\
embarrass”. So it’s better for us to speak to each other in the \\
Arabic language, because there are some matters that are \\
important. The purpose of these matters is not to air our dirty \\
laundry, but rather to talk directly with our Arab public.} […]\textit{)}
 \end{xlist}
\end{exe}

Ayman Odeh returns to negative politeness strategies (“allow me please”), and the languages are again referenced as indexes of nationality: Hebrew for Jewish Israelis and Arabic for Palestinians. Normally, the Palestinians are in the minority, and Ayman Odeh is in the position of having to lobby for their rights. Yet at this session, most Jewish Israeli members of Knesset had left the room, and so the Palestinians were in the majority, and Ayman Odeh joked about his political principles of standing up for minorities. He was hoping that this joke would organize affiliations, as self-deprecating jokes tend to do, by bringing the audience together in alliance with him \citep[318–319]{meyer2000a}. Later, he built on this idea of alliance by engaging in positive politeness, reinforcing the common ground, by saying that Eli Avidar – who was born in Alexandria – was Egyptian and therefore included in the in-group that speaks Arabic. On the basis of speaking Arabic to Arabs, albeit from the Knesset dais, he then launched into Arabic for a substantial discussion – not for short greetings or slogans, but to agonize over the divisions in Palestinian society – which would have been unthinkable in 2015.

\hspace*{-.4pt}The Palestinian speakers found themselves in an unprecedented situation: they had one Knesset session to sort out political differences internal to Palestinian society in Israel, with the government – including Israeli right-wing and far-right parties, among them Eli Avidar’s Yisrael Beitenu – in some sort of agreement with the Islamists, and the opposition, Likud, out of the room. The issue of connecting excluded communities to the electricity grid, including 100,000 Bedouin living in officially unrecognized villages in the Negev/Naqab, is a human rights problem relating to the right to a decent standard of living \citep{forum2020a}. The Palestinian representatives in the Knesset had been lobbying for years for these people’s rights. However, the two Palestinian parties were now in disagreement over the framing: was the issue one of discrimination against a national minority (as the Joint List presented it in the debate), or was it to do with socioeconomic disadvantage (as the Islamists presented)? For those who claimed discrimination, negative politeness made sense; the powerful group that engaged in discrimination was kept at a respectful distance. For those who claimed disadvantage, positive politeness was apt, since the group which had a socioeconomic advantage, including the Palestinian Islamist politicians, was able to offer help to those in need.


\begin{exe}\exi{(2)} \begin{xlist}
\exi{c.}\label{hawker:ex2c}
\textbf{\textit{Aida Touma-Suleiman (Joint List):}} 

[…] \textit{The ones who had prevented the planning and delayed the \\
approval of the comprehensive plans were the state, the various \\
governments, and the Ministry of the Interior. And why am I \\
emphasizing this? Because the discourse must not descend to \\
the level of: “We are willing to accept the narrative and the \\
story and the discourse of those governments that oppressed \\
the Arab population and deprived it and take the reckoning on \\
ourselves, according to this discourse.”}

\textbf{\textit{Chair Mansour Abbas (United Arab List):}}

\textit{\ul{Says in Arabic}: Speak in Arabic, Aida. Everyone here is Arab.}

\textbf{\textit{Aida Touma-Suleiman (Joint List):}}

\textit{(\ul{Speaks in Arabic; henceforth translation}: I speak in Arabic\\
 and I speak in Hebrew, and I was just going to switch to Arabic\\
  to try to understand this law accurately.} […]\textit{)}
   \end{xlist}
\end{exe}


Aida Touma-Suleiman gave her speech in Hebrew, though she claimed the “us” of the oppressed Arab national minority. She reiterated the framing of the electricity issue as one of discrimination. The Chair invited her to speak in Arabic, not on the basis of the official legalistic rule, but on the basis of the tacit norm: you can speak Arabic in the in-group, providing everyone wants to be addressed as Arabs. Mansour Abbas reassured her on the point of identification of interlocutors. Her reaction echoed the pro-multilingualism manifesto of the emergent Palestinian middle class: knowing several languages is cultural capital. Speaking and analysing in two languages allowed Aida Touma-Suleiman to “understand this law accurately”. There was no issue of politeness. Negative politeness is only for addressing the overlords, to let them know that you resent their dominance.

\subsection{Protests against Arabic audibility}

All is well, then: Palestinians can speak Arabic to Palestinians in the Knesset now? Not quite, since the lack of institutional translation betrayed a problem regarding the identity of Israeli state institutions, and how languages spoken therein create that identity. The problem of translation soon became manifest as protests against the very audibility of Arabic on 5 January 2022, as the Knesset session on the Electricity Law continued into its second day. The day started with Itamar Ben-Gvir, representative of the Jewish Power party (a Jewish supremacist organization), calling Walid Taha (United Arab List) a “terrorist”, and receiving a caution from the Chair Mickey Levy (Yesh Atid, a centrist secularist party). Walid Taha, as the sponsor of the bill, then explained the proposed law in Hebrew \citep[82–89]{knesset2022-a}, and was interrupted multiple times by Jewish Israeli Knesset members of the National Religious party, the Likud party, and Shas, a party that represents Orthodox Jews of Middle Eastern and North African heritage. After yet another interruption, Walid Taha switched to Arabic, without preliminaries. The debate soon broke down, with multiple heckles hindering scheduled speeches, until Mansour Abbas took over chairing, and managed to return the debate to its apparently normal course, in Hebrew.\\


\begin{exe}\ex\label{hawker:ex3}
\textit{Asking for translations into Hebrew in the Knesset, 5 January 2022.}\\

\begin{linenumbers*}
\textbf{\textit{Walid Taha (Chair of the Committee on Internal Affairs and\\
Environmental Protection):}}

\textit{(\ul{Speaks in Arabic; henceforth translation}: I appeal to everyone \\
who is following the discussions now around the Electricity \\
Law, the law that they have tried in every way to thwart and \\
prevent from arriving here today, to vote on it in both readings, \\
the second and the third. }[…]\textit{)}

\textbf{\textit{Keti Kathrin Shitrit (Likud):}}

\textit{Walid, we don’t understand, what’s going on?}

\textbf{\textit{Walid Taha:}}

\textit{You didn’t understand the needs of Arab society for decades \\
either, Keti.}

\textbf{\textit{Keti Kathrin Shitrit (Likud):}}

\textit{Lies. This won’t help you. It’s a lie, a lie.}

\textbf{\textit{Itamar Ben-Gvir (Jewish Power):}}

\textit{- - - like in Syria. He thinks he’s in Syria.}

\textbf{\textit{Walid Taha:}}

\textit{(\ul{Speaks in Arabic; henceforth translation}: 96 hours were\\
shortened} – [about Itamar Ben-Gvir] \textit{that’s a racist fascist,\\
speaking in a vulgar way that corresponds to his vulgar\\
manner – I was saying that the Committee shortened the debate\\
time to 14 hours }[…]\textit{.)}

\textbf{\textit{Keti Kathrin Shitrit (Likud):}}

\textit{Really, Mickey, we don’t have anything to do here, he is\\
speaking in Arabic, let’s leave. Really, why are you} [plur.] \\
\textit{here?}

\textbf{\textit{Walid Taha:}}

\textit{(\ul{Speaks in Arabic; henceforth translation}: Still, they were not \\
satisfied, and declared a boycott of the Knesset debates, left the \\
Knesset and went home yesterday} […]. \textit{Now that I am \\
speaking, opposition Knesset members are protesting that I \\
speak in Arabic.)}

\textit{\textbf{Shouts:} This is crazy … in the Knesset. Disgrace. Simply\\ unbelievable.}

\textbf{\textit{Walid Taha:}}

\textit{(\ul{Speaks in Arabic; henceforth translation}: They are so angry, \\
they could explode. They are angry that I am speaking in \\
Arabic, and I will speak Arabic whenever I want. Avi Dichter,\\
do you hear, I will speak Arabic and whoever doesn’t like it\\
can go and drink the water of the Dead Sea.)}

\textit{\textbf{Avi Dichter (Likud):}}

\textit{(\ul{Says in Arabic}: Walid, why didn’t you speak Arabic when you\\
voted against the peace agreements with the Emirates? Until\\
today you didn’t say anything in Arabic.} [the back-and-forth\\
between Walid Taha and Avi Dichter, mostly in Arabic,\\
continues for 10 pages \citep[92–101]{knesset2022-a}].\textit{)}

\textit{\textbf{Avi Dichter (Likud):}}

\textit{You are not a reasonable man. If you were a reasonable man,\\
you would speak in Arabic and translate into Hebrew for the\\
98 per cent who don’t understand you. As it is, you are not\\
speaking to the Knesset. You are in the Knesset and you are not\\
speaking to the Knesset.} 

\textbf{\textit{Walid Taha:}}

\textit{Me, when you speak in Hebrew, I do understand. That you\\
don’t understand when I speak Arabic, that’s your problem.\\
Why didn’t you learn Arabic?}

\textit{\textbf{Itamar Ben-Gvir (Jewish Power):}}

\textit{- - - You are a guest.}

\textit{\textbf{David Amsalem (Likud):}}

\textit{You are right, it’s our problem that they allowed you to speak\\
Arabic here. You} [plur.] \textit{are our problem.} […] \textit{It’s a problem\\
that you are still speaking here.} […] \textit{Here they are, the pair\\
who are stealing the country, on the podium.}
 
\textit{\textbf{Chair Mansour Abbas:}}

- - -

\textbf{\textit{Walid Taha:}}

\textit{\ul{(Speaks in Arabic; henceforth translation:)}}

\textit{\textbf{David Amsalem (Likud):}}

\textit{Tell me, should we bring you coffee and baklava? Aren’t you\\
ashamed? Look at this, in the Knesset of Israel, two Arabs are\\
talking among themselves. Look at what we have come to. They are\\
making a mockery of us.}

\textit{\textbf{Shouts:} We want translation - - -}

\textit{\textbf{Chair Mansour Abbas:}}

\textit{Alright. Avi Dichter and Yariv Levin know Arabic, they know\\
what we are talking about.}

\textit{\textbf{David Amsalem (Likud):}}

\textit{Soon you’ll get electricity, you took 53 billion, and now you’ve\\
turned the Knesset into Arabic, do you understand?}

\textit{\textbf{Chair Mansour Abbas:}}

\textit{David Amsalem, calm down, at the end of the day it’s just an\\
Electricity Law.}

\textit{\textbf{David Amsalem (Likud):}}

\textit{At the end of the day it’s the theft of a third of the State of\\
Israel.}

\textit{\textbf{Ofir Katz (Likud):}}

\textit{Think of the Jews here, that we are a minority, be considerate\\
of us.}

\textit{\textbf{Chair Mansour Abbas:}}

\textit{Ofir, alright. Well done.} […] \textit{Look, you can ask Avi Dichter;\\
whatever he} [Walid Taha] \textit{had said in Hebrew, he said in\\
Arabic.}

\textit{\textbf{Keti Kathrin Shitrit (Likud):}}

\textit{If he won’t speak Hebrew, let them bring us earphones.}
 \end{linenumbers*}
\end{exe}

Walid Taha’s insistence on speaking Arabic and refusing to self-translate appeared hostile, based on the lack of politeness markers. However, his ten-minute exchange in Arabic with Avi Dichter seemed to be pleasant to both of them, as they are smiling at each other, or to themselves, as observed in the video recording between minutes 36 and 47 \citep{knesset2022-c}. They seemed to be on the same discursive team, using the Arabic they had in common for some form of positive politeness, enjoying the jousting. In contrast, around them the reactions to the Arabic became increasingly angry – David Amsalem looked apoplectic – until Kathrin Shitrit mentioned the need for earphones for simultaneous interpretation \citep[131]{knesset2022-a}. Then, the discussion returned to Hebrew until the end of the session, when the law was passed. 

The trigger for the switch to Arabic seems to be the turn immediately preceding it – a complaint by Moshe Abutbul that Walid Taha had ignored problems that Ultra-Orthodox Jews face which are supposedly similar to those faced by Palestinian and other citizens of Israel regarding the supply of electricity.\footnote{Searching for evidence that Orthodox religious school (Yeshiva) dormitories were deprived of electricity, I found that some Yeshivas could not afford their electricity bills, and had their supply temporarily cut \citep{bureau2014a}.} Haredi Jews (as the Ultra-Orthodox prefer to call themselves) are among the poorest sectors of Israeli society for a combination of reasons, but mostly by communal choice \citep[14]{o2020a}. Faced with this confusion of the issue, and provoked by constant heckling from rightwing opposition to the Electricity Law, Walid Taha turned abruptly to Arabic to address pointedly the members of the Joint List who had criticized the framing of the bill, and to appeal to them for support (lines 3–7). This switch prompted the first request for translation, from Kathrin Shitrit (line 9), which Walid Taha dismissed irritably with the quip, “You didn’t understand the needs of Arab society [for electricity] for decades, either” (lines 11–12), implying, “Well, too bad for you, not understanding Arabic.” The debate went downhill from there. 

In this mood, Walid Taha adhered to Arabic for declarations of defiance (lines 37–40) which mimicked the manipulation of the indexicality of one nation, one language, explored in Example \xref{hawker:ex1}. However, he did not soften his defiance with polite greetings and self-translations, as seen in Example \xref{hawker:ex1}; apparently, politeness is not necessary when one is in government. What is meant by ‘manipulation' is that the speaker speaks in a language that s/he claims as defining their national identity, in symbolic reference to the equation of one nation, one language. They are not speaking that language simply because they are of that nationality; in practice, they are speaking several languages as a skilled multilingual, for multiple pragmatic purposes. The symbolic reference to, rather than practical use of, the one national language is on a higher order of indexicality. The speech act of displaying now unapologetic multilingualism is intended to signify political power, or at least entitlement to the particular platform above the hecklers’ objections. 

The provocation of having to listen to political theatre in Arabic in the Knesset incensed some of the interlocutors, but also overwhelmed the structures of Knesset debate. The Chair, Mickey Levy, apparently gave up on intervening for order after Kathrin Shetrit challenged him to restore the supremacy of Hebrew with, “he is speaking in Arabic, let’s leave. Really, why are you here?” (lines 24–26). There was no keeping to time; Walid Taha and Avi Dichter jousted away; heckling was unchecked for a while. Itamar Ben-Gvir’s racist “You are a guest” (line 58) – meaning that an Arab is barely tolerated in the Knesset (and in Itamar Ben-Gvir’s vision of the Land of Israel) – was not countered with a caution. Mansour Abbas arrived to start his shift as chair, amid the heckling, with an aside to the speaker at the dais, Walid Taha. He settled into his chair at minute 47 of the video recording of the debate \citep{knesset2022-c}. The stenographer noted an effort to translate the Arabic speech, but the effort failed, and the record leaves the turns as blanks (lines 65–67). Even this minimal Knesset translation service, the last line of republican discursive defence, had broken down.

The day before, it had seemed acceptable for Arabic speakers to address other Arabic speakers in Arabic in the Knesset, since that had maintained the norm of using Arabic within the in-group. However, this level of Arabic audibility underestimated the strength of the second norm: the silencing of Arabic on Zionist platforms. By having a brief exchange in Arabic on the Knesset podium, Mansour Abbas and Walid Taha undid the Zionist nature of that stage. For David Amsalem, they had transformed the Knesset into an Arab coffee house, offensive to the history of Zionist institution-building: “Tell me, should we bring you coffee and baklava? Aren’t you ashamed? Look at this, in the Knesset of Israel, two Arabs are talking among themselves. Look at what we have come to. They are making a mockery of us” (lines 69–72). Such is the power of discourse to transform situations – from Zionist to non-Zionist – on the back of language ideologies \citep{irvine2009a}. Soon, the very existence of Israel was at stake; the idea that marginalized Bedouin villages could connect to electricity and might contemplate the reversal of decades of infrastructural neglect amounted to “the theft of a third of the State of Israel” (lines 63, 84–85). All this emotion was caused by Arabic in the Knesset.

 Ofir Katz repeated the request for translation by returning to Ayman Odeh’s joke (Example \ref{hawker:ex2b}), saying, “Think of the Jews here, that we are a minority, be considerate of us” (lines 87–88). The joke did not make Mansour Abbas laugh (he reacted with a sarcastic “Well done”; line 90), because the affiliations were not correct; Ofir Katz, as opposed to Ayman Odeh, was not actually fighting for minority rights, and could not therefore claim self-deprecation. Kathrin Shetrit finally requested institutional simultaneous interpreting, after nearly one hour of arguing (line 94). Institutional simultaneous interpreting has been provided in the Knesset on the annual “Arabic language day”, a performance of the liberal utopia periodically enacted by the Joint List since 2017 \citep[53–55]{hawker2019a}. The rest of the year, the earphones stay in the cupboard.
 
Mansour Abbas reassured the angry interlocutors (lines 90–92) that despite the absence of explicit signposting, Walid Taha had indeed self-translated his Arabic, with Avi Dichter as his witness, whose reliability was underpinned by being Jewish Israeli. Hebrew dominance was reinstated until the end of the session, patiently chaired. It did not matter whether the Arabic had in fact been translated into Hebrew for comprehension purposes; what mattered was that the norm was explicitly re-established that Arabic speakers were responsible for their own interpreting. The law was passed, and Benjamin Netanyahu called it a “black day for Zionism and democracy” \citep{mualem2022a}. At the time of writing, no villages had been newly connected to electricity, and the governing coalition looked fragile \citep{abu2022a}. 

\subsection{Politeness that breaks norms}

Islamists joining the Israeli government broke a political taboo, and they also inadvertently changed the norms of language choices for the Palestinians and other Arabs who speak Hebrew in addition to their native Arabic. The starkest description of that change in norms (and perhaps the most absurd description to a reader outside Israeli and Palestinian realities) is David Amsalem’s lament: “Look at this, in the Knesset of Israel, two Arabs are talking among themselves. Look at what we have come to” (Example \ref{hawker:ex3}, lines 69–72). Look, indeed: since the Zionism of the Knesset platform is discursively constituted by Hebrew dominance, Arabs speaking Arabic raises questions as to the precise identity of the state institution that is being (re-)formed. Putting to one side the radical vision of a putative liberal state where every citizen would thrive, let’s apply the language ideology test. If Israel were a republican state that manages minorities, then Arabic should have been translated in the Knesset. If it were an ethnonationalist state that excludes minorities, then Arabic speakers would be delegitimized. The litmus test of language ideologies, at least in January 2022, indicates an inclination towards the latter.

The discursive events of 4 and 5 January 2022 were not exceptions, but points in a trajectory of change. Palestinian politicians in Israel have been speaking defiantly in Arabic on Zionist platforms for ten years \citep[39–50]{hawker2019a}, and the Islamists have simply taken it to another level \citep{leal2022a}. The differences between 2015 and 2022 are in the newer lack of announcement of switches to Arabic, the positive rather than negative politeness strategies, and the occasional refusal to self-translate. Overall, I will submit that the changes are related to the interactive organizational work done by politeness. There has long been a stereotype that Jewish citizens of Israel, especially those born to families with a socialist-Zionist-pioneering ethos, will speak to the point, without polite adornments, while Palestinian citizens of Israel, especially those who are born to traditionalist patriarchal families, will speak indirectly, paying attention to interlocutors’ face-saving needs \citep{katriel1986a}. Mansour Abbas, described by Israeli commentators as “soft-spoken and affable” \citep{karsh2022a} is the personification of the traditionalist patriarch, and his performance of that type of politeness served him well in chairing difficult Knesset debates.

The stereotypes of contrasting Jewish Israeli and Palestinian politeness exist on the level of the first order of indexicality, in that they can be inherited, unselfconscious styles. Yet they can also be found in elaborations on higher orders of indexicality. These elaborations exist in Palestinians’ self-conscious pride in being rude, such as when Walid Taha said, “Avi Dichter, do you hear, I will speak Arabic and whoever doesn’t like it can go and drink the water of the Dead Sea” (Example \ref{hawker:ex3}, lines 38–40). Another elaboration exists in Avi Dichter’s policing of what is a “reasonable” way for a Palestinian to speak: “If you were a reasonable man, you would speak in Arabic and translate into Hebrew” (Example \ref{hawker:ex3}, lines 48–49). These are explicit affirmations of what is polite and impolite in the situation. My contribution to the analysis of politeness in political debates is to move away from stereotypes or elaborations on them, and towards organization of power relations. When Palestinian politicians use negative politeness strategies in asking permission to speak Arabic, they are showing regard for a raciolinguistic hierarchy of discursive legitimacy \citep[251]{heller2017a}, which serves to expose that hierarchy where there is a pretence that there is none. When they include Jewish Arabic speakers such as Eli Avidar and Avi Dichter in the in-group that can be addressed in Arabic, using positive politeness strategies, they are showing that the cultural capital of speaking several languages is supranational. It is to this accumulation of cultural capital that I turn in the discussion of class formation that follows.

\section{Discussion: Combining conflict and late capitalism in the multilingualism analysis}

Israel has developed a knowledge- and technology-based economy, with some of the highest inequality and poverty rates among high-income nations \citep{o2021a}. Since the 1990s, in line with the neo-liberalization of the economy globally as well as in Israel, tertiary education has increased exponentially. At the same time, by 2015, the salary gaps between tertiary-educated and uneducated workers grew, and yet Palestinian and other Arab university graduates earned an average monthly income of USD 1,885, as compared to USD 3,149 among Jewish Israelis, and Palestinian and other Arab university graduates were more likely to be un- or under-employed \citep{ayalon2016a}. These graduates want political engagement, voting for the Joint List, but they also want the aspects of the good life that consumerist late capitalism can offer \citep{carmeli2004a}. Clearly, this good life is not compatible with disconnection from the electricity grid. What is more, this educated, aspirational emergent middle class has been making multilingual political jokes about discrimination against them since 2009 \citep{henkin2009a}. I argue that there are articulated links between the emergent middle class in consumerist late capitalism and enregistering multilingual discourse.

I still argue for those articulations, but would like to use this paragraph to reflect. I had thought that any increase in Arabic audibility on Zionist platforms, discursively transforming the identity of those platforms, would be accompanied by some steps towards liberal multiculturalism, as the performances of the Palestinian politicians had augured over the past ten years. The manifesto on the cultural capital of multilingualism had been a promise that I had taken as a commissive speech act. In this liberal vision, there was supposed to be dignity, respect, and perhaps even resolution of conflict. The speakers themselves had said so, in the words of Ayman Odeh reported in Section 2 and Aida Touma-Suleiman in Section 3.1 above. In linguistic anthropology, we appreciate our participants’ perspectives, yet I depart from my participants’ self-perception to offer my own perspective on the evidence. 

\begin{quote}
While participant perspectives are privileged, they are not necessarily made into the cornerstone of the analysis. Researchers take a critical stance with regard to all texts and appeal to critical ethnography, which allows them to triangulate the data, and to critical discourse analysis, which enables them to uncover hidden ideological meanings.  \citep[25]{multilingual2004a}
\end{quote}

Combing through the evidence with the fine-toothed analysis of politeness, what I find does not resemble dignity, respect, or resolution of conflict. What I find is supranational opportunistic alliances for power maximization, which gives speakers in power the entitlement to dominate, in whatever language. I find selective invocation of reinvented norms and of fragile and unfair institutional rules. I also find suppression of both conflict and conflict resolution: “It’s just an Electricity Law” (Example \ref{hawker:ex3}, line 81). Speakers are not making any steps towards liberal visions of freedom. What they are advancing towards is the offering of late capitalism: improvements in economic conditions are negotiated on behalf of the collective, yet for the benefit of individuals, both for politicians winning points at elections and voters seeing specific material gains. Wider political issues regarding the definition of the common good and conceptions of freedom are suspended \citep{davis2008a}. Late capitalism and its commodification of language, including of multilingualism’s cultural capital, is put to the service not of the liberal side of the emergent middle class, but of the consumerist side. And yet, the most painful realization is knowing that this offering is an improvement on the apartheid experienced by Palestinians at present  \citep{international2022a}. The push towards Palestinian access to the bounties of consumerism might undo some aspects of apartheid and co-opt others \citep{taha2020a}. Meanwhile, to simplify debates a little, institutions experiencing bouts of multilingual discourse could take the interpreters’ earphones out of the cupboard where they have been gathering dust – the precarious jobs of interpreters are indeed a harbinger of the late-capitalist future.

\section{Conclusion}
I have brought together evidence from political speeches that code-switch between Arabic and Hebrew to show that there is a trajectory of change in multilingual styles. The commodification of multilingualism that comes with late capitalism has appreciated the cultural capital of middle-class speakers who use several languages. They use their languages skilfully for pragmatic purposes of politeness and of group identification, and to convey political messages. Already departed by several indexical orders from the nationalist conflict patterns of one nation, one language, the new patterns do not align neatly with liberal values of dignity, respect, and conflict resolution. Rather, the multilingual discourse is the code of the emergent middle class that sees its relative power as an opportunity to push for greater access to consumerist material benefits. This opportunity challenges an oppressive reality but might not lead to the type of freedom that the speakers themselves had articulated with multilingual discourse. 

\section*{Acknowledgements}
My sincere thanks go to the members of the CoTiSp team, Valentina Serreli, Katrin Pfadenhauer, and Sofia Rüdiger, who encouraged me with patience and kindness as I wrote this chapter. My greatest intellectual debts go to Deborah Cameron, who mentored my academic career and taught me to identify structural biases against speeches by marginalized speakers, to Alexandre Duchêne, who has trusted my expert judgment enough to make me trust myself and who introduced me to the scholarship on late capitalism, and to Clive Holes, who has defended my focus on Israel/Palestine and taught me to discern pragmatic purposes of Arabic political speeches. I am grateful to politicians and their constituents in Israel/Palestine who allowed me to attend their meetings and learn from them how to be a citizen. I thank my friends who have guided me through the Israel/Palestine political landscape: Alaa, Helal, Maha, Noa, Saleh, Tal, and Yael.

\printbibliography[heading=subbibliography, notkeyword=this]

\end{document}
