\documentclass[output=paper]{langscibook}
\ChapterDOI{10.5281/zenodo.10497371}

\author{Jacopo Falchetta \orcid{0000-0002-0467-9079} \affiliation{University of Bergamo and IREMAM, Aix-en-Provence}}


\title[The ``unnecessary'' use of French in Moroccan Arabic]{The ``unnecessary'' use of French in Moroccan Arabic: Social discriminant or
collaborative enterprise?}
\shorttitlerunninghead{The ``unnecessary'' use of French in Moroccan Arabic}
%TODO: Im abstract keine Abkürzungen?


\abstract{
   The practice of code-switching between Moroccan Arabic and French is well documented among speakers of the former variety. Previous scientific stances on Moroccan Arabic-French code-switching (especially expressed in works almost entirely concerned with formal structure rather than sociopragmatic functions) merely saw bilinguals’ recurring to French as an effort to upgrade their social status by using a ``prestigious'' language. Based on a sample of a corpus collected by the author, this contribution presents the first steps of a study which, by adopting an interactionalist approach, aims to identify the sociopragmatic functions that speakers fulfil through code-switching. The results show that, in communities where not all speakers are highly proficient, code-switching can also be used to fill the existing gaps between the speakers’ linguistic repertoires. Such gaps are notably due to different levels of exposure to non-native French or being taught in private schools where French is the teaching language. It is also argued that, even in those cases in which code-switching is in fact due to the positive connotations of French, it is more fruitful to go beyond the ``prestige versus stigmatization'' dichotomy by looking at the indexical meanings associated with the forms employed. Examining the contexts of exposure to, learning, and use of these forms is suggested as a possible method to address such indexicalities.
}

\IfFileExists{../localcommands.tex}{
  \addbibresource{../localbibliography.bib}
  % add all extra packages you need to load to this file

\usepackage{tabularx,multicol}
\usepackage{url}
\urlstyle{same}

\usepackage{listings}
\lstset{basicstyle=\ttfamily,tabsize=2,breaklines=true}

\usepackage{langsci-basic}
\usepackage{langsci-optional}
\usepackage{langsci-lgr}
\usepackage{langsci-osl}
% \usepackage{./langsci/styles/langsci-lgr}
% \usepackage{./langsci/styles/langsci-osl}
% \usepackage{langsci-gb4e}

\usepackage{tikz}
\usetikzlibrary{patterns,calc}
\pgfdeclarepatternformonly{south east lines}{\pgfqpoint{-0pt}{-0pt}}{\pgfqpoint{3pt}{3pt}}{\pgfqpoint{3pt}{3pt}}{
    \pgfsetlinewidth{0.6pt}
    \pgfpathmoveto{\pgfqpoint{0pt}{3pt}}
    \pgfpathlineto{\pgfqpoint{3pt}{0pt}}
    \pgfpathmoveto{\pgfqpoint{.2pt}{-.2pt}}
    \pgfpathlineto{\pgfqpoint{-.2pt}{.2pt}}
    \pgfpathmoveto{\pgfqpoint{3.2pt}{2.8pt}}
    \pgfpathlineto{\pgfqpoint{2.8pt}{3.2pt}}
    \pgfusepath{stroke}}
    
\usepackage{stmaryrd}
\usepackage{wasysym}
\usepackage{multirow}
\usepackage{caption}
\usepackage{subcaption}
\usepackage{mathrsfs}
\usepackage{qtree}

\usepackage{linguex}


  %pminos do not split footnotes
% \interfootnotelinepenalty=10000 %Footnote in Laporte chapters has to be split SN


%\DeclareIndexNameFormat{default}{%
%\nameparts{#1}%
%\usebibmacro{index:name}%
%{\index[names]}%
%{\namepartfamily}%
%{\namepartgiveni}%
% {}% L1
% {}% L2
%{\namepartprefix}% generates spurious space L3
%{\namepartsuffix}% generates spurious space L4
%}

%  {\DeclareIndexNameFormat{default}{%
%     \usebibmacro{index:name}{\index[names]}{#1}{#3}{#5}{#7}}}

%\DeclareIndexNameFormat{default}{%
%  \usebibmacro{index:name}{\sindex[nom]}{#1}{#3}{#5}{#7}}

%\DeclareIndexNameFormat{default}{%
%  \usebibmacro{index:name}{\sindex[person]}{#1}{#3}{#5}{#7}}
%\DeclareIndexNameFormat{default}{%
%\nameparts{#1} \usebibmacro{index:name}{\sindex[person]]}{\namepartfamily}{‌​\namepartgiven}{\nam‌​epartprefix}{\namepa‌​rtsuffix}}

%\newcommand{\smiley}{:)}

%\renewbibmacro*{index:name}[5]{%
%\usebibmacro{index:entry}{#1}%
%{\iffieldundef{usera}{}{\thefield{usera}\actualoperator}\mkbibindexname{#2}{#3}{#4}{#5}}}

% \newcommand{\noop}[1]{}

%remove for final
%\overfullrule=1mm

\newcommand{\tobi}[2]}}
\renewcommand{\S}[1]{\tobi{#1}{\textsc{*}}}

% this volume references
% puts: [this volume]
% already defined: \citetv
%\newcommand{\citepv}[1]{(\citeauthor{#1} \citeyear*{#1} [this volume])}
\newcommand{\citealtv}[1]{\citeauthor{#1} \citeyear*{#1} [this volume]}

%parentheses around example number
\newcommand{\pref}[1]{(\ref{#1})}

% in-text examples

\newcommand{\lnex}[1]{\textit{#1}} %target lang word
\newcommand{\lnlit}[1]{(lit.: `#1')} %literal reading
\newcommand{\lnlat}[1]{(#1)} % latinization
\newcommand{\lntrans}[1]{`#1'} %translation
\newcommand{\lnexl}[2]%
{\lnex{#1}{} \lnlat{#2}} % ex with latinization
\newcommand{\lnexlat}[3]{\lnex{#1}{} \lnlat{#2}{} \lntrans{#3}} % ex with latinization and tranl.

%ch01
\newcommand{\co}[1]{\mbox{\textbf{#1}}}

%ch09

\newcommand{\cyrbulg}[1]{\begin{otherlanguage*}{bulgarian}#1\end{otherlanguage*}}


%ch10
\newcommand{\nlp}{{\small NLP}}
\newcommand{\mwe}{{\small MWE}}
\newcommand{\rae}{{\small RAE}}
\newcommand{\lvc}{{\small LVC}}
\newcommand{\pos}{{\small P}o{\small S}}
%\newcommand{\todo}[1]{ \textcolor{red}{#1} }

%\renewcommand{\labelenumi}{\theenumi}
%\ainamefmt{{vv}{ll}{, ff}{, jj}} % fullname

\newcommand{\biberror}[1]{{\color{red}#1}}

\newcommand{\osenovaitem}{--~} 
  %% hyphenation points for line breaks
%% Normally, automatic hyphenation in LaTeX is very good
%% If a word is mis-hyphenated, add it to this file
%%
%% add information to TeX file before \begin{document} with:
%% %% hyphenation points for line breaks
%% Normally, automatic hyphenation in LaTeX is very good
%% If a word is mis-hyphenated, add it to this file
%%
%% add information to TeX file before \begin{document} with:
%% %% hyphenation points for line breaks
%% Normally, automatic hyphenation in LaTeX is very good
%% If a word is mis-hyphenated, add it to this file
%%
%% add information to TeX file before \begin{document} with:
%% \include{localhyphenation}
\hyphenation{
    Beck-man
    Ngu-yen
    back-chan-nel
    back-chan-nels
    mo-not-o-nous
    ste-reo-typ-i-cal
}

\hyphenation{
    Beck-man
    Ngu-yen
    back-chan-nel
    back-chan-nels
    mo-not-o-nous
    ste-reo-typ-i-cal
}

\hyphenation{
    Beck-man
    Ngu-yen
    back-chan-nel
    back-chan-nels
    mo-not-o-nous
    ste-reo-typ-i-cal
}
 
  \togglepaper[2]%%chapternumber
}{}



\newleipzig{1, 2, 3}{1, 2, 3}{1st, 2nd, 3rd person}
\newleipzig{ACT}{act}{active voice}
\newleipzig{CONJ}{conj}{conjunction}
\newleipzig{DEF}{def}{definite}
\newleipzig{DEM}{dem}{demonstrative}
\newleipzig{F}{f}{feminine}
\newleipzig{GEN}{gen}{genitive}
\newleipzig{INDF}{indf}{indefinite}
\newleipzig{IPFV}{ipfv}{imperfective}
\newleipzig{M}{m}{masculine}
\newleipzig{PL}{pl}{plural}
\newleipzig{PFV}{pfv}{perfective}
\newleipzig{PREP}{prep}{preposition}
\newleipzig{PRVB}{prvb}{preverb}
\newleipzig{PTCP}{ptcp}{participle}
\newleipzig{REL}{rel}{relative particle}
\newleipzig{SG}{sg}{singular}


\newleipzig{ARG}{arg}{argument-introducting particle}
\newacronym{CS}{CS}{code-switching}
\newacronym{EC}{EC}{Embedded Constituent}
\newacronym{FR}{FR}{French}
\newleipzig{INTRJ}{intrj}{interjection}
\newacronym{MA}{MA}{Moroccan Arabic}
\newacronym{SA}{SA}{Standard Arabic}



\begin{document}
\maketitle

\section{Introduction}
The study of \gls*{CS} between French and \gls*{MA} has generally been focused on the formal aspects of the alternation between the two varieties, trying to explain \textit{how} they co-exist in a single string of speech (most notably \citealt{bentahila_syntax_1983,bentahila_patterns_1995, lahlou_morpho-syntactic_1991, ziamari_development_2007, ziamari_moroccan_2009,ziamari_determiner_2018}). Conversely, \textit{why} speakers alternate between the two codes (i.e., the sociopragmatic functions of French-\gls*{MA} \gls*{CS} in the interactional context) has been questioned less frequently and, when it has, it has rarely been supported through in-depth interactional analyses of the socio-pragmatic reasons leading speakers to engage in \gls*{CS}. In most of the previous works on French and \gls*{MA} mixing, this practice has been attributed to the ambition to elevate one’s own social status, although other studies have seen it as motivated by participation in an urban bilingual culture, or by aesthetic and expressive intent. Nevertheless, it is argued here that there can also be more urgent motives pushing speakers to engage in this practice, as will be shown through an analysis of verbal interactions occurring in the framework of a guided production test. It is also argued that, even in those cases in which the prestigious status of French seems relevant to what is happening, it is useful to attempt to qualify the notion of ``prestige''— which has long been employed in the explanation of sociolinguistic phenomena as a sort of black box — by taking other aspects of the problem into consideration, such as contexts of language learning and indexicalities.


This chapter presents the partial results of a work in progress which aims to show what functions are fulfilled through \gls*{CS} by young, male speakers of \gls*{MA}, and how diverging lexical repertoires, possibly due to unequal learning opportunities, are connected to the particular practices observed and to the functions identified. First, however, a general diachronic picture of the contact between French and \gls*{MA} varieties will be given in order to provide a historical background to the present sociolinguistic status of French and \gls*{MA} in Morocco. Following this, a brief review of the previous literature on \gls*{MA}-French \gls*{CS} will be sketched. An operative definition of \gls*{CS} will be formulated before describing the data corpus, and then a provisional classification of sociopragmatic functions will be attempted based on the analysis of the \gls*{CS} occurrences. Finally, some observations will be made on the implications of the results and the steps that further research should take.


\section{The status of French and Moroccan Arabic in Morocco}
French has been enjoying a privileged status in Morocco since the beginning of the French Protectorate (1912--1956). During this crucial period, it not only functioned as the official language of administration and public affairs, but also as the main teaching language of the school system established by the colonizers. It also became associated with the transmission of technical and all sorts of specialized knowledge. In this context, the introduction of new French terms connected to technological advancements must have been the main motive that pushed \gls*{MA} speakers to start borrowing massively from French vocabulary (cf. \citealt[355]{brunot_emprunts_1949}), although borrowings were and are by no means limited to technical domains.\footnote{Besides, \acrshort*{SA} or \gls*{MA} words were also used in several cases to translate some of those innovative terms \citep[356]{brunot_emprunts_1949}.} 

After independence from France (1956), a much greater part of the Moroccan society became exposed to French — in spite of the government’s more or less immediate adoption of \gls*{SA} as the only official language \citep{moatassime_langages_2006, sayahi_diglossia_2014}. This is because, while during the Protectorate access to education was limited to the colonizers’ and the members of a few Moroccan élite families \citep[197--198]{redouane_arabisation_1998},  “the spread of education, the sociodemographic changes (population growth and movement)” that came with the political independence, “and finally the role of mass media” contributed to increasing occasions of contact with French to an unprecedented degree \citep[42]{sayahi_diglossia_2014}.


The role played by education was significantly curbed between the 1970s and the 1980s, when the teaching of all school subjects was switched from French to Arabic with what are known as the Arabization policies. In the Moroccan context, these did not affect scientific and other university faculties (e.g., Economics, Medicine, etc.), which maintain the exclusive use of French to this day. As a consequence, the side effect of Arabization was a problematic language gap, as most students who attended state school used to find themselves abruptly switching from Arabic to French when starting their BA programmes. Of course, the same did not apply to those students whose families could afford to send them to private primary and secondary school institutes where the teaching language is French for all subjects (\citealt[210--212]{ennaji_multilingualism_2005}, \citealt[81]{pellegrini_enseignement_2019}).\footnote{This issue was re-addressed in 2014, when the teaching of scientific subjects at secondary-school level was reconverted to French \citep[81–82]{pellegrini_enseignement_2019}.}


Today, French competes with \gls*{SA} as the variety employed in formal and learned contexts, but maintains a dominant role in the scientific arena as well as in the private business sector. Ideologically, French is also the language associated with Western acculturation and/or modernity in public discourse and questionnaire-based interviews about individuals’ language attitudes.\footnote{On language attitudes in Morocco towards French in general, see \citet[193--195]{ennaji_multilingualism_2005} and included bibliography; for an overview of Moroccan civil society’s positions vis-à-vis the use of \gls*{SA}, French, and English in education and the media, see \citet[109--129]{pellegrini_enseignement_2019}; on the association of French with sciences and modernity as it emerges from individual interviews, see \citet{tamer_language_2003,tamer_what_2006, ennaji_multilingualism_2005, boutieri_two_2012, chakrani_work_2014}.} 

As for \gls*{MA}, the other variety involved in the \gls*{CS} practices analyzed here, it is native to most of the Moroccan population.\footnote{According to the 2014 national census, about 92 per cent of Moroccans speak (non-Standard) Arabic as their mother tongue, while 26.1 per cent were classified as Berber speakers \citep{noauthor_rgph_nodate}. While it is possible that a number of Arabic-Berber bilinguals declared that they spoke \gls*{MA} only, the number of Berber speakers is known to be declining, at least in Morocco.} It has traditionally been seen as standing in a diglossic \citep{ferguson_diglossia_1959} relationship as the L variety against \gls*{SA} (the H variety), although now it is generally admitted that this strictly dichotomous view, which has been revisited by the author himself \citep{ferguson_diglossia_1991}, is inadequate to represent the more fluid linguistic reality of Morocco and the other Arabic-speaking countries. Since the complex relationship and interplay between \gls*{SA} and Arabic colloquial varieties is not at issue in this work, it will suffice here to say very briefly that “in linguistic practice, vernacular Arabic exists in a \textit{symbiotic} relationship with Standard Arabic”, whereby “at the level of ideology [\dots] vernacular Arabic exists in a \textit{subordinate} position vis-à-vis Standard Arabic” (\citealt[4]{hachimi_contextualizing_2022}; my emphasis). However, even though \gls*{MA} still enjoys no official status in Morocco, it is now enlarging its domains of use at the expense of \gls*{SA} and French, especially in formal and learned contexts and youth cultural expressions. Ideologically, \textit{darija} (/da:ri:ʒa/)\footnote{A \gls*{MA} word indicating the variety of Arabic spoken in ordinary daily situations. It is generally used in opposition to \textit{fusha} (/fusˁħa:/), a term which brings together both the classical and modern versions of \gls*{SA}.} is now reported to be positively evaluated, especially by young people, and play a role in processes of identity construction (cf. \citealt{de_ruiter_les_2006, de_ruiter_marche_2014, caubet_darija_2017}).

\section{Previous research on \gls*{MA}-French code-switching}

\gls*{CS} between \gls*{MA} and French in spontaneous verbal exchanges has been studied since at least the 1970s. It has most often been observed among educated speakers, especially university students and graduates (e.g., \citealt{abbassi_sociolinguistic_1977,bentahila_syntax_1983,bentahila_patterns_1995,lahlou_morpho-syntactic_1991,ennaji_multilingualism_2005, ziamari_development_2007,ziamari_moroccan_2009,ziamari_determiner_2018,post_impact_2015}), as people with a certain degree of bilingualism are assumed to alternate codes more frequently. In line with an interest in the syntactic and morphological constraints of code alternation, the vast majority of these works have mainly focused on the structural features of \gls*{CS}, such as allowed switching points, inter- versus intra-sentential \gls*{CS}, and similar formal issues. Studies adopting this perspective have undoubtedly shed light on important formal dynamics of \gls*{CS} and serve as a basis for the type of research proposed here. However, few of them have addressed the sociopragmatic functions that the use of \gls*{CS} fulfils in a given verbal interaction. In the majority of cases, the main motive mentioned for engaging in \gls*{MA}-French \gls*{CS} is the speaker’s desire to enhance their speech style by demonstrating their knowledge of a socially prestigious and valuable language. The excerpt below exemplifies this.


\begin{quote}
	The technique of studding Arabic discourse with French lexical items is a means whereby [bilinguals educated under the Arabisation policies] manage to \textit{distinguish themselves} from their less educated contemporaries who would not be in a position to switch to French at all; the presence of French vocabulary signals that the speakers, although they are basically speaking Arabic, \textit{do have the requisite knowledge of French to be able to call upon it when they feel like it}. (\citealt[84]{bentahila_patterns_1995}; my emphasis)
\end{quote}

\noindent % TODO: Hier ist Ziamari erwähnt ohne ein Buch zu zitieren
Nonetheless, a few exceptions to this trend are found in later works. Ziamari, for example, makes the association between French-\gls*{MA} \gls*{CS} and urban youth practices. Her community of study is made up of students of Francophone faculties at the University of Meknes, Morocco.


\begin{quote} 
	Codeswitching is mainly associated with an \textit{urban environment.} [\dots] It is this urban environment, where different languages come into closest contact and \textit{where there is the greatest incidence of bilingualism,} which favors the emergence of the practice. Codeswitching in Morocco [...], while it does occur in various social categories, is essentially \textit{a feature of the speech of young bilinguals}. (\citealt[276]{ziamari_development_2007}; my emphasis)
\end{quote}


More recent studies (e.g., \citealt{chakrani_sociolinguistic_2010, chakrani_work_2014, khoumssi_attitudes_2020}) have taken a greater interest in speakers’ attitudes towards \gls*{CS}. Rather than analyzing samples of code-switched utterances, the authors made use of questionnaires in which consultants were explicitly asked about their opinion of \gls*{CS} and other language varieties. These works provide more information about contexts of use than social functions and meanings of \gls*{MA}-French \gls*{CS}, observing that it is usually employed in informal communication in the classroom or peer groups. Conversely, \citet{post_impact_2015} joins the study of verbal productions and language attitudes through a systematic analysis of the relations between extra-linguistic factors and \gls*{CS} structure and frequency of use; however, motivations for \gls*{CS} at the interactional level are not touched upon. To my knowledge, \citegen{ziamari_moroccan_2009} is the only study that addresses \gls*{CS} in spontaneous communicative contexts from an entirely pragmatic perspective by finding correspondences between use of French and the information structure of \gls*{MA} utterances.\footnote{More precisely, Ziamari finds that French constituents embedded in \gls*{MA} speech are frequent in focus and topic positions.} Other studies focusing on Maghrebi artistic texts have been more sensitive to the issue of the socio-pragmatic potential of French-Maghrebi Arabic \gls*{CS}, e.g. in the expression of humor (cf. \citealt{caubet_alternance_1998} for Algerian Arabic) or in the aesthetic search for expressivity in song lyrics (cf. \citealt{bentahila_language_2002}, \citealt{davies_code_2006,davies_code_2008} for \gls*{MA}, and \citealt{caubet_jeux_2002} for Algerian Arabic). This chapter will therefore explore the possibility of enlarging knowledge of the sociopragmatic functions of \gls*{MA}-French \gls*{CS} in spontaneous exchanges by analyzing verbal interactions involving members of a group of young male Moroccan peers, in order to reveal their purposes for engaging in code alternation.

\section{The data}
The corpus that will be exploited for this analysis originally served as a basis for the study of three linguistic variables in the language use of \gls*{MA} speakers in the town of Temara, Morocco.\footnote{For more details on the variables and results of the analysis, see \citet{falchetta_social_2019}.} The interest in studying linguistic variation in Temara, a former rural suburb of the capital, Rabat, lies in its peculiar sociodemographic situation. Since the years of the French protectorate, Temara has been attracting huge numbers of migrants from other areas of the country, especially after Rabat’s residential areas became saturated, and this has led to a steep demographic growth of the town’s population, particularly from the 1970s onwards (from circa 20,000 in 1972 to over 300,000 in 2014). Linguistically, the convergence of immigrants from different regions has led to the encounter of different regional varieties of \gls*{MA}, which has given way to phenomena of dialect contact such as levelling and reallocation of diatopically distinctive features. 

What is of interest for this study is the social background of the interviewees, all of whom are children of rural migrant parents and were living in Temara at the time the data were collected. While the parents had, in most cases, received little or no education, they had managed to raise their children in more than decent material conditions and fund their education in state or (less frequently) private schools. As a result, most of the interviewees had been educated to at least the final year of secondary school, although a minority of them had not gone past middle school. In addition, given the proximity of Rabat, routes between the latter and Temara are well-served, and Temari youth are used to commuting to the capital city frequently to attend university, work in skilled or specialized positions, or engage in leisure activities that are not available in the town where they live (such as clubbing or skating).

The data are taken from a corpus of speech samples obtained by means of an experiment which I designed specifically to elicit one of the variables and test the speakers’ tendency to employ the two variants involved (more details in \citealt{falchetta_social_2019}). This experiment, to which I shall henceforth refer as ``the test'', required (at least) two participants\footnote{All the test sessions analyzed in this chapter involved two participants each.} to engage in narratives and verbal exchanges for the fulfilment of the following tasks: one of them was asked to recount a speechless video (a hidden-camera prank) to the other, who had not watched it; the latter was subsequently asked some questions, allegedly with the aim of ensuring the first participant had been clear and exhaustive in summarizing the prank.\footnote{In fact, the real purpose of the questions to the second participant was to elicit words containing the targeted variable.} The choice of having the consultants summarize practical jokes that were shown in routines was motivated by the exigency to minimize the observer’s paradox \citep[208--209]{labov_sociolinguistic_1973}: this verbal task closely resembled a type of activity in which \gls*{MA} speakers had been seen engaging spontaneously (telling a peer about a funny hidden-camera prank), and the routine nature of the video made it easier for the recounter to memorize and verbalize the amusing situation. If there is obviously no way that one can be assured the consultants would have used the same wording and register in the researcher’s absence, the test was still effective in triggering loose exchanges between the participants, and in eliciting the desired variable.

The participants were 17 pairs of male residents of Temara; three videos were verbalized by each of the participants, meaning each of the 17 sessions included summaries of and answers to questions on six videos. Each session lasted between 30 and 60 minutes, including the time required for the participants to watch the respective videos. 

The test was conducted between mid-July and mid-August 2017. Two criteria were used for the recruitment of the participants: they had to be prior acquaintances and used to speaking \gls*{MA} between one another.\footnote{The latter criterion was relevant for speakers whose first native language was a variety of Berber.} Their ages ranged between 18 and 38, with an average of 25.6. The data have not been transcribed; excerpts from the sessions were written down for the phonological analysis using IPA characters, and the same will be done here for the analytical purposes of this chapter.

The key advantage of this type of test is that it makes it possible to collect samples of speech from an unlimited number of speakers while keeping the communicative context and purposes constant, to the benefit of the comparability of language use. While the test was administered with the purpose of collecting data on variation \textit{within} \gls*{MA}, the recorded exchanges between the participants in the test sessions often involved the use of French forms in mainly \gls*{MA} utterances, which thus makes them useful for an analysis of \gls*{MA}-French \gls*{CS} in interactions between \gls*{MA} native speakers. Since the prank videos at issue took place in Canada, they were checked one by one and it was made sure none of them contained any English or French signage that could have triggered \gls*{CS} in the speech of the viewing participants.\footnote{The only potential trigger, the writing \textit{Location} (French for ‘renting’) on a lorry, was not exploited by any participant, as no one used this term.}

One of the aims of this preliminary analysis of French-\gls*{MA} \gls*{CS} in my data was to assess whether the two factors of French proficiency and connectedness to an urban environment were indeed relevant to the frequency and extent of this practice in an individual’s speech, as has been contended in previous literature.\footnote{However, for this study, the discussion will be limited to the recurring sociopragmatic functions that \gls*{CS} appeared to fulfil across the interviewees’ sample, regardless of the influence of these social factors.} For this reason, 16 out of 34 consultants were selected for further analysis, so that the sample would cover different degrees of exposure to French and different types of personal social networks (urbanized versus non-urbanized). The 8 sessions thus selected yielded approximately 166 minutes of native \gls*{MA} speech, or an average of 20 minutes per session. The distribution of the participants’ sample according to each of these two variables is reported in \tabref{falchetta:tab:1} and \tabref{falchetta:tab:2} respectively. As shown in \tabref{falchetta:tab:1}, most interviewees did not receive a Francophone education for most of their schooling years, and presumably few of them used French on a daily basis at work. As for attending a university faculty with French as the teaching language, this did not seem to improve their ease in speaking the language; on the contrary, during interviews conducted with some participants before the test, many of those who attended university courses in French reported not being competent enough in the language to fully understand classes. Therefore, this study presents an important difference with respect to previous works \citep{abbassi_sociolinguistic_1977,bentahila_syntax_1983,bentahila_patterns_1995,lahlou_morpho-syntactic_1991, ziamari_development_2007, ziamari_moroccan_2009, ziamari_determiner_2018}: while these were explicitly focused on analyzing \gls*{CS} among a population of fluent French speakers, my young Temari consultants’ proficiency was not a matter of concern (as the test originally had purposes unrelated to \gls*{CS} analysis) and ranged from low to high according to the participant.\footnote{However, it should be noted that all works, including mine, base their considerations on the participants’ proficiency either on assumptions or on self-evaluations.} Nevertheless, even those less competent in French did engage in \gls*{CS} during their test performance, as will be shown in the next section.

\begin{table} % TODO: Ist das okay mit den Zeilenumbrüchen so?
	\begin{tabularx}{\textwidth}{QQQQ}\midrule\toprule
		\textbf{Number of participants} & \textbf{Higher education in French}                          & \textbf{Pre-higher education in French} & \textbf{French as a working language} \\\midrule
		3                               &                                                              &                                         &                                       \\ 
		7                               & \surd                                                        &                                         &                                       \\ 
		1                               &                                                              & \surd                                   &                                       \\ 
		3                               & \surd                                                        &                                         & \surd                                 \\ 
		2                               & \surd                                                        & \surd                                   & \surd                                 \\
		Total: 16                       &% \multicolumn{3}{>{\centering\setlength\hsize{3\hsize} }L}{}                                                                                  
		 \\ \bottomrule\midrule
	\end{tabularx}
	\caption{Distribution of the participants according to sources of exposure to French.}
	\label{falchetta:tab:1}
\end{table}

\begin{table}
	\begin{tabularx}{\textwidth}{QQQ}\midrule\toprule
		\textbf{Number of participants} & \textbf{Urbanized social networks}
\footnotemark[14] 
		& \textbf{Non-urbanized social networks} \\\midrule
		9        & & \surd                                  \\ 
		7      & \surd &\\ 
		Total: 16         & %\multicolumn{2}{>{\centering\setlength\hsize{2\hsize}}L}{}                                                                                                                                                                               
		 \\ \bottomrule\midrule
	\end{tabularx}
	\caption{Distribution of the participants according to type of social network.}
	\label{falchetta:tab:2}
\end{table}
%TODO: You should not add footnotes to table or figures (5.7.4)

% TODO: Kinda sketchy solution
\footnotetext[14]{By this, I mean social networks that include work- or leisure-related contacts with speakers
	from larger urban centres, such as Rabat or Casablanca.}

\setcounter{footnote}{14}


\section{Formal features of \gls*{MA}-French \gls*{CS}}

According to \citet[285]{manfredi_language_2015}, “a distinction” between \gls*{CS} and the closely related phenomenon of borrowing “is necessary and possible”. They define \gls*{CS} as “the presence of lexical or sentential material belonging to different linguistic systems, provided that its \textit{different origin is still transparent in the speaker’s output} in one or more grammatical domains” (\citealt[286]{manfredi_language_2015}; my emphasis). Therefore, in order to distinguish \gls*{CS} from borrowing, it is crucial to define the criteria by which the linguist can state whether “the material in the speaker’s output” enjoys “native or foreign status” \citep[286]{manfredi_language_2015}. Keeping this in mind, in this preliminary analysis an item was classified as ‘code-switched' if at least one semantically equivalent lexical item was employed in the recipient language (i.e., \gls*{MA}) with equal or greater frequency. In this sense, it is assumed that a given item enjoys ‘foreign status' insofar as it is seemingly ‘unnecessary' to the speakers of the recipient language, as they do have synonyms of that item in their native repertoire. \citet[300, 306]{manfredi_language_2015} also occasionally classify an item of “foreign” origin as \gls*{CS} (rather than borrowing) by using a similar criterion, although the main point of their study is that prosodic and intonational regularities can be important cues for the identification of \gls*{CS}.\footnote{Further developments of the analysis presented here will also aim to check whether such regularities match the instances of \gls*{CS} identified in the Temara corpus.} Using the criterion of the absence of an equally or more widespread synonym, 157 instances of French-derived forms were classified as \gls*{CS} in the sample analyzed, or an average slightly short of ten instances per participant, with individual figures ranging from 2 to 32.\footnote{Occurrences of \gls*{CS} in speech directed at the researcher (e.g., during the answering task) were omitted, to avoid switches to French due to accommodation to a non-native speaker. However, cases in which the same term used in speech directed at the researcher was also employed with the other participant, or was adopted after the other participant had employed it for the same referent, were retained in the count.} 

From a formal point of view, all speech in the eight sessions analyzed was mainly in \gls*{MA}, with limited constituents being switched to French. According to the Matrix Language Framework \citep{myers-scotton_contact_2002}, it can then be stated that \gls*{MA} almost invariably constituted the Matrix Language and French the Embedded Language. Concerning the extension of the \glspl*{EC}, they hardly went beyond noun-phrase level, with just one interviewee producing two entire clauses in French. This matches Bentahila \& Davies' (\citeyear{bentahila_patterns_1995}) data, which revealed that bilinguals with a post-Arabization education (like my participants) mainly engaged in a similar type of \gls*{CS} — with \glspl*{EC} mostly limited to noun phrases — more than 20 years before my data were collected. By contrast, bilinguals who received a Francophone (pre-Arabization) education engaged in a radically different type of \gls*{CS} in Bentahila and Davies’s data, with a greater use of clause- and sentence-wide \glspl*{EC} and with inter-sentential switches more frequent than intra-sentential switches. Below are some examples of \gls*{CS} from the Temara corpus, with different types of \glspl*{EC} involved, including

\begin{itemize}
	\item morpheme-level \glspl*{EC};
	\item noun-phrase-level \glspl*{EC}; and
	\item prepositional- and adverbial-phrase-level \glspl*{EC}.
\end{itemize}

\noindent
In some cases, \gls*{CS} occurs below word level, with a French lexical morpheme being combined with \gls*{MA} inflectional morphemes, as in Examples \xref{falchetta:ex:1} and \xref{falchetta:ex:2}.

\begin{exe} 
	\ex\label{falchetta:ex:1} \gls*{MA} (Temara corpus; GS-7)\footnotemark \\
	\gll 	hi:ja 		rˁa 	ka:-tǝbqa 					m\textbf{fi:ks}ja 	\\
	3\F\SG{} 	\ARG{} 	\PRVB-stay;\IPFV.3\F\SG{}	fix;\PTCP.\F{} 		\\
	\glt 	“It remains \textbf{fixed}” (< FR \textit{fixer} ‘to fix’)
	\ex\label{falchetta:ex:2} \gls*{MA} (Temara corpus; GG-11) \\
	\gll 	{wa:ħǝd l-ma:la:bi:s} 	lli 	ka:-\textbf{tʔatir}i 							bǝzza:f 	d 		n-na:s \\
	\INDF-clothes 			\REL{} 	\PRVB-\textbf{attract};\IPFV.3\F\SG{}			many		\GEN{}	\DEF-people \\
	\glt	“Clothes that \textbf{attract} lots of people” (< FR \textit{attirer} ‘to attract’)
\end{exe}
\footnotetext{Every interviewee is represented by a two- or three-letter code, which is always indicated next to the source. The number following the code indicates the total number of \glspl*{EC} calculated for the participant. Switched items on which the example is focused are in bold.}

\noindent
When at least one lexeme is switched at noun-phrase level, \gls*{CS} can be limited to the noun alone (as in Example \ref{falchetta:ex:3}) or involve associated determiners (Example \ref{falchetta:ex:4}) and noun complements (Example \ref{falchetta:ex:5}).

\nogltOffset

\begin{exe}
	\ex\label{falchetta:ex:3} \gls*{MA} (Temara corpus; DS-9) \\
	\gll 	za:jdi:n   					li:-ha   	hu:ma   f   l-\textbf{volym}\footnotemark \\
	increase;\PTCP.\ACT.\PL{}	to-3\F\SG{}	3\PL{}	in 	\DEF-\textbf{volume} \\
	\glt 	“They turned up its \textbf{volume}” (< FR \textit{volume} ‘volume’)
	\ex\label{falchetta:ex:4} \gls*{MA} (Temara corpus; IDG-10) \\
	\gll 	fa:ʃ   	jǝmʃi   			d-dǝrri			tǝ-jʒi:w   				\textbf{de}     \textbf{turist} \\
	\CONJ{}	go;\IPFV.3\M\SG{}	\DEF-child.\M{}	\PRVB-come;\IPFV.3\PL{}	\textbf{\INDF.\PL{}} \textbf{tourist} \\
	\glt	“When the little kid leaves, \textbf {some tourists} come” (< FR \textit{des touristes} ‘tourists’)
	\ex\label{falchetta:ex:5} \gls*{MA} (Temara corpus; EMK-26) \\
	\gll 	mu:hi:mm	dǝk   	\textbf{la} 	\textbf{sal}	\textbf{d-atãt} \\
	important	\DEM{}	\textbf{\DEF{}}			\textbf{room}	\textbf{\PREP-waiting} \\
	\glt	“Anyway, it’s that type of \textbf{waiting room}” (< FR \textit{salle d’attente} ‘waiting room’)
\end{exe}

\resetgltOffset

\footnotetext{Concerning the treatment of definite articles preposed to French \glspl*{EC}, it was observed that the /l-/ preceding masculine singular nouns starting with a consonant was often phonetically realized as [ǝl] and therefore sensibly different from French <le> ([lǝ]). For this reason, it is to be interpreted as the \gls*{MA} article /l-/ ([lǝ]); accordingly, it gets assimilated to a following sun (coronal) letter (cf. Example  \ref{falchetta:ex:12}). Conversely, /la/ and /le/ are to be interpreted as the French feminine singular and plural articles respectively, and therefore as part of \glspl*{EC}. \citet[152]{boumans_modelling_2000} make similar observations in the context of Algerian Arabic-French \gls*{CS}. The issue of the confusion between the \gls*{MA} definite article and French <l’>, i.e., the form preceding singular nouns beginning with a vowel, will not be dealt with here as none of the examples happens to present such a case.}

\noindent
Several cases of switched prepositional (see Example \ref{falchetta:ex:6}) and adverbial phrases (Example \ref{falchetta:ex:7}) or conjunctions (Example \ref{falchetta:ex:8}) are also found, albeit more rarely.

\begin{exe}
	\ex\label{falchetta:ex:6} \gls*{MA} (Temara corpus; DN-15) \\
	\gll \textbf{ã} \textbf{fas}   mʕa:-hum\\
	\textbf{\PREP{}} \textbf{face} with-\Tpl{} \\
	\glt “\textbf{Facing} them” (< FR \textit{en face} ‘opposite’)\footnotemark
	\ex\label{falchetta:ex:7} \gls*{MA} (Temara corpus; GS-7) \\
	\gll \textbf{otomatikmã}  hu:wa   tǝ-jdxǝl   l-ha:di:k   lli   tǝħt\\
	\textbf{automatically} \Third\M\SG{} \PRVB-enter;\Third\M\SG{} to-\DEM{} \REL{} below \\
	\glt “\textbf{Automatically}, he falls into that thing below” (< FR \textit{automatiquement} ‘automatically’)
	\ex\label{falchetta:ex:8} \gls*{MA} (Temara corpus; LT-32) \\
	\gll \textbf{a}   \textbf{ʃak}   \textbf{fwa}   ʃi   wa:ħǝd   kǝ-jʒi\\
	\textbf{\PREP{}} \textbf{each} \textbf{time} \INDF{} one \PRVB-come;\Third\M\SG{} \\
	\glt “\textbf{Every time} somebody comes \dots” (< FR \textit{à chaque fois} ‘every time’)
\end{exe}
\footnotetext{In this case, the French adverbial locution is used in combination with the \gls*{MA} preposition /mʕa/ ‘with’ to form the \gls*{MA} prepositional locution /\textit{ã fas} mʕa/ ‘opposite’.}

\noindent
As regards the phonetic integration of French-derived forms, this varies from full integration into \gls*{MA} phonetics to full adaptation to the prescribed French pronunciation. Generally speaking, more proficient speakers tend towards the latter, although a certain degree of adaptation to local phonotactics is found in most cases. For instance, \tabref{falchetta:tab:3} shows different pronunciations of the vowels in the French \textit{feu rouge} [fø ʀuʒ] ‘traffic light’, as produced by three participants. These are ranked by increasing (assumed) exposure to French in descending order. Based on this example, it would actually seem that greater exposure does lead to pronunciations closer to the standard. Conversely, the second column shows that the same factor did not necessarily entail greater engagement in \gls*{CS} during the test. In any case, analyses of a broader sample would be necessary to state to what extent proficiency is influential in the degree of integration of switched items.\footnote{\citegen[159--163]{post_impact_2015} analysis of the correlation between \gls*{CS} practices  and proficiency in French shows trends in the rate of different types of French constituents switched, but does not provide absolute numbers of switched constituents.}


\begin{table}
	\begin{tabularx}{\textwidth}{llQl}
		\midrule\toprule
		\textbf{Interviewee} & \textbf{\glspl*{EC}} & \textbf{Description} & \textbf{Realization} \\\midrule
		DC & 5 & no schooling in French up to (and including) secondary level, unemployed & [fi:rʉʒ] \\
		DN & 15 & no schooling in French up to (and including) secondary level, uses French as a working language, frequently goes to Rabat for work and leisure & [fø:rʉʒ] \\
		LPI & 5 & all schooling in French, uses French as a working language, works in Rabat & [fø:ru:ʒ]\\ \bottomrule\midrule
	\end{tabularx}
	\caption{Different phonetic integrations of \textit{feu rouge} ‘traffic light’.}
	\label{falchetta:tab:3}
\end{table}


\noindent
While the number of \glspl*{EC} in the corpus is not quantitatively significant compared to other studies, certain sociopragmatic functions fulfilled through \gls*{CS} appeared to recur across individual uses. As I shall demonstrate in the next section, the interest in pinpointing these functions lies in the new light they shed on \gls*{MA} speakers’ \gls*{CS} practices with respect to the data reported in previous literature.

\section{Sociopragmatic functions of \gls*{CS}}
\subsection{The purpose of classifying functions}
In order to show the various pragmatic functions fulfilled through the use of \gls*{CS}, an interactional analysis was carried out on each verbal exchange in which speakers made use of \glspl*{EC}. Following this, a tentative classification of the pragmatic functions was made based on those that recurred the most across the corpus. This endeavour freely follows the model provided and the analysis carried out by Gumperz %TODO: Autor hier ohne Buchreferenz
with his list of conversational functions of \gls*{CS}, which he bases on parallel examples taken from three different pairs of alternated languages \citep[75--84]{gumperz_discourse_1982}. However, it should be noted that the aim of the classification made in the present work is not to elaborate a comprehensive taxonomy, but rather to show how \gls*{MA}-French \gls*{CS} can also serve as an instrument to achieve greater inter-comprehension and/or convey information in a more effective way. This is already suggested by the context of these exchanges, which is the fulfilment of a verbal task that depends on the correct transmission of information from the speaker to the listener.

\subsection{Information-conveying \gls*{CS}}
Two types of communicative strategies have been identified in which \gls*{CS} is a means for achieving mutual understanding. Both of them can be assimilated to Gumperz’s “reiteration” strategy, which is when “a message in one code is repeated in the other code, either literally or in somewhat modified form. In some cases such repetitions may serve to clarify what is said, but often they simply amplify or emphasize a message” \citep[78]{gumperz_discourse_1982}. It is argued that, in the exchanges analyzed here, clarification is what is aimed at. 

The first strategy, or sociopragmatic function of \gls*{CS}, can be defined as one of ‘clarification through translation'. This is obtained by juxtaposing a French-derived lexeme with its (Standard or Moroccan) Arabic translation, or vice versa. Even though the purpose is apparently that of making sure that the listener identifies the intended referent correctly, the speaker does not seem to wait for the interlocutor to request more clarification, but rather assumes that the latter may or may not understand if only the Arabic or the French form were employed. Examples \xref{falchetta:ex:9} and \xref{falchetta:ex:10} are taken from two different sessions.

\begin{exe}
	\ex\label{falchetta:ex:9} \gls*{MA} (Temara corpus; GP-10) \\
	\gll ha:da   wa:ħǝd  {ǝ\dots} l-\textbf{maʒisiẽ} sa:ħi:r\\
	\DEM{} \INDF{} [hesitation] \DEF-\textbf{magician} magician \\
	\glt “This is a\dots \textbf{magician}, a magician” (< FR \textit{magicien} ‘magician’)
	\ex\label{falchetta:ex:10} \gls*{MA} (Temara corpus; DN-15) \\
	\gll ka:-tħǝjjǝd   d-dʒa:ki:tˁa   ka:-tħǝjjǝd   \textbf{la}   \textbf{vɛst} \\
	\PRVB-take.off;\Third\F\SG{} \DEF-jacket \PRVB-take.off;\Third\F\SG{} \textbf{\DET} \textbf{jacket} \\
	\glt “She takes off her jacket, she takes off her \textbf{jacket}” (< FR \textit{la veste} ‘the jacket’)
\end{exe}

\noindent
As the pranks took place in Canada, it may be observed that in these two cases, which resemble many others in the corpus, the bilingual denomination is used for referents that are potentially associated with a foreign culture, namely a particular type of entertainer and an item of female clothing. Nonetheless, an Arabic lexeme does exist for each of these entities and is employed by the interviewees themselves. Therefore, while the foreignness of the signified may have an influence on the hesitation between the two languages, the choice of employing two signifiers, each one drawn from a different language, has pragmatic implications that go beyond language-culture association, as will be discussed below. 

The second sociopragmatic function can be termed ‘validation through translation'. In this order of cases, one participant employs French to translate or reformulate a string of speech that the other has just expressed in Arabic. Here, the foreign language appears to help the code-switcher make sure he understood his interlocutor’s statement. Examples \xref{falchetta:ex:11} and \xref{falchetta:ex:12} are taken from two different sessions to illustrate this kind of use.

\begin{exe}
	\ex\label{falchetta:ex:11} \gls*{MA} (Temara corpus) \\
	\begin{xlist}
		\ex POP-1\footnotemark \\
		\gll da:ba wa:ħǝd smi:t-u {wa:ħǝd l-mǝtˁʕǝm} bħa:l l-mǝtˁʕǝm gǝlti bħa:l\\
		now one name-\Third\M\SG{} \INDF-restaurant like \DEF-restaurant say;\PFV.\Ssg{} like \\
		\glt “So it’s a\dots \ how’s it called\dots \ a restaurant, like a restaurant\dots \ you’d say\dots \ like\dots”
		\ex PT-7 \\
		\gll ʔa:h   \textbf{rɛstorã} \\
		yes \textbf{restaurant} \\
		\glt “Yeah, a \textbf{restaurant}!” (< FR \textit{restaurant} ‘restaurant’)
	\end{xlist}
\end{exe}
\begin{exe}
	\ex\label{falchetta:ex:12} \gls*{MA} (Temara corpus)
	\begin{xlist}
		\ex GP-10 \\
		\gll ha:da  wa:ħǝd   ka:mi:ju   dja:l   l-ħǝbs   fi:-h   ʒu:ʒ \\
		\DEM{} \INDF{} lorry \GEN{} \DEF-prison in-\Third\M\SG{} two \\
		\glt “This is a prison lorry, with two\dots”
		\ex LT-32 \\
		\gll dja:l \\
		\GEN{} \\
		\glt “A what\dots?”
		\ex GP \\
		\gll dja:l l-ħǝbs   lli   tǝ-jku:n   fi:-h   l-msǝʒʒǝn   hi:ja b-a:ʃ   tǝ-jħǝwwlu:-h   mǝn \\
		\GEN{} \DEF-prison \REL{} \PRVB-be.\Third\M\SG{} in-3\M\SG{} \DEF-imprisoned \Third\F\SG{} \PREP-\REL{} \PRVB-transfer;\IPFV.\Third\PL-\Third\M\SG{} \PREP \\
		\glt “A prison [lorry], the one that has inmates inside, that is, to move them from\dots”
		\ex LT \\
		\gll ʔa:h dja:l t-\textbf{trãspͻr} dja:l \textbf{le} \textbf{pʁizonje} \\
		yes \GEN{} \DEF-\textbf{transport} \GEN{} \textbf{\DEF.\PL} \textbf{prisoner} \\
		\glt “Oh yeah! For the \textbf{transport} of \textbf{prisoners}!” (< FR \textit{transport} ‘transport’ and \textit{les prisonniers} ‘the prisoners’)
		\newpage
		\ex GP \\
		\gll hu:wa   ha:da:k \\
		\Third\M\SG{} \DEM{} \\
		\glt “Exactly!”
		\ex LT \\
		\gll vwala \\
		\INTRJ \\
		\glt “There you go!”
	\end{xlist}
\end{exe}
\footnotetext{Letters within the example indicate different speech turns in a single exchange. Each turn is introduced by the interviewee’s code and his rate of \glspl*{EC} (the latter is only reported on the first turn).}


Interestingly, this use seems to imply that French can be more effective than Arabic in clarifying certain kinds of referents for some \gls*{MA} speakers in Morocco; this is particularly clear in Example \xref{falchetta:ex:12}, where LT does not seem to grasp GP’s tentative description of a prison lorry until he asks and obtains validation for his mixed French-\gls*{MA} reformulation of the description itself. The fact that the two examples both refer to concepts that were either introduced or renewed by the French colonizers during the protectorate may explain why their association with the European language appears to be strong for some participants. 

As seen in the examples above, in which the main communicative goal was conveying information in a clear and effective manner, \gls*{CS} can fulfil sociopragmatic functions that go beyond the signalling of one’s linguistic skills — at least, one would argue, among speakers with low proficiency in, and/or lower exposure to, French. In parallel, what these uses of \gls*{CS} reveal is that individuals with different social, educational, and attitudinal profiles differ in their abilities to produce not only French, but also \gls*{MA} forms for specific items. The last excerpt reported is particularly eloquent in this respect. Further proof of this comes from the variety of lexemes used by the test participants for a single referent. Since the same hidden-camera videos were shown in different sessions, it is possible to see how participants diverged in the forms they used to denominate the same object. This is especially observed in lexical choices concerning objects crucial to the prank described. More often than not, the sets of words thus obtained contain both French and \gls*{MA} or \gls*{SA} items. Two examples are reported in \tabref{falchetta:tab:4} and \tabref{falchetta:tab:5}.

\begin{table}
	\begin{tabularx}{\textwidth}{lXl}\midrule\toprule
		& \textbf{\textit{Noticeboard}} & \textbf{Language variety} \\\midrule
		LT-32 & /\textit{banjɛr}/ (< \textit{bannière} ‘banner’) & French \\
		IDG-10 & /wǝrqa/, /blˁa:ka/ & \gls*{MA}, \gls*{MA} \\
		FG-9 & /\textit{pano}/ (< \textit{panneau} ‘board’) & French \\
		GS-7 & /blˁa:ka/ & \gls*{MA} \\
		OL-2 & /blˁa:ka/ & \gls*{MA} \\ \bottomrule \midrule
	\end{tabularx}
	\caption{Different denominations for a noticeboard.}
	\label{falchetta:tab:4}
\end{table}

\begin{table}
	\begin{tabularx}{\textwidth}{lXl}\midrule\toprule
		& \textbf{\textit{Waiting room}} & \textbf{Language variety} \\\midrule
		EMK-26 & /\textit{sal datãt}/ (< \textit{salle d’attente} ‘waiting room’) & French \\
		DN-15 & /qa:ʕat ʔintidˁa:r/ & \gls*{SA} \\
		DS-9 & /\textit{sal datãt}/ & French \\
		FG-9 & /qa:ʕa d-ʔi:nti:dˁa:r/ & \gls*{SA}/\gls*{MA} \\
		GM-5 & /\textit{sal datãt}/ & French \\
		EY-4 & /sˁa:la dja:l l-ʔi:nti:dˁa:r/ & \gls*{SA}/\gls*{MA} \\ \bottomrule\midrule
	\end{tabularx}
	\caption{Different denominations for a doctor’s waiting room.}
	\label{falchetta:tab:5}
\end{table}


The choice of whether to designate the same entity with an \gls*{MA} word (which could also be a relatively ancient borrowing, such as /blˁa:ka/ in \tabref{falchetta:tab:4}, which is presumably an old adaptation either from the Spanish \textit{placa} or the French \textit{plaque,} both meaning ``plaque'') or through \gls*{CS} could be due to the expectation of which form the interlocutor would understand most quickly, or simply to the chosen lexeme being the most familiar and therefore the most immediate one for the speaker. Whatever the reason may be, it is clear that different people are used to sourcing words from different varieties (French, \gls*{MA}, or \gls*{SA}) to denominate the same referent; additionally, as seen above, they are often aware that other individuals may not easily understand the lexeme with which they are most familiar. It is argued here that such divergence in the speakers’ lexical repertoires needs to be accounted for on the basis of the educational and social background of their language development.

\largerpage
The uses of \gls*{CS} illustrated so far appear to be motivated by a need to recur to French material for the sake of mutual understanding. For this reason, they cannot be motivated by the social status that French confers on an individual’s speech. That said, other instances of \gls*{CS} have been observed in the corpus in which the connotations of French must have played a role in the speaker’s linguistic choices. Nevertheless, what is advocated in the following paragraphs is the need to go beyond mere notions of ‘prestige' and ‘social status' when discussing the reasons that underlie the use of French for stylistic purposes.

\subsection{Stylistic adjustment through \gls*{CS}}
When \gls*{CS} is not used to clarify referential meaning, it serves the purpose of exploiting the indexical field\footnote{Peirce defines an index as a sign that “signifies its object solely by virtue of being really connected with it”, and not by virtue of laws (like linguistic signs) and visual resemblance (like icons; \citealt[361]{peirce_algebra_1933}). I use the concepts of ``indexicality'' and ``indexical field'', based on \citegen{peirce_algebra_1933} ``index'', following their development by \citet{silverstein_shifters_1976,silverstein_indexical_2003} and \citet{eckert_variation_2008}.} associated with the French form in order to produce certain communicative effects. The sociopragmatic functions that are fulfilled in this group of cases will be grouped under the umbrella term ‘stylistic adjustment' for the time being. Previous literature mainly restricted its view to these types of uses, attributing them to a generic indexation of prestige or young urban culture associated with the use of French, or French-\gls*{MA} \gls*{CS}. What is argued here is that a more in-depth look at the meaning of these code-mixing practices allows us to take into account how the indexicalities of the French forms employed originate from their juxtaposition with entities (people, objects, attitudes, settings, etc.) which are involved in the speakers’ familiarization with, exposure to, and/or use of such forms. A reading through the concept of indexicalities may also clarify what leads speakers to consider French forms as more prestigious, or what social values make them more appealing to young urban bilinguals, thus elucidating previous findings. The limited amount of data analyzed here does not make it possible to generalize as to the observations made in the remainder of this section; however, the few examples taken from the corpus provide a glimpse of the variety of indexical meanings that can underlie French-\gls*{MA} \gls*{CS} in the community studied and in the Moroccan context in general.

\begin{exe}
	\ex\label{falchetta:ex:13} \gls*{MA} (Temara corpus; GP-10) \\
	\gll tǝ-jt\textbf{kaʃʃ}a   mu:rˁa:-h \\
	\PRVB-\textbf{hide}.\Third\M\SG{} behind-\Third\M\SG{} \\
	\glt “He \textbf{hides} behind it” (< FR \textit{cacher} ‘to hide’)\footnotemark
	\ex\label{falchetta:ex:14} \gls*{MA} (Temara corpus; EMK-26) \\
	\gll \textbf{ã} \textbf{fẽ} \textbf{d-kõt} ka:-jʒi   wa:ħǝd-rˁ-rˁa:ʒǝl \\
	\textbf{\PREP} \textbf{end} \textbf{of-count} \PRVB-come.\Third\M\SG{} \INDF-man \\
	\glt “\textbf{In the end}, a man comes \dots” (< FR \textit{en fin de compte} ‘ultimately’)
\end{exe}
\footnotetext{While the original French verb only bears a transitive meaning, the fused \gls*{MA}-French verb used by GP is rendered intransitive by the \gls*{MA} infix /-t-/, which has a passivizing or reflexivizing grammatical function.}


Neither of the switches occurring in Examples \xref{falchetta:ex:13} and  \xref{falchetta:ex:14} stems from a problem of mutual understanding, as they both occur during the summary of one of the pranks that the speaker has viewed (i.e., neither GP nor EMK was seeking clarification from the other participant). A prestige-based interpretation would then claim that both consultants were \textit{elevating} the style of speech. However, if the intended meaning of ‘elevating' is ‘making appropriate to the communicative context', then it seems equivalent to say that GP and EMK code-switched in order to \textit{adjust} their style (hence the definition of this pragmatic function). 

A reading from the perspective of indexicalities leads to the search for associations which the speaker makes with the use of the French forms involved in the switch. It was suggested that this search could be conducted by looking at the context of learning or habitual use of such forms, or exposure to them. This is admittedly complicated, as each single form might demand a dedicated enquiry. A possible starting point is the formal aspect of the switched constituent; while \textit{/ã fẽ dkõt/} (Example  \ref{falchetta:ex:14}) is an adverbial locution pronounced according to standard rules, the switched item in /tǝ-jt\textit{kaʃʃ}a/ (Example \ref{falchetta:ex:13}) is a single lexical morpheme that has been adapted to local morphology and phonotactics.\footnote{See the gemination of /ʃ/, which is non-geminated in the standard French verb [kaʃe].} This suggests that EMK may have taken its locution from native or prescriptive (e.g., classroom) French speech, while GP may have acquired its mixed-form verb from interactions with Moroccan peers (or it may be an original form he created by following similar models of mixed-form verbs). However, the use of a switched form that is not fully adapted to \gls*{MA} phonetics does not necessarily stem from cultivated contexts of acquisition (see the switch in Example \ref{falchetta:ex:15}). 

\begin{exe}
	\ex\label{falchetta:ex:15} 
	\gll \textbf{dirɛktǝmã}   hi:ja   ka:-ddu:rˁ   l   l-kbi:r \\
	\textbf{directly} \Third\F\SG{} \PRVB-turn;\Third\F\SG{} to \DEF-big \\
	\glt “She \textbf{immediately} turns towards the grown-up man” (< FR \textit{directement} ‘directly’)
\end{exe}


\largerpage
Unlike EMK’s adverbial locution in Example \xref{falchetta:ex:14}, the French-derived adverb employed by DC in Example \xref{falchetta:ex:15} is not located at the head of an utterance in standard French. However, its \gls*{MA} counterpart /ni:ʃa:n/ does admit this syntactic collocation. This suggests that DC’s use of /\textit{dirɛktǝmã}/ derives from a direct translation of the \gls*{MA} adverb into French, and that, considering its collocation, this use may not have been picked up from proficient speakers of French~— a possible indication that this, too, is a kind of use acquired from peers or youth speech. 

Besides the context of acquisition or use, another important source of information is speech by the same interviewee outside the interview — which is unfortunately only available in a few cases. The consultant whose speech is reported in Example \xref{falchetta:ex:2} is one such case. During a previous field study in the same town, GG had also taken part in a group interview. In the course of both the interview and the test, he was often observed to adjust his own speech with respect to his usual, off-recorder way of speaking. This adjustment consisted of a more frequent engagement in both \gls*{MA}-French and \gls*{MA}-\gls*{SA} \gls*{CS}. By doing so, he was adopting a style which he obviously thought more appropriate for the research context. The overall view of GG’s linguistic choices thus sheds more light on his use of /ka:-t\textit{ʔatir}i/ in the utterance quoted in Example \xref{falchetta:ex:2}; this occurred in his description of the sexually appealing attire of an actress, whose way of dressing was functional to the prank he was summarizing. The use of an ‘adjusted' French form at this point of the narration may be designed to diminish the potential vulgarity of the situation described. The same interviewee also resorted to French in another, similarly problematic context.

\begin{exe}
	\ex\label{falchetta:ex:16} \gls*{MA} (Temara corpus; GG-11) \\
	\gll mankẽ da:jri:n li:-h l-qa:dˁi:b dja:l-u   l-\textbf{penis}  dja:l-u\\
	dummy put;\PTCP.\ACT.\PL{} to-\Third\M\SG{} \DEF-penis \GEN-\Third\M\SG{} \DEF-\textbf{penis} \GEN-\Third\M\SG{} \\
	\glt “A \textit{dummy} to which they added a penis, a \textbf{penis}” (< FR \textit{pénis} ‘penis’)
\end{exe}


In Example \xref{falchetta:ex:16}, too, while reporting on a different prank, GG chose to code-switch to French (after using the \gls*{SA} word for the same referent) as he needed to refer to the male organ. The association between French and school, a context in which politeness is presumably given particular value, may underlie the preference for using this language to express sensitive content in the two examples taken from GG’s test session; in  \xref{falchetta:ex:16}, another language linked to schooling (\gls*{SA}) is even resorted to in addition to French. Therefore, the general tone adopted by GG throughout his contributions to the author’s research helps frame his pragmatic uses of \gls*{CS} in Examples \xref{falchetta:ex:2} and \xref{falchetta:ex:16} as attempts to adjust to the communicative situation by building on the indexical association between French, its learning environment, and the social attitudes promoted within such an environment. Ultimately, it is this association that makes French a safe haven when one has to express contextually sensitive information.\footnote{\citet[236]{caubet_jeux_2002} also identifies the use of a French word in \gls*{CS} with Algerian Arabic as sounding “neutral and more scientific or technical”. \gls*{CS} used to express taboo words has also been found in \gls*{MA}-French \gls*{CS} in song lyrics \citep[200--202]{bentahila_language_2002} and even, occasionally, in \gls*{MA}-\gls*{SA} \gls*{CS} in the \gls*{MA} dubbing of soap-operas \citep[233--236]{ziamari_ana_2013}.}

\section{Discussion and provisional conclusions}
The uses of \gls*{MA}-French \gls*{CS} illustrated in the first part of the analysis add a new perspective on this practice, which is not only used to signal one’s linguistic skills, or as a group marker, but can also be, and often is, a means to improve the flow of information or to negotiate the meaning of what is being communicated. This aspect is not likely to be adequately appreciated if the focus of the analysis is limited to the structural features of the switched constituents. More importantly, it has been shown that cases in which this practice facilitates the conveyance of information reveal that it can also be a collaborative enterprise, whereby supposed gaps in the interlocutors’ linguistic repertoires are filled by referring to the same entity or concept in two different language varieties at the same time. This communicative strategy also has the effect of (partially) levelling inequalities in the speakers’ exposure to French, since those who are less familiar with the language have the chance to learn new words by hearing them associated with their translation in Arabic. 

Nevertheless, the fact that some speakers are more at ease when a certain entity or concept is referred to in French, even though the \gls*{MA} form is more widespread in the community (i.e., when the French form is, apparently, ‘unnecessary'), also reveals that they have contrasting repertoires. It is easy to connect this linguistic inequality to the similarly unequal access to adequate French learning in Moroccan society, and different degrees of exposure to French inevitably entail different levels of proficiency in or familiarity with this language. In this sense, \gls*{CS} is also undoubtedly a social discriminant, insomuch as it depends on the ability that the speaker has to engage in it, with obvious consequences for the presence and frequency of \glspl*{EC} in their speech. However, previous studies (especially \citealt{post_impact_2015}) have also pointed out how personal attitudes towards the linguistic varieties at stake may heavily affect the individual’s inclination to mix their native variety with foreign forms; in the Moroccan context, this is all the more relevant, since French is still seen today as a language through which values alien to the local society are being imported. This means that some speakers may choose not to make use of this distinctive linguistic skill even if they are able to. Another problematic point is the extent to which habitual \gls*{MA}-French code-switchers choose to accommodate to non-\gls*{CS} speakers. To clarify this, \gls*{CS} frequencies will have to be compared in the same interviewee’s speech in contexts other than the hidden-camera test (i.e., group interviews and spontaneous conversations), for those speakers for whom these data are available. 

Concerning the cases in which \gls*{CS} serves the purpose of adjusting speech style, these suggest that practices which could all be included under the term ‘prestige' can actually be based on different indexical meanings attributed to the same language. Even though additional data are needed to define more precisely what kind of associations are made by the speakers in reference to these forms, the mere existence of different contexts of learning implies that indexicalities are potentially quite divergent; while a French form learnt in the classroom may index politeness and cultural elitism, another French form learnt from a peer group may be associated with ‘youth' or ‘street' language, ‘thug' registers and so on. Of course, hypotheses on contexts of use/exposition need more data to be supported and/or nuanced. Extending the sample of participants (including to women), making it more equally representative of speakers who have been schooled in Arabic or French, or asking the participants themselves for their feedback on the matter are all steps that may help in this endeavour.

The analysis of this small sample of interviewees has shed light on several elements that further research on \gls*{CS} (involving this and other pairs of varieties) should take into account, especially in contexts in which bilingualism is due to an ex-colonial, non-native language constituting a necessary tool for seeking employment and/or social upgrading. In these contexts, collaborative \gls*{CS} may be an option for speakers, and can be identified by analyzing what happens in interactions between speakers who usually code-switch with different frequencies, or have different proficiency levels in their non-native language. Besides, carrying out proficiency tests while at the same time collecting data on language use may help determine whether contact between proficient \gls*{CS} and non-proficient speakers contributes to the spread of linguistic knowledge. As for non-collaborative, stylistic \gls*{CS}, an analysis that rests on the concept of indexicalities makes it possible to move beyond the dichotomous interpretations of prestigious versus stigmatized speech, by revealing the multiple meanings that the embedding of one language into another may bear for bilinguals.  All these efforts require extending the analysis to contexts of acquisition as well as use and exposure to the foreign forms, in order to get a fuller view of the sociolinguistic life of the speaker, and clarify why ‘unnecessary' use of foreign forms under a purely linguistic perspective is actually necessary according to the individuals’ social and pragmatic point of view.



\section*{Abbreviations}
\begin{tabularx}{.45\textwidth}{>{\scshape}lQ}
1, 2, 3 & 1\textsuperscript{st}, 2\textsuperscript{nd}, 3\textsuperscript{rd person}\\
act & active voice\\
arg &argument-introducing particle\\
conj & conjunction\\
cs & code-switching\\
def & definite\\
dem & demonstrative\\
ec & Embedded Constituent\\
f & feminine\\
fr & French\\
gen & genitive\\
indf & indefinite\\
\end{tabularx}
\begin{tabularx}{.45\textwidth}{>{\scshape}lQ}
intrj & interjection\\
ipfv & imperfective\\
m & masculine\\
ma & Moroccan Arabic\\
pfv & perfective\\
pl & plural\\
prep & preposition\\
prvb & preverb\\
ptpc & participle\\
rel & relative particle\\
sg & singular\\
sa & Standard Arabic\\
\\
\end{tabularx}




\printbibliography[heading=subbibliography, notkeyword=this]

\end{document}
