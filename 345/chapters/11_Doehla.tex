\documentclass[output=paper]{langscibook}
\ChapterDOI{10.5281/zenodo.10497391}

\author{Hans-Jörg Döhla \orcid{0000-0003-0047-3492} \affiliation{University of Tübingen}}


\title[Imitating the Arabic model]{Imitating the Arabic model: The case of valency-increasing operations in the Old Spanish translation of the Arabic \textit{Kalīla wa-Dimna}}

\abstract{
    Learned language contact, explicitly the translation of Arabic works into Old Spanish, offers quite a range of possibilities for the detailed study of morphosyntactical and semantic structures in the target language of Old Spanish. This chapter investigates causative constructions in Old Spanish using the medieval translation (ca. 1251 AD) of the Arabic \textit{Kalīla wa-Dimna} as its primary source. In this context, three main formal strategies can be identified: (1) zero morphology or conversion; (2) denominal derivations of the patterns \textit{a-N-ar} and \textit{en-N-ar}; and (3) analytic constructions by means of the causative verbs \textit{fazer} ‘to make’, \textit{mandar} ‘to order’, and \textit{enbiar} ‘to send’. In the case of (1), it can be shown that the valency increase of Old Spanish intransitive verbs is motivated by the Arabic model that displays a stem II or IV verb that has factitive and causative meaning. As far as formal strategy (2) is concerned, the motivation behind its augmented use in Old Spanish can be traced not only to the Arabic semantic model, but also to the formal similarity of the \textit{a}- prefix, found both in Arabic and Old Spanish. Finally, strategy (3) also follows the analytical constructions found in the Arabic model, especially regarding the original semantics of the auxiliary causative verbs. The analysis of the Old Spanish causative constructions confirms that translation is an act of negotiation and compromise and is influenced deliberately where the given repertoire of the target language exhibits a general predisposition that permits such contact-induced structures.
}

\IfFileExists{../localcommands.tex}{
  \addbibresource{../localbibliography.bib}
  % add all extra packages you need to load to this file

\usepackage{tabularx,multicol}
\usepackage{url}
\urlstyle{same}

\usepackage{listings}
\lstset{basicstyle=\ttfamily,tabsize=2,breaklines=true}

\usepackage{langsci-basic}
\usepackage{langsci-optional}
\usepackage{langsci-lgr}
\usepackage{langsci-osl}
% \usepackage{./langsci/styles/langsci-lgr}
% \usepackage{./langsci/styles/langsci-osl}
% \usepackage{langsci-gb4e}

\usepackage{tikz}
\usetikzlibrary{patterns,calc}
\pgfdeclarepatternformonly{south east lines}{\pgfqpoint{-0pt}{-0pt}}{\pgfqpoint{3pt}{3pt}}{\pgfqpoint{3pt}{3pt}}{
    \pgfsetlinewidth{0.6pt}
    \pgfpathmoveto{\pgfqpoint{0pt}{3pt}}
    \pgfpathlineto{\pgfqpoint{3pt}{0pt}}
    \pgfpathmoveto{\pgfqpoint{.2pt}{-.2pt}}
    \pgfpathlineto{\pgfqpoint{-.2pt}{.2pt}}
    \pgfpathmoveto{\pgfqpoint{3.2pt}{2.8pt}}
    \pgfpathlineto{\pgfqpoint{2.8pt}{3.2pt}}
    \pgfusepath{stroke}}
    
\usepackage{stmaryrd}
\usepackage{wasysym}
\usepackage{multirow}
\usepackage{caption}
\usepackage{subcaption}
\usepackage{mathrsfs}
\usepackage{qtree}

\usepackage{linguex}


  %pminos do not split footnotes
% \interfootnotelinepenalty=10000 %Footnote in Laporte chapters has to be split SN


%\DeclareIndexNameFormat{default}{%
%\nameparts{#1}%
%\usebibmacro{index:name}%
%{\index[names]}%
%{\namepartfamily}%
%{\namepartgiveni}%
% {}% L1
% {}% L2
%{\namepartprefix}% generates spurious space L3
%{\namepartsuffix}% generates spurious space L4
%}

%  {\DeclareIndexNameFormat{default}{%
%     \usebibmacro{index:name}{\index[names]}{#1}{#3}{#5}{#7}}}

%\DeclareIndexNameFormat{default}{%
%  \usebibmacro{index:name}{\sindex[nom]}{#1}{#3}{#5}{#7}}

%\DeclareIndexNameFormat{default}{%
%  \usebibmacro{index:name}{\sindex[person]}{#1}{#3}{#5}{#7}}
%\DeclareIndexNameFormat{default}{%
%\nameparts{#1} \usebibmacro{index:name}{\sindex[person]]}{\namepartfamily}{‌​\namepartgiven}{\nam‌​epartprefix}{\namepa‌​rtsuffix}}

%\newcommand{\smiley}{:)}

%\renewbibmacro*{index:name}[5]{%
%\usebibmacro{index:entry}{#1}%
%{\iffieldundef{usera}{}{\thefield{usera}\actualoperator}\mkbibindexname{#2}{#3}{#4}{#5}}}

% \newcommand{\noop}[1]{}

%remove for final
%\overfullrule=1mm

\newcommand{\tobi}[2]}}
\renewcommand{\S}[1]{\tobi{#1}{\textsc{*}}}

% this volume references
% puts: [this volume]
% already defined: \citetv
%\newcommand{\citepv}[1]{(\citeauthor{#1} \citeyear*{#1} [this volume])}
\newcommand{\citealtv}[1]{\citeauthor{#1} \citeyear*{#1} [this volume]}

%parentheses around example number
\newcommand{\pref}[1]{(\ref{#1})}

% in-text examples

\newcommand{\lnex}[1]{\textit{#1}} %target lang word
\newcommand{\lnlit}[1]{(lit.: `#1')} %literal reading
\newcommand{\lnlat}[1]{(#1)} % latinization
\newcommand{\lntrans}[1]{`#1'} %translation
\newcommand{\lnexl}[2]%
{\lnex{#1}{} \lnlat{#2}} % ex with latinization
\newcommand{\lnexlat}[3]{\lnex{#1}{} \lnlat{#2}{} \lntrans{#3}} % ex with latinization and tranl.

%ch01
\newcommand{\co}[1]{\mbox{\textbf{#1}}}

%ch09

\newcommand{\cyrbulg}[1]{\begin{otherlanguage*}{bulgarian}#1\end{otherlanguage*}}


%ch10
\newcommand{\nlp}{{\small NLP}}
\newcommand{\mwe}{{\small MWE}}
\newcommand{\rae}{{\small RAE}}
\newcommand{\lvc}{{\small LVC}}
\newcommand{\pos}{{\small P}o{\small S}}
%\newcommand{\todo}[1]{ \textcolor{red}{#1} }

%\renewcommand{\labelenumi}{\theenumi}
%\ainamefmt{{vv}{ll}{, ff}{, jj}} % fullname

\newcommand{\biberror}[1]{{\color{red}#1}}

\newcommand{\osenovaitem}{--~} 
  %% hyphenation points for line breaks
%% Normally, automatic hyphenation in LaTeX is very good
%% If a word is mis-hyphenated, add it to this file
%%
%% add information to TeX file before \begin{document} with:
%% %% hyphenation points for line breaks
%% Normally, automatic hyphenation in LaTeX is very good
%% If a word is mis-hyphenated, add it to this file
%%
%% add information to TeX file before \begin{document} with:
%% %% hyphenation points for line breaks
%% Normally, automatic hyphenation in LaTeX is very good
%% If a word is mis-hyphenated, add it to this file
%%
%% add information to TeX file before \begin{document} with:
%% \include{localhyphenation}
\hyphenation{
    Beck-man
    Ngu-yen
    back-chan-nel
    back-chan-nels
    mo-not-o-nous
    ste-reo-typ-i-cal
}

\hyphenation{
    Beck-man
    Ngu-yen
    back-chan-nel
    back-chan-nels
    mo-not-o-nous
    ste-reo-typ-i-cal
}

\hyphenation{
    Beck-man
    Ngu-yen
    back-chan-nel
    back-chan-nels
    mo-not-o-nous
    ste-reo-typ-i-cal
}
 
  \togglepaper[11]%%chapternumber
}{}

%\makeglossaries
\newleipzig{JUS}{jus}{jussive}
\newleipzig{ACT}{act}{active}
\newleipzig{PN}{pn}{proper name}
\newleipzig{CLAUSE}{clause}{??} %TODO: Weder im Original noch im Paket definiert
\newleipzig{PREP}{prep}{preposition}


\begin{document}
\maketitle

\section{Introduction}\label{doehla:sec:1}

The main focus of this chapter is the contrastive historical analysis of the Old Spanish translation, \textit{Calila e Dimna}, and its famous and widespread Arabic model, \textit{Kalīla wa-Dimna} (see Section \ref{dohla:sec:2}), with particular attention paid to Old Spanish factitive and causative constructions (see Section \ref{dohla:sec:5}). In this study, translation as such is regarded as a particular form of language contact, where structures and semantic concepts may be imitated in the target language by the translators following patterns found in the source model \citep{dohla_traduccion_2008-1,hasler_ubersetzung_2001}.\footnote{Translations are popular windows for matter borrowings, that is the adoption of entire constructions of the source language, including the phonological shape and its corresponding meaning \citep[15]{sakel_types_2007}, to enter the target language. There are also Arabisms found in \textit{Calila e Dimna} \citep{dohla_libro_2009}, although these are few in number (44) considering the overall amount of 2,535 lexemes (types).} At the same time, it can be observed that structures of the target language text diverge from those of the model language in order to comply with the linguistically acceptable repertoire of the target language, especially if the target language, like written Old Spanish, is still a young language lacking certain syntactic and semantic fine-tuning \citep{bossong_probleme_1979,galmes_de_fuentes_influencia_1996,huffman_syntactical_1973}. As far as the Arabic–Old Spanish translations of the thirteenth century are concerned, the studies by \citet{bossong_probleme_1979,bossong_creatividad_2008} and \citet{dohla_traduccion_2008-1,dohla_libro_2009} demonstrate how thoroughly the translators operated while translating the bulk of texts from Arabic, predominantly using structures from their available Romance repertoire and modifying them creatively by derivational and analytic strategies. 

The main research question of this study concerns the morphological strategies used by the translators in order to create factitive and causative meanings and the possible influence of the formal side of the derivational and analytic constructions exhibited in the Arabic model.
The study itself is intended to be purely descriptive and non-statistical (except for \autoref{doehla:tab:4}, taken from \citealt{sanaphre_villanueva_analytic_2010}), since only the full corpus of translated texts of the thirteenth century (see \citealt{faulhaber_semitica_2004}) would fully justify statistical analyses. Unfortunately, the full corpus, Old Spanish and Arabic, has not yet been edited for large-scale comparative studies. 

Before getting to the descriptive analyses in Section \ref{dohla:sec:5}, Sections \ref{dohla:sec:3} and \ref{dohla:sec:4} are dedicated to the differences between Arabic and (Old) Spanish morphology and to general aspects of factitive and causative constructions.

\section{Contact between Spanish and Arabic and \textit{Calila e Dimna}}\label{dohla:sec:2}

When compared to other areas where Romance languages are spoken, the Iberian Peninsula of the Middle Ages stands out due to the coexistence of two different faith systems (Christianity and Islam), several different realms (the northern Christian shires and emerging kingdoms versus the Muslim sphere of control in the southern regions), and, most importantly, two different language families (Indo-European and Semitic). As can be imagined and as is well known, out of this coexistence a number of language-contact scenarios emerged. In the first place, the Christians living under Muslim rule have to be mentioned. According to Latin sources of the Christian North, they called themselves \textit{Mozarabs},\footnote{In fact, the term \textit{Muzaraues} \citep[497]{perez_lexicon_2010} can be found in a Latin document from the northern region of León as far back as 1024. However, the adoption of Arabisms in the Latin of the Christians in the north can be dated back to the ninth century, with \textit{citara} ‘curtain’ (812), \textit{barrius} ‘quarter’ (887), \textit{adorra} ‘kind of shirt’ (887), \textit{zuramen} ‘Moorish tunic’ (887), and \textit{adria} ‘special kind of share’ (896) \citep[172, 95, 13, 805, 815]{perez_lexicon_2010}. This early appearance of Arabisms in Latin documents of the northern Christian realms can be linked sociohistorically to the migration of Mozarabs from the south to the northern no-man’s-land, which was repopulated actively by the Christians. On the other hand, even though the term \textit{Mozarab} comes from Arabic, there are no traces of the use of the Arabic term among Muslim writers to refer to Christians in al-Andalus \citep[VIII]{simonet_historia_1983}.} which points to the fact that they were ‘Arabicized’ (not ‘Islamized’), since the term derives from the Classical Arabic \textit{mustaʕrab} ‘Arabicized’ \citep[383]{corriente_dictionary_2008-1}. Besides this inclination toward the Arabic language \citep[69]{bossong_maurische_2007}, their inherited language was Ibero-Romance, which was also used by Muslims and Jews in al-Andalus.\footnote{This Ibero-Romance language has been called \textit{mozárabe/Mozarabic} by some scholars (\citealt{beale-rivaya_at_2016,galmes_de_fuentes_dialectologimozarabe_1983}, \citealt[XVII]{hitchcock_mozarabs_2008}) and, more appropriately, \textit{romandalusí} by \citet{corriente_romania_2008}. The term proposed by Corriente suggests a Romance origin of this language used not only by the Christian, i.e., Mozarabic, population of al-Andalus but by all other groups, too \citep[98]{corriente_romania_2008}.} Apart from a high number of early Arabisms, especially the bilingual \textit{ḫaraǧāt} (Spanish \textit{jarchas}) \citep{corriente_poesidialectal_1997,corriente_romania_2008}, the final verses of the Andalusi poem pattern called \textit{muwaššaḥ}, which bear witness to the creative and aesthetic use of Arabic-Romandalusi (see Note 3) code-switching, must be mentioned. While this kind of language contact took place on a popular, albeit court, level in the case of the \textit{ḫaraǧāt}, the thirteenth century marked a turning point in language use, when the most powerful of the advancing Christian kingdoms, Castile, promoted its vernacular Castilian language as the official language of the court. Thus, the first document written in Old Spanish is the \textit{Tratado de Cabreros} of 1206 \citep{wright_tratado_2000}, a treaty between Alfonso VIII of Castile and Alfonso IX of León, 24 years before the unification of Castile and León under Fernando III in 1230.
 This first use of Old Castilian as the official language of the court paved the way for an early elaboration process \citep[273]{haugen_implementation_1983},\footnote{In the terminology of \citet[273]{haugen_implementation_1983}, “elaboration” is “the continued implementation of a norm to meet the function of a modern world. […] A modern language of high culture needs a terminology for all the intellectual and humanistic disciplines, including the cultural underworld that runs from low to popular”. The norm-driven implementation of Castilian in Spain was not followed up until the eighteenth century.} even before the first vernacular grammar was written, which was published by Antonio de Nebrija in 1492. The elaboration process was twofold; one originated from the desire to express mostly Christian and Classical topics in the vernacular in prose and rhyme, while another made Old Castilian the target language of the translation of scientific and wisdom literature from Arabic. These Arabic documents had been falling into the hands of the Christians, advancing their \textit{reconquista} from the northern parts of the peninsula toward the south, from the eleventh century onward, thus already leading to Arabic–Latin translations in Toledo in the twelfth century. However, as far as the patron of the translations of the thirteenth century, Alfonso X the Wise, is concerned, a clear preference for the vernacular can be detected, as can be read in the foreword of the \textit{Lapidario} (1245), a treatise on the magic properties of (gem) stones.

\begin{quote}
    [E]t dizien le Yhuda Mosca el Menor, que era mucho entendudo en la arte de astronomia et sabie et entendie bien el arauigo et el latin. Et de que por este iudio, su fisico, ouo entendido el bien et la grand pro que en el iazie, mando gelo trasladar de arauigo en lenguaie castellano por que los omnes lo entendiesses meior. \citep[19]{rodriguez_m_montalvo_lapidario_1981} 


    [A]nd he was called Yehuda Mosca Junior, who was very adept in the art of astronomy and he knew and understood Arabic and Latin well. And since, with the help of this Jew, he [the infant Alfonso] had understood the greatness and the great advantage which was lying in it [the Lapidario], he ordered him to translate it from Arabic into the Castilian language, so that people would understand it better. (Own translation)
\end{quote}


Thus, under the patronage of Alfonso X the Wise (approximately 1250‒1282), there was a wave of translations\footnote{In general, the translation process was usually undertaken by a team of translators, with one translator reading the text aloud or translating it into Andalusi Arabic. Then, another member of the team would translate the text heard into Old Castilian, which was then written down (\citealt[3]{hilty_libro_1954}, \citealt[XXIX]{hilty_aly_2005}, \citealt[19]{rodriguez_m_montalvo_lapidario_1981}).} into Old Castilian, comprising many topics, such as astronomy/astrology, zoology, veterinary medicine, agriculture, mineralogy, and oriental wisdom. This special type of language contact \citep{dohla_traduccion_2008-1,hasler_ubersetzung_2001} initiated a process of acculturation which, according to \citet[6]{bossong_probleme_1979}, is the “Prozeß der Universalisierung und/oder Komplektisierung einer R[esponse]-Sprache unter dem Einfluß und durch Anregung einer S[timulus]-Sprache” ‘the process of universalization and/or complectization of the response language under the influence of a stimulus language’ (own translation). In this context, universalization refers to the elaboration of the lexicon, whereas complectization alludes to the development of syntactic expressivity. Vestiges of both processes are clearly detectable in the Arabic–Old Castilian translations of the thirteenth century. However, even though there are studies regarding the lexicon \citep{bossong_canones_1978,bossong_probleme_1979,dohla_libro_2009} and contact-induced syntactic structures (in chronological order: \cite{tallgren_acerca_1934,dietrich_beitrage_1937,hottinger_kalila_1958,huffman_syntactical_1973}), this whole area of investigation still displays several desiderata, which can best be summed up in a general comparative grammar of Arabic and Old Spanish based on the translations of the thirteenth century. 

The linguistic material discussed in the following sections is taken from the Alfonsine Old Castilian version of the famous Arabic mirror of princes called \textit{Kalīla wa-Dimna}. Its origin lies in the Old Indian \textit{Pañcatantra} ‘Five stories’, a collection of fables and allegories compiled sometime between the third and fourth centuries. It was translated into Middle Persian and expanded at the same time in the sixth century, and from there it was rendered into Arabic in 750 AD by Ibn al-Muqaffaʕ. This Arabic translation has been transmitted in many copies, which can be subclassified into different lines of manuscript traditions, one of them being the Iberian branch with a particular order and number of chapters. Without going into further detail, at this point, it is sufficient to note that the translation, which was presumably ordered by the infant Alfonso in 1251 (Alfonso the X from 1252 onward), has survived to the present day in two manuscripts. One of them (manuscript A; kept at the library of the Real Monasterio de San Lorenzo de El Escorial, sig. h-III-9), dates from the first third of the fifteenth century, and quite accurately reflects the model of Arabic manuscripts that are considered part of the Iberian branch.\footnote{For example ms. Arabe 3478, supplément 1795, Bibliothèque Nationale, Paris, cited here as \citet{ibn_al-muqaffa_kitab_nodate}; ms. 4095/3, Süleymaniye Kütüphanesi, Istanbul, cited here in the edition of \citet{azzam_kitab_1941}.} The other (manuscript B; kept at the library of the Real Monasterio de San Lorenzo de El Escorial, sig. x-III-4), dates from 1467 and exhibits influences from other Arabic manuscript traditions and, possibly, from the Hebrew translation of \textit{Kalīla wa-Dimna} elaborated by Rabbi Joël in Spain in the twelfth and thirteenth centuries as well \citep[57, 78]{dohla_libro_2009}. This means that manuscript B, despite its similarities with manuscript A, is not a mere copy of manuscript A. As far as the examples of the Old Castilian \textit{Calila e Dina} are concerned, they are taken from \citet{dohla_libro_2009}. The Arabic model sentences are taken from those manuscripts specified in Note 6.

\section{Arabic and Spanish contrastive morphology}\label{dohla:sec:3}

Before presenting a brief typology of formal valency-increasing strategies as well as addressing the concrete valency-increasing operations found in the Old Spanish version of \textit{Kalīla wa-Dimna} in more detail and contrasting them with the respective source constructions of the Arabic model, it seems reasonable to dedicate a brief paragraph to the differences between Arabic and (Old) Spanish concerning their basic morphological strategies. 

In general, the main difference between Arabic, a Semitic language of the Afroasiatic language family, and (Old) Spanish, a Romance language of the Indo-European language family, lies in the use of transfixation as the major derivational device in Arabic. In this context, “[a] \textbf{transfix} may be defined as a discontinuous affix that disrupts the base to which it is attached” \citep[552]{broselow_56_2000}. Other authors refer to this morphological type as “templatic morphology” \citep[81-82]{lieber_introducing_2009}, “root and pattern morphology”\footnote{\citet{broselow_56_2000} distinguishes two general patterns of transfixation; one that she calls “segmental transfixation”, where the opposition between two derivations lies in the exchange of vowels, such as in the Arabic \textit{ḥazina} ‘to be sad’ versus \textit{ḥazana} ‘to make sad’, and a second that corresponds to the root and pattern morphology.} \citep[99]{holes_modern_2004}, or as Spanish “interdigitalización” ‘interdigitation’ \citep[21]{bossong_creatividad_2008}. 

In practice, as opposed to the exclusively linear, sequential, or concatenative morphology found in Romance languages, Arabic morphological derivation departs from a lexical root consisting of a fixed sequence of three consonants\footnote{\citet{wehr_arabisches_1952} lists 2,967 roots with three consonants and 362 with four consonants \citep[7]{drisner_arabic_2015}. The roots with three consonants also contain those roots where one consonant is represented by the glottal stop (\textit{hamza}) or by the half-consonants \textit{wāw} and \textit{yāʔ}, or where the second and the third consonants are identical, thus appearing as one geminated consonant. However, the same derivational principles are applied to these particular types of roots, even though they might experience some phonetical alternations and adaptions.} in most cases. From there, “Arabic is like a mathematical game. You take the root of a word, […], and you start playing” \citep[4]{drisner_arabic_2015}, creating different patterns according to fixed structures by adding the vowels /a, i, u/, and/or /aː, iː, uː/, by omitting a vowel between root consonants and/or by lengthening the middle consonant (see \autoref{doehla:tab:1}). At the same time, it should be mentioned that Arabic, just as Spanish does, makes use of other types of affixes, such as prefixes and suffixes, for inflectional and derivational morphology, which can also be combined with basic lexemes, that is with stems already modified by transfixation.
This way, verbal, nominal and adjectival derivations are created, all of them belonging more or less to the same semantic field. For example, the consonant sequence \textit{k-t-b} construes lexemes that have something to do with the domain of ‘writing’, thus making the nominal derivations displayed in  \autoref{doehla:tab:1} possible, among others.



\begin{table}
    \begin{tabularx}{\textwidth}{llQ}\midrule\toprule
        lexeme  & pattern    & meaning                                                             \\ \midrule
        \textit{kātib}   & \textit{C\textsubscript{1}āC\textsubscript{2}iC\textsubscript{3}} & ‘writer,   scribe, secretary’                                       \\
        \textit{kuttāb}  & \textit{C\textsubscript{1}uC\textsubscript{2}C\textsubscript{2}āC\textsubscript{3}} & ‘writer,   scribe, secretary (\Pl{})’ [but also \Sg{} ‘Koran school’] \\
        \textit{maktab}  & \textit{maC\textsubscript{1}C\textsubscript{2}aC\textsubscript{3}} & ‘office,   desk’ (\textit{nomen loci}, literally ‘place where you write’)    \\
        \textit{makātib} & \textit{maC\textsubscript{1}āC\textsubscript{2}iC\textsubscript{3}} & ‘office,   desk (\Pl{})’                                               \\
        \textit{kitāb}   & \textit{C\textsubscript{1}iC\textsubscript{2}āC\textsubscript{3}} & ‘book’                                                              \\
        \textit{kutub}   & \textit{C\textsubscript{1}uC\textsubscript{2}uC\textsubscript{3}} & ‘book (\Pl{})’ \\
        \textit{miktāb}  & \textit{mi-C\textsubscript{1}C\textsubscript{2}āC\textsubscript{3}} & ‘stylus’   (\textit{nomen instrumenti}, literally ‘device you write with’)\\ \bottomrule\midrule
    \end{tabularx}
    \caption{Some basic derived nouns of the root \textit{k-t-b} in Classical Arabic}
    \label{doehla:tab:1}
\end{table}

As can be noted, the so-called ‘broken plural’ forms \textit{kuttāb}, \textit{makātib}, and \textit{kutub} are not formed by simply adding a plural suffix, but by reshaping the form of the construction, applying different transfix patterns. 

At the same time, the triliteral root interlocks with a pattern to produce a number of so-called verbal forms that usually indicate semantically related meanings. \autoref{doehla:tab:2} displays eight derived forms of the root \textit{k-t-b}.\footnote{For the basic verbal form, like \textit{kataba} (stem I), Arabic grammarians have always used the simplest of all verbal forms, which corresponds to \Third\SG.\M.\PFV{} of stem I. However, whenever I provide the meaning of a basic verbal form, I use the English infinitive; thus, instead of the literal ‘he wrote’, I use ‘to write’.}
%TODO: Fussnote in Tabelle is nicht erlaubt!.


\begin{table}
    \begin{tabularx}{\textwidth}{lllQ}\midrule\toprule
        stem & lexeme & pattern & meaning \\\midrule
        I & \textit{kataba} & \textit{C\textsubscript{1}aC\textsubscript{2}aC\textsubscript{3}a} & ‘to write’ \\
        II & \textit{kattaba} & \textit{C\textsubscript{1}aC\textsubscript{2}C\textsubscript{2}aC\textsubscript{3}a} & ‘to make write’ \\
        III & \textit{kātaba} & \textit{C\textsubscript{1}āC\textsubscript{2}aC\textsubscript{3}a} & ‘to exchange letters, to correspond’ \\
        IV & \textit{ʔaktaba} & \textit{ʔaC\textsubscript{1}C\textsubscript{2}aC\textsubscript{3}a} & ‘to make write, to dictate’ \\
        VI & \textit{takātaba} & \textit{taC\textsubscript{1}āC\textsubscript{2}aC\textsubscript{3}a} & ‘to write to each other, to exchange letters’ \\
        VII & \textit{inkataba} & \textit{inC\textsubscript{1}aC\textsubscript{2}aC\textsubscript{3}a} & ‘to subscribe’ \\
        VIII & \textit{iktataba} & \textit{iC\textsubscript{1}<t>aC\textsubscript{2}aC\textsubscript{3}a} & ‘to copy, to be registered, to subscribe, to contribute’ \\
        X & \textit{istaktaba} & \textit{ista-C\textsubscript{1}C\textsubscript{2}aC\textsubscript{3}a} & ‘to ask to write, to dictate, to make write, to have a copy made’\\ \bottomrule\midrule
    \end{tabularx}
    \caption{The derived verbal stems of the root \textit{k-t-b} in Classical Arabic}
    \label{doehla:tab:2}
\end{table}



Eight of the possible ten derived, and commonly used, verb stems (\citealt{alward_translating_2019}, \citealt[§§163‒173]{fischer_grammatik_1972}, \citealt[100–105]{holes_modern_2004}) are to be found in the \textit{kataba} entry in Wehr's dictionary (\citeyear[812]{wehr_dictionary_1976}; see \autoref{doehla:tab:2}). Stems II, III, and IV are created ‒ departing from \textit{kataba} ‒ by gemination of the middle consonant (II), by lengthening the vowel /a/ between \textit{C\textsubscript{1}} and \textit{C\textsubscript{2}} (III), and by extending the root by means of the prefix /ʔa-/ while omitting any vowel between \textit{C\textsubscript{1}} and \textit{C\textsubscript{2}}. Stem VI is produced by adding the prefix \textit{ta-} to stem III, whereas stem VIII is formed by the insertion of the infix /-t-/ after \textit{C\textsubscript{1}}. Finally, stem VII is created by adding the prefix \textit{n-} to stem I, preceded by \textit{i-}, in order to avoid an onset with two consonants, whereas in stem X, the root pattern \textit{-ktaba} receives the prefix \textit{st-}, preceded by the same prosthetic vowel as in stem VII. As can be seen in \autoref{doehla:tab:2}, each stem is connected to the semantic field of ‘writing’. In particular, stems II and IV are used to construe factitive and causative meanings in Arabic. The main semantic values of the other stems are as follows: stem III: conative; stem V: reflexive, effective; stem VI: reflexive, reciprocal; stem VII: reflexive, passive; stem VIII: reflexive; stem X: reflexive (\citealt[153]{corriente_gramatica_2002}, \citealt[§§163‒172]{fischer_grammatik_1972}, \citealt[100-105]{holes_modern_2004}). I will come back to the concrete application of these root-transfix patterns, especially of those representing causative values (stems II and IV), in the following sections. 

This formal flexibility of the consonantal roots, in combination with different nuances in meaning, is used productively and creatively by the \textit{bulaġāʔ} ‘eloquent (\Pl{})’, such as Ibn al-Muqaffaʕ, who translated \textit{Kalīla wa-Dimna} from Middle Persian into Arabic. The following two examples illustrate the creative potential of Arabic derivational morphology, which is difficult to imitate using the concatenative morphology of Romance and Germanic languages. 

The first example demonstrates the application of a common linguistically aesthetic technique found in Classical Arabic literature and, thus, in \textit{Kalīla wa-Dimna}: double constructional parallelism, where both the form, that is the derivational pattern, and the meaning of two adjacent nouns are synomorph and synonym at the same time, as shown in Example \xref{doehla:ex:1}.\footnote{In the following Arabic examples, the transcriptions make use of the special characters recommended by the \textit{Deutsche Morgenländische Gesellschaft}. However, my transcriptions include the following three special principles: any long vowel or any consonant that is not represented in the written \textit{rasm} is in superscript; any /a/ represented by \textit{yāʔ} in writing receives a grave accent (\textit{à}); the omission of the vowel represented by the letter \textit{ʔalif}, usually in definite articles and some derived verbal stems (VII, VIII, X), in combined reading of the individual elements of a sentence, is represented by the single quotation mark '.}

\begin{exe}
    \ex\label{doehla:ex:1} (Arabic; Semitic, Afroasiatic) \\
    \gll {bahāʔ-u-h\textsuperscript{ū}} {wa-ǧamāl-u-h\textsuperscript{ū}} \\
    beauty-\NOM-\POSS.\Third\SG{} and-beauty-\NOM-\POSS.\Third\SG{} \\
    \glt ‘his beauty and his beauty’\footnote{Of course, in order to reflect the construction of synonyms, we could translate this as ‘his magnificence and his beauty’.}
\end{exe}


The Old Spanish equivalent in \textit{Calila e Dimna} is \textit{su beldad e su fermosura} \citep[AI.31e]{dohla_libro_2009}, both synonyms, but not synomorphs, since both nouns do not follow the same derivational pattern as the Arabic nouns do (\textit{C\textsubscript{1}aC\textsubscript{2}āC\textsubscript{3}}). \textit{Beldad} is formed with the nominal suffix \textit{-dad}, whereas \textit{fermosura} exhibits the nominal suffix \textit{-ura}. The locating of these Old Spanish synonyms in translation from Arabic may serve to shed some light on synonymity in Old Spanish. 

The second example is even more striking and even less imitable by Old Spanish derivational means. In this case, the verb stem IV is formed and combined with a meaning that does not appear as such in any of the many Arabic dictionaries. The Arabic \textit{nahr} means ‘river’, derived from stem I of the verb \textit{nahara}, ‘to flow copiously, to stream forth’. Taking the root \textit{n-h-r} as the morphological and ‘river’ as the semantic point of departure, Ibn al-Muqaffaʕ derives a verb following the pattern of stem IV, \textit{ʔanhara}, as can be seen in Example \xref{doehla:ex:2}.

\begin{exe}
    \ex\label{doehla:ex:2}
    \gll fa-ya-kūn-u maṯal-ī fī ḏālika miṯla 'l-kalb-i 'llaḏī {\textbf{yu-nhir-u}\footnotemark[12]} \\
    so-\Third\SG.\M-be.\IPFV-\IND{} example-\POSS.\First\SG{} in \DEM.\DIST{} similar.to \Def-dog-\GEN{} \REL.\Third\SG{} \Third\SG.\M-come.by.river.\IPFV-\IND{} \\
    \glt ‘So my example would be similar to the dog who was coming 	by a river’. \citep[38]{cheikho_version_1905}
\end{exe}

\footnotetext[12]{The conjugation of the perfective verbal paradigm makes use of suffixes, whereas the imperfective paradigm uses prefixes (and occasional suffixes as well). Both paradigms also differ in the assignment of vowels to the stem patterns.}
\setcounter{footnote}{12}


This semantic innovation, created by Ibn al-Muqaffaʕ, was analyzed correctly and translated adequately by the Medieval Spanish translator, using the complex analytic construction \textit{que yva por un rrio} ‘who was going by a river’. 

After comparing both Old Spanish manuscripts with the Arabic manuscripts of the Iberian branch \citep{dohla_libro_2009}, it can be stated that the translator(s) of the Arabic \textit{Kalīla wa-Dimna} were skilled in both Arabic and the Old Spanish repertoire available to them, only leaving a few ‘dark’ passages in Old Spanish and only displaying a low number of misreadings (or mishearings). 

With this in mind, in Section \ref{dohla:sec:4}, I present a brief overview of the actual construction analyzed in the subsequent paragraphs.

\section{Factitives and causatives as valency-increasing operations morphology}\label{dohla:sec:4}
Valency-decreasing (by means of anticausative, passive, antipassive, incorporative, reflexive, reciprocal, and medial constructions) and valency-increasing operations (\citealt{comrie_causatives_1993,dixon_changing_2000, haspelmath_107_2004}, \citealt[175–222]{payne_describing_1997}) are very common in the languages of the world. Among the latter constructions, causatives in particular have been attracting the attention of linguists for decades (see \citealt{comrie_language_1989, lehmann_latin_2016, shibatani_grammar_1976, shibatani_causative_2002, song_causatives_1996,song_toward_2001}). 

In this paragraph, as a first step, I depart from an onomasiological point of view as far as the two relevant valency-increasing operations – factitive and causative – are concerned. This seems to be the logical consequence resulting from the previous section, where the flexibility and creativity of the Arabic root and pattern morphology was contrasted with the linear, semantically less flexible morphology of Old Spanish. 

Thus, the term \textit{factitive} is used in this study to refer to those constructions where the meaning side represents a complex event, where the caused embedded action usually denotes a qualitative state (\textit{to be someone}), a change of state (\textit{to become something/somebody}), or an intransitive\footnote{Intransitive verbs can be stative (non-active) or active, meaning the only actant may take the role of the undergoer or the agent.} action (\textit{to fall}), meaning the embedded action or declarative statement is monovalent. In all three cases, the causer, the additional actant who causes the embedded action to happen, is responsible for the qualitative state (\textit{to make someone be something/somebody}), the change of state (\textit{to make someone become something/somebody}), or the intransitive action (\textit{to make fall, to fell}). Thus, the valency-increasing operation results from the addition of the causer as the agent (i.e., syntactic subject) of the factitive construction. In this way, the former subject, be its semantic role that of the agent or the undergoer, is demoted to object status. A case in point is the application of apophony, or ablaut, in Indo-European languages such as English, German, or Latin, where vowel fronting is used to create a factitive meaning, as in the English \textit{to f\textbf{a}ll} > \textit{to f\textbf{e}ll}, the German \textit{f\textbf{a}llen} > \textit{f\textbf{ä}llen}, and the Latin \textit{c\textbf{a}dere} > \textit{c\textbf{ae}dere}. The same segmental transfixation can be found in some Arabic verbs like \textit{ḥazina} ‘to be sad’ versus \textit{ḥazana} ‘to make sad, to sadden’ (both are verbs with the pattern of stem I) \citep[174]{wehr_dictionary_1976}. But Arabic also makes use of the derived stems II and IV to express the same meaning of ‘to make sad’: \textit{ḥazzana} (II) and \textit{ʔaḥzana} (IV). 

The term \textit{causative} is used in this study to refer to a similar complex event as in the case of factitives, but this time, the embedded action is represented by a verb, which is transitive or divalent. Hence, the embedded action that already comprises an agent and a patient is extended by the influence of another agent, the causer, who acts on the embedded agent in a way that s/he performs the action with the embedded patient. The embedded agent is degraded to the role of causee (see \autoref{doehla:fig:1}), which in several languages is coded as a direct object, so that, syntactically, there are two direct objects representing the extralinguistic referents of the embedded action, as in \textit{he made his son write a song} or \textit{he made him write it}.

\begin{figure}
    \includegraphics[width=0.6\textwidth]{Doehla_figure1}
    \caption{The complex event expressed by causative constructions comprising three actants: the causer (Z) of the embedded action, which is performed by the causee (X), who is the agent of the embedded action that is executed upon the patient (Y).}
    \label{doehla:fig:1}
\end{figure}


The same double object constructions can be observed in Arabic taken from the monolingual dictionary \textit{Al-Munǧid}, illustrating the use of \textit{ʔaktaba} and its meaning (see Example \ref{doehla:ex:3}).

\begin{exe}
    \ex\label{doehla:ex:3}
    \gll {ʔaktab-a-h\textsuperscript{ū}} {'l-qaṣīdat-a:} {ʔaml-ā-h\textsuperscript{ū}} {ʔiyyā-hā} \\
    make.write.\PFV-\Third\SG.\M-\OBJ.\Third\SG.\M{} \Third\SG.\M\Def-ode.\F-\ACC{} dictate.\PFV-\Third\SG.\M-\OBJ.\Third\SG.\M{} \ACC-\OBJ.\Third\SG.\F{} \\
    \glt ‘he made him write the ode: he dictated it to him’ \citep[671]{al-mungid_-mungidu_2005}
\end{exe}

Stem IV of the Arabic root \textit{k-t-b}, ‘to make write, to dictate’, may take two direct objects. Problems only occur when both objects are pronominalized, since verbs may only agglutinate one direct object pronoun at a time. Therefore, Classical Arabic has developed the ‘object carrier’ \textit{ʔiyyā-} which extracts one object from the verb in order to bypass the ‘one object constraint’. 

As already noted, there are several linguistic strategies found in the languages of the world for expressing factitive and causative events formally. The typological possibilities are summed up in \autoref{doehla:tab:3}, following \citet[926]{lehmann_latin_2016} and considering only the verbal element of the causativization strategies, leaving aside the treatment of nominal elements.

\begin{table}
    \begin{tabularx}{\textwidth}{QQQllQ}\midrule \toprule
        \multicolumn{6}{l}{reduction of complexity $\longrightarrow$} \\\midrule
        morpho-syntactic process & lexical--syntactic & analytic     & synthetic    & fusional    & zero \\\midrule
        verbal strategy         & complex sentence  & periphras-tic & derivational & alternation & valency conversion\\ \bottomrule\midrule
    \end{tabularx}
    \caption{Typology of causativization strategies (following \citealt[926]{lehmann_latin_2016})}
    \label{doehla:tab:3}
\end{table}

So far in this chapter, we have seen examples of analytic (the English \textit{to make write}), derivational (the Arabic stem II \textit{kattaba}), and alternational verbal strategies (the English \textit{to fall} versus \textit{to fell}). Even though “Latin does not have a productive morphological process for the formation of causative constructions” \citep[918]{lehmann_latin_2016}, it uses other means like complex sentences or compound verbs \citep{hoffmann_lateinische_2018,lehmann_latin_2016}. (Old) Spanish and other Romance languages have also developed formal morphological and periphrastic means to express causativization. Thus, in what follows, I will present the three most common and productive verbalization strategies of factitive and causative events found in the Old Spanish version of \textit{Kalīla wa-Dimna} and compare them to the Arabic model. As already pointed out, the research question behind this analysis is to see whether the Arabic model may have influenced the translator(s)' choice as far as the formal side of the Old Spanish factitives and causatives is concerned.

\section{Factitive and causative constructions in the Old Spanish translation of \textit{Kalīla wa-Dimna}}\label{dohla:sec:5}
\subsection{Zero morphology and conversion}\label{dohla:sec:5.1}


The first set of examples of valency-increase consists of verbs that, in principle, are intransitive, but are used with two actants in \textit{Calila e Dimna}. These transitive uses of otherwise intransitive verbs mostly fly under the radar of Hispanic lexicologists, because the verb does not display any morphological change that would indicate a difference in use and meaning. Moreover, the fact that the Old Spanish of the thirteenth century in general, and that used in the translated texts from Arabic in particular,\footnote{Classical Arabic, like Biblical Hebrew, follows the Proto-Semitic VS(O) order. However, the Modern Arabic dialects and Modern Hebrew prefer an SV(O) order.} is a VS(O) language \citep{bossong_sintaxis_2006,lopez_garcia_como_2000,neumann-holzschuh_satzgliedanordnung_1997} makes the detection of transitive uses a difficult undertaking, because any superficial analysis would read the NP after the verb as the subject of the sentence, simply following the entrenched idea already existing in the lexicologist’s mind. Only a meticulous syntactic and semantic analysis could find the cases in point, as in the following examples from \textit{Calila e Dimna}.

The first example concerns the Old Spanish verb \textit{encaresçer}, which has the basic meaning ‘to become expensive’. It is a deadjectival derivation from \textit{caro} ‘expensive’ by means of the circumfix \textit{en}-\ADJ{}-\textit{esçer}, the suffixal part of which was already used in Latin to create an inchoative verb from an adjectival root (see below) and in Vulgar Latin to generate a factitive meaning \citep[§316]{vaananen_introduction_1978}. This basic meaning of \textit{encaresçer} is also attested in \textit{Calila e Dimna}, where we can find the sentence presented in Example \xref{doehla:ex:4}.

\begin{exe}
    \ex\label{doehla:ex:4}
    \gll Et acaesçio que encaresçio la miel e la manteca \\
    and happen.\Third\SG.\PST.\PFV{} \COMPL{} become.expensive.\Third\SG.\PST.\PFV{} \Def.\F{} honey and \Def.\F{} fat \\
    \glt ‘And it happened that honey and fat became (more) 	expensive’. \citep[A.VI.9]{dohla_libro_2009}
\end{exe}


The word order in the subordinate clause introduced by the complementizer \textit{que} is VS. The fact that there is a mismatch as far as the agreement between the sentence-initial verb (in \SG{}) and the subject-NP (in \PL{}, since there are two nouns) is concerned, is certainly due to the Arabic model, which displays the following text:

\begin{exe}
    \ex\label{doehla:ex:5}
    \gll wa-wāfaq-a ḏālika \textbf{ġal-ā} 'l-saman-u {wa-'l-ʕasal-u\footnotemark[15]} \\
    and-become.\PFV-\Third\SG.\M{} \DEM.\DIST{} become.expensive.\PFV-\Third\SG.\M{} \Def-fat.\M-\NOM{} and-\Def-honey.\M-\NOM{} \\
    \glt ‘and it happened (that) fat and honey became (more) expensive’ \citep[fo. 139r]{ibn_al-muqaffa_kitab_nodate}
\end{exe}
\footnotetext[15]{In the actual manuscript, \textit{saman}- and \textit{ʕasal}- are vocalized with -\textit{a} (\ACC{}), which does not seem to be correct. It should be -\textit{u}, which marks the nominative case.}


In Example \xref{doehla:ex:5}, it is possible to observe the same construction and agreement in the Arabic pattern as found in the Spanish translation.

In contrast to this example of the basic use of \textit{encaresçer} in Old Spanish, the following attested use (see Example \ref{doehla:ex:6}) in \textit{Calila e Dimna} certainly diverges from the latter by its increase in valency and by a degree of difficulty regarding the assignment of the correct meaning in the given context. The particular chapter where the sentence in \xref{doehla:ex:4} can be found (chapter VI in \citealt{dohla_libro_2009}), deals with the benefits of friendship. Thus, a raven wants to befriend a mouse, who is hiding in a hole in the ground, frightened to come out because of the apparent danger, that is the presence of the raven on a nearby tree. Nevertheless, the raven keeps on talking to the mouse, trying to convince it to become its friend. In the course of the dialogue, the raven utters the following sentence:

\begin{exe}
    \ex\label{doehla:ex:6}
    \begin{xlist}
        \ex
        \gll e non me encarezca-s la cosa \\
        and \NEG{} \First\SG.\OBJ{} make.expensive.\PRS.\SBJV-2\SG{} \Def{} thing \\
        \glt \citep[AIII.31a]{dohla_libro_2009} \\
        \ex
        \gll e non encarezca-s el {amor\footnotemark[16]} \\
        and \NEG{} make.expensive.\PRS.\SBJV-2\SG{} \Def{} love \\
        \glt \citep[BIII.31a]{dohla_libro_2009}
    \end{xlist}
\end{exe}
\footnotetext[16]{Interestingly, the word \textit{friendship} does not appear in any Arabic manuscript. However, one of the aesthetic techniques of the translator(s) was to play around with the sequence of phonemes between the source and the target language. This way, \textit{amor} was chosen as the equivalent of \textit{ʔumūr}, the plural of \textit{ʔamr} ‘matter’, just because of phonological similarities concerning the consonants. Another case in point is the translation of \textit{ṣabr} ‘patience’ with \textit{sufrimiento} because of the \textit{sufr}- onset. \textit{Sufrimiento} actually means ‘suffering’.}

With the basic meaning of \textit{encaresçer} in mind and considering the transitive construction, the literal meaning of the two sentences would be ‘and do not make the matter expensive for me’ and ‘and do not make the friendship expensive’, respectively. The verb clearly exhibits the \Second\SG{} -\textit{s} (\textit{encarezca-s}) and a direct object and, thus, is used divalently. However, when looking at the Arabic model in Example \xref{doehla:ex:7}, the actual meaning of \textit{encaresçer} becomes clear.

\begin{exe}
    \ex\label{doehla:ex:7}
    \gll wa-lā tu-ṣaʕʕib ʔal-ʔumūr-a \\
    and-\NEG{} \Second\SG.\M-make.difficult.\JUS{} \Def-matter;\PL-\ACC{} \\
    \glt‘and do not make the matter difficult’ \citep[fo. 48v]{azzam_kitab_1941}
\end{exe}


The Classical Arabic verbal form \textit{tuṣaʕʕib} is in the jussive mood; it is basically the indicative imperfective form of the verb with elision of the final vowel. The derivational pattern follows stem II, that is with factitive/causative meaning. Thus, \textit{no encarezcas} means ‘do not make difficult’, which is not surprising at all considering the fact that the adjective \textit{caro}, apart from ‘expensive’, also had the meaning ‘difficult’ in Old Spanish \citep[138]{kasten_tentative_2001}. In this way, the translator(s) extended the meaning of the Old Spanish \textit{encaresçer} according to the Arabic model: 
\begin{itemize}
\item[]\textit{ṣaʕuba} (stem I) ‘to be difficult’ / \textit{ṣaʕb} ‘difficult’ → \textit{ṣaʕʕaba} (stem II) ‘to make difficult’
\item[]\textit{caro} ‘difficult’ → \textit{encaresçer} ‘to make difficult’
\end{itemize}

It is interesting to observe that none of the vocabularies dealing exclusively with the lexicon of \textit{Calila e Dimna} \citep{holmes_etymological_1936,perez_vocabulario_1943, stinson_etymological_1967} and none of the major Old Spanish dictionaries (\citealt{alonso_diccionario_1996}, \citealt{kasten_tentative_2001}) indicate the correct meaning of \textit{encaresçer} in the context of the above-mentioned text passage. 

But there are other examples of the same kind where the Old Spanish translation follows the Arabic model, increasing the valency of an otherwise intransitive Old Spanish verb.

\begin{exe}
    \ex
    \begin{xlist}
        \ex\label{doehla:ex:8a}
        \gll e festina al tardinero \\
        and make.hurry.\Third\SG.\PRS.\IND{} \OBJ;\Def{} slow.person \\
        \glt ‘and makes the slow person hurry’ \citep[I.174t]{dohla_libro_2009}
        \ex\label{doehla:ex:8b}
        \gll wa-yu-sarriʕ-u 'l-baṭīʔ-a \\
        and-\Third\SG.\M-make.hurry.\IPFV-\IND{} \Def-slow.person-\ACC{} \\
        \glt ‘and makes the slow person hurry’ \citep[fo. 49r]{ibn_al-muqaffa_kitab_nodate}
    \end{xlist}
\end{exe}


In \xref{doehla:ex:8a}, the verb \textit{festinar}, which is normally used intransitively, just like the classical Latin \textit{festīnāre}, in the sense of ‘to hurry, to hasten’, is extended to a divalent scope in order to translate the factitive meaning of \textit{sarraʕa} ‘to urge someone, to hurry someone’. This is a stem II verb pattern, derived from the basic stem \textit{saruʕa} ‘to be fast, to hurry’. 

All cases of valency conversions found in the Old Spanish translation of \textit{Kalīla wa-Dimna} share the property of zero morphology, thus not overtly indicating the change in meaning and syntactic scope. In the two examples discussed here, only \textit{encaresçer} follows the Arabic model as much as it can, given the morphological possibilities of Old Spanish. After editing the two Old Spanish manuscripts and comparing them meticulously with the Arabic model \citep{dohla_libro_2009}, it can be said that the attitude of the translator(s) can be determined as one that makes every effort to express as much as possible of the Arabic model by means of the Old Spanish lexicon. At the same time, the translator(s) are adventurous and creative and do not hesitate to introduce lexico-semantic contextual innovations based on their solid knowledge of the Old Castilian language.

\largerpage
\subsection{Denominal derivations}\label{dohla:sec:5.2}
The second set of verbs that are used to create a factitive meaning consist of denominal and deverbal derivations. In this case, a nominal root, for example, is taken with its basic meaning as the point of departure for verbalization by adding the prefix \textit{a}- or \textit{en}-, cutting off the final vowel of the noun and adding whatever conjugational pattern (following the \textit{a}-conjugation) is required by the given context. Alternatively, a verb is prefixed by \textit{a}-, thus creating the same factitive/causative meaning. Apart from being a productive method of creating factitive verbal constructions with the help of morphological means in \textit{Calila e Dimna}, both strategies are also present in other Romance languages, such as French, Italian, and Portuguese, and Judeo-Spanish. However, as I will show in the following examples, the Old Spanish translator(s) of \textit{Kalīla wa-Dimna} made use of this construction to reproduce the Arabic stem II or stem IV derivations that have a factitive/causative meaning. Cases in point are the following three passages from \textit{Calila e Dimna} with the respective Old Spanish translation and the Arabic model.

\begin{exe}
    \ex\label{doehla:ex:9}
    \begin{xlist}
        \ex\label{doehla:ex:9a}
        \gll et esfuerça al cobarde, e {\textbf{encobarda/acouarda}} al esforçado \\
        and strengthen.\Third\SG.\PRS{} \OBJ;\Def{} coward and make.coward.\Third\SG.\PRS{} \OBJ;\Def{} brave \\
        \glt ‘and it [fate, fortune] strengthens the coward and makes a coward out of the brave’ \citep[A/BI.174t]{dohla_libro_2009}
        \ex\label{doehla:ex:9b}
        \gll wa-yu-šaǧǧiʕ-u 'l-ǧabān-a {\textbf{wa-yu-ǧabbin-u}} 'l-šaǧāʕ-a \\
        and-\Third\SG.\M-encourage.\IPFV-\IND{} \Def-coward-\ACC{} and-\Third\SG.\M-make.coward.\IPFV-\IND{} \Def-brave-\ACC{} \\
        \glt ‘and it encourages the coward and makes a coward out of the brave’ \citep[fo. 49r]{ibn_al-muqaffa_kitab_nodate}
    \end{xlist}
\end{exe}
\begin{exe}
    \ex\label{doehla:ex:10}
    \begin{xlist}
        \ex\label{doehla:ex:10a}
        \gll e \textbf{apriuado} \textbf{lo} mas que a todos sus vasallos \\
        and make.confidant.3\SG.\PST.\PFV{} \OBJ.\Third\SG.\M{} more than \OBJ{} all.\PL{} \POSS.\Third\PL{} vassal.\PL{} \\
        \glt ‘and he [the king] made him confidant, closer than all his other vassals’ \citep[AXII.16]{dohla_libro_2009}
        \ex\label{doehla:ex:10b}
        \gll et \textbf{puso-le} \textbf{en} \textbf{mayor} \textbf{pryuança} que a ninguno de sus vasallos \\
        and put.\Third\SG.\PST.\PFV-\OBJ.\Third\SG.\M{} \LOC{} more confidence than \OBJ{} none \GEN{} \POSS.\Third\PL{} vassal.\PL{} \\
        \glt ‘and he granted him more confidence than any other of his vassals’ \citep[BXII.16]{dohla_libro_2009}
        \ex\label{doehla:ex:10c}
        \gll {wa\textbf{-ʔaḫaṣṣ-a-h\textsuperscript{ū}}} dūna {ʔaṣḥāb-i-h\textsuperscript{ī}} \\
        and-make.private.\PFV-\Third\SG.\M-\OBJ.\Third\SG.\M{} more.than companion.\PL-\GEN-\POSS.\Third\SG{} \\
        \glt ‘and he made him confidant, more than his companions’ \citep[94v]{azzam_kitab_1941}\footnotemark[17]
    \end{xlist}
\end{exe}

\footnotetext[17]{The edition of \citet{azzam_kitab_1941} contains \textit{aḫtaṣṣa}, stem VIII. In \citet{ibn_al-muqaffa_kitab_nodate}, the actual verb is missing.}
\setcounter{footnote}{17}

\begin{exe}
    \ex\label{doehla:ex:11}
    \begin{xlist}
        \ex\label{doehla:ex:11a}
        \gll El rrey non \textbf{apriua} a los omnes por la priuança de sus padres, … \\
        \Def{} king \NEG{} make.confidant \OBJ{} \Def.\M.\PL{} men because.of \Def.\F.\SG{} confidence \GEN{} \POSS.\Third\PL{} forefather.\PL{} {} \\
        \glt ‘The king does not make men confidants because of the confidence of their forefathers’. \citep[AI.41a]{dohla_libro_2009}
        \ex\label{doehla:ex:11b}
        \gll ʔinna 'l-sulṭān-a lā \textbf{yu-dnī} 'l-riǧāl-a li-qurb-i ʔabāʔ-i-him \\
        truly \Def-sultan-\ACC{} \NEG{} \Third\SG.\M-make.confidant.\IPFV;\IND{} \Def-men-\ACC{} because.of-proximity-\GEN{} forefather.\PL-\GEN-\POSS.\Third\PL.\M{} \\
        \glt ‘The sultan does not make men confidants because of the proximity of their forefathers’. \citep[fo. 29v]{ibn_al-muqaffa_kitab_nodate}
    \end{xlist}
\end{exe}


In Example \xref{doehla:ex:9a}, the translator(s) created a new word that can only be found in \citet[I, 443]{muller_diccionario_1987}, %TODO: Hier fehlt "-2005"
with the single medieval reference of \textit{Calila e Dimna}. All other pieces of textual evidence are from later centuries, from the end of the fifteenth century onward (\textit{acobardar} earlier than \textit{encobardar}, the latter being very rare). Here, the translator(s) followed a similar derivational pathway as in Arabic, always making use of the morphological repertoire available to them in Old Spanish:

\begin{itemize}
\item[]\textit{cobarde} ‘coward’ → \textit{ser cobarde} ‘to be a coward’ → \textit{encobardar/acovardar} ‘to make (someone a) coward’
\item[]\textit{ǧabān} ‘coward’ → \textit{ǧabuna} ‘to be a coward’ → \textit{ǧabbana} ‘to make (someone a) coward’.
\end{itemize}

However, the translator(s) were not able to reproduce the aesthetic morphological rhyme parallelism applied in the following Arabic formula with \textit{C\textsubscript{1}-C\textsubscript{2}-C\textsubscript{3}} and \textit{D\textsubscript{1}-D\textsubscript{2}-D\textsubscript{3}} representing two different consonantal roots:

\begin{itemize}
\item[]\textit{wa-yuC\textsubscript{1}aC\textsubscript{2}C\textsubscript{2}iC\textsubscript{3}u ’l-D\textsubscript{1}aD\textsubscript{2}āD\textsubscript{3}a wa-yuD\textsubscript{1}aD\textsubscript{2}D\textsubscript{2}iD\textsubscript{3}u ’l-C\textsubscript{1}aC\textsubscript{2}āC\textsubscript{3}a}
\end{itemize}

Nevertheless, they were still eager to get as close as possible to the etymological pattern found in the Arabic model, thus using \textit{es\textbf{for}çar} – \textit{es\textbf{for}çado} and \textit{\textbf{cobarde}} – \textit{en\textbf{cobard}ar/a\textbf{covard}ar}, where both pairs are etymologically related through derivation just as \textit{\textbf{š}a\textbf{ǧ}ǧa\textbf{ʕ}a} – \textit{\textbf{š}a\textbf{ǧ}ā\textbf{ʕ}} and \textit{\textbf{ǧ}a\textbf{b}ba\textbf{n}a} – \textit{\textbf{ǧ}a\textbf{b}ā\textbf{n}}. 

In the next two passages presented above, there is another denominal \xref{doehla:ex:10} and a deverbal derivation \xref{doehla:ex:11}. In both cases, the resulting verbal form exhibits the \textit{a}- prefix, \textit{apriuadar} and \textit{apriuar}. Whereas \textit{*privadar} is not documented according to \textit{Real Academia Española} (\citeyear{real_academia_espanola_banco_nodate}) and \citet{pharies_diccionario_2002}, \textit{privar} can be found in other medieval texts with the meaning ‘to dispose of’ \citep[569]{kasten_tentative_2001}, which has been inherited from its Latin etymon \textit{privāre}. The Old Spanish \textit{privar} in the meaning ‘to make someone a confidant’ also appears once in \textit{Calila e Dimna} \citep[AI.140b]{dohla_libro_2009}. Unfortunately, none of the Arabic models exhibits the same sentence (‘and you made him a confidant’), which could be an indication as to the nature of the Arabic verb serving as a model. 

Nevertheless, it can be stated that \textit{aprivadar} and \textit{aprivar} follow the morphological pattern of Arabic stem IV derivations (\textit{ʔaḫaṣṣa} and \textit{ʔadnà}, respectively) that have an overt \textit{ʔa}- prefix. Of course, this \textit{a}- prefix is certainly not a matter borrowing from Arabic.\footnote{\citet[189]{pharies_diccionario_2002} states that as far as the development of parasynthetic verbal constructions is concerned, “se desarrollan en romance las llamadas estructuras parasintéticas, como \textit{en} … \textit{ecer} y a … \textit{ar}, cuyo element prefijal es semánticamente vacío” (‘the so-called parasynthetic constructions were developed in Romance, just like \textit{en} … \textit{ecer} and \textit{a} … \textit{ar}, the prefixes of which are semantically empty’ (own translation)).} It is definitely taken from Romance matter by making use of an already existing construction, but it is used productively to emphasize the factitive meaning of the verb. Thus, the translator(s) chose linguistic structures as Romance as possible by using original Old Spanish matter, but they also displayed a certain “voluntad de dejarse influir” \citep[230]{galmes_de_fuentes_influencia_1996} ‘willingness to be influenced’, to adapt to Arabic structures where their intuition of acceptable Romance grammar would permit. In the case of the translator(s) of \textit{Kalīla wa-Dimna}, for example, they did not accept another valency-increasing strategy (that does not generate a factitive/causative meaning), namely the \textit{figura etymologica}. This is a special case of paronomasia, where the monovalent nature of an intransitive verb is changed to divalent by adding an etymologically related noun as a direct object, often in order to generate an intensive adverbial meaning, as in the following Arabic example from \textit{Kalīla wa-Dimna}.

\begin{exe}
    \ex\label{doehla:ex:12}
    \gll fa-baynamā humā ka-ḏālik ʔiḏ \textbf{ḫār-a} 'l-ṯawr-u {\textbf{ḫuwār-a\textsuperscript{n}}} {šadīd-a\textsuperscript{n}} \\
    and.then-while 2.\DU{} like-\DEM.\DIST{} and.then moo.\PFV-\Third\SG.\M{} \Def-bull-\NOM{} mooing-\ACC.\INDF{} intense-\ACC.\INDF{} \\
    \glt ‘and then, while they were like this, then the bull mooed 	intensely’ (literally ‘… the bull mooed an intense mooing’) \citep[fo. 30r/v]{ibn_al-muqaffa_kitab_nodate}
\end{exe}


In Example \xref{doehla:ex:12}, the intransitive verb \textit{ḫāra} ‘to moo’ is extended by adding the verbal noun (Ar. \textit{maṣdar}) of the same verb \textit{ḫāra} as a direct object, both verb and object thus containing the same three consonants \textit{ḫ-w-r}. This type of object is called “inner” or “absolute” object \citep[§§376–377]{fischer_grammatik_1972}. As I mentioned before, there is no single instance of the \textit{figura etymologica} construction found in the two manuscripts of the Old Spanish translation of \textit{Kalīla wa-Dimna}. The respective passages in manuscripts A and B are the following:

\begin{exe}
    \ex\label{doehla:ex:13}
    \gll Et estando amos asy, bramo {Çençeba /} el buey muy {fuerte…} \\
    and be.\ACT.\PTCP{} both like.this moo.\Third\SG.\PST.\PFV{} \PN{} \Def{} bull very strong \\
    \glt ‘and while both were like this, Çençeba / the bull mooed intensely’ \citep[A/BI.45a]{dohla_libro_2009}
\end{exe}


Both manuscripts display the adjective \textit{fuerte} ‘strong’ in the function of an adverb. However, \citet[201]{galmes_de_fuentes_influencia_1996} cites the sentence “E estando amos así, bramió Çençeba muy fuerte bramido”, a construction that is supposed to reproduce literally the \textit{figura etymologica}. Unfortunately, this sentence is a pure invention by \citet[76]{alemany_y_bolufer_antigua_1915}, from where it was taken. Yet the Old Spanish \textit{figurae etymologicae} can be found in scientific translations from Arabic during the second half of the thirteenth century and, even more frequently, in the Aljamiado-Morisco literature of the sixteenth and seventeenth centuries: \textit{lloró lloramiento muy grande} ‘and he cried heavily’ (\textit{Libro de las batallas}, 16th c.; \citealt[203]{galmes_de_fuentes_influencia_1996}).

\subsection{Analytic causative constructions by means of \textit{fazer}, \textit{mandar}, and \textit{enbiar}}\label{dohla:sec:5.3}

%\noindent
The last set of verbal constructions with increased valency concerns analytic ones with the following structure in the Old Spanish version of 
\textit{Calila e Dimna: fazer} ‘make’ + \Inf{} \citep{alfonso_vega_verbos_2006, aranda_ortiz_expresion_nodate, sanaphre_villanueva_analytic_2010}.\footnote{There are also complex lexico-syntactic constructions with the pattern \textit{fazer} \COMPL{} \CLAUSE{}.}  This pattern follows one that can also be found in other Romance languages. Cases in point can be found in the following passage.

\begin{exe}
    \ex\label{doehla:ex:14}
    \begin{xlist}
        \ex\label{doehla:ex:14a}
        \gll Et esto \textbf{le} \textbf{faze} \textbf{descuydar} de sy e de su fazienda et \textbf{faze-lo} \textbf{olvidar} aquello en que esta, et \textbf{faze-le} \textbf{dexar} la carrera por que se ha de saluar \\
        and \DEM.\PROX{} \OBJ.\Third\SG.\M{} make.\Third\SG.\PRS{} neglect.\Inf{} \GEN{} himself and \GEN{} \POSS.\Third\SG{} affair and make.\Third\SG.\PRS-\OBJ.\Third\SG{} forget.\Inf{} \DEM.\DIST{} \LOC{} what be.\Third\SG.\PRS{} and make.\Third\SG.\PRS-\OBJ.\Third\SG{} leave.\Inf{} \Def{} path through which \REFL{} have.\Third\SG.\PRS{} \PREP{} save \\
        \glt ‘and this [the little sweetness of this world] makes him [the 	human being] neglect himself and his affair, and it makes him 	forget what [the miserable situation] he is in, and it makes 	him leave the path through which he has to save himself’ \citep[BC.69e]{dohla_libro_2009}
        \ex\label{doehla:ex:14b}
        \gll {fa\textbf{-yu-šġil-u-h\textsuperscript{ū}}} ʕan {nafs-i-h\textsuperscript{ī}} {wa\textbf{-yu-lhī-h\textsuperscript{ī}}} ʕan {ʔamr-i-h\textsuperscript{ī}} {wa\textbf{-yu-nsī-h\textsuperscript{ī}}} ʕan {šaʔn-i-h\textsuperscript{ī}} {wa\textbf{-ya-ṣudd-u-h\textsuperscript{ū}}} ʕan sabīl-i {naǧāt-i-h\textsuperscript{ī}} \\
        and.then-\Third\SG{}.\M-distract.\IPFV-\IND-\OBJ.\Third\SG.\M{} \ABL{} soul-\GEN-\POSS.\Third\SG.\M{} and-\Third\SG.\M-distract.\IPFV;\IND-\OBJ.\Third\SG.\M{} \ABL{} affair-\GEN-\POSS.\Third\SG.\M{} and-\Third\SG.\M-make.forget.\IPFV;\IND-\OBJ.\Third\SG.\M{} \ABL{} affair-\GEN-\POSS.\Third\SG.\M{} and-\Third\SG.\M-turn.away.\IPFV-\Inf-\OBJ.\Third\SG.\M{} \ABL{} path-\GEN{} salvation-\GEN-\POSS.\Third\SG.\M{} \\
        \glt ‘and then it distracts him away from himself, and it distracts him away from his affair, and it makes him forget his affair, and it turns him away from the path of salvation’ \citep[fo. 21r]{ibn_al-muqaffa_kitab_nodate}
    \end{xlist}
\end{exe}


As can be seen in Example \xref{doehla:ex:14b}, all analytic causative constructions with \textit{fazer} + \Inf{} go back to stem IV verbs in Arabic (\textit{ʔašġala}, \textit{ʔalhā}, \textit{ʔansā}),\footnote{Apparently, the translator(s) fused the first two Arabic sentences, since both verbs are synonyms.} except for the last sentence, where the verb \textit{ṣadda} (stem I) inherently contains the causative meaning. The Old Spanish translator(s) chose analytic constructions by means of \textit{fazer} + \Inf{} due to the fact that they were already in use productively in the thirteenth century (see \citealt{sanaphre_villanueva_analytic_2010}), and in other text genres not translated from Arabic to Old Spanish. This way, they used a common causative construction of the following kind, as can be seen in \autoref{doehla:fig:2}.

\begin{figure}
    \includegraphics[width=0.6\textwidth]{Doehla_figure2}
    \caption{Causative construction with \textit{fazer} found in \textit{Calila e Dimna}, (see Example \ref{doehla:ex:13})} 
    \label{doehla:fig:2}
\end{figure}


The causee, that is the ‘human being’ in Example \xref{doehla:ex:14a}, is demoted to the object position, overtly indicated by the object pronoun le/lo (proclitic and enclitic) in Old Spanish and the suffixed object pronoun in Arabic. So far, the translation of Arabic stem IV verbs by means of an analytic causative construction follows the expected pattern. However, Old Spanish causatives may also be expressed by other analytic verbal constructions, among others, by \textit{enbiar} and \textit{mandar}, which are, together with \textit{fazer}, the most common causative analytical verbs. A pertinent example can be found in the following passage from \textit{Calila e Dimna}, where all three analytical constructions are present.

\begin{exe}
    \ex\label{doehla:ex:15}
    \begin{xlist}
        \ex\label{doehla:ex:15a}
        \gll Mucho \textbf{me} \textbf{as} \textbf{fecho} \textbf{aboreçer} la pryuança de Sençeba, e \textbf{yo} \textbf{enbiar} \textbf{le} \textbf{he} \textbf{dezir} loque tengo en coraçon, et \textbf{mandar} \textbf{le} \textbf{he} \textbf{que} \textbf{se} \textbf{vaya} do quisiere. \\
        much \OBJ.\First\SG{} have.\AUX.\Second\SG.\PRS{} make.\PTCP{} hate.\Inf{} \Def{} proximity \GEN{} \PN{} and \First\SG{} send.\Inf{} \OBJ.\Third\SG.\M{} have.\AUX.\First\SG.\PRS{} say.\Inf{} what have.\First\SG.\PRS{} \LOC{} heart and order.\Inf{} \OBJ.\Third\SG.\M{} have.\AUX.\First\SG.\PRS{} \COMPL{} \REFL{} go.\Third\SG.\PRS.\SBJV{} where want.\Third\SG.\FUT.\SBJV{} \\
        \glt ‘You have made me hate a lot the proximity of Sençeba, and I 	will let him know what I have in mind, and I will order him to 	go wherever he wants to’. \citep[AI.157a]{dohla_libro_2009}
        \ex\label{doehla:ex:15b}
        \gll la-qad	\textbf{tarak-ta-nī} \textbf{kārih-a\textsuperscript{n}} {li-muǧāwarat-i} šanzaba {wa\textbf{-ʔana}} \textbf{mursil-u\textsuperscript{n}} \textbf{ʔilay-h\textsuperscript{ī}} \textbf{wa-ḏākir-a\textsuperscript{n}} \textbf{lah\textsuperscript{ū}} mā waqaʕ-a fī nafs-ī {wa\textbf{-ʔāmar-u-h\textsuperscript{ū}}} \textbf{'l-laḥāq-a} ḥayṯu ʔaḥabb-a \\
        truly-already let.\PFV-\Second\SG.\M-\OBJ.\First\SG{} hate.\ACT.\PTCP-\ACC.\INDF{} \PREP-proximity-\GEN{} \PN{} and-\First\SG{} send.out.\ACT.\PTCP-\NOM.\INDF{} to-\OBJ.\Third\SG.\M{} and-tell.\ACT.\PTCP-\ACC.\INDF{} \OBJ;\Third\SG.\M{} what happend.\IPFV-\IND-\POSS.\Third\SG.\M{} \LOC{} soul-\POSS{}.\First\SG{} and-\First\SG;order.\IPFV-\IND-\POSS.\Third\SG.\M{} \Def-entering-\ACC{} where want.\PFV-\Third\SG.\M{} \\
        \glt ‘Truly, you already made me hate the proximity of Šanzaba, and I will send out to him and tell him what I have in mind, and I will order him to go wherever he wants to’. \citep[fo. 45r]{ibn_al-muqaffa_kitab_nodate}
        \saveex % See localcommands.tex for definition
    \end{xlist}
\end{exe}


Interestingly, the decision made by the translator(s) in choosing an analytical verbal construction is somewhat predetermined by the Arabic model, in that each Old Spanish causative auxiliary verb can be traced back to exactly the same verbal semantics in the Arabic model:

\begin{itemize}
\item[]\textit{taraka} ‘to let’ → \textit{fazer} ‘to make, to let’
\item[]\textit{mursil} ‘sending out’ (← stem IV verb \textit{ʔarsala} ‘to send out’) → \textit{enbiar} ‘to send’
\item[]\textit{ʔamara} ‘to order’ → \textit{mandar} ‘to order’
\end{itemize}

Besides this, the causative verb is also followed by an infinitive form in Arabic, namely the active participle (gerund) or the \textit{maṣdar} (verbal noun), \textit{kārih/ḏākir} and \textit{laḥāq}. The use of the infinitive can also be observed in the Old Spanish translation, except for the last sentence with \textit{mandar}, which exhibits a subordinate clause introduced by the complementizer \textit{que}, a structure that was on the rise in the thirteenth century \citep{sanaphre_villanueva_analytic_2010}. However, the respective sentence in manuscript B (the passage in Example \ref{doehla:ex:15a} is from manuscript A) also displays the infinitive after \textit{mandar}.

%%\begin{exe}
%%    \exr{doehla:ex:15} %TODO: This is somewhat of a workaround to continue with the 15c numbering. You can see the commands used for this in localcommands.tex. However I strongly advice against using this, and instead consider going on with number 16 at this point.
%%    \begin{xlist}
%%        \resumeex % See localcommands.tex for definition
%%        \ex\label{doehla:ex:15c}
%%        \gll et mandar-le-he salyr de my tierra \\
%%        and order.\OBJ.\Third\SG.\M-have.\AUX.\First\SG.\PRS{} leave.\Inf{} \ABL{} \POSS.\First\SG{} land \\
%%        \glt ‘and I will order him to leave my land’. \citep[BI.157a]{dohla_libro_2009}
%%    \end{xlist}
%%\end{exe}

\begin{exe} \exi{(15)} %%Alternative solution..
    \begin{xlist}
\exi{c.} \label{doehla:ex:15c}
        \gll et mandar-le-he salyr de my tierra \\
        and order.\OBJ.\Third\SG.\M-have.\AUX.\First\SG.\PRS{} leave.\Inf{} \ABL{} \POSS.\First\SG{} land \\
        \glt ‘and I will order him to leave my land’. \citep[BI.157a]{dohla_libro_2009}
    \end{xlist}
\end{exe}


First of all, it has to be stated that, to my knowledge, there are no studies concerning analytical causative constructions in Arabic. Those that deal with causatives are usually restricted to derivational verb patterns \citep{saad_causatives_1974,ford_three_2009}. However, the fact that they appear in \textit{Kalīla wa-Dimna} is certainly worthy of a separate study in the future. 

As far as the Old Spanish analytic causative constructions by means of \textit{fazer}, \textit{mandar}, and \textit{enbiar} are concerned, it must be mentioned that all three verbal constructions were already in use before the translation of \textit{Kalīla wa-Dimna} into Old Spanish in 1251, since all three of them appear in \textit{El Cantar de Mio Çid}, the Spanish national epic, often considered to be the oldest documented Spanish literary work, despite the discrepancies regarding the dating of the only existing manuscript and its chronological relationship to the actual original composition \citep[276–290]{montaner_frutos_cantar_2011}. However, it has to be emphasized that historical research regarding the innovative emergence of causative constructions out of the Latin heritage, a language without morphological \citep{lehmann_latin_2016} but with increasingly productive analytical causatives in Late Latin \citep[109]{hoffmann_lateinische_2018}, still leaves a lot to be researched and certainly deserves a dedicated in-depth study, considering the oldest vestiges of Proto-Romance and Castilian, which can be dated with accuracy. 

The situation is slightly better concerning the Middle Ages, but it is still more or less limited to the studies of \citet{alfonso_vega_verbos_2006}, \citet{davies_evolution_1995,davies_syntactic_2000}, and \citet{sanaphre_villanueva_analytic_2010}, with only the latter providing reliable statistical data for \textit{fazer}, \textit{mandar}, and \textit{enbiar}, which I summarize in \autoref{doehla:tab:4}.

\begin{table}
    \begin{tabularx}{\textwidth}{L{3cm}QQQQ}\midrule\toprule
        & 13th c.         & 14th c.         & 15th c.         & 16th c.         \\\midrule
        corpus   size         & 257,222   words & 257,355   words & 257,631   words & 258,200   words \\\midrule
        \multicolumn{5}{c}{\textbf{\textit{mandar}}}                                                                    \\\midrule
        total instances     & 521             & 363             & 495             & 457             \\
        \textbf{causative instances} & \textbf{389   (75\%)}& \textbf{273   (75\%)}    & \textbf{331   (67\%)}    & \textbf{255   (55\%)}    \\
        \textit{mandar} \Inf{}            & 292             & 147             & 223             & 149             \\
        \textit{mandar que} \CLAUSE{}     & 97              & 126             & 105             & 106             \\
        \textit{mandar a} \Inf{}          & ‒               & ‒               & 3               & ‒               \\\midrule
        \multicolumn{5}{c}{\textbf{\textit{fazer}}}                                                                     \\\midrule
        total   instances     & 2,809           & 3,053           & 1,449           & 1,792           \\
        \textbf{causative   instances} & \textbf{357   (13\%)}    & \textbf{244   (8\%)}     & \textbf{221   (15\%)}    & \textbf{400   (22\%)}    \\
        \textit{fazer} \Inf{}             & 344             & 239             & 210             & 371             \\
        \textit{fazer que} \CLAUSE{}      & 13              & 5               & 11              & 29              \\\midrule
        \multicolumn{5}{c}{\textit{\textbf{enbiar}}}                                                                    \\\midrule
        total   instances     & 237             & 277             & 182             & 140             \\
        \textbf{causative   instances} & \textbf{61   (26\%)}     & \textbf{84   (30\%)}     & \textbf{39   (20\%)}     & \textbf{42   (30\%)}     \\
        \textit{enbiar} \Inf{}            & 35              & 59              & 2               & 19              \\
        \textit{enbiar que} \CLAUSE{}     & 11              & 4               & 1               & 6               \\
        \textit{enbiar a} \Inf{}          & 15              & 21              & 34              & 17              \\
        \textit{enbiar a que}   \CLAUSE{} & ‒               & ‒               & 2               & ‒\\ \bottomrule\midrule
    \end{tabularx}
    \caption{Frequencies of \textit{mandar}, \textit{fazer}, and \textit{enbiar} analytic causative constructions from the thirteenth to the sixteenth century. The absolute numbers are taken from \citet[88 and 148]{sanaphre_villanueva_analytic_2010}.}
    \label{doehla:tab:4}
\end{table}


As can be seen in \autoref{doehla:tab:4}, analytic causative constructions with \textit{mandar} and \textit{fazer} are far more numerous than those with \textit{enbiar}. The fact that \textit{mandar} exhibits higher absolute frequencies than \textit{fazer} from the thirteenth to the fifteenth century could be linked to the respective text genres and discourse traditions that were used in the corpus, like chronicles and medieval epic poems, where constructions with \textit{mandar} (and even \textit{enbiar}) are expected to be quite numerous. Interestingly, \citet[4]{sanaphre_villanueva_analytic_2010} did not consider any legal works that make frequent use of causative constructions. In fact, the impact of linguistic structures of the judicial discourse tradition on the development of the Spanish language has, so far, not been analyzed thoroughly. In the same way, the linguistic properties and traditions of the royal scriptorium of Alfonso X and its influence on contemporary and later forms of Old Spanish have never been studied systematically. This royal scriptorium not only comprises the translated texts, the linguistic structures of which are often motivated by the Arabic model, but also the bulk of legal and historical texts of the thirteenth century. 

Of course, the analytic causative structures found in the Old Spanish language of \textit{Calila e Dimna} are taken from the Romance repertoire the translator(s) were able to make use of. However, the Arabic model served as source and multiplier, so that the use of \textit{enbiar}, \textit{fazer}, and \textit{mandar} increased in numbers in the given texts, themselves serving as models and sources for other Old Spanish works of the fourteenth and fifteenth centuries.\footnote{Such as \textit{Calila e Dimna} for \textit{El conde Lucanor} by Don Juan Manuel (14th century). Both works are taken into consideration by \citet{sanaphre_villanueva_analytic_2010}.} Nevertheless, an overall contrastive study of causative constructions in \textit{Calila e Dimna} in particular and in other translated texts in general is still a desideratum.

\section{Conclusion}\label{dohla:sec:6}
As I have shown throughout this preliminary study, this special type of learned language contact scenario, namely the translation of Arabic works into Old Spanish, offers quite a range of possibilities for the detailed study of morphosyntactical and semantic structures in the target language Old Spanish. The analyses confirm that translation is an act of negotiation \citep{eco_mouse_2004} and compromise \citep[169]{hasler_ubersetzung_2001} and is influenced deliberately where the given repertoire of the target language permits such contact-induced structures. On other occasions (e.g., \citealt{dohla_contact-induced_2022}), I have spoken of the preexistence of a general predisposition for some structures to be accepted, which leads to an increase in frequency of these patterns in language-contact situations or in translated texts. The analyses of the examples in Section \ref{dohla:sec:3} and Section \ref{dohla:sec:5} have demonstrated the creative potential of the translator(s) as well as their limits. 

At the same time, I have indicated throughout this chapter where there are still desiderata for future research. In this respect, this preliminary study serves as a starting point for the establishment of a large-scale project with the goal of an Arabic–Old Spanish historical contrastive grammar.


\section*{Abbreviations}
\begin{tabularx}{.45\textwidth}{lQ}
JUS & jussive		\\
ACT & active		\\
\end{tabularx}
\begin{tabularx}{.45\textwidth}{lQ}
PN & proper name		\\
PREP & preposition
\end{tabularx}

%\printglossaries
\printbibliography[heading=subbibliography, notkeyword=this]

\end{document}
