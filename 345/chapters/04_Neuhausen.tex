\documentclass[output=paper]{langscibook}
\ChapterDOI{10.5281/zenodo.10497375}
  
\author{Miriam Neuhausen \orcid{0009-0007-0260-531X} \affiliation{University of Freiburg}}

\title{Language change and stance in a remote Mennonite community in Canada}

\abstract{
    This study investigates Canadian Raising patterns in the Pennsylvania German English speech of nine Old Order Mennonites in southern Ontario with a focus on linguistic context and stance-taking. Although this community is traditionally isolated and resistant to change, some speakers are now increasingly in contact with the local English-speaking community and have begun to participate in the ongoing shift towards Canadian Raising, a process that is largely complete in the wider Canadian English speech community. The extent to which linguistic resources are available to the speakers and used for social work, such as stances, hinges on the individual degree of contact with the English community. The speakers deal differently with the new vowel; /aʊ/-raising, imbued with the social meaning of Canadianness, might be adopted or avoided, while /aɪ/-raising, below the speakers’ social awareness, indicates contact with the linguistic resources. The context of this study is perfectly suited for understanding local and translocal dynamics in a dialect contact situation, particularly in relation to stylistic practices in Third Wave sociolinguistics.
}

\IfFileExists{../localcommands.tex}{
  \addbibresource{../localbibliography.bib}
  % add all extra packages you need to load to this file

\usepackage{tabularx,multicol}
\usepackage{url}
\urlstyle{same}

\usepackage{listings}
\lstset{basicstyle=\ttfamily,tabsize=2,breaklines=true}

\usepackage{langsci-basic}
\usepackage{langsci-optional}
\usepackage{langsci-lgr}
\usepackage{langsci-osl}
% \usepackage{./langsci/styles/langsci-lgr}
% \usepackage{./langsci/styles/langsci-osl}
% \usepackage{langsci-gb4e}

\usepackage{tikz}
\usetikzlibrary{patterns,calc}
\pgfdeclarepatternformonly{south east lines}{\pgfqpoint{-0pt}{-0pt}}{\pgfqpoint{3pt}{3pt}}{\pgfqpoint{3pt}{3pt}}{
    \pgfsetlinewidth{0.6pt}
    \pgfpathmoveto{\pgfqpoint{0pt}{3pt}}
    \pgfpathlineto{\pgfqpoint{3pt}{0pt}}
    \pgfpathmoveto{\pgfqpoint{.2pt}{-.2pt}}
    \pgfpathlineto{\pgfqpoint{-.2pt}{.2pt}}
    \pgfpathmoveto{\pgfqpoint{3.2pt}{2.8pt}}
    \pgfpathlineto{\pgfqpoint{2.8pt}{3.2pt}}
    \pgfusepath{stroke}}
    
\usepackage{stmaryrd}
\usepackage{wasysym}
\usepackage{multirow}
\usepackage{caption}
\usepackage{subcaption}
\usepackage{mathrsfs}
\usepackage{qtree}

\usepackage{linguex}


  %pminos do not split footnotes
% \interfootnotelinepenalty=10000 %Footnote in Laporte chapters has to be split SN


%\DeclareIndexNameFormat{default}{%
%\nameparts{#1}%
%\usebibmacro{index:name}%
%{\index[names]}%
%{\namepartfamily}%
%{\namepartgiveni}%
% {}% L1
% {}% L2
%{\namepartprefix}% generates spurious space L3
%{\namepartsuffix}% generates spurious space L4
%}

%  {\DeclareIndexNameFormat{default}{%
%     \usebibmacro{index:name}{\index[names]}{#1}{#3}{#5}{#7}}}

%\DeclareIndexNameFormat{default}{%
%  \usebibmacro{index:name}{\sindex[nom]}{#1}{#3}{#5}{#7}}

%\DeclareIndexNameFormat{default}{%
%  \usebibmacro{index:name}{\sindex[person]}{#1}{#3}{#5}{#7}}
%\DeclareIndexNameFormat{default}{%
%\nameparts{#1} \usebibmacro{index:name}{\sindex[person]]}{\namepartfamily}{‌​\namepartgiven}{\nam‌​epartprefix}{\namepa‌​rtsuffix}}

%\newcommand{\smiley}{:)}

%\renewbibmacro*{index:name}[5]{%
%\usebibmacro{index:entry}{#1}%
%{\iffieldundef{usera}{}{\thefield{usera}\actualoperator}\mkbibindexname{#2}{#3}{#4}{#5}}}

% \newcommand{\noop}[1]{}

%remove for final
%\overfullrule=1mm

\newcommand{\tobi}[2]}}
\renewcommand{\S}[1]{\tobi{#1}{\textsc{*}}}

% this volume references
% puts: [this volume]
% already defined: \citetv
%\newcommand{\citepv}[1]{(\citeauthor{#1} \citeyear*{#1} [this volume])}
\newcommand{\citealtv}[1]{\citeauthor{#1} \citeyear*{#1} [this volume]}

%parentheses around example number
\newcommand{\pref}[1]{(\ref{#1})}

% in-text examples

\newcommand{\lnex}[1]{\textit{#1}} %target lang word
\newcommand{\lnlit}[1]{(lit.: `#1')} %literal reading
\newcommand{\lnlat}[1]{(#1)} % latinization
\newcommand{\lntrans}[1]{`#1'} %translation
\newcommand{\lnexl}[2]%
{\lnex{#1}{} \lnlat{#2}} % ex with latinization
\newcommand{\lnexlat}[3]{\lnex{#1}{} \lnlat{#2}{} \lntrans{#3}} % ex with latinization and tranl.

%ch01
\newcommand{\co}[1]{\mbox{\textbf{#1}}}

%ch09

\newcommand{\cyrbulg}[1]{\begin{otherlanguage*}{bulgarian}#1\end{otherlanguage*}}


%ch10
\newcommand{\nlp}{{\small NLP}}
\newcommand{\mwe}{{\small MWE}}
\newcommand{\rae}{{\small RAE}}
\newcommand{\lvc}{{\small LVC}}
\newcommand{\pos}{{\small P}o{\small S}}
%\newcommand{\todo}[1]{ \textcolor{red}{#1} }

%\renewcommand{\labelenumi}{\theenumi}
%\ainamefmt{{vv}{ll}{, ff}{, jj}} % fullname

\newcommand{\biberror}[1]{{\color{red}#1}}

\newcommand{\osenovaitem}{--~} 
  %% hyphenation points for line breaks
%% Normally, automatic hyphenation in LaTeX is very good
%% If a word is mis-hyphenated, add it to this file
%%
%% add information to TeX file before \begin{document} with:
%% %% hyphenation points for line breaks
%% Normally, automatic hyphenation in LaTeX is very good
%% If a word is mis-hyphenated, add it to this file
%%
%% add information to TeX file before \begin{document} with:
%% %% hyphenation points for line breaks
%% Normally, automatic hyphenation in LaTeX is very good
%% If a word is mis-hyphenated, add it to this file
%%
%% add information to TeX file before \begin{document} with:
%% \include{localhyphenation}
\hyphenation{
    Beck-man
    Ngu-yen
    back-chan-nel
    back-chan-nels
    mo-not-o-nous
    ste-reo-typ-i-cal
}

\hyphenation{
    Beck-man
    Ngu-yen
    back-chan-nel
    back-chan-nels
    mo-not-o-nous
    ste-reo-typ-i-cal
}

\hyphenation{
    Beck-man
    Ngu-yen
    back-chan-nel
    back-chan-nels
    mo-not-o-nous
    ste-reo-typ-i-cal
}
 
  \togglepaper[4]%%chapternumber
}{}



%\makeglossaries
\newacronym{OOM}{OOM}{Old Order Mennonites}
\newacronym{PG}{PG}{Pennsylvania German}
\newacronym{PGE}{PGE}{Pennsylvania German English}

\begin{document}
\maketitle

\section{Introduction}
The \gls*{OOM} comprise a religious community of Swiss-German origin \citep[72]{burridge_throw_1998} who reject modern technology, including the internet, mobile phones, and cars \citep[39–41]{epp_mennonites_2012}. They have resisted not only social change but also linguistic change to an extent; unlike the majority of immigrant groups in North America, they have maintained \gls*{PG} as L1 for nearly the past 400 years \citep[203]{burridge_steel_2002}. Similarly, their L2 English has also been comparably resistant to the surrounding language changes and still displays old features of Canadian English, such as the lack of Canadian Raising. However, due to increasing exposure to the English-speaking community, some \glspl*{OOM} have begun to participate in the ongoing change towards Canadian Raising, a process that is already complete in the local mainstream English community. 
	

Research on language and identity has shown that identity plays a major role in linguistic variation \citep{eckert_linguistic_2000,johnstone_mobility_2006,nycz_stylistic_2018}. Playing with (context-dependent) social meaning, speakers can use linguistic variants to create locally meaningful identities \citep{podesva_phonation_2007}. In choosing between non-raised and raised vowels, \glspl*{OOM} can create Pennsylvania German identities that linguistically distance them from local mainstream speakers or show linguistic integration with mainstream society. 

While linguistic research on \glspl*{OOM} in Canada is mostly restricted to \gls*{PG} \citep{richter_phonemic_1969,burridge_creating_1992}, the majority of studies conducted on \gls*{PGE} is based on communities in the US \citep{huffines_english_1984,huffines_intonation_1986,kopp_phonologie_1997,kopp_phonology_1999} – and only anecdotally in some cases \citep{springer_study_1980,shields_germanisms_1987}. Furthermore, the existing literature on both \gls*{PG} and \gls*{PGE} is fairly dated, with the only exception being \citegen{anderson_bidialectalism_2011} PhD thesis discussing dialect contact and salience in \gls*{PGE}.

Drawing on six sociolinguistic interviews with nine \glspl*{OOM}, I investigate the linguistic context of Canadian Raising, before exploring stance – both quantitatively and qualitatively – as a possible factor accounting for some of the observed linguistic variation. It should be noted that the nine speakers do not have the same amount of contact with English; as the language change of Canadian Raising is still in its incipient stages in the \gls*{OOM} community, a high degree of inter- and intra-speaker variation can be expected.


\section{The Old Order Mennonites}
In previous research, members of religiously conservative communities have been shown to produce sociophonetically different speech from their secular neighbours. For instance, a Mormon community in Alberta displayed less Canadian Raising \citep{meechan_mormon_1999} and /æ/-raising \citep{rosen_vowel_2015} than their immediate secular neighbours. And even within a Mormon community there may be differences: \citet{baker-smemoe_linguistic_2015} report significant linguistic differences based on how active Mormons in Utah were; inactive members fronted pre-nasal /ʌ/ significantly less than active members. 

Unlike the vast majority of immigrants, the \glspl*{OOM} have maintained their L1 \gls*{PG} since the early eighteenth century \citep[216]{draper_mennonites_2010}, when they left Europe for freedom of religion, which was promised in Pennsylvania \citep[131–132]{frantz_religion_2017}. Despite its Swiss German roots, \gls*{PG} is based on Palatinate German and shaped by its long-term close contact with English. The Anabaptists likely originally spoke Upper German and Swiss German \citep[317]{raith_relig_1996}. 
When they fled Zurich and were staying in the Palatinate, they shifted to Palatinate German within one generation (Gratz \& Geiser 1973 ctd. in \cite[317]{raith_relig_1996}).
The shift to Palatinate German thus took place before the migration to Pennsylvania and provided the basis for what would later become known as \gls*{PG} \citep[316]{raith_relig_1996}.
Today, members of the community are usually bilingual and acquire the two languages successively – \gls*{PG} at home as L1 and English in school as L2 \citep[85–86]{burridge_throw_1998}.

Distinct views prevail in the community concerning \gls*{PG} and \gls*{PGE}. Speakers frequently comment on linguistic variation in \gls*{PG}, for example the increase of English words, without evaluating it. By contrast, strong prescriptivism prevails concerning English \citep[85]{burridge_throw_1998}. For example, during fieldwork, I observed that the speakers were aware of the fact that their English deviated from their secular neighbours’ and even corrected each other’s usage of English, such as one speaker’s production of \textit{legacy} with [dʒ] instead of [g]. 

Even though the community attempts to sustain its isolation and restrict interactions with the secular world, numerous \glspl*{OOM} work in English-speaking domains, for example, local corner shops, quilt stores, and Canada’s biggest farmers’ market. With more than 200 vendors, the market is the largest year-round farmers’ market in Canada (see also their website, \url{https://stjacobsmarket.com/about-us/}, for more information). Situated in Waterloo County, it advertises itself as being “home to the largest population of Old Order Mennonites in Canada” and therefore attracting local \gls*{OOM} farmers who travel to the market by horse and buggy. Every day, numerous tourists from Toronto and beyond visit the market and buy traditional Mennonite-made goods, such as quilts and cheese curds. As a consequence, while some Old Order Mennonites are in regular contact with English-speaking locals and tourists, others work in \gls*{PG}-speaking domains, for example, farms and domestic work, and have barely contact with English.

\gls*{PGE} may be referred to as a religion-based ethnolinguistic repertoire (\citealt[110]{rosen_vowel_2015}; \citealt[142]{benor_mensch_2011}), since it is not only the cultural heritage that keeps the community together, but also – and in particular – religion, serving as a “source of ethnic regional differentiation” \citep[147]{frantz_religion_2017} from mainstream society. The notion of linguistic repertoire (as opposed to variety) implies that speakers of a group do not behave uniformly but choose linguistic variants both consciously and unconsciously to perform identities \citep{benor_mensch_2011}.

\section{Previous research}\label{neuhausen:sec:3}
\subsection{Canadian Raising}\label{neuhausen:sec:3.1}

Canadian Raising, first noted by \citet{joos_phonological_1942}, describes the raised onsets [ʌɪ] and [ʌʊ] of the vowels /aɪ/ and /aʊ/ prior to voiceless consonants and /t/-flaps \citep{dailey-ocain_canadian_1997,rosenfelder_canadian_2007,fruehwald_spread_2008}. Raising is, however, not always restricted to this environment and may also occur prior to voiced consonants (such as /r/) and nasals or word-finally. Examples of this are provided in case studies in Canada, as in Victoria, BC \citep{rosenfelder_canadian_2007}, but also in the US, as in Ann Arbor \citep{dailey-ocain_canadian_1997} and Philadelphia \citep{fruehwald_spread_2008}.

The earliest attestations of Canadian Raising in Ontario can be traced back to the 1880s (\citealt[148]{thomas_origin_1991}, \citealt[107]{chambers_canadian_2006}), 100 years after the first \gls*{OOM} settlers arrived there \citep[17]{epp_mennonites_2012}. Due to increasing contact with the English-speaking community, some \glspl*{OOM} now find themselves in the early stages of the language change and have begun to adopt Canadian Raising.

Canadian Raising is commonly linked to general Canadian English speech \citep{chambers_canadian_1989,niedzielski_effect_1999}, despite its attestations in the US \citep{labov_sociolinguistic_1972,roberts_internal_2016} and in the UK \citep{moore_natural_2018}. In particular the two lexical items \textit{out} and \textit{about} have become stereotypes to both American and Canadian speakers in the Labovian sense \citep[200]{labov_study_1971}. These two words are often produced with (phonetically inaccurate) hyper-raised nuclei: “oot and aboot” \citep[50]{nycz_changing_2013}. Investigating Canadian speakers who had moved to the New York City region, \citet[56–57]{nycz_changing_2013} found that half of the 15 speakers raised in all raising contexts, while the other half raised only the vowels in \textit{out} and \textit{about}. By contrast, regarding /aɪ/-raising, there seems to be much less social awareness \citep[76]{chambers_canadian_1989}. 

Concerning the speech of the \glspl*{OOM}, it can be expected that the two lexical items \textit{out} and \textit{about} pattern differently from other lexical items in general and from /aɪ/-raising in particular. Despite the differing degrees of contact with the mainstream Canadian community, all nine \glspl*{OOM} maintain close ties with settlements in the US through regular visits, letter correspondence and Old Order journals like the \textit{Brotherhood Journal}. Therefore, despite their restricted geographic mobility, it can be assumed that these speakers are aware of the two shibboleths \textit{out} and \textit{about}.

\subsection{Stance and identity}\label{neuhausen:sec:3.2}
When language variants become linked to social meaning, speakers can use them to create locally meaningful identities. Linguistic variants have multiple potential social meanings that are constantly negotiated within the context of language, interaction, and sociocultural values \citep[139]{du_bois_stance_2007}. Once a linguistic variant has gained social meaning and is no longer an \textit{indicator} but a \textit{marker} in \citegen[192–193]{labov_study_1971} sense, speakers can use it to do identity work. That way, a variant can become enregistered as indexing group membership \citep[94]{eckert_three_2012}. For example, in Pittsburgh, a set of linguistic features indicating socioeconomic class was first linked to place and then associated with a “Pittsburghese” dialect \citep{johnstone_mobility_2006}. Similarly, a white American boy at a California high school called Brand One draws on features of African American Vernacular English because they index (black) masculinity \citep{buchholtz_you_1999}. 

The \glspl*{OOM} have different linguistic means to index otherness in \gls*{PGE}: Firstly, features that are archaic in Canadian English can be used to index separation from the mainstream community and secondly, L1 transfer from \gls*{PG} may index membership in a local (\gls*{PG}-speaking) Mennonite community. Importantly, however, \citet[496]{podesva_phonation_2007} notes that the association evoked by a given linguistic variant remains “open to interpretation, on the part of both the linguist and [the interlocutor]”. In other words, it is not just a matter of the speaker interpreting a variable to use it for social work; their interlocutors (and the researchers) also need to be able to read and interpret it as such. 

Analysing stylistic practices, such as stance, from a Third Wave perspective (cf. \citealt{eckert_three_2012}) is essential to understanding how speakers use language to create social identities. Stance-taking occurs when speakers evaluate and position the object they are talking about and align themselves with regard to the object and listener \citep{du_bois_stance_2007}. Through the accumulation of these stance-taking acts, speakers create identities (Rauniomaa 2003 and Du Bois 2002 ctd. in \citealt[596]{buchholtz_identity_2005}) that are constantly negotiated and renegotiated. 

There is no one way of conceptualising stance. Some linguists examine affective and epistemic dimensions \citep{gadanidis_integrating_2021}, while others include alignment \citep{barnes_copula_2018} or explore investment, interlocutor positioning, and voicing \citep{bohmann_stance_2021}. Including alignment, affective stance, and topic, \citet{nycz_stylistic_2018} investigated Canadian Raising in the speech of mobile Canadian speakers residing in Washington, DC. She detected more raising, associated with Canada, when ambivalence or emotional distance from the US was expressed and less raising when closeness or positive affect was shown. Following \citegen{nycz_stylistic_2018} definition of stance, I evaluate its usefulness for the analysis of Canadian Raising in the speech of nine \glspl*{OOM}.

\section{Methodology}\label{neuhausen:sec:4}
\subsection{Procedure}\label{neuhausen:sec:4.1}

I conducted sociolinguistic interviews with \glspl*{OOM} based in Waterloo County, Ontario, between 2018 and 2019. The fieldwork took place within the framework of Sali Tagliamonte’s Ontario Dialects Project (\citeyear{tagliamonte_linguistic_2003,tagliamonte_directions_2007,tagliamonte_transmission_2010}; cf. \url{http://ontariodialects.chass.utoronto.ca}), which documents linguistic variation in English across Ontario.


As a cultural outsider, I entered the community as a “friend of a friend” \citep[53]{milroy_language_1980}. Residing outside the community, I spent five months with the community, familiarising myself with the culture and the languages. Even though we found common ground thanks to my German-European background, I represented a cultural outsider at all times.

Interviewing members of the \gls*{OOM} community who openly reject modern technology proved quite difficult at times, particularly as many \glspl*{OOM} felt extremely uncomfortable around the recording device. It was impossible to eliminate the “observer’s paradox” \citep[113]{labov_principles_1972}, but the range of informal topics resulted in comparably casual and emotionally coloured speech. Being the interviewer in all speech situations, I should note that I generally do not raise the diphthongs /aɪ/ and /aʊ/, which may or may not have affected the participants during the interview situation. 

I used a Roland R-09 recording device and its integrated stereo microphone for the interviews. Recorded data were digitised at the sampling rate of 44.1 kHz and submitted for acoustic analysis. For the acoustic analysis, I measured the height of the vowel onset, i.e. the F\textsubscript{1} value at 20 per cent duration. Future work should include tracking the entire vowel trajectory, as done by \citet{moore_natural_2018}. Fronting was not included in the analysis as an initial exploration of the data did not indicate any changes in F\textsubscript{2}. 

Preparing the data for quantitative analysis, I took the following steps: Where necessary, I removed noise in Audacity \citep{ash_audacity_2015} before normalising the sound.\footnote{Normalising sound in Audacity does not change the sound quality but amplifies sound without cutting off sound waves.} I segmented and transcribed the interviews in ELAN \citep{max_planck_institute_for_psycholinguistics_elan_2019}, before force-aligning and extracting the vowels in FAVE \citep{rosenfelder_fave_2014}. I manually checked approximately 15 per cent of the relevant contexts for accuracy in the force-aligned PRAAT script. Where necessary, I modified the alignment and re-ran the vowel extraction. After the removal of unstressed vowels in R \citep{r_core_2020}, I removed tokens shorter than 80ms ($\sim$ 8.65\%, \textit{n} = 67 for /aʊ/ and $\sim$ 11.3\%, \textit{n} = 301 for /aɪ/; see also \citealt[55]{nycz_changing_2013}). As a next step, I removed outliers beyond three standard deviations for each vowel and formant individually ($\sim$ 1.13\%, \textit{n} = 8 for /aʊ/, and $\sim$ 1.81\%, \textit{n} = 44 for /aɪ/). After this, I was left with a data set of 700 /aʊ/ tokens and 2,378 /aɪ/ tokens for the quantitative analysis. I then normalised the vowels based on the vowel-extrinsic and speaker-intrinsic Lobanov method in R.\footnote{For more information on the recommended order of operations in sociophonetic analysis for the purpose of comparability, please consider \citet{stanley_order_2022}.} 

For the quantitative analysis, I manually coded for /t/-flapping and for stance; /t/-flapping was coded both acoustically and auditorily and stance was defined by topic and alignment (cf. \autoref{neuhausen:sec:4.3}). I then matched the codes with the measured vowels using Python \citep{python_software_foundation_python_2019}. In R, I categorised following context into raising contexts (voiceless sounds and /t/-flaps), non-raising contexts (voiced sounds and pauses), and all cases of \textit{out} and \textit{about} (cf. \citealt[115]{chambers_canadian_1973,moreton_origins_2007,sadlier-brown_homogeneity_2012,nycz_changing_2013}). Preceding segments were grouped into the different manners of articulation and word-initial contexts. Syllable type was manually coded per word as heterosyllabic or tautosyllabic. For the qualitative analysis, raising was determined auditorily. For reasons of simplicity, the auditory analysis is binary (as opposed to the quantitative approach).

\subsection{Speakers}\label{neuhausen:sec:4.2}
This data set comprises nine speakers in six interview situations, totalling 8.25 hours of speech. Each of these interviews covers 1.25 to 1.75 hours of speech, of which approximately three quarters represent \gls*{PGE} and one quarter \gls*{PG}. As the analysis is restricted to snippets of these speakers’ speech, patterns emerging from the acoustic analysis should be considered suggestive rather than representative of general patterns in the \gls*{OOM} community. 

All nine speakers grew up in the Township of Woolwich or in Wellington County and have spent their entire lives in the area. The speakers constitute a homogeneous group in terms of the following variables:
\begin{itemize}
    \item Religious affiliation: baptised Old Order Mennonites
    \item Ethnicity: white European Canadians
    \item L1: \gls*{PG} acquired at home
    \item L2: \gls*{PGE} acquired in school
    \item Education: eight years of either public or parochial school\footnote{The parochial school system was established in the 1960s \citep[41]{epp_mennonites_2012} and reinforced social and linguistic isolation. Prior to that, children from the \gls*{OOM} community went to public school taught by non-members and attended by ‘English’ children, i.e. children growing up in mainstream Canadian society. The types of school are not statistically significant in this analysis, but future research could explore the role of school in linguistic variation further.} 
    \item Non-mobility: transportation modes restricted to horse and buggy rides and the occasional ride in a car driven by a non-member
\end{itemize}


Language competence was not measured, as all speakers are fluent in both languages and can switch effortlessly between the languages. Two speakers are in their 20s (Chloe and Leah), four speakers are in their 40s and 50s (Ada, Elisa, Naomi, and Rachel), and the remaining three speakers in their 60s (Isaac, Katie, and Phoebe).\footnote{All names were anonymised.} The current data set covers eight women and one man. It was more difficult for me to recruit men as interview partners because they tend to have less contact with cultural outsiders – their lives are often restricted to \gls*{PG}-speaking domains. Another reason is gender segregation in the community; a young woman would easier interview other women than men. 

All speakers in this data set are somewhat open to cultural outsiders as they agreed to being recorded. In a community that openly objects to modern technology, this already presupposes some openness towards and trust in cultural outsiders. I did not record speakers that were completely shielded from outside communities. Based on my impressionistic observation, these speakers may not raise at all; however, it does not seem likely that linguists will gain access to these speakers in the near future. 

Despite their homogeneous sociodemographic background, the \glspl*{OOM} – as any other community – are a highly heterogeneous group of speakers. While the language change of Canadian Raising is already complete in the wider Canadian population, the \glspl*{OOM} have only begun to participate in the process. Thus, the nine \glspl*{OOM} display great inter-speaker variation which can be observed in \autoref{neuhausen:fig:1}. The x-axis visualises the degree of raising, where lower numbers indicate more raising. Ada and Naomi feature the highest degree of raising, while Leah, Isaac, and Phoebe display the least. Both Ada and Naomi are independent, unmarried women and work in the English-speaking domain, while the three speakers with the smallest degree of raising – Isaac, Leah, and Phoebe – work in the \gls*{PG}-speaking space. 

The y-axis visualises the speakers’ respective standard deviation of raising and therefore shows how much speakers vary in their raising behaviour. Higher numbers indicate greater variation. The graph illustrates that Naomi, Ada, and Chloe, plotted at the top, are among the speakers who shift the most between raising and non-raising. Strikingly, all three women are surrounded by English speakers in their everyday lives. By contrast, Isaac – the only male speaker in this data set – appears to follow a consistent pattern, not raising much in the first place and hardly varying this behaviour; he may thus avoid the innovative variant or not have full access to it (yet). The remaining speakers, Katie, Rachel, Elisa, Phoebe, and Leah, behave similarly in terms of how much they vary between their raised and non-raised variants.

\begin{figure}
    \centering
    \includegraphics[width=0.6\textwidth]{Neuhausen_Figure1.png}
    \caption{Inter-speaker variation of the onsets of  /aɪ/ and /aʊ/ is displayed on the x-axis and intra-speaker variation on the y-axis. Higher numbers on the x-axis represent a lower degree of raising, i.e. speakers on the right raise less than speakers on the left. Higher numbers on the y-axis indicate more intra-speaker variation, i.e. speakers on the top vary greatly in their raising behaviour, while speakers on the bottom do not vary much.}
    \label{neuhausen:fig:1}
\end{figure}

\subsection{Stance}\label{neuhausen:sec:4.3}
The coding process was text- and content-based; any influence of phonetic context was avoided. Following \citet{nycz_stylistic_2018}, I coded for topic and affective stance/ alignment. The different codes are explained and examples of each are provided in the following. 

Topic was divided into \gls*{OOM}-specific topics, mainstream topics, and neutral topics.\footnote{If a stretch of speech represented multiple topics, I chose the one that was most relevant for the speaker’s overall statement.} The first include narratives and descriptions of community-specific norms and customs, like the concept of a maid in \xref{neuhausen:ex:1}. Maids are young girls who help other young families with child care and domestic work before they are allowed to work for wages at the age of 16. Mainstream topics cover subjects related to the (English-speaking) secular mainstream community, as in \xref{neuhausen:ex:2}. Neutral was coded for when neither of the former two applied, as in \xref{neuhausen:ex:3}.

\begin{exe}
    \ex\label{neuhausen:ex:1} My sister went out and helped another family, uhm, if they maybe have a whole bunch of small kids, that’s what we usually do our first couple of years, ’til we’re old enough to work away, ’til sixteen or whatever. (Rachel, \gls*{OOM}).\footnote{To increase legibility, I inserted commas and full stops for pauses. Words indicated in single quotation marks in square brackets represent a translation; an ellipsis inserted in square brackets indicates missing parts of speech. Extracts may be shortened and restricted to relevant context.}

    \ex\label{neuhausen:ex:2} It is interesting with the people we meet from all over the world, like you. They are for the history that we didn’t get. (Katie, mainstream)

    \ex\label{neuhausen:ex:3} The teacher didn’t like when we whispered, like we always, like sometimes was when she’d catch us but then she would get us to write out: “Let’s not whisper! Let’s not whisper!” (Naomi, neutral)
\end{exe}


Speakers may express closeness to or distance from a given topic, which is captured by the alignment coding. Following \citet{nycz_stylistic_2018}, I coded for both alignment and affect, which resulted in the following levels: aligned, non-aligned, positive, negative or neutral. The aligned stance expresses both solidarity and closeness, for example, expressions of belonging or fitting in, as in \xref{neuhausen:ex:4}. Non-aligned is used for feelings of not belonging to or not agreeing with a community, as in \xref{neuhausen:ex:5}. Here, Chloe distances herself from another community, another branch of Mennonites, by explaining that they do not share the same variety of German. Positive and negative affect covers the evaluation of a described object. An example of positive affect is provided in \xref{neuhausen:ex:6} and of negative affect in \xref{neuhausen:ex:7}. Neutral alignment, illustrated in \xref{neuhausen:ex:8}, represents stretches of speech where no explicit affective or alignment stance was used.

\begin{exe}
    \ex\label{neuhausen:ex:4} Although it’s still a very pleasant life in our thinking, is a great opportunity to raise a family in a setting where they’ve got something to do. (Isaac, aligned)


    \ex\label{neuhausen:ex:5} Or there’s Mexican Mennonites, […] their German is different from ours. (Chloe, non-aligned)

    \ex\label{neuhausen:ex:6} They’d know that’s the icing on the cake, so that used to be fun to gather eggs and then feed them! (Phoebe, positive)
    
    \ex\label{neuhausen:ex:7} And grammar was my worst subject. I never understood why we had to analyse sentences. I did not see any sense in that, I didn’t understand it. […] I quit analysing sentences when I quit school. (Isaac, negative)

    \ex\label{neuhausen:ex:8} A few times that we didn’t go to school that I remember of, like where it snowed so bad that– or it was extremely cold or something that they cancelled school. (Leah, neutral)
\end{exe}


The data set covers 475 aligned codes, 463 non-aligned codes, 225 positive codes, 35 negative codes, and 1,880 neutral codes. Due to the comparably low number of negative alignment levels, they were excluded from the statistical analysis. After the removal of outliers and the negative alignment codes, I was left with 1,085 vowels coded for \gls*{OOM} topics, 481 for mainstream topics, and 1,477 vowels for neutral topics. Translated into vowels, this left me with 694 tokens of /aʊ/ and 2,349 tokens of /aɪ/. 

For the analysis of topic and alignment, contact with English needs to be assessed for every speaker individually. As the \gls*{OOM} community finds itself in a long-term contact situation with ongoing linguistic changes, it is vital to understand to what extent individual speakers have contact with English and are open to cultural outsiders. Scholars have addressed the issue in different ways. \citet[150–151]{hazen_identity_2000}, for example, draws a line between local- and expanded-identity speakers. Local-identity speakers are oriented towards the core of the community, while expanded-identity speakers are oriented towards other communities; this orientation shows both socially and linguistically. In the case of the \glspl*{OOM}, all speakers have somewhat extended identities, as they allowed me to record them and are thus necessarily open towards people beyond the community lines. \citet{baker-smemoe_linguistic_2015} distinguish between active and non-active members in a Mormon community, which shows linguistically. Active members, as opposed to inactive members, self-identified as participating at least weekly or nearly weekly in organised religious activities. This distinction is not useful for the present community, as all \glspl*{OOM} are active members. It becomes clear that the issue of which measures to use strongly hinges on local context. In lesser-researched communities, such relevant community-specific social factors may only emerge during ethnographic fieldwork (cf. \citealt{stanford_clan_2009,neuhausen_understanding_2023}) and can only then be used to accurately define what contact with another language (or community) means in a given community. 

I developed a community-specific scale attempting to quantify the speakers’ individual degree of contact with English and openness to non-members. The following categories serve to help measure it:
\begin{itemize}
    \item using English at least sometimes at the work place,
    \item working part-time/full-time in an English domain,
    \item having contact with English family members/neighbours/friends outside of work,
    \item seeking contact with cultural outsiders, and
    \item being an unmarried woman (who often leave the \gls*{PG}-speaking home farms to live in apartments in the English neighbourhood and/or who may run their own businesses with English customers).
\end{itemize}


Based on these categories, I calculate the individual degrees of contact with English as follows: The starting point for each speaker is score 0 and represents almost no contact with English; the higher the score, the more contact with English a speaker is predicted to have. For every category that pertains to a given speaker, a score of +1 is added. This measure is then transformed into categorical variables ranging from "(almost) no" contact to "regular" contact with English. According to this measure, Elisa and Leah have the least contact with English; Rachel, Isaac, Katie, and Chloe are grouped as speakers with some contact with English; Phoebe and Ada represent speakers with moderate contact; and Naomi is the speaker with the highest score, with regular contact with English (see also \autoref{neuhausen:fig:2}). It is noteworthy that here, Naomi’s status aligns with her vowels being the most strongly raised and most strongly varied, as can be seen in \autoref{neuhausen:fig:1}. Despite the measure pointing to some speakers having “(almost) no” contact with English, it is important to keep in mind that all speakers have some contact with English as they all acquired English as L2 and live in a community that uses English as lingua franca with outside communities.

\begin{figure}
    \centering
    \includegraphics[width=0.6\textwidth]{Neuhausen_Figure2_B.png}
    \caption{The categorised contact with English is visualised on the x-axis. Different colours indicate different speakers; the different width of each bar represents the relative frequency of vowels analysed per speaker.}
    \label{neuhausen:fig:2}
\end{figure}

\section{Statistical analysis}\label{neuhausen:sec:5}
The statistical analysis includes both linguistic context and stance. For the linear mixed-effects model, the F\textsubscript{1} value was treated as an independent variable, with linguistic context and stance as dependent variables. Speaker and word were incorporated as random effects and vowel was fitted as random slope for speaker. In order to investigate the linguistic context of Canadian Raising in the speech of the \glspl*{OOM}, the following language-internal variables were included in the model: vowel, duration, following voicing context, preceding manner of articulation, and syllable type. Phonetic stress and age were not significant and not included in the model.

\hspace*{-2.8pt}Concerning language-external factors, stance can be expected to strongly hinge on the degree of contact with English as the \glspl*{OOM} use English to varying degrees. Thus, I examined the interaction of contact with English and alignment and topic.

All categorical variables were sumcoded. The reference levels of the sumcoded variables were set to word-initial contexts for preceding manner of articulation and non-raising contexts for following voicing contexts, i.e. word-final and voiced contexts. Neutral codes were used as reference levels for alignment and topic and the least contact with English was used as reference level for contact with English.

Multicollinearity and heteroscedasticity were both tested for, using the performance package \citep{ludecke_performance_2021}. They are not a concern; for the same model without the interaction, the highest corrected VIF scores are at 1.95 for alignment and 1.94 for topic (see also \citealt[160]{levshina_how_2015}). In the following, the results of the model are briefly discussed and visualised in form of predictions. The predictions based on the statistical model were calculated using the ggeffects \citep{ludecke_ggeffects_2018} and the lme4 packages \citep{bates_fitting_2015}. The p-values were calculated using the broom.mixed library \citep{bolker_robinson_2022}. The calculated R$^2$ value for the model (from the performance package) predicts that 30.3\% of variance can be accounted for by the model.


%\begin{table}[H]
%%  \caption{The mixed-effects model treats raising as an independent variable; linguistic context and stance – i.e. the interaction of contact, alignment, and topic – are treated as fixed effects. Speaker and word were incorporated as mixed effects, with vowel as random slope for speaker. Only significant interactions of stance were included in \autoref{neuhausen:tab:1}. The table was created using the gt package \citep{iannone_gt_2022}.}
%           \includegraphics[width=\textwidth]{Neuhausen_Table1.png}
%
%%    \label{neuhausen:tab:1}
%\end{table}

  \renewcommand{\arraystretch}{1.2}
\setlength{\tabcolsep}{1.2pt}
\begin{table}
    \caption{The mixed-effects model treats raising as an independent variable; linguistic context and stance – i.e. the interaction of contact, alignment, and topic – are treated as fixed effects. Speaker and word were incorporated as mixed effects, with vowel as random slope for speaker. Lower estimates displayed in \autoref{neuhausen:tab:1} indicate more raising. Only significant interactions of stance were included in \autoref{neuhausen:tab:1}.}
    \label{neuhausen:tab:1}
    %\includegraphics[width=\textwidth]{Neuhausen_Table1.png}
{\fontsize{6.7pt}{8.04pt}\selectfont
  \begin{tabularx}{\textwidth}{C r r rr rr r}\midrule\toprule
  \multicolumn{8}{c}{\shortstack{\footnotesize{Mixed model predicting Canadian Raising in the speech of Old Order Mennonites}\\ \tiny{ \textit{N=} 3,043 (\textit{n=}694 for  /aʊ/, \textit{n=}2,349 for  /aɪ/) }} }\\
  \midrule
%%%%  &&&&&\multicolumn{2}{l}{\shortstack[c]{95\% confidence interval}}&\\
%%%%  \cline{6-7}\\
%%%%  term &estimate&  \shortstack[r]{standard error}      & t-value &   p-value&  
%%%% \shortstack[l]{lower bound} & \shortstack[l]{upper bound}& signif.\\
  &&&&&\multicolumn{2}{c}{95\% confidence interval}&\\
  \cline{6-7}
  term &estimate& standard error    & t-value &   p-value&  lower bound & upper bound & signif.\\
  
  \midrule
(intercept)  		&0.1387& 0.2135& 0.6495& 0.5162&-0.2804&0.5577&\\
Vowel: /aʊ/    	&0.2082& 0.0403& 5.1623& 0.0000& 0.1265&0.2898&***\\
Duration [in s]	&3.5748& 0.2612&13.6859&0.0000& 3.0626&4.0870&***\\
<out>/<about>         &-0.3742& 0.1373& -2.7260&0.0098&-0.6524&-0.0960& **\\
Raising context (voiceless \&  /t/-flaps)   &0.1359&0.0735 &1.8497&0.0707&-0.0119&0.2836&\\
Prec. plosive        &-0.0633&0.0515&-1.2290&0.2197&-0.1645&0.0379&\\        
Prec.  fricative        &0.2029&0.0760&2.6717&0.0080&0.0534&0.3525&**\\
Prec.  vowel       &0.0781&0.0773&1.0112&0.3120&-0.0734&0.2296&\\
Prec. nasal        &-0.2537&0.0599&-4.2359&0.0000&-0.3714&-0.1360&***\\        
Prec. rhotic        &0.0634&0.0732&0.8654&0.3874&-0.0806&0.2073&\\
Prec. sibilant      &-0.2353&0.0727&-3.2383&0.0013&-0.3779&-0.0972&**\\
Prec. lateral      &-0.3819&0.0889&-4.2949&0.0000&-0.5567&-0.2071&***\\
Prec. approximant       &0.0399&0.1246&0.3200&0.7494&-0.2064&0.2862&\\
Prec. affricate        &0.0487&0.2362&0.2064&0.8366&-0.4161&0.5136&\\
Tautosyllabic syllable       &-0.0968&0.0344&-2.8183&0.0055&-0.1647&-0.0290&**\\ 
{[Moderate contact - aligned stance] }      &0.7997&0.3104&2.5760&0.0100&0.1910&1.4083&**\\
{[Aligned stance - mainstream topic]    }   &0.9271&0.4017&2.3079&0.0211&0.1394&1.7148&*\\
 Moderate contact - aligned stance -       mainstream topic      &-1.4124&0.4059&-3.4796&0.0005&-2.2083&-0.6165&***\\
{[Regular contact - aligned stance -       mainstream topic]       }
              &2.1791&0.8324&2.6177&0.0089&0.5468&3.8113&**\\
\midrule
\multicolumn{8}{>{\hsize=\textwidth}Q}{Random effects structure: Random intercepts for participant and 
word; by-participant random slopes for vowel}\\
                     \midrule \bottomrule
 \end{tabularx}    
 }
 
\end{table}


 In the following, the model will be interpreted based on the predicted values of F\textsubscript{1}. The predicted values of F\textsubscript{1} are adjusted for the following linguistic context: /aʊ/ vowels; at a duration of 0.16 seconds; in the items \textit{out} and \textit{about}; with preceding plosives; in tautosyllabic contexts. The adjusted values for the following social factors are: speakers with moderate contact to English; producing non-aligned stances; and mainstream topics. 



\largerpage
Regarding the vowel, less raising is predicted for /aʊ/ (predicted value for F\textsubscript{1} = 0.7) than for /aɪ/ (predicted value for F\textsubscript{1} = -0.4, p = 8.17e-6). This finding is interesting as /aɪ/ is arguably below the speakers’ conscious level, while /aʊ/ may indicate Canadianness. While the spread of /aɪ/-raising indicates that Canadian Raising has entered the community, the lack of /aʊ/-raising may suggest that speakers avoid the variant, which is socially imbued with Canadian identity outside their community (see also \autoref{neuhausen:fig:3}). In other words, /aɪ/ may be more systematic in both raising and non-raising contexts, while for /aʊ/-raising, speakers may feel the need to mark the distinction between the raised and non-raised variant of /aʊ/ more strongly as it is linked to the social meaning of Canadianness. It should be noted, however, that more than three thirds of the data set cover tokens of /aɪ/ (77.19\%, \textit{n} = 2,349).


\begin{figure}
    \includegraphics[width=0.6\textwidth]{Neuhausen_Figure3.png}
    \caption{The two vowels /aɪ/ and /aʊ/ are indicated on the x-axis and the predicted amount of raising is plotted on the y-axis, where lower numbers indicate more raising. /aɪ/, plotted at the top, is predicted to feature more raising than /aʊ/, plotted at the bottom. This may suggest that speakers avoid the variant that carries the social meaning of Canadian identity.}
    \label{neuhausen:fig:3}
\end{figure}


The predicted probability score of raising with a vowel duration of 0.5 seconds is 1.49 (see \autoref{neuhausen:fig:4}). It increases to 3.27 for a vowel duration of 1 second. In other words, the longer the duration, the less raising is predicted in these contexts (p = 2.21e-41). In American English, vowels tend to be longer when preceding voiced consonants \citep[119]{kendall_sociophonetics_2021}. This aligns with the present findings, where longer vowels, assumably prior to voiced contexts and pauses, are predicted to feature less raising (see also \autoref{neuhausen:fig:4}).


\begin{figure}
    \includegraphics[width=0.6\textwidth]{Neuhausen_Figure4.png}
    \caption{Duration (in seconds) is displayed on the x-axis, while predicted raising is indicated on the y-axis. Longer vowels are associated with less raising.}
    \label{neuhausen:fig:4}
\end{figure}



\largerpage
Following voicing context was grouped into raising and non-raising contexts. A third level was included describing all vowels in \textit{out} and \textit{about}. \citet{nycz_changing_2013} demonstrates that these words show different raising patterns in the speech of mobile Canadian speakers in the New York City region. 
The same pertains to the present data set: the words \textit{out} and \textit{about} are significantly more raised (0.27, p = 9.77e-3) and predicted to feature more raising than vowels in raising contexts (0.78, p = 7.07e-2) and in non-raising contexts (0.89; see also \autoref{neuhausen:fig:5}).
 This finding aligns with Nycz’ study who found that some mobile Canadian speakers constantly raised while others only raised the vowels in \textit{out} and \textit{about}.\footnote{Unlike in her data where \textit{out} and \textit{about} occurred in more than half of all occurrences, \textit{out} and \textit{about} only account for 31.56\% of the /aʊ/ data set.} Remember that /aʊ/ generally co-occurs with less raising than /aɪ/; yet, the two cultural shibboleths of \textit{out} and \textit{about} significantly co-occur with raising, which potentially indicates that the two words carry social meaning. This may also suggest that \textit{out} and \textit{about} accelerate the language change as shibboleths towards /aʊ/-raising or are selectively used for raising, while the remaining tokens of /aʊ/ lag behind.



\begin{figure}][p]
    \includegraphics[width=0.6\textwidth]{Neuhausen_Figure5.png}
    \caption{Following voicing context is displayed on the x-axis. By far the most raising is predicted for the words \textit{out} and \textit{about}. Vowels prior to voiceless contexts (including /t/-flaps), are linked to more raising than in non-raising contexts (following voiced sounds and pauses).}
    \label{neuhausen:fig:5}
\end{figure}


\begin{figure}[p]
    \includegraphics[width=0.9\textwidth]{Neuhausen_Figure6_B.png}
    \caption{The x-axis represents the preceding manner of articulation. From left to right: Vowels following laterals, nasals, and sibilants are predicted to be the most raised, followed by plosives, approximants, affricates, rhotics, and other vowels in that order. Vowels following fricatives and word boundaries are predicted to be raised the least.}
    \label{neuhausen:fig:6}
\end{figure}

\largerpage
In terms of preceding manner of articulation, laterals (-0.5, p = 2.19e-5), nasals (0.08, p = 2.71e-5), sibilants (0.10, p = 1.25e-3), and plosives (0.27, p = 2.20e-1) are predicted to favour raising in that order. Examples of preceding laterals are \textit{line} and \textit{realised}, of preceding nasals \textit{mind} and \textit{amount}, of preceding sibilants \textit{outside} and \textit{shout}, and of preceding plosives \textit{kind} and \textit{about}. These are followed by approximants (0.38, p = 7.49e-1, \textit{while} and \textit{otherwise}), affricates (0.39, p = 8.37e-1, \textit{child} and \textit{lunch hour}), rhotics (0.40, p = 3.87e-1, \textit{right} and \textit{around}), other vowels (0.42, p = 3.12e-1, \textit{so I} and \textit{go out}), and fricatives (0.54, p = 8.01e-3, \textit{find} and \textit{without}). The smallest degree of raising is predicted in word-initial vowels (\textit{I} and \textit{out}, see also \autoref{neuhausen:fig:6}). 

According to \citet[79]{chambers_canadian_1989}, tautosyllabic contexts favour Canadian Raising over heterosyllabic contexts. Investigating the distribution of raising across the two syllable types is worthwhile, as the community finds itself in the initial stages of the language change towards Canadian Raising and it cannot be assumed that all speakers in the present data set have access to the linguistic constraints. Supporting Chamber’s statement for the \gls*{OOM} community, in the present data set, vowels in tautosyllabic contexts (0.7, p = 5.47e-3) are predicted to occur more often as raised tokens than in the latter (0.47; see also \autoref{neuhausen:fig:7}). This finding may suggest that the \glspl*{OOM} indeed have access to (at least some of) the linguistic resources of Canadian Raising.

\begin{figure}
    \includegraphics[width=0.6\textwidth]{Neuhausen_Figure7.png}
    \caption{Heterosyllabic and tautosyllabic contexts are indicated on the x-axis. The graph supports previous literature on Canadian Raising, predicting raising to favor tautosyllabic contexts \citep[79]{chambers_canadian_1989}.}
    \label{neuhausen:fig:7}
\end{figure}



Stance, defined as the interaction of contact with English, topic, and alignment, represents the only social factor in the model. For the analysis, I only consider the significant interaction with the highest amount of interaction levels with three levels. As one of the two significant interactions (regular contact--aligned stance--mainstream topic) only features four tokens, I only focus on the other significant three-level interaction: moderate contact—aligned stance—mainstream topic (\textit{n} = 66). For speakers with moderate contact with English, the aligned stance is predicted to occur with more raising when covering mainstream topics (0.03, p = 5.10e-4, as opposed to 0.78, 0.55, and 1.26 for speakers with almost no contact, some contact, and regular contact in that order). This is also visualised in \autoref{neuhausen:fig:8}. It is an interesting finding, as it suggests that rhetorical closeness may entail linguistic closeness for speakers with moderate contact with English. In other words, speakers who rhetorically align with mainstream topics may also converge linguistically. In the following section, the significant interaction of aligned stance, mainstream topic, and moderate contact with English is assessed qualitatively in the speech of Phoebe.

\begin{figure}[t]
    \includegraphics[width=.9\textwidth]{Neuhausen_Figure8.png}
    \caption{The combination of moderate contact, aligned stance, and mainstream topic (n=66) is predicted to feature raised vowels.}
    \label{neuhausen:fig:8}
\end{figure}



\section{Phoebe}
Phoebe, in her sixties, is active in both \gls*{PG} and English domains, but spends significantly more time in the former, where she plays an important social role. In terms of raising degree and intra-speaker variation, her behaviour clusters with the other speakers (with the exception of Naomi and Ada, cf. \autoref{neuhausen:fig:1}). 

Having grown up with English-speaking family members, she likely has full access to the linguistic constraints of Canadian Raising and appears confident when speaking English. Phoebe expresses strong sentiments, showing solidarity with and distancing herself from concepts linked to the English-speaking community. In \xref{neuhausen:ex:9}, she emphasises that from a spiritual perspective, there is no difference between people, no matter what community they are affiliated with. In \xref{neuhausen:ex:10}, she distances herself from church services in the English-speaking mainstream community, where sermons are (necessarily) monolingual and restricted to English only. In her view, having only one language to pray in, the mainstream community misses out on some of the depth and meaningfulness that the German language can add in the bilingual context of the Mennonite church. Here, she distances herself from mainstream society and implicitly states that the Mennonites profit from bilingual sermons, as speakers can use the resources of more than one language to express their needs (and prayers).

\begin{exe}
    \ex\label{neuhausen:ex:9} So yeah, we’re all just people anyhow, regardless what colour hats we wear or dresses or cars or vehicles we drive or – […] we’re all just people.

    \ex\label{neuhausen:ex:10} There’s something missing, that added depth isn’t there because they’re confined to their use of their English vocabulary.
\end{exe}


In the following, two stretches of speech produced by Phoebe will be analysed in more depth. Both address mainstream topics and represent aligned stances. Raised vowels are indicated in italics, non-raised vowels are underlined. In utterance \xref{neuhausen:ex:11}, Phoebe describes how her younger brother Elo acquired English by spending time with his cousin Jacob. Growing up in an English-only environment outside the \gls*{OOM} community, Jacob brought English into the games with his \gls*{PG}-speaking cousin. Phoebe expresses her solidarity with her English cousins and simultaneously distances herself from them by describing them as “English”: “we have good memories of our \textit{English} cousins”. Solidarity is further expressed by her emphasis of how close the two boys were: Firstly, sharing the same birthday and year, they were called twins by their families; secondly, they spent a lot of time together (“they often came”); and thirdly, they were still children. Phoebe makes the point that children “get along well” even without speaking the same language. In the following stretch of speech, she expresses her adoration of how Elo started to learn English without effort—“without thinking”—thanks to his socialising with his English cousin. Here, Phoebe expresses admiration of how easily children adapt to new situations and how they profit from interacting with other (English) children. 

Linguistically, in terms of /aʊ/, she does not raise in non-raising contexts, such as \textit{our} and \textit{how}. In raising contexts, she only raises the vowel in the shibboleth \textit{out}, but not in \textit{without}, which represents a raising context. By contrast, for /aɪ/, all non-raising contexts feature non-raised variants, i.e. \textit{my} and \textit{time}, and the only raising context features a raised vowel, i.e. \textit{like}. This utterance may support what the quantitative findings suggest: /aʊ/ is not raised across the board but in \textit{out} and \textit{about}, while /aɪ/ may be consistently raised in raising contexts. Regarding /aɪ/, Phoebe conforms to the traditional raising contexts, potentially indicating that she has full access to the linguistic constraints.

\begin{exe}
    \ex\label{neuhausen:ex:11} We have good memories of \ul{our} English cousins and that definitely, \ul{my} youngest brother Elo had a twin with an English cousin and you know \ul{how} children are; they don’t need to know the same language to get along well, they just know \ul{by} their actions and \ul{by} the tone of voice. Uhm, they were both, he was four—they were both four, of course, ’cause they were twins—and after they left, they often came, but after they left this one \ul{time} when he was four, \ul{my} mother was making a cake—or was it, \ul{I’m} not sure who was making the cake—but, uhm, we told him to bring their wooden spoon, \textit{like} to stir it, and he got it \textit{out} of the drawer and he said: “This?” That was his first English word that we know that he said \ul{without} thinking, you know, it was just there because he had been interacting so much with Jacob, his cousin.
\end{exe}


In \xref{neuhausen:ex:12}, she describes how her mother always emphasised that there was no difference between people, whether they were Mennonites or not. Phoebe strongly agrees with her mother and adds that it may take longer for some people to figure this out, but they will do so eventually. Rhetorically, she aligns with members of mainstream society here as she does not make a difference between them and herself. 

In terms of /aʊ/-raising, the first two tokens of /aʊ/, i.e. \textit{out}, are not raised despite being in raising contexts. The third occurrence of \textit{out} is audibly just a little bit raised. Concerning /aɪ/-raising, non-raising contexts feature non-raised vowels, i.e. \textit{my}, \textit{line}, and \textit{find}, and raising contexts feature raised vowels, i.e. \textit{life}. The only exception is represented by \textit{why}, a raised vowel in a non-raising context.

\begin{exe}
    \ex\label{neuhausen:ex:12} And \ul{my} mother, she’s the one who always constantly was saying: “We’re all just people.” And you know what? The bottom \ul{line} is: we are all just people! Some people \ul{find} it \ul{out} sooner, some people \ul{find} it \ul{out} later in \textit{life} but eventually there will come a \ul{time} when you will \textit{find out} and discover first hand: we are all just people! \textit{Mir sin all jusht leit} [‘we are all just people’], no kidding! And she was good with that and she also said: “What happens to others can happen to us. \textit{Why} do we think we’re better than they are?” Mh. Amen. \textit{Sell is wahr}. [‘That is the truth.’]
\end{exe}


It is interesting that in this stretch of speech, in two out of three times, Phoebe does not raise the vowel in \textit{out}. Following the quantitative findings, one would expect to see a strong tendency for raised \textit{out} and \textit{about}. This may suggest that Phoebe deviates from a more general raising pattern for \textit{out} and \textit{about} by avoiding to raise in these contexts. As she is one of three speakers who have moderate or regular contact with English, this raises the question whether the other two speakers with a similar degree of contact behave similarly and whether increased contact with English can lead to the avoidance of raised \textit{out} and \textit{about}. If speakers are confident enough speaking English, they may not feel the urge to speak ‘proper’ English anymore, i.e. linguistically conform to their monolingual English neighbours, and use the raising of these vowels for identity work. Doing so, Phoebe may distance herself linguistically in these contexts. Moreover, while she conforms to the traditional raising contexts for /aɪ/ in \xref{neuhausen:ex:11}, she over-raises /aɪ/ in one instance in \xref{neuhausen:ex:12}. This may be a first indicator that she has access to the linguistic constraints of /aɪ/-raising but may manipulate both vowels /aʊ/ and /aɪ/ to make a statement, particularly as the over-raised vowel occurs in a rhetorical question summarising her statement. It should be noted that the qualitative analysis remains very superficial at this point and only provides a snapshot of the linguistic complexity that is going on in the data.

\section{Discussion}
According to the quantitative analysis, linguistic context mostly correlates with the expected raising behaviour. This applies to what is known about vowel duration and syllable type. Yet, the qualitative analysis also reveals that following voicing contexts do not consistently feature the raising pattern documented in previous studies. Maybe this cannot be expected in a contact situation with ongoing linguistic changes and speakers with varying degrees of contact with English. 

Across the board, the vowels in \textit{out} and \textit{about} behave significantly differently from voicing contexts in the statistical model. It seems likely that the shibboleths \textit{out} and \textit{about} are salient to all speakers subject to the analysis. The speakers maintain close ties with their American settlements and all of them are in contact with English to varying degrees (even the most isolated ones still live in a community that is surrounded by English). Potentially, \textit{out} and \textit{about} accelerate the language change towards /aʊ/-raising as shibboleths. However, as a snapshot of Phoebe’s speech illustrates, not everybody raises \textit{out} and \textit{about} across the board. This is particularly interesting because Phoebe has moderate contact with English and it is safe to assume that she is aware of the social meaning imbued on these two words. 

The quantitative analysis demonstrates that /aɪ/ is raised more than /aʊ/. The phenomenon of Canadian Raising provides a particularly interesting linguistic variable for the analysis of variation in a contact situation as the present speakers are likely aware of the social meaning attached to raised /aʌ/ (particularly in \textit{out} and \textit{about}), which can be used or avoided for identity work. 

Being able to use innovative linguistic forms for identity work hinges on sufficient contact with the new linguistic variant. Therefore, I attempted to develop a measure for the individual speaker contact with English. This is a highly commu\-nity-specific matter; in a different community, the same measure will be based on different local social factors. Ethnolinguistic fieldwork is vital for researchers to understand what categories may be socially relevant in a given community and can be used for that purpose. More research is required to develop solid measures that can potentially even be transferred to other multilingual/diaspora communities.

It may not come as a surprise that the interaction of contact and stance is only significant for speakers who are in regular contact with English. Only speakers with at least moderate contact with English may use the new vowel for identity work, while speakers with fewer contact points may still be in the process of acquiring the linguistic constraints. A possible explanation as to why the interaction is not significant for the only speaker with regular contact with English may be that she is already linguistically integrated in mainstream society and conforms to the traditional raising pattern in all alignments and topics.

The variable age was not significant in the quantitative analysis; this may be the case because nine speakers are not enough to provide a wide age range. In contexts with ongoing change, age should also be included in the measure of contact to assess ongoing change. As younger speakers are increasingly in touch with non-members, such as English customers, their way of speaking may (consciously or not) be converging towards the English spoken outside the community. Other (older) speakers, such as Phoebe, may avoid raised /aʊ/ for two reasons: either countering the ongoing movement towards raising the vowel in \textit{out} and \textit{about} or marking themselves as members of a different community.

\section{Conclusion}
The \glspl*{OOM} represent a community in a long-term contact situation with varying degrees of contact with the language changes happening around them. This paper describes an attempt to capture contact with English without relying on speaker surveys. 

All in all, the combination of both a quantitative approach and a qualitative analysis yields fascinating insights and raises more questions. According to the quantitative analysis, the analysed \glspl*{OOM} adopt Canadian Raising overall in the expected linguistic contexts but not to the same extent as previously documented. The \glspl*{OOM} raise mostly in shorter vowels and tautosyllabic syllables, but they raise the two shibboleths \textit{out} and \textit{about} significantly more than vowels in other raising contexts. The qualitative analysis, however, shows that linguistic context alone cannot fully explain individual raising patterns. 

The greater amount of overall variation in /aʊ/ may be explained through openness to social (and linguistic) change. The status of \textit{out} and \textit{about} as indexes of Canadianness is well-established across community boundaries, which makes these two items easily available to \gls*{OOM} speakers for a wide range of stance-taking acts. By variously avoiding or adopting it in the unfolding interviews, they can use the feature to position themselves with regard to topics and objects talked about. Depending on the speaker, the cumulative result of such stance-taking is to emphasise a traditional Mennonite identity or to represent selective identification with the Canadian mainstream. For the two speakers who are in moderate contact with English, the interaction of raising with topic and alignment is significant. The analysed snippets of Phoebe’s raising behaviour suggest that she does not simply use the lexical shibboleths idiosyncratically but rather avoids them and overraises in some /aɪ/ contexts.

Despite remaining suggestive, the findings of this paper shed light on the importance of qualitative research and understanding individual biographies. Due to the scope of this paper, the qualitative approach has remained superficial but already indicates interesting reasonings. Speakers seem to deal with the ongoing linguistic change individually. This has become apparent in Phoebe’s under-raising of \textit{out} and \textit{about}, the vowels in which, according to the quantitative analysis, are predicted to be raised. As the participation in ongoing language change very much hinges on contact with English, the speakers’ individual social contexts must be outlined and understood in order to paint the full picture and develop appropriate measures for the quantification of contact with English. 

Stance may not tell us everything, especially not in the initial stages of a language change. What it does, however, is provide us with a fine-grained picture that enables us to study how speakers use a newly available linguistic resource to position and align themselves with regard to topics and listeners. Such stance-taking is first and foremost part of their individual efforts to express identity through language, but in the long run it will lead to the emergence of more consolidated ethnolinguistic repertoires for the community.


\section*{Abbreviations}
\begin{tabularx}{\textwidth}{ll@{\qquad}ll}
OOM& Old Order Mennonites & PGE&Pennsylvania German English \\
PG&Pennsylvania German\\
\end{tabularx}


\section*{Acknowledgements}
I would like to thank the Mennonite community and the Elmira locals for their heart-warming hospitality and their willingness to share their time and life stories with me as well as letting me become part of their world. The field trip was made possible by a DAAD scholarship and the generous support of Christian Mair and Sali Tagliamonte. Last but not least, I would like to thank the two anonymous reviewers for their productive feedback.



\printbibliography[heading=subbibliography, notkeyword=this]


\end{document}
