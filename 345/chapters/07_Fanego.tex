\documentclass[output=paper]{langscibook}
\ChapterDOI{10.5281/zenodo.10497381}

\author{Klaudia Dombrowsky-Hahn\orcid{ 0000-0003-0600-2018}\affiliation{University of Bayreuth} and Axel Fanego Palat\orcid{ 0000-0003-3284-9562}\affiliation{Goethe University Frankfurt}}

\title{Mobile and complex: A West African linguistic repertoire}

\abstract{Migration is one of the sources of individual multilingualism. Patterns of mobility are typically more complex than a simple move from an original home to a new residence; they can involve trajectories including internal, rural–urban, south–south, south–north, and circular migration. An individual’s experience of migration is reflected in their linguistic repertoire. Migrants commonly acquire new linguistic resources, expanding their repertoire throughout their itinerary. This is especially true of mobile people from West Africa, where urban and rural multilingualism is common in many regions. 

In our project entitled “African people in the Rhine-Main region – a project on linguistic integration”, we study language repertoires of speakers from different African countries. Through multimodal methods, including the collection of language portraits, accompanying narratives, and interviews, we get to know mobile people’s biographies and their histories of language acquisition. The data can also be analysed with a view to contact phenomena. 

In this chapter, we take a close look at the use of German during an extended interview conversation with one speaker, Kajatu, a woman born in Guinea. We focus on three examples from different tiers of language structure: the semantics of the spatial preposition \textit{in}, morphosyntactic properties of genitive constructions, and phonetic–phonological details of nasalization processes. In all three, we find evidence that the speaker draws on her entire linguistic repertoire, marked by several West African and European languages. Differences between Kajatu’s use of German and standard norms cannot simply be attributed to  ‘automatic’ processes of native language interference. Instead, individual usage patterns emerge and stabilize that can sometimes be traced back to one of the various other languages in her repertoire. In this sense, the linguistic forms on the levels of phonetics, morphosyntax, and (lexical) semantics index the individual’s biography and identity.
}

\IfFileExists{../localcommands.tex}{
  \addbibresource{../localbibliography.bib}
  \usepackage{langsci-optional}
\usepackage{langsci-gb4e}
\usepackage{langsci-lgr}

\usepackage{listings}
\lstset{basicstyle=\ttfamily,tabsize=2,breaklines=true}

%added by author
% \usepackage{tipa}
\usepackage{multirow}
\graphicspath{{figures/}}
\usepackage{langsci-branding}

  
\newcommand{\sent}{\enumsentence}
\newcommand{\sents}{\eenumsentence}
\let\citeasnoun\citet

\renewcommand{\lsCoverTitleFont}[1]{\sffamily\addfontfeatures{Scale=MatchUppercase}\fontsize{44pt}{16mm}\selectfont #1}
   
  %% hyphenation points for line breaks
%% Normally, automatic hyphenation in LaTeX is very good
%% If a word is mis-hyphenated, add it to this file
%%
%% add information to TeX file before \begin{document} with:
%% %% hyphenation points for line breaks
%% Normally, automatic hyphenation in LaTeX is very good
%% If a word is mis-hyphenated, add it to this file
%%
%% add information to TeX file before \begin{document} with:
%% %% hyphenation points for line breaks
%% Normally, automatic hyphenation in LaTeX is very good
%% If a word is mis-hyphenated, add it to this file
%%
%% add information to TeX file before \begin{document} with:
%% \include{localhyphenation}
\hyphenation{
affri-ca-te
affri-ca-tes
an-no-tated
com-ple-ments
com-po-si-tio-na-li-ty
non-com-po-si-tio-na-li-ty
Gon-zá-lez
out-side
Ri-chárd
se-man-tics
STREU-SLE
Tie-de-mann
}
\hyphenation{
affri-ca-te
affri-ca-tes
an-no-tated
com-ple-ments
com-po-si-tio-na-li-ty
non-com-po-si-tio-na-li-ty
Gon-zá-lez
out-side
Ri-chárd
se-man-tics
STREU-SLE
Tie-de-mann
}
\hyphenation{
affri-ca-te
affri-ca-tes
an-no-tated
com-ple-ments
com-po-si-tio-na-li-ty
non-com-po-si-tio-na-li-ty
Gon-zá-lez
out-side
Ri-chárd
se-man-tics
STREU-SLE
Tie-de-mann
} 
  \togglepaper[7]%%chapternumber
}{}

\newleipzig{ƁE}{{\scriptsize ƁE}}{noun class morpheme – class \textsc{ɓe} (human plural)}
\newleipzig{AUGM}{augm}{augmentative}
\newleipzig{NDE}{nde}{noun class morpheme – class \textsc{nde}}
\newleipzig{AFF}{aff}{affirmative}
\newleipzig{QUAL}{qual}{auxiliary in quality expressing clause}
\newleipzig{RELPFV}{relpfv}{relative perfective}
\newleipzig{REV}{rev}{reversative}

\begin{document}

\maketitle

\section{Introduction: Mitigating structural imprints and fluid repertoires}\label{fanego:sec:1}

Many migrants from West Africa are highly multilingual. They learn some languages along their paths of migration, while their proficiency in others results from multilingualism in the places where they used to live, places they may consider home. We will focus on one speaker, Kajatu [ka'ɟatu],\footnote{Kajatu is the speaker’s fictitious name.} born in Guinea. At the time of the interview in 2019, she had been living in Germany for over twenty years. Kajatu lived in several countries before settling in Germany. Her linguistic biography is therefore remarkable, but not exceptional – which is all the more reason for us to deem her story worth documenting. Kajatu’s story, the narrative she builds in the interview, is told mostly in German, and her German is clearly characterized by her complex linguistic biography. The lessons we learned from engaging with Kajatu and her way of using German have several layers, and conveying these layers is the main goal of this chapter. 

First of all, one significant insight is that Kajatu’s linguistic competence is based on a repertoire that draws on several languages on a constant basis. Yet her linguistic competence constitutes one repertoire, not an ‘assortment of languages’. By this we mean that she does not constantly shift back and forth between languages, and she does not insert chunks from one language into another in a patchwork fashion; instead, her communication is holistic. Secondly, a closer look at Kajatu’s German reveals that it is not a random mixture of input from whatever language first comes to her mind when she speaks German, or a lack of certain lexical or grammatical means. Instead, many non-standard features of her German partially indicate stabilized usage patterns. These draw on her full repertoire, including languages other than German, which can be traced through structural analysis. Thirdly, where there is variation in the way a particular structure is instantiated, there is the potential for indexical significance, for example when Manding-like structures appear more frequently when Kajatu talks about her Maninka-speaking grandmother. Variation is therefore not a sign of insufficient competence or insecurity in terms of knowledge of German.  

We looked into examples from three different sub-domains of grammar, because we wanted to scrutinize conventional ideas of imperfect learning and (‘native language’) interference – to the extent this is possible on the basis of a fairly limited set of data, as in our case study. Phonetics/phonology, morphosyntax, and (lexical) semantics appear to differ in how grammatically entrenched they are, arguably mirrored (and perhaps caused) by their order in first language acquisition. This would suggest that interference effects in second/later language learning are not equally likely for all domains of grammar. A functional motivation for this could be that phonetic and phonological features are more strongly driven by language-specific norms and conventions, deeply engrained in early first language acquisition. In contrast, lexical semantics pertains to the construal of meaning. This field is, in principle, infinite and constantly addresses new communicative needs, which can arise irrespective of an individual’s alleged first language. The speaker’s lexical semantic repertoire seems more likely to be affected by linguistic experiences beyond first language acquisition, and not as immutable as their phonetic-phonological habitus, for instance. Our examples do not provide straightforward corroboration for this. Instead, with regard to all three features (the semantics of the German preposition \textit{in} ‘in’, genitive constructions, and processes affecting the pronunciation of vowel-nasal combinations), non-standard usage patterns attested in Kajatu’s speech defy such a clear ordering. The explanation, we believe, lies in a refined understanding of linguistic repertoires and speaker agency.


We will address some theoretical notions in Section \ref{fanego:sec:2}, which follows this introduction. Section \ref{fanego:sec:3} will then provide some useful background information on the linguistic setting in parts of West Africa relevant to Kajatu’s experiences. These will be presented in greater detail in Section \ref{fanego:sec:4}, before we turn to specific evidence of language contact in her speech and use of German in Section \ref{fanego:sec:5}. This will be followed by a discussion and conclusions in Section \ref{fanego:sec:6}.

\section{Repertoires and languages: Theoretical notions and methodological approaches}\label{fanego:sec:2}

We align with \citet[4ff]{matras_language_2020} in choosing a repertoire approach to multilingualism. According to Matras, a multilingual speaker’s repertoire is made primarily of linguistic structures, among others word forms, phonological rules, and constructions. During language socialization, these structures become associated with social activities, including factors such as interlocutors, institutional settings, and conversational topics. Their separation into languages or classification as linguistic systems and their labelling, as well as conventions and constraints as to when and how to use and mix them, is learned through metalinguistic activities. 


Multilingual individuals, especially if they learn languages in an unguided way, draw on \textit{all} the linguistic resources in their complex repertoires to achieve successful communication, even if these choices are not always “conscious, deliberate, or strategic” \citep[7]{matras_language_2020}. A multilingual speaker’s complex repertoire is constantly present and available, and the speaker draws on it rather than switching single “language systems” on and off \citep[9]{matras_language_2020}. When drawing on their entire repertoire and exploiting its full expressive potential while complying with interlocutors’ expectations, speakers become creative communicators. A similar approach is advocated for multilingualism in African languages by \citet{lupke_repertoires_2013} and \citet{storch_communicative_2016}. Theirs is a refined linguistic understanding of agency – a crucial concept when emphasizing language as an activity rather than as a structure (see \citealt{pennycook_language_2010}). Interestingly, post-structuralist and critical approaches to language \citep{makoni_disinventing_2007, makoni_disinventing_2012,pennycook_repertoires_2018} converge in some of these views, with cognitive linguistic positions insisting on the significance of usage-based models, radical construction, or emergent grammar, although they differ in others, for example when they emphasize that linguistic repertoires and practices are inherently messy \citep{storch_communicative_2016}. 

\hspace*{-1.5pt}Overall, these influential voices emphasize ideas of real-time construal of meaning, language as an activity, and fluid repertoires. How then can the emergence of those stable structures and usage patterns that we see in Kajatu’s use of German be accounted for? We suggest that her use of formal substance, structures, and construction types across conventional language boundaries is a good starting point for finding out about how language impacts us, and how we in turn mould language for our own purposes. The task we have set for ourselves is to highlight how, in the case of one speaker’s German, a broad range of languages is drawn upon when speaking German. This does, in fact, lead to certain structural choices in Kajatu’s use of German, afforded by other languages – for example varieties of Fula and Manding – that are not ‘switched off’, but remain available to Kajatu at all times. When a speaker does this in a regular fashion, a traditional system-based view of language would describe this as one system ‘interfering’ with another.

\citet[76ff]{matras_language_2020} rejects the concept of interference as a meaningful explanation for why speakers rely on structures of one of their languages when, in fact, they speak another. The most common interference scenario that springs to mind is that of native language structures surfacing in a ‘foreign’ language. The term ‘interference’ suggests that the ‘correct’ acquisition of a target language by an individual learner is hampered by categories and structures of their native language (see \citealt{weinreich_languages_1953}). Notions of ‘imperfect learning’ possibly leading to substrate effects in language-contact scenarios, specifically with a view to language shift, is but one example which can be extrapolated from individual language competence to sociolinguistic effects at a macro level.

A repertoire approach emphasizes achieving successful communication as the primary goal of using language. From that perspective, ‘interference’ is not a meaningful term. Rather, structures of one language used when speaking another are viewed “as enabling factors that allow language users to build bridges between different subgroups within their overall repertoire of linguistic forms and to use these bridges to maintain communication” \citep[78]{matras_language_2020}. 

This paves the way to a contact-linguistic approach to migrants’ language-learning practices that overcomes notions of (inextricably deficient) second language acquisition. The speaker whose communicative creativity is studied in this chapter uses several languages, and we will show that resources from all of them are creatively employed in the interaction. Capturing these phenomena systematically can be a daunting task, which is why a few words about how we went about this are apt.

The interview which forms the core of our study was conducted by the first author of this chapter. She met Kajatu through a mutual acquaintance in the context of a research project on linguistic practices of Africans living in the Frankfurt area in Germany. In this particular project, a range of methods were used to obtain data. Central to all of them was a multimodal approach that aimed to uncover the speakers’ heteroglossic practices and their own interpretations of the experience of these practices, as described by \citet{busch_language_2018}. Our interlocutors were first asked to visualize their linguistic repertoire using a drawing of a body silhouette representing them. They were requested to insert languages, varieties, or registers in or around the body silhouette using coloured pens at their convenience (see \figref{fanego:fig:kajatus_lng_portrait} for an example).
These visual images produced by our interlocutors formed the starting point for interviews that we conducted with 25 people in 2019 and 2020. Based on the drawings and supplemented with ethnographic observations, these conversations about the speakers’ languages and varieties and how they learned them were often quite long; the interview with Kajatu lasted an hour. In addition to being rich and informative personal narratives centring on the theme of mobility and language learning, they are also textual data that lend themselves to structural analysis of language-contact phenomena. However, the structural analysis is not detached from contextual information. 

Taken together, the different perspectives result in a linguistic ethnography approach that combines the analysis of language structures with insights drawn from observation and the interpretation of personal narratives as recorded in the interview. Before we examine Kajatu’s case, we will provide some information on language settings – both in terms of sociolinguistic scenarios and with regard to significant typological information – to help the reader situate Kajatu’s trajectory and experiences.

\section{Language settings in West Africa as a backdrop to Kajatu’s repertoire}\label{fanego:sec:3}
\largerpage[-1]
Pular, Maninka, and Susu, the Guinean languages in Kajatu’s repertoire, are the most important lingua francas in the capital, Conakry, and in three of the four regions of the country respectively. Susu is the lingua franca of Basse Guinée (Lower Guinea) in western Guinea and the main language in the capital. Pular is widely spoken in Moyenne Guinée, the central region, and Maninka in Haute Guinée, the north-eastern part of the country and in urban centres of Région Forestière in south-eastern Guinea. Although 35 languages are spoken in the country, many inhabitants speak the three listed above \citep{barry_francais_2014}; thus, Kajatu’s multilingualism can be considered as typical of Guineans. 

Bambara (or Bamanankan) and Fulfulde are widespread as an L1 and L2 in Mali, a country where 63 languages are spoken according to \textit{Ethnologue} \citep{eberhard_ethnologue_2021}. Both have the status of national languages in Mali. In addition to being spoken as an L1 in the regions of Ségou, Koulikoro, Sikaso, and Kayes and by many inhabitants of the capital, Bamako, Bambara is the most common lingua franca, gaining prominence throughout the country \citep{cisse_langues_2020, dumestre_strategies_1994,dumestre_bambara_1998}. Approximately 80 per cent of the Malian population use it to a greater or lesser extent \citep{eberhard_ethnologue_2021}. Bambara is the de facto (albeit not de jure) language of Malian politics \citep{cisse_langues_2020, dumestre_dynamique_1994}, but it hardly plays a role as language of instruction in schools. Bambara has become an important language in other West African countries as well as in France and other European countries through migration. Members of the diaspora commonly regard it as the national language of Mali, but associate it with a status of West African lingua franca at the same time \citep{galtier_dynamique_1995,van_den_avenne_bambara_1998,van_den_avenne_experience_2001,van_den_avenne_changer_2004}. 

Fulfulde is one of the five regional lingua francas in Mali. It is spoken as an L1 and L2 in the regions of Mopti and Kayes. It is also present as an L1 in other parts of Mali, especially since 2012, when terrorism forced numerous people from central Mali to flee their home regions. 

Pular and Malian (or Massina) Fulfulde are varieties of Fula, also known as Fulfulde (in a wider, more generic sense) or Peul, a language originally spoken in 19 countries between Mauritania, Senegal, Gambia, the two Guineas, Sierra Leone in the west, and Sudan in the east and which includes a huge range of varieties. Guinean Pular, also known as Pular of Futa Jallon, the main area of distribution, is the Fula variety best represented in the diaspora \citep{mohamadou_presentation_2017} – whether in other African countries, such as Angola (\citealt[336, 339]{niedrig_bildungsinstitutionen_2003}, A. Diallo pers. comm.), or in Europe, for instance among African refugees in Hamburg, Germany \citep[336, 339]{niedrig_bildungsinstitutionen_2003}.\footnote{\citet{niedrig_bildungsinstitutionen_2003} reports on a research project with refugee youths from West Africa and other parts of the continent living in Hamburg at that time. A total of 32 out of the 73 people she consulted were proficient in Pular (Fula). Some of them had started to learn the language only after their arrival in Germany.} 

French is the only official language in West African countries with a French colonial past, including Guinea, Mali, Senegal, and Côte d’Ivoire, all places where Kajatu spent extended periods of time. While there was continuity in maintaining the colonizer’s language in the administrative and education systems in the other former French colonies after independence, Guinea suspended French as the main language of education between 1968 and 1984 in an effort to decolonize \citep{barry_francais_2014}. Since its reintroduction as the only official language in administration and education, the importance of French has been growing, especially among young people planning to migrate, study, or travel abroad. They consider a knowledge of French to be a great advantage \citep{diallo_quelques_2021}. Only an estimated 25 per cent of the Guinean population and a minority of Malians speak French \citep{cisse_langues_2020}.\footnote{According to \textit{Ethnologue}, French is spoken by about 16 per cent of the Malian population \citep{eberhard_ethnologue_2021}.} When Kajatu lived in Guinea and Mali, French was taught and learned almost exclusively at school. Kajatu did not have much schooling, if any. She learned French during her stays abroad, especially in Senegal and Côte d’Ivoire. Accordingly, she considers it an international language rather than associating it with her home country.

\begin{table}
    \begin{tabular}{llll}\midrule\toprule
        Language & Cluster & Language family & Mainly spoken in \\\midrule
        Pular    & Fula    & Atlantic        & Guinea           \\
        Fulfulde & Fula    & Atlantic        & Mali             \\
        Susu     &         & Mande           & Guinea           \\
        Maninka  & Manding & Mande           & Guinea           \\
        Bambara  & Manding & Mande           & Mali\\ \bottomrule\midrule
    \end{tabular}
    \caption{Genetic classification of African languages relevant to this chapter.}
    \label{fanego:tab:gen_class_afrcn_lngs}
\end{table}

\noindent
\tabref{fanego:tab:gen_class_afrcn_lngs} summarizes the genetic classification of the African languages in Kajatu’s repertoire. They belong to the Mande and Atlantic language families; Pular and Malian Fulfulde are geographical varieties of Fula, which belongs to the Atlantic languages, while Susu, Maninka, and Bambara are members of the Mande family. Maninka and Bambara are closely related dialects of the Manding\footnote{\url{https://www.ethnologue.com/subgroup/286/}}
cluster. Susu is less closely related to Manding. Mande and Atlantic languages have long been in contact in many parts of West Africa, but they differ considerably from each other. The main typological differences between them are summarized in \tabref{fanego:tab:feat_mnd_atl_lngs}.

\begin{table}
    \begin{tabularx}{\textwidth}{QQQ}\midrule\toprule
        Typological feature               & Fula         & Manding                     \\\midrule
        word order in   transitive clause 
        & S-V-O-X   & S-AUX-O-V-X                 \\
        noun phrase (genitive   modifier) & N+Gen & Gen+N                       \\
        \tablevspace
        nominal morphology                & noun class suffixes   on the noun, initial consonant mutation as remnant of prefixes & absence of noun   classes \\
        \tablevspace
        adpositions                       & prepositions                                                                           & postpositions{\slash}prepositions\\ \bottomrule\midrule
    \end{tabularx}
    \caption{Typological features of Mande and Atlantic languages (cf. \citealt[151]{guldemann_languages_2018}, \citealt[18–22]{williamson_niger-congo_2000}).}
    \label{fanego:tab:feat_mnd_atl_lngs}
\end{table}

\largerpage
Mande word order in transitive clauses differs from the word order in Atlantic languages by the preverbal position of the object; adjuncts follow the verb. In Atlantic languages, the object is in the postverbal position. In a noun phrase with a genitive modifier, the head is initial in Atlantic languages, and final in Mande. Noun class systems are a characteristic feature of Atlantic languages. Fula marks noun classes on lexical nouns by means of suffixes, whereas nouns in Mande do not have any similar morphological marking. Adpositions in Atlantic languages precede the noun (i.e., they are prepositions), while Mande languages have mostly postpositions. Speakers proficient in languages of both families, Atlantic and Mande, have quite a rich repertoire of constructions at their disposal when they start to learn a typologically distant language such as German. Kajatu is clearly one such case.

\section{Kajatu and her language-learning experiences}\label{fanego:sec:4}

\begin{figure}
    \includegraphics[width=0.5\textwidth]{Fanego_Dombrowski_Bilddatei}
    \caption{Kajatu’s language portrait.}
    \label{fanego:fig:kajatus_lng_portrait}
\end{figure}

\noindent
Kajatu, about 50 years old, draws seven points in the silhouette that represents her body (\autoref{fanego:fig:kajatus_lng_portrait}) and attributes them to the following languages: the Guinean languages Pular, Susu, and Maninka (which, according to the French tradition, she calls “Malinke”), the Malian languages Bambara and Fulfulde, French, and German. Kajatu was born and partly raised in Guinea’s capital, Conakry. She considers Guinean Pular her “Muttersprache” (‘mother tongue’), as it is the language that was used at home when she was a child. Pular is both her parents’ main language. They were Fulbe who originally came from Dalaba, a town situated in the Mamou region in central Guinea. Kajatu learned her grandmother’s language, Maninka, during the family’s stays in Dalaba while on holiday. As a child, she moved to Bamako, the capital of Mali, to accompany her elder sister, who married a Malian Fula man. She grew up in Bamako from then on and spent 15 years there. She learned Bambara and the Malian Fula variety Fulfulde, which Kajatu refers to as “Fulfulde Bamako” or “Pular von Bamako”. She also spent several years in Senegal and in Côte d'Ivoire, moving there with her elder sister and her husband repeatedly for extended periods of time. There, the language she used outside the family was French. As a young woman, she travelled a lot for trading, among other places to Senegal and Nigeria, before she came to Germany more than 20 years ago and married a German. Kajatu has three grown-up children and three younger foster children. German is the only family language. She uses Pular with her family in Guinea, Pular, Maninka, and Susu with her Guinean friends on a daily basis, Bambara with her Malian friends, and French with other friends and colleagues from West Africa outside Guinea. Kajatu travels frequently to Guinea and other countries in West Africa, and even when she does not travel, she speaks all the languages she mentions in connection with her language portrait with her relatives and friends in the Rhine-Main region and on the phone on a daily basis. 

In \citegen{klein_second_1986} terms, Kajatu’s learning of Pular and Maninka can be classified as bilingual first language acquisition \citep[15]{klein_second_1986} with a clear dominance of Pular, which she learned from birth within her family environment. In contrast, the acquisition of Maninka was temporary, since it took place only in interactions with her grandmother during short stays in Dalaba. Kajatu started to learn Bambara and the Malian variety of Fula when she moved to Bamako. She does not mention a date or her age at the time of the move to Bamako, but some elements in the narrative suggest that she was at least four years old at the time. This would mark the earliest contact with and possible beginning of the learning process with regard to Bambara and Malian Fulfulde. According to Kajatu, she grew up in Bamako, where she also learned Bambara: “Bambanan, das habe ich in Bamako […] weil ich bin gewachsen in Bamako.” (‘Bambara, I have [learned it] in Bamako, since I grew [up] in Bamako’). 

Her learning of Bambara and Malian Fula would qualify as child second language acquisition in Klein's terms (\citeyear[15]{klein_second_1986}), ranging between the age of three to four and puberty – the critical period for language acquisition (see also \citealt[72ff]{matras_language_2020}). 

It is difficult to tell whether Kajatu started to learn the additional languages Susu and French before or after puberty. The acquisition of German, in any case, would count as a clear instance of adult second language acquisition. 

Kajatu learned all her languages, including French, spontaneously in everyday communication without systematic guidance. In Guinea and Mali, French is taught almost exclusively in the formal context of school lessons,\footnote{This has changed in the last few years, as noted in \citet{diallo_quelques_2021}.} but there are several (indirect) indications that Kajatu had little or no schooling. When asked how she learned French, she replied using the impersonal indefinite pronoun \textit{man} in \xref{fanego:ex:1}.
\newpage

\begin{exe}
    \ex\label{fanego:ex:1} German \\
    \gll Französisch, man lern das in die Schule. \\
    French \INDF{} learn that in \Def.\SG.\F.\ACC{}? school \\
    \glt ‘French, one learns it at school.’ [Kajatu 03:13] %TODO: Ist das ein Zitat?
\end{exe}

\noindent
The strategy of a generic account instead of a personal one is often used as a mechanism to avoid talking about personal experience when it comes to sensitive topics in personal narrations \citep[35]{de_fina_interview_2019}. By saying that French is learned at school, Kajatu remains vague and perhaps purposefully leaves an insinuation of having attended school in place, without committing to it. Given that she spent time in Côte d’Ivoire, un-monitored learning by communicating in French is likely to have taken place. As Kajatu states concerning Côte d’Ivoire, “fast alle spricht Französisch” (‘Almost everybody speaks French’) [Kajatu 12:40]. 

We only know of one context of formal education that she underwent: adult education courses in Germany, which she attended irregularly and half-heartedly. Therefore, in her own view, she achieved little success in the acquisition of German.

Next to Pular, her “mother tongue”, Kajatu categorizes Susu and Maninka as languages of her homeland. She considers Susu Guinea’s main language: “Susu ist Hauptsprache von Guinea, [\dots] alle spricht das” (‘Susu is the main language of Guinea; everybody speaks it’), but also asserts, “Französisch [\dots], alle Herrn muss das verstehen” (‘French, all people have to understand it’) [Kajatu 16:35]. She does not evaluate her own proficiency in these languages, except for Malian Fulfulde, which she believes she speaks very well, and German, which she considers to be “not so good”.

%TODO: Wofür die Klammer hier?

\begin{exe}
    \ex\label{fanego:ex:2} German \\
    War nicht so lange [ref. der Besuch der Volkshochschulkurse], aber die Reste hab ich in die unterwegs genommen, überall ein bisschen aber wegen das mein Deutsch ist nicht so gut, aber mein Mann ist Deutscher. \\
    \glt ([It] was not so long [referring to her attendance of German classes at evening school], but the rest I grasped along the way, everywhere a little bit, but for that reason my German is not so great. But my husband is German.) [Kajatu 14:28]
\end{exe}

\noindent
She appears critical of herself in this regard, but we should bear in mind other factors that come into play. For example, the interview was conducted in German, and down-playing her competence is certainly affected by politeness strategies in the research interaction. At the same time, she is likely to measure her own German skills against the expectations of others and her own expectations in a context that emphasizes formal aspects of ‘correctness’ in language use. For all practical purposes, she is a proficient user of German, her husband’s language used constantly in the family home.

\section{Tiers of language structure in a repertoire understanding}\label{fanego:sec:5}
Building on the information (including typological properties) concerning the languages that characterize Kajatu’s repertoire (in Section \ref{fanego:sec:3}) and the overview of her biography and mobility in West Africa (in Section \ref{fanego:sec:4}), we will now look into three different sets of constructions more closely. Each pertains to a different level of linguistic analysis. We will move from the meanings of \textit{in} in Section \ref{fanego:sec:5.1} to genitive constructions in Section \ref{fanego:sec:5.2} and, finally, to nasalization phenomena in Section \ref{fanego:sec:5.3}. This order follows an (often implicit) assumption concerning speakers’ deliberate leverage, arguably decreasing from lexical choices to morphosyntax and, in particular, phonology that are learned early on, not used or influenced consciously, non-deliberate, and, therefore, hard to unlearn.

\subsection{The meanings of \textit{in}}\label{fanego:sec:5.1}
In the interview, Kajatu mostly uses the preposition \textit{in} to express static containment. In this sense, her use is no different from standard German practice – “to live in Germany; to be in the house, etc.”, as illustrated in \xref{fanego:ex:3}.

\begin{exe}
    \ex\label{fanego:ex:3} German \\
    \glll ich bin \textbf{in} Guinea {ge-bor-en} \\
    ʔis bin \textbf{ʔin} ginea geborɛn \\
    I be.\First\SG.\PRS{} \textbf{in} Guinea \PTCP-be.born-\PTCP{} \\
    \glt ‘I was born \textbf{in} Guinea.’ [Kajatu 02:20]
\end{exe}

\noindent
This is typically the case with notions of genuine stable containment. In certain cases, she uses the German \textit{in} in metaphorical containment settings. In \xref{fanego:ex:4}, we see \textit{in die Arbeit} (‘in work’) instead of either \textit{bei der Arbeit} or \textit{auf der Arbeit}, both of which are more common expressions in standard German meaning ‘at work’.

\begin{exe}
    \ex\label{fanego:ex:4} German \\
    \glll Ich bin die {Einzig-e} \textbf{in} die Arbeit, wer in Guinea komm-t \\
    ʔis	bin	di	{ʔã͡ɪ̃sige} \textbf{ʔin} di ʔarba͡ɪtə vɛ ʔin ginea kɔm-tə\\
    I be.\First\SG.\PRS{} \Def.\SG.\F.\NOM{} only.one-\SG.\F.\NOM{} \textbf{in} \Def.\F.\ACC? work \REL{} in Guinea come-\Third\SG.\PRS{} \\
    \glt ‘I am the only one \textbf{at} work who comes from Guinea.’ [Kajatu 18:22]
\end{exe}

\noindent
This instance of a less common use of the German \textit{in} probably relates to the fact that the German prepositions \textit{bei} (for loose spatial association, like the French \textit{chez} and English \textit{by}) and \textit{auf} ‘on’ (for – mostly vertical – support/attachment, as in the English \textit{on}) are in fact not very intuitive in this particular context. 

The semantic range of \textit{in} in Kajatu’s German systematically extends beyond its standard German scope of meaning in sentences that contain expressions of translational motion – not just “into” but, more strikingly and surprisingly, also “\ul{out of}” a containing GROUND as in \xref{fanego:ex:5}.

\ea\label{fanego:ex:5} German \\
    \glll mein-e Eltern komm-t \textbf{in} ein-e Stadt heiß-t Dalaba \\
    ma͡in-ə ʔɛltɛn kɔm-t \textbf{ʔin} ʔa͡ɪnə sat ha͡ɪs dalaba \\
    \First\SG.\POSS-\PL.\NOM{} parents come-\Third\SG.\PRS{} \textbf{in} \INDF-\SG.\F.\ACC? town be.called-\Third\SG.\PRS{} Dalaba \\
    \glt ‘\dots my parents come \textbf{from} a town called Dalaba.’ [Kajatu 02:32]
\z


\noindent
This is not a slip of the tongue or a nonce occurrence. In \xref{fanego:ex:4} to \xref{fanego:ex:6}, we find comparable examples of translational motion expressions containing \textit{in}, but with a reversal in the directional meaning.

\begin{exe}
    \ex\label{fanego:ex:6} German \\
    \glll Haben.wir ge-sag ok, aber wiesowieso in \dots bei uns in Heimat wir wohn-en immer zusammen bis man heirate is, man geht raus \textbf{in} Wohnung \\
    ʔãmə gəzak okɛ͡ɪ ʔabɐ vizovizo ʔin \dots ba͡ɪ ũs ʔin ha͡ɪmatə vi͡ɐ voːnənə ʔimɐ sʊzaːmən bis man ha͡ɪratə ʔis man geːt raus \textbf{ʔin} voːnũŋ \\
    have.\First\PL{} \PTCP-say ok but anyhow in \dots at ours in home we live-\First\PL.\PRS{} always together until \INDF{} marry.\PTCP? be.\Third\SG.\PRS{} \INDF{} go.\Third\SG.\PRS{} out \textbf{in} flat \\
    \glt ‘We said, ok, but in any case, in our home country, we live always together, until one gets married, one moves out \textbf{from} home.’ [Kajatu 11:02]
\end{exe}

\noindent
In part, Kajatu’s use of the German spatial preposition \textit{in} in \xref{fanego:ex:4} to \xref{fanego:ex:6} may go back to a systematic difference between verb- and satellite-framing languages \citep{croft_revising_2010, talmy_semantics_1975, talmy_lexicalization_1985}. All languages Kajatu used prior to learning German are characterized as verb-framing in that theoretical approach. They express PATH notions including directionality by lexical verbs. In contrast, predominantly satellite-framing languages like German express them through prepositional phrases (including choice of preposition and case). In contrast, the locative phrases with \textit{in} as used by Kajatu represent (container-like) GROUNDS, not PATH-denoting expressions. 

Kajatu may at times struggle to express motion events with their conceptual components GROUND and PATH arranged differently in basic construction types in German. 

In Pular, the meaning of ‘being/coming from’ corresponding to Example \xref{fanego:ex:5} would be expressed as in Example \xref{fanego:ex:7}:

\begin{exe}
    \ex\label{fanego:ex:7} Fula (Pular, Futa Jallon) \\
    \gll maw-ɓe an ɓen ko saare inn-etee-nde Dalabaa nde iw-i \\
    parent-\ƁE{} \POSS \Fsg{} \Def.\ƁE{} \Foc{} town.\NDE{} be.called-\PTCP.\PASS-\NDE{} Dalaba \Def.\NDE{} come.from-\PFV{} \\
    \glt ‘My parents come from a town called Dalaba.’ [M. Diallo, elicited example]
\end{exe}

\noindent
The phrase “from a town called Dalaba” is a locative argument licensed by the (PATH-containing) motion verb \textit{iwugol} ‘to come (from)’. It is not introduced by a preposition. The three constructed examples in \xref{fanego:ex:8} show how PATH-related notions depend entirely on the lexical meaning of the verb. Again, there are no prepositions specifying directional (\textit{from}/\textit{to}; \ref{fanego:ex:8a}, \ref{fanego:ex:8b}) or boundary-related notions (such as semantic nuances distinguishing \textit{into}/\textit{to}/\textit{towards}; \ref{fanego:ex:8b}, \ref{fanego:ex:8c}).
\newpage

\begin{exe}
    \ex\label{fanego:ex:8} Fula (Pular, Futa Jallon), constructed examples
    \begin{xlist}
        \ex\label{fanego:ex:8a} 
        \gll Ko Labe mi iw-i. \\
        \FOC{} Labe I come.from-\REL.\PFV{} \\
        \glt ‘I come from Labe.’
        \ex\label{fanego:ex:8b} 
        \gll Mi yah-ay Labe. \\
        I go-\IPFV{} Labe \\
        \glt ‘I will go to Labe.’
        \ex\label{fanego:ex:8c} 
        \gll Ko Labe mi iw-t-i. \\
        \FOC{} Labe I come.from-\REV-\REL.\PFV{} \\
        \glt ‘It is to Labe that I returned.’
    \end{xlist}
\end{exe}
In Bambara, such locative phrases without an overt postposition exist, too. A place name carries inherent locative noun properties as illustrated in \xref{fanego:ex:9a}. \\


    \begin{exe}
        \ex\label{fanego:ex:9} Bambara, constructed examples
        \begin{xlist}
            \ex\label{fanego:ex:9a} 
            \gll ù bɛ́ bɔ́ Dalaba \\
            they \IPFV.\AFF{} exit Dalaba \\
            \glt ‘They come from Dalaba.’
            \ex\label{fanego:ex:9b} 
            \gll ń bɔ́-ra só-\` kɔ́nɔ \\
            I exit-\PFV{} house-\ART{} inside \\
            \glt ‘I came out of/went out of the house/left (from inside) the house.’
            \ex\label{fanego:ex:9c} 
            \gll ń dòn-na só-\` kɔ́nɔ \\
            I enter-\PFV{} house-\ART{} inside \\
            \glt ‘I entered/went into the house.’
        \end{xlist}
    \end{exe}


\noindent
In contrast, \xref{fanego:ex:9b} and \xref{fanego:ex:9c} contain a postposition expressing interior space. In Bambara, this is the semantically transparent word \textit{kɔ́nɔ} whose full lexical meaning ‘belly’ is available alongside the use as a postposition with the (approximate) meaning ‘in(side)’. It is important to note that the postposition profiles a region in space that relates to the GROUND-expressing noun, not the PATH component of the translational motion; the latter is contained within the verb. The postpositional locative phrase does not change in shape, irrespective of whether, as in \xref{fanego:ex:9c}, the inside of the house is entered into, or as in \xref{fanego:ex:9b}, moved out from (see \citealt{dombrowsky-hahn_motion_2012} for a comprehensive account of the construal and expression of motion events in Bambara). 

We should also draw attention to a general-purpose locative preposition, \textit{ka}, in Pular. It serves to construe nouns as locative nouns in very general terms, as in \textit{ka kammu} ‘at, in, towards the sky’. Such locative expressions introduced by the preposition \textit{ka} occur in a wide range of contexts. A restriction exists concerning the noun denoting the GROUND, which has to be unequivocally identifiable, expressed by a definite noun introduced in earlier discourse, or a unique or salient place (e.g., a town’s [main] square or market, or a settlement’s [only] river). The semantic range of the preposition is not restricted in terms of topological notion. 

\begin{exe}
    \ex\label{fanego:ex:10} Fula (Futa Jallon)
    \begin{xlist}
        \ex\label{fanego:ex:10a} 
        \gll o {ƴett-i\footnotemark} horde \textbf{ka} tenkere \\
        she take-\PFV{} calabash \textbf{ka} shelf \\
        \glt ‘She took a calabash \textbf{from} the shelf.’
        \ex\label{fanego:ex:10b} 
        \gll o yah-i \textbf{ka} caangol \\
        she go-\PFV{} \textbf{ka} river \\
        \glt ‘Elle alla \textbf{à} la rivière [She went to the river.]’
        \ex\label{fanego:ex:10c} 
        \gll o jas-i ngayka \textbf{ka} njaareendi \\
        she dig-\PFV{} pit \textbf{ka} sand \\
        \glt ‘She dug a pit \textbf{in} the sand.’ [\citealt[84]{labatut_initiation_nodate}, confirmed by A. Diallo, pers. comm.]
    \end{xlist}
\end{exe}
\footnotetext{<ƴ> in the orthography of Fula corresponds to the implosive palatal consonant noted in the IPA as [ʄ]. It differs from the approximant <y> [j]. ƴ [ʄ] is one of three implosive stops in the language, next to [ɓ] and [ɗ].}

\noindent
As \xref{fanego:ex:10b} illustrates, prepositional phrases introduced by \textit{ka} occur also in sentences referring to translational motion. As with the postpositional phrases in Bambara (\ref{fanego:ex:9b}, \ref{fanego:ex:9c}), such locatives are GROUND phrases; they do not express PATH notions. These are part of the lexical make-up of motion verbs in relevant expressions, as illustrated in \xref{fanego:ex:11}. In these examples, the directionality of translational motion does not affect the GROUND phrase \textit{ka suudu}. The few examples illustrate that \textit{ka suudu} can mean ‘into the house’, ‘out of the house’, or ‘to the house’, depending on the verb’s meaning.

\begin{exe}
    \ex\label{fanego:ex:11} Fula (Pular, Futa Jallon)
    \begin{xlist}
        \ex\label{fanego:ex:11a} 
        \gll mi yalt-ay \textbf{ka} suudu \\
        I move.out.of-\IPFV{} \textbf{ka} house \\
        \glt ‘I will leave/move \textbf{out of} the house.’
        \ex\label{fanego:ex:11b} 
        \gll mi naat-i \textbf{ka} suudu	\\
        I enter-\PFV{} \textbf{ka} house \\
        \glt ‘I entered/went \textbf{into} the house.’
        \ex\label{11c} 
        \gll mi ar-t-i \textbf{ka} suudu \\
        I come-\REV-\PFV{} \textbf{ka}	house \\
        \glt ‘I came back \textbf{(in)to} the house.’ [M. Diallo, elicited examples]
    \end{xlist}
\end{exe}

\noindent
As in Bambara, place names do not usually require a preposition (see also (\ref{fanego:ex:7}) and (\ref{fanego:ex:8}) above). 

In contrast to Bambara, the preposition \textit{ka} refers not only to interior space or containing GROUNDS. In this regard, the two languages differ, but as to how they express PATH in motion events, Fula and Bambara both rely on the same conceptual pattern. It is the verb that expresses PATH, not a (spatial) adposition phrase. In Fula, we see this very clearly in the contrast between \xref{fanego:ex:8a} and \xref{fanego:ex:8c}, where the only difference is the use of a related (derived) verb stem in \xref{fanego:ex:8c}. 

Conceptually, the verb-framing pattern prevails in both Bambara and Fula. Despite this similarity, Fula (with its multi-purpose preposition \textit{ka} introducing general locative nouns) seems more likely than Bambara (with its semantically very transparent postposition \textit{kɔ́nɔ} ‘in[side]’) to have served Kajatu as a conceptual model for the construal and expression of GROUNDS in German. The choice of the German \textit{in} and its broadening in semantic scope may be a plain frequency effect, given that \textit{in} is a common preposition in German. 

Another strategy would be to omit the use of any preposition with German locatives. In fact, such examples – structurally similar to the Fula example in \xref{fanego:ex:8} – are not uncommon when Kajatu uses German. 

\begin{exe}
    \ex\label{fanego:ex:12} German 
    \begin{xlist}
        \ex\label{fanego:ex:12a} 
        \gll habe ge-flogen Senegal, Elfenbeinküste, Lomé, Togo\\
        have.\First\SG.\PRS{} \PTCP-fly.\PTCP{} Senegal Côte~d’Ivoire Lomé Togo \\
        \glt ‘I flew [to] Senegal, Côte d’Ivoire, Lomé, Togo.’
        \ex\label{fanego:ex:12b} 
        \gll aber wir waren Bamako zusammen \\
        but we be.\First\PL.\PST{} Bamako together \\
        \glt ‘But we’ve been [to?/in?] Bamako together.’ [Kajatu 30:12]
    \end{xlist}
\end{exe}

\noindent
Most examples of this type contain place names, and most express locative being in/at a place rather than translational motion, with some exceptions, like \xref{fanego:ex:12a}, or ambiguous cases \xref{fanego:ex:12b}. A larger sample of Kajatu’s German could help us understand better when she uses the preposition \textit{in} in the non-standard manner and when she resorts to not using any preposition at all for the GROUND phrase. Whatever triggers this distinction, it goes to show that the distribution of both constructions in Kajatu’s German is not random but indicative of conventional and recurrent patterns. It may represent a genuine functional split in her individual grammatical usage, impacted, but not predetermined, by the equivalent expressions in Fula and Bambara. We are not dealing with a plain ‘transfer’ of structures here that are, in principle, foreign to German. The emergence of usage patterns in Kajatu’s German rather draws equally on the various linguistic resources which we may associate with separate languages, but which are at Kajatu’s disposal simultaneously. 

In the following sections, we will scrutinize this idea in other tiers of language structure – those that are arguably characterized more strongly by conventional norms, usage patterns, and formal means usually taken to belong to an individual language in a systemic perspective.

\subsection{Noun phrases with a genitive modifier} \label{fanego:sec:5.2}
Noun phrases with a genitive modifier present several differences in Pular (as representative of Fula), Bambara (representing Manding), and German. Among others, the languages differ with respect to the order of head and modifier. Roughly speaking, in Bambara, all noun phrases with a genitive modifier show the order of genitive modifier and head. In Pular, they show the reverse word order, that is, the head precedes the genitive modifier. Examples \xref{fanego:ex:13} to \xref{fanego:ex:16} illustrate the constructions in Pular and Bambara.

%\protectedex{
    \begin{exe}
        \ex\label{fanego:ex:13} Manding (Bambara) \\
        \glll modifier head \\
        ń fà-\` \\
        \First\SG{} father-\ART{} \\
        \glt ‘my father’
        \ex\label{fanego:ex:14} Manding (Bambara) \\
        \glll modifier head \\
        Adama fà-\` \\
        Adama father-\ART{} \\
        \glt ‘Adama’s father’
        \ex\label{fanego:ex:15} Fula (Pular, Futa Jallon) \\
        \glll head modifier \\
        baaba an \\
        father \POSS.\Fsg{} \\
        \glt ‘my father’
        \ex\label{fanego:ex:16} Fula (Pular, Futa Jallon) \\
        \glll head modifier \\
        baaba Adama \\
        father Adama \\
        \glt ‘Adama’s father’
    \end{exe}
%}

\noindent
Neither Pular nor Bambara mark case on lexical nouns, contrary to German. In German, noun phrases with a genitive modifier vary according to case, dedicated possessive pronouns that agree in gender with the head, and the changing word order depending on the presence or absence of the preposition \textit{von}, which functions as a link between head and modifier. 

There are two types of construction which do not have the same distribution in German. One is a construction with the genitive case illustrated in \xref{fanego:ex:17}. It pertains to a more formal style, but occurs in colloquial speech as well, in particular with pronominal possessive modifiers (\textit{mein Vater} ‘my father’).

%\protectedex{
    \begin{exe}
        \ex\label{fanego:ex:17} German
        \begin{xlist}
            \ex\label{fanego:ex:17a} 
            \glll modifier head \\
            mein Vater \\
            \POSS.\First\SG.\M.\NOM{} father.\M.\SG.\NOM{} \\
            \glt ‘my father’ \\
            \ex\label{fanego:ex:17b} 
            \glll modifier head \\
            Adama-s Vater \\
            Adama-\GEN{} father.\M.\SG.\NOM{} \\
            \glt ‘Adama’s father’
        \end{xlist}
    \end{exe}
%}

\noindent
The other construction has, as illustrated in \xref{fanego:ex:18}, an overt prepositional linker. It generally belongs to a more colloquial register.

%\protectedex{
    \begin{exe}
        \ex\label{fanego:ex:18} German
        \begin{xlist}
            \ex\label{fanego:ex:18a} 
            \glll head {} link modifier \\
            der Vater von mir \\
            \Def.\M.\SG.\NOM{} father.\M.\SG.\NOM{} of \First\SG.\DAT{} \\
            \glt ‘my father (lit. the father of me)’ \\
            \ex\label{fanego:ex:18b} 
            \glll head {} link modifier \\
            der Vater von Adama \\
            \Def.\M.\SG.\NOM{} father.\M.\SG.\NOM{} of Adama.\DAT{} \\
            \glt ‘Adama’s father (lit. the father of Adama)’
        \end{xlist}
    \end{exe}
%}
\noindent
In the first type, the modifier is the possessive pronoun, agreeing in gender, case, and number with the noun it modifies \xref{fanego:ex:17a}, or a modifying noun in the genitive case \xref{fanego:ex:17b}. In both cases, the modifier precedes the head as in \textit{mein Vater} ‘my father’ or \textit{Adamas Vater} ‘Adama’s father’. In the second type, illustrated in \xref{fanego:ex:18a} and \xref{fanego:ex:18b}, the preposition \textit{von} links head and modifier, the latter being a pronoun or a noun in dative case.

Kajatu often uses the modifier–head genitive construction type in German. This order corresponds to the Bambara model. At times, however, expressions used by Kajatu are reminiscent of the Pular pattern, for instance in \textit{Fulfulde Bamako} ‘Fulfulde of Bamako’. In standard speech, it would have required the linking preposition \textit{von} ‘of’ in German (\textit{Fulfulde von Bamako}).

During the interview, Kajatu uses a few complex constructions whose modifier is itself a noun phrase with a genitive modifier. One of them, shown in \xref{fanego:ex:19}, conforms to the Manding (Bambara and Maninka) model,\footnote{The prosody of the nominal phrase \textit{ˌmeine ˈPapa ˈMutter} ‘my dad mother’ – distinctly audible in the recording – clearly distinguishes the genitive construction from the compound \textit{ˌmeine ˈPapamutter} ‘my dad-mother’, which, theoretically, would also be possible.} as shown in \xref{fanego:ex:20}, which differs from both the Pular \xref{fanego:ex:21} and the German constructions \xref{fanego:ex:22}.

%\protectedex{
    \begin{exe}
        \ex\label{fanego:ex:19} German \\
        \gll [[ˈmein-e ˈPapa] ˈMutter] \\
        \POSS.\Fsg-? dad mother.\F.\NOM{} \\
        \glt ‘my dad’s mother’ [Kajatu 11:30]
    \end{exe}
%}

\noindent
Both the noun phrase, which is itself a modifier, and the entire noun phrase are head-final like the corresponding Bambara construction \xref{fanego:ex:20}. They differ from the head-initial Pular counterpart \xref{fanego:ex:21}.

%\protectedex{
    \begin{exe}
        \ex\label{fanego:ex:20} Manding (Bambara) \\
        \gll [[ń fà-\`{}] bá-\`{}] \\
        \First\SG{} father-\ART{} mother-\ART{} \\
        \glt ‘my father’s mother’ \\
        \ex\label{fanego:ex:21} Fula (Pular, Futa Jallon) \\
        \gll [neene [baaba an]] \\
        mother father \POSS.\First\SG{} \\
        \glt ‘my father’s mother’
    \end{exe}
%}

\noindent
In the corresponding standard German, the genitive modifier is head-final, but the overall construction is head-initial and displays the preposition \textit{von} as a link between head and modifier, see \xref{fanego:ex:22}.

%\protectedex{
    \begin{exe}
        \ex\label{fanego:ex:22} German
        \begin{xlist}
            \ex\label{fanego:ex:22a} Standard German \\
            \gll [die Mutter von [mein-em Papa]] \\
            \Def.\F.\SG.\NOM{} mother.\SG.\F.\NOM{} of \POSS.\First\SG-\M.\DAT{} father \\
            \glt ‘my dad’s mother, i.e., my paternal grandmother’ \\
            \ex\label{fanego:ex:22b} German literary style \\
            \gll [[mein-es Vater-s] Mutter] \\
            \POSS.\First\SG-\M.\GEN{} father.\SG-\M.\GEN{} mother.\SG.\F.\NOM{} \\
            \glt ‘my father’s mother’
        \end{xlist}
    \end{exe}
%}

\noindent
Literary style as in \xref{fanego:ex:22b} is unlikely to have triggered Kajatu’s utterance in \xref{fanego:ex:20} despite the same word order. Note that using the colloquial form of address \textit{Papa} ‘Daddy, Pa’ is stylistically awkward in this construction type (\textit{meines Papas Mutter} ‘my Daddy’s mother’).

The other two complex genitive constructions of the three Kajatu uses in the interview correspond to standard German grammar (“der Mann von meiner Schwester” ‘my sister’s husband’ [Kajatu 05:31], “die Leiche von seine Papa” ‘the corpse of his father’ [Kajatu 42:22]).\footnote{Standard German requires the dative in this syntactic frame: \textit{die Leiche von seinem Papa} ‘the corpse of his father’.} The third, “meine Papa Mutter” ‘my father’s mother’, which follows the Bambara/Manding model, is uttered when Kajatu relates that she learned Maninka from her paternal grandmother when visiting her. This suggests that the conversation topic triggered this particular construction.

\subsection{Phonology: Nasal consonants and nasalized vowels} \label{fanego:sec:5.3}
The realization of syllables with a nasal coda in German words provides evidence that Kajatu makes use of her entire repertoire. \autoref{fanego:tab:3} provides a list of Kajatu’s realizations of the monosyllabic word \textit{dann} ‘then’, a temporal adverb that occurs frequently in the biographical narrative. The examples show the following realizations:
\begin{itemize}
    \item A nasalized vowel and a final velar–nasal consonant: [dãŋ]
    \item A nasalized vowel without a syllable coda: [dã]
    \item A nasalized vowel and a coda homorganic with the following consonant: [dãm] before [b] or [m], in the latter case resulting in the gemination of the final nasal as in [dãmːa͡ɪnə]
    \item An oral vowel and the labiodental nasal [daɱ] before a labiodental [v]
    \item An oral vowel and the final [n], corresponding to the German standard pronunciation: [dan]
    \item The corresponding form with a nasalized vowel [dãn]
\end{itemize}

\begin{table}
    \begin{tabularx}{\textwidth}{lXX}\midrule\toprule
        a. & d\textbf{ãŋ} … va nis   me                           & dann war nicht mehr   (06:19)                           \\
        b. & d\textbf{ãŋ} … vi͡ɐ sin   fɛha͡ɪɾatətə               & dann wir sind   verheiratet (09:30)                     \\
        c. & d\textbf{ã} binis                                    & dann bin ich (08:37)                                    \\
        d. & d\textbf{ã} vɛn di kindɐ   ʔis gəkɔmə                & dann wenn die   Kinder ist gekommen (22:50)             \\
        e. & d\textbf{ãm}ːa͡ɪnə swɛstɐ   va gəha͡ɪratət in Bamako & dann meine   Schwester war geheiratet in Bamako (03:32) \\
        f. & d\textbf{ãm} ... binis                               & dann bin ich (08:41)                                    \\
        g. & da\textbf{ɱ} vaː ʔis   diːzə.                         & dann war ich diese   (07:37)                            \\
        h. & d\textbf{ãn} habis ha͡ɪmveː                          & dann hab ich Heimweh   (32:26)                          \\
        i. & da\textbf{n} mʊsən   vi͡ɐ mit das maxɛn               & dann müssen wir mit   das machen (33:33)\\ \bottomrule\midrule
    \end{tabularx}
    \caption{Kajatu’s realizations of \textit{dann} ‘then’.}
    \label{fanego:tab:3}
\end{table}

In careful speech, the standard German \textit{dann} is realized as [dan] with a final alveolar nasal consonant; in less careful speech in informal German, different instances of regressive assimilation are encountered as well, for instance the geminated labial nasal resulting from a following [m]. Therefore, the assimilations perceived in Kajatu’s speech do not necessarily result from phonological patterns in languages acquired during childhood. Some of her realizations, however, suggest that they are constructions inspired by her previously learned languages, among others the velar coda [dãŋ] and the realization with a nasalized vowel with or without a coda, for instance [dãm] and [dã]. A look at nasal consonants and nasal vowels in Pular and Manding allows us to test the hypothesis. 

Unlike Pular and Manding, German has no nasal vowels. In Pular and the Manding varieties, nasal vowels interact with nasal consonants in a syllable coda. The processes observed suggest that nasal consonants in a syllable coda are underspecified for the place of articulation. The specification of the place of articulation is conditioned by the context to its right.

Depending on the context, in Pular there are three possible processes, P1, P2, and P3, schematized in \autoref{fanego:fig:2}.


\begin{figure}
\fittable{
    \begin{tikzpicture}    
\node[left, align=left] at (0,0){ CVN} ;
\node[right, align=left] at (1,0.5){CṼŋ / \_\_\_\_\#}   ;         
\node[right, align=left] at (1,0){ CṼ / \_\_\_\_C};
\node[right, align=left] at (1,-.5){CṼN [ɑ\textsc{place}] /\_\_\_\_   [C, ɑ\textsc{place}] };
\node[right, align=left] at (6.2,.5){P1 final velarization};   
\node[right, align=left] at (6.2,0){ P2 vowel nasalization};
\node[right, align=left] at (6.2,-.8){P3 vowel nasalization +   regressive\\  place assimilation};  
   \draw[->] (0,0)--(1,.5);
   \draw[->] (0,0)--(1,0);
    \draw[->] (0,0)--(1,-.5);   
\end{tikzpicture}
}

    \caption{Phonological processes concerning nasal consonants as a syllable coda in Pular.}
    \label{fanego:fig:2}
\end{figure}


P1 (CVN → CṼŋ / \_\_\_\_\#) stands for the process in Pular by which a vowel preceding a coda N becomes nasalized, and the coda itself is realized as a velar nasal. It applies when the coda N occurs at the end of a respiratory unit, as in \xref{fanego:ex:23}, or when the speaker hesitates before proceeding with the sentence.

\begin{exe}
    \ex\label{fanego:ex:23} Fula (Pular), realization according to P1 \\
    \glll a yar-ii ndiyan? \\
    [ʔa jar-iː ndijãŋ] \\
    \Second\SG{} drink-\PFV.\ABS{} water \\
    \glt ‘Did you drink water?’
\end{exe}

\noindent
When a nasal syllable coda N is followed by a consonant (e.g., \textit{won-de} ‘live-INF’), either process P2 or P3 applies.

P2, (CVN → CṼ / \/ \_\_\_\_C), describes the nasalization of the vowel preceding a nasal consonant in a syllable coda, and the deletion of the nasal consonant, when followed by another consonant \citep[44]{diallo_grammaire_2000}. 

P3, schematized as CṼN [ɑ\textsc{place}]/ \/\_\_\_\_ [C, ɑ\textsc{place}], is similar to P2 in that the vowel becomes nasalized. The nasal consonant is not deleted but assimilates to the place of articulation of the following consonant.

The infinitive \textit{won-de} is thus realized as [wɔ̃de] in \xref{fanego:ex:24} or as [wɔ̃nde] in \xref{fanego:ex:25} (A. Diallo, pers. comm.).

\begin{exe}
    \ex\label{fanego:ex:24} Fula (Pular), realization according to P2 \\
    \glll [ˈwɔ̃de] \\
    won-de \\
    be-\Inf{} \\
    \glt ‘be, live’
    \ex\label{fanego:ex:25} Fula (Pular), realization according to P3 \\
    \glll [ˈwɔ̃nde] \\
    won-de \\
    be-\Inf{} \\
    \glt ‘be, live’
\end{exe}

\noindent
When the consonant following the N of the coda is a nasal consonant, P3 results in the gemination of the consonant, as in \xref{fanego:ex:26}.

\begin{exe}
    \ex\label{fanego:ex:26} Fula (Pular), realization according to P3 \\
    \glll [ˈwĩnd\textbf{ãm}:ɔˈlɛːtɛr] {} {}\\
    winnd-an mo leeter \\
    write-\BEN{} him/her letter \\
\end{exe}

\noindent
There are different views on nasality for the varieties of Manding, Bambara, and Maninka (cf. among others \citealt{creissels_malinke_2009}, \citealt{vydrin_vowel_2020}). According to the most widely accepted view, the Bambara system of phonemes includes nasalized vowels Ṽ, and there are no closed syllables in the language (\citealt[18]{dumestre_grammaire_2003}, \citealt{vydrin_cours_2019}). According to \citet[17]{vydrin_cours_2019}, when nasalized vowels are followed by a consonant, /Ṽ/ is realized [VN], that is, as an oral vowel followed by a nasal consonant homorganic with the adjacent consonant. However, this is not the only realization possible, as will be shown below. 

The analysis of nasality differs for Kita Maninka: \citet[16–17]{creissels_malinke_2009} posits the existence of CVN syllables, where N is an underspecified nasal consonant manifest only through the nasalization of the vowel [Ṽ] and, when it is followed by a plosive, through the realization of a nasal segment homorganic with the plosive. When applied to the Manding varieties learned and used by Kajatu, Creissels’ analysis brings to the fore significant differences, as well as certain similarities, between Pular and Manding. According to this analysis, we have to admit a (C)VN syllable with an underspecified nasal coda next to the usual (C)V syllable. The realization of the (C)VN syllable is (C)Ṽ before a pause (cf. B1, illustrated in (\ref{fanego:ex:27})), but when followed by a consonant, several processes are at work according to the speaker’s dialect and idiolect, and, possibly, according to the context of occurrence (cf. \autoref{fanego:fig:3} and Examples \ref{fanego:ex:27} to \ref{fanego:ex:30}).

\begin{exe}
    \ex\label{fanego:ex:27} Manding (Bambara), realization according to B1: Ṽ before sentence boundary \\
    \gll à ká {b\textbf{òn} [b\textbf{ò̰}]} \\
    \Third\SG{} \QUAL.\AFF{} big \\
    \glt ‘it is big’
    \ex\label{fanego:ex:28} Manding (Bambara), realization according to B2: nasalized vowel Ṽ before consonant \\
    \gll à bínà ókáz\textbf{ɔ̂n} {\textbf{s}ɔ̀rɔ̀ [ókáz\textbf{ɔ̰̂}ː \textbf{s}ɔ̀rɔ̀]} \\
    \Third\SG{} \FUT.\AFF{} opportunity find \\
    \glt ‘he will get an opportunity (to come by car)’ \citep[19]{diallo_assimilation_2003}
    \ex\label{fanego:ex:29} Manding (Bambara), realization according to B3: Ṽ and regressive assimilation
    \begin{xlist}
        \ex 
        \gll án má à yé [\textbf{á̰m}ːáːǃjé] \\
        \First\PL{} \PFV.\NEG{} \Third\SG{} see \\
        \glt ‘we have not seen her’ (\citealt[68]{sauvant_grammaire_1956}, retranscribed by \citealt[15]{diallo_assimilation_2003})
        \ex 
        \gll ókázɔ̂n {gale [ókázɔ̰́ŋgálé]} \\
        opportunity first \\
        \glt ‘the first opportunity’ \citep[22]{diallo_assimilation_2003}
    \end{xlist}
    \ex\label{fanego:ex:30} Manding (Bambara), realization according to B4: V and regressive assimilation \\
%    \begin{xlist}
%        \ex 
        \gll {bón-ba [b\textbf{óm}bá]} {bón-dá [b\textbf{ón}dá]}\\
        house-\AUGM{} house-door \\
        \glt ‘big house’\qquad ‘house door’ \citep[17]{vydrin_cours_2019}
%        \ex 
%        \gll {bón-dá [b\textbf{ón}dá]} \\
%        house-door \\
%        \glt ‘house door’ \citep[17]{vydrin_cours_2019}
%    \end{xlist}
\end{exe}

\noindent
Some processes correspond in Bambara and Pular. When a consonant follows a (C)VN syllable, the vowel becomes nasalized and the nasal consonant is either deleted (B2[=P2]) or its place of articulation assimilates to that of the following consonant (B3[=P3]). Bambara has an additional realization when a consonant follows, called B4 in \autoref{fanego:fig:3}, in that the nasal assimilates to the following consonant and the preceding vowel is not nasalized.

Bambara and Pular also differ with respect to the realization of (C)VN syllables before a pause; they are realized with a final nasal vowel (C)Ṽ in Bambara (B1), whereas they end on a velar nasal in Pular (P1).



\begin{figure}
    \begin{tikzpicture}    
\node[left, align=left] at (0,0){(C)VN} ;

\node[right, align=left] at (1,0.5){(C)Ṽ / \_\_\_\#}   ;         
\node[right, align=left] at (1,0){  (C)Ṽ / \_\_\_\_C  };
\node[right, align=left] at (1,-.5){(C)ṼN   [ɑ\textsc{place}] /\_\_\_\_ [C, ɑ\textsc{place}] };
\node[right, align=left] at (1,-1.5){(C)VN [ɑ\textsc{place}]   / \_\_\_\_[C, ɑ\textsc{place}]};

\node[right, align=left] at (6.5,.5){B1 final vowel nasalization};   
\node[right, align=left] at (6.5,0){B2 vowel nasalization};
\node[right, align=left] at (6.5,-.75){B3 vowel nasalization + \\ regressive place assimilation};
  \node[right, align=left] at (6.5,-1.75){B4 regressive place \\ assimilation};
   \draw[->] (0,0)--(1,.5);
   \draw[->] (0,0)--(1,0);
    \draw[->] (0,0)--(1,-.5);   
        \draw[->] (0,0)--(1,-1.5);   
\end{tikzpicture}
\caption{Phonological processes concerning nasal consonants as a syllable coda in Bambara.}
\label{fanego:fig:3}
\end{figure}



\noindent
Kajatu’s realizations of the German \textit{dann} are equivalent to all possible realizations of CVN syllables in German, Pular, and Bambara/Maninka, with the exception of \textit{dã} before a pause (which would correspond to B1). The adverb ‘then’ does not occur in isolation or at the end of an utterance in Kajatu’s speech, and before a suspensive pause, Kajatu realizes it corresponding to P1. Next to the Pular-specific process P1 and the Bambara-specific process B4, the processes equivalent in both languages, P2/B2 and P3/B3, are attested as well. It is thus impossible to assign a more important role to either the L1 or the L2. On the basis of the data at our disposal, it is difficult to discover a regular distribution pattern and, consequently, to say whether we are dealing with free variation or whether this repertoire of realizations indexes something other than the speaker’s individual multilingualism.

\section{Discussion and conclusion}\label{fanego:sec:6}
We have dissected Kajatu’s German, choosing three features, and built an argument based on a description of linguistic data that make Kajatu’s German look very non-standard and exceptional. Her use of German is certainly quite unique, as the result of her particular biography and her mobility throughout her life. The linguistic codes and experiences with various language ecologies are specific to Kajatu’s case. But the mechanisms at work, we are convinced, are not unusual. A multilingual individual moving between spaces and among societies in West Africa, each characterized by various languages in different ways, will necessarily lead to a complex language repertoire. The German language data from the interview that we have focused on bear witness to this. Kajatu communicates proficiently, and that is sufficient to warrant a rich description and a systematic analysis, including that of structural properties and regular patterns. 

Kajatu’s personal narrative emphasizes continuities rather than ruptures between first and second language acquisition. There are no clear first language interference effects, but a broad range of linguistic resources, originating in the various codes at her disposal, which she relies on when speaking German. 

What shapes and determines Kajatu’s choices and possibly directions of transfer (for want of a better term) when moving between languages and drawing on different languages cannot be predicted easily. Different hypotheses have been launched and discussed at length with regard to how transfer or interference take place in speakers’ minds. In addition to the putatively compelling effects of an individual’s native language, it has been suggested that other languages learned afterwards may be equally significant and leave an imprint on languages learned even later in life – perhaps because the effort is more conscious, or perhaps because the learning processes occur closer to each other in time (see \citealt{vildomec_multilingualism_1963}, \citealt[154]{gass_second_2008}).

Typological and other similarities could be expected to play a role as well, possibly facilitating certain ‘routes of transfer’ of particular structures. The fact that certain areas of grammar are more highly structured than others, organized into strict paradigms and deeply entrenched categories (phonology more so than morphosyntax, and both outranking the lexicon), suggests a sequence from early to later in first language acquisition, which may reflect their stability and likelihood to be maintained when an individual speaker learns and uses a ‘non-native’ language. 

Overall then, tracing possible ‘interference’ scenarios is anything but simple or straightforward. And yet, choices of constructions, linguistic items, and structural properties are not purely chaotic either, even in the case of speakers like Kajatu, where notions of native competence, sequencing language acquisition into categories like those offered by \citet{klein_second_1986}, are pushed to their limits. 

In Kajatu’s German, we find structures that do not commonly form part of the repertoire of ‘native’ speakers. Some of these relate to comparable structures in one of her other languages. The complex genitive constructions that correspond to the word order they have in Maninka or Bambara are a Manding feature. The CṼŋ realization of syllables with a nasal coda at the end of a respiratory unit can be found in Pular, but not other languages. The same is true for Kajatu’s generalization of the German preposition \textit{in}, which reflects the wide range of functions of \textit{ka} in Pular when compared to the more restricted meaning of the Bambara \textit{kɔ́nɔ} ‘in’. 


Not all instances of transfer have to be associated with a language of origin, though. Some typological features are common to all Manding or even all Mande languages in Kajatu’s repertoire (i.e., Susu, in addition to Maninka and Bambara), and a specific ‘loan trajectory’ associated with a donor language is not a reasonable assumption. The same is true for those features that cannot be attributed to either Fula or Manding. Both converge in the realization of syllable codas as nasal vowels (with the nasal consonant assimilating or being deleted). We should also bear in mind that these are partly found in informal German as well. 

To return to the bigger picture, we can trace features in Kajatu’s German back to different linguistic codes in her repertoire. We do not assume that there are various grammars in her mind that compete or impact on each other in real-time speech production. This, however, does not imply the absence of clear usage patterns, stabilizing choices that characterize Kajatu’s German in a particular way. 

Structural features that are associated with one language or linguistic code, but occur in another (in this case, German), can be found in all tiers of linguistic structure. They are not ranked from phonetics to lexical semantics with a decreasing likelihood of ‘native-language impact’, nor are they ordered in terms of which language is ‘most native-like’ for Kajatu.

Contact linguistics offers a number of fundamental generalizations concerning the likelihood of transfers of language structures across varieties. The logic of borrowing scales derives from assumptions about what language structures lend themselves to being transferred between languages in contact. Situations of language shift may also lead to similarities across languages, as when phonetic–phonological features are preserved and introduced into the languages shifted towards. Our observation that various sub-domains of language structure do not differ systematically in terms of how stable and native-language-proportioned they are appears to go against these – quite widely accepted – tenets. How can they be reconciled with the case under study?

We believe that different kinds of language ecologies need to be distinguished. A predominant tendency towards monolingual ideologies and practices in a given area or community may favour mechanisms captured adequately, for example by the ‘orderly’ notion of a borrowing scale. 


Other settings and scenarios may be substantially different. It could be argued that this is simply a matter of scale, assuming that extremely multilingual scenarios may complicate the picture, but do not force us to assume different qualitative mechanisms (and limitations) concerning the transfer of language structures. Most importantly, the logic of first language acquisition as essentially distinct from (inherently imperfect) learning by (adult) second language learners could and would be upheld in this view, which relates individual processes of language learning to macro-sociolinguistic effects on entire speech communities (cf. \citealt{thomason_language_1988}, see also \citealt{trudgill_sociolinguistic_2011}). 

On a more micro-level scale, the case discussed in this chapter cautions us against merely extrapolating patterns from individual language-learning dynamics to processes at a macro level. It also illustrates the significance of what has long been known, but still has not informed linguistic theory-building widely enough; namely, the possibility of fundamentally different processes and mechanisms which are highly diverse in a linguistic context.

That is one reason we regret not having had more chances to attend situations in which Kajatu interacts with fully fledged speakers of Fula, Bambara, or other languages. If we are right in assuming that she communicates by speaking according to necessary context cues, but drawing on her entire repertoire, we might expect structures from various languages to be used in such situations. The standard view might propose lexical borrowing from German as a possible, perhaps likely feature of African languages used in the local diaspora, but not others. We are not so sure about the latter, but that remains to be seen.

%%\printglossaries


\section*{Abbreviations}
\begin{tabularx}{\textwidth}{lQ}
\textsc{\small AFF}	&Affirmative\\
AUGM &	Augmentative\\
ƁE	& noun class morpheme – class ƁE (human plural)\\
NDE &	noun class morpheme – class NDE \\
QUAL &	auxiliary in quality expressing clause \\
RELPFV	& relative perfective \\
\end{tabularx}


\section*{Acknowledgements}

First of all, we would like to express our gratitude to Kajatu for her willingness to share with us her experiences about language learning and her linguistic repertoire. A jaaraama! We would also like to express our great appreciation to our colleague, Abdourahmane Diallo, for his assistance with the Pular examples and to Klaudia’s friend M. Diallo for her help in translating selected sentences into Pular. We are grateful to the Alliance of the Rhine-Main-Universities for the financial support that made this research possible. Finally, we wish to thank the organizers and the participants of the CoTiSp conference for their valuable questions and suggestions.



\printbibliography[heading=subbibliography, notkeyword=this]

\end{document}
