\documentclass[output=paper]{langscibook}
\ChapterDOI{10.5281/zenodo.10497377}

\author{Fabio Gasparini \orcid{0000-0002-9196-8695} \affiliation{Free University of Berlin} }

\title{Why a language dies: The case of Bəṭaḥrēt in Oman }

\abstract{This chapter gives an account of my experiences conducting fieldwork on Bəṭaḥrēt (commonly referred to as Baṭḥari),
 a Semitic–Afroasiatic language that is critically endangered and spoken by fewer than 10 elders in the eastern part of the governorate of Dhofar, Oman.
  Using data and observations collected during fieldwork in the area between 2016 and 2017, I will address community and speakers’ attitudes in order
   to understand why the Baṭāḥira tribe have switched almost completely to the dominant Arab–Bedouin identity, which will inevitably lead to the disappearance of traditional Baṭāḥira heritage in just a few decades. 
While the influence of colonialism and the growth of Arab nationalism since the start of the twentieth century played a crucial role in shaping the contemporary cultural landscape elsewhere in the Middle East (\citealt{khalidi_origins_1991}; \citealt{miller_linguistic_2003}), Dhofar, its people, and its cultures remained disconnected and almost unknown to outsiders until the 1970s, when the country was unified and underwent a process of rapid Arabization. Prior to this, its inhabitants lived in seminomadic tribal groups. The Baṭāḥira have undergone a process of identity reshaping since then. The shift toward the local Bedouin Arabic culture inevitably also involved a process of language shift; in fact, Arabic has now replaced Bəṭaḥrēt in every social domain within the local community.
}

\IfFileExists{../localcommands.tex}{
   \addbibresource{../localbibliography.bib}
   \usepackage{langsci-optional}
\usepackage{langsci-gb4e}
\usepackage{langsci-lgr}

\usepackage{listings}
\lstset{basicstyle=\ttfamily,tabsize=2,breaklines=true}

%added by author
% \usepackage{tipa}
\usepackage{multirow}
\graphicspath{{figures/}}
\usepackage{langsci-branding}

   
\newcommand{\sent}{\enumsentence}
\newcommand{\sents}{\eenumsentence}
\let\citeasnoun\citet

\renewcommand{\lsCoverTitleFont}[1]{\sffamily\addfontfeatures{Scale=MatchUppercase}\fontsize{44pt}{16mm}\selectfont #1}
  
   %% hyphenation points for line breaks
%% Normally, automatic hyphenation in LaTeX is very good
%% If a word is mis-hyphenated, add it to this file
%%
%% add information to TeX file before \begin{document} with:
%% %% hyphenation points for line breaks
%% Normally, automatic hyphenation in LaTeX is very good
%% If a word is mis-hyphenated, add it to this file
%%
%% add information to TeX file before \begin{document} with:
%% %% hyphenation points for line breaks
%% Normally, automatic hyphenation in LaTeX is very good
%% If a word is mis-hyphenated, add it to this file
%%
%% add information to TeX file before \begin{document} with:
%% \include{localhyphenation}
\hyphenation{
affri-ca-te
affri-ca-tes
an-no-tated
com-ple-ments
com-po-si-tio-na-li-ty
non-com-po-si-tio-na-li-ty
Gon-zá-lez
out-side
Ri-chárd
se-man-tics
STREU-SLE
Tie-de-mann
}
\hyphenation{
affri-ca-te
affri-ca-tes
an-no-tated
com-ple-ments
com-po-si-tio-na-li-ty
non-com-po-si-tio-na-li-ty
Gon-zá-lez
out-side
Ri-chárd
se-man-tics
STREU-SLE
Tie-de-mann
}
\hyphenation{
affri-ca-te
affri-ca-tes
an-no-tated
com-ple-ments
com-po-si-tio-na-li-ty
non-com-po-si-tio-na-li-ty
Gon-zá-lez
out-side
Ri-chárd
se-man-tics
STREU-SLE
Tie-de-mann
}
   \boolfalse{bookcompile}
   \togglepaper[5]%%chapternumber
}{}


\begin{document}
\maketitle 

\section{Introduction}
Bəṭaḥrēt is one of the six Semitic–Afroasiatic Modern South Arabian languages,\footnote{Henceforth MSALs.} the other five being Mehri, Hobyōt, Ḥarsūsi, Soqotri, and Jibbāli (also known as Sheḥri, Śḥerɛ̄t, or Gəblɛ̄t). Modern South Arabian is an endangered group of unwritten minority languages currently spoken by around 200,000 people in Eastern Yemen and Soqotra, Western Oman, and the southernmost part of Saudi Arabia. Bəṭaḥrēt is listed by the \textit{UNESCO Atlas of the world languages in danger} \citep{moseley_atlas_2010} as “critically endangered,” and is the most endangered among the MSALs. It is spoken to varying degrees of fluency by fewer than 10 elders belonging to the Baṭāḥira tribe and thus destined to disappear in a few decades, at best. Current literature uses the name \textit{Baṭḥari} for the language, which is the Arabized form (with the \textit{nisba} ‘relation’ suffix \textit{-i}). However, native speakers refer to it as \textit{Bəṭaḥrēt}; therefore, this label will be used throughout this chapter.


Minorities can only be defined within the wider social context in which they are found. As \citet[2]{miller_linguistic_2003}  says, “the concept of minority implies the notion of inequality either in terms of demographic […] or sociopolitical weight.” Theoretically, this description aptly describes the case of the Baṭāḥira until recent times; according to local narratives concerning tribal life before the 1970s (the time of the unification of Oman), this small tribe used to live in geographic and political isolation, with few – albeit constant – relationships with the neighboring Janaba, Ḥarāsīs, and Mahra tribes \citep{morris_thoughts_2017}. However, today, the Baṭāḥira tend not to identify themselves as a separate minority within the multifaceted cultural landscape of contemporary Oman. Younger generations look to a unifying national Omani identity and the traditional Bedouin cultural heritage of neighboring Janaba as a means of self-representation. This allows them to feel like an integrated part of the Arab world on a wider scale, while at the same time maintaining a well-marked regional identity as a means of preserving an ethno-anthropological peculiarity (albeit partially divergent from that of their ancestors).

This case study aims to portray my experiences with the Baṭāḥira tribe while conducting fieldwork. In doing so, I will highlight the processes that led the Baṭāḥira to abandon their tribal language and traditional culture. This chapter is based on certain considerations initially presented as an introduction to my unpublished PhD thesis \citep{gasparini_bathari_2018} and on a series of field notes. 

\section{The Baṭāḥira}

This section aims to give a brief and basic description of the setting in which Bəṭaḥrēt speakers previously lived and live today. A comprehensive, rich ethnographic portrait can be found in \citet{morris_collection_nodate}. However, a broad outline at least is needed to understand why the language is on the verge of extinction and why it remained understudied for such a long time – and will also afford a better understanding of the fieldwork conditions. 

\subsection{The history of the tribe}

Until recently, very little was known in general about the small Baṭāḥira tribe. Mentions of the tribe in literature are almost non-existent, most likely because it occupies an area that is far from hospitable, hard to reach, entirely desert and barren – apart from a few springs. This area must have appeared considerably less welcoming than the lush, reassuring coasts of Salalah, which meant that Western visitors and scholars preferred to focus their interests far from the area inhabited by the Baṭāḥira. 

The first documented report to mention the tribe was written by the famous British explorer Bertram \citet[100]{thomas_among_1929}. His words about his encounter with a member of the tribe are rather unflattering and are a source of annoyance to some Baṭāḥira today. A list of words is presented in a later paper \citep{thomas_four_1937}. In addition to a brief mention by \citet{dostal_remarks_1960}, some minor information can be found in \citet{janzen_nomaden_1967}. The only person who has been able to conduct fieldwork with the Baṭāḥira was Miranda J. Morris, who worked in the area between the 1970s and 1980s and returned many years later, starting from 2013 and until 2016. She collected many recordings that had been left unpublished until recently, and much of the material collected at the time – together with more recent recordings \citep{morris_documentation_2016} – is publicly available. Furthermore, a printed edition of a selection of texts is nearing publication \citep{morris_collection_nodate} and a descriptive grammar is in the making \citep{morris_descriptive_nodate}. Morris is also the author of two papers concerning Bəṭaḥrēt; one \citep{morris_preliminary_1983} deals with Bəṭaḥrēt songs and poems, while the other \citep{morris_thoughts_2017} reports on thoughts and problems concerning the study of endangered languages (particularly addressing Hobyōt and Bəṭaḥrēt), with various considerations and samples from the languages in question. Further fieldwork and linguistic analysis on the language was recently conducted by Gasparini 
(\citeyear{gasparini_phonetics_2017}; \citeyear{gasparini_bathari_2018};  \citeyear{gasparini_emphasis_2021}; \citeyear{gasparini_nominalizzazione_2021})
and \citet{lucas_modern_2020}.


Nowadays, the Baṭāḥira are settled in the coastal area of the far east of Dhofar, near the border with the al-Wusṭā governorate. Members of the tribe are scattered from the villages of Likbi to Ṣowḳǝrǝ, with most of them living in Ashwaymiyah (\autoref{gasparini:map}). 


\begin{figure}
    \includegraphics[width=\textwidth]{Gasparini_map}
    \caption{Map of the area inhabited by the Baṭāḥira (Base layer: ESRI World Imagery. Data edited by E. Croce, F. Gasparini, and M. J. Morris, 2022).}
    \label{gasparini:map}
\end{figure}


A clan belonging to the tribe, the Bayt Kdaš, moved to the area around Salalah in the past; they still live there, and having been assimilated into the Śḥerɛ̄t-speaking communities, they no longer speak Bəṭaḥrēt. 
Reports from some of the members of the tribe (although it should be noted that the Baṭāḥira living in Salalah are particularly insistent on this point) say that their tribal territory once reached the two Wādi Ġadōn, which can be found approximately 30 km to the west of Ṯamrīt and 10 km east of Ṣowḳǝrǝ, respectively, stretching south near the mountains that divide Salalah from the desert inland. Migration and invasions by the Janaba from the northeast and from the Mahra coming from Yemen reportedly pushed the Baṭāḥira toward the area which they inhabit today. \citet[54–55]{altabuki_tribal_1982} places this event up to around 300 years ago, but the absence of historical records prevents further investigation. Because of these invasions, the tribe lost control of their former land, suffered a dramatic decrease in numbers, and were subjugated to the conquerors. A recurrent folk etymology connects the tribe’s name to the term \textit{baṭḥ} ‘dust, sand,’ which would allude to them previously being as numerous as grains of sand. The dramatic reduction in their numbers is explained through two disjointed myths. The first refers to the slaughtering of \textit{nāḳat Ṣāliḥ} ‘the she-camel of the Prophet Ṣāliḥ,’ for which God sent a plague of insects that ate all the Baṭāḥira’s belongings as punishment, leaving this once prosperous people to starve to death. The second myth is connected to the memory of fierce fighting against the so-called \textit{Burtuġaliyīn} ‘the Portuguese.’ Narratives related to these events constitute common lore among the younger members of the tribe and are imbued with a strong sense of pride. During my stay in Ashwaymiyah, I was shown a cave along the beach of Warx, an ancient settlement at the mouth of a \textit{wādī} to the east of Ashwaymiyah. The cave, which can only be reached by boat, is where the foreign invaders would be imprisoned; a large number of graves grouped together in the same area is ascribed to a great massacre of women and children perpetrated by the invaders – according to traditional folklore; this caused the drastic decrease in numbers of the tribe.

Any statement concerning the position of the Baṭāḥira within the context of past and current tribal power relationships should best be avoided. According to Thomas’ (\citeyear{thomas_among_1929}) informants, the Baṭāḥira were considered to belong to the lowest social scale. However, intertribal marriage between Baṭāḥira women and men from the Janaba and the Ḥarsūsi-speaking tribes in the north was relatively frequent (less so for Baṭāḥira men); the Baṭāḥira had ownership over some fishing sites and frankincense trees; in times of famine, it was common for the bedouins living inland to share food with the Baṭāḥira fishermen and vice versa. In all likelihood, the situation was not as simplistic as Thomas portrayed it; furthermore, these issues are still relevant to the Baṭāḥira and constitute a salient part of their identity, and must therefore be approached carefully by nonmembers of the community.

\hspace*{-1.6pt}While the interactions between the Portuguese and Oman are well documented in the northern part of the country, the same cannot be said for Dhofar, where the records are scarce and far from exhaustive. What is known is mostly due to sailors, travelers, and adventurers; Portuguese sailors, members of Vasco de Gama’s navy, did indeed stop in Hallāniyāt (formerly known as Kuria Muria), an island not far from the shores of Ashwaymiyah, for several months between 1502 and 1503. The raising of one of the sunken ships from that expedition, together with written records from the period, confirms this historical event \citep{mearns_portuguese_2016}. However, there is no tangible evidence of any conflict between the Baṭāḥira and the Portuguese navy, which apparently had good relations with the islanders, while no contact with the inland population is documented. Some truth may yet be found as to the origin of these narratives, but given the lack of historical records, these reports can only be treated as oral folk tradition.

\subsection{Environment and lifestyle}
The area traditionally inhabited by the Baṭāḥira is characterized by severely dry weather. The climate is not affected by the \textit{xarīf} season (that of tropical monsoons, between June and August), which makes the plain of Salalah prosperous and fertile. The desert and desolate landscapes of the area inhabited by the Baṭāḥira were not particularly hospitable, and a paucity of natural springs, vegetation, pastures, and wild animals to hunt made traditional life very hard, according to recollections of the eldest members of the tribe (who are also the last remaining speakers of Bəṭaḥrēt). 

The Baṭāḥira’s diet was composed almost exclusively of fish, shellfish, and turtles (abundant off the coast of Oman), camel and goat milk (from animals they bred), and, occasionally, rice and dates, depending on the time of the year and trade. Narratives and personal stories constantly emphasize the absolute lack of sufficient means of survival, with chronic starvation and diseases cyclically decimating the numbers of the tribe. The elders understandably carry with pride the fact of them being able to survive nonetheless in such harsh conditions. Water was fetched by women from various coastal springs, often located miles away from their areas of settlement. Living a seminomadic way of life, the Baṭāḥira would be either cave-dwellers or build small stone houses, which are still recognizable, especially in Mingíy, another abandoned settlement west of Šarbithat. Daily activities were carried out almost exclusively during daylight hours, as leaving the camp during the night was extremely dangerous and done only in case of emergency.

These harsh conditions, made worse by chronic starvation and disease, eventually came to an end after the unification of Oman in the 1970s. The Baṭāḥira completely gave up their nomadic lifestyle and now live a quiet, relatively healthy life in modern houses with all the standard comforts of the contemporary world. The whole tribe quickly switched to Arabic (and other MSALs), and at the time of writing, there are fewer than 10 relatively fluent Bəṭaḥrēt native speakers remaining.
 
The reasons that led to this swift process of language shift will be specifically addressed in the following sections.

\section{Working with the Baṭāḥira: The difficulties of fieldwork}
The following section provides a general description of my experiences in the field with the Baṭāḥira.\footnote{Unfortunately, further fieldwork after 2018 was initially prevented by a lack of additional funding and, after the start of my postdoctoral studies at Freie Universität in April 2020, by the global COVID-19 pandemic. This prevented me from traveling to Oman until recently (January 2022), when I finally managed to get back in the field. Data from the current fieldwork were not included due to time constraints.}  While my primary goal was to gather material for what would later become my PhD dissertation \citep{gasparini_bathari_2018}, I also tried to analyze what I witnessed from a sociolinguistic point of view, by paying attention to attitudes and opinions toward the use of language within the community.

\subsection{Fieldwork description}
In order to find a way to connect with the Baṭāḥira community, I got in touch with Dr. Miranda J. Morris from the University of St. Andrews, Scotland, who very kindly agreed to help me by sharing her local connections and part of her Bəṭaḥrēt recordings. An initial meeting with her and her main field collaborator, Khalifa Hamoud alBaṭḥari, a member of the community himself, took place in November 2015 in St. Andrews, Scotland, where the two were at that time working on Morris’ corpus of ethnographic recordings. During my stay, I was able to discuss the possibility of a period of fieldwork in Oman with Khalifa alBaṭḥari, to which he eventually agreed. 

Social scientists approach field research in different ways, according to their epistemological assumptions and goals. Often, the main goal of fieldwork in linguistics is to conduct research in the service of academia and the linguist’s individual career, whereas the potentially exploitative effects of being researched on are tentatively minimized but do not necessarily become crucial to the researcher. Within this framework, fieldwork is conducted in such a way that the linguist collects linguistic data for their own purposes, with little consideration for the speakers’ needs and desires in respect to their language and culture. This kind of approach is referred to as \textit{ethical} \citep[15]{cameron_researching_1992}. \textit{Advocacy} research \citep[15]{cameron_researching_1992} brings the researcher to actively use their knowledge to promote the researched’s needs. Finally, the \textit{empowering} model \citep[22]{cameron_researching_1992} requires the researcher to work in strict collaboration with speech community members, with the aim of building a truly reciprocating relationship between the researcher and the community. The latter approach surely is the best way to care for language data, especially when dealing with endangered and minority communities. 

The initial stages of my fieldwork were conducted over the course of two stays between October and November 2016 and March and April 2017. During my first stay, I settled in Šəlīm, a small town – mainly inhabited by local workers from South Asia and some Mahra – on the plateau surrounding the plain of Ashwaymiyah. I paid daily visits to the elder Bəṭaḥrēt speakers together with Khalifa alBaṭḥari, who would arrange the meetings for me; otherwise, the elders would not have agreed to work with me, as happened to other researchers who previously tried to work on Bəṭaḥrēt. Due to temporal and financial constraints, I was unable to spend enough time in the community to completely earn people’s trust and better understand community needs. I could not perform immersion fieldwork, understood as the observation of how language is actively used by the community during a consistent period of time \citep{aikhenvald_linguistic_2007}, since Bəṭaḥrēt is effectively a moribund language known by only a handful of elders. Furthermore, at the start of my research, my interest in the community was primarily of a purely linguistic nature, which led me to adopt an \textit{ethical} approach to my fieldwork. It was not long before I understood how inadequate my linguistic analysis would have been without a better understanding of the cultural setting. I found myself working on a language none of the members of the community had any interest in working on with me (apart from Khalifa, who is not a proficient speaker); the amount of work the last remaining speakers did with Morris right before my arrival was more than enough for them to pay her back for everything she had done for the tribe in the past. This is how I came to terms with the importance of developing a reciprocal relationship with the community. The production of a descriptive grammar did not meet primary community needs, but reminiscing and illustrating personal histories and traditional heritage (and being shown respect for their knowledge) was the main priority to many of the speakers. It became necessary to mediate between my needs as a researcher and the needs of the speech community. Therefore, I soon abandoned mostly unsuccessful direct elicitation from the contact language (Arabic) and tried to focus on monologic, monolingual, unscripted speech, discussing relevant aspects of traditional culture and spending my time with Khalifa alBaṭḥari translating the recordings into Arabic when not conducting interviews. This way I learnt how honour and respect were the most important values for the elders of the community, and without building solid trust first I would not have achieved much in my fieldwork.

My second round of fieldwork saw some considerable improvements. I was able to secure an apartment in Ashwaymiyah, a small yet lively center today famous for its rich fish trade and where most of the tribe is settled. This made the whole project much easier due to daily contact with the community, and I also had the chance to meet elders outside the controlled context of interviews. 

During my second stint, I was able to get in touch with the younger members of the tribe, some of whom were the grandsons and nephews of the elder Bəṭaḥrēt speakers. I would spend my evenings after fieldwork sitting outside coffee shops and talking to the curious young men who would approach me, gathering their impressions of my work and their cultural heritage through informal discussions. 

I would now like to turn the focus from the observer to the observed, that is to say the interviewees I worked with.

\subsection{The interviewees}
Two main groups of interviewees were considered, namely the speakers of Bə\-ṭaḥ\-rēt and the younger members of the tribe (up to 30 years old), all of them living in Ashwaymiyah.

The first group, which was also my target group for my primary linguistic inquiry, included the elderly men and women of the tribe whose mother tongue is Bəṭaḥrēt and who were adjudged by Khalifa and the community itself to be reliable speakers. Of the few who were left, I was able to work with six men and three women. Their exact ages were not clear, but they were certainly born years before Sultan Qaboos acceded to the throne, by which time they were young adults, meaning they would now be between 60 and 80 years old. All of them are illiterate, unlike most of their descendants, who underwent free schooling after the unification of Oman. 

Nowadays, all the interviewees are bilingual with Arabic, which has become their daily means of communication, and all of them but one know at least one other MSAL (either Mehri or Ḥarsūsi) as a consequence of frequent intertribal marriage.

Whenever I was left to spend time by myself in Ashwaymiyah, I would have the chance to socialize with the younger male members of the Baṭāḥira tribe living there (educated children and young adults, some of whom attended or were attending university in Muscat). They are for the most part the (great-)grandsons/nephews of the elder Bəṭaḥrēt speakers and were genuinely curious about my interest in the old language of the tribe, which was completely obscure to them. Of the dozen young men I would frequently meet during those evenings, two became my main interlocutors. This second group would later come to be a precious – and unexpected – source of knowledge. Local gender restrictions did not allow me to interact with members of the opposite sex who were the same age as me. However, as their male peers and the elders themselves reported, not even the young women of the tribe have any competence in Bəṭaḥrēt, despite the fact that they spend more time in the household in close contact with female elders. Šəlim did not offer any such chances of socialization, since it is mostly inhabited by South Asian workers, and the small community of Omani residents – with whom I had no contact – belong to the Mahra tribe. 

Day by day, I was able to collect the impressions of these young men through informal discussions during or after dinner. I did not record these sessions since they would happen by chance, mostly out in the open air and surrounded by other people passing by. However, I did make extensive use of handwritten notes.



\section{Linguistic repertoires and community attitudes toward Bəṭaḥrēt}
Nowadays, the Baṭāḥira tribe shares the area with other local tribes: Janaba families can be found east of Ashwaymiyah; Śḥerɛ̄t-speaking families populate most of the villages along the coast toward Salalah; and the Mahra inhabit the inner land. There is also the overwhelming presence of South Asian laborers working either in nearby oil fields and local small shops or as servants in private houses. Furthermore, intertribal marriage and subsequent relocation have always been common practice. Ashwaymiyah has also become an important fishing harbor due to its particularly rich waters. For this reason, the area is well known to traders, fishermen, and even tourists interested in fishing. All this shows that the local community has become an integrated part of the wider Omani society.

My report concerning the speech community of Ashwaymiyah, where speech community is understood as “a whole of people, of undetermined size, sharing access to a whole of language varieties and united by some sort of socio-political grouping”\footnote{Own translation; the original reads: “un insieme di persone, di estensione indeterminata, che condividano l'accesso a un insieme di varietà di lingua e che siano unite da una qualche forma di aggregazione socio-politica.”} \citep[60]{berruto_fondamenti_2002}, does cover two important segments of the local population; however, I could not get past very formal levels of interaction with the generation between these two. This is much to my regret, as a closer look at their communicative habits would have been critical since they are the generation where the change in the community’s linguistic behavior happens. However, there is no direct evidence that their linguistic repertoire is divergent from that of their fellow tribesmen as reported here.

Arabic is undoubtedly the primary language of the community. Morris (\citeyear{morris_preliminary_1983}; \citeyear{morris_thoughts_2017}) refers to the variety spoken by the Baṭāḥira by the label \textit{Janaba Arabic}, which is the dialectal variety of Arabic spoken by the neighboring Janaba tribe, with whom the Baṭāḥira have always been in close contact. It is important to note that Modern Standard Arabic carries an ‘overt' prestige only on certain formal occasions, for example when talking to a foreigner such as myself who would not likely be taken for a native speaker of Arabic. In fact, the younger, educated members of the tribe would talk to me in Modern Standard Arabic, presumably seeing our interactions as pertaining to a very formal register. The elders, however, do not have access to this variety since they were not formally educated – and they would occasionally be mocked for this.

Few of those young men show any interest in their linguistic heritage. Some can understand a few words and basic expressions, but none of them has any real competence in the language. The most common reason for their lack of interest is that Bəṭaḥrēt would be completely useless in their daily lives, since no one outside their hometown would understand them – not even the neighboring Mahra, with whose language there is only scarce intelligibility. The young Baṭāḥira view Arabic as a powerful tool to enrich themselves and move to bigger cities (mainly Salalah, Muscat, or the Emirates). In fact, evident prestige was accorded to those who could master Modern Standard Arabic at a higher level. A good example of this was the case of M.; 24 years old, with a degree in engineering and a high level of Modern Standard Arabic, he would be automatically elected as the most entitled one to guide a group conversation with me, and would be addressed as a medium between me and the rest of the group. Most of the other youngsters spoke only Arabic (in its vernacular and standard forms) and the local Pidgin Gulf Arabic variety, while one individual was fluent in Ḥarsūsi, as his mother came from the Ḥarāsīs tribe. 

The time I spent with the elders was for the most part dedicated to learning more about Bəṭaḥrēt. Usually, our recording sessions would revolve around a series of ethnographic topics. When questioned about the reasons which led them not to speak Bəṭaḥrēt with their children anymore, all the elders agreed in stating the uselessness of Bəṭaḥrēt in the rapidly changing post-revolutionary Oman. The last remaining speakers do not seem to have any will or see any need to teach the language; they feel too old to start, lack the energy to see it through and are scarcely interested in speaking it anyway. It is a common opinion – not only in the area, but in the Arab-speaking world in general (see \cite{kaye_diglossia_2001}) – that ‘proper' languages are taught at school, while daily, home speech has to be considered vernacular. In fact, all of the remaining speakers consider Bəṭaḥrēt to be a virtually dead language – which is true from a sociolinguistic point of view, with the use of the language being maintained in no social domain whatsoever apart from arranged interviews. There is only one occasion which sees the speakers use Bəṭaḥrēt; during social gatherings, some traditional Bəṭaḥrēt songs are occasionally sung, mainly by the men of the tribe, and recordings are shared using smartphones. The overall meaning of these songs is not always preserved, though, and sometimes not even the elders remember the meaning of the words. 

However, it must be recognized that, after Morris’ work, a rediscovered sense of pride in the adventurous narratives of past daily struggles to survive in extreme conditions is palpable. The young men I met feel at least great respect for their grandfathers and their material culture, even though they do not have the will to maintain it. Their social networks are usually wider than those of their grandparents, and frequently include individuals and tribes from other villages. Furthermore, they socialize also with the great numbers of immigrant workers from South Asia (who often speak Gulf Pidgin Arabic) and foreigners in general, mostly (but not only) for the sakes of daily life and work-related reasons. This situation is common throughout the whole Gulf area \citep[137]{holes_language_2011} and has radically changed the sociolinguistic landscape of the region. 

\hspace*{-3.5pt}The suppressed Omani revolution and the subsequent ``renaissance'' overturned the traditional way of life of the Baṭāhira. In the new Omani society that emerged, the role of Arabic was that of a unifying and empowering tool that the tribe was very keen to fully adopt for the sake of their involvement in the globalized world. Therefore, there was no reason left to maintain their native tongue – this was the point at which parents ceased transmitting it to their children. Language shift usually happens at a slower pace and requires many generations to reach its end \citep{dorian_pidgins_1989}, yet this is not true in the case of the Baṭāḥira, where only one generation was needed to nearly complete the process. Bəṭaḥrēt disappeared from daily life in the community, and young people now show clear signs of cultural shift toward an Arab identity, under the heavy influence of a dominant image of Arab–Bedouin heritage.

\subsection{``Iḥna bɛ̄du'': Identity replacement in the youngest generations}
The discourse concerning cultural representation and self-identification has become one of real interest within both ethno-anthropological frameworks and daily discourse in the globalized world (\citealt{fardon_production_1995}; \citealt{hannerz_fluxos_1997}). It is striking to note how quickly the process of Arabization led the Baṭāḥira to adopt an Arab–Bedouin identity almost completely. The new generations are eager to present and identify themselves primarily as Bedouin and not necessarily as an ethnically separate group from the neighboring tribes. It must be noted that the term \textit{bɛ̄du} ‘Bedouin’ is usually used by the elders to refer to the lifestyle of the neighboring Arab, Ḥarsūsi, and Mahri tribes living inland (\textit{bʕéli əbādiyə}, ‘people of the desert’) who made a heavier use of camels (\textit{bʕéli həbɛ̄ʕar }‘camel people’), as opposed to the fishermen living near the shore (\textit{bʕéli ərawnə}, ‘people of the sea’); it is never applied (to my understanding, at least) to the Baṭāḥira as a group, but seldomly to the few Baṭāḥira who owned animals and periodically lived inland as Bedouins. According to this use, the term simply acknowledges the fact that the Bedouins inhabited the desert and does not entail much more; on the contrary, the Bayt Kdaš, the group living in Salalah, seem to idealize the concept of the Bedouin lifestyle in a similar but stronger fashion to the younger generations living in Ashwaymiyah and frequently report it as a distinctive feature of their tribe’s past. Calling themselves  Bedouins implies the adherence to an all-round identity, in stark contrast to the fomer use. This is not unexpected: on the one hand, in a socially composite city such as Salalah, the Bedouin identity is best fitting for the Bayt Kdaš, since it allows them to proudly elaborate their history from a different perspective; on the other hand, the younger Baṭāḥira were already schooled within the framework of the national education system, which promotes the common Arab identity, where monolingual education in Arabic played a crucial role to the expense of minorities. This also raises the question of whether this widespread self-representation reflects the more or less overt feelings of their own households – possibly suggesting that a systematic process of identity replacement may have taken place in their parents’ generation – or if cultural assimilation into their intertribal cohorts was induced through daily and continuous contact. Today, only the Baṭāḥira elders directly experienced the hunger and struggle of pre-unification times. The elders undoubtedly recognize the existence of a system of traditions once belonging and peculiar to the Baṭāḥira, such as material culture, fishing techniques, taboos, rituals, and so forth. They are also very aware that the Arab–Bedouin culture has now completely replaced the way the elders used to live, and that traditional cultural heritage has lost its vitality due to obsolescence. Most of Baṭḥari lore, in fact, stems from a way of life which has completely disappeared today. This knowledge has lost its immediate, practical utility for its community, and as such there is no apparent reason to keep it alive. One evident example is the whole practice of \textit{ʕawf} ‘taboo.’ The imposition of temporary or permanent taboos on certain actions – such as fishing or food consumption – indirectly regulated the exploitation of natural resources by limiting or prohibiting the catching of specific fish species at a given moment of the month or the year, thus allowing for their ecologically sustainable reproduction. Since the Baṭāḥira have severed their relationship with the sea, passing on these habits to their descendants is no longer significant to the elders nor to their heirs. The value of traditional culture thus becomes evident if it is put in relation to the environment in which it developed. The disconnection between these two levels has had dramatic consequences for the survival of the Baṭḥari culture and language, and has been claimed with regard to the rest of the MSAL-speaking peoples \citep{watson_language_2017}. Another example of dramatic cultural loss relates to the field of traditional medicine and ethnobotanical knowledge, which can be of great importance for modern-day research nonetheless (see, for example, \cite{leonti_traditional_2013}).

On the part of the elders, the strong will to integrate into the new, modern Omani society and improve living conditions inevitably meant getting rid of any memory related to a past of hunger and poverty, intrinsically linked to the traditional way of life of the Baṭāḥira. With the language itself being a vestige and a constant reminder of those times, the need to get past their isolation was so urgent that parents started to talk to their children only in Arabic – education and media did the rest. Before Morris’ return to the field in 2014, the few speakers left reportedly had not spoken Bəṭaḥrēt for decades, and it was only thanks to Morris’ continuous efforts that they managed to recall their long-unspoken mother tongue (Morris p.c.).

Now it is interesting to note the value of Bəṭaḥrēt within the Bayt Kdaš, a clan of the Baṭāḥira tribe that moved to Salalah some time ago and has quite a different story from that of the rest of the tribe. It is not known at what point in history this part of the tribe moved to the city. This event might have taken place in a relatively distant time, since the members of the clan I was able to talk to generically refer to their ancestors when talking about who among them migrated to the mountains of Salalah. Despite belonging to the same tribe, the Bayt Kdaš had no contact with the rest of the tribe for decades, and connections have only been reestablished in recent years. Some of them have now bought or built new houses in the area of Ashwaymiyah and often spend time there. Their life in the lush area around Salalah was radically different from that of the rest of the tribe and they could enjoy much prosperous conditions. In the peculiar urban context of Salalah a kind of balanced diglossia (MSALs being used at home and as tribal languages while Arabic is used in other environments) was established over time, probably because of tribal prestige and power relations; differently from the situation in Ashwaymiyah, there never was a social demand to stop speaking MSALs and be fully integrated to the Arab identity, allowing for culture maintenance. In this scenario, the Bayt Kdaš fully switched to Śḥerɛ̄t and Arabic and integrated into the local community through marriages with the Śḥeró tribe. Interestingly, this group is the only one to have shown great interest in my research, seeing in it a way to rediscover their origins and build a strong tribal identity through the tokenization of Bəṭaḥrēt itself. However, I avoided being actively involved with this identity discourse, since I did not want to gaslight the local community in any way by forcing the debate over a very sensitive topic within local society.

\subsection{The remaining MSALs}
The sociolinguistic situation of the other ethnic groups living in Dhofar and speaking MSALs is considerably harder to evaluate, due to the higher numbers of speakers; nonetheless, a dedicated study on the topic would surely reveal a greater level of complexity, requiring many different layers of analysis. The identity discourse within other more numerous groups may be even more sensitive and, needless to say, would require the interested researcher to apply a certain degree of caution. In any case, the ongoing effects of Arabization are clear, with younger generations of Dhofaris from MSAL-speaking families progressively losing their linguistic competence in favor of Arabic \citep{morris_thoughts_2017} – in this regard, the extremely complex sociolinguistic situation of a city like Salalah would be a fascinating study in itself, as it was hinted in previous sections. Among other things, this loss of linguistic competence is due to the exclusive use of Arabic in education, a lack of written material, and very limited use of MSALs in the media. Arabic is considered to be the one and only language of Oman, necessary to secure a good job and to travel abroad (especially to the Emirates, whose charm and cultural influence over the younger generations is getting stronger). Meanwhile, MSALs are seen by many as the vernacular, dialectal medium and are therefore used in family or local contexts – mostly because they are unwritten, which seems to be a critical factor in determining speakers’ opinions. However, things seem to be changing; some members of the local communities are now engaged in documentation and revitalization projects,\footnote{See, for example, a Wikitongue-funded Language Revitalization Accelerator project by Said Baquir and Abdullah alMahri.}  and hopefully, their direct involvement will mitigate, if not invert, the process of language and culture loss. 


\section{Conclusion}
Since the rise of modern nation-states and the ideologies behind them in Europe in the twenty-first century, the distinction between an official language and local or vernacular languages has been seen as crucial in order to enhance a shared identity from the point of view of central governments, often struggling against conflicting identities coexisting under the same political entity \citep{bonfiglio_deconstructing_2010}. Apparently, this would not happen – or would happen to a lesser extent – within the pre-modern empires, where no strict language policy was set up and minorities would have a certain amount of independence from the central government \citep{stolz_arabic_2015}, meaning the use and demise of a language was strictly dependent on the alternate fates of its community of speakers.

The situation of language minorities in the Arab world has been considered by many scholars. For example, \citet{zaborski_minority_1997}, \citet{owens_close_2007}, and \citet{stolz_arabic_2015} give general historical and synchronic accounts on the topic, while \citet{miller_linguistic_2003} focuses her analysis on the contemporary situation of Egypt and Sudan. \citet{grandguillaume_arabisation_1983} outlines the situation of the Maghreb and \citet{stolz_arabization_2015} and \citet{maddy-weitzman_berber_2011} focus on Algeria and the status of Berber. It becomes clear that most of the countries in the Arabic-speaking world historically adopted policies favoring the quasi-exclusive use of Arabic in its standard, official variety, to the detriment not only of other linguistic minorities but also the local dialectal varieties of Arabic itself. From the speakers’ perspective, the acquisition of the official language was considered a means of cultural redemption, granting individual acceptance and facilitating integration into society. A recurring concept in many definitions of what a modern nation is refers to an aggregation of people inhabiting a delimited territory and speaking the same language, and this same idea can be found at the basis of pan-Arabist movements from the late twenty-first century onward \citep{suleiman_language_2013}. The construction of a shared, collective memory is thus seen as a necessary step in the building of a nation. Whatever might be detrimental to this vision is left behind – but this is a colonial legacy. Commenting on a draft version of this paper, Frisone (p.c.) posed the question: “If there were no borders or nations, would this link between identity and memory be so strong?” If they exist, it is because the European and sovereign social organization has spread the nationalistic idea of identification between the people and their history. When this concept is developed into a foundation myth, it builds a more or less coercive bond to a specific territory; borders, in turn, create a sense of belonging to an ethnicity by spatially delimiting the extension of a shared identity. It is entirely a caged macroconstruction of which all the systems of state power, inside and outside Europe, made use nonetheless: its effects are especially visible in the domain of colonial heritage. 

It is safe to say that the influence of colonialism and the growth of Arab nationalism during the twentieth century played a crucial role in shaping the contemporary linguistic situation in the Middle Eastern area \citep{khalidi_origins_1991}. The positivist idea of the need for an official language to unify such a wide area under the same macro-identity led to the rise of Modern Standard Arabic as a shared official language, which was undoubtedly beneficial in many respects. One major consequence, however, is that most of the minority languages, which were already struggling, and communities scattered across this extensive territory were put at risk under the pressure of culturally hegemonic (and sometimes violent) central governments. In fact, until very recent times, “[m]inorities’ languages [were] almost totally excluded from public life and at best [were] accepted in their ``folkloric'' forms [… and] almost never taught” \citep[8]{miller_linguistic_2003}.

In this sense, Oman is a very peculiar case when compared to the rest of the Arabic world. Until its unification, Dhofar was a widely unknown land, where the lack of a central government prevented modern infrastructures from developing, leaving local lifestyles almost untouched by the outside world for centuries. What we do know about its history is still fragmented; unlike Northern Oman, the area inhabited by the MSAL-speaking people has precious few historical records, especially for the time between the fall of the ancient South Arabian kingdoms and the late nineteenth century. Oman as we know it today was born after the Dhofari revolution, which took place between the 1960s and 1970s. This brought unification to this very diverse land, under the rule of the late Sultan Qaboos bin Said al Said, who rose to power after overthrowing his father, Said bin Taimur, in a palace coup in 1970 and transformed the newly unified nation from a poor, underdeveloped country into a modern state. A major consequence of the war and the socioeconomic change in the subsequent decades has been the blurring of traditional social distinctions \citep[266]{peterson_omans_2004}, with loyalty to Sultan Qaboos being put before tribal rivalries (although these were never completely extinguished). 

If compared to the policies towards minorities adopted by other Middle Eastern countries, a noteworthy element of Sultan Qaboos’ reign – and that of his successor, Haytham bin Tariq Al Sa'id, for now at least – is that they never directly acted against the heterogeneity of Omani ethnic composition by means of repression or forced cultural substitution, albeit the undeniable hegemonic position of Arabic and Arab identity within the Omani context. It must also be noted that Śḥerɛ̄t and MSAL-speaking groups in general in the western part of Dhofar were considered strictly linked to the separatist rebellion and looked at with suspicion by the government: in this perspective, to indirectly destabilize the maintenance of an important identity element such as language benefits the interests of the central power. It is also true that there are no ongoing safeguard programs addressing minority languages. The only part of the constitution dealing with this topic is Article 3, where it is stated that the official language of the state is the Arabic language.\footnote{\textarab{لغة الدولة الرسمية اللغة العربية} “the official language of the country is Arabic”.} The preservation of minority languages is left to the enterprising spirit of individual speakers, but without any inclusive language policy, remaining competent in the traditional language often means being cut off from the contemporary world. The ramifications of the loss of language for identity and culture are likely to be enormous and will undoubtedly transform Dhofari society significantly \citep[260]{peterson_omans_2004}.

Like all aspects of a culture, the place of language as a core cultural value is culturally determined \citep{smolicz_language_1980} and negotiated within individual and community identities. The major frame of analysis postulates that “language features are the link which binds individual and social identities together” \citep[317]{tabouret-keller_language_2017}, but when this ceases, it might well be that other tools are used as an instrument of identity, be it tribal history or whatever is at hand or comes to mind. It seems that for the last remaining speakers of Bəṭaḥrēt, the only place where the full assumption of tribal identity and a real sense of unity for the whole social group can be found – that is, the only instances where they would say \textit{iḥna l-Baṭāḥira} ‘us, the Baṭāḥira’ – is connected to the memory of the past, of a world now completely vanished after the advent of modernity. Far from being idealized, this was a world where starvation and poverty were commonplace, and one that they are quite happy to have exchanged for survival and an easier way of life. What undoubtedly remains is a great sense of pride for having been able to survive to such conditions. This is in stark contrast with the part of the tribe in Salalah, which gave new meaning to their tribal belonging in order to better fit within the context of the city. As for the situation in Ashwaymiyah, the language and cultural shift swiftly took place within a couple of generations. There seems to be an unsaid schism between them and the younger generations that is not easy to address directly.

\bigskip

\section*{Acknowledgements}

My thanks to the Baṭāḥira community for their kindness and for cooperating with me for the purposes of my research. I am also tremendously grateful to Miranda J. Morris, whose help has had an incredible impact on the quality of my work and my learning of the language and culture of the Baṭāḥira. I would like to thank Enrico Croce for producing the map used in Figure 1 and Gloria Frisone for her valuable comments on an early version of this paper. Finally, I also warmly thank Kamala Russell for her precious comments and suggestions, which greatly improved the paper. This research is part of the project Describing the Modern South Arabian Baṭḥari language [ref. 40.20.0.007SL], funded by Fritz Thyssen Stiftung.


\printbibliography[heading=subbibliography, notkeyword=this]

\end{document}
