\documentclass[output=paper]{langscibook}
\ChapterDOI{10.5281/zenodo.10497393}

\author{Yaron Matras \orcid{0000-0002-9840-0277} \affiliation{Aston Institute for Forensic Linguistics, Birmingham, and Department of Hebrew Language, University of Haifa}}

\title{Reconciling the global and local in language contact}
\abstract{Contributions to this collection cover a continuum of phenomena broadly located between two poles of language contact perspectives: The idea of the multilingual repertoire as a dynamic and fluid pool of resources which individuals deploy at their meaningful discretion, and the social conventions and values that shape users’ attitudes to languages and the hierarchical relations between them. Linked to those are the transformations in the use and shape of individual structural features: The replication of word forms with their phonetic characteristics from another language, the re-configuration of morphological and syntactic constructions blending features from more than one language, the re-combination of word forms with scripts, and more. This chapter deals with the dichotomy between ‘named languages’ and the practice that has been described as ‘translanguaging’. It proposes an integrated model of language contact. At the core of the model is the view of contact as a balancing act of pull factors that impact strategies to manage an integrated repertoire of features. Structural changes in language (contact induced changes, convergence, or borrowing) are understood as new local practice routines that can be disseminated and shared across a practice community. As variables the model takes into consideration the functional properties of individual structural categories and the motivation to innovate practice around them, patterns of action routines and types of talk, as well as discourses about language.
}


\IfFileExists{../localcommands.tex}{
   \addbibresource{../localbibliography.bib}
   \usepackage{langsci-optional}
\usepackage{langsci-gb4e}
\usepackage{langsci-lgr}

\usepackage{listings}
\lstset{basicstyle=\ttfamily,tabsize=2,breaklines=true}

%added by author
% \usepackage{tipa}
\usepackage{multirow}
\graphicspath{{figures/}}
\usepackage{langsci-branding}

   
\newcommand{\sent}{\enumsentence}
\newcommand{\sents}{\eenumsentence}
\let\citeasnoun\citet

\renewcommand{\lsCoverTitleFont}[1]{\sffamily\addfontfeatures{Scale=MatchUppercase}\fontsize{44pt}{16mm}\selectfont #1}
  
   %% hyphenation points for line breaks
%% Normally, automatic hyphenation in LaTeX is very good
%% If a word is mis-hyphenated, add it to this file
%%
%% add information to TeX file before \begin{document} with:
%% %% hyphenation points for line breaks
%% Normally, automatic hyphenation in LaTeX is very good
%% If a word is mis-hyphenated, add it to this file
%%
%% add information to TeX file before \begin{document} with:
%% %% hyphenation points for line breaks
%% Normally, automatic hyphenation in LaTeX is very good
%% If a word is mis-hyphenated, add it to this file
%%
%% add information to TeX file before \begin{document} with:
%% \include{localhyphenation}
\hyphenation{
affri-ca-te
affri-ca-tes
an-no-tated
com-ple-ments
com-po-si-tio-na-li-ty
non-com-po-si-tio-na-li-ty
Gon-zá-lez
out-side
Ri-chárd
se-man-tics
STREU-SLE
Tie-de-mann
}
\hyphenation{
affri-ca-te
affri-ca-tes
an-no-tated
com-ple-ments
com-po-si-tio-na-li-ty
non-com-po-si-tio-na-li-ty
Gon-zá-lez
out-side
Ri-chárd
se-man-tics
STREU-SLE
Tie-de-mann
}
\hyphenation{
affri-ca-te
affri-ca-tes
an-no-tated
com-ple-ments
com-po-si-tio-na-li-ty
non-com-po-si-tio-na-li-ty
Gon-zá-lez
out-side
Ri-chárd
se-man-tics
STREU-SLE
Tie-de-mann
}
   \boolfalse{bookcompile}
   \togglepaper[12]%%chapternumber
}{}



\begin{document}
\maketitle

\section{Introduction}
In 2011 a new question was introduced into the UK national census asking respondents to specify their ‘main language’. The results gave the first statistical picture of the diversity of languages spoken in the country. But the question was not designed to capture the country’s multilingualism. In fact, it was worded and formatted in such a way that would effectively obscure valuable information on multilingualism (cf. \cite{matras_multilingualism_2015}): A hierarchy was introduced by asking respondents to name another language only if their ‘main language’ was not English. That will have excluded the home languages of those who took ‘main language’ to mean the language spoken most hours of the day, for example, at work or a place of study, or the language favoured with peers or media. Respondents who opted to name a ‘main language’ other than English could only name a single language. The reality of multilingual repertoires could not be captured. Census takers were then asked to self-assess their proficiency in English. The overall purpose of the question was evidently to give authorities an indication of the extent to which speakers of other languages had lower levels of English. In the public discourse of the conservative political establishment that had been in government in the UK since 2010, use of other languages was linked to lack of integration and productivity and even to radicalisation and ideological extremism. Multilingualism was seen as a potential citizenship deficiency.


\section{Counting languages vs translanguaging}

At the same time the numbers tell a story. First, they represent the action of self-declaring another language in full awareness of the pressures to conform. Indirectly and unintentionally the census offers a platform for declarative agency or what \citet{stroud_multilingual_2018} calls ‘acts of linguistic citizenship’. They also show some trends: In 2011 around 7.7 per cent of the population of England and Wales declared a ‘main language’ other than English. Ten years on, in the 2021 Census, the figure had risen to 8.9 per cent. Yet the proportion of those who declared that they did not know English remained exactly the same, at 0.3 per cent of the population. These figures demystify the notion that being multilingual necessarily comes at the expense of knowing English, particularly since it is likely that many respondents, for the reasons explained above, under-reported their use of other home languages.

Nevertheless, the representation of languages as countable, discrete entities skews the reality of many households. In 2013, we produced a short film about the languages of Manchester.\footnote{\url{https://www.youtube.com/watch?v=pmTDzsPrBp8} (Last accessed: 4 December 2023).}  In an allusion to the census question we asked one interviewee what her ‘main language’ was. She responded with little hesitation: “Main language? Every day I speak three languages at the same time!” For multilingual persons the question framed in a monolingual mindset is counter-intuitive. In the Northwest of England, the most widespread community language is Panjabi (and closely related variants Mirpuri, Pahari, and Potwari). Yet the highest census numbers for a ‘main language’ other than English in the region appear for Urdu. Forced to choose one over the other(s) people who originate from Pakistan preferred to declare a publicly recognised, official written language over regional varieties. Speakers of Romani almost never declared their home language. They listed the national languages of their respective countries of origin in central and eastern Europe, usually assuming that their language would not qualify as ‘main’ since it lacks institutional status and often not even knowing what the English term for the language is (they call their language \textit{rromanes}, lit. ‘in the manner of the Roms’).

The census is a blatant example of the way perception of language is shaped by nation-state ideologies. Some authors have been strongly critical of any enumeration of languages, referring to it polemically as ‘linguistic accounting’, ‘demolinguistics’, or ‘headcount of languages’ (see \citealt[19–49]{pennycook_metrolingualism_2015}; \citealt[187–188]{king_multilingual_2016}; \citealt[56–64]{stevenson_language_2017}). They juxtapose the listing of languages to first-hand investigations of linguistic practices, sometimes referred to as ‘languaging’. The focus on practice is strengthened by an appreciation that increased mobility and new forms of mediality and institutional participation create ever more complex domains of interaction. Captured by concepts such as ‘ethnoscapes’ and ‘super-diversity’ (\citealt{appadurai_global_1992}; \citealt{vertovec_super-diversity_2007}) the multiplicity of interaction options leads to a lower degree of predictability of links between language, place, identity, and community, with methodological implications for the analysis of relations between linguistic forms, participants, place, and institutions (cf. \citealt{blommaert_sociolinguistics_2010}; \citealt{blommaert_language_2011}; \citealt{arnaut_language_2016}; \citealt{arnaut_engaging_2017}). Terms such as ‘translanguaging’, ‘metrolingualism’, ‘heteroglossia’, and ‘crossing’ have been used to capture the dynamic fluidity of moves among linguistic forms (\citealt{rampton_crossing_1995}; \citealt{blackledge_multilingualism_2010}; \citealt{garcia_translanguaging_2014}; \citealt{pennycook_metrolingualism_2015}; \citealt{wei_translanguaging_2018}). The notion of ‘translanguaging’ in particular has been celebrated almost with a sense of triumphalism: It stands for a paradigm shift that not only replaces the view of languages as fixed entities with clear demarcation boundaries but also calls for social engagement and intellectual resistance against ideologies that foster that view (cf. \citealt{creese_routledge_2018}; \citealt{moore_translanguaging_2020}).

This critical, post-structuralist view of the links between language and social representations aligns itself with an established strand that theorises the use and processing of multiple languages: Multilingualism is not the added accumulation of several monolingual modes \citep{grosjean_neurolinguists_1989}. Instead, it is a complex set of features \citep{jorgensen_polylingual_2008} blended together in an individual’s overall repertoire. That repertoire includes acquired norms and conventions according to which in a given interaction context features and sets of features are selected and others inhibited (\citealt{matras_language_2009}/\citeyear{matras_language_2020}; see also \citealt{green_mental_1998}). Such notions of repertoire problematise ‘language’ as a pre-determined set of structures and view it instead as a dynamic, emerging pattern of practices, detaching it from fixed notions of pre-defined groups or speech communities and viewing groups as emerging and evolving networks of practice and people as moving in between and among them (\citealt{busch_linguistic_2012}; \citealt{blommaert_superdiverse_2013}). The view of language contact as one closed system interfering with another has been replaced by a view in which plurality of form is the default and closed systems or ‘named languages’ are derived social constructions.

\section{Contact, categories, and repertoire management}
The very premise of contact linguistics questions a founding principle of modern historical linguistics, namely the idea that languages are only pre-destined to diverge from one another. Contacts between populations and the multilingualism that they create increase similarities between languages and can lead to convergence \citep{trubetzkoy_proposition_1928}. Already the earliest examinations of the effect of contact questioned whether some forms were exempted entirely or partially from such processes \citep{whitney_mixture_1881}. In due course attention was given to the likelihood that some components were more easily ‘borrowed’ from one language into another \citep{thomason_language_1988}. The question was asked whether typological parameters could predict and account for ease of borrowing (\citealt{moravcsik_universals_1978}; \citealt{campbell_proposed_1993}; \citealt{stolz_universelle_1997}; \citealt{matras_borrowability_2007}) and whether those in turn might reveal something about the inner functions of language categories and their status within the speech production process itself (\citealt{myers-scotton_four_2000}, \citealt{matras_language_2009}/\citeyear{matras_language_2020}). 

Frequency-based trends are not always meaningful: Nouns can be at the top of the borrowability hierarchy simply because they are the most common category in most languages and because they represent new objects and concepts that enrich semantic expression when cultures come into contact. But an implicational hierarchy such as ‘but > or > and’ (where ‘>’ indicates greater likelihood of borrowing, and the implicational arrangement suggests that the presence of higher elements on the hierarchy is a pre-condition for lower elements) calls for an explanatory model \citep{matras_utterance_1998}: If the semantic-pragmatic operation of contrast outranks that of disjunction, and the latter outranks addition, in their respective susceptibility to borrowing, then that will have its roots in the ease with which bilingual users maintain a separation of forms by ‘language’ around the respective processing operation, or, instead opt to give up that separation in favour of generalising just a single form (usually the one that can be used in both in- and out-group communication, or simply in a wider set of interaction settings). The hierarchy, attested universally (cf. \citealt{hober_borrowing_2022}; \citealt{stolz_pero_2021}; \citealt{grant_contact_2012}; \citealt{matras_borrowability_2007}), suggests that on the cline, the operation of contrast is more likely to serve as a trigger to users to give up separation of features within the repertoire. It is in other words more difficult to maintain such separation around the function of contrast, with implications for the processing of broken causal chains and turn management. The historical event that we regard as ‘borrowing’ between languages is thus seen and explained as triggered by factors that involve the management of a complex repertoire of features, across a complexity of ever evolving interaction settings. Borrowing is in reality a change in practice routines. It is the product of an innovation that is gradually propagated across a network of language users or a practice community. Through the inertia of our structuralist intellectual upbringing, we tend to view the outcome of that change in practice routine as a structural change in a particular named language which we label ‘borrowing’.

\section{Towards an integrated model}

I have so far alluded to two major innovative developments in the emergence of an epistemology of language contact: The move away from examining closed, self-contained ‘systems’ in contact and onto appreciating the existence of a complex and dynamic, wholesale repertoire of features; and the realisation that the fate of forms and features in linguistic settings that are complex and dynamic depends on their category status, i.e. on their function in the mental processing of information and knowledge and the structuring of interactional turns. Work within the translanguaging paradigm, innovative as it may be, has so far been rather reluctant to engage with the structural transformations that language contact brings about. Structural and typological approaches in their turn have by and large shied away from embracing critical approaches that seek to de-construct the idea of language as a fixed ‘system’. The empirical contributions to the present volume show that there is a reality behind both: The dynamic fluctuation of features in an individual’s repertoire, and the metalinguistic perception of ‘languages’ as emblems of identity that can be enumerated, labelled, and evaluated and whose integrity can be either carefully maintained or intentionally disrupted and interrogated.

For this reason, we need an integrated theory of language contact. Such a theory must be explicitly equipped to account for contact-induced structural change in terms of the factors that motivate users to alter their practice routine when managing their complex repertoire of features. It must also account for the factors that motivate networks of users to converge around innovative practices. It must recognise that linguistic practice is driven not merely by aesthetic attributes that are associated with individual features but also by the function that different features and categories of features assume in the process of knowledge transfer and knowledge processing that is at the core of communicative interaction. At the same time, it must also acknowledge that users’ practice routines can themselves become the subject of discourse – labelled, enumerated, and qualified. 

In \citeauthor{matras_language_2009} (\citeyear{matras_language_2009}/\citeyear{matras_language_2020}; see also \citealt{matras_theorizing_2021a}) I outline the principles of such an integrated theory: Users manage their repertoire of linguistic features balancing three pull-factors: a) the wish to accommodate to context-bound expectations on the part of the listener by selecting those features that are deemed purposeful and permissible while inhibiting those that are not; b) exploiting the full expressive potential of the repertoire, making use of as many features (including word forms, constructions, suprasegmental features, and discourse-management routines) as possible to maximise expressiveness; and c) managing processing load effectively by reducing where possible the need to deploy the selection and inhibition mechanism by generalising features across interaction contexts and settings (‘levelling’). The balancing act is a constant one, prompting users to negotiate and re-negotiate the choice of features locally, i.e. in each and every interaction and often utterance. Yet it is also guided by the conventions of established practice routines. The latter can be subject to more global meta-discourses that may contain and constrain individual users’ flexibility to deploy features in a way that arises directly from the local balancing act. When altering practice routines users can draw on at least two distinct strategies: The deployment of linguistic ‘matter’ (phonological shapes or forms) and of linguistic ‘pattern’ (form-meaning relationship). For some constructions the choice is constrained by their very nature: Combinations of words are always ‘patterns’. For others, both can be options: Definite articles and some lexical items can be replicated as word-forms or calqued through grammaticalisation or semantic extension, respectively. Attitudes to language may come into play: Nativising a form through pattern replication may be preferred as a way of preserving the integrity of a set of features that are associated with a particular interaction setting Finally, the role of a feature and its category or function value in the processing of knowledge in communicative interaction will determine or partly determine its susceptibility to borrowing: Users are motivated to eliminate the need to select and inhibit, i.e. to choose among functionally equivalent or near-equivalent items, and maintain the separation of sub-sets, when the processing burden is most intense. That accounts, among other things, for the high borrowability of expressions of contrast and of elements that help monitor and direct the interaction such as discourse markers (‘pragmatic’ markers). Similarly, the motivation to generalise features across the repertoire is greater when those features represent unique or particular knowledge spaces. This accounts, among other things, for the borrowability of so-called ‘cultural loans’, including names of institutions and culture-specific practices.

\section{Usage and ideology on a continuum}
Before I return to elaborate on the theory and its variables, I wish to reference the contributions to this volume and the way in which they demonstrate the existence of a continuum between flexible and dynamic repertoire management and the social reality of metalinguistic discourses about named languages and the social construction of demarcation boundaries around them. 

Strict demarcation boundaries among languages are identified in interviews that elicit attitudes and in institutional settings where language choices are strictly defined. Sandra Schlumpf describes how metalinguistic discourses reveal the acceptance of hierarchies among named languages in Equatorial Guinea. Users associate languages with the practice routines in which they are deployed, attributing values accordingly. In effect, users differentiate (and label) different sets of features within their overall repertoires according to the communicative practices that they represent. Whether a language – an identifiable set of features that carries a label – has high or low prestige appears to be in part indexical to the settings and contexts in which it is deployed, particularly when compared to other languages whose deployment in institutional settings might be considered superior. Nancy Hawker discusses how Palestinian Arabic in Israel is viewed as an in-group language. The display of multilingualism, i.e. of proficiency in both the in-group language and in Hebrew, the majority and principal state language, is regarded as a valorisation of assets. In effect, it testifies to the user’s greater ability to assume flexibility among different communicative practices in a variety of settings, particularly in institutional settings as well as addressing different user networks. In the social-political context of Israeli society, user networks are associated with different populations and a cultural and political boundary. Switching language can signal defiance or audience selection; in other words, it can disrupt or accommodate to established practice routines. As Fabio Gasparini describes, repertoires are subject to historical changes and the Bəṭaḥrēt language is losing ground in Oman as Arabic infiltrates all domains of communication in the local community. This may be regarded as a consequence of the infiltration of nation-state ideologies into a community that has been situated on the fringe of such ideologies for much of its history. Changing ideologies leads directly to radical changes in practice routines.

The contrast of languages can be meaningful, in Gumperz’ (\citeyear{gumperz_discourse_1982}) terms, also within a single interaction. Jacopo Falchetta describes the associations of repertoire elements with different social contexts. That allows users to exploit meaningful contrasts as a socio-pragmatic function in the use of Moroccan Arabic and French. While each set of features represents the sum of interaction settings with which they are linked, the alternation among them is itself a social determinant of a user’s background, notably the user’s ability to deploy a complexity of feature sets, testifying to their immersion in multiple networks of users and multiple practice settings. Similarly, Marta Rodríguez García describes how in Gibraltar Yanito is a permanent negotiation of repertoire components and feature sets. In fact, the label itself captures users’ perception of their alternating deployment of sets of features that are otherwise, in institutional settings, considered to be separate languages, as an integrated whole. The practice routine of selecting features at users’ discretion within the same interaction and network of users is acknowledged in the meta-discourse as a variety in its own right.

In her discussion of Sofia’s linguistic landscapes Emilia Slavova shows how writing systems, normally subjected to more tightly regulated and institutionalised language use, can also be deployed at users’ discretion as part of the resources of the complex linguistic repertoire. They become combinable in new ways with word forms associated with different written languages. If monolingualism is considered in public discourse to be the norm that is linked to nation states and language education, and a means of conformity, valorising the individual and good citizenship, then the absence of inhibition when deploying repertoire features to maximise creative expressiveness is in some ways an act of defiance (Slavova mentions usage “in unexpected ways”), one through which users assume agency to draw on past experience but subvert existing routines and give legitimacy to new forms of practice (cf. \citealt{emirbayer_what_1998}; \citealt{liddicoat_agency_2020}). I am reminded of a young trilingual child’s theatrical use of a one-off mixed utterance, using English and German in an interaction context that is normally reserved for Hebrew (\citealt{matras_language_2009}/\citeyear[38]{matras_language_2020}): ‟Aba, where do I get a \textit{Lappen} so I can \textit{wisch} my \textit{Gesicht}?” – ‘Daddy, where do I get a \textit{wash cloth} so I can \textit{wipe} my \textit{face}?’ The subversion of the routine is two-layered: First, in the unexpected choice of English as the predication language of the utterance when addressing an interlocutor with whom Hebrew is the established routine; and second, in inserting lexical items from German into the English utterance. Below I will return briefly to this aspect of performativity and its role in explaining certain types of language contact outcomes.

Klaudia Dombrowsky-Hahn and Axel Fanego Palat discuss a German West African woman's deployment of various features of her repertoire in a way that does not always conform to the expected monolingual norm but enables communication in that it shows creative agency in forming constructions drawing on an array of resources. The replication of French impersonal constructions when speaking German, the generalisation of a preposition with a variety of verbs independently of the direction of motion, and the use of modifier-head juxtaposition to express possession can all be approached as an emerging practice that allows the user to navigate language-learning in settings shaped by frequent mobility and a range of participation networks. The outcome is a set of highly individualised innovations and features. Linda Bäumler’s chapter lends further insights into the potential cumulative effect of such choices shaped by participants’ networks of practice: Whether or not English sounds are directly replicated in loanwords or replaced by Spanish equivalents has to do with a sense of ‘affinity’, that is, awareness of and appropriation of the wholesale repertoire. Affinity can emerge and be reinforced through factors such as interaction experience or exposure to media. It is in part an emotional state rather than a strictly objective circumstance such as geographical proximity, all the more so since in a globalised world immersion in practice routines can be remote and mediated. Miriam Neuhausen similarly shows how a process of phonological change in a minority or diaspora language aligns itself with a parallel process in the contact language: Users seem to blend together their ‘management’ of linguistic resources, adopting a change wholesale irrespective of the named language, in other words, indiscriminately in all settings and with all sets of interlocutors with whom the relevant sound pattern is used. Here too, the extent of the change in Pennsylvania German is linked to the extent of exposure to settings in which English is used. 

Hans-Jörg Döhla’s discussion of a ‘learned language contact scenario’ shows how such individual choices are triggered by a comparable sense of affinity. The translators seek to preserve the aesthetic appearance of the model text where a juxtaposition of factive and causative is used as a recurring template. It is the aesthetic format that becomes a repertoire feature, one that is preserved when addressing a separate audience of readers through the creative process of exhausting the expressive potential of Spanish verb derivational constructions. Codified by the translated texts, innovation is then disseminated among the practice community of readers to become a characteristic feature of the genre. The process bears resemblance to Inga Hennecke’s account of pragmatic markers in Manitoba French. Here, too, a practice community emerges, albeit in spoken discourse rather than through the reception of a written genre, characterised through the way certain features of the complex repertoire are managed. We have a good example of the inherent link between repertoire management and the role of certain functional categories in the motivation to generalise the mapping of meaning to form (French \textit{comme} adopting the wider meanings of English \textit{like}) and of actual word forms (English \textit{so}) across the repertoire. The author hints that the wish to ‘level’ these features is in some sense pre-determined by the properties of the category of pragmatic markers: Their semantic and syntactic detachability as well as difficulties in translation equivalence.\footnote{Hennecke takes issue with the explanation I provided in \citet{matras_language_2020} where I traced lapses in the selection and inhibition mechanism, i.e. bilingual speech production errors, to cognitive factors. But, of course, we are dealing with different kinds of data here, and since Hennecke’s chapter does not address such lapses in control, the cognitive triggers may not be evident. Still, I would argue that local shifts in meaning and distribution of individual markers have their roots in one-off lapses of control where repertoire features are blended. The stage that Hennecke examines is one in which such occurrences have become accepted and conventionalised.}

\section{From practice to contact languages: Variables of a theory}

The case studies show us how users’ linguistic repertoires comprise word forms, form-meaning mappings and constructions, phonetic articulations as well as experiences, values and attitudes, all of which are associated with a range of experiences of various communicative interaction settings. These features can cluster in partly distinguishable sets that are subject to more or less strict selection control, but they can also be re-grouped into new configurations either at the level of local interaction or that of the individual user’s emerging practice preference, and be disseminated among a practice network involving other users. Users may or may not exploit the contrastive potential of the affinity between features and associated interaction settings as a means to express difference and to replicate or interrogate social hierarchies.

The labelling and enumeration of languages, and the realisation that users engage in the practice of languaging drawing on their full repertoire of features, are therefore not at all mutually exclusive and so they need not constitute theoretically juxtaposed perspectives. They are instead complementary, provided we can view both through a shared and integrated theoretical lens. Returning to the model of repertoire management briefly outlined above (cf. \citealt{matras_language_2009}/\citeyear{matras_language_2020}), I would like to propose that such an integrated theory must give consideration to a differentiated set of variables. These include a) structural features (constructions, words, morphs, phones, suprasegmentals), b) linguistic action routines (institutional forms of discourse, pragmatic organisation of discourse modes, roles, turn taking, types of talks and distribution of illocutions), c) social and institutional settings that impact language and give rise to an array of language practice routines, and d) metalinguistic awareness and discourses about language. To a considerable extent these are also the elements that are alluded to in discussions of the super-diverse repertoire (\citealt{busch_linguistic_2012}; \citealt{blommaert_superdiverse_2013}) and in those of language ecology \citep{pennycook_language_2010}. The challenge as I see it is to link these dimensions explicitly to a theory of structural change in contact situations, explaining change as the product of innovations in the management of repertoire features, and explaining innovations as motivated by the goals of communication and the different procedures of mental processing of information that are triggered by different kinds of structural categories. 

If we return to the ‘mixed’ utterance of the trilingual child quoted above, here the conscious defiance of the practice routines of feature separation is performative, aiming to achieve a particular effect on the listener and so on the relationship between speaker and listener at a given moment in the interaction. The structure of the utterance strongly resembles the conventionalised patterns that have been labelled ‘Mixed Languages’ \citep{bakker_mixed_2003}: These are languages that display contact outcomes that are deemed to be unconventional, combining, for instance, grammatical inflection from one source language with core lexicon from another, or nominal inflection from one with verb inflection from another, or borrowing wholesale function word paradigms such as pronouns which are normally not borrowed wholesale from one language to another. For that reason they are deemed worth of the explicit label of being ‘mixed’. Traditional historical linguistic approaches have defined Mixed Languages in terms of the genetic tree-model of language diversification as languages whose genetic ancestry cannot be determined \citep{thomason_language_1988}. Yet if our ‘critical’ or post-structuralist approach to language contact prompts us to abandon the tree-model as the principal prism through which we view language change and to think instead of convergence and re-configuration as a default, we must find a new way to conceptualise Mixed Languages. I propose that Mixed Languages arise from the performative practices of combining features that are not normally combined through the everyday pull factors (maximising expressive potential or easing the burden of processing load – the two principal motivations to introduce permeations into the management of the complex repertoire). Instead they are quintessential expressions of agency, subverting established routines in order to perform new identities 
(cf. \citealt{matras_repertoire_2021b}). 
Our integrated model of repertoire management thus allows us to link so-called ‘translanguaging’ practice with the perception of language boundaries, the functional value of structural categories (and the ease with which they are generalised across interaction settings or ‘named languages’), forms of illocution, agency, and the perpetuation of new practices through dissemination across an emerging practice community. In other words, Mixed Languages are exceptional because they arise from utterances that are purposefully defiant of everyday patterns of mixing. They are products of a particular mode of repertoire management.

In a similar vein, Creoles, traditionally viewed in historical linguistics as the expansion of pidgins formed out of a need for restricted communication (but see critique of that view in \citealt{mufwene_creoles_2021}) can be viewed as features of the repertoires of many individuals, adopted as an emerging shared practice in a newly formed practice community. It is not their ‘genetics’ that give them substance but rather their constitution as an assembly of features that enable communicative interaction in new and emerging settings. An integrated theory of language contact is challenged to link the global functions of individual structural categories to local processes of repertoire management, and in that way to account for new and changing action routines.

\printbibliography[heading=subbibliography, notkeyword=this]

\end{document}
