\documentclass[output=paper
		  ]{langscibook}
\ChapterDOI{10.5281/zenodo.10497367}

%\documentclass[output=paper]{langscibook}

\author{Katrin Pfadenhauer\orcid{ 0000-0002-7411-2932}, Sofia Rüdiger\orcid{ 0000-0003-4370-8052} \& Valentina Serreli \orcid{ 0000-0002-4944-3107} \affiliation{University of Bayreuth}}

\title{Introduction}
\abstract{This introductory chapter presents a short summary of the field of contact linguistics, setting the stage for the following chapters and establishing the interdisciplinary and multi-methodological mindset of the authors in this edited volume.}


\IfFileExists{../localcommands.tex}{
   \addbibresource{../localbibliography.bib}
   % add all extra packages you need to load to this file

\usepackage{tabularx,multicol}
\usepackage{url}
\urlstyle{same}

\usepackage{listings}
\lstset{basicstyle=\ttfamily,tabsize=2,breaklines=true}

\usepackage{langsci-basic}
\usepackage{langsci-optional}
\usepackage{langsci-lgr}
\usepackage{langsci-osl}
% \usepackage{./langsci/styles/langsci-lgr}
% \usepackage{./langsci/styles/langsci-osl}
% \usepackage{langsci-gb4e}

\usepackage{tikz}
\usetikzlibrary{patterns,calc}
\pgfdeclarepatternformonly{south east lines}{\pgfqpoint{-0pt}{-0pt}}{\pgfqpoint{3pt}{3pt}}{\pgfqpoint{3pt}{3pt}}{
    \pgfsetlinewidth{0.6pt}
    \pgfpathmoveto{\pgfqpoint{0pt}{3pt}}
    \pgfpathlineto{\pgfqpoint{3pt}{0pt}}
    \pgfpathmoveto{\pgfqpoint{.2pt}{-.2pt}}
    \pgfpathlineto{\pgfqpoint{-.2pt}{.2pt}}
    \pgfpathmoveto{\pgfqpoint{3.2pt}{2.8pt}}
    \pgfpathlineto{\pgfqpoint{2.8pt}{3.2pt}}
    \pgfusepath{stroke}}
    
\usepackage{stmaryrd}
\usepackage{wasysym}
\usepackage{multirow}
\usepackage{caption}
\usepackage{subcaption}
\usepackage{mathrsfs}
\usepackage{qtree}

\usepackage{linguex}


   %pminos do not split footnotes
% \interfootnotelinepenalty=10000 %Footnote in Laporte chapters has to be split SN


%\DeclareIndexNameFormat{default}{%
%\nameparts{#1}%
%\usebibmacro{index:name}%
%{\index[names]}%
%{\namepartfamily}%
%{\namepartgiveni}%
% {}% L1
% {}% L2
%{\namepartprefix}% generates spurious space L3
%{\namepartsuffix}% generates spurious space L4
%}

%  {\DeclareIndexNameFormat{default}{%
%     \usebibmacro{index:name}{\index[names]}{#1}{#3}{#5}{#7}}}

%\DeclareIndexNameFormat{default}{%
%  \usebibmacro{index:name}{\sindex[nom]}{#1}{#3}{#5}{#7}}

%\DeclareIndexNameFormat{default}{%
%  \usebibmacro{index:name}{\sindex[person]}{#1}{#3}{#5}{#7}}
%\DeclareIndexNameFormat{default}{%
%\nameparts{#1} \usebibmacro{index:name}{\sindex[person]]}{\namepartfamily}{‌​\namepartgiven}{\nam‌​epartprefix}{\namepa‌​rtsuffix}}

%\newcommand{\smiley}{:)}

%\renewbibmacro*{index:name}[5]{%
%\usebibmacro{index:entry}{#1}%
%{\iffieldundef{usera}{}{\thefield{usera}\actualoperator}\mkbibindexname{#2}{#3}{#4}{#5}}}

% \newcommand{\noop}[1]{}

%remove for final
%\overfullrule=1mm

\newcommand{\tobi}[2]}}
\renewcommand{\S}[1]{\tobi{#1}{\textsc{*}}}

% this volume references
% puts: [this volume]
% already defined: \citetv
%\newcommand{\citepv}[1]{(\citeauthor{#1} \citeyear*{#1} [this volume])}
\newcommand{\citealtv}[1]{\citeauthor{#1} \citeyear*{#1} [this volume]}

%parentheses around example number
\newcommand{\pref}[1]{(\ref{#1})}

% in-text examples

\newcommand{\lnex}[1]{\textit{#1}} %target lang word
\newcommand{\lnlit}[1]{(lit.: `#1')} %literal reading
\newcommand{\lnlat}[1]{(#1)} % latinization
\newcommand{\lntrans}[1]{`#1'} %translation
\newcommand{\lnexl}[2]%
{\lnex{#1}{} \lnlat{#2}} % ex with latinization
\newcommand{\lnexlat}[3]{\lnex{#1}{} \lnlat{#2}{} \lntrans{#3}} % ex with latinization and tranl.

%ch01
\newcommand{\co}[1]{\mbox{\textbf{#1}}}

%ch09

\newcommand{\cyrbulg}[1]{\begin{otherlanguage*}{bulgarian}#1\end{otherlanguage*}}


%ch10
\newcommand{\nlp}{{\small NLP}}
\newcommand{\mwe}{{\small MWE}}
\newcommand{\rae}{{\small RAE}}
\newcommand{\lvc}{{\small LVC}}
\newcommand{\pos}{{\small P}o{\small S}}
%\newcommand{\todo}[1]{ \textcolor{red}{#1} }

%\renewcommand{\labelenumi}{\theenumi}
%\ainamefmt{{vv}{ll}{, ff}{, jj}} % fullname

\newcommand{\biberror}[1]{{\color{red}#1}}

\newcommand{\osenovaitem}{--~}
   %% hyphenation points for line breaks
%% Normally, automatic hyphenation in LaTeX is very good
%% If a word is mis-hyphenated, add it to this file
%%
%% add information to TeX file before \begin{document} with:
%% %% hyphenation points for line breaks
%% Normally, automatic hyphenation in LaTeX is very good
%% If a word is mis-hyphenated, add it to this file
%%
%% add information to TeX file before \begin{document} with:
%% %% hyphenation points for line breaks
%% Normally, automatic hyphenation in LaTeX is very good
%% If a word is mis-hyphenated, add it to this file
%%
%% add information to TeX file before \begin{document} with:
%% \include{localhyphenation}
\hyphenation{
    Beck-man
    Ngu-yen
    back-chan-nel
    back-chan-nels
    mo-not-o-nous
    ste-reo-typ-i-cal
}

\hyphenation{
    Beck-man
    Ngu-yen
    back-chan-nel
    back-chan-nels
    mo-not-o-nous
    ste-reo-typ-i-cal
}

\hyphenation{
    Beck-man
    Ngu-yen
    back-chan-nel
    back-chan-nels
    mo-not-o-nous
    ste-reo-typ-i-cal
}

   \boolfalse{bookcompile}
   \togglepaper[00]%%chapternumber
}{}



\begin{document}
\maketitle

\section{Previous research on language contact: An overview}

Language contact is a major force in the evolution of language and language change; it presupposes the coexistence and interaction of different languages at the individual, community, and societal level. In essence, language contact depends on the existence and extent of contact between speakers. It is also fundamentally inter-idiolectal, meaning that groups are only in contact to the extent that their individuals and their idiolects are in contact with each other (\citealt{mufwene_ecology_2001}; \citealt{mufwene_individuals_2012}).
 
Since the 1950s, linguistic studies on language contact have attempted to individuate the processes and outcomes of contact in linguistic systems, producing models to explain contact-induced change; that is, the effect of contact in language structures (see \citealt{weinreich_languages_1953}; \citealt{thomason_language_1988}; \citealt{thomason_language_2001}; \citealt{van_coetsem_general_2000}; \citealt{heine_language_2005}; \citealt{bakker_contact_2013}; for a general overview, see \citealt{winford_introduction_2003}; \citealt{matras_language_2009}; \citealt{hickey_handbook_2010}). Among the subfields of interest within the broad area of language contact are the study of code-switching and code-mixing (e.g., \citealt{muysken_bilingual_2000}; \citealt{myers-scotton_contact_2002}; \citealt{gardner-chloros_code-switching_2009}), dialect contact (e.g., \citealt{trudgill_dialects_1986}; \citealt{kerswilll_creating_2000}), and pidginization and creolization (e.g., \citealt{arends_pidgins_1994}; \citealt{siegel_emergence_2008}; \citealt{aboh_emergence_2015}; \citealt{velupillai_pidgins_2015}; \citealt{knorr_creolization_2018}). 

While most of the studies mentioned above share a conception of languages or varieties as discrete and bounded systems, a more recent approach to language contact, starting from the conceptualization of language as a social practice, has focused on how linguistic resources – meaning either language systems or single features – are used by speakers to create social meaning and to negotiate identities (e.g., \citealt{sanchez_moreano_practicas_2021}). Contact-induced variation is thus observed in language repertoires rather than languages as discrete systems, and new frameworks have been devised to help describe and explain the fluidity of language practices, including \textit{crossing} \citep{rampton_crossing_1995}, \textit{languaging} (\citealt{canagarajah_ecology_2007}; \citealt{makoni_disinventing_2007}; \citealt{jorgensen_languaging_2011}), \textit{polylingualism} \citep{jorgensen_polylingual_2008}, and \textit{translanguaging} \citep{garcia_translanguaging_2014}. Finally, a broader view of contact in context is found in studies that assume language evolution takes place within local linguistic ecologies and is influenced by the complex interrelation between demographic, social, economic, and intrinsically linguistic factors (\citealt{mufwene_ecology_2001}; \citealt{pennycook_language_2010}). 

Languages are brought into contact in a number of ways, the most obvious of which is migration – voluntary or forced, individual or large-scale, temporary or permanent – whereby the interaction of speakers with different histories and repertoires results in the reinterpretation of existing affiliations and identities and the negotiation of new ones (\citealt{aalberse_heritage_2019}; \citealt{canagarajah_changing_2019}). Yet in the era of globalization, characterized by easier communication and circulation of information, heterogeneity and diversity can also be found within local populations, both in metropoles at the center of power and in smaller communities on the periphery (\citealt{blommaert_sociolinguistics_2010}; \citealt{pietikainen_sociolinguistics_2016}).

\section{Global and local perspectives on language contact}

This edited volume recognizes both traditional and innovative language contact research and brings together contributors whose expertise covers various languages, with the goal of examining both general phenomena of language contact and specific features in a comparative approach. The mixture of research on Romance languages, Semitic languages, and Germanic languages is unique in its scope and leads to fruitful synergistic effects. While English, so often ubiquitous, is indeed at the center of some of the studies, a particular asset of this volume is not only the variety of contact settings included, but also the range of subject languages under investigation. Furthermore, the inclusion of Arabic studies enriches the volume due to the increasing relevance of Arabic in contact settings, for instance in the context of migration.

A particular focus is on past and present contact scenarios between languages of unbalanced political and symbolic power. Insights from diachronic and synchronic language-contact research have important linguistic and societal implications, especially considering current global migration streams. The contributions in this volume describe phenomena of language contact between and with Romance languages, Semitic languages, and various forms of English. This sees a diverse range of contact constellations and settings covered in the volume (these are grouped together here but still roughly follow the order of the chapters): English and Bulgarian (Slavova); English and Spanish (in Mexico and Spain: Bäumler; in Gibraltar: Rodríguez García); Moroccan Arabic and French (Falchetta); Pennsylvania German English and Canadian English (Neuhausen); Arabic and Hebrew (Hawker), Baṭḥari and Arabic (Gasparini); German, French, and West African languages (Dombrowsky-Hahn \& Fanego Palat); Spanish, French, Bantu languages, the Portuguese-based creole Fá d’Ambô, and the English-based pidgin Pichi (Schlumpf); Manitoban French and English (Hennecke); and Old Spanish and Arabic (Döhla). 

\section{Structure and scope of this volume}

The contributions in this volume are grouped into four thematic sections, depending on the respective main focus of each chapter: I) globalization and power relations; II) language contact and group identity; III) migration and language contact; and IV) microlinguistic outcomes of contact. The volume concludes with a final discussion chapter by Yaron Matras, who brings together the different perspectives from the individual chapters and consolidates them in his integrated theory of language contact (\citealt{matras_language_2009}/\citeyear{matras_language_2020}).

The first thematic section of the volume is concerned with language contact scenarios involving asymmetrical power relations due to globalization. Many factors, including transnational flows, migration (see Section III of this volume), displacement, tourism, and the internet, have contributed to an increase in language contact scenarios around the world – scenarios that have been extensively discussed in the research community (see, for example, the contributions in \citealt{collins_globalization_2009} and \citealt{coupland_handbook_2010} for an overview; also see \citealt{blommaert_sociolinguistics_2010}). The three contributions in this section focus specifically on settings involving languages in uneven power relations, such as local and global linguistic forces in the urban landscape of Sofia, Bulgaria (Slavova), the language of the former colonizer – namely, French – in the Moroccan setting (Falchetta), and a diaglossic situation involving various languages – British English, Gibraltar English, Andalusian Spanish, and Yanito – in Gibraltar (Rodríguez García).

The section begins with Emilia Slavova’s survey on translanguaging in the linguistic landscape of Sofia. This mainly involves Bulgarian and English, but also includes other languages, and becomes all the more salient in the script choices employed by venues in the inner city. Next, Jacopo Falchetta examines the use of French lexical items in colloquial Arabic in Morocco, showing how they fulfil prestige-related functions while also (potentially) contributing to language learning. These forms of code-switching and code-mixing thus constitute part of the habitual and indexical linguistic practices of the surveyed speakers. Code-switching practices are also at the center of Marta Rodríguez García’s contribution on Yanito in Gibraltar. Rodríguez García focuses on young adults – a sociodemographic group whose sociolinguistic realities have been neglected in previous linguistic research – and shows how thematic factors and topics, discursive elements, and idiosyncratic factors influence switching between the various languages present in the linguistic ecology of Gibraltar. 

Section II focuses on the connection between language contact and group identities and ideologies. The three chapters in this section show that the local evolution of linguistic repertoires is subject to changes related to different factors – demographic, social, economic, political, and intrinsically linguistic. Such changes affect individual linguistic repertoires or idiolects, through the filter of the speaker’s attitudes, and dominant ideologies, and the extent to which such ideologies circulate within and across groups. In turn, regular individual patterns of behavior converge toward a norm or practice routine. Miriam Neuhausen investigates the performative use of Canadian Raising patterns by Old Order Mennonites in southern Ontario. She combines quantitative and qualitative analyses and shows that members of the community manipulate the use of linguistic features to foreground their Mennonite identity or show identification with the Canadian mainstream (it being understood that an individual’s behavior is related to the extent of their exposure to English). The dialect contact situation described involves a closed community of Pennsylvania German English speakers and the wider community of Canadian English speakers, affording Neuhausen a unique look at the local and translocal forces at play. The community under investigation in Fabio Gasparini’s chapter is also closed, and in fact even smaller, as his informants make up the last ten speakers of Baṭḥari in Oman. Gasparini shows how vernacular and standard varieties of Arabic have replaced Baṭḥari in every domain within the local community and presents this linguistic shift as part of a process of identity reshaping and a shift toward the local Bedouin Arabic culture. In other words, language change is related to demographic, economic, and social changes in the Baṭāḥira community and the nation-state ideologies that accompany them. Community boundaries and nation-state ideologies also play an essential role in Nancy Hawker’s chapter on the use of Arabic and Hebrew by Palestinian and Jewish Israeli politicians. Hawker analyses the pragmatic functions related to the use of Arabic by Palestinian politicians in the Knesset (the Israeli parliament) and aptly shows how the extent to which existing norms are manifested, contested, and renegotiated by speakers ultimately depends on the speakers’ relative power. However, this power is related to their identities, and their linguistic behavior depends on which of these identities the speakers intend to foreground in any given interaction. Altogether, the three chapters in Section II deal with language ideologies, identification processes, and the construction of the speakers’ identities in relation to their in-group and out-group relations. They show that language, as a means through which boundaries are constructed and communicated, allows for the erasure of difference – through accommodation (Neuhausen) and language shift (Gasparini) – or the foregrounding of difference – through maintenance of diverging features (Neuhausen) and the display of multilingualism (Hawker).

Section III highlights two studies investigating matters of language contact and migration. Migration and diaspora have been widely theorized and investigated in linguistic work on contact settings (see, e.g., \citealt{piller_language_2016} for a compilation of influential publications). Ever more important roles in settings involving language contact and migration are played here by the notions of transnational labor migration (e.g., \citealt{lorente_scripts_2018}), neoliberalism and postmodern societies (e.g., \citealt{allan_neoliberalism_2017}), and superdiversity (e.g., \citealt{creese_routledge_2018}). Both chapters in this section focus on African migrant groups in Europe, but adopt very different methodological approaches and analytical perspectives. Klaudia Dombrowsky-Hahn and Axel Fanego Palat analyze the history of language acquisition processes in the migration trajectories of people from West Africa who settled in the German Rhine-Main region, and show that linguistic interferences in their use of German and French can be attributed to their wide language repertoires as mobile speakers rather than their L1 only. Sandra Schlumpf then explores the attitudes of Equatoguinean migrants in Madrid toward the different languages spoken in Equatorial Guinea, explaining these attitudes on the basis of language ideologies and the sociohistorical and cultural background of the speakers. She also looks at the questions of power, prestige, and social inclusion/exclusion implied in such attitudes, with the aim of questioning traditional colonial and Eurocentric hierarchies, ideologies, and approaches. 

Section IV features three chapters, each of which discusses individual structural phenomena that are the result of specific language contact situations. The contributions are concerned with different aspects of linguistic systems, from (morpho)syntax to phonetics. Inga Hennecke discusses processes of contact-in\-duced language change in pragmatic markers, using data from Franco-Manitoban, a contact variety of Canadian French isolated for a long time from other varieties of French and in strong language contact with English. Hennecke explains why English and French markers show different outcomes and why certain markers undergo contact-induced language change, while others of the same system do not. Linda Bäumler investigates the variation in integration of Anglicisms among Spanish-speaking natives from Mexico and Spain. By analyzing the phonetic realization of a given grapheme, and whether speakers imitate the English model or follow the Spanish equivalent, Bäumler demonstrates that the urban versus rural variable is mainly responsible for the variation, while regional differences are neutralized in the era of globalization. The typologically driven chapter by Hans-Jörg Döhla on TMA-marking in Spanish closes the section. He considers the grammaticalization pathways of the morphological devices observed in the tense–aspect intersection on the basis of Old Spanish, as translated from Arabic literature and the Hebrew bible in the Middle Ages, and additional data from Spanish creole languages.

The volume concludes with a final discussion chapter by Yaron Matras, who draws together the different strands of research presented in the individual contributions and describes the emerging meta themes. Besides providing a comprehensive synthesis of the presented approaches to language contact, the discussion chapter offers an outlook on where the field is headed and incentives for follow-up studies. This synthesis by Matras, who has broad expertise in the fields of contact linguistics and multilingualism, not only consolidates the findings but also adds yet another layer of insight to the exploration of cross-language contact dynamics, which is particularly important in light of the plethora of contact settings covered in the volume. 

\section*{Acknowledgements}

We are very grateful for the financial support provided by the \textit{WiN-Academy of the University of Bayreuth}, which enabled us to host the \textit{Language Contact through Time and Space} conference in 2021, on which the contributions in this edited volume are based. In particular, we would like to thank Mabel Braun for her patience and encouragement in our endeavors.
 
We would also like to express our gratitude to all our colleagues who were involved in the peer review process and who provided invaluable feedback and input. Tobias Berner and Davide Frapporti provided invaluable help with LaTeX. Our thanks also go to the series editors at Language Science Press, Isabelle Léglise and Stefano Manfredi, who have been most helpful in the process of putting this volume together. And, of course, \textit{thank you}, \textit{Danke}, \textit{merci}, \textit{gracias}, and \textarab{شكرًا} to our contributors for filling this volume with their insights and perspectives on language contact around the world.

The volume editors and authors of the introductory chapter are listed in alphabetical order, which is indicative of equal contribution.

\printbibliography[heading=subbibliography, notkeyword=this]

\end{document}
