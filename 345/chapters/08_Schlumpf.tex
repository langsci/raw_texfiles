\documentclass[output=paper]{langscibook}
\ChapterDOI{10.5281/zenodo.10497383}

\author{Sandra Schlumpf \orcid{0000-0001-6316-1694} \affiliation{University of Basel}}


\title[Asymmetrical power relations between languages of Equatorial Guinea]{Asymmetrical power relations between languages of Equatorial Guinea: Views from the migration context in Madrid}

\abstract{Equatorial Guinea is a small yet culturally and linguistically diverse country in Central Africa. It is characterized by complex contact situations between languages that differ widely in terms of status, prestige, and official recognition: three official European languages (Spanish, French, and Portuguese), several African languages (which all belong to the Bantu family), and two languages of mixed origins (the Portuguese-based Creole Fá d’Ambô and Pichi, a local variety of English). This linguistic setting shows several asymmetrical power relations, for example between European and African languages, between the African languages of different ethnic groups, and between standard English and Pichi.
In this chapter, I focus on language attitudes of Spanish-speaking Equatoguineans toward three languages spoken in their country: Bubi, Fang, and Pichi. I work with a corpus of 24 semi-structured life-story interviews conducted between 2017 and 2018 with Equatoguineans living in Madrid, Spain. I explain the observed language attitudes in this interview corpus with the help of language ideologies and the sociohistorical and cultural background of the speakers. This allows me to recognize specific implications of their language attitudes, which sometimes show clear asymmetries in the distribution of prestige and power between the languages analyzed. Finally, I mention specific problems relevant to the topic of this chapter, such as the maintenance of colonial hierarchies in postcolonial times, the transmission of conservative and often Eurocentric language ideologies and classifications of language phenomena, the projection of Western standards on African realities, and the lack of understanding of completely individual linguistic settings.
}


\IfFileExists{../localcommands.tex}{
   \addbibresource{../localbibliography.bib}
   % add all extra packages you need to load to this file

\usepackage{tabularx,multicol}
\usepackage{url}
\urlstyle{same}

\usepackage{listings}
\lstset{basicstyle=\ttfamily,tabsize=2,breaklines=true}

\usepackage{langsci-basic}
\usepackage{langsci-optional}
\usepackage{langsci-lgr}
\usepackage{langsci-osl}
% \usepackage{./langsci/styles/langsci-lgr}
% \usepackage{./langsci/styles/langsci-osl}
% \usepackage{langsci-gb4e}

\usepackage{tikz}
\usetikzlibrary{patterns,calc}
\pgfdeclarepatternformonly{south east lines}{\pgfqpoint{-0pt}{-0pt}}{\pgfqpoint{3pt}{3pt}}{\pgfqpoint{3pt}{3pt}}{
    \pgfsetlinewidth{0.6pt}
    \pgfpathmoveto{\pgfqpoint{0pt}{3pt}}
    \pgfpathlineto{\pgfqpoint{3pt}{0pt}}
    \pgfpathmoveto{\pgfqpoint{.2pt}{-.2pt}}
    \pgfpathlineto{\pgfqpoint{-.2pt}{.2pt}}
    \pgfpathmoveto{\pgfqpoint{3.2pt}{2.8pt}}
    \pgfpathlineto{\pgfqpoint{2.8pt}{3.2pt}}
    \pgfusepath{stroke}}
    
\usepackage{stmaryrd}
\usepackage{wasysym}
\usepackage{multirow}
\usepackage{caption}
\usepackage{subcaption}
\usepackage{mathrsfs}
\usepackage{qtree}

\usepackage{linguex}


   %pminos do not split footnotes
% \interfootnotelinepenalty=10000 %Footnote in Laporte chapters has to be split SN


%\DeclareIndexNameFormat{default}{%
%\nameparts{#1}%
%\usebibmacro{index:name}%
%{\index[names]}%
%{\namepartfamily}%
%{\namepartgiveni}%
% {}% L1
% {}% L2
%{\namepartprefix}% generates spurious space L3
%{\namepartsuffix}% generates spurious space L4
%}

%  {\DeclareIndexNameFormat{default}{%
%     \usebibmacro{index:name}{\index[names]}{#1}{#3}{#5}{#7}}}

%\DeclareIndexNameFormat{default}{%
%  \usebibmacro{index:name}{\sindex[nom]}{#1}{#3}{#5}{#7}}

%\DeclareIndexNameFormat{default}{%
%  \usebibmacro{index:name}{\sindex[person]}{#1}{#3}{#5}{#7}}
%\DeclareIndexNameFormat{default}{%
%\nameparts{#1} \usebibmacro{index:name}{\sindex[person]]}{\namepartfamily}{‌​\namepartgiven}{\nam‌​epartprefix}{\namepa‌​rtsuffix}}

%\newcommand{\smiley}{:)}

%\renewbibmacro*{index:name}[5]{%
%\usebibmacro{index:entry}{#1}%
%{\iffieldundef{usera}{}{\thefield{usera}\actualoperator}\mkbibindexname{#2}{#3}{#4}{#5}}}

% \newcommand{\noop}[1]{}

%remove for final
%\overfullrule=1mm

\newcommand{\tobi}[2]}}
\renewcommand{\S}[1]{\tobi{#1}{\textsc{*}}}

% this volume references
% puts: [this volume]
% already defined: \citetv
%\newcommand{\citepv}[1]{(\citeauthor{#1} \citeyear*{#1} [this volume])}
\newcommand{\citealtv}[1]{\citeauthor{#1} \citeyear*{#1} [this volume]}

%parentheses around example number
\newcommand{\pref}[1]{(\ref{#1})}

% in-text examples

\newcommand{\lnex}[1]{\textit{#1}} %target lang word
\newcommand{\lnlit}[1]{(lit.: `#1')} %literal reading
\newcommand{\lnlat}[1]{(#1)} % latinization
\newcommand{\lntrans}[1]{`#1'} %translation
\newcommand{\lnexl}[2]%
{\lnex{#1}{} \lnlat{#2}} % ex with latinization
\newcommand{\lnexlat}[3]{\lnex{#1}{} \lnlat{#2}{} \lntrans{#3}} % ex with latinization and tranl.

%ch01
\newcommand{\co}[1]{\mbox{\textbf{#1}}}

%ch09

\newcommand{\cyrbulg}[1]{\begin{otherlanguage*}{bulgarian}#1\end{otherlanguage*}}


%ch10
\newcommand{\nlp}{{\small NLP}}
\newcommand{\mwe}{{\small MWE}}
\newcommand{\rae}{{\small RAE}}
\newcommand{\lvc}{{\small LVC}}
\newcommand{\pos}{{\small P}o{\small S}}
%\newcommand{\todo}[1]{ \textcolor{red}{#1} }

%\renewcommand{\labelenumi}{\theenumi}
%\ainamefmt{{vv}{ll}{, ff}{, jj}} % fullname

\newcommand{\biberror}[1]{{\color{red}#1}}

\newcommand{\osenovaitem}{--~}
   %% hyphenation points for line breaks
%% Normally, automatic hyphenation in LaTeX is very good
%% If a word is mis-hyphenated, add it to this file
%%
%% add information to TeX file before \begin{document} with:
%% %% hyphenation points for line breaks
%% Normally, automatic hyphenation in LaTeX is very good
%% If a word is mis-hyphenated, add it to this file
%%
%% add information to TeX file before \begin{document} with:
%% %% hyphenation points for line breaks
%% Normally, automatic hyphenation in LaTeX is very good
%% If a word is mis-hyphenated, add it to this file
%%
%% add information to TeX file before \begin{document} with:
%% \include{localhyphenation}
\hyphenation{
    Beck-man
    Ngu-yen
    back-chan-nel
    back-chan-nels
    mo-not-o-nous
    ste-reo-typ-i-cal
}

\hyphenation{
    Beck-man
    Ngu-yen
    back-chan-nel
    back-chan-nels
    mo-not-o-nous
    ste-reo-typ-i-cal
}

\hyphenation{
    Beck-man
    Ngu-yen
    back-chan-nel
    back-chan-nels
    mo-not-o-nous
    ste-reo-typ-i-cal
}

   \boolfalse{bookcompile}
   \togglepaper[8]%%chapternumber
}{}


\begin{document}
\maketitle

\section{Introduction}\label{schlumpf:sec:intro}
Equatorial Guinea is the only Spanish-speaking country in Africa today, aside from the sensitive case of Western Sahara. It is located on the geographical and ideological periphery of the Spanish-speaking map and is often forgotten or treated as a small, exotic appendix in linguistic handbooks on the Spanish language. This is also true of Equatorial Guinea’s history, cultures, and, in particular, literature (e.g., \citealt{mbomio_bacheng_originalidad_2011}; \citealt{trujillo_historia_2012}; \citealt{repinecz_raza_2019}; \citealt{riochi_siafa_guinea_2021}). This situation of invisibility has been intensified by historical and political circumstances since the country gained independence in 1968, such as both Equatoguinean regimes (especially the first dictatorship of Francisco Macías Nguema Bidyogo from 1968 to 1979), the Francoist dictatorship during the first wave of mass emigration from Equatorial Guinea to Spain, and the categorization of Equatorial Guinea as “classified material” by the Spanish Law of Official Secrets in the 1970s \citep[290-292]{schlumpf_construccion_2019}. Even today, people in Spain show little interest in this former colony in Africa, and as such, they have very limited knowledge about it. The country and its history are hardly ever part of the school or university curriculum, and it is only recently that a more critical, postcolonial, or even decolonial perspective on Spain’s colonial role in Equatorial Guinea has emerged (e.g., \citealt{aixela-cabre_colonial_2020}).

Equatoguinean society is characterized by its multiethnicity and multilingualism, representing a good example of the African polyglossia pyramid described by \citet[210]{wolff_language_2016}. Despite the small size of the country 
(approximately 28,000 km$^2$), it presents a diverse linguistic setting with languages of different origins, levels of official status, and functionalities. Several asymmetrical power relations can be observed (cf. \citealt{martin_rojo_language_2017}), for example between the African languages of the Bantu family and the official European languages, between the African languages of different ethnic groups, between the local English variety known as Pichi\footnote{There are different hypotheses about the linguistic status of Pichi, the most frequent are those who describe it as a Pidgin English \citep{lipski_pidgin_1992}, an Afro-Caribbean English Lexifier Creole (\citealt{yakpo_pichi_2013}; \citealt{yakpo_wayward_2013}) or as a (new) language \citep{lipski_isnt_2012}. As the aim of this paper is not the discussion of the linguistic classification of Pichi, I use the term \textit{variety} as a more neutral option.} and standard English, and between Equatoguinean Spanish and other varieties of Spanish, especially the Peninsular variety of the former colonial power.

In this chapter, I focus on Equatoguineans’ language attitudes toward different languages spoken in their country, a topic which has barely been analyzed to date.\footnote{Some exceptions are the classical studies of the 1980s and 1990s done by Antonio Quilis and Celia Casado-Fresnillo (\citealt{quilis_actitud_1983}; \citeyear{quilis_nuevos_1988}; \citealt{quilis_lengua_1993}), the very short follow-up study by \citet{chirila_identidad_2015}, and the publications by Schlumpf (\citeyear{schlumpf_guineoecuatorianos_2018}; \citeyear{schlumpf_african_2020}; \citeyear{schlumpf_espanol_2020}).}  What language perceptions and attitudes can be found today within the Equatoguinean community living in Madrid? Which functions, values, and levels of prestige do Equatoguineans connect to different languages? Which positive and negative associations arise? I will look at these questions with regard to three languages spoken in Equatorial Guinea: two Bantu languages (Bubi and Fang) and Pichi. None of them has the status of an official language in contemporary Equatorial Guinea.

In the following section, I present my corpus of sociolinguistic interviews, which forms the core of this study. Then, after a review of the importance of investigating language attitudes in postcolonial contexts (Section \ref{schlumpf:sec:importance}), I will continue with a description of the attitudes of Equatoguineans toward two African Bantu languages: Bubi and Fang (Section \ref{schlumpf:sec:attitudes}). I will then offer an excursus concerning the use of the terms \textit{lengua} ‘language’ and \textit{dialecto} ‘dialect’ in my corpus (Section \ref{schlumpf:sec:terminology}). Finally, I will present the language attitudes toward Pichi (Section \ref{schlumpf:sec:pichi}). The chapter finishes with a short conclusion (Section \ref{schlumpf:sec:conclusion}).

\section{Corpus of sociolinguistic interviews with Equatoguineans in Madrid}
\label{schlumpf:sec:corpus}

The results that I present in the following sections are based on my corpus of semi-directed, sociolinguistic interviews, which I conducted with Equatoguineans in Madrid, Spain, from 2017 to 2018. All the interviewees were born in Equatorial Guinea but were living in the Autonomous Community of Madrid at the time of their interview. The Equatoguinean community in Spain is of special relevance because it is the largest outside the country. As of January 1, 2021, approximately 23,000 people born in Equatorial Guinea were living in Spain. Almost 40 percent (8,791 people as of January 1, 2021) were registered in the Autonomous Community of Madrid, hence the interest of this central region around the Spanish capital for this survey.\footnote{For more information about the interview corpus and the Equatoguinean community in Spain, see \citet[349-358]{schlumpf_spanisch_2021}. The latest demographic data are drawn from the Spanish National Institute of Statistics: \url{https://www.ine.es/dynt3/inebase/es/index.htm?type=pcaxis&path=/t20/e245/p08/&file=pcaxis&dh=0&capsel=1} (accessed in March 2022).} 

The corpus of interviews with Equatoguineans on which this study is based was collected in a migration context. The interviews are composed of different thematic modules around the following main topics: the interviewees’ arrival in Spain, their past in their country of origin (Equatorial Guinea), their adaptation to life in Madrid and differences to their home country, their education and professional situations, their families, and their plans and expectations for the future. All the modules include linguistic and sociolinguistic questions, for example concerning the use and functions of different languages, the prestige and values of languages and varieties, and specific dialectological features.

In total, I carried out 46 interviews, all of which were between 40 and 90 minutes in length. From these, I selected 24 interviews that form my main corpus. These 24 interviews are evenly distributed among three sociolinguistic variables (see Table 1): ethnic group (B = Bubi/F = Fang), sex (H = male [\textit{hombre}]/M = female [\textit{mujer}]), and duration of residence in Spain (–8 = up to eight years/+8 = more than eight years). For each possible combination of the three variables, the main corpus contains three participants.\footnote{The same three variables are used to establish the participants’ codes, meaning the code of the first participant is 01\_–8HB (= number of the participant + time spent in Spain + sex + ethnic group). These codes identify each speaker providing the quotes presented in this chapter.}


\begin{table}[h]
\caption{Equatoguineans in Madrid: composition of the main corpus (code of participant number 1: 01\_–8HB).}
\label{schlumpf:tab:1}
\begin{tabular}{llllll}\midrule\toprule
&\multicolumn{2}{c}{Bubi (B)}                                                                                                                                                                             
&\quad & \multicolumn{2}{c}{Fang (F)}\\[3pt]
& Male (H)  & Female (M) &&Male (H)  & Female (M)\\
&[\textit{hombre}]  & [\textit{mujer}] &&[\textit{hombre}]  & [\textit{mujer}]\\
\midrule
Up to 8 years in&	3 [01–03]	&3 [04–06]&& 3 [13–15]	&3 [16–18]\\
 Spain (–8)&&&&&\\[3pt]
More than 8 years &	3 [07–09]&	3 [10–12]&& 3 [19–21]&	3 [22–24]\\
 in Spain  (+8) &&&&&\\
\bottomrule\midrule
\end{tabular}
\end{table}

All interviews were audio recorded, after which the 24 interviews for the main corpus were transcribed entirely.\footnote{For the most part, the transcriptions follow the recommendations created for the \textit{Proyecto para el Estudio Sociolingüístico del Español de España y de América} (PRESEEA), summarized in \citet{moreno_fernandez_marcas_2021}. However, in this chapter, I present the examples in their unlabeled version (i.e., without metalinguistic annotations) and using some additional marks: “/” = short pause; “//” = longer pause; “-" = interruption; “:” = lengthening; “h:” = exhalation; “.h” = short inhalation; “.h:” = longer inhalation. The codes “Ie.” and “Ir.” refer to the interviewee and the interviewer, respectively.}  In addition, the main corpus was prepared for digital processing and analysis. Firstly, all utterances of the Equatoguinean interviewees were digitized and lemmatized using SketchEngine (corpus size: 185,185 words).\footnote{SketchEngine (https://www.sketchengine.eu/) is a platform that provides both corpus management and advanced text analysis. It was developed for linguistic researchers, lexicographers, translators, language teachers, and students. Tagging of texts in Spanish is supported by the FreeLing morphological tagger with a morphological dictionary obtained from different open-source projects with over 555,000 word forms. More details about SketchEngine can be found in \citet{kilgarriff_sketch_2014}.}  Secondly, they were introduced into the online spreadsheet–database hybrid Airtable,\footnote{Airtable (https://airtable.com/) is a free online spreadsheet–database hybrid. It enables the user to annotate all concordances in a database, making them easy to retrieve, change, or extend, and includes all the functionalities common in spreadsheet tools, such as straightforward importing or exporting and intuitive filtering and sorting of entries. In addition, Airtable supports easy annotation schema definition with checkboxes, drop-down lists, and comments.} which allowed me to create concordances and annotate and analyze the linguistic, discursive, and thematic phenomena.

The results presented in the following section are based on the qualitative analysis of the 24 interviews and on the concordances in Airtable, which contain all the occurrences of the names of the languages studied, and certain related linguistic terms: \textit{lengua} ‘language’, \textit{idioma} ‘language’, \textit{dialecto} ‘dialect,’ and \textit{lenguaje} ‘language’.

\section{Language attitudes toward different languages spoken in Equatorial Guinea}\label{schlumpf:sec:language}

\subsection{On the importance of language attitudes in postcolonial and decolonial studies}\label{schlumpf:sec:importance}

In this chapter, I understand \textit{attitudes} as follows, according to \citet[596, emphasis in original]{gallois_attitudes_2007}:

\begin{quote}
This concept […] represents the judgements that people tend to make and generalise about an object (social or otherwise) outside themselves. Attitudes are formed toward a particular entity, called the attitude object, and are generally theorised to contain three main components: cognitive, conative (behavioural), and affective. The \textit{cognitive component} in attitude formation and maintenance represents a person’s beliefs and thoughts about the attitude object, without any positive or negative tone. The \textit{conative component} is a predisposition to behave in accordance with the beliefs. Finally, the \textit{affective component} represents an emotional reaction, positive or negative, that accompanies the beliefs.
\end{quote}

More precisely, according to \citet[177-178]{moreno_fernandez_principios_2015}, a \textit{language attitude} refers to the following:

\begin{quote}
[A] manifestation of the social attitude of individuals, distinguished by focusing on and referring specifically to both the language and the use made of it in society […]. The attitude toward language and its use becomes especially attractive when one appreciates to its full extent the fact that languages are not only carriers of certain linguistic forms and attributes, but are also capable of transmitting social meanings or connotations, as well as sentimental values. Cultural norms and marks of a group are transmitted or emphasized through language.\footnote{Own translation. The original quote is: “La \textit{actitud lingüística} es una manifestación de la actitud social de los individuos, distinguida por centrarse y referirse específicamente tanto a la lengua como al uso que de ella se hace en sociedad […]. La actitud ante la lengua y su uso se convierte en especialmente atractiva cuando se aprecia en su justa magnitud el hecho de que las lenguas no sólo son portadoras de unas formas y unos atributos lingüísticos determinados, sino que también son capaces de transmitir significados o connotaciones sociales, además de valores sentimentales. Las normas y marcas culturales de un grupo se transmiten o enfatizan por medio de la lengua.”}
\end{quote}

Therefore, the analysis of language attitudes gives important insights into the values that certain languages and linguistic varieties have in a specific context, and about what roles languages play in the identity of individuals and speech communities. I consider this a very important question, especially in postcolonial contexts, where usually several, very different languages – and different communities with their specific cultural and linguistic backgrounds – exist together in a certain space and come into contact on a daily basis.

This approach to language attitudes is strongly connected to the study of language ideologies, understood as systems of ideas that connect linguistic phenomena with concrete cultural, political and/or social contexts \citep[19–20]{del_valle_glotopolitica_2007}. As already stated by \citet[518]{blommaert_language_2006}:

\begin{quote}
The study of language ideology grew out of linguistic anthropology and shares the basic preoccupation in this tradition of investigating the nexus of language and culture. It does so by introducing another level of cultural structuring in language: the language-ideological, indexical metalinguistic level. This level drives the development of linguistic structure […] and it organizes the social, political, and historical framing of language and language use.
\end{quote}

Both language attitudes and language ideologies are always bound to a particular cultural context, which is “structured across intersubjective and institutional scales” \citep[117]{garcia_language_2017}. Therefore, language attitudes, which also reflect the language ideologies of the speakers, allow us to understand what functions and values different languages in multilingual contexts have at a precise moment in time. This enables us to uncover possible asymmetries and power relations between languages and language communities, and how these are reflected in the beliefs of the speakers \citep[1]{bouchard_postcolonial_2022}. Terminology is also important in this regard (see Section \ref{schlumpf:sec:terminology}).

Finally, the study of language attitudes of speakers of a lesser-known variety of the Spanish language responds to the call to decolonialize current knowledge in the sense of making visible other ways of seeing that have not been shown before. \citet{lara_delgado_pensamiento_2015} calls this the \textit{empowerment of the epistemologies} which traditionally have been silenced, whereas \citet[318]{deumert_grammar_2020} talks about the “willingness to take seriously the knowledge traditions of the dispossessed”. This should lead to the revision and correction of established knowledge. In other words, “it is necessary to incorporate other analyses that allow us to have a general view and visibilize other components that are established through: epistemic, spiritual, racial/ethnic and gender/sexual hierarchies; products of […] the European/Euro-North American capitalist/patriarchal modern/colonial system” \citep[49–50]{gomez_velez_estudios_2017}.\footnote{Own translation. The original quote is: “Es necesario incorporar otros análisis que permitan tener una mirada general y visibilizar otros componentes que se establecen a través de: las jerarquías epistémicas, espirituales, raciales/étnicas y de género/sexualidad; productos del […] sistema europeo/euro-norteamericano capitalista/patriarcal moderno/colonial.”}  In fact, placing the focus on the Spanish-speaking Equatoguineans affords insights into a community hardly ever studied from a sociolinguistic perspective. Thus, underrepresented voices and beliefs become visible. Moreover, a decolonial perspective makes it possible to disclose colonial legacies in the description and recognition of languages and communicative practices. In fact, until today: “Representations and meanings attached to languages, linguistic forms, and practices structure and stratify social spaces in a way that reproduces colonial hierarchies” \citep[5]{bouchard_postcolonial_2022}. These consequences of coloniality and colonial power asymmetries in linguistic contexts are what Veronelli calls the \textit{coloniality of language} (see \citealt{veronelli_coloniality_2015}; \citeyear{veronelli_sobre_2016}; \citeyear{veronelli_colonialidad_2019}).

In order to find out about possible relations, disencounters, and asymmetrical power relations between different languages spoken in Equatorial Guinea, I will look at the language attitudes toward Bubi, Fang, and Pichi expressed by Equatoguineans who live in the Autonomous Community of Madrid, the political heart of the former colonial power.

\subsection{Equatoguineans’ attitudes toward two Bantu languages: Bubi and Fang}\label{schlumpf:sec:attitudes}

Bubi and Fang are the two most frequently spoken Bantu languages in Equatorial Guinea. Both are part of the Bantu languages of Zone A according to Guthrie’s famous classification of the Bantu languages, and therefore are included in the so-called Western Bantu languages (cf. \citealt{guthrie_bantu_1953}; \citeyear{guthrie_western_1971}).

In my 24 interview transcripts, the term \textit{bubi} appears 388 times; 191 occurrences refer to the Bubi language, while the other 197 refer to a Bubi person or the Bubi ethnic group. In the case of \textit{fang}, the total number of occurrences in the interviews is 528, of which 283 refer to the Fang language and 240 to the Fang ethnic group or a Fang person; a few cases are not entirely clear. I focus mainly on the occurrences that refer to the Bubi and Fang languages, although in some cases, I also rely on instances that appear in relation to the Bubi and Fang people.

In what follows, I will examine the positive and negative topics that appear in relation to the two languages. The clearly positive opinions stand out as proof that Bubi and Fang are cultural elements of great importance for their speakers. The following four points can be highlighted.

Firstly, several Equatoguinean interviewees associate Bubi and Fang with personal and cultural values such as identity and pride, roots, origin, customs, local music, and food (see Example \ref{schlumpf:ex1}).

\begin{exe}\ex\label{schlumpf:ex1}
%\begin{quote}(1)
	\textit{[hablando de sus hijos hipotéticos:] Ie.: […] intentaría / si puedo económicamente // que todas las vacaciones se vayan a Guinea // porque creo que hay valores de Guinea que no tienen que perder […] en casa les intentaría educar / a la manera de Guinea / entre medias […] intentaría que hablen bubi […] cuando consiga yo aprender […] y que sepan de Guinea / que sepan sus costumbres su cultura // para que estén orgullosos de dónde son que in- intentar que no lo renieguen / eso: y: eso es lo que intentaría} (03\_-8HB)\newline
[talking about his hypothetical children:] Ie.: […] I would try / if I can afford it // that they go for every vacation to Guinea // because I believe that there are Guinean values that don’t have to be lost […] at home I would try to raise them / the Guinean way / in part […] I would try to make them speak Bubi […] when I manage to learn it […] and that they know about Guinea and its customs its culture so that they can be proud of where they are from try that they don’t deny it / that and that’s what I would try (03\_-8HB).
%\end{quote}
\end{exe}

The two Bantu languages represent a connection with the speakers’ country of origin and relatives who still live in Equatorial Guinea. The Bubi language is a way for the speakers to move mentally and emotionally to Equatorial Guinea, as a form of therapy and to remember; it is associated with the Bubi culture, music, and traditional food. The Fang language is an important “mother tongue” and therefore something peculiar and important (e.g., “I am Fang and being Fang I consider that I have to learn and know Fang”, 23\_+8MF). Without any doubt, both languages are integral parts of the multicultural tradition of Equatorial Guinea.

Secondly, some speakers cannot express certain things in a language other than the African languages (see Example \ref{schlumpf:ex2}). These languages are described by the speakers as something that is their own (\textit{yours, theirs}), as opposed to Spanish, which is after all a language imposed during the time of colonization.

\begin{exe}\ex\label{schlumpf:ex2}
%\begin{quote}(2)	
\textit{Ie.: […] [el bubi] tiene otros componentes que no tiene el español entonces eeh tiene: / .h o: o sea te sientes un poco mejor expresándote en bubi que en español porque .h: el bubi lo sientes como una lengua propia tuya […] lo otro como una lengua: .h eeh extraña porque hay palabras muchas veces que quieres decir y: y no puedes pero en bubi sí:} (08\_+8HB)\\
Ie.: […] [Bubi] has other components that Spanish doesn’t have so it has ooh that is you feel a bit better expressing yourself in Bubi than in Spanish because you feel Bubi as your own language […] the other as a ooh strange language because there are words you want to say many times and and you can’t but in Bubi you can (08\_+8HB)
%\end{quote}
\end{exe}

In fact, interviewees from both ethnic groups describe the communicative situations in which they are able to use their African languages in the Spanish migration context as joyful moments, since they feel comfortable speaking and listening to them.

Thirdly, both Bubi and Fang interviewees express their desire to transmit their African languages to their children, also in the context of migration, even though this is rather difficult, as I will analyze later.

Finally, in some interviews, I noted a certain awareness of language revitalization (e.g., rap in Bubi). Individual attempts to recover or even learn African languages are described, particularly in the context of migration. For instance, one young Bubi interviewee always tries to spend more time with her grandmother in order to practice Bubi, and others say that they speak and learn more Bubi or Fang in Spain than when they were living in Equatorial Guinea. It seems, though, that the awareness of the cultural value of these languages is more apparent, at least for some of the Equatoguinean migrants, when the distance to their homeland and own communities is greater.

At first sight, the negative opinions associated with Bubi and Fang seem to outweigh the positive ones. However, it soon becomes clear that many of these comments in fact express a positive attitude toward these African languages, albeit indirectly. The following six topics can be summarized.

The first topic is mentioned by many of the interviewees: they regret the problem of the intergenerational transmission of Bubi and Fang, mainly in the Spanish migratory context, but to a lesser extent also in their country of origin. They point to this as the main challenge related to their African languages and express, almost unanimously, their desire to be able to transmit this linguistic legacy to their children. They find it painful and sad that young people no longer speak their own languages (some still understand them, but do not speak them); as such, several of the interviewees fear that the Bubi language in particular will be lost in the near future (not only in the Spanish diaspora, but maybe also in Equatorial Guinea itself). In the case of Fang, negative practical implications are also noted, since when returning to Equatorial Guinea, a lack of knowledge of Fang can cause communication problems or lead to discrimination. Others, however, think that Spanish is sufficient to get by in Equatorial Guinea. The causes that explain the problem of linguistic transmission in Spain are multiple: lack of contact with other people who speak the same language; mixed, multilingual couples; the different situation of women in Spain with regard to work (and, therefore, the relatively limited time they spend with their children); and the parents’ own lack of linguistic proficiency. Especially in the case of Bubi, it seems that the grandparents’ generation, and grandmothers in particular, are fundamental for the transmission of the language to the younger generations. This same fact is also implicitly present in a number of opinions that associate the Bubi language with the village, the rural environment, and the elderly or the aged.

Whereas this first topic – the problem of linguistic transmission to subsequent generations – indirectly expresses a very positive attitude toward the African languages, the second point is clearly negative. Two Fang interviewees question the usefulness of the Fang language (see Examples \ref{schlumpf:ex3} and \ref{schlumpf:ex4}); they believe that speaking Spanish is enough to live in Equatorial Guinea, and that in Spain or in other countries, Fang does not offer any advantages to them or their children. Another person adds that, since Fang is not an official language in Equatorial Guinea, it has no value outside the country (17\_-8MF).

\begin{exe}\ex\label{schlumpf:ex3}
%\begin{quote}(3)
	\textit{Ie.: no h: no me parece importante de mi cultura […] me parece más importante otras cosas de cultura .h como enseñar al que no sabe // como enseña:r pues las tradiciones no: un: no un dialecto […] no me parece importante un dialecto\\
Ir.: vale / .h si tuvieras hijos por ejemplo: ¿les enseñarías hablar fang también?\\
Ie.: ¿si tuviese hijos aquí: [en España]?\\
Ir.: sí aquí\\
Ie.: no / no les enseñaría hablar fang […] porque el fang pff / ¿qué les va a aportar? […] cero […] si va a Guinea: / y tal no va a tener problema: .h / hablando español // lógicamente sí va a tener problema / que no hable fang // porque dirá alguno “pero si tu padre es fang” y ¿a mí qué?} (19\_+8HF)\\
Ie.: no I don’t think it’s important in my culture [...] I think it’s more important other things of culture like teaching the one who doesn’t know // like to teach well traditions not a not a dialect […] I don’t think that a dialect is important\\
Ir.: okay / if you had children for example would you teach them to speak Fang?\\
Ie.: if I had children here [in Spain]?\\
Ir.: yes here\\
Ie.: no / I wouldn’t teach them to speak Fang […] because Fang pff / what will it bring to him/her? […] zero […] if he/she goes to Guinea / and so he/she won’t have a problem / speaking Spanish // of course he/she will have a problem / with not speaking Fang // because someone will say “but your father is Fang” and so what? (19\_+8HF)
%\end{quote}

\ex\label{schlumpf:ex4}
%\begin{quote}(4)
	\textit{Ie.: […] no no no o sea [el fang] no me aportaría nada […] solo se habla en Guinea en Camerún y en Gabón / y cuando estoy aquí: / no no me aporta nada el fang […] porque: en la calle yo tengo que: defenderme / y si: hablo fang ¿quién me va a entender? ¿quién me va a ayudar? .h o sea que: nada\\
Ir.: mhm mhm / .h // a pesar de ello ¿el fang para ti es algo: / importante?\\
Ie.: sí: claro / porque también forma parte de: / de mi vida […] me identifica que yo soy guineana y soy fang} (18\_-8MF)\\
Ie.: […] no no no I mean [Fang] would not bring me anything […] it is only spoken in Guinea in Cameroon and in Gabon / and when I am here / Fang does not bring me anything […] because I have to defend myself in the street / and if I speak Fang who will understand me? who is going to help me? so nothing\\
Ir.: mhm mhm // despite this Fang for you is something / important?\\
Ie.: yes of course / because it also forms part of / of my life […] it identifies me that I am Guinean and I am Fang (18\_-8MF)
%\end{quote}
\end{exe}

A third point triggering negative comments that comes up in more than one interview has to do with the linguistic interferences between the African languages Bubi and Fang and other languages, especially Spanish. This is linked to a long-lasting tradition of negative language attitudes concerning the multilingual repertoires of Spanish-speaking Equatoguineans. To cite just one example of an early publication about the ``problem of polylingualism'':

\begin{quote}
To all the vernacular languages […] we must add the very particular problem of bilingualism. This, the bilingualism, and, by analogy, polylingualism, is one of the most pressing and complex problems, and that until now has not been given the pedagogical importance it requires, not only because of its psychological effects, but also because of the political and social effects it entails. \citep[52]{castillo_barril_influencia_1969}\footnote{Own translation. The original quote is: “A todas las lenguas vernáculas […] habrá que sumar el problema particularísimo del bilingüismo. Este, el bilingüismo, y, por analogía, el polilingüismo, es uno de los problemas más acuciantes y complejos y que hasta ahora no se le ha dado la importancia pedagógica que requiere, no sólo por sus efectos psicológicos, sino por los políticos y sociales que implican.”}
\end{quote}

Indeed, this opinion is quite widespread, both among certain linguists and among the multilingual speakers themselves, and it is well known from very different linguistic contexts. \citet[307–308]{morales_de_walter_puertorriquenos_2008}, for example, describes the language attitudes of Puerto Ricans toward their daily code-switching between Spanish and English as follows (italics added):

\begin{quote}
Code-switching within or between sentences has received \textit{much criticism}, especially from educators who see in this use \textit{possible deficiencies in one or both languages} or, in any case, carelessness in speaking. The bilinguals themselves are hesitant in their assessment, although in Brentwood 82 percent of the adults alternated Spanish and English, 39 percent of them pointed out that this was \textit{not correct}, and especially criticized the alternation within the same sentence. 77 percent of the students alternated codes and, like their parents, a good proportion of them believed that this was \textit{not appropriate}. Users of code-switching think it is a \textit{problem}, but they justify it. […] In general, parents and children, although they use it, are quite critical when they have to give their opinion on language switching; they think that this is one of the causes that makes their way of speaking \textit{stigmatized}.\footnote{Own translation. The original quote is: “La alternancia de códigos dentro de una oración o entre oraciones ha recibido muchas críticas, especialmente de los educadores que ven en ese uso posibles deficiencias en una o ambas lenguas o, en todo caso, descuido al hablar. Los propios bilingües dudan en su valoración, aunque en Brentwood un 82\% de los adultos alternaba español e inglés, el 39\% de ellos señalaba que eso no era correcto, y criticaban especialmente la alternancia dentro de la misma oración. El 77\% de los estudiantes también alternaban códigos y, como sus padres, una buena proporción de ellos creía que esto no era adecuado. Los usuarios del cambio de códigos piensan que es un problema, pero lo justifican. […] En general, padres e hijos, a pesar de que lo usan, son bastante críticos cuando tienen que dar su opinión sobre la alternancia de lenguas; piensan que esa es una de las causas que hace que su modo de hablar esté estigmatizado.”}
\end{quote}

In a similar way, in my interview corpus, it is stated that Bubi and Fang are being “corrupted” (i.e., mixed), that the Fang language “hinders” Equatoguinean Spanish, and that the mix of languages in discourse is something negative. Although some argue that frequent code switching is simply a natural expression of their multilingual identity, negative attitudes are more frequent.

A fourth negative language attitude is mentioned about the Bubi language. One Bubi interviewee thinks that the Bubi feel inferior when they speak Bubi (they have a “mental problem,” 09\_+8HB). He argues that other people mock them when they speak Bubi, a situation that contrasts with what he observes in other Equatoguinean ethnic groups. Indeed, several people believe the Bubi are submissive and conformist, and that they have been discriminated against since Equatorial Guinea’s independence. This is explained by the fact that both presidents of independent Equatorial Guinea have been Fang, which is the reason for the clear dominance of Fang in politics and practically all areas of life, a phenomenon known as \textit{Fanguization} (cf. \citealt[62–69]{aixela-cabre_colonists_2013}). As some interviewees explain, the Fang are in the government, in the bureaucracy, in everything. They constitute the majority and dominant ethnic group, and they feel somehow superior. The Bubi, on the contrary, are a minority ethnic group with fewer opportunities, they have been largely excluded from power, and they have been persecuted and marginalized for decades. It is interesting to note that, according to \citet[93]{bolekia_boleka_aproximacion_2003}, this division between Bubi and Fang has its roots in colonial times. He explains that the Spanish governors and businessmen told the Bubi negative stories about the Fang (e.g., portraying the Fang as savage, brutal invaders) and vice versa (e.g., portraying the Bubi as weak, lazy, and inferior). During the process of independence, this ethnic conflict was exacerbated.

A fifth negative topic mentioned in some interviews is connected with the previous point. Several interviewees, both Bubi and Fang, confirm and criticize the fact that the Fang always speak Fang or speak it too much. As they explain, today, the Fang are not only the majority ethnic group in the country as a whole – and especially in mainland Equatorial Guinea (Río Muni), which is their original home – but also in the capital Malabo, on the island of Bioko, the traditional territory of the Bubi ethnic group. According to some of the interviewees, the Fang feel that they are already the owners of the capital, and there are some Bubi who want the Fang to return to the continent. It is because of this situation that one Fang participant describes the Fang as “invaders” and “braggarts,” and he adds – laughing – that in Equatorial Guinea, they are called “Japanese” because there are so many of them (20\_+8HF). As a result of this Fang presence all over the country, they apparently always address people in Fang automatically, regardless of whether the other person speaks Fang or not; some even add that the Fang want to impose their language on other ethnic groups. Hence, there is a clear difference in power and status between Fang as the dominant language and the other autochthonous languages of Equatorial Guinea, which makes it more difficult to use the latter in public and formal domains.

This last point has another negative consequence: in the opinion of several interviewees, both Bubi and Fang, the Fang find it difficult to speak Spanish well, precisely because they speak Fang so often. They think that the Fang maintain their language more than the Bubi, that they speak it more often at home and everywhere (both in Equatorial Guinea and in Spain), and that, for this reason, the Fang are not as interested in speaking Spanish as the Bubi. Others like to qualify this idea and specify that linguistic proficiency depends on the education of each person and that in all the ethnic groups, there are some who speak Spanish better than others. Interestingly, the widespread idea that the Fang speak Spanish worse than the Bubi – a kind of shared mental space, or shared image of linguistic reality built on individual experiences and common beliefs – is based not primarily on linguistic facts, but above all on the ethnic, cultural, social, and political differences in Equatorial Guinea. The inferior position of the Bubi in Equatorial Guinea is compensated via the image of their linguistic superiority in Spanish. This shows the importance of taking the background of a linguistic community into account when describing and interpreting their language attitudes. In this case, it is necessary to recognize how the ethnic conflicts in the speakers’ home country are reflected in the language attitudes of Equatoguineans in the Spanish migration context toward both Spanish and the Bantu languages spoken by the two ethnic groups, Bubi and Fang (cf. \citealt{schlumpf_african_2020}).

In summary, it is important to bear the following points in mind regarding attitudes toward the African languages Bubi and Fang:

\begin{itemize}
\item Firstly, it is necessary to emphasize the great cultural value that these languages carry. They are a sign of ethnic identity, hence the frequent allusions to subjects such as culture, customs, and origin, and the idea that they represent a means for Equatoguineans abroad to connect with their home country. Altogether, a positive internal prestige is shown here, which allows identification with the group (\textit{in-group language}).
\sloppy
\item Secondly, and in a more negative sense, it is worth recalling the frequent critiques of the multilingual, mixed repertoires of the Equatoguineans, who naturally switch between different languages in their spontaneous discourse with their compatriots, both in Madrid and in Equatorial Guinea itself.\footnote{More data on the maintenance of the African languages in the migration context can be found in \citet{schlumpf_african_2020}.} This reflects a general lack of recognition of intense language-contact situations and the linguistic implications this has, especially in (post)colonial contexts.
\item Thirdly, when comparing the two Bantu languages studied, clear differences in power are noted. Fang constitutes the dominant language in Equatorial Guinea, because the Fang are the majority group and due to the political circumstances in Equatorial Guinea since its independence. Bubi, on the contrary, is a dominated language. Within the Bubi community, this situation has produced a certain feeling of inferiority on the one hand, and a shared idea of linguistic superiority in Spanish on the other.
\sloppy
\item Finally, in addition to this internal difference between Bubi and Fang concerning status and prestige, I have highlighted that these African languages are in constant opposition to other languages. Spanish is the first official language and the lingua franca of Equatorial Guinea, as well as the dominant language in the migration context in Madrid. The Bantu languages of Equatorial Guinea, in spite of their cultural and linguistic relevance, especially in informal contexts, do not enjoy any form of official status in the country; in some instances, this fact is openly criticized by the Equatoguineans interviewed, but in other cases, it is used to demonstrate the perceived uselessness of these languages, at least outside Equatorial Guinea.
\end{itemize}


\subsection{The use of linguistic terminology as a reflection of colonial power structures: \textit{Lengua} versus \textit{dialecto}}
\label{schlumpf:sec:terminology}

Some of the aspects described in the previous section can be corroborated by how the Equatoguinean interviewees denominate the different languages spoken in their country of origin. In particular, the use of the terms \textit{lengua} and \textit{dialecto} to refer to different languages is highly revealing for the purpose of this topic and exemplifies a widespread phenomenon observed in various contact situations with a hierarchical structure (e.g., in Latin America).\footnote{  	For some further details on this topic and examples of interview passages, see \citet{schlumpf_castizar_2021}.}

The term \textit{lengua}\footnote{The term \textit{lengua} is used a total of 131 times, of which 55 refer to local Bantu languages and 29 to Spanish. The remaining occurrences refer to other languages or to \textit{lengua} in general.} is used to refer to both the African languages and Spanish. However, the most frequent semantic and morphological combinations are different. On the one hand, the references to the African languages (in this case, Bubi and Fang) highlight, above all, their cultural and identity values (e.g., \textit{lengua materna} ‘mother tongue,’ \textit{lengua local} ‘local language,’ \textit{lengua vernácula} ‘vernacular language,’ \textit{mi/tu lengua} ‘my/your language,’ \textit{nuestras lenguas} ‘our languages,’ \textit{sus lenguas propias} ‘their own languages’). On the other hand, the largest number of occurrences that refer to Spanish show the combination \textit{lengua oficial} ‘official language’ (16 of 29 occurrences). Furthermore, on a few occasions, Spanish in Equatorial Guinea is described as a \textit{lengua extranjera} or \textit{lengua extraña} ‘foreign/strange language’ and as \textit{una lengua impuesta por los colonizadores} ‘a language imposed by the colonizers.’ This clearly shows the main association of Spanish with its \textit{official} status in Equatorial Guinea, while the Bantu languages Bubi and Fang represent a symbol of the \textit{local} traditions, whose main communicative settings are Equatoguinean households. As such, Spanish enjoys overt prestige, whereas the two African languages represent important in-group languages and enjoy a kind of covert prestige, according to Labov’s (\citeyear{labov_social_2006}) and Trudgill’s (\citeyear{trudgill_sex_1972}) terminology. This difference reproduces a clearly colonialist view on the sociocultural realities in Equatorial Guinea, a view that over history and in many colonialized countries has led to the imposition of foreign (European) languages to the detriment of local languages. The examples of my interviews demonstrate that this colonial and Eurocentric discourse continues to circulate today, and has even been converted into a properly African discourse.

\largerpage[-1]
If we look at the occurrences of the term \textit{dialecto}\footnote{The term \textit{dialecto} is used a total of 33 times and shows a very clear semantic distribution, since 26 of the 33 cases refer to an African language. The remaining cases are not very clear; they refer to \textit{dialecto} in general or to different dialects of a specific language (e.g., dialects of Spanish).} that appear in the interview corpus, most of them refer to one of the Bantu languages of Equatorial Guinea. Here, the syntactic combinations with possessives stand out (e.g., \textit{mi/su/nuestro dialecto} ‘my/his or her/our dialect’), as does the expression \textit{dialecto de casa} ‘home dialect.’ Spanish and other European languages are never labelled as dialects, except when referring explicitly to a certain diatopic variety. This is highly revealing when considering that “the name ‘dialect’ has often served to mask a situation of linguistic subordination and reinforce power relations, not only between the linguistic varieties, but also between groups of speakers and language communities” \citep[56]{moustaoui_srhir_concept_2016}.\footnote{The same author explains the following: “attitudes towards ‘dialects’ considered socially as non-standard, are attitudes that reflect the structure of the society where they are spoken and the power relations that exist between the various groups that compose that society – its speakers and their social classes” \citep[51]{moustaoui_srhir_concept_2016}.} 

The passages in the interviews in which a clear and direct terminological opposition is established between \textit{lengua} and \textit{dialecto} are of particular interest. All these examples show and, at the same time, reproduce linguistic hierarchies between European \textit{languages} and African \textit{dialects}. Even today, in most African countries, European languages occupy the position of official languages (the exceptions are found in the north of the continent, cf. \citealt[65]{barbosa_da_silva_politica_2011}) and represent “the languages of public and intellectual discourse” \citep[21]{zeleza_inventions_2006}. On the other hand, the colonial period saw autochthonous languages relegated to non-official and informal contexts, and therefore to certain speech registers – a situation that largely continues to this day.

This terminological and conceptual problem is another reminder of the objectives of decolonial theories. In fact, \citet{lara_delgado_pensamiento_2015} points out the need to eliminate the “epistemic racism which refers to a hierarchy of colonial domination where the knowledge produced by Western (imperial and oppressed) subjects within the zone of being is considered a priori as superior to the knowledge produced by the non-Western colonial subjects in the zone of non-being.”\footnote{Own translation. The original quote is: “Racismo epistémico que se refiere a una jerarquía de dominación colonial donde los conocimientos producidos por los sujetos occidentales (imperiales y oprimidos) dentro de la zona del ser es considerada a priori como superior a los conocimientos producidos por los sujetos coloniales no-occidentales en la zona del no-ser.”} The dichotomy between \textit{languages} and \textit{dialects} is a good example of these hierarchies of colonial domination that continue to be transmitted and consolidated today. In this context, “the theories of decoloniality go beyond decolonization, and propose ‘other’ alternatives that seek to subvert the hegemonic power, in order to make visible the effects that colonialization and coloniality have brought in power, in knowledge and in being” \citep[51]{gomez_velez_estudios_2017}.\footnote{Own translation. The original quote is: “Así las cosas, las teorías de la decolonialidad van más allá de la descolonización, y plantean alternativas ‘otras’ que buscan subvertir el poder hegemónico, para visibilizar los efectos que ha traído la colonización y la colonialidad en el poder, en el saber y en el ser.”} One very important source of these hierarchies constructed in colonial times is the predominant, Eurocentric perspective on history, culture, and society. I understand this concept as “the cognitive perspective produced in the long time of the whole Eurocentered world of colonial/modern capitalism, and which \textit{naturalizes} the experience of the people in this pattern of power. That is, it makes them perceive it as \textit{natural}, consequently, as given, not susceptible to be questioned” (\citealt[94]{castro-gomez_colonialidad_2007}, emphasis in original).\footnote{Own translation. The original quote is: “Se trata de la perspectiva cognitiva producida en el largo tiempo del conjunto del mundo eurocentrado del capitalismo colonial/moderno, y que \textit{naturaliza} la experiencia de las gentes en este patrón de poder. Esto es, la hace percibir como \textit{natural}, en consecuencia, como dada, no susceptible de ser cuestionada.”} This Eurocentric system of norms is thus perceived as “a superior and universal pattern of reference,” and consequently, “[t]he other forms of being, the other forms of organization of society, the other forms of knowledge are transformed not only into different, but into lacking, archaic, primitive, traditional, premodern” \citep[10]{lander_ciencias_2000}.\footnote{Own translation. The original quote is: “Esta es una construcción eurocéntrica, que piensa y organiza a la totalidad del tiempo y del espacio, a toda la humanidad, a partir de su propia experiencia, colocando su especificidad histórico-cultural como patrón de referencia superior y universal. […] Las otras formas de ser, las otras formas de organización de la sociedad, las otras formas del saber, son trasformadas no sólo en diferentes, sino en carentes, en arcaicas, primitivas, tradicionales, premodernas.”} In order to decolonialize our knowledge, it is important to recognize the problematic consequences of this Eurocentric vision and to invert established power structures via the critical analysis of traditional discourses, concepts, and use of terminology. Upgrading the Bantu languages of Equatorial Guinea to the category of \textit{language} (as opposed to \textit{dialect}) would be an important step toward the positive recognition of these linguistic codes – both within the scientific community and among the speakers themselves.

\subsection{Equatoguineans’ attitudes toward Pichi}\label{schlumpf:sec:pichi}

Having analyzed the speakers’ attitudes toward Bubi and Fang (Section \ref{schlumpf:sec:attitudes}) and their usage of the terms \textit{lengua} and \textit{dialecto} (Section \ref{schlumpf:sec:terminology}), I will now turn to the Equatoguineans’ language attitudes toward Pichi, also known as Pichinglis(h) or Pidgin English.\footnote{For more information about Pichi and its linguistic features, history, and functions in Equatorial Guinea, see \citet{lipski_pidgin_1992} and Yakpo (\citeyear{yakpo_gramatica_2010}; \citeyear{yakpo_pichi_2013}; \citeyear{yakpo_wayward_2013}; \citeyear{yakpo_only_2016}; \citeyear{yakpo_o_2016}; \citeyear{yakpo_nacimiento_2018}; \citeyear{yakpo_grammar_2019}).} This local variety of English is mainly spoken in the capital Malabo and its surroundings, yet it can also be heard in other areas of Bioko and, increasingly, in the city of Bata. According to \citet[276–278]{yakpo_wayward_2013}, 70 percent of Bioko’s population use Pichi regularly. One of the main functions of Pichi is that of serving as the lingua franca between speakers of different ethnic groups. However, it is also used in informal conversations between speakers who share the same ethnic background, especially Bubi. Moreover, Pichi allows communication with Africans from other countries, where similar varieties of English can be found (e.g., Cameroon, Ghana, Nigeria, and Sierra Leona). Yet despite all this, Pichi enjoys neither prestige nor official recognition in Equatorial Guinea \citep[25–26]{yakpo_o_2016}.

In my interview corpus, a total of 164 references to Pichi were found: \textit{pichi} (122 occurrences), \textit{inglés} (28 occurrences), \textit{pichinglish/pichinglis} (5 and 6 occurrences respectively) and \textit{broken English} (3 occurrences). As far as general statements are concerned, and although opinions also differ, most interviewees say that it is mainly the Bubi who speak Pichi, especially people from the island (Bioko) and people from Malabo. The main contexts in which Pichi is used are the street and conversations with friends, among young people. Pichi is also spoken in some Equatoguinean households, although in other families it is never used. By contrast, Pichi is not represented at all in the school environment – where Spanish is the only language used – or in other (semi-)official contexts.

Looking at the positive and negative evaluations that appear in relation to Pichi, there is a large imbalance between approximately 60 occurrences of negative content compared to fewer than 30 occurrences of content presented as something positive (cf. \citealt[218ff]{yakpo_only_2016};  \citeyear{yakpo_o_2016}). In summary, Pichi is seen as something positive in terms of its communicative utility and its particular value as a dynamic and ever-changing code, used in particular by young people in Malabo/on Bioko. Some young Bubi participants even describe it as a ``secret code'' of their generation, although elder generations also use a more traditional version of Pichi. Some interviewees point out its basic value as an additional language next to Spanish and the Bantu languages spoken in the country; according to another person, knowledge of Pichi makes it easier to learn English at school (06\_-8MB).

On the other hand, the negative opinion predominates that Pichi is somewhat incorrect, impure, and non-standard (cf. \citealt[35–36]{yakpo_o_2016}). It is bad English and therefore an incomplete or ``broken''
code of a complete language: a kind of English, but not English; something like English, but not the ``real'' one (cf. \citealt[63]{woolard_language_1994}); a cheap replica of English; a corrupted version of English; a mixed code (\textit{dialect}) with influences of other (local) languages (see Example \ref{schlumpf:ex5}). Similar linguistic terminology and descriptions are documented in other studies on Pichi, including “broken English” \citep[36]{bolekia_boleka_lenguas_2001}, “corrupted English” \citep[356]{castillo_rodriguez_language_2013}, “broken-inglis” (\citealt{lipski_pidgin_1992}; \citeyear[117]{lipski_spanish_2004}), “inglés-africano,” “broken English,” and “inglés roto” \citep[5]{de_zarco_dialecto_1937}. According to two of my interviewees, knowledge of Pichi can make the acquisition of standard English at school difficult (see Example \ref{schlumpf:ex6}), which contrasts with the interviewee who said that it helps with English acquisition. In addition, Pichi is sometimes associated with delinquency and bandits.

\begin{exe}\ex\label{schlumpf:ex5}
%\begin{quote}(5)
	\textit{Ie.: […] [el pichinglish] es una réplica del: bueno una réplica por decirlo así del inglés pero muy barato o sea: es .h: sí se ha modificado totalmente o sea: no es inglés ni es nada es una // a lo mejor algo que suena inglés pero: / eh la: la gran mayoría de palabra:s se han inventao: se: se han modificao} (05\_-8MB)\\
Ie.: […] [Pichinglish] is a replica of well a replica so to speak of English but very cheap I mean it is yes it has been totally modified I mean it is not English nor is it anything it is a // maybe something that sounds English but / eh the the great majority of words has been invented has has been modified (05\_-8MB)
%\end{quote}

\ex\label{schlumpf:ex6}
%\begin{quote}(6)
	\textit{Ie.: no yo no quise aprender pichi porque yo: el pichi me parece algo cutre […] porque es el inglés mal hablado ¿sabes? .h una persona que ya ha aprendido pichi y que / habla pichi / es / difícil que sepa hablar el inglés} (16\_-8MF)\\
Ie.: no I didn’t want to learn Pichi because I Pichi seems to me somewhat shabby […] because it is badly spoken English you know? a person who has already learnt Pichi and who / speaks Pichi / it’s / difficult that he/she knows how to speak English (16\_-8MF)
%\end{quote}
\end{exe}

\largerpage
Clearly, the fact that most negatively influences the evaluations of Pichi – despite its common use in Equatorial Guinea as a kind of second lingua franca – is its close cognation with English (cf. \cite[218–220]{yakpo_o_2016}). This is the reason why it is almost always evaluated \textit{in comparison to} English, rather than as an independent linguistic code. All this despite the fact that standard English is not even currently spoken in Equatorial Guinea (except in the oil camps). This can be compared with typical diglossic situations according to the original definition of diglossia by Charles A. Ferguson (\citeyear{ferguson_diglossia_1959}): the coexistence of a clearly superposed and prestigious variety of a language (\textit{high variety}) and another variety of the same language, more regional, usually a more informal and oral code (\textit{low variety}). Although standard English is not part of the linguistic setting of Equatorial Guinea, its existence and prestige (and the Equatoguineans’ awareness of it) are enough to consolidate the \textit{low variety} status of Pichi. This direct comparison to ``real'' English leads to an even more negative interpretation of Pichi than, for example, the Bantu languages Bubi and Fang.\footnote{A similar comparison can be found between different languages and varieties in the Maghreb. Similar to Pichi in Equatorial Guinea, the local Arabic variety known as Darija is characterized by its long-standing inferiority vis-à-vis Modern Standard Arabic and has only recently gained some recognition and sociolinguistic prestige among its speakers (see \citealt{moustaoui_srhir_transforming_2019}; \citealt{moustaoui_srhir_arabe_2019}). On the other hand, Tamazight, although only officially recognized in Morocco in 2011, has always been a sign of the cultural and linguistic identity of the Berber community vis-à-vis the majority Arab population. In the two Spanish exclaves in Northern Morocco, it can also be observed that Tamazight has a stronger integrating function in Melilla than Darija has in Ceuta, because the latter is seen as a “deformed” and “impure” form of Classical Arabic \citep[39]{tilmatine_contacto_2011}. Similar to Tamazight in the Maghreb, the Bantu languages Bubi and Fang in Equatorial Guinea represent important cultural symbols for their ethnic groups, although they do not enjoy any official status in the country.}  Although these two languages also seem to be somehow evaluated as less prestigious than Spanish, they symbolize the local traditions of the different ethnic groups. In summary, they assume important cultural values for the speakers and are part of the speakers’ ethnic identity, which is not the case for Pichi (cf. \citealt[26ff.]{yakpo_only_2016}).


\section{Conclusion}\label{schlumpf:sec:conclusion}

In this final section, I summarize some major points in this chapter and highlight a number of specific problems relevant to the topic. These problems can be used to understand, analyze, and question traditional power relations between languages and linguistic varieties in colonial and postcolonial settings.

The description of the language attitudes of the Equatoguineans in Madrid showed that Bubi and Fang enjoy a high level of prestige as part of the cultural and ethnic identity of the speakers. They both fulfill particular communicative functions, in interactions with elder generations or when talking about specific topics, for instance. Pichi, on the other hand, has an important function as Equatorial Guinea’s second lingua franca and as an in-group code for younger generations. It also enables speakers to communicate with Africans from other countries. However, both the two Bantu languages and Pichi find themselves in an uncomfortable situation of comparison and inferiority vis-à-vis standardized European languages – namely, the official Romance languages (especially Spanish, the language of the former colonial power) and standard English. Bubi and Fang, in the opinion of most of the interviewees, are somehow less valuable for official purposes or as languages used in the educational system than Spanish. As for Pichi, it is considered to be less prestigious than English and is frequently described as “poor,” “corrupted,” or “broken” English. These power relations between European and African languages and varieties are reflected in the terminological opposition between \textit{lengua} and \textit{dialecto}, a legacy of colonial domination and hierarchizations. Whereas the former is used with very different associations when relating to Spanish and to Bubi and Fang (the official status of Spanish versus the local values of the two African languages), the latter almost exclusively refers to the Bantu languages. All this has serious implications for the speakers themselves. Negative evaluations of their African mother tongues and the commonly used Pichi turn into a negative self-perception and a critical opinion toward an integral part of their own cultures and communities.

When it comes to differences between the two ethnic groups studied, Bubi and Fang, I have shown that feelings of inferiority in the case of the former have been propagated since Equatorial Guinea’s independence in 1968 under both Fang regimes (Francisco Macías Nguema Bidyogo, 1968 to 1979, and Teodoro Obiang Nguema Mbasogo, since 1979), a process known as \textit{Fanguization}. Until today, most influential positions in the country are held by Fang.

Considering the specific context of my study – the migration context in Madrid – and the daily experiences of my interviewees as Equatoguineans in this Spanish setting, it can be observed that the negative evaluations (and self-evaluations) are fostered by external forces such as social exclusion, limited access to the labor market, and even racism. The widespread lack of knowledge about Equatorial Guinea and the languages spoken in the country, as well as the non-recognition of the Equatoguinean Spanish as part of the Hispanophone world, do not improve the attitudes of the Equatoguineans vis-à-vis their own linguistic repertoires; nor do so theories that favor monolingual speakers of standard (European) languages. Experiences of discrimination further foster feelings of sociocultural inferiority as part of a minority.

To conclude, I wish to emphasize from a theoretical point of view that the negative attitudes of the Equatoguineans interviewed toward their own local languages help to perpetuate not only colonial hierarchies and power relations, but also the transmission of conservative ideas and language ideologies. Bubi and Fang receive mostly negative evaluations, unless they are related exclusively to culture and local traditions, and they are always compared to (standard) European languages, which serve as superior counterpoints. In this way, Western standards and Eurocentric views on cultural and linguistic phenomena, established and spread during the colonial period, ultimately become African views as well. It will be crucial to compare the findings of the present study with the data collected in Equatorial Guinea itself in 2022. This will allow to better understand the influence of the migration context in the Madrid corpus; and to get more profound insights into the contemporary language attitudes and ideologies of Equatoguineans who live in their own country.

It is of utmost importance that work is done to effect a change of perspective in investigations of Africa-related topics or other colonialized territories. African phenomena must be evaluated from an inner-African perspective, avoiding constant references to European models. It is time to leave the colonial past behind and decolonialize perceptions of historically constructed dichotomies and hierarchies. This will enable researchers to focus positively on the future; after all, the recognition and consolidation of local structures, cultures, and knowledge are crucial to solving not only social problems, but political issues as well.

\section*{Acknowledgements}
The support of several funding institutions and individuals was crucial for the realization of my research project on the Equatoguinean community in Madrid, and I wish to express my gratitude to all of them. The interviews with Equatoguineans in Madrid were carried out during two research stays funded by the Swiss National Science Foundation (SNSF) (project no. 173468: “Sociolinguistic integration of immigrants from Equatorial Guinea in Madrid”) and via a research scholarship (“Giner de los Ríos”) from the University of Alcalá. The transcriptions of the interviews were realized with the help of Lilli Geyer-Schuch, Sara Carreira, Laura Renna, Tabea Dürr, Emanuel Branco (University of Basel), and Adriana Orjuela López (University of Freiburg). The digitization of the interviews was carried out in collaboration with Anđelka Zečević (Faculty of Mathematics, University of Belgrade) within the project “Digital analyses of sociolinguistic data. Linguistic features in interviews with Spanish speakers from Equatorial Guinea” (project no. 190022) financed by the SNSF. The databases analyzed in this chapter were annotated together with Sara Carreira and Lilli Geyer-Schuch (University of Basel). Finally, I would like to thank Johannes Ritter (University of Basel) for his valuable support during the preparation and careful revision of this chapter; and the two anonymous reviewers for their helpful suggestions.


\printbibliography[heading=subbibliography, notkeyword=this]


\end{document}
