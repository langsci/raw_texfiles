\chapter{Yongning Na tones in dynamic-synchronic perspective}
\label{chap:yongningnatonesinadynamicsynchronicperspective}

% Temporarily widen the epigraph block
{\setlength{\epigraphwidth}{.55\textwidth} 
\epigraph{The past century of phonetic research has illuminated our understanding of the production of sound, the properties of the acoustic signals, and to a~certain extent, the perception of speech sounds. But the search for the originating causes of sound change itself remains one of the most recalcitrant problems of phonetic science.}{\citet[1]{labov1979}}

%\vspace{-0.7cm}

\epigraph{It is often claimed that the causes of phonetic changes remain unknown. However, this assertion is incorrect. Crucially, there are multiple factors at play, not just one single cause. A common mistake among those who have addressed this question is that, upon identifying a cause of phonetic change, they have assumed it to be the sole cause and attempted to attribute all changes to it.}{\citet[175]{grammont1933}\footnotemark}{}\footnotetext{\textit{Original text:} On enseigne partout qu’elles [les causes des changements phonétiques] sont encore inconnues et mystérieuses. C’est inexact. Mais il n’y a pas une cause, il y en a un grand nombre, et l’erreur de la plupart de ceux qui se sont occupés de la question a été précisément, lorsqu’ils ont reconnu une cause de changements phonétiques, de croire qu’elle était la seule cause et de vouloir tout y ramener.}

} % end of: \setlength{\epigraphwidth}{.53orwhatever\textwidth}
%\footnote{\textit{Original text:} On enseigne partout qu’elles [les causes des changements phonétiques] sont encore inconnues et mystérieuses. C’est inexact. Mais il n’y a pas une cause, il y en a un grand nombre, et l’erreur de la plupart de ceux qui se sont occupés de la question a été précisément, lorsqu’ils ont reconnu une cause de changements phonétiques, de croire qu’elle était la seule cause et de vouloir tout y ramener.} % Cité par Martinet (1955: 35).

% My first literal translation: 
% It is said everywhere that the causes of phonetic changes are still unknown and mysterious. This is incorrect. But there is not one cause, there are many, and the mistake of most of those who have dealt with the question has been precisely that, when they identified a cause of phonetic changes, they believed that it was the only cause and wanted to reduce everything to it.

%Command \noindent added to avoid having a first indent in cases where a paragraph starts after an epigraph without an intervening title.
{\noindent}The synchronic description proposed in the present volume provides a~foundation for examining the
historical dynamics of tone in Na. As mentioned in \sectref{sec:theoreticalbackdrop}, a~dynamic approach to synchrony brings out patterns of synchronic
\isi{variation} which, in turn, offer glimpses into diachronic evolution. 

%The study of variability is especially crucial to the study of tone, as 
Variability in tone
patterns tends to be high in level-tone systems with rich \isi{morphotonology}, and {diachronic} change
tends to occur at a~faster pace than in other areas of the linguistic system (such as syntax). The view that tonological models
should be designed in such a~way as to accommodate
patterns of \isi{variation} is found e.g.~in a~discussion of \ili{Bambara}, a~\ili{Mande} language:

\begin{quotation}
Clearly, any hypothesis about the system that underlies the tonal
productions of Bambara speakers should be able to account, with minimal adjustments, for observed
patterns of \isi{variation}.~\citep[8]{creissels1992}
    
    \medskip 
    %\footnote{
    {\noindent}\textit{Original text:} il est clair que toute hypothèse sur l’organisation du système
sous-jacent à un corpus de productions tonales de bambarophones doit pouvoir au prix d’un minimum
d’aménagements rendre compte de possibilités éventuelles de {variation}.
\end{quotation}

As more data becomes available on the dialectal diversity of Na, it will be possible to investigate patterns of \is{language contact}contact and \isi{variation} with increasing precision. This chapter discusses five topics, in no particular order, and making no attempt at exhaustiveness: {structural} gap-filling, {analogy}, syllable reduction, disyllabification, and the influence of {bilingualism} with {Mandarin}. 


\section[Gap-filling in tonal paradigms]{Gap-filling in tonal paradigms: The example of subject-plus-verb phrases}
\label{sec:howthesuffixacquiresitslmorhtoneafteramtoneverb}
\is{gap-filling|textbf}

When a~phonological unit drifts far enough from its original pronunciation for a new combination to occupy the abandoned slot, structural gap-filling can occur, leading to changes in the phonological system. For instance, in Yongning Na it is likely that the sound [\ipa{ʁ}] was originally an empty-onset filler (see Appendix A, \sectref{sec:theinitialvoiceduvularfricativeasaphonemicizedemptyonsetfiller}). The syllable /\ipa{ʁo}/ in
/\ipa{ɑ˩ʁo˧}/ ‘house’ is \is{comparative method (historical linguistics)}reconstructed as a~simple *\ipa{o} at the proto-\ili{Naish} stage \citep{jacquesetal2011}. It remains
phonemically onsetless in \ili{Laze} ([\ipa{ɑ˥wu˥}], phonemically /\ipa{ɑ˥u˥}/) and in \ili{Naxi}
([\ipa{mi˧wu˩}], phonemically /\ipa{mi˧u˩}/). Once *\ipa{o} syllables came to be realized as [\ipa{ʁo}] at the surface phonological level in Yongning Na, no [\ipa{o}] or [\ipa{wo}] syllables remained. But this phonetic slot was subsequently filled by syllables
with other origins: the syllable /\ipa{wo}/ is now firmly attested in words such as /\ipa{wo˥}/ ‘hard’, /\ipa{wo˩\textsubscript{b}}/ ‘{classifier} for teams of oxen’, /\ipa{wo˩kɤ\#˥}/ ‘swing’, and /\ipa{wo˩˥}/ ‘turnip leaves’. The emergence of [\ipa{wo}] syllables precipitated the phonemicization of what was originally an empty-onset-filler:
the syllable /\ipa{ʁo}/ in /\ipa{ɑ.ʁo}/ ‘house’ must now be analyzed as composed of two phonemes, an initial
/\ipa{ʁ}/ and the vowel /\ipa{o}/.

\begin{sidewaystable}[p]
	\caption{\label{tab:thetonepatternsofsubjectplusverbBIS}The tone patterns of subject-plus-verb combinations, in
		surface phonological transcription.}
	\begin{tabularx}{\textheight}{ l@{\hspace{6mm}} Q l@{\hspace{6mm}} l@{\hspace{6mm}} l@{\hspace{6mm}} l@{\hspace{6mm}} Q }
		\lsptoprule
		& tone of verb & & & & &\\ \cmidrule{2-7}	
		tone of noun & H & M\textsubscript{a} & M\textsubscript{b} & L\textsubscript{a} & L\textsubscript{b} & MH\\ \midrule
		LM, LH & L.H & L.M+M & L.M+M & L.H & L.H & L.MH\\
		M & M.M+L & M.M+M & M.M+M & M.L & M.L & M.MH\\
		L & M.M+L & L.L  & M.M+M & L.L & L.L~/ M.L & L.L\\
		H & M.M+L & M.M+L & M.M+L & M.MH & M.MH & M.L\\
		MH & M.H & M.H & M.H & M.MH & M.MH & M.H\\ \addlinespace \hdashline \addlinespace
		M & M.M.M+L & M.M.M+M & M.M.M+M & M.M.L & M.M.L & M.M.MH\\
		\#H & M.M.M+L & M.M.M+L & M.M.M+L & M.M.MH & M.M.MH & M.M.L\\
		MH\# & M.M.MH & M.M.MH & M.M.MH & M.M.MH & M.M.MH & M.M.H\\
		H\$ & M.M.M+L & M.M.M+L & M.M.M+L~/ M.M.M+H & M.M.MH & M.M.MH & M.H.L\\
		L & L.L.L & L.L.L & L.L.L & L.L.L & L.L.L & L.L.H\\
		L\# & M.L.L & M.L.L & M.L.L & M.L.L & M.L.L & M.L.L\\
		LM+MH\# & L.M.M+L & L.M.M+L & L.M.M+L & L.M.MH & L.M.MH & L.M.H\\
		LM+\#H & L.M.M+L & L.M.M+M & L.M.M+M & L.M.L & L.M.MH & L.M.MH\\
		LM & L.M.M+L & L.M.M+M & L.M.M+M & L.M.L & L.M.L & L.M.MH\\
		LH & L.H.L & L.H.L & L.H.L & L.H.L & L.H.L & L.H.L\\
		H\# & M.H.L & M.H.L & M.H.L & M.H.L & M.H.L & M.H.L\\
		\lspbottomrule
	\end{tabularx}
\end{sidewaystable}

Gap-filling can also take place in tonal paradigms. This section is devoted to a~plausible case in subject-plus-verb constructions. 

\is{form!surface|(}
\is{gap-filling}

\tabref{tab:thetonepatternsofsubjectplusverbcombinationsinsurfacephonologicaltranscription}, repeated here as \tabref{tab:thetonepatternsofsubjectplusverbBIS}, presents the tonal behaviour of subject-plus-verb combinations, covering both \is{monosyllables}monosyllabic and disyllabic subjects. Two contexts were used to arrive at underlying tone categories: S+V and
S+V+\textsc{per\-fec\-tive}. For instance, ‘the guests arrive’ is /\ipa{hĩ˧-bæ˧
	tsʰɯ˧˥}/, and addition of the \textsc{perfective} yields /\ipa{hĩ˧-bæ˧
	tsʰɯ˧-ze˥}/ ‘the guests have arrived’. The tone pattern of this subject-predicate combination, /M.M.MH/, is analyzed as \mbox{//MH\#//}: an MH \is{tonal contour}contour associating to the last syllable.

A~key analytical challenge raised by the dataset in \tabref{tab:thetonepatternsofsubjectplusverbBIS} concerns the surface phonological tone sequences ending in M+L, M+M, and M+H. In these notations, the tone following the ‘+’ sign is that carried by the \textsc{perfective} \mbox{/\ipa{-ze˧}/} when suffixed to the subject-plus-verb combination. The {question} is how the \textsc{perfective} suffix acquires its \mbox{/L/}, \mbox{/M/} or \mbox{/H/} tone in these combinations. 

The full set of relevant expressions is comprised of M.M+L, M.M.M+L, M.M+M, M.M.M+M, M.M.M+H, L.M+M, L.M.M+M, and L.M.M+L. Among these, those
ending in /M+L/, as in (\ref{ex:tigerwalked}), and those ending in /M+M/, as in (\ref{ex:tigerdied}), are commonplace. On the other hand, only one expression ends in /M+H/: it results from the
combination of a~\mbox{//H\$//}-tone subject and a~\mbox{//M\textsubscript{b}//}-tone verb, as in (\ref{ex:shecatjumped}). 

\begin{exe}
	\ex
	\label{ex:tigerwalked}
	\ipaex{lɑ˧ se˧-ze˩}\\ 
	\gll lɑ˧		se˥		-ze˧\textsubscript{b}\\
	tiger	to\_walk	\textsc{pfv}\\
	\glt ‘the tiger walked’ (input tones: M on noun and H on verb) \textit{(SubjectVerb.31)}  \pandoi{0004477\#W31}
\end{exe}

\begin{exe}
	\ex
	\label{ex:tigerdied}
	\ipaex{lɑ˧ ʂɯ˧-ze˧}\\ 
		\gll lɑ˧		ʂɯ˧\textsubscript{a}		-ze˧\textsubscript{b}\\
		tiger	to\_die		\textsc{pfv}\\
		\glt ‘the tiger died’ (input tones: M on noun and M on verb) \textit{(SubjectVerb.22)}  \pandoi{0004477\#W22}
\end{exe}
	
\begin{exe}
	\ex
	\label{ex:shecatjumped}
	\ipaex{hwɤ˧mi˧ tsʰo˧-ze˥}\\ 
		\gll hwɤ˧mi˥\$		tsʰo˧\textsubscript{b}	-ze˧\textsubscript{b}\\
		she\_cat	to\_jump	\textsc{pfv}\\
		\glt ‘the she-cat jumped’ (input tones: H\$ on noun and M\textsubscript{b} on verb) \textit{(SubjectVerb.85-87)}  \pandoi{0004477\#W85}
\end{exe}
	
The hypothesis proposed here is that the pattern ending in /M+H/
is an \is{innovative (phonological form)}{innovation}.

The \textsc{perfective} can receive one of three tones in subject-plus-verb plus \textsc{perfective} constructions: \mbox{/M/}, \mbox{/H/} or \mbox{/L/}. Cases in which
the \textsc{perfective} carries \mbox{/M/} tone are the simplest: the morpheme does not receive any tonal specification from what
precedes, and surfaces with its lexical M tone. The surface strings \mbox{/M.M}+\mbox{M/} (for \is{monosyllables}monosyllabic nouns) and \mbox{/M.M.M}+\mbox{M/} (for
disyllabic nouns) can therefore be analyzed as \mbox{//M//}. Likewise, \mbox{/L.M}+\mbox{M/} (for \is{monosyllables}monosyllabic nouns) and \mbox{/L.M.M}+\mbox{M/} (for
disyllabic nouns) can be analyzed as \mbox{//LM//}. 

Cases where the \textsc{perfective} receives /H/ tone look like typical instances of the \is{floating tone}floating H tone, \mbox{//\#H//}. This
tone, which does not surface \is{form!in isolation}in isolation but can be manifested on a~following syllable, is
frequently observed in Yongning Na. It is the lexical tone of a~class of nouns, exemplified by ‘little brother’, which surfaces \is{form!in isolation}in isolation as /\ipa{gi˧zɯ˧}/ ‘little brother’ but yields /\ipa{gi˧zɯ˧ ɲi˥}/ when followed by the \isi{copula}, as explained in~\sectref{sec:afloatinghtonewithcomparativeevidencepointingtoitsorigin}.

As noted in \sectref{sec:wordfinalandmorphologicalnucleusfinalHtones}, the \mbox{//H\$//} tone shows signs of variability: it is the lexical tone with the greatest number of \is{morphotonology}morphotonological variants across different morphosyntactic contexts. In subject-plus-verb constructions, its combination with a~\mbox{//M\textsubscript{b}//}-tone verb allows
for two possibilities: M.M.M+L and M.M.M+H. The latter tonal string, M.M.M+H, is not attested in any
of the other subject-plus-verb combinations. This distribution suggests the hypothesis that this
\is{variants}variant is an \is{innovative (phonological form)}{innovation} that filled an empty slot in the surface phonological patterns.

Under the hypothesis that the tone pattern M.M.M+H in subject-plus-verb plus \textsc{perfective} combinations represents an \is{innovative (phonological form)}{innovation}, the pre-existing M.M.M+L surface pattern could have been analyzed as \mbox{//\#H//}. 
%prior to this \is{innovative (phonological form)}{innovation} the M.M.M+L surface pattern could have been analyzed as \mbox{//\#H//}. 
The \is{floating tone}floating H tone was not manifested directly but triggered
a~lowering of following tones~-- in this instance, a~lowering of the tone of the \textsc{perfective}
morpheme.\footnote{The floating H tone of Yongning Na, transcribed as \mbox{//\#H//}, exists not only as a~lexical category on nouns (as discussed in \sectref{sec:afloatinghtonewithcomparativeevidencepointingtoitsorigin}), but also as the output of some \is{morphotonology}morphotonological rules, such as those that apply in {compound} nouns. Cases where an H tone does not surface but lowers the following tones (to L) are found in several areas of Yongning Na morphotonology. For instance, the phrase /\ipa{mv̩˩tɕo˧ se˧}/ ‘to walk downward’ depresses a~following \textsc{perfective} suffix (//\ipa{-ze˧\textsubscript{b}}//, which has lexical M tone) to L: /\ipa{mv̩˩tɕo˧ se˧-ze˩}/ ‘(s)he walked downward’ (\textit{SpatialOrientation.59-60}  \pandoi{0004559\#W59}). This is interpreted as evidence of the presence of a~floating H tone in the expression ‘to walk downward’: //\ipa{mv̩˩tɕo˧ se\#˥}// (see \sectref{sec:themarkingofspatialorientationonverbs}).}

This analysis is reflected in \tabref{tab:subjectverbcombinations}, which leaves out the
problematic \is{variants}variant M.M.M+H for the combination of a~subject carrying H\$ tone with a~verb carrying M\textsubscript{b} tone.

\begin{sidewaystable}[p]
	\caption{\label{tab:subjectverbcombinations}A phonological analysis of the tones of subject-plus-verb combinations, 
		% positing that tone \#H is reflected in the lowering of the tone of the postverbal morpheme, and 
		leaving aside the M.M.M+H variant of the combination of an H\$-tone subject and an M\textsubscript{b}-tone verb.}
	\begin{tabularx}{\textheight}{ l Q Q Q Q Q Q }
		\lsptoprule
		& tone of verb & & & & &\\ \cmidrule{2-7}
		tone of noun & H & M\textsubscript{a} & M\textsubscript{b} & L\textsubscript{a} & L\textsubscript{b} & MH\\ \midrule
		LM, LH & LM & LM & LH & LH & LH & LM+MH\#\\
		M & M & M & \#H & M.L & M.L & M.MH\\
		L & L & M & \#H & L & L & L\\
		H & \#H & \#H & \#H & MH\# & MH\# & L\#\\
		MH & H\# & H\# & H\# & MH\# & MH\# & H\#\\ \addlinespace \hdashline \addlinespace
		M & M & M & \#H & \#H & \#H & MH\#\\
		\#H & \#H & \#H & \#H & MH\# & MH\# & \#H\\
		MH\# & MH\# & MH\# & MH\# & MH\# & MH\# & \#H\\
		H\$ & \#H & \#H & \#H & MH\# & MH\# & H\#\\
		L & L & L & L & L & L & L+H\#\\
		L\# & L\#-- & L\#-- & L\#-- & L\#-- & L\#-- & L\#--\\
		LM+MH\# & LM--+\#H  & LM--+\#H  & LM--+\#H  & LM+MH\# & LM+MH\# & LM+H\$\\
		LM+\#H & LM-- & LM-- & LM--+\#H  & LH-- & LM+MH\# & LM+MH\#\\
		LM & LM-- & LM-- & LM--+\#H  & LH-- & LH-- & LM+MH\#\\
		LH & LH-- & LH-- & LH-- & LH-- & LM+MH\# & LH--\\
		H\# & H\#-- & H\#-- & H\#-- & H\#-- & H\#-- & H\#--\\
		\lspbottomrule
	\end{tabularx}
\end{sidewaystable}


\tabref{tab:subjectverbcombinations} is a~\is{comparative method (historical linguistics)}reconstruction of the set of \is{tone rules}tone rules that applied in
subject-plus-verb constructions prior to the appearance of the M.M.M+H \is{variants}variant. If this reconstruction represents a~historical reality, it is likely to have shallow time depth: the amount of observed idiolectal and dialectal variation suggests that such a~change could have taken place within a~couple of generations. At this \is{comparative method (historical linguistics)}reconstructed stage, 
%represented in \tabref{tab:subjectverbcombinations}, 
a~tone rule must be posited whereby the \mbox{//\#H//} tone in subject-plus-verb constructions does not
surface, but depresses following tones to L. In view of the general architecture of the Na tone
system, this rule is not an \textit{ad hoc} device to explain away an unaccountable observation:
a~rule to the same effect operates in other morphosyntactic contexts.

In the present state of the language, on the other hand, the M.M.M+H \is{variants}variant has settled in, and the simplest phonological interpretation of this pattern is as the
manifestation of a~\is{floating tone}floating H tone~-- an interpretation that conflicts with the earlier system. Interpretation of the M.M.M+H pattern as \is{form!underlying}underlying \mbox{//\#H//} by \is{language acquisition}language learners entails an in-depth modification of the tonal system: as the \mbox{//\#H//} slot in the system comes to be occupied by the new, \is{innovative (phonological form)}innovative form,
the M.M.M+L surface pattern~-- previously analyzable as reflecting an underlying \mbox{//\#H//}~-- requires a~new phonological interpretation, as do the other surface patterns ending in /M+L/ in \tabref{tab:thetonepatternsofsubjectplusverbBIS}. 

One possible phonological reanalysis 
%in view of the entire system 
would be to interpret these cases as involving a~\is{floating tone}floating L tone, //\#L//, which would thus enter the language’s tonal system. The surface phonological
patterns in subject-plus-verb plus \textsc{perfective} constructions would then be interpreted as shown in \tabref{tab:analysisofthetonesofsubjectverbcombinationspositingfloatingLtones}, where the newly introduced //\#L//
category is highlighted, bringing out its relatively pervasive presence in the table.

\begin{sidewaystable}[p]
	\caption{\label{tab:analysisofthetonesofsubjectverbcombinationspositingfloatingLtones}A phonological analysis of the tones of subject-plus-verb combinations positing floating L tones.}
	\begin{tabularx}{\textheight}{ l Q Q Q Q Q Q }
		\lsptoprule
		& tone of verb & & & & &\\ \cmidrule{2-7}
		tone of noun & H & M\textsubscript{a} & M\textsubscript{b} & L\textsubscript{a} & L\textsubscript{b} & MH\\ \midrule
		LM, LH & LH & LM & LM & LH & LH & LM+MH\#\\
		M & \shadedcell \#L & M & M & M.L & M.L & M.MH\\
		L & \shadedcell \#L & L & M & L & L & L\\
		H & \shadedcell \#L & \shadedcell \#L & \shadedcell \#L & MH\# & MH\# & L\#\\
		MH & H\# & H\# & H\# & MH\# & MH\# & H\#\\ \addlinespace \hdashline \addlinespace
		M & \shadedcell \#L & M & M & \shadedcell \#L & \shadedcell \#L & MH\#\\
		\#H & \shadedcell \#L & \shadedcell \#L & \shadedcell \#L & MH\# & MH\# & \shadedcell \#L\\
		MH\# & MH\# & MH\# & MH\# & MH\# & MH\# & \shadedcell \#L\\
		H\$ & \shadedcell \#L & \shadedcell \#L & \lshadedcell \#H / \#L & MH\# & MH\# & H\#\\
		L & L & L & L & L & L & L+H\#\\
		L\# & L\#-- & L\#-- & L\#-- & L\#-- & L\#-- & L\#--\\
		LM+MH\# & LM--+\#H  & LM--+\#H  & LM--+\#H  & LM+MH\# & LM+MH\# & LM+H\$\\
		LM+\#H & LM--+\#H  & LM-- & LM-- & LH-- & LM+MH\# & LM+MH\#\\
		LM & LM--+\#H  & LM-- & LM-- & LH-- & LH-- & LM+MH\#\\
		LH & LH-- & LH-- & LH-- & LH-- & LM+MH\# & LH--\\
		H\# & H\#-- & H\#-- & H\#-- & H\#-- & H\#-- & H\#--\\
		\lspbottomrule
	\end{tabularx}
\end{sidewaystable}

Devoting an entire section to the discussion of a single tonal \is{variants}variant may seem
dreadfully disproportionate. This \is{variants}variant deserves special attention, however, because it exemplifies the
constant tension between \is{form!underlying}underlying forms and surface phonological forms, shedding light on types of
evolution taking place in level-tone systems. From the point of view of surface phonological forms,
the \is{innovative (phonological form)}innovative pattern discussed here can be viewed in the light of a~\isi{simplification}: at the
(hypothetical) \is{conservative (phonological form)}conservative stage presented in \tabref{tab:subjectverbcombinations}, the combination of \mbox{//H\$//} and \mbox{//M\textsubscript{b}//} involves an H tone in the input but none in the output; in the \is{innovative (phonological form)}innovative form, by contrast, an H tone is present in both input and output, creating a~closer match between the two. From the
perspective of underlying forms, on the other hand, this development introduces an analytical puzzle for linguists~-- and quite possibly for \is{language acquisition}language learners as well. Such cases allow for
several analytical options and hence hold potential for {diachronic} change.
\is{form!surface|)}

\section{Analogy}
\label{sec:analogy}

\subsection{General principles}
\label{sec:generalprinciples}

\is{analogy|textbf}Analogy is a process of language change whereby a~linguistic form is altered to resemble another form or pattern within the language system. The typical result is that less common word forms are reshaped to conform with more common patterns, levelling down irregularities. But the driving force behind analogy is not likely to be a~preference for regularity: irregular forms can endure over centuries, so long as they are frequent enough in discourse that they do not go out of memory. 

The real driving force is likely to be that a~speaker is in doubt about a~certain form, and coins a~plausible form on-the-fly by imitating other forms that exist in the language.\footnote{This perspective was proposed by Nathan Hill (p.c.\ 2015).} For instance, at some point, a~speaker of \ili{English} who was in doubt about the past tense of \textit{to dive} reasoned that \textit{dove} is to \textit{dive} as \textit{drove} is to \textit{drive}, and introduced an \is{innovative (phonological form)}innovative form, \textit{dove}, which has now become standard in North America, replacing the earlier form \textit{dived}. From a~morphological point of view, {analogy} can be viewed as a~process of
regularization. From the point of view of phonetic change, on the other hand, the piecemeal modifications
introduced by {analogy} tend to obfuscate regular correspondences.

Case studies of analogical reanalysis reveal the complexity of individual situations. For
instance, in the \ili{Bantu} language \ili{Eton}, the stem of the \isi{possessive} ‘my’ ends in
/\ipa{ɔ}/ in association with nouns of classes 1 and 3: /\ipa{-amɔ}/, and in /\ipa{a}/ elsewhere: /\ipa{-ama}/. This
irregularity is due to a~mechanism of analogical morphophonological reanalysis that changed the
original /\ipa{a}/ of the class 1/3 forms to /\ipa{ɔ}/. In \ili{Eton}, there is a~|\ipa{ɔ}| \isi{morphoneme} whose
morphologically conditioned realizations include /\ipa{wa}/. Commonly occurring
sequences of /\ipa{w}/+/\ipa{a}/, although separated by a~morpheme {boundary}, were reinterpreted as
realizations of this \isi{morphoneme} \citep{vandevelde2008}. This example illustrates how morphophonological
{analogy} can disregard morphological boundaries. %This is not the least of the paradoxes of \isi{analogy},
%which has the potential to create morphophonological alternations \citep{blevinsetal2009} as well
%as to inhibit phonetic change in some contexts \citep{blevinsetal2009b}.

Analogy is, by definition, irregular and unpredictable. One may nonetheless believe that evidence from case studies gradually adds up, allowing for an increasingly refined understanding of this phenomenon.

\begin{quotation}
	[I]t is possible to some extent
	to constrain the space of hypotheses involving {analogy}, and research
	on the general principles of {analogy} is of utmost importance for historical
	linguistics. \citep[239]{jacques2016}
\end{quotation}

To date, studies about the principles of {analogy} \citep[e.g.][]{kurylowicz1944, lahiri2000, hill2007, blevinsetal2009, juge2013, hill2014} contain little discussion of tone, while studies on tone (\cites[e.g.][]{pike1948}{fromkinTONE1978}{pulleyblank1986}[229-231]{gussenhoven2004}) contain little on {analogy}. Yet it seems intuitively clear that \isi{morphotonology}, like other aspects of morphophonology, can be subject to analogical levelling.


\subsection{Analogy in Yongning Na morphotonology}
\label{sec:applicationtoyongningna}

Traces of \isi{analogy} are found among the tones of compounds and affixed forms, as was pointed out in several sections (\sectref{sec:anindependentsetoffactscompoundgivennamesandtermsofaddress}, \ref{sec:lexicalizedcompoundsofnadjstructure}, \ref{sec:lhtoneroots}, \ref{sec:concludinggeneralobservations}, and \ref{sec:!nominalization}). 
It appears highly plausible that the tantalizingly similar, but not identical tonal paradigms of
\is{classifiers}classifiers~-- H\textsubscript{a} and H\textsubscript{b}, M\textsubscript{a} and M\textsubscript{b}, MH\textsubscript{a} and MH\textsubscript{b}, L\textsubscript{a}, L\textsubscript{b} and L\textsubscript{c} (see Chapter~\ref{chap:classifiers})~-- have undergone
a~degree of analogical levelling without becoming fully identical. The existence of variants for
some combinations, along with consultants' occasional hesitations and confusions (errors) during elicitation
sessions, all point to the presence of contradictory pressures: on the one hand, a~tendency towards analogical
\isi{simplification}, and on the other, a~counteracting tendency to maintain the distinct identity of the
different classes. 

This is an area where the description of a~single language variety reaches its limits, and
a~variationist approach would be called for. Such a~study could begin with the closest language
varieties, studying phenomena of accommodation among speakers within the hamlet under study, before extending the investigation to dialects spoken in the Yongning basin and its immediate vicinity. 

%A~\is{comparative method (historical linguistics)}{diachronic}-comparative analysis confirms the plausibility of \isi{analogy} as a~major factor in the history of this aspect of Yongning Na morphotonology. But in order to understand this argument, a~hypothesis concerning the origin of the system needs to be set out first.

%\begin{quotation}
%	In Burmo-Qiangic languages other than \ili{rGyalrongic} (except the Burmish branch), final stops are invariably lost. In the case of \ili{Naish} loss of final obstruents had already happened at the proto-\ili{Naish} stage.
%	
%	There is some evidence that the final stops in pre-proto-\ili{Naish}\footnote{Proto-\ili{Naish} only includes materials which can be shown to have been present in the common ancestor of the three languages that constitute the present-day \ili{Naish} subgroup (Yongning Na, Naxi, and \ili{Laze}); pre-proto-\ili{Naish}, on the other hand, is a~construct in which the correspondences found between the three languages are projected as far back as the comparison with archaic languages allows. Within the \il{Sino-Tibetan}Sino-Tibetan family, archaic languages include \ili{rGyalrongic} languages (Khroskyabs, Horpa, Situ, \ili{Japhug}, Tshobdun and Zbu), Written \ili{Tibetan}, and Old \ili{Burmese}. For further information on the \is{comparative method (historical linguistics)}reconstruction of proto-\ili{Naish}, the reader is referred to \citet{jacquesetal2011}.} left a~trace in the patterning of tonal alternation in the \is{numerals}numeral-plus-classifier paradigms. ({\dots}) [T]he comparison of the three \ili{Naish} languages Na, \ili{Laze} and Naxi reveals that numerals under 10 can be classified into groups based on their tonal alternations. The numerals 3, 7, 9 and 10 have specific alternations, but \{1, 2\}, \{4, 5\} and \{6, 8\} respectively always have the same tonal patterns. The group \{6, 8\} is particularly significant, as it is the only group of non-contiguous numerals, and both 6 and 8 have final obstruents in \isi{conservative} languages (\ili{Tibetan} \textit{drug} and \textit{brgʲad}, for instance).
	
%	Thus, it can be hypothesized that (i)~although final stops were lost, they were partially transphonologized as tonal contrasts, and (ii)~the development of the classifier system in \ili{Naish} predates the loss of final stops. \citep[143]{jacques.morphology2017}
%\end{quotation}

%“Somewhat paradoxically, in rGyalrong languages, otherwise known for their polysynthetic and irregular verbal morphology \citep{jackson14morpho, jacques12incorp}, numerals and classifiers present relatively simple and predictable alternations” \citep[135]{jacques.morphology2017}. Jacques's argument is that these alternations are cognate with the \is{numerals}numeral-plus-classifier paradigms in Lolo-\ili{Burmese} and \ili{Naish} (cases of \isi{suppletion} found across Burmo-Qiangic constitute evidence of shared innovations, not parallel innovations) but have been thoroughly simplified by analogical levelling.

%\begin{quotation}
%The  fact that some numerals have two competing prefixal forms (for instance \ipa{kɯmŋu-} vs.\ \ipa{kɯmŋɤ-} for \ipa{kɯmŋu} `five') shows that \isi{analogy} is still synchronically at work in the system, and therefore that a~massive generalization of one particular allomorph is likely to have occurred several times in the history of \ili{Japhug} and other \ili{rGyalrongic} languages, on the basis of phonological alternations otherwise attested in the language. \citep[147]{jacques.morphology2017}
%\end{quotation}

\section{Syllable reduction}
\label{sec:syllablereduction}

Syllable reduction is a salient topic in the phonology of Naxi, a language closely related to Na. The topic was encountered in Naxi narratives and investigated in a phonetic study.

\begin{quotation}
The Western dialect of \ili{Naxi} has four lexical tones: High, Mid, Low and Rising; the latter is rare in the lexicon. Rising contours on monosyllables are frequent in connected speech, however, as a result of a process of syllable reduction: reduction of a morpheme carrying the High tone results in re-association of its tone to the syllable that precedes it in the sentence, creating a rising contour. An experiment (with one speaker and five listeners) establishes that there is not only one rising contour that originates in tonal reassociation, as reported in earlier descriptions, but two: Low-to-High and Mid-to-High~-- as could be expected by analogy with phenomena observed in Niger-Congo languages and elsewhere. A second set of experiments ({\dots}) investigates the reduction of Mid- and Low-tone syllables: they reduce to [\ipa{ə̄}] and [\ipa{ə̀}], respectively, and coalesce with the preceding syllable (in Naxi, syllabic structure is simply \textit{consonant+glide+vowel}). Unlike High-tone syllable reduction, this process stops short of complete tonal de-linking. \citep[237]{michaudetal2007d}
\end{quotation}

The topic was taken up by a native speaker, He Likun \zh{和丽昆}, who carried out a highly detailed study. Not only did he bring out further patterns, but he also advanced the discussion about the applicability of the notions of floating tones and tonal reassociation, confirming that complete tonal de-linking (producing floating tones) is not the majority case. 

\begin{quotation}
Previous studies report that some syllables carrying high tone in \ili{Naxi} get deleted (in function words), leaving only a “floating tone” which reassociates to the previous syllable in the sentence. Re-examination of this phenomenon reveals that syllable reduction is extremely common in present-day Naxi, and that it is not limited to high-tone syllables: there are as many as twelve possible tonal combinations in syllable reduction. These reduced syllables behave in ways that are tantalizingly similar to floating tones, but do not entirely match this concept. The study of reduction and coarticulation phenomena in \ili{Naxi} opens a window onto the phonetic complexity of tonal variation and change; it provides insights into evolutionary paths in tone systems, and their ties to processes of \isi{grammaticalization}. \citep[][]{he_naxiyu_2021}
\end{quotation}
 
In Na, syllable reduction leading to tonal change is nowhere as salient as in Naxi. Nonetheless, it is attested in at least one case, described in \sectref{sec:the1st2ndand3rdpersonpronouns}. The 3\textsuperscript{rd}-person \is{pronouns}pronoun, when combined with the possessive, appears in two forms: /\ipa{ʈʂʰɯ˧=bv̩˧}/ and /\ipa{ʈʂʰɯ˧=bv̩˩}/. These
are not tonal variants but semantically distinct forms. The latter, /\ipa{ʈʂʰɯ˧=bv̩˩}/, is
a~reduced form of /\ipa{ʈʂʰɯ˧=ɻ̩˩=bv̩˩}/, where /\ipa{=ɻ̩˩}/ is the \isi{associative plural}. Thus,
/\ipa{ʈʂʰɯ˧=bv̩˩}/ means ‘their’, whereas /\ipa{ʈʂʰɯ˧=bv̩˧}/ simply means ‘her/his’. In the expression /\ipa{ʈʂʰɯ˧=bv̩˩}/, the
\isi{associative plural} /\ipa{=ɻ̩˩}/ is fully elided: it leaves no segmental trace, only a~tonal
difference on the \isi{possessive} particle. This example illustrates how segmental
simplifications can contribute to the expansion of \isi{morphotonology}, a~type of {diachronic} change that is especially well\babelhyphen{nobreak}attested in {Bantu} languages.

\section{Disyllabification}
\label{sec:disyllabification}

\is{disyllabification|textbf}
\is{disyllabification|(}

As mentioned at the outset of Chapter~\ref{chap:compoundnouns}, many roots that were once phonologically
distinct have become \is{homophony}homophonous in Na, as in other \il{Sino-Tibetan}Sino-Tibetan languages that have undergone
considerable \isi{phonological erosion} (such as \ili{Tujia}, Bai, \ili{Namuyi}, and \ili{Shixing}). As a~consequence, there
is a~strong tendency towards disyllabification. 

If each tonal combination of two
monosyllables were to create a~distinct tonal category for the resulting \is{disyllables}disyllable, this would
multiply the number of tones by squaring: six tones on monosyllables could theoretically yield 6×6=36 tonal categories on
disyllables. The actual number is much smaller: eleven tone categories for disyllabic nouns. Investigating the relationship between the tones of monosyllables and those of disyllables
holds promise for an understanding of the dynamics of the tone system.


\subsection{A dynamic analysis of compound nouns}
\label{sec:adynamicanalysisofcompoundnouns}


The analysis of \is{compounds}compound nouns in Chapter~\ref{chap:compoundnouns} aimed to elucidate the relationship between input nouns
and the resulting \is{compounds}compound. The notations chosen for the tones of compounds were designed to bring out their internal
makeup. For instance, the combination of a~\#H-tone determiner and an LM-tone head yields an M.H
surface phonological tone pattern, as in (\ref{ex:horseskin}).

\begin{exe}
	\ex
	\label{ex:horseskin}
	\ipaex{ʐwæ˧-ɣɯ˥}\\ 
		\gll ʐwæ˥		ɣɯ˩˧\\
		horse		skin\\
		\glt ‘horse’s skin’ (\textit{DetermCompounds6.24} \pandoi{0004454\#W24}, \textit{7.67-68} \pandoi{0004456\#W67}, \textit{12.44} \pandoi{0004462\#S44})
\end{exe}

The process leading from the input tones to the tone of the \is{compounds}compound can be interpreted as follows: the lexical tone of the determiner, being a~\is{floating tone}floating H tone (which is never realized on the lexical item itself but only on a~following syllable), associates to the
second syllable of the \is{compounds}compound. The assignment of surface
tones then takes place according to the general rules governing the association of \#H tone in
Yongning Na. Since the \isi{tone group} contains a~following syllable that can host the floating tone (namely, the
second syllable of the \is{compounds}compound), the H tone attaches there, while the first syllable of the \is{compounds}compound
receives M by default (through Rule~2). In Chapter~\ref{chap:compoundnouns}, the notation used for this combination was \#H--, using the symbol
‘--' to stand for the last syllable of the first component of the \is{compounds}compound. The complexity of this notation is justified by the fact that it reflects a~hypothesis about how the \is{compounds}compound's tone pattern obtains. Such notations are referred to below as \textit{source-oriented}.

In terms of end result, on the other hand, the \is{compounds}compound in (\ref{ex:horseskin}) belongs to the tone category H\#: it
carries a~final H tone that does not shift. Disyllabic compounds made up of the combination of a~\#H-tone
determiner and an LM-tone head thus feed into the lexical category of H\# disyllables. This equally valid notation, H\#, is referred to below as \textit{result-oriented}.

Likewise, the source-oriented notation --L corresponds to the
result-oriented notation L\#: assigning an L tone after the \is{juncture (inside a tone group)}juncture between the two parts of the disyllabic \is{compounds}compound yields the same outcome as assigning a~final L tone to the entire
expression. \tabref{tab:sourceorientedandresultorientednotationsofthetonesofcompoundsthreeexamples}
provides a~summary.

\begin{table}%[t]
\caption{Source-oriented and result-oriented notations of the tones of compounds: three examples.}
\begin{tabularx}{\textwidth}{ l Q l l }
\lsptoprule
	 &  & \multicolumn{2}{l}{phonological analysis}\\ \cmidrule{3-4}
	input tones & surface phonological tone & source-oriented & result-oriented\\\midrule
	\#H and LM & M.H & \#H-- & H\#\\
	M and LM & M.L & --L & L\#\\
	M and L & M.L & --L & L\#\\
\lspbottomrule
\end{tabularx}
\label{tab:sourceorientedandresultorientednotationsofthetonesofcompoundsthreeexamples}
\end{table}


The table presenting the tone patterns of disyllabic compounds (\tabref{tab:surfacemonosyllabicmonosyllables} in Chapter~\ref{chap:compoundnouns}) is rewritten below as \tabref{tab:thetonesofdisyllabiccompounds}, adopting a~result-oriented notation that eliminates all references to the \is{juncture (inside a tone group)}juncture between
the two components of the \is{compounds}compound (transcribed by means of the symbol ‘--' in \tabref{tab:sourceorientedandresultorientednotationsofthetonesofcompoundsthreeexamples}). Each
row corresponds to a~tonal category of determiners, and each column to a~tonal category of
heads.\footnote{The same treatment cannot be extended to compounds of more than two syllables: describing their tone patterns without reference to the {juncture} between the input nouns would require a~complete change of notation, for instance specifying the tone of each syllable individually.
%it is not possible to describe the tone patterns of these compounds without referring to the {juncture} between the two input nouns, except by changing the entire notation, for instance specifying the tone of each syllable.
}

\begin{table}%[t]
\caption{The tones of disyllabic compounds, adopting a~result-oriented notation. The four combinations transcribed differently from \tabref{tab:sourceorientedandresultorientednotationsofthetonesofcompoundsthreeexamples} are set in italics.}
{\renewcommand{\arraystretch}{1.25}
\begin{tabularx}{\textwidth}{ l@{\hspace{8mm}} Q Q Q Q Q }
\lsptoprule
	tone & LH; LM & M & L & \#H & MH\#\\ \midrule
	LM & LM & LM & LM & \tikzmark{6a} LM+\#H & \tikzmark{5a} LM+MH\#\\
	LH & LH & L & LH & \hspace*{\fill}\tikzmark{6e} & \hspace*{\fill}\tikzmark{5e}\\
	M & \textit{L\#} & \#H & \textit{L\#} & \#H & MH\#\\
	L & \tikzmark{1a} L & & & & \hspace*{\fill}\tikzmark{1e}\\
	\#H & \textit{H\#} & \tikzmark{2a}\#H & & \hspace*{\fill}\tikzmark{2e} & \textit{L\#}\\
	MH & \tikzmark{3a} H\# & & \hspace*{\fill}\tikzmark{3e} & \tikzmark{4a} H\$ & \hspace*{\fill}\tikzmark{4e}\\
\lspbottomrule
\end{tabularx}}
\DrawBox[dashed]{1a}{1e}
\DrawBox[dashed]{2a}{2e}
\DrawBox[dashed]{3a}{3e}
\DrawBox[dashed]{4a}{4e}
\DrawBox[dashed]{5a}{5e}
\DrawBox[dashed]{6a}{6e}
\label{tab:thetonesofdisyllabiccompounds}
\end{table}

All of the tone categories observed on \is{disyllables}disyllabic nouns in Yongning Na appear in \tabref{tab:thetonesofdisyllabiccompounds}, with the sole exception of M. This reveals that the synchronically productive \is{tone rules}tone rules that apply in compounds feed into all
of the tone categories of disyllables except M.

\subsection{Possible origins of disyllables in view of their tone}
\label{sec:possibleoriginsfordisyllablesonthebasisoftheirtoneabirdseyeview}


\tabref{tab:possibleoriginsofdisyllabicitemsinviewofcurrentlyproductivetonerules} flips around the \is{morphotonology}morphotonological rules set out in Chapters~\ref{chap:compoundnouns} and~\ref{chap:combinationsofnounswithgrammaticalwords} to provide
an overview of possible origins of \is{disyllables}disyllabic items in view of currently productive tone rules. A~ ‘–’ indicates that no example was found. H\# tone as an outcome of \is{suffixes}suffixation is labelled as \textit{dubious} because no firmly established pattern of {correspondence} has been identified between monosyllables and suffixed forms carrying \mbox{//H\#//} tone; only isolated tokens exist, whose analysis remains problematic. For instance, /\ipa{tse˧mi˥}/ ‘cigarette lighter’ has \mbox{//H\#//} tone and looks like it results from {suffixation}, but no corresponding {monosyllable} has been found, so that there is no possibility to establish a~tone {correspondence} between root and suffixed form.

The bird’s-eye view in \tabref{tab:possibleoriginsofdisyllabicitemsinviewofcurrentlyproductivetonerules} can provide a~clue for the analysis of disyllabic words whose \isi{etymology} is
uncertain. 

\begin{table}%[t]
\caption{Possible origins of disyllabic items, in view of currently productive tone rules.}
\begin{tabularx}{\textwidth}{ Q Q Q Q }
\lsptoprule
	tone & compounding & \is{suffixes}suffixation & \is{prefixes}prefixation\\\midrule
	M & -- & yes & yes\\
	\#H & yes & yes & --\\
	MH\# & yes & -- & yes\\
	H\$ & yes & yes & yes\\
	L & yes & yes & yes\\
	L\# & yes & -- & yes\\
	LM+MH\# & yes & -- & --\\
	LM+\#H & yes & yes & --\\
	LM & yes & yes & --\\
	LH & yes & yes & --\\
	H\# & yes & \textit{dubious} & --\\
\lspbottomrule
\end{tabularx}
\label{tab:possibleoriginsofdisyllabicitemsinviewofcurrentlyproductivetonerules}
\end{table}

\subsection{Recovering the tones of nouns on the basis of compounds}
\label{sec:recoveringthetonesofnounsonthebasisofcompounds}

It is tempting to try to work backwards from the tones of compounds to those of their constituting elements. For
instance, ‘elder sibling (brother or sister)’ is most commonly realized as /\ipa{ə˧mv̩˩}/ (tone: L\#), but
it has a~\is{variants}variant with MH\# tone: /\ipa{ə˧mv̩˧˥}/. The tone of the coordinative compound
/\ipa{ə˧mv̩˧-gi˥zɯ˩}/ ‘brothers’ (made up of ‘elder sibling’ + ‘younger brother’) is the one
expected for an input MH\# tone, not an input L\# tone. This could suggest
that the MH\# \is{variants}variant of ‘elder sibling’, /\ipa{ə˧mv̩˧˥}/, was the form that entered into the formation of the \is{compounds}compound. Seen in this light,
the relative rarity of the MH\# \is{variants}variant in present-day speech, where /\ipa{ə˧mv̩˩}/ is far more common, suggests
that the MH\# \is{variants}variant /\ipa{ə˧mv̩˧˥}/ is not a~recent \is{innovative (phonological form)}{innovation} but a~form which is currently losing ground to
/\ipa{ə˧mv̩˩}/.

However, the greatest care must be exercised when attempting to recover earlier tones through such inferences, as different tone rules may have applied at different stages of the language’s history. As pointed out by Nathan Hill (p.c.\ 2016), it remains entirely possible that the MH\# \is{variants}variant /\ipa{ə˧mv̩˧˥}/ for ‘elder sibling’ was not inherited but instead inferred from the \is{compounds}compound~-- whatever the origin of the \is{compounds}compound's tone may have been~--, making it an \is{innovative (phonological form)}{innovation}. 

\is{disyllabification|)}



\section{The influence of bilingualism with Mandarin}
\label{sec:theinfluenceofbilingualismwithchinese}

\is{bilingualism|textbf}
\il{Mandarin|textbf}

\is{language contact}Language contact is well known to be a~key factor in linguistic
change. an exemplary illustration of how the study of present-day contact dynamics can shed light on prosodic systems is provided by \citet{kubozono2007} in his analysis of  \il{Japanese!Kagoshima}Kagoshima Japanese. The Kagoshima dialect has two prosodic patterns at word level: one (Tone A) with a~high tone on the penultimate syllable (i.e.\ a~fall in pitch from the penultimate to the last syllable) and the other (Tone B) with a~high tone on the final syllable (i.e.\ no pitch fall within the word). At the time of study, this dialect was undergoing tonal change due to influence from \il{Japanese!Tokyo}Tokyo {Japanese}, the national standard: words that involve an abrupt pitch fall in \il{Japanese!Tokyo}Tokyo tended to be reinterpreted as carrying Tone B in \il{Japanese!Kagoshima}Kagoshima, and those without one as Tone A. This sheds light on the issue whether the \il{Japanese!Kagoshima}Kagoshima prosodic system is best analyzed as tonal or accentual: “the tonal changes in {question} can best be understood if an accentual analysis of \il{Japanese!Kagoshima}Kagoshima {Japanese} \isi{prosody} is adopted in preference to the traditional tonal analysis” (\citealt[348]{kubozono2007}; further supporting evidence from a~follow-up study of twenty speakers is reported by \citealt{otaetal2016}).

Since the present volume adopts a~synchronic perspective, past
\is{language contact}contact between Na, \ili{Tibetan}, Chinese, \ili{Pumi}, \ili{Lisu}, \ili{Naxi} and other
languages is not investigated (apart from brief remarks in \sectref{sec:thetonegroupasbuildingblockofutterancesanditsroleinconveyinginformationstructure}). Instead, this section focuses
on the current landscape of \isi{language contact}, in which \ili{Mandarin} has, by
far, the leading role. To take the example of the main consultant,
{Mandarin} is the only language other than her mother tongue of which she
has any knowledge.\footnote{Since moving to Lijiang (2010), she has had relatively frequent	contacts with {Naxi} speakers, however, leading to at least one amendment to her Na vocabulary. The Na refer to the {Naxi} as //\ipa{nɑ˩hĩ\#˥}//, a~calque modelled on the structure of the {Naxi} word /\ipa{nɑ˩çi˧}/,
	made up of an {endonym} which is segmentally identical in both languages (/\ipa{nɑ}/) combined with the word for
	‘person, human being’: {Naxi} /\ipa{çi˧}/, Na //\ipa{hĩ˥}//. Initially, the main consultant used the Na pronunciation //\ipa{nɑ˩hĩ\#˥}// when
	speaking with {Naxi} people in Lijiang. But to a~{Naxi} listener, this pronunciation does not sound right, as \ili{Naxi} does not have nasalization in the syllable for ‘person, human being’. Whether at the suggestion of \ili{Naxi}
	speakers or through spontaneous accommodation, she shifted to referring to the
	{Naxi} as /\ipa{nɑ˩ɕi˥}/, denasalizing the second syllable. This amounts to borrowing the word from
	\ili{Naxi}, instead of calquing it using Na morphemes.}
The guiding principle in focusing on the influence of \ili{Mandarin} is that “extracting as much historical information from clear \is{language contact}contact phenomena as possible before attempting greater time depths may be the order of investigation most likely to be fruitful” \citep[485]{souag2010}.


\ili{Mandarin} is a~relatively recent arrival in this area. The feudal chieftain of Yongning
spoke Na, and up until the mid-20\textsuperscript{th} century the Na language held a~dominant position in the Yongning basin. Few Han Chinese had settled in Yongning, and those who did learnt Na, which was the locally
dominant language and served as the lingua franca in the Yongning marketplace for speakers
of other languages, such as \ili{Pumi}, \ili{Lisu}, and \ili{Nosu} ({Nuosu}). While
there can be no doubt that the Na language received various influences in the course of its history, \isi{bilingualism} was not widespread among native speakers of Yongning Na: locally, the expectation was that speakers of other languages would learn some Yongning Na, rather than the other way round. It is reported by Mrs.\ Latami that numerous Na speakers had very little command of other languages, or none at all. This situation is somewhat uncommon in
this region, at the Sichuan-Yunnan border. For instance, the
small Na-speaking community in the neighbouring county of Muli \zh{木里} (Shuiluo \zh{水落} township) is bilingual in \ili{Shixing} (Xumi) and has some command of
\ili{Tibetan} and \ili{Pumi}, while the variety of Na spoken in Guabie \zh{瓜別} has long been subject to influence from other languages, in particular \ili{Pumi} and \ili{Nosu}.

While Yongning still preserves the role of a~meeting place and market
centre for inhabitants of surrounding mountain villages,
for instance for the \ili{Nosu} and \ili{Pumi} \citep[85]{wellens2006}, language shift from Na to
\ili{Mandarin} is now under way in Yongning. All Na speakers today have at least some
command of \ili{Mandarin}. While some members of the community express regret that their language is falling into gradual disuse, proficiency in \ili{Mandarin}~-- one of the keys to success in present-day society~-- is widely valued far above proficiency in Na. The blending of Na with \ili{Mandarin} is not stigmatized but instead accepted with tolerance, an attitude that facilitates linguistic change. 
%While every language exhibits a~degree of \isi{variation} at any given moment, linguistic change in the strict sense occurs only when innovative, non-traditional forms gain acceptance within the speech community.
%While a~pool of \isi{variation} is present at every moment and in any language, linguistic change in the strict sense requires the acceptance of innovative, deviant forms by the community of speakers. 


Within China, all level-tone systems are currently subject to the same pressure towards alignment with the tone system of {Mandarin}. The sociolinguistic landscape of Yongning Na offers interesting opportunities for
studying the effects of \isi{bilingualism}~-- specifically, \textit{non-egalitarian bilingualism}, to adopt a~notion from \citet{haudricourt1961b}~-- in languages with widely different tone systems. The account of Na
tone presented in Chapters~\ref{chap:thelexicaltonesofnouns} to \ref{chap:toneassignmentrulesandthedivisionoftheutteranceintotonegroups} of this volume
makes it clear how much this system differs from those of the \il{Sinitic}Chinese dialects
to which Na speakers are currently exposed, namely \il{Mandarin!Southwestern}Southwestern Mandarin
and \il{Mandarin!Standard}Standard Mandarin. As discussed in Chapter~\ref{chap:arealandtypologicaldiscussion}, tones
in Yongning Na are phonetically simple, consisting of three level tones and their combinations, whereas \ili{Mandarin} tones are phonetically \is{complex tones}complex. The Yongning Na tone system also includes certain oppositions that are
neutralized when a~word is said \is{form!in isolation}in isolation. 

When they learn
\ili{Mandarin}, Na speakers come to terms with a~tonal system that is structured in a fundamentally different way. In \ili{Mandarin}, leaving aside marginal cases
of toneless syllables and \isi{tone sandhi}, each syllable has its own tone, which surfaces as such \is{form!in isolation}in isolation. In contrast, the discrepancy between underlying forms and \is{form!surface}surface forms in Yongning Na makes its tones 
difficult to handle for bilingual
speakers who receive greater exposure to \ili{Mandarin} than to Na.\footnote{On the effects of bilingualism with Mandarin on the tone systems of other minority languages in rural Southwest China, see the two case studies presented by \citet{stanfordandevans2012}. These concern (i)~\ili{Sui} (Tai-Kadai family), which has a system of six phonetically \isi{complex tones} in unchecked syllables and two in checked syllables, and (ii)~Southern Qiang~/ \ili{Rma} (Sino-Tibetan family), whose tone system is based on a~binary opposition between H and L levels.}


\subsection{Loss of tone categories that do not surface in isolation}
\label{sec:thelossoftonecategoriesnotreflectedinsurfaceformsinisolation}

Among younger speakers, especially those who went to boarding school and predominantly use Southwestern \ili{Mandarin}, there is a~tendency to overlook the 
tonal oppositions that are neutralized \is{form!in isolation}in isolation, such as that between the \mbox{//H\#//} and \mbox{//H\$//} tones. The surface\is{form!surface} tone
pattern of a~word is reinterpreted as its underlying pattern, causing an upheaval in the architecture of the tonal system. 

For instance, the family name of the main language consultant is realized \is{form!in isolation}in isolation as
/\ipa{lɑ˧tʰɑ˧mi˥}/, while its \is{associative plural}{associative} form (‘the
Latami clan, the Latami family’) is /\ipa{lɑ˧tʰɑ˧mi˧=ɻ̩˥}/. This reveals that the family name belongs to tone category H\$, %(the  \textit{hopping} H tone), 
indicating that the underlying form is //\ipa{lɑ˧tʰɑ˧mi˥\$}//. But if these alternations are ignored and the M.M.H surface tone pattern in /\ipa{lɑ˧tʰɑ˧mi˥}/ is taken at face value, the word will instead be interpreted as carrying an H tone on its final syllable, leading to its classification within phonological category \mbox{//H\#//}, i.e.\ //\ipa{lɑ˧tʰɑ˧mi˥\#}//. As a~consequence, when building phrases, e.g.~forming an \is{associative plural}{associative plural} (‘the
Latami clan’) by adding the \is{suffixes}suffix
/\ipa{=ɻ̩˩}/, the H tone will be left sitting on the final syllable of the noun, yielding $\ddagger${\kern2pt}\ipa{lɑ˧tʰɑ˧mi˥=ɻ̩˩}. This pattern is observed in the speech of consultant F5, a~speaker proficient in both Na and \ili{Mandarin}, who was 35 years old at the time of fieldwork.

An especially difficult opposition to \is{language acquisition}acquire is that between the LM and LH tone patterns over disyllables, as it only surfaces when the noun is followed by a~\is{clitics}clitic. For instance, //\ipa{bo˩mi˧}// ‘sow,
female pig’ and //\ipa{bo˩ɬɑ˥}// ‘boar, male pig’ carry the same tones
not only \is{form!in isolation}in isolation (/\ipa{bo˩mi˥}/ and /\ipa{bo˩ɬɑ˥}/) but also in most other contexts. The few
environments in which this tonal opposition becomes apparent include constructions such as /\ipa{bo˩mi˧=bv̩˧}/ ‘of (a) sow’ vs.\ /\ipa{bo˩ɬɑ˥=bv̩˩}/
‘of (a) boar’. Even there, the H component of the lexical LH pattern is not detectable as such at a~phonetic level: given tonotactic restrictions, the sequences L.H.L and L.M.L are neutralized on the phonological surface, so that there is no phonetic cue to the presence of an H tone in /\ipa{bo˩ɬɑ˥=bv̩˩}/. The H tone in //\ipa{bo˩ɬɑ˥}// is only manifested indirectly: it induces lowering of the following \isi{possessive} clitic to L. In view of the complexity of this area of the \is{morphotonology}morphotonological system, it does not come as a~surprise that the lexical opposition between LM and LH
has been lost by some speakers, such as F5 and F6, for whom the tones of ‘sow’ and ‘boar’ have fully merged.



\subsection{Simplification of morphosyntactic tone rules}
\label{sec:thesimplificationofmorphosyntactictonerules}

\is{tone rules}

Examining the tables in Chapters \ref{chap:classifiers} to \ref{chap:verbsandtheircombinatoryproperties}, the complexity of Na tonal morphosyntax may look
mind-boggling. But the rules are productive, and the syntactic structures at
issue (object-plus-verb, \is{compounds}compound nouns{\dots}) are so frequent that they were not
particularly challenging for \is{language acquisition}first-language learners steeped in a~Na-speaking environment. For speakers with limited exposure and practice, however, these tonal patterns become
problematic: they are difficult to recall and apply if one does not practise the language regularly. This holds true for the visiting linguist as well as for many Na speakers below the age of sixty, many of whom use Mandarin more than Na in daily life. 

Deviant tonal patterns serve as harbingers of language change. To the linguist, such patterns serve as early indicators of ongoing structural reorganization within the tonal system. These range from occasional slips of the tongue to entrenched habits.

%1st edition: Evidence about ongoing language change can be gathered from deviant tonal patterns, which to the linguist are harbingers of language change. 
%%Other possible reformulation: 
%Deviant tonal patterns serve as harbingers of language change, serving as early indicators of ongoing structural reorganization within the tonal system. These range from occasional slips of the tongue to entrenched habits.

Instances of hesitation and tonal \isi{simplification} can be observed in the speech of M23, a~bilingual
language consultant. In subject-plus-verb and object-plus-verb constructions, as well as in compounds, the //L// tone, which surfaces as \mbox{/M/} \is{form!in isolation}in isolation, tends to be neutralized with \mbox{//M//}. For instance, consultant M23
consistently realizes ‘the sheep came’ as /\ipa{jo˧ {\kern2pt}|{\kern2pt} tsʰɯ˩-ze˩}/ (example (\ref{ex:sheeparrived}) in Chapter~\ref{chap:verbsandtheircombinatoryproperties}), instead of the expected form in (\ref{ex:sheeparrivedCLASSICAL}). In the view of the main consultant (Mrs.\ Latami, F4), the M.L.L pattern for ‘the sheep came’ is a~mere \is{variants}variant, and is moreover condemned as a~slip of the tongue; in M23's speech, it has become the
usual form. 

\begin{exe}
	\ex
	\label{ex:sheeparrivedCLASSICAL}
	\ipaex{jo˩ tsʰɯ˩-ze˩˥}\\
	\gll jo˩	tsʰɯ˩\textsubscript{a}	-ze˧\textsubscript{b}\\
	sheep		to\_come.\textsc{pst}		\textsc{pfv}\\
	\glt ‘the sheep have come’
\end{exe}

Similarly, in M23's speech the phrase ‘sheep’s muzzle’ is //\ipa{jo˧-ɲi˧gɤ˧}//, i.e.\ a~simple concatenation of the surface forms of the two nouns, diverging from the \is{conservative (phonological form)}{conservative} pattern found in F4’s speech, shown in (\ref{ex:sheepnose}).

\begin{exe}
	\ex
	\label{ex:sheepnose}
	\ipaex{jo˩-ɲi˩gɤ˩˥}\\
	\gll jo˩	ɲi˧gɤ˧\\
	sheep		nose/muzzle\\
	\glt ‘sheep's muzzle’
\end{exe}

In the \is{conservative (phonological form)}{conservative} form, (\ref{ex:sheepnose}), the tonal pattern of the \is{compounds}compound is phonologically identical with the lexical L tone of the determiner, //\ipa{jo˩}// ‘sheep’. This yields //\ipa{jo˩-ɲi˩gɤ˩}// ‘sheep’s muzzle’, which surfaces
as /\ipa{jo˩-ɲi˩gɤ˩˥}/ following the postlexical addition of a~final H tone, in accordance with Rule~7: all-L
tone groups are not allowed in Yongning Na, and if a~\isi{tone group} only contains L tones, a~postlexical H tone is added to its final syllable. A speaker needs a~good command of Na \isi{morphotonology} to implement this \is{conservative (phonological form)}{conservative} tonal pattern. The L tone of ‘sheep’, which does not
even surface \is{form!in isolation}in isolation, extends over no fewer than three syllables in
succession in the \is{compounds}compound, whereas the lexical tone of the head noun does not get to express itself. The underlying form of ‘nose, muzzle’ is //\ipa{ɲi˧gɤ˧}//, which surfaces as /{\kern2pt}\ipa{ɲi˧gɤ˧}/ \is{form!in isolation}in isolation but is changed to /{\dots}\ipa{ɲi˩gɤ˩}/ in (\ref{ex:sheepnose}). Such tonal processes are alien to \ili{Mandarin}. In light of this typological discrepancy, it is unsurprising that less proficient
speakers of Yongning Na who are exposed to \ili{Mandarin} on a~day-to-day basis should have hesitations and (occasionally or regularly) opt for a~direct concatenation of the tones as they surface in
isolation, as is the case in their second language (\ili{Mandarin}), instead of applying the \is{morphotonology}morphotonological rules of Na.

These examples illustrate the complexity of phenomena of language \isi{variation} and change. As noted earlier, a~given change may constitute a~\isi{simplification} from one point of view and a~complexification from other points of
view. Adopting an M.L.L tone pattern (/\ipa{jo˧ tsʰɯ˩-ze˩}/) for ‘the sheep arrived’, instead of the conservative L.L.L pattern in (\ref{ex:sheeparrivedCLASSICAL}), can be seen as a~\isi{simplification}
insofar as the subject bears the same tone as \is{form!in isolation}in isolation. However, it simultaneously increases the frequency of contexts in which the lexical L tone of ‘sheep’ does not
\is{form!surface}surface, making it more complex for \is{language acquisition}language learners to arrive at the identity of this word’s underlying
lexical tone.


\subsection{Straightening out irregular tone patterns}
\label{sec:thestraighteningoutofirregulartonepatterns}

In addition to losing some tone categories, less proficient speakers tend to regularize irregular
patterns, for want of the detailed knowledge of exceptions that fluent speakers possess. For instance, the word for ‘powder, flour’ is
/\ipa{tsɑ˧bɤ˧}/, which bears M tone. According to the synchronic rules that govern the tone patterns of \is{compounds}compound nouns,
its combination with /\ipa{lv̩˧mi˧}/ ‘stone’ should yield a~simple M-tone output,
$\dagger${\kern2pt}\ipa{lv̩˧mi˧-tsɑ˧bɤ˧}. (This \is{compounds}compound means ‘fine sand’.) However, in the speech of the older generation of
speakers, compounds involving /\ipa{tsɑ˧bɤ˧}/
‘powder, flour’ have an irregular tone pattern: the second part (the head noun ‘powder, flour’) systematically carries L tone. These compounds belong in a~set of expressions referred to as ‘\ili{Tibetan} compounds' (see \sectref{sec:thenounflourpowder} and \sectref{sec:anindependentsetoffactscompoundgivennamesandtermsofaddress}): they contain \ili{Tibetan} loanwords and consistently have L tone on their second part. Examples were provided in \tabref{tab:powder} and \tabref{tab:Names}. 

This irregularity is levelled out in the speech of less proficient speakers, who produce these compounds with an M.M.M.M tonal string
(phonological analysis: /M/). Thus, instead of the conservative realizations, they use forms such as $\ddagger${\kern2pt}\ipa{lv̩˧mi˧-tsɑ˧bɤ˧} ‘fine sand’,
$\ddagger${\kern2pt}\ipa{qʰɑ˧dze˧-tsɑ˧bɤ˧} ‘sweetcorn flour’, and $\ddagger${\kern2pt}\ipa{dze˧ɭɯ˧-tsɑ˧bɤ˧} ‘wheat flour’.


\subsection{Cases where MH tone fails to unfold: Towards a~syllable-tone system?}
\label{sec:caseswheremhtonefailstosplitintotwolevelshintofapotentialforevolutiontowardsasyllabletonesystem}

In Yongning Na, plentiful synchronic evidence supports the analysis of \is{tonal contour}contours as sequences of levels. There are nonetheless some weak hints of a~tendency for MH contours to be treated as units
associated with a~syllable. When saying MH-tone \is{monosyllables}monosyllabic nouns in the frame ‘This is~{\dots}’,
Mrs.\ Latami (consultant F4) occasionally realized the target noun with a~rising \is{tonal contour}{contour}, as illustrated in (\ref{ex:isasheep}). The standard realization is provided in (\ref{ex:isasheepcorrect}).

\begin{exe}
	\ex
	\label{ex:isasheep}
	\ipaex{$\ddagger${\kern2pt}ʈʂʰɯ˧ {\kern2pt}|{\kern2pt} tsʰɯ˧˥ ɲi˩.{\kern3pt}{≈}{\kern3pt}$\ddagger${\kern2pt}ʈʂʰɯ˧ {\kern2pt}|{\kern2pt} tsʰɯ˧˥ ɲi˥.}\\ 
		\gll ʈʂʰɯ˥		tsʰɯ˧˥	ɲi˩\\
		\textsc{dem.prox}	goat	\textsc{cop}\\
		\glt ‘This is a sheep.’ 
\end{exe}

%\Hack{\newpage}

\begin{exe}
	\ex
	\label{ex:isasheepcorrect}
	\ipaex{ʈʂʰɯ˧ {\kern2pt}|{\kern2pt} tsʰɯ˧ ɲi˥.}\\ 
	\gll ʈʂʰɯ˥		tsʰɯ˧˥	ɲi˩\\
	\textsc{dem.prox}	goat	\textsc{cop}\\
	\glt ‘This is a sheep.’
\end{exe}

When her attention was
drawn to these discrepancies, the consultant said: /\ipa{ɖɯ˧-bæ˧ lɑ˧ ɲi˥}/, “It’s just the
same”. Intonationally, realization of the MH \is{tonal contour}contour on the syllable to which it is lexically
attached tends to occur when the word is \is{emphasis}emphasized.

This tendency surfaces here and there in the recordings. An example is the phonetic realization of /\ipa{ʈʂʰæ˧-pɤ˥to˩}/ ‘even a~deer’ as [\ipa{ʈʂʰæ˧˥-pɤ˥to˩}] (in the recording \textit{NounsEven.7} \pandoi{0004566\#W7}), where: 

\begin{itemize}
    \item {[\ipa{pɤ˥}] carries an H tone due to reassociation of the H part of the MH \is{tonal contour}contour of /\ipa{ʈʂʰæ˧˥}/ ‘deer’, as expected; and}
    \item {[\ipa{ʈʂʰæ˧˥}] exhibits an MH \is{tonal contour}contour, reflecting incomplete dissociation of the H component of its phonological MH \is{tonal contour}contour.}
\end{itemize}

This is only a~weak tendency, which by no means warrants the conclusion that Yongning Na is on its
way towards adopting a~syllable-tone system. A~detailed cross-linguistic phonetic study would be
required to assess how widespread such tendencies are among the world’s level-tone
systems. Such a~study might reveal that Yongning Na is not exceptional in this respect. The
computation of tone sequences is not a~purely mechanical process, and slips of the tongue whereby
an H tone does not fully dissociate from the syllable to which it is lexically attached may not come
as a~surprise to linguists familiar with level-tone systems. 
Even so, this synchronic tendency
appeared well worth mentioning in relation to the influence currently exerted by {Mandarin}.

\ili{Naxi}, a~close relative of Yongning Na, can be hypothesized to constitute an example of a language which the influence of \ili{Mandarin} is nudging towards reinterpretation of its tones as units attached to the syllables, rather than as High, Mid and Low levels that can combine among themselves. 
%has few phenomena of tone change. In the A-sher dialect, the
%reduction of a~morpheme carrying H tone results in reassociation of this tone to the syllable to the preceding syllable \citep{michaud2006d,michaudetal2007d}. Informal observations and discussions with
%\ili{Naxi} speakers living in the city of Lijiang suggest that even these limited instances of tone change
%are disappearing from Lijiang \ili{Naxi}. For instance, in A-sher, the conditional morpheme is /\ipa{se˥}/, but it is preceded by a~\is{floating tone}floating H tone (its full form can be transcribed as //\ipa{˥}~\ipa{se˥}//). This floating H is presumably the historical outcome of the reduction of a~disyllabic form, *\ipa{ɭɯ˥~se˥}. The phenomenon is still reported in a~dictionary compiled from 1995 to 2012: “This word is fairly unique in that it triggers the mid
%or low tone of the preceding syllable to become a~rising tone”
%\citep[337]{pinsonetal2012}. Impressionistic observations made in Lijiang around 2010 suggest that in the speech of younger speakers this morpheme is being simplified to /\ipa{se˥}/, with loss of the preceding floating H. It is not unlikely that increasing use of \ili{Mandarin} is exerting influencing on \ili{Naxi}, leading to the reinterpretation of its tones as units attached to the syllables, rather than as levels that can combine
%among themselves.
Of the four tones of \ili{Naxi}, High, Mid, Low, and Rising, the last is especially revealing in this
respect. It is clearly an \is{innovative (phonological form)}{innovation} that emerged in a~system previously containing only three levels: High, Mid and Low. A~plausible historical scenario is that rising contours developed through processes of syllable reduction and subsequently became lexicalized in certain words. Once such rising tones had become part of the \ili{Naxi} lexicon, the way was paved for the assignment of a~/LH/ tonal sequence to \ili{Mandarin} words with rising tone. Borrowings from \il{Mandarin!Southwestern}Southwestern Mandarin then consolidated this marginal tonal category by giving it substantial lexical expansion \citep{michaud2003b,michaud2006d,michaudetal2007d}. In recent decades, bilingualism with \ili{Mandarin} gradually tilted the perception of \ili{Naxi} tones
towards a~syllable-tone system~-- to the point that it is now an open question whether the rising tone of \ili{Naxi}
should be analyzed as a~sequence of levels (L+H or M+H) or as an indecomposable phonological unit (akin to contours in \ili{Mandarin}). Given that most \ili{Naxi} speakers are highly proficient in Southwestern \ili{Mandarin}, it is unclear to what extent they maintain a~cognitive distinction between the tone systems of the two languages. To venture a~hypothesis, the lexical rising tone of \ili{Naxi} currently appears to function primarily as an indecomposable {contour}.


\subsection{A topic for future research: Influence of language contact on intonation}
\label{sec:presentdaysociolinguisticsituationofyongningnacontactwithacomplextonesystemmandarinchinese}
\label{sec:presentdaysociolinguisticsituationeffectsontone}
\label{sec:presentdaysociolinguisticsituationeffectsonintonation}

\is{language contact}

Intonation is particularly susceptible to carry-over from one language to another in the speech of bilinguals (see e.g.\ \citealt{lai_language_2022}). A~striking example involving \ili{Vietnamese} speakers in a~\ili{French}-speaking environment is reported by \citet[401]{dungetal1998}. In the case of Yongning Na, a~full-fledged study of \isi{intonation} should include a~description and analysis of the \isi{intonation} patterns used by Na speakers when speaking {Mandarin}, comparing them with their \isi{intonation} in Na and investigating cases of interaction between the two. This is not an easy topic, however. The main consultant uses {Mandarin} reluctantly and hesitantly, lacking confidence and feeling awkward in the language. Her own assessment is that she will never be able to speak “proper Chinese'' and will make do with “pig Chinese'' until her last breath. 

A~further complication is that the variety of Chinese to which she was occasionally exposed until 2000 was Southwestern \il{Mandarin!Southwestern}Mandarin, whereas since 2000 she has had regular exposure to Standard \il{Mandarin!Standard}Mandarin through television. While the tone systems of these two dialects are almost identical phonologically, the four tones' phonetic templates differ sufficiently to complicate accommodation to their respective \isi{intonation} patterns. Consultant F4 does her part in dialogues with {Mandarin}-speaking relatives and acquaintances, but recording these interactions and transcribing them with the speakers would have run counter to the implicit focus of our collaboration on her mother tongue, Yongning Na. The consultant's comfort zone was respected, and no pieces in {Mandarin} were recorded. 

As a~(lame) consolation for not being able to offer data on this topic, here is an example from another language illustrating how \isi{language contact} can exert a~strong influence on \isi{intonation}. \ili{Wolof}, one of the nontonal languages of the Niger-Congo family, has been described as typologically unusual in that it does not have any intonational marking of focus \citep{riallandetal2001}. Information structure is conveyed through verbal morphology: there are three “nonfocusing conjugations'' and three “focusing conjugations''. The latter “vary according to the syntactic status of the focused constituent: subject, verb or complement (in the wide sense of any constituent that is neither subject nor main verb)'' \citep[895]{riallandetal2001}. As \ili{Wolof} ascends “from its origins in the heartland of Senegal to the status of urban vernacular and national \textit{lingua franca}'' \citep[142]{mclaughlin2008}, it is increasingly acquired as a~second language by speakers of diverse linguistic backgrounds. 

Informal discussion with a~speaker of \ili{Wolof} who also speaks some \ili{Bambara} suggests that, in her speech, focus is clearly marked intonationally: the focused constituent `Peer' (a proper name) in (\ref{ex:woloffoc}) is pronounced with a~raised pitch, unlike in (\ref{ex:wolofnonfoc}), where it is not under focus. 


\begin{exe}
	\ex
	\label{ex:woloffoc} % :)
	\gll {\dots} Peer moo ko lekk\\
	{} Peer \textsc{subjemph.3sg} \textsc{opr} to\_eat\\
	\glt ‘It was Peer who ate it.’ (\textsc{subjemph} stands for \textit{subject-emphatic}, and \textsc{opr} for \textit{object pronoun}.)
\end{exe}

\begin{exe}
	\ex
	\label{ex:wolofnonfoc} % :)
	\gll {\dots} Peer lekk na\\
	{} Peer to\_eat \textsc{pft.3sg}\\
	\glt ‘Peer has eaten.’ (No focused constituent.)
\end{exe}

This attests to the possibility of carry-over of \isi{intonation} patterns from \ili{Bambara} to \ili{Wolof} in bilingual speakers. Such \is{language contact}contact-induced influences can have far-reaching consequences for the evolution of a~language's \isi{intonation} system.
