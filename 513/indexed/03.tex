\chapter{Nominal classifiers}
\label{chap:classifiers}
\largerpage[2]
\is{classifiers|(}
\is{numerals|(}

A nominal classifier \is{classifiers|textbf} is “a type of limited noun that occurs only after numerals ({\dots}), and whose
selection is determined by a~preceding (overt or implicit) noun”
\citep[88]{matisoff1973a}. While classifiers are, morphologically speaking, nouns, their tonal behaviour is highly specific and exhibits remarkable complexity. This justifies dedicating a separate chapter to their analysis rather than including them within Chapter~\ref{chap:thelexicaltonesofnouns} on the lexical tones of nouns.\footnote{In the first edition, this chapter was Chapter 4. It has been repositioned to follow immediately after the chapter on nouns.}

The term ‘classifier’ is understood here in the
syntactic sense of any noun that may appear directly after
a~numeral. This includes measure words (‘inch', ‘armspan', ‘heap', {\dots}) and time words (‘day',
‘month', ‘year', {\dots}), which immediately follow the numeral. In view of their great importance in the language and the richness of their tone patterns, classifiers are dealt with in a~chapter of their own. The main focus is on their tonal behaviour in numeral-plus-classifier phrases (\sectref{sec:numeralplusclassifierphrases}), but classifiers
also appear in demonstrative-plus-classifier constructions (\sectref{sec:demonstrativeplusclassifierphrases}). Moreover, numeral-plus-classifier and
demonstrative-plus-clas\-si\-fi\-er phrases can interact tonally with a~preceding noun (\sectref{sec:interactionnoun}).

Devoting tens of pages to nominal classifiers, after an extensive chapter on the tones of nouns (Chapter~\ref{chap:thelexicaltonesofnouns}), may test the patience of even the most willing reader. Will the archipelago of tables never come to an end?
Is there no more concise way to describe tone in Yongning Na than by recording lump after lump of morphotonological detail?

Indeed, the tone patterns of classifiers phrases are a~disappointment to the linguist, in that they cannot be accounted for by phonological sandhi, nor do they follow a~single, well-defined set of \is{morphotonology}morphotonological rules.\footnote{For clarifications on the terms ‘{tone sandhi}’, ‘morphotonology’, and ‘tonal morphology’, see \sectref{sec:definitionofterms}.} Instead, the tonal paradigms of the various categories of classifiers need to be learnt one by one. The tables documenting these patterns are, admittedly, the most heavy-going in this volume. Yet paradoxically, this chapter, which deals with the most irregular part of the Yongning Na tone system, carries a~reassuring message: there \textit{is} a~limit to the seemingly endless complexity of the system, even in its least predictable areas. I~venture to claim (taking full responsibility for this immodest stance) that this chapter comes close to an exhaustive description of its object, thereby showing that a~comprehensive account of tone in Yongning Na is not beyond reach.

%They have no more synchronic motivation than other irregular patterns such as \isi{suppletion} between the classifier /\ipa{v̩˧}/, in association with the numeral ‘one', and /\ipa{kv̩˧˥}/, with higher numerals, to count human beings.


\section{Numeral-plus-classifier phrases}
\label{sec:numeralplusclassifierphrases}

Within the \ili{Naish} language group, tonal alternations in numeral-plus-classifier phrases are
vestigial in \ili{Naxi} (as spoken in the plain of Lijiang), limited in \ili{Laze}, and ubiquitous in Yongning
Na. In the Alawua dialect of Yongning Na, the tone of the
classifier is affected by the numeral to such an extent that arriving at the classifier's \is{form!underlying}underlying tone is far from straightforward. For instance, the classifier for ‘day’ carries an H tone in /\ipa{ɲi˧-ɲi˥}/ ‘two
days’, an LH tone in /\ipa{so˩-ɲi˩˥}/ ‘three days’, and an M tone in /\ipa{ʐv̩˧-ɲi˧}/ ‘four days’; and the
classifier for ‘year’ has an MH tone in /\ipa{ɲi˧-kʰv̩˧˥}/ ‘two years’, an LH tone in /\ipa{so˩-kʰv̩˩˥}/
‘three years’, and an L tone in /\ipa{ʐv̩˧-kʰv̩˩}/ ‘four years’. This section proposes a~synchronic description and analysis.\footnote{An earlier version of this section was published as
  \citet{michaud2013d}. Among other differences, subcategories of tones for classifiers are now distinguished using subscript letters, e.g.~M\textsubscript{a} and M\textsubscript{b} instead of the earlier
  notation as M1 and M2.}

For each noun documented during fieldwork, a~note was made of its typical
classifier(s), and this piece of information was included in the dictionary of Yongning Na \citep{michaud_et_al_na_dict_2024}, following the example of Eugénie Henderson’s dictionary
of Bwe \ili{Karen} \citep{henderson1997}. Naturally, this does not capture the full range of \is{stylistics}stylistic possibilities in
classifier choice, which is best studied through actual language use: “a noun can be accompanied
by various classifiers depending on context,
so it may not be adequate to describe one
of these as the primary classifier at a~lexical level” \citep[167]{francois2000a}.\footnote{\textit{Original text:} le même nom peut être accompagné de plusieurs classificateurs selon les contextes, sans qu'il soit légitime d'en privilégier un comme fondamental dès le lexique.} Nonetheless, this information proved valuable in assembling a comprehensive list of classifiers, yielding over one hundred monosyllabic items. (Disyllabic classifiers will be addressed separately in
\sectref{sec:disyllabicclassifiersandimplicationsforthetonesofnumerals}.)


\subsection{Elicitation procedures}
\label{sec:elicitationprocedures}
\label{sec:recordings}

The logical first step in studying the tones of classifiers was to conduct systematic elicitation with numerals.\footnote{Classifiers are bound forms and cannot be elicited in isolation (i.e.\ without an accompanying numeral or demonstrative). Numerals, too, do not usually appear on their own, but it is possible, at a~push, to elicit them in isolation. The forms from one to ten are: /\ipa{ɖɯ˩˥}/ ‘1’, /\ipa{ɲi˩˥}/ ‘2’, /\ipa{so˧}/ ‘3’, /\ipa{ʐv̩˧}/ ‘4’, /\ipa{ŋwɤ˧}/ ‘5’, /\ipa{qʰv̩˧˥}/ ‘6’, /\ipa{ʂɯ˧}/ ‘7’, /\ipa{hõ˧˥}/ ‘8’, /\ipa{gv̩˧}/ ‘9’, and /\ipa{tsʰe˧}/ ‘10’.} The first dataset, elicited in 2009, covered numerals from 1 to 30. However, the results were less than fully consistent, partly because the consultant was unaccustomed to lengthy
counting tasks. The same noun-plus-classifier combination was sometimes realized with different tone patterns from one elicitation session to another. When such discrepancies were
pointed out, the consultant (Mrs.\ Latami) would identify one pattern as correct and dismiss the others as
mistaken. Yet these \textit{a posteriori} judgments also wavered: a~\is{variants}variant that had been
brushed aside as \is{mistakes}erroneous might come up again in a~later session, with the consultant now insisting on its correctness.

At first, I wrongly assumed that only one tone pattern could be
correct. Over time, however, it became clear that some of the phrases allow two variants. For
instance, the combination of ‘9’ with the classifier for threads can be realized as
either /\ipa{gv̩˧-kʰɯ˥}/ or /\ipa{gv̩˧-kʰɯ˩}/.
%(To preview the result of the analysis proposed in
%\sectref{sec:aboutvariantsoftonepatterns}: in the first \is{variants}variant, the phrase is parsed
%into two tone groups, whereas in the second, it is parsed as one \isi{tone group}.)
Taking into
account both the occasional occurrence of elicitation errors and the existence of genuine variants, a~comprehensive description could finally be reached, dissipating earlier perplexity and frustration.

The data collected in 2009, covering numerals from 1 to 30, reveals that some classifiers have tantalizingly similar yet not identical behaviour. For instance, the classifier for days and the classifier for steps (as in steps of a staircase) follow the same tone patterns for numerals up to thirteen but differ for 14, 15, 16, 18, 19, and 22, as shown in \tabref{tab:stepsanddays}. Thus, ‘fourteen steps' is //\ipa{tsʰe˩ʐv̩˩-ɖwæ˥\$}// (tone: L+H\$; surface phonological form: /\ipa{tsʰe˩ʐv̩˩-ɖwæ˥}/), whereas ‘fourteen days' is //\ipa{tsʰe˩ʐv̩˩-ɲi˩}// (tone: L; surface phonological form: /\ipa{tsʰe˩ʐv̩˩-ɲi˩˥}/).

\begin{table}[p]
	\caption{Surface phonological form of the classifiers for days and for steps, in association with numerals from 1 to 30.}
	\begin{tabularx}{.85\textwidth}{ l Q Q l}
		\lsptoprule
		\textsc{num} & \textsc{clf}.days & \textsc{clf}.steps & comparison\\ \midrule
		1 & \ipa{ɖɯ˧-ɲi˥} & \ipa{ɖɯ˧-kʰwɤ˥} & same\\
		2 & \ipa{ɲi˧-ɲi˥} & \ipa{ɲi˧-kʰwɤ˥} & same\\
		3 & \ipa{so˩-ɲi˩˥} & \ipa{so˩-kʰwɤ˩˥} & same\\
		4 & \ipa{ʐv̩˧-ɲi˧} & \ipa{ʐv̩˧-kʰwɤ˧} & same\\
		5 & \ipa{ŋwɤ˧-ɲi˧} & \ipa{ŋwɤ˧-kʰwɤ˧} & same\\
		6 & \ipa{qʰv̩˧-ɲi˥} & \ipa{qʰv̩˧-kʰwɤ˥} & same\\
		7 & \ipa{ʂɯ˧-ɲi˧} & \ipa{ʂɯ˧-kʰwɤ˧} & same\\
		8 & \ipa{hõ˧-ɲi˥} & \ipa{hõ˧-kʰwɤ˥} & same\\
		9 & \ipa{gv̩˧-ɲi˧} & \ipa{gv̩˧-kʰwɤ˧} & same\\
		10 & \ipa{tsʰe˩-ɲi˩˥} & \ipa{tsʰe˩-kʰwɤ˩˥} & same\\
		11 & \ipa{tsʰe˩ɖɯ˩-ɲi˩˥} & \ipa{tsʰe˩ɖɯ˩-kʰwɤ˩˥} & same\\
		12 & \ipa{tsʰe˩ɲi˩-ɲi˩˥} & \ipa{tsʰe˩ɲi˩-kʰwɤ˩˥} & same\\
		13 & \ipa{tsʰe˩so˩-ɲi˩˥} & \ipa{tsʰe˩so˩-kʰwɤ˩˥} & same\\
		14 & \ipa{tsʰe˩ʐv̩˩-ɲi˩˥} & \ipa{tsʰe˩ʐv̩˩-kʰwɤ˥} & \textit{different}\\
		15 & \ipa{tsʰe˩ŋwɤ˩-ɲi˩˥} & \ipa{tsʰe˩ŋwɤ˩-kʰwɤ˥} & \textit{different}\\
		16 & \ipa{tsʰe˩qʰv̩˩-ɲi˩˥} & \ipa{tsʰe˩qʰv̩˩-kʰwɤ˥} & \textit{different}\\
		17 & \ipa{tsʰe˩ʂɯ˩-ɲi˩˥} & \ipa{tsʰe˩ʂɯ˩-kʰwɤ˩˥} & same\\
		18 & \ipa{tsʰe˩hõ˩-ɲi˩˥} & \ipa{tsʰe˩hõ˩-kʰwɤ˥} & \textit{different}\\
		19 & \ipa{tsʰe˩gv̩˩-ɲi˩˥} & \ipa{tsʰe˩gv̩˩-kʰwɤ˥} & \textit{different}\\
		20 & \ipa{ɲi˧tsi˧-ɲi˧} & \ipa{ɲi˧tsi˧-kʰwɤ˧} & same\\
		21 & \ipa{ɲi˧tsi˧ɖɯ˧-ɲi˥} & \ipa{ɲi˧tsi˧ɖɯ˧-kʰwɤ˥} & same\\
		22 & \ipa{ɲi˧tsi˧ɲi˧-ɲi˧} & \ipa{ɲi˧tsi˧ɲi˧-kʰwɤ˥} & \textit{different}\\
		23 & \ipa{ɲi˧tsi˧so˩-ɲi˩˥} & \ipa{ɲi˧tsi˧so˩-kʰwɤ˩˥} & same\\
		24 & \ipa{ɲi˧tsi˧ʐv̩˧-ɲi˧} & \ipa{ɲi˧tsi˧ʐv̩˧-kʰwɤ˧} & same\\
		25 & \ipa{ɲi˧tsi˧ŋwɤ˧-ɲi˧} & \ipa{ɲi˧tsi˧ŋwɤ˧-kʰwɤ˧} & same\\
		26 & \ipa{ɲi˧tsi˧qʰv̩˧-ɲi˥} & \ipa{ɲi˧tsi˧qʰv̩˧-kʰwɤ˥} & same\\
		27 & \ipa{ɲi˧tsi˧ʂɯ˧-ɲi˧} & \ipa{ɲi˧tsi˧ʂɯ˧-kʰwɤ˧} & same\\
		28 & \ipa{ɲi˧tsi˧hõ˧-ɲi˥} & \ipa{ɲi˧tsi˧hõ˧-kʰwɤ˥} & same\\
		29 & \ipa{ɲi˧tsi˧gv̩˧-ɲi˧} & \ipa{ɲi˧tsi˧gv̩˧-kʰwɤ˧} & same\\
		30 & \ipa{so˧tsʰi˧-ɲi˧} & \ipa{so˧tsʰi˧-kʰwɤ˧} & same\\
		\lspbottomrule
	\end{tabularx}
	\label{tab:stepsanddays}
\end{table}

This finding led to the decision to include the full range of numerals from 1 to 100 in a~second set of recordings. In total, 2,810 numeral-plus-classifier phrases were recorded. Within this dataset, the ranges
[50..59] and [80..89] were omitted, as previous elicitation had shown that their tone
patterns were identical to those of [40..49] and [60..69], respectively. Shortening the list
of numerals in this manner helped mitigate consultant fatigue.

The entire set of transcribed recordings is available online, with both a~surface phonological
transcription and an indication of the underlying tonal pattern.

No amount of continuous speech would be sufficient to document all the nu\-mer\-al-plus-clas\-si\-fi\-er
combinations from 1 to 100, hence the choice to resort to systematic elicitation. However, some of
these combinations are also attested in the transcribed narratives.

%\newpage
\subsection{Results: Nine tonal categories for monosyllabic classifiers}
\label{sec:results}
%\subsubsection{How the tonal categories were brought out and labelled}
\label{sec:howthetonalcategorieswerebroughtoutandlabelled}

The nine classifier tone categories (H\textsubscript{a}, H\textsubscript{b}, MH\textsubscript{a}, MH\textsubscript{b}, M\textsubscript{a}, M\textsubscript{b}, L\textsubscript{a}, L\textsubscript{b}, and L\textsubscript{c}) are illustrated in \tabref{tab:oneexampleofeachofthetonalcategoriesofclassifiers}.

\begin{table}[H]
	\caption{One example of each of the nine tonal categories of {monosyllabic} classifiers.}
	\begin{tabularx}{\textwidth}{ Q Q l }
		\lsptoprule
		classifier & tone & description: classifier for{\dots}\\ \midrule
		\ipa{ɖwæ˥\textsubscript{a}} & H\textsubscript{a} & steps (of stairs)\\
		\ipa{ɲi˥\textsubscript{b}} & H\textsubscript{b} & days\\
		\ipa{hɑ̃˧˥\textsubscript{a}} & MH\textsubscript{a} & nights\\
		\ipa{kv̩˧˥\textsubscript{b}} & MH\textsubscript{b} & people, persons\\
		\ipa{nɑ˧\textsubscript{a}} & M\textsubscript{a} & tools\\
		\ipa{dzi˧\textsubscript{b}} & M\textsubscript{b} & pairs of non-separable objects, e.g.~shoes\\
		\ipa{dze˩\textsubscript{a}} & L\textsubscript{a} & pairs of separable objects, e.g.~pots, bottles\\
		\ipa{dzi˩\textsubscript{b}} & L\textsubscript{b} & trees, bamboos\\
		\ipa{ʐɤ˩\textsubscript{c}} & L\textsubscript{c} & lines, patterns (in weaving or drawing)\\
		\lspbottomrule
	\end{tabularx}
	\label{tab:oneexampleofeachofthetonalcategoriesofclassifiers}
\end{table}

These nine tonal categories were brought out by distributional analysis, grouping together {monosyllabic} classifiers that have the same tonal behaviour in combination with numerals.
%In turn, these categories were
%grouped into four sets on the basis of their similarities. For instance, the tone categories H\textsubscript{a} and H\textsubscript{b} (illustrated by the classifier for days and the classifier for steps of stairs, respectively) are identical
%except for fourteen of the numerals: {[14..16], [18..19], 22, 32, 42, 52, 62, 72, 82, 92,
%  99}.
The choice of labels for these categories was guided by structural considerations. The tone patterns in association with the numerals 6 and 8
%‘6’ and ‘8’
are not highly informative, as nearly all tonal oppositions are
neutralized in this context (yielding only H\# and H\$). Likewise, numeral-plus-classifier phrases
involving 3 and 10 only exhibit two distinct patterns. The numerals 4 and 5 allow for the identification of four tonal groups, but if these patterns were taken as indicative of the classifiers’ lexical tones, the system
would comprise only two High tones (\#H and H\#) and two Low tones (L\# and L), with no Mid
tones or contour tones. This would be strikingly different from the lexical tone system of the other {monosyllabic}
nouns found in Yongning Na, which includes H, M, L, and two types of rising contours, analyzed as
MH and LM.

On the other hand, the tone patterns associated with the numerals 1 and 2 provide meaningful labels for tonal categories of classifiers. These four patterns are H, MH, M, and L, all of which also occur as lexical
tones for nouns. They are therefore adopted as category labels, with subscript letters added to distinguish \is{subcategories of lexical tones}subcategories (two for H, MH, and M, and three for L), ordered by decreasing frequency (e.g.~category M\textsubscript{a} contains more classifiers than M\textsubscript{b}). The same \is{subcategories of lexical tones}approach is used for verbs, where two \is{subcategories of lexical tones}subcategories of L tones (L\textsubscript{a} and L\textsubscript{b}) and three
\is{subcategories of lexical tones}subcategories of M tones (M\textsubscript{a}, M\textsubscript{b}, and M\textsubscript{c}) are recognized (see Chapter~\ref{chap:verbsandtheircombinatoryproperties}).

However, tonal \is{subcategories of lexical tones}subcategories are
established separately for nouns and verbs, with no basis for identifying them across parts of speech. For instance, the label ‘L\textsubscript{a}’ used for verbs does not refer
to the same category as ‘L\textsubscript{a}’ for classifiers. To preclude potential misunderstandings, one could use distinct arrays of subscript letters, such as Greek letters for classifier \is{subcategories of lexical tones}subcategories and Latin letters for verb \is{subcategories of lexical tones}subcategories. But it seemed advisable to use Latin letters throughout to avoid an extravagant profusion of symbols.

A comprehensive summary of the tone patterns these nine categories yield in combination with numerals from 1 to 100 is presented in
Tables~\ref{tab:1to25hmh} to \ref{tab:76to100ml}. These tables encapsulate, in tightly packed form, the
information required to generate the surface phonological tone patterns of all
numeral-plus-classifier phrases in the Alawua dialect of Yongning Na.

The mass of information presented in Tables~\ref{tab:1to25hmh} to \ref{tab:76to100ml} may
appear overwhelming at first glance. Were it not for the clear evidence from recorded data, one might suspect that this
multiplicity of tone patterns was an artefact of the elicitation procedures.

Variants are separated by a~slash (/). For typographical clarity, the information is split between two sets of tables: the first set (Tables~\ref{tab:1to25hmh} to \ref{tab:76to100hmh}) presents the categories with H and MH tones (H\textsubscript{a}, H\textsubscript{b}, MH\textsubscript{a}, and MH\textsubscript{b}), while the second (Tables~\ref{tab:1to25ml} to \ref{tab:76to100ml}) covers the categories with M and L tones (M\textsubscript{a}, M\textsubscript{b},
L\textsubscript{a}, L\textsubscript{b}, and L\textsubscript{c}).

Numeral-plus-classifier phrases typically form a single \isi{tone group}.\footnote{The \textit{tone group} is the unit within which tonal processes apply in Yongning Na. It could also be referred to as \textit{phonological phrase}. Chapter~\ref{chap:toneassignmentrulesandthedivisionoftheutteranceintotonegroups} is devoted to this morphotonological unit, which is fundamental to Na {prosody}.} However, speakers can choose to split them into two groups for expressive (\is{emphasis}emphatic) purposes, as will be
discussed in Chapter~\ref{chap:toneassignmentrulesandthedivisionoftheutteranceintotonegroups}. The \is{juncture (inside a tone group)}juncture indicated by ‘--’ does
not mark a division into separate tone groups but rather separates the phrase into two structural parts: one for the tens and another for the units and classifier. This \is{juncture (inside a tone group)}juncture appears after the first two syllables in phrases containing numerals above twenty. Such phrases consist of two syllables referring to the tens (‘two-ten’ for twenty, ‘three-ten’ for thirty, etc.), followed by the last digit and the classifier~-- except in round numbers (twenty, thirty, etc.), where no zero is indicated and the classifier follows directly, as in /\ipa{ʐv̩˧tsʰi˩-kʰv̩˩}/ ‘forty years’. For instance, /\ipa{so˧tsʰi˧so˩-ɲi˩}/ ‘33 days’ can be represented as /\ipa{so˧tsʰi˧--so˩-ɲi˩}/ to show the \is{juncture (inside a tone group)}juncture between the two parts (glosses for the four syllables: ‘three~-ten~--~three~-~\textsc{clf}.days’).

% Checked on proofs: how to get all the figures exactly HERE.

%\clearpage
%\FloatBarrier
%\afterpage{

		% \label{tab:1to100hmh}  %% Commented out on April 30th, 2025: no subtables, only sequentially numbered tables in the entire volume.
		\begin{table}[p!]% rather than "H" which places the table at top of page
			\caption{\label{tab:1to25hmh}The underlying tone patterns of the nine categories of numeral-plus-classifier phrases. H and MH tones. Numerals from 1 to 25.}
			\begin{tabularx}{\textwidth}{ P{10mm} Q Q Q Q }
			\lsptoprule
				 & H\textsubscript{a} & H\textsubscript{b} & MH\textsubscript{a} & MH\textsubscript{b}\\\midrule
				1 & H\$ & H\$ & MH\# & MH\#\\
				2 & H\$ & H\$ & MH\# & MH\#\\
				3 & L & L & L & L\\
				4 & \#H & \#H & L\# & L\\
				5 & \#H & \#H & L\# & L\\
				6 & H\$ & H\$ & H\# & H\$\\
				7 & \#H & \#H & MH\# & MH\#\\
				8 & H\$ & H\$ & H\# & H\$\\
				9 & \#H & \#H & L\# & L\\
				10 & L & L & L & L\\
				11 & L & L & L & L\\
				12 & L & L & L & L\\
				13 & L & L & L & L\\
				14 & L+H\$ & L & L+H\# & L+H\#\\
				15 & L+H\$ & L & L+H\# & L+H\#\\
				16 & L+H\$ & L & L+H\# & L+H\#\\
				17 & L & L & L & L\\
				18 & L+H\$ & L & L+H\# & L+H\#\\
				19 & L+H\$ & L & L+H\# & L+H\#\\
				20 & \#H & \#H & MH\# & MH\#\\
				21 & H\$ & H\$ & MH\# & MH\#\\
				22 & H\$ & \#H & MH\# & MH\#\\
				23 & --L & --L & --L & --L\\
				24 & \#H & \#H & --L\# & --L\\
				25 & \#H & \#H & --L\# & --L\\
			\lspbottomrule
			\end{tabularx}
		\end{table}

		\begin{table}[p!]% rather than "H" which places the table at top of page
			\caption{\label{tab:26to50hmh}The underlying tone patterns of the nine categories of numeral-plus-classifier phrases. H and MH tones. Numerals from 26 to 50.}
			\begin{tabularx}{\textwidth}{ P{10mm} Q Q Q Q }
			\lsptoprule
				 & H\textsubscript{a} & H\textsubscript{b} & MH\textsubscript{a} & MH\textsubscript{b}\\\midrule
				26 & H\$ & H\$ & H\# & H\$\\
				27 & \#H & \#H & MH\# & MH\#\\
				28 & H\$ & H\$ & H\# & H\$\\
				29 & \#H & \#H & --L\# & --L\\
				30 & \#H & \#H & MH\# & MH\#\\
				31 & H\$ & H\$ & MH\# & MH\#\\
				32 & H\$ & \#H & MH\# & MH\#\\
				33 & --L & --L & --L & --L\\
				34 & \#H & \#H & --L\# & --L\\
				35 & \#H & \#H & --L\# & --L\\
				36 & H\$ & H\$ & H\# & H\$\\
				37 & \#H & \#H & MH\# & MH\#\\
				38 & H\$ & H\$ & H\# & H\$\\
				39 & \#H/--L & \#H & --L\# & --L\\
				40 & L\#-- & L\#-- & L\#-- & L\#--\\
				41 & L\#--H\$ / L\#-- & L\#--H\$ / L\#-- & L\#--MH\# / L\#-- & L\#--MH\#\\
				42 & L\#--H\$ / L\#-- & L\#--\#H / L\#-- & L\#--MH\# / L\#-- & L\#--MH\#\\
				43 & L\#--L & L\#--L / L\#-- & L\#--L & L\#--L\\
				44 & L\#--\#H / L\#-- & L\#--\#H / L\#-- & L\#--L\# & L\#--L\\
				45 & L\#--\#H / L\#-- & L\#--\#H / L\#-- & L\#--L\# & L\#--L\\
				46 & L\#--H\$ & L\#--H\$ & L\#--H\# & L\#--H\$\\
				47 & L\#--\#H & L\#--\#H & L\#--MH\# & L\#--MH\#\\
				48 & L\#--H\$ & L\#--H\$ & L\#--H\# & L\#--H\$\\
				49 & L\#--\#H / L\#--L & L\#--\#H / L\#-- & L\#--L\# & L\#--L\\
				50 & L\#-- & L\#-- & L\#-- & L\#--\\
			\lspbottomrule
			\end{tabularx}
		\end{table}

		\begin{table}[p!]% rather than "H" which places the table at top of page
		\caption{\label{tab:51to75hmh}The underlying tone patterns of the nine categories of numeral-plus-classifier phrases. H and MH tones. Numerals from 51 to 75.}
		\begin{tabularx}{\textwidth}{ P{10mm} l Q Q Q }
		\lsptoprule
			 & H\textsubscript{a} & H\textsubscript{b} & MH\textsubscript{a} & MH\textsubscript{b}\\\midrule
			51 & L\#--H\$ / L\#-- & L\#--H\$ / L\#-- & L\#--MH / L\#-- & L\#--MH\#\\
			52 & L\#--H\$ / L\#-- & L\#--\#H / L\#-- & L\#--MH / L\#-- & L\#--MH\#\\
			53 & L\#-- & L\#-- & L\#--L & L\#--L\\
			54 & L\#--\#H / L\#-- & L\#--\#H / L\#-- & L\#--L\# & L\#--L\\
			55 & L\#--\#H / L\#-- & L\#--\#H / L\#-- & L\#--L\# & L\#--L\\
			56 & L\#--H\$ & L\#--H\$ & L\#--H\# & L\#--H\$\\
			57 & L\#--\#H & L\#--\#H & L\#--MH\# & L\#--MH\#\\
			58 & L\#--H\$ & L\#--H\$ & L\#--H\# & L\#--H\$\\
			59 & L\#--\#H / L\#-- & L\#--\#H / L\#-- & L\#--L\# & L\#--L\\
			60 & LM--H\$ & LM--H\$ & LM--H\# & LM--\#H\\
			61 & LM--H\$ & LM--H\$ & LM--H\# & LM--\#H\\
			62 & LM--H\$ / LM--\#H & LM--\#H & LM--H\# & LM--H\#\\
			63 & LM--H\$ / LM--L & LM--H\$ / LM--L & LM--H\# & LM--H\#\\
			64 & LM--\#H & LM--\#H & LM--H\# & LM--H\#\\
			65 & LM--\#H & LM--\#H & LM--H\# & LM--H\#\\
			66 & LM--H\$ & LM--H\$ & LM--H\# & LM--H\#\\
			67 & LM--\#H & LM--\#H & LM--H\# & LM--H\#\\
			68 & LM--H\$ & LM--H\$ & LM--H\# & LM--H\#\\
			69 & LM--\#H / LM--L & LM--L & LM--H\# & LM--H\#\\
			70 & L\#-- & L\#-- & L\#-- & L\#--\\
			71 & L\#--H\$ / L\#-- & L\#--H\$ & L\#--MH\# / L\#-- & L\#--MH\#\\
			72 & L\#--H\$ / L\#-- & L\#--\#H & L\#--MH\# / L\#-- & L\#--MH\#\\
			73 & L\#--L & L\#--L & L\#--L & L\#--L\\
			74 & L\#--\#H & L\#--\#H & L\#--MH\# / L\#-- & L\#--L\\
			75 & L\#--\#H & L\#--\#H & L\#--MH\# / L\#-- & L\#--L\\
		\lspbottomrule
		\end{tabularx}
		\end{table}

		\begin{table}[p!]% rather than "H" which places the table at top of page
		\caption{\label{tab:76to100hmh}The underlying tone patterns of the nine categories of numeral-plus-classifier phrases. H and MH tones. Numerals from 76 to 100.}
		\begin{tabularx}{\textwidth}{ P{10mm} Q Q Q Q }
		\lsptoprule
			 & H\textsubscript{a} & H\textsubscript{b} & MH\textsubscript{a} & MH\textsubscript{b}\\\midrule
			76 & L\#--H\$ & L\#--H\$ & L\#--H\# / L\#-- & L\#--H\#\\
			77 & L\#--\#H & L\#--\#H & L\#--MH\# / L\#-- & L\#--MH\#\\
			78 & L\#--H\$ & L\#--H\$ & L\#--H\# / L\#-- & L\#--H\#\\
			79 & L\#--\#H / L\#--L & L\#--\#H & L\#--L\# & L\#--L\\
			80 & LM--H\$ & LM--H\$ & LM--H\# & LM--H\#\\
			81 & LM--H\$ & LM--H\$ & LM--H\# & LM--H\#\\
			82 & LM--H\$ & LM--\#H & LM--H\# & LM--H\#\\
			83 & LM--H\$ & LM--H\$ / LM--L & LM--H\# & LM--H\#\\
			84 & LM--\#H & LM--H\$ & LM--H\# & LM--H\#\\
			85 & LM--\#H & LM--\#H & LM--H\# & LM--H\#\\
			86 & LM--H\$ & LM--H\$ & LM--H\# & LM--H\#\\
			87 & LM--\#H & LM--\#H & LM--H\# & LM--H\#\\
			88 & LM--H\$ & LM--H\$ & LM--H\# & LM--H\#\\
			89 & LM--\#H & LM--\#H & LM--H\# & LM--H\#\\
			90 & L\#-- & L\#-- & L\#-- & L\#--\\
			91 & L\#--H\$ & L\#--H\$ & L\#--MH\# / L\#-- & L\#--MH\# / L\#--\\
			92 & L\#--H\$ & L\#--\#H & L\#--MH\# / L\#-- & L\#--MH\# / L\#--\\
			93 & L\#--L & L\#--L & L\#--L & L\#--L\\
			94 & L\#--\#H & L\#--\#H & L\#--L\# / L\#-- & L\#--L\\
			95 & L\#--\#H & L\#--\#H & L\#--L\# / L\#-- & L\#--L\\
			96 & L\#--H\$ & L\#--H\$ & L\#--H\# / L\#-- & L\#--H\# / L\#--\\
			97 & L\#--\#H & L\#--\#H & L\#--MH\# / L\#-- & L\#--MH\# / L\#--\\
			98 & L\#--H\$ & L\#--H\$ & L\#--H\# / L\#-- & L\#--H\# / L\#--\\
			99 & L\#--L & L\#--\#H & L\#--L\# / L\#-- & L\#--L\\
			100 & \#H & \#H & MH\# & MH\#\\
		\lspbottomrule
		\end{tabularx}
		\end{table}




		% \label{tab:1to100ml}  %% Commented out on April 30th, 2025: no subtables, only sequentially numbered tables in the entire volume.
		\begin{table}[p!]% rather than "H" which places the table at top of page
			\caption{\label{tab:1to25ml}The underlying tone patterns of the nine categories of numeral-plus-classifier phrases. M and L tones. Numerals from 1 to 25.}
			\begin{tabularx}{\textwidth}{ P{10mm} Q Q Q Q Q }
			\lsptoprule
				 & M\textsubscript{a} & M\textsubscript{b} & L\textsubscript{a} & L\textsubscript{b} & L\textsubscript{c}\\\midrule
				1 & M & M & L\# & L\# & L\#\\
				2 & M & M & L\# & L\# & L\#\\
				3 & M & M & L & M & M\\
				4 & L & L & H\# & H\# & H\#\\
				5 & L & L & H\# & H\# & H\#\\
				6 & H\# & H\$ & H\# & H\# & H\#\\
				7 & M & M & L\# & L\# & L\#\\
				8 & H\# & H\$ & H\# & H\# & H\#\\
				9 & L & L & H\# & H\# / L\# & H\#\\
				10 & M & M & L & M & L\\
				11 & M & M & L\# & L\# & L\#\\
				12 & M & M & L\# & L\# & L\#\\
				13 & M & M & L\# & L\# & L\#\\
				14 & L+H\# & L & L+H\# & L+H\# & L+H\#\\
				15 & L+H\# & L & L+H\# & L+H\# & L+H\#\\
				16 & L+H\# & L & L+H\# & L+H\# & L+H\#\\
				17 & M & M & L\# & L\# & L\#\\
				18 & L+H\# & L & L+H\# & L+H\# & L+H\#\\
				19 & L+H\# & L & L+H\# & L+H\# & L+H\#\\
				20 & M & M & --L & --L & --L\\
				21 & M & M & --L\# & --L\# & --L\#\\
				22 & M & M & --L\# & --L\# & --L\#\\
				23 & --L / M & M / --L & --L & --L & --L\\
				24 & --L & --L & H\# & H\# & H\#\\
				25 & --L & --L & H\# & H\# & H\#\\
			\lspbottomrule
			\end{tabularx}
		\end{table}

		\begin{table}[p!]% rather than "H" which places the table at top of page
			\caption{\label{tab:26to50ml}The underlying tone patterns of the nine categories of numeral-plus-classifier phrases. M and L tones. Numerals from 26 to 50.}
			{\fontsize{10}{11}\selectfont
				\begin{tabularx}{\textwidth}{ P{10mm} Q Q Q Q Q }
					\lsptoprule
						 & M\textsubscript{a} & M\textsubscript{b} & L\textsubscript{a} & L\textsubscript{b} & L\textsubscript{c}\\\midrule
						26 & H\# & H\$ & H\# & H\# & H\#\\
						27 & M & M & --L\# & --L\# & --L\#\\
						28 & H\# & H\$ & H\# & H\# & H\#\\
						29 & --L & --L & H\# & --L\# / H\# & H\#\\
						30 & M & M & --L\# & --L & --L\#\\
						31 & M & M & --L\# & --L\# & --L\#\\
						32 & M & M & --L\# & --L\# & --L\#\\
						33 & --L & --L & --L & --L & --L\\
						34 & --L & --L & H\# & H\# & H\#\\
						35 & --L & --L & H\# & H\# & H\#\\
						36 & H\# & H\$ & H\# & H\# & H\#\\
						37 & M & L & --L\# & --L\# & --L\#\\
						38 & H\# & H\$ & H\# & H\# & H\#\\
						39 & --L & --L & H\# / --L\# & H\# / --L\# & H\# / --L\#\\
						40 & L\#-- & L\#-- & L\#-- & L\#-- & L\#--\\
						41 & L\#-- & L\#--M / L\#-- & L\#--L\# / L\#-- & L\#--L\# / L\#-- & L\#--L\# / L\#--\\
						42 & L\#-- & L\#--M / L\#-- & L\#--L\# / L\#-- & L\#--L\# / L\#-- & L\#--L\# / L\#--\\
						43 & L\#-- & L\#--M / L\#-- & L\#--M / L\#--L & L\#--M / L\#-- & L\#--M / L\#--\\
						44 & L\#--H\# / L\#-- & L\#-- & L\#--H\# / L\#-- & L\#--H\# / L\#-- & L\#--H\# / L\#--\\
						45 & L\#--H\# / L\#-- & L\#-- & L\#--H\# / L\#-- & L\#--H\# / L\#-- & L\#--H\# / L\#--\\
						46 & L\#--H\# / L\#-- & L\#--H\$ / L\#-- & L\#--H\# / L\#-- & L\#--H\# / L\#-- & L\#--H\# / L\#--\\
						47 & L\#--M / L\#-- & L\#--M / L\#-- & L\#--L\# / L\#-- & L\#--L\# / L\#-- & L\#--L\# / L\#--\\
						48 & L\#--H\# / L\#-- & L\#--H\$ / L\#-- & L\#--H\# / L\#-- & L\#--H\# / L\#-- & L\#--H\# / L\#--\\
						49 & L\#--H\# / L\#-- & L\#--L & L\#--H\# / L\#-- & L\#--H\# / L\#-- & L\#--H\# / L\#--\\
						50 & L\#-- & L\#-- & L\#-- & L\#-- & L\#--\\
					\lspbottomrule
				\end{tabularx}
			}
		\end{table}

		\begin{table}[p!]% rather than "H" which places the table at top of page
			\caption{\label{tab:51to75ml}The underlying tone patterns of the nine categories of numeral-plus-classifier phrases. M and L tones. Numerals from 51 to 75.}
			{\fontsize{10}{11}\selectfont
				\begin{tabularx}{\textwidth}{ P{5mm} Q Q Q Q Q }
				\lsptoprule
					 & M\textsubscript{a} & M\textsubscript{b} & L\textsubscript{a} & L\textsubscript{b} & L\textsubscript{c}\\\midrule
					51 & L\#-- & L\#--M / L\#-- & L\#--L\# / L\#-- & L\#--L\# / L\#-- & L\#--L\# / L\#--\\
					52 & L\#-- & L\#--M / L\#-- & L\#--L\# / L\#-- & L\#--L\# / L\#-- & L\#--L\# / L\#--\\
					53 & L\#-- & L\#--M / L\#-- & L\#--M / L\#--L & L\#--M / L\#-- & L\#--M / L\#--\\
					54 & L\#--H\# / L\#-- & L\#-- & L\#--H\# / L\#-- & L\#--H\# / L\#-- & L\#--H\# / L\#--\\
					55 & L\#--H\# / L\#-- & L\#-- & L\#--H\# / L\#-- & L\#--H\# / L\#-- & L\#--H\# / L\#--\\
					56 & L\#--H\# / L\#-- & L\#--H\$ / L\#-- & L\#--H\# / L\#-- & L\#--H\# / L\#-- & L\#--H\# / L\#--\\
					57 & L\#--M / L\#-- & L\#--M / L\#-- & L\#--L\# / L\#-- & L\#--L\# / L\#-- & L\#--L\# / L\#--\\
					58 & L\#--H\# / L\#-- & L\#--H\$ / L\#-- & L\#--H\# / L\#-- & L\#--H\# / L\#-- & L\#--H\# / L\#--\\
					59 & L\#--H\# / L\#-- & L\#--L & L\#--H\# / L\#-- & L\#--H\# / L\#-- & L\#--H\# / L\#--\\
					60 & LM--H\# & LM--H\$ & LM--H\# & LM--H\# & LM--H\#\\
					61 & LM--H\# / L+MH\#--M & LM--H\$ & LM--H\# & LM--H\# & LM--H\#\\
					62 & LM--H\# / L+MH\#--M & LM--H\$ & LM--H\# & LM--H\# & LM--H\#\\
					63 & LM--H\# / L+MH\#--M & LM--H\$ & LM--H\# & LM--H\# & LM--H\#\\
					64 & LM--H\# / L+MH\#--L & LM--H\$ & LM--H\# & LM--H\# & LM--H\#\\
					65 & LM--H\# / L+MH\#--L & LM--H\$ & LM--H\# & LM--H\# & LM--H\#\\
					66 & LM--H\# & LM--H\$ & LM--H\# & LM--H\# & LM--H\#\\
					67 & LM--H\# / L+MH\#--M & LM--\#H & LM--H\# & LM--H\# & LM--H\#\\
					68 & LM--H\# & LM--H\$ & LM--H\# & LM--H\# & LM--H\#\\
					69 & LM--H\# / L+MH\#--L & LM--H\$ / LM--L & LM--H\# & LM--H\# & LM--H\#\\
					70 & L\#-- & L\#-- & L\#-- & L\#-- & L\#--\\
					71 & L\#-- / L\#--M & L\#--M / L\#-- & L\#--L\# / L\#-- & L\#--L\# / L\#-- & L\#--L\# / L\#--\\
					72 & L\#-- / L\#--M & L\#--M / L\#-- & L\#--L\# / L\#-- & L\#--L\# / L\#-- & L\#--L\# / L\#--\\
					73 & L\#-- / L\#--M & L\#--M / L\#-- & L\#--L & L\#--M / L\#-- & L\#--M / L\#--\\
					74 & L\#-- / L\#--H\# & L\#--L & L\#--H\# / L\#-- & L\#--H\# / L\#-- & L\#--H\# / L\#--\\
					75 & L\#-- / L\#--H\# & L\#--L & L\#--H\# / L\#-- & L\#--H\# / L\#-- & L\#--H\# / L\#--\\
				\lspbottomrule
				\end{tabularx}
			}
		\end{table}

		\begin{table}[p!]% rather than "H" which places the table at top of page
			\caption{\label{tab:76to100ml}The underlying tone patterns of the nine categories of numeral-plus-classifier phrases. M and L tones. Numerals from 75 to 100.}
			{\setlength\tabcolsep{4.5pt}
			{\fontsize{10}{11}\selectfont
				\begin{tabularx}{\textwidth}{ l P{22mm} Q Q Q Q }
				\lsptoprule
					 & M\textsubscript{a} & M\textsubscript{b} & L\textsubscript{a} & L\textsubscript{b} & L\textsubscript{c}\\\midrule
					76 & L\#--H\# / L\#-- & L\#--H\$ / L\#-- & L\#--H\# / L\#-- & L\#--H\# / L\#-- & L\#--H\# / L\#--\\
					77 & L\#--M / L\#-- & L\#--M / L\#-- & L\#--L\# / L\#-- & L\#--L\# / L\#-- & L\#--L\# / L\#--\\
					78 & L\#--H\# / L\#-- & L\#--H\$ / L\#-- & L\#--H\# / L\#-- & L\#--H\# / L\#-- & L\#--H\# / L\#--\\
					79 & L\#--L / L\#--H\# & L\#--L & L\#--H\# / L\#-- & L\#--H\# / L\#-- & L\#--H\# / L\#--\\
					80 & LM--H\# & LM--H\$ & LM--H\# & LM--H\# & LM--H\#\\
					81 & LM--H\# / L+MH\#--M & LM--H\$ & LM--H\# & LM--H\# & LM--H\#\\
					82 & LM--H\# / L+MH\#--M & LM--H\$ & LM--H\# & LM--H\# & LM--H\#\\
					83 & LM--H\# / L+MH\#--M & LM--H\$ & LM--H\# & LM--H\# & LM--H\#\\
					84 & LM--H\# / L+MH\#--L & LM--H\$ & LM--H\# & LM--H\# & LM--H\#\\
					85 & LM--H\# / L+MH\#--L & LM--H\$ & LM--H\# & LM--H\# & LM--H\#\\
					86 & LM--H\# & LM--H\$ & LM--H\# & LM--H\# & LM--H\#\\
					87 & LM--H\# / L+MH\#--M & LM--\#H & LM--H\# & LM--H\# & LM--H\#\\
					88 & LM--H\# & LM--H\$ & LM--H\# & LM--H\# & LM--H\#\\
					89 & LM--H\# / L+MH\#--L & LM--H\$ / LM--L & LM--H\# & LM--H\# & LM--H\#\\
					90 & L\#-- / L\#-- & L\#-- & L\#-- & L\#-- & L\#--\\
					91 & L\#--M / L\#-- & L\#--M / L\#-- & L\#--L\# / L\#-- & L\#--L\# / L\#-- & L\#--L\#\\
					92 & L\#--M / L\#-- & L\#--M / L\#-- & L\#--L\# / L\#-- & L\#--L\# / L\#-- & L\#--L\#\\
					93 & L\#--M / L\#-- & L\#--M / L\#-- & L\#--M / L\#-- & L\#--M / L\#-- & L\#--M / L\#--\\
					94 & L\#--L / L\#--H\# & L\#-- & L\#--H\# / L\#-- & L\#--H\# / L\#-- & L\#--H\#\\
					95 & L\#--L / L\#--H\# & L\#-- & L\#--H\# / L\#-- & L\#--H\# / L\#-- & L\#--H\#\\
					96 & L\#--H\# / L\#-- & L\#--H\$ / L\#-- & L\#--H\# & L\#--H\# / L\#-- & L\#--H\#\\
					97 & L\#--M / L\#-- & L\#--M / L\#-- & L\#--L\# / L\#-- & L\#--L\# / L\#-- & L\#--L\#\\
					98 & L\#--H\# / L\#-- & L\#--H\$ / L\#-- & L\#--H\# & L\#--H\# / L\#-- & L\#--H\#\\
					99 & L\#--L / L\#--H\# & L\#--L & L\#--H\# & L\#--H\# / L\#-- & L\#--H\# / L\#\\
					100 & M & M & L\# & L\# & L\#\\
				\lspbottomrule
				\end{tabularx}}
			}
		\end{table}

	\clearpage
% } % for 'afterpage' command

Entries in Tables~\ref{tab:1to25hmh} to \ref{tab:76to100ml} that begin with ‘--’ do not have any specified tone on their first part; this
part receives a~Mid tone, by default. Using again ‘33 days’ as an example, its tonal pattern is --L, meaning that a~Low tone associates after the \is{juncture (inside a tone group)}juncture, yielding /{\dots}\ipa{so˩-ɲi˩}/. Since Low tones do not spread
regressively (‘right-to-left’), the first part of the phrase receives a default M tone (/\ipa{so˧tsʰi˧}{\dots}/),
resulting in the final output /\ipa{so˧tsʰi˧so˩-ɲi˩}/.

Similarly, entries whose tone pattern ends with ‘--’ lack a specified tone on their second part, which receives its tones through the phonological tone rules governing tone groups in
the Alawua dialect of Yongning Na. For example, ‘forty years’ has the tone pattern L\#--: a final L tone associates to the first half of the phrase, yielding /\ipa{ʐv̩.tsʰi˩}{\dots}/. As previously mentioned, tones do not spread leftward; the first syllable
thus receives a~default Mid tone, giving /\ipa{ʐv̩˧tsʰi˩}{\dots}/. At this point, a~phonological rule applies (see \sectref{sec:alistoftonerules}):
Rule~5, which states that “All syllables following an H.L or M.L sequence receive L tone”. This means that
a~tone cannot be flanked by higher tones within a~\isi{tone group}: sequences such as |~MLM~|, |~HMH~|, |~HLM~| or |~MLMH~| do not occur. Consequently, the only possible tone on the second
part of the phrase is L. The final output is /\ipa{ʐv̩˧tsʰi˩-kʰv̩˩}/, as shown in \figref{fig:tone40years}.

\begin{figure}
	\caption[{A detailed representation of tone-to-syllable association for ‘forty years'.}]{A detailed representation of tone-to-syllable association for the numeral-plus-classifier phrase /\ipa{ʐv̩˧tsʰi˩-kʰv̩˩}/ ‘forty years'.}
	\begin{tikzpicture}
	\node (1) at (0.5,-1.5) {L\#};
	\node (4) at (3.5,-1.5) {MH\textsubscript{a}};
%	\node (4) at (3.5,-0.5) {MH\#};

	\node (2) at (0,-2.5) {σ};
	\node (3) at (1,-2.5) {σ};
	\node (5) at (3.5,-2.5) {σ};

	\node [anchor=mid] (s1l) at (0.5,-3) {/\ipa{ʐv̩.tsʰi}/ ‘forty’};
	%  \node (s1ll) at (0.5,-2.5) {lexical tone: MH\#};

	\node [anchor=mid] (s1lll) at (3.5,-3) {/\ipa{kʰv̩}/ ‘year’};
	%	\node [anchor=mid] (s1lll) at (3,-2) {/\ipa{bv̩}/ \textsc{poss}};
	%  \node (s1llll) at (4,-2.5) {lexical tone: L};

%	\node[text width=40mm] (s1) at (-3,-0.75) {Stage 1:\\ input};
	\node[text width=50mm] (s1) at (-3,-1.75) {Stage 1:\\ input};

	\node (12) at (0.5,-4) {L\#};
%	\node (42) at (2,-4) {MH\#};

	\node (22) at (0,-5.5) {σ};
	\node (32) at (1,-5.5) {σ};
	\node (52) at (2,-5.5) {σ};

	\node[text width=50mm] (s2) at (-3,-4.75) {Stage 2:\\ \is{anchorage}anchoring of the phrase's\\ L\#-- tone (see \tabref{tab:26to50hmh})\\ to its
		phonologically\\ specified locus};

	\draw[decoration={markings,mark=at position 1 with
		{\arrow[scale=2,>=stealth]{>}}},postaction={decorate}] (12) -- (32);

	%	\draw[decoration={markings,mark=at position 1 with {\arrow[scale=2,>=stealth]{>}}},postaction={decorate}] (42) -- (52);


%
%	\node (13) at (1,-7) {L};
%	%	\node (63) at (1.5,-7) {H};
%	\node (43) at (2,-7) {MH\#};
%
%	\node (23) at (0,-8.5) {σ};
%	\node (33) at (1,-8.5) {σ};
%	\node (53) at (2,-8.5) {σ};
%
%	\node[text width=40mm] (s3) at (-3,-7.75) {Stage 3:\\ one-to-one mapping\\ of levels to available syllables};
%
%	\draw[decoration={markings,mark=at position 1 with
%		{\arrow[scale=2,>=stealth]{>}}},postaction={decorate}] (13) -- (33);
%	%	\draw[decoration={markings,mark=at position 1 with {\arrow[scale=2,>=stealth]{>}}},postaction={decorate}] (43) -- (53);
%

	\node (14) at (0,-7) {M};
	\node (64) at (1,-7) {L};
%	\node (44) at (2,-7) {MH\#};

	\node (24) at (0,-8.5) {σ};
	\node (34) at (1,-8.5) {σ};
	\node (54) at (2,-8.5) {σ};

	\node[text width=50mm] (s4) at (-3,-7.5) {Stage 3:\\ addition of default\\ M tone};

	\draw[decoration={markings,mark=at position 1 with
		{\arrow[scale=2,>=stealth]{>}}},postaction={decorate}] (14) -- (24);
	\draw (64) -- (34);
	%	\draw (44) -- (54);


	\node (14) at (0,-10) {M};
	\node (64) at (1,-10) {L};
	\node (44) at (2,-10) {L};

	\node (24) at (0,-11.5) {σ};
	\node (34) at (1,-11.5) {σ};
	\node (54) at (2,-11.5) {σ};

	\node[text width=50mm] (s4) at (-3,-10.5) {Stage 4:\\ assignment of L tone\\ by {phonological rule}:\\ M.L can only be followed\\ by L};

	\draw (14) -- (24);
	\draw (64) -- (34);
%	\draw (44) -- (54);

		\draw[decoration={markings,mark=at position 1 with
			{\arrow[scale=2,>=stealth]{>}}},postaction={decorate}] (44) -- (54);

	\end{tikzpicture}
	\label{fig:tone40years}
\end{figure}

\subsection[The tones of classifiers and of corresponding free forms]{Why little evidence about the tones of classifiers can be gleaned from the free forms in which they originate}
\label{sec:nohelpfromFullForms}

In the preceding sections, the tones of classifiers were analyzed on the basis of their synchronic distributional properties. Mention needs to be made of other potential sources of evidence, explaining why they have not, so far, provided decisive insights into the phonological nature of classifiers' tonal categories.

In principle, relevant evidence could come from classifiers that transparently correspond to a~free form: either a~noun or a~verb. For example, the classifier for blows, /\ipa{dɑ˧˥}/, is a~\isi{cognate object} of the verb /\ipa{dɑ˧˥}/ ‘to hit, to strike’, as illustrated in (\ref{ex:strikeablow}).

%\Hack{\newpage}

\begin{exe}
	\ex
	\label{ex:strikeablow}
	\ipaex{ɖɯ˧-dɑ˧ tʰi˥-dɑ˩}\\
	\gll ɖɯ˧	dɑ˧˥	tʰi˧-	dɑ˧˥\\
	one	\textsc{clf}.blows			\textsc{dur}		to\_strike\\
	\glt ‘to strike a blow’ \textit{(Sister3.135)} \pandoi{0004344\#S135}
\end{exe}

The tonal {correspondence} with the verb appears transparent: both the verb and its \is{grammaticalization}grammaticalized form as a~classifier carry MH tone. Similarly, ‘mountain, hill’ is /\ipa{ʁwɤ˧}/, and as a~classifier it yields /\ipa{ʁwɤ˧}/
‘heap (of something)’, which is phonologically identical to the free noun, including its M tone. Likewise, ‘beam’, /\ipa{ɖʐo˥}/, is identical in form to its
\isi{self-classifier}, /\ipa{ɖʐo˥}/.

%\newpage
However, such tonal identity is not found in all examples. A~different pattern for H-tone nouns is exemplified by /\ipa{kɯ˥}/ ‘star’, which yields /\ipa{kɯ˧}/ as a~\isi{self-classifier}, belonging to the M\textsubscript{b} category. Similarly, ‘bowl’ is /\ipa{qʰwɤ˩˧}/ (LM tone) as a noun but /\ipa{qʰwɤ˧˥}/ (MH\textsubscript{a} category) when used as a~classifier meaning ‘bowlful’.

To date, too few classifiers can be straightforwardly linked to full nouns (or verbs) for the search of tonal correspondences between lexical word and classifier to yield robust generalizations. While isolated cases may suggest a tonal relationship between free forms and classifiers, the overall picture remains too inconsistent for this to serve as a reliable means of determining classifier tonal categories.

\subsection{Borrowed classifiers}
\label{sec:nohelpfromLoans}

Classifiers borrowed from \il{Sinitic}Chinese constitute another potential source of evidence for the tonal categories of Yongning Na classifiers. Specifically, it can be assumed that the tone \is{subcategories of lexical tones}subcategories assigned to recent borrowings function synchronically as unmarked, default categories.
%: H\textsubscript{a} or H\textsubscript{b}, MH\textsubscript{a} or MH\textsubscript{b}, and so on.

\tabref{tab:LoanCLF} presents the eight observed \is{loanwords}loanwords: four with L tone, three with MH tone, and one with H tone.

\begin{table}%[t]
	\caption{Classifiers borrowed from Chinese.}
	\begin{tabularx}{\textwidth}{ l l P{19mm} l l P{44.8mm} }
		\lsptoprule
		\multicolumn{3}{l}{borrowed classifier} & \multicolumn{3}{l}{Chinese donor form}\\
		\cmidrule(lr){1-3} \cmidrule(lr){4-6}
		form & tone & \textsc{clf} for: & script & \textit{Pinyin} & note\\\midrule
		\ipa{po˩\textsubscript{b}} & L\textsubscript{b} & pack & \zh{包} & \textit{bāo} & presumably a recent loan\\
		\ipa{tsʰe˩\textsubscript{b}} & L\textsubscript{b} & thumb, inch & \zh{寸} & \textit{cùn} & presumably a recent loan\\
		\addlinespace \hdashline \addlinespace
		\ipa{mæ˩\textsubscript{a}} & L\textsubscript{a} & monetary unit & \zh{元} & \textit{yuán} & Middle Chinese *ngjwon \citep{baxter2000}; Old Chinese *[ŋ]o[r] \citep{baxteretal2014} \\
		\ipa{mæ˩\textsubscript{a}} & L\textsubscript{a} & 10,000 & \zh{萬} & \textit{wàn} & Middle Chinese *mjonH \citep{baxter2000}; Old Chinese *C.ma[n]-s \citep{baxteretal2014} \\
		\addlinespace \hdashline \addlinespace
		\ipa{tɕi˧˥\textsubscript{a}} & MH\textsubscript{a} & pound & \zh{斤} & \textit{jīn} & presumably a recent loan \\
		\ipa{pv̩˧˥\textsubscript{a}} & MH\textsubscript{a} & step, stride & \zh{步} & \textit{bù} & presumably a recent loan \\
		\ipa{te˧˥\textsubscript{a}} & MH\textsubscript{a} & generations & \zh{代} & \textit{dài} & presumably a recent loan \\
		\addlinespace \hdashline \addlinespace
		\ipa{mo˥\textsubscript{a}} & H\textsubscript{a} & acre & \zh{亩} & \textit{mǔ} & presumably a recent loan \\
		\lspbottomrule
	\end{tabularx}
	\label{tab:LoanCLF}
\end{table}

The number of examples is small, and a detailed study of the successive layers of \il{Sinitic}Chinese \isi{loanwords} in Na (following the method applied to \ili{Hani} by \citealt{sagartetal2001}) remains a task for the future. Nonetheless, keeping these major limitations in mind, an interesting pattern emerges: the two subsets of L-tone classifiers in \tabref{tab:LoanCLF}, each containing two items, appear to align with a distinction between recent and earlier loanwords.

The classifiers for ‘pack’ (e.g.\ a pack of cigarettes) and ‘thumb, inch’ (used in its standard sense as a measurement of length, ⅓~dm) are excellent candidates for recent loanwords, and both carry L\textsubscript{b} tone. By contrast, ‘monetary unit, yuan’ and ‘10,000’ are not part of the same layer of loanwords. Their history remains to be worked out, but their nasal initial consonant, in contrast with the glide-initial onset of their Mandarin counterparts, constitutes sufficient evidence that they are not recent loans. Both carry L\textsubscript{a} tone. This pattern suggests that the current default subcategory of L tones for classifiers is L\textsubscript{b}.

The three MH-tone classifiers in \tabref{tab:LoanCLF} are all presumably recent loans and all carry MH\textsubscript{a} tone. Thus, the current default subcategory of MH tones for classifiers appears to be MH\textsubscript{a}.

As for the H-tone classifier /\ipa{mo˥\textsubscript{a}}/, ‘Chinese acre’, corresponding to \zh{亩} \textit{mǔ}, it is presumably a recent loan, but given that it is the sole H-tone item in the dataset, no general conclusions can be drawn.

The way forward here would consist in a~comprehensive study of Chinese loanwords in Na.


\subsection[Indirect confirmation for the H, MH, M, and L categories]{Indirect confirmation for the H, MH, M, and L categories of classifiers from frequency in surface forms}
\label{sec:indirectsupport}

The distributional frequency of the various tonal levels (the three phonological primitives: H, M, and L) provides some degree of support~-- albeit weak and indirect~-- for the four tonal ‘super-categories’ of classifiers proposed here, namely H, MH, M, and L. Under the (admittedly simplistic) assumption that
the contribution of the classifier's lexical tone is statistically reflected in the tonal
patterns of numeral-plus-classifier phrases, the labels H, MH, M, and L make good sense.

Averaging over the entire range of tone patterns from 1 to 100, it emerges that the classifiers that carry a~High tone after 1 and 2 (labelled H\textsubscript{a}
and H\textsubscript{b} in Tables~\ref{tab:1to25hmh} to \ref{tab:76to100ml}) also show the highest proportion of H tones overall~-- simply counting the ‘H’ letters in the relevant columns, regardless of the tonal pattern in which they appear, be it H\#, \#H, H\$, MH or MH\#~-- and the lowest proportion of L tones. Conversely, the three tonal
categories of classifiers that carry a~Low tone after ‘1’ and ‘2’ (analyzed as L\textsubscript{a}, L\textsubscript{b}, and L\textsubscript{c}) exhibit the
lowest proportion of H tones and the highest of L tones (based, here again, on a raw count of ‘H’ and ‘L’ letters in the relevant columns).

The other two classifier categories, M and MH (with subgroups M\textsubscript{a} and M\textsubscript{b}, and MH\textsubscript{a}
and MH\textsubscript{b}), lie in an intermediate position between these two extremes. As expected, M\textsubscript{a} and M\textsubscript{b} show a~higher proportion of M
tones and a~lower proportion of H tones than MH\textsubscript{a} and MH\textsubscript{b}. These broad comparisons, while lacking {demonstrative} value, help convey an overall sense of
the data's structure.

Another indirect approach to this dataset consists in examining occasional
mistakes made by the consultant. Errors in tonal realization may shed light on the underlying categories by revealing which tones are more susceptible to confusion or substitution.


\subsection[Degree of tonal complexity and frequency of mistakes]{Degree of tonal complexity and frequency of mistakes in the recordings}
\label{sec:aboutmistakenrealizationsintherecordings}

\is{complexity}
\is{mistakes|textbf}

As mentioned earlier (\sectref{sec:elicitationprocedures}), the task of producing long series of numeral-plus-classifier phrases
was challenging for the consultant. Among the 2,810 recorded tokens, 7\% have a~mistaken tone
pattern,\footnote{This figure includes some items that were deleted from the sound files at an early
  stage of the study, before the decision was made to preserve the recordings unchanged.} i.e.\ a~tone pattern that Mrs.\ Latami (consultant F4) consistently judged to be incorrect (a tonal slip of the
tongue) when we revisited the data after the recording sessions.

The overall amount of errors is comparable to that obtained in tonal elicitation on different languages: e.g.\ \citet[369]{zerbian_sequences_2023} report discarding 8\% of tokens due to disfluencies or mispronunciations. In detail, these mistakes partly reflect the degree of complexity of the tone patterns in question. The notion
that mistakes can provide insights about language dates back at least to Henri Frei’s
\textit{Grammar of mistakes} (\citeyear{frei1929}); this strand of research was also pursued by \citet{fromkin1973},
\citet{rossietal1998}, and \citet{nooteboom2011}. The usefulness of speech errors and word games to gain
insights into tonal systems was demonstrated by \citet[180–181]{hombert1986b}. Speakers of \ili{Mandarin},
\ili{Cantonese}, \ili{Minnan} and \ili{Thai} tend to move the tones along with the syllables when changing
a~C\textsubscript{1}V\textsubscript{1}C\textsubscript{2}V\textsubscript{2} sequence into
C\textsubscript{1}V\textsubscript{2}C\textsubscript{2}V\textsubscript{1} or
C\textsubscript{2}V\textsubscript{2}C\textsubscript{1}V\textsubscript{1}, whereas speakers of
\ili{Bakwiri} (also known as Bakweri; \ili{Bantu} branch of Niger-Congo) tend to leave tone patterns unchanged. (See also
\citet{wanetal1998} on \ili{Mandarin}.)

\figref{fig:numberofmistakesintherecordednumeralplusclassifierphrasesasafactoroftherangeoftens} presents the distribution of mistakes according to the range of tens, showing the proportion of
%\is{mistakes}
mistakes occurring in numeral-plus-classifier phrases for numerals between 1 and 9 (leftmost bar), 10 and 19 (second bar),
etc. The ranges 50--59 and 80--89 are not represented, as they were excluded from the recordings to reduce consultant fatigue. As noted in the description of the elicitation procedure in \sectref{sec:elicitationprocedures}, this decision was based on the fact that, in the \is{morphotonology}morphotonological system of Yongning Na, the tone patterns of 50--59 are always identical with those for 40--49, and those of 80--89 mirror those of 60--69.

\figref{fig:numberofmistakesintherecordednumeralplusclassifierphrasesasafunctionofthelastdigitunits} shows the distribution of
%\is{mistakes}
mistakes according to the last digit: how many
mistakes concern numbers ending in 1 (namely 1, 11, 21, 31, 41, and so on), in 2, etc.

\begin{figure}[t!]
  \caption{Number of mistakes in the recorded numeral-plus-classifier
    phrases as a~factor of the range of tens.}
\begin{tikzpicture}
  \begin{axis}[
      width=\textwidth,
      bar width=8mm,
      height=50mm,
      ymajorgrids,
      ylabel
      near ticks,
      xlabel near ticks,
      ylabel={Number of mistakes},
      xlabel={Range of numerals},
      tick pos=left,
      ymin=0,
      ymax=34,
      ytick={0,5,10,15,20,25,30},
      yticklabels={0,5,10,15,20,25,30},
      symbolic x coords={1, 10, 20, 30, 40, 60, 70, 90},
      %x tick label style  = {text width=1cm,align=center},
      xtick={1,10,20,30,40,60,70,90},
      xticklabels={1--9, 10--19, 20--29, 30--39, 40--49, 60--69, 70--79,
        90--99}]
    \pgfplotsset{ytick style={draw=none}}
    \pgfplotsset{major grid style={dashed}}
    \pgfplotsset{every x tick label/.append style={font=\scriptsize}}
    \pgfplotsset{every y tick label/.append style={font=\scriptsize}}
    \addplot[ybar,fill=lsRichGreen] coordinates { (1, 6) (10, 5) (20, 33) (30, 32) (40, 20) (60, 16)
      (70, 26) (90, 18) };
  \end{axis}
\end{tikzpicture}
\label{fig:numberofmistakesintherecordednumeralplusclassifierphrasesasafactoroftherangeoftens}
\end{figure}


\begin{figure}%[t!]
  \caption{Number of mistakes in the recorded numeral-plus-classifier phrases as a~function of the
    last digit (units).}
\begin{tikzpicture}
  \begin{axis}[
      ymin=0,
      width=\textwidth,
      bar width=6mm,
      height=50mm,
      ymajorgrids,
      ylabel near ticks,
      xlabel near ticks,
      ylabel={Number of mistakes},
      xlabel={Last digit of numeral},
      tick pos=left,
      ymin=0,
      ymax=34,
      ytick={0,5,10,15,20,25,30},
      yticklabels={0,5,10,15,20,25,30},
      symbolic x coords={0, 1, 2, 3, 4, 5, 6, 7, 8, 9},]
    \pgfplotsset{xtick style={draw=none}}
    \pgfplotsset{ytick style={draw=none}}
    \pgfplotsset{major grid style={dashed}}
    \pgfplotsset{every x tick label/.append style={font=\scriptsize}}
    \pgfplotsset{every y tick label/.append style={font=\scriptsize}}
    \addplot[ybar,fill=lsRichGreen] coordinates { (0, 5) (1, 14) (2, 25) (3, 10) (4, 25) (5,20) (6,
      1) (7, 23) (8, 3) (9, 30) };
  \end{axis}
\end{tikzpicture}
\label{fig:numberofmistakesintherecordednumeralplusclassifierphrasesasafunctionofthelastdigitunits}
\end{figure}


The dataset is not perfectly symmetrical. In particular:
\begin{enumerate}[label=(\roman*)]
    \item some tonal categories are represented more extensively than others;
    \item certain combinations were repeated at recording; and
    \item there is a slight overrepresentation of tokens in the range [1..10] relative to higher numerals.
\end{enumerate}

Nevertheless, some observations can be made. Numerals ending in 6 and 8 (e.g.~6, 16, 26; 8, 18, 28) show a notably lower rate of mistakes. Likewise, numerals beginning with 6 or 8 (e.g.~60, 61, 62, 63; 80, 81, 82, 83) are slightly less prone to errors than those in adjacent ranges of ten. This is unsurprising, as numeral-plus-classifier phrases containing 6 and 8 are the least complex in tonal terms, due to the \isi{neutralization} of many tonal distinctions following these numerals (as noted at the outset of \sectref{sec:howthetonalcategorieswerebroughtoutandlabelled}).

By contrast, numeral-plus-classifier phrases containing numbers ending in 7 and 9, which exhibit the greatest diversity in tone patterns, are among the most frequently mistaken. The highest concentration of mistakes is found in phrases containing \{2, 4, 5, 7, 9\}.


\subsection{About variants of tone patterns}
\label{sec:aboutvariantsoftonepatterns}
%\largerpage
As noted at the outset of this chapter, some phrases allow for \is{variants}variant tone patterns. This phenomenon is more frequent for higher numerals than for numerals below twenty. For instance, the combination of the numeral 47 with the classifier for round objects can be realized either as
/\ipa{ʐv̩˧tsʰi˩--ʂɯ˧-ɭɯ˥}/ or as /\ipa{ʐv̩˧tsʰi˩--ʂɯ˩-ɭɯ˩}/. No observed numeral-plus-classifier phrase has more than two acceptable tone
patterns.

Many of these variants can be explained in light of the existence of two possible ways to parse a numeral-plus-classifier phrase: as a single tone group or as two tone groups. If a numeral-plus-classifier phrase is treated as a single \isi{tone group},
then its tone pattern is subject to the phonological rules that apply within a~tone
group (see \sectref{sec:alistoftonerules}).

For instance, the numeral 44 in association with the classifier for tools, /\ipa{nɑ˧\textsubscript{a}}/, allows for the following two variants: /\ipa{ʐv̩˧tsʰi˩--ʐv̩˧-nɑ˥}/ and /\ipa{ʐv̩˧tsʰi˩--ʐv̩˩-nɑ˩}/. The first of these, /\ipa{ʐv̩˧tsʰi˩--ʐv̩˧-nɑ˥}/, is not a~well-formed \isi{tone group}, as it violates Rule~5: within a \isi{tone group}, all syllables following an M.L sequence receive L tone. The phrase /\ipa{ʐv̩˧tsʰi˩--ʐv̩˧-nɑ˥}/ is therefore to be analyzed as consisting of two tone groups: /\ipa{ʐv̩˧tsʰi˩ {\kern2pt}|{\kern2pt} ʐv̩˧-nɑ˥}/ (tone
pattern: L\# for the first tone group, H\# for the second). If this phrase were treated as a single \isi{tone group}, the tones of its
last two syllables would be lowered to L. This is precisely what happens in the alternative attested \is{variants}variant: /\ipa{ʐv̩˧tsʰi˩ʐv̩˩-nɑ˩}/ (tone pattern: L\#--). The two variants can
therefore be described as (i)~a~form consisting of two tone groups and (ii)~a~simplified
form, whose tonal pattern results straightforwardly from its treatment as a~single tone
group. (The same phenomenon is reported in the Lataddi dialect: see \citealt{dobbsetal2016}.)

The same applies to all tonal patterns in the ranges [40..59], [70..79], {\linebreak}and [90..99], as the first
two syllables (corresponding to 40, 50, 70 and 90, respectively) carry a~Mid-plus-Low
pattern. This pattern precludes any tone other than L on the following syllables within the same
\isi{tone group} (by Rule~5). One would therefore expect all
of these combinations to have two variants. This holds true as a~general rule: whenever the consultant
provided a~complex form and I attempted a~simplified form, the latter was never rejected. On the other hand, for some combinations, only the simpler form is acceptable. For instance, for 44 with a~classifier of category M\textsubscript{b}, such as /\ipa{ɭɯ˧\textsubscript{b}}/ (the classifier
for round objects), the only correct (acceptable) form is /\ipa{ʐv̩˧tsʰi˩ʐv̩˩-ɭɯ˩}/ (tone pattern: L\#--).

If one supposes, by {analogy} with category M\textsubscript{a}, that $\dagger${\kern2pt}\ipa{ʐv̩˧tsʰi˩ {\kern2pt}|{\kern2pt} ʐv̩˧-ɭɯ˥} was acceptable at
an earlier stage of the language, then it must have fallen out of use. For category M\textsubscript{a}, where two variants are currently in common use, the consultant’s intuition is that the complex \is{variants}variant,
/\ipa{ʐv̩˧tsʰi˩ {\kern2pt}|{\kern2pt} ʐv̩˧-nɑ˥}/, sounds somewhat “slow” and “clumsy”; it conveys special
\isi{emphasis} and is only appropriate as part of an \is{expressivity}expressive strategy to highlight the figure in question. In summary, the integration of numeral-plus-classifier phrases into a single tone group is the general rule.

Interestingly, when a~phrase ends in two Low-tone syllables, it is possible to test whether these
Low tones result from the levelling down of originally non-Low tones (as in the case of
/\ipa{ʐv̩˧tsʰi˩--ʐv̩˩-nɑ˩}/, mentioned above) or whether they reflect an underlying Low tone. When the phrase is divided into two tone groups, if the second group has an underlying L tone, it receives a~postlexical final H tone through the application of Rule~7: “If a~tone
group only contains L tones, a~postlexical H tone is added to its last syllable”. For example, ‘23
years’ (category MH\textsubscript{a}) can be realized either as /\ipa{ɲi˧tsi˧--so˩-kʰv̩˩}/ or as /\ipa{ɲi˧tsi˧ {\kern2pt}|{\kern2pt}
  so˩-kʰv̩˩˥}/, revealing that its underlying tone pattern is M--L. By contrast, with the classifier for
tools, it would be incorrect to say \ipa{$\ddagger${\kern2pt}ʐv̩˧tsʰi˩ {\kern2pt}|{\kern2pt} ʐv̩˩-nɑ˩˥}: the acceptable {variant} with a~division into
two groups is /\ipa{ʐv̩˧tsʰi˩ {\kern2pt}|{\kern2pt} ʐv̩˧-nɑ˥}/.
%This neatly explains why the \is{tonal contour}contour-creating final H tone is only allowed for some of the phrases.

A~device for forcing the division of the phrase into two
tone groups consists of inserting the syllable /\ipa{lɑ˧}/ ‘and’ before the last digit, as in /\ipa{ʂɯ˧tsʰi˩ lɑ˩ {\kern2pt}|{\kern2pt} qʰv̩˧-ʁwɤ˥}/ ‘79 heaps’.


\subsection[Disyllabic classifiers]{Disyllabic classifiers, and what they reveal about the tones of numerals}
\label{sec:disyllabicclassifiersandimplicationsforthetonesofnumerals}

Only a~few \is{disyllables}disyllabic classifiers were observed. One is a~reduplicated {monosyllable}:
/\ipa{ʈʂʰe˧{$\sim$}ʈʂʰe˧}/ ‘the width of a~room’.\footnote{Roselle Dobbs (p.c.\ 2016) points out that the classifier /\ipa{ʈʂʰe˧{$\sim$}ʈʂʰe˧}/ ‘the width of a~room’ may be reduplicated from the verb /\ipa{ʈʂʰe˧\textsubscript{b}}/ ‘to stretch’.} Others are nouns, such as /\ipa{ʝi˧qʰv̩\#˥}/ ‘bull’s horn’, which can be used as a~classifier, as bull horns were traditionally used for drinking or pouring liquids, for instance to pour water into a~pot. This noun competes, in its classifier function, with
another classifier specifically referring to hornfuls of liquids, /\ipa{qʰv̩˧tʰv̩˧}/, which is more
commonly used.
%The classifier for strides (large steps) is disyllabic:
%/\ipa{pɤ˧ʁɑ˧}/.

‘Bottle’, /\ipa{to˩bi\#˥}/, can be used either as a~noun, as in /\ipa{to˩bi\#˥ {\kern2pt}|{\kern2pt} ɖɯ˧-ɭɯ˧}/ ‘a bottle’, or as a~classifier, as in /\ipa{ʐɯ˧ {\kern2pt}|{\kern2pt} ɖɯ˧-to˩bi˩}/ ‘a bottle of liquor/spirits’. The classifier for
ladlefuls is /\ipa{bv̩˩dze˩}/, and that for handfuls (using both hands) is /\ipa{lo˩dzi˩}/.

The disyllabic classifiers observed to date fall into one of four tonal categories: M, \#H, L, and LM+\#H. In terms of their behaviour in association with numerals, these four tonal categories cluster into two sets: M and \#H on the one
hand, L and LM+\#H on the other. Tables~\ref{tab:hornfuls} and \ref{tab:bottles} present the data. Clearly, the relevant \is{juncture (inside a tone group)}juncture for tonal association in these phrases is the one preceding the classifier, even for numerals below twenty. Describing the tone pattern of /\ipa{ɖɯ˧-to˩bi˩}/ ‘one bottle’ requires recognition of a~\is{juncture (inside a tone group)}juncture preceding the classifier, i.e.\ /\ipa{ɖɯ˧~-- to˩bi˩}/.


\begin{table}[t!]
\caption{\label{tab:hornfuls}The tonal behaviour of disyllabic classifiers with lexical M or \#H tone.}
\begin{tabularx}{\textwidth}{ P{20mm} Q Q Q }
  \lsptoprule
  	numeral & example form & output tone & meaning\\ \midrule
	1 & \ipa{ɖɯ˧-qʰv̩˧tʰv̩\#˥} & \#H & one hornful\\
	2 & \ipa{ɲi˧-qʰv̩˧tʰv̩\#˥} & \#H & two hornfuls\\
	3 & \ipa{so˩-qʰv̩˩tʰv̩˩˥} & L & three hornfuls\\
	4 & \ipa{ʐv̩˧-qʰv̩˧tʰv̩\#˥} & \#H & four hornfuls\\
	5 & \ipa{ŋwɤ˧-qʰv̩˧tʰv̩\#˥} & \#H & five hornfuls\\
	6 & \ipa{qʰv̩˧-qʰv̩˧tʰv̩\#˥} & \#H & six hornfuls\\
	7 & \ipa{ʂɯ˧-qʰv̩˧tʰv̩\#˥} & \#H & seven hornfuls\\
	8 & \ipa{hõ˧-qʰv̩˧tʰv̩\#˥} & \#H & eight hornfuls\\
	9 & \ipa{gv̩˧-qʰv̩˧tʰv̩\#˥} & \#H & nine hornfuls\\
	10 & \ipa{tsʰe˩-qʰv̩˩tʰv̩˩˥} & L & ten hornfuls\\
\lspbottomrule
\end{tabularx}
\end{table}

\begin{table}[t!]
\caption{\label{tab:bottles}The tonal behaviour of disyllabic classifiers with lexical L or LM+\#H tone.}
\begin{tabularx}{\textwidth}{ P{20mm} Q Q Q }
  \lsptoprule
  	numeral & example form & output tone & meaning\\ \midrule
	1 & \ipa{ɖɯ˧-to˩bi˩} & --L & one bottle\\
	2 & \ipa{ɲi˧-to˩bi˩} & --L & two bottles\\
	3 & \ipa{so˩-to˩bi˩˥} & L & three bottles\\
	4 & \ipa{ʐv̩˧-to˥bi˩} & \#H-- & four bottles\\
	5 & \ipa{ŋwɤ˧-to˥bi˩} & \#H-- & five bottles\\
	6 & \ipa{qʰv̩˧-to˥bi˩} & \#H-- & six bottles\\
	7 & \ipa{ʂɯ˧-to˩bi˩} & --L & seven bottles\\
	8 & \ipa{hõ˧-to˥bi˩} & \#H-- & eight bottles\\
	9 & \ipa{gv̩˧-to˥bi˩} & \#H-- & nine bottles\\
	10 & \ipa{tsʰe˩-to˩bi˩˥} & L & ten bottles\\
\lspbottomrule
\end{tabularx}
\end{table}


%Overall, there is much less diversity of tones for disyllabic classifiers than for \is{monosyllables}monosyllabic
%ones.

\subsection[Two numerals plus a~classifier]{Expressing approximation with two numerals and a~classifier}
\label{sec:twonumerals}

Two numerals can accompany a~classifier, conveying an approximative number. This can be likened to the coordinative construction ‘\textsc{num} or \textsc{num}’ in {English}, as in ‘one or two’ and ‘two or three’. This construction is less common than coordinative compounds consisting of two numeral-plus-classifier phrases, such as /\ipa{ɖɯ˧-ɲi˧ -- ɲi˧-ɲi˧}/ ‘one day-two days’ and /\ipa{ɲi˧-ɲi˧ -- so˧-ɲi˥}/ ‘two days-three days’, presented in \sectref{sec:coordinativecompounds}. Table~\ref{tab:twonum} sets out the results of elicitation with the noun ‘day’. A~dash ‘--’ indicates that the combination is not in use. This is not to say that there is a~hard-and-fast rule against the combinations marked with a~dash in Table~\ref{tab:twonum}, only that they seemed less felicitous to the consultant for reasons that may involve considerations of \isi{homophony} along with frequency of use and semantics.

For instance, the expression \ipa{ɖɯ˧-ɲi˧ ɲi˧}, literally ‘one-two days’, actually means ‘a few days’, somewhat like ‘a~couple of days’ in {English}. Since this expression easily covers the range from one to four, it renders the combinations ‘two-three’ and ‘three-four’ unnecessary. Among higher numbers, a~reason why the combination ‘seven-eight’ is deemed acceptable may be that it corresponds to ‘a week or so’, a~span of time that holds particular relevance in the consultant's current conceptualization of time, given that weeks are meaningful units due to the grandchildren's school schedule. Seen in this light, the combination ‘six-seven’ should be about as relevant as ‘seven-eight’. Indeed, the consultant (Mrs.\ Latami) confirmed that it could be acceptable at a~push: \ipa{qʰv̩˧-ʂɯ˧ ɲi˧} ‘six or seven days’.

\begin{table}%[t]
	\caption{Expressions conveying an approximative number: two numerals plus a~classifier.}
	\begin{tabularx}{\textwidth}{ l l l }
		\lsptoprule
		numerals & association of two numerals & meaning\\\midrule
		1 and 2 & \ipa{ɖɯ˧-ɲi˧ ɲi˧}  &  ‘a few days’\\
		2 and 3 & -- &\\
		3 and 4 & -- &\\
		4 and 5 & \ipa{ʐv̩˧-ŋwɤ˧ ɲi˧} & ‘four or five days’\\
		5 and 6 & \ipa{ŋwɤ˧-qʰv̩˧ ɲi˧} & ‘five or six days’\\
		6 and 7 & -- &\\
		7 and 8 & \ipa{ʂɯ˧-hõ˧ ɲi˧} & ‘seven or eight days’\\
		8 and 9 & -- &\\
		9 and 10 & \ipa{gv̩˧-tsʰe˧ ɲi˥} & ‘nine or ten days’\\
		\lspbottomrule
	\end{tabularx}
	\label{tab:twonum}
\end{table}

The fact that the expressions in \tabref{tab:twonum} do not form a~fully productive par\-a\-digm in the consultant's speech led to the conclusion that it would not be appropriate to attempt systematic elicitation of phrases consisting of two numerals and a~classifier. There was a~concern that elicitation of unusual forms~-- many of them new coinages~-- would not yield consistent results (as discussed in \sectref{sec:examinationoftranscribedtextsanddirectelicitation}). This is one of many topics that remain for future research.

%\largerpage % No longer necessary for 2nd ed.
\subsection[Conclusions]{Conclusions about numeral-plus-classifier phrases}
\label{sec:Conclusions}

From the mass of information set out above, it is clear that the tone patterns of
numeral-plus-classifier phrases encapsulate information that is not derived from phonological rules. The
system presented in Tables~\ref{tab:1to25hmh} to \ref{tab:76to100ml} is both regular and productive, in that all classifiers of a~given
tone category follow the same tone patterns. As this system lends itself straightforwardly to computer
implementation, a~simple script was written.\footnote{As of 2025, the script is available at \url{https://github.com/alexis-michaud/na/blob/master/SCRIPTS/NaTone/NaTone.pl}.} This script takes as input
the classifier’s tone category, its segmental composition, and a~numeral (or range of numerals) from 1
to 100. The data in Tables~\ref{tab:1to25hmh} to \ref{tab:76to100ml} is stored within the script, allowing for tone patterns to be retrieved through table lookup. The surface phonological tone pattern is then assigned to the phrase based on the general rules governing tonal association in this dialect (rules that are encoded into the script), such as the association of simple L and M tones to all syllables within their domain, and the “left-to-right” association of tone sequences.

For instance, when the numeral 44 and the classifier for tools, /\ipa{nɑ˧\textsubscript{a}}/ (tonal category: M\textsubscript{a}), are provided as input, the script yields the following
two variants for ‘44 tools’: /\ipa{ʐv̩˧tsʰi˩--ʐv̩˧-nɑ˥}/ (tone pattern: L\#--H\#) and /\ipa{ʐv̩˧tsʰi˩--ʐv̩˩-nɑ˩}/ (tone
pattern: L\#--).

In the current version of the script, all of the information set out in Tables~\ref{tab:1to25hmh} to \ref{tab:76to100ml} is encoded in full,
specifying the tone patterns of 900 combinations (9 tone categories of classifiers × 100
numerals). This allows for straightforward table-lookup but is uneconomical from the perspective
of linguistic modelling. The addition of some rules could significantly reduce the number of
combinations that need to be explicitly specified.

In particular, the tone patterns of numerals in the range [40..49] are identical
to those in [50..59]; likewise for [60..69] and [80..89]. Numerals ending in 1 also share
identical tone patterns with those ending in 2, with a~few exceptions in category H\textsubscript{b}. Furthermore, the information provided for each \is{subcategories of lexical tones}subcategory (H\textsubscript{a} and H\textsubscript{b}, M\textsubscript{a} and M\textsubscript{b}, and so on) could be streamlined by
designating one \is{subcategories of lexical tones}subcategory as the default and only specifying forms for the other
\is{subcategories of lexical tones}subcategories where they diverge from this default.

However, even after such {simplification}, a substantial number of
tonal patterns would still need to be specified individually. For instance, neither H\textsubscript{a} nor H\textsubscript{b} can be straightforwardly derived from the other. While the presence of an L tone in all phrases from 10 to 19 might suggest that the patterns for H\textsubscript{b} constitute a~simplified version of those for H\textsubscript{a}, there is also a~complicating factor for H\textsubscript{b} that does not apply to H\textsubscript{a}, namely the presence of different tones after numerals ending in 1 and 2. This observation is a~striking \isi{counterexample} to the pan-\ili{Naish} generalization that
the numerals 1 and 2 always have the same tone patterns~-- a~generalization that holds for \ili{Naxi}
and \ili{Laze}, as well as for all the other available Na data.

Finally, an idiosyncratic tone pattern is observed for /\ipa{to˥}/ ‘armful’: this classifier belongs to the H\textsubscript{a} category, yet the combination ‘11 armfuls’ is realized as
/\ipa{tsʰe˧ɖɯ˧-to˧}/ instead of the expected $\dagger${\kern2pt}\ipa{tsʰe˩ɖɯ˩-to˩˥}.


\section{Demonstrative-plus-classifier phrases}
\label{sec:demonstrativeplusclassifierphrases}

A \is{demonstratives}demonstrative and a~following classifier are always integrated into a single \isi{tone group}, as illustrated in (\ref{ex:dress}), where the proximal \is{demonstratives}demonstrative //\ipa{ʈʂʰɯ˥}// and the classifier for pairs or sets //\ipa{dzi˧\textsubscript{b}}// combine as /\ipa{ʈʂʰɯ˧-dzi˧˥}/ ‘this set'.

\begin{exe}
	\ex
	\label{ex:dress}
	\ipaex{no˩bv̩˧ {\kern2pt}|{\kern2pt} ʈʂʰɯ˧-dzi˧˥ {\kern2pt}|{\kern2pt} le˧-ʑi˩, {\kern2pt}|{\kern2pt} tʰi˧-mv̩˧-kʰɯ˧˥.}\\
	\gll no˩bv̩˧		ʈʂʰɯ˥					dzi˧\textsubscript{b}			le˧-		ʑi˩\textsubscript{b}	tʰi˧-	mv̩˧\textsubscript{a}	-kʰɯ˧˥\\
	given\_name		\textsc{dem.prox}	\textsc{clf}.sets/pairs				\textsc{accomp}	to\_bring\_along		\textsc{dur}		to\_put\_on		\textsc{caus}\\
	\glt ‘Nobbu brought that set [of clothes], and made [her] wear [the clothes].’ \textit{(BuriedAlive3.73)} \pandoi{0004538\#S73}
\end{exe}

The
elicitation procedure for demonstrative-plus-classifier phrases was similar to that used for numeral-plus-classifier phrases, and similar
puzzles were encountered.

As explained in the previous section, the tonal categories of classifiers were established on the
basis of their behaviour when combined with numerals. The nine tonal categories of monosyllabic classifiers are: H\textsubscript{a} and H\textsubscript{b}; MH\textsubscript{a} and
MH\textsubscript{b}; M\textsubscript{a} and M\textsubscript{b}; and L\textsubscript{a}, L\textsubscript{b}, and L\textsubscript{c}. As for the proximal \is{demonstratives}demonstrative, /\ipa{ʈʂʰɯ\#˥}/, and the
distal \is{demonstratives}demonstrative, /\ipa{tʰv̩\#˥}/, both carry a lexical \#H tone. The expectations were that,
in \is{demonstratives}demonstrative-plus-classifier phrases, (i)~there would be no difference between proximal and distal
demonstratives, since they have the same lexical tone, and (ii)~all classifiers within each of the
nine tonal categories would have the same tonal behaviour.

The first prediction was verified: phrases containing /\ipa{ʈʂʰɯ\#˥}/ ‘this’ and /\ipa{tʰv̩\#˥}/
‘that’ always have the same tonal patterns. The second prediction, on the other hand, was not borne out: some of the tonal categories for classifiers proved to be less than fully homogeneous. The
account provided below begins with the simplest cases and progresses towards the category displaying the greatest degree of divergence.

When combined with demonstratives, H\textsubscript{a} and H\textsubscript{b} behave identically, as do MH\textsubscript{a} and MH\textsubscript{b}. This is the
simplest part of the system: the opposition between H\textsubscript{a} and H\textsubscript{b} is neutralized in this context, as is the opposition between MH\textsubscript{a} and MH\textsubscript{b}. Categories M\textsubscript{a} and M\textsubscript{b}, by contrast, behave differently
from each other, but in a~consistent and straightforward manner, each displaying only one possible pattern:
a demonstrative plus an M\textsubscript{a}-tone classifier yields L\#, e.g.~/\ipa{tʰv̩˧-nɑ˩}/ (classifier for tools); a demonstrative plus an M\textsubscript{b}-tone classifier yields \#H, e.g.~/\ipa{tʰv̩˧-ɭɯ\#˥}/ (generic classifier).

Among Low-tone classifier categories, L\textsubscript{a} and L\textsubscript{c} are relatively straightforward. All L\textsubscript{a}-tone
classifiers have the same behaviour, allowing two variants: H\# and H\$. Both variants are firmly
attested. The speaker expresses a~preference for the former, but this slight imbalance appears to be
uniform across examples, suggesting that no clear (lexicalized) preference exists for one variant over the other in association with a~specific classifier. L\textsubscript{c}-tone
classifiers, on the other hand, allow no fewer than three variants: MH\#, H\#, and H\$.

Category L\textsubscript{b} is the most problematic. In the production data, three variants are attested: H\#, H\$, and
MH\#. However, the distribution of these variants is not uniform across classifiers of this category. For some classifiers (e.g.~/\ipa{mi˩\textsubscript{b}}/, the classifier for animals, and
/\ipa{kʰɯ˩\textsubscript{b}}/, the classifier for long objects), MH\# is by far the most frequent pattern, with H\$ appearing as an occasional {variant} and H\# rarely attested. For other classifiers, by contrast, H\# and H\$ occur with comparable frequency, whereas MH\# is seldom found.

When several \is{variants}variants were proposed by the investigator and the consultant was
asked to evaluate their acceptability, a~similar picture emerged: MH\# was strongly preferred for some classifiers, with H\$ judged as
an acceptable {variant}, whereas H\# was dispreferred (either rejected as incorrect or considered only marginally acceptable). For other classifiers, the preferred form was with H\# tone, H\$ was
an acceptable {variant}, and MH\# was not. \tabref{tab:thetwosubcategoriesoflbtoneclassifiersbasedontheirbehaviourinassociationwithdemonstratives} presents these two subsets, provisionally labelled
as ‘type I’ and ‘type II’.

\begin{table}%[t]
  \caption{The two subcategories of L\textsubscript{b}-tone classifiers, based on their behaviour in association with demonstratives.}
\begin{tabularx}{\textwidth}{ P{20mm} Q P{15mm} P{25mm} }
\lsptoprule
	type & tone pattern in association with {demonstrative} & form & classifier for\\ \midrule
	L\textsubscript{b}, type I & MH\# most common; & \ipa{bo˩\textsubscript{b}} & headdresses\\
	 & H\$ attested;  & \ipa{dv̩˩\textsubscript{b}} & small groups\\
	 & H\# dispreferred or refused & \ipa{dzi˩\textsubscript{b}} & trees\\
	 &  & \ipa{jo˩\textsubscript{b}} & ounces\\
	 &  & \ipa{kʰɯ˩\textsubscript{b}} & long objects\\
	 &  & \ipa{lo˩\textsubscript{b}} & valleys\\
	 &  & \ipa{ɬi˩\textsubscript{b}} & armspans\\
	 &  & \ipa{mi˩\textsubscript{b}} & animals\\
	 &  & \ipa{pʰv̩˩\textsubscript{b}} & fields\\
	 &  & \ipa{tʰv̩˩\textsubscript{b}} & sets of ten\\
	 &  & \ipa{tɕʰi˩\textsubscript{b}} & meals\\
	 &  & \ipa{wɤ˩\textsubscript{b}} & loads\\
	 &  & \ipa{wo˩\textsubscript{b}} & teams of oxen\\ \midrule
	L\textsubscript{b}, type II & H\# most common; & \ipa{pɤ˩\textsubscript{b}} & ladders, doors\\
	 & H\$ attested; & \ipa{po˩\textsubscript{b}} & packs\\
	 & MH\# dispreferred or refused & \ipa{ʁo˩\textsubscript{b}} & types, sorts\\
	 &  & \ipa{ʂɯ˩\textsubscript{b}} & times\\
	 &  & \ipa{tsʰe˩\textsubscript{b}} & leaves\\
	 &  & \ipa{ʈv̩˩\textsubscript{b}} & large chunks\\
\lspbottomrule
\end{tabularx}
\label{tab:thetwosubcategoriesoflbtoneclassifiersbasedontheirbehaviourinassociationwithdemonstratives}
\end{table}

\newpage
The consultant’s acceptance of variants fluctuated somewhat from session to session, but the distinction
between types I and II was confirmed over the course of an extensive series of elicitation sessions. Additional evidence is provided by examples from recorded texts, which reflect the same patterns.

The tone patterns for all categories of classifiers are set out in \tabref{tab:thetonepatternsofdemonstrativeplusclassifierphrases}, using the distal
demonstrative /\ipa{tʰv̩\#˥}/ as an example. (As noted above, the tone patterns for the
proximal \is{demonstratives}demonstrative, /\ipa{ʈʂʰɯ\#˥}/, are identical.) The corresponding online recordings are \textit{DemClf} \pandoi{0004830}, \textit{DemClf2} \pandoi{0004832}, and \textit{DemClf3} \pandoi{0004834}.

\begin{table}%[t]
\caption{The tone patterns of demonstrative-plus-classifier phrases.}
\begin{tabularx}{\textwidth}{ l P{30mm} Q l }
\lsptoprule
	\multicolumn{2}{l}{tone pattern} & \multicolumn{2}{l}{example}\\\cmidrule(lr){1-2}\cmidrule(lr){3-4}
	tone of \textsc{clf} &  tone of~\textsc{dem}+\textsc{clf} phrase & classifier for &
   \textsc{dem}+\textsc{clf} phrase\\ \midrule
	H\textsubscript{a}, H\textsubscript{b} & H\$ / \#H & chunks & \ipa{tʰv̩˧-kʰwɤ˥\$} / \ipa{tʰv̩˧-kʰwɤ\#˥}\\
	MH\textsubscript{a}, MH\textsubscript{b} & L\# & cattle & \ipa{tʰv̩˧-pʰo˩}\\
	M\textsubscript{a} & L\# & tools & \ipa{tʰv̩˧-nɑ˩}\\
	M\textsubscript{b} & \#H & \textit{generic} & \ipa{tʰv̩˧-ɭɯ\#˥}\\
	L\textsubscript{a} & H\# / H\$ & quantities & \ipa{tʰv̩˧-mɤ˥ / tʰv̩˧-mɤ˥\$}\\
	L\textsubscript{b}, type I & MH\# / H\$ & animals & \ipa{tʰv̩˧-mi˧˥ / tʰv̩˧-mi˥\$}\\
	L\textsubscript{b}, type II & H\# / H\$ & doors & \ipa{tʰv̩˧-pɤ˥ / tʰv̩˧-pɤ˥\$}\\
	L\textsubscript{c} & MH\# / H\# / H\$  & plains & \ipa{tʰv̩˧-di˧˥ / tʰv̩˧-di˥ / tʰv̩˧-di˥\$}\\
\lspbottomrule
\end{tabularx}
\label{tab:thetonepatternsofdemonstrativeplusclassifierphrases}
\end{table}

The distinction between ‘type I’ and ‘type II’ within L\textsubscript{b}-tone classifiers constitutes a striking additional layer of complexity within the classifier system's nine tonal categories. L\textsubscript{b} itself is a~subdivision within the broader L category of tones; its
further division into types I~and II thus constitutes a~subdivision within a~subdivision. From the perspective of phonological output, type I within the L\textsubscript{b} category yields an outcome (MH\# / H\$)
that does not coincide with that of any other category within the system.



\section{Tonal interactions with a~preceding noun}
\label{sec:interactionnoun}
\label{sec:introductioninteractionnoun}

Occasionally, tonal interaction occurs between a~numeral-plus-classifier or
{\linebreak}demonstrative-plus-classifier phrase and the preceding noun, as in (\ref{ex:wowsheisgivingyouthreesilvercoins}).
\begin{exe}
  \ex
  \ipaex{ə˧mi˧! {\kern2pt}|{\kern2pt} pæ˧kʰwɤ˧ so˧-ɭɯ˥ ki˩ mæ˩!}\\
  \label{ex:wowsheisgivingyouthreesilvercoins}
  \gll ə˧mi˧	pæ˧kʰwɤ\#˥	so˩	ɭɯ˧	ki˧\textsubscript{a}	mæ˧\\
  \textsc{intj}	silver\_coin	three	\textsc{clf}	to\_give	\textsc{disc}.\textsc{ptcl}\\
  \glt ‘Wow! [(S)he] is giving you three silver coins!’
\end{exe}

%\newpage
According to the main consultant's recollections, (\ref{ex:wowsheisgivingyouthreesilvercoins}) is the
type of comment that uncles and aunts would make when a~child received a significant monetary gift on the occasion of their coming-of-age ceremony.\footnote{Explanations are provided in the narrative \textit{ComingOfAge2.58ff.} \pandoi{0004588\#S58}.} Offering a single coin would be inappropriate, as
gifts should come in pairs. Two coins constituted a generous gift, while three exceeded expectations, corresponding at the time to approximately half a~month's salary.

The tonal \is{derivation!tonal}derivation for the phrase ‘three silver coins’ is as follows:
\begin{enumerate}[label=(\roman*), itemsep=0pt]
\item The input tones are: \#H for ‘silver coin’, //\ipa{pæ˧kʰwɤ\#˥}//; L for ‘three’; and M
  for the classifier, //\ipa{ɭɯ˧}//.
\item The numeral-plus-classifier phrase ‘three-\textsc{clf}’ carries M tone: //\ipa{so˧-ɭɯ˧}//.
\item The phrase resulting from the combination of (i) and (ii) carries an H\# tone: //\ipa{pʰæ˧kʰwɤ˧
  so˧-ɭɯ˥\#}//.
\end{enumerate}

Other examples of such tonal interaction include (\ref{ex:thatgirl}) and (\ref{ex:thatuncle}).

\begin{exe}
	\ex
	\label{ex:thatgirl}
	\ipaex{mv̩˩zo˩ tʰv̩˩-ɭɯ˩˥}\\
	\gll mv̩˩zo˩	tʰv̩˥	ɭɯ˧\textsubscript{b}\\
	 girl	\textsc{dem.dist}	\textsc{clf}.generic\\
	\glt ‘that girl’ \textit{(Coming\-Of\-Age2.60; Tiger2.139, 141, 147)} \pandoi{0004588\#S60}
\end{exe}

\begin{exe}
	\ex
	\label{ex:thatuncle}
	\ipaex{ə˧v̩˧ tʰv̩˧-v̩˧}\\
	\gll ə˧v̩˧˥	tʰv̩˥	v̩˧\\
	uncle	\textsc{dem.dist}	\textsc{clf}.individual\\
	\glt ‘that uncle’ \textit{(Tiger2.109)} \pandoi{0004545\#S109}
\end{exe}

The \is{form!underlying}underlying tonal category of such
expressions can be established using the same tests as for nouns, such as the addition of the
\isi{copula}. The expression ‘three silver coins’ /\ipa{pʰæ˧kʰwɤ˧ so˧-ɭɯ˥}/ in (\ref{ex:wowsheisgivingyouthreesilvercoins}) yields /\ipa{pʰæ˧kʰwɤ˧ so˧-ɭɯ˥ ɲi˩}/,
revealing that its tone is H\#: //\ipa{pʰæ˧kʰwɤ˧ so˧-ɭɯ˥\#}//. By contrast, ‘that uncle’, /\ipa{ə˧v̩˧ tʰv̩˧-v̩˧}/ (\ref{ex:thatuncle}),
yields /\ipa{ə˧v̩˧ tʰv̩˧-v̩˧ ɲi˥}/, showing that its tone is \#H: //\ipa{ə˧v̩˧ tʰv̩˧-v̩\#˥}//. (As for (\ref{ex:thatgirl}), its surface form is sufficient to determine its underlying tone: it can only be //L//.)

Why is tonal change only occasional? Here are some speculations.

First, if tonal change were systematic, it would entail the \isi{neutralization} of some tonal distinctions
among classifiers. The number of possible tone patterns in three-syllable or longer tone groups is not as high as the combinatorial diversity in shorter tone groups. For instance, after disyllabic nouns carrying an L\# tone, determiners would undergo tone lowering to L (by application of Rule~5: “All syllables following an H.L or M.L sequence receive L tone”), leading to a complete \isi{neutralization} of tonal oppositions among classifiers. %Even if the majority of contrasts among different categories of classifiers were retained,
Tonal change would thus decrease distinctiveness, as \isi{neutralization} makes surface tonal strings less rich in phonological information. It would moreover increase
\isi{opacity}, as more cases would arise in which the \is{form!surface}surface phonological tone of numeral-plus-classifier determiners differed from their \is{form!underlying}underlying tone. Given the richness of the tone system of classifiers, increased \isi{opacity} would be likely to create pressure towards \isi{simplification}. The fact that tonal interaction between a~noun and a~following determiner occurs only occasionally may thus contribute to the preservation of the current system.

Functionally, the availability of alternative forms expands the range of \is{stylistics}stylistic possibilities.

% Here: a subsection header was deleted.
\label{sec:anasymmetricalsystemsometonesaremorepronetochangethanothers}

While discourse factors are paramount in determining whether interaction occurs between the noun and the phrase that determines it, the tonal category may also play a~role. All else being equal, some
tones appear to resist interaction because the resulting tonal output would be non-trivial, requiring the speaker to recall a specific, morphologically conditioned tone combination rule. Other tones, by contrast, are more prone to interaction, as integrating the noun with the following determiner phrase is simply a~matter of applying phonological rules.

For instance, when the noun's tone is L\#, it feels easiest to integrate this noun with the following phrase, as the phonological consequence is a straightforward levelling down of the tones of the following syllables within the \isi{tone group}: the noun's two syllables get M and L tone, respectively, and all syllables following this M.L sequence receive L tone by Rule~5. Conversely, division into two tone groups requires the additional effort of preserving the tones of the determiner phrase from lowering. While this is possible, it is less economical. As the path of least resistance leads to adopting the pattern in which the noun is integrated with its determiner, keeping it in a tone group of its own requires a~\is{stylistics}stylistic motivation: an intention to \is{emphasis}emphasize the two constituents.

The implication for the analysis of the tone system is that some tones are more combination-prone than others, as the tone combination rules in which they are involved follow directly from phonological principles and thus tend to be exceptionless and straightforward to implement. The \is{tone spreading}spreading of L\# onto following syllables stands as the best example. Other tones, by contrast, are less combination-prone because the tone changes they undergo are more complex and allow several variants. The H\$ tone category is a~case in point.

Such asymmetries within the tone system, with some tones being more change-prone than others, would
be reflected in the frequency of tonal interactions between a~noun and a~following determiner
(N+\textsc{dem}+\textsc{clf}, N+\textsc{num}+\textsc{clf}). Systematic verification of these hypotheses must be deferred to a later stage.

% \largerpage[-2] % Not useful in final page layout for 2nd edition
\section[Concluding note]{Concluding note}
\label{sec:generalconclusion}

While the tone patterns of classifiers may appear staggeringly complex, phrases containing classifiers are frequent in discourse, a~factor known to favour the preservation of irregular morphology \citep{wu-etal-2019-morphological,round2024natural}. As for the system's {diachronic} origin, some reflections are set out in \sectref{sec:applicationtoyongningna}.

\is{classifiers|)}
\is{numerals|)}

If anyone has read this entire chapter with uninterrupted excitement and engagement, they are invited to imagine how much more elaborate the picture would have been had Yongning Na had the fortune to be used as a~medium to express number-theoretic concepts such as ordinals and fractions. \ili{Tibetan}, for instance, has ordinals as well as ways of expressing fractions and other mathematical concepts \citep{liu2010tib}, which tend to diffuse into neighbouring languages under the cultural influence of \ili{Tibetan} (e.g.~\ili{Nar-Phu}, spoken in the Manang district of Nepal: \citealt[342]{noonan2003}). \ili{Dzongkha}, a~Bodic language spoken in Bhutan, displays fascinating complexities such as the use of fractions within its counting system \citep{mazaudon1985a}. But in Yongning Na, there are neither ordinal numbers nor fractions. Expressions that contain ordinal numbers in other languages (such as {English} and Chinese) correspond to turns of phrase with cardinal numbers in Yongning Na.

For instance, in (\ref{ex:seventh}), ‘on the eighth day’ is expressed as ‘one day [after] seven days’, i.e.\ ‘the day after seven days had elapsed’.\footnote{The context of example (\ref{ex:seventh}) is as follows. A~young woman choked after swallowing boiled eggs too hastily; her in-laws, believing she had died, buried her. She regained consciousness when grave robbers set her upright to strip her of the valuable garments in which she had been interred. When she returned to her mother's home, the mother was terrified, fearing that her daughter had become a~ghost. According to ghost lore, spirits are thought to return within a limited period of seven days, so the mother instructed her daughter to remain in hiding until the seventh day had passed, and only then return home.} In (\ref{ex:bridge}), the equivalent of ‘the first person who will come' is ‘whoever comes', understood by inference from context as ‘the \textit{first} person who comes'.\footnote{This example requires some explanation of context. The parents of a~sick infant are advised to build a~bridge and to request the first person who crosses this bridge to bestow a~new name upon their child. This act constitutes a symbolic adoption, intended to grant the child a~fresh start in life. The stranger gives the infant a~new name along with a~small token, such as a~button from their garment. Example (\ref{ex:bridge}) is part of the instructions given to the parents regarding what they should do for their child.} Other examples are found in \textit{Renaming.29, 38} \pandoi{0004534\#S29} and \textit{Funeral.2} \pandoi{0004571\#S2}.

 \begin{exe}
 	\ex
 	\ipaex{ʂɯ˧-hɑ̃˧ gv̩˧ {\kern2pt}|{\kern2pt} ɖɯ˧-ɲi˥, {\kern2pt}|{\kern2pt} ə˧mi˧ lɑ˧ {\kern2pt}|{\kern2pt} mv̩˩˥ {\kern2pt}|{\kern2pt} tʰi˧-to˧{$\sim$}to˧, {\kern2pt}|{\kern2pt} ŋv̩˩ ɲi˥ tsɯ˩ {\kern2pt}|{\kern2pt} mv̩˩!}\\
 	\label{ex:seventh}
 	\gll ʂɯ˧-hɑ̃˧˥					gv̩˧\textsubscript{c}			ɖɯ˧-ɲi\$˥					ə˧mi˧		lɑ˧		mv̩˩˥		tʰi˧-		to˧{$\sim$}to˧		ŋv̩˩\textsubscript{a}		-ɲi˩		tsɯ˧˥		mv̩˧\\
 	seven-\textsc{clf}.nights	to\_elapse		one-\textsc{clf}.days	mother		and		daughter		\textsc{dur}	to\_hug		to\_cry			\textsc{certitude}		\textsc{rep}		\textsc{affirm}\\
 	\glt ‘On the eighth day (\textit{literally}: the day after seven days had gone by), mother and daughter fell into each other's arms and wept!' \textit{(BuriedAlive3.129)} \pandoi{0004538\#S129}
 \end{exe}

% Dirty hack for good page layout: avoiding orphaned line starting an example
%{\newpage} Not necessary for 2nd edition

 \begin{exe}
 	\ex
 	\ipaex{hĩ˧ {\kern2pt}|{\kern2pt} ɲi˩ le˧-tsʰɯ˩-ɻ̩˩-dʑo˩~{\dots}}\\
 	\label{ex:bridge}
 	\gll hĩ˥		ɲi˩							le˧-					tsʰɯ˩\textsubscript{a}		ɻ̩˩								-dʑo˥\\
 	person		\textsc{interrog}.who	\textsc{accomp}		to\_come					to\_turn\_towards	\textsc{top}\\
 	\glt ‘The first person who comes along{\dots} (\textit{literally:} whoever comes along{\dots})' \textit{(Renaming.14)} \pandoi{0004534\#S14}
 \end{exe}

\is{language contact}Contact with \il{Sinitic}Chinese, including training in mathematics at school, now exerts pressure towards the adoption of ordinals and fractions, but this has resulted is the wholesale borrowing of Chinese forms.

% Indexing stuff
%\is{in isolation (realization of word in isolation)|see{form!in isolation}}
