\chapter{Compound nouns}
\label{chap:compoundnouns}

% indexing 'compound' for whole chapter
\is{compounds|(}
 
Tonal processes within the noun phrase constitute a~major part of the Yongning Na tone
system. They also shed light on evolutionary processes. In Na, as in other {Sino-Tibetan} languages
that have undergone considerable \isi{phonological erosion} (such as \ili{Tujia} \zh{土家语}, \ili{Bai} \zh{白语}, \ili{Namuyi} \zh{纳木依语}, and \ili{Shixing} \zh{史兴语}), many
roots that were once phonologically distinct have become \is{homophony}homophonous. As a~result, there
is a~strong tendency towards \isi{disyllabification}. The study of synchronic tonal processes reveals which processes~-- such as compounding and {affixation}~-- feed into which tonal categories of
\is{disyllables}disyllabic nouns. It also highlights, by contrast, those disyllabic nouns whose tones deviate from expected patterns based on currently productive rules. In turn, this draws attention to these
outlier nouns, raising the question of how they acquired their tone~-- whether they date back to
a~time when different tonal rules applied, for instance.

Tonal phenomena within the noun phrase in Na will be presented by starting with
compounds (this chapter), then examining combinations between
nouns and grammatical morphemes (Chapter~\ref{chap:combinationsofnounswithgrammaticalwords}).

Compounding is a~highly productive word formation process in Na. “Compounding is the
prevalent morphological process” \citep[344]{lidz2010}. This holds true for many other languages of East and Southeast Asia as well (for {Sinitic}, see \citealt[passim]{arcodia2012}). \is{compounds!determinative|textbf}Determinative compounds (the \is{compounds!tatpuruṣa|textbf}\textit{tatpuruṣa} compounds of Sanskrit grammar) are
more common than coordinative compounds\is{compounds!coordinative|textbf} (Sanskrit \is{compounds!dvandva|textbf}\textit{dvandva}). In Yongning Na, these two types follow distinct tonal rules. For instance, 
% determinative compounds, such as ‘tiger’s skin’, and coordinative compounds, such as ‘mother and
% daughter’, do not follow the same tone rules. For instance, 
the determinative compound ‘nanny goat’s
back’ /\ipa{tsʰɯ˧mi˧-gv̩˧dv̩˥}/ carries H\# tone (a final H tone), whereas the coordinative compound
‘father and mother’ /\ipa{ə˧dɑ˧-ə˧mi\#˥}/ carries \#H tone (a \is{floating tone}floating H tone), even though the
input tones are the same: both ‘nanny goat’ and ‘father’ have H\$ tone, and both ‘mother’ and ‘back’
have M tone. Determinative and coordinative compounds will therefore be examined separately.



\section{Determinative compound nouns. Part I: The main facts}
\label{sec:determinativecompoundnouns}

In Yongning Na, determinative compounds follow the order \textit{determiner plus head}, as is generally the case in \il{Sino-Tibetan}Sino-Tibetan
\citep{michailovsky2011}.

In some tonal languages, \isi{possessive} constructions (genitival syntagms) and compounds (complex
lexemes) are distinguished by their tone patterns. In Kita \ili{Malinke} (\ili{Mande} branch of Niger-Congo), for instance, ‘the meat of the cow’ is
/\ipa{mìsí sùbû}/, whereas ‘beef, cow meat’ is /\ipa{mìsì-súbú}/ \citep{creisselsetal1993}. The
latter is characterized by tonal compactness (\textit{compacité tonale}): the tone pattern of the
compound is determined by that of its first component, the determiner. Similarly, in Yongning Na, \isi{possessive} constructions do not involve tonal changes, whereas compounds do~-- although the tone changes are more complex than in \ili{Malinke}, as will be explained further below. The two constructions are conspicuously different: in \isi{possessive}
constructions, the \isi{possessive} marker /\ipa{=bv̩˧}/ is inserted after the determiner, before the head noun (e.g.~/\ipa{hwɤ˧li˧˥}/ ‘cat’, /\ipa{ɬv̩˧˥}/ ‘brains’, /\ipa{hwɤ˧li˧=bv̩˥ {\kern2pt}|{\kern2pt} ɬv̩˧˥}/ ‘brains of the
cat’). The determiner and the \isi{possessive} particle form a~single \isi{tone group}, while the head noun retains its tone pattern as it appears \is{form!in isolation}in isolation. By contrast, in compounds, tones are not simply concatenated from their constituent elements. The present
analysis proceeds in increasing order of abstraction, moving from the \is{form!surface}surface phonological patterns of
compounds to their underlying tonal system.

Determinative compounds are sometimes classified into “free” and “fixed” combinations. The former consist of two nouns that are not
habitually associated, such as /\ipa{gi˧nɑ˧mi˧-njɤ˥ɭɯ˩}/ ‘bear’s eye’, where the two nouns are combined into
a~noun phrase only in the context of a~specific utterance. The
latter constitute lexicalized combinations, such as /\ipa{ʑi˩hṽ̩\#˥}/ ‘body hair (of humans)’, literally ‘monkey’s hair’. Cross-linguistically, compounds tend to stray away from regular morphophonological
patterns as well as from the semantics that one would expect on the face of their
constituting elements. However, the degree of divergence may vary: in some cases, the meaning is specialized while the phonological form
remains indistinguishable from that of a~newly coined compound; conversely, the phonological form may
be irregular while the meaning remains fully transparent on a strictly synchronic basis. 

\begin{quotation}
	Thus in \ili{Zarma}, \textit{háw bíì} /ox/black/ is a~syntagm with a~perfectly regular form, which would
	be expected to mean ‘black ox’, but which refers to the buffalo~-- an animal that resembles the ox,
	and whose colour is black. In \textit{cùrò bíì}, one easily recognizes \textit{cúrò} ‘bird’ and
	\textit{bíì} ‘black’, but the meaning is ‘guinea fowl’; in this case, semantic specialization is
	accompanied by a~tonal irregularity: a~syntagm meaning ‘black bird’ would be expected to have the
	form \textit{cúrò bíì}.~\citep[121]{creissels1991}
    
    \medskip 
    %\footnote{
    {\noindent} \textit{Original text:} Ainsi en zarma, \textit{háw bíì} /bœuf/noir/ est un syntagme de
		formation parfaitement régulière dont on attendrait qu’il signifie «~bœuf noir~», mais qui
		désigne le buffle (animal semblable au bœuf et de couleur noire). Dans \textit{cùrò bíì}, nous
		reconnaissons facilement \textit{cúrò} «~oiseau~» et \textit{bíì} «~noir~», mais la signification
		est «~pintade~» ; dans ce cas, le figement sémantique s’accompagne d’une irrégularité tonale : le
		syntagme signifiant «~oiseau noir~» serait \textit{cúrò bíì}.
\end{quotation}

The lack of a straightforward match between semantic regularity and morphophonological
similarity must be taken into account when analyzing a~tone
system. Semantics cannot serve as the sole criterion for identifiying
morphophonologically irregular compounds. The present chapter therefore does not distinguish between “fixed” and “free” combinations but rather between regular
and irregular ones. From a~\is{morphotonology}morphotonological perspective, the relevant parameter is
whether a~compound's tone pattern follows productive rules.

\subsection{The role of the number of syllables}
\label{sec:theroleofthenumberofsyllables}

Tonal changes in compounding are only observed when the second term~-- the head~-- has fewer than
three syllables. That is, tone changes occur in compounds of the form σ+σ, σ+σσ, σσ+σ, σσ+σσ, σσσ+σ or
σσσ+σσ. By contrast, no tone change takes place when the head has three or more syllables (e.g.\ in σ+σσσ, σσ+σσσ or σσσ+σσσ). What matters, then, is not the total number of syllables
in the compound but the number of syllables in the head. This is illustrated is examples (\ref{ex:lugulake}--\ref{ex:bearseye}).

\begin{exe}
  \ex
  \begin{xlist}
    \ex \label{ex:lugulake}
    \ipaex{lo˧ʂv̩˩ {\kern2pt}|{\kern2pt} -hi˩nɑ˧mi\#˥}\\
	\gll lo˧ʂv̩˩		hi˩nɑ˧mi\#˥\\
	Loshu~(village~name)		lake\\
    \glt ‘Lugu Lake’ (\textit{literally:} ‘the lake of Loshu’)
    \ex \label{ex:yongningplain}
    \ipaex{ɬi˧di˩-di˩mi˩}\\
	\gll ɬi˧di˩		di˧mi˧\\
		Yongning~(place~name)		large\_plain\\
    \glt ‘Yongning plain’
    \ex \label{ex:bearseye}
    \ipaex{gi˧nɑ˧mi˧-njɤ˥ɭɯ˩}\\
	\gll gi˧nɑ˧mi\#˥		njɤ˩ɭɯ˧\\
	bear		eye\\
    \glt ‘bear’s eye’
  \end{xlist}
\end{exe}

The place names in (\ref{ex:lugulake}) and (\ref{ex:yongningplain}) share the same syntactic structure: Loshu \ipa{lo˧ʂv̩˩} (Chinese: Luòshuǐ \zh{落水}) is the name of a~village on the shore of Lugu Lake, and \ipa{ɬi˧di˩} is the name of Yongning. In both cases, the relationship is one of determiner and
head: ‘the lake of \ipa{lo˧ʂv̩˩}’ and ‘the plain of \ipa{ɬi˧di˩}’. In (\ref{ex:lugulake}), both constituents retain their lexical tones: /\ipa{lo˧ʂv̩˩}/ ‘Luoshui’ and /\ipa{hi˩nɑ˧mi\#˥}/ ‘lake’. The
compound ‘Lugu Lake’, /\ipa{lo˧ʂv̩˩ {\kern2pt}|{\kern2pt} -hi˩nɑ˧mi\#˥}/, thus consists of two tone
groups (this is indicated by the symbol ‘\ipa{|}’, which stands for a~\isi{tone group} boundary). If it were a single \isi{tone group}, its expected tone pattern would be $\dagger${\kern2pt}\ipa{lo˧ʂv̩˩-hi˩nɑ˩mi˩},
by application of Rule~5: “All syllables following an H.L or M.L sequence receive L tone” (for
a~list of the tone rules, see \sectref{sec:alistoftonerules}). In (\ref{ex:yongningplain}), by contrast, the expected
tone change does take place: the lexical tone of ‘plain’ is M
(/\ipa{di˧mi˧}/), but within this compound, it is lowered to L by application of Rule~5. Example (\ref{ex:bearseye}), from /\ipa{gi˧nɑ˧mi\#˥}/
‘bear’ and /\ipa{njɤ˩ɭɯ˧}/ ‘eye’, shows that tonal change still occurs when the determiner has three syllables, provided that the head has no more than two.

It is clear from the data set out below (in Tables~\ref{tab:surfacemonosyllabicmonosyllables} to \ref{tab:surfacetriisyllabic}) that heads undergo more tonal changes than determiners in compounding. In particular, there
are numerous cases where an H tone that is lexically associated with the determiner shifts to
the final syllable of the head. When this occurs in a~σσ+σ compound, the distance between the H tone's original position and its surface position in the compound is no greater than one syllable. In a~σσ+σσ compound, this distance increases to two syllables. 
%(In this rule-of-thumb calculation, the H tone is considered to be associated to the lexical word's last syllable: remember that H tones never appear on a~word-initial syllable.) 
If the same process applied to a σσ+σσσ compound, the H tone would move three syllables away from its original lexical position, making the shift significantly more complex. This is by no means a~cognitive impossibility: highly intricate tonal phenomena are firmly attested cross-linguistically, and by some estimates the Alawua dialect of Yongning Na qualifies as a~complex system (see \sectref{sec:morphophonologicalcomplexity}). 
%Nevertheless, the observed pattern, whereby the compound is divided into two parts, suggests a~strategy to bypass tonal computation and thus place a limit on generative complexity. 
Nevertheless, the observed pattern, whereby the compound is divided into two parts, appears to reflect a~tendency for tonal computation to be circumvented, thereby placing a limit on generative complexity.
The asymmetry, whereby σσσ+σσ compounds undergo tonal change while σσ+σσσ compounds do not, makes intuitive sense in light of the greater computational load required for the latter. 

There is thus a~preference (in this particular language and dialect) for tonal processes that do not involve tonal shifts of more than two syllables at a~time. Long-distance tone movement is avoided. This observation will be taken up in \sectref{sec:longdistancedispreferred}. 


\subsection{How the tone patterns were collected}
\label{sec:howthetonepatternswerecollected}
\largerpage[-1] %longdistance
Some lexicalized compounds appear in narratives, such as /\ipa{ə˧mi˧-ʁæ˧ʈv̩˥}/ {\linebreak}‘mother’s neck’ in \textit{Tiger2.86} \pandoi{0004545\#S86}. Others were encountered during vocabulary elicitation sessions and are consigned in the dictionary of Yongning Na \citep{michaud_et_al_na_dict_2024}. To obtain a comprehensive set of tonal combinations for determiners and heads,
systematic elicitation was also conducted. 

The main language consultant, Mrs.\ Latami Dashilamu (F4), was reluctant to
accept semantically implausible combinations. Over time, she came to understand that the unusual pairings I proposed were intended to elicit specific tonal sequences, yet she remained firmly committed to common-sense usage. For instance, compounds such as ‘flea’s back’ and ‘flea’s liver’ were deemed acceptable~--they did not stretch
plausibility too far in the consultant's view~--, and I gratefully recorded them. But she would not have accepted
combinations as far-fetched as ‘chief’s beetle’s kidney basket’~-- an example from \citet{hyman2007a}, later discussed by \citet[225]{evans2010}. 

Compounds that did not make sufficiently good sense in her view, such as ‘woman’s blood’, were produced either
as \isi{possessive} constructions~-- an {English} equivalent would be ‘blood of a/the woman’, as opposed to the
desired ‘woman’s blood’~-- or else as ungrammatical juxtapositions of citation forms. 

Nevertheless, it was possible to elicit all tonal combinations in the end, by searching through the lexicon for the least implausible pairings and discussing possible
contexts with the consultant. In the case of ‘woman’s blood’, her argument was that there is nothing inherently distinct about the blood of women, as opposed to that of men: it is simply human blood. To work around this, I imagined a scenario in which a~man-eating demon craves \textit{woman’s} \textit{blood}. 

Each combination was then checked multiple times by using different lexical items with the same
tones and by eliciting tokens across elicitation sessions. 
This cautious approach made the elicitation process slower than it might have been with a~consultant willing to generate arbitrary combinations among nouns. However, such conservative behaviour has its advantages:
%The elicitation of data from four speakers about this particular point of the tonal
%grammar extended over the first three field trips, from 2006 to 2008. 
%A~consultant’s {conservative} behaviour may have some
%advantages, however: 
one might suspect that a~consultant whose imagination roamed entirely free
from the trammels of common sense could occasionally take similar liberties with the language’s
tonal system, too.

Transcribed recordings of over 1,500 compounds are available online from the Pangloss Collection (references: \textit{DetermCompounds1} \pandoi{0004450} through \textit{DetermCompounds16} \pandoi{0004491}). For each compound, these recordings provide the input tones and the lexical forms of both the determiner and the head, as illustrated in \figref{fig:CompOnPangloss}. In all cases where examples also
appear in texts, the tone
patterns obtained through elicitation are identical with those found in spontaneous speech. \tabref{tab:examplewordsusedtoelicitbodypartcompoundnouns} provides an example noun for each tonal category used in the elicitation of body-part compounds.

\begin{figure}
	\includegraphics[width=\textwidth]{figures/CompOnPangloss.png}
	\caption{First lines of the document \textit{DetermCompounds1} \pandoi{0004450} as displayed in the online interface.}
	\label{fig:CompOnPangloss}
\end{figure}

\begin{table}[t]
\caption{Example words used to elicit body-part compound nouns.}
\begin{tabularx}{\textwidth}{ l Q Q Q l }
\lsptoprule
	tone & determiners & meaning & heads & meaning\\\midrule
	LM & \ipa{bo˩˧} & pig & \ipa{ɣɯ˩˧} & skin\\
	M & \ipa{lɑ˧} & tiger & \ipa{bv̩˧} & intestine\\
	L & \ipa{jo˩} & sheep & \ipa{mɤ˩} & fat\\
	\#H & \ipa{ʐwæ˥} & horse & \ipa{sɤ˥} & blood\\
	MH\# & \ipa{ʈʂʰæ˧˥} & deer & \ipa{ɬv̩˧˥} & brains\\ \addlinespace \hdashline \addlinespace
	M & \ipa{po˧lo˧} & ram & \ipa{gv̩˧dv̩˧} & back\\
	\#H & \ipa{ʐwæ˧zo\#˥} & colt & \ipa{ɲi˧gɤ\#˥} & nose, snout\\
	MH\# & \ipa{hwɤ˧li˧˥} & cat & \ipa{qv̩˧ʈʂæ˧˥} & throat\\
	H\$ & \ipa{hwɤ˧mi˥\$} & she-cat & \ipa{hu˧mi˥\$} & stomach\\
	L & \ipa{kʰv̩˩mi˩} & dog & \ipa{nv̩˩mi˩} & heart\\
	L\# & \ipa{dɑ˧ʝi˩} & mule & \ipa{ɬi˧pi˩} & ear\\
	LM+MH\# & \ipa{õ˩dv̩˧˥} & wolf & \ipa{ʝi˩ʈʂæ˧˥} & waist\\
	LM+\#H & \ipa{nɑ˩hĩ\#˥} & Naxi person & \ipa{njæ˩qʰæ\#˥} & eye sand, rheum\\
	LM & \ipa{æ˩mi˧} & hen & \ipa{njɤ˩ɭɯ˧} & eye\\
	LH & \ipa{bo˩ɬɑ˥} & boar & \ipa{hi˩ʐæ˥} & uvula\\
	H\# & \ipa{hwæ˧tsɯ˥} & rat & \ipa{ʁæ˧ʈv̩˥} & neck\\
\lspbottomrule
\end{tabularx}
\label{tab:examplewordsusedtoelicitbodypartcompoundnouns}
\end{table}

\subsection[Surface phonological tone patterns]{The facts: Surface phonological tone patterns}
\label{sec:thefacts}

The tone patterns of compound nouns in Alawua are set out in Tables~\ref{tab:surfacemonosyllabicmonosyllables} to \ref{tab:surfacetriisyllabic} as a~function of the tones
of their constituent elements. The leftmost column indicates the tone of the determiner, while the top row indicates the tone of the head. For example, ‘tiger’ /\ipa{lɑ˧}/ has lexical M, and ‘skin’ /\ipa{ɣɯ˩˧}/ has lexical LM. The tone of the compound ‘tiger’s skin’ can be found at the intersection of row M and column LM in Table~\ref{tab:surfacemonosyllabicmonosyllables}. The cell at this intersection contains ‘M.L’, indicating 
that the surface phonological tone pattern of the compound is M.L: /\ipa{lɑ˧-ɣɯ˩}/ ‘tiger’s skin’. Tables~\ref{tab:surfacemonosyllabicmonosyllables} and
\ref{tab:surfacemonosyllabicdisyllables} present data for compounds with monosyllabic heads,
and Tables~\ref{tab:surfacedisyllabicmonosyllables} and \ref{tab:surfacedisyllabicdisyllables} cover disyllabic
heads. 

\begin{table}%[b]
\caption{\label{tab:surfacemonosyllabicmonosyllables}Surface phonological representation of the tones of
  compound nouns. Monosyllabic determiner and monosyllabic head. Leftmost column: tone of determiner; top row: tone of head.}
{\renewcommand{\arraystretch}{1.2}
\begin{tabularx}{\textwidth}{ Q Q l Q P{21mm} Q }
\lsptoprule
	tone & LH; LM & M & L & H & MH\\ \midrule
	LM & \tikzmark{0a}L.M &  & \hspace*{\fill}\tikzmark{0e} & \tikzmark{1a}L.M, L.M.H & \tikzmark{2a}L.MH\\
	LH & L.H & L.L & L.H & \hspace*{\fill}\tikzmark{1e} &  \hspace*{\fill}\tikzmark{2e}\\
	M & M.L & M.M, M.M.H & M.L & M.M, M.M.H & M.MH\\
	L & \tikzmark{3a}L.LH &  &  &  &  \hspace*{\fill}\tikzmark{3e}\\
	H & M.H & \tikzmark{4a}M.M, M.M.H &  & \hspace*{\fill}\tikzmark{4e} & M.L\\
	MH & \tikzmark{5a}M.H &  &  \hspace*{\fill}\tikzmark{5e} & \tikzmark{6a}M.H, M.M.H & \hspace*{\fill}\tikzmark{6e}\\
\lspbottomrule
\end{tabularx}}
\DrawBox{0a}{0e}
\DrawBox{1a}{1e}
\DrawBox{2a}{2e}
\DrawBox{3a}{3e}
\DrawBox{4a}{4e}
\DrawBox{5a}{5e}
\DrawBox{6a}{6e}
\end{table}

\begin{table}%[p]%[t] % previously: tb
\caption{\label{tab:surfacemonosyllabicdisyllables}Surface phonological representation of the tones of
  compound nouns. Disyllabic determiner and monosyllabic head. Leftmost column: tone of determiner; top row: tone of head.}
{\renewcommand{\arraystretch}{1.2}
\begin{tabularx}{\textwidth}{ l Q Q Q P{21mm} Q }
\lsptoprule
	tone & LH; LM & M & L & H & MH\\ \midrule
	M & M.M.L & M.M.M, M.M.M.H & M.M.L & M.M.M, M.M.M.H & M.M.L\\
	\#H & M.M.H & \tikzmark{1a}\hbox{M.M.M, M.M.M.H} &  & \hspace*{\fill}\tikzmark{1e} & M.M.L\\
	MH\# & M.M.H & \tikzmark{2a}M.M.MH &  &  \hspace*{\fill}\tikzmark{2e} & M.M.H\\
	H\$ & M.M.H & M.M.M, M.M.M.H & M.M.H, M.M.M.H & M.M.M, M.M.M.H & M.H.L\\
	L & L.L.H & \tikzmark{3a}L.L.LH &  & \hspace*{\fill}\tikzmark{3e} & L.L.H\\
	L\# & \tikzmark{4a}M.L.L &  &  &  &  \hspace*{\fill}\tikzmark{4e}\\
	LM+MH\# & L.M.H & L.M.MH & \tikzmark{5a}\hbox{L.M.H, L.M.M.H} & & \hspace*{\fill}\tikzmark{5e}\\
	LM+\#H & L.M.H & L.M.M, L.M.M.H & L.M.H & L.M.M, L.M.M.H & L.M.H\\
	LM & L.M.L & L.M.M & L.M.L & L.M.M, L.M.M.H & L.M.MH\\
	LH & \tikzmark{6a}L.H.L &  &  &  & \hspace*{\fill}\tikzmark{6e}\\
	H\# & \tikzmark{7a}M.H.L &  &  &  &  \hspace*{\fill}\tikzmark{7e}\\
\lspbottomrule
\end{tabularx}}
\DrawBox[dashed]{1a}{1e}
\DrawBox[dashed]{2a}{2e}
\DrawBox[dashed]{3a}{3e}
\DrawBox[dashed]{4a}{4e}
\DrawBox[dashed]{5a}{5e}
\DrawBox[dashed]{6a}{6e}
\DrawBox[dashed]{7a}{7e}
\end{table}




As with simple nouns, compounds must be elicited in at least two contexts to determine their lexical tone
categories, since oppositions such as \mbox{//M//} vs.\ \mbox{//\#H//} are neutralized \is{form!in isolation}in isolation. 
Compound nouns were therefore elicited both (i)~\is{form!in isolation}in isolation and (ii)~in a carrier frame: (\ref{ex:carrierthisisatheREP}). 

\clearpage

\begin{sidewaystable}%[p!]
\caption{\label{tab:surfacedisyllabicmonosyllables}Surface phonological representation of the tones of compound nouns. Monosyllabic determiner and disyllabic head. Leftmost column: tone of determiner; top row: tone of head.}
{\renewcommand{\arraystretch}{1.35}
\begin{tabularx}{\textwidth}{ Q Q Q Q Q Q Q Q Q }
\lsptoprule
	tone & M & \#H & MH\# & H\$ & L & L\# & LM+MH\#; LM+\#H; LM; LH & H\#\\\midrule
	LM; LH & L.M.M & L.M.M, L.M.M.H & L.M.MH~/ L.H.L & L.M.H, L.M.M.H & L.M.L & L.M.L~/ L.L.H & L.H.L ($=$L.M.L) & L.M.H~/ L.L.H\\
	M & M.M.M & M.M.M, M.M.M.H & M.M.MH & M.M.H, M.M.M.H & M.L.L & M.M.L & M.L.L & M.M.H\\
	L & \tikzmark{1a}L.L.LH &  &  \hspace*{\fill}\tikzmark{1e} & L.L.H & L.L.LH & L.L.H & L.H.L & L.L.H\\
	\#H & M.M.H & M.M.M, M.M.M.H & M.H.L & M.L.L~/ M.M.H & M.H.L &
   M.M.H & M.H.L & M.M.H\\
	MH\# & M.M.H & M.M.M, M.M.M.H & M.M.MH & M.H.L & M.H.L & M.M.H & M.H.L & M.M.H\\
\lspbottomrule
\end{tabularx}}
\DrawBox[dashed]{1a}{1e}
\end{sidewaystable}



\begin{sidewaystable}[p]
\caption{\label{tab:surfacedisyllabicdisyllables}Surface phonological tones of compound nouns. Disyllabic determiner and disyllabic head. Leftmost column: tone of determiner; top row: tone of head.}
{\renewcommand{\arraystretch}{1.6}
{\fontsize{9}{10.75}\selectfont
\begin{tabularx}{\textwidth}{ l P{16mm} Q Q P{19mm} P{15mm} Q P{23mm} P{16mm} }
\lsptoprule
	tone & M & \#H & MH\# & H\$ & L & L\# & LM+MH\#;\par LM+\#H; LM; LH & H\#\\\midrule
	M & M{\dots}M & M{\dots}M,\par M{\dots}M.H & M{\dots}MH & M.M.M.H,\par M{\dots}M.H~/ M.M.H.L & M.M.L.L & M.M.M.L & M.M.L.L & M.M.M.H\\
	\#H & \tikzmark{1a}M.M.M.H & M{\dots}M,\par M{\dots}M.H & M.M.H.L & M.M.M.H,\par M{\dots}M.H~/ M.M.H.L~/ M.M.M.H & M.M.H.L & M.M.M.H & M.M.H.L & M.M.M.H\\
	MH\# &  & M{\dots}MH & M.M.H.L & M.M.H.L~/ M.M.M.H & M.M.H.L & M.M.M.H & M.M.H.L & M.M.M.H\\
	H\$ &  ~\par\hspace*{\fill}\tikzmark{1e} & M{\dots}M,\par M{\dots}M.H & \tikzmark{2a}\hbox{M.M.H.L~/ M.H.L.L} & ~\par\hspace*{\fill}\tikzmark{2e} & M.M.H.L & M.M.M.H & M.M.H.L & M.M.M.H\\
	L & L.L.L.H & L.L.L.LH & L.L.H.L & L.L.L.H & L.L.H.L & L.L.L.H & L.L.H.L & L.L.L.H\\
	L\# & M.L.L.L & \tikzmark{3a}M.L.L.L &  &  &  &  &  &  \hspace*{\fill}\tikzmark{3e}\\
	L+MH\# & \tikzmark{4a}L.M.M.H & \tikzmark{5a}L.M.M.M, L.M.M.M.H & \tikzmark{6a}L.M.H.L &  & ~\par\hspace*{\fill}\tikzmark{6e} & \tikzmark{7a}L.M.M.H & \tikzmark{8a}L.M.H.L & \tikzmark{9a}L.M.M.H\\
	LM+\#H &  \hspace*{\fill}\tikzmark{4e} &  & L.M.H.L & L.M.M.H, &
   L.M.H.L &  \hspace*{\fill}\tikzmark{7e} & \hspace*{\fill}\tikzmark{8e} &\\
	LM & L.M.M.M & \hspace*{\fill}\tikzmark{5e} & L.M.M.MH & L.M.M.M.H & L.M.L.L & L.M.M.L & L.M.L.L & \hspace*{\fill}\tikzmark{9e}\\
	LH & \tikzmark{10a}L.H.L.L &  &  &  &  &  &  & \hspace*{\fill}\tikzmark{10e}\\
	H\# & \tikzmark{11a}M.H.L.L &  &  &  &  &  &  &
   \hspace*{\fill}\tikzmark{11e}\\
\lspbottomrule
\end{tabularx}}
\DrawBox[dashed]{1a}{1e}
\DrawBox[dashed]{2a}{2e}
\DrawBox[dashed]{3a}{3e}
\DrawBox[dashed]{4a}{4e}
\DrawBox[dashed]{5a}{5e}
\DrawBox[dashed]{6a}{6e}
\DrawBox[dashed]{7a}{7e}
\DrawBox[dashed]{8a}{8e}
\DrawBox[dashed]{9a}{9e}
\DrawBox[dashed]{10a}{10e}
\DrawBox[dashed]{11a}{11e}}
\end{sidewaystable}


\begin{table}%[t]
\caption{\label{tab:surfacetriisyllabic}Surface phonological
  representation of the tones of compound nouns. Compounds with a~{trisyllabic} \#H-tone determiner: ‘bear’+body
  part.}
{\setlength\tabcolsep{5pt}
\begin{tabularx}{\textwidth}{ l l l Q }
\lsptoprule
	\multicolumn{2}{l}{head} & \multicolumn{2}{l}{compound}\\ \cmidrule(r){1-2}\cmidrule(l){3-4}
	form  & tone & surface form & surface tone\\ \midrule
	\ipa{ɣɯ˩˧} ‘skin’ & LM & \ipa{gi˧nɑ˧mi˧-ɣɯ˥} & M.M.M.H, M.M.M.H.L\\
	\ipa{bv̩˧} ‘intestine’ & M & \ipa{gi˧nɑ˧mi˧-bv̩˧} & M.M.M.M, M{\dots}M.H\\
	\ipa{mɤ˩} ‘grease’ & L & \ipa{gi˧nɑ˧mi˧-mɤ˧˥} & M.M.M.MH\\
	\ipa{sɤ˥} ‘blood’ & H & \ipa{gi˧nɑ˧mi˧-sɤ˧} & M.M.M.M, M{\dots}M.H\\
	\ipa{ɬv̩˧˥} ‘brains’ & MH & \ipa{gi˧nɑ˧mi˧-ɬv̩˩} & M.M.M.L\\ \addlinespace \hdashline \addlinespace
	\ipa{gv̩˧dv̩˧} ‘back’ & M & \ipa{gi˧nɑ˧mi˧-gv̩˧dv̩˧} & M.M.M.M.M, M{\dots}M.H\\
	\ipa{ɲi˧gɤ˧} ‘nose’ & \#H & \ipa{gi˧nɑ˧mi˧-ɲi˧gɤ˧} & M.M.M.M.M, M{\dots}M.H\\
	\ipa{qv̩˧ʈʂæ˧˥} ‘throat’ & MH\# & \ipa{gi˧nɑ˧mi˧-qv̩˥ʈʂæ˩} & M.M.M.H.L\\
	\ipa{hu˧mi˥\$} ‘stomach’ & H\$ & \ipa{gi˧nɑ˧mi˧-hu˧mi˥} & M.M.M.M.H, M{\dots}M.H\\
	\ipa{nv̩˩mi˩} ‘heart’ & L & \ipa{gi˧nɑ˧mi˧-nv̩˥mi˩} & M.M.M.H.L\\
	\ipa{ɬi˧pi˩} ‘ear’ & L\# & \ipa{gi˧nɑ˧mi˧-ɬi˧pi˥} & M.M.M.M.H\\
	\ipa{ʝi˩ʈʂæ˧˥} ‘waist’ & LM+MH\# & \ipa{gi˧nɑ˧mi˧-ʝi˥ʈʂæ˩} & M.M.M.H.L\\
	\ipa{njɤ˩ɭɯ˧} ‘eye’ & LM & \ipa{gi˧nɑ˧mi˧-njɤ˥ɭɯ˩} & M.M.M.H.L\\
	\ipa{ʁæ˧ʈv̩˥} ‘neck’ & H\# & \ipa{gi˧nɑ˧mi˧-ʁæ˧ʈv̩˥} & M.M.M.M.H\\
\lspbottomrule
\end{tabularx}}
\end{table}





This elicitation procedure is the same as that applied earlier to simple nouns (see \sectref{sec:overviewofthesystem}). Accordingly, frame (\ref{ex:carrierthisisatheREP}) is reproduced from example (\ref{ex:carrierthisisathe}) of Chapter~\ref{chap:thelexicaltonesofnouns}. The lexical tones of the {demonstrative} and the {copula} are indicated in the interlinear glosses, following the findings of Chapter~\ref{chap:thelexicaltonesofnouns}. The tone-group {boundary} separating the {demonstrative} from what follows is also indicated, by means of the vertical bar ‘\ipa{|}’. No tone is given for the {copula} in the surface phonological representation (first line) because its surface tone varies according to the tone category of the target noun.

 \begin{exe}
 	\ex
 	\label{ex:carrierthisisatheREP}
	\ipaex{ʈʂʰɯ˧ {\kern2pt}|{\kern2pt} {\_\_\_\_\_\_\_\_\_} {\kern2pt}ɲi.}\\
 	\gll ʈʂʰɯ˥ {\_\_\_\_\_\_\_\_\_} ɲi˩\\
 	\textsc{dem.prox} \textit{{target item}}	\textsc{cop}\\
 	\glt	‘This is \mbox{(a/the)} \ipa{{\_\_\_\_\_\_\_\_\_}}.’
 \end{exe}

% \newpage 
In cases where the \isi{copula} surfaces with its lexical L tone, a single tone pattern is recorded in the corresponding table cell. For instance, ‘tiger’s skin’ /\ipa{lɑ˧-ɣɯ˩}/ (tone: /M.L/) yields  /\ipa{lɑ˧-ɣɯ˩ ɲi˩}/ ‘is tiger’s skin’, with L tone on the copula. The table thus simply indicates ‘M.L’, with no additional notation, meaning that the pattern remains unchanged upon addition of the \isi{copula}.

Conversely, in cases where the \isi{copula} bears a~tone other than its lexical L, the altered tonal string is noted after a comma.
%that is obtained when adding the \isi{copula} is indicated after a~comma. 
For instance, the information ‘M.M, M.M.H’ provided at the intersection of row M and column M in Table~\ref{tab:surfacemonosyllabicmonosyllables} indicates that the tonal string of the
compound is M.M when said \is{form!in isolation}in isolation, e.g.~/\ipa{lɑ˧-bv̩˧}/ ‘tiger’s intestine’, but that
adding the \isi{copula} yields M.M.H: /\ipa{lɑ˧-bv̩˧ ɲi˥}/ ‘is tiger’s intestine’.

In \tabref{tab:surfacemonosyllabicmonosyllables}, there are no examples of simple M.M: all disyllabic M.M compounds plus {copula} yield M.M.H, so that the ‘M.M.H’ part in ‘M.M, M.M.H’ may seem redundant. However, this is not superfluous, since in Yongning Na, a surface M.M sequence always requires disambiguation: /M.M/ constitutes the {neutralization} of underlying \mbox{//M//} and \mbox{//\#H//}, exemplified in Chapter~\ref{chap:thelexicaltonesofnouns} by //\ipa{po˧lo˧}// ‘ram’ (M) and //\ipa{ʐwæ˧zo\#˥}// ‘colt’ (\#H).

When tonal variants exist, alternatives are separated by slashes. For example, in row \#H, column H\$ of \tabref{tab:surfacedisyllabicmonosyllables}, the entry ‘M.L.L~/ M.M.H’  indicates that
compounds in this category may have either of these two patterns, e.g. /\ipa{ʐwæ˧-hu˩mi˩}/ or
/\ipa{ʐwæ˧-hu˧mi˥}/ for ‘horse’s stomach’. 

\
To enhance readability, sequences of four M tones (M.M.M.M) are
abbreviated as ‘M{\dots}M’. Additionally, adjacent cells with identical contents are
grouped using dashed-line boxes. For instance, since all disyllabic compounds with an L-tone determiner follow the pattern /L.LH/, the entire ‘L' row in Table~\ref{tab:surfacemonosyllabicmonosyllables} is enclosed in a single box labelled ‘L.LH'. However, the process of grouping cells has not been extended indiscriminately. For example, in the bottom left corner of Table~\ref{tab:surfacemonosyllabicmonosyllables}, the ‘H’ and ‘MH’ rows were not merged, as the tonal pattern /M.H/ likely arises for different reasons in these two cases (\sectref{sec:Htonedems} and \sectref{sec:MHtonedems}). 




The purpose of Tables~\ref{tab:surfacemonosyllabicmonosyllables} to \ref{tab:surfacetriisyllabic} is to set out the facts in a~legible and unambiguous manner. This is a step towards the long-term goal of developing a~full-fledged linguistic model, incorporating novel approaches to capturing both regularities and irregularities in paradigms
\citep{sagotetal2013,bonami_computational_2017}. 

In view of the rarity of trisyllabic nouns, only one trisyllabic determiner was included in the data: /\ipa{gi˧nɑ˧mi\#˥}/ ‘bear’ (tone: \#H). The data is set out in
\tabref{tab:surfacetriisyllabic}.

\subsection{Analysis of underlying tone patterns}
\label{sec:analysisintounderlyingtonepatterns}

The surface\is{form!surface} tonal sequences observed in compounds, as reported in Tables~\ref{tab:surfacemonosyllabicmonosyllables} to \ref{tab:surfacetriisyllabic}, can in most cases be reduced to simple tone categories, such as //L//, \mbox{//LM//} or \mbox{//LH//}, by analyzing them in light of the phonological \is{tone rules}tone rules set out in \sectref{sec:alistoftonerules}. The following two examples illustrate this process. 



\begin{itemize}
	\item{The
		sequence /L.LH/ can be interpreted as the realization of a~simple //L// tone: it spreads over the two
		syllables of the compound, yielding //L.L//, and this sequence is further supplemented by a~postlexical
		H tone due to Rule~7 (“If a~{tone group} only contains L tones, a~postlexical H tone is added to its
		last syllable”).}
	\item{The sequence /M.M/ in /\ipa{lɑ˧-bv̩˧}/ ‘tiger’s intestine’ could reflect an underlying \mbox{//M//} or \mbox{//\#H//} tone pattern (recall that \mbox{//\#H//} is a floating H tone, which is only realized after the lexical item at issue). The fact
		that the \isi{copula} receives an H tone when following this compound (/\ipa{lɑ˧-bv̩˧ ɲi˥}/ ‘is tiger’s intestine’) reveals that
		the \is{form!underlying}underlying tone pattern is \#H.}
\end{itemize}

The results of this analysis are presented in Tables~\ref{tab:abstractmonosyllabicmonosyllables} to \ref{tab:abstracttrisyllabic}, which provide all the information
needed to generate the \is{form!surface}surface phonological patterns of compounds, following the standard tone-to-syllable association
rules.






In describing these patterns, reference must be made to a~\is{juncture (inside a tone group)}juncture internal to the
\isi{tone group}, which separates the determiner from the head. This \is{juncture (inside a tone group)}juncture is indicated by the
symbol ‘--’, which was already used in the description of numeral-plus-classifier phrases in Chapter~\ref{chap:classifiers}. Thus, --L refers to an L tone attaching to the second part of an expression. 

For example, the output --L indicated in \tabref{tab:abstractdisyllabicdisyllables} for a~disyllabic determiner carrying M tone (first row) and a~disyllabic head carrying L tone (sixth column) means that the determinative compound carries L tone on its second part, i.e.\ after the \is{juncture (inside a tone group)}juncture between the two nouns. Thus, the combination of /\ipa{po˧lo˧}/ ‘ram’ and /\ipa{nv̩˩mi˩}/ ‘heart’ yields a~compound with L tone on its second part, hence /\ipa{po.lo.nv̩˩mi}/ (L tone associates to the first syllable after the \is{juncture (inside a tone group)}juncture between the two nouns). 

The association of tones to the other three syllables then follows the regular \is{tone rules}phonological tone rules of Yongning Na. First, the L tone spreads onto the following syllable (by Rule~1), yielding /\ipa{po.lo.nv̩˩mi˩}/. Then the first two syllables receive M tone (by Rule~2), producing /\ipa{po˧lo˧-nv̩˩mi˩}/. 

% \begin{itemize}
%     \item The L tone spreads onto the following syllable (by Rule~1), yielding /\ipa{po.lo.\-nv̩˩\-mi˩}/.
%     \item The first two syllables receive M tone (by Rule~2), producing /\ipa{po˧lo˧-nv̩˩mi˩}/. 
% \end{itemize}
A~step-by-step representation of this process is provided in \figref{fig:toneLround}. A~similar representation is shown in \figref{fig:tonepoundHround} for \#H-- (a~\is{floating tone}floating H tone attaching to the first part of the expression). In this case, the \#H tone is inserted before the \is{juncture (inside a tone group)}juncture between the two nouns, i.e.\ it associates to the second syllable of the compound. From there: 
\begin{itemize}
    \item The H tone attaches to the following syllable (this is the defining property of the \#H tone category, which associates \textit{after} the word to which it is lexically attached), i.e.\ on the third syllable. 
    \item The first and second syllables receive M tone by Rule~2. 
    \item The fourth syllable receives L tone by Rule~4.
\end{itemize}

\begin{figure}
	\caption{First illustration of the anchoring of tones relative to a~morpheme break inside a~complex expression (notation: ‘--’): step-by-step representation of tonal association of the --L tone of the compound /\ipa{po˧lo˧-nv̩˩mi˩}/ ‘rat's stomach'.}
	\begin{tikzpicture}
	\node (1) at (0.2,-0.5) {M};
	\node (4) at (3.5,-0.5) {L};
	
	\node (2) at (-0.3,-1.5) {σ};
	\node (3) at (0.7,-1.5) {σ};
	\node (5) at (3,-1.5) {σ};
	\node (6) at (4,-1.5) {σ};
	
	\node [anchor=mid] (s1l) at (0.2,-2) {/\ipa{po.lo}/ ‘ram’};
	%  \node (s1ll) at (0.5,-2.5) {lexical tone: MH\#};
	
	\node [anchor=mid] (s1lll) at (3.6,-2) {/\ipa{nv̩.mi}/ ‘heart'};
	%  \node (s1llll) at (4,-2.5) {lexical tone: L};
	
	\node[text width=40mm] (s1) at (-3,-0.75) {Stage 1:\\ input nouns};
	
	
	
	\node (12) at (1.5,-4) {--L};
	
	\node (22) at (0,-5.5) {σ};
	\node (32) at (1,-5.5) {σ};
	\node (52) at (2,-5.5) {σ};
	\node (7) at (3,-5.5) {σ};
	
	\node[text width=40mm] (s2) at (-3,-4.75) {Stage 2:\\ \is{morphotonology}morphotonological rules\\ determine the tone\\ of the compound\\ (see \tabref{tab:abstractdisyllabicdisyllables})};
	
	
	
	
	\node (9) at (1.5,-7) {L};

	\node (23) at (0,-8.5) {σ};
	\node (33) at (1,-8.5) {σ};
	\node (53) at (2,-8.5) {σ};
	\node (8) at (3,-8.5) {σ};
	
	\node[text width=40mm] (s3) at (-3,-7.75) {Stage 3:\\ association of L tone\\ to its specified locus:\\ after the \is{juncture (inside a tone group)}juncture\\between the two nouns};
	
	% arrow from L tone: 
	\draw[decoration={markings,mark=at position 1 with
		{\arrow[scale=2,>=stealth]{>}}},postaction={decorate}] (9) -- (53);
	
	
	\node (44) at (2,-10) {L};
	
	\node (24) at (0,-11.5) {σ};
	\node (34) at (1,-11.5) {σ};
	\node (54) at (2,-11.5) {σ};
	\node (8) at (3,-11.5) {σ};
	
	\node[text width=40mm] (s4) at (-3,-10.5) {Stage 4:\\ assignment of L tone\\ by {phonological rule}:\\ L-tone spreading \\ (Rule 1)};
	
	\draw (44) -- (54);	
	\draw[decoration={markings,mark=at position 1 with
		{\arrow[scale=2,>=stealth]{>}}},postaction={decorate}] (44) -- (8);	
	
	\node (14) at (0,-13) {M};
	\node (64) at (1,-13) {M};
	\node (44) at (2,-13) {L};
	\node (91) at (3,-13) {L};
	
	\node (24) at (0,-14.5) {σ};
	\node (34) at (1,-14.5) {σ};
	\node (54) at (2,-14.5) {σ};
	\node (92) at (3,-14.5) {σ};
	
	\node[text width=40mm] (s4) at (-3,-13.5) {Stage 5:\\ addition of M tones\\ to toneless syllables\\ (Rule~2)};
	
	\draw[decoration={markings,mark=at position 1 with
		{\arrow[scale=2,>=stealth]{>}}},postaction={decorate}] (14) -- (24);	
	\draw[decoration={markings,mark=at position 1 with
		{\arrow[scale=2,>=stealth]{>}}},postaction={decorate}] (64) -- (34);	
	\draw (44) -- (54);
	\draw (91) -- (92);
	\end{tikzpicture}
	\label{fig:toneLround}
\end{figure}


\begin{figure}
	\caption{Second illustration of the anchoring of tones relative to a~morpheme break inside a~complex expression (notation: ‘--’): step-by-step representation of tonal association of the \#H-- tone of the compound /\ipa{hwɤ˧li˧-qv̩˥ʈʂæ˩}/ ‘cat's sound / cat sounds'.}
	\begin{tikzpicture}
	\node (1) at (0.2,-0.5) {M};
	\node (4) at (3.5,-0.5) {L};
	
	\node (2) at (-0.3,-1.5) {σ};
	\node (3) at (0.7,-1.5) {σ};
	\node (5) at (3,-1.5) {σ};
	\node (6) at (4,-1.5) {σ};
	
	\node [anchor=mid] (s1l) at (0.2,-2) {/\ipa{hwɤ.li}/ ‘cat’};
	%  \node (s1ll) at (0.5,-2.5) {lexical tone: MH\#};
	
	\node [anchor=mid] (s1lll) at (3.6,-2) {/\ipa{qv̩.ʈʂæ}/ ‘sound'};
	%  \node (s1llll) at (4,-2.5) {lexical tone: L};
	
	\node[text width=40mm] (s1) at (-3,-0.75) {Stage 1:\\ input nouns};
	
	
	
	\node (12) at (1.5,-4) {\#H--};
	
	\node (22) at (0,-5.5) {σ};
	\node (32) at (1,-5.5) {σ};
	\node (52) at (2,-5.5) {σ};
	\node (7) at (3,-5.5) {σ};
	
	\node[text width=40mm] (s2) at (-3,-4.75) {Stage 2:\\ \is{morphotonology}morphotonological rules\\ determine the tone\\ of the compound\\ (see \tabref{tab:abstractdisyllabicdisyllables})};
	
	
	\node (9) at (1.5,-7) {\#H};
	
	\node (23) at (0,-8.5) {σ};
	\node (33) at (1,-8.5) {σ};
	\node (53) at (2,-8.5) {σ};
	\node (8) at (3,-8.5) {σ};
	
	\node[text width=40mm] (s3) at (-3,-7.75) {Stage 3:\\ the \is{floating tone}floating H tone\\ (\#H) is inserted\\ before the \is{juncture (inside a tone group)}juncture\\between the two nouns};
	
	% arrow from L tone: 
	\draw[decoration={markings,mark=at position 1 with
		{\arrow[scale=2,>=stealth]{>}}},dashed,postaction={decorate}] (9) -- (33);
	
	% attempting a bent line
	%	\draw [->] (33.north) to [out=150,in=30] (53.north)
	
	\node (44) at (1,-10) {H};
	
	\node (24) at (0,-11.5) {σ};
	\node (34) at (1,-11.5) {σ};
	\node (54) at (2,-11.5) {σ};
	\node (8) at (3,-11.5) {σ};
	
	\node[text width=40mm] (s4) at (-3,-10.5) {Stage 4:\\ the floating H tone\\ gets anchored\\to the next syllable};
	
	\draw[decoration={markings,mark=at position 1 with
		{\arrow[scale=2,>=stealth]{>}}},postaction={decorate}] (44) -- (54);	
	
	\node (14) at (0,-13) {M};
	\node (64) at (1,-13) {M};
	\node (44) at (2,-13) {H};
	\node (91) at (3,-13) {L};
	
	\node (24) at (0,-14.5) {σ};
	\node (34) at (1,-14.5) {σ};
	\node (54) at (2,-14.5) {σ};
	\node (92) at (3,-14.5) {σ};
	
	\node[text width=40mm] (s4) at (-3,-13.5) {Stage 5:\\ addition of L after H\\ (by Rule~4),\\ and of M tones\\ to toneless syllables\\ (by Rule~2)};
	
	\draw[decoration={markings,mark=at position 1 with
		{\arrow[scale=2,>=stealth]{>}}},postaction={decorate}] (14) -- (24);	
	\draw[decoration={markings,mark=at position 1 with
		{\arrow[scale=2,>=stealth]{>}}},postaction={decorate}] (64) -- (34);	
	\draw[decoration={markings,mark=at position 1 with
		{\arrow[scale=2,>=stealth]{>}}},postaction={decorate}] (91) -- (92);	
	\draw (44) -- (54);
	\end{tikzpicture}
	\label{fig:tonepoundHround}
\end{figure}


Four tonal categories of head nouns always behave the same way in compounds: the opposition among LM, LH,
LM+\#H, and LM+MH\# is neutralized. These tone categories for heads are therefore pooled together in Tables~\ref{tab:abstractmonosyllabicmonosyllables} to \ref{tab:abstracttrisyllabic}. Among determiners, the opposition between LH and LM on monosyllables is also
neutralized; accordingly, these two tones are pooled together in the tables.



\begin{table}
\caption{\label{tab:abstractmonosyllabicmonosyllables}The underlying tonal categories of compound
  nouns. Monosyllabic determiner and monosyllabic head. The tone of the determiner is indicated in the leftmost column, and the tone of the head in the top row.}
{\renewcommand{\arraystretch}{1.35}
\begin{tabularx}{\textwidth}{ Q  Q  Q  Q  Q  Q }
\lsptoprule
	tone & LM; LH & M & L & H & MH\\ \midrule
	LM; LH & LH & LM & LH & LM+\#H & LM+MH\#\\
	M & --L & \#H & --L & \#H & MH\#\\
	L & \tikzmark{1a}L &  &  &  & \hspace*{\fill}\tikzmark{1e}\\
	H & \#H-- & \tikzmark{2a}\#H &  &
   \hspace*{\fill}\tikzmark{2e} & --L\\
	MH & \tikzmark{3a}H\# &  &  \hspace*{\fill}\tikzmark{3e} & \tikzmark{4a}H\$ & \hspace*{\fill}\tikzmark{4e}\\
\lspbottomrule
\end{tabularx}}
\DrawBox[dashed]{1a}{1e}
\DrawBox[dashed]{2a}{2e}
\DrawBox[dashed]{3a}{3e}
\DrawBox[dashed]{4a}{4e}
\end{table}

\begin{table}
\caption{\label{tab:abstractmonosyllabicdisyllables}The underlying tonal categories of compound
  nouns. Disyllabic determiner and monosyllabic head. The tone of the determiner is indicated in the leftmost column, and the tone of the head in the top row.}
{\renewcommand{\arraystretch}{1.35}
\begin{tabularx}{\textwidth}{ Q P{18mm} Q Q Q Q }
\lsptoprule
	tone & LH; LM & M & L & H & MH\\ \midrule
	M & --L & \#H & --L & \#H & \tikzmark{1a}--L\\
	\#H & \tikzmark{99a}H\# & \tikzmark{2a}\#H &  &
   \hspace*{\fill}\tikzmark{2e} & \hspace*{\fill}\tikzmark{1e}\\
	MH\# & \hspace*{\fill}\tikzmark{99e} & \tikzmark{3a}MH\# &  &  \hspace*{\fill}\tikzmark{3e}& H\#\\
	H\$ & \#H-- & \#H & H\$ & \#H & H\#--\\
	L & L+H\# & \tikzmark{4a}L &  & \hspace*{\fill}\tikzmark{4e} & L+H\#\\
	L\# & \tikzmark{5a}L\#-- &  &  &  & \hspace*{\fill}\tikzmark{5e}\\
	LM+MH\# & \tikzmark{6a}LM+MH\#-- & LM+MH\# & \tikzmark{7a}LM+H\$ &  & \hspace*{\fill}\tikzmark{7e}\\
	LM+\#H &  \hspace*{\fill}\tikzmark{6e}& LM+\#H & LM+H\# & \tikzmark{8a}LM+\#H & LM+H\#\\
	LM & LM--L & LM & LM--L &
   \hspace*{\fill}\tikzmark{8e} & LM+MH\#\\
	LH & \tikzmark{9a}LH &  &  &  & \hspace*{\fill}\tikzmark{9e}\\
	H\# & \tikzmark{10a}H\#-- &  &  &  & \hspace*{\fill}\tikzmark{10e}\\ 
\lspbottomrule
\end{tabularx}}
\DrawBox[dashed]{1a}{1e}
\DrawBox[dashed]{2a}{2e}
\DrawBox[dashed]{3a}{3e}
\DrawBox[dashed]{4a}{4e}
\DrawBox[dashed]{5a}{5e}
\DrawBox[dashed]{6a}{6e}
\DrawBox[dashed]{7a}{7e}
\DrawBox[dashed]{8a}{8e}
\DrawBox[dashed]{9a}{9e}
\DrawBox[dashed]{10a}{10e}
%\DrawBox[dashed]{11a}{11e}
\DrawBox[dashed]{99a}{99e}
\end{table}


\begin{sidewaystable}%[p]
\caption{\label{tab:abstractdisyllabicmonosyllables}The underlying tonal categories of compound nouns. Monosyllabic determiner and disyllabic head. The tone of the determiner is indicated in the leftmost column, and the tone of the head in the top row.}
{\renewcommand{\arraystretch}{1.35}
{\fontsize{9}{10.75}\selectfont
\begin{tabularx}{\textwidth}{ l P{7mm} P{12mm} Q l P{10mm} P{22mm} P{16mm} P{26mm} }
\lsptoprule
	tone & M & \#H & MH\# & H\$ & L & L\# & LM+MH\#; LM+\#H; LM; LH & H\#\\\midrule
	LM; LH & LM & LM+\#H  & LM+MH\#~/ L+\#H-- & LM+H\$ & L+\#H-- & L+\#H--~/ L+H\# & L+\#H-- & LM+H\#~/ L+H\#\\
	M & M & \#H & MH\# & H\$ & --L & --L\# & --L & H\#\\
	L & \tikzmark{1a}L &  & \hspace*{\fill}\tikzmark{1e} & L+H\# & L & L+H\# & L+\#H-- & L+H\#\\
	\#H & \tikzmark{2a}H\# & \tikzmark{3a}\#H & \#H-- & --L~/ H\# & \tikzmark{4a}\#H-- & \tikzmark{5a}H\# & \tikzmark{6a}\#H-- & H\#\\
	MH & \hspace*{\fill}\tikzmark{2e} & \hspace*{\fill}\tikzmark{3e} &
   MH\# & \#H-- & \hspace*{\fill}\tikzmark{4e} & \hspace*{\fill}\tikzmark{5e} & \hspace*{\fill}\tikzmark{6e} & \#H\\
\lspbottomrule
\end{tabularx}}
\DrawBox[dashed]{1a}{1e}
\DrawBox[dashed]{2a}{2e}
\DrawBox[dashed]{3a}{3e}
\DrawBox[dashed]{4a}{4e}
\DrawBox[dashed]{5a}{5e}
\DrawBox[dashed]{6a}{6e}
}
\end{sidewaystable}

\begin{sidewaystable}[p]
\caption{\label{tab:abstractdisyllabicdisyllables}The underlying tonal categories of compound nouns. Disyllabic determiner and disyllabic head. The tone of the determiner is indicated in the leftmost column, and the tone of the head in the top row.}
{\renewcommand{\arraystretch}{1.65}
{\fontsize{10}{10.75}\selectfont
\begin{tabularx}{\textwidth}{ l@{\hspace{16pt}} P{12mm} P{12mm} P{16mm} P{22mm} P{20mm} P{12mm} Q P{12mm} }
  \lsptoprule tone & M & \#H & MH\# & H\$ & L & L\# & LM+MH\#;\hack{\par} LM+\#H;\hack{\par} LM; LH & H\#\\\midrule
  M & M & \tikzmark{1a}\#H & MH\# & H\$ / \#H-- & --L & --L\# & --L & \tikzmark{2a}H\#\\
  \#H & \tikzmark{3a}H\# & \hspace*{\fill}\tikzmark{1e} & \tikzmark{4a}\#H-- & H\$ / \#H-- / H\# & \tikzmark{17a}\#H-- & \tikzmark{15a}H\# & \#H-- &\\
  MH\# &  & MH\# & \hspace*{\fill}\tikzmark{4e} & \#H-- / H\# &  &  & MH\#-- &\\
  H\$ & \hspace*{\fill}\tikzmark{3e} & \#H & \tikzmark{16a}\#H-- / H\#-- & \hspace*{\fill}\tikzmark{16e} & \hspace*{\fill}\tikzmark{17e} & \hspace*{\fill}\tikzmark{15e} & \#H-- &  \hspace*{\fill}\tikzmark{2e}\\
  L & L+H\# & L & L+H\# & L+H\# & L+\#H-- & L+H\# & L+\#H-- & L+H\#\\
  L\# & \tikzmark{14a}L\#-- &  &  &  &  &  &  & \hspace*{\fill}\tikzmark{14e}\\
  LM+MH\# & \tikzmark{5a}LM+H\# & \tikzmark{12a}LM+\#H & \tikzmark{13a}LM+MH\#-- & LM+MH\#--/ H\# & LM+MH\#-- & \tikzmark{11a}LM+H\# & \tikzmark{10a}LM+MH\#-- & \tikzmark{9a}LM+H\#\\
  LM+\#H & \hspace*{\fill}\tikzmark{5e} &  & \hspace*{\fill}\tikzmark{13e} & \tikzmark{8a}LM--H\$ & LM+\#H-- & \hspace*{\fill}\tikzmark{11e} & \hspace*{\fill}\tikzmark{10e} &\\
  LM & LM-- &  \hspace*{\fill}\tikzmark{12e} & LM+MH\# & \hspace*{\fill}\tikzmark{8e} & LM--L & LM--L\# & LM--L & \hspace*{\fill}\tikzmark{9e}\\
  LH & \tikzmark{6a}LH &  &  &  &  &  &  & \hspace*{\fill}\tikzmark{6e}\\
  H\# & \tikzmark{7a}H\#-- &  &  &  &  &  &  & \hspace*{\fill}\tikzmark{7e}\\
  \lspbottomrule
\end{tabularx}
\DrawBox[dashed]{1a}{1e}
\DrawBox[dashed]{2a}{2e}
\DrawBox[dashed]{3a}{3e}
\DrawBox[dashed]{4a}{4e}
\DrawBox[dashed]{5a}{5e}
\DrawBox[dashed]{6a}{6e}
\DrawBox[dashed]{7a}{7e}
\DrawBox[dashed]{8a}{8e}
\DrawBox[dashed]{9a}{9e}
\DrawBox[dashed]{10a}{10e}
\DrawBox[dashed]{11a}{11e}
\DrawBox[dashed]{12a}{12e}
\DrawBox[dashed]{13a}{13e}
\DrawBox[dashed]{14a}{14e}
\DrawBox[dashed]{15a}{15e}
\DrawBox[dashed]{16a}{16e}
\DrawBox[dashed]{17a}{17e}
}}
\end{sidewaystable}

\newpage
\begin{table}%[t]
\caption{\label{tab:abstracttrisyllabic}Examples and underlying tonal categories of compound nouns with a~{trisyllabic}, \#H-tone determiner: ‘bear’+body part.}
\begin{tabularx}{\textwidth}{ l l Q Q }
\lsptoprule
	\multicolumn{2}{l}{head} & \multicolumn{2}{l}{compound}\\
   \cmidrule(r){1-2} \cmidrule(l){3-4}
	form  & tone & underlying form & underlying tone\\\midrule
	\ipa{ɣɯ˩˧} ‘skin’ & LM & \ipa{gi˧nɑ˧mi˧-ɣɯ˥} & H\#\\
	\ipa{bv̩˧} ‘intestine’ & M & \ipa{gi˧nɑ˧mi˧-bv̩\#˥} & \#H\\
	\ipa{mɤ˩} ‘grease’ & L & \ipa{gi˧nɑ˧mi˧-mɤ˧˥} & MH\#\\
	\ipa{sɤ˥} ‘blood’ & H & \ipa{gi˧nɑ˧mi˧-sɤ\#˥} & \#H\\
	\ipa{ɬv̩˧˥} ‘brains’ & MH & \ipa{gi˧nɑ˧mi˧-ɬv̩˩} & L\#\\ \addlinespace \hdashline \addlinespace
	\ipa{gv̩˧dv̩˧} ‘back’ & M & \ipa{gi˧nɑ˧mi˧-gv̩˧dv̩\#˥} & \#H\\
	\ipa{ɲi˧gɤ˧} ‘nose’ & \#H & \ipa{gi˧nɑ˧mi˧-ɲi˧gɤ\#˥} & \#H\\
	\ipa{qv̩˧ʈʂæ˧˥} ‘throat’ & MH\# & \ipa{gi˧nɑ˧mi˧-qv̩˥ʈʂæ˩} & \#H--\\
	\ipa{hu˧mi˥\$} ‘stomach’ & H\$ & \ipa{gi˧nɑ˧mi˧-hu˧mi˥\$} & H\$\\
	\ipa{nv̩˩mi˩} ‘heart’ & L & \ipa{gi˧nɑ˧mi˧-nv̩˥mi˩} & \#H--\\
	\ipa{ɬi˧pi˩} ‘ear’ & L\# & \ipa{gi˧nɑ˧mi˧-ɬi˧pi˥} & H\#\\
	\ipa{ʝi˩ʈʂæ˧˥} ‘waist’ & LM+MH\# & \ipa{gi˧nɑ˧mi˧-ʝi˥ʈʂæ˩} & \#H--\\
	\ipa{njɤ˩ɭɯ˧} ‘eye’ & LM & \ipa{gi˧nɑ˧mi˧-njɤ˥ɭɯ˩} & \#H--\\
	\ipa{ʁæ˧ʈv̩˥} ‘neck’ & H\# & \ipa{gi˧nɑ˧mi˧-ʁæ˧ʈv̩˥} & H\#\\
\lspbottomrule
\end{tabularx}
\end{table}



% The subsection below needs to start on a clean page, after all floats have been unpiled.


\section{Determinative compound nouns. Part II: Discussion}
\label{sec:determinativecompoundnounsII}
\label{sec:aboutproductivetonerulesincompounding}

Let us now proceed to an analysis of the patterns presented above. At this stage, it may be useful to consider possible ways in which the tones of the determiner and head could combine. Some languages, such as \ili{Mandarin}, have no tone change at all in compounds, demonstrating that tone change is not necessary to compounding. A~first theoretical possibility would thus be the simple concatenation of the input tones. Under such a~configuration, given the \textit{determiner-head} order of Yongning Na compounds, the tone of the determiner would be expressed first. Given the constraints on tone sequences within a~\isi{tone group} (summarized as Rules
1--7), this would leave little room for the tone of the head to be expressed. Since an H
tone precludes the realization of any other
tone on subsequent syllables (by phonological rules 4 and 5), determiners carrying a~lexical tone that includes an H level (i.e.\ one of \#H, H\$, MH\#, LM+MH\#, LM+\#H, and H\#) would \is{neutralization}neutralize all tonal oppositions on head nouns. Likewise, the L level in the lexical categories L and L\# would spread
rightward throughout the compound. The tone of the head could be expressed only when the determiner has M tone: M plus L would yield --L, M plus \#H would yield \#H, and so forth.

A~second possibility would be the complete \isi{neutralization} of tonal oppositions on either the determiner or the head. Neutralization of oppositions on the determiner would be unexpected, given the determiner-first order of compounds and the predominantly perseverative directionality of tone \is{tone spreading}spreading and tone reassociation in Yongning Na. The loss of tonal oppositions among determiners would drastically reduce the number of tonal patterns at the surface phonological level. Such neutralization of tonal oppositions on the head is attested in the neighbouring language \ili{Shixing} (also known as Xumi), where only the tone of the determiner is expressed: all tonal oppositions on the head are neutralized, such that the 3×3 tonal combinations among nouns reduce to just three tone patterns on compounds \citep{chirkovaetal2009}. %The existence of a~greater number of lexical tones in Yongning Na may
%explain in part why such massive \isi{neutralization} did not take place in this language. 

A~third possibility would be the assignment of a~replacive tone in compounds. This is reported to be widespread in the \ili{Mande} subgroup of the Niger-Congo family, e.g.~in \ili{Dan-Gwɛɛtaa} \citep{vydrin2016} and \ili{Kpelle} (\cites{welmers1969}[132]{welmers1973}[239]{konoshenko2014syntax}). In \ili{Naish}, the lower-level subgroup within Sino-Tibetan to which Yongning Na belongs, no clear instance of replacive tone has been observed to date. The closest parallel in \ili{Naish} is found in the \ili{Laze} language, where compounds with input tones M and H, M and L, and M and MH tend to adopt an H.H tone pattern as they lexicalize \citep{michaud2008a}. However, the H.H pattern is not the only one carried by compounds: if the tone of the head is L, the compound receives a different tone pattern.

The state of
affairs found in Yongning Na bears some similarities to both the first and second theoretical possibilities: there is a~tendency for output tone patterns to consist of the concatenation of the input tones, alongside a degree of \isi{neutralization} of tonal oppositions (such that the set of output tone patterns is smaller than the set of input combinations). On the one hand, approximately half of the patterns can be straightforwardly explained in terms of successive association of the two input tones, followed by
application of the general rules governing tonal adjustments within a~tone
group. On the other hand, the remaining patterns could not be fully accounted for using a set of rules.

The tone patterns of determinative compounds thus present a~composite picture (play on words intended). Since not all patterns align neatly with any of the three theoretical possibilities listed above, it must be acknowledged that compounds have \is{tone rules}tone rules of their own, which differ subtly from the straightforward concatenation of input tones. 

This conclusion may come as a~slight disappointment to the linguist, whose job is to
account for all observations through a~model that is as simple and elegant as possible. 
As
in many other domains of the Yongning Na tone system, the relationship between the lexical tones of words and the tones that surface in context cannot be reduced to a set of sandhi rules operating at a purely phonological level. (Clarifications concerning the terms ‘{tone sandhi}’, ‘morphotonology’, and ‘tonal morphology’ are provided in \sectref{sec:definitionofterms}.) The behaviour of the three-syllable noun ‘bear’, /\ipa{gi˧nɑ˧mi\#˥}/, when used as
a~determiner, illustrates this point: it patterns almost like disyllabic \#H-tone nouns, such as ‘colt’,
/\ipa{ʐwæ˧zo\#˥}/, but not quite. When the head is an L-tone {monosyllable}, the output tone is MH\#, rather than the expected \#H.

In view of these observations, the present discussion of the combinations in Tables~\ref{tab:abstractmonosyllabicmonosyllables} to \ref{tab:abstracttrisyllabic} is arranged in order of increasing complexity. 

The simpler cases can be described as follows: the tone of the determiner expresses itself first, and subsequently, the tone of the head surfaces to the extent permitted by the tones already assigned. The tone patterns of compounds in which the determiner has H\#, LH or L\# tone are so straightforward as to appear trivial: in all three cases, only the tone of the determiner is expressed. When the determiner carries H\# tone (i.e.\ an H tone associated with its last syllable), tonal oppositions on the head are
neutralized, regardless of the number of syllables: see the final rows of Tables~\ref{tab:abstractmonosyllabicmonosyllables} to \ref{tab:abstracttrisyllabic}. The resulting compound
carries H\#--, meaning that the determiner's final syllable bears an H tone. This can be analyzed as the direct association of the H\# tone with the determiner: H on its final syllable and, by default, M on its first syllable. The lowering of all following tones to L follows from
Rules 4 and 5: “A syllable following an H-tone syllable receives L tone” and “All syllables
following an H.L or M.L sequence receive L tone”. This is represented in \figref{fig:toneMHcomp}, using the compound ‘rat's stomach' (/\ipa{hwæ˧tsɯ˥-hu˩mi˩}/) as an example. In this compound, no trace remains of the H\$ tone (the ‘flea' tone) carried by the head.

\begin{figure}[p]
	\caption[{A hypothesis about how H\# tone on the determiner associates to the entire compound.}]{A hypothesis about how H\# tone on the determiner associates to the entire compound. Example: /\ipa{hwæ˧tsɯ˥-hu˩mi˩}/ ‘rat's stomach'.}
	\begin{tikzpicture}
	\node (1) at (0.2,-0.5) {H\#};
	\node (4) at (3.5,-0.5) {H\$};
	
	\node (2) at (-0.3,-1.5) {σ};
	\node (3) at (0.7,-1.5) {σ};
	\node (5) at (3,-1.5) {σ};
	\node (6) at (4,-1.5) {σ};
	
	\node [anchor=mid] (s1l) at (0.2,-2) {/\ipa{hwæ.tsɯ}/ ‘rat’};
	%  \node (s1ll) at (0.5,-2.5) {lexical tone: MH\#};
	
	\node [anchor=mid] (s1lll) at (3.6,-2) {/\ipa{hu.mi}/ ‘stomach'};
	%  \node (s1llll) at (4,-2.5) {lexical tone: L};
	
	\node[text width=40mm] (s1) at (-3,-0.75) {Stage 1:\\ input};
	
	
	
	\node (12) at (0.5,-4) {H\#};
	\node (42) at (2.5,-4) {H\$};
	
	\node (22) at (0,-5.5) {σ};
	\node (32) at (1,-5.5) {σ};
	\node (52) at (2,-5.5) {σ};
	\node (7) at (3,-5.5) {σ};
	
	\node[text width=40mm] (s2) at (-3,-4.75) {Stage 2:\\ \is{anchorage}anchoring of H\# to\\ its
		phonologically\\ specified locus};
	
	\draw[decoration={markings,mark=at position 1 with
		{\arrow[scale=2,>=stealth]{>}}},postaction={decorate}] (12) -- (32);
	
	
	
	\node (9) at (0,-7) {M};
	\node (13) at (1,-7) {H};
%	\node (63) at (1.5,-7) {H};
	\node (43) at (2.5,-7) {H\$};
	
	\node (23) at (0,-8.5) {σ};
	\node (33) at (1,-8.5) {σ};
	\node (53) at (2,-8.5) {σ};
	\node (8) at (3,-8.5) {σ};
	
	\node[text width=40mm] (s3) at (-3,-7.75) {Stage 3:\\ addition of default\\ M tone, by Rule~2.\\ H\$ remains unassociated\\ (and is deleted)};
	
	\draw (13) -- (33);
%	\draw[decoration={markings,mark=at position 1 with {\arrow[scale=2,>=stealth]{>}}},postaction={decorate}] (63) -- (53);
% M tone: 
	\draw[decoration={markings,mark=at position 1 with
		{\arrow[scale=2,>=stealth]{>}}},postaction={decorate}] (9) -- (23);
	
	
	\node (14) at (0,-10) {M};
	\node (64) at (1,-10) {H};
	\node (44) at (2.5,-10) {L};
	
	\node (24) at (0,-11.5) {σ};
	\node (34) at (1,-11.5) {σ};
	\node (54) at (2,-11.5) {σ};
	\node (8) at (3,-11.5) {σ};
	
	\node[text width=40mm] (s4) at (-3,-10.5) {Stage 4:\\ assignment of L tone\\ by Rules 4 and 5.};
	
	\draw[decoration={markings,mark=at position 1 with
		{\arrow[scale=2,>=stealth]{>}}},postaction={decorate}] (44) -- (54);
	\draw[decoration={markings,mark=at position 1 with
		{\arrow[scale=2,>=stealth]{>}}},postaction={decorate}] (44) -- (8);
	\draw (14) -- (24);
	\draw (64) -- (34);
	
	
	\node (14) at (0,-13) {M};
	\node (64) at (1,-13) {H};
	\node (44) at (2,-13) {L};
	\node (91) at (3,-13) {L};
	
	\node (24) at (0,-14.5) {σ};
	\node (34) at (1,-14.5) {σ};
	\node (54) at (2,-14.5) {σ};
	\node (92) at (3,-14.5) {σ};
	
	\node[text width=40mm] (s4) at (-3,-13.5) {Stage 5:\\ resulting surface-\\ phonological tone};
	
	\draw (14) -- (24);
	\draw (64) -- (34);
	\draw (44) -- (54);
	\draw (91) -- (92);
	\end{tikzpicture}
	\label{fig:toneMHcomp}
\end{figure}

This same analysis extends to the two other
tone categories of disyllabic determiners after which all tonal oppositions are neutralized: LH and L\#. The realization of either of these tone patterns on the first part of the compound (the determiner) precludes the occurrence of any
tone other than L on the subsequent syllables, by Rules 4 and 5:
\begin{itemize}
    \item LH tone, unfolding as L.H over the two syllables of the determiner, triggers the assignment of L tone to the following syllable by Rule~4 (``The syllable following a H-tone syllable receives L tone''), and further, the assignment of L tone to any subsequent syllables by Rule~5 (``The syllables following an H.L or M.L sequence receive L tone'').
    \item L\# tone, unfolding as M.L over the two syllables of the determiner (through association of an L tone to the second syllable of the determiner, and of a default M tone to its first syllable), likewise depresses the tones of following syllables to L by Rule~5 (``The syllables following a H.L or M.L sequence receive L tone'').
\end{itemize}


In the remaining cases, which constitute a~majority, tonal oppositions on the head are not
entirely neutralized: establishing the tone of the compound requires knowledge of the head noun's tone. This observation raises doubts about the adequacy of the representation proposed in \figref{fig:toneMHcomp}, which assumes a determiner-driven tonal association process. If the process were a straightforward step-by-step association~-- first applying the determiner's tone, then the head's~--, one would expect the complete \isi{neutralization} of all tonal oppositions on the head when the determiner carries MH\# tone. On the {analogy} of \figref{fig:toneMHcomp}, one would expect 
%the behaviour shown in \figref{fig:toneMHcompFALSE}, i.e.\ that 
the tone pattern for ‘cat's ear’ to be $\dagger${\kern2pt}\ipa{hwɤ˧li˧-ɬi˥pi˩}. But the observed tone is H\#: /\ipa{hwɤ˧li˧-ɬi˧pi˥}/. 

\begin{figure}[p]
	\caption{Tone-to-syllable association expected for ‘cat's ear’ under the mistaken hypothesis of determiner-driven tone association:  $\ddagger${\kern2pt}\ipa{hwɤ˧li˧-ɬi˥pi˩}, as contrasted with the observed pattern: /\ipa{hwɤ˧li˧-ɬi˧pi˥}/.}
	\begin{tikzpicture}
	\node (1) at (0.5,-0.5) {MH\#};
	\node (4) at (3,-0.5) {L\#};
	
	\node (2) at (0,-1.5) {σ};
	\node (3) at (1,-1.5) {σ};
	\node (5) at (2.5,-1.5) {σ};
	\node (91) at (3.5,-1.5) {σ};
	
	\node [anchor=mid] (s1l) at (0.5,-2) {/\ipa{hwɤ.li}/ ‘cat’};
	%  \node (s1ll) at (0.5,-2.5) {lexical tone: MH\#};
	
	\node [anchor=mid] (s1lll) at (3,-2) {/\ipa{ɬi.pi}/ ‘ear’};
	%  \node (s1llll) at (4,-2.5) {lexical tone: L};
	
	\node[text width=40mm] (s1) at (-3,-0.75) {Stage 1:\\ input};
	
	
	
	\node (12) at (0.5,-4) {MH\#};
	\node (42) at (2.5,-4) {L\#};
	
	\node (22) at (0,-5.5) {σ};
	\node (32) at (1,-5.5) {σ};
	\node (52) at (2,-5.5) {σ};
	\node (90) at (3,-5.5) {σ};
	
	\node[text width=40mm] (s2) at (-3,-4.75) {$\ddagger${\kern2pt}Stage 2:\\ \is{anchorage}anchoring of MH\# to\\ its
		phonologically\\ specified locus};
	
	\draw[decoration={markings,mark=at position 1 with
		{\arrow[scale=2,>=stealth]{>}}},postaction={decorate}] (12) -- (32);
	
	
	
	\node (13) at (1,-7) {M};
	\node (63) at (1.5,-7) {H};
	\node (43) at (2.5,-7) {L\#};
	
	\node (23) at (0,-8.5) {σ};
	\node (33) at (1,-8.5) {σ};
	\node (53) at (2,-8.5) {σ};
	\node (92) at (3,-8.5) {σ};
	
	\node[text width=40mm] (s3) at (-3,-7.75) {$\ddagger${\kern2pt}Stage 3:\\ one-to-one mapping\\ of levels to available syllables};
	
	\draw[decoration={markings,mark=at position 1 with
		{\arrow[scale=2,>=stealth]{>}}},postaction={decorate}] (13) -- (33);
	\draw[decoration={markings,mark=at position 1 with
		{\arrow[scale=2,>=stealth]{>}}},postaction={decorate}] (63) -- (53);
	
	
	\node (14) at (0,-10) {M};
	\node (64) at (1,-10) {M};
	\node (44) at (2,-10) {H};
	\node (97) at (3,-10) {L};
	
	\node (24) at (0,-11.5) {σ};
	\node (34) at (1,-11.5) {σ};
	\node (54) at (2,-11.5) {σ};
	\node (94) at (3,-11.5) {σ};
	
	\node[text width=40mm] (s4) at (-3,-10.5) {$\ddagger${\kern2pt}Stage 4:\\ addition of default M,\\ and assignment\\ of final L by\\ phonological rule};
	
	\draw[decoration={markings,mark=at position 1 with
		{\arrow[scale=2,>=stealth]{>}}},postaction={decorate}] (14) -- (24);
	\draw[decoration={markings,mark=at position 1 with
		{\arrow[scale=2,>=stealth]{>}}},postaction={decorate}] (97) -- (94);
	\draw (64) -- (34);
	\draw (44) -- (54);
	
	
	\node (14) at (0,-13) {M};
	\node (64) at (1,-13) {M};
	\node (44) at (2,-13) {M};
	\node (96) at (3,-13) {H};
	
	\node (24) at (0,-14.5) {σ};
	\node (34) at (1,-14.5) {σ};
	\node (54) at (2,-14.5) {σ};
	\node (95) at (3,-14.5) {σ};
	
	\node[text width=40mm] (s4) at (-3,-13.5) {\textit{Observed pattern:\\ final H tone.}};
	
	\draw (14) -- (24);
	\draw (64) -- (34);
	\draw (44) -- (54);
	\draw (96) -- (95);
	\end{tikzpicture}
	\label{fig:toneMHcompFALSE}
\end{figure}

The double daggers $\ddagger$ added to the labels ‘Stage 2', ‘Stage 3', and ‘Stage 4' in \figref{fig:toneMHcompFALSE} highlight the hypothetical nature of that representation: it is proposed solely to illustrate the \textit{ad hoc} character of the analysis in \figref{fig:toneMHcomp}. Tone association in compounds is not consistently driven by the determiner. Nonetheless, the representation in \figref{fig:toneMHcomp} does correspond to a genuine pattern, albeit one specific to H\# tone. This tone is \is{anchorage}anchored to a~word's final syllable; in compounds, when the first element of the compound (the determiner) carries H\# tone, this tone remains moored to the determiner's final syllable.

A~few combinations in Tables~\ref{tab:abstractmonosyllabicmonosyllables} to \ref{tab:abstracttrisyllabic} appear counter-intuitive in view of the input tones. For
instance, when an M-tone determiner is combined with an M-tone monosyllabic head, the resulting compound carries a~\is{floating tone}floating H tone, \#H. This outcome is all the more unexpected as M is generally expected to be inactive (see~\sectref{sec:analysisofmasadefaulttone}). By contrast, an M-tone determiner and an M-tone \textit{disyllabic} head yield a compound with a simple M tone. 
Crucially, the unexpected \#H
output does not result from a~general {phonological rule} of Na that turns any sequence of two M tones into a~\is{floating tone}floating H. Rather, it results from the \textit{morpho}{phonological rules} that apply specifically within this syntactic construction. Since these rules cannot be straightforwardly reduced to a~single generalization, they are best presented in table form, as in Tables~\ref{tab:abstractmonosyllabicmonosyllables} to \ref{tab:abstracttrisyllabic}.

%{\largerpage} %%May be useful here
The discussion of the more {complex tone} patterns that follows is arranged by the tone category of the determiner.

\subsection{LM-tone determiners}

An LM tone on the determiner invariably results in the assignment of L on the first syllable of the compound
and M on the second syllable. A~relatively high number of tone sequences are allowed after
/L.M/, allowing some tone categories of the
head to be expressed in full. Over three syllables, three patterns are observed: /L.M.L/, /L.M.M/, and /L.M.H/. Over four syllables, $\ddagger${\kern2pt}L.M.L.M and $\ddagger${\kern2pt}L.M.L.H are ruled out, as the sequence /M.L/ can only be followed by /L/, by virtue
of Rule~5 (“All syllables following an H.L or M.L sequence receive L tone”). The tones that are compatible with an initial LM pattern are
observed to surface as expected: LM plus \#H, LM plus MH\#, LM plus H\$, and LM plus
H\# are realized straightforwardly as concatenations of the two input tones. As anticipated, an M-tone head has no effect on the final tone pattern, which is simply LM. Likewise, the combinations LM plus \#L and
LM plus H\# in quadrisyllabic compounds surface as such.

Some of the \is{form!surface}surface patterns are analytically indeterminate. For instance, the combination of a~monosyllabic LM determiner and a~monosyllabic MH\# head yields L.MH
(e.g.~/\ipa{bo˩-ɬv̩˧˥}/ ‘pig’s brains’). This could be analyzed in two ways: either as L--MH\# (L on the first part, and
MH\# on the second) or as LM+MH\# (where LM contributes L to the first syllable and M to the second, and MH\# assigns an MH \is{tonal contour}contour to the last syllable,
which in this case is also the second syllable). Since both analyses generate the same
output, the simpler approach is to describe the pattern as the concatenation of the two input
tones, i.e.\ //LM+MH\#//. Under this analysis, the same applies to \is{trisyllables}{trisyllabic} (σσ+σ) and quadrisyllabic (σσ+σσ) compounds: the 
%different 
surface patterns (L.M.MH for σσ+σ and
L.M.M.H for σσ+σσ) follow straightforwardly from the general rules of tone-to-syllable mapping
summarized in~\sectref{sec:asummaryoftonetosyllableassociationrules}.

In a~disyllabic compound with an LM-tone determiner, an L or LM tone on the head cannot express itself, as the
determiner’s LM \is{tonal contour}contour has already projected its endpoint (M) on the second syllable. While the M
tone functions as a default tone in some contexts (see~\sectref{sec:analysisofmasadefaulttone}), in this case it is specified as the endpoint
of an LM \is{tonal contour}contour and thus behaves as a~fully specified tone. Such cases are not typologically unusual, a point to which the discussion in Chapter~\ref{chap:arealandtypologicaldiscussion} will return. In \is{derivation!tonal}derivational terms,
this suggests that at the point when the head's tone
could come into play, both syllables of the compound are already specified for tone,
leading to the \isi{neutralization} of LM, M, and L as the second components of σ+σ compounds with
an LM determiner.

By contrast, in three- and four-syllable compounds, there is still room for the expression of an L level originating in an L or LM tone on the head. These combinations result in L.M.L in trisyllabic compounds and L.M.L.L in quadrisyllabic ones, in accordance with Rule~5 (“All syllables following an H.L or M.L sequence receive L tone”).

To sum up, the tones of all compounds with an LM determiner obtain through the concatenation of the tones of their two components, subject to the phonological rules that apply at the level of tone
groups (as recapitulated in Chapter~\ref{chap:toneassignmentrulesandthedivisionoftheutteranceintotonegroups}).

\subsection{M-tone determiners}
\label{sec:Mtonedet}

After M-tone determiners, L and LM tones on heads are
neutralized due to Rule~5 (“All syllables following an H.L or M.L sequence receive L tone”). Beyond these two categories, one would expect the tone of the second component of the compound to express itself
fully, given that M behaves in some respects as a~default tone (see \sectref{sec:analysisofmasadefaulttone}) and is generally phonologically neutral (i.e.\ inert). This prediction is not entirely borne out, however. 

As noted at the outset of this chapter, the tone pattern \#H (a~\is{floating tone}floating H tone) that results from the combination of an M-tone determiner with a~monosyllabic M-tone
head does not conform to the regularities observed for the other combinations. Similarly, the \#H--
\is{variants}variant observed for a disyllabic M-tone determiner plus an H\$-tone head remains unexplained. A~third
unexpected outcome is the --L pattern arising from the combination of a disyllabic M-tone determiner with a monosyllabic MH\# head. Given that in all other analogous cases (σσ+σσ, σ+σσ, and σ+σ), MH\# surfaces as expected, one would anticipate the same here. These three cases confirm an earlier observation: not all patterns can be explained in terms of a~set of rules applying
throughout the morphophonological system. 

Mrs.\ Latami (the main consultant for this study) appears to have \is{language acquisition}acquired a~large number of tone
patterns individually during childhood, learning \isi{morphotonology} in a~manner comparable to how children acquire morphology in \ili{rGyalrongic} or \ili{Kiranti} languages, to cite two subgroups of Sino-Tibetan known for their flamboyant morphology \citep{michailovsky1975a,vandriem1990,sun2000a,jacques2004}. However, it may now be too late to investigate children's acquisition of Yongning Na \isi{morphotonology} in the form documented in this volume: by the mid-2010s, school-age children in Alawua were exposed to a~great deal of {Mandarin}, making it unlikely that much of the \isi{morphotonology} was being transmitted. (The influence of {bilingualism} with {Mandarin} is examined in \sectref{sec:theinfluenceofbilingualismwithchinese}.)

\subsection[L-tone determiners]{L-tone determiners and what they reveal for the analysis of the head noun}

The first of the seven phonological tone rules of Yongning Na is that L tone spreads progressively (“left-to-right”) onto syllables that are unspecified
for tone. The tone patterns of compounds with L-tone determiners thus provide an interesting
testing-ground for determining whether lexical tones that begin with an M tone in their surface
form are underlyingly specified for tone on their first syllable. If they were, that initial M tone would
be expected to block L-tone \is{tone spreading}spreading; conversely, if the first syllable is unspecified for tone,
it should receive an L tone through \is{tone spreading}spreading.

The observed patterns support the analysis that disyllables with High tone (\#H, H\$, and
H\#) are unspecified for tone on their first syllable: when the head carries one of these tones, an L-tone determiner spreads L onto the first syllable of the head. 

The same reasoning can be extended to the L\# tone, a~type of L tone that associates in word-final position. L plus L\# yields H\#, as in the compound /\ipa{kʰv̩˩mi˩-ɬi˩pi˥}/ ‘dog’s ear’, where
the L tone on the first syllable of the head is analyzed as resulting from L-tone \is{tone spreading}spreading (cf.\ /\ipa{kʰv̩˩mi˩}/ ‘dog’ and /\ipa{ɬi˧pi˩}/ ‘ear’). (The final H in this compound's tone pattern is discussed further below.)

The weight of this argument is admittedly somewhat diminished by the fact that tone patterns after
an L-tone determiner cannot be generated in full through the application of a~set of general rules. It remains unclear to what extent the processes involving L tone in this particular morphosyntactic context
(determinative compounds) relate to the general rule of L-tone \is{tone spreading}spreading. When the determiner and the head are both
monosyllabic, all tonal oppositions are neutralized: the compound
carries a simple L tone. In the other three length combinations (σσ+σσ, σσ+σ, σ+σσ), the picture is more
complex. A~\is{floating tone}floating H tone (\#H) on the head is always disregarded, yielding L. In
combination with an M-tone head, the result is also L, except for σσ+σσ, which yields L+H\# (surface phonological tone sequence: L.L.L.H). The presence of a~final H is not due to a~general rule limiting the propagation of L: for instance, the compound
//\ipa{jo˩-gv̩˩dv̩˩}// ‘sheep’s back’ (input tones: L plus \#H) carries a~simple L tone, which spreads over both syllables of
the head. (It surfaces as /\ipa{jo˩-gv̩˩dv̩˩˥}/, with a~final rise, due to postlexical H-tone
addition: Rule~7.) Similarly, an L plus L combination in disyllables yields an outcome that is incompatible with the hypothesis that the tones of the determiner and head are simply concatenated: L+\#H-- (surface
form: L.L.H.L).

As with M-tone determiners (\sectref{sec:Mtonedet}), the behaviour of L-tone determiners resists phonological generalizations. In some cases it could seem as if \isi{dissimilation} were at play. For example, in /\ipa{kʰv̩˩mi˩-nv̩˥mi˩}/ ‘dog’s heart’ (from an L plus L input), the H tone on the first
syllable of the head might be interpreted as a~case of \isi{dissimilation}, hypothesizing a process whereby the L tone on the head dissimilates to an initial H tone. But L plus L produces a~simple L output in other length configurations (σσ+σ, σ+σ, and σ+σσ), which shows that there is no general (phonological) mechanism of \isi{dissimilation} between successive L tones. Once again, the evidence suggests that tone combination rules are not derived from a uniform set of phonological principles. They need to be learnt individually.


\subsection{H-tone determiners: \#H, H\#, and H\$}
\label{sec:Htonedems}

As with L-tone determiners, the tone of compounds formed with determiners having an H tone (\#H, H\# or H\$) varies depending on the number of syllables in the head noun. Attempts to generate the tones of these
compounds from the input tones via a~set of rules were unsuccessful. Nonetheless, a~general
observation can be made:

\begin{quotation}
\#H and H\$ (the \is{floating tone}floating H tone and the `hopping' H tone) are never observed to reassociate more than
one syllable to their right.
\end{quotation}

This constraint is specific to determinative compounds and does not apply elsewhere in the
\isi{morphotonology}. For example, in (\ref{ex:byLatami}), the H\$ tone shifts two syllables away from
the word to which it is lexically attached (the family name /\ipa{lɑ˧tʰɑ˧mi˥\$}/ ‘Latami’).

\begin{exe}
	\ex
	\label{ex:byLatami}
	\ipaex{lɑ˧tʰɑ˧mi˧=ɻ̩˧ ɳɯ˥}\\
	\gll lɑ˧tʰɑ˧mi˥\$		=ɻ̩˩		ɳɯ˧\\
	Latami~(family~name)	\textsc{associative}	\textsc{a}\\
	\glt ‘by the Latamis’ (Field notes.)
\end{exe}

In compounds with a~monosyllabic head, the \#H tone is preserved in six out of ten cases. Since this
tone attaches at the end of the word that carries it, this preservation effectively corresponds to a~one-syllable shift to the
right from its original position. When the head is disyllabic, the \#H tone is never
present in the output, as if it could not move more than one syllable away from its original
position without changing its nature. Instead, in eight of the sixteen cases, the compound carries a~fixed, word-final H tone (H\#).

Under this analysis, the H\# tone observed in compounds where both the determiner and the head carry H\# tone should be attributed to the head rather than the determiner.

The same generalization accounts for the fact that the H\$ tone never surfaces in compounds with
a~disyllabic head. On the other hand, no hypothesis can be proposed as to why it surfaces when the head is an L-tone
monosyllable but does not surface in combination with any other monosyllabic head. Interestingly, the seven
cases that have two or three variant forms all involve an H\$-tone head noun, suggesting that this tone category is relatively more unstable than the other types of H tones.

Compounds with a~\#H-tone determiner yield the same output tone regardless of whether the determiner is
monosyllabic or disyllabic. This finding corroborates the initial hypothesis that the tone
category of monosyllables illustrated by /\ipa{ʐwæ\#˥}/ ‘horse’ is phonologically identical to that of disyllables such as /\ipa{gi˧zɯ\#˥}/ ‘little brother’.\footnote{As was explained in \sectref{sec:thenotationoftonalcategoriesinlexicalentries}, for typographical simplicity, and given the absence of an opposition among different types of H tones on
monosyllabic nouns, ‘horse’ is transcribed as /\ipa{ʐwæ˥}/ rather than /\ipa{ʐwæ\#˥}/,
omitting the information on the segmental anchoring of its H tone.}

\subsection{MH-tone determiners}
\label{sec:MHtonedems}

In determinative compounds, MH tone, like other tones containing an H level, is never observed to move
more than one syllable to the right. When the determiner is disyllabic and the head monosyllabic (σσ+σ), an MH tone on the determiner moves onto the last syllable of the compound. In σ+σ, on the other hand,
the MH tone does not move as a~whole: it appears to remain associated with the determiner, and to project
its H level onto the head~-- except when the head has a~\#H or MH tone. 

Once again, these observations remain fragmentary: no set of rules can be formulated to systematically generate the tones of these compounds from the input tones.

\subsection{Determiners carrying LM+MH\# tone or LM+\#H tone}

The behaviour of LM+MH\# and LM+\#H when they appear on determiners provides further evidence for their
phonological analysis. These two
lexical categories unfold as L.M over the first two syllables of polysyllabic
compounds, which supports the analysis proposed in \sectref{sec:overviewofthesystem}: both involve an LM tonal sequence. If LM+MH\# were instead analyzed as L+MH\#, dispensing with the first M in the expression ‘LM+MH\#’, its initial L tone would be
expected to spread, creating a~sequence of L tones followed by a~final MH \is{tonal contour}contour. Likewise, reanalyzing LM+\#H as L+\#H would not be promising at all, despite the apparent gain in descriptive economy: if the first part of this tone were simply L, its behaviour in compounds would be incomprehensible.

Beyond this general observation, the interpretation of individual tone combinations is not
straightforward. The cases that seem to make good sense in terms of the input tones do not greatly
outnumber those that appear \is{opacity}opaque. For instance, when LM+MH\# is followed by a~monosyllabic head carrying \#H or MH, the resulting compound receives a~pattern comprising a~`hopping' H tone (H\$), exactly like in compounds where a monosyllabic MH-tone determiner associates with a \#H-tone or MH-tone head. But the parallel ends here: when the head is an L-tone {monosyllable}, an MH-tone determiner yields a~final H tone (H\#), whereas
an LM+MH\# determiner produces a~`hopping' (H\$).


\subsection[Cases of neutralization of tonal oppositions on the head]{About cases of neutralization of tonal oppositions on the head}
\label{sec:abouttheneutralizationoftonaloppositionsonthehead}

Four tonal categories of disyllabic head nouns always behave identically: the distinctions among
LM, LH, LM+\#H, and LM+MH\# are neutralized. This \isi{neutralization} is not a~direct consequence of the seven phonological tone rules set out in \sectref{sec:alistoftonerules} and reiterated at various points in this volume. 
%(including the ‘Quick reference’ section). 
In
principle, one could imagine a~\is{morphotonology}morphotonological rule whereby an L-tone determiner and a~head carrying LM+MH\# tone would yield L--LM+MH\# in the compound through simple concatenation of the
two tones. Under such a rule, the expected output for ‘sheep’s waist’ (with input tones L and LM+MH\#) would not be /\ipa{jo˩-ʝi˥ʈʂæ˩}/ but rather $\dagger${\kern2pt}\ipa{jo˩-ʝi˩ʈʂæ˧˥}. This form would not violate any well-formedness constraints. What, then, might explain its absence? 

As observed at the outset of \sectref{sec:determinativecompoundnounsII}, in simpler cases, the tone of the determiner is realized first, followed by the tone of the head to the extent that it is compatible with the tones already assigned. The four categories LM, LH,
LM+\#H, and LM+MH\# all begin with an L tone; in almost all cases, expression of this L tone on the head
results in the creation of an /H.L/ or /M.L/ sequence at the \is{juncture (inside a tone group)}juncture between the determiner and the head. Consider, for instance, a compound consisting of an M-tone determiner such as /\ipa{po˧lo˧}/ ‘ram’ and an LM+MH\#-tone head such as /\ipa{ʝi˩ʈʂæ˧˥}/ ‘waist’. 
%Taking the simple example of an M-tone determiner, such as /\ipa{po˧lo˧}/ ‘ram’, and an LM+MH\#-tone head, such as /\ipa{ʝi˩ʈʂæ˧˥}/ ‘waist’. 
The tone
assignment process can be hypothesized as follows: 

\begin{enumerate}[label=(\roman*)]
	\item  The M tone of the
	determiner associates first, yielding M tone on the first two syllables of the compound /\ipa{po.lo-ʝi.ʈʂæ}/
	‘ram’s waist’, hence /\ipa{po˧lo˧-ʝi.ʈʂæ}/.
	
	\item  Since M is a non-spreading tone, the tone of the head can surface. By left-to-right association, the first syllable of the head receives L tone, corresponding to the first tone level in the LM+MH\#
	pattern. This results in /\ipa{po˧lo˧-ʝi˩ʈʂæ}/.
	
	\item  The final syllable receives L tone due to one of the phonological rules that govern tone association in the Alawua dialect of Yongning Na: “All syllables following an H.L or M.L sequence receive L tone” (Rule~5). This yields /\ipa{po˧lo˧-ʝi˩ʈʂæ˩}/.\footnote{An alternative approach would be to assume that the full lexical tone pattern of the head is initially assigned, yielding $\ddagger${\kern2pt}\ipa{po˧lo˧-ʝi˩ʈʂæ˧˥}, and that this form, which is disallowed due to the presence of a~trough in the middle (the L in the M.M.L.MH
	sequence), is subsequently repaired by deleting the final MH and replacing it with L. Under this view, Rule~5 could
	be reformulated as Rule~5’: “All syllables following an H.L or M.L sequence are lowered to L”. Rule
	5’ would apply after full association of the LM+MH\# tone pattern, whereas under the present
	account, Rule~5 applies as soon as the M.L sequence is created, i.e.\ as soon as an L tone associates to
	the syllable /{\dots}\ipa{-ʝi˩}{\dots}/. Since both approaches yield the same outcome, the choice between them depends on theoretical
	preferences.}
\end{enumerate}

The situation exemplified by /\ipa{po˧lo˧-ʝi˩ʈʂæ˩}/ ‘ram’s waist’ is widespread: there are many cases where the tone of the compound can be analyzed as resulting from the successive association of the determiner and head's tones, with any necessary adjustments dictated by phonological tone rules. Why, then, does the combination of input tones L and LM+MH\# deviate from this pattern? 

A possible explanation is that, at an earlier historical stage, this input (L and LM+MH\#) yielded L.L.MH (phonologically: L+MH\#) for a~σ+σσ string, through successive association. If the rest of the system at that stage resembled its present state, this L+MH\# output would have been an outlier. Among compounds with a L-tone determiner, it would have been the only one to contain anything other than L and H tones, as it would have been the only combination in which a~tone pattern beginning in L could be fully realized on the head without contravening phonological constraints. In all other cases, the opposition between LM, LH, LM+\#H, and LM+MH\# would have been neutralized by Rule~5: “All syllables following
an H.L or M.L sequence receive L tone”. Thus, L+MH\# would have stood as the sole \isi{counterexample} to this local pattern of \isi{neutralization}. It is conceivable that speakers regularized this unique output (*L+MH\#) through \isi{analogy} with the more common pattern. 

In other words, the proposed scenario is as follows. First, a near-complete pattern of \isi{neutralization} emerged due to a~{phonological rule}, with a single \is{exceptions}exception: *L+MH\# from input L and LM+MH\#. Later, this exception was eliminated through \isi{analogy}, rendering the pattern fully exceptionless within this corner of Yongning Na \isi{morphotonology}. This development enhanced the internal consistency of the tone system by ensuring greater uniformity in the inventory of tone patterns of compounds. It resulted in the current pattern of full \isi{neutralization} of LM, LH, LM+\#H, and LM+MH\# tones on head nouns in compounds. On the other hand, it also disrupted the straightforward correspondence between input tones and output tones by introducing an \is{exceptions}exception to the general principle of successive association. (The simplicity of the successive-association pattern makes it appealing to the phonologist, but apparently less so to speakers of the language: rule simplicity is one thing; simplicity of output forms is another.) As a result, the new output for input L and LM+MH\# became more aligned with other compound tone patterns but it had to be learnt independently.

This account is admittedly purely speculative. However, formulating hypotheses of this kind can be useful in constructing a coherent understanding of the tone combination rules, rather than viewing them as a~mere collection. 
%of observations. 
Successive association of input tones in compounds may never have been an exceptionless rule at any {diachronic} stage; indeed, it is conceivable that cases where output tones directly reflect successive association were \textit{fewer} in number at some point in the past than they are now. Nonetheless, reasoning in terms of a prototypical pattern (in this case, successive association) remains useful: it allows for the systematic identification and analysis of deviations, providing insight into the {structural} factors that may have shaped these patterns. Further progress in this strand of research will require broader dialectal coverage than is currently available. The topic of the {diachronic} dynamics of the Yongning Na tone system is taken up in Chapter~\ref{chap:yongningnatonesinadynamicsynchronicperspective}.


\subsection{A tendency to avoid long-distance movement of tones}
\label{sec:longdistancedispreferred}

It was noted at the outset of this chapter, when discussing compounds
of five syllables and more, that Yongning Na has a~tendency to avoid
processes leading to long-distance tone movement~-- specifically,
reassociation of an H tone more than two syllables away from the word
to which it is lexically attached. This tendency is confirmed by
observations about the tone patterns that appear in compounding. A
\is{floating tone}floating H tone (\#H) on the determiner, combined with an M tone on the head,
yields a~final H tone (H\#) on the compound, rather than a~\is{floating tone}floating H tone
(\#H). Consequently, the H tone does not shift further away, as a~{floating} tone would. It seems
as if a~determiner's \is{floating tone}floating H tone loses its ability to {float} in the process of compounding. To put it impressionistically: the H tone only floats once; at the outcome of the
compounding process, it does not retain any
potential for further movement. It appears highly significant that the
only σσ+σσ compounds that retain a~\is{floating tone}floating H tone result from an input
in which the head originally carried this tone, i.e.\ cases where the
\is{floating tone}floating H tone does not move in the process of compounding. 

Likewise, when the determiner carries H\$ or MH\#, the H level in the lexical tone tends to remain close to the determiner's edge, 
%last syllable, 
even though its \is{anchorage}anchoring may be modified in compounding (as set out in \sectref{sec:Htonedems}). 
%For instance, H\$ tone yields H\$, \#H--, \#H or H\#-- on compounds (with a~range of different modes of association) depending on the tone
%of the head.

The association of an L tone to long stretches of syllables does not
constitute a~\isi{counterexample} to the tendency to avoid long-distance
tone movement. Tone \is{tone spreading}spreading,
neutralizing all tonal oppositions on a~portion of the tone
group, does not involve tonal movement in the sense of reassociating a tone away from the word to which it is lexically attached.
%The process is
%summarized in Chapter~\ref{chap:toneassignmentrulesandthedivisionoftheutteranceintotonegroups} as Rule~5: “All syllables following an H.L or M.L sequence receive L tone”.

\subsection{Brief remarks about slips of the tongue}
\label{sec:slipsofthetongue}

Hesitations, variants, and tonal slips of the tongue can offer insights into the tone system. A~full-fledged study of this captivating topic would require more fine-grained tools than those used so far. The {boundary} between an acceptable \is{variants}variant and a~commonly occurring \is{mistakes}mistake is
not entirely clear, and the main consultant’s judgments sometimes wavered between the two. While the greatest care was exercised to verify the data, the initial dichotomy between
\is{mistakes}mistakes and acceptable variants would need to be followed up with targeted experiments
to assess variant acceptability on a more precise scale. (On the notion of
gradient acceptability, see \citealt{kirbyetal2007,coetzee2008constraints,goldrick2011}.) An attempt at quantifying and analyzing \is{mistakes}mistakes is made in the discussion of numeral-plus-classifier phrases (\sectref{sec:aboutmistakenrealizationsintherecordings}); for compounds, only preliminary observations can be offered here.

In studying slips of
the tongue, it is useful to consider, for each surface phonological tone pattern, (i)~the rules that can give rise to it, and (ii)~the morphosyntactic constructions in which it occurs. For instance, the L tone category does not appear in any σσ+σ determinative compound (see \tabref{tab:abstractmonosyllabicdisyllables}), so that cases where a~σσ+σ compound is mistakenly realized with an L tone cannot be attributed to \isi{analogy} with other σσ+σ compounds. (Examples include the realization of ‘dog’s brains’ as $\ddagger${\kern2pt}\ipa{kʰv̩˩mi˩-ɬv̩˩} instead of
/\ipa{kʰv̩˩mi˩-ɬv̩˥}/ in the recording \textit{DetermCompounds12} \pandoi{0004463\#W121}.) A possible explanation is interference between tone rules that apply to compounds of different syllabic structures, given that σ+σσ compounds with the same tonal input yield an L tone (see \tabref{tab:abstractdisyllabicmonosyllables}). Alternatively, the erroneous pattern (the slip of the tongue) could result from misapplying a~{phonological rule} of tone \is{tone spreading}spreading: “L tone spreads progressively onto syllables unspecified for tone” (Rule~1; see \sectref{sec:alistoftonerules}). In this case, the \is{mistakes}mistake would consist in grouping the two nouns together with a~solely phonological adjustment instead of a~morphophonological one.

Tonal slips of the tongue~-- perhaps better termed “slips of the larynx”?~-- in the Na data also suggest a~special closeness between certain tonal categories. The M tone category, for instance, appears more prone to confusion
with \#H than with other tones. This is illustrated by errors involving /\ipa{lɑ˧}/ ‘tiger’, in which the expected tone pattern is replaced with that of a~\#H-tone noun: $\ddagger${\kern2pt}\ipa{lɑ˧-ɬv̩˩} ‘tiger’s brains’
instead of /\ipa{lɑ˧-ɬv̩˧˥}/, and $\ddagger${\kern2pt}\ipa{lɑ˧-hu˩mi˩} ‘tiger’s stomach’ instead of
/\ipa{lɑ˧-hu˧mi˥\$}/. (Both errors occur twice in the recordings: \textit{DetermCompounds6} \pandoi{0004454}
and \textit{DetermCompounds7} \pandoi{0004456}.) There are also cases of substitution among H\$, MH\#, and H\#, such as $\ddagger${\kern2pt}\ipa{hwɤ˧li˧-sɤ˥}
instead of /\ipa{hwɤ˧li˧-sɤ˧˥}/ for ‘cat’s blood’, and $\ddagger${\kern2pt}\ipa{hwɤ˧li˧-ɬv̩˧˥} instead of
/\ipa{hwɤ˧li˧-ɬv̩˥}/ for ‘cat’s brains’. Finally, the speaker occasionally confuses LM+H\# and
LM+MH\# tones.



\subsection{Perspectives for comparison across speakers}
\label{sec:crossspeakerdifferences}
\largerpage

The dataset analyzed here was provided by the primary consultant, Mrs.\ Latami (F4). Additional data was elicited from three other speakers: F5, who is F4’s daughter-in-law; M21, a~relative of F4,
belonging to the same generation; and M23, who is M21’s son. Information about these three speakers was provided in \sectref{sec:otherlanguageconsultants}. To summarize briefly, all three are less proficient speakers than F4~-- M21 due to long stays away from Yongning, and F5 and M23 due to a~generation gap: both are fluent speakers of \il{Mandarin!Southwestern}Southwestern Mandarin.

Unsurprisingly, some cross-speaker differences emerge. Analyzing these differences is crucial to understanding the dynamics of the tone system. Some reflect lexical \isi{variation}, with certain nouns belonging to different categories in different consultants' speech. For instance, ‘flea’ is /\ipa{kv̩˧ʂe˥\$}/ (LH
tone) in F4’s speech but /\ipa{kv̩˧ʂe\#˥}/ (\#H tone) in M21’s. Similarly, ‘boar’ is /\ipa{bo˩ɬɑ˥}/ (LH tone) in F4’s speech but /\ipa{bo˩ɬɑ˧˥} in F5's and /\ipa{bo˩ɬɑ˧}/ in M21’s.\footnote{The dictionary records variants for the various consultants: see \citet[x-xi]{michaud_et_al_na_dict_2024}.} The tonal difference for ‘boar’s snout’~-- /\ipa{bo˩ɬɑ˥-ɲi˩gɤ˩}/ (LH) in F4's speech and /\ipa{bo˩ɬɑ˧ɲi˧gɤ\#˥}/ (LM+\#H) in M21's~-- does not appear to stem from differences in the tonal rules governing compounds, but from a difference in the lexical tone of the determiner. Both compounds follow the
regularities summarized in Tables~\ref{tab:abstractmonosyllabicmonosyllables} to \ref{tab:abstracttrisyllabic}, but the input combination is different.

A full assessment of each consultant's tone system and lexicon is therefore a~necessary first step before selecting compound nouns for elicitation. Only on this basis can cross-speaker differences in the tone rules be reliably brought out. 

To begin with a~comparison within the same generation, \tabref{tab:differencesbetweenspeakersf4andm21} presents two differences
between F4 and M21.


\begin{table}%[t]
\caption{Differences between speakers F4 and M21 in the tones of compounds.}
{\renewcommand{\arraystretch}{1.15}
\begin{tabularx}{\textwidth}{ P{26mm} P{27mm} P{27mm} Q }
%\begin{tabularx}{\textwidth}{ P{15mm}@{\hspace{5mm}} P{15mm} P{10mm} Q }
\lsptoprule
	input tones & \multicolumn{3}{c}{output tones, with example compounds}\\
	 & F4 & M21 & meaning\\ \midrule
	M+M (σ+σ) & \#H  & L\# &\\ 
	 &  \ipa{lɑ˧-bv̩\#˥} & \ipa{lɑ˧-bv̩˩} & tiger’s intestine\\ 
	\addlinespace \hdashline \addlinespace
	\#H+H\$ (σσ+σσ) & H\#, H\$ or \#H--  & --L  &\\
	  &  \ipa{ʐwæ˧zo˧-hu˧mi˥} &  \ipa{ʐwæ˧zo˧-hu˩mi˩} & colt’s stomach\\
\lspbottomrule
\end{tabularx}}
\label{tab:differencesbetweenspeakersf4andm21}
\end{table}

The output obtained when combining two M tones over monosyllables differs between the two speakers, and in both cases, it is somewhat unexpected: why not simply concatenate the input tones, yielding M tone for the compound, as occurs in the other M+M compounds (σσ+σσ,
σσ+σ, and σ+σσ)? For both F4 and M21, the outcome deviates from the apparent default. Likewise, the combination of \#H and H\$ over disyllables (σσ+σσ) differs between the two speakers, who
both rejected the other’s \is{variants}variant when I tested it on them. 

An intriguing characteristic of these patterns is their closeness: despite the differences, M21’s patterns bear a family resemblance to F4’s. For instance, M21 assigns --L tone to the phrase ‘colt’s stomach’ (σσ+σσ), a tone also found in F4’s data, albeit in compounds where the input nouns with these tones are monosyllabic (σ+σ). Such observations suggest that subtle processes of phonological convergence may be at play, whereby individuals bound by social ties tend to accommodate to one another's tone patterns. One could speculate that speakers seeking to foster a~sense of community, for instance in relaxed conversations with relatives, occasionally adopt new patterns to emphasize shared linguistic ground over divergence. Given that a single surface tone pattern is open to several phonological interpretations (e.g.~/M.M/ may be the realization of underlying \mbox{//M//} or \mbox{//\#H//}), this process of accommodation could lead to a~proliferation of \is{variants}variant forms across speakers. In other words, speakers' adoption of tone patterns from their customary interlocutors could explain the extent of the observed pool of \is{morphotonology}morphotonological \isi{variation}.

By a~related process, speakers may tend to select from within this pool of \isi{variation} those patterns that they believe will be most accessible to their addressee, avoiding variants that they feel are most sharply at variance with the addressee's linguistic habits. 
Conversely, self-assertive speakers wishing to distance themselves from the addressee might favour patterns that they feel are most divergent from their interlocutor’s norms, potentially 
leading to the spread and eventual stabilization of a particular \is{variants}variant within that speaker’s idiolect.

To test these hypotheses, one could examine dialogues, comparing F4's tone patterns in conversations with different family members~-- whose own tone systems would need to be documented with the greatest possible precision. This line of research holds promise for identifying key mechanisms in the evolution of the tone system, while also offering insights into its synchronic state. However, carrying out such a study would be far from straightforward. As anthropologists are well aware, speech recording in the uplands of Socialist Asia is a sensitive activity \citep{milan2024_gathering}, posing particular challenges for the collection of {spontaneous} conversations from multiple speakers.

\subsection{Exceptional items}
\label{sec:exceptionalitems}

Some compounds have lexical tones that differ from those expected under
the synchronic tone rules. The irregularity may stem either from the determiner, as in the first four lines of \tabref{tab:compoundsdiffer}, or from the head, as in the last line. These examples
are examined individually below.

\begin{sidewaystable}%[t]
    \caption{Compounds whose tones differ from those expected under the synchronic tone rules.}
{\renewcommand{\arraystretch}{1.35}
\begin{tabularx}{\textwidth}{P{50mm} P{30mm} P{30mm} l l l }
    \lsptoprule
 observed compounds & tone pattern & expected pattern & irregular word & meaning & tone\\ 
	\midrule
 \ipa{nɑ˩hĩ˧-kʰɯ˥dʑi˩} ‘Naxi leggings’; \ipa{nɑ˩hĩ˧-bɑ˧lɑ˥} ‘Naxi clothes’ & LM+\#H--; \newline LM+H\#-- & LM+MH\#--; \newline LM+\#H-- & 	\ipa{nɑ˩hĩ\#˥} & Naxi & LM+\#H\\
 \ipa{ɲi˧gɤ˧-dʑɯ˧˥} ‘mucus’ & MH\# & \#H & \ipa{ɲi˧gɤ\#˥} & nose & \#H\\
 \ipa{mv̩˧-ʁo˥} ‘sky’ & H\# & \#H & 	\ipa{mv̩˥} & sky & \#H\\
 \ipa{ʝi˧bv̩˧˥} ‘cows’ stable’ & MH\# & \#H & \ipa{ʝi˥} & ox & \#H\\
\addlinespace \hdashline \addlinespace \ipa{lv̩˧mi˧-tsɑ˩bɤ˩} ‘fine sand’; \ipa{qʰɑ˧dze˧-tsɑ˩bɤ˩} ‘sweetcorn flour’; \ipa{dze˧ɭɯ˧-tsɑ˩bɤ˩} ‘wheat flour’ & --L & M & \ipa{tsɑ˧bɤ˧} & powder & M\\
    \lspbottomrule
\end{tabularx}}
\label{tab:compoundsdiffer}
\end{sidewaystable}

\subsubsection{The noun ‘Naxi’}
\label{sec:thenounnaxi}

The noun ‘Naxi’ (/\ipa{nɑ˩hĩ\#˥}/, tone: LM+\#H) yields irregular tonal patterns in two quadrisyllabic compounds (see the recording \textit{DetermCompounds16} \pandoi{0004491}):
\begin{enumerate}[label=(\roman*)]
\item With MH\# tone: /\ipa{nɑ˩hĩ˧-kʰɯ˥dʑi˩}/ ‘Naxi leggings’. The expected pattern would be
  $\dagger${\kern2pt}\ipa{nɑ˩hĩ˧-kʰɯ˧dʑi˧˥} (underlying tone: LM+MH\#--), but this form is
  not acceptable. The observed tone is LM+\#H--. A compound that follows the regular tone pattern
  is /\ipa{nɑ˩hĩ˧-ŋwɤ˧pʰæ˧˥}/ ‘Naxi tile’.
\item With L tone: /\ipa{nɑ˩hĩ˧-bɑ˧lɑ˥}/ ‘Naxi clothes’. The expected pattern would be
  $\dagger${\kern2pt}\ipa{nɑ˩hĩ˧-bɑ˥lɑ˩} (underlying tone: LM+\#H--, which can also be described as
  LM+MH\#--), but this form is also not acceptable. The observed tone is LM+H\#. A regularly patterned compound is /\ipa{nɑ˩hĩ˧-sɯ˥tʰi˩}/ ‘Naxi knife’.
\end{enumerate}
The noun ‘Naxi’ has a~special status in Yongning Na. On the one hand, it refers to an ethnic group perceived
as distinct from the Na: the Naxi of the Lijiang plain, some of whom settled in Yongning
in the early 20\textsuperscript{th} century but have retained their distinct costumes and language (on the cultural divide between the Naxi and Na, see Appendix~\ref{chap:historyanthropologysociology}, in particular the end of \sectref{sec:shih19932010andweng1993}). On the other hand, it
consists of the \isi{endonym} of the Na combined with the word for ‘person, human being’, which creates a potential for ambiguity: its
independence from ‘Na’ is problematic. The exceptional treatment of this noun may reflect a deliberate effort to avoid its perception as
a~compound meaning ‘Na person’.


\subsubsection{The nouns ‘nose’ and ‘hair’}
\label{sec:thenounsnoseandhair}

‘Nasal mucus’ is /\ipa{ɲi˧gɤ˧-dʑɯ˧˥}/ (tone: MH\#); based on its component nouns, {\linebreak}/\ipa{ɲi˧gɤ\#˥}/ ‘nose’ and
/\ipa{dʑɯ˩}/ ‘water’, the expected tone would be \#H ($\dagger${\kern2pt}\ipa{ɲi˧gɤ˧-dʑɯ\#˥}). Similarly, ‘hair (on the head)’,
/\ipa{ʁo˧hṽ̩˧˥}/, also has MH\# tone, whereas the expected tone would be \#H ($\dagger${\kern2pt}\ipa{ʁo˧hṽ̩\#˥}), given that both input nouns carry H tone: /\ipa{ʁo˥}/ ‘head; top’ and /\ipa{hṽ̩˥}/ ‘hair’.

The \isi{etymology} of these two words is self-evident. However, mucus is not merely
a~type of water (‘nose-water’). The difference between hair on the head and on the body may seem subtler,
yet many languages distinguish between the two with distinct lexical roots, e.g.~\ili{French} and \ili{Lao} \citep[187-188]{enfield2006body}. In Na, 
‘hair (on the head)’ and ‘body hair’ are both lexicalized compounds; the latter is /\ipa{ʑi˩hṽ\#˥}/, literally ‘ape hair’. The irregular tone patterns of ‘nasal mucus’ and ‘hair (on the head)’ may thus reflect early \isi{lexicalization}. The discrepancy between their tones and the output of the currently
productive tone rules for compounds could be due to earlier \is{tone rules}tone rules that applied at the time of their formation. Alternatively, their divergence from the regular tone pattern may have arisen as these compounds evolved into lexical units rather than transparent compositions. While this second possibility might seem less plausible at first glance, item-by-item tone
change accompanying \isi{lexicalization} is a~salient characteristic of the tone system of \il{Laze|textbf}Laze,
a~language closely related to Na.\footnote{Laze has four lexical tones for monosyllables (for predicates: H, M, L, and MH; for nouns: H, M, L, and a~floating H tone). In
	theory, this could yield up to sixteen tone patterns for disyllables, but only seven are
	attested. Tone
	changes occur as {lexicalization} takes place. Numerous combinations of two input tones other than H yield H+H. For instance, ‘dog’ is /\ipa{kʰɯ˧}/ and ‘to beat’ is /\ipa{ɖɯ˩}/, a combination which should
	yield M.L but instead surfaces as H.H. This process plays a key role in the lexical
	integration of disyllables in {Laze} \citep{michaud2008a,michaud2009a,michaudetal2012c}.} This
possibility should therefore not be lightly dismissed.

\subsubsection{The noun ‘flour, powder’}
\label{sec:thenounflourpowder}

The word for ‘powder, flour’ is /\ipa{tsɑ˧bɤ˧}/, with M tone. According to the synchronically
productive rules, combining this noun with the M-tone determiners /\ipa{lv̩˧mi˧}/ ‘stone’, /\ipa{qʰɑ˧dze˧}/
‘sweetcorn’, and /\ipa{dze˧ɭɯ˧}/ ‘wheat’ should yield a~simple M-tone output,
e.g.~$\dagger${\kern2pt}\ipa{lv̩˧mi˧-tsɑ˧bɤ˧} for ‘fine sand’. But the attested forms, shown in \tabref{tab:powder}, all follow an M.M.L.L tone pattern, corresponding to underlying --L (an L tone on the second part of the compound). 

\begin{table}%[t]
	\caption{Compounds with /\ipa{tsɑ˧bɤ˧}/ ‘flour, powder’.}
	{\renewcommand{\arraystretch}{1.35}
		\begin{tabularx}{\textwidth}{ P{21mm} P{30mm} P{28mm} Q }
			%\begin{tabularx}{\textwidth}{ P{15mm}@{\hspace{5mm}} P{15mm} P{10mm} Q }
			\lsptoprule
			determiner & \multicolumn{3}{c}{compound with /\ipa{tsɑ˧bɤ˧}/ ‘flour, powder’ as head noun}\\
			& expected tone: M & attested tone:  --L  & meaning\\ \midrule
			/\ipa{lv̩˧mi˧}/ ‘stone’ & $\dagger${\kern2pt}\ipa{lv̩˧mi˧-tsɑ˧bɤ˧}  & \ipa{lv̩˧mi˧-tsɑ˩bɤ˩} & fine sand\\ 
			/\ipa{qʰɑ˧dze˧}/ ‘sweetcorn’ &  $\dagger${\kern2pt}\ipa{qʰɑ˧dze˧-tsɑ˧bɤ˧} & \ipa{qʰɑ˧dze˧-tsɑ˩bɤ˩} & sweetcorn flour\\ 
			/\ipa{dze˧ɭɯ˧}/ ‘wheat’ &  $\dagger${\kern2pt}\ipa{dze˧ɭɯ˧-tsɑ˧bɤ˧} &  \ipa{dze˧ɭɯ˧-tsɑ˩bɤ˩} & wheat flour\\
			\lspbottomrule
		\end{tabularx}}
		\label{tab:powder}
	\end{table}

The principle applied in the present study is that two lexical items of similar phonological structure whose tones differ in at least one morphosyntactic context must be assigned to different lexical tone categories. Mechanical application of this principle would lead to recognizing 
%\rephrase{of the word}%
%{of a word such as }
/\ipa{tsɑ˧bɤ˧}/ ‘flour, powder’ as the sole member of an umpteenth (twelfth) tonal category of disyllabic nouns. But it makes much more sense to seek an explanation for the irregular behaviour of the compounds in \tabref{tab:powder}.

The word /\ipa{tsɑ˧bɤ˧}/ is a~\ili{Tibetan} \is{loanwords}loanword (from \textit{rtsam pa} ‘roasted flour’). The tone lowering observed in the latter part of the compounds in \tabref{tab:powder} is part of a broader pattern. It echoes observations about a~larger set of words of \ili{Tibetan} origin: given names. Compounds involving \ili{Tibetan} loanwords will be referred to for short as ‘\ili{Tibetan} compounds’. This topic will be taken up in the analysis of given names in \sectref{sec:anindependentsetoffactscompoundgivennamesandtermsofaddress}.

\subsubsection{The noun ‘sky’}
\label{sec:thenounsky}

The disyllabic form of the noun ‘sky’ is /\ipa{mv̩˧ʁo˥\$}/. The \is{monosyllables}monosyllabic root for ‘sky’ is
/\ipa{mv̩˥}/. If the second syllable were /\ipa{ʁo˥}/ ‘head; top’, the expected tone of the compound would be \#H tone, based on the regular patterns set out in Tables~\ref{tab:abstractmonosyllabicmonosyllables} to \ref{tab:abstracttrisyllabic}. But the second syllable could also be the \is{postpositions}postposition ‘on’, which is itself likely to have \is{grammaticalization}grammaticalized from ‘head’. This \is{postpositions}postposition is no longer in common use (the common form is //\ipa{bi˩}// ‘on; at’: see \sectref{sec:ltoneencliticspluralandassociativeplural}), making it difficult to study its combinatorial properties and establish its lexical tone.


\section{Coordinative compounds}
\label{sec:coordinativecompounds}

\subsection{The main facts}
\label{sec:themainfactscoordinativecompounds}

In the closely related language \ili{Naxi}, the tones of coordinative compounds are simply the
concatenation of those of their constituents. For instance, /\ipa{ɲi˧nv̩˩-jæ˧kæ˩zɯ˧}/ ‘wife and husband’ is built from /\ipa{ɲi˧nv̩˩}/ ‘wife’ and /\ipa{jæ˧kæ˩zɯ˧}/ ‘husband’.\footnote{This tonally inert compound is of little phonological interest; on the other hand, it has some ethnolinguistic significance. As the {Naxi} of Lijiang like to point out, this compound places the wife before the
husband, in contradiction of Confucian principles. About the family structure of the Na and Naxi, and its history, see \sectref{sec:anthropologicalresearchthefascinationofnafamilystructure}.} In Yongning Na, by contrast, coordinative compounds are tonally active to a~degree comparable with determinative compounds.

However, coordinative compounds are less common than determinative compounds and more difficult to
elicit systematically. Syntactically, coordinative constructions can apply to any pair of nouns. This is illustrated by the names of public houses in \ili{English}: combinations like “Fox and Hounds” or “Dog and Duck” refer to hunting traditions, while others, such as “Bear and
Ragged Staff”, refer to heraldry. Once the pattern is established, new coordinative compounds can be created at will, such as the humorous “Snail and
Salad”, where the relationship between the two terms~-- and their relationship to the food served in
the pub~-- is offered to the customer’s fancy. 

In Yongning Na, any two nouns can be linked by
means of the \is{conjunctions}conjunction /\ipa{lɑ˧}/ ‘and’, but coordinative compound nouns are not as easy to coin: the two nouns must refer to entities that are commonly paired together.

Three main sources of coordinative compounds were identified: 
\begin{itemize}
    \item Pairs of animal names denoting the two sexes, or one sex and an offspring, such as ‘ewe
and ram’, ‘ram and ewe’, ‘ewe and lamb’, and ‘ram and lamb’; 
    \item Pairs of kinship terms, such as ‘uncle and nephew’ and ‘mother and daughter’;
    \item Successive numerals followed by the same classifier, as in ‘two or three years’, ‘five or six months’, or ‘four or five days’.
\end{itemize}

A~broad sample of the first
two types can be found in the online recording \textit{CoordCompounds} \pandoi{0004561}, while the third type appears in
\textit{CoordCompounds2} \pandoi{0004662} and \textit{CoordCompounds3} \pandoi{0004869}. Since the consultant (Mrs.\ Latami, F4) preferred to remain within the bounds of common
sense, non-matching pairs such as ‘mother and nephew’ or ‘grandmother and brother’
were avoided. 

The data is set out in Tables \ref{tab:coordinativecompoundsmono}-\ref{tab:examplesofcoordinativeHpound}. As elsewhere, a~slash separates
variants. Some tone patterns proposed by the investigator but rejected by the consultant
are marked with a~double dagger $\ddagger$ in the output column. For example, given that the combination /\ipa{ʐv̩˩ɬi˩-ŋwɤ˩ɬi˩˥}/{\kern2pt}\ipa{≈}{\kern2pt}/\ipa{ʐv̩˩ɬi˩-ŋwɤ˥ɬi˩}/ ‘four or five months’ has two possible variants, an attempt was made to extend the L-tone \is{variants}variant /\ipa{ʐv̩˩ɬi˩-ŋwɤ˩ɬi˩˥}/ to other combinations of two L-tone nouns, such as
‘nephews and nieces’ /\ipa{ze˩v̩˩-ze˥mi˩}/. The consultant rejected the L-tone {variant} ($\ddagger${\kern2pt}\ipa{ze˩v̩˩-ze˩mi˩˥}). This piece of information is indicated in the output column by ‘($\ddagger${\kern2pt}L)’.

Three suffixes appear repeatedly in the table: the female/{\allowbreak}{augmentative} \is{suffixes}suffix /\ipa{-mi˩}/, the male
\is{suffixes}suffix /\ipa{-pʰv̩˥}/, and the child/{\allowbreak}male/{\allowbreak}{diminutive} suffix /\ipa{-zo˥}/. These suffixes are discussed in more detail in~\sectref{sec:thegendersuffixes}.


\begin{table}
  \caption{Coordinative compounds of fewer than four syllables, arranged by input tones.}
  \begin{tabularx}{\textwidth}{ l Q l l }
    \lsptoprule
  	compound & meaning & input & output\\ \midrule
	\ipa{mv̩˧di˧˥} & universe (‘sky’+‘earth’) & H and LM & MH\#\\ \addlinespace \hdashline \addlinespace
	\ipa{zo˧mv̩˥} & child (‘son’+‘daughter’) & H and LH & H\#\\ \addlinespace \hdashline \addlinespace
	\ipa{ə˧mi˧-mv̩˩} & mother and daughter & M and LH & --L\\ \addlinespace \hdashline \addlinespace
	\ipa{ə˧mi˧-zo\#˥} & mother and son & M and H & \#H\\ \addlinespace \hdashline \addlinespace
	\ipa{ə˧dɑ˧-mv̩˥} & father and daughter & H\$ and LH & H\#\\ \addlinespace \hdashline \addlinespace
	\ipa{ə˧dɑ˧-zo\#˥} & father and son & H\$ and H & \#H\\
\lspbottomrule
  \end{tabularx}
\label{tab:coordinativecompoundsmono}
\end{table}


Most examples are quadrisyllabic, formed from two input disyllables (σσ+σσ). The two disyllabic examples
(σ+σ) at the top of \tabref{tab:coordinativecompoundsmono} are written without a~hyphen, reflecting the intuition that
they are more strongly integrated than the others. 

\begin{table}[b]
  \caption{Quadrisyllabic compounds with M as the first input tone.}
  \begin{tabularx}{\textwidth}{ l Q l l }
    \lsptoprule
  	compound & meaning & input & output\\ \midrule
	\ipa{ə˧pʰv̩˧-ʐv̩˧v̩\#˥} & great-uncle and great-nephews & M and M & \#H\\
	\ipa{ə˧pʰv̩˧-ʐv̩˧mi\#˥} & great-uncle and great-nieces &  &\\
	\ipa{ə˧si˧-ə˧pʰv̩\#˥} & 3\textsuperscript{rd}-generation ancestors &  &\\
	\ipa{ə˧si˧-ʐv̩˧mi\#˥} & (great-)grandmother and granddaughters &  &\\
	\ipa{ə˧si˧-ʐv̩˧v̩\#˥} & great-grandmother and grandsons &  &\\ \addlinespace \hdashline \addlinespace
	\ipa{gv̩˧dv̩˧-gv̩˧mi˧} & (human) body & M and M & M\\
	\ipa{jo˧mi˧-po˧lo˧} & ewe and ram &  &\\
	\ipa{ɖʐwæ˧mi˧-ɖʐwæ˧pʰv̩˧} & male and female sparrow &  &\\
	\ipa{ɖɯ˧ɬi˧-ɲi˧ɬi˧} & one or two months &  &\\ \addlinespace \hdashline \addlinespace
	\ipa{ɲi˧ɬi˧-so˩ɬi˩} & two or three months & M and M & --L\\ \addlinespace \hdashline \addlinespace
	\ipa{bæ˧mi˧-bæ˧pʰv̩\#˥} & female and male duck & M and \#H & \#H\\
	\ipa{bæ˧mi˧-bæ˧zo\#˥} & female duck and duckling &  &\\
	\ipa{bv̩˧mi˧-bv̩˧zo\#˥} & female yak and yak calf &  &\\ \addlinespace \hdashline \addlinespace
   \ipa{ʂɯ˧ɬi˧-hõ˧ɬi\#˥} & seven or eight months & M and H\$ & \#H\\ \addlinespace \hdashline \addlinespace
	\ipa{ə˧mi˧-ze˩mi˩} & aunt and niece & M and L & --L\\
	\ipa{ə˧mi˧-ze˩v̩˩} & aunt and nephew &  &\\
	\ipa{bv̩˧mi˧-bv̩˩ʂwæ˩} & female and male yak &  &\\
	\ipa{dzo˧mi˧-dzo˩pʰv̩˩} & female and male lizard &  &\\
	\ipa{so˧ɬi˧-ʐv̩˩ɬi˩} & three or four months &  &\\
   \lspbottomrule
  \end{tabularx}
\label{tab:examplesofcoordinativecompoundsarrangedbyinputtones}
\end{table}


All four trisyllabic examples follow the structure
\textit{disyllable plus monosyllable} (σσ+σ). This pattern arises from morphological and semantic factors rather than a~phonological preference. All kinship terms referring to one's elders (e.g.\ elder sister or brother, mother, aunts, father, uncles) include a schwa prefix (presented in \sectref{sec:thekinshipprefix}), making them disyllabic, whereas `daughter' and `son' are monosyllabic. Since terms for elders conventionally precede those for younger family members, the four kinship pairs in \tabref{tab:coordinativecompoundsmono} (`mother and daughter', `mother and son', `father and daughter', and `father and son') naturally conform to the σσ+σ pattern. Thus, the prevalence of this structure stems from the morphological properties of kinship terms and their semantic ordering rather than from a~phonological preference for an initial disyllabic
\largerpage[2]
term.

Hexasyllabic compounds can also be
formed, such as /\ipa{ŋwɤ˩-ɬi˩mi˩-qʰv̩˥-ɬi˩mi˩}/ (\textit{Dog2.64)} \pandoi{0004660\#S64}) ‘the fifth and sixth months’. 
\clearpage





%Table 6c
%Table 6. in manuscript
\begin{table}[p]
  \caption{Quadrisyllabic compounds with \#H as the first input tone.}
\scalebox{0.92}{
  \begin{tabularx}{\textwidth}{ P{31mm} Q l l }
    \lsptoprule
  	compound & meaning & input & output\\ \midrule
	\ipa{gi˧zɯ˧-go˧mi\#˥} & little brothers and sisters & \#H and M & \#H\\
	\ipa{bæ˧zo˧-bæ˧mi\#˥} & duckling and female duck  &  &\\
	\ipa{bæ˧pʰv̩˧-bæ˧mi\#˥} & male duck and female duck  &  &\\ \addlinespace \hdashline \addlinespace
	\ipa{tsʰɯ˧zo˧-to˧qɑ˥} & kids and little nanny goats & \#H and M & H\#\\ \addlinespace \hdashline \addlinespace
	\ipa{ʐv̩˧v̩˥-ʐv̩˩mi˩} & grandchildren & \#H and \#H & H\#--\\ \addlinespace \hdashline \addlinespace
	\ipa{hwɤ˧pʰv̩˧-hwɤ˧zo\#˥} / \ipa{hwɤ˧pʰv̩˧-hwɤ˥zo˩} & tom-cat and kitten & \#H and \#H & \#H~/ \#H--\\
	\ipa{ho˧mi˧-ho˧pʰv̩\#˥} / \ipa{ho˧mi˧-ho˥pʰv̩˩} & female and male pheasant &  &\\
	\ipa{dʑi˧mi˧-dʑi˧zo\#˥} / \ipa{dʑi˧mi˧-dʑi˥zo˩} & female and baby buffalo &  &\\
	\ipa{dʑi˧zo˧-dʑi˧mi\#˥} / \ipa{dʑi˧zo˧-dʑi˥mi˩} & baby and female buffalo &  &\\
	\ipa{lɑ˧mi˧-lɑ˧pʰv̩\#˥} / \ipa{lɑ˧mi˧-lɑ˥pʰv̩˩} & female and male tiger &  &\\
	\ipa{lɑ˧mi˧-lɑ˧zo\#˥} / \ipa{lɑ˧mi˧-lɑ˥zo˩} & female and baby tiger &  &\\
	\ipa{ʐv̩˧ɲi˧-ŋwɤ˧ɲi\#˥} / \ipa{ʐv̩˧ɲi˧-ŋwɤ˥ɲi˩} & four or five days &  &\\ \addlinespace \hdashline \addlinespace
	\ipa{ʁv̩˧pʰv̩˧-ʁv̩˧mi\#˥} & male and female crane & \#H and MH & \#H\\ \addlinespace \hdashline \addlinespace
	\ipa{hwɤ˧pʰv̩˧-hwɤ˧mi˥} & tom-cat and she-cat & \#H and H\$ & H\#\\
	\ipa{hwɤ˧zo˧-hwɤ˧mi˥} & cats: kitten and parents &  &\\ \addlinespace \hdashline \addlinespace
	\ipa{ŋwɤ˧ɲi˧-qʰv̩˩ɲi˩} & five or six days & \#H and H\$ & --L\\
	\ipa{ʂɯ˧ɲi˧-hõ˩ɲi˩} & seven or eight days &  &\\ \addlinespace \hdashline \addlinespace
	\ipa{ʐwæ˧zo˧-ʐwæ˥mi˩}~/ \ipa{ʐwæ˧zo˧-ʐwæ˧mi˥} & colt and mare & \#H and L & \#H--/ H\#\\
	\ipa{pʰɤ˧pʰv̩˧-pʰɤ˥mi˩} / \ipa{pʰɤ˧pʰv̩˧-pʰɤ˧mi˥} & male and female hyena &  &\\ \addlinespace \hdashline \addlinespace
	\ipa{gv̩˧ɲi˧-tsʰe˩ɲi˩} / \ipa{gv̩˩ɲi˩-tsʰe˩ɲi˥} & nine or ten days & \#H and L & --L~/ L+H\#\\ \addlinespace \hdashline \addlinespace
	\ipa{kʰv̩˧zo˥-kʰv̩˩mv̩˩} / \ipa{kʰv̩˧zo˧-kʰv̩˧mv̩˥} & male and female puppies & \#H and H\# & H\#-- / H\#\\
	\lspbottomrule
  \end{tabularx}}
\label{tab:examplesofcoordinativecFLOATINGH}
\end{table}

\begin{table}%[t]
  \caption{Quadrisyllabic compounds with MH\# as the first input tone.}
  \begin{tabularx}{\textwidth}{ P{29mm} Q l l }
    \lsptoprule
  	compound & meaning & input & output\\ \midrule
   \ipa{ə˧ʑi˧-ə˧pʰv̩˧˥} & elders, grandparents & MH\# and M & MH\#\\ \addlinespace \hdashline \addlinespace
   \ipa{æ˧mv̩˧-go˧mi˥} & sisters, female siblings & MH\# and M & H\#\\ \addlinespace \hdashline \addlinespace
   \ipa{ə˧ʑi˧-ʐv̩˥v̩˩} & grandmother and grandsons & MH\# and \#H & \#H--\\
	\ipa{ə˧ʑi˧-ʐv̩˥mi˩} & grandmother and granddaughter &  &\\
	\ipa{æ˧mv̩˧-gi˥zɯ˩} & brethren, brothers &  &\\ \addlinespace \hdashline \addlinespace
	\ipa{ɖɯ˧kʰv̩˧-ɲi˥kʰv̩˩} & one or two years & MH\# and MH\# & MH\#--\\ \addlinespace \hdashline \addlinespace
	\ipa{ə˧v̩˧-ze˥v̩˩} & uncle and nephew & MH\# and L & MH\#--\\
	\ipa{ə˧v̩˧-ze˥mi˩} & uncle and niece &  &\\
	\ipa{zo˧hṽ̩˧-mv̩˥zo˩} & descendants &  &\\ \addlinespace \hdashline \addlinespace
	\ipa{ɲi˧kʰv̩˧-so˧kʰv̩˥} ($\ddagger${\kern2pt}\ipa{ɲi˧kʰv̩˧-so˥kʰv̩˩}) & two or three years & MH\# and L & H\#\\ \addlinespace \hdashline \addlinespace
	\ipa{ʂɯ˧kʰv̩˧-hõ˥kʰv̩˩} / \ipa{ʂɯ˧kʰv̩˧-hõ˧kʰv̩˥} & seven or eight years & MH\# and H\# & MH\#-- / H\#\\
   \lspbottomrule
  \end{tabularx}
\label{tab:examplesofcoordinativeMH}
\end{table}

%Table 6e
%Table 6. in manuscript
\begin{table}[p]
  \caption{Quadrisyllabic compounds with H\$ as the first input tone.}
\scalebox{0.95}{
  \begin{tabularx}{\textwidth}{ P{33mm} Q l P{15mm} }
    \lsptoprule
  	compound & meaning & input & output\\ \midrule
	\ipa{tsʰɯ˧mi˧-po˧lo˥ } & nanny goat and billy goat & H\$ and M & H\#\\ \addlinespace \hdashline \addlinespace
	\ipa{ə˧dɑ˧-ə˧mi\#˥ } & father and mother, parents & H\$ and M & \#H\\ \addlinespace \hdashline \addlinespace
	\ipa{qʰv̩˧ɬi˥-ʂɯ˩ɬi˩ } & six or seven months & H\$ and M & H\#--\\ \addlinespace \hdashline \addlinespace
	\ipa{ə˧ɲi˧tsʰi˧ɲi\#˥} & these days & H\$ and \#H & \#H\\
	\ipa{ə˧ʝi˧-tsʰi˧ʝi\#˥} & these years &  &\\
	\ipa{ʈʂʰæ˧mi˧-ʈʂʰæ˧zo\#˥} & doe and stag &  &\\ \addlinespace \hdashline \addlinespace
	\ipa{hwɤ˧mi˧-hwɤ˧zo\#˥} / \ipa{hwɤ˧mi˧-hwɤ˧zo˥\$} & she-cat and kitten & H\$ and \#H & \#H~/ H\$\\ \addlinespace \hdashline \addlinespace
   \ipa{hwɤ˧mi˧-hwɤ˥pʰv̩˩} / \ipa{hwɤ˧mi˧-hwɤ˧pʰv̩\#˥ / hwɤ˧mi˧-hwɤ˧pʰv̩˥\$} & she-cat and tom-cat &
   H\$ and \#H & \#H-- / \#H~/ H\$\\ \addlinespace \hdashline \addlinespace
   \ipa{qʰv̩˧ɲi˥-ʂɯ˩ɲi˩} / \ipa{qʰv̩˧ɲi˧-ʂɯ˥ɲi˩ / qʰv̩˧ɲi˧-ʂɯ˧ɲi\#˥} & six or seven days & H\$ and \#H & H\#-- / \#H-- / \#H\\
	\ipa{($\ddagger${\kern2pt}qʰv̩˧ɲi˧-ʂɯ˧ɲi˥\$)} &  &  &\\
	\ipa{hõ˧ɲi˥-gv̩˩ɲi˩} /  & eight or nine days &  &\\
	\ipa{hõ˧ɲi˧-gv̩˥ɲi˩} /  &  &  &\\
	\ipa{hõ˧ɲi˧-gv̩˧ɲi\#˥} &  &  &\\
	(\ipa{$\ddagger${\kern2pt}hõ˧ɲi˧-gv̩˧ɲi˥\$}) &  &  &\\ \addlinespace \hdashline \addlinespace
   \ipa{ɲi˧ɲi˧-so˧ɲi˥} ($\ddagger${\kern2pt}\ipa{ɲi˧ɲi˥-so˩ɲi˩}) & two or three days & H\$ and MH\# & H\#\\ \addlinespace \hdashline \addlinespace
	\ipa{ɖɯ˧ɲi˧-ɖɯ˥hɑ̃˩} ($\ddagger${\kern2pt}\ipa{ɖɯ˧ɲi˧-ɖɯ˧hɑ̃˥}) & one day and one night & & \#H--\\ \addlinespace \hdashline \addlinespace
	\ipa{ʈʂʰæ˧mi˧-ʈʂʰæ˧zo\#˥} & doe and fawn & H\$ and H\$ & \#H\\
	\ipa{ɖɯ˧ɲi˧-ɲi˧ɲi\#˥} & one or two days &  &\\ \addlinespace \hdashline \addlinespace
	\ipa{hõ˧ɬi˥-gv̩˩ɬi˩} & eight or nine months & H\$ and L & H\#--\\
	\ipa{ɲi˧ɲi˥} {\kern2pt}|{\kern2pt} \ipa{-so˩ɲi˩˥} & two or three days &  &\\ \addlinespace \hdashline \addlinespace
	\ipa{ə˧ʝi˧-ʂɯ˥ʝi˩} & in the past & H\$ and LM+\#H & \#H--\\
\lspbottomrule
  \end{tabularx}}
\label{tab:examplesofcoordinativeDOLLAR}
\end{table}

\begin{table}%[t]
  \caption{Quadrisyllabic compounds with L as the first input tone.}
  \begin{tabularx}{\textwidth}{ P{30mm} Q l l }
    \lsptoprule
  	compound & meaning & input & output\\ \midrule
	\ipa{bv̩˩ʂwæ˩-bv̩˥mi˩} / \ipa{bv̩˩ʂwæ˩-bv̩˩mi˩} & male yak and female yak & L and M & L~/
   L+\#H--\\
   \ipa{kɤ˩pʰv̩˩-kɤ˩mi˥} / \ipa{kɤ˩pʰv̩˩-kɤ˥mi˩} & male and female falcon &  &\\
	\ipa{dzo˩pʰv̩˩-dzo˩mi˩} / \ipa{dzo˩pʰv̩˩-dzo˥mi˩} & male and female lizards &  &\\ \addlinespace \hdashline \addlinespace
   \ipa{mv̩˩zo˩-ə˥mi˩} / \ipa{mv̩˩zo˩-ə˩mi˥} & young woman and (her) mother & L and M & L+\#H-- / L+H\#\\
	\ipa{gv̩˩ɬi˩-tsʰe˥ɬi˩} / \ipa{gv̩˩ɬi˩-tsʰe˩ɬi˥} & nine or ten months &  &\\ \addlinespace \hdashline \addlinespace
	\ipa{mv̩˩zɯ˩-ni˥mi˩} & brothers and sisters & L and \#H & L+\#H--\\ \addlinespace \hdashline \addlinespace
\ipa{ʐwæ˩mi˩-ʐwæ˩zo˩} & mare and colt & L and \#H & L\\
	\ipa{so˩ɲi˩-ʐv̩˩ɲi˩} & three or four days &  &\\ \addlinespace \hdashline \addlinespace
   \ipa{ŋwɤ˩ɬi˩-qʰv̩˥ɬi˩} & five or six months & L and H\$ & \#H--\\ \addlinespace \hdashline \addlinespace
	\ipa{ʝi˩mi˩-ʐɤ˥qo˩} & cow and calf & L and L & L+\#H-- ($\ddagger${\kern2pt}L)\\
	\ipa{ze˩v̩˩-ze˥mi˩} & nephews and nieces &  &\\
	\ipa{ʝi˩bv̩˩-ʝi˥mi˩} & bull and cow &  &\\
	\ipa{pɤ˩mi˩-pɤ˥pʰv̩˩} & female and male frog &  &\\
	\ipa{pʰɤ˩mi˩-pʰɤ˥zo˩} & female hyena and hyena pup  &  &\\ \addlinespace \hdashline \addlinespace
	\ipa{ʐv̩˩ɬi˩-ŋwɤ˩ɬi˩˥} / \ipa{ʐv̩˩ɬi˩-ŋwɤ˥ɬi˩} & four or five months & L and L & L~/ L+\#H--\\
	\ipa{so˩kʰv̩˩-ʐv̩˩kʰv̩˥} & three or four years & L and L\# & L+H\#\\
   \lspbottomrule
  \end{tabularx}
\label{tab:examplesofcoordinativeL}
\end{table}


%Table 6g
%Table 6. in manuscript
\begin{table}%[t]
\caption{Quadrisyllabic compounds with L\# as the first input tone.}
  \begin{tabularx}{\textwidth}{ Q Q P{20mm} l }
    \lsptoprule
  	compound & meaning & input & output\\ \midrule
	\ipa{ʐwæ˧sɯ˩-ʐwæ˩zo˩} & stallion and colt & L\# and \#H & L\#--\\ \addlinespace \hdashline \addlinespace
	\ipa{ʐwæ˧sɯ˩-ʐwæ˩mi˩} & stallion and mare & L\# and L & L\#--\\
	\ipa{gv̩˧kʰv̩˩-tsʰe˩kʰv̩˩} & nine or ten years &  &\\ \addlinespace \hdashline \addlinespace
	\ipa{ʐv̩˧kʰv̩˩-ŋwɤ˩kʰv̩˩} & four or five years & L\# and L\# & L\#--\\ \addlinespace \hdashline \addlinespace
	\ipa{ŋwɤ˧kʰv̩˩-qʰv̩˩kʰv̩˩} & five or six years & L\# and H\# & L\#--\\
\lspbottomrule
  \end{tabularx}
\label{tab:examplesofcoordinativeLpound}
\end{table}

\begin{table}%[t]
  \caption{Quadrisyllabic compounds with LM as the first input tone.}
  \begin{tabularx}{\textwidth}{ l Q P{26mm} l }
    \lsptoprule
  	compound & meaning & input & output\\ \midrule
   \ipa{ɑ˩ʁo˧-ʑi˧dv̩˧} & the household & LM and M & LM--\\ \addlinespace \hdashline \addlinespace
	\ipa{pv̩˩tsɯ˧-pv̩˥mi˩} & small and large combs & LM+MH\# and L & LM+MH\#--\\
	\ipa{pɤ˩tɕi˧-pɤ˥mi˩} & tadpole &  &\\ \addlinespace \hdashline \addlinespace
	\ipa{æ˩mi˧-æ˧ʂwæ˥} & hen and cock & LM and H\# & LM+H\#\\
	\ipa{æ˩mi˧-æ˧tsɯ˥} & hen and chicks &  &\\
	\ipa{bo˩mi˧-bæ˧bv̩˥} & sow and piglets &  &\\ \addlinespace \hdashline \addlinespace
	\ipa{ʐæ˩pʰv̩˧-ʐæ˩mi˩} & male and female leopard & LM and LM+\#H & LM--L\\ \addlinespace \hdashline \addlinespace
	\ipa{dv̩˩mi˧-dv̩˥pʰv̩˩} & female and male weasels & LM+\#H and LM & LM+\#H--\\ \addlinespace \hdashline \addlinespace
   \ipa{ɑ˩mi˧-ɑ˥pʰv̩˩} & female and male goose  & \multirow{2}{26mm}{LM+\#H and LM+\#H} & LM+\#H--\\
   \ipa{ɖɯ˩zo˧-ɖɯ˥mi˩} & female and male mule  &  &\\
   \lspbottomrule
  \end{tabularx}
\label{tab:examplesofcoordinativeLM}
  \end{table}

\clearpage

\begin{table}%[t]
  \caption{Compounds of four to six syllables with H\# as the first input tone.}
  \begin{tabularx}{\textwidth}{ l Q l l }
    \lsptoprule
  	 compound & meaning & input & output\\ \midrule
   \ipa{qʰv̩˧kʰv̩˥-ʂɯ˩kʰv̩˩} & six or seven years & H\# and MH\# & H\#--\\ \addlinespace \hdashline \addlinespace
	\ipa{hõ˧kʰv̩˥-gv̩˩kʰv̩˩} & eight or nine years & H\# and L\# & H\#--\\ \addlinespace \hdashline \addlinespace
	\ipa{æ˧ʂwæ˥-æ˩mi˩} & cock and hen & H\# and LM & H\#--\\
	\ipa{ŋwɤ˩ɬi˩mi˩-qʰv̩˥ɬi˩mi˩} & the fifth and sixth months & &\\
\lspbottomrule
\end{tabularx}
\label{tab:examplesofcoordinativeHpound}
\end{table}

% The subsection below needs to start on a clean page, after all floats have been unpiled.

\subsection{Discussion: Tonal variability and lexical diversity}
\label{sec:discussiontonalvariabilityandsemanticlexicaldiversity}

% \todo{check ref to subtables} %% Alexis's comment, April 30th: Done! 
The existence of variants was already observed for some determinative compounds (see Tables~\ref{tab:abstractmonosyllabicmonosyllables} to \ref{tab:abstracttrisyllabic}), but the overall
proportion of combinations allowing variants was low: only 7 out of 257, all of which had
H\$ tone on the head noun. In the case of coordinative compounds, on the other hand, less regularity is
observed. Not only are there tonal variants, but also compounds with identical input tones that yield different outputs. Among quadrisyllabic compounds, two different outputs (on different examples) are found for no fewer than six tonal combinations: those with input tones M and M, \#H and M, MH\# and
M, H\$ and M, H\$ and \#H, and L and \#H. A~seventh combination, \#H and \#H, even has three
different outputs. Overall, for quadrisyllabic compounds, roughly one in four tone combinations has two or three distinct outputs.

The general picture is thus one of great tonal diversity. However, this does not mean that coordinative compounds are characterized by a~general looseness in tone patterns, whereby two or three tonal variants would be acceptable for any given combination of input tones. In cases where the same tonal input yields different outputs in different coordinative compounds, an attempt was made to substitute one pattern for the other. For instance, input nouns with H\$ and M tones yield a~compound with \#H tone in the case of ‘little brothers and little sisters’ (//\ipa{gi˧zɯ˧-go˧mi\#˥}//) but a~compound with H\# tone in the case of ‘kids and little nanny goats’ (//\ipa{tsʰɯ˧zo˧-to˧qɑ˥}//). On this basis, tones \#H and H\# were tested as substitutes for each other in coordinative compounds. The consultant rejected these substitutions, as documented in \tabref{tab:braveattempts}. This indicates that each coordinative compound acquires a~habitual tone pattern to the exclusion of others. Further comparison across speakers (initially within a single family, then extending to other micro-dialects) will be necessary to trace the paths through which these idiosyncratic preferences develop. 

\begin{table}%[t]
	\caption{Attempted tonal variants for coordinative compounds.}

% 	\begin{tabularx}{\textwidth}{Qlllll}
% 		\lsptoprule
% 	    meaning & input tones &	\multicolumn{2}{l}{attested form} & \multicolumn{2}{l}{attempted variant}\\ \cmidrule(lr){1-2}\cmidrule(lr){3-4}\cmidrule(lr){5-6}
% %		 & full form & tone & attempted {variant} & tone\\\midrule
% 		little brothers and little sisters & H\$ and M & \ipa{gi˧zɯ˧-go˧mi\#˥} & \#H & $\ddagger${\kern2pt}\ipa{gi˧zɯ˧-go˧mi˥} & H\#\\
% 		father and mother & H\$ and M & \ipa{ə˧dɑ˧-ə˧mi\#˥} & \#H & $\ddagger${\kern2pt}\ipa{ə˧dɑ˧-ə˧mi˥} & H\#\\
% 		kids and little nanny goats & H\$ and M & \ipa{tsʰɯ˧zo˧-to˧qɑ˥} & H\# & $\ddagger${\kern2pt}\ipa{tsʰɯ˧zo˧-to˧qɑ\#˥} & \#H\\
% 		nanny goat and billy goat & H\$ and M & \ipa{tsʰɯ˧mi˧-po˧lo˥} & H\# & $\ddagger${\kern2pt}\ipa{tsʰɯ˧mi˧-po˧lo\#˥} & \#H\\
% 		ancestors & MH\# and M & \ipa{ə˧ʑi˧-ə˧pʰv̩˧˥} & MH\# & $\ddagger${\kern2pt}\ipa{ə˧ʑi˧-ə˧pʰv̩˥} & H\#\\
% 		sisters & MH\# and M & \ipa{æ˧mv̩˧-go˧mi˥} & H\# & $\ddagger${\kern2pt}\ipa{æ˧mv̩˧-go˧mi˧˥} & MH\#\\
% 		\lspbottomrule
% 	\end{tabularx}


	\fittable{
	\begin{tabular}{llll}
		\lsptoprule
	    meaning & input tones &	 {attested form} & {attempted variant}\\
	    \midrule
		little brothers and little sisters 	& H\$ and M & \ipa{gi˧zɯ˧-go˧mi\#˥}  	& $\ddagger${\kern2pt}\ipa{gi˧zɯ˧-go˧mi˥} \\
		father and mother 			& H\$ and M & \ipa{ə˧dɑ˧-ə˧mi\#˥} 	& $\ddagger${\kern2pt}\ipa{ə˧dɑ˧-ə˧mi˥} \\
		kids and little nanny goats 		& H\$ and M & \ipa{tsʰɯ˧zo˧-to˧qɑ˥}  	& $\ddagger${\kern2pt}\ipa{tsʰɯ˧zo˧-to˧qɑ\#˥} \\
		nanny goat and billy goat 		& H\$ and M & \ipa{tsʰɯ˧mi˧-po˧lo˥}  	& $\ddagger${\kern2pt}\ipa{tsʰɯ˧mi˧-po˧lo\#˥} \\
		ancestors 				& MH\# and M & \ipa{ə˧ʑi˧-ə˧pʰv̩˧˥} 	& $\ddagger${\kern2pt}\ipa{ə˧ʑi˧-ə˧pʰv̩˥} \\
		sisters 				& MH\# and M & \ipa{æ˧mv̩˧-go˧mi˥}  	& $\ddagger${\kern2pt}\ipa{æ˧mv̩˧-go˧mi˧˥} \\
		\lspbottomrule
	\end{tabular}
	}

	\label{tab:braveattempts}
\end{table}

Tonal diversity relates in subtle ways to the semantic diversity of coordinative compounds. Importantly, the meaning of such compounds is not always predictable from their two
constituents. For instance, /\ipa{hwɤ˧zo˧-hwɤ˧mi˥}/, composed of the words for ‘kitten’ and ‘she-cat’, does not
mean ‘kitten and she-cat’ (i.e.\ a mother and her offspring) but rather refers to cats in general, as a~species. Meanwhile, the terms for male and female puppies, /\ipa{kʰv̩˧zo\#˥}/ and
/\ipa{kʰv̩˧mv̩\#˥}/ respectively, have been repurposed as names for human newborns: an unattractive name is intentionally chosen to ward off malevolent spirits believed to threat\-en the child's life. This practice is known in Chinese as “milk name” (\zh{乳名} \textit{rǔmíng}) and constitutes one of the many “names intended to avoid attracting the unwanted attention of gods, sparing the name-bearers the misfortunes wrought by the god's wrath or jealousy” \citep[118]{chen2016}. The child's real name is given only after a~couple of months, or sometimes as late as one full year after birth. Over time, the terms
/\ipa{kʰv̩˧zo\#˥}/ and /\ipa{kʰv̩˧mv̩\#˥}/, along with their compound /\ipa{kʰv̩˧zo˥-kʰv̩˩mv̩˩}/, have become
culturally specialized and are no longer used to refer to actual puppies. 

Coordinative compounds thus range from novel elicited combinations, which a consultant may never have conceptualized before (e.g.\ ‘male and female
jackal’), to highly lexicalized expressions. The overall number of examples
is too small to determine with confidence, within this diverse landscape, which outputs are currently productive. A logical next step in investigating this issue would be to assess the degree of \isi{lexicalization} of the various compounds. 

The tone patterns observed thus far are summarized in
Tables~\ref{tab:thetonepatternsofcoordinativecompoundscombinationswithmonosyllabicsecondnoun}
and
\ref{tab:thetonepatternsofcoordinativecompoundscombinationsamongdisyllables}. When
different (and mutually exclusive) patterns are attested for
different compounds, these patterns are separated by a~semi-colon. In
cases of free \isi{variation} (i.e.\ over the same compound), the patterns are
separated by a~slash. To save space, tone categories for
which no examples have been found are simply omitted from the table. 


\begin{table}%[t]
\caption{\label{tab:thetonepatternsofcoordinativecompoundscombinationswithmonosyllabicsecondnoun}The tones of coordinative compounds with {monosyllabic} second noun. A~{question} mark indicates that no example was found.}
\begin{tabularx}{.75\textwidth}{ l@{\hspace{7mm}} l@{\hspace{7mm}} l@{\hspace{7mm}} Q Q }
\lsptoprule
	\multirow{2}{12mm}[-1.6mm]{type of 1\textsuperscript{st}~noun} & \multirow{2}{11mm}[-1.6mm]{tone of 1\textsuperscript{st}~noun} & \multicolumn{3}{l}{tone of 2\textsuperscript{nd} noun}\\ \cmidrule{3-5}
	 &  & LM  & L  & \#H\\\midrule
	monosyllables & \#H & MH\# & H\# & ?\\ \addlinespace \hdashline \addlinespace
	disyllables & M & ? & --L & \#H\\
	 & H\$ & ? & H\# & \#H\\
\lspbottomrule
\end{tabularx}
\end{table}


\begin{sidewaystable}[p]
\caption{\label{tab:thetonepatternsofcoordinativecompoundscombinationsamongdisyllables}The tones of coordinative compounds consisting of two disyllables. A~{question} mark indicates that no example was found.}
{\renewcommand{\arraystretch}{1.35}
{\setlength\tabcolsep{5pt}
\begin{tabularx}{\textwidth}{ Q l l l l l l l l }
\lsptoprule
& 2\textsuperscript{nd} noun\\ \cmidrule{2-9}
	tone of 1\textsuperscript{st}~noun & M & \#H & MH\# & H\$ & L & LM+\#H & LM & H\#\\\midrule
	M & M; \#H  & \#H & ? & ? & --L & ? & ? & ?\\
	\#H & H\#; \#H & H\#--; H\#/\#H-- & \#H & H\# & \#H-- / H\# & ? & ? & H\# / H\#--\\
	MH\# & MH\#; H\# & MH\#-- & ? & ? & MH\#-- & ? & ? & ?\\
	H\$ & H\#; \#H & \#H; \#H-- / H\# & \#H-- & \#H & ? & \#H-- & ? & ?\\
	L & L+H\# / L+\#H-- & L; L+\#H-- & ? & ? & L+\#H-- & ? & ? & ?\\
	L\# & ? & L\#-- & ? & ? & L\#-- & ? & ? & ?\\
	LM+MH\# & ? & ? & ? & ? & LM+MH\#-- & ? & ? & ?\\
	LM+\#H & ? & ? & ? & ? & ? & LM+MH\#-- & LM+MH\#-- & ?\\
	LM & LM-- & ? & ? & ? & ? & LM--L & ? & LM+H\#\\
	LH & ? & ? & ? & ? & ? & ? & ? & ?\\
	H\# & ? & ? & ? & ? & ? & ? & H\#-- & ?\\
\lspbottomrule
\end{tabularx}}}
\end{sidewaystable}



Sixty percent of the combinations are identical with those found on
determinative compounds. There is a~considerable proportion of combinations for which no single example was found, however. In addition to the cells containing a~{question} mark in the table, one must also take into account the empty columns, which are omitted
from the table for conciseness. In total, only 35 combinations were observed out of a~theoretically possible 223 (6×6 for σ+σ compounds, 11×6 for σσ+σ compounds, and 11×11 for σσ+σσ compounds). Since there is no inherent restriction on input combinations, the gaps in Tables~\ref{tab:thetonepatternsofcoordinativecompoundscombinationswithmonosyllabicsecondnoun}
and
\ref{tab:thetonepatternsofcoordinativecompoundscombinationsamongdisyllables} must be considered accidental. This is due in part to the limitations of available materials, but the scarcity of examples, coupled with the diversity of their tone patterns, also raises \is{morphotonology}morphotonological questions: coordinative compounds are much less common than determinative compounds. It may be that there is no such thing as a~full-fledged set of productive \is{tone rules}tone combination rules governing the tone patterns of coordinative compounds, and that new coinages are based on \isi{analogy} with the best example at hand: an established (lexicalized) compound perceived (on grounds that may fluctuate) as falling under the same tone rules. A~touch of \isi{expressivity} may also be at play in the process: greater weight placed on one of the two elements in the compound, for \is{stylistics}semantic-stylistic reasons, could contribute to the selection of one tone pattern over another. This could help explain the observed diversity. 


\section{Compound given names and other ``Tibetan compounds''}
\label{sec:anindependentsetoffactscompoundgivennamesandtermsofaddress}

In Yongning, given names are of \ili{Tibetan} origin. They consist of a~combination of two disyllabic names. For instance, /\ipa{ʝi˧tɕi˧-ɖɯ˩mɑ˩}/ is composed of /\ipa{ʝi˧tɕi˧}/ and /\ipa{ɖɯ˩mɑ\#˥}/; similarly, /\ipa{ɖɯ˩ɖʐɯ˧-tsʰɯ˩ɻ̩˩}/ combines /\ipa{ɖɯ˩ɖʐɯ˧}/ and /\ipa{tsʰɯ˧ɻ̩\#˥}/. \tabref{tab:Names} provides a~list of disyllabic names and attested combinations. The corresponding forms in Written \ili{Tibetan} were proposed by Nathan Hill and Tsering Samdrup (p.c.\ 2016), and remain to be confirmed by eliciting the written forms from a~monk in Yongning. Question marks indicate uncertain identifications. A~dash ‘--’ signals the absence of attestations in the set of recorded texts, indicating the need for additional data.

\begin{table}[p!!]
	\caption{Yongning Na given names and identifications with {Tibetan} names proposed by Nathan Hill and Tsering Samdrup.}
	{\renewcommand{\arraystretch}{1.13}
		\begin{tabularx}{\textwidth}{ P{17mm} P{35mm} Q }
		\lsptoprule
		name & \ili{Tibetan} & attested combinations\\ \midrule
		\ipa{ɖɯ˩ɖʐɯ˧} & Rdo rje & \ipa{ɖɯ˩ɖʐɯ˧-tsʰɯ˩ɻ̩˩}, \ipa{ɖɯ˩ɖʐɯ˧-ɬɑ˩-tsʰo˩}\\
		\ipa{ɖɯ˩mɑ\#˥} & Sgrol ma & \ipa{ɖɯ˩mɑ˧-ɬɑ˩tsʰo˩}, \ipa{ɖɯ˩mɑ˧-pv̩˩ʈʰɯ˩}\\
		\ipa{dʑɤ˩tsʰi\#˥} & Bde skyid? & \ipa{dʑɤ˩tsʰi˥-ɖɯ˩mɑ˩}, \ipa{dʑɤ˩tsʰi˥-pv̩˩ʈʰɯ˩}\\
		\ipa{gv̩˧mɑ˧} & ? & \ipa{gv̩˧mɑ˧-tsʰɯ˩ɻ̩˩}\\
		\ipa{ʝi˧ʂɯ˥} & Ye shes & \ipa{ʝi˧ʂɯ˥-ti˩ɖo˩}\\
		\ipa{ʝi˧tɕi˧} & Yid ches? & \ipa{ʝi˧tɕi˧-ɖɯ˩mɑ˩}, \ipa{ʝi˧tɕi˧-ɬɑ˩mv̩˩}\\
		\ipa{kɤ˧zo\#˥} & Skal bzang? & \ipa{kɤ˧zo˧-tsʰɯ˩ɻ̩˩}\\
		\ipa{ki˧zo\#˥} & Skal bzang? & \ipa{ki˧zo˧-ɖɯ˩mɑ˩}, \ipa{ki˧zo˧-ɬɑ˩mv̩˩}\\
		\ipa{lɑ˩mɑ˩} & Bla ma & --\\
		\ipa{ɬɑ˧mv̩˥\$} & Lha mo & --\\
		\ipa{ɬɑ˧tsʰo\#˥} & Lha mtsho & --\\
		\ipa{nɑ˧dʑi\#˥} & Rnam rgyal & --\\
		\ipa{ɲi˩mɑ\#˥} & Nyi ma & --\\
		\ipa{no˩bv̩˧} & Nor bu & \ipa{no˩bv̩˧-tsʰɯ˩ɻ̩˩}\\
		\ipa{no˧no˧} & ? & \ipa{no˧no˧-ɖɯ˩mɑ˩}\\
		\ipa{pæ˩pʰæ˧˥} & Spen pa? & --\\
		\ipa{pi˧mɑ˧} & Padma & \ipa{pi˧mɑ˧-ɬɑ˩mv̩˩}, \ipa{pi˧mɑ˧-ɬɑ˩tsʰo˩}\\
		\ipa{pʰi˧tsʰo\#˥} & Phun tshogs & \ipa{pʰi˧tsʰo˧-ɖɯ˩ɖʐɯ˩}\\
		\ipa{pv̩˩ʈʰɯ˧} & Bu phrug? Bu khrid?  & --\\
		\ipa{ɻ̩˩ʈʂʰe\#˥} & Rin chen & \ipa{ɻ̩˩ʈʂʰe˧-ɖɯ˩mɑ˩}, \ipa{ɻ̩˩ʈʂʰe˧-tsʰɯ˩ɻ̩˩}\\
		\ipa{tɑ˩dʑɤ\#˥} & Dar rgyes? & --\\
		\ipa{ʈæ˧ʂɯ˧} & Bkra shis & \ipa{ʈæ˧ʂɯ˧-ɖɯ˩mɑ˩}, \ipa{ʈæ˧ʂɯ˧-ɬɑ˩mv̩˩}, \ipa{ʈæ˧ʂɯ˧-pæ˩pʰæ˩}, \ipa{ʈæ˧ʂɯ˧-ʈæ˩ʈv̩˩}, \ipa{ʈæ˧ʂɯ˧-tsʰi˩ti˩}\\
		\ipa{ʈæ˩ʈv̩\#˥} & Dgra 'dul? & --\\
		\ipa{tɕʰi˧ɖv̩\#˥} & Spyi 'dul? & --\\
		\ipa{ti˧ɖo˥} & ? & --\\
		\ipa{tsʰi˧ti\#˥} & ? & --\\
		\ipa{tsʰɯ˧ɻ̩\#˥} & Tshe ring & \ipa{tsʰɯ˧ɻ̩˧-lɑ˩mv̩˩}, \ipa{tsʰɯ˧ɻ̩˧-ɖɯ˩mɑ˩}, \ipa{tsʰɯ˧ɻ̩˧-ɬɑ˩mv̩˩}, \ipa{tsʰɯ˧ɻ̩˧-pʰi˩tsʰo˩}\\
		\lspbottomrule
	\end{tabularx}}
	\label{tab:Names}
\end{table}

% The full name of F4’s grandmother was /\ipa{ɖɯ˩mɑ˧-pv̩˩ʈʰɯ˩}/ (compounded from the LM-tone names /\ipa{ɖɯ˩mɑ˧}/ and /\ipa{pv̩˩ʈʰɯ˧}/). Her siblings addressed her as /\ipa{pv̩˩ʈʰɯ˧}/, using the second part of the given name (as it is pronounced on its own, and not as it appears in the compound, where both of its syllables carry L tone), and her nephews and nieces, as well as other persons from the village, addressed her respectfully as /\ipa{ə˧mi˧-pv̩˩ʈʰɯ˩}/, i.e.\ using the term of address for aunts. The lowering of the last syllable in /\ipa{ə˧mi˧-pv̩˩ʈʰɯ˩}/ is due to the general prohibition  prohibits tonal troughs such as $\ddagger${\kern2pt}M.M.L.M within a~tone group.

Compound given names constitute a specific area in the \isi{morphotonology} of Yongning Na. Their tonal behaviour differs from that of coordinative compounds. The tones of the second name are lowered to L in all cases, even when they could theoretically be expressed without contravening any {phonological rule}. For instance, /\ipa{ɖɯ˩ɖʐɯ˧}/ and /\ipa{tsʰɯ˧ɻ̩\#˥}/ could combine as $\dagger${\kern2pt}\ipa{ɖɯ˩ɖʐɯ˧-tsʰɯ˧ɻ̩\#˥}, by successive association of the tones of the two components. This tone pattern is attested elsewhere in the language, as seen in four-syllable expressions such as /\ipa{nɑ˩bɑ˧-ʁɑ˧ɭɯ\#˥}/ (Nabbahralee, the name of a~mountain). 

The lowering to L applies exclusively to given names and does not affect compound names consisting of a~term of address followed by a~two-syllable given name. For instance, a~woman named \textit{Gisso} /\ipa{ki˧zo\#˥}/ may be addressed as /\ipa{ə˧mi˧-ki˧zo\#˥}/ (“Mother Gisso, Aunt Gisso”) by her nephews and nieces. This term of address is not realized as $\dagger${\kern2pt}\ipa{ə˧mi˧-ki˩zo˩}, as would be expected if it followed the pattern observed in compound given names. 

The lowering of the final two syllables in given names, as in /\ipa{ɖɯ˩ɖʐɯ˧-tsʰɯ˩ɻ̩˩}/ and all other examples in \tabref{tab:Names}, cannot be put down to the general tone rules of Yongning Na. Nor can it be explained as the result of \isi{lexicalization} at an earlier stage of the language's history, since the lowering process operates as an exceptionless synchronic rule, applying to all compound given names. Speakers retain a~clear awareness of the two components of these names, each of which has its own lexical tone. One of the two elements serves as the usual term of address, which may be the first or, less commonly, the second. For instance, the 
%full given name of M18 is /\ipa{ʈæ˧ʂɯ˧-tsʰɯ˩ɻ̩˩}/, but people address him as /\ipa{ʈæ˧ʂɯ˧}/. The 
name that was bestowed on me by a~priest of the Yongning monastery is \textit{Yishi Diddeo} /\ipa{ʝi˧ʂɯ˥-ti˩ɖo˩}/; Mrs.\ Latami (F4) chose to address me by the second part, \textit{Diddeo} /\ipa{ti˧ɖo˥}/, which she pronounced with its original lexical tone rather than pronouncing it as a fragment extracted from the compound (which would have resulted in L tone).\footnote{F4's son (M18) argued that the shortened name should be \textit{Yishi} /\ipa{ʝi˧ʂɯ˥}/, but F4 maintained her choice without further explanation other than personal preference. The decision partly reflects concerns about homonymy within the extended family: the pool of available names is small, making homonymy a genuine issue. Moreover, concerns about inappropriate use of proper names run deep in Asian cultures. Strict conventions govern the use of personal names, as evidenced, for instance, by the taboo against uttering (or writing) the names of important persons (emperors, but also one's elders), a~practice known in China as \textit{bìhuì} \zh{避讳} \citep[discussed in detail by][]{adamek2012}. In Yongning Na, such concerns are not limited to elders, as illustrated by the following anecdote. In the course of telling the story \textit{BuriedAlive2}, consultant F4 realized that the name of a protagonist, \ipa{/ʈæ˧ʂɯ˧-no˩bv̩˩}/, coincided with that of a~household member. She immediately substituted another name, /\ipa{no˩bv̩˧-tsʰɯ˩ɻ̩˩}/, to avoid any undesirable association between her family member and the character's shameful behaviour (\textit{BuriedAlive2.114} \pandoi{0004537\#S114}). The consultant was so concerned by this coincidence that she would have preferred me to delete the recording entirely. But 
%we had spent much time working on the transcription, so 
I~was not overjoyed at the prospect of outright deletion. Fortunately, the consultant found comfort in the explicit disclaimer which is present within the narrative, to the effect that the character's name was /\ipa{no˩bv̩˧-tsʰɯ˩ɻ̩˩}/ and \textit{not} /\ipa{ʈæ˧ʂɯ˧-no˩bv̩˩}/, and she agreed that we should complete the transcription and allow access to this linguistic document.} This is one of many pieces of evidence demonstrating that compounds such as /\ipa{ʝi˧ʂɯ˥-ti˩ɖo˩}/ remain readily decomposable. 

Should the tonal lowering of the latter part of compound given names be considered as another instance of a~tone rule applying in a~highly specific morphosyntactic context: a~tone combination rule that holds in given names, and nowhere else~-- not even in other sets of proper names, such as place names?

Interestingly, a~similar process of lowering is found in compounds involving the word for ‘powder, flour’: /\ipa{tsɑ˧bɤ˧}/ (as reported in \sectref{sec:thenounflourpowder}). According to the synchronically
productive rules, the combination of this word with /\ipa{lv̩˧mi˧}/ ‘stone’, /\ipa{qʰɑ˧dze˧}/
‘sweetcorn’, and /\ipa{dze˧ɭɯ˧}/ ‘wheat’ should yield a~simple M-tone output,
e.g.~$\dagger${\kern2pt}\ipa{lv̩˧mi˧-tsɑ˧bɤ˧} for ‘fine sand’. But the observed forms, shown in \tabref{tab:powder}, all carry an M.M.L.L tone pattern, corresponding to underlying --L (an L tone on the second part of the compound), just like the compound given names in \tabref{tab:Names}. The label ‘\ili{Tibetan} compounds’ is proposed for these compounds. From a~synchronic point of view, this label may seem invalid, since none of the consultants has any knowledge of \ili{Tibetan} (either spoken or written) and hence any clear awareness of \ili{Tibetan} loanwords as such. The hypothesis here is that at some earlier point in time~-- one to four centuries ago?~-- speakers of Yongning Na who had some knowledge of \ili{Tibetan} (a~smattering would have been enough) applied a~specific tonal treatment to compounds containing \ili{Tibetan} words, by imitation of what they perceived as the \ili{Tibetan} pattern. \ili{Tibetan} was a~prestige language in Yongning at least since the fourteenth century, and up until the mid-twentieth century (see Appendix~\ref{chap:historyanthropologysociology}, \sectref{sec:historicaloutline}), so it is not unlikely that processes of imitation of prosodic patterns took place. This may have happened at the time when the \ili{Tibetan} words were borrowed, or at a~later stage, when speakers of Yongning Na who were aware of the status of these words as \ili{Tibetan} in origin made efforts to copy (what they felt to be) \ili{Tibetan} prosodic patterns. Cases of \is{language contact}contact with a~prestigious language can result in a~range of unusual changes by imitation, including hypercorrections: see \citet[99-103]{meillet1936} on a~possible case concerning {Germanic} and {Romance}, and \citet{ferlus2001eng} on cases in the \il{Vietic languages}Vietic subgroup of \ili{Austroasiatic}; a~highly speculative application to {Tibetan} has also been proposed \citep{ferlus2003a}. 

Supposing that imitation of \ili{Tibetan} was at play in these compounds, it remains to be explained why lowering tone to L on the latter part of compounds had a~\ili{Tibetan} ring to Na ears, and at which point in history the adoption of this prosodic pattern took place. For want of a~command of \ili{Tibetan}, I~am not in a~position to investigate topics of historical \is{language contact}contact with this language~-- one of many topics that remain for future investigations.



\section{Compound nouns containing adjectives}
\label{sec:compoundnounscontainingadjectives}

The tonal categories of \is{adjectives}adjectives will be brought out in \sectref{sec:adjectivesasdistinctfromverbs}. As background to the present discussion, here is a~preview of the results: adjectives fall into four tonal categories~-- L, M, H, and MH~--, with L-tone items further subdivided into L\textsubscript{a} and L\textsubscript{b}.


Compounds containing adjectives are a~difficult topic in Yongning Na, not least because they arise through \isi{lexicalization} and cannot be elicited systematically. Unlike \textit{noun plus noun} compounds, which can be explored through on-the-fly coinage by consultants to bring out a~full set of synchronic rules, compounds containing adjectives do not lend themselves to such elicitation. As a~preliminary to studying lexicalized compounds, it is useful to examine adjectival phrases.

\subsection{A productive construction: \textsc{N}+\textsc{Adj}+{relativizer}}
\label{sec:productiveconstruction}

%
%Liberty Lidz notes that “the constituent order for Na adjectival phrases is
%\textsc{N}+\textsc{Adj}, which is consistent with Na’s OV constituent order” \citep[215]{lidz2010}, as in (\ref{ex:averybigfish}).
%
%\begin{exe}
%  \ex
%  \label{ex:averybigfish}
%  \glll {ni³³ zɔ³³} dɯ⁵⁵ ʐwæ¹³ dɯ³³ mi³¹\\
%  fish big \textsc{ints} one \textsc{cls}\\
%  \zh{鱼} \zh{大} \zh{很} \zh{一} \zh{量词}\\
%  \glt ‘a very big fish’ (example~187 from \citealt[215]{lidz2010})
%\end{exe}

In Yongning Na, \isi{adjectives} are associated with nouns through the construction \textsc{N}+\textsc{Adj}+{relativizer}/{nominalizer}
\mbox{/\ipa{-hĩ˥}/}. For instance, //\ipa{ɖɯ˩\textsubscript{a}}// ‘big’\footnote{As explained on the first page of the introduction, morpheme-level transcriptions indicate lexical tone using tone symbols supplemented by subscript letters \textsubscript{a} \textsubscript{b} \textsubscript{c} to distinguish subcategories of lexical tones. The {subcategories} for verbs and adjectives are set out in \tabref{tab:Utonesofverbs} of \sectref{sec:overview}. 
%The pound symbol \# is also part of the apparatus for transcribing the different categories of lexical tones, as explained in \sectref{sec:afloatinghtonewithcomparativeevidencepointingtoitsorigin}.
} yields //\ipa{ɖɯ˩-hĩ˩}// ‘(which is) big’, and //\ipa{ʂɯ˧˥}// ‘new’ yields //\ipa{ʂɯ˧-hĩ˥\$}// ‘(which is) new’. These follow the noun as
a~separate \isi{tone group}, as illustrated in (\ref{ex:bigfish})-(\ref{ex:newbowl}).

\begin{exe}
	\ex
	\label{ex:bigfish}
	\ipaex{ɲi˧zo˧ {\kern2pt}|{\kern2pt} ɖɯ˩-hĩ˩˥}\\
	\gll ɲi˧zo˧	ɖɯ˩\textsubscript{a}	-hĩ˥\\
	fish	large		\textsc{nmlz}\\
	\glt ‘big fish’
\end{exe}

\begin{exe}
	\ex
	\label{ex:newcloth}
	\ipaex{pʰi˧ {\kern2pt}|{\kern2pt} ʂɯ˧-hĩ˥}\\
	\gll pʰi˧	ʂɯ˧˥		-hĩ˥\\
	linen\_cloth		new		\textsc{nmlz}\\
	\glt ‘brand new linen cloth’
\end{exe}

\begin{exe}
	\ex
	\label{ex:newladle}
	\ipaex{tɕʰo˩˥ {\kern2pt}|{\kern2pt} ʂɯ˧-hĩ˥}\\
	\gll tɕʰo˩˧		ʂɯ˧˥		-hĩ˥\\
	ladle		new		\textsc{nmlz}\\
	\glt ‘new ladle’
\end{exe}

\begin{exe}
	\ex
	\label{ex:newbowl}
	\ipaex{qʰwɤ˧˥ {\kern2pt}|{\kern2pt} ʂɯ˧-hĩ˥}\\
	\gll qʰwɤ˧˥		ʂɯ˧˥		-hĩ˥\\
	bowl		new		\textsc{nmlz}\\
	\glt ‘new bowl’
\end{exe}


No tonal interaction takes place between the noun and \is{adjectives}adjective. If an \is{intensifiers}intensifier is
substituted for the relativizer, the construction becomes a~statement, as in (\ref{ex:fishisbig}). The construction (\ref{ex:fishisbig2}) likewise means ‘the fish is big’.

\begin{exe}
	\ex
	\label{ex:fishisbig}
	\ipaex{ɲi˧zo˧ {\kern2pt}|{\kern2pt} ɖɯ˧ {\kern2pt}|{\kern2pt} ʐwæ˩˥.}\\
	\gll ɲi˧zo˧		ɖɯ˩\textsubscript{a}		ʐwæ˩\\
	fish		large		\textsc{ints}\\
	\glt ‘The fish is very big.’
\end{exe}

\begin{exe}
	\ex
	\label{ex:fishisbig2}
	\ipaex{ɲi˧zo˧ {\kern2pt}|{\kern2pt} ɖɯ˧.}\\
	\gll ɲi˧zo˧		ɖɯ˩\textsubscript{a}\\
	fish		large\\
	\glt ‘The fish is big.’
\end{exe}

In example (\ref{ex:averybigfish}), the
{numeral}-plus-classifier phrase has the effect of nominalizing a~construction that would otherwise
mean ‘the fish is/was really big’, rather than ‘a very big fish’.

\begin{exe}
  \ex
  \label{ex:averybigfish}
  \glll {ni³³ zɔ³³} dɯ⁵⁵ ʐwæ¹³ dɯ³³ mi³¹\\
  fish big \textsc{ints} one \textsc{cls}\\
  \zh{鱼} \zh{大} \zh{很} \zh{一} \zh{量词}\\
  \glt ‘a very big fish’ (example~187 from \citealt[215]{lidz2010}; her glosses and tone marking. \textsc{cls}: classifier; \ipa{⁵⁵}: High tone; \ipa{³³}: Mid tone; \ipa{³¹}: Low-falling tone; \ipa{¹³}: Low-rising tone.)
\end{exe}

The \isi{word order} Noun+Adjective within a~noun phrase always signals a~lexicalized item. To use a~textbook example from {English}, the~phrasal, non-lexicalized expression ‘black bird' contrasts with ‘blackbird' (a specific species). A similar distinction exists in Yongning Na. For instance, //\ipa{ʐɯ˧nɑ˩}//, composed of //\ipa{ʐɯ˧}//
‘liquor/spirits’ and //\ipa{nɑ˩\textsubscript{b}}// ‘black’, does not mean ‘black liquor’ in a descriptive sense (i.e.\ liquor of a~black colour) but refers specifically to a~type of strong, high-quality spirits. This is a~disyllabic noun requiring its own dictionary entry; it is not a~phrasal construction that simply attributes a~quality to the noun's referent. Likewise, the nouns //\ipa{ə˧mi˧-ɖɯ˩}// and
//\ipa{ə˧mi˧-tɕi˩}//, referring to the mother’s older sisters and younger sisters respectively, are
lexical units, despite their transparent internal structure: they consist of //\ipa{ə˧mi˧}//
‘mother’ plus the adjectives //\ipa{ɖɯ˩\textsubscript{a}}// ‘large’ and //\ipa{tɕi˩\textsubscript{a}}// ‘small’. 
%The conceptual
%difference among aunts (mother’s older sisters and younger sisters) is clear in Na culture, witness
%the existence of distinct terms of address: /\ipa{ə˧jɤ˩}/ for ‘mother’s older sister’ and
%/\ipa{ə˧tɕi˩}/ for ‘mother’s younger sister.

Compounds formed with the adjectives ‘big’ and ‘small’ also exist for maternal uncles, //\ipa{ə˧v̩˧˥}//: //\ipa{ə˧v̩˧-tɕi˥}// for ‘mother's younger brother’, and //\ipa{ə˧v̩˧-ɖɯ˧˥}// for ‘mother's elder brother’. However, constructions incorporating the
{relativizer} \mbox{//\ipa{-hĩ˥}//} are more common. To specify whether one is referring to the elder or younger maternal uncle, one typically says /\ipa{ə˧v̩˧˥ {\kern2pt}|{\kern2pt} ɖɯ˩-hĩ˩˥}/ for the former and /\ipa{ə˧v̩˧˥
  {\kern2pt}|{\kern2pt} tɕi˩-hĩ˩˥}/ for the latter, as shown in (\ref{ex:unc1}--\ref{ex:unc2}). 

\begin{exe}
	\ex 
	\begin{xlist}
		\ex
		\label{ex:unc1}
		\ipaex{ə˧v̩˧˥ {\kern2pt}|{\kern2pt} ɖɯ˩-hĩ˩˥}\\
		\gll 	ə˧v̩˧˥		ɖɯ˩\textsubscript{a}	-hĩ˥\\
		uncle		big		\textsc{nmlz}\\
		\glt ‘mother's elder brother, elder maternal uncle'
		
		\ex
		\label{ex:unc2}
		\ipaex{ə˧v̩˧˥ {\kern2pt}|{\kern2pt} tɕi˩-hĩ˩˥}\\
		\gll 	ə˧v̩˧˥		tɕi˩\textsubscript{a}	-hĩ˥\\
		uncle		small		\textsc{nmlz}\\
		\glt ‘mother's younger brother, younger maternal uncle' \textit{(Caravans.75, 76, 78, 79, 177--179, 196, 259, Elders3.23, 31, 32)} \pandoi{0004530\#S75}
	\end{xlist}
\end{exe}

A study of successive occurrences within the same text confirms that the construction with the
{relativizer}/{nominalizer} \mbox{//\ipa{-hĩ˥}//}, although seemingly cumbersome, is the
standard construction to associate an adjective with a~noun. Notably, this construction is not replaced by a~synthetic, compact \textsc{N}+\textsc{Adj} construction at later occurrences. For instance, example (\ref{ex:bigchildB}) appears only a few sentences after (\ref{ex:bigchildA}) within the same narrative, yet it retains the nominalizer construction.

\begin{exe}
	\ex
	\label{ex:bigchildA}
	\ipaex{mv̩˩zo˩˥ {\kern2pt}|{\kern2pt} ɖɯ˩-hĩ˩˥, {\kern2pt}|{\kern2pt} zo˧mv̩˥ {\kern2pt}|{\kern2pt} ɖɯ˩-hĩ˩˥ {\kern2pt}|{\kern2pt} ɖɯ˧-ɭɯ˧ dʑo˩ tsɯ˩.}\\
	\gll mv̩˩zo˩	ɖɯ˩\textsubscript{a}	-hĩ˥	zo˧mv̩˥		ɖɯ˩\textsubscript{a}	-hĩ˥	ɖɯ˧-ɭɯ˧		dʑo˩\textsubscript{b}	tsɯ˧˥\\
	young\_lady		large	\textsc{nmlz}	child	large	\textsc{nmlz}	one-\textsc{clf}	\textsc{exist}	\textsc{rep}\\
	\glt ‘It is said that [this couple] had a~big girl, a~big child (=a child who thought
	very seriously for her age).’ \textit{(Reward.59)} \pandoi{0004446\#S59}
\end{exe}

\begin{exe}
	\ex
	\label{ex:bigchildB}
	\ipaex{mv̩˩zo˩˥ {\kern2pt}|{\kern2pt} ɖɯ˩-hĩ˩-ki˥ ({\dots})}\\
	\gll mv̩˩zo˩	ɖɯ˩\textsubscript{a}	-hĩ˥	 -ki˧\\
	young\_lady		large		\textsc{nmlz}	\textsc{dat}\\
	\glt ‘to his elder daughter, [the father
	said{\dots}]’ \textit{(Reward.65)} \pandoi{0004446\#S65}
\end{exe}


\subsection{Lexicalized compounds with a \textsc{N}+\textsc{Adj} structure}
\label{sec:lexicalizedcompoundsofnadjstructure}

The adjectives that appear in lexicalized combinations with nouns in the examples provided above are
//\ipa{nɑ˩\textsubscript{b}}// ‘black, dark’, //\ipa{ɖɯ˩\textsubscript{a}}// ‘large’, and //\ipa{tɕi˩\textsubscript{a}}// ‘small’. Is it merely coincidental that
‘black’ is also the adjective used in the textbook example of \ili{English} \textit{black bird} and
\textit{blackbird}? The compound noun \textit{blackbird} refers to \textit{Turdus merula}, a~species
of thrush, whereas the noun-adjective combination \textit{black bird} refers to any bird of
black colour. The former, \textit{blackbird}, carries stress on the first element of the compound
(for short: “first-element stress”), whereas the latter, \textit{black bird}, carries primary stress on \textit{bird}. The meaning of \textit{black bird} is compositional, being deducible from the
meanings of its elements, whereas the meaning of
\textit{blackbird} is not. First-element stress is
generally interpreted as a~marker of degree of {lexicalization}. A~study of
\ili{English} compounds observes that “the number of adjectives that work in the way that \textit{black} does in our
\textit{exemple}-\textit{type} seems to be very restricted” \citep[9]{bauer2004}. Examples are
shown in \tabref{tab:typesofadjectivesthatappearincompoundnounsinenglish}. However, after an extensive corpus-based investigation exploring factors such as the frequency of the particular collocations and contrasting patterns of premodification, the author concludes that the gaps in \ili{English} adjective-plus-noun compounds are likely to be accidental. In Na, as in \ili{English}, the set of adjectives appearing in compound nouns is not closed.


\begin{table}%[t]
\caption{Types of adjectives that appear in compound nouns in {English} \citep[from][9]{bauer2004}.}
{\renewcommand{\arraystretch}{1.35}
\begin{tabularx}{\textwidth}{ Q Q Q }
\lsptoprule
	type of adjectives & examples & example compounds\\\midrule
	some colour adjectives & \textit{black}, \textit{blue}, \textit{brown}, \textit{green}, \textit{grey}, \textit{red}, \textit{white} & \textit{blackboard}, \textit{blue-tit},
   \textit{brownstone}, \textit{greenfly}, \textit{greyhound}, \textit{redfish}, \textit{whiteboard}\\
   \textit{grand} in words of family relationships & \textit{grand} & \textit{grandfather}\\
   a~miscellaneous set of {monosyllabic} gradable adjectives & \textit{broad}, \textit{dry}, \textit{free}, \textit{hard}, \textit{hot}, \textit{mad},
   \textit{small}, \textit{sweet} (among others) & \textit{broadcloth}, \textit{dry-cell}, \textit{freepost}, \textit{hardboard}, \textit{hotbed}, \textit{madman},
   \textit{small-arm}, \textit{sweetcorn}\\
   a~small set of non-gradable {monosyllabic} adjectives & \textit{blind}, \textit{dumb}, \textit{first}, \textit{quick} (= ‘alive’),
   \textit{square}, \textit{whole} & \textit{blindside}, \textit{dumbcluck}, \textit{first-day}, \textit{quicksand}, \textit{squaresail}, \textit{wholestitch}\\ a~very
   small number of disyllabic adjectives & \textit{bitter}, \textit{narrow}, \textit{silly} & \textit{bitter-cress}, \textit{narrow-boat},
   \textit{sillyseason}\\
\lspbottomrule
\end{tabularx}}
\label{tab:typesofadjectivesthatappearincompoundnounsinenglish}
\end{table}

In {English}, there is {variation} across speakers (and even for a single speaker) in stress assignment, both in intuition-based judgments and in actual speech. In Yongning Na, by contrast, noun-plus-adjective compounds are conspicuously different from attributive constructions, as the latter require a~{relativizer}. This structural difference makes these compounds easy to identify. Their tonal analysis, however, is less straightforward. Examples are shown in tabular form, classified by the tone of the adjective: L\textsubscript{a} in \tabref{tab:adjective-plus-nouncompoundtoneLa}, L\textsubscript{b} in \tabref{tab:black}, M in \tabref{tab:adjective-plus-nouncompoundtoneM}, and H in \tabref{tab:adjective-plus-nouncompoundtoneH}.
%\footnote{The subscript letters \textsubscript{a} \textsubscript{b} \textsubscript{c} indicate lexical tone subcategories. The {subcategories} for verbs and adjectives are detailed in \tabref{tab:Utonesofverbs} (\sectref{sec:overview}).} 
(No compounds with MH-tone adjectives have yet been observed.) All these items are lexicalized: for instance, the phrase /\ipa{ʈʂʰæ˧nɑ˥}/ refers to a~legendary stag that only spirits are able to hunt, and is thus distinct from an attributive
construction meaning ‘black-coloured deer’.

The compounds exhibit some tonal diversity. Among the five compounds associated with {monosyllabic} roots carrying H tone, three have
L\# tone, and two have H\# tone. 

\clearpage


\begin{table}%[t]
	\caption{Examples of compounds containing the L\textsubscript{a}-tone adjectives /\ipa{mo˩\textsubscript{a}}/ ‘old’, /\ipa{ɖɯ˩\textsubscript{a}}/ ‘large’, /\ipa{tɕi˩\textsubscript{a}}/ ‘small’, and /\ipa{pʰv̩˩\textsubscript{a}}/ ‘white’. Note that no {monosyllabic} form is attested synchronically for ‘stone’ and ‘ard’.}
	\begin{tabularx}{\textwidth}{ l l P{27mm} l l Q }
		\lsptoprule
		\multicolumn{3}{l}{head noun} & \multicolumn{3}{l}{compound}\\
		\cmidrule(r){1-3} \cmidrule(l){4-6}
		form & tone & meaning & form & tone & meaning\\\midrule
		\ipa{hĩ˥} & H & person & \ipa{hĩ˧mo˥} & H\# & elderly person\\
		\ipa{ʐwæ˥} & H & horse & \ipa{ʐwæ˧mo˥} & H\# & old horse\\
		\ipa{si˥} & H & wood & \ipa{si˧mo˥} & H\# & old wood, old tree\\
		\ipa{lv̩˧mi˧} & ? & stone & \ipa{lv̩˧mo˥} & H\# & old stones\\
		\ipa{tsʰo˩} & L & human being & \ipa{tsʰo˩mo˩} & L & old man\\
		\ipa{æ˩gv̩˩} & ? & ard\footnote{The ard, also known as scratch plough, is the type of ploughing implement used in Yongning. Unlike the plough, the ard has a~symmetrical share that traces a shallow furrow but does not invert the soil \citep{haudricourtetal1955}.} & \ipa{æ˩mo˥} & LH & used ard (out of use)\\ 
		\addlinespace \hdashline \addlinespace
		\ipa{ʁo˥} & H & head & \ipa{ʁo˧ɖɯ˧˥} & MH\# & tadpole\\
		\ipa{zo˥} & H & son & \ipa{zo˧ɖɯ˧} & M & eldest son\\
		\ipa{mv̩˩˥} & LH & daughter & \ipa{mv̩˩ɖɯ˩} & L\# & eldest daughter\\
		\ipa{ə˧mi˧} & M & mother & \ipa{ə˧mi˧-ɖɯ˩} & L\# & mother's elder sister\\
		\ipa{ə˧v̩˧˥} & MH\# & maternal uncle & \ipa{ə˧v̩˧-ɖɯ˧˥} & MH\# & mother's elder brother\\
		\ipa{ə˧bo˥\$} & H\$ & paternal uncle & \ipa{ə˧bo˧-ɖɯ˧˥} & MH\# & father's elder brother\\
		\addlinespace \hdashline \addlinespace
		\ipa{mv̩˩˥} & LH & daughter & \ipa{mv̩˩tɕi˥} & LH & youngest daughter\\
		\ipa{zo˥} & H & son & \ipa{zo˧tɕi˥} & H\# & youngest son\\
		\ipa{ə˧mi˧} & M & mother & \ipa{ə˧mi˧-tɕi˩} & L\# & mother's younger sister\\
		\ipa{ə˧v̩˧˥} & MH\# & maternal uncle & \ipa{ə˧v̩˧-tɕi˥} & H\# & mother's younger brother\\
		\ipa{ə˧bo˥\$} & H\$ & paternal uncle & \ipa{ə˧bo˧-tɕi˥} & H\# & father's younger brother\\
		\addlinespace \hdashline \addlinespace
		\ipa{tɕɯ˧} & M & cloud &  \ipa{tɕɯ˧pʰv̩˩} & L\# & white cloud\\
		\lspbottomrule
	\end{tabularx}
	\label{tab:adjective-plus-nouncompoundtoneLa}
\end{table}


\begin{table}%[t]
\caption{Examples of compounds containing the L\textsubscript{b}-tone adjective /\ipa{nɑ˩\textsubscript{b}}/ ‘black’.}
\begin{tabularx}{\textwidth}{ l l P{27mm} l l Q }
\lsptoprule
	\multicolumn{3}{l}{head noun} & \multicolumn{3}{l}{compound}\\
   \cmidrule(r){1-3} \cmidrule(l){4-6}
	form & tone & meaning & form & tone & meaning\\\midrule
	\ipa{hṽ̩˥} & H & hair & \ipa{hṽ̩˧nɑ˩} & L\# & wild animal\\
	\ipa{ʂe˥} & H & meat & \ipa{ʂe˧nɑ˩} & L\# & lean meat\\
	\ipa{si˥} & H & wood & \ipa{si˧nɑ˥} & H\# & deep forest\\
	\ipa{kʰv̩˥} & H & dog & \ipa{kʰv̩˧nɑ˥} & H\# & dog \textit{(in formal speech)}\\
	\ipa{tɕʰi˥} & H & thorn & \ipa{tɕʰi˧nɑ˥} & H\# & prinsepia\\
	\ipa{ʐɯ˧} & M & liquor/spirits & \ipa{ʐɯ˧nɑ˩} & L\# & high-quality spirits\\
	\ipa{njɤ˩˥} & LH & eye & \ipa{njɤ˧nɑ˩} & L\# & eyeball\\
	\ipa{ʈʂʰæ˧˥} & MH & deer & \ipa{ʈʂʰæ˧nɑ˥} & H\# & legendary black stag\\
\lspbottomrule
\end{tabularx}
\label{tab:black}
\end{table}

\begin{table}%[t]
	\caption{Examples of compounds containing the M-tone adjectives /\ipa{pv̩˧}/ ‘dry’, /\ipa{bæ˧}/ ‘stupid’, /\ipa{tʰi˧}/ ‘clever’, /\ipa{tsʰi˧}/ ‘hot’, and /\ipa{ʂæ˧}/ ‘long’.}
	\begin{tabularx}{\textwidth}{ l l P{27mm} l l Q }
		\lsptoprule
		\multicolumn{3}{l}{head noun} & \multicolumn{3}{l}{compound}\\
		\cmidrule(r){1-3} \cmidrule(l){4-6}
		form & tone & meaning & form & tone & meaning\\\midrule
	\ipa{hɑ˥} & H & food & \ipa{hɑ˧pv̩˩} & L\# & dry cooked rice (as opposed to gruel)\\
		\addlinespace \hdashline \addlinespace
	\ipa{zo˥} & H & son & \ipa{zo˧bæ˩} & L\# & idiot\\
		\addlinespace \hdashline \addlinespace
	\ipa{mv̩˩˥} & LH & daughter & \ipa{mv̩˩tʰi˩} & L & clever woman\\
		\addlinespace \hdashline \addlinespace
	\ipa{dʑɯ˩} & L & water & \ipa{dʑɯ˩tsʰi˩} & L & hot water\\
		\addlinespace \hdashline \addlinespace
	\ipa{zɯ˧} & M & life & \ipa{zɯ˧ʂæ˧} & M & long life\\
		\lspbottomrule
	\end{tabularx}
	\label{tab:adjective-plus-nouncompoundtoneM}
\end{table}

\clearpage

\begin{table}
	\caption{Examples of compounds containing the H-tone adjectives /\ipa{qʰæ˥}/ ‘cold’ and /\ipa{ɖæ˥}/ ‘short’.}
	\begin{tabularx}{\textwidth}{ l l P{27mm} l l Q }
		\lsptoprule
		\multicolumn{3}{l}{head noun} & \multicolumn{3}{l}{compound}\\
		\cmidrule(r){1-3} \cmidrule(l){4-6}
		form & tone & meaning & form & tone & meaning\\\midrule
	\ipa{dʑɯ˩} & L & water &  \ipa{dʑɯ˩qʰæ˩} & L & cold water\\
		\addlinespace \hdashline \addlinespace
	\ipa{zɯ˧} & M & life & \ipa{zɯ˧ɖæ\#˥} & \#H & short life\\
		\lspbottomrule
	\end{tabularx}
	\label{tab:adjective-plus-nouncompoundtoneH}
\end{table}


This difference in tone does not appear to correlate with the degree of semantic specialization of the
compounds: whether they are transparent or non-transparent. For instance, the compounds ‘wild animal’ and ‘lean meat’ share the same tone pattern (L\#) and \is{derivation!tonal}derive from the same input tones (H and L\textsubscript{b}), yet they differ in their degree of transparency. The noun /\ipa{hṽ̩˧nɑ˩}/, literally ‘black hair’, does not refer to a~type of hair but means ‘wild animal’, denoting the \textit{possessor} of dark
hair by synecdoche.\footnote{As a zoological aside, the association of darker fur with wildness (i.e.\ lower predisposition to
domestication) is supported by studies of animal domestication \citep{trut1999}. This phenomenon appears to stem from a link between stress levels and melanin production, which manifests in darker fur \citep{burchilletal1986}: wild animals experience higher stress levels than their domesticated counterparts. Wild yaks, for instance, have
darker hair than domestic yaks \citep{leslieetal2009}.} (In the Indian linguistic tradition, /\ipa{hṽ̩˧nɑ˩}/, literally ‘black hair’, meaning ‘wild animal’, would be classified as a~\is{compounds!bahuvrīhi|textbf}\textit{bahuvrīhi} compound, denoting a~referent by means of a~salient characteristic.) By contrast, the compound /\ipa{ʂe˧nɑ˩}/,
‘lean meat’, is relatively transparent: it designates a specific type of meat~-- lean meat as opposed to fat meat, /\ipa{ʂe˧mɤ˧˥}/. Traditionally, pigs
and cattle were only slaughtered once a~year, so that fresh meat was the {exception}; preserved lean meat was the norm and had a dark brown colour.

The set of three H\#-tone compounds from like input tones (H and L\textsubscript{b}) is not homogeneous in this respect either. While the meaning of /\ipa{tɕʰi˧nɑ˥}/ (literally ‘black thorn’) is highly specific~-- it refers to prinsepia~--, /\ipa{kʰv̩˧nɑ˥}/ (literally ‘dark dog, black dog’) remains fully general. The latter does not denote any specific subset within \textit{Canis familiaris}; the only nuance distinguishing it from the more common form /\ipa{kʰv̩˩mi˩}/ is that it belongs to formal, elevated speech.

A parameter that appears more relevant than the overall degree of semantic specialization is the extent to which the adjective's literal meaning remains present in the compound. The hypothesis here is that the two tonal subsets differ in that, in the first subset, the adjective retains its literal interpretation, whereas in the second, it is semantically bleached.


From this perspective, the coherence of the two subsets is as follows: in the expressions ‘wild animal’ and ‘lean meat’, the \is{adjectives}adjective
/\ipa{nɑ˩\textsubscript{b}}/ retains its face value of ‘black, dark’. By contrast, as noted above, /\ipa{tɕʰi˧nɑ˥}/ refers to a distinct species of plant (prinsepia) rather than to dark thorns, and /\ipa{kʰv̩˧nɑ˥}/ designates ‘dog’ without any implication of hair colour. 

The third and final item in the second set, /\ipa{si˧nɑ˥}/ (‘wood’+‘dark’, for ‘deep forest’), is the least clear-cut case. A dense forest is naturally darker than a sparse one, making it difficult to determine the extent to which the adjective retains its face value of ‘dark’. Hypothesizing that the L\#-tone compounds are those in which the adjective is understood literally, whereas H\#-tone compounds (with the same input tones: H plus L\textsubscript{b}) make a~semantically bleached use of the adjective, it follows that the etymological notion of darkness in /\ipa{si˧nɑ˥}/ ‘deep forest’ has become bleached. Discussions with Mrs.\ Latami support this hypothesis. 

Analysis of this issue is complicated by the scarcity of adjectives that share the same tone as ‘black’ (//\ipa{nɑ˩\textsubscript{b}}//). The only other example identified so far is //\ipa{dʑɤ˩\textsubscript{b}}// ‘good’ (see \sectref{sec:adjectivesasdistinctfromverbs}).

A~similar lack of one-to-one {correspondence} between input tones and output tones is also found for compounds containing the adjective /\ipa{ɖɯ˩\textsubscript{a}}/ ‘large, big’. The compound nouns /\ipa{zo˧ɖɯ\#˥}/
‘eldest son’ and /\ipa{ʁo˧ɖɯ˧˥}/ ‘tadpole’ (literally ‘big head’) have different tonal patterns (\#H and
MH\#, respectively), although both are made up of a~noun root that has H lexical tone and the
adjective /\ipa{ɖɯ˩\textsubscript{a}}/ ‘big’.

To venture speculative hypotheses about the origin of these variegated tone patterns: 
\begin{itemize}
    \item \emph{Variation in the adjective itself:} One possibility is that the adjective in tonal\-ly different compounds is not the same. In synchrony, there
    exists an adjective /\ipa{nɑ˥}/ (not \is{homophony}homophonous with /\ipa{nɑ˩\textsubscript{b}}/ ‘black, dark’) meaning ‘important, serious (e.g.~a~wound)’. Some adjectival compounds may have incorporated this
    adjective or another, now-lost adjective pronounced [\ipa{nɑ}]. However, this hypothesis does not look promising: it is not suitable to the pair of words ‘eldest son’ and ‘tadpole’, where the adjective appears recognizably identical (‘big’). The etymology of ‘tadpole’ as ‘big head’ is super-clear: an apt designation for this small and distinctive creature (even though such terms are prone to playful deformation or replacement by \is{expressivity}expressive coinages).
    
    \item \emph{Truncation from longer words:} Some of these compounds may not originate from the direct combination of
    a~{monosyllabic} noun and adjective but instead \is{derivation!morphological}derive from the truncation of longer nouns. For instance, the syllable /\ipa{si˧}/ in /\ipa{si˧nɑ˥}/ could result from the truncation of
    the disyllabic /\ipa{si˧ɕi˧˥}/ ‘forest’.\footnote{\citet[50-52]{creissels1982} documents synchronic {variation} in \ili{Mandinka}, where frequently compounding nouns tend to get extracted from compounds with their compound-internal tone, which eventually replaces their original lexical tone. Such processes, operating on a word-by-word basis, contribute to the irregularity of tonal correspondences across dialects, complicating {diachronic} comparison and \isi{reconstruction}.} Evidence for the possibility of such truncation comes from /\ipa{tʰo˧ɕi˧˥}/ ‘pine forest’, which combines a~{monosyllable} for ‘pine’ with the \textit{second} syllable of /\ipa{si˧ɕi˧˥}/ ‘forest’. However, this explanation seems implausible for ‘eldest son’ and ‘tadpole’, which appear to \is{derivation!morphological}derive straightforwardly from the {monosyllabic} nouns ‘son’ and
    ‘head’, respectively. If one nonetheless tries to push this hypothesis, one might speculate that
    ‘tadpole’ was originally built on the basis of a~disyllabic noun, which could still survive in
    another dialect.

    \item \emph{Reanalysis of the second morpheme:} The second part of the compound may not always function as an adjective. Instead, it could be analyzed as a~\is{suffixes}suffix in some cases and as an adjective in others, analogous to the distinction proposed by \citet[182]{lidz2010} for the morpheme /\ipa{mɔ¹³}/, which functions as an adjective meaning ‘old’ but also as a~\is{suffixes}suffix meaning ‘dear’ (indicating respect). Under this analysis, the H\#-tone compounds could be morphologically parsed as /\ipa{tɕʰi˧-nɑ˥}/ ‘prinsepia’, /\ipa{kʰv̩˧-nɑ˥}/ ‘dog’, and /\ipa{si˧-nɑ˥}/ ‘deep forest’, contrasting with the L\#-tone compounds /\ipa{hṽ̩˧nɑ˩}/ ‘wild animal’ and /\ipa{ʂe˧nɑ˩}/ ‘lean meat’. 

    \item \emph{Different historical layers:} The tonal differences may reflect distinct historical strata, corresponding to different tonal rules that applied at different diachronic stages. For instance, ‘deep forest’, ‘dog’, and ‘prinsepia’ may predate ‘wild animal’ and ‘lean meat’, as the former compounds appear less semantically transparent. This scenario bears similarities to the previous one. 
\end{itemize}


So far, consistency in the tone patterns of adjectival compounds appears to be limited to synchronically trivial
patterns. For instance, it does not come as a~surprise that Mid-tone /\ipa{ʐɯ˧}/ ‘liquor, spirits’ and Low-tone
/\ipa{nɑ˩\textsubscript{b}}/ ‘black, dark’ yield a~compound with an M+L surface tone pattern, /\ipa{ʐɯ˧nɑ˩}/: this looks
like a~straightforward case of concatenation. The same tone pattern is also found with another adjective that has
the same lexical tone, /\ipa{dʑɤ˩\textsubscript{b}}/ ‘good’. Disyllabic /\ipa{kɯ˧ dʑɤ˩}/ was readily extracted from the expression /\ipa{kɯ˧ dʑɤ˩ hɑ̃˩ dʑɤ˩}/ ‘auspicious
day’ (/\ipa{kɯ˧}/ means ‘star’, and /\ipa{hɑ̃˧˥}/ ‘evening, night’ is a
term used to count days). The L\# tone pattern of /\ipa{kɯ˧ dʑɤ˩}/, literally ‘good star’, results in the following two syllables receiving L tone,
through Rule~5. 

The tone pattern of /\ipa{kɯ˧ dʑɤ˩}/ ‘good star’ is the same as that of /\ipa{ʐɯ˧nɑ˩}/ ‘high-quality liquor’. This shared tone could suggest that both compounds belong to the same historical layer, but more examples would be needed to investigate this possibility further.

To sum up: in view of the limited number of examples found to date and their heterogeneity, pooling them all into a~summary table of tonal outcomes for \textsc{N}+\textsc{Adj} compounds does not
appear particularly illuminating at this stage. Provisionally, the examples are simply listed in \tabref{tab:adjective-plus-nouncompoundtoneLa}, arranged by adjective, by decreasing number of examples.

%Table 10
%Table 10. in manuscript
%\begin{table}[t]
%\caption{Further examples of adjective-plus-noun compounds.}
%\begin{tabularx}{\textwidth}{ l Q l P{15mm} l P{32mm} }
%\lsptoprule
%	\multicolumn{2}{l}{noun} & \multicolumn{2}{l}{adjective} & \multicolumn{2}{l}{compound}\\\midrule
%\lspbottomrule
%\end{tabularx}
%\end{table}

As a~general observation, the tonal patterns in \is{adjectives}noun-plus-adjective compounds are not identical to those of noun-plus-verb combinations (described in Chapter~\ref{chap:verbsandtheircombinatoryproperties}). For instance, ‘hot water’, /\ipa{dʑɯ˩tsʰi˩}/, has L tone, \is{derivation!tonal}derived from an input of L and M on the noun and adjective, respectively. In contrast, noun-plus-verb combinations (whether object plus verb or subject plus verb) with the same tonal input yield M as the output.

The {diachronic} trend is for \is{monosyllables}monosyllabic nouns to become less frequent, being replaced by disyllables. As disyllabic nouns become lexicalized, the tonal {correspondence} between noun-plus-adjective combinations and the roots ceases to be transparent to speakers. This makes the tone patterns of disyllables more susceptible to replacement~-- whether through \isi{analogy} or \is{language contact}contact among dialects~-- than in cases where the root still exists as an independent \is{monosyllables}monosyllabic form. In the current state of the language, ‘stone’ and ‘ard’ are attested only as disyllables, except in compounds with the adjective ‘old’, where they are found in \is{monosyllables}monosyllabic form. This raises the question of whether it is justified to extract a~{monosyllable} from the disyllabic compounds containing ‘old’. 

The root for ‘stone’ could be \is{comparative method (historical linguistics)}\is{reconstruction}reconstructed with an H tone, as *\ipa{lv̩˥}, on the basis of its tonal behaviour in adjectival compounds: the compound /\ipa{lv̩˧mo˥}/ ‘old stone’ carries H\# tone, like the compounds formed by adding ‘old’ to the H-tone monosyllables ‘person’, ‘horse’, and ‘wood’. However, closer examination reveals that ‘old stone’ is not in common use in Yongning Na and has no clear independent meaning. It appears to be a~recent coinage, attested only in a~saying~-- example (\ref{ex:whydontyou})~-- where it serves as a~parallel to /\ipa{si˧mo˥}/ ‘old wood’. Its tone pattern also follows that of ‘old wood’. (Use of the symbol ‘F’ for ‘Focalization’ in the sentence-level transcription is explained in \sectref{sec:focalization}.)

\begin{exe}
	\ex
	\label{ex:whydontyou}
		\ipaex{lv̩˧mo˥ F {\kern2pt}|{\kern2pt} dʑɯ˧ {\kern2pt}|{\kern2pt} le˧-qv̩˩; {\kern2pt}|{\kern2pt} si˧mo˥ F {\kern2pt}|{\kern2pt} le˧-dze˩ kv̩˩! {\kern2pt}|{\kern2pt} no˧ F {\kern2pt}|{\kern2pt} ə˧tse˧ {\kern2pt}|{\kern2pt} le˧-ʂɯ˧ mɤ˧-tʰɑ˧˥ {\kern2pt}|{\kern2pt} di˩!}\\
		\gll lv̩˧mo˥		dʑɯ˩	le˧					qv̩˩\textsubscript{a}	si˧mo˥		le˧-	dze˩\textsubscript{a}	-kv̩˧˥		no˩		ə˧tse˧	le˧-	ʂɯ˧\textsubscript{a}	mɤ˧-	tʰɑ˧˥ 	di˩\textsubscript{a}\\
		old.stones		water	\textsc{accomp}	to\_carry\_away			old\_wood	\textsc{accomp}	to\_cut		\textsc{abilitive}		\textsc{2sg}	\textsc{interrog.}why	\textsc{accomp}		to\_die		\textsc{neg}	\textsc{permissive}	\textsc{exist.spatial}\\
		\glt ‘Old stones are carried away by the stream; and old wood gets chopped down! And you, why won't you die?' \textit{Context:} jeering an elderly person. Na tradition assigns human beings a~lifespan of sixty years; people getting past seventy are considered to be well past their expected lifespan. (Field notes.)
\end{exe}

Further analysis will require gathering more examples, sorting them into sets according to their tone patterns, identifying the historical layers that they belong to, and examining their process of formation. As a~first step in this direction, the following paragraph discusses items that are currently on the verge of \isi{lexicalization}.

\subsection{\textsc{N}+\textsc{Adj} combinations in the process of lexicalization}
\label{sec:nadjitemscurrentlyintheprocessoflexicalization}

In between adjectival constructions such as /\ipa{ə˧v̩˧˥ {\kern2pt}|{\kern2pt} ɖɯ˩-hĩ˩˥}/ ‘mother’s elder brother’ (example (\ref{ex:unc1}) above) on the one hand, and lexical items such as /\ipa{ə˧v̩˧-ɖɯ˧˥}/ (also meaning ‘mother’s elder brother’: see \tabref{tab:adjective-plus-nouncompoundtoneLa}) on the other, there are cases that offer insights into the process of \isi{lexicalization}. ‘Elderly person’ is /\ipa{hĩ˧mo˥}/, from /\ipa{hĩ˥}/ ‘person’ and /\ipa{mo˩\textsubscript{a}}/ ‘old’. In a~set of twenty texts, this noun appears fifteen times, always in the plural, as /\ipa{hĩ˧mo˥=ɻæ˩}/; the fact that it is followed by a~\is{clitics}clitic shows that it is a~full-fledged noun. But there is a~higher number of occurrences (twenty-three) of /\ipa{hĩ˧ mo˥-hĩ˩}/, which also means ‘elderly person’, again from \mbox{/\ipa{hĩ˥}/} ‘person’ and /\ipa{mo˩\textsubscript{a}}/ ‘old’, but with addition of the relativizer \mbox{/\ipa{-hĩ˥}/}. This is not quite like the adjectival construction presented in \sectref{sec:productiveconstruction}: in that construction, the noun constitutes a~\isi{tone group} on its own, e.g.~/\ipa{tɕʰo˩˧ {\kern2pt}|{\kern2pt} ʂɯ˧-hĩ\#˥}/ ‘new ladle’ (example (\ref{ex:newladle}) above), whereas ‘elderly person’ is realized as /\ipa{hĩ˧ mo˥-hĩ˩}/, in one \isi{tone group}. At a~push, it would be possible to say /\ipa{hĩ˧ {\kern2pt}|{\kern2pt} mo˩-hĩ˩˥}/ ‘a person that is old’, but this is judged decidedly awkward in the contexts where /\ipa{hĩ˧ mo˥-hĩ˩}/ is attested. The interpretation that can be proposed is that /\ipa{hĩ˧mo˥}/ is on its way towards \isi{lexicalization}~-- as evidenced by the tonal interaction between its two constituent morphemes~-- but the perception of its second syllable as an adjective remains strong enough that the relativizer is commonly added after it. The fact that the agent marker ($\ddagger${\kern2pt}\ipa{hĩ˧mo˥ ɳɯ˩}) or the topic marker ($\ddagger${\kern2pt}\ipa{hĩ˧mo˥ ʈʂʰɯ˩}) cannot be added shows that /\ipa{hĩ˧mo˥}/ is not yet fully lexicalized. It is compulsory to add an intervening plural or relativizer: /\ipa{hĩ˧mo˥=ɻæ˩ ɳɯ˩}/, /\ipa{hĩ˧mo˥=ɻæ˩ ʈʂʰɯ˩}/, /\ipa{hĩ˧ mo˥-hĩ˩ ɳɯ˩}/, and /\ipa{hĩ˧ mo˥-hĩ˩ ʈʂʰɯ˩}/.

%\newpage 
For purposes of synchronic description, the notations adopted are /\ipa{hĩ˧mo˥{\allowbreak}=ɻæ˩}/ and /\ipa{hĩ˧ mo˥-hĩ˩}/. In /\ipa{hĩ˧mo˥=ɻæ˩}/, the sequence /\ipa{hĩ˧mo˥}/ is transcribed as a~lexical unit, with no hyphen or blank space between its two syllables. In /\ipa{hĩ˧ mo˥-hĩ˩}/, the first syllable is analyzed as a~noun, and separated by a~blank space from the adjective that follows. This notational distinction aims to draw attention to the versatility of disyllables made up of a~noun and an adjective. To take another example of this phenomenon, /\ipa{zo˧bæ˩}/, from /\ipa{zo˥}/ ‘son; man’ and /\ipa{bæ˧}/ ‘stupid; dumb (unable to speak)’, has clearly nominal uses, meaning ‘dumb man; stupid man’. More than twenty examples are found in the \textit{Lake} narrative, one of whose main protagonists is a~dumb person. The noun can be followed by the agent adposition: /\ipa{zo˧bæ˩ ɳɯ˩}/ (\textit{Lake3.29} \pandoi{0004348\#S29}, \textit{Lake4.24} \pandoi{0004350\#S24}); no intervening {relativizer{\slash}nominalizer} is needed for quantization purposes, as evidenced by examples (\ref{ex:onedumb})--(\ref{ex:thatdumb}).

\begin{exe}
	\ex
	\label{ex:onedumb}
	\ipaex{zo˧bæ˩ ɖɯ˩-v̩˩}\\
	\gll zo˧bæ˩		ɖɯ˧-v̩˧\\
	dumb\_person	one-\textsc{clf}.individual\\
	\glt ‘a dumb person’ \textit{(Lake4.4)} \pandoi{0004350\#S4}
\end{exe}

\begin{exe}
	\ex
	\label{ex:thisdumb}
	\ipaex{zo˧bæ˩ ʈʂʰɯ˩-v̩˩}\\
	\gll zo˧bæ˩		ʈʂʰɯ˥	v̩˧\\
	dumb\_person	\textsc{dem.prox}	\textsc{clf}.individual\\
	\glt ‘this dumb person’ \textit{(Lake4.6)} \pandoi{0004350\#S6}
\end{exe}

\begin{exe}
	\ex
	\label{ex:thatdumb}
	\ipaex{zo˧bæ˩ tʰv̩˩-v̩˩}\\
	\gll zo˧bæ˩		tʰv̩˥	v̩˧\\
	dumb\_person	\textsc{dem.dist}	\textsc{clf}.individual\\
	\glt ‘that dumb person’ \textit{(Lake4.12-14)} \pandoi{0004350\#S12}
\end{exe}

The expression /\ipa{zo˧bæ˩}/ can also directly precede a~verb, as in /\ipa{zo˧bæ˩ {\kern2pt}|{\kern2pt} go˩bo˧ di˧˥}/ ‘the dumb man drove cattle’ (\textit{Lake4.19} \pandoi{0004350\#S19}). In addition to such typically nominal uses, the word also functions predicatively (as an adjective): in a~context of self-deprecation where someone calls himself stupid, /\ipa{zo˧bæ˩}/, another person may comfort him by saying (\ref{ex:notdumb}).

\begin{exe}
	\ex
	\label{ex:notdumb}
	\ipaex{mɤ˧-zo˧bæ˩!}\\
	\gll mɤ˧-		zo˧bæ˩\\
	\textsc{neg}	stupid\\
	\glt ‘[No, you are] not stupid!’
\end{exe}

The antonym of /\ipa{zo˧bæ˩}/ in this adjectival sense is /\ipa{zo˧tʰi˧}/ ‘clever; clever person’, which has the same structure, from /\ipa{zo˥}/ ‘son; man’ and /\ipa{tʰi˧}/ ‘able; capable; clever; sharp’. The two words have different tones (L\# tone for /\ipa{zo˧bæ˩}/, vs.\ M tone for /\ipa{zo˧tʰi˧}/) despite their constituting morphemes having the same tones. This tonal difference alerts us to the possibility that the two words may belong to different historical layers. Syntactically, the two words also differ: it is not possible to say $\ddagger${\kern2pt}\ipa{mɤ˧-zo˧tʰi˧} ‘not clever’ on the {analogy} of /\ipa{mɤ˧-zo˧bæ˩}/ ‘not stupid’ (\ref{ex:notdumb}). The first syllable of these two words is sufficiently \is{grammaticalization}grammaticalized for them to be used as adjectives for men and women alike, but in their nominal use they refer exclusively to men: /\ipa{zo˧bæ˩}/ means ‘stupid man’, and /\ipa{zo˧tʰi˧}/ ‘clever man’. The corresponding words for women are /\ipa{mv̩˩-bæ˧mi˩}/ ‘stupid woman’ and /\ipa{mv̩˩tʰi˩}/ ‘clever woman’.


\subsection{Two lexicalized compounds with an \textsc{Adj}+N structure}
\label{sec:alexicalizedcompoundofadjnstructure}

So far, only two lexicalized compounds with an \textit{adjective plus noun} structure have been identified: /\ipa{pv̩˧lv̩˧}/
‘nonirrigated farmland; dry land’, clearly related to /\ipa{pv̩˧}/ ‘dry’ and /\ipa{lv̩˧}/
‘field’, and /\ipa{ɖɯ˩-hĩ˩}/ ‘important people, great personages (including elders)’,\footnote{Occurrences are found in \textit{Sister.13} \pandoi{0004342\#S13}, \textit{14}, \textit{34}, \textit{Sister3.31} \pandoi{0004344\#S31}, \textit{36}, \textit{38}, \textit{41}, and \textit{BuriedAlive3.5} \pandoi{0004538\#S5}.} from /\ipa{ɖɯ˩\textsubscript{a}}/ ‘large’ and /\ipa{hĩ˥}/ ‘person’.

These words exhibiting the reverse order from the other compounds do not show phonological signs of antiquity, such as consonantal or vocalic differences relative to their etymological components. \is{comparative method (historical linguistics)}{Comparative evidence} from \ili{Naxi} appears relevant here. Naxi has an exact equivalent to Na /\ipa{pv̩˧lv̩˧}/ ‘nonirrigated farmland’: the corresponding Naxi form is /\ipa{pv̩˩ɭɯ˧}/, with the same \textsc{Adj}+\textsc{N}
structure, the same meaning, and the same morphological transparency (in \ili{Naxi}, ‘dry’ is
/\ipa{pv̩˩}/, and ‘field’ is /\ipa{ɭɯ˧}/). \ili{Naxi} also has another \textsc{Adj}+\textsc{N} compound
containing ‘dry’: /\ipa{pv̩˩dy˩}/ ‘dry land (as opposed to water)’ \citep[55]{pinsonetal2012}.\footnote{There is a typographical error in the phonetic transcription in \citet[55]{pinsonetal2012}, where /\ipa{pv̩˩dv̩˩}/ should read /\ipa{pv̩˩dy˩}/, as confirmed by the orthographic transcription.} The existence of a cognate in Naxi suggests that the compound may have some time depth, possibly even predating the split between Na and Naxi.

As for ‘important person; great personage’, the expression found in contemporary \ili{Naxi} is /\ipa{hi˧-ɖɯ˩}/, which follows the more common \textsc{N}+\textsc{Adj} ordering. Further dialectal evidence will be necessary to assess the historical depth of the Na compound and make progress in the analysis generally.
%This appears in the text Weresow: /\ipa{hi˧-ɖɯ˩ wɑ˩ ji˥, ɖɯ˧-mə˞˩ ʈʂʰu˩-be˧ kæ˧ le˧˥-tsʰɯ˩}/ ‘being an important person, [the to-mba priest] left [the
%  place where he had conducted a~ritual] rather early’. 


\subsection{A lexicalized compound with a V+\textsc{Adj} structure}
\label{sec:alexicalizedcompoundofvadjstructure}

As a~final observation on compounds, one single lexicalized compound with a \textsc{V}+\textsc{Adj} structure has been observed so far: /\ipa{tsʰo˧ɖɯ˩}/ ‘group dance’, a~type of dance involving anywhere from ten
to about a~hundred participants. Its components are /\ipa{tsʰo˧\textsubscript{b}}/ ‘to jump’ (in a~nominalized reading)
and /\ipa{ɖɯ˩\textsubscript{a}}/ ‘large’. An alternative interpretation whereby /\ipa{ɖɯ˩\textsubscript{a}}/ would have an {adverbial}
reading (‘jumping a~lot’) is implausible, as /\ipa{ɖɯ˩\textsubscript{a}}/ does not have attested
{adverbial} uses.

This is not a~productive construction: for instance, it is not possible to create a compound meaning ‘banquet’ from
the verb ‘to eat’ and the adjective ‘large’. Still, the existence of the word /\ipa{tsʰo˧ɖɯ˩}/ ‘group dance’ can be
taken as evidence of occasional permeability between word classes.
The
distinction between nouns and verbs is more or less clear-cut from one language to another (\citealt{launey1994}, \textit{passim}). In \ili{Naish}
languages, some items straddle the boundary between categories. For example, Na /\ipa{kɤ˧ʈʂɯ˩}/,
like \ili{Naxi} /\ipa{kɯ˧ʈʂɯ˩}/, has both verbal and nominal uses: ‘to speak’ and ‘speech; language’ in \ili{Naxi}, and ‘to tell’ and ‘speech’ in Na.


\section{Concluding note}
\label{sec:concludingnotes}

The\is{complexity} complexity of tone combination rules in various types of compounds in the Alawua dialect of Yongning Na, along with the existence of exceptions to these synchronic rules, arguably sheds light on the considerable diversity of
patterns from one dialect to another~-- including differences among speakers from the same village, and even within a~single idiolect. While some compounds happen to carry
the same tone sequence as would the simple juxtaposition of their constituent elements, most display tone patterns that reflect their status as compounds. Among these, tonal evidence allows for distinguishing between determinative and coordinative compounds in a~few cases, whereas in the others, speakers need to rely on semantic information and context.

% end of tag indexing 'compound' for whole chapter
\is{compounds|)}
