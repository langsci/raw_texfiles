\chapter{Historical and ethnological perspectives} 
\label{chap:appendixb}
\label{chap:historyanthropologysociology}

\largerpage[2]

While the preceding chapters have focused primarily on the tonal system and morphosyntax of Yongning Na, in keeping with the volume's title, the broader context in which this language is spoken invites further reflection. 
Thinking beyond disciplinary boundaries is central to the study of unwritten languages and orally transmitted traditions. The intertwined challenges for linguists and anthropologists were memorably articulated by Jacqueline Thomas, founder of the \textit{Langues et Civilisations à Tradition Orale} (LACITO) research centre within \textit{Centre National de la Recherche Scientifique} (CNRS): (i)~the study of language is only meaningful when grounded in a~thorough understanding of the culture in which it is embedded, and (ii)~the study of a society requires full command of its language. 

In this spirit, the present Appendix is a~modest attempt to contribute a~historical and anthropological perspective to complement the linguistic analyses developed in the main chapters. It reflects on the “chains of societies” (\textit{chaînes de sociétés}: \citealt{amselle1990}; see also \citealt[329--331]{tryon1998}) that shaped Na ethnicity over time. It opens with a~concise overview of the history of the Na and related groups (\sectref{sec:historicaloutline}), then turns to questions of ethnic classification and self-identification (\sectref{sec:ethnicclass}). The final section examines anthropological research on kinship and family structure (\sectref{sec:anthropologicalresearchthefascinationofnafamilystructure}).\footnote{This chapter was initially part of the introduction. However, a~reviewer observed~-- quite reasonably~-- that the encyclopaedic quotations and extended discussions of topics at some remove from the tonal system of Yongning Na risked tiring the reader and weighed down the introductory chapter. This was a~fair point: given the book’s title, \textit{Tone in Yongning Na}, readers cannot be assumed to expect several dozen pages on historical, ethnographic, and anthropological matters. These materials have therefore been relocated to the present Appendix.}

%Readers who wish to go straight to the heart of tonal matters can safely skip the 50-page historical and ethnographic section (\sectref{sec:presentationofthenalanguageandnasocietyandreviewofearlierstudies}) and start from \sectref{sec:chronologyofthestudyelicitationproceduresandonlinematerials} or even from Chapter~\ref{chap:thelexicaltonesofnouns}.

%This long section (almost fifty pages) goes into more detail than is strictly necessary to the discussion of tones, and focused tonologists can safely skip it. / It provides background information about the Na and their language. 


\section{The history of Yongning in outline}
\label{sec:historicaloutline}

\subsection{Prehistory}
\label{sec:prehistory}

The \ili{Naxi} scholars Guo
Dalie \zh{郭大烈} and 
He Zhiwu \zh{和志武} argue that the name 
\textit{Móshā} \zh{摩沙}, found in a~fourth-century chronicle, unambiguously refers to “the \ili{Naxi}”, a~concept which they define as including the speakers of Yongning Na \citep[102-103]{guoetal1994}.\footnote{“This is the first certain and unequivocal mention of the {Naxi} in recorded history.” \textit{Original text:} \zh{这是纳西族在历史上首次明确无误的记录。}} In doing so, they project the ethnic identity of present-day \ili{Naxi} communities back a millennium and a half. They proceed to track the group's history through a~sequence of changing designations in {Chinese} sources: from \textit{Móshā} \zh{摩沙} in the Jin dynasty, to \textit{Mósuò} \zh{磨些} in the Tang dynasty,\footnote{The character \zh{些}, normally read \textit{xiē}, is read \textit{suò} in this context, as also in other contemporary toponyms: see \citet[170n9]{pelliot1904deux}.} and later to \textit{Mósuò} \zh{麽些}, \textit{Mósuò} \zh{摩些}, \textit{Mósuō} \zh{摩娑}, and \textit{Mòsuò} \zh{末些} in the Yuan dynasty and subsequent periods. 

As to earlier origins, Guo and He suggest an admixture of Qiang \zh{羌} people with an indigenous population, left unnamed.

\begin{quote} 
	In the formation of the \ili{Naxi} people, the main component consisted of aboriginals, who blended with Qiang \zh{羌} people, and later assimilated some other peoples at their periphery; conversely, in peripheral areas, some \ili{Naxi} got assimilated into other peoples. 
	\citep[24]{guoetal1994}
    
    \medskip 
    %\footnote{
    {\noindent}\textit{Original text:} 
		\zh{纳西族的形成以土著为主,融合了北来羌人,以后又同化了周围其他一些民族,边缘地区则是纳西族被其他民族同化。}
        %} 
\end{quote}

%Command \noindent added to avoid having an indent. Proofreader suggestion: since this sentence continues the argument, it is better not to indent. 
{\noindent}A~further link is proposed between the \ili{Naxi} and the Shiguanzang \zh{石棺葬} culture, attested during the first millenium \,BC in regions overlapping with present-day Na and \ili{Naxi} settlement \citep[66-67]{guoetal1994}. This cultural association of incoming \il{Sino-Tibetan}Sino-Tibetan peoples with a~certain type of graves raised hopes similar to the association of Kurgan pit-graves with the Indo-Europeans \citep{gimbutas1977,anthony2010}. Shiguanzang graves are typically found on tablelands near river confluences and include short bronze swords of a~type also known from the northern steppes. The absence of securely identified associated settlements raises the possibility that the builders were nomadic, metal-using pastoralists, contrasting with the earlier inhabitants of the area, whose numerous settlements suggest a~sedentary, agricultural society in transition from the Neolithic to the Bronze Age.\footnote{\citet[64-66]{guoetal1994} identify this nomadic population with the \ili{Yi} \zh{夷} peoples mentioned in {Chinese} chronicles, whom they regard as ancestors of both the present-day \ili{Yi} \zh{彝}, and of the Na and \ili{Naxi}. It is not altogether clear how this identification is to be reconciled with their other claim that the Sino-Tibetan component in Naxi ethnogenesis came from the Qiang \zh{羌}.} 

This scenario fits broader models of \ili{Sino-Tibetan} migration from the Yellow River valley \citep[]{jacques_migrations_2024}, in which the Yangshao \zh{仰韶} culture (c.~5000 BC 
%Modified from `c.~5000\,BC` for overfull hbox reasons
to 3000\,BC) is hypothesized as a~point of origin. One of the main {reconstructed} lines of migration is “south-west down the river valleys along the eastern edge of the \ili{Tibetan} plateau through what has been called the \textit{ethnic corridor}” (\citealt[236]{lapolla2001}; emphasis in original).

Yet such reconstructions raise long-standing methodological issues. In particular, the proposals advanced by Guo Dalie and He Zhiwu invite scrutiny on points such as the extent to which archaeological evidence can be reliably fitted into broad cultural typologies, and the basis on which the names of peoples were assigned in Chinese historical chronicles. As \citet[31]{hein2013} notes, “Most excavation reports describing and interpreting burial material from Southwest China tend to associate grave type with archaeological culture; hence their urgent desire to arrive
at a~clear classification of burial types; however, ({\dots}) one cultural or ethnic group can be characterized by
a~number of different burial rituals, while other practices might be common across such
boundaries”. 

A~1,400-page study of cultural geography and interregional contacts in prehistoric times offers a~more fine-grained exploration than was possible at the time of writing of Guo and He's \textit{History}. Systematic examination of the available evidence leads to the identification of no fewer than “four subregions
showing fairly distinct archaeological assemblages, burial patterns, and subsistence
systems, indicating that they were probably inhabited by different cultural groups” \citep[588]{hein2013}. 

The first of these is the Anning river \zh{安宁河} valley. Settlements dating to the third millenium\,BC yield relatively consistent archaeological finds, including “coarse sand-tempered low fired red-brown ceramics (mainly large urns with finger-tip impressed appliqué strip below the rim, \textit{bo} and \textit{wan} bowls, vases, and a~few lids and rarely spouts), accompanied by a~few polished stone woodworking tools, arrowheads, and among the surface finds also perforated
stone-knives” \citep[559]{hein2013}. These commonalities are interpreted as implying that the communities in these settlements “probably shared
[a] similar cultural tradition and thus identified themselves as part of the same larger group” \citep[589]{hein2013}. At one of the sites, Dayangdui \zh{大洋堆}, evidence points to outside influence around 2000\,BC, followed by cultural {assimilation}: 

\begin{quotation}
	Both in ceramic quality and
	execution, the early Dayangdui ceramics ({\dots}) strongly resemble ceramics from sites in 
	Gansu \zh{甘肅} and Qinghai \zh{青海} attributed to the Qijia \zh{齊傢} culture. It is therefore not
	unlikely that the earth-pit graves at Dayangdui were built by a~group of Qijia origin. This would
	suggest a~date between 2200 and 1800\,BC ({\dots}). \citep[562]{hein2013}
	
	The middle and late Dayangdui assemblages do not contain any metal objects, however,
	and they show a~mixture of both early Dayangdui and local Neolithic traits %\footnote{The original reads “local Neolithic trades”; this is presumed to be a~typographical error and is replaced here by “traits”.} 
    that indicate some
	form of acculturation of the group of immigrants. As no similar sites of clear foreign origin have
	been identified in the Anning River Valley, it is likely that migration of whole groups from the
	North occurred rarely. \citep[594]{hein2013}
\end{quotation}	

%Command \noindent added to avoid having an indent. Proofreader suggestion: since this sentence continues the argument, it is better not to indent. 
{\noindent}In this area, megalithic graves appeared, and were subsequently imitated in neighbouring regions: 	

\begin{quotation}
	Graves with stone-construction parts are common throughout Southwest China, but
	megalithic graves seem to be unique to the Anning River Valley. The ceramics associated with
	these graves indicate a~local origin of this burial tradition in the Xichang area. This impression is
	supported by the fact that all early megalithic graves ({\dots}) are located in Xichang, while the megalithic graves in other regions
	such as Dechang, Mianning, Puge, and Xide all date to Phase IIa at the earliest. Why this kind of
	burial mode arose is uncertain, but its overall development and spread is relatively clear: it
	started with small constructions used for a~single instance of interment of several people,
	possibly in a~secondary mode of burial. During or after the burial, communal drinking rituals
	took place which seem to have become more extensive over time, as the large number of
	drinking vessels both in later graves and related ceramic pits shows. ({\dots})
    
    As far as daily life and mode of subsistence are concerned, the tool assemblages from
	megalithic graves and related settlement sites in the Anning River Valley show an agricultural
	and probably settled mode of living involving the planting of rice and probably other cereals,
	often supplemented by hunting, and in some places fishing. Only the sites in Puge show a
	continued primary reliance on hunting. Metal seems to have mainly been used for personal
	ornaments and only secondarily weapons or tools. \citep[595-599]{hein2013}
	
\end{quotation}	

Archaeological remains from the Neolithic to Bronze Age collected in Yongning in 1958 are considered to be connected to the megalithic graves \citep[933]{hein2013}. It is relevant to the history of Yongning that the spread of this burial mode is hypothesized to have taken place through cultural diffusion~-- presumably by persons who had participated in communal rituals and later reproduced these practices in their home settlements~-- rather than through military conquest. It is for future excavations to verify the existence of this connection. 

The second subregion is one of remote mountains, a~harsh environment where “groups of different origins conducted different kinds of
burial rituals next to each other, apparently respecting each other's monuments and even
adopting part of each others' burial customs and objects. In this meeting place of different groups,
cultural and other forms of identity (or at least their expression in the choice of grave form, burial
mode, and object assemblage) thus seem to have been extremely fluid” \citep[602]{hein2013}.

The third subregion is one of fertile valleys to the Southeast. The inhabitants of the earliest settlements “practiced a~hunter-gatherer lifestyle, using caves and open-air sites either as seasonal or hunting camps rather than living in permanent settlements”, later “practising incipient agriculture in a~particularly
congenial environment, living either in permanent or semi-permanent settlements” \citep[605-606]{hein2013}.

The fourth and final subregion comprises the high-altitude mountains, pla\-teaus and valleys of the Southwest, a~geographical area that includes Yongning. 

\begin{quotation}
	The people living in Yanyuan \zh{盐源} and Ninglang \zh{宁蒗} ({\dots}) belonged to a~clearly
	separate cultural group for whom armed combat~-- sometimes combined with horseback-riding~-- was a~central part of their life and identity. The emphasis on horse-riding, the interment of horse heads and sheep shoulder blades in
	graves and the overall metal assemblage (in particular the staff heads) seen at Yanyuan are
	essentially foreign to the research area. Pictorial evidence for horse-riding is known from the
	Dian \zh{滇} culture context, but horse skulls or long bones have never been. The interment of horse
	bones is instead common in the Northern Steppe and the Ordos region, and elements of horse
	gear similar to those seen in Yanyuan have been reported from there as well. ({\dots})
	It is therefore likely that the burying group of the “warrior graves” in Yanyuan is of a
	northern origin, be it the upper Minjiang \zh{岷江} or even the steppe. \citep[616-618]{hein2013}
\end{quotation}


%\il{Sinitic}Chinese historical sources on Na and \ili{Naxi} history as summarized by \citet{chavannes1912}
%(reprinted in \citealt{bacot1913}); the historical outline presented by \citet{gros1996}; and Chapter 9 of \citet{mathieu2003}. 


%
%"That such a~labor-intensive custom as erecting megalithic graves was taken up
%by groups that were originally culturally distinct, is likely the outcome of the personal encounter
%with the powerful rituals surrounding them and the bonds they created between those who
%participated in them, forming a~supra-local sense of community throughout the Anning River
%Valley and neighboring regions, leading to the emergence of a~new kind of identity that
%transcended the previous cultural and local group boundaries without necessarily destroying
%them, as the continuation of local particularities in each of the various location shows." 624
%
%"However, it is not enough to point out single objects or features and then jump to
%inferences such as "migration," "influence," or "contact," which are so readily used as blanket
%explanations for similarities between the material remains in different places. In a~first step, we
%have to define what we mean by "contact," what different types of contact there may be, and how
%we can identify them in the archaeological record. The challenging terrain of the research area
%furthermore makes it imperative to consider possible routes and reasons for contact."

Yanyuan \zh{盐源}, a~strategic area due to its abundance in rock salt, was conquered in 225\,AD by the Chinese, who referred to the local population as the “Mosha” \zh{摩沙}. Returning to the hypothesis advanced by \ili{Naxi} historians Guo Dalie and He Zhiwu~-- that these “Mosha” are the ancestors of today's \ili{Naxi}~--, one might rephrase the claim as follows: military defeat prompted clans from this warrior population to retreat westward, conquering new territories that included Yongning, the banks of the Yangtze, and later the plain of Lijiang~-- areas which remain their principal settlements to this day. 

However, there does not appear to be decisive evidence linking the “warrior culture” of Yanyuan specifically to the \ili{Naxi}, as opposed to other \ili{Sino-Tibetan} groups who later migrated into more distant regions of present-day Yunnan and Myanmar. A~modest piece of evidence relevant to this question concerns the path that ritual practitioners dictate to the soul of the deceased for its journey back to the ancestral homeland. Such ritual journeys, which have been studied in many parts of East and Southeast Asia, have been shown to preserve memories of historical migration routes \citep{blackburn2004, gaenszle2012, mckhann2012}. Among the \ili{Naxi}, different clans maintain different itineraries, yet all these ritual paths pass through Yongning~-- where they join with those of the Na~-- and continue northward \citep[50-55]{guoetal1994}. As the distance from Yongning increases, the identification of place names soon becomes uncertain or impossible. The designation of Minya Konka \zh{贡嘎山} as the endpoint of the journey is not unlikely to be a~later addition, inspired by the prestige of this 7,500-meter peak, which serves as a~mooring point for traditions and beliefs of various peoples of the eastern Himalayas. On the other hand, it is worth noting that the path for returning souls does not pass through Yanyuan; there is thus no evidence that the group's forebears ever dwelt further east than the Yongning area. Naturally, such ethnological evidence does not carry considerable weight, as ritual paths may have been altered at any point in their {oral transmission}. Still, it suggests that the identification of the “Mosha” \zh{摩沙} as direct ancestors of the Na and \ili{Naxi} should not be treated as established fact.

\begin{quotation}
	Local oral narratives and scholarly writings which discuss the origins 
	and migrations of hill peoples of the far eastern Himalaya often share the 
	same propositions. Firstly, both types of sources tend to plot routes of  
	migration between an assumed original homeland area or origin place and 
	a present-day dwelling location; direction of movement and itineraries 
	are of shared importance here. Second, they both claim identification of
	contemporary populations with their purported ancestors from past times 
	and distant places, with implicit and explicit claims of ethnic continuity. \citep[83]{huber2012}
\end{quotation}

Whatever the precise relationship between the “Mosha” and the present-day Na and \ili{Naxi}, the conquest of the Yanyuan area by Chinese forces in 225\,AD stands as a~major historical landmark. It established ties with the heart of the Chinese empire that were never entirely severed thereafter, even during periods when the Chinese central power was weakest, such as the four centuries that followed. 

\subsection{Empires and indigenous chieftains}
\label{sec:feudal}

%: archaeology reveals major discontinuities during the Han dynasty, as “the thousand year old tomb-building culture of the Zuo \zh{笮} was suddenly replaced by the earth graves culture of the Kunming \zh{昆明} and Sui \zh{嶲} tribes, and, in the north east, by the tomb building culture of the Han” \citep[366]{mathieu2003}. 
%\il{Sinitic}Chinese chronicles provide glimpses into a~troubled history, allowing for the building of hypotheses such as that the Yongning area was conquered in the first century \,AD “by the Kunming \zh{昆明} and Sui \zh{嶲} from what is today Dali \zh{大理}” \citep[367]{mathieu2003}. 

%{\largerpage[-1]} % Added on April 22nd, 2025

In 794\,AD, the Nanzhao \zh{南诏} kingdom, centred on the fertile lands around Erhai Lake (\zh{洱海}, today
a~Bai-speaking area), conquered a~broad region that included both Yongning and Lijiang \zh{丽江}.
Following the fall of Nanzhao in 902, the kingdom of 
Dali (\zh{大理国}, 937--1253), likewise centred
on Erhai Lake, exercised control over Yongning and Lijiang, although both continued to be ruled by indigenous chieftains.

At the outset of the Yuan dynasty, a~new feudal chieftain (\textit{tǔsī} \zh{土司}) was installed in Yongning by the Mongolian forces, who passed
through Yongning en route to their conquest of the Dali kingdom. A~chapter of the imperial geography \textit{Yuan Yi Tongzhi} \zh{《元一统志》}{\kern-4pt}, dated 1286, records the name of Yongning as \zh{楼头} (present-day \ili{Mandarin} pronunciation: \textit{lóutóu}). Using the system proposed by \citet{coblin2007}, this name reconstructs as *\ipa{ləw dəw}, which is clearly cognate with the present-day name of Yongning: \ili{Naxi} /\ipa{ly˧dy˩}/ and Na /\ipa{ɬi˧di˩}/, as discussed in Chapter 1, \sectref{sec:presentationofthenalanguageandnasocietyandreviewofearlierstudies}. It is likely that the authors based their transcriptions on the pronunciation used in Lijiang, a~more prominent centre than Yongning in terms of both population and accessibility. In that sense, their record does not constitute direct evidence for the language spoken in Yongning at the time. Nevertheless, it provides a~reasonable basis on which to hypothesize linguistic continuity in Yongning since at least the thirteenth century. 

The Yongning chieftain who surrendered to the Mongolians in 1253 claimed a~genealogy of thirty-one generations since his ancestors had conquered the region. %Assuming linguistic continuity, Yongning Na would have been introduced into the area at 
Assuming an average of twenty to twenty-five years per generation, this would place the initial conquest in the period between 500 and 650\,AD. 
Of course, %it may also be that an earlier form of the language was already spoken in and around Yongning earlier on, and 
this change in the ruling class (which still requires confirmation from other sources) may not have entailed significant linguistic change. It is therefore impossible at present to determine how far back in time linguistic continuity can be assumed in Yongning.
%would have been introduced into the area at

The introduction of \ili{Tibetan} Buddhism dates to roughly the same period as the Mongolian conquest, with the missionary
efforts of monks from Muli beginning in 1276 \citep[389]{guoetal1994}. In 1356, a~Kagyupa (\textit{bka' brgyud pa})
monastery was established. Two centuries later, in 1556, a~large Gelugpa (\textit{dge-lugs pa}) monastery was founded
in Yongning (\ili{Tibetan} name: \textit{dgra} \textit{med} \textit{dgon} \textit{pa}). Earlier cults persisted, with a~division of ritual labour between the Buddhist clergy and the Na \textit{Ddabe} \ipa{dɑ˧pɤ˧} ritual
practitioners.\footnote{Twentieth-century reports suggest that monks and \textit{Ddabe} \ipa{dɑ˧pɤ˧} coexisted peacefully, with
an established division of ritual responsibilities: the \textit{Ddabe} \ipa{dɑ˧pɤ˧} were called upon for rites associated with livestock, such as the slaughtering of pigs; Buddhist monks officiated on prescribed days of the religious calendar; and for major ceremonies, such as funerals, both groups participated. This long-standing coexistence, and the occasions for
interaction it provided, appear to had led to a certain amount of
convergence. The \textit{Ddabe} \ipa{dɑ˧pɤ˧} have adopted certain ritualized behaviours characteristic of Buddhist monks. They prepare rituals with great attention to detail, ensuring in advance that all the necessary paraphernalia~-- such as butter, candles,
water, and different types of flour~-- are available. This contrasts with the practices of the ritual specialists among the \ili{Yi}, known as \textit{Bimo} \zh{毕摩} \citep{Nevot_MastersofPsalmody}, who have a~habit of requesting objects and accessories at unpredictable moments during a ceremony, as if responding to sudden inspiration (Latami Dashi, p.c. 2008). Furthermore, the gestures performed by the \textit{Ddabe} during rituals have come to resemble those of Buddhist monks. Conversely, some Buddhist monks are reported to study the Na
system of horoscopy~-- one of the traditional domains of expertise of the \textit{Ddabe}. The study of Na rituals, particularly in comparison with those of the Naxi, holds great promise for understanding Na cultural dynamics (see \citealt{mathieu2015} and references therein). Such an endeavour, however, requires expertise in \ili{Tibetan} philology, an in-depth knowledge of \ili{Tibetan} Buddhism, and a range of other specialized skills beyond my own field of study.} From that point onward, however, Buddhism became the dominant religion in Yongning. (At the time of the Communist takeover, there were over 700 monks at the Yongning monastery.) This contributed to an increasing cultural divergence between Yongning and the Lijiang plain. In Lijiang, no school of
\ili{Tibetan} Buddhism succeeded in establishing and maintaining a~central role: its religious history was characterized by abrupt turns and shifting allegiances \citep[22]{gros2019chronology}. The \ili{Naxi} \textit{Dobbaq} \ipa{to˧mbɑ˩} ritual tradition~-- likely shaped in large part by Bön influences~-- nearly attained the status of an official cult \citep{jackson1979}.

\begin{photofigure}[t!!]
	\caption{The Yongning monastery. The dialect under study is spoken in Alawua, a~hamlet adjacent to the monastery. Autumn 2006.}
	\includegraphics[width=\textwidth]{figures/MonasteryAndPlain.jpg}
\end{photofigure}

\begin{photofigure}[t]
	\caption{Worshippers at the Yongning monastery. Autumn 2006.}
	\includegraphics[width=\textwidth]{figures/worshippers.jpg}
\end{photofigure}

%{\largerpage[-1]} % Added on April 22nd, 2025

During the Yuan and Ming dynasties, persistent warfare took place among the feudal chieftains of Yongning,
Lijiang, and Yanyuan \citep[430--431]{guoetal1994}. In 1545, Yongning
united with the neighbouring territories of the five \textit{suǒ} \zh{所} 
(Zuosuo \zh{左所}, Yousuo \zh{右所}, 
Qiansuo \zh{前所}, Housuo \zh{后所}, and 
Zhongsuo \zh{中所}). The Yangtze River served
as the boundary between the territories of Yongning and Lijiang.

During the Qing dynasty (1644--1912), the \ili{Naxi} chiefdom of Lijiang was brought under direct Chinese administration. In contrast, the feudal chieftain system remained in place in Yongning, owing to the failure of successive military expeditions to subjugate the Liangshan \zh{凉山} \ili{Yi} region, which serves as a~natural gateway to Yongning \citep[460]{guoetal1994}. This contributed to an ever-deepening cultural differentiation between Lijiang, which became increasingly Sinicized, and peripheral regions such as Yongning. The Yongning chieftains actively resisted Chinese settlement, prohibiting the cultivation of rice and forming alliances with \ili{Yi} chieftains to obtain their military support. These developments are reported to have hastened the migration and settlement of \ili{Yi} families into the Yanyuan and
Yanbian areas, leading to the displacement of earlier populations~-- speakers of closely related language
varieties. These varieties are referred to as “\ili{Naxi} dialects” by Guo Dalie and He Zhiwu (p.\ 461), but are here designated as “\ili{Naish} languages”, as explained in \sectref{sec:thepositionofnaandnaxiwithinsinotibetan}. The \ili{Laze}, a~small group of approximately four hundred people who relocated from Yanbian to their
present location in Muli towards the end of the nineteenth century, appear to be among the speakers of \ili{Naish} languages who departed from Yanbian as
it became increasingly dominated by \ili{Yi}-speaking communities (on the \ili{Laze} language, see \citealt{huang2009}).

%{\largerpage[-1]} % Added on April 22nd, 2025

During the 1920s, Yongning became a~node in trade routes linking Tibet, the \ili{Yi} territories centred on Ninglang, Lijiang, and inland China. “This was the first time in
history that the Moso had frequent visitors from the outside on a~considerable scale. ({\dots})~this period was also the onset of the syphilis epidemic” \citep{shihetal2002}, an epidemic curbed in the 1950s.\footnote{This epidemic is alluded to by Goullart, who stayed in Lijiang in the 1940s: “The Nakhi men ({\dots}) knew well enough that most of the Liukhi [\ipa{ly˧ çi˧}, the {Naxi} term for the Na of Yongning] tribe was infected with venereal disease”
	%	, and it was only this dread of almost certain infection that made the
	%	Nakhi and other sensible men give a~wide berth to the Liukhi enchantresses” 
	(Chapter 3 of \citealt{goullart1955}).} 

\subsection{The People's Republic of China}
\label{sec:prc}

{\largerpage} % Added on April 22nd, 2025

After the founding of the People's Republic of China in 1949, the central government laid out plans for the graduated integration of frontier areas along the outer margins of the dynastic power realm~-- those known during the Qing dynasty as ‘vassal states’ (\textit{fānshǔ} \zh{藩屬}).

\begin{quotation}
	Instead of ‘mobilizing the masses’ like in the Yunnan interior and elsewhere in China, land reform on the volatile frontier took the form of “uniting the feudal to fight feudalism” (\textit{lianhe fengjian fan fengjian} \zh{联合封建反封建}), a scheme designed to win over the ethnic elite. The CCP united front work saw a~revival of traditions in forms of pledging allegiance, conferring official titles, and the ritual of court audiences~-- those who cooperated were invited to serve in the new government and local dignitaries were taken to Beijing to have an audience with the CCP leaders. \citep[43-44]{guoState2008}
\end{quotation}

\begin{sloppypar} % Indispensible to avoid overfull hbox.

The Liangshan \zh{凉山} \ili{Yi} area, which constitutes the gateway to Yongning, proved a~hard nut to crack. Armed rebellion broke out in 1956. By 1958, large-scale ``pacification" operations had been completed \citep[228-231]{guoState2008}, and administration of the area by the Communist state was instituted. Since then, reforms have been applied essentially top-down. The guiding policy has been that the majority group points the way forward, leading minority groups towards modernity and eventual assimilation. 

In a~study of the Drung, an ethnic
group residing in an even less accessible area of Yunnan, \citet{gros2014} argues that these policies represent a~fundamental shift from earlier relationships of
vassalage between local powers and the state(s). Under the previous system, feudal chieftains paid tribute and received titles in return; the balance of this exchange~-- how much tribute was offered, and how much recognition and autonomy granted in return~-- was assessed by both parties. By contrast, top-down state policies
do not partake in any logic of exchange~-- \textit{don et contre-don}, as emphasized in the classic
study by Marcel \citet{mauss1990}. 

Post-1956 events remain a~highly sensitive topic. Mazard's observation about the Nusu (\ili{Yi}) also applies to the Na:

\end{sloppypar} % Indispensible to avoid overfull hbox.

\begin{quotation}
	They generally
	treat the years from 1958 to around 1979 as a~single historical period circumscribed
	by the collectivisation and de-collectivisation of their land. ({\dots}) Why do many of them ({\dots}) identify this period as a~whole with the
	Cultural Revolution, even though they know and employ the term ‘Great Leap
	Forward’ as well?
	
	One reason may be that the CCP allows overt (though
	limited) criticism of the Cultural Revolution; not so the Great Leap Forward (or the Anti-Rightist
	Campaign). The Party has never denounced the Great Leap Forward as a~{mistake} ({\dots}). The
	Cultural Revolution has its villains (the Gang of Four); so do the Civil War (the Guomindang) and
	the Second World War (the {Japanese}). Discussion of the Cultural Revolution is not easy, but it is
	possible. When Nusu elders allude to their suffering as occuring under the ‘Cultural Revolution’
	(even if it took place in 1959), they lay claim to a~permitted register of complaints. \citep[172]{mazard2011}
\end{quotation}

Recent history (since the 1980s) will only be addressed here indirectly, through a~discussion of ethnic classification (\sectref{sec:ethnicclass}) and a~review of studies on the impact of tourism on Na society (\sectref{sec:presentdaysociologicalstudiestheimpactoftourismsincethe1990s}).

\section{Ethnic classification: Naxi, Mongolian, Mosuo or Na}
\label{sec:ethnicclass}

%\largerpage[-2]

Ethnic categorization as defined by the state “crafted the prism through which the modern Chinese state, and increasingly the people of China and the world at large, have come to view and understand non-Han Chinese identity” \citep[5]{mullaney2010}. This system of categorization, which to this day (as of 2025) appears on each individual's identity card (\textit{shēnfen zhèng} \zh{身份证}, literally ‘certificate of identity’), has such a~strong bearing on present-day identities that it warrants separate discussion.

\begin{quotation}
	{\dots}~in the China of Chiang Kai-shek, the Nationalist regime vociferously argued that the country was home to only one people, “the Chinese people” (\textit{Zhonghua minzu} \zh{中華民族}), and that the supposedly distinct groups of the republic were merely subvarieties of a~common stock. At the same time, a~counterdiscourse emerged among Chinese scholars in the newly formed disciplines of ethnology and linguistics, a~discourse in which China was reimagined as home to many dozens of unique ethnic groups~-- a~newly imported concept also translated using the term \textit{minzu} \zh{民族}. (\citealt[2]{mullaney2010}; see also \citealt{bulag2012})
\end{quotation}

The first census of the People's Republic of China, in 1953-1954, recorded over four hundred different ethnic identities, more than half of which concerned Yunnan, a~province bordered on all sides by multi-ethnic areas (Vietnam, Laos, Myanmar, Tibet, Sichuan, Guizhou, and Guangxi). Until recently, little information was available about the process of ethnic identification (\textit{mínzú shíbié} \zh{民族识别}) whereby these identities were grouped into the roughly fifty nationalities (\textit{mínzú} \zh{民族}) officially recognized by the state. Sources made available in the first decade of the 21\textsuperscript{st}century reveal how small teams of researchers from a~range of social science backgrounds assessed possibilities for groupings, and assigned names to these groupings, all under a~tight timeline (less than six months). “In the years following the end of the project, cultural and scientific works rewrote the history of China and its
diversity in an effort to promote a~so-called “historic” and “ancestral” model of the 56 \textit{minzu}
components” \citep{frangville2011}. 

At first, the categories remained open to some fine-tuning: adjustments were made during the 1960s and 1970s, including the recognition of two ethnic minorities that had been absent from the 1954 classification. In 1987, however, it was declared that no further nationalities would be recognized, and the figure of fifty-six was made final.

\begin{quotation}
	{\dots}~the idea of China as a~“unified, multinational country” (\textit{tongyi de duominzu guojia} \zh{统一的多民族国家}) is a~central, load-bearing concept within a~wide and heterogeneous array of discourses and practices in the contemporary People's Republic. China is a~plural singularity, this orthodoxy maintains, composed of exactly fifty-six ethnonational groups (\textit{minzu} \zh{民族}): the Han ethnic majority, which constitutes over ninety percent of the population, and a~long list of fifty-five minority nationalities who account for the rest. Wherever the {question} of diversity is raised, this same taxonomic orthodoxy is reproduced, forming a~carefully monitored orchestra of remarkable reach and consistency: anthropology museums with the requisite fifty-six displays, “nationalities doll sets” with the requisite fifty-six figurines, book series with the requisite fifty-six “brief histories” of each group, Olympic ceremonies with fifty-six delightfully costumed children, and the list goes on. Fifty-six stars, fifty-six flowers, fifty-six \textit{minzu}, one China. \citep[1]{mullaney2010}
\end{quotation}

The teams tasked with surveys for the national project of ethnic identification in 1953-1954 worked independently in each province, and decisions were likewise made on a~province-by-province basis. The Na living in Yunnan were classified as part of the \ili{Naxi} minority. Those living in Sichuan were classified as Mongolian; this surprising choice was presumably influenced by the lingering prestige of the name ‘Mongolian’ in the region since the Mongolian army’s formidable crossing of the Himalayas (passing through Yongning) and victory over the Song dynasty.

\begin{quotation}
	Due to historical tensions between the Na and the \ili{Naxi}, when the Sichuan Na learned that they
	would be classified as \ili{Naxi} in the early 1950s, they protested by taking over the county
	government offices. As the federal government limits recognition to the fifty-six
	ethnicities, local officials were perplexed as to what to do, and a~face-saving compromise
	was established such that the Sichuan Na could be classified as Mongolian, on the basis
	that the Mongols had invaded the area seven hundred years previously, and perhaps the
	Na were descendants of these Mongols. Although this designation is within historical
	memory, the Na in Sichuan have clearly adopted their designation as Mongolian, and
	colorful plastic plaques of Genghis Khan hang prominently on the walls in homes.
	%Sichuan Na also disavow designation as Mosuo, likely because of the associations with
	%the term ‘Mosuo’ developed in the tourist industry. 
	\citep[9]{lidz2010}
\end{quotation}

%Command \noindent added to avoid having an indent. Proofreader suggestion: since this sentence continues the argument, it is better not to indent. 
{\noindent}The claim of Mongolian descent apparently predates the systematic process of “ethnic identification” undertaken by the People's Republic of China. 

\begin{quotation}
	Moso chiefs in both Yunnan and Sichuan Provinces claimed that they were descendants of the Mongols. According to Joseph Rock, who personally befriended key members of the Yongning chief's family ({\dots}), the general superintendent (\textit{zongguan} \zh{總管}) of Yongning at that time “was proud of his Mongol origin, for he was a~descendant of one of the Mongol officers left by Kublai Khan in Yung-ning to govern that territory” \citep[359]{rock1947}. Abundant historical records indicate that it was commonplace for the Mongolian conquerors to leave troops of Mongol or non-Mongol ethnic background to govern the newly subjugated territories. ({\dots}) The problem is that in all cases other than the Moso chief's, there is evidence~-- such as records on stelae, tombs and tombstones (some inscribed in Mongolian), records of genealogy, language or vocabulary, and legends in one form or another~-- to substantiate the claim. The Moso aristocrats, however, had nothing to support their claim of Mongol ancestry. \citep[40-41]{shih2010}
\end{quotation}

%Command \noindent added to avoid having an indent. Proofreader suggestion: since this sentence continues the argument, it is better not to indent. 
{\noindent}Shih suggests that the new chieftain was “a Xifan [\ili{Pumi}] officer in the Mongol troops left by Kublai Khan to
rule Yongning” \citep[51]{shih2010}. One piece of evidence he adduces is the identification of the Yongning chieftain's ethnicity as “Xifan” \zh{西番} in Ming-dynasty chronicles. However, the interpretation of the term “Xifan” as referring specifically to the \ili{Pumi} is not self-evident: the label may have been used in a~broad sense that included speakers of the language ancestral to Yongning Na. To this day, the \ili{Namuyi} of Muli, who are speakers of a~{Naic} language (about which see Chapter 1, \sectref{sec:thepositionofnaandnaxiwithinsinotibetan}), are included among the “Xifan” \zh{西番}, a~cover term for various non-{Tibetan} groups. Another piece of evidence is ethnographic: at the succession of the Yongning chieftain, the \ili{Pumi} would perform a~ritual akin to \textit{Sikee} /\ipa{sɯ˧kʰɯ˩}/, the Na ritual associated to the giving of a~household member to another house (typically, the giving of a~young woman); this is consistent with Shih's hypothesis that their ethnic group was the donor of the Yongning chieftain himself. Shih states his interpretation as follows:

\begin{quotation}
	In the case of \textit{sike} [phonetic transcription: /\ipa{sɯ˧kʰɯ˩}/], a~household member was given to become a~particular person's wife in another household. Because the status of wife necessarily ended with the life of the woman in {question}, \textit{sike} was a~one-time ritual between the two families concerned. In the succession ritual, however, as the historical records suggest, when a~member of the \ili{Pumi} was given to become the chieftain of Yongning, a~territory dominated by the Moso, the status of chieftaincy was perpetual, as was the ritual of interrogation. In both cases, the rituals were performed to dramatize a~reassertion of the unbreakable blood bond between the deceased and her or his natal family. \citep[48]{shih2010}
\end{quotation}

%\largerpage[-2]

Inclusion in the Mongolian minority proved a~comfortable fiction: paradoxically, it granted the Na of Sichuan a~place of their own within the landscape of the officially recognized ethnic groups of Sichuan. Initially the ‘Mongolian’ label was taken very seriously, and efforts were made to teach the Mongolian script to \ili{Naish} speakers of Sichuan~-- with predictably poor results. The initial ethnic identification has not been revised since, due to a~national policy of keeping the Pandora’s box of ethnic labels firmly closed. Their fictitious cousins in Mongolia do not appear to have raised objections to the term ‘Mongolian’ being applied to this small group in Sichuan; and the label has had the effect of preventing the Na from being pooled together with geographically closer neighbours such as the \ili{Naxi}. By contrast, the classification of the Na of Yunnan as part of the \ili{Naxi} rendered them a~minority within a~minority, reducing the number and influence of their representatives at various institutional levels. 

In light of the historical overview summarized in \sectref{sec:historicaloutline}, it is not difficult to understand why the Na tend to think of themselves as distinct from the \ili{Naxi}, despite the conspicuous similarities between their respective languages. Resentment over inclusion in the \ili{Naxi} minority has fuelled a~search for recognition as a~separate group. From this vantage point, the \isi{endonym} /\ipa{nɑ˩˧}/ is less than ideal: /\ipa{nɑ˩˧}/, presumed to mean ‘black, dark’, also appears in the \isi{endonym} of the \ili{Naxi}: /\ipa{nɑ˩çi˧}/, where /\ipa{çi˧}/ means ‘person, human being’. Use of nearly identical endonyms could cast doubt on the legitimacy of a~clear-cut separation. 

Instead, the Na of Yunnan have come to favour the \isi{exonym} ‘Mosuo’ (\textit{Mósuō} \zh{摩梭}), a~name formerly used in Chinese records, which was officially replaced after 1949 by ‘\ili{Naxi}' (\textit{Nàxī} \zh{纳西}). Thus, the dictionary of pictograms originally published by Li Lin-ts’an \zh{李霖灿}, Chang K’un \zh{张琨} and Ho Ts’ai \zh{和才} as 
	\textit{Dictionary of Mo-So hieroglyphics} (\citeyear{lietal1953}) was reprinted in \citeyear{lietal2001} on the mainland under the title \textit{Dictionary of {Naxi} pictograms}; all occurrences of \textit{Mósuò} \zh{麽些} in the book were replaced by \textit{Nàxī} \zh{纳西}.

The historical variants of the name ‘Mosuo’ are shown in \tabref{tab:thenamesofthenaendonymsandexonyms} (Chapter 1, \sectref{sec:endoexo}). This designation's origin is unclear. \citet[132]{chavannes1912} cites Chinese chronicles as indicating that the Mo-so tribe was formed during the Nanzhao period (737-902) out of two distinct elements, the Mo and the So. 

Reviving the demised term ‘Mosuo’ to designate the Na of the Yongning area is a~strategic move in support of claims to recognition as a~group distinct from the \ili{Naxi}, since the names ‘Mosuo’ and ‘\ili{Naxi}’ are clearly distinct. Moreover, the term ‘Mosuo’ carries a~double advantage: it is a~term of great antiquity, having appeared in Chinese chronicles since the first millenium \,AD, and it had fallen into disuse by the mid-twentieth century, which lends it a~certain quaint charm and a~touch of mystery. In 1990, following vocal protests against the label ‘\ili{Naxi}’, the ‘Mosuo’ of Yunnan were granted recognition at the provincial level as a~separate subgroup within the \ili{Naxi} minority. This label now appears poised to become standard in Sichuan as well \citep[10-11]{lidazhu2015}. La Mingying \zh{喇明英}, a~member of the Sichuan Academy of Social Sciences who identifies officially as ‘Mongolian’ but expresses a~preference for the label ‘Na’, reports that self-identification as ‘Mosuo’ is gaining ground among the younger generations on the Sichuan side of the designated ‘Mosuo’ tourist area. 

\begin{quotation}
	Disagreement among the Na about names [ethnonyms] is a~cause for disputes; people sometimes even come to blows. Acknowledging one's ethnic identity, and having a~sense of belonging to the community, constitute the most basic and essential factors in “ethnic identification”. The multiplication of ethnic denominations for the Na of the Lugu Lake area generates great perplexity about their ethnic identity, to the point of causing prejudice to their sense of belonging to a~community and to their ethnic cohesion. \citep[53]{lamingying2015} 
    %\footnote{
    
    \medskip \noindent \textit{Original text:} \zh{因称呼的分歧,纳人内部时有争论甚至打架的情况发生。对民族身份的承认和群体归属感是“民族认同”最基本的要素。泸沽湖地区纳人族称的多元化在很大程度上对其民族身份的认同造成很大的困惑,甚至影响了纳人的群体归属感和民族凝聚力。}
    %}
\end{quotation}

Before the Communist takeover, the Na of Yongning were divided into three hereditary castes: the family of the chieftain, /\ipa{sɯ˧pʰi˧}/, constituted the nobility; this group was distinct from the commoners, /\ipa{dze˧kʰɤ˧˥}/, who formed the majority (about 640 families in the late 1950s). A~third, smaller group of about 280 families were serfs, /\ipa{wɤ˧}/. Historically, when outsiders joined the community~-- whether as~war captives or as~immigrants from areas near and far~--, they would be integrated into the serf caste. This caste also accommodated commoners who had been stripped of their rank as punishment for rebellion \citep{liu1981}. 

The administrative assignment of ethnic identity~-- whether as ‘\ili{Naxi}’ or ‘Mongolian’~-- effaced these earlier distinctions of caste. It also coincided with the end of the practice of integrating newcomers into Na society through the caste system. The steady influx of settlers into Yongning, recorded in successive editions of
the county Annals (\textit{xiànzhì} \zh{县志}; the entire collection
bears the 
name {\kern-3pt}\zh{《中华人民共和国地方志丛书》}{\kern-4pt}, “Collection of local chronicles of the People’s
Republic of China”), has led to the cohabitation of individuals whose official ethnic identities remain distinct~-- some are designated as “Han”, others as “Mosuo”, “\ili{Naxi}”, “\ili{Pumi}”, “\ili{Yi}”, and so on~--, in keeping with the state ideology of a~multi-ethnic, unified China. 



\section[Anthropological research: The fascination of the Na family structure]{Anthropological research: The fascination of the Na's kinship system and family structure}
\label{sec:anthropologicalresearchthefascinationofnafamilystructure}

The rich \isi{morphotonology} of the Na language, which forms the central topic of this book, appears to have gone unnoticed until the early 21\textsuperscript{st} century. By contrast, the Na kinship system and family structure have long been famous far beyond the circles of specialized ethnologists. The following excerpt is drawn from the highly exoticized account of Peter Goullart, a~Russian-born traveller, explorer, and author who lived among the \ili{Naxi} in Lijiang from 1942 to 1949, during the final years of intense caravan traffic (1920s-1940s).

\newenvironment{amquote}{\list{}{\rightmargin0pt\leftmargin7mm} 
\item\relax}
{\endlist}

\begin{amquote}
	The arrival of the members of a~certain matriarchic tribe, living about seven days by caravan
	north of Likiang, always created a~furore in Likiang. Whenever these men and women passed through
	the market or Main Street on their shopping expeditions, there was indignant whispering, giggling
	and squeals of outraged modesty on the part of Likiang women and girls, and salacious remarks from
	men. They were the inhabitants of the Yungning duchies across the Yangtze at the apex of the great
	bend. The Nakhi [\ipa{nɑ˩-çi˧}] called them Liukhi [\ipa{ly˧-çi˧}] and they called themselves Hlihin [\ipa{ɬi˧-hĩ˧}]. The structure of their
	society was entirely matriarchal. The property passed from mother to daughter. Each woman had
	several husbands and the children always cried, ‘We have mama but no papa.’ The mother’s husbands
	were addressed as uncles and a~husband was allowed to stay on only as long as he pleased the
	woman, and if he didn’t, could be thrown out without much ceremony. The Yungning country was a~land of free love, and all efforts of the Liukhi women were concentrated on enticing more lovers
	in addition to their husbands. Whenever a~\ili{Tibetan} caravan or other strangers were passing
	Yungning, these ladies went into a~huddle and secretly decided where each man should stay. The
	lady then commanded her husbands to disappear and not to reappear until called. She and her
	daughters prepared a~feast and danced for the guest. Afterwards the older lady bade him to make a~choice between ripe experience and foolish youth. ({\dots})
	
	With their lips heavily rouged and eyes painted, they walked slowly, or rather undulated, through
	the streets, swaying their hips, smiling and casting an amorous eye on this man or that. That
	alone was enough to incense the less sophisticated Nakhi women. But when they walked slowly along
	hanging on the neck of a~husband or a~lover, and being held by the waist, this was too much for
	even the brazen Nakhi women, who spat or giggled nervously.~\citep[Chapter~3]{goullart1955}
	%  ({\dots})
	%
	%Only twice was my path crossed by Liukhi women and in both cases it resulted in a~mild scandal.
\end{amquote}


%Command \noindent added to avoid having an indent. Proofreader suggestion: since this sentence continues the argument, it is better not to indent. 
{\noindent}This sample of travellers’ accounts of a~purported “land of free love” suffices to explain the enduring fascination exerted by the Na on the general public. The present review does not seek to provide comprehensive coverage of the extensive anthropological, ethno-historical, and sociological literature on the Na. Rather, its aim is to convey a~sense of how the field has developed over time, of the historical evolution of research perspectives, and of the implications for local perceptions of social scientists conducting fieldwork in Yongning. 


\subsection[Surveys conducted in the 1960s]{A major source of information: Surveys conducted in the 1960s}
\label{sec:themainsourceofinformationonnafamilystructuresurveysconductedinthe1960s}

A~key resource for the study of Na society is the detailed sociological report based on surveys conducted in the 1960s \citep{bianjizuguojiaminweiminzushehuilishidiaochayunnanshengbianjizu1986}. The data, organized by village, reveal subtle variations both across villages and among individual households. Most subsequent scholarship has drawn extensively on the findings recorded in the three volumes of this report, which together amount to nearly one thousand pages.

The survey clarifies that, until the 1950s, the typical family structure in the Yongning basin was
matrilinear, with lifelong matrilocal residence. In non-tech\-ni\-cal terms, this means that brothers and sisters lived their entire lives in their mother’s house, together with maternal relatives: cousins, aunts and uncles, and the mother's mother along with her siblings. 

This configuration bears some resemblance to that found among the Minang\-kabau (Indonesia), as described by \citet{hadler2008}, and the Nayar (India), as described by \citet{fuller1976}. 

\begin{quotation}
	Men marry into an extended family, but remain attached to their mothers' houses. They return to that house daily to work the fields, convalesce there in times of sickness, and are eventually buried in the maternal family graveyard. A~husband and father is an evanescent figure. In the words of a~Minangkabau aphorism, “The \textit{urang sumando} is like a~horsefly on the tail of a~buffalo, or like ashes on a~burned tree trunk. [When a~little wind blows, it is gone.]” ({\dots}) Minangkabau culture has been termed matrifocal because, although men can be part of the lives of their wives and children, it is mother-centredness that grounds the family. \citep[6]{hadler2008}
\end{quotation}

%Command \noindent added to avoid having an indent. Proofreader suggestion: since this sentence continues the argument, it is better not to indent. 
{\noindent}Among the Nayar of India, the husband resided with his sister and visited his wife at night \citep{fuller1976}; in pre-1956 Na society, lovers met discreetly at the woman’s home. Among the Na, as among the Minangkabau and the Nayar, the answer to the “matrilineal puzzle”~\citep{richards1950}~-- the potential conflict of authority between father and maternal uncle~-- was to place authority with the uncle. Fathers did not play a~prominent social role; men's commitments were to their sisters' children, not to their own, who were raised in another household. “According to tradition, it is the \textit{mamak} (maternal uncle) who provides male authority in the lives of children” \citep[6]{hadler2008}. Likewise, among the Na, the male figure of authority was the maternal uncle, as illustrated in proverbs such as (\ref{ex:eagle}).

\begin{exe}
	\ex
	\label{ex:eagle}
	\ipaex{mv̩˧ʁo˥ {\kern2pt}|{\kern2pt} dze˩-hĩ˩-dʑo˥, {\kern2pt}|{\kern2pt} kɤ˩-nɑ˧mi˧; {\kern2pt}|{\kern2pt} di˧qo˧ se˧-dʑo˩, {\kern2pt}|{\kern2pt} ə˧v̩˧˥.}\\
	\gll mv̩˧ʁo˥\$	dze˩\textsubscript{a}	-hĩ˥	-dʑo˥	kɤ˩-nɑ˧mi˧	di˧qo˧	se˥	-dʑo˥	ə˧v̩˧˥\\
	heavens		to\_fly		\textsc{nmlz}	\textsc{top}	eagle	plain	to\_walk		\textsc{top}		maternal\_uncle\\
	\glt ‘As the Eagle is greatest of all that fly in the sky, so the Uncle is greatest of all that walk the earth.' (Field notes.)
\end{exe}

%Command \noindent added to avoid having an indent. Proofreader suggestion: since this sentence continues the argument, it is better not to indent. 
{\noindent}A~key difference, however, is that while the Minangkabau maintained (if tenuous) marital ties, and the Nayar also practised marriage, this institution was largely marginal among the Na of the Yongning plain prior to the social upheaval of the post-1956 period. The chieftains, through their interactions with patrilineal Chinese, {Tibetan}, {Naxi}, or {Pumi} peers, had wives~-- at least as a~diplomatic façade. Among commoners, on the other hand, there was no system of marital exchange between clans or families, and neither dowries nor brideprice. 

%\largerpage[-2] 
The method of data collection used in the 1960s survey bears the stamp of its historical context: a~time when the young
People’s Republic of China was taking stock of its new western possessions. Clearly, unconditional obedience to instructions was expected from the human subjects of the survey. It seems that this objective
was achieved: that respondents provided detailed statements about their family histories and sexual partnerships. 

The fact that all the data was eventually published, including the real names of those who had entrusted private information to the visiting ethnographers, is at odds with contemporary concerns about the protection of personal data, as outlined in anthropological
handbooks (e.g.~\citealt{fluehrlobban2014}, which reproduces the Code of Ethics of the
American Anthropological Association). On the other hand, the social structures described in the survey have undergone such profound transformations since then that the report constitutes an irreplaceable document. 
%\footnote{A consultant told me in 2008 that during the Cultural Revolution, cereal rations in Yongning were made conditional upon the possession of a~marriage certificate. Beyond this 	report, indicative of a~perception of a~historical watershed, an anthropologist would want to obtain fuller documentation, verifying this information with additional consultants and investigating how such policies were implemented across different villages and how the local society responded. “The Cultural Revolution” is sometimes used as a~cover term, since (as mentioned at the end of \sectref{sec:historicaloutline}) the ruling party allows some degree of criticism of the Cultural Revolution, unlike other episodes such as the Great Leap Forward \citep[172]{mazard2011}.} 
    
    To venture an analogy with the history of exploratory techniques in experimental phonetics: the results of the 1960s survey are akin to the X-ray data collected from the 1930s to the 1970s~-- a window of time opened by technical advances that made cineradiography possible, and closed by the realization that the high radiation doses involved posed serious health risks to those being filmed. This body of legacy data remains a~precious resource for the study of the world’s languages~\citep{fant1960,leroyetal1974,bothoreletal1986,bouarourouetal2008}.


\subsection[Marxist interpretation of the Na family structure]{Marxist interpretation: Na family structure as a~confirmation of Morgan’s theory}
\label{sec:marxistinterpretationnafamilystructureasaconfirmationofmorganstheory}
% \largerpage[-2] % xyz still relevant?

In the early 1980s, several of the researchers involved in the 1960s survey published books drawing on these
materials \citep{zhanetal1980,yanetal1984}, before the release of the original report. These authors adopted an evolutionary perspective, leading them to the conclusion that Na society represented a “living fossil” 
(\textit{huóhuà shí} \zh{活化石}): a~remnant of a~matriarcal society predating patriarchy, providing decisive proof of the historical reality of Lewis Henry Morgan’s theory (1877), as embraced by Marx. According to this theory, family structure evolved from the consanguine family, through the
matrilineal clan, to the patrilineal nuclear family. 

From the standpoint of anthropological theory, Na family structure was thus treated as confirmation of a preconceived framework~-- ironically, one that itself well deserved the label of “living fossil”. Morgan's theory had long been discredited in the West and only survived in China by the power of dogma. As a~result, the new data, rather than contributing to theoretical innovation, merely served to reinforce inherited assumptions. 

This may be likened to the interpretations once given of sunspots (dark patches on
the surface of the Sun): in the 9\textsuperscript{th} century AD, they were interpreted as planetary transits
obscuring part of the Sun \citep[93]{wilson1917}; in the 17\textsuperscript{th} century, they were taken as evidence
of the sun’s decay, confirming the prevailing pessimistic view of the world’s gradual decadence, as expressed in the works of Walter Raleigh and Thomas Browne. “Scientific evidence can only answer the questions that scientists think fit to ask” \citep[21]{hampson1968}.

To preview a~topic taken up below, the portrayal of Na society as a~“living fossil”,
popularized through tourism promotion for the Na area, created no small amount of
resentment among the Na (as reported, for instance, by \citealt[132]{shih2010}).


\subsection[Cai Hua’s \textit{A society without fathers or husbands}]{Bringing Na family structure to the attention of Western anthropologists: Cai Hua’s \textit{A society without fathers or husbands}}
\label{sec:bringingnafamilystructuretotheattentionofwesternanthropologistscaihuas19972001asocietywithoutfathersorhusbands}
\nocite{cai1997,cai2001}

While Na society was straightforwardly pigeonholed within one of the evolutionary stages postulated by
Marxist-Leninist anthropology (namely matriarchy), it did not fit into postwar
Western models of kinship. A~glance at the first pages of Murdock’s classic
study of family structure (\citeyear{murdock1949}: 1--3) suffices to show that the Na family falls outside his typology. Murdock’s takes the “first and most basic” form of family
organization to be the nuclear family (“a married man and woman with their offspring”), with the other two recognized types being “combinations” thereof: the polygamous
family “consists of two or more nuclear families affiliated by plural marriages”, while the extended
family “consists of two or more nuclear families affiliated through an extension of the parent-child
relationship rather than of the husband-wife relationship, i.e., by joining the nuclear family of
a~married adult to that of his parents” (p. 2). One can readily imagine the excitement with which
a~researcher working in Western anthropological circles might pursue the theoretical implications of
the observations made in Yongning, which contradict two of Murdock’s assumptions: the universality of marriage, and the universality of the nuclear family.

Such was the perspective adopted by 
Cai Hua \zh{蔡华}, a~Yunnan-born anthropologist who completed
a~Ph.D.\ at the \textit{École des hautes études en sciences sociales} in Paris, under the supervision of Kristofer Schipper,
Françoise Héritier and Olivier Herrenschmidt. His book \textit{Une société sans père ni mari: les Na de Chine} (A society without fathers or husbands: the Na of China) \citep{cai1997}
was the first to present a~study of Na family structure to a~non-Chinese-reading audience. 
%The title itself signals the author’s angle: presenting the Na as a~{counterexample} to generalizations previously taken as well established, %in a book that brought to the attention of the international anthropological community a~social structure 
%challenging key assumptions in kinship theory. 
The title signals the author’s angle: bringing to international anthropological attention a~social structure that undermines foundational assumptions of kinship theory. 
Its diffusion was
facilitated by an {English} translation \citep{cai2001}. As an illustration of the enthusiastic reception that greeted this heady blend of theoretical import and ethnographic colour, an article in the \textit{New York Review of
	Books} \citep{geertz2001} highlights the challenge posed to Lévi-Strauss’s views on kinship
\citep{levistrauss1949}, while also providing a~racy summary of the more titillating aspects of the ethnography:

% \largerpage[-2] % xyz still relevant? 
\begin{quotation}
	Sexual intercourse takes place between casual, opportunistic lovers, who develop no broader, more
	enduring relations to one another. The man “visits,” usually furtively, the woman at her home in
	the middle of the night as impulse and opportunity appear, which they do with great
	regularity. Almost everyone of either sex has multiple partners, serially or simultaneously;
	simultaneously usually two or three, serially as many as a \hfill hundred 
	%\newpage\noindent
	or two. There are no nuclear
	families, no in-laws, no stepchildren.~\citep{geertz2001}\footnote{The same two aspects~--
		scientific significance and sexual fascination~-- recur in reviews of the book. Here is another
		example: “le propos a~de quoi mettre sens dessus dessous la théorie anthropologique qui fait
		reposer le principe même des sociétés humaines sur l'alliance de mariage. Mais la lecture de ce
		livre à la fois savant et ingénu est aussi recommandée à ceux que ce problème laisse froids~: le
		tableau des mœurs libertines des Na est digne des plus joyeux fantasmes qui circulaient en
		Europe dans les années 70”~\citep{journet1998}.}
\end{quotation}

% \largerpage[-2] % xyz still relevant?
From the vantage point of a~Western audience, an additional asset of the study is that its author is Chinese. The workings of a~society “without fathers or husbands” appear all the more fascinating when narrated by an anthropologist from a~patriarchal, marriage-centred culture in which Na society has often been disparaged as “backward”. In examining Na social organization, Cai
Hua is careful to distance himself from predecessors such as Yan Ruxian, explicitly criticizing the evolutionary assumptions in their writings. This earned him praise from commentators who emphasized the author's scientific achievement of freeing himself from two potential sources of bias~-- Marxist ideology and Chinese cultural preconceptions concerning kinship and sexuality.
%that might have hampered his analysis, setting aside both Marxist ideology and culturally specific Chinese views on kinship and sexuality. 
The tone of such assessments was not devoid of condescension. One 
%the somewhat patronizing praise of
% colleagues who emphasize the author’s scientific achievement: freeing himself from two
% preconceptions that could have biased his research, namely Marxist ideology and Chinese
% prejudice against forms of kinship and sexuality that are remote from Chinese
% culture (e.g.\ \citealt[57--58]{cartieretal1998}).
reviewer, for instance, highlighted the difficulty for a “young Yunnanese researcher” of rising above preconceptions, and credited his stay in Paris for enabling this achievement \citep[57--58]{cartieretal1998}.\footnote{“To narrate the transformations imposed on Na society is, in itself, a notable achievement on the part of a Han ethnologist~-- especially if he manages to take some distance from Marxist ideology and from the specifically Chinese prejudices surrounding kinship and sexuality systems that differ sharply from his own culture. Particular credit is therefore due to Cai Hua, a young Yunnanese scholar who came to Paris to further his training, for having succeeded in this demanding task by, as it were, wiping the slate clean of most of the prejudices that could have hindered his research.”
%“Narrating the transformations imposed on Na society was in itself a~highly laudable contribution on the part of a~Han ethnologist, especially if he was able to distance himself from Marxist ideology and from specifically Chinese prejudice relating to patterns of kinship and sexuality far removed from his own culture. Cai Hua, a young Yunnanese researcher who came over to Paris to complement his training, therefore deserves to be congratulated for having succeeded in this difficult task by literally wiping the slate clean of most of the prejudices that could have hampered his research.” 
\textit{Original text:} Relater ({\dots}) les transformations
	imposées à la société Na représentait en soi une contribution fort honorable de la part d’un
	ethnologue Han, surtout s’il se montrait capable de prendre ses distances par rapport à
	l’idéologie marxiste et aux préjugés proprement chinois relatifs à des formes de parenté et de
	sexualité très éloignées de sa culture. On saura donc tout particulièrement gré à Cai Hua, un
	jeune chercheur yunnanais venu à Paris compléter sa formation ({\dots}), d’avoir réussi ce difficile
	exercice en faisant littéralement table rase de la plupart des préjugés qui auraient pu
	handicaper sa recherche.} 
    
    The book gained international visibility, and elicited a~response from Claude Lévi-Strauss~\citep{levistrauss2004}.

The less positive side of this success story lies in the fact that, in his eagerness to highlight the originality of Na
society and its challenge to kinship theory, Cai Hua sometimes overstates the case. A~reader with access to the Chinese literature reviewed his
argument in detail and concluded that “in setting out to make certain points, Cai picks his
unreferred cases rather selectively and ignores the cases which do not fit his argument”
\citep{wellens2003}. The book is selective in its presentation of the data, in order to
dramatize the uniqueness of this society, presented as “the ‘other’ of the Han Chinese: a~society free of the constrictions of Confucian morality” \citep[147]{wellens2003}. 

Cai Hua became a~professor in anthropology at Peking University, where he continued to specialize in kinship studies. However, he did not publish a~Chinese version of his book. One possible reason for this may be that he was
aware the work~-- initially written for a~Western audience~-- would encounter greater scrutiny from scholars familiar with the relevant ethnographic sources and, increasingly, with the linguistic evidence. To characterize Yongning Na society as “a society without fathers or husbands” is, to say the least, an overstatement. The concept of ‘father’ is not absent from the language: the
word \textit{adda} /\ipa{ə˧dɑ˥\$}/ %\footnote{The combination \ipa{˥\$} in this word's transcription refers to one of the lexical tone categories of Yongning Na: see Chapter 2, \sectref{sec:wordfinalandmorphologicalnucleusfinalHtones}.} 
unambiguously means ‘father’. If the aim is to frame the social pattern in contrast to mainstream Chinese marriage customs, a~more accurate description 
% proposed by He
% Xueguang \zh{和学光} (p.c.\ 2008)  %% in fact: is used in earlier publications, such as the 1966 film about Yongning society.
is \textit{bù} \textit{qǔ} \textit{bú} \textit{jià} 
\zh{不娶不嫁}: men do not
\textit{take} a~wife into their own family (\textit{qǔ} \zh{娶}), and women do not \textit{leave} their natal family to join their partner’s (\textit{jià} \zh{嫁}).

In his subsequent publications in foreign languages, Cai Hua continued to lend a~prominent place to Na society within comparative kinship studies. In a~book published in {French} \citep{cai2008}, he discusses four types of family structure: Chinese, Na, {French}, and Samo
(Burkina Faso), the latter having been studied by Cai's Ph.D.\ advisor Françoise Héritier. These four are
arranged into a~system of binary oppositions: the Chinese and Na family structures are classified as monolateral, whereas those of the French and Samo are described as bilateral; the Chinese as masculine and Na as feminine. The French family structure is described as bilateral-symmetrical, and the Samo family as bilateral-asymmetrical. The publisher’s blurb emphasizes that the book proposes “new
epistemological approaches that call into {question} a~certain Western rationalism and would, apparently, also be useful to other human and social sciences”.\footnote{\textit{Original text:} {\dots}~de nouvelles propositions épistémologiques
	qui remettent en {question} un certain rationalisme occidental et seraient utiles, selon toute
	apparence, aux autres sciences humaines et sociales.} The contrast between the author’s
anthropological publications in Chinese and in {French} offers a~telling illustration of the
enduring divide between ‘Western’ and ‘Chinese’ academic traditions~-- the former apparently encouraging
epistemological boldness, albeit at times at the expense of typological breadth and empirical precision.


\subsection[Studies of Na society in comparative perspective]{Beyond the initial scoop: Studies of Na society in comparative perspective}
\label{sec:shih19932010andweng1993}

The initial scoop~-- encountering a~society with an uncommon family structure~-- opens onto a~wealth of issues for anthropologists to explore. Two important Ph.D.\ dissertations about the Na were completed in 1993: those of Shih Chuan-kang (\citeyear{shih1993}) and Weng Naiqun (\citeyear{weng1993}). These were followed in \citeyear{chao1995} by Emily Chao’s dissertation, which has a~stronger focus on the \ili{Naxi}. A~Chinese translation of Shih’s dissertation was published in \citeyear{shih2008}, and
an expanded {English} edition, incorporating additional fieldwork results, appeared in \citeyear{shih2010}. These
studies provide in-depth analyses of Na society, based on new ethnographic data.

A~critical note may be ventured regarding certain aspects of Shih Chuan-kang's publications: his conclusions on linguistic
matters are at times hasty, as when he accepts the {folk etymology} of the place name ‘Yongning’ (discussed in Chapter 1, \sectref{sec:presentationofthenalanguageandnasocietyandreviewofearlierstudies}). Likewise, there are reasons for caution regarder Shih’s interpretation of the \isi{exonym} ‘Mosuo’ used in
Chinese historical sources: 

%\Hack{\newpage}
\begin{quotation}
	In the summer of 2001, I made another field trip to Yongning under the auspices of the National Science Foundation. While being jolted around in a~Mitsubishi SUV on the way from Lijiang to Yongning, I was ruminating yet again over the candidate words for which the term \textit{Mosuo} and its variants might have been transliterated. When I was mulling over the phrase \textit{mosi}, the legend about the {English} word \textit{kangaroo} suddenly occurred to me.
	
	In the 1770s, the story goes, when Captain Cook and his explorers in Australia saw a~large quadruped hopping animal they had never seen in Europe, they asked: “What is the name of this animal?” “Kangaroo,” the aborigines replied. The British assumed this must be the name of the animal and introduced the word into the {English} vocabulary as such. It turned out, according to the legend, that the word was not the name of the animal. Rather it meant “I don't understand.”
	
	Inspired by this legend, I wondered how I could have missed the point for so long. In both the {Naxi} and Naru [i.e.\ Na] languages, \textit{mosi} means “not know,” which can be used as an independent phrase to answer a~{question}. The pronunciation of this phrase is identical in both languages. I had asked this phrase in the field countless times but never thought it was the answer to my long-standing {question}.
	
	Neither the historical accuracy of the kangaroo legend nor the exact meaning of the word \textit{kangaroo} in the aboriginal language bear any direct relevance to the origin of the word in {question}. Rather, the significance of this legend is that it vividly depicts a~conceivable scenario in which cultural and linguistic misunderstandings could arise during the initial contact of different cultures. It is not difficult to envision another such scenario: One of the first Chinese speakers to get in touch with the group under consideration asked: “Who are you?” Responding to a~language that he did not understand, the person said: “\textit{Mosi},” meaning “I don't know (what you are talking about).” The Chinese speaker just took it as an answer to his {question} and recorded or repeated this “name of the people” in the closest sounds in his own language.~\citep[25--26]{shih2010}
\end{quotation}

%Command \noindent added to avoid having an indent. Proofreader suggestion: since this sentence continues the argument, it is better not to indent. 
{\noindent}“Kangaroo legend” is indeed an apt label. ‘Kangaroo' does not mean ‘I don't know'; it refers to a~species of kangaroo in \ili{Guugu Yimidhirr}, a~language of the \ili{Pama-Nyungan} family. To relate early Chinese references to Naish peoples to /\ipa{mɤ˧-sɯ˥}/ ‘[I] don’t know’ is to construct a~new legend in cheerful disregard of linguistic method. Shih compares present-day \il{Mandarin!Southwestern}Southwestern {Mandarin} pronunciations
with present-day Na \citep[26--27]{shih2010}, but etymological investigations at such historical depth require recourse to \is{comparative method (historical linguistics)}reconstructed forms. The earliest Chinese term, \zh{摩沙}, goes back to the Jin dynasty
(265--420\,AD). Reconstructions of \ili{Old Chinese} suggest that the character \zh{沙} may have had a pronunciation close to
*\ipa{sræ} or *\ipa{sræj} \citep{baxter2000}. This does not match up well with the proposed proto-{Naish} form of ‘to know’, which is \is{comparative method (historical linguistics)}reconstructed as *\ipa{si} \citep{jacquesetal2011} or *\ipa{sɯ9} (\citealt[230]{li_yuanshi_2021}), i.e.\ with a~high vowel, unlike in the form reconstructed for Chinese.\footnote{In the reconstructions proposed by \citet{li_yuanshi_2021}, such as *\ipa{sɯ9} for `to know', the numeral (here `9') refers to the tonal correspondence.}

To the linguist, these small lapses in the linguistic dimension of an anthropologist’s work
serve as a~cautionary example: it is essential to avoid oversimplification when stepping outside one's field of expertise. Just as imprecision in linguistic analyses may cast the shadow of a~doubt over an anthropologist's broader conclusions, linguists too run the risk of overlooking key insights if they adopt reductive views of social phenomena, or neglect the inherently social nature of language.

% \largerpage[-2]%longdistance xyz still relevant?
On historical topics, a~salient feature of Shih’s study is his insistence that the \ili{Naxi} and the Na constitute two distinct peoples. He proposes separate ancestries for the two groups, tracing
one back to the tribes referred to as \textit{Máoniúzhǒng} 
\zh{牦牛种} in Han-dynasty \il{Sinitic}Chinese chronicles, and the other to the 
\textit{Rǎnmáng}
\zh{冉駹} of the same sources. Attempts to identify present-day ethnic minorities with the names
given to “barbarian” tribes in early Chinese texts remain the subject of ongoing debate in Chinese
scholarship. These identifications are, however, highly speculative \citep{gros2014b}. Shih’s assertion that “patrilineal descent has been the norm for thousands of years” among the forebears of the {Naxi} is not supported by convincing evidence. One may form the impression that the author~-- who expresses sustained
sympathy for Na society throughout his work, as reflected in the 2010 title \textit{Quest} \textit{for}
\textit{Harmony}~-- adopts the perspective of his Na interlocutors in asserting a~clear distinction between themselves and the Naxi, reifies this perceived difference as a~binary opposition between Na and {Naxi} as ethnic categories,
and projects this dichotomy into the indefinite past.

\begin{quotation}
	Shih is so preoccupied with establishing the exceptional nature of the Moso case that he rather too hastily dismisses the comparative relevance of analogous practices observed in certain neighbouring regions along the Sino-Tibetan frontier. Yet one of the merits of his book lies precisely in the clarity of its ethnographic account, which offers a~solid foundation for regional comparisons and contributes more broadly to
	the anthropological debate on kinship. \citep{gros2011} 
        
    \medskip 
    %\footnote{
    {\noindent} \textit{Original text:} préoccupé d’établir l’exceptionnalité du cas Moso, l’auteur écarte un peu trop rapidement l’intérêt comparatif de cas assez similaires relevés dans certaines régions voisines de la bordure sino-tibétaine. C’est pourtant un des mérites de son ouvrage que de nous livrer une ethnographie claire à même de servir à une entreprise comparative régionale, comme d’alimenter le débat au sein de l’anthropologie de la parenté plus généralement.
    %}
\end{quotation}

% \largerpage[-2] % xyz still relevant?
An unfortunate consequence of Shih Chuan-kang’s entrenched belief in the great historical depth of
the Na-vs.-{Naxi} divide is that it leads him to dismiss the work of researchers who
hypothesize a~common ancestry for the two groups and who offer a historical account of their gradual divergence. An extreme version of this hypothesis is put forward by
\citet[33–46]{jackson1979}, who notes “strong resemblances with regard to their kinship
patterns in particular”, and suggests that the main differences between {Naxi} and Na societies only
go back about three centuries. In his view, these differences stem primarily from the deep Sinicization
of {Naxi} culture since the 18\textsuperscript{th} century. Shih
vigorously rejects Jackson’s theses, and the \citeyear{shih2010} edition of his book makes no mention of Christine Mathieu’s (\citeyear{mathieu2003}) study, \textit{A history and anthropological study of the ancient kingdoms of the Sino-Tibetan borderland~-- {Naxi} and Mosuo},
which explicitly seeks to explore the historical relationship between the Na and the Naxi.

It must be acknowledged that 
%Jackson’s study calls for a~thorough revision in light of more recent
%documentation. The volume of available material has increased dramatically since the time of his research, which goes a~long way towards explaining his occasional inaccuracies, such as
%the interpretation of the \ili{Naxi} name of the Na, /\ipa{ly˧-çi˧}/ (romanized as \textit{Lü-khi}), as
%“the people of Lü, the Chinese name for the area” (p. 36), when it actually means ‘the people of the
%Centre’, and is an exact parallel (cognate) to the Na \isi{endonym} /\ipa{ɬi˧-hĩ˧}/.\footnote{For the sake of simplicity, this noun is provided here in surface phonological transcription. Its underlying form is //\ipa{ɬi˧-hĩ\#˥}//, with a~{floating} High tone. This tonal category is analyzed in Chapter 2, \sectref{sec:afloatinghtonewithcomparativeevidencepointingtoitsorigin}.} Also, 
some of Jackson’s
formulations are deliberately provocative. He favours broad-stroke sketches of {Naxi} history, highlighting turning points such as “1723 A.D. when the Mu family was ignominiously dismissed and the area was ‘nationalized’ by the Chinese” (p. 35). In that year, Lijiang was
placed under direct imperial rule, and the Mu \zh{木} family of feudal chieftains, who had ruled the area since the Yuan dynasty, ceased to exercise real power. This was undoubtedly a~major landmark in {Naxi}
history. However, it is worth emphasizing that the deliberate introduction of Chinese culture and
Confucian ideology had begun much earlier: the Mu rulers’ unswerving allegiance to the Chinese state dates back to the inception of their rule in the 14\textsuperscript{th} century. Viewed in this light, the formal incorporation of Lijiang into the Chinese administrative system in the 18\textsuperscript{th} century was not a~sudden break with the past, but the consequence of a~course set in motion by the Mu rulers centuries earlier.

% is not without links to decisions that were made by the ruling family several centuries earlier.

\begin{quotation}
	The Mu paid tribute to the imperial court and guarded the frontier on behalf of the Chinese emperors. To develop their realm, they pacified, conscripted, and taxed the local tribes (against fierce resistance), and they also called on large numbers of Chinese migrants from the interior~– peasants, artists, craftsmen, literati, Taoist and Chinese Buddhist adepts~– who worked on their estates, joined their armies, populated garrisoned villages and towns in tribal territories, and assimilated into the {Naxi} population. The Mu kings prided themselves on their civilization, in other words: their Sinicization. They were soldiers, and they became scholars, poets and calligraphists.  They built palaces in Chinese style; they also built Confucian, Taoist and Buddhist temples, and dedicated arches to the chastity of their wives in Confucian fashion. \citep[359-360]{mathieu2015}
\end{quotation}

% \largerpage[-2] % xyz still relevant?
A well-documented typological parallel for the emergence of cultural differences through Sinicization is the case of Vietnam: intensive Sinicization over the course of the first millenium \,AD led to differences in mentality between the Vietnamese~-- speakers of an \ili{Austroasiatic} language deeply influenced by Chinese~-- and their \ili{Austroasiatic} neighbours.\footnote{Haudricourt draws a~parallel with the Germanization of the Czechs, a~Western Slavic group. Readers unfamiliar with Haudricourt's style should be aware that he had a~taste for thought-provoking formulations. “The Vietnamese are what they are because, fundamentally, they are culturally Chinese. This is exactly like the Czechs: they speak a~\ili{Slavic} language, but their civilization is German. Literary \ili{German} was the Prague variety of German as used by the imperial administration of the House of Luxembourg. This explains why there have always been insoluble national problems between the Czechs and the Slovaks. The Vietnamese have roughly the same history: they have assimilated enough Chinese culture to become unclassifiable in the eyes of their neighbours” \citep[97-98]{haudricourtetal1987}. \textit{Original text:} Les Vietnamiens sont ce qu'ils sont parce qu'en fait ils sont chinois. C'est exactement comme les Tchèques, Marcel Mauss m'avait fait remarquer avant la guerre que les Tchèques parlent une langue slave mais qu'ils ont une civilisation allemande. L'allemand littéraire c'était l'allemand de Prague utilisé par l'administration impériale de la dynastie des Luxembourg. Ce qui explique qu'il y ait toujours eu des problèmes nationaux insolubles entre les Tchèques et les Slovaques. Les Vietnamiens ont à peu près la même histoire, ils ont assimilé assez de civilisation chinoise pour se rendre inclassables aux yeux de leurs voisins.} 
Again adopting the linguist’s (admittedly narrow) perspective, there appears to be evidence supporting the view of a~gradual divergence between the Na and the Naxi. A~close examination of kinship terms in \ili{Naxi} suggests that words for paternal relatives are mostly borrowings or recent coinages, as they are not \is{comparative method (historical linguistics)}cognate across dialects. The same holds true of terms relating to marriage, such as ‘husband’, ‘wife’, and ‘daughter-in-law’. By contrast, terms for maternal relatives appear to be of greater antiquity, with cognates in Na and {Laze}. Viewed in this light, the hypothesis of a~divergence in family structure linked to the {Naxi}'s progressive exposure to Chinese (Confucian) influence deserves serious consideration. Jackson’s formulation (p. 37) is: “This is the missing key to the
confusion on Nakhi kinship: legal patrilinearity yet traditional matrilinearity”.


% \largerpage[-2] % xyz still relevant?
Last but not least in this review of studies of Na kinship and family structure is a~two-volume set of collected works in Chinese: \citet{latami2006}. The collection is organized by themes (volume I: ethnology and anthropology; volume II: language, customs, religion, culture, music, and book reviews), and spans the period from 1960 to 2005. Some minor limitations should be noted, such as incomplete information on the original publication details of reprinted materials, and typographical issues affecting Latin-script passages and International Phonetic Alphabet notation.


\subsection[Present-day sociological studies]{Present-day sociological studies: The impact of tourism since the 1990s}
\label{sec:presentdaysociologicalstudiestheimpactoftourismsincethe1990s}

Since the 1990s, tourism has developed at a~staggering pace in the Yongning Na area. A~number of books, both in Chinese and in Western languages, cater to the tourist industry by presenting
idealized depictions of Na society set against its picturesque backdrop: Lugu Lake and the Yongning
plain~\citep[e.g.][]{refflet2006,lamu1998}. There also exists a~substantial body of scholarly literature addressing the effects of tourism on Na society. Particularly striking contrasts could be observed in the
2010s between rituals and songs as practised among villagers in Lhirezzee (Lijiazui \zh{利家嘴})~-- a~settlement located far from the main tourist areas~-- and the performances staged for tourists on the shores of Lugu Lake \citep{milan2013}. Anthropologists have documented in-depth transformations linked to tourism, such as decreasing reliance on matrilineal kin as material wealth increases.

\begin{quotation}
	{\dots}~increased individual access to resources is associated with	diminished importance of the kinship group in organising behaviour, a~shift away from matrilineal inheritance and {erosion} of the non-conjugal visiting system. ({\dots}) Respondents in tourist-impacted areas showed more deviation from matrilineal ideology in terms of household composition and preference for marriage, trends that seem to be associated more with wealth than with cultural {assimilation}. \citep[171]{mattison2010}
\end{quotation}

Foreign sociologists and anthropologists often adopt a~critical perspective,
observing that “official representations of China’s ethnic minorities have created an image of
minority people as dangerous, feminine, and erotic”, and that in the case of the
Na/{\allowbreak}Mosuo, “early state categorizations of Mosuo gender practices have led to
representations of Mosuo ethnicity built around notions of women freely available for sex, to whom
present lovers have no future commitments, or of a~land where women rule. Matriarchy and sexual
availability are central in tourists’ desire to visit the Mosuo” (\citealt[449–450]{walsh2005}; see also \citealt{schein1997, blumenfield2010}). These authors bring to light an ironic reversal: “the cultural characteristics the Maoist government tried to change became celebrated as markers of Mosuo cultural uniqueness and value”
\citep[457]{walsh2005}.

The detailed analyses offered by Stéphane \citet{gros2001} in his study of the
Drung (Dulong) ethnic group also apply more broadly to other groups in Yunnan and throughout China: representations of
“minority” identities are constructed to align with national objectives. Assimilationist policies,
which manifested with particular intensity during the Great Proletarian Cultural Revolution (1966--1976), have resulted in dual narratives of pre-Liberation and post-Liberation society. Relative toleration during the 1980s, China’s “Reforms and Opening Up” decade \citep{zhu2014}, led to the recognition of acceptable cultural features that need not be eliminated along with “bad” inheritances from the past~-- notably, \textit{religion} (\textit{zōngjiào} \zh{宗教}) as (precariously) distinguished from \textit{superstition} (\textit{míxìn} \zh{迷信}).

After 1989, the {conservative} backlash was accompanied by a process of folklorization:
officially defined ethnic minorities were presented through timeless, monolithic representations, commodified to serve the tourist industry and contribute to the country's GDP. A~constant across these periods is the deployment of “minority” imagery as a~foil for asserting the homogeneity and modernity of
the Han “majority” \citep[31]{gros2001}. Another constant is sustained pressure towards {assimilation}.

\begin{quotation}
	Today, on the mere surface, “leisure culture” represents market reasoning rather than a~statist logic. This is, after all, what hegemony is all about: naturalization of ruling technologies. \citep[242]{sigley2013} 
\end{quotation}

For readily understandable reasons, scholars with institutional or family ties to mainland China tend to avoid such lines of criticism. A~substantial body of literature instead focuses on how to achieve a~reasonable balance between the competing imperatives of
economic development and “cultural preservation” 
(\textit{wénhuà bǎohù} \zh{文化保护}). This perspective is in line with the national mottos of promoting economic growth while maintaining social harmony. Useful overviews are provided by \citet{knodel1995} and \citet{he2008}. The following English-language excerpt is indicative of this approach:

\begin{quotation}
	With its unique natural landscape of a~plateau
	lake and matriarchal culture, Lugu Lake region has
	recently become an attractive destination for tourists
	and researchers. Although the present environmental
	conditions in Lugu Lake region are good,
	rapid economic and tourism development in
	recent years has impacted on the regional environment
	and the traditional Mosuo culture. ({\dots}) Unlike the attitude
	of other races to nature, the Mosuo’s attitudes
	towards nature greatly benefit environmental
	protection. \citep[49-51]{yanetal2008}
\end{quotation}

%Command \noindent added to avoid having an indent. Proofreader suggestion: since this sentence continues the argument, it is better not to indent. 
{\noindent}This article culminates in a~series of policy recommendations, the first of which reads: “Government should play a~more positive role in conservation of traditional Mosuo culture and the local environment through increased investment” \citep[54]{yanetal2008}.

On a~less predictable note, the \citeyear{lamu2007} collection of articles by Mosuo scholar and poet Lamu Gatusa \zh{拉木·嘎吐萨} preserves (in novelized form) a~number of striking pieces of local history. Publications by Mosuo anthropologist Latami Wangyong (Latami Dashi) \zh{拉他咪王勇(拉他咪达石){\kern-4pt}} address a~broad range of topics and locales, including Mosuo villages on the Ninglang plain \citep{latami2009, latami2016}. 
