\chapter{Introduction}
\label{chap:introduction}
\label{chap:1}

%\epigraph{The most important task in linguistics today~-- indeed, the only really important task~-- is to get out into the field and describe languages, while this can still be done.}{\citep[144]{dixon1997}}

The aim of this book is to provide a~description and analysis of the tone system of 
Yongning Na (also known as Mosuo), a~Sino-Tibetan language spoken in Southwest China.

The richness of this system becomes apparent even from a~single sentence. Example (\ref{ex:ihavetogoandtakemyluggagenow}) is the 
first one that I transcribed. At the time, I had just arrived at my future teacher’s house; my luggage had been left at another house along the main road, about fifty metres away. I asked my teacher’s son, who speaks fluent Mandarin, to translate an explanation for me: “I have brought 
a~lot of stuff; I have to go back [to the main road] and pick it up now.” This yielded (\ref{ex:ihavetogoandtakemyluggagenow}). Later, I elicited (\ref{ex:ihavetogoimafraidihavetoleave}) as a~simpler form. 

\begin{exe}
  \ex \label{1}
  \begin{xlist}
    \ex
    \label{ex:ihavetogoandtakemyluggagenow}\label{1a}
    \gll njɤ˧	ʑi˩	bi˩	-zo˩	-ho˥.\\
    \textsc{1sg}	to\_take	to\_go	\textsc{obligative}	\textsc{desiderative}\\
    \glt ‘I have to go and take [my luggage] now.' (Field notes, 2006)

    \ex
    \label{ex:ihavetogoimafraidihavetoleave}\label{1b}
    \gll	njɤ˧	bi˧	-zo˧	-ho˩.\\
    \textsc{1sg}	to\_go	\textsc{obligative}	\textsc{desiderative}\\
    \glt ‘I have to go. / I’m afraid I have to leave.' (Field notes, 2006)
  \end{xlist}
\end{exe}

The contrast in lexical tone on the main verb (/\ipa{ʑi˩}/ ‘to take’ in \ref{ex:ihavetogoandtakemyluggagenow}, 
versus /\ipa{bi˧}/ ‘to go’ in \ref{ex:ihavetogoimafraidihavetoleave}) affects the tones of the following syllables, 
extending all the way to the end of the sentence. %% Clarified “difference in the lexical tone” to “contrast in lexical tone” for conciseness.

This book offers an analysis of the underlying system, encompassing lexical tone categories, phonological rules that operate within tone groups, and morphotonological rules that govern combinations within various types of phrases. 
As a~preview of the results concerning lexical tones, examples 
(\ref{ex:ihavetogoandtakemyluggagenow}--\ref{ex:ihavetogoimafraidihavetoleave}) are provided below
(as \ref{ex:ihavetogoandtakemyluggagenow2}--\ref{ex:ihavetogoimafraidihavetoleave2}) with
morpheme-level transcriptions. These transcriptions indicate lexical tones using tone symbols supplemented by subscript letters \textsubscript{a}, \textsubscript{b}, and \textsubscript{c}, which differentiate subcategories of lexical tones. 
The following chapters examine the system in depth, detailing how the tone sequences of entire sentences obtain from the lexical tones. 

\begin{exe}
\ex
\begin{xlist}
\ex
\label{ex:ihavetogoandtakemyluggagenow2}
\ipaex{njɤ˧ ʑi˩ bi˩-zo˩-ho˥.}\\
\gll njɤ˩ 	ʑi˩\textsubscript{a}		bi˧\textsubscript{c}	-zo˧\textsubscript{a}		-ho˩\\
\textsc{1sg}	to\_take		to\_go	\textsc{obligative}	\textsc{desiderative}\\
\glt ‘I have to go and take [my luggage] now.' 

\ex
\label{ex:ihavetogoimafraidihavetoleave2}
\ipaex{njɤ˧ bi˧-zo˧-ho˩.}\\
\gll njɤ˩ 	bi˧\textsubscript{c}	-zo˧\textsubscript{a}		-ho˩\\
\textsc{1sg}	to\_go	\textsc{obligative}	\textsc{desiderative}\\
\glt ‘I have to go. / I’m afraid I have to leave.'
\end{xlist}
\end{exe}

To set the stage for the analyses developed in this volume, the introduction provides essential background on the Na language and its study. 

Section \ref{sec:presentationofthenalanguageandnasocietyandreviewofearlierstudies} presents the language and its dialects, followed by a~review of previous research in Section~\ref{sec:previousstudiesofthenalanguage} (with complementary historical and ethnological perspectives in Appendix~\ref{chap:historyanthropologysociology}). The research programme behind the present study is set out in Section \ref{sec:chronologyofthestudyelicitationproceduresandonlinematerials}, while Section~\ref{sec:collaborationwithconsultants} introduces the
language consultants and the data elicitation methods. Section~\ref{sec:OnlineResources} then presents the online primary documentation on Yongning Na~-- comprising audio and video recordings with their accompanying philological apparatus~-- alongside the dictionary and other resources, all conceived as part of a~coherent Open Science environment. 

Finally, a~concise grammatical sketch in Section~\ref{sec:sketch} summarizes key linguistic properties, such as the structure of {noun} and {verb} phrases. 
%These general characteristics provide the background for the morphotonological analyses developed in the subsequent chapters.  %% Removed on proofs; the opening sentence already establishes the purpose of the section with sufficient clarity. Removing the final sentence improves the pacing and prevents an overly didactic tone. It shows confidence in the reader's ability to follow the structure without continual reminders.


\section{The Na language}
\label{sec:presentationofthenalanguageandnasocietyandreviewofearlierstudies}


\subsection{Endonym and exonyms}
\label{sec:endoexo}

Yongning Na is a~Sino-{Tibetan} language spoken in Southwest China, straddling the border between the provinces of Yunnan and Sichuan, at a~latitude of
27{\textdegree}50’~\textsc{n} and a~longitude of 100{\textdegree}41’~\textsc{e} (see Map~\ref{map:1-1}). Speakers refer to the language as /\ipa{nɑ˩ʐwɤ˥}/, meaning ‘Na language’. The structure of this {compound}, made up of a~{noun} and a~verb, is shown in (\ref{ex:nalang}).

\begin{exe}
	\ex
	\label{ex:nalang}
	\ipaex{nɑ˩ʐwɤ˥}\\
	\gll nɑ˩˧		\hspace{0.3cm}ʐwɤ˩\textsubscript{b}\\
	\isi{endonym}:~Na			\hspace{0.3cm}to\_speak\\
	\glt ‘the language of the Na’, i.e.\ ‘the Na language’
\end{exe}

\begin{mapfigure}[p!] 
	\caption{A sketch map of the Yongning area. \textit{Designed by Jérôme Picard. Sources: Geofabrik, ASTER GDEM (a product of METI and NASA), and OpenStreetMap.}}
%	\includegraphics[width=.8\textwidth]{figures/map/27mai.jpg}
%	\includegraphics{figures/map/PNG_CMYK.png} % for test only: rasterized image
% \includegraphics{figures/map/PNG_Grayscale.png} % for test only: rasterized image
\includegraphics[scale=.99]{figures/map/PDF_CMYK_1point4.pdf} % vector image for hardcopy
%\includegraphics{figures/map/PDF_Grayscale_1point4.pdf} % vector image for softcopy
%\includegraphics{figures/map/PDF_RGB_1point4.pdf} % vector image for online PDF

%	\includegraphics{figures/map/YongningMap.jpg}

	%	\includegraphics[width=.95\textwidth]{figures/ch1mapSN.pdf}
	\label{map:1-1}
\end{mapfigure}

The name ‘Yongning Na’ was introduced by Liberty \citet{lidz2006}, combining the people's \isi{endonym} with the name of the region where the language is spoken. This naming aligns with principles such as: (i)~“language names (like city names) are loanwords, not code-switches” \citep{haspelmath2017}, and (ii)~“language names may have a modifier-head structure”, whereby ‘Yongning Na’ conveys the intended meaning of ‘the Yongning variety of the Na language’.

Yongning \zh{永宁} is a plain near Lugu Lake \zh{泸沽湖}, a~lake covering approximately fifty square kilometres (see Map~\ref{map:1-1}). The lake fosters a~microclimate that makes the area suitable for agriculture despite the high altitude (about 2,800 metres above sea level).
% Map~\ref{map:1-1} locates this language on a~map of Asia showing the current geographical distribution of \il{Sino-Tibetan}Sino-Tibetan languages.\footnote{Issues of language classification are addressed in \sectref{sec:thepositionofnaandnaxiwithinsinotibetan}.} 



%\begin{mapfigure}[t]
%	\caption{The distribution of Sino-Tibetan languages. \textit{Source: Glottolog 2.7.}}
%	\includegraphics[width=\textwidth]{figures/map/Glottolog.jpg}
%	%	\includegraphics[width=.95\textwidth]{figures/ch1mapSN.pdf}
%	\label{map:1-1}
%\end{mapfigure}

%\hyperref[fig:map]{see map}

\begin{photofigure}[t]
	\caption{The Yongning plain seen from the west, with Gemu Mountain (in~Na: \ipa{kɤ˧mv̩˧˥}) in the background. The Lake is behind the pass on the right-hand side. Autumn 2006.}
	\includegraphics[width=\textwidth]{figures/YongningPlain08375.jpg}
\end{photofigure}

Ethnonymy reflects the high degree of ethnic, cultural and linguistic intricacy of the Sino-Tibetan borderlands \citep{gros_sino-tibetan_2023}. \tabref{tab:thenamesofthenaendonymsandexonyms} presents (i)~two endonyms, (ii)~the name by which the \ili{Naxi} \zh{纳西} (a~closely related ethnic group) refer to the Na, and (iii)~a~Chinese exonym found in various sources, under various avatars, for close to two thousand years, and which currently enjoys renewed favour for reasons discussed in Appendix~\ref{chap:historyanthropologysociology} (\sectref{sec:ethnicclass}). 

%%test
%\begin{figure}
%	% [t] to place at top; here: full page
%%	\includegraphics[width=\textwidth]{figures/map/ExportCarteMichaudjpg_14avril_ROGNEE_vuAM.jpg}
%	\includegraphics[width=\textwidth]{figures/map/mapPLACEHOLDER.jpg}
%	\caption{A sketch map of the Yongning area. \textit{Designed by Jérôme Picard. Sources: Geofabrik, ASTER GDEM (a product of METI and NASA) and OpenStreetMap.}}
%	\label{fig:map}
%\end{figure}


\begin{sidewaystable}[p]
	\caption{The names of the Na: endonyms and exonyms.}
	{\renewcommand{\arraystretch}{1.35}
		\begin{tabularx}{\textheight}{ Q l@{\hspace{5mm}} P{42mm} P{42mm} P{37mm} }
			\lsptoprule
			transcription & language	& romanized equivalents &	Chinese equivalents & meaning\\ \midrule
			\ipa{nɑ˩˧}	& Na	& Na \citep{cai1997,lidz2010} &  \textit{Nà} \zh{纳} \citep{yang2006} & ‘black’\\
			\ipa{ɬi˧-hĩ˧} & 	Na	& Hli-khin \citep{rock1963}, Hli-hing \citep{shih1993}	& \textit{Lǐxīn} \zh{里新}
			\citep[15]{shih2008} & ‘People of the Centre’\\
			\ipa{ly˧-çi˧}	& Naxi & 	Lü-khi \citep{rock1963}	& \textit{Lǚxī} \zh{吕西} \citep[8]{guoetal1994} & as above: ‘People of the Centre’\\
			\textit{origin not yet established}	& Chinese &  Moso
			\citep{cordier1908,nishida1985,shih1993,mckhann1998,luo2008}, Mo-So, Mosuo \citep{knodel1995} & \textit{Móshā} \zh{摩沙}, \textit{Mósuò}\footnote{The character \zh{些}, standardly read \textit{xiē}, is read \textit{suò} in this context, as also in other place names from the same period: see \citet[170n9]{pelliot1904deux}.} \zh{磨些}, \textit{Mósuò} \zh{麽些}, \textit{Mósuò} \zh{摩些}, \textit{Mósuō} \zh{摩娑}, \textit{Mòsuò} \zh{末些}, \textit{Móhuò} \zh{磨获}, \textit{Mòsuān} \zh{莫狻}, \textit{Mósuō} \zh{摩梭}	&
			\textit{not yet established}\\ \lspbottomrule
		\end{tabularx}}
		\label{tab:thenamesofthenaendonymsandexonyms}
	\end{sidewaystable}
	
{\largerpage} % Introduced on April 15th, 2025 to avoid an orphaned line, 'the Na coexist with the Pumi 普米, who call themselves ʈʰóŋmə ‘white people''.
The most likely interpretation of the \isi{endonym} \ipa{nɑ˩˧} is that it means ‘black’. Use of ethnonyms meaning ‘black’ or ‘white’ is widespread in the area. In Yongning, the Na coexist with the \ili{Pumi} \zh{普米}, who call themselves \ipa{ʈʰóŋmə} ‘white people’. 

%\begin{quotation}
%	In Southwest China, there is no noticeable difference between the characteristic skin colors or facial features of different ethnic groups; \isi{variation} within groups is at least as broad as \isi{variation} between groups.~({\dots}) Occasionally someone will comment that people of one ethnic group or another might be shorter or darker or have curlier hair than another, but these are not the important distinctions; the important distinctions are linguistic and cultural. So terms like “White Miao” or “Black \ili{Yi}” or “Red Lahu” do not refer to the phenotypic characteristics of their bodies, but rather to the things they wear. 
%\end{quotation}

\begin{quotation}
The designation \ipa{ʈʰóŋ} ‘white’ sets the {Pumi} apart from some surrounding ethnic groups whom they designate as \ipa{nʲæ̌} ‘black’: the \ipa{ɡoŋnʲæ̌} ‘{Nuòsū} (Yí) \zh{彝}’ (‘black skin’) and the \ipa{nʲæmə̌} ‘Na (Mósuō) \zh{摩梭}’ (‘black person’). \citep[2]{daudey2014}
\end{quotation}

%Command \noindent added to avoid having an indent. Proofreader suggestion: since this sentence continues the argument, it is better not to indent. 
{\noindent}Among the \ili{Yi} \zh{彝} (formerly known as ‘{Lolo}’), there is a~distinction between ‘black’ and ‘white’ castes. “The {Nasoid} groups are also known as Black {Lolo}, and the assimilated groups connected with them~-- either {Nasoid} groups who have become Sinicized, or others who have become assimilated to the {Nasoid} groups, often by capture or conquest~-- are called ‘white’ to denote the fact that they do not ‘fit’ in the {Nasoid} clan structure” \citep[53]{bradley1979}.

\begin{quotation}
In northeastern Yunnan and western Guizhou, the designation \textit{Nasu} (the black ones) refers to a~group of 
\ili{Yi} who were the overlords of a~series of feudal kingdoms between the 9\textsuperscript{th} and the 20\textsuperscript{th} centuries;
they were often contrasted to other, subordinate groups who referred to themselves as white.  In the Liangshan region of southwestern Sichuan, on the other hand, Black bones (\textit{Nuoho}, called Black \ili{Yi} in 
Chinese), and White bones (\textit{Quho}, called White \ili{Yi} in Chinese), refer to the aristocratic and commoner castes into which the society is divided~-- the term \textit{nuo}, or ‘Black' also means ‘heavy', ‘important', or ‘serious'. At the same time, the aristocratic caste is also associated with darker colored clothing ({\dots}).
%aristocratic women often dress entirely in black. 
%In this case, it appears that historically the color of the clothing is derived from
%the color name given to the people, rather than the other way around. 
What is most important here is to realize that the association of people and colors in this region has little or nothing to do with the imagined color of the people themselves, but rather is part of a~complex symbolic system that both reflects and is reflected in the styles and colors of people's clothing. \citep[102]{harrell2009}
\end{quotation}

The Na name for Yongning is \ipa{ɬi˧di˩}. \citet[23]{shih2010} interprets it as meaning ‘the peaceful land’, connecting it to the verb \ipa{ɬi˥} ‘to rest, to relax'. This \isi{folk etymology} fits nicely with the author’s celebration of Na society's ideals of harmony, reflected in the title of his work: \textit{Quest} \textit{for} \textit{harmony}:
\textit{the} \textit{Moso} \textit{traditions} \textit{of} \textit{sexual} \textit{union}
\textit{and} \textit{family} \textit{life}. However, phonetic correspondences with \ili{Naxi} do not support this interpretation. Instead, they indicate that the Na name for Yongning, \ipa{ɬi˧di˩}, means ‘the central land, the heartland’. The linguistic evidence is as follows.

Yongning is called \ipa{ly˧dy˩} in \ili{Naxi} \citep[201]{heetal2011}. This cannot be a~recent borrowing from Na, as Na does not have a~rounded close front vowel /\ipa{y}/. If the contemporary Na form were borrowed into \ili{Naxi}, it would be interpreted as \ipa{li˧di˩} by Naxi ears, with a~straightforward correspondence for /\ipa{d}/ and /\ipa{i}/ (shared by both languages) and a~reinterpretation of Na /\ipa{ɬ}/ as Naxi /\ipa{l}/, since Naxi does not have an unvoiced lateral. The presence of the vowel /\ipa{y}/ in \ili{Naxi} \ipa{ly˧dy˩} strongly suggests that the word is cognate with its Na counterpart. While it could be a~calque~-- a~root-for-root translation from Na to \ili{Naxi} by a~bilingual speaker who understood the word’s morphological structure~--, it is not a~phonetic \is{loanwords}loanword.

The second syllable of the noun is easily analyzed: it means ‘earth, place, land' (Na: \ipa{di˩˥}, \ili{Naxi}: \ipa{dy˩}), a~root commonly found in place names across both languages, where it serves as a~locative nominalizer \citep[559]{lidz2010}. The first syllable, however, is less straightforward. Based solely on Na data, it could have a~number of different interpretations. It might indeed be connected to the verb ‘to rest', \ipa{ɬi˥}, as in the \isi{folk etymology} of Yongning as ‘the land of rest, the peaceful land’ adopted by Shih Chuan-kang.\footnote{The main language consultant reports this {folk etymology} in the document \textit{Folk Etymology}, available online \pandoi{0004606}.
%	 This recording had not yet been transcribed and translated at the time of publication of the present volume.
} 
	 But it could equally be related to ‘moon', \ipa{ɬi˧} (as in the disyllable \ipa{ɬi˧mi˧} ‘moon'), to ‘ear', \ipa{ɬi˧} (as in \ipa{ɬi˧pi˩} ‘ear'), to ‘middle', \ipa{ɬi˧} (as in \ipa{ɬi˧gv̩˧}, 
     %% Footnote commented out on April 27th, 2025.
     %\footnote{For the sake of simplicity, this noun is provided here in surface phonological transcription. Its underlying form is //\ipa{ɬi˧gv̩\#˥}//, with a~{floating} High tone, analyzed in \sectref{sec:afloatinghtonewithcomparativeevidencepointingtoitsorigin}.} 
     ‘middle part'), or to ‘Bai' (an ethnic group), through truncation of the disyllable \ipa{ɬi˧bv̩˧} ‘Bai'. Any of these roots, combined with \ipa{di˩˥} ‘earth, place', would yield the form \ipa{ɬi˧di˩} by application of regular tone rules. 

Moreover, the search needs to be extended further in view of the existence of some tonal irregularities. Certain disyllabic words exhibit tone patterns that deviate from those predicted by synchronically {productive} rules (as discussed in \sectref{sec:exceptionalitems} and \sectref{sec:htoneroots}). This suggests that tonal constraints should be relaxed when investigating the etymology of \ipa{ɬi˧di˩}. Broadening the search to \ipa{ɬi} roots across all tonal categories reveals additional candidates: nouns such as ‘roebuck' (\ipa{ɬi˩}), ‘turnip' (in \ipa{ɬi˩bi˩}), ‘trousers' (in \ipa{ɬi˩qʰwɤ˩}), and ‘wrath, anger' (in \ipa{ɬi˩ʁɑ˩}), as well as verbs like \ipa{ɬi˩} ‘to measure' and \ipa{ɬi˧˥} ‘to dry in the sun'. Language-internal evidence thus allows for a~wide range of hypotheses: does \ipa{ɬi˧di˩} mean ‘the peaceful land, the land of rest', ‘the land of the moon', ‘the land of ears', ‘the land of the middle', ‘the land of the Bai people', ‘the land of the roebuck', ‘the land of turnips', ‘the land of trousers', ‘the land of wrath', ‘the land of measurements', or ‘the land of sun-drying'? 

Semantic plausibility alone cannot resolve this question. A~study of toponymy in various languages of China \citep{yangliquan2011} demonstrates its remarkable creativity, with meanings that are often surprising.

Decisive evidence comes from comparison with \ili{Naxi}. Among all the possibilities, only one is supported by the existence of a~cognate in \ili{Naxi}. The root \ipa{ly˥} in \ili{Naxi} means ‘centre’, as does \ipa{ɬi˧} in Na. This leads to the conclusion that \ipa{ɬi˧di˩} (and \ili{Naxi} \ipa{ly˧dy˩}) can be interpreted as ‘the central land, the heartland’. 

Crucially, this interpretation rests on linguistic evidence provided by the \is{comparative method (historical linguistics)}comparative method. By identifying cognates between Na and \ili{Naxi} words,
%~-- technically known as \textit{}~-- and examining their phonetic correspondences, 
a~definitive conclusion can be reached. While \ipa{ɬi˧di˩} can have more than ten different interpretations in Na, and \ipa{ly˧dy˩} in \ili{Naxi} can likewise be interpreted in a~range of different ways (such as ‘land of grain’, ‘land of Asian crabapple, \textit{Malus asiatica}’, ‘central land', ‘land of watchfulness', or ‘quaking land'), historical comparison %brings out cognate words between Na and \ili{Naxi} and 
points decisively to the ‘central land' hypothesis. 

‘The Centre, the Central Land’, is an apt designation given the geographic specificity of the Yongning basin, set within a~rugged landscape of steep slopes and high mountains. It is also compelling in light of the basin's historical significance for Naish peoples (see Appendix~\ref{chap:historyanthropologysociology}, in particular \sectref{sec:historicaloutline}). 

From a~linguistic perspective, the greatest diversity within the \ili{Naish} language group~-- the lower-level subgrouping to which Na belongs (\sectref{sec:dialectclassificationyongningnaandnaxi})~-- is indeed concentrated in and around the Yongning basin, within a~radius of less than 100 kilometres. The name ‘People of the Centre’, \ipa{ɬi˧-hĩ˧}, 
%\footnote{For simplicity, this noun is provided here (and in \tabref{tab:thenamesofthenaendonymsandexonyms}) in surface phonological transcription. Its underlying form is //\ipa{ɬi˧-hĩ\#˥}//, with a~{floating} High tone, analyzed in \sectref{sec:afloatinghtonewithcomparativeevidencepointingtoitsorigin}.} 
originally referred to the
inhabitants of the Yongning plain. Ironically, this name is no longer commonly used in the dialect under study, despite the dialect being located squarely within the Yongning plain. Instead, it survives among a~community of speakers who relocated from Yongning to the peripheral region of Shuiluo \zh{水落} (in the neighbouring county of Muli \zh{木里}) several centuries ago.


\subsection{Dialect classification}
%\subsection{Dialect classification and issues of phylogeny}
\label{sec:dialectclassificationyongningnaandnaxi}

%\subsection{Dialect classification: The heritage of mid-20\textsuperscript{th} century surveys}
\label{sec:dialectclassification}

The language spoken in Yongning was investigated in 1979 by linguists He 
Jiren \zh{和即仁} and Jiang Zhuyi \zh{姜竹仪}, who classified it among the Eastern \ili{Naxi} dialects \citep[4, 104--116]{heetal1985}. The division of \ili{Naxi} into Eastern and Western dialects was initially proposed cautiously as a~working hypothesis, based on relatively brief fieldwork conducted in 1956 and 1957 as part of the national survey of languages within the borders of the People’s Republic of China.

\begin{quotation}
	From our analysis and comparison of the available linguistic and cultural materials, we propose a~preliminary division between two dialects, Western and Eastern. But due to the very short amount
	of time [that could be devoted to this research] and the shortcomings of our experience, it is
	difficult to tell whether this division tallies with the actual language
	situation{\dots}~\citep[120]{heetal1988}
    
    {\medskip} %Added on April 27th, at the recommendation of S. Nordhoff, instead of the footnote: \footnote{\textit{Original text:} \zh{我们从现有的语言和人文材料来加以分析和比较,将纳西语初步分划为西部和东部两个方言。不过时间短促,经验不足,这样分划不知是否符合客观现实情况{\dots}{\dots}}}
    {\noindent} \textit{Original text:} \zh{我们从现有的语言和人文材料来加以分析和比较,将纳西语初步分划为西部和东部两个方言。不过时间短促,经验不足,这样分划不知是否符合客观现实情况{\dots}{\dots}}
\end{quotation}

%Command \noindent added to avoid having an indent. Proofreader suggestion: since this sentence continues the argument, it is better not to indent. 
{\noindent}The Western dialect area thus proposed by and large corresponds to the area ruled by the Naxi chieftains of Lijiang from the 14\textsuperscript{th} to the 18\textsuperscript{th} centuries (see Appendix~\ref{chap:historyanthropologysociology}, \sectref{sec:feudal}). The Eastern dialect area is located to its east and north-east, across the Yangtze River, in parts of the present-day counties of Ninglang \zh{宁蒗}, Yanyuan \zh{盐源}, Muli \zh{木里}, and Yanbian \zh{盐边}. Within the Eastern dialect area, three sub-dialects were distinguished by He and Jiang: Yongning \zh{永宁}, Guabie \zh{瓜别}, and Beiquba \zh{北蕖坝}. 

This classification became the standard in Chinese scholarship. It was also adopted by the Summer Institute of Linguistics in its inventory of languages, which serves as an international standard (ISO 639-3): \textit{Ethnologue}: \textit{Languages} \textit{of} \textit{the} \textit{World} \citep{gordon2005}. \il{Naxi|textbf}Naxi initially appeared in Ethnologue under the language code \textsc{nbf}, encompassing all the dialects mentioned above (i.e.\ giving the name “\ili{Naxi}” the same extension as in Chinese scholarship). In 2010, the “Eastern” dialects of \ili{Naxi} were given a separate entry under the romanized name “Narua” (code: \textsc{nru}). The original language code \textsc{nbf} was retired, following the split into (i)~\ili{Naxi} proper (new code: \textsc{nxq}), corresponding to “Western \ili{Naxi}” in Chinese scholarship, and (ii)~“Narua” (code: \textsc{nru}), corresponding to “Eastern \ili{Naxi}”. The division of dialects within “Narua” remains as proposed by He and Jiang for “Eastern \ili{Naxi}”. 

% For typesetting reasons, this paragraph was modified: By contrast, the Glottolog database adopts a \textit{doculect}-based approach, which aims to “improve on the ISO 639-3 language identifiers in terms of quality, transparency and anchoring” \citep[918]{swj_glottocodes}. This approach groups language data~-- ultimately emanating from idiolects of specific speakers, and recorded in specific publications~-- into successively larger categories such as subdialects, dialects, languages, subfamilies and families \citep{nordhoff2011glottolog}.
By contrast, the Glottolog database adopts an innovative approach, setting itself the ambitious goal of “improv[ing] on the ISO 639-3 language identifiers in terms of quality, transparency and anchoring” \citep[918]{swj_glottocodes}. It does so by structuring language labels from the level of the \textit{doculect}~-- ``a linguistic variety as it is documented in a given resource" \citep[342]{good_languoid_2013}~-- into progressively broader categories, such as subdialects, dialects, languages, subfamilies, and families \citep{nordhoff2011glottolog}. This approach is based on the insight that language data ultimately originates from idiolects of specific speakers and is recorded in specific sources. In version 5.1 of Glottolog, Yongning Na is listed under the code \textit{yong1288}, with a distinct entry for the Na dialect spoken in Lataddi, on the eastern shore of Lugu Lake (Glottocode: \textit{lata1234}). These two dialects together form a grouping labelled ‘Narua’ (Glottocode: \textit{yong1270}). While the label ‘Narua’~-- a~romanization of /\ipa{nɑ˩ʐwɤ˥}/ ‘Na language’, shown in (\ref{ex:nalang})~-- is thus shared by Ethnologue and Glottolog, its characterization differs between the two databases. 

The total number of speakers of Na (“Eastern Naxi”) was estimated at around 40,000 in early surveys \citep[107]{heetal1985}, a figure reiterated by \citet{yang2009}. The Ethnologue entry reports a slightly higher figure of 47,000 as of 2012, based on the Summer Institute of Linguistics' sources \citep{lewisetal2016}. 
%\footnote{
The figure of 47,000 speakers includes groups that do not identify their native language by the name /\ipa{nɑ˩ʐwɤ˥}/, which highlights a~drawback of the name ‘Narua’.
%} % Removed footnote, integrating to main text.

No large-scale dialectal comparison was conducted in the half-century that followed the first dialect survey. The list of “subfamilies” 
(\textit{zhīxì} \zh{支系}) of the “Naxi nationality” (\textit{Nàxīzú} \zh{纳西族})
provided by \citet[5–9]{guoetal1994} could serve as a~useful reference for such a~survey, keeping in mind that this list was essentially based on ethnological criteria rather
than linguistic data. Reliable descriptions of the language varieties that these authors grouped under the label
“Naxi” are required for fine-grained dialectological and comparative research. The present volume
aims to contribute to this long-term endeavour by offering a~synchronic description of the tone system of one
specific language variety.

\subsection{The position of Na and Naxi within Sino-Tibetan}
\label{sec:thepositionofnaandnaxiwithinsinotibetan}

The position of \ili{Naxi} and Na within \il{Sino-Tibetan}Sino-Tibetan is a~topical issue in \il{Sino-Tibetan}Sino-Tibetan historical
linguistics. \ili{Naxi} was classified as a~member of the “Lolo branch” (\ili{Yi}) by \citet{shafer1955};
however, Shafer clarified that this language, to which he referred as “Mosso”, was among those for
which there was “[t]oo little data or too irregularly recorded” (p. 103, note 37). His classificatory
proposal for \ili{Naxi}, as for the other languages placed in the “Unclassified” set within the “Lolo
branch”, was thus tentative. 

\citet{bradley1975} revisited the issue based on advances in the
comparative study of Lolo (\ili{Yi}) languages. He noted that “[w]hile a~large proportion of Nahsi
vocabulary is plausibly cognate to Proto-{Burmese}-Lolo (*BL) and Proto-Loloish (*L) forms
reconstructed in \citealt{bradley1975b}, there is only limited systematic regularity of
\is{comparative method (historical linguistics)}correspondence. Moreover, the tonal and other developments postulated for *BL and *L by Matisoff are
not reflected in Nahsi”. The lack of regular \is{comparative method (historical linguistics)}correspondences, and the absence in \ili{Naxi} of shared
innovations deemed characteristic of Loloish and Burmese-Lolo, led Bradley to conclude that \ili{Naxi} is
“certainly not a~Loloish language, and probably not a~Burmish language either” (p. 6).

Some scholars, especially in mainland China, nonetheless maintain the classification of \ili{Naxi} as
a~member of the \ili{Yi}/{\allowbreak}Lolo group. \citet{gaietal1990} base this renewed claim on the high
percentage of phonetically similar words between \ili{Naxi} and \ili{Yi}/{\allowbreak}Lolo, though without
verifying the regularity of sound correspondences. \citet{lama2012} includes \ili{Naxi} among the set of
thirty-seven Lolo-{Burmese} languages among which he proposes subgroupings using two methods: (i)~searching for
candidates for the status of shared innovations, and (ii)~conducting automated computation. The latter approach consists of performing Bayesian inference of phylogeny using \textsc{MrBayes}, and computing phylogenetic networks by means of  \textsc{SplitsTree}. These two programmes are applied to a~300-word list from the thirty-seven languages. No judgments of cognacy
are passed on the 300 word sets fed as input to the computational procedure, which apparently
assumes cognacy in all cases as a~default hypothesis. By definition, these methods generate subgrouping proposals
without questioning the premise that the languages at issue all belong to the same branch.

In light of the conclusion reached by \citet{bradley1975}, if one looks outside Lolo-{Burmese} for
languages most closely related to \ili{Naxi} and Na, suggestive evidence comes from comparison with the
neighbouring languages \ili{Shixing} \zh{史兴语} (also known as Xumi; see \citealt{huangetal1991} and \citealt{chirkovaetal22013}) and \ili{Namuyi} \zh{纳木依语} (\citealt{lama1994}; see also \citealt{sun2001}, \citealt{yang2006}, and \citealt{lakhietal2010}). However, full-fledged comparison
has not been carried out yet, and the state of \isi{phonological erosion} in these languages is a~major
impediment to \is{comparative method (historical linguistics)}comparative studies.

Sun Hongkai included \ili{Shixing} (Xumi) and \ili{Namuyi} within a~“Qiangic” language group which he defined on
the basis of typological similarities. Other languages in the group include Qiang (\ili{Rma}), \il{rGyalrongic}rGyalrongish, \ili{Pumi}, \ili{Muya} (Minyag), \ili{Queyu} and \ili{Zhaba}. 

\begin{quotation}
    Naxi is often regarded as a language of the Yi branch, yet it is widely acknowledged that it does not fit neatly within this classification \citep[14]{bradley1979}. Reciprocal constructions in Naxi verbs are fully identical to those of the Qiang branch. 
    Furthermore, Naxi dialects exhibit a number of phonological, lexical, and grammatical features that align with those of Qiangic languages.
    Naxi is thus a `borderline' language between the Qiang and Yi branches, possessing characteristics from both. That is, in genealogical classification, Naxi has traits of both language groups: some of its features are specific to the Yi branch, while others are distinctive of Qiangic languages. Among these, the system of reciprocal constructions stands out as a~key grammatical feature linking Naxi to Qiangic.~\citep[14]{sun1984}
    
    {\medskip} %Added on April 27th, at the recommendation of S. Nordhoff, instead of the footnote.
    {\noindent} \textit{Original text:}
 \zh{纳西语经常被认为是彝语支的语言,大家知道,纳西语在彝语支中是不太合套的一种语言(详情请参阅}David Bradley\zh{《}Proto-Loloish\zh{》第14页),纳西语动词的互动范畴和羌语支完全一致。此外纳西语方言中还有一些语音、词汇和语法现象,与羌语支语言相一致。纳西语是羌语支和彝语支之间的“临界”语言。即在语言谱系分类上兼有两种语言集团的不同的特征。也就是说,纳西语同时兼有彝语支和羌语支所特有的某些特征,而动词互动范畴,则是纳西语兼有羌语支的一种重要的语法特征。} 
\end{quotation}
 
Sun Hongkai thus proposes that \ili{Naxi} (understood as encompassing both \ili{Naxi} and Na)
is an “intermediate language” (\textit{línjiè yǔyán} \zh{临界语言}) between Loloish and Qiangic
\citep{sun1984}. This compromise view projects the presence of \ili{Yi}-like and Qiangic-like
typological features into the indefinite past of {Naxi}. By a~similar reasoning, the presence of words
of \ili{Sinitic} (\il{Sinitic}Chinese), \il{Tai-Kadai}Tai-Kadai, and \il{Mon-Khmer}Mon-Khmer origin in \ili{Vietnamese} could lead to its classification as an intermediate language straddling the divide between these three language families.  Historical
linguists, however, favour an approach in which borrowings and other changes in the language
are gradually identified, layer after layer, ultimately resulting in a~detailed account of the
language’s evolution that includes the influences to which it was subjected through the
ages. Thus, Maspero, in his study on \ili{Vietnamese}, identified \il{Sinitic}Chinese elements as belonging to a~later layer than \ili{Tai-Kadai} and \il{Mon-Khmer}Mon-Khmer elements:

\begin{quotation}
	Pre-Annamite was born out of the fusion of a~\il{Mon-Khmer}Mon-Khmer dialect with a~{Tai} dialect; the fusion may
	even have involved a~third language, which remains unidentified. At a~later period, the Annamite
	language borrowed a~huge number of Chinese words.~\citep[118]{maspero1912} 
    
    {\medskip} %Added on April 27th, at the recommendation of S. Nordhoff, instead of the footnote.
    {\noindent} \textit{Original text:} Le préannamite est né de la fusion d'un dialecte mon-khmer, d'un dialecte thai et peut-être même d'une troisième langue encore inconnue, et postérieurement, l'annamite a~emprunté une masse énorme de mots chinois.
\end{quotation}

Four decades later, Haudricourt attempted to disentangle the \il{Tai-Kadai}Tai-Kadai and \il{Mon-Khmer}Mon-Khmer components, emphasizing that the notion of “language fusion” can be misleading:

\begin{quotation}
	If we admit that there is no such thing as “fusion” between languages, and that genealogical
	relatedness must be assessed on the basis of core vocabulary and grammatical structure, we are led
	to consider that the modern form of a~language is not determined by its genealogical origin, but
	by the influences to which it is subjected in the course of its history.~(\citealt[121--122]{haudricourt1953a})    
    
    {\medskip} %Added on April 27th, at the recommendation of S. Nordhoff, instead of the footnote.
    {\noindent} \textit{Original text:} Si l'on admet qu'il n'y a~pas de «~fusion~» de langues, et que l'apparentement généalogique doit être fondé sur le vocabulaire de base et la structure grammaticale, on sera conduit à penser que ce qui donne sa forme moderne à une langue n'est pas son origine généalogique, mais les influences qui s'exercent sur elle au cours de son histoire.
\end{quotation}

Haudricourt identified a~greater proportion of \il{Mon-Khmer}Mon-Khmer words in basic vocabulary as opposed to words of \il{Tai-Kadai}Tai stock, and concluded that \ili{Vietnamese} is a~\il{Mon-Khmer}Mon-Khmer language. Importantly, this proposed phylogenetic affiliation in no way constitutes a~denial of the considerable influence of \isi{language contact} in the course of history~-- a~point underscored by \citet[268-271]{dimmendaal2011}, citing \citet{manessy1990}. 

\begin{quotation}
	It is important to realise that there is no principled way in which one can argue that language x has become “mixed”, i.e.\ embedded with foreign language material, to an extent where it should be classified as non-genetic, or multi-genetic. There are scales or degrees of borrowing, and it is precisely for this reason that it is \textit{not} a~useful taxonomic principle to talk about non-genetic or multi-genetic developments. \citep[271]{dimmendaal2011}
\end{quotation}

There is no clear dividing line between cases of \is{language contact}contact that are considered to result in language replacement~-- where the vocabulary inherited from an earlier language is regarded as a~substratum, comprising piecemeal vestiges of a~language that has been replaced by another (e.g.~Basque or Celtic elements in \ili{Romance} languages)~-- and cases where the earlier component remains substantial enough to justify the classification of the modern language as belonging to that earlier component’s language family, such as the \il{Mon-Khmer}Mon-Khmer component in \ili{Vietnamese}.

Mixed languages with split ancestry are nonetheless recognized as a special case, and handled accordingly in historical linguistics. ``A real challenge to the Tree model is the existence of mixed languages with split ancestry \citep[]{bakker2017}, such as \ili{Michif} (verbs and verbal morphology
from \ili{Cree}, nouns and nominal morphology from \ili{French}) or \ili{Media Lengua} (grammar from
\ili{Quechua}, lexicon from \ili{Spanish}). Mixed languages are exceptional, and given their rarity,
the Tree model remains generally applicable as long as such cases are correctly identified'' \citep{ledgeway_family_2025}. Neither \il{Vietic languages}Vietic languages nor Naic languages fall into this special category.

Returning to \ili{Naxi} and Na, the traditional tools of \is{comparative method (historical linguistics)}comparative-historical phonology appear to be the most reliable means of unravelling this language’s history and clarifying its relationship to other languages within the \il{Sino-Tibetan}Sino-Tibetan family. There is widespread agreement regarding this method: “it is only by searching for lexical and morphological parallels on all sides and by establishing the phonetic equations for such parallels that we can finally decide the genetic relationship of a~doubtful group” \citep[98]{shafer1955}. %To what extent this endeavour is successful depends, as Shafer was keenly aware, on the empirical basis: the abundance and reliability of available data. 

The observation on which Sun Hongkai places considerable emphasis as an indicator of phylogenetic relatedness~-- namely, the use of reduplication in reciprocal constructions in Naxi as well as in the languages he groups together as “Qiangic”~-- does not carry any significant weight as a~diachronic argument. Similar constructions are found in a variety of languages, and there is no barrier to such a construction being borrowed in a situation of \isi{language contact} or being independently innovated from a~pre-existing use of \isi{reduplication} to convey \isi{intensification}. This latter possibility is hypothesized, for instance, in the case of \ili{Tok Pisin}, a creole language spoken throughout Papua New Guinea \citep[6]{fedden_reciprocal_2003}. 

\begin{figure}[tb] % earlier comment: put on page to avoid pagebreak in fn 12
    \resizebox{1.04\textwidth}{!}{
    \begin{forest}
        where n children=0{tier=lect}{},
        for tree = {reversed, grow'=east, parent anchor=east, child anchor=west, anchor=west, l sep=1em, s sep=0}%
        [Sino-\\Tibetan, align=left
          [Burmo-\\Qiangic\\(Burmo-\\rGyal-\\rongic), align=left, tier=prim, l sep=2em
            [Lolo-\\Burmese, align=left, tier=sec 
              [Loloish, tier=ish 
                [Lahu] 
                [Lisu] 
                [\textit{etc.}]
              ] 
              [Burmish, tier=ish 
                [Burmese] 
                [\textit{etc.}]
              ]
            ]
            [Na-\\Qiangic, align=left, tier=sec
              [Qiangic, tier=ter
                [rGyalrongic 
                  [rGyalrongish\\(East\\rGyalrongic), align=left, tier=ish, l sep=2em 
                    [Situ] 
                    [Japhug] 
                    [Tshobdun] 
                    [Zbu]
                  ]
                  [Lavrung, tier=ish 
                    [Thugsrje] 
                    [Njorogs]
                  ] 
                  [Horpa 
                    [Rtau] 
                    [Stodsde]
                  ] 
                ]
                [Qiang, tier=ish 
                  [Northern Qiang (Rma)] 
                  [Southern Qiang (Rma)]
                ]
                [Muya, tier=ish 
                  [Northern Muya] 
                  [Southern Muya]
                ]
                [Queyu]
                [Zhaba?]
                [{Pumi (Prinmi)}, tier=ish, calign with current 
                  [Northern Pumi] 
                  [Southern Pumi]
                ]
                [Tangut]
              ]
              [Ersuish, tier=ish 
                [Ersu] 
                [Lizu] 
                [Tosu]
              ]
              [Naic, tier=ter 
                [Namuyi] 
                [Shixing (Xumi)] 
                [Naish, tier=ish 
                  [Naxi] 
                  [\textbf{Na (Mosuo)}] 
                  [Laze]
                ]
              ]
            ]
          ]
          [Bodic\\(including\\Tibetan), align=left, tier=prim]
          [Sinitic\\(Chinese), align=left, tier=prim]
          [other\\primary\\branches, align=left, tier=prim]
        ]
    \end{forest}}
    \caption{A family tree showing the position of Yongning Na within Sino-Tibetan as hypothesized by \citet{jacquesetal2011}. Yongning Na is highlighted (second from bottom).}\il{Sino-Tibetan|textbf}
    \label{fig:atentativefamilytreeshowingthepositionofyongningnawithinaburmoqiangicbranchofsinotibetan}
\end{figure}

% \begin{figure}[tb] %earlier comment: put on page to avoid pagebreak in fn 12
% % 	\centering
% 	\resizebox{1.04\textwidth}{!}{
% 	\begin{forest}
% 		where n children=0{tier=lect}{},
% 		for tree = {reversed,grow'=east,parent anchor=east,child anchor=west,anchor=west,l sep=1em,s sep = 0}%
% 		[Sino-\\Tibetan,align=left
% 		  [Burmo-\\Qiangic\\(Burmo-\\Rgyal-\\rongic),align=left,tier=prim,l sep=2em
% 			[Lolo-\\Burmese,align=left,tier=sec [Loloish,tier=ish [Lahu] [Lisu] [\textit{etc.}]] [Burmish,tier=ish [Burmese] [\textit{etc.}]] ]
% 			[Na-\\Qiangic,align=left,tier=sec
% 			  [Qiangic,tier=ter
% 			[Rgyalrongic [Rgyalrongish\\(East\\Rgyalrongic),align=left,tier=ish,l sep=2em [Situ] [Japhug] [Tshobdun] [Zbu]] [Lavrung,tier=ish [Thugsrje] [Njorogs]] [Horpa [Rtau] [Stodsde]] ]
% 			[Qiang,tier=ish [Northern Qiang (Rma)] [Southern Qiang (Rma)] ]
% 			[Muya,tier=ish [Northern Muya] [Southern Muya] ]
% 			[Queyu]
% 			[Zhaba?]
% 			[{Pumi (Prinmi)},tier=ish,calign with current [Northern Pumi] [Southern Pumi] ]
% 			[Tangut]
% 			  ]
% 			  [Ersuish,tier=ish [Ersu] [Lizu] [Tosu]]
% 			  [Naic,tier=ter [Namuyi] [Shixing (Xumi)] [Naish,tier=ish [Naxi] [\textbf{Na (Mosuo)}] [Laze]] ]
% 			]
% 		  ]
% 		  [Bodic\\(including\\Tibetan),align=left,tier=prim]
% 		  [Sinitic\\(Chinese),align=left,tier=prim]
% 		  [other\\primary\\branches,align=left,tier=prim]
% %		  [{{\textit{other}}{\\}{\textit{\primary}}{\\}{\textit{\branches}}},align=left,tier=prim]
% 		]
% 	\end{forest}}
% 	\caption{A tentative family tree showing the position of Yongning Na within Sino-Tibetan as hypothesized by \citet{jacquesetal2011}. Yongning Na is highlighted (second from bottom).}\il{Sino-Tibetan|textbf}
% 	\label{fig:atentativefamilytreeshowingthepositionofyongningnawithinaburmoqiangicbranchofsinotibetan}
% \end{figure}

\figref{fig:atentativefamilytreeshowingthepositionofyongningnawithinaburmoqiangicbranchofsinotibetan} presents a~family tree proposed in a~preliminary \is{comparative method (historical linguistics)}comparative study \citep{jacquesetal2011}, based on Yongning Na, Lijiang \ili{Naxi}, and \ili{Laze}~-- a~language spoken in Muli County. Within the \il{Sino-Tibetan}Sino-Tibetan family, waves of mutual influence across languages are so pervasive that concerns regarding the applicability of the tree model have been expressed for decades \citep{benedict1972,matisoff1978}. However, the use of a~tree representation does not imply a disregard for the importance of areal diffusion. 

In a~context of continuing debates about models and methods (see e.g.~\citealt{francois2014, jacques.listSAVE}), it is essential to highlight the usefulness of the tree model as one of the tools in the historical linguist’s toolbox. It provides a framework for formulating working hypotheses about degrees of phylogenetic closeness between languages~-- hypotheses that form the basis for comparative efforts and for proposing reconstructions at various historical depths \citep{ledgeway_family_2025}.  
%The researcher’s aim when proposing a~tree model is not to float new proposals about classification for classification’s sake, but to clarify assumptions made in historical comparison. 

{\largerpage}

The aim of proposing a tree model is not to advance classifications for their own sake but to clarify the assumptions underlying historical comparison. The central objective is to document the evolution from the hypothesized common ancestor of a~language group to the attested language varieties. 

\begin{quotation}
We may assert or hypothesize a genetic relation on the basis of [regular sound correspondences]. But the proof of the linguistic pudding remains in the follow-up, the systematic exploitation, the full implementation of the comparative method, which alone can demonstrate, not just a linguistic genetic relationship, but a linguistic history. \citep[295]{watkins1990}
\end{quotation}


The proposal that Yongning Na, Lijiang \ili{Naxi}, and \ili{Laze} form a~\il{Naish|textbf}{Naish} lower-level
subgroup is supported by shared innovations, notably structural similarities in the tones of {numeral}-plus-classifier phrases that cannot have been acquired through \is{language contact}contact (for details, see \citealt{michaud2011c}). It is further proposed that
the {Naish} subgroup joins with \ili{Shixing} (Xumi) and \ili{Namuyi} into a~\il{Naic|textbf}Naic subgroup. At a~third, more speculative
level, {Naic} is hypothesized to join with {Ersuish} (referred to as {Ersuic} in the historical study of \citealt{yu2012}; see also the reference grammar of \ili{Ersu} by \citealt{zhang2013ersu}) and with a broad `Qiangic' group {\largerpage} that includes \ili{rGyalrongic}, \ili{Rma} (Qiang), \ili{Muya}, \ili{Pumi}, \ili{Tangut}, and more. The proposed grouping, termed `\ili{Na-Qiangic}',\footnote{The label `\ili{Na-Qiangic}' reflects a Na-centred perspective, arising in the context of reconstruction work focusing on Naish languages. In broader contexts, this choice warrants revision. The `Qiangic' component in the label `Na-Qiangic' also risks causing confusion due to the varied uses made of the exonym `Qiang' \zh{羌}. Historically, this term has been used in Chinese records with a broad meaning and now serves as the designation for an officially recognized ethnic group, speakers of the \ili{Rma} language \citep{sims_towards_2016,sims_methodological_2024}. Given the significance of the rGyalrongic group for Sino-Tibetan historical linguistics \citep{hill2019historical,zhang_study_2019}, recent scholarship has adopted terminology that replaces `Qiang' with `rGyalrong' in intermediate phylogenetic labels. For instance, Guillaume Jacques and colleagues \citep{jacques_phylogenies_2021} refer to the node at which Lolo-Burmese joins the grouping containing rGyalrongic languages as `Burmo-rGyalrongic', instead of the label `Burmo-Qiangic' used in \figref{fig:atentativefamilytreeshowingthepositionofyongningnawithinaburmoqiangicbranchofsinotibetan} and retained by \citet{hill2021scholarship}.} is further connected to Lolo-{Burmese} to form a~higher-level
grouping referred to as `Burmo-{Qiangic}' or `Burmo-rGyalrongic', which is placed on a~par with Bodic, {Sinitic}, and other primary branches. This working hypothesis encourages systematic searches for cognates between {Naic} languages, {Ersuish}, and
Qiangic. Comparison with Lolo-{Burmese} languages is also
essential for advancing the historical study of {Naish} languages. All the languages listed in
\figref{fig:atentativefamilytreeshowingthepositionofyongningnawithinaburmoqiangicbranchofsinotibetan}
are uncontroversially related, as members of the \il{Sino-Tibetan}Sino-Tibetan family, therefore the \is{comparative method (historical linguistics)}comparative analysis
of data from all these languages makes sense. 

In the present era, computational approaches offer a useful addition to the historical linguist's toolbox \citep{wu_computer-assisted_2020}. These approaches, which encompass computational methods for language phylogeny, have seen significant advances, including the development of techniques for automating cognate detection and conducting Bayesian phylogenetic inference \citep{rama-list-2019-automated}. Accordingly, it would seem advisable to subject the representation in \figref{fig:atentativefamilytreeshowingthepositionofyongningnawithinaburmoqiangicbranchofsinotibetan}~-- which is based on Guillaume Jacques's expert intuitions about the Sino-Tibetan family~-- to computational phylogenetic testing. However, the eight languages at the bottom of \figref{fig:atentativefamilytreeshowingthepositionofyongningnawithinaburmoqiangicbranchofsinotibetan} (including Yongning Na) were intentionally excluded from a~large-scale study of fifty Sino-Tibetan languages that aimed to produce dated phylogenies \citep{sagart_dated_2019}. This exclusion stemmed from the fact that the historical phonologies of these languages have not yet been sufficiently developed to permit reliable cognate identification and the establishment of sound correspondences with the rest of the family (Guillaume Jacques, p.c.). 

{\largerpage[-1]}

Nonetheless, another study published in the same year \citep{zhang_phylogenetic_2019} incorporated six of these more challenging languages~-- Yongning Na, Naxi, Xumi (Shixing), Namuyi, Lyuzu (Lizu), and Ersu~-- among a set of 109 Sino-Tibetan languages. This study produced a ``clade credibility tree'' estimating divergence times for the emergence of the major Sino-Tibetan branches and their subsequent diversification. A portion of this tree is presented in \figref{fig:ZhangMenghanFamily}, with modifications: the figure excludes groupings outside Burmo-rGyalrongic (such as Sinitic, Bodic, Karen, Kuki-Chin-Naga, and Himalayish/Kiranti), and omits some details about lower-level branches not directly relevant to the discussion at hand. 

\begin{figure}[p] 
% 	\centering
\resizebox{1.00\textwidth}{!}{

\begin{forest}
  forked edges,
  /tikz/every pin edge/.append style={Latex-, shorten >=2.5pt%, 
  %darkgray
  },
  /tikz/every pin/.append style={%darkgray
  %, font=\sffamily
  },
  /tikz/every label/.append style={%darkgray
  %, font=\sffamily
  },
  before typesetting nodes={
    delay={
      where content={}{coordinate}{},
    },
    where n children=0{tier=terminus, label/.process={Ow{content}{right:#1}}, content=}{},
  },
  for tree={
    grow'=0,
    s sep'+=8pt, % vertical separation between lines
    l sep'+=5pt, % Parameter for separation between levels
  },
  l sep'+=5pt,
  % tikz+={
  %   \node [draw=blue, circle, ellipse, densely dashed, fit to={1,tree}, pin={[pin distance=50pt,name=clades]-40:clades}] {};
  %   \node (p) [draw=blue, circle, ellipse, densely dashed, fit to={l,tree}] {};
  %   \draw [every pin edge] (p) -- (clades);
  % }
%[ % ST
    % [ %Burmo Bodish+ Nung
        [%Burmo Bodish
            [ % Burmo Qiangic  
                [ % Lolo-Burmese
                    [ [ [[Lahu][Lisu]] [Sani (Yi)]] [[Youle Jinuo][[Mpi][Hani]]]]
                    [  [[[[Achang][Atsi]][[[Bola][Langsu]][Leqi]]][[Burmese][Marma]]][[Central Nusu][[Northern Nusu][Southern Nusu]]]  ]  
                ] % Lolo-Burmese
                          [ %Na-rG-IC (dont Ersu et Naic)
        [ %rG-IC sans Naic
            [ %rG-IC sans Ersu: rG-ish et PumiQiang
                    [ %rG-ish
                        [Caodeng (Tshobdun)] [Maerkang (Japhug) rGyalrong]
                    ]
                    [ %MuyaPumiQiang 
                    [[[Daofu][Ergong]][Muya (Minyak)]]
                    [[[Pumi][Yajiang Queyu]][[Mawo Qiang (Rma)][Longxi Qiang (Rma)]]]
                    ]
            ]
            [ % Ersu
                [Ersu]
                [Lyuzu (Lizu)]
            ]
        ]
        [ % Naic
            [Namuyi]
            [ %Naic sans Namuyi
                [ % Naish
                    [Naxi (Lijiang dialect)] 
                    [\textbf{Na (Yongning dialect)}]
                ]
                [Xumi (Shixing)]
            ]
        ]    
                ] %Na-rGic
            ] 
            [{Bodic, Nungish{\dots}}]
        ]
    %     [{Nungish, Himalayish{\dots}}]
    % ] %Burmo Bodish+ Nung
%     [Himalayish]
% ] % finST
\end{forest}
 } % Closing bracket for 'resizebox'
	\caption{A portion of the maximum clade credibility tree for 109 Sino-Tibetan languages shown in Figure 1 of \citet{zhang_phylogenetic_2019}. Yongning Na is highlighted (third from bottom).}\il{Sino-Tibetan|textbf}
	\label{fig:ZhangMenghanFamily}
\end{figure}

A~technical difference between \figref{fig:atentativefamilytreeshowingthepositionofyongningnawithinaburmoqiangicbranchofsinotibetan} and \figref{fig:ZhangMenghanFamily} lies in the branching structure employed by \citet{zhang_phylogenetic_2019}. The binary branching model adopted in their study allows for a more fine-grained resolution at nodes where \figref{fig:atentativefamilytreeshowingthepositionofyongningnawithinaburmoqiangicbranchofsinotibetan} features polytomies (forks with three or more branches). For instance, while \figref{fig:atentativefamilytreeshowingthepositionofyongningnawithinaburmoqiangicbranchofsinotibetan} places the Naish subgroup (Lijiang Naxi and Yongning Na\footnote{A detail is that Yongning Na is referred to in \citet{zhang_phylogenetic_2019} as ``Naxi Yongning'', mirroring the convention of the Chinese source used in the Sino-Tibetan Etymological Dictionary and Thesaurus (STEDT) database, from which the data is extracted. The name is rewritten in \figref{fig:ZhangMenghanFamily} as ``Na (Yongning)''.}) in a node alongside \ili{Namuyi} and Xumi (\ili{Shixing}), \figref{fig:ZhangMenghanFamily} resolves this differently: Xumi (\ili{Shixing}) branches off first, followed by \ili{Namuyi} at the subsequent binary node. Despite this slight difference, the overall representation in \figref{fig:ZhangMenghanFamily} closely mirrors that in \figref{fig:atentativefamilytreeshowingthepositionofyongningnawithinaburmoqiangicbranchofsinotibetan}. 

The remarkable similarity between the two trees raises questions about its underlying cause. One possibility is that Zhang et al.'s classification draws on that proposed in Glottolog 3.1 \citep{hammarstrom_glottolog_2017}, which incorporates \citet{jacquesetal2011} as one of its sources. While the computational part of Zhang et al.'s study can be reproduced using the article's accompanying data files, a significant limitation is the lack of access to their full set of proposed cognates. This absence makes it impossible to scrutinize the etymological groupings proposed by the authors or to evaluate how they addressed the practical challenges of diachronic comparison for the Naic languages included in their analysis. 



% \begin{figure}[p] 
% % 	\centering
% 	\resizebox{1.00\textwidth}{!}{

% % \begin{forest}
% % for tree={grow’=0,draw},
% % forked edges,
% % [/
% % [home
% % [saso
% % [Download]
% % [TeX]
% % ]
% % [alja]
% % [joe]
% % ]
% % [usr
% % [bin]
% % [share]
% % ]
% % ]
% % \end{forest}
 
%         \begin{forest}
% 		where n children=0{tier=lect}{},
% %  for tree={grow’=0,draw},
%         for tree={grow’=0},
%         forked edges,
% [/
% [home
% [
% [Download]
% [TeX]
% ]
% [alja]
% [joe]
% ]
% [usr
% [bin]
% [share]
% ]
% ]
% %		for tree = {reversed,grow'=east,parent anchor=east,child anchor=west,anchor=west,l sep=1em,s sep = 0}%
% 		% [Sino-\\Tibetan,align=left
% 		%   [Burmo-\\Qiangic\\(Burmo-\\Rgyal-\\rongic,align=left,tier=prim,l sep=2em
% 		% 	[Lolo-\\Burmese,align=left,tier=sec [Loloish,tier=ish [Lahu] [Lisu] [\textit{etc.}]] [Burmish,tier=ish [Burmese] [\textit{etc.}]] ]
% 		% 	[Na-\\Qiangic,align=left,tier=sec
% 		% 	  [Qiangic,tier=ter
% 		% 	[Rgyalrongic [Rgyalrongish,tier=ish [Situ] [Japhug] [Tshobdun] [Zbu]] [Lavrung,tier=ish [Thugsrje] [Njorogs]] [Horpa [Rtau] [Stodsde]] ]
% 		% 	[Qiang,tier=ish [Northern Qiang] [Southern Qiang] ]
% 		% 	[Muya,tier=ish [Northern Muya] [Southern Muya] ]
% 		% 	[Queyu]
% 		% 	[Zhaba?]
% 		% 	[{Pumi (Prinmi)},tier=ish,calign with current [Northern Pumi] [Southern Pumi] ]
% 		% 	[Tangut]
% 		% 	  ]
% 		% 	  [Ersuish,tier=ish [Ersu] [Lizu] [Tosu]]
% 		% 	  [Naic,tier=ter [Namuyi] [Shixing (Xumi)] [Naish,tier=ish [Naxi] [Na (Mosuo)] [Laze]] ]
% 		% 	]
% 		%   ]
% 		%   [Bodic\\(including\\Tibetan),align=left,tier=prim]
% 		%   [Sinitic\\(Chinese),align=left,tier=prim]
% 		%   [other\\primary\\branches,align=left,tier=prim]
% %		  [{{\textit{other}}{\\}{\textit{\primary}}{\\}{\textit{\branches}}},align=left,tier=prim]
% %		]
% 	\end{forest}}
% 	\caption{Part of the Sino-Tibetan family tree in \citet{zhang_phylogenetic_2019}.}\il{Sino-Tibetan|textbf}
% 	\label{fig:ZhangMenghanFamily}
% \end{figure}

% \citet{zhang_phylogenetic_2019}  The tree representations in the article offer a quantification of plausibility values for specific branches. Figures 3 and 4 in the article's Extended Data set out reliability values on the internal nodes, calculated from four-point analysis for the maximum clade credibility tree of 109 Sino-Tibetan languages.

{\largerpage}

Systematic comparison between {Naic} and Lolo-{Burmese} holds great potential for clarifying the extent to which their typological similarities result from (i)~inheritance, (ii)~parallel changes~-- unrelated developments,
from a~typologically similar starting point: e.g.~the development of retroflex consonants from
initial consonant clusters~--, or (iii)~\isi{language contact}. This research agenda, outlined in a review of Naish language studies \citep{li2015}, has been implemented in a comparative monograph \citep{li_yuanshi_2021}, which seeks to reexamine the phylogenetic position of Naish languages based on a new reconstruction of proto-Naish. 

Li Zihe's study is based on six dialects, three of which are documented first-hand: Baoshan \zh{宝山} (\zh{宝山乡石头城}), Malimasa \zh{玛丽玛萨} (\zh{塔城镇汝柯村}), and Ninglang \zh{宁蒗} (\zh{宁蒗县大兴镇堰坝村}). The other three~-- Lijiang \zh{丽江}, Bowan \zh{波湾}, and Yongning \zh{永宁} (the dialect described in this book)~-- are incorporated from secondary sources. In total, 730 words are reconstructed for proto-Naish and compared with (i)~830 reconstructed proto-Loloish forms and (ii)~2,000 words from Japhug rGyalrong \citep[175-204]{li_yuanshi_2021}. The study finds that Naish and Loloish share more cognates than Naish shares with rGyalrongish or Loloish shares with rGyalrongish. On this basis, the author concludes that Naish and Loloish form a single branch, to which rGyalrongish connects at a higher taxonomic level \citep[206]{li_yuanshi_2021}. 

Despite the evident care and effort invested in Li Zihe's comparative study, a striking discrepancy in the materials used for Lolo-Burmese and rGyalrong raises concerns about the robustness of the conclusions. On the one hand, the Lolo-Burmese data draws on reconstructions at the level of the entire subgroup; on the other hand, the comparison with rGyalrongic relies exclusively on synchronic data from \ili{Japhug}, a single modern variety. This imbalance complicates the interpretation of the findings. A~more balanced comparison, using reconstructions for rGyalrongish, might yield different results, but such an undertaking is currently hindered by the lack of a comprehensive reconstruction of Proto-rGyalrongish  (a.k.a.\ East rGyalrongic) or \ili{rGyalrongic}. There has been recent progress in this field: in 2023, a lexical database for historical comparison within \ili{rGyalrongic} was released \citep{lai2023lexical}, providing cognacy judgments for words (or parts of words) across the subgroup. However, as of version 0.2, the database does not include reconstructed forms for the 303 words (concepts) it contains, leaving this critical piece of the puzzle incomplete. Further advances in this area are eagerly anticipated, as the stakes are high for \il{Sino-Tibetan}Sino-Tibetan historical linguistics as a whole.

In summary, there remains considerable room for progress in the \is{comparative method (historical linguistics)}historical-comparative study of Naish languages within their broader \il{Sino-Tibetan}Sino-Tibetan context. Greater clarity in the phylogenetic relationships among these languages will depend on continued progress in documentation and \is{comparative method (historical linguistics)}reconstruction.

\section{A review of studies on Yongning Na}
\label{sec:previousstudiesofthenalanguage}

{\largerpage}

Historical sources in Chinese offer fascinating glimpses into the language spoken in Yongning centuries ago. The \textit{Yuan {Yi} 
Tongzhi} {\kern-3pt}\zh{《元一统志》}{\kern-4pt}, a~book dated 1286, records the {Chinese} phonetic equivalents for present-day Lijiang and Yongning as \zh{样渠头} and \zh{楼头} (present-day {Mandarin}: \textit{yàngqútóu} and \textit{lóutóu}), respectively. In the variety of {Chinese} recorded in the 14\textsuperscript{th}-century rhyme table \textit{Zhongyuan 
Yinyun} {\kern-3pt}\zh{《中原音韵》}{\kern-4pt}, the initial of \zh{头} is unvoiced. However, using the \is{comparative method (historical linguistics)}reconstruction of {\apostrophe}Phags-pa\footnote{{\apostrophe}Phags-pa is an alphabetic script devised during the 1260s by the Tibetan monk Drogön Chogyal Phagpa, at the request of emperor Kublai Khan, to transcribe the main languages of the empire: \ili{Mongolian}, Chinese, \ili{Tibetan}, and \ili{Uyghur} (as well as \ili{Sanskrit}). {\apostrophe}Phags-pa was used from 1269 until the mid-14\textsuperscript{th} century, %end of the Yuan \zh{元} dynasty a century later, 
so that a fair amount of parallel-text Chinese-{\apostrophe}Phags-pa materials were produced, which constitute important materials for the study of Chinese historical phonology.}) by \citet{coblin2007}, it is interpreted as *\ipa{dəw}, i.e.\ with the same voicing feature as in present-day Na and \ili{Naxi}. The 
name \zh{楼头} reconstructs as *\ipa{ləw dəw} \citep[487]{jacquesetal2011}, which is evidently cognate with the current names of Yongning in Na (\ipa{ɬi˧di˩}) and \ili{Naxi} (\ipa{ly˧dy˩}), discussed above (\sectref{sec:endoexo}). This historical record establishes that the name dates back at least eight centuries. Moreover, it provides evidence on a~disputed point of Chinese historical phonology, suggesting that the standard dialect of Yuan dynasty Chinese (Northern Mandarin) retained voiced obstruents \citep[487]{jacquesetal2011}. 

Regarding tone, however, these early records offer little insight. %In deference to the present volume's focus on linguistic tone, it appears relevant to point out that 
{\apostrophe}Phags-pa does not indicate tone: segmentally identical syllables under the same tone are spelt the same way \citep[13]{coblin2007}. The limited data and the lack of detailed knowledge about tone in both Yuan dynasty Chinese and Yuan-era {Naish} languages leave many questions unanswered. It is plausible that tone was simply disregarded in the process of selecting Chinese equivalents for local terms. Similarly, the notes of explorers of the turn of the 20\textsuperscript{th} century provide little information about the language and less about tone. For discussion of such early sources, readers are referred to \citet{michaudetal2010} and references therein. This review focuses on contemporary linguistic research on Na.  

\subsection{Information about Yongning Na in \textit{A brief description of the Naxi language}}
\label{sec:heandjiang1985}

{\largerpage}

He Jiren \& Jiang Zhuyi’s \citeyear{heetal1985} \textit{A brief description of the Naxi language} focuses essentially on the dialects spoken in the Lijiang plain, i.e.\ Naxi proper, not Na. However, the volume includes a~word list for Yongning Na, along with observations on phonology,
syntax, and dialectal diversity (pp. 107--116; see also \citealt{jiang1993}).

The transcription in this work is not phonemic and may lack consistency. Only four tones are
transcribed over monosyllables: LM (\ipa{˩˧}), M (\ipa{˧}), ML (\ipa{˧˩}), and H (\ipa{˥}). In contrast, this volume identifies six categories (LM, LH, M, L, H, and MH: see
Chapter~\ref{chap:thelexicaltonesofnouns}). He \& Jiang's analysis relied heavily on their knowledge of \ili{Naxi}, a~language He Jiren spoke natively and Jiang Zhuyi learnt as a second language. \ili{Naxi} has a~four-way tonal distinction over monosyllables:
High, Mid, Low (realized phonetically as low-falling),
and Rising. When analyzing Yongning Na, He \& Jiang failed to differentiate between its low-rising and mid-rising tones. They also reported a distinction between M and H tones in word-initial position, which does not
exist. 

Although these discrepancies might be due to dialectal variation, it seems unlikely, given the phonetic similarity between their word list and the data presented here, that the tone system of their consultants differed considerably from that of the consultants who collaborated with me. More plausibly, these differences reflect the challenges of tonal analysis based on brief fieldwork. As noted by \citet[329]{matisoff2004}, phonemic and tonal analyses from short-term data collection often differ substantially across researchers due to incomplete phonemicization. 
%For example, for \ili{Shixing} (a.k.a. Xumi), \citet{sun1983} reports 58 initials, whereas
%\citet{huangetal1991} report 43, for two subvarieties which are mutually intelligible
%\citep{chirkova2009}.

In the Alawua dialect of Yongning Na, as described in this volume, the phonological distinction between M and H tones only becomes apparent in specific contexts, such as when the word is followed by the {copula} (see Chapter~\ref{chap:thelexicaltonesofnouns}). \is{form!in isolation}In isolation, the M and H tones are neutralized to M (see \sectref{sec:thesynchronicfacts}); phonetic realizations of this tone, which can be understood as an \isi{architoneme} in a~\is{Praguean phonology}Praguean
framework, fluctuate freely within the upper half of the speaker’s tonal
range. This variability can lead an investigator working under the assumption of a contrast between H- and M-tone monosyllables to perceive pitch differences that seem to corroborate such a hypothesis. Such a process of pre-existing assumptions affecting linguistic observation
happened to me during the early stages of fieldwork: my initial transcriptions distinguished H-tone
words and M-tone words, but it turned out later that there was no such opposition \is{form!in isolation}in isolation. 

{\largerpage}

Consequently, the distinction between H and M tones made by He
and Jiang based on words pronounced \is{form!in isolation}in isolation is unwarranted (unless, as mentioned above, there is
a~considerable gap between the tone systems of the dialects at issue). For instance, ‘field’, glossed as ‘earth’
(\textit{dì} \zh{地}), is transcribed by He and Jiang with a~High tone: /\ipa{lv̩˥}/. However, this word actually carries a Mid
tone in Alawua: //\ipa{lv̩˧}// (the double slashes are used in this volume to distinguish lexical forms
from surface phonological ones). Similarly, ‘man’ is transcribed as /\ipa{xĩ˧}/, with a Mid tone, which corresponds to its realization \is{form!in isolation}in isolation. However, its behaviour in other contexts demonstrates that its lexical
tone is in fact H: //\ipa{hĩ˥}//.

The distribution of Low(-falling) and Low-rising tones in He \& Jiang’s data is a~puzzle to me. In the Alawua variety, no monosyllables are realized with Low tone \is{form!in isolation}in isolation. Examples from their data include ‘plain, flatlands’, ‘water’, and ‘goose’, transcribed as /\ipa{dy˧˩}/,
/\ipa{dʑi˧˩}/ and /\ipa{o˧˩}/, respectively. My own findings reveal distinct underlying tones for these items: 
\begin{itemize}
    \item LH for ‘earth, plain’ (//\ipa{di˩˥}//, realized as such in isolation: /\ipa{di˩˥}/);
    \item L for ‘water’ (//\ipa{dʑɯ˩}//, realized in isolation as M: /\ipa{dʑɯ˧}/);
    \item LM for ‘goose’ (//\ipa{ɑ˩˧}//, realized in isolation as LH: /\ipa{ɑ˩˥}/).
\end{itemize}

My best guess is that, in the speech of
He \& Jiang’s consultants, L was a~(relatively infrequent) free {variant} of LH in {citation forms}. Another possibility is that their word list combines data from several speakers, and is not dialectally
homogeneous. 

Such heterogeneity is evident in their Naxi data. Although the authors attribute this material to the Qinglong \zh{青龙} (now Changshui \zh{长水}) dialect, which was spoken by Jiang Zhuyi’s teacher He Zhiwu \zh{和志武}, He Jiren clarified (p.c. 2002) that he contributed some data based on his native dialect, Yangxi \zh{漾西}. In the absence of indications regarding the source of each piece of data, disentangling dialectal variation becomes exceedingly difficult.

  
He \& Jiang’s transcription of vowels and consonants also has some issues, as is to be
expected in initial field notes. Nasality is marked in only two syllables: /\ipa{xĩ}/ (as in
‘man’, transcribed /\ipa{xĩ˧}/; my data: //\ipa{hĩ˥}//) and /\ipa{ɣə̃r}/ (the sole example is ‘bone’,
transcribed /\ipa{ʂa˧ɣə̃r˧}/; my transcription: //\ipa{ʂæ˩ɻ̩̃˩}//). By contrast, the present study identifies eight distinct nasal rhymes. Another notable difference is that He \&
Jiang do not transcribe the uvular consonants reported in Appendix~\ref{chap:vowelsandconsonants} of this volume. Such
discrepancies may reflect genuine phonological differences between the variety they studied and the one documented here. However, it is not implausible that certain contrasts were overlooked.

Conversely, some of the vowel distinctions transcribed by He \& Jiang may not correspond to actual phonemic contrasts. Their word list includes both /\ipa{li}/ (as in /\ipa{li˧}/ ‘to look’) and /\ipa{lie}/ (as in /\ipa{lie˩˧}/
‘tea’). In my data ‘tea’ and ‘to look’ share the same initial and rhyme. The vowel /\ipa{i}/ is
slightly diphthongized towards [\ipa{e}], approaching [\ipa{lie}], which could account for the variation in perception before the investigator’s ear became attuned to the
vowel system of Yongning Na. Once again, it remains theoretically possible that these two words contained different phonemes in the dialect investigated by He \& Jiang.


\subsection{A~linguist's study of kinship terms: \citet{fu1980}}
\label{sec:fu1980astudyofkinshipterms}

The linguist Fu Maoji \zh{傅懋勣} visited Yongning in May and June 1979 together with He Jiren and Jiang Zhuyi. He collected data in the
village of Jjabbu /\ipa{dʑɤ˩bv̩˧-ʁwɤ˩}/ (Jiabowa \zh{甲波瓦}) for a~study on kinship terms, which he presented at the 12\textsuperscript{th} International Conference on {Sino-Tibetan} Languages and Linguistics (Paris,
1979) and later published in both Chinese and {French} translation (\citealt{fu1980,fu1983}). His paper is a~testimony to the broader appeal of Na family structure beyond the circle of professional anthropologists (see Appendix~\ref{chap:historyanthropologysociology}, \sectref{sec:anthropologicalresearchthefascinationofnafamilystructure}). The article includes an appendix with notes on phonetic transcription, which, when examined in the light of a~comprehensive phonemic analysis, reveal some groundbreaking observations~-- most notably, the recognition of the uvular phonemes /\ipa{q}/, /\ipa{qʰ}/ and /\ipa{ʁ}/. Fu Maoji also documented the approximant /\ipa{ɹ}/, noting its occurrence both as an initial consonant before a~vowel and as a syllabic element in its own right. This aligns with the analysis proposed in this volume (see Appendix~\ref{chap:vowelsandconsonants}), where the notation adopted is as a~retroflex, /\ipa{ɻ}{\kern2pt}/. 

However, alongside these perceptive insights, Fu Maoji also proposed some more puzzling analyses. For instance, his inventory of uvular consonants includes the fricative /\ipa{χ}/, which closer scrutiny of the target dialect would likely reveal as non-contrastive with velar realizations, consistent with findings in all recorded Naxi and Na dialects to date. If, during their joint fieldwork, He Jiren and Jiang Zhuyi examined Fu Maoji's notations, they may well have been skeptical about the inclusion of /\ipa{χ}/ in the inventory. Such doubts could have led them to question Fu Maoji's correct identification of /\ipa{q}/ and /\ipa{qʰ}/ as distinct phonemes. Several factors may have contributed to their reservations. First, the uvular sounds [\ipa{q}] and [\ipa{qʰ}] are also found in \ili{Naxi}, where allophonic {variation} of velar stops extends well into the uvular region. This could lead speakers of {Naxi} to assume that uvulars in other {Naish} varieties are likewise allophonic variants of velars. Moreover, in Yongning Na, the phonological environments where /\ipa{k}/ and /\ipa{kʰ}/ contrast with /\ipa{q}/ and /\ipa{qʰ}/ are relatively restricted (for details, see Appendix A, \sectref{sec:velaranduvularstops}).

 Other problematic aspects of Fu Maoji's notation include: (i)~a distinction between plain and laryngeally constricted /\ipa{u}/, /\ipa{v̩}/ and /\ipa{z̩}/ rhymes, (ii)~a proposed set of three rhotic vowels in addition to the approximant rhyme [\ipa{ɹ}], and (iii)~an analysis in which /\ipa{i}/ contrasts with /\ipa{e}/ but is realized as [\ipa{e}] after apical and apical-dental consonants. These notes remain some way from constituting a functional phonemic notation. 
 
 As for tone, Fu Maoji's classification into three categories~-- mid-rising, high-flat, and mid-falling~-- demonstrates a commendable effort to establish the tonal system on its own terms rather than interpreting it through the lens of the \ili{Naxi} system. However, his analysis did not fully reach the point where the relevant categories would clearly emerge. The fragmentary nature of his report may explain why He Jiren and Jiang Zhuyi did not draw upon his data in their 1985 book. Nevertheless, it is important to acknowledge that conducting fieldwork in Yongning in 1979 was, in itself, a significant achievement. 


\subsection{An outline of Yongning Na by Yang Zhenhong (\citeyear{yang2009})}
\label{sec:yang2009}

An outline of Yongning Na was published by Yang Zhenhong \zh{杨振洪}, a~native speaker of the language
from Abbuwua /\ipa{ə˧bv̩˧-ʁwɤ˧}/ village (Abuwa \zh{阿布瓦村}), near the present-day site of Yongning high school. The original version, written in Chinese, appeared as \citet{yang2006d}, followed by an {English} translation by Liberty Lidz, incorporating improvements made in consultation with the author, and published as \citet{yang2009}. 

The structure of this outline largely follows that of He \& Jiang’s description of Naxi. Some aspects of its treatment of phonetics and phonology leave room for further refinement. Among consonants, uvular and retroflex stops are not granted phonemic
status. As for tone, the analysis is based on the four-tone system of Naxi, which imposes certain limitations. Informal discussions with the author in 2011 suggest that the dialect in question in fact possesses a greater number of tonal categories. 

Like many other researchers, Yang Zhenhong employs descriptive
tools developed for syllable-tone systems such as those of \ili{Sinitic} languages, which do not provide an entirely satisfactory framework for describing tone in Na, as explained in \sectref{sec:thenotationoftonalcategoriesinlexicalentries}. A similar challenge arose in the study of \ili{Pumi} (also known as Prinmi), where significant advances were only achieved when researchers with expertise in other types of tonal systems~-- such as those of \ili{Japanese} dialects~-- became involved (see
\citealt{ding2001,ding2006,jacques2011a}).


\subsection{Lexical materials}
\label{sec:dictionary2013}

{\largerpage}

\textit{An anthology of everyday words and expressions in the Mosuo language} \citep{zhibaetal2013} presents vocabulary and expressions organized by semantic field. %The authors are a~native speaker from the Lugu Lake area and a scholar who, at the time, was completing a Ph.D. dissertation on Mosuo language and culture at Yunnan University. 
%The book covering both the Yongning plain and the Lugu Lake area, with the Yongning plain as the primary research site (p. 2). 
The second author clarified (p.c.\ 2024) that the book was intended for a general readership rather than as a contribution to linguistic research.
% \footnote{\zh{“我们编写这本书的初衷是为了让更多的普通百姓看得懂,并非是本体的语言学研究。”}} 
Given this stated objective, it would be inappropriate to assess the work from a strictly phonetic or phonological perspective or to hold it to the standards of specialized linguistic studies. %For instance, the name of the sacred mountain Gemu /\ipa{kɤ˧mv̩˧˥}/ is transcribed as /\ipa{gə⁵⁵mu⁵⁵}/, with a~voiced initial (p. 17 and elsewhere), whereas the adjective /\ipa{dʑɤ˩\textsubscript{b}}/ ‘good’ is transcribed as /\ipa{tɕɑ¹³}/, with an unvoiced initial. Some phonemes, such as uvulars, do not appear in the transcriptions. 
For the purpose of the present review of Na lexicography, it suffices to note that the volume was not designed as a reference work for academic research.

Dictionaries on neighbouring Naxi provided a brilliant example to be emulated \parencite{heetal2011,pinsonetal2012}. The \textit{Na (Mosuo)~– English~– Chinese dictionary} produced as part of the same language documentation suite as the present work is presented in section \sectref{sec:dictionary}.

\subsection{Collections of oral literature: Na ritual texts}
\label{sec:collectionsoforalliteraturenaritualtexts}

In the Yongning plain, {Tibetan} Buddhism (of the Gelugpa school) coexists with a~local tradition of ritual practitioners known as \textit{Ddabe} \ipa{dɑ˧pɤ˧}. Unlike among the
Naxi, rituals are not written down, although certain written characters are used for calendrical calculations in divination (\citealt{yang1985}; \citealt[163-189]{lidazhu2015}). The absence of a~written tradition partly explains why Na rituals have received
less scholarly attention than those of the Naxi. The relatively small size and limited socio-political influence of Yongning, especially in comparison with Lijiang, as well as the difficulty of access to the area until the late 20\textsuperscript{th} century, have further contributed to the lack of documentation. 

A bilingual collection published by a~native speaker \citep{azeming2013} presents ritual texts in a fourfold parallel format: (i)~a phonetic approximation of each syllable by means of a~Chinese
character, (ii)~a transcription in the International Phonetic Alphabet, (iii)~a syllable-for-syllable Chinese gloss,
and (iv) a~full-line translation into Chinese. 
(Most lines contain five, seven, or eight
syllables.) This volume were best reserved for readers already familiar with the language, who will
be able to identify part of the glosses by making allowance for nonstandard transcription habits influenced by
the \textit{Pinyin} system used for romanizing Chinese. For instance, /\ipa{p}/, /\ipa{t}/, and
/\ipa{k}/ appear to represent the aspirated phonemes /\ipa{pʰ}/, /\ipa{tʰ}/, and /\ipa{kʰ}/,
as in \textit{Pinyin}. Thus, the word for ‘white’, transcribed as
/\ipa{puə}/ in this book (p. 4), in fact has an aspirated initial in Na. Tone is not indicated.

During the 2010s, Latami Wangyong (Latami Dashi) launched a~documentation programme with the goal of publishing an extensive collection of translated and annotated rituals, accompanied by video recordings.  
This work has resulted in the online release of a set of rituals in 2025 in the Pangloss Collection (\pandoi{0009179} through \pandoi{0009216}). 

\subsection{Liberty \citet{lidz2010}, \textit{A descriptive grammar of Yongning Na (Mosuo)}}
\label{sec:lidz2010}

{\largerpage}

By far the most comprehensive description and analysis of Yongning Na to date is Liberty Lidz’s
Ph.D.\ dissertation, \textit{A descriptive grammar of Yongning Na (Mosuo)} \citep{lidz2010}. This work focuses on the variety of Yongning Na spoken in the village of Loshu \ipa{lo˧ʂv̩˩} (Luoshui \zh{落水}),
on the shore of Lugu Lake. Based on in-depth fieldwork, the dissertation provides a~detailed account of the morphosyntax
of the language and includes 150 pages of
transcribed and annotated narratives.

Regarding tone, it has been noted that “[t]he tonal system of Luoshui Narua calls for further
analysis. Surface phonological tones are transcribed employing three tonal levels, but a~reanalysis
in terms of two levels would seem possible in many cases. Mention is made of prolific {tone sandhi}
processes, but tantalisingly, these processes are not elaborated on” \citep[69]{dobbsetal2016}. It is not unusual for reference grammars to leave open some issues of {prosody}, such as the status of tone or stress \citep[26]{zeitoun2007}. In the case of Yongning Na, a tacit division of labour has emerged between Liberty Lidz and myself, whereby the task of conducting a detailed investigation into tone has been left to me. The present volume is dedicated to this endeavour.


\section{The research programme and its theoretical underpinnings}
\label{sec:chronologyofthestudyelicitationproceduresandonlinematerials}

\subsection[The aim: A detailed description of a level-tone system]{The aim: Describing a~level-tone system of East Asia}
\label{sec:theaimofthepresentvolumetoprovideanindepthdescriptionofaneastasianleveltonesystem}

Tonal alternations permeate numerous aspects of the morphosyntax of Yongning Na. Crucially, they are
not the outcome of a~small set of phonological rules, but rather of a complex system of morphotonological patterns that are restricted to
specific morphosyntactic contexts. As Guillaume Jacques (p.c.\ 2009) has observed, in Yongning Na~-- as in
other \ili{Naish} languages and in \ili{Pumi}~--, irregular morphology mostly consists of irregular \isi{morphotonology}. The richness of this aspect of the language warrants a~book-length study, applying what \citet[152]{scherer1885} described as “the old philological virtue of exactitude”
 to this intricate system, in pursuit of a comprehensive and rigorous account.

A search for book-length descriptions of similar systems in other languages suggests that such
reference works remain relatively scarce, even within the extensive
\ili{Bantu} branch of the Niger-Congo language family, renowned for its rich \isi{morphotonology}.

\begin{quotation}
  Theoretical linguistics is primarily concerned with advancing the theoretical enterprise, and
  tends to produce short pieces~-- chapters, articles, squibs. It does not have the writing of
  grammars as a~priority, and few of the theoretical grammars of African languages written during
  the heyday of transformational theory during the 1960s and 1970s have stood the test of time. ({\dots})
  Are there enough grammars of sub-Saharan African, especially {Bantu}, languages? The answer is
  no. ({\dots}) The overwhelming impression is that of the small number of real grammars, and the number
  is not increasing.~\citep[xxiii--xxiv]{nurse2011}
\end{quotation}

Beyond the issue of sheer numbers, there is also a question of breadth and depth of coverage in existing book-length descriptions. Linguistic fieldwork consists in “going into a~community
where a~language is spoken, collecting data from fluent native speakers, analysing the data, and
providing a~comprehensive description, consisting of grammar, texts and dictionary”
\citep[12]{dixon2007}. This holistic approach entails decisive advantages for understanding a language as a~whole, as it functions in its social setting. However, the breadth of such studies can occasionally come at the expense of depth in specific domains, such as tone. “No variety of Bambara has heretofore been the object of a~systematic tonological
description aiming at full coverage”\footnote{\textit{Original text:} aucun parler bambara ({\dots}) n’a jusqu’ici fait
	l’objet de ce qui mériterait d’être considéré comme une description tonologique systématique visant
	à l’exhaustivité.} (\citealt[199]{creissels1992}; see also \citealt{clements2000};
\citealt{hyman2005a}). “Even the ‘well described’ languages often suffer from a~lack of examples, by which to test the descriptive or theoretical
claims” \citep[xxiii]{nurse2011}. Similar concerns arise in studies of tone systems from various regions of the world. In Mesoamerican linguistics, for instance, the following call for more extensive documentation of \isi{tonal morphology} underscores the urgency of such research:

\begin{quotation}
	We need full paradigms in grammars of tonal languages, not just rules, abstract representations or examples of how a~given form is used in a~natural context. This is a~cordial invitation to descriptive linguists
	to enrich the field with new data on inflection. It matters. It matters in a~time
	when most languages with complex morphology are dying. By doing so, we will
	be paying tribute both to the languages and to the field of linguistics, because in
	a~hundred years from now, when all of us are gone, it will only be our data that
	shall remain for future linguists to continue increasing our understanding of our
	human languages. \citep[134]{palancar2016}
\end{quotation}

In the field of \il{Sino-Tibetan}Sino-Tibetan studies, tone systems have occasionally been misrepresented, as noted by \citet{sun2003b} for \ili{Tibetan} and by \citet[188]{post2015} for Tani and languages of Northeast India more generally. Fortunately, languages
with level-tone systems and rich morphotonological structures are now receiving increasing scholarly attention. Two grammars of
\ili{Pumi}, a~language employing two tonal levels, have been published \citep{daudey2014,ding2014}. The present study of Yongning Na tone is intended as a~contribution to this broader movement within Sino-Tibetan linguistics. 

At this stage, the primary goal is to establish a~precise description, which serves as the necessary basis for further research. Consequently, this volume includes numerous tables that lay out the paradigms in full. There is room for some progress in terms of economy of description: for instance, by identifying a~set of morphotonological combination patterns as default for a~given category of morphemes and reformulating the data for other categories as \textit{identical to the standard pattern, except for{\dots}} rather than presenting all paradigms in tabular form. The long-term research agenda includes modelling the \isi{morphotonology} of Yongning Na through computer implementation (finite-state modelling), a prospect discussed in the conclusion (Chapter~\ref{chap:conclusion}).


\subsection{Theoretical backdrop}
\label{sec:theoreticalbackdrop}

This study is \textit{theory-informed}, not \textit{theory-driven}: the primary objective is not to use selected data to engage in current phonological or morphological debates, but rather to provide an in-depth description of the language as it functions. This aim is widely shared among linguists, and is ultimately more significant than theoretical divergences. A key guiding principle is the utmost caution in avoiding Procrustean models. The theoretical backdrop to the present research is intended to remain as unobtrusive as possible, in keeping with the priorities of language description and analysis. Linguistic models will be invoked only as necessary. 

In essence, the method used here adheres to the fundamental principles of classical structural-functional phonology, as
set out in standard handbooks of phonological description (e.g.\ \citealt[15,
  34--47]{martinet1956}). It is difficult to extract a single quotation that would neatly summarize
the core tenets of this approach. Martinet devoted an entire volume to articulating \textit{A
  functional view of language} \citep{martinet1962}. The following excerpt is drawn from another of his {English}-language works, \textit{The internal conditioning of phonological systems}.

\begin{quotation}
  A~dynamic conception of language presupposes that we do not deal with it as we would with a~dead
  body in the morgue, but try to look at it as a~means of satisfying some of the human needs, and
  essentially that of communication. In other terms, it derives from a~functional view of language~({\dots}). 
  [E]xperience has shown that even if language is often used for the satisfaction of
  other needs as, for instance, that of communion, it is, in the last analysis, mutual understanding
  that determines the choices of the speakers. ({\dots}) At every point in time, with every speaker,
  what is said and how it is said will show a~balance between the desire to communicate, and
  inertia, be it individual, i.e.\ reduction of energy, or social, i.e.\ preservation of traditional
  forms at the expense of personal comfort and communicative efficiency.~\citep[2–3]{martinet1996}
\end{quotation}

A~major source of linguistic change lies in the constant competition between two opposing forces: phonological integration, on the one hand, and phonetic economy, on the
other. Phonological integration tends to \is{gap-filling}fill structural gaps within phonological systems, whereas
phonetic economy tends to create such gaps. A~straightforward example can be drawn from tone. A system with five level tones (Top, High, Mid, Low, Bottom) could be considered phonologically economical, as it maximizes the use of tone to convey lexical contrasts, while also enabling immense possibilities for \isi{tonal morphology} and \isi{morphotonology}. However, such a system is phonetically uneconomical, as distinguishing among a~large number of level tones is perceptually difficult~-- for instance, differentiating sequences such as a Top followed by a High, a Top followed by a Mid, and a High followed by a Mid.

In synchronic description, attention to the conflicting factors that are constantly at play in speech communication leads to the adoption of the method advocated by
Martinet under the name of \is{dynamic synchrony|textbf}\textit{dynamic synchrony} \citep{martinet1990}. The primary focus remains synchronic, but the approach is enriched by observations on tensions within
the system, assessing which of the competing variants are \is{innovative (phonological form)}innovative and which are {conservative}. 

At present, little is known about the \is{comparative method (historical linguistics)}{diachrony} of level-tone systems in Sino-{Tibetan}. Case studies in {dynamic synchrony} can yield converging evidence on {diachronic} tonal changes and their conditioning factors. Ultimately, what is needed is an approach that formulates generalizations about sound
change that transcend any single language or language family. The goal is to establish an inventory of types of sound change and to arrive at an improved understanding of the
conditions under which they arise. Haudricourt characterizes such an approach as \is{panchronic phonology|textbf}\textit{panchronic}
(\citealt{haudricourt1940,haudricourt1973b}; see also \citealt{hagegeetal1978}). 
%\is{panchronic phonology}
Panchronic phonology derives its general laws inductively from a~typological survey of precise {diachronic} events whose analysis reveals the common conditions under which they occur. In turn, 
%\is{panchronic phonology}
{panchronic} laws can shed light on
individual historical developments. The premise is that, within the broader set of potential changes, the direction of evolution in a~given language is partly conditioned by the structure of its phonological system: which phonemic oppositions it maintains, the phonotactic constraints governing them, their role in morphophonology, and so on. (For examples, see \citealt{jacques2011a,lai_lenition_2023}.) 

A challenging mid- to long-term objective for historical linguistics will be to model the origins and evolution of level-tone systems with the same degree of precision as in studies of classical \is{tonogenesis}tonogenetic processes in \ili{Sinitic}, \ili{Austroasiatic} and \il{Tai-Kadai}Tai-Kadai. \is{tonogenesis}Tonogenetic developments in these families constitute a~success story of 
%\is{panchronic phonology}
{panchronic} phonology, demonstrating how a voicing opposition can be transphonologized into a~phonation-type opposition, a~tonal contrast, or a~vowel quality distinction \citep{haudricourt1965a,ferlus1979}. In Mon, for instance, the voicing contrast on initial consonants gave rise to two phonation types; 
in \ili{Vietnamese}, it led to an increase in the number of lexical tones; while in Khmer, it resulted in a proliferation of vowel qualities. 

Phonation-type oppositions have several phonetic cues. In addition to \is{phonation types}phonation type proper, they include differences in pitch, as well as differences in vowel articulation. This was already noted for a~{conservative} variety of Khmer by \citet{henderson1952}, and is confirmed by experimental studies of Mon (\citealt{abramsonetal2015} and references therein). One or the other of these phonetic cues can become dominant: either vowel quality stabilizes as the new distinctive property (as in Khmer), or the distinctions become tonal (as in \ili{Vietnamese}). A~major structural factor in this bifurcation is whether the language already possesses tones at the time of transphonologization. If the language already has tones, the loss of voicing
oppositions induces a~split within the tonal system; otherwise, it leads to a~vowel quality contrast, producing a~two-way split in the vowel inventory. Comparative studies of numerous East and Southeast Asian languages corroborate this model while also offering opportunities for refinements through the study of \is{tonogenesis}tonogenetic processes in progress (e.g.~\citet{brunelle2012} on \ili{Cham}; \citet{kirby2014} on \ili{Khmer}; \citet{yangetal2015} on \ili{Lalo}; and \citet{pittayaporn.kirby.laryngeal2017} on \ili{Cao Bằng Tai}). 

{\largerpage[-1]} % Added on April 21st, 2025

Panchronic phonology holds similar promise for the study of the area of Sino-Tibetan to which Na belongs. A landmark study of \isi{tonogenesis} in \ili{Rma} ({Qiāng} \zh{羌}), \ili{Pumi} (Prinmi), and \ili{Tangut} concludes that ``far from being an exotic case of \isi{tonogenesis} without concomitant
segmental change ({\dots}), [the primary tone split found in Rma, Prinmi, and Tangut] is strikingly similar to the classic model described for the development of tones from \ili{Old Chinese} to \ili{Middle Chinese} \citep[]{sagart1999a} and also from proto-\il{Vietic languages}Vietic to \ili{Vietnamese} \citep[]{haudricourt1954b} in which final stops induced low-rising tones and smooth syllables
developed high tones'' \citep[8]{Sims2021_tonogenesis}. 

Panchronic phonology is close~-- at least in my view~-- to \textit{\is{evolutionary phonology}evolutionary phonology}, a~theoretical framework that seeks to integrate insights from historical phonology and experimental phonetics. This approach aims to establish “a~general link between neogrammarian discoveries, advances in modern phonetics, and phonological theory” \citep[xiii]{blevins2004}. Its emphasis on the phonetic bases of change, following \citet{ohala1989}, encourages a~mutually enriching dialogue between experimental phonetics and historical phonology. Like 
%\is{panchronic phonology}
{panchronic} phonology, evolutionary phonology represents a long-term research programme that holds promise of an increasing degree of
precision and explicitness in modelling historical change. 

While some differences exist in their stated principles and methodologies (see \citealt[601]{labov1994}, \citealt{andersen2006}, \citealt{iverson2006}, \citealt{mazaudonetal2007} and \citealt{smithetal2008}), I would argue that their common objective~-- explaining
synchronic states in terms of the processes that give rise to them and formulating general laws of
sound change~-- is ultimately more significant than their theoretical divergences. In practice, scholars working within \is{panchronic phonology}{panchronic} phonology, \is{evolutionary phonology}evolutionary phonology, or other
approaches to the typology of sound systems and sound change share the same fundamental goals. 

This may seem far too much {diachronic} background for a~synchronic monograph. However, I believe that synchronic descriptions such as the present study benefit from being situated within a {diachronic} research agenda. In historical linguistics, as in phonetics/phonology, “the devil is in the detail” \citep{nolan2003}, and the patient, fine-grained analysis of tonal patterns can constitute a good basis for developing a~feel for the {diachronic} evolution of tone systems. Chapter~\ref{chap:yongningnatonesinadynamicsynchronicperspective} is devoted to examining synchronic \isi{variation} and {diachronic} change from this perspective.


\section{Field trips and collaboration with consultants}
\label{sec:collaborationwithconsultants}

{\largerpage[-1]} % Added on April 21st, 2025

The present results are based on data collected since 2006. Four field trips to the village of Alawua were conducted from 2006 to 2009. Excluding time spent on travel and organizational tasks, the duration of these stays was 50 days in 2006, 35 days in 2007, 58 days in 2008, and 40 days in 2009. From July 2011 to October 2012, I had the wonderful opportunity of staying in China for long-term fieldwork. As my main consultant had by that time moved to the town of Lijiang to care for a granddaughter, I was based in this town as well, working with her for an average of two hours a~day. In 2013, 2014, 2015, 2016, 2018, and 2024, I made short field trips to Lijiang and Yongning, continuing to work mainly with the same consultant.

Standard procedures for data collection, as reflected in the ‘Method’ section of papers in phonetics journals, tend to avoid any mention of personal familiarity between the investigator and the subjects. Such mentions are not merely deemed irrelevant but even suspect, since exchanges with consultants beyond providing instructions are viewed as a~contaminant: a~threat to the objectivity of the experiment. “The subjects were unaware of the purpose of the experiment” is considered a~commendable state of affairs. However, linguists with experience in collaborating with consultants, whether in a language lab or in the field, are well aware of how profoundly the relationship established with the consultant influences research. A~close examination of data collection in language laboratories suggests that key aspects in consultant selection and the formulation of instructions are often overlooked \citep{niebuhretal2015}. %Rarely are concerns raised about the bias introduced by the use of professional linguists or multilingual students as subjects, despite well-documented differences across speakers: clearly, different people have different abilities (see e.g.~\citealt{audibertetal2008}).

% Earlier formulation: 
% Worries are seldom voiced about the bias
% introduced by the use of professional linguists, or multilingual students, as subjects, despite the existence of well-documented differences across speakers: clearly, different people have different abilities (see e.g.~\citealt{audibertetal2008}). 

Of course, different research purposes call for different data collection methods, and it
would be thoroughly unreasonable to expect all investigators to develop personal familiarity
with the subjects of their research. Nevertheless, in the process of
describing a~language, mutual understanding between the investigator and consultants is essential. In this light, the personal details presented here are not mere fieldwork anecdotes: they constitute relevant information on data collection.
%\footnote{Further thoughts about data collection are set out in \citet{niebuhretal2015}.}


\subsection{First steps in the search for consultants}
\label{sec:firststepsinthesearchforconsultants}

{\largerpage}

On my first field trip, Mr.~Latami Wangyong \zh{拉他咪王勇} (Latami Dashi \zh{拉他咪达石}),
%{\kern-5pt}, 
a~native speaker of Na and a~researcher in ethnology based in the Ninglang county seat, to whom I had been introduced by Picus Ding, accompanied me to his family’s house, where I was invited to reside during all my stays in Yongning. He volunteered to work as a~consultant (his code in the database of \ili{Naish} speakers is M18). Mr.~Latami has near-native command of both \il{Mandarin!Southwestern}Southwestern {Mandarin} and \il{Mandarin!Standard}Standard (Beijing) Mandarin. It was immediately obvious to both of us, when we began an elicitation session, that his long years of daily practice of {Mandarin} had taken their toll on his proficiency in Yongning Na.

He offered to help find a~speaker who had a~relatively homogeneous linguistic experience, having lived continuously in Yongning since childhood. For my part, I wanted to work with a~male speaker for a~technical reason: spectrographic and electroglottographic analysis, two techniques that I planned to use, are easier to perform on data from male speakers. We also agreed to look for a~speaker whose age ranged between 35 and 65 years old. Younger speakers have limited command of the language; as for the oldest speakers, they are often the most proficient in Na, but at a~certain age, speech becomes less audible and communication with strangers more difficult.

Mr.~Latami therefore set out to find a~suitable consultant in the neighbourhood. The procedure, as he narrated it to me, was as follows. He invited a~candidate over to his place after dinner, treated him to liquor and sunflower seeds, and launched a~conversation about how the language was being lost by the younger generation. He then explained that there was currently a~foreigner staying in the house who wanted to study and record everyday language, and he asked if the person would agree to work as a~language consultant.


Several acquaintances were thus invited, said they would consider the proposal, and eventually declined. There may have been a~number of reasons for this. Mr.~Latami’s perspective was that, being unfamiliar with the nature of linguistic fieldwork, they were suspicious of potential misuse of the information they would provide and wary of the reputational damage they might suffer if their name became associated with debatable materials about Na language and culture. There are enough examples of ludicrously simplified depictions of Na culture produced to cater to the tourist industry (as reviewed in Appendix~\ref{chap:historyanthropologysociology}, \sectref{sec:presentdaysociologicalstudiestheimpactoftourismsincethe1990s}) to justify their cautious stance.

On the other hand, these individuals had known Mr.~Latami since he was a~child and had reason to trust that he would not collaborate in a~research project that might harm the community’s image. In his own research on Na culture, Mr.~Latami takes care to gather viewpoints from a~range of relatives and acquaintances. After finalizing a~draft of one of his books, he circulates copies to Na people literate in Chinese and solicits their feedback. Only after incorporating their criticisms, corrections, and comments into the final version does he proceed with publication. While this offers no absolute guarantee against resentment among community members concerning the book’s contents, it may help to allay suspicions.

{\largerpage}

The reluctance of these Na speakers to participate in data collection provides insight into traditional Na society as a~highly conservative agricultural society where deviant behaviour is met with sharp reproach. This has linguistic consequences: “the strong networks typical of rural life” \citep[379]{milroyetal1985} not only favour archaism but also foster innovations that increase morphological complexity~-- a~process opposite to what occurs in cases of creolization. For instance, reflecting on the development of person agreement marking on complementizers in \ili{Bavarian German} (as reported by \citealt{bayer1984}), Peter Trudgill speculates that such developments occur only in tightly-knit rural communities. The example given is (\ref{ex:minga}), where \textit{ob} ‘whether’ receives the second-person agreement marker \textit{-st} \citep[82, 112--113]{trudgill2011}.

\begin{exe}
  \ex
	 \label{ex:minga}
	 \gll {\dots} obst du noch Minga kummst\\
     {} whether you-\textsc{sg} to Munich come\\
     \glt ‘whether you are coming to Munich’
\end{exe}

Returning to Yongning Na, the development of the rich \isi{morphotonology} described in the present volume may have been favoured by the same social factors that initially led potential consultants to decline sharing their knowledge of Na with me. 
%I~am aware that this link is hypothetical and may resemble a~“just-so story”: an \textit{ad hoc} and unverifiable speculation. )
(Some initial observations on the social dynamics of Yongning Na \isi{morphotonology} are set out in Chapter~\ref{chap:yongningnatonesinadynamicsynchronicperspective}.)

Mr.~Latami Dashi’s mother supported her son’s efforts to reassure potential consultants, explaining that they need not be intimidated by the tasks proposed to them. She clarified that the goal was not to collect folklore but to study the everyday language, and that the initial stage of the work was as simple as saying words such as ‘head’ and ‘hand’. While she did not succeed in persuading others, she eventually convinced herself and volunteered as a~consultant.


\subsection{Main language consultant}
\label{sec:mainlanguageconsultantmrs}
%\largerpage

{\largerpage}

My Na language teacher is Mrs.\ Latami Daeshilamu /\ipa{lɑ˧tʰɑ˧mi˥ ʈæ˧ʂɯ˧-ɬɑ˩mv̩˩}/. (Her code in the database of speakers of \ili{Naish} languages is F4.) She was born in 1950 into a~family of commoners~-- the majority group among the Na, distinct on the one hand from the chieftain's family, which constituted the nobility, and on the other hand from the serfs. Her birthplace is the hamlet called Alawua /\ipa{ə˧lɑ˧-ʁwɤ˧}/, 
     %% Footnote commented out on April 27th, 2025.
%\footnote{For the sake of simplicity, this noun is provided here in surface phonological transcription. Its underlying form is //\ipa{ə˧lɑ˧-ʁwɤ\#˥}//, with a~{floating} High tone. This tonal category is analyzed in \sectref{sec:afloatinghtonewithcomparativeevidencepointingtoitsorigin}.} 
close to the monastery of Yongning, called
\textit{dgra med dgon pa} in {Tibetan}, a~name rendered in Chinese as \textit{Zhāměisì} \zh{扎美寺}. The
full address is: Yúnnán province, Lìjiāng municipality, Nínglàng Yí
autonomous county, Yǒngníng district, Ālāwǎ 
village (\zh{云南省丽江市宁蒗彝族自治县永宁乡阿拉瓦村}). 
%This place is referred to in this volume as ‘Alawua'.
The founding of this hamlet is recounted in the narrative \textit{Elders3} \pandoi{0004532} (for information on the narratives and other online resources, see \sectref{sec:transcribedandtranslatednarrativesandphonologicalmaterials}). My teacher later established a~home of her own in a~neighbouring hamlet, slightly closer to
the road leading to the Yongning marketplace.

My teacher is attached to traditions, closely associating them with the teachings of her grandmother, whom
she remembers as an outstanding character who tactfully managed a~large household. In narratives,
she refers to her grandmother as /\ipa{ə˧si˧}/, ‘great-grandmother, ancestor of the third
generation’~-- the {term of address} used by her own children, thereby pointing her out as a~model to the next
generation. Locally, my teacher is regarded as a~connoisseur of Na customs: over the past two decades or so, villagers experiencing doubts about ceremonial procedures have come to her for guidance.

At the same time, she is acutely aware of the profound transformations Na society has undergone since her childhood and does not cling to a bygone past. She is an open-minded character, gaily deriding in retrospect the prejudices that once prevailed in Na villages. For instance, in an account of the introduction of vegetables such as courgettes and eggplants in Yongning, she recalls that distrustful and indignant villagers would warn, “Don't eat those: they are grown in shit!” Yet, these new crops were eventually adopted, along with traditional Chinese methods for fertilizing soil. (See the narrative \textit{Housebuilding2} \pandoi{0004549}.)

In her childhood, my teacher was one of the actors in a~film about the Na and their unusual matrilineal and matrilocal family structure %“without fathers or husbands” 
(discussed in Appendix~\ref{chap:historyanthropologysociology}, \sectref{sec:anthropologicalresearchthefascinationofnafamilystructure}): \textit{‘A-zhu’
  marriage among the Naxi of Yongning}.\footnote{I was not able to access this film. Chinese title: {\kern-3pt}\zh{《永宁纳西族的阿注婚姻》}{\kern-4pt}. Black and white. Duration: about 60~minutes (six film portions of ten minutes each). Production date: approximately 1966. Advisor: Qiū Pǔ
  \zh{秋浦}. Scenario: Zhān Chéngxù \zh{詹承绪} and Yáng Guānghǎi \zh{杨光海}. Director: Yáng
  Guānghǎi \zh{杨光海}. Photography: Yuán Yáozhù \zh{袁尧柱}. Sound recording: Zhào Déwàng \zh{赵德旺}. 
  Animations: Zhèng Chéngyáng \zh{郑成杨}. Narration: Zhōu Qìngyú \zh{周庆瑜}. Summary: “Before Liberation, the Naxi of the people's commune of Yongning, in Ninglang Yi Autonomous County, lived in a feudal society, but they preserved distinctive characteristics of primeval matrilineal societies. They had matrilineal households in which the maternal side constituted the core of the family. They retained a style of marriage (‘A-zhu’ marriage) in which men did not take wives into their family, and women did not marry into another family. This documentary film records this form of matrimony according to the facts.” \textit{Original text:} \zh{在云南省宁蒗彝族自治县永宁公社的纳西族,解放前处于封建领主社会,但长期以来还保存着原始母系社会特征,保存着以母系为核心的母系家庭,保存着男不娶,女不嫁的“阿注婚姻”。男阿注到女阿注家过夜,晚上来白天走的“半同居”婚姻生活。本片对这种婚姻形式,特点和母系家庭作了如实的记录。} The film is part of a~series about “ethnic minorities” initiated in 1957: \zh{少数民族社会历史科学纪录片}.} 
  
Later, one of her sons became
an anthropologist specializing in Na society, bringing her into contact with many of his 
colleagues. She has thus witnessed how the Na of Yongning have become an object of curiosity and fantasy, and
how Na culture has been folklorized for the promotion of the tourist industry.\footnote{One example among many is a~report for the {French} tabloid
  \textit{Paris Match}. The magazine's special issue “China is changing” («~La Chine change~», May 2001) contained no fewer
  than ten pages dedicated to the “Mosuo” (pp. 52--61), including an interview with Latami Dashi and
  his mother, Latami Daeshilamu (p. 60).} Her experiences and reflections have shaken some of the
beliefs passed down to her by her grandmother, such as Buddhist
faith. While conscientiously performing the prescribed rituals in her daily life, her belief in Buddhist teachings such as reincarnation has waned, though this has not diminished her commitment to the ideals of benevolence and respect for others. The narratives recorded reflect her awareness of the cultural relativity of the  %waning 
customs and traditions of the Na, to which she
nonetheless remains deeply attached.

\begin{photofigure}[t]
	\caption{The main language consultant, Mrs.\ Latami Daeshilamu (\ipa{lɑ˧tʰɑ˧mi˥ ʈæ˧ʂɯ˧-lɑ˩mv̩˩}), shopping at the Yongning marketplace. Spring 2008.}
	\includegraphics[width=0.7\textwidth]{figures/Ama08355.jpg}
\end{photofigure}

{\largerpage}

Unlike more traditional parents, who viewed the monastery as the most prestigious prospect for
boys, she encouraged her children~-- girls as well as boys~-- to study in the Chinese school system,
which she considered a~better gateway to a life free from daily toil in the fields. All four of her 
children have since found employment outside Yongning: one in the county town of Ninglang, two
in Lijiang, and one in faraway Shenzhen (Guangdong). Although she lived continuously in the village
from birth and hardly ever left the Yongning plain (and, indeed, seldom left her own village)
until she moved to Lijiang in 2010 to care for a newborn granddaughter, she always had a keen awareness of the wider world.

Her belief that, beneath differences in local customs, the human heart is the same everywhere occasionally emerges in the narratives she agreed to record along the course of our 
collaboration. She often draws parallels between the situations depicted in her
stories and those of the present day. For instance, reflecting on the aspirations of apprentice monks to find
a~good master, she pointed out the {analogy} with my own study of Na, which likewise depended on the guidance of dedicated teachers~-- just as her grandson required good instructors at his university in Kunming. Following her instruction, I
have always addressed her as ‘mother’ (\ipa{ə˧mɑ˧}) and have been the grateful recipient of her affectionate care
throughout my stays. She is a~model of tact, expertly fine-tuning relations both within the family and
beyond. Despite her declining health and the heavy workload of a~mother and grandmother, she has been a~patient and encouraging teacher. While she may well take pride in having raised four children under harsh
circumstances, she possesses a sharp sense of humour and has never been one to pose as a guardian angel,
muse, or Madonna (to borrow Baudelaire’s impassioned wording: “l’Ange gardien, la Muse et la Madone”).

%\largerpage
She was a~stutterer in adolescence but later overcame this difficulty. I might not have
noticed had I not been informed, but I now interpret the rare cases of stammering in
recorded audio documents as a remnant of this earlier difficulty. She is also known in the community for speaking fast. Finally, although she has never suffered from any major
otorhinolaryngological ailments, she has noticed changes in her voice over the years and can no longer sing the high-pitched songs of the Na. The reason is partly social: in local custom, singing is regarded as an activity for young people. The voices of singers over forty (or fifty at the latest) are deemed unattractive, and it is unusual for women past fifty to sing songs. When my teacher agreed to perform a~song~-- understanding its value for language documentation and recognizing that most younger Na speakers no longer learn traditional songs~-- it became apparent that lack of practice over the years had left her unable to
sing them.

She never travelled to other Na villages
beyond the Yongning plain, such as Pujjo \ipa{pʰv̩˧dʑo˧} 
%\footnote{For simplicity, this noun is provided here in surface phonological transcription. Its underlying form is //\ipa{pʰv̩˧dʑo\#˥}//, with a~{floating} High tone (analyzed in \sectref{sec:afloatinghtonewithcomparativeevidencepointingtoitsorigin}).} 
(Labai \zh{拉伯}) or those in Muli county. She speaks a~little
\il{Mandarin!Southwestern}Southwestern Mandarin and incorporates common {Chinese} words into her Na speech, but
her proficiency remained very limited until she moved to Lijiang in 2010 and found herself in
a~predominantly Chinese-speaking environment. Apart from the initial vocabulary elicitation, which
was conducted in Chinese, fieldwork was carried out entirely in Yongning Na, with the consultant providing explanations in her own
language without translation into Chinese. While this inevitably slowed the elicitation process compared to working with a~bilingual consultant, monolingual fieldwork has its own advantages,
allowing me to develop a~better command of the language.

In addition to phonological materials, the set of texts by consultant F4 expanded from a single traditional story to over one hundred monologues, addressed to the investigator, on topics of her own choosing. She does not consider herself a~skilled performer of oral literature, having never been formally trained as a custodian of oral traditions~-- a~role reserved for men in the local ritualist tradition (called \textit{Ddabe} \ipa{dɑ˧pɤ˧}, and related to the Naxi \textit{Dobbaq} \ipa{to˧mbɑ˩}: see \citealt{fangetal1995,lidazhu2015}). However, she yielded to the investigator's eagerness for narratives and audio recordings in general. Aware of my interest in history, ethnology, and sociology, she spoke about life in Yongning in \textit{the old times}, a~conveniently vague era encompassing her childhood and youth. The result is a~collection of monologues that belong to ordinary, casual speech as opposed to codified performance of oral literature (following the distinction set out in \citealt{dournes1990}). 

At the same time, she is keenly aware of the stakes involved in narrating {oral history} \citep{milan2024_gathering}, including the didactic and mobilizational power of personal stories in “an {oral history} regime” \citep[100-105]{bulag2010}. She therefore exercises caution, avoiding topics that she perceives as sensitive. This results in a~degree of stylistic evenness and uniformity, limiting both the range of speaking styles and the diversity of content. No claim is made that this corpus (presented in some detail in \sectref{sec:transcribedandtranslatednarrativesandphonologicalmaterials}) constitutes a balanced dataset in any sense. From the perspective of \isi{morphotonology}, the primary objective was to assemble a sufficiently rich dataset to confirm at least some of the tonal combinations obtained through elicitation, cross-checking them against materials that, in the phonetician's admittedly coarse classification, fall under “spontaneous speech” as opposed to phonological elicitation.

\subsection{Other language consultants}
\label{sec:otherlanguageconsultants}

{\largerpage}

During my first field trip (2006), as I began working with Mrs.~Latami (consultant F4), I reflected on possibilities to extend the study to additional speakers, mindful of the textbook arguments for working with multiple informants: 

\begin{quotation}
  Data from varied sources can guard against distortions resulting from dressage, the observer’s
  paradox, faulty questioning, or prescriptive influences of one individual’s idiolect. Working
  with several speakers will provide the researcher with points of comparison so that he or she can
  learn to distinguish between reliable and unreliable data.~\citep[180–181]{chelliahetal2011}
\end{quotation}

In the household where I stayed, two family members lived in Yongning year-round: Mrs.~Latami and
her daughter-in-law Gisso, the wife of her second son. Gisso (born in 1973)
agreed to assist with basic elicitation. As shown in a documentary film by Franck Guillemain about my linguistic fieldwork in Yongning \citep{guillemain_Na_2020}, 
%\footnote{CNRS~-- Service audiovisuel d'ARDIS (UAR2259). \textit{À l'écoute des Na de Yongning}. 2020. [Video.] Canal-U. \url{https://doi.org/10.60527/k0sw-n830.} In French. Also available in Chinese: \zh{倾听~-- 法国语言学家米可永宁摩梭语采风记录}. \url{https://www.canal-u.tv/149717}.} 
Gisso excels at summarizing ongoing activities in a single word or short phrase and at describing the use of farming tools and implements. Her precise, confident elocution is a real asset to the linguist. She does not hesitate to correct my grammatical errors, enabling me to move beyond basic intelligibility to a more precise command of the language. 

However, Gisso declined to record continuous texts such as narratives or dialogues, explaining that she did not feel up to the task. Unlike Latami Daeshilamu,
who, after a~few days of vocabulary elicitation, agreed to record a~narrative, Gisso maintained her
initial decision, providing only short responses to my questions, apart from a~few brief songs recorded in 2007. 

Both Mrs.~Latami and Gisso had limited spare time, so I would work with one or the other depending on who was available. One day in December 2007, when both were occupied, Gisso asked her niece to “replace”
her for a~work session, offering an opportunity to get insights into the speech of a younger-generation speaker. Qiddeu (F6), born in 1987, was a high school student in the Ninglang county
seat, returning home only for holidays. Vocabulary elicitation revealed that her lexicon was limited 
and that her phonological system was highly simplified, closely aligning with Mandarin Chinese. These observations are discussed in 
a~joint book chapter with Latami Dashi: “A description of endangered phonemic oppositions in Mosuo
(Yongning Na)” (\citealt{michaudetal2011}; Chinese translation: \citealt{mikemike-alexis-michaudetal2010}).

In 2008, two additional speakers were recorded. The first was Mr.~Ho Jjacee \ipa{ho˧dʑɤ˧tsʰe˥} (He Jiaze \zh{何甲泽}), hereafter M21, born in 1942. A~retired cadre (\textit{gànbù} \zh{干部}), he had spent two years in Kunming and three in Yongsheng. However, elicitation tasks proved
challenging due to hearing difficulties~-- he reported complete deafness in
one ear and severely reduced sensitivity in the other. His experience of various dialects also
contributed to linguistic variability beyond what would have been ideal for phonological consistency. In particular, eliciting stable tone patterns for {compound}
nouns proved unfeasible: patterns given in one session were sometimes rejected in another, only to be
reaffirmed later with full confidence. 

Later, M21’s youngest son, Ddeezzhi \ipa{ɖɯ˩ɖʐɯ˧}
(He Duzhi \zh{何独知}), born in 1974, kindly agreed to participate. His command of Na appears comparable to that of F5. Like F5, M21 and M23 declined to
record anything beyond vocabulary and isolated sentences. %This is interpreted as an indirect indication that they received some exposure to the oral traditions, and still take them seriously enough to consider that
%they are not to be treated casually. The speakers who accept most readily to tell stories are not
%necessarily those who are most familiar with them. On the contrary: some speakers who have a~strong
%footing in another language and culture (in the case of \ili{Naish}: Chinese) can tell simplified versions
%of traditional stories all the more easily as they mean less to them.

\subsection{Examination of transcribed texts and direct elicitation}
%\section{Elicitation methods}
\label{sec:elicitationmethods}
\label{sec:examinationoftranscribedtextsanddirectelicitation}

{\largerpage}

“Texts are the lifeblood of linguistic fieldwork. The only way to understand the grammatical
structure of a~language is to analyse recorded texts in that language” \citep[22]{dixon2007}. Following classical methods in linguistic fieldwork, observations from continuous speech (in this study, primarily narratives) are verified and further investigated through
elicitation. A~fair amount of transcribed narrative has been collected and transcribed~-- more than four hours in total~-- but a~considerably larger corpus would be needed to study the language’s tone system on
the basis of these texts alone. Not all possible tonal combinations in {compound} nouns occur in
the texts, and no amount of continuous speech would suffice to obtain all the combinations of numerals and classifiers required for the study
of {numeral}-plus-classifier phrases (Chapter~\ref{chap:classifiers}). Systematic elicitation was therefore carried out to investigate one area after another of Yongning Na's tonal grammar.

Larry Hyman, reflecting on his study of the tones of another {Sino-Tibetan} language (\ili{Thlangtlang
Lai}), makes the following observation:
%LaTeX symbol for maths: × is composed as $\times$
\begin{quotation}
  Clearly the speaker had never heard or conceptualized noun phrases such as “pig’s friend’s
  grave’s price”, “chief’s beetle’s kidney basket”~({\dots}). It{\linebreak} would not impress any psychologist, and
  it would definitely horrify an anthropologist. ({\dots}) However, when I need to get 3×3×3×3=81 tonal combinations to test my rules, the available data may be limited, or the
  language may make it difficult to find certain tone combinations. I am personally thankful that
  speakers of Kuki-Chin languages are willing to entertain such imaginary notions. It is most
  significant that the novel utterances are produced with the appropriate application of tone
  rules.~\citep[34]{hyman2007a}
\end{quotation}

%Command \noindent added to avoid having an indent. Proofreader suggestion: since this sentence continues the argument, it is better not to indent. 
{\noindent}Using this method, L.\ Hyman completed the elicitation of tonal data for Thlang\-tlang Lai within six hours
\citep[9]{hyman2007a}. Such swift progress is possible when the linguist is fortunate to work with
consultants who possess strong metalinguistic abilities, engage with linguistic reasoning, and
collaborate as colleagues. A well-known example is François Mandeville, a~speaker
of Chipewyan (Athabas\-kan family) who “possessed the extraordinary ability to dictate texts and to explain forms with
lucidity and patience” \citep[132]{li1964}. 

{\largerpage}

The relationship with consultants in Yongning Na was somewhat different. Over the
course of our collaboration, they developed an understanding of a~linguist’s interests and aims. In
particular, after dozens of hours of joint work, the main consultant became familiar with the investigator’s body
language. She would understand, for instance, a~repetition-beseeching glance upward from the laptop screen, or
volunteer an explanation when a~lengthy pause suggested that the investigator was experiencing
doubt. However, she did not develop metalinguistic intuitions beyond recognizing full \isi{homophony} between two words. She did not reach the stage of identifying a word's tone with another word's, much less of naming a tonal category. Nor did she become a~collaborator in the sense of learning to transcribe her language or engaging in the inventory and analysis of phonemes and tonemes, i.e.~gaining training in linguistics. As a result, she did not provide feedback on analytical choices or on the transcriptions. 

This clarification is provided in light of the recommendation that “linguists should make it a~practice to explicitly
indicate whether the tonal categories have been recognised by native speaker consultants
and whether words cited have been confirmed in those categories by native speakers, or
whether those categories are the result of the linguist’s own analysis” \citep[638]{morey2014}. I~would be delighted to work with native speakers as fellow linguists and hope that there will be opportunities for this in the future.\footnote{I was fortunate to co-supervise the M.A.\ thesis of a~native speaker who conducted a~comparative study \citep{a2016} on the tone systems of her own dialect~-- Shèkuǎ \zh{舍垮}~-- and that of Alawua (Ālāwǎ \zh{阿拉瓦}), examined in the present volume. This was, of course, of the highest interest for me. However, A Hui's work did not include verification of my analyses for Alawua: she worked out the tone categories of her own dialect and compared them with transcriptions I provided, without questioning my notations. The two systems are fairly different: Shèkuǎ has only two tone levels (High and Low), whereas Alawua distinguishes High, Mid and Low.}

\subsection{The issue of cross-speaker differences}
\label{sec:theissueofcrossspeakerdifferences}

One of the findings from the systematic study of {compound} nouns (reported in Chapter~\ref{chap:compoundnouns}) was the high
degree of cross-speaker difference.

It is a~general observation that tone is particularly susceptible to \is{variation!dialectal}dialectal variation. Existing reports suggest that the greater Yongning area has not only some of the richest tone systems within the Naish-speaking
area but also the greatest dialectal diversity. In this light,
differences observed within the same village, and even within the same family, did not come as
a~great surprise.

Another dimension of diversity is the gap between age groups created by the ongoing shift to
Chinese. Mrs.~Latami’s four children are more proficient in Chinese than in Yongning Na. All four left Yongning for
work. Two married Han Chinese spouses with no command of Na; one married a~Na speaker from
a~different dialect area (Luggu \ipa{lo˧gv̩˩}, known in Chinese as Běiqúbà \zh{北渠坝}), but mutual comprehension difficulties led the couple to communicate
in Chinese instead. Another married a~Na spouse from Yongning (Gisso, F5) but worked in Shenzhen (Guangdong) and rarely returned
home. The next generation~-- her four grandsons and granddaughters~-- has even less command of Na, despite being cared for by Mrs.~Latami in their early years. 

{\largerpage}

There are also notable tonal
differences between Mrs.~Latami and her
daughter-in-law Gisso, despite their living under the same roof. Documenting the tone system of multiple
speakers is not simply a~matter of verifying data: each speaker's system must be analyzed independently, with comparisons drawn only at a~later stage. In addition to this analytical requirement, the investigation faced a practical limitation: transcribed texts are necessary to corroborate at least part of the patterns obtained
through systematic elicitation, but consultants F5, M21, and M23 declined to record narratives. Even if they had agreed, assembling a~sizeable collection of texts for each of the four speakers would have represented
a~formidable workload~-- not to mention the difficulty for the investigator of
keeping the four systems distinct.
% avoiding unwarranted carry-over of transcription habits developed
% when working on data from one speaker.

This raised a fundamental choice: whether to prioritize tonological depth or sociolinguistic breadth.

\subsection{Tonological depth vs.\ sociolinguistic breadth}
\label{sec:adilemmabreadthofcoverageofthetonesystemvsbreadthofsociolinguisticcoverage}

It was clearly not feasible to explore all areas of the tone system in equal
detail for all four speakers. This led to the following dilemma: either limiting the scope of
investigation~-- for instance, focusing on compound nouns, or on numeral-plus-classifier phrases~--
but eliciting data from a~broad sample of speakers to achieve decent sociolinguistic coverage;
or extending the study to more parts of the linguistic system, aiming for a~comprehensive
picture of the tonal grammar of a single speaker, with occasional extensions to other speakers.

The second approach was preferred: attempting as complete a~description as possible of the linguistic system. Work with Mrs.~Latami (consultant F4) appeared more promising because of her much stronger
proficiency in Na compared to the two younger consultants (F5 and M23) and her relatively stable
linguistic trajectory, in contrast to M21, whose long exposure to various Na dialects and to
Chinese had introduced additional complexities. Basing the study primarily on data carefully verified with Mrs.~Latami provided a~reliable foundation for later comparative work, including cross-speaker and cross-dialect studies. 

The ultimate research goal, conceived as a~collective endeavour, is to document in fine
detail the synchronic tone systems of a~number of research locations within the Na-speaking area and, on
this solid empirical basis, to conduct comparative analyses and gradually reconstruct the historical
evolution of these tone systems. Ideally, this would lead to a~full account of the origin and
development of tone systems in {Naish}, shedding light on the stages that led from a~non-tonal
system to each of the present-day varieties. This is, of course, a~long-term undertaking. The immediate task is to provide a~detailed synchronic description of the tone system of one dialect.

The analyses of Yongning Na tone presented in this volume are therefore based on data from  Mrs.~Latami, unless otherwise mentioned.

%\largerpage
\section{Online corpus, dictionary, and other tools}
\label{sec:OnlineResources}

A guiding principle in the present research is that a~close association between documentation and
research is highly beneficial to both \citep{garellek_toward_2020}. If, as suggested by \citet{whalen2004},
“the study of endangered languages has the potential to revolutionize linguistics” and
“the vanguard of the revolution will be those who study endangered languages”, then it is all the
more regrettable that “enormous amounts of data~-- often the only information we have on
disappearing languages~-- remain inaccessible both to the language community itself, and to ongoing
linguistic research” (\citealt{thiebergeretal2006}; see also
\citealt{woodbury2003,woodbury2011}). 

“[L]anguage documentation as a~paradigm in linguistic
research” has been recognized as offering several key benefits: 
\begin{quotation}
\begin{enumerate}[label=(\roman*)]
	\item  making analyses accountable to the primary material on which they are
based; 

	\item  providing future researchers with a~body of linguistic material to analyse in ways not foreseen by the original collector of the data; and, equally importantly,

	\item  acknowledging the responsibility of the linguist to create records that can be
accessed by the speakers of the language and by their descendants. \citep[1]{thiebergeretal2016}
\end{enumerate}
\end{quotation}

The necessity of making primary data available has been emphasized specifically in relation to the study of tone at a~symposium on “Cross-linguistic studies of tonal phenomena”:
 
\begin{quotation}
The point to be made is extremely simple: declare the status of the primary data for what it is, and allow an evaluation of the data and its subsequent interpretation to take place in view of the declared status of the data. ({\dots}) I~in no way imply that the wealth of impressionistic data which we have on tonal phenomena are inherently wrong or that they misrepresent the situation in any particular language. ({\dots}) What I do, however, argue for is an unashamed scientific approach to the handling of tonal data, abiding with generally accepted criteria of scientific practice. This will not only raise the credibility of analyses, but may even lead to objective empirical testing of particular theories. \citep[366]{roux2001}
\end{quotation}

Accordingly, the recordings conducted in Yongning have been made freely available
online, document by document, since 2011. 

\label{sec:onlinematerials}
\subsection{Transcribed narratives and phonological materials}
\label{sec:transcribedandtranslatednarrativesandphonologicalmaterials}

These recordings consist mainly of
narratives and lexical or phonological elicitation sessions. They are accompanied by metadata
(providing information about the recordings) and, as far as possible, by full transcriptions
and translations. A~list of documents is provided in the Abbreviations section, with one-click links to these resources.

The data is hosted by the Pangloss Collection,\footnote{Address of the interface since 2021: \url{https://pangloss.cnrs.fr/}. Direct link to the Yongning Na corpus: \url{https://pangloss.cnrs.fr/corpus/Yongning_Na}.} a~language archive developed at the \textit{Langues et Civilisations à Tradition Orale} (LACITO) research centre within \textit{Centre National de la Recherche Scientifique} (CNRS) \citep{jacobsonetal2001,michailovskyetal2014,adamou_pangloss_2025}. The goal of this archive is to preserve and disseminate oral literature and other linguistic materials in (mainly) endangered or poorly documented languages,
providing simultaneous access to sound recordings and text annotation. 

% \begin{quote}
%     Paired with advances in digital media, accessible corpora of annotated language data not only allow for verification of current analyses; they will, in time, provide answers to as yet unknown research questions, as well as providing a~cultural record of value to the broader community. \citep{thiebergeretal2016}
% \end{quote}

The Na media files are mostly audio, although a few video recordings were made in 2018. Some of the audio files are accompanied by an electroglottographic signal, which allows for
high-precision measurement of the voice’s fundamental frequency, as well as other glottal
parameters.\footnote{For an introduction to electroglottography, see the original report on its invention: \citet{fabre1957};
a~synthesis: \citet{baken1992}; some caveats: \citet{orlikoff1998}; discussions on measurable parameters: \citet{henrichetal2004b}, \citet{michaud2004b}, \citet{herbst_electroglottography_2020}; and applications to the study of specific linguistic issues,
e.g.~\citet{brunelleetal2010} and \citet{kuangetal2014}.} Documents that include
an electroglottographic signal are marked with a~special icon in the list of resources. % (\figref{fig:autolist}). 

% \begin{figure}%[t]
% 	\includegraphics[width=\textwidth]{figures/AutoList.png}
% 	\caption{Screen shot of the beginning of the list of Yongning Na resources in the Pangloss Collection.}
% 	\label{fig:autolist}
% \end{figure}

% The list of resources is generated automatically from the archive's catalogue. As of 2017, it was arranged by date of deposit. This does not make it easy to locate a~document in this list, or to get a~feel for the state of the corpus. A~description of the corpus is therefore provided as a~static HTML document. It is available by clicking on the language name (‘Na’) at top of page. This description is arranged by dialect and by type of contents. Explanations about data collection are also provided. A~passage from this presentation (which is maintained and gradually improved over the years) is shown in \figref{fig:static}.
% %
% % \begin{figure}  %[t]
% % \includegraphics[width=\textwidth]{figures/ListOfResources.png}
% % \caption{Static HTML page presenting the Na resources.}
% % \label{fig:static}
% % \end{figure}
% %
%  For instance, the recordings of {numeral}-plus-classifier phrases which appear as a~list towards the bottom of \figref{fig:autolist} are arranged in table form on the presentation shown in \figref{fig:static}, making it much easier to see at a~glance which recordings are available.
 \figref{fig:apassagefromoneofthedocumentsasdisplayedonthewebinterface} shows
a~passage from one of the documents as displayed on the web interface, featuring transcription, translation, and time-aligned audio. The original files can also be downloaded without requiring login. 



\begin{figure}[p] 
\includegraphics[width=.85\textwidth]{figures/Pangloss_Sister.jpg}
\caption{A passage from one of the documents as displayed on the web interface: transcription, translation, and time-aligned audio.}
\label{fig:apassagefromoneofthedocumentsasdisplayedonthewebinterface}
\end{figure}

The Na documents are archived with provisions for long-term conservation and will remain accessible regardless of any future changes
to the web address of the Pangloss Collection’s interface. Direct links to specific locations within a
text are provided “to offer readers the means to interact instantly with digital versions of the primary data, indexed by transcripts'' \citep{thieberger2009}. Seamless navigation between linguistic descriptions and data is the main improvement introduced in this second edition, fulfilling a hope expressed in the first edition (``It is hoped that, by the time the next edition of this volume is released, tools for resolving a~multimedia document's identifier will be all set up and working, allowing for links from the digital book that will direct the user straight to the relevant passage in the online data''). 

The availability of these audio and electroglottographic data with synchronized transcriptions allows interested readers to develop a first-hand sense of the data and facilitates further research into a~broad range of phonetic topics. Given the substantial time investment required to carry out a~state-of-the-art experimental \is{experimental phonetics}phonetic study, it is simply not feasible for
linguists engaged in the description of an entire language (or multiple languages) to conduct a dedicated phonetic investigation to substantiate and refine every observation they make. However, it is possible to collect a~sufficiently extensive dataset to enable colleagues with a specialized interest in phonetics to undertake such studies. 
The electroglottographic signal has not yet been systematically analyzed, apart from occasional use in auditory verification (the pitch can sometimes be perceived more clearly when	listening to the electroglottographic signal than when listening to the audio recording). This signal could	serve in the future as the basis for a~phonetic study of tone implementation in Yongning Na (see \sectref{sec:keyfactorsinthephoneticimplementationoftone}).
%Taking the case of recorded data on {numeral}-plus-classifier phrases (analyzed in Chapter~\ref{chap:classifiers}), the following are two examples of phonetic phenomena that could be investigated in the future using the Yongning Na recordings.

%\begin{enumerate}[label=(\roman*)]
%\item The implementation of tone. The electroglottographic signal has not been exploited so
%  far, except for its occasional use in auditory verification (the pitch can be clearer when
%  listening to the electroglotto-graphic signal than when listening to the audio). This signal could
%  serve in future for a~phonetic study of the implementation of tone in Yongning Na. There is
%  a~large gap between phonological representations in terms of sequences of level tones, on the one
%  hand, and observed fundamental frequency curves, on the other: “both F\textsubscript{0} height and F\textsubscript{0} velocity are
%  relevant parameters ({\dots}) even for the simplest \is{level tones}level tone languages” \citep[1]{Yu2010}. A
%  study of the implementation of tone sequences in Na would be a~useful addition to the existing
%  literature on contextual tonal \isi{variation} and segmental effects on tone, as studied e.g.~by as
%  studied e.g.~by \citet{abramson1979a}, \citet{gsell1985} and \citet{gandouretal1992} for
%  \ili{Thai}, and \citet{xu1997,xu1998} for \ili{Mandarin}.
%\item The weakened (hypo-articulated) realization of repeated words. When a~consultant
%  pronounces a~sequence of \is{numerals}numeral-plus-classifier phrases, such as /\ipa{ɖɯ˧-kʰwɤ˥ {\kern2pt}|{\kern2pt} ɲi˧-kʰwɤ˥ {\kern2pt}|{\kern2pt}
%    so˩-kʰwɤ˩˥ {\kern2pt}|{\kern2pt} ʐv̩˧-kʰwɤ˧}/ (‘one piece, two pieces, three pieces, four pieces{\dots}’), the tone
%  of the classifier changes (High, High, Low-to-High, Mid{\dots}) but its consonants and vowels do
%  not. As a~result, the speaker’s attention focuses on the realization of the new information
%  (essentially: the correct tone sequence for the phrase); in terms of the continuum from
%  hyper-articulation to hypo-articulation \citep{lindblom1990}, the classifier is
%  hypo-articulated. Specifically, the unvoiced lateral /\ipa{ɬ}/ in classifiers such as /\ipa{ɬi˧}/
%  ‘month’ and /\ipa{ɬi˩}/ ‘armspan’ is occasionally realized as voiced, despite the existence of
%  a~voicing opposition between /\ipa{ɬ}/ and /\ipa{l}/ in Na. Pursuing such observations would shed
%  light on the field of allophonic dispersion of Na phonemes. Due to the nature of the corpus,
%  numerous tokens of each \is{numerals}numeral and classifier are available, offering a~good basis for
%  statistical treatments.
%\end{enumerate}

%\subsubsection*{(i) The implementation of tone}


%\subsubsection*{(ii) The weakened (hypo-articulated) realization of repeated words}

%\largerpage[2] % Old 'hack' for page layout. No longer relevant. This is among the elements to check for the 2nd edition.

Importantly, the documents are also open to entirely different uses, including aesthetic appreciation of a~voice captured through the wonder of high-fidelity recording, and preserved unaltered through the magic of digital storage. I, for one, am not insensitive to the luxury of listening to the Yongning Na recordings at leisure. Once a~recording session has concluded, the qualms and concerns of fieldwork recede, leaving room for ``the fecund miracle of communication within solitude'', to borrow Proust's evocative description of reading.\footnote{\textit{Original text:}~la lecture, dans son essence originale, dans ce miracle fécond d'une communication au sein de la solitude (\dots) (\textit{Journées de lecture}, in \textit{Pastiches et mélanges}, Paris: Gallimard, 1919, p. 257).}

%The other online materials and tools include a dictionary and a~group library used in research are shared over the Internet.

A~dictionary of Yongning Na is also shared over the Internet, as well as a~group library aiming to facilitate the sharing of publications relating to Naish languages and cultures. 

\subsection{Dictionary}
\label{sec:dictionary}

%{\largerpage[-1]} % Added on April 21st, 2025

In the classical tradition of linguistic fieldwork, a language description should include a dictionary, alongside a grammar and a collection of texts. Together, these three components constitute what is often referred to as the “Boasian trilogy” \citep{foley1999}, in reference to Franz Boas’s foundational work collecting North American languages \parencite{boas1902,boasetal1911}. 

The lexicographic data collected during fieldwork on Yongning Na were progressively organized into a dictionary of Yongning Na \citep{michaud_et_al_na_dict_2024}, available both as a PDF and in XML database format. %Version 1.0 (2015) and 1.1 (2016)The first edition  online. %(i)~as an online dictionary in HTML format, (ii)~as a~PDF document, and (iii)~in database format. 
It is designed not only as a tool for academic research but also as a gateway to Na culture and language. In addition to the version tailored for English-speaking readers,\footnote{\url{https://shs.hal.science/halshs-01204638/}} separate versions have been prepared for Chinese-speaking\footnote{\url{https://shs.hal.science/halshs-01744420}} and French-speaking\footnote{\url{https://shs.hal.science/halshs-01204645}} audiences. Successive versions of the dictionary are numbered, providing a stable point of reference while allowing for continuous refinement over time. Like this book, and like the materials in the Pangloss Collection, the dictionary is freely available online under a Creative Commons license.


\subsection{Group library}
\label{sec:bibliography}

An online bibliography of {Naish} studies~-- covering research on {Naxi}, Na, {Laze}, and related languages~-- was
initiated in 2015 as a~Zotero group \citep{duong2010zotero} under the name ‘{Naish} languages and cultures’.\footnote{\url{https://www.zotero.org/groups/395957/naish_languages_and_cultures/}} 
% References will gradually be
% added and enriched to provide multilingual information: for Chinese-language references, the author
% name will be provided in Chinese characters, as well as in romanized Chinese (\textit{Pinyin}
% transcription); and titles and other relevant information will be translated into {English}. 
The primary focus is on linguistic research, but the team of contributors also aims to include
relevant ethnological, anthropological, historical, and sociological studies. 

This bibliography is publicly available, and those interested in contributing to its enrichment and maintenance are
encouraged to get in touch.


\section{A grammatical sketch of Yongning Na}
\label{sec:sketch}

%{\largerpage[-1]} % Added on April 21st, 2025

This final introductory section provides a brief grammatical sketch of Yongning Na, outlining structural properties such as the organization of noun and verb phrases. These elements serve as background for the discussion of \isi{morphotonology} in the chapters that follow. More substantial sketches can be found in \citet{lidz2016,lidz2019}.

%\subsection{Morphology}
%\label{sec:morphology}

%``Na is quite analytic, and does not have inflectional morphology" \citep[238]{}. There is no inflectional morphology in Na. 
There are no stem alternations in Na, unlike in conservative Sino-Tibetan languages such as rGyalrong \citep{gong2018,zhang2020}. Suppletion is attested only in the verb ‘to go’, which has distinct forms for different grammatical contexts: {nonpast} /\ipa{bi˧\textsubscript{c}}/ (\ref{ex:alone}), {past} /\ipa{hɯ˧}/ (\ref{ex:gone}), {past perfective} /\ipa{hɤ˩\textsubscript{a}}/ (\ref{ex:Caravans35went}), and {imperative} /\ipa{hõ˧}/ (\ref{ex:imperat}). Note that the {nonpast} form in (\ref{ex:alone}) is translated as a preterite, a common tense for narratives in written {English}.
 
\begin{exe}
	\ex
	\label{ex:alone}
	\ipaex{ɖɯ˧-v̩˧ lɑ˧ \textbf{bi˥} {\kern2pt}|{\kern2pt} pi˧-dʑo˩ {\kern2pt}|{\kern2pt} ʐv̩˩-kv̩˩ tɕɯ˥-kv̩˩ mæ˩!}\\
	\gll ɖɯ˧-v̩˧						lɑ˧		\textbf{bi˧\textsubscript{c}}			pi˥		-dʑo˥				ʐv̩˩-kv̩˩					tɕɯ˧˥		-kv̩˧˥		mæ˧\\
	one-\textsc{clf}.individual		only	\textbf{to\_go.\textsc{nonpast}}		to\_say		\textsc{top}		four-\textsc{clf}	to\_lead	\textsc{abilitive}	\textsc{obviousness}\\
	\glt ‘If only one [man] went, he could lead four [horses]! / If a man went alone, he would lead up to four horses!’ \textit{(Caravans.119)} \pandoi{0004531\#S119}
\end{exe}

%\Hack{\newpage}

\begin{exe}
  	\ex
  	\label{ex:gone}
  	\ipaex{ɻ̩˩ʈʂʰe˧-ɖɯ˩mɑ˩ {\kern2pt}|{\kern2pt} tsʰi˧ɲi˧ {\kern2pt}|{\kern2pt} ə˧tse˧ \textbf{hɯ˧}-ɻ̩˩?}\\
  	\gll ɻ̩˩ʈʂʰe˧-ɖɯ˩mɑ˩	tsʰi˧ɲi\#˥	ə˧tse\$˥							\textbf{hɯ˧\textsubscript{c}}						-ɻ̩˩\\
  	given\_name				today		\textsc{interrog}:why		\textbf{to\_go.\textsc{pst}}		\textsc{inceptive}\\
  	\glt ‘Why has Erchei Ddeema gone away today? / How come Erchei Ddeema has left the house?’ \textit{(BuriedAlive2.14)} \pandoi{0004537\#S14}
\end{exe}
  
\begin{exe}
 	\ex
 	\label{ex:Caravans35went}
 	\ipaex{“ɳæ˧=ɻ̩˩-se˩ {\kern2pt}|{\kern2pt} ə˧mv̩˩ le˩-ʝi˩{$\sim$}ʝi˩-ze˩! {\kern2pt}|{\kern2pt} se˧kʰɯ˩-ʁo˩ni˩ le˩-po˩ ʝi˩{$\sim$}ʝi˩-ze˩!” {\kern2pt}|{\kern2pt} pi˧-kv̩˩ mæ˩, {\kern2pt}|{\kern2pt} ho˧di˧ \textbf{hɤ˧}-dʑo˥!}\\
 	\gll ɳæ˧=ɻ̩˩		-se˧		ə˧mv̩˩				le˧-				ʝi˧\textsubscript{c} {$\sim$}	-ze˧				se˧kʰɯ˩-ʁo˩ni˩			le˧-				po˧˥			ʝi˧\textsubscript{c} {$\sim$}	-ze˧			pi˥							-kv̩˧˥					mæ˧							ho˧di˧		\textbf{hɤ˩\textsubscript{a}}		-dʑo˧\\
 	\textsc{2pl}	\textsc{top}	elder\_brother	\textsc{accomp}		to\_come	\textsc{red}	\textsc{pfv}	satin\_headdress		\textsc{accomp}	to\_bring		to\_come			\textsc{red}	\textsc{pfv}	to\_say		\textsc{abilitive}	\textsc{obviousness} 	Sichuan		\textbf{to\_go.\textsc{pst.pfv}}		\textsc{prog}\\
 	\glt ‘When [one's brothers] had gone away [on a caravan] to Sichuan, [people in the village] would say: “You, your brother is going to come back! [He] will bring back a~satin headdress [for you]!”~’ \textit{(Caravans.35) }\pandoi{0004531\#S35}
\end{exe}
 
\begin{exe}
	\ex
	\label{ex:imperat}
	\ipaex{no˧ {\kern2pt}|{\kern2pt} wɤ˩˥ {\kern2pt}|{\kern2pt} tsʰo˧qʰwɤ˩ \textbf{hõ˩}!}\\
	\gll no˩			wɤ˩˥			tsʰo˧qʰwɤ˩		\textbf{hõ˧}\\
	\textsc{2sg}	again			tonight			\textbf{to\_go.\textsc{imperative}}\\
	\glt ‘Go again tonight!’ \textit{(Reward.63)} \pandoi{0004447\#S63}
\end{exe}
 
\is{reduplication}Reduplication of verbs and adjectives is widely used to convey reflexivity and \isi{intensification} (see \sectref{sec:reduplication}). 

In the lexicon, a major force driving the creation of disyllables is the pressure of \isi{homophony}, resulting from the dramatic \isi{phonological erosion} of the {Naish} languages compared with proto-{Sino-Tibetan}. For instance, [{\kern1.3pt}\ipa{ʝi˧}] can mean ‘jar’, ‘ox’, ‘to do’, ‘to draw’, or ‘to inform’. The word ‘jar’ is typically disyllabic: [{\kern1.3pt}\ipa{ʝi˧mi˧}], incorporating the morpheme for ‘mother’, which has grammaticalized functions as both a~feminine suffix and an augmentative suffix (see \sectref{sec:thegendersuffixes}). In this case, the disyllabic form has entirely supplanted the original \is{monosyllables}monosyllable. 

%\subsection{Syntax}
%\label{sec:syntax}

\subsection{Word order}
\label{sec:wordorder}
\is{word order|textbf}

%{\largerpage[-1]}

Word order is S+O+V, or, more precisely: “[i]n unmarked, non-idiomatic,
pragmatically neutral constructions, subject-object-verb {word order} is
most common” \citep[48]{lidz2007}. Adverbials can appear in various positions. Agent marking is not obligatory: “agents are unmarked more frequently than
they are marked, both in conversation and narrative” \citep[51]{lidz2011}. Like the {agent} adposition /\ipa{ɳɯ˧}/, the {dative} \is{suffixes}suffix /\ipa{-ki˧}/ is optional. These morphemes allow for deviations from the standard {word order}; they also serve a disambiguating function, as illustrated in (\ref{ex:teachyou}). In this example, the presence of a noun phrase marked as {dative} clarifies that the verb /\ipa{so˩\textsubscript{a}}/ is to be understood as ‘to teach’ rather than ‘to study; to imitate’, as the latter meaning does not take a~{dative} particle. 

The context of (\ref{ex:teachyou}) is as follows: the speaker saw me sitting ready for an elicitation session, silently hoping that someone would spare some time for me. There was genuine uncertainty as to who that person would be, as I was working with two consultants within the household at the time. No subject is indicated in the first part of the sentence (‘After feeding the pigs’), since it is self-evident from the situation that the speaker will perform this action. By contrast, in the second part of the sentence, both semantic roles are explicitly mentioned, providing a thorough response to the addressee's implicit concerns.

\begin{exe}
	\ex
	\label{ex:teachyou}
	\ipaex{bo˩-hɑ˧ {\kern2pt}|{\kern2pt} le˧-ki˧ le˧-se˩-ze˩, {\kern2pt}|{\kern2pt} njɤ˧ ɳɯ˧ {\kern2pt}|{\kern2pt} no˧-ki˧ {\kern2pt}|{\kern2pt} nɑ˩ʐwɤ˧ so˩-bi˩!}\\
	\gll bo˩˧	hɑ˥		le˧-	ki˧\textsubscript{a}			le˧-		se˩\textsubscript{a}		-ze˧		njɤ˩				ɳɯ˧		no˩					-ki˧			nɑ˩ʐwɤ˥				so˩\textsubscript{a}						-bi˧\\
	pig		food		\textsc{accomp}		to\_give	\textsc{accomp}	to\_finish			\textsc{pfv}	\textsc{1sg}	\textsc{a}	\textsc{2sg}	\textsc{dat}	Na\_language		to\_teach/to\_study		\textsc{imm.fut}\\
	\glt ‘After feeding the pigs, I will teach you Na (myself) / I shall come over for a language lesson!’ (Sentence transcribed on the fly in 2008)
\end{exe}

As is well-attested across the Himalayas, nominals, including pronouns, need not be overtly expressed if they can be inferred from the discourse context. This applies to any definite argument of the verb, the head of a relative clause, or the head of a complex noun phrase. Another areal characteristic is that topic markers are among the most frequently occurring morphemes.

\subsection{Tense, aspect and modality}
\label{sec:tenseaspectmodality}

%{\largerpage[-1]}

Tense, aspect and modality are expressed through a~number of postverbal particles and some preverbal particles, such as the \textsc{durative} /\ipa{tʰi˧}-/, the \textsc{accomplished} /\ipa{le˧}-/, the \textsc{perfective} /-\ipa{ze˧}/ and the \textsc{experiential} /-\ipa{dʑɯ˧}/. The verb ‘to go’ /\ipa{bi˧\textsubscript{c}}/ has \is{grammaticalization}grammaticalized into an immediate future marker, while /\ipa{se˩\textsubscript{a}}/ ‘to complete’ functions as a \textsc{completion} marker~-- both occurring in postverbal position. 

Na stands at a~great typological distance from languages such as Wolof, where a single predicative marker associated to the verb encodes the bulk of the grammatical content \citep{creisselsetal1998, guerin2015}. Instead, Na patterns with Loloish (\ili{Yi}) languages such as \ili{Lahu} \citep{matisoff1973a} and \ili{Lalo} \citep{bjorverud1998}, where particles have rich combinatorial potential. The interplay among these particles gives rise to a~wealth of highly dialect-specific nuances, best appreciated through contextualized examples. 


\subsection{Question formation}
\label{sec:qformation}
In yes/no questions, the verb is preceded by an interrogative particle, /\ipa{-ə˩}/, as illustrated in (\ref{ex:areyoumydaughter}). In \textit{wh}-questions, exemplified by (\ref{ex:whogoeswithmother}), the interrogative \is{pronouns}pronoun occupies the same slot as a noun phrase would in a statement. 

\begin{exe}
	\ex
	\label{ex:areyoumydaughter}
	\ipaex{njɤ˧ mv̩˩ {\kern2pt}|{\kern2pt} ə˩-ɲi˩˥?}\\
	\gll njɤ˩				mv̩˩˥		ə˩-						ɲi˩\\
	\textsc{1sg}		daughter		\textsc{interrog}		\textsc{cop}\\
	\glt ‘Is this my daughter? / Are you my daughter?’ \textit{(BuriedAlive3.107)} \pandoi{0004538\#S107}
\end{exe}

\begin{exe}
	\ex
	\label{ex:whogoeswithmother}
	\ipaex{ɲi˩ ɳɯ˥ {\kern2pt}|{\kern2pt} ə˧mɑ˧-qɑ˧ tɕʰo˧ bi˧˥?}\\
	\gll ɲi˩		ɳɯ˧						ə˧mɑ˧		-qɑ˧˥						tɕʰo˩	bi˧\textsubscript{c}\\
	\textsc{interrog}.who		\textsc{a}		mother		\textsc{comitative}		together				to\_go\\
	\glt ‘Who is going to go together with mama? / Who is going to accompany mama?’ \textit{(FoodShortage2.17)} \pandoi{0004657\#S17}
\end{exe}


\subsection{Existential sentences}
\label{sec:existential}
Na, like many other {Sino-Tibetan} languages, has several \is{existentials}existential verbs. One is preferred for animate referents, as illustrated in (\ref{ex:isathome}). Another is “used with things that stand, protrude, or are perpendicular to a
plane” \citep[358]{lidz2010}, as in (\ref{ex:barley}). A third one applies to “objects within a container” \citep[361]{lidz2010}~-- including abstract containers such as narratives, as shown in (\ref{ex:instory}). 

\begin{exe}
	\ex
	\label{ex:isathome}
	\ipaex{hĩ˧ ɑ˥ʁo˩ dʑo˩}\\
	\gll hĩ˥		ɑ˩ʁo˧		dʑo˩\textsubscript{b}\\
	person		home		\textsc{exist.animate}\\
	\glt ‘there are people at home’ \textit{(Reward.11)} \pandoi{0004446\#S11}
\end{exe}

\begin{exe}
	\ex
	\label{ex:barley}
	\ipaex{tɕʰi˧ tʰi˧-di˥-ɲi˩ mæ˩!}\\
	\gll tɕʰi˥		tʰi˧-			di˩\textsubscript{a}		-ɲi˩		mæ˧\\
	thorn		\textsc{dur}	\textsc{exist}		\textsc{certitude}	\textsc{obviousness}\\
	\glt ‘There are some thorns [on barley], not? / [barley] has beard, hasn't it!’ \textit{(FoodShortage.43)} \pandoi{0004557\#S43}
\end{exe}

\begin{exe}
	\ex
	\label{ex:instory}
	\ipaex{dɑ˧pɤ˧ qʰwæ˧-qo˩ {\kern2pt}|{\kern2pt} qv̩˧ɻ̩˧ {\kern2pt}|{\kern2pt} tʰi˧-ʑi˥-kv̩˩ mæ˩!}\\
	\gll dɑ˧pɤ˧						qʰwæ˧		-qo˧		 qv̩˧ɻ̩\#˥						tʰi˧-				ʑi˥						-kv̩˧˥										mæ˧\\
	priest\_of\_local\_religion		message		inside		name\_of\_a\_mountain	\textsc{dur}	\textsc{exist}		\textsc{abilitive}		\textsc{obviousness}\\
	\glt ‘Mount Gheu’er \ipa{qv̩˧ɻ̩\#˥} is mentioned in the tales of the Daba priests!’ \textit{Literally:} ‘Mount Gheu’er \ipa{qv̩˧ɻ̩\#˥} exists in the tales of the Daba priests’ \textit{(Mountains.120)} \pandoi{0004573\#S120}
\end{exe}

In contrast to these, there is an \is{existentials}existential for diffuse entities, namely those that can neither be counted nor contained (and thus quantified), illustrated in (\ref{ex:athand}). Finally, there is a~generic \is{existentials}existential verb, exemplified in (\ref{ex:nothingtoeat}). 

\begin{exe}
	\ex
	\label{ex:athand}
	\ipaex{ə˧tso˧-mɤ˧-ɲi˩ {\kern2pt}|{\kern2pt} le˧-ʂe˧, {\kern2pt}|{\kern2pt} le˧-ʝi˥!}\\
	\gll ə˧tso˧-mɤ˧-ɲi˩		le˧-		ʂe˧\textsubscript{a}		le˧-	ʝi˥\\
	all\_sorts\_of\_things	\textsc{accomp}		to\_get		\textsc{accomp}		\textsc{exist}\\
	\glt ‘We get all sorts of things (all the necessary paraphernalia for a ritual, a feast{\dots}) [so that] we have it (at hand for when we need it) / We get all sorts of things ready (for the ritual / the feast)!’ (Field notes.)
\end{exe}

\begin{exe}
	\ex
	\label{ex:nothingtoeat}
	\ipaex{dzɯ˧-di˧ mɤ˧-dʑo˧˥}\\
	\gll dzɯ˥		-di˩				mɤ˧-			dʑo˧\textsubscript{b}\\
	to\_eat			\textsc{nmlz}		\textsc{neg}		\textsc{exist}\\
	\glt ‘there was nothing to eat / there was no food left’ \textit{(FoodShortage2.4)} \pandoi{0004657\#S4}
\end{exe}

\citet[359]{lidz2010} reports an additional {existential} verb in the Luoshui dialect, “used for the passing of time”: “/\ipa{ku33}/ \textsc{exist.t} [Existential: Used with past existence of time] seems to have something of a~connotation of ‘pass,’ and may be a~fairly recent {grammaticalization} from a lexical verb”. In the Alawua dialect (studied in the present volume), this verb, /\ipa{gv̩˧\textsubscript{c}}/, retains the lexical meaning ‘to flow, to go by, to elapse (of time); to take place, to occur (of an event)’.

\subsection{Sentence-final particles}
\label{sec:sentfinparticles}

Among sentence-final particles, several function as epistemic or evidential markers: /\ipa{tsɯ˧˥}/ for hearsay, /\ipa{mv̩˧}/ for affirmation, /\ipa{le˩}/ for exclamation, and /\ipa{mæ˧}/ to convey obviousness. Combinations of these particles allow for a~considerable range of nuances. Their meaning in context arises through an interplay with intonational cues: the three-level tone system does not impose strong constraints on a sentence’s \isi{intonation}, leaving ample room for the expression of nuances of doubt, surprise, and other attitudes and emotions through intonational means, such as overall raising of the pitch register, pitch \isi{range expansion}, \isi{lengthening}, and changes in \is{phonation types}phonation type. (This topic is taken up in Chapter~\ref{chap:fromsurfacephonologicalformstophoneticrealizationintonationandtonalimplementation}.)


\subsection{Noun and noun phrase}
\label{sec:nounNP}

The classifier, preceded by a~{numeral} or a~{demonstrative}, follows the head noun, as illustrated in (\ref{ex:lake33}) with the phrase ‘a~dumb person’. A~\is{numerals}numeral and a~classifier can also form a~well-formed phrase on their own. Such a phrase may receive a~{suffix}, as in ‘a~family’ in the same example. 

\begin{exe}
	\ex
	\label{ex:lake33}
	\ipaex{ɖɯ˧-ʑi˩=ɻ̩˩-dʑo˩ {\kern2pt}|{\kern2pt} zo˧bæ˩ {\kern2pt}|{\kern2pt} ɖɯ˧-v̩˧ dʑo˩ tsɯ˩ {\kern2pt}|{\kern2pt} mv̩˩!}\\
	\gll ɖɯ˧	ʑi˩\textsubscript{b}					=ɻ̩˩					-dʑo˥			zo˧bæ˩			ɖɯ˧		v̩˧							dʑo˩\textsubscript{b}		 tsɯ˧˥			mv̩˧\\
	one			\textsc{clf}.households		\textsc{associative}	\textsc{top}	dumb\_man	one		\textsc{clf}.individual		\textsc{exist}			\textsc{rep}		\textsc{affirm}\\
	\glt ‘It is said that, in a~household, there was a~dumb man.’ \textit{(Lake3.3)} \pandoi{0004348\#S3}
\end{exe}

The counting system is decimal. 

The system of classifiers is extensive, with many items still transparently related to nouns. Some classifiers function as \is{self-classifier}‘self-classifiers’, occurring without a head noun; for instance, /\ipa{ɖɯ˧-kʰv̩˧˥}/ ‘one year’ is acceptable, whereas {$\ddagger$}{\kern2pt}\ipa{kʰv̩˧˥ ɖɯ˧-kʰv̩˧˥} (‘year one-year’) is not. Classifiers are discussed in Chapter~\ref{chap:classifiers}.

The order in noun compounds follows a \textit{determined+determiner} pattern. By contrast, a~small number of lexicalized compounds involving \isi{adjectives} have the reverse order, e.g.~‘dry field’ /\ipa{pv̩˧lv̩˧}/ (‘dry’{\allowbreak}+‘field’). Compounds are discussed in Chapter~\ref{chap:compoundnouns}.

There is no grammatical gender or agreement. 

The demonstrative system distinguishes between proximal and distal forms. The proximal \is{demonstratives}demonstrative, /\ipa{ʈʂʰɯ˥}/, is \is{homophony}homophonous with the third-person singular \is{pronouns}pronoun. The pronominal system has inclusive and exclusive first-person pronouns, as well as \is{associative plural}associative forms referring to ‘me and my kin’, ‘you and your kin’, etc. Syntactically, pronouns in Na behave like nouns~-- as in languages such as \il{Sinitic}Chinese, \ili{Japanese}, \ili{Bambara}, and \ili{Zarma} \citep[29]{creissels1995syntaxe}~--  rather than functioning as pronominal indices.

%\newpage
The clause relativizer is /-\ipa{hĩ˥}/, as illustrated in (\ref{ex:familyGIVE}).

\begin{exe}
	\ex
	\label{ex:familyGIVE}
	\ipaex{go˧mi˧ {\kern2pt}|{\kern2pt} tʰi˧-ki˧-hĩ˧ {\kern2pt}|{\kern2pt} ʈʂʰɯ˧-ʑi˥{\dots}}\\
	\gll go˧mi˧		tʰi˧-		ki˧\textsubscript{a}			-hĩ˥		ʈʂʰɯ˥			ʑi˩\textsubscript{b}\\
	younger\_sister		\textsc{dur}	to\_give			\textsc{rel}		\textsc{dem.prox}		\textsc{clf}.households\\
	\glt ‘the family to which the younger sister pledged herself{\dots} / the household which the woman entered upon marriage{\dots}’ \textit{(Sister3.18)} \pandoi{0004344\#S18}
\end{exe}


\subsection{Verb and verb phrase}
\label{sec:vvphrase}

Directionality is marked before the verb, using /\ipa{gɤ˩}-/ ‘upward’, /\ipa{mv̩˩}-/ ‘downward’, or disyllabic expressions (\sectref{sec:themarkingofspatialorientationonverbs}). 

Verb serialization is employed in various constructions, including resultatives, equivalents of Chinese ‘\textit{de} \zh{得}’ constructions, and expressions of movement \citep[397-405]{lidz2010}. Imperatives are conveyed solely through intonational means, except for the verb ‘to go’, which has a distinct imperative form. The syntactic structure is identical to that of statements, except in the case of the verb ‘to go’, which has a~distinct {imperative} form, as discussed at the outset of this section (\sectref{sec:sketch}).

\is{adjectives}Adjectives function as \isi{stative verbs}, as they “can take aspect marking, be negated, and can be modified by the \is{intensifiers}intensifier” \citep[362]{lidz2010}. The intensifiers are /\ipa{ɖwæ˧˥}/, which precedes the adjective (‘very \textsc{Adj}’) or verb (‘V a~lot’), and /\ipa{ʐwæ˩}/, which means ‘extremely’ and occurs exclusively with adjectives. The {copula} is not used with adjectives but may be added after any verb to express certainty \citep[354]{lidz2010}.

In addition to examining the lexical tones of nouns (Chapter~\ref{chap:thelexicaltonesofnouns}) and verbs (\sectref{sec:thelexicaltonesofverbs}), the following chapters address various syntactic phrase types in which morphotonological adjustments occur. These include phrases containing nominal classifiers (Chapter~\ref{chap:classifiers}); {compound} nouns (Chapter~\ref{chap:compoundnouns}); combinations of content words with grammatical elements~-- nouns in Chapter~\ref{chap:combinationsofnounswithgrammaticalwords}, and verbs and adjectives in \sectref{sec:reduplication}-\ref{sec:combinationsofadjectiveswithgrammaticalmorphemes}~--; object and verb (\sectref{sec:objectandnonprefixedverb}-\ref{sec:objectandprefixedverb}); and subject and verb (\sectref{sec:subjectandverb}).
