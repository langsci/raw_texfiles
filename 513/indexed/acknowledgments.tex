\addchap{Acknowledgments}
% \begin{refsection}

\epigraph{[Writing a~grammar] takes an individual who loves language in general, the target language in particular, and is trained and happy to spend time doing this work.}{\citep[xxiv]{nurse2011}}

%Command \noindent added to avoid having a first indent in cases where a paragraph starts after an epigraph without an intervening title.
{\noindent}I am deeply grateful to my teachers for their patience and inspiration over the years. Laurent Danon-Boileau, my first linguistics tutor, encouraged students to explore lesser-documented languages. 
% – a~field about which I knew nothing. 
I knew at once that this was what I wanted to do. In 1994, reading Michel Launey’s newly published description of Classical Nahuatl as “an omnipredicative grammar” \citep{launey1994}, my imagination was fired by his mention (p. 16) of the professional and personal coincidences that led him to study a~language remote in time and space, ultimately yielding groundbreaking insights into human language. That story resonated with me, planting the idea that I, too, might contribute by exploring an unfamiliar linguistic and cultural world. I finally had the chance to experience this wonderful blend of discovery, excitement and fulfilment in my fieldwork in Southwest China. 

A special thanks to Jacqueline Vaissière, who took on the challenge of raising a~novice steeped in the cloudiest romanticism to the status of Doctor in phonetics, and who has been a wise and supporting mentor ever since. 

My work on Yongning Na began in 2006, the same year I joined the \textit{Langues et Civilisations à Tradition Orale} (LACITO) research centre within \textit{Centre National de la Recherche Scientifique} (CNRS). LACITO, with its deep commitment to immersion fieldwork, has been an intellectually stimulating and supportive home base. I am grateful to CNRS for making it possible for me to carry out extended fieldwork in China in 2011--2012, with a~temporary
affiliation at the \textit{Centre d’Études Français sur la Chine contemporaine} (CEFC). My thanks also to the Institute of Linguistics at Academia Sinica for hosting me for three months in early 2011. From late 2012 to mid-2016, I was based at the International Research Institute MICA in Hanoi, in an environment that facilitated close collaboration with colleagues from across Asia and beyond. I especially appreciated the support and encouragement of the institute's directors, Phạm Thị Ngọc Yến (succeeded in 2015 by Nguyễn Việt Sơn) and Eric Castelli.

In China, I was fortunate to receive practical and administrative assistance from the Dongba Culture Research Institute \zh{丽江市东巴文化研究院} in Lijiang and the Horse-Tea Road Culture Research Centre \zh{云南大学茶马古道文化研究所} in Kunming. At Yunnan University, I am grateful to Duan Bingchang \zh{段炳昌}, Wang Weidong \zh{王卫东}, Zhao Yanzhen \zh{赵燕珍}, and Yang Liquan \zh{杨立权} for their skillful handling of fieldwork-related administrative matters. 

My deepest gratitude goes to Latami Wangyong \zh{拉他咪王勇} (Latami Dashi \zh{拉他咪达石}), who welcomed me into his world and supported my work with his mother, Mrs.~Latami Daeshilamu \zh{拉他咪打史拉姆} (\ipa{lɑ˧tʰɑ˧mi˥ ʈæ˧ʂɯ˧-ɬɑ˩mv̩˩}). My main language consultant, Mrs.~Latami Daeshilamu, and the other Yongning Na speakers who collaborated with me showed extraordinary patience and generosity. Their contribution to this project is immeasurable. I also thank Picus Ding for introducing me to Latami Wangyong.

I am indebted to many fellow researchers for stimulating discussions and invaluable feedback on drafts: Lamu Gatusa \zh{拉木·嘎吐萨} (pen-name: Shi Gaofeng \zh{石高峰}), Liberty Lidz, Christine Mathieu, Pascale-Marie Milan and Ho Sana \zh{何撒娜}. My heartfelt thanks to Roselle Dobbs, for her sharp editorial eye and deep dives into linguistic nuance, and to Nathan Hill for a careful revision of the entire manuscript. I also extend my appreciation to the three anonymous reviewers, whose wonderfully thorough and helpful feedback helped sharpen this book. Many others provided insightful comments on draft chapters: Chen Yen-ling \zh{陳彥伶}, Katia Chirkova, Denis Creissels, Stéphane Gros, Guillaume Jacques, Martine Mazaudon, Boyd Michailovsky, Frédéric Pain, Phạm Thị Thu Hà, Annie Rialland, Martine Toda, and Meng Yang. 

A special note of thanks to Séverine Guillaume, the engineer in charge of the Pangloss Collection, for her assistance in archiving the annotated audio recordings that underpin this work. I am also grateful to Luise Dorenbusch for converting the original draft to \LaTeX{}, as well as to 
 %and for working out \textit{haute couture} technical solutions for various typesetting issues; 
Guillaume Jacques, Thomas Pellard, and Sebastian Nordhoff for their \LaTeX{} expertise. Jérôme Picard kindly created the map. 

Many thanks to the series editor, Martin Haspelmath, and to Language Science Press coordinator Sebastian Nordhoff, for taking care of the whole editorial process, and for making the experience of publishing a second edition uncannily simple and straightforward. I am also deeply appreciative of the many volunteers who proofread the volume (the list of names appears in the colophon: the legal notices page). Any remaining errors are mine alone.

Many thanks to my wife and my daughter for their patience and support. 

I am also grateful to readers of the first edition who pointed out passages in need of improvement. Their keen observations helped refine this second edition. 

%I would also like to express my gratitude for the support, advice and inspiration I received from many outstanding colleagues during the period when this second edition was in preparation. I consider myself extremely fortunate to work in a~vibrant and collegial international environment. Special thanks go to Laurent Besacier, Danièle Bourcier, Isabelle Bril, Maximin Coavoux, Lise Crevier-Buchman, Didier Demolin, Karën Fort, Zlatka Guentchéva, Samia Naïm, Solange Rossato, Roland Trouville, Bonny Sands, Mark Van de Velde, Elizabeth Zeitoun and Jeanne Zerner.

% Right overhead missing; added with special command
\rohead{Acknowledgments}

Fieldwork on Yongning Na was funded through two grants from the \textit{Agence Nationale de la Recherche} (ANR, France): ``Phylogenetic assessment of Southern Qiangic" (PASQi, ANR-07-JCJC-0063) and ``Himalayan corpora" (Himalco, ANR-12-CORP-0006). The present book is also a~contribution to the work packages “Evolutionary approaches to phonology: New goals and new methods” and “LabField: Bringing the lab to the field” of 
the Labex project “Empirical Foundations of Linguistics” (ANR-10-LABX-0083). 

More broadly, the research presented here took shape over a~period during which I was involved in several other ANR-funded projects, including “Computational Language Documentation by 2025” (CLD2025, ANR-19-CE38-0015-04), “Probing neural representations for typological signal” (DeepTypo, ANR-23-CE{\allowbreak}38-0003), and “Glottalization in the light of Machine Learning” (Glot-TAL, ANR-24-CE38-3766) as well as an NSF project, “DEL: Discovering and demonstrating linguistic features for language documentation” (DEL, NSF-1761548). Although the second edition of this book was not an intended outcome of these projects, they provided a~stimulating research environment, fostering discussions, collaborations, and methodological advances that indirectly contributed to its development. Many thanks to Martine Adda, Gilles Adda, Katia Chirkova, Elisabeth Delais-Roussarie, Séverine Guillaume, Guillaume Jacques, Aimée Lahaussois, Graham Neubig, Claire Pillot-Loiseau, Rachid Ridouane, Trần Thị Thủy Hiền and Guillaume Wisniewski for having me on board their projects.
% \printbibliography[heading=subbibliography]
% \end{refsection}

So many people have contributed to this project that I am sure I have inadvertently left someone out. My apologies to those whose names should be here.