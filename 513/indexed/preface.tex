\addchap{\lsPrefaceTitle}

\epigraph{That looked good. Yes, that looked very good. In fact it went on looking better and better, straight along — until by-and-by it grew into positive \textit{proof}. And then he put the matter at once out of his mind, for he had a private instinct that a proof once established is better left so.}{(Mark Twain, \textit{The man that corrupted Hadleyburg}, 1899.)}

I have not put this book out of my mind~-- as a settled, authoritative account~-- since its publication in April 2017. Data collection and analysis have continued, leading to improvements in the dictionary, which recently reached version 2 \citep{michaud_et_al_na_dict_2024}, as well as in the online collection of audio and video materials, where the number of fully transcribed and translated narratives reached twenty-nine in 2024. In the course of continuing documentation and description work, the analysis presented in the book has undergone further verification. The findings have held up well, and the volume’s core analytical content remains unchanged. This second edition is a~wonderful opportunity to %correct isolated errors and 
incorporate some refinements. 

To facilitate navigation between Na texts and their linguistic analysis, this edition introduces one-click links from the examples discussed in the book to the online corpus of Yongning Na hosted in the Pangloss Collection. This enhancement promotes direct consultation of the primary data,
%This addition provides direct access to primary data, 
encouraging readers to engage more deeply with the material. 
% Regex to check that no DOI lacks its 'S' or 'W' element in the address: \d{7}\\#\d

%Visual refinements have also been made to enhance readability. Notably, the colour gradients in the legend of Map~\ref{map:1-1} (Chapter~\ref{chap:introduction}) have been improved, making it clearer. 

The discussion of intonation was expanded, aiming to provide a~cogent, detailed argument about this important and delicate topic (in Chapters~8 and 9). In the first edition of this book, there was no independent chapter on intonation. Observations about Na intonation were contained within a chapter entitled \textit{From surface phonological tone to phonetic realization}, which covered both intonation and the phonetic implementation of tone. While this title was not technically incorrect~-- given the broad definition of intonation adopted here, whereby intonation encompasses all aspects of the concrete phonetic realization of speech that are not directly predictable from phonemic contrasts~--, it was nonetheless misleading. The phrasing suggested that the entire chapter would be devoted to the phonetic implementation of tonal sequences, whereas its scope was broader. 

The decision to avoid the term \textit{intonation} in the chapter title reflected a~pragmatic concern about the work's reception. In a~scholarly landscape where the \is{autosegmental-metrical models}autosegmental-metrical approach enjoys widespread acceptance, including in the field of language documentation and description \citep{himmelmann_prosodic_2008}, I wished to contribute to a~broader scholarly dialogue by raising fundamental questions anew and examining several frameworks for intonation analysis. The aim is not only to explore diverse perspectives 
but also to make a~case for an alternative approach, arguing that the \is{autosegmental-metrical models}autosegmental-metrical model is not the most suitable for the study of intonation in tonal languages. However, I was aware that a~critical engagement with the mainstream model could be perceived as contentious, potentially jeopardizing a~manuscript's acceptance for publication. A~degree of caution thus seemed advisable.  

The second edition of this book provides an opportunity to devote two full chapters to %the important topic of 
intonation. Chapter \ref{chap:IntonationTheoryAndTypology} engages with the theoretical landscape of intonation studies, establishing theoretical and typological foundations through an open discussion of the key issues at stake. Chapter~\ref{chap:IntonationDescriptionAndAnalysis} turns to the specifics of Yongning Na, presenting the detailed empirical work delving into its intonation system as well as issues of phonetic implementation.

The journey of this book since its initial publication has been enriched by feedback from its readers. Through personal correspondence and published reviews \citep{alves_review_2017, konoshenko_review_2018, gros_review_2018, rialland_compterendu, LiZH2023_review}, scholars have pointed out areas where further clarification would be helpful. Their precious suggestions have been gratefully addressed through targeted revisions. 

Finally, the entire volume has undergone a fresh round of editorial polishing. Writing about tone in an accessible manner is no easy task; the complexity of the subject matter makes clarity all the more essential. The bulk of the effort in preparing this second edition has therefore been devoted to improving fluency, precision, and stylistic consistency. 

As I present this revised edition, I only wish I had made the topic even more accessible to a wider audience, and I apologize to readers who may still find the volume a demanding read. 

%I will now get back to synchronic and diachronic research with the hope of being able to deliver fresh insights into the fascinating Yongning Na language and culture.