\chapter{Combinations of nouns with grammatical elements}
\label{chap:combinationsofnounswithgrammaticalwords}

This chapter brings together data on a~wide range of constructions containing nouns, from morphological derivation~-- mostly nouns with gender suffixes~-- and {reduplication}, to combinations of nouns with particles in discourse. 


\section{Derivational affixes: Gender suffixes and kinship prefix}
\label{sec:thegendersuffixes}
\is{derivation!morphological|(}
\subsection{Introduction to the three main gender suffixes}
\label{sec:thegendersuffixesintro}

The most common \is{derivation!morphological}derivational affixes in Na are the gender suffixes /\ipa{-mi}/, \mbox{/\ipa{-zo}/}, and /\ipa{-pʰv̩}/,\footnote{At this stage, no tone is indicated for these three suffixes, as establishing their tonal category is a~key objective of \sectref{sec:thegendersuffixes}.} which carry
the meanings ‘female, mother’, ‘son, young’, and ‘male’, respectively. For instance, in /\ipa{ɬi˧mi˧}/
‘female roebuck’, /\ipa{ɬi˩zo˩}/ ({variant}: /\ipa{ɬi˧zo\#˥}/) ‘young roebuck’, and /\ipa{ɬi˩pʰv̩˩}/
({variant}: /\ipa{ɬi˧pʰv̩\#˥}/) ‘male roebuck’ \citep[177-179]{lidz2010}.\footnote{Liberty Lidz's notations are /\ipa{-mi33}/, /\ipa{-zɔ33}/, and /\ipa{-pʰu33}/. Consultant F4's word for ‘grandmother's brother’, /\ipa{ə˧pʰv̩˧}/, is glossed by L.~Lidz as ‘grandfather (father of mother or father)’ in the Luoshui dialect and transcribed as /\ipa{ɑ33-pʰv̩33}/ (p. 766), but it also appears as /\ipa{ɑ33-pʰu33}/ in some instances, e.g.~pp. 269, 503, and 763.}

The suffixes /\ipa{-mi}/ and /\ipa{-zo}/ also serve as augmentative and diminutive markers,
respectively. The association of ‘mother’ with ‘large’ and ‘son’ with ‘small’ is common in languages of the
area (see \citealt{mazaudon2003b} on \ili{Tamang} and the cross\babelhyphen{nobreak}linguistic discussion by \citealt{matisoff1992}). For
instance, /\ipa{kʰɤ˧˥}/ ‘basket (carried on the back)’ yields /\ipa{kʰɤ˧mi˥\$}/ ‘large basket’ and
/\ipa{kʰɤ˧zo\#˥}/ ‘small basket’. 

In many suffixed forms, the original augmentative or diminutive
meaning has faded. For instance, {monosyllabic} /\ipa{ljɤ˩˥}/ and suffixed /\ipa{ljɤ˩mi˥}/ both refer to the same object, namely the major
(supporting) beams of a structure. There is no diminutive counterpart ($\dagger${\kern2pt}\ipa{ljɤ˩-zo\#˥}) to /\ipa{ljɤ˩mi˥}/. Similarly, for purlins, called /\ipa{ʐv̩˩ɭɯ˥}/, an augmentative form $\dagger${\kern2pt}\ipa{ʐv̩˩ɭɯ˥-mi˩} is not available. Instead, size distinctions are made using
adjectives such as /\ipa{tɕi˩\textsubscript{a}}/ ‘small’ and /\ipa{ɖɯ˩\textsubscript{a}}/ ‘large’. (On the association of adjectives with nouns in Yongning Na, see \sectref{sec:productiveconstruction}.) 

Etymologically, the second syllable of /\ipa{ɲi˧mi\#˥}/ ‘sun’ and /\ipa{ɬi˧mi˧}/ ‘moon’ {\linebreak}probably originates in the same morpheme /\ipa{mi}/. Since these two words have the same structure in \ili{Naxi} (/\ipa{ɲi˧me˧}/,
/\ipa{he˧me˧}/) and \ili{Laze} (/\ipa{ɲie˧mie˧}/, /\ipa{ɬie˧mie˧}/), two  closely related languages, their disyllabic status appears to have historical depth. The \is{suffixes}suffix /\ipa{zo}/ in /\ipa{ɲi˧zo\#˥}/ ‘fish’, another
disyllabic noun without a~{monosyllabic} counterpart, also has a parallel in \ili{Laze} (/\ipa{ze˧}/ ‘son’,
/\ipa{ɲi˩ze˥}/ ‘fish’), whereas \ili{Naxi} retains a~{monosyllabic} form, /\ipa{ɲi˧}/.


The discussion below examines gender suffixes in names of animals and peoples, as well as their
augmentative and diminutive uses. Needless to say, words
that contain a~/\ipa{mi}/ syllable of different origin were excluded, such as the name of the Yongning monastery,
/\ipa{ɖæ˩mi˧}/, a~\is{loanwords}loanword from \ili{Tibetan} \textit{dgra med}. 

The suffixes for \textit{female}, \textit{young} and \textit{male} are related to the free morphemes /\ipa{mi˩˧}/, /\ipa{zo˥}/, and /\ipa{pʰv̩˧}/, which
appear in contexts such as (\ref{ex:isfemale})\babelhyphen{nobreak}(\ref{ex:ismale}). Here, the evolution from the nouns to the \is{derivation!morphological}derivational affixes is straightforward. 

\begin{exe}
	\ex
	\label{ex:isfemale}
	\ipaex{ʈʂʰɯ˧ {\kern2pt}|{\kern2pt} mi˩ ɲi˥.}\\
	\gll ʈʂʰɯ˥	mi˩˧	ɲi˩\\
	\textsc{dem.prox}		female	\textsc{cop}\\
	\glt ‘This is a~female.’
\end{exe}

\begin{exe}
	\ex
	\label{ex:isyoung}
	\ipaex{ʈʂʰɯ˧ {\kern2pt}|{\kern2pt} zo˧ ɲi˥.}\\
	\gll ʈʂʰɯ˥	zo˥	ɲi˩\\
	\textsc{dem.prox}		son/young/male	\textsc{cop}\\
	\glt ‘This is a~young/male.’
\end{exe}

\begin{exe}
	\ex
	\label{ex:ismale}
	\ipaex{ʈʂʰɯ˧ {\kern2pt}|{\kern2pt} pʰv̩˧ ɲi˩.}\\
	\gll ʈʂʰɯ˥	pʰv̩˧	ɲi˩\\
	\textsc{dem.prox}		male	\textsc{cop}\\
	\glt ‘This is a~male.’
\end{exe}

The free forms /\ipa{zo˥}/ ‘young/male’ and /\ipa{pʰv̩˧}/ ‘male’ have different tones (H and M, respectively), whereas the suffixes /\ipa{-zo}/ and /\ipa{-pʰv}/ always have the same tonal patterns, even sharing the same tonal variants, as in the example of ‘roebuck’. However, \isi{neutralization} of tonal oppositions on nouns in the process of \isi{grammaticalization} as gender suffixes is not thoroughgoing: the tonal behaviour of /\ipa{-zo}/ and /\ipa{-pʰv}/ differs from that of the female \is{suffixes}suffix /\ipa{-mi}/. 

From a~tonal point of view, there is thus evidence that these three \is{derivation!morphological}derivational elements have become distinct from the free nouns in which they originate. This is reminiscent of classifiers: the study of classifiers provided in Chapter \ref{chap:classifiers} reveals that the tone system of classifiers is not identical to that of free nouns and that the tone of a~classifier is not necessarily the closest equivalent of that of the noun from which it \is{derivation!morphological}derives. To repeat an example from \sectref{sec:howthetonalcategorieswerebroughtoutandlabelled}, there are two tonal correspondences among classifiers for H-tone nouns: one illustrated by ‘beam’, /\ipa{ɖʐo˥}/, which has /\ipa{ɖʐo˥\textsubscript{a}}/ (category H\textsubscript{a}) as its self\babelhyphen{nobreak}classifier, and the other illustrated by /\ipa{kɯ˥}/ ‘star’, which yields /\ipa{kɯ˧\textsubscript{b}}/ (M\textsubscript{b} tone category) as a~self\babelhyphen{nobreak}classifier. Seen in this light, differences in tone between a~noun as a~full form and as a~\is{derivation!morphological}derivational \is{suffixes}suffix are not a~particularly out-of-the-way finding in the context of Yongning Na \isi{morphotonology}.


The difference in tone patterns between /\ipa{-mi}/ and the other two suffixes suffices to establish that suffixes are not toneless. But ascertaining the tones of these suffixes is not an easy matter. A~simple test consists in combining them with an M-tone noun: the M tone has properties that make it suitable for use in tonal tests. M can be followed by any tone (unlike H, which can only be followed by L), and it does not spread (unlike L), so it would seem to offer the best possible context for the lexical tone of the following morpheme to manifest itself in. This works out well for verbs: a~useful tonal test consists in observing a~verb's tone after an M-tone morpheme such as the {negation} \is{prefixes}prefix, /\ipa{mɤ˧-}/. In this context, H-tone verbs surface with H tone, MH-tone verbs with MH tone, and so on (see \sectref{sec:overview}). But the three gender suffixes all yield the same result after an M-tone {monosyllable}: for instance, /\ipa{lɑ˧}/ ‘tiger’ yields /\ipa{lɑ˧mi\#˥}/ ‘female tiger’, /\ipa{lɑ˧zo\#˥}/ ‘baby tiger’, and /\ipa{lɑ˧pʰv̩\#˥}/ ‘male tiger’. After disyllabic M-tone nouns, on the other hand, the results are different: thus, /\ipa{si˧gɯ˧}/ ‘lion’ yields /\ipa{si˧gɯ˧-mi˩}/ ‘female lion’, with L tone on the suffix, and /\ipa{si˧gɯ˧-zo\#˥}/ ‘baby lion’, with a~floating H tone. On this slender basis, the two sets of suffixes are provisionally transcribed as carrying lexical L and H tone, respectively. They are transcribed hereafter as //\ipa{-mi˩}//, //\ipa{-zo˥}//, and //\ipa{-pʰv̩˥}//. It must be cautioned that this tentative identification does by no means encapsulate all the information about the tonal behaviour of these two types of suffixes, which is set out in tabular form below.


\subsection{The facts}
\label{sec:thegendersuffixesfacts}

Tables~\ref{tab:gendertwosyllableslm} to \ref{tab:genderthreesyllables}
present the data concerning the suffixes //\ipa{-mi˩}//, //\ipa{-zo˥}// and \mbox{//\ipa{-pʰv̩˥}//}. The examples are arranged by tone of the suffixed form. Tables~\ref{tab:gendertwosyllableslm} to \ref{tab:gendertwosyllablesrest} present disyllables, and Tables \ref{tab:genderthreesyllablesdisyllablesm} and \ref{tab:genderthreesyllables} present 
trisyllables. Nouns in which the suffixes are augmentative or diminutive
and not gender suffixes are italicized (i.e.\ nouns that do not refer
to animals or ethnic groups). As elsewhere, a~slash separates
variants. A~dash ‘--’ in a~cell indicates that the
form does not exist: for instance, it is not possible to use a word suffixed with /\ipa{-zo}/ for ‘piglet’ (the attested form is
/\ipa{bæ˧bv̩˥}/). A~double dagger $\ddagger$ preceding a~noun indicates that it is
a~form that was proposed by the investigator and rejected by the
consultant. For instance, “\ipa{ʐæ˩mi\#˥} (\ipa{$\ddagger${\kern2pt}ʐæ˩mi˩})” for ‘female
leopard’ indicates that the tonal {variant} $\ddagger${\kern2pt}\ipa{ʐæ˩mi˩} was proposed by
the investigator on the \isi{analogy} of the existence of an L-tone {variant}
/\ipa{ʑi˩mi˩}/ for ‘female ape’, whose root belongs in the same
category as ‘leopard’, and that it was rejected by the consultant.


	% \label{tab:gendertwo}  %% Commented out on April 30th, 2025: no subtables, only sequentially numbered tables in the entire volume.
    
%	\begin{table}[t]
%		\caption{\label{tab:gendertwosyllableslm}Nouns with gender suffixes or \textsc{augmentative/{\allowbreak}diminutive} suffixes. Disyllabic
%			words. LM\babelhyphen{nobreak}tone roots.}
%		\begin{tabularx}{\textwidth}{ P{21mm} P{16mm} P{20mm} Q Q }
%			\lsptoprule
%			\multirow{3}{21mm}{correspon\-dences} & root & \multicolumn{3}{l}{suffixed forms}\\ \cmidrule{3-5}
%			& & //\ipa{-mi˩}// & //\ipa{-zo˥}// & //\ipa{-pʰv̩˥}//\\
%			& & ‘female'/{\allowbreak}\textsc{aug} & ‘child'/{\allowbreak}\textsc{dim} & ‘male'\\ \midrule
%			\multirow{2}{21mm}{first type} & sow & \ipa{bo˩mi˧} & -- & \ipa{bo˩pʰv̩˧}\\
%			& hen & \ipa{æ˩mi˧} & -- & --\\
%			& \textit{thumb} & \ipa{lo˩mi˧} & -- & --\\
%			& bee & \ipa{dze˩mi˧} & -- & --\\ \addlinespace \hdashline \addlinespace
%			\multirow{4}{21mm}[1.5\baselineskip]{second type} & yak & \ipa{bv̩˧mi˧} & \ipa{bv̩˧zo\#˥ / bv̩˩zo˩} & \ipa{bv̩˧pʰv̩\#˥ / bv̩˩pʰv̩˩}\\
%			& sparrow & \ipa{ɖʐwæ˧mi˧} & \ipa{ɖʐwæ˧zo\#˥ / ɖʐwæ˩zo˩} & \ipa{ɖʐwæ˧pʰv̩\#˥ / ɖʐwæ˩pʰv̩˩}\\
%			& hawk & \ipa{kɤ˧mi˧} & \ipa{kɤ˧zo\#˥ / kɤ˩zo˩} & \ipa{kɤ˧pʰv̩\#˥ / kɤ˩pʰv̩˩}\\
%			& \textit{food steamer} & \ipa{bv̩˧mi˧ } & \ipa{bv̩˩zo˩ / bv̩˧zo\#˥} & --\\ \addlinespace \hdashline \addlinespace
%			\multirow{5}{21mm}[2\baselineskip]{third type} & weasel & \ipa{dv̩˩mi\#˥ / dv̩˩mi˧} & \ipa{dv̩˩zo\#˥ / dv̩˩zo˧} & \ipa{dv̩˩pʰv̩\#˥ / dv̩˩pʰv̩˧}\\
%			& goose & \ipa{ɑ˩mi\#˥ / ɑ˩mi˧} & \ipa{ɑ˩zo\#˥ / ɑ˩zo˧} & \ipa{ɑ˩pʰv̩\#˥ / ɑ˩pʰv̩˧}\\
%			& lizard & \ipa{dzo˩mi\#˥ / dzo˩mi˧} & \ipa{dzo˩zo\#˥ / dzo˩zo˧} & \ipa{dzo˩pʰv̩\#˥ / dzo˩pʰv̩˧}\\
%			& \textit{ladle} & \ipa{tɕʰo˩mi\#˥ / tɕʰo˩mi˧} & \ipa{tɕʰo˩zo\#˥ / tɕʰo˩zo˧} & --\\
%			& Na (people) & \ipa{nɑ˩mi\#˥ / nɑ˩mi˧} & \ipa{nɑ˩zo\#˥ / nɑ˩zo˧} & --\\ \addlinespace \hdashline \addlinespace
%			\multirow{2}{21mm}[.5\baselineskip]{fourth type}	& jackal & \ipa{pʰɤ˩mi˩} & \ipa{pʰɤ˧zo\#˥ / pʰɤ˩zo˩} & \ipa{pʰɤ˧pʰv̩\#˥ / pʰɤ˩pʰv̩˩}\\
%			& \textit{road, path} & \ipa{ʐɤ˩mi˩} & -- & --\\
%			\lspbottomrule
%		\end{tabularx}
%	\end{table}

	\begin{sidewaystable}[p]
		\caption{\label{tab:gendertwosyllableslm}Nouns with gender suffixes or {augmentative/{\allowbreak}diminutive} suffixes. Disyllabic
			words. LM\babelhyphen{nobreak}tone roots. Nouns in which
the suffixes are augmentative or diminutive and not gender suffixes are italicized.}
		\begin{tabularx}{\textwidth}{ P{21mm} P{19mm} Q Q P{42mm}}
			\lsptoprule
			\multirow{2}{21mm}{correspon\-dences} & root & \multicolumn{3}{l}{suffixed forms}\\ \cmidrule{3-5}
			& & //\ipa{-mi˩}// ‘female'/{\allowbreak}\textsc{aug} & //\ipa{-zo˥}// ‘child'/{\allowbreak}\textsc{dim} & //\ipa{-pʰv̩˥}// ‘male'\\ \midrule
%\multicolumn{3}{l}{suffixed forms}\\ \cmidrule{3-5}
%& & //\ipa{-mi˩}// & //\ipa{-zo˥}// & //\ipa{-pʰv̩˥}//\\
%& & ‘female'/{\allowbreak}\textsc{aug} & ‘child'/{\allowbreak}\textsc{dim} & ‘male'\\ \midrule
			\multirow{2}{21mm}{first type} & sow & \ipa{bo˩mi˧} & -- & \ipa{bo˩pʰv̩˧}\\
			& hen & \ipa{æ̃˩mi˧} & -- & --\\
			& \textit{thumb} & \ipa{lo˩mi˧} & -- & --\\
			& bee & \ipa{dze˩mi˧} & -- & --\\ \addlinespace \hdashline \addlinespace
			\multirow{4}{21mm}[1.5\baselineskip]{second type} & yak & \ipa{bv̩˧mi˧} & \ipa{bv̩˧zo\#˥ / bv̩˩zo˩} & \ipa{bv̩˧pʰv̩\#˥ / bv̩˩pʰv̩˩}\\
			& sparrow & \ipa{ɖʐwæ˧mi˧} & \ipa{ɖʐwæ˧zo\#˥ / ɖʐwæ˩zo˩} & \ipa{ɖʐwæ˧pʰv̩\#˥ / ɖʐwæ˩pʰv̩˩}\\
			& hawk & \ipa{kɤ˧mi˧} & \ipa{kɤ˧zo\#˥ / kɤ˩zo˩} & \ipa{kɤ˧pʰv̩\#˥ / kɤ˩pʰv̩˩}\\
			& \textit{steamer} & \ipa{bv̩˧mi˧} & \ipa{bv̩˩zo˩ / bv̩˧zo\#˥} & --\\ \addlinespace \hdashline \addlinespace
			\multirow{5}{21mm}[2\baselineskip]{third type} & weasel & \ipa{dv̩˩mi\#˥ / dv̩˩mi˧} & \ipa{dv̩˩zo\#˥ / dv̩˩zo˧} & \ipa{dv̩˩pʰv̩\#˥ / dv̩˩pʰv̩˧}\\
			& goose & \ipa{ɑ˩mi\#˥ / ɑ˩mi˧} & \ipa{ɑ˩zo\#˥ / ɑ˩zo˧} & \ipa{ɑ˩pʰv̩\#˥ / ɑ˩pʰv̩˧}\\
			& lizard & \ipa{dzo˩mi\#˥ / dzo˩mi˧} & \ipa{dzo˩zo\#˥ / dzo˩zo˧} & \ipa{dzo˩pʰv̩\#˥ / dzo˩pʰv̩˧}\\
			& \textit{ladle} & \ipa{tɕʰo˩mi\#˥ / tɕʰo˩mi˧} & \ipa{tɕʰo˩zo\#˥ / tɕʰo˩zo˧} & --\\
			& Na (people) & \ipa{nɑ˩mi\#˥ / nɑ˩mi˧} & \ipa{nɑ˩zo\#˥ / nɑ˩zo˧} & --\\ \addlinespace \hdashline \addlinespace
			\multirow{2}{21mm}[.5\baselineskip]{fourth type}	& jackal & \ipa{pʰɤ˩mi˩} & \ipa{pʰɤ˧zo\#˥ / pʰɤ˩zo˩} & \ipa{pʰɤ˧pʰv̩\#˥ / pʰɤ˩pʰv̩˩}\\
			& \textit{road, path} & \ipa{ʐɤ˩mi˩} & -- & --\\
			\lspbottomrule
		\end{tabularx}
%	\end{table}
	\end{sidewaystable}	
	
	\begin{table}[t!]
		\caption{\label{tab:gendertwosyllableslh}Nouns with gender suffixes or {augmentative/{\allowbreak}diminutive} suffixes. Disyllabic words. LH\babelhyphen{nobreak}tone roots. Nouns in which
the suffixes are augmentative or diminutive and not gender suffixes are italicized.}
		\begin{tabularx}{\textwidth}{ P{29mm} P{15mm} P{20mm} Q Q }
			\lsptoprule
			correspondences & root & \multicolumn{3}{l}{suffixed forms}\\ \cmidrule{3-5}
			& & //\ipa{-mi˩}// ‘female'/{\allowbreak}\textsc{aug} & //\ipa{-zo˥}// ‘child'/{\allowbreak}\textsc{dim} & //\ipa{-pʰv̩˥}// ‘male'\\ \midrule	
			\multirow{2}{32mm}{first type} &	leopard & \ipa{ʐæ˩mi\#˥ ($\ddagger${\kern2pt}ʐæ˩mi˩)} & \ipa{ʐæ˩zo\#˥ ($\ddagger${\kern2pt}ʐæ˩zo˩)} & \ipa{ʐæ˩pʰv̩\#˥ ($\ddagger${\kern2pt}ʐæ˩pʰv̩˩)}\\ 
			& monkey & \ipa{ʑi˩mi\#˥ / ʑi˩mi˩} & \ipa{ʑi˩zo\#˥ / ʑi˩zo˩} & \ipa{ʑi˩pʰv̩\#˥ / ʑi˩pʰv̩˩}\\ 
			& buffalo & \ipa{tʰɑ˩mi\#˥ ($\ddagger${\kern2pt}tʰɑ˩mi˩)} & \ipa{tʰɑ˩zo\#˥} & \ipa{tʰɑ˩pʰv̩\#˥}\\ 
			& muntjac & \ipa{tɕʰɯ˩mi\#˥ / tɕʰɯ˩mi˩} & \ipa{tɕʰɯ˩zo\#˥ / tɕʰɯ˩zo˩} & \ipa{tɕʰɯ˩pʰv̩\#˥ / tɕʰɯ˩pʰv̩˩}\\ 
			& \textit{plane (tool)} & \ipa{tʰi˩mi\#˥ ($\ddagger${\kern2pt}tʰi˩mi˩)} & \ipa{tʰi˩zo\#˥ ($\ddagger${\kern2pt}tʰi˩zo˩)} & --\\ \addlinespace \hdashline \addlinespace
			second type (isolated example) & \textit{slope} & \ipa{to˩mi˩} & \ipa{to˩zo˩} & --\\ \addlinespace \hdashline \addlinespace
			third type & \textit{plain} & \ipa{di˧mi˧} & -- & --\\ 
			 & woman & \ipa{mv̩˧mi˧} & \ipa{mv̩˩zo˩} & --\\ 
			\lspbottomrule
		\end{tabularx}
	\end{table}
	
	\begin{table}[t!]
		\caption{\label{tab:gendertwosyllablesm}Nouns with gender suffixes or {augmentative/{\allowbreak}diminutive} suffixes. Disyllabic words. M-tone roots. Nouns in which
the suffixes are augmentative or diminutive and not gender suffixes are italicized. Only one type of {correspondence}.}
		\begin{tabularx}{\textwidth}{ P{20mm} P{38mm} Q Q }
			\lsptoprule
			root & \multicolumn{3}{l}{suffixed forms}\\ \cmidrule{2-4}
			& //\ipa{-mi˩}// ‘female'/{\allowbreak}\textsc{aug} & //\ipa{-zo˥}// ‘child' & //\ipa{-pʰv̩˥}// ‘male'\\ \midrule
			tiger & \ipa{lɑ˧mi\#˥} & \ipa{lɑ˧zo\#˥} & \ipa{lɑ˧pʰv̩\#˥}\\
			goral & \ipa{se˧mi\#˥} & \ipa{se˧zo\#˥} & \ipa{se˧pʰv̩\#˥}\\
			\textit{message} & \ipa{qʰwæ˧mi\#˥} & -- & --\\
			\lspbottomrule
		\end{tabularx}
	\end{table}
	
	
	\begin{table}[t!]
		\caption{\label{tab:gendertwosyllablesl}Nouns with gender suffixes or {augmentative/{\allowbreak}diminutive} suffixes. Disyllabic
			words. L-tone roots. Nouns in which
the suffixes are augmentative or diminutive and not gender suffixes are italicized. Only one type of {correspondence}.}
		{\setlength\tabcolsep{4pt}
		\begin{tabularx}{\textwidth}{ P{15mm} P{34mm} Q Q }
			\lsptoprule
			root & \multicolumn{3}{l}{suffixed forms}\\ \cmidrule{2-4}
			& //\ipa{-mi˩}// ‘female'/{\allowbreak}\textsc{aug} & //\ipa{-zo˥}// ‘child'/{\allowbreak}\textsc{dim} & //\ipa{-pʰv̩˥}// ‘male'\\ \midrule
			daughter & \ipa{mv̩˧mi˧} & -- & --\\
			sheep & \ipa{jo˧mi˧} & \ipa{jo˧zo\#˥ / jo˩zo˩} & \ipa{jo˧pʰv̩\#˥ / jo˩pʰv̩˩}\\
			roebuck & \ipa{ɬi˧mi˧} & \ipa{ɬi˧zo\#˥ / ɬi˩zo˩} & \ipa{ɬi˧pʰv̩\#˥ / ɬi˩pʰv̩˩}\\
			\textit{bottle} & \ipa{kɤ˧mi˧} & \ipa{kɤ˩zo˩ ($\ddagger${\kern2pt}kɤ˧zo\#˥)} & --\\
			\textit{river} & \ipa{dʑɯ˧mi˧} & -- & --\\
			\lspbottomrule
		\end{tabularx}}
	\end{table}
	
	\begin{table}[t!]
		\caption{\label{tab:gendertwosyllablesh}Nouns with gender suffixes or {augmentative/{\allowbreak}diminutive} suffixes. Disyllabic words. H-tone roots. Nouns in which
the suffixes are augmentative or diminutive and not gender suffixes are italicized.}
		\begin{tabularx}{\textwidth}{ P{21mm} P{25mm} P{20mm} Q Q }
			\lsptoprule
			\multirow{2}{21mm}{correspon\-dences} & root & \multicolumn{3}{l}{suffixed forms}\\ \cmidrule{3-5}
			& & //\ipa{-mi˩}// ‘female'/{\allowbreak}\textsc{aug} & //\ipa{-zo˥}// ‘child'/{\allowbreak}\textsc{dim} & //\ipa{-pʰv̩˥}// ‘male'\\ \midrule
			first type & cow & \ipa{ʝi˩mi˩} & \ipa{ʝi˧zo\#˥} & \ipa{ʝi˧pʰv̩\#˥}\\
			& horse & \ipa{ʐwæ˩mi˩} & \ipa{ʐwæ˧zo\#˥} & \ipa{ʐwæ˧pʰv̩\#˥}\\
			& dog & \ipa{kʰv̩˩mi˩} & \ipa{kʰv̩˧zo\#˥} & \ipa{kʰv̩˧pʰv̩\#˥}\\ \addlinespace \hdashline \addlinespace
			second type & \ili{Pumi} (people) & \ipa{bɤ˧mi\#˥} & \ipa{bɤ˧zo\#˥} & --\\
			& pheasant & \ipa{ho˧mi\#˥} & \ipa{ho˧zo\#˥} & \ipa{ho˧pʰv̩\#˥}\\
			& \textit{cooking pan} & \ipa{v̩˧mi\#˥} & \ipa{v̩˧zo\#˥} & --\\ \addlinespace \hdashline \addlinespace
			third type & \textit{door} & \ipa{kʰi˧mi˧} & \ipa{kʰi˧zo\#˥} & --\\
			& \textit{canal} & \ipa{qʰæ˧mi˧} & \ipa{qʰæ˧zo\#˥} & --\\
			& \textit{tree trunk}  & \ipa{ɻ̩̃˧mi˧} & -- & --\\
			\lspbottomrule
		\end{tabularx}
	\end{table}
	
	
	\begin{table}[t!]
		\caption{\label{tab:gendertwosyllablesmh}Nouns with gender suffixes or {augmentative/{\allowbreak}diminutive} suffixes. Disyllabic words. MH\babelhyphen{nobreak}tone roots. Nouns in which
the suffixes are augmentative or diminutive and not gender suffixes are italicized.}
		\begin{tabularx}{\textwidth}{ P{29mm} P{14mm} Q Q Q }
			\lsptoprule
			correspondences & root & \multicolumn{3}{l}{suffixed forms}\\ \cmidrule{3-5}
			& & //\ipa{-mi˩}// ‘female'/{\allowbreak}\textsc{aug} & //\ipa{-zo˥}// ‘child'/{\allowbreak}\textsc{dim} & //\ipa{-pʰv̩˥}// ‘male'\\ \midrule
			first type & cat & \ipa{hwɤ˧mi˥\$} & \ipa{hwɤ˧zo\#˥ / hwɤ˧zo˥\$} & \ipa{hwɤ˧pʰv̩\#˥ / hwɤ˧pʰv̩˥\$}\\
			& doe & \ipa{ʈʂʰæ˧mi˥\$} & \ipa{ʈʂʰæ˧zo\#˥ / ʈʂʰæ˧zo˥\$} & \ipa{ʈʂʰæ˧pʰv̩\#˥ / ʈʂʰæ˧pʰv̩˥\$}\\
			& goat & \ipa{tsʰɯ˧mi˥\$} & \ipa{tsʰɯ˧zo\#˥ / tsʰɯ˧zo˥\$} & \ipa{tsʰɯ˧pʰv̩\#˥ / tsʰɯ˧pʰv̩˥\$}\\
			& crane & \ipa{ʁv̩˧mi˥\$} & \ipa{ʁv̩˧zo\#˥ / ʁv̩˧zo˥\$} & \ipa{ʁv̩˧pʰv̩\#˥ / ʁv̩˧pʰv̩˥\$}\\
			& wasp & \ipa{tɕɯ˧mi˥\$} & \ipa{tɕɯ˧zo\#˥ / tɕɯ˧zo˥\$} & \ipa{tɕɯ˧pʰv̩\#˥ / tɕɯ˧pʰv̩˥\$}\\
			& \textit{basket} & \ipa{kʰɤ˧mi˥\$} & \ipa{kʰɤ˧zo˥\$ ($\ddagger${\kern2pt}kʰɤ˧zo\#˥)} & --\\
			& \textit{needle} & \ipa{ʁo˧mi˥\$} & \ipa{ʁo˧zo\#˥ ($\ddagger${\kern2pt}ʁo˧zo˥\$)} & --\\
			& \textit{scales} & \ipa{tɕɯ˧mi˥\$} & \ipa{tɕɯ˧zo˥\$ ($\ddagger${\kern2pt}tɕɯ˧zo\#˥)} & --\\
			& \textit{bowl} & \ipa{qʰwɤ˧mi˥\$} & \ipa{qʰwɤ˧zo˥\$ ($\ddagger${\kern2pt}qʰwɤ˧zo\#˥)} & --\\
			& \textit{stomach} & \ipa{hu˧mi˥\$} & -- & --\\ \addlinespace \hdashline \addlinespace
			second type (isolated example) & \textit{building} & \ipa{ʑi˧mi˧} & -- & --\\
			\lspbottomrule
		\end{tabularx}
	\end{table}
	
	
	\begin{table}[p!]
		\caption{\label{tab:gendertwosyllablesrest}Nouns with gender suffixes or {augmentative/{\allowbreak}diminutive} suffixes. Disyllabic words without a~corresponding {monosyllable}. Nouns in which
the suffixes are not gender suffixes are italicized.}
		\begin{tabularx}{\textwidth}{ P{22mm} P{26mm} P{20mm} Q Q }
			\lsptoprule
			\multirow{2}{22mm}{possible root tone} & meaning & \multicolumn{3}{l}{suffixed forms}\\ \cmidrule{3-5}
			& & //\ipa{-mi˩}// ‘female'/{\allowbreak}\textsc{aug} & //\ipa{-zo˥}// ‘child'/{\allowbreak}\textsc{dim} & //\ipa{-pʰv̩˥}// ‘male'\\ \midrule
			M or H & 	water buffalo & \ipa{dʑi˧mi\#˥} & \ipa{dʑi˧zo\#˥ ($\ddagger${\kern2pt}dʑi˩zo˩)} & \ipa{dʑi˧pʰv̩\#˥}\\
			& 	granddaughter & \ipa{ʐv̩˧mi\#˥} & -- & --\\
			& 	\textit{sun} & \ipa{ɲi˧mi\#˥} & -- & --\\ \addlinespace \hdashline \addlinespace
			L or LM & duck & \ipa{bæ˧mi˧} & \ipa{bæ˧zo\#˥} & \ipa{bæ˧pʰv̩\#˥}\\
			& 	\textit{large vat} & \ipa{dzo˧mi˧} & \ipa{dzo˧zo\#˥} & --\\
			& 	\textit{sword} & \ipa{ʁæ˧mi˧} & \ipa{ʁæ˧zo\#˥} & --\\ \addlinespace \hdashline \addlinespace
			\multirow{2}{22mm}{LH, L, H or MH} & 	\textit{tummy, belly} & \ipa{bi˧mi˧} & -- & --\\
			& 	fox & \ipa{ɖɤ˧mi˧} & -- & --\\
			& 	little sister & \ipa{go˧mi˧} & -- & --\\
			& 	\textit{moon} & \ipa{ɬi˧mi˧} & -- & --\\
			& 	\textit{king, lord} & \ipa{ʁo˧mi˧} & -- & --\\
			& 	louse & \ipa{ʂe˧mi˧} & -- & --\\
			& 	wife & \ipa{ʈʂʰv̩˧mi˧} & -- & --\\ \addlinespace \hdashline \addlinespace
			LM or LH & 	frog & \ipa{pɤ˩mi˩} & -- & \ipa{pɤ˩pʰv̩˩}\\ \addlinespace \hdashline \addlinespace
			LM, LH or H & 	\textit{tongue} & \ipa{hi˩mi˩} & -- & --\\
			& 	\textit{large comb} & \ipa{pv̩˩mi˩} & -- & --\\
			& 	\textit{axe} & \ipa{bi˩mi˩} & -- & --\\
			& 	\textit{heart} & \ipa{nv̩˩mi˩} & -- & --\\
			& 	niece & \ipa{ze˩mi˩} & -- & --\\
			& 	mule & \ipa{ɖɯ˩mi\#˥} & \ipa{ɖɯ˩zo\#˥} & \ipa{ɖɯ˩pʰv̩\#˥}\\ \addlinespace \hdashline \addlinespace
			LM or H & 	\textit{bow} & \ipa{ʐv̩˩mi˩} & \ipa{ʐv̩˧zo\#˥} & --\\ \addlinespace \hdashline \addlinespace
			any tone except LH & 	fish & -- & \ipa{ɲi˧zo\#˥} & --\\ \addlinespace \hdashline \addlinespace
			unclear & 	hwamei (bird) & \ipa{tɕɯ˩mi˥} & -- & --\\
			& 	\textit{cigarette lighter} & \ipa{tse˧mi˥} & -- & --\\
			\lspbottomrule
		\end{tabularx}
	\end{table}




	% \label{tab:genderthree}  %% Commented out on April 30th, 2025: no subtables, only sequentially numbered tables in the entire volume.
	\begin{table}%[t]
		\caption{\label{tab:genderthreesyllablesdisyllablesm}Nouns with gender suffixes or augmentative/diminutive
			suffixes. Three\babelhyphen{nobreak}syllable words derived from M-tone disyllables.}
		
		{\setlength\tabcolsep{4pt}
			\begin{tabularx}{\textwidth}{ P{18mm} Q l l l }
				\lsptoprule
				\multirow{2}{18mm}{correspon\-dences} & root & \multicolumn{3}{l}{suffixed forms}\\ \cmidrule{3-5}
				& & //\ipa{-mi˩}// ‘female' & //\ipa{-zo˥}// ‘child' & //\ipa{-pʰv̩˥}// ‘male'\\ \midrule
				\multirow{2}{18mm}{first\\ type} & rabbit & \ipa{tʰo˧li˧-mi˩} & \ipa{tʰo˧li˧-zo\#˥} & \ipa{tʰo˧li˧-pʰv̩\#˥}\\
				& snake & \ipa{ʐv̩˧bæ˧-mi˩} & \ipa{ʐv̩˧bæ˧-zo\#˥} & \ipa{ʐv̩˧bæ˧-pʰv̩\#˥}\\
				& lion & \ipa{si˧gɯ˧-mi˩} & \ipa{si˧gɯ˧-zo\#˥} & \ipa{si˧gɯ˧-pʰv̩\#˥}\\
				& earth\-worm  & \ipa{dʑɯ˧dv̩˧-mi˩} & \ipa{dʑɯ˧dv̩˧-zo\#˥} &
				\ipa{dʑɯ˧dv̩˧-pʰv̩\#˥}\\  \addlinespace \hdashline \addlinespace
				\multirow{2}{18mm}{second\\ type} & demon & \ipa{si˧bv̩˧-mi\#˥} & \ipa{si˧bv̩˧-zo\#˥} & --\\
				& ghost & \ipa{tsʰo˧qʰwɤ˧-mi\#˥} & \ipa{tsʰo˧qʰwɤ˧-zo\#˥} & --\\
				& Bai (people) & \ipa{ɬi˧bv̩˧-mi\#˥} & \ipa{ɬi˧bv̩˧-zo\#˥} & --\\
				\lspbottomrule
			\end{tabularx}}
		\end{table}
		
		%Table 4b.
		\begin{table}%[t]
			\caption{\label{tab:genderthreesyllables}Nouns with gender suffixes or augmentative/diminutive
				suffixes. Three\babelhyphen{nobreak}syllable words without a~corresponding disyllable.}
			\begin{tabularx}{\textwidth}{ l Q l l l }
				\lsptoprule
				root tone & root & \multicolumn{3}{l}{suffixed forms}\\ \cmidrule{3-5}
				& & //\ipa{-mi˩}// ‘female' & //\ipa{-zo˥}// ‘child' & //\ipa{-pʰv̩˥}// ‘male'\\ \midrule
				L & bird & \ipa{v̩˩dze˩-mi˩} & \ipa{v̩˩dze˩-zo˩} & \ipa{v̩˩dze˩-pʰv̩˩}\\ \addlinespace \hdashline \addlinespace
				L\# & bat & \ipa{dze˧bɤ˩-mi˩} & \ipa{dze˧bɤ˩-zo˩} & \ipa{dze˧bɤ˩-pʰv̩˩}\\
				& owl & \ipa{mo˧jo˩-mi˩} & \ipa{mo˧jo˩mi˩-zo˩} & \ipa{mo˧jo˩mi˩-pʰv̩˩}\\ \addlinespace \hdashline \addlinespace
				LM+MH\# & wolf & \ipa{õ˩dv̩˧-mi˥} & \ipa{õ˩dv̩˧-zo\#˥} & \ipa{õ˩dv̩˧-pʰv̩\#˥}\\ \addlinespace \hdashline \addlinespace
				H\# & camel & \ipa{njɤ˧mv̩˥-mi˩} & \ipa{njɤ˧mv̩˥mi˩-zo˩} &
				\ipa{njɤ˧mv̩˥mi˩-pʰv̩˩}\\ \addlinespace \hdashline \addlinespace
				unclear & cicada & \ipa{dʑɯ˧dze˧mi\#˥} & -- & --\\
				& vulture & \ipa{se˩gwɤ˩-mi˧} & -- & --\\
				\lspbottomrule
			\end{tabularx}
		\end{table}

	
For two of the items in Table~\ref{tab:gendertwosyllableslm} that lack a~{monosyllabic} counterpart~-- ‘bee' and ‘thumb', which are not synchronically attested as monosyllables~--, the tone of the root can be inferred through \is{reconstruction!internal}internal reconstruction. Since there is a~substantial number of examples (seven) of LM\babelhyphen{nobreak}tone
disyllables corresponding to LM\babelhyphen{nobreak}tone monosyllables, and there is no other attested source for
LM\babelhyphen{nobreak}tone disyllables, /\ipa{dze˩mi˧}/ ‘bee’ and /\ipa{lo˩mi˧}/ ‘thumb’ can be hypothesized to derive historically from LM\babelhyphen{nobreak}tone roots: \ipa{*dze˩˧} and \ipa{*lo˩˧}. 

But for most of the items that lack a~\is{monosyllables}monosyllabic counterpart,
	internal \is{comparative method (historical linguistics)}reconstruction does not lead to a~clear\babelhyphen{nobreak}cut conclusion, because several tone categories of
	roots feed into the same tone categories of \is{disyllables}disyllables. For instance, /\ipa{ɲi˧zo\#˥}/ ‘fish’ may
	have originated in a~\is{monosyllables}monosyllabic root of any tone category except H, and M-tone words with the
	/\ipa{-mi˩}/ \is{suffixes}suffix, such as /\ipa{ɖɤ˧mi˧}/ ‘fox’, may be \is{derivation!morphological}derived from any of the following four
	tone categories of monosyllables: LH, L, H or MH. The last two items in \tabref{tab:gendertwosyllablesrest}, ‘hwamei (a species of bird)' and ‘(cigarette) lighter', carry a~tone that does not correspond to any of the attested correspondences.
	
	The purpose of Tables \ref{tab:gendertwosyllableslm} to \ref{tab:genderthreesyllables} is to provide a~bird’s eye view of the tonal correspondences. It does not
	present information about \isi{etymology} and frequency of use: for instance, that /\ipa{ɻ̩̃˧mi˧}/ ‘treek trunk’ (in \tabref{tab:gendertwosyllablesh}) etymologically means ‘big bone’; that
	/\ipa{po˧lo˧}/ is a~more common form for ‘ram’ than the /\ipa{-pʰv̩˥}/ suffixed form /\ipa{jo˧pʰv̩\#˥}/{\kern2pt}\ipa{≈}{\kern2pt}/\ipa{jo˩pʰv̩˩}/ (literally ‘male sheep’) shown in \tabref{tab:gendertwosyllablesl}; or that
	/\ipa{-zo˥}/ suffixed forms are more common than /\ipa{-pʰv̩˥}/ suffixed forms to refer to male mules and
	water buffalo (\tabref{tab:gendertwosyllablesrest}). Some such facts are adduced in the discussion below; they can be looked up in the corresponding entries in the dictionary \citep{michaud_et_al_na_dict_2024}.
	
	%check on proofs: is it useful to introduce a
	\clearpage
	%here? xyz

	\subsection{Discussion}
	\label{sec:thegendersuffixesdisc}
	
	The tonal correspondences between \is{monosyllables}monosyllabic roots and \is{disyllables}disyllables are not one\babelhyphen{nobreak}to\babelhyphen{nobreak}one. The
	diversity of these correspondences suggests that suffixed disyllables have varying degrees
	of \isi{lexicalization} and historical depth. It would thus be misleading to consider all suffixed forms as the result of a~synchronic, currently productive morphological
	process. 
    
    Semantically, there is a~continuum from disyllables with a~clearly female meaning, such
	as ‘sow’, to those in which the semantic contribution of the \is{suffixes}suffix has become bleached,
	e.g.~/\ipa{kʰv̩˩mi˩}/, which simply means ‘dog’ rather than specifically ‘she\babelhyphen{nobreak}dog’. After such semantic bleaching, suffixes must be added anew to specify gender. For example, the terms for ‘camel calf’ and ‘male camel’ are based on /\ipa{njɤ˧mv̩˥mi˩}/ ‘camel’ and come out as /\ipa{njɤ˧mv̩˥mi˩-zo˩}/ and 
    /\ipa{njɤ˧mv̩˥mi˩-pʰv̩˩}/, 
%    /\ipa{njɤ˧{\linebreak}mv̩˥mi˩-pʰv̩˩}/, 
    respectively. While these forms are readily understandable, the main consultant finds them awkward. Does this imply that /\ipa{-mi˩}/ in /\ipa{njɤ˧mv̩˥mi˩}/ ‘camel’ is still perceived as carrying a {female} meaning? Not necessarily: the slight weirdness of /\ipa{njɤ˧mv̩˥mi˩-zo˩}/ ‘camel calf’ and /\ipa{njɤ˧mv̩˥mi˩-pʰv̩˩}/ ‘male camel’ seems to stem from the sequences /{\dots}\ipa{mi.zo}/ and /{\dots}\ipa{mi.pʰv}/, where the \is{suffixes}suffix, as it were, re\babelhyphen{nobreak}activates the gender denotation of \mbox{/\ipa{-mi˩}/}.
	
	An examination of forms that allow two tonal realizations, such as /\ipa{hwɤ˧zo\#˥}/ and
	/\ipa{hwɤ˧zo˥\$}/ for ‘male kitten’, reveals that tonal variants are item\babelhyphen{nobreak}specific
	(lexicalized). Among the nine words suffixed with \mbox{/\ipa{-zo˥}/} or \mbox{/\ipa{-pʰv̩˥}/} corresponding
	to an MH\babelhyphen{nobreak}tone root, only four allow both \#H and H\$ variants. Interestingly, of the four object	names in this set~-- ‘basket’, ‘needle’, ‘scales’, and ‘bowl’~--, none allows
	tonal variation: each belongs unambiguously to one category (\#H for ‘needle’ and H\$ for the
	other three). By contrast, four of the five animal names allow both variants. The distinction between gender and size suffixes is not rigid: some object names also permit tonal variation. For instance, ‘ladle’ follows the pattern of exceptionless
	tonal duality for LM\babelhyphen{nobreak}tone roots: /\ipa{tɕʰo˩mi\#˥}/{\kern2pt}\ipa{≈}{\kern2pt}/\ipa{tɕʰo˩mi˧}/ and
	/\ipa{tɕʰo˩zo\#˥}/{\kern2pt}\ipa{≈}{\kern2pt}/\ipa{tɕʰo˩zo˧}/. Nevertheless, there appears to be a~statistical tendency for animal names
	to retain greater tonal flexibility, which may reflect speakers' stronger perception of their internal structure.
	
	In some cases, it is possible to identify specific factors that have influenced the current tonal outcomes. For instance, ‘little bottle’
	/\ipa{kɤ˩zo˩}/ lacks the expected \#H-tone {variant} \ipa{$\ddagger${\kern2pt}kɤ˧zo\#˥}. To the consultant,
	the latter form immediately evoked the given name /\ipa{kɤ˧zo\#˥}/. If ‘little
	bottle’ once had both L and \#H tonal variants, pressure to avoid \isi{homophony} may have contributed to the exclusive retention of /\ipa{kɤ˩zo˩}/.
	
	The discussion below follows the order of root noun tone categories as presented in Tables \ref{tab:gendertwosyllableslm} to \ref{tab:genderthreesyllables}. For each tone category, suffixed forms with //\ipa{-mi˩}// are examined first, followed by those with //\ipa{-zo˥}// and //\ipa{-pʰv̩˥}//.
	
	
	\subsubsection{LM\babelhyphen{nobreak}tone roots}
	\label{sec:lmtoneroots}
	
	
	Tables~\ref{tab:gendertwosyllableslm} to \ref{tab:gendertwosyllablesrest} reveal no fewer than four tonal correspondences between LM-tone monosyllables and their suffixed
	forms: LM, as in /\ipa{bo˩mi˧}/ ‘sow’ and /\ipa{æ˩mi˧}/ ‘hen’; M, as in /\ipa{bv̩˧mi˧}/ ‘female yak’,
	/\ipa{ɖʐwæ˧mi˧}/ ‘female sparrow’, and /\ipa{kɤ˧mi˧}/ ‘female falcon’; LM+\#H, as in /\ipa{dv̩˩mi\#˥}/
	‘female weasel’, /\ipa{ɑ˩mi\#˥}/ ‘goose’, /\ipa{dzo˧{\linebreak}mi\#˥}/ ‘female lizard’, and /\ipa{nɑ˩mi\#˥}/ ‘Na
	woman’; and finally L, as in /\ipa{pʰɤ˩mi˩}/ ‘female jackal’.
	
	Three LM\babelhyphen{nobreak}tone nouns outside the semantic domain of animal names also take the /\ipa{-mi˩}/ \is{suffixes}suffix,
	exhibiting three of the four tonal patterns attested above. One has M tone: /\ipa{bv̩˧mi˧}/ ‘big food steamer’ (from
	/\ipa{bv̩˩˧}/ ‘food steamer’). Another has LM+\#H tone: /\ipa{tɕʰo˩mi\#˥}/ ‘big ladle’ (from
	/\ipa{tɕʰo˩˧}/ ‘ladle’). The third has L tone: /\ipa{ʐɤ˩mi˩}/ ‘road, path’~-- which no longer specifically means ‘large road’: the \is{monosyllables}monosyllable /\ipa{ʐɤ˩˧}/, likewise meaning ‘road,	path’, is falling into disuse.
	
	Given the diversity of these patterns, it is difficult to establish the relative chronology of the tone
	rules that produced the four types (LM, M, LM+\#H, and L). Nonetheless, a~few hints may be detected. While LM\babelhyphen{nobreak}tone monosyllables correspond to no fewer than four different tonal categories in
	suffixed forms, LM\babelhyphen{nobreak}tone suffixed forms correspond exclusively to LM\babelhyphen{nobreak}tone monosyllables. In other words,
	LM\babelhyphen{nobreak}tone monosyllables are the sole source of LM\babelhyphen{nobreak}tone disyllables. The words at issue, namely ‘sow’, ‘hen’, ‘bee’, and ‘thumb’, belong to basic vocabulary, suggesting that the LM::LM {correspondence} between \is{monosyllables}monosyllable and disyllable reflects an older
	pattern.
	
	The tone patterns for the //\ipa{-zo˥}// and //\ipa{-pʰv̩˥}// suffixes stand in regular tonal correspondence to those of
	the \is{suffixes}suffix //\ipa{-mi˩}//. In the second type of correspondences in Tables~\ref{tab:gendertwosyllableslm} to \ref{tab:gendertwosyllablesrest}, forms suffixed with //\ipa{-zo˥}// or //\ipa{-pʰv̩˥}// always have \#H (with
	L as a~\is{variants}variant) when the form suffixed with //\ipa{-mi˩}// has M. In the third type, words with any of these three suffixes have LM+\#H, with LM as
	a~\is{variants}variant. 
	
	On the other hand, these patterns differ widely from those observed in other syntactic
	structures, such as \is{compounds}compound nouns and noun-verb or noun-adjective combinations. This confirms that different syntactic structures are associated with distinct tone rules~-- further complicated by numerous lexicalized \is{irregularities}oddities.
	
	\subsubsection{LH-tone roots}
	\label{sec:lhtoneroots}
	
	Monosyllables with LH tone correspond to disyllables carrying LM+\#H. However, two items also
	have an L-tone \is{variants}variant, namely /\ipa{ʑi˩mi\#˥}/{\kern2pt}\ipa{≈}{\kern2pt}/\ipa{ʑi˩mi˩}/ ‘female monkey’ and
	/\ipa{tɕʰɯ˩mi\#˥}/{\kern2pt}\ipa{≈}{\kern2pt}/\ipa{tɕʰɯ˩mi˩}/ ‘female muntjac’. In contrast, /\ipa{ʐæ˩mi\#˥}/ ‘female
	leopard’ does not allow this \is{variants}variant: the form \ipa{$\ddagger${\kern2pt}ʐæ˩mi˩} is not acceptable. 
    
    Outside the semantic field
	of animal names, /\ipa{tʰi˩mi\#˥}/ ‘large plane’ (from /\ipa{tʰi˩˥}/ ‘plane [carpentry tool]’)
	cannot be realized as \ipa{$\ddagger${\kern2pt}tʰi˩mi˩}. Conversely, the word for ‘large slope’ (from /\ipa{to˩˥}/
	‘slope’) is /\ipa{to˩mi˩}/, and \ipa{$\ddagger${\kern2pt}to˩mi\#˥} is not acceptable. 
    
    At present, it remains uncertain whether two different tone rules applied at different times~-- in which case the
	existence of two variants would be a~development due to \isi{analogy} or dialect \is{language contact}contact~-- or whether both variants used to coexist for all items, with some later losing one of the two. %The fact that
%	variants are present for the majority of items within the same {correspondence} class is consistent with the
%	latter hypothesis.
	
	Two additional patterns are attested, each represented by a single example. These words may be relatively old: M tone
	in /\ipa{di˧mi˧}/ ‘plain’ (compare /\ipa{di˩˥}/ ‘earth, land’), and LH tone in /\ipa{ljɤ˩mi˥}/ ‘major beam’.
	
	\subsubsection{M-tone roots}
	\label{sec:mtoneroots}
	
	Monosyllables with M tone yield disyllables with \#H tone: /\ipa{lɑ˧mi\#˥}/ ‘female tiger’ and
	/\ipa{se˧mi\#˥}/ ‘female goral’. This pattern is also found outside the semantic domain of animal
	names, as in /\ipa{qʰwæ˧mi\#˥}/ ‘message; letter’. That this rule is currently productive was confirmed
	by adding the augmentative \is{suffixes}suffix to the M-tone noun /\ipa{qwæ˧}/ ‘bed mat’, yielding /\ipa{qwæ˧mi\#˥}/ ‘large bed mat’.
	
	The situation is more complex for trisyllables formed with the \is{suffixes}suffix /\ipa{-mi˩}/. Two patterns are attested: --L, as in /\ipa{si˧gɯ˧-mi˩}/
	‘lionness’, /\ipa{ʐv̩˧bæ˧-mi˩}/ ‘female snake’, and /\ipa{tʰo˧li˧-mi˩}/ ‘hare’; and \#H, as in
	/\ipa{ɬi˧bv̩˧-mi\#˥}/ ‘woman of the Bai ethnic group’. The latter pattern matches the correspondence found for monosyllables. 
    
    To determine which of the two is currently productive, the \is{suffixes}suffix was added
	to a~word that does not normally take it: ‘earthworm’ /\ipa{dʑɯ˧dv̩˧}/, as earthworms are hermaphroditic. The elicitation scenario was as follows: a~child wonders whether there are such things as male and female earthworms and asks, ‘Are
	there such things as female earthworms?~/ Do female earthworms exist?’ The speaker formulated
	the {question} without hesitation, using the suffixed form
	/\ipa{dʑɯ˧dv̩˧-mi˩}/ ‘female earthworm’ with --L tone  (\ref{ex:femaleearthworm}).
	
%\Hack{\newpage}

	\begin{exe}
		\ex
		\label{ex:femaleearthworm}
		\ipaex{dʑɯ˧dv̩˧-mi˩ {\kern2pt}|{\kern2pt} ə˩-dʑo˩˥?}\\
		\gll dʑɯ˧dv̩˧-mi˩		ə˩-				dʑo˩\textsubscript{b}\\
		female\_earthworm		\textsc{interrog}	\textsc{exist}\\
		\glt ‘Are there such things as female earthworms?/Do female earthworms exist?’ (Field notes.)
	\end{exe}
	
	
%	In this second type, the tone patterns for the /\ipa{-zo˥}/ and /\ipa{-pʰv̩˥}/ suffixes are identical with those for /\ipa{-mi˩}/.
	
	
	\subsubsection{L-tone roots}
	\label{sec:ltoneroots}
	
	Monosyllabic words with L tone yield disyllables with M tone: /\ipa{jo˧mi˧}/ ‘ewe’,
	/\ipa{ɬi˧mi˧}/ ‘female roebuck’, and /\ipa{mv̩˧mi˧}/ ‘woman’. This last example is
	attested in a~proverb shown in (\ref{ex:womanARCH}).
	
	\begin{exe}
		\ex
		\label{ex:womanARCH}
		\ipaex{mv̩˧mi˧ ʈʂʰwɤ˩ mɤ˩-ɖɯ˩,  {\kern2pt}|{\kern2pt}  kʰv̩˧nɑ˥ ʐo˩ mɤ˩-ɖɯ˩.}\\
		\gll mv̩˧mi˧	ʈʂʰwɤ˥						mɤ˧-			ɖɯ˧\textsubscript{b}		 kʰv̩˧nɑ˥			ʐo˩							mɤ˧-	ɖɯ˧\textsubscript{b}\\
		woman	dinner~(evening~meal)		\textsc{neg}	to\_get							black\_dog		lunch		\textsc{neg}	to\_get\\
		\glt ‘No dinner for the [married] woman, no lunch for the black dog.’ \textit{Explanation:} ‘No dinner for the married woman’:  if a~married woman visits her natal home during the day, she cannot stay for the evening meal, as she has obligations in the household she has married into. ‘No lunch for the black dog’: here, ‘black dog’ refers to dogs in general. Dogs receive only two meals a~day, one in the morning and one in the evening. (\textit{Sister.130, 131, 139, 158,
		171} \pandoi{0004341\#S130} and \textit{Sister3.3, 113, 117} \pandoi{0004344\#S3})
	\end{exe}
	
	
	The word /\ipa{mv̩˧mi˧}/ ‘woman’ is no longer intelligible to younger speakers, such as M23. It is likely that the {correspondence} between L-tone roots and M-tone disyllabic forms reflects
	an older tone rule. The same pattern (M tone) is found in
	/\ipa{kɤ˧mi˧}/ ‘large bottle’ and /\ipa{dʑɯ˧mi˧}/ ‘large river’.
	
	The tone patterns of the suffixes /\ipa{-zo˥}/ and /\ipa{-pʰv̩˥}/ are identical for roots with L and LM tones.
	
	
	\subsubsection{H-tone roots}
	\label{sec:htoneroots}
	
	
	For words \is{derivation!morphological}derived from H-tone monosyllables, 
    %Suggestion by S. Nordhoff (accepted): \rephrase{there are}{we find } 
    three tonal patterns are observed: \#H tone, as in
	/\ipa{ho˧mi\#˥}/ ‘female pheasant’ and /\ipa{bɤ˧mi\#˥}/ ‘\ili{Pumi} woman’; L tone, as in /\ipa{ʐwæ˩mi˩}/ ‘mare’, /\ipa{ʝi˩mi˩}/
	‘cow’, and /\ipa{kʰv̩˩mi˩}/ ‘dog’ (discussed further below); and M tone, as in /\ipa{kʰi˧mi˧}/ ‘main door’, /\ipa{qʰæ˧mi˧}/ ‘canal; large ditch’,
	and /\ipa{ɻ̩̃˧mi˧}/ ‘tree trunk’ (etymologically ‘large bone’).
	
	The first pattern makes good synchronic sense: the tones of the \is{monosyllables}monosyllable and disyllable
	correspond neatly with each other, and the semantic relationship is also clear~-- the base noun (‘pheasant’, ‘\ili{Pumi}’) is gender-neutral, while the suffixed form specifies gender (‘female pheasant’, ‘\ili{Pumi} woman’). The second and third patterns are less transparent phonologically. Concordantly, the semantic relationship is less clear in some cases. 
    
    The words for
	‘mare’, ‘cow’ and ‘dog’ illustrate three stages in the gradual evolution of the \is{suffixes}suffix’s
	meaning. The \is{monosyllables}monosyllable for ‘horse’, /\ipa{ʐwæ˥}/, is in common use, and the \is{derivation!morphological}derived word for ‘mare’,
	/\ipa{ʐwæ˩mi˩}/, simply specifies gender. By contrast, the \is{monosyllables}monosyllable /\ipa{ʝi˥}/ for ‘cow’
	is not in frequent use; there are over ten different disyllables pronounced /\ipa{ʝi}/, six of
	them with H tone. Here, the \is{suffixes}suffix /\ipa{-mi˩}/ serves a disambiguating function. While it retains its meaning of `female', it can be said to function as a general animal \is{suffixes}suffix just as much
	as a~female \is{suffixes}suffix. The third example, /\ipa{kʰv̩˩mi˩}/ ‘dog’, represents a further stage in this evolution: it applies to dogs regardless of sex, and the \is{monosyllables}monosyllable /\ipa{kʰv̩˥}/ for
	‘dog’ is seldom used (but attested, witness \textit{Dog.1, 3, 45} \pandoi{0004442} and \textit{Dog2.68, 74-77, 79} \pandoi{0004660\#S68}). These observations suggest that \#H is the tone of more recently \is{derivation!morphological}derived words, while L tone reflects an earlier stage and remains lexically preserved in some older words.
	
	This conjecture is supported by examples from outside the semantic field of animal names. The disyllable /\ipa{sɑ˩mi˩}/ \textit{‘Cannabis indica’} (the psychotropic plant) corresponds to the \is{monosyllables}monosyllabic /\ipa{sɑ˥}/
	\textit{‘Cannabis sativa’} (used for fibre production) and has L tone. Here, the suffix conveys an augmentative meaning, referring to the larger size of the leaves. (As an aside: a comparable \is{derivation!morphological}derivation exists in \ili{Mandarin}, where ‘cannabis’ \textit{dàmá} \zh{大麻} is formed from ‘hemp’ \textit{má} \zh{麻} by addition of the augmentative \textit{dà} \zh{大} ‘large’.) Again, this suggests that L was the earlier tone. By
	contrast, more recent and less clearly lexicalized disyllables, such as /\ipa{v̩˧mi\#˥}/ ‘large pot’ (cf.\ /\ipa{v̩˥}/ ‘pot’), carry \#H tone, as does /\ipa{sɯ˧ɻ̩̃˧mi\#˥}/ ‘backbone, spine’ (compare /\ipa{sɯ˧ɻ̩̃\#˥}/ ‘tree trunk’).
	
	The tone patterns of the suffixes /\ipa{-zo˥}/ and /\ipa{-pʰv̩˥}/ are identical for roots with H and M tones.
	
	
	\subsubsection{MH-tone roots}
	\label{sec:mhtoneroots}
	
	MH-tone monosyllables all correspond to disyllables with H\$ tone. This pattern is attested in the following animal names: /\ipa{hwɤ˧mi˥\$}/ ‘she\babelhyphen{nobreak}cat’, /\ipa{ʈʂʰæ˧mi˥\$}/ ‘hind’, /\ipa{tsʰɯ˧mi˥\$}/ ‘nanny goat’,
	/\ipa{ʁv̩˧mi˥\$}/ ‘female crane’, and /\ipa{tɕɯ˧mi˥\$}/ ‘wasp’. Beyond the semantic domain of
	animal names, the same correspondence is found for /\ipa{kʰɤ˧mi˥\$}/ ‘large basket’, /\ipa{ʁo˧mi˥\$}/ ‘big needle’, /\ipa{tɕɯ˧mi˥\$}/
	‘large scales (for weighing things)’, and /\ipa{hu˧mi˥\$}/ ‘stomach, bowels’. %The latter noun, now more commonly used than \is{monosyllables}monosyllabic /\ipa{hu˧˥}/, has no connotation of ‘big, large’.
	
	Words suffixed with /\ipa{-zo˥}/ or /\ipa{-pʰv̩˥}/ can carry either \#H, a~pattern which is widely
	attested with these suffixes, or H\$, the tone found in items suffixed with /\ipa{-mi˩}/.
	
	\subsubsection{Some observations on other lexical tones}
	\label{sec:someobservationsaboutotherlexicaltones}
	
	As predicted by Rule~5 (“All syllables following an H.L or
	M.L sequence receive L tone”; see \sectref{sec:alistoftonerules}), L\# tone spreads onto the {suffix}, yielding M.L.L:
	/\ipa{dze˧bɤ˩-mi˩}/ ‘female bat’, /\ipa{dze˧bɤ˩-zo˩}/ ‘little bat, pup’, and /\ipa{dze˧bɤ˩-pʰv̩˩}/
	‘male bat’. On this basis, a~disyllabic form \ipa{*mo˧jo˩} can be confidently extracted from
	/\ipa{mo˧jo˩-mi˩}/ ‘owl’. (Recall that the asterisk indicates a~\is{comparative method (historical linguistics)}reconstructed form, not an ungrammatical one.)
	
	The tone patterns of ‘cicada’, ‘vulture’, ‘hwamei (bird)’, and ‘(cigarette) lighter’ have no attested parallel elsewhere, making it impossible, in the present stage of our knowledge, to infer the roots' tones.
	
	
	\subsubsection{Concluding observations}
	\label{sec:concludinggeneralobservations}

	
	The patterns documented in Tables \ref{tab:gendertwosyllableslm} to \ref{tab:genderthreesyllables} are summarized in Tables~\ref{tab:tcorrmono} and \ref{tab:tcorrdi}. It must be emphasized that the distinction between currently productive patterns and older patterns for the /\ipa{-mi˩}/ {suffix} remains speculative. As elsewhere, a~slash (/) separates variants.
	

		% \label{tab:tcorrd}  %% Commented out on April 30th, 2025: no subtables, only sequentially numbered tables in the entire volume.
        
		%Early numbering: Table 5.
		\begin{table}%[t]
			\caption{\label{tab:tcorrmono}Tonal correspondences between {monosyllabic} base forms and disyllables containing the suffixes /\ipa{-mi˩}/,
				/\ipa{-zo˥}/ and /\ipa{-pʰv̩˥}/, with tentative indications on whether the tone pattern is currently
				productive.}
						\begin{tabularx}{\textwidth}{ l l Q l }
				\lsptoprule
				& \multicolumn{2}{l}{/\ipa{-mi˩}/} & /\ipa{-zo˥}/, /\ipa{-pʰv̩˥}/\\\cmidrule(lr){2-3}\cmidrule(lr){4-4}
				& older? & productive? & \textit{no distinctions in productiveness}\\ \midrule
				LM & LM; L & M; LM+\#H / LM & LM+\#H / LM; \#H / L\\
				LH & M; LH; L & LM+\#H / L & LM+\#H / L\\
				M &  & \#H & \#H\\
				L &  & M & \#H / L\\
				H & L; M & \#H & \#H\\
				MH &  & H\$ & \#H / H\$\\
				\lspbottomrule
			\end{tabularx}
		\end{table}
		
		\begin{table}%[t]
			\caption{\label{tab:tcorrdi}Tonal correspondences between disyllabic base forms and {trisyllabic} nouns containing the suffixes /\ipa{-mi˩}/, /\ipa{-zo˥}/ and /\ipa{-pʰv̩˥}/, with tentative indications on whether the tone pattern is currently productive.}
			\begin{tabularx}{\textwidth}{ l l Q l }
				\lsptoprule
				& \multicolumn{2}{l}{/\ipa{-mi˩}/} & /\ipa{-zo˥}/, /\ipa{-pʰv̩˥}/\\\cmidrule(lr){2-3}\cmidrule(lr){4-4}
				& older? & productive? & \textit{no distinctions in productiveness}\\ \midrule
				M & \#H & --L (L\#) & \#H\\
				H &  & \#H &\\
				L &  & L & L\\
				L\# &  & L\#-- & L\#--\\
				LM+MH\# &  & LM+H\# & LM+\#H\\
				H\# &  & H\#-- & --\\
				\lspbottomrule
			\end{tabularx}
		\end{table}


	Roots with the same lexical tones correspond to suffixed forms with diverse tones, with as many as
	four types of correspondences for LM tone. The total number of noun subsets in Tables \ref{tab:gendertwosyllableslm} to \ref{tab:genderthreesyllables}, excluding
	disyllabic roots, is 14. Given that /\ipa{-mi˩}/ and /\ipa{-zo˥}/, /\ipa{-pʰv̩˥}/ fall into different tonal categories, this results in 2×14=28 potentially distinct tonal types of
	suffixed nouns. Considering that many types have variants, the actual number of tonal patterns attested on suffixed nouns
	could be considerable.
	%; one could expect it to cover the entire range of eleven existing tone patterns for disyllables. 

    Yet the observed set of tones remains relatively constrained, limited to six primary patterns: \{M, \#H, H\$, L,
	LM, LM+\#H\}, apart from two outliers~-- tones attested in only one example each. This suggests that some tone categories encompass a large number of \is{suffixes}suffixed or compound nouns, whereas others are not fed by any currently productive morphological processes. 


    From a \is{phonostylistics}phonostylistic perspective, such asymmetries contribute to giving different lexical tone categories a specific feel of their own. Depending on their frequency among recent coinages (as well as among loanwords), certain tonal patterns might carry more or less salient \textit{overtones} (if the pun may be allowed) of novelty vs.\ classicality. A prevalent pattern might be perceived as straightforward, mainstream, and workaday, and conversely, a rarely-occurring one might feel quaint and archaic-sounding. The \is{phonostylistics}phonostylistic associations of tonal categories, along with those of the various \is{morphotonology}morphotonological rules, constitute an interesting avenue for future research, as one of the various \is{stylistics}stylistic topics that could be investigated through the study of the corpus of Yongning Na narratives.
	
	For forms suffixed with /\ipa{-zo˥}/ and /\ipa{-pʰv̩˥}/, only five patterns are attested: \{\#H, H\$, L, LM, LM+\#H\}. One additional pattern, M tone, is
	attested for /\ipa{-mi˩}/. The relatively lower complexity of tone patterns for
	/\ipa{-zo˥}/ and /\ipa{-pʰv̩˥}/ may have something to do with their more restricted lexical distribution:
	since nouns suffixed with ‘male’ and ‘child’ markers are fewer in number, they may have undergone greater tonal \isi{simplification} through the analogical extension of productive patterns.	
	

	\subsection{Other suffixes for ‘male’}
		\label{sec:othermale}

	
	In addition to the currently productive \is{suffixes}suffix /\ipa{-pʰv̩˥}/ for ‘male’, there exist other, non-productive suffixes. These are mentioned here for completeness, though the small number of examples greatly limits possibilities for analysis of their tone patterns.
		
		\subsubsection{The suffix /\ipa{-ʂwæ˧}/}
	\label{sec:thesuffixformale}


    The free form /\ipa{ʂwæ˧}/ currently means ‘castrated/neutered male’.
	%: see (\ref{ex:ismale2}). %% A proofreader noted that "the example immediately follows the colon, so there is no need to cross-reference the example here".
	
	\begin{exe}
		\ex
		\label{ex:ismale2}
		\ipaex{ʈʂʰɯ˧ {\kern2pt}|{\kern2pt} ʂwæ˧ ɲi˩.}\\
		\gll ʈʂʰɯ˥	ʂwæ˧	ɲi˩\\
		\textsc{dem.prox}		castrated\_male	\textsc{cop}\\
		\glt ‘This is a~castrated male.’
	\end{exe}
	
	
	The morpheme /\ipa{ʂwæ˧}/ may, however, have previously meant ‘male’, as suggested by the noun /\ipa{æ̃˧ʂwæ˥}/ meaning ‘rooster, cock’ (compare /\ipa{æ̃˩˧}/ ‘chicken’). Notably, this noun has no counterpart with the \is{suffixes}suffix /\ipa{-pʰv̩˥}/; a form such as \ipa{$\ddagger${\kern2pt}æ̃˩pʰv̩\#˥} is not available. Roselle Dobbs (p.c.\ 2016) reports that in Lataddi \zh{喇塔地}, ‘grandfather’ and ‘rooster’ sound comically similar. It is conceivable that an earlier form $\dagger${\kern2pt}\ipa{æ̃˩pʰv̩\#˥} ‘rooster’ fell into disuse in the Alawua dialect (studied here) due to phonetic similarity with ‘grandfather’ (even in the absence of full homophony). 
	
	The \is{suffixes}suffix /\ipa{-ʂwæ}/ also appears in three other words, in which it carries the meaning ‘castrated male’ (see \tabref{tab:namesofanimalswiththesuffix}). Interestingly, its
	tone pattern differs between ‘cock’ and ‘castrated yak’, despite both having roots with the same lexical tone
	(LM). This tonal divergence may be related to the \is{suffixes}suffix's different meanings in these words: ‘male’ in one case and ‘castrated male’ in
	the other. It is reasonable to assume that /\ipa{æ̃˧ʂwæ˥}/ ‘cock’ has greater time depth.
	
	Concerning the tone of the \is{suffixes}suffix, two of the words in which it appears (‘castrated yak’ and ‘castrated male sheep’) display tone patterns that are among possible variants for the suffixes \mbox{//\ipa{-zo˥}//} ‘baby, male’ and \mbox{//\ipa{-pʰv̩˥}//} ‘male’, which are tentatively analyzed as bearing H tone. However, the third word, /\ipa{tsʰɯ˧ʂwæ˥}/ ‘wether, castrated male goat’, follows a different tonal pattern from words suffixed in \mbox{//\ipa{-zo˥}//} ‘baby, male’ or \mbox{//\ipa{-pʰv̩˥}//} ‘male’: /\ipa{tsʰɯ˧zo\#˥}/{\kern2pt}\ipa{≈}{\kern2pt}/\ipa{tsʰɯ˧zo˥\$}/ and /\ipa{tsʰɯ˧pʰv̩\#˥}/{\kern2pt}\ipa{≈}{\kern2pt}/\ipa{tsʰɯ˧{\allowbreak}pʰv̩˥\$}/ (see \tabref{tab:gendertwosyllablesmh}). This constitutes evidence that the \is{suffixes}suffix /\ipa{-ʂwæ}/ does not share the same tonal behaviour as the \smash{\mbox{//\ipa{-zo˥}//}} and \smash{\mbox{//\ipa{-pʰv̩˥}//}} suffixes. Here, the \is{suffixes}suffix /\ipa{-ʂwæ}/ is provisionally labelled as carrying M tone and transcribed as \mbox{//\ipa{-ʂwæ˧}//}. However, it should be emphasized that this label primarily serves to distinguish its tone from that of the ‘baby, male’ and ‘male’ suffixes, which are provisionally transcribed as //\ipa{-zo˥}// and //\ipa{-pʰv̩˥}//, respectively.
	
	\begin{table}%[t]
		\caption{\label{tab:namesofanimalswiththesuffix}Names of animals with the suffix /\ipa{ʂwæ˧}/.}
		\begin{tabularx}{\textwidth}{ Q l l l }
			\lsptoprule
			tone of root & meaning of root & suffixed form & meaning\\ \midrule
			LM & chicken & \ipa{æ̃˧ʂwæ˥} & rooster (not castrated)\\
			LM & yak & \ipa{bv̩˩ʂwæ˩} & castrated yak\\
			L & sheep & \ipa{jo˩ʂwæ˩} & wether, castrated male sheep\\
			MH & goat & \ipa{tsʰɯ˧ʂwæ˥} & wether, castrated male goat\\
			\lspbottomrule
		\end{tabularx}
	\end{table}

	\subsubsection{The suffix /\ipa{-v̩}/}
	\label{sec:thesuffixv}

    
	\citet[179]{lidz2010} proposes that the \is{suffixes}suffix in /\ipa{zɛ³¹-wu³³}/ ‘nephew' (in the speech of Mrs.\ Latami, consultant F4: /\ipa{ze˩v̩˩}/) and /\ipa{ʐu³¹-wu³³}/ ‘grandson' (F4: /\ipa{ʐv̩˧v̩\#˥}/) \is{derivation!morphological}derives from the root for ‘uncle/senior male
	relative’, which appears in /\ipa{ə˧v̩˧˥}/ ‘maternal uncle'. She further hypothesizes that this root constitutes the origin of the classifier for individuals (F4: /\ipa{v̩˧}/). Progress in comparative studies of Naish languages will hopefully allow for testing these hypotheses. 
	
		\subsubsection{The suffix /\ipa{-ʁo}/}
		\label{sec:thesuffixro}

	The word for ‘castrated horse’ is /\ipa{ʐwæ˧ʁo˩}/. Horses have been the object of great care and interest in this part of the Himalayas for at least two millenia (\citealt{wang1980}; \citealt[10320]{sagart_dated_2019}; \citeauthor{jacques_OriginsST} to appear), so it is unsurprising that words belonging to this semantic field are numerous, some of them probably very old. This isolated example, however, is clearly insufficient for linguistic analysis.
	

	\subsection{The kinship prefix /\ipa{ə˧-}/}
	\label{sec:thekinshipprefix}
	
	Another non-productive but readily identifiable {affix} is the kinship \is{prefixes}prefix \mbox{/\ipa{ə˧-}/}, found in all kinship terms referring to one’s elders. It is common to various languages of the area, such as \ili{Rma} (Qiang) 
	\citep[158–159]{evansetal2007}, \ili{Yi}, and \ili{Mandarin}. \tabref{tab:kinshiptermswiththeprefix} presents the examples that were observed in
	Yongning Na, where this \is{prefixes}prefix is “the only common noun \is{prefixes}prefix” \citep[167]{lidz2010}.
	

	\begin{table}%[t]
		\caption{\label{tab:kinshiptermswiththeprefix}Kinship terms with the prefix /\ipa{ə˧-}/.}
		\begin{tabularx}{\textwidth}{ Q l l }
			\lsptoprule
			kinship term & tone & meaning\\ \midrule
			\ipa{ə˧mɑ˧} & M & mother ({term of address})\\
			\ipa{ə˧mi˧} & M & mother; aunt\\
			\ipa{ə˧pʰv̩˧} & M & brother of the mother’s mother\\
			\ipa{ə˧si˧} & M & mother’s mother's %\babelhyphen{nobreak}
            mother; ancestor\\
			\ipa{ə˧ɖo˧} & M & boyfriend/girlfriend, lover\\
			\ipa{ə˧ʑi˧˥} & MH\# & mother’s mother\\
			\ipa{ə˧v̩˧˥} & MH\# & mother's brother\\
			\ipa{ə˧dɑ˥\$} & H\$ & father\\
			\ipa{ə˧bo˥\$} & H\$ & father's brother\\
			\ipa{ə˧ɕjɤ˩} & L\# & boyfriend/girlfriend, lover\\
			\ipa{ə˧jɤ˩} & L\# & mother’s elder sister\\
			\ipa{ə˧mv̩˩} & L\# & elder sibling (brother or sister)\\
			\ipa{ə˧tɕi˩} & L\# & mother’s younger sister\\
			\ipa{ə˧zɯ˩ ≈ ə˩zɯ˩} & L\# ≈ L & dual: us two\\
			\ipa{ə˧-sɯ˩kv̩˩ ≈ ə˩-sɯ˧kv̩˥} & --L ≈ LMH & 1\textsuperscript{st} person plural, inclusive\\
			\lspbottomrule
		\end{tabularx}
	\end{table}
	
	Monosyllabic forms do not exist, and no convincing method to extract the tone of the root has been
	found. It is tempting to hypothesize that the \is{prefixes}prefix does not contribute a~tonal specification, and that
	the tone of the disyllable reflects that of the root: disyllables with M, MH\#, and H\$ tone would
	originate in roots with M, MH, and H tone, respectively, and disyllables with L\# tone would originate
	in roots with L, LM or LH tone. However, this reasoning remains highly speculative, and no evidence has been found
	to explore this issue further. The root /\ipa{mi}/ in /\ipa{ə˧mi˧}/ ‘mother’ is presumably related to
	the free form /\ipa{mi˩˧}/ ‘female’, and the root /\ipa{pʰv}/ in /\ipa{ə˧pʰv̩˧}/ ‘grandmother's elder brother’ to the free form /\ipa{pʰv̩˧}/ ‘male’. However, unlike these free forms, the kinship terms /\ipa{ə˧mi˧}/ ‘mother’ and /\ipa{ə˧pʰv̩˧}/ ‘brother of the mother’s mother’ have the same tone, making it problematic to infer the tones of the roots from those of the prefixed kinship terms.
	
	From a~static\babelhyphen{nobreak}synchronic point of view, it is also difficult to reach firm conclusions due
	to the limited amount of data: one \is{prefixes}prefix and four suffixes. With this qualification, one may
	observe that disyllables with M, H\$ or L tone can result from either \is{suffixes}suffixation or
	\is{prefixes}prefixation; that disyllables with \#H, H\#, LM, LH or LM+\#H tone can result from \is{suffixes}suffixation but not from
	\is{prefixes}prefixation; and that disyllables with MH\# or L\# tone can result from \is{prefixes}prefixation but not from
	\is{suffixes}suffixation. The possible origins of the various tonal categories of disyllables (by \is{suffixes}suffixation, \is{prefixes}prefixation, and compounding) are examined in \sectref{sec:possibleoriginsfordisyllablesonthebasisoftheirtoneabirdseyeview}.
	
	With regard to the tone of the kinship \is{prefixes}prefix, it seems reasonable to analyze it as M, since it always appears with M tone except for two variants shown in the last two lines of \tabref{tab:kinshiptermswiththeprefix}. This is not substantially different from an analysis in which the \is{prefixes}prefix is underlyingly toneless, given that M behaves in some respects as a~default tone, as discussed in \sectref{sec:analysisofmasadefaulttone}.
	
	Kinship terms in the Luoshui dialect \citep[167]{lidz2010} are similar to those in Alawua (the dialect studied here). For instance, Luoshui /\ipa{ɑ³³ʐɯ³³}/
	‘grandmother’ is cognate with Alawua /\ipa{ə˧ʑi˧˥}/. Only three terms from
	L.~Lidz’s list are not attested in Alawua. One of these is /\ipa{ɑ³³pɔ³¹}/, for ‘uncle: father’s
	elder or younger brother’, which could be a~\is{loanwords}borrowing from \ili{Mandarin} \textit{ābó} \zh{阿伯}
	‘brother\babelhyphen{nobreak}in\babelhyphen{nobreak}law; father’s older brother’. Borrowing is facilitated by the similar structure of kinship terms for one's elders in both
	languages, with a~similar \is{prefixes}prefix (in \ili{Mandarin}: \textit{ā} \zh{阿}). A~different term is in use in
	Alawua: /\ipa{ə˧bo˥\$}/, whose voiced initial suggests that it is not a~recent \is{loanwords}borrowing from
	{Mandarin}, since Mandarin does not retain voiced stops.\footnote{In \textit{Pinyin} romanization, \textit{b} stands for a~voiceless bilabial stop: /\ipa{p}/, not /\ipa{b}/.} Ethnological data sheds light on the fact that the terms
	for uncles on the father’s side do not correspond neatly across dialects: the social relationship
	with one’s father (and his household) was traditionally loose (see Appendix~\ref{chap:historyanthropologysociology}, \sectref{sec:anthropologicalresearchthefascinationofnafamilystructure}); accordingly, kinship terms for paternal relatives were not as specific as those for maternal relatives. 
	
	The peculiar structure of Na families invites linguistic speculation concerning the origin and
	evolution of the terms currently used for relatives on the father’s
	side. Fu Maoji (\citeyear[23]{fu1980}; \citeyear[38–39]{fu1983}) hypothesizes that /\ipa{ə˧bo˥\$}/ ‘uncle on the father’s side’ formerly referred to male
	relatives of the father’s generation on the father’s side, i.e.\ the father and his brothers, and
	that the introduction of the term /\ipa{ə˧dɑ˥\$}/ ‘father’ led to the specialization of
	/\ipa{ə˧bo˥\$}/ to refer specifically to uncles on the father’s side. This would imply that people had a~term for their paternal uncles, pooled together with their father under the term /\ipa{ə˧bo˥\$}/,
	before they had a~specific term for ‘father’. This seems paradoxical: since children did not live in the same household as their
	paternal uncles, their link to these uncles was through the father, making the existence of a term for ‘father’ a~logical prerequisite for conceptualizing the broader category of \textit{male relatives on the father's side, of the father's generation}. 
    
    However, Christine Mathieu (p.c.\ 2016) quotes Lamu Gatusa \zh{拉木·嘎吐萨} as reporting a~word in the dialect of Labai \zh{拉柏} that is cognate with /\ipa{ə˧bo˥\$}/ and refers precisely to this concept: male relatives on the father's side, belonging to the father's generation~-- i.e.\ the father and his brothers. Distinctions can be made by adding the adjective ‘small’ to refer to the father's younger brothers and ‘big’ to refer to his elder brothers. Seen in this light, Fu Maoji's reasoning could be plausible in a~conceptual universe that is structured around extended families and clans rather than nuclear families. If an individual is primarily identified in terms of belonging to a~household, and to a~generation within the family, it is conceivable that fathers were not distinguished terminologically from their brothers. The hypothesis of an undifferentiated term for the father and his brothers must be examined in light of the fact that ties with the father's family were traditionally loose and distant, both economically and socially.
	
	Synchronically, the extension of kinship terms from the traditional household (i.e.\ the mother’s side) to the paternal family is occasionally observed. For instance, speakers may publicly address their father as /\ipa{ə˧v̩˧˥}/ ‘uncle on the mother’s side’ (field notes, consultant F4). The \is{stylistics}stylistic effect is to both convey closeness~-- through inclusion in the household~-- and honour, as maternal uncles are characters to whom the highest respect is due (as explained in Appendix~\ref{chap:historyanthropologysociology}, \sectref{sec:themainsourceofinformationonnafamilystructuresurveysconductedinthe1960s}). 
    
    In light of such flexibility in terms of address, one could venture
	an alternative hypothesis about the origin of the term for ‘uncle on the father’s side’: namely, that the words /\ipa{ə˧v̩˧˥}/ ‘uncle on the mother’s side’ and /\ipa{ə˧bo˥\$}/ ‘uncle on the father’s side’ in the Alawua dialect may have originally referred to the mother’s
	elder and younger brothers, respectively. Under this hypothesis, /\ipa{ə˧bo˥\$}/~-- corresponding to a~socially less important and prestigious role than /\ipa{ə˧v̩˧˥}/~-- would have been reassigned to uncles on the father’s side, while /\ipa{ə˧v̩˧˥}/ would have been generalized to all of the
	mother’s brothers irrespective of age. This process would have preserved the social hierarchy between /\ipa{ə˧v̩˧˥}/ as the more
	important social figure and /\ipa{ə˧bo˥\$}/ as the less important one, while shifting the distinction from one of age to one of lineage (maternal vs.\ paternal). 
    
    This hypothesis does not seem excessively far-fetched in view of the existence of distinct terms for aunts: ‘mother's elder sister’, /\ipa{ə˧jɤ˩}/, and ‘mother's younger sister’, /\ipa{ə˧tɕi˩}/. This lends some plausibility to the idea that an earlier stage of the language may also have distinguished older and younger uncles. However, this hypothesis remains highly speculative. 
		
	The second term reported by Liberty Lidz that is not found in the present research data (Alawua dialect) is /\ipa{ɑ³³mɔ¹³}/ as another term for
	‘grandmother’. The third is /\ipa{ɑ³³lɑ³¹}/, referring to great\babelhyphen{nobreak}great\babelhyphen{nobreak}grandparents: in Alawua, the
	term /\ipa{ə˧si˧}/ is used for all ancestors of the great\babelhyphen{nobreak}grandmother’s generation and above.
	
\is{derivation!morphological|)}


\section{Reduplication}
\label{sec:reduplicationofnounphraseswithadiscussionofsomemarginalcasesofreduplication}

In languages that do not have morphophonological templates specific to reduplication, it is not always easy to distinguish between reduplication and other types of repetition or copying. For instance, \citet[301]{moravcsik1978} considers \textit{very very} in the English example \textit{He is very very bright} as a~case of reduplication. This is debatable, as the intensifier can be repeated more than once (either an even number of times, 2×n, or an odd number of times: \textit{He is very, very, very bright}), with gradual rather than categorical semantic\babelhyphen{nobreak}stylistic effects. 

In Yongning Na, reduplication is not too difficult to delimit on a~phonological basis, as it follows specific tonal templates and carries clearly identifiable syntactic and semantic values. \is{reduplication}Reduplication is most commonly observed with verbs: see \sectref{sec:reduplication}. Reduplication involving nouns is nowhere as frequent: the only well\babelhyphen{nobreak}attested case is the \isi{reduplication} of \is{numerals}numeral\babelhyphen{nobreak}plus\babelhyphen{nobreak}classifier phrases. 


\subsection{Reduplication of numeral\babelhyphen{nobreak}plus\babelhyphen{nobreak}classifier phrases}
\label{sec:numclred}

Reduplication of a~phrase consisting of the \is{numerals}numeral ‘one’ (analyzed as carrying LH tone: //\ipa{ɖɯ˩˥}//) plus a~\is{classifiers}classifier indicates iteration. The entire reduplicated
phrase is integrated into a~single \isi{tone group}~-- a crucial unit in Na morphotonology, discussed in Chapter \ref{chap:toneassignmentrulesandthedivisionoftheutteranceintotonegroups}. The tonal pattern of the first half of the phrase
determines that of the second, by application of the phonological rules set out in \sectref{sec:alistoftonerules}:

\begin{itemize}
	\item{After an H-tone classifier, the second half of the phrase receives L
		tone by application of Rules 4 and 5, e.g.~//\ipa{ɖɯ˧-ɲi˥\$}// → /\ipa{ɖɯ˧-ɲi˥{$\sim$}ɖɯ˩-ɲi˩}/ ‘day after day’
		(\textit{Reward.155} \pandoi{0004446\#S155}, \textit{BuriedAlive2.85} \pandoi{0004536\#S85}, \textit{Caravans.259} \pandoi{0004530\#S259}).}
	\item{After an M-tone classifier, the second part remains
		unaffected, as in //\ipa{ɖɯ˧-ʁwɤ˧}// → /\ipa{ɖɯ˧-ʁwɤ˧{$\sim$}ɖɯ˧-ʁwɤ˧}/ ‘one heap after another’
		(\textit{Housebuilding.51} \pandoi{0004448\#S51}).}
	\item{After an L-tone classifier, the second part is lowered to L by application of Rule~5,
		e.g.~//\ipa{ɖɯ˧-ʑi˩}// → /\ipa{ɖɯ˧-ʑi˩{$\sim$}ɖɯ˩-ʑi˩}/ ‘one family after another’ (\textit{Healing.94} \pandoi{0004540\#S94},
		\textit{Caravans.237} \pandoi{0004530\#S237}).}
	\item{After an MH-tone classifier, the H part of the \is{tonal contour}contour lands on the first syllable of the second half: //\ipa{ɖɯ˧-kɤ˧˥}// →
		/\ipa{ɖɯ˧-kɤ˧{$\sim$}ɖɯ˥-kɤ˩}/ ‘one tree after another’ (\textit{Housebuilding.28} \pandoi{0004448\#S28}). The final L tone in this expression arises through the application of Rule~4.}
\end{itemize}


% \subsection[A reduplicated nominal suffix]{Addition of the reduplicated suffix /\ipa{-ʂo˧{$\sim$}ʂo˩}/ to nouns, conveying abundance}
\subsection[A reduplicated nominal suffix]{A reduplicated nominal suffix, conveying abundance}
\label{sec:additionofreduplicatedsuffix}

Addition of the \is{suffixes}suffix /\ipa{-ʂo˧{$\sim$}ʂo˩}/ to nouns conveys abundance. Examples include
/\ipa{mɤ˩-ʂo˩{$\sim$}ʂo˥}/ ‘smeared with grease, covered with grease’ (\ref{ex:Lake315}), /\ipa{ʂe˧-ʂo˧{$\sim$}ʂo˥}/ ‘rich in meat, with lots of meat’ (\ref{ex:Dog35}), and /\ipa{si˧-ʂo˧{$\sim$}ʂo˥}/ ‘packed with wood’ (\ref{ex:Housebuilding281}). 

\begin{exe}
	\ex
	\label{ex:Lake315}
	\ipaex{mv̩˩kʰv̩˧˥ {\kern2pt}|{\kern2pt} le˧-tsʰɯ˩-dʑo˩, {\kern2pt}|{\kern2pt} ɲi˧to˧ {\kern2pt}|{\kern2pt} ʈʂʰɯ˧-qo˧ {\kern2pt}|{\kern2pt} le˧-tɑ˧˥, {\kern2pt}|{\kern2pt} ʈʂʰɯ˧-qo˧ {\kern2pt}|{\kern2pt} le˧-tɑ˧˥, {\kern2pt}|{\kern2pt} \textbf{mɤ˩-ʂo˩{\textasciitilde}ʂo˥} tsɤ˩ tsɯ˩ {\kern2pt}|{\kern2pt} mv̩˩!}\\
	\gll mv̩˩kʰv̩˧˥		le˧-					tsʰɯ˩\textsubscript{a}		-dʑo˥		ɲi˧to˧		ʈʂʰɯ˧-qo˧		le˧-tɑ˧˥	mɤ˩		\textbf{-ʂo˧{\textasciitilde}ʂo˩}		tsɤ˧	tsɯ˧˥	mv̩˧\\
	evening		\textsc{accomp}			to\_come.\textsc{pst}	\textsc{top}	mouth	here	up\_to		animal\_fat		\textbf{\textsc{abundance}}	to\_become	\textsc{rep}		\textsc{affirm}\\
	\glt ‘The story goes that in the evening, when [the dumb man] came back, [his] mouth was smeared with grease up to here, up to here!’ \textit{(Lake3.15)} \pandoi{0004348\#S15}
\end{exe}

\begin{exe}
	\ex
	\label{ex:Dog35}
	\ipaex{kʰv̩˩mi˩-ki˥, {\kern2pt}|{\kern2pt} ə{\dots} ʂe˧! {\kern2pt}|{\kern2pt} ɲi˧zo˧ {\kern2pt}|{\kern2pt} ɖɯ˧-kʰwɤ˥, {\kern2pt}|{\kern2pt} æ̃˩-ʂe˧ {\kern2pt}|{\kern2pt} ɖɯ˧-kʰwɤ˥, {\kern2pt}|{\kern2pt} bo˩-ʂe˧ {\kern2pt}|{\kern2pt} ɖɯ˧-kʰwɤ˥, {\kern2pt}|{\kern2pt} ʝi˧-ʂe˧ {\kern2pt}|{\kern2pt} ɖɯ˧-kʰwɤ˥, {\kern2pt}|{\kern2pt} kʰv̩˩mi˩-ki˥ {\kern2pt}|{\kern2pt} tʰv̩˧-hɑ̃˩-dʑo˩, {\kern2pt}|{\kern2pt} \textbf{ʂe˧-ʂo˧~ʂo˥} {\kern2pt}|{\kern2pt} tʰi˧-ki˧-kv̩˧ tsɯ˥ {\kern2pt}|{\kern2pt} mv̩˩!}\\
	\gll kʰv̩˩mi˩	-ki˧		ə{\dots}					ʂe˥		ɲi˧zo\#˥		ɖɯ˧-kʰwɤ˥\$			æ̃˩-ʂe\#˥		ɖɯ˧-kʰwɤ˥\$		bo˩-ʂe\#˥		ɖɯ˧-kʰwɤ˥\$		ʝi˧-ʂe\#˥		ɖɯ˧-kʰwɤ˥\$		kʰv̩˩mi˩	-ki˧		tʰv̩˧-hɑ̃˩		-dʑo˥		ʂe˥		\textbf{-ʂo˧{\textasciitilde}ʂo˩}	tʰi˧-		ki˧\textsubscript{a}		-kv̩˧˥		tsɯ˧˥		mv̩˧\\
	dog		\textsc{dat}		\textit{hesitation}		meat	fish	one-\textsc{clf}.piece		chicken\_meat		one-\textsc{clf}.piece		pork		one-\textsc{clf}.piece	beef		one-\textsc{clf}.piece	dog		\textsc{dat}	that\_day	\textsc{top}	meat	\textbf{\textsc{abundance}}		\textsc{dur}	to\_give	\textsc{abilitive}	\textsc{rep}	\textsc{affirm}\\
	\glt ‘To the dog, erm{\dots} [one would give] meat! A~piece of fish, a~piece of chicken, a~piece of pork, a~piece of beef{\dots} On that day [New Year's Eve], one would give the dog plenty of meat!' \textit{(Dog.35)} \pandoi{0004442\#S35}
\end{exe}

\begin{exe}
	\ex
	\label{ex:Housebuilding281}
	\ipaex{nɑ˩-ʑi˧mi˧ ʈʂʰɯ˧ {\kern2pt}|{\kern2pt} ɖɯ˧-ʈʂɤ˥{\textasciitilde}ʈʂɤ˩-ki˩-ze˩-se˩ {\kern2pt}|{\kern2pt} \textbf{si˧-ʂo˧{\textasciitilde}ʂo˥}-ɲi˩!}\\
	\gll nɑ˩˧		ʑi˧mi˧		ʈʂʰɯ˧						ɖɯ˧-			ʈʂɤ˧\textsubscript{a}	{\textasciitilde}	ki˧\textsubscript{a}		-ze˧	-se˩	si˥		\textbf{-ʂo˧{\textasciitilde}ʂo˩}	-ɲi˩\\
	Na~(\isi{endonym})	house		\textsc{top}		\textsc{delimitative}		to\_count	\textsc{activity}	to\_give		\textsc{pfv}	\textsc{completion}		wood		\textbf{\textsc{abundance}}	\textsc{certitude}\\
	\glt ‘The Na house, if one is going to count every part of it, [one will realize that] it is packed with wood!' (i.e.\ a~huge deal of wood goes into its construction) \textit{(Housebuilding.281)} \pandoi{0004448\#S281}
\end{exe}

This suffix does not have a non\babelhyphen{nobreak}reduplicated counterpart and does not belong to a~broader set of reduplicated nominal suffixes. One might therefore wonder whether it should be analyzed as a~reduplicated form or rather as a~simple suffix that happens to consist of two identical syllables. 
The reason for analyzing it as a~reduplicated form lies in its structural parallel with reduplicated suffixes added to adjectives (presented in \sectref{sec:thereduplicationofadjectives}). 

The reduplicated \is{suffixes}suffix /\ipa{-ʂo˧{$\sim$}ʂo˩}/ can be added to a~wide range of
nouns, including count nouns such as those denoting persons. For instance, a~household with numerous young men may be described
as /\ipa{pʰæ˧tɕi˥-ʂo˩{$\sim$}ʂo˩}/ ‘teeming with youngsters’. The broad semantic applicability of this \is{suffixes}suffix allows for the elicitation of an entire set, presented in \tabref{tab:thetonalbehaviourofthereduplicatedsuffixdependingonthetoneoftheprecedingnoun}. As elsewhere, the ‘+’ sign in the transcription of surface tone
patterns indicates the tone of the \isi{copula} when placed after the expression as a~test to ascertain
the type of syllabic {anchoring} of a~final H tone.

\begin{table}%[t]
	\caption{\label{tab:thetonalbehaviourofthereduplicatedsuffixdependingonthetoneoftheprecedingnoun}The tonal behaviour of the reduplicated suffix /\ipa{-ʂo˧{$\sim$}ʂo˩}/ depending on the tone of the preceding noun.}
	\begin{tabularx}{\textwidth}{ Q l l l Q }
		\lsptoprule
		example & tone & example & surface pattern & analysis\\ \midrule
		dust & LM & \ipa{ɖæ˩-ʂo˧{$\sim$}ʂo˩} & L.M.L & LM+L\#\\
		pimple & LH & \ipa{ʝi˩-ʂo˥{$\sim$}ʂo˩} & L.H.L (=L.M.L) & LH--\\
		star & M & \ipa{kɯ˧-ʂo˧{$\sim$}ʂo˥} & M.M.H+L & H\#\\
		grease & L & \ipa{mɤ˩-ʂo˩{$\sim$}ʂo˥} & L.L.H+L & L+H\#\\
		meat & H & \ipa{ʂe˧-ʂo˧{$\sim$}ʂo˥} & M.M.H+L & H\#\\
		mushroom & MH & \ipa{mo˧-ʂo˧{$\sim$}ʂo˥} & M.M.H+L & H\#\\ \addlinespace \hdashline \addlinespace
		dew & M & \ipa{ɖʐv̩˧qʰɑ˧-ʂo˧{$\sim$}ʂo˩} & M.M.M.L & L\#\\
		fly & \#H & \ipa{bv̩˧ɻ̩˧-ʂo˧{$\sim$}ʂo˥} & M.M.M.H+L & H\#\\
		paste & MH\# & \ipa{ho˧dʑɯ˧-ʂo˧{$\sim$}ʂo˥} & M.M.M.H+L & H\#\\
		mud & H\$ & \ipa{ɖʐæ˧qʰæ˧-ʂo˧{$\sim$}ʂo˥} & M.M.M.H+L & H\#\\
		egg & L & \ipa{æ̃˩ʁv̩˩-ʂo˩{$\sim$}ʂo˥} & L.L.L.H+L & L+H\#\\
		cake/bread & L\# & \ipa{dze˧dv̩˩-ʂo˩{$\sim$}ʂo˩} & M.L.L.L & L\#--\\
		bean chaff & LM+MH\# & \ipa{nv̩˩tsɑ˧-ʂo˧{$\sim$}ʂo˥} & L.M.M.H+L & LM+H\#\\
		potato & LM+\#H & \ipa{jɤ˩jo˧-ʂo˧{$\sim$}ʂo˥} & L.M.M.H+L & LM+H\#\\
		soybeans & LM & \ipa{nv̩˩ɭɯ˧-ʂo˧{$\sim$}ʂo˩} & L.M.M.L & LM+L\#\\
		button & LH & \ipa{pv̩˩ɭɯ˥-ʂo˩{$\sim$}ʂo˩} & L.H.L.L & LH--\\
		youngster & H\# & \ipa{pʰæ˧tɕi˥-ʂo˩{$\sim$}ʂo˩} & M.H.L.L & H\#--\\
		\lspbottomrule
	\end{tabularx}
\end{table}

The tone of the \is{suffixes}suffix /\ipa{-ʂo{$\sim$}ʂo}/ can be hypothesized to be L\# (hence the notation
//\ipa{ʂo˧{$\sim$}ʂo˩}// adopted here), based on its behaviour after M-tone and LM\babelhyphen{nobreak}tone disyllables. The tone patterns documented in \tabref{tab:thetonalbehaviourofthereduplicatedsuffixdependingonthetoneoftheprecedingnoun} are not entirely straightforward, however; they differ from those of disyllabic postpositions, discussed in \sectref{sec:spatialpostpositions}.


\section{Possessive constructions containing pronouns}
\label{sec:possessiveconstructionscontainingpronouns}

Possessive constructions were discussed in Chapter~\ref{chap:thelexicaltonesofnouns}, where the behaviour of nouns in association with the \isi{possessive} /\ipa{=bv̩˧}/ served as one of the tests for
determining lexical tone categories. Possessive constructions containing pronouns, however, do not follow quite the same tonal patterns. This is one of several respects in which pronouns are special.


\subsection[The 1$^{st}$, 2$^{nd}$ and 3$^{rd}$ person pronouns]{The 1\textsuperscript{st}, 2\textsuperscript{nd} and 3\textsuperscript{rd} person pronouns}
\label{sec:the1st2ndand3rdpersonpronouns}


A~\is{pronouns}pronoun's tonal category is established by matching up its tone \is{form!in isolation}in isolation with its tone when a~\isi{copula} is added (a test that is also useful for nouns, as explained in \sectref{sec:dynamicview}). On the basis of the tones in /\ipa{njɤ˩ ɲi˩˥}/ ‘it's me’, /\ipa{no˩ ɲi˩˥}/ ‘it's you’ and /\ipa{ʈʂʰɯ˧ ɲi˥}/ ‘it's her/him’, the \textsc{1sg}, \textsc{2sg}, and \textsc{3sg} pronouns are analyzed as //\ipa{njɤ˩}//, //\ipa{no˩}//, and //\ipa{ʈʂʰɯ˥}//, respectively.

To build a~\isi{possessive} construction with a~\is{pronouns}pronoun, the \isi{possessive} /\ipa{=bv̩˧}/ is generally
used, as illustrated in (\ref{ex:longagofiveofmyfamilyswerestolen}). The forms are /\ipa{njɤ˧=bv̩˩}/, /\ipa{no˧=bv̩˩}/, and /\ipa{ʈʂʰɯ˧=bv̩˧}/ for the 1\textsuperscript{st},
2\textsuperscript{nd}, and 3\textsuperscript{rd} persons.
\begin{exe}
  \ex
  \label{ex:longagofiveofmyfamilyswerestolen}
  \ipaex{ə˧ʝi˧-ʂɯ˥ʝi˩, {\kern2pt}|{\kern2pt} njɤ˧=bv̩˩ {\kern2pt}|{\kern2pt} ʐwæ˧ ʈʂʰɯ˧, {\kern2pt}|{\kern2pt} ŋwɤ˩-kv̩˩ ʈʂæ˥ {\kern2pt}|{\kern2pt} po˧ hɯ˧-ɲi˥!}\\
  \gll ə˧ʝi˧-ʂɯ˥ʝi˩	njɤ˩	=bv̩˧	ʐwæ˥	ʈʂʰɯ˧	ŋwɤ˧	kv̩˧˥	ʈʂæ˧˥ po˧˥			hɯ˧\textsubscript{c}		-ɲi˩\\
  in\_the\_past	\textsc{1sg}	\textsc{poss}	horse	\textsc{top}	five	\textsc{clf}	to\_rob to\_take\_away	to\_go.\textsc{pst}	\textsc{certitude}\\
  \glt ‘Once, long ago, five of my family’s horses were stolen!’ \textit{Literally:} ‘Long ago, my horses (=my
  family’s horses), five were stolen and taken away!’ \textit{(Caravans.183)} \pandoi{0004530\#S183}
\end{exe}


This is unlike the pattern for nouns: L-tone nouns with the \isi{possessive} yield L+M, and H-tone nouns yield M+H.

A~further complication is that the 3\textsuperscript{rd}-person \is{pronouns}pronoun appears in two forms: /\smash{\ipa{ʈʂʰɯ˧=bv̩˧}}/ and /\smash{\ipa{ʈʂʰɯ˧=bv̩˩}}/. These
are not tonal variants but semantically distinct forms. The latter, /\smash{\ipa{ʈʂʰɯ˧=bv̩˩}}/, is
a~reduced form of /\smash{\ipa{ʈʂʰɯ˧=ɻ̩˩=bv̩˩}}/, where /\smash{\ipa{=ɻ̩˩}}/ is the \isi{associative plural}. Thus,
/\smash{\ipa{ʈʂʰɯ˧=bv̩˩}}/ means ‘their’, whereas /\smash{\ipa{ʈʂʰɯ˧=bv̩˧}}/ simply means ‘her/his’. The
associative plural /\smash{\ipa{=ɻ̩˩}}/ is fully elided: it leaves no segmental trace, only a~tonal
difference on the \isi{possessive} particle. This example provides an insight into the expansion of \isi{morphotonology} through segmental
simplifications~-- a~type of {diachronic} change that is especially well\babelhyphen{nobreak}attested in {Bantu} languages. This
topic will be further explored in section \sectref{sec:syllablereduction} of Chapter~\ref{chap:yongningnatonesinadynamicsynchronicperspective}, where the Yongning Na tone system is examined from a~dynamic\babelhyphen{nobreak}synchronic perspective.

A~\is{pronouns}pronoun may also directly precede the noun to build a~\isi{possessive} construction, as in (\ref{ex:letmumdie}).

%\Hack{\newpage}

\begin{exe}
  \ex
  \label{ex:letmumdie}
  \ipaex{njɤ˧ mv̩˩{\dots} {\kern2pt}|{\kern2pt} ə˧zɯ˩ {\kern2pt}|{\kern2pt} ʂɯ˧-bi˧, {\kern2pt}|{\kern2pt} ə˧mi˧ {\kern2pt}|{\kern2pt} tʰi˧-ʂɯ˧-kʰɯ˧!}\\
  \gll njɤ˩		mv̩˩˥		ə˧zɯ˩		ʂɯ˧	-bi˧	ə˧mi˧	tʰi˧-	ʂɯ˧		-kʰɯ˧˥\\
  1\textsc{sg}		daughter	1\textsc{pl}.\textsc{incl}		to\_die	\textsc{imm\_fut}	mother	\textsc{dur}	to\_die	\textsc{caus}\\
  \glt ‘My dear daughter{\dots}  We are going to die [=we can’t avoid death, now that the tiger is
    at our door]; let Mum die [=let me sacrifice myself, so you can survive]!' \textit{(Tiger.16)} \pandoi{0004444\#S16}
\end{exe}

Combinations of pronouns and nouns, as in (\ref{ex:letmumdie}), were systematically elicited. The
results are identical for the 1\textsc{sg} and 2\textsc{sg} pronouns, //\ipa{njɤ˩}// and
//\ipa{no˩}//. They are shown in \tabref{tab:thetonesofpossessiveconstructionsconsistingofa1sgpronounandanoun}. The corresponding recording is: \textit{PossessPro} \pandoi{0004575}.

%Table 1.
\begin{table}%[t]
\caption{\label{tab:thetonesofpossessiveconstructionsconsistingofa1sgpronounandanoun}The tones of {possessive} constructions consisting of a~1\textsc{sg} pronoun and a~noun.}
\begin{tabularx}{\textwidth}{ Q Q P{25mm} l l }
\lsptoprule
	tone & head & meaning & example & tone pattern\\ \midrule
	LM & \ipa{bo˩˧} & pig & \ipa{njɤ˧ bo˩} & L\#\\
	LH & \ipa{mv̩˩˥} & daughter & \ipa{njɤ˧ mv̩˩} & L\#\\
	M & \ipa{zɯ˧} & life, existence & \ipa{njɤ˧ zɯ\#˥} & \#H\\
	L & \ipa{dʑɯ˩} & water & \ipa{njɤ˧ dʑɯ\#˥} & \#H\\
	\#H & \ipa{hĩ˥} & human being & \ipa{njɤ˧ hĩ\#˥} & \#H\\
	MH\# & \ipa{tsʰɯ˧˥} & goat & \ipa{njɤ˧ tsʰɯ˧˥} & MH\#\\ \addlinespace \hdashline \addlinespace
	M & \ipa{po˧lo˧} & ram & \ipa{njɤ˧ po˧lo˧} & M\\
	\#H & \ipa{ʐwæ˧zo\#˥} & colt & \ipa{njɤ˧ ʐwæ˧zo\#˥} & \#H\\
	MH\# & \ipa{hwɤ˧li˧˥} & cat & \ipa{njɤ˧ hwɤ˧li˧˥} & MH\#\\
	H\$ & \ipa{kv̩˧ʂe˥\$} & flea & \ipa{njɤ˧ kv̩˧ʂe˥\$} & H\$\\
	L & \ipa{kʰv̩˩mi˩} & dog & \ipa{njɤ˧ kʰv̩˩mi˩} & --L\\
	L\# & \ipa{dɑ˧ʝi˩} & mule & \ipa{njɤ˧ dɑ˧ʝi˩} & L\#\\
	LM+MH\# & \ipa{ʝi˩ʈʂæ˧˥} & waist & \ipa{njɤ˧ ʝi˩ʈʂæ˩} & --L\\
	LM+\#H & \ipa{bi˩ʈʂʰɤ\#˥} & whiskers & \ipa{njɤ˧ bi˩ʈʂʰɤ˩} & --L\\
	LM & \ipa{bo˩mi˧} & sow & \ipa{njɤ˧ bo˩mi˩} & --L\\
	LH & \ipa{bo˩ɬɑ˥} & boar & \ipa{njɤ˧ bo˩ɬɑ˩} & --L\\
	H\# & \ipa{kʰv̩˧nɑ˥} & dog & \ipa{njɤ˧ kʰv̩˧nɑ˥} & H\#\\
\lspbottomrule
\end{tabularx}
\end{table}

%\newpage 
In all these phrases, the \is{pronouns}pronoun //\ipa{njɤ˩}// carries the same tone as \is{form!in isolation}in isolation: an M tone. (Neutralization of //L//, \mbox{//M//}, and //H// to /M/ \is{form!in isolation}in isolation is the general rule for {monosyllabic} nouns and pronouns: see \sectref{sec:monosyllabicnouns}.) The patterns for
\is{monosyllables}monosyllabic nouns cannot be obtained through the application of a~simple set of general rules. On
the other hand, the patterns for disyllables are extremely simple. They consist of the succession of
the \is{pronouns}pronoun, as said \is{form!in isolation}in isolation: /\ipa{njɤ˧}/, followed by the noun, which carries the same tone
as when it appears on its own, except that some tone levels are deleted to comply with the phonological
constraints on a~well\babelhyphen{nobreak}formed \isi{tone group}. This affects the tone categories in which an L tone is
attached to the first syllable of the noun: L, LM+MH\#, LM+\#H, LM, and LH. For these categories, the sequence
found on the first two syllables (M tone on the \is{pronouns}pronoun, and L tone on the initial syllable of the
disyllabic noun) is incompatible with any tone other than L on following syllables, by Rule~5
(“All syllables following an H.L or M.L sequence receive L tone”: see \sectref{sec:alistoftonerules}). For instance, \ipa{$\ddagger${\kern2pt}njɤ˧ bo˩mi˧}
(obtained through simple concatenation) would not be a~well\babelhyphen{nobreak}formed \isi{tone group}; this is repaired to
/\ipa{njɤ˧ bo˩mi˩}/, where the final M is lowered to L. The representation in \figref{fig:concat} assumes that the M part of the LM tone is initially associated but subsequently deleted. Alternatively, one could consider that this M does not associate at all. Psycholinguistic experiments would be necessary to approach more closely the processes taking place in speakers' brains.

\begin{figure}[tb]
	\caption{Tone-to-syllable association in /\ipa{njɤ˧ bo˩mi˩}/ ‘my sow’.}
	\begin{tikzpicture}
	\node (1) at (0.5,-0.5) {L};
	\node (4) at (3,-0.5) {LM};
	
	\node (2) at (0.5,-1.5) {σ};
%	\node (3) at (1,-1.5) {σ};
	\node (5) at (2.5,-1.5) {σ};
	\node (91) at (3.5,-1.5) {σ};
	
	\node [anchor=mid] (s1l) at (0.5,-2) {/\ipa{njɤ}/ \textsc{1sg}};
	%  \node (s1ll) at (0.5,-2.5) {lexical tone: MH\#};
	
	\node [anchor=mid] (s1lll) at (3,-2) {/\ipa{bo.mi}/ ‘sow’};
	%  \node (s1llll) at (4,-2.5) {lexical tone: L};
	
	\node[text width=48mm] (s1) at (-3,-0.75) {Stage 1:\\ input};
%	\node[text width=48mm] (s1) at (-3,-0.75) {Stage 1:\\ input};
	
	
	
	\node (12) at (0.5,-3.2) {M};
	\node (42) at (2.5,-3.2) {L};
	\node (99) at (3.5,-3.2) {M};
	
	\node (22) at (0.5,-4.7) {σ};
%	\node (32) at (1,-5.5) {σ};
	\node (52) at (2.5,-4.7) {σ};
	\node (90) at (3.5,-4.7) {σ};
	
	\node[text width=48mm] (s2) at (-3,-3.95) {Stage 2:\\ separate tonal association\\  for the two words;\\ //L// surfaces as /M/};
	
	\draw[decoration={markings,mark=at position 1 with
		{\arrow[scale=2,>=stealth]{>}}},postaction={decorate}] (12) -- (22);
	\draw[decoration={markings,mark=at position 1 with
			{\arrow[scale=2,>=stealth]{>}}},postaction={decorate}] (42) -- (52);
	\draw[decoration={markings,mark=at position 1 with
				{\arrow[scale=2,>=stealth]{>}}},postaction={decorate}] (99) -- (90);
			
	\node (13) at (1,-6) {M};
	\node (63) at (2,-6) {L};
	\node (43) at (3,-6) {M};
	\node (98) at (4,-6) {L};
	
	\node (23) at (3,-6.7) {=}; % 6.75 puis 6.6 puis 6.65
	\node (3) at (3,-7.2) {};
	\node (33) at (1,-7.5) {σ};
	\node (53) at (2,-7.5) {σ};
	\node (92) at (3,-7.5) {σ};
	
	\node[text width=48mm] (s3) at (-3,-6.75) {Stage 3:\\ joining into one \isi{tone group};\\ replacement of M tone\\ by L, following Rule~5};
	
	\draw (13) -- (33);
	\draw (63) -- (53);
%	\draw (43) -- (92); % barred association line
	\draw (43) -- (3); % barred association line
		
	\draw[decoration={markings,mark=at position 1 with
		{\arrow[scale=2,>=stealth]{>}}},postaction={decorate}] (98) -- (92);
	% Now to final (surface) stage:
	
	\node (14) at (1,-8.8) {M};
	\node (64) at (2,-8.8) {L};
	\node (44) at (3,-8.8) {L};
	
	\node (34) at (1,-10.3) {σ};
	\node (54) at (2,-10.3) {σ};
	\node (94) at (3,-10.3) {σ};
	
	\node[text width=48mm] (s4) at (-3,-9.55) {Stage 4:\\ resulting surface \\ phonological tone};
	
	\draw (14) -- (34);
	\draw (64) -- (54);
	\draw (44) -- (94); 

	\end{tikzpicture}
	\label{fig:concat}
\end{figure}


The construction in \tabref{tab:thetonesofpossessiveconstructionsconsistingofa1sgpronounandanoun} exemplifies minimal tonal integration between two elements. It can be
described as concatenation of the two parts of the expression, followed by adjustments dictated by phonological rules.

For disyllables, the tone patterns following the 3\textsuperscript{rd} person \is{pronouns}pronoun //\ipa{ʈʂʰɯ˥}// are
identical with those found after the 1\textsuperscript{st} and 2\textsuperscript{nd} persons. For
monosyllables, on the other hand, a contrast arises in the case of L-tone nouns: compare //\ipa{njɤ˧ dʑɯ\#˥}// ‘my water’ with //\ipa{ʈʂʰɯ˧ dʑɯ˧}// ‘her/his water’. This asymmetry poses yet another
challenge to the \is{language acquisition}language learner, who must (i)~distinguish the tone patterns associated with these two sets of pronouns when they combine with a~{monosyllabic} noun, while (ii)~disregarding this
difference when the following noun is disyllabic. The full set is shown in  \tabref{tab:thetonesofpossessiveconstructionsconsistingofa3sgpronounandanoun}.

\begin{table}%[t]
\caption{\label{tab:thetonesofpossessiveconstructionsconsistingofa3sgpronounandanoun}The tones of {possessive} constructions consisting of a~3\textsc{sg} pronoun and a~noun.}
\begin{tabularx}{\textwidth}{ Q Q l l l }
\lsptoprule
	tone & example & meaning & with \textsc{3sg} & tone pattern\\ \midrule
	LM & \ipa{bo˩˧} & pig & \ipa{ʈʂʰɯ˧ bo˩} & L\#\\
	LH & \ipa{mv̩˩˥} & daughter & \ipa{ʈʂʰɯ˧ mv̩˩} & L\#\\
	M & \ipa{zɯ˧} & life, existence & \ipa{ʈʂʰɯ˧ zɯ\#˥} & \#H\\
	L & \ipa{dʑɯ˩} & water & \ipa{ʈʂʰɯ˧ dʑɯ˧} & M\\
	\#H & \ipa{hĩ˥} & human being & \ipa{ʈʂʰɯ˧ hĩ\#˥} & \#H\\
	MH\# & \ipa{tsʰɯ˧˥} & goat & \ipa{ʈʂʰɯ˧ tsʰɯ˧˥} & MH\#\\ \addlinespace \hdashline \addlinespace
	M & \ipa{po˧lo˧} & ram & \ipa{ʈʂʰɯ˧ po˧lo˧} & M\\
	\#H & \ipa{ʐwæ˧zo\#˥} & colt & \ipa{ʈʂʰɯ˧ ʐwæ˧zo\#˥} & \#H\\
	MH\# & \ipa{hwɤ˧li˧˥} & cat & \ipa{ʈʂʰɯ˧ hwɤ˧li˧˥} & MH\#\\
	H\$ & \ipa{ə˧dɑ˥\$} & father & \ipa{ʈʂʰɯ˧ ə˧dɑ\#˥} & \#H\\
	L & \ipa{kʰv̩˩mi˩} & dog & \ipa{ʈʂʰɯ˧ kʰv̩˩mi˩} & --L\\
	L\# & \ipa{dɑ˧ʝi˩} & mule & \ipa{ʈʂʰɯ˧ dɑ˧ʝi˩} & L\#\\
	LM+MH\# & \ipa{ʝi˩ʈʂæ˧˥} & waist & \ipa{ʈʂʰɯ˧ ʝi˩ʈʂæ˩} & --L\\
	LM+\#H & \ipa{bi˩ʈʂʰɤ\#˥} & whiskers & \ipa{ʈʂʰɯ˧ bi˩ʈʂʰɤ˩} & --L\\
	LM & \ipa{bo˩mi˧} & sow & \ipa{ʈʂʰɯ˧ bo˩mi˩} & --L\\
	LH & \ipa{bo˩ɬɑ˥} & boar & \ipa{ʈʂʰɯ˧ bo˩ɬɑ˩} & --L\\
	H\# & \ipa{kʰv̩˧nɑ˥} & dog & \ipa{ʈʂʰɯ˧ kʰv̩˧nɑ˥} & H\#\\
\lspbottomrule
\end{tabularx}
\end{table}

 
Finally, a~looser construction also exists: a~simple juxtaposition of the \is{pronouns}pronoun and the
noun, each in its own tone group, as in (\ref{ex:goneawaytowork}). 

 
% \Hack{\newpage}

\begin{exe}
  \ex
  \label{ex:goneawaytowork}
  \ipaex{njɤ˧ {\kern2pt}|{\kern2pt} ɻ̩˩ʈʂʰe˧-ɖɯ˩mɑ˩ ʈʂʰɯ˩-dʑo˩, {\kern2pt}|{\kern2pt} no˧sɯ˩kv̩˩ {\kern2pt}|{\kern2pt} tʰv̩˧-ɲi˧ {\kern2pt}|{\kern2pt} lo˧ ʝi˧ hɯ˧ tsɯ˩.}\\
  \gll njɤ˩		ɻ̩˩ʈʂʰe˧-ɖɯ˩mɑ˩		ʈʂʰɯ˧	-dʑo˥	no˧sɯ˩kv̩˩ tʰv̩˧-ɲi˧	lo˧ 	ʝi˧	hɯ˧\textsubscript{c}	tsɯ˧˥\\
  1\textsc{sg}		given\_name	\textsc{top}	\textsc{top}	\textsc{2pl}.\textsc{excl} that.day	work	to\_do	go.\textsc{pst}	\textsc{rep}\\
  \glt ‘As for my [daughter] Erchei Ddeema{\dots} that day, you (=the members of your
  family) had gone away to work.' \textit{(BuriedAlive2.132)} \pandoi{0004536\#S132}
\end{exe}

\newpage
This is not a~\isi{possessive}
construction in the strict sense; rather, the \is{pronouns}pronoun serves as a~topic. The context for this example is as follows. A~young woman is unhappy with a~marriage arranged by her family and commits a~small offence, which takes on huge proportions. Her mother then comes to the mother\babelhyphen{nobreak}in\babelhyphen{nobreak}law’s house to make things right by talking the matter over. The young woman’s mother first recounts the entire story, clarifying the actions of both parties~-- the members
of the two families. In this situation, the first-person \is{pronouns}pronoun in the construction /\ipa{njɤ˧ {\kern2pt}|{\kern2pt}
  ɻ̩˩ʈʂʰe˧-ɖɯ˩mɑ˩}/ highlights that the mother is speaking on her daughter’s behalf and, as head of the family, assumes some responsibility for her daughter’s actions.

\largerpage
\subsection{The pronoun ‘oneself’}
\label{sec:thepronounoneself}

The \is{pronouns}pronoun /\ipa{õ˧˥}/ ‘oneself’ has a~different behaviour from the 1\textsuperscript{st},
2\textsuperscript{nd}, and 3\textsuperscript{rd} person pronouns. 
It can be followed by the \isi{possessive} marker /\ipa{=bv̩˧}/, as illustrated in (\ref{ex:ourown}).
\begin{exe}
  \ex
  \label{ex:ourown}
  \ipaex{ə˧ʝi˧-tsʰi˧ʝi˧, {\kern2pt}|{\kern2pt} ɖʐɯ˧qo˩ {\kern2pt}|{\kern2pt} le˧-hwæ˧, {\kern2pt}|{\kern2pt} ɖɯ˧-tʰɑ˧˥ {\kern2pt}|{\kern2pt} pi˧-bi˧-bi˧, {\kern2pt}|{\kern2pt} õ˧=bv̩˥ {\kern2pt}|{\kern2pt} ʈʂʰɯ˧-kʰwɤ˥-dʑo˩, {\kern2pt}|{\kern2pt} dʑɤ˩ ʈʂɤ˧ ɲi˥, {\kern2pt}|{\kern2pt} pi˧-zo˩, {\kern2pt}|{\kern2pt} ɑ˩ʁo˧ tʰi˧-tɕɯ˥!}\\
  \gll ə˧ʝi˧-tsʰi˧ʝi˧   ɖʐɯ˧qo˩     le˧-    hwæ˧\textsubscript{a}   ɖɯ˧\textsubscript{b}    -tʰɑ˧˥  pi˥     -bi-bi    õ˧˥    =bv̩˧   ʈʂʰɯ˥   kʰwɤ˥\textsubscript{a}      -dʑo˥    dʑɤ˩\textsubscript{b}      ʈʂɤ˧\textsubscript{a}     ɲi˩   pi˥     -zo     ɑ˩ʁo˧   tʰi˧-     tɕɯ˥\\
  nowadays  market  \textsc{accomp}     to\_buy    to\_obtain   \textsc{permissive}	to\_say  \textsc{permissive}   oneself     \textsc{poss}   \textsc{dem.prox}   \textsc{clf}.chunks     \textsc{top}        good  to\_count\_as   \textsc{certitude}      to\_say  \textsc{advb}   home    \textsc{dur}    to\_lay\_up\\
  \glt ‘These days, even if we can buy things at the market and find what we want, well, we consider the things that are our own [productions] as good (=as better than what we buy at the market), so we store them at home!’ (\textit{Benevolence.15}) \pandoi{0004582\#S15}
\end{exe}

Example (\ref{ex:ourown}) is the only instance found in texts where the sequence /\ipa{õ˧=bv̩˥}/ ‘one’s own’ appears independently, rather than as part of the construction /\ipa{õ˧{\linebreak}=bv̩˥-õ˩}/ ‘one’s own’, ‘one’s proper’, ‘oneself’, which occurs 59 times in the available corpus (as of 2025). The latter construction is typically followed by a~noun or verb. An example is shown in (\ref{ex:weeatourownproduce}).

\begin{exe}
  \ex
  \label{ex:weeatourownproduce}
  \ipaex{õ˧=bv̩˥-õ˩ hɑ˩, {\kern2pt}|{\kern2pt} õ˧=bv̩˥-õ˩ dzɯ˩!}\\
  \gll õ˧˥		=bv̩˧	õ˧˥		hɑ˥		õ˧˥		=bv̩˧	õ˧˥		dzɯ˥\\
  oneself	\textsc{poss}	oneself		food		oneself	\textsc{poss}	oneself		to\_eat\\
  \glt ‘One’s own food, one eats it oneself!~/ We eat our own produce!’ \textit{Context:} this statement summarizes the traditional self\babelhyphen{nobreak}sufficiency of the Na, who grew their own food crops. \textit{(Agriculture.68)} \pandoi{0004440\#S68}
\end{exe}

The symmetry of the construction /\ipa{õ˧=bv̩˥-õ˩}/, which begins and ends with the morpheme /\ipa{õ˧˥}/ ‘self’, makes it an \is{iconicity}iconic, Narcissus-style mirror image of its meaning: namely, pointing to \textit{self}. 

The widespread use of the phrase /\ipa{õ˧=bv̩˥-õ˩}/ (its ``reproductive success'', to hijack a term from evolutionary biology) may also be linked to another matter of linguistic form, namely its tonal simplicity. Following this phrase, tonal oppositions are fully neutralized, as a following word can only carry L tones, due to a~constraint formulated in the present description as Rule~5: “All syllables following an H.L or M.L sequence receive L tone”.

Outside constructions involving the possessive morpheme, the \is{pronouns}pronoun /\ipa{õ˧˥}/ typically combines directly with a~following noun or verb, as
in (\ref{ex:thisismyownbrother}) and (\ref{ex:onesmeat}).
\begin{exe}
  \ex
  \label{ex:thisismyownbrother}
  \ipaex{õ˧ ə˧mv̩˥ ɲi˩-ze˩!}\\
  \gll õ˧˥	ə˧mv̩˩		ɲi˩	-ze˧\\
  self	elder\_sibling	\textsc{cop}	\textsc{pfv}\\
  \glt ‘This is my own brother!’ \textit{Context:} a~young woman recognizes a~ragged
  stranger attending her wedding as being her long\babelhyphen{nobreak}lost brother. (\textit{Sister.57} \pandoi{0004341\#S57}; \textit{Sister3.58} \pandoi{0004344\#S58})
\end{exe}

\begin{exe}
	\ex
	\label{ex:onesmeat}
	\ipaex{õ˧-ʂe˥, õ˩ ʈʰæ˩!}\\
	\gll õ˧˥	ʂe˥		õ˧˥	ʈʰæ˧˥\\
	self	meat	self	to\_bite\\
	\glt ‘Each person eats her/his own slab of meat!’ \textit{Context:} describing
	table manners. Each family member used to receive
	one slice of meat and eat it up. This differs from Han Chinese custom, where each guest picks food
	mouthful by mouthful, with chopsticks, from dishes placed on the table. (Field notes.)
\end{exe}

The construction /\ipa{õ˧˥}/ plus N, ‘[my/one’s] own N’, has a~tonal behaviour of its
own. On the {analogy} of
(\ref{ex:onesmeat}), new maxims can be coined, such as (\ref{ex:ones2}), (\ref{ex:ones3}), and (\ref{ex:ones4}). 

\begin{exe}
	\ex
	\label{ex:ones2}
	\ipaex{õ˧-dʑɯ˥, õ˩ ʈʰɯ˩!}\\
	\gll õ˧˥	dʑɯ˩	õ˧˥	ʈʰɯ˩\textsubscript{b}\\
	self	water	self	to\_drink\\
	\glt ‘Each drinks from her own bottle!’ \textit{Context:} a~toddler has grabbed another’s bottle; parents
	prevent her from drinking from it. (Field notes.)
\end{exe}

\begin{exe}
	\ex
	\label{ex:ones3}
	\ipaex{õ˧-ɖæ˥, õ˩ bæ˩!}\\
	\gll õ˧˥	ɖæ˩˧	õ˧˥	bæ˩\textsubscript{a}\\
	self	dust	self	to\_sweep\\
	\glt ‘One must sweep one’s
	own garbage.’ (Elicited example)
\end{exe}

\begin{exe}
	\ex
	\label{ex:ones4}
	\ipaex{õ˧-lv̩˥, õ˩ li˩!}\\
	\gll õ˧˥	lv̩˧		õ˧˥	li˧\textsubscript{a}\\
	self	field	self	to\_look\_after\\
	\glt ‘One must look after one’s own fields.’ (Elicited)
\end{exe}

Combinations were systematically elicited. The full set of tonal combinations is presented in \tabref{tab:thetonesofpossessiveconstructionsconsistingofnoun}.

\begin{table}%[t]
\caption{\label{tab:thetonesofpossessiveconstructionsconsistingofnoun}The tones of {possessive} constructions consisting of /\ipa{õ˧˥}/+N.}
\begin{tabularx}{\textwidth}{ P{16mm}@{\hspace{2mm}} P{17mm}@{\hspace{2mm}} P{23mm}@{\hspace{2mm}} P{36mm}@{\hspace{2mm}} l@{\hspace{1mm}} }
\lsptoprule
	tone & example & meaning & with /\ipa{õ˧˥}/ & tone pattern\\ \midrule
	LM & \ipa{ɖæ˩˧} & dirt, dust & \ipa{õ˧-ɖæ˥} & H\#\\
	LH & \ipa{mv̩˩˥} & daughter & \ipa{õ˧-mv̩˥\$} & H\$\\
	M & \ipa{lv̩˧} & field & \ipa{õ˧-lv̩˥\$} & H\$\\
	L & \ipa{dʑɯ˩} & water & \ipa{õ˧-dʑɯ˥\$} & H\$\\
	\#H & \ipa{zo˥} & son & \ipa{õ˧-zo\#˥} & H\#\\
	MH\# & \ipa{tsʰɯ˧˥} & goat & \ipa{õ˧-tsʰɯ˥\$} & H\$\\ \addlinespace \hdashline \addlinespace
	M & \ipa{go˧mi˧} & younger sister & \ipa{õ˧-go˧mi˥} & H\#\\
	\#H & \ipa{ʐwæ˧zo\#˥} & colt & \ipa{õ˧-ʐwæ˧zo\#˥} & \#H\\
	MH\# & \ipa{ə˧mv̩˧˥} & elder sibling & \ipa{õ˧-ə˧mv̩˧˥ / õ˧-ə˥mv̩˩} & MH\# / MH\#--\\
	H\$ & \ipa{ə˧dɑ˥\$} & father & \ipa{õ˧-ə˧dɑ˥ / õ˧-ə˧dɑ\#˥} & H\# / \#H\\
	L & \ipa{kʰv̩˩mi˩} & dog & \ipa{õ˧-kʰv̩˥mi˩} & MH\#--\\
	L\# & \ipa{dɑ˧ʝi˩} & mule & \ipa{õ˧-dɑ˧ʝi˥} & H\#\\
	LM+MH\# & \ipa{ʝi˩ʈʂæ˧˥} & waist & \ipa{õ˧-ʝi˥ʈʂæ˩} & MH\#--\\
	LM+\#H & \ipa{bi˩ʈʂʰɤ\#˥} & whiskers & \ipa{õ˧-bi˥ʈʂʰɤ˩} & MH\#--\\
	LM & \ipa{ɑ˩ʁo˧} & home & \ipa{õ˧-ɑ˥ʁo˩} & MH\#--\\
	LH & \ipa{bo˩ɬɑ˥} & boar & \ipa{õ˧-bo˥ɬɑ˩} & MH\#--\\
	H\# & \ipa{kʰv̩˧nɑ˥} & dog & \ipa{õ˧-kʰv̩˧nɑ˥} & H\#\\
\lspbottomrule
\end{tabularx}
\end{table}


The behaviour of /\ipa{õ˧˥}/ in association with disyllables coincides with that of determinative
compounds containing an MH\babelhyphen{nobreak}tone determiner. With monosyllables, however, the tone patterns only
coincide with those of determinative compounds for nouns with LM or MH\# tone.


\section[Monosyllabic enclitics, suffixes, and postpositions]{Monosyllabic morphemes appearing after nouns: Enclitics, suffixes, and postpositions}
\label{sec:enclitics}

The analysis of morphemes as affixes, clitics, postpositions, serial verbs, “particles”, and other parts of speech raises interesting issues that differ widely from one language to another, as illustrated by the diversity of proposals and viewpoints presented in a~collective volume about the notion of “word” \citep{dixonetal2002b}. \citet[43]{aikhenvald2002} proposes that there is “a multidimensional continuum, from a~fully bound to a~fully independent morpheme”. The approach adopted here is to start out from tone patterns and move towards an analysis in light of the morphemes' tonal behaviour, rather than proceeding from predefined morphosyntactic categories. This method provides independent evidence for syntactic analysis and part-of-speech classification. For instance, the dative /\ipa{-ki˧}/ and the \isi{possessive} /\ipa{=bv̩˧}/ turn out to have exactly the same tonal behaviour, distinct from that of the agentive /\ipa{ɳɯ˧}/ (see \sectref{sec:encliticsthatcarrymtonewhenfollowingamtonenoun}), suggesting that the first two belong to the same morphosyntactic category, whereas the latter constitutes a separate class. This finding can be taken as confirmation for the observation~-- based on syntactic behaviour~-- that the dative and \isi{possessive} are “almost suffixal” \citep[155]{lidz2010}, whereas the agentive is best analyzed as a~case adposition.

Enclitics, suffixes, and postpositions are grouped into four subsets on the basis of their behaviour following M-tone nouns. 
Those that surface\is{form!surface} with L tone in this context are considered to carry lexical L tone; likewise, those that surface with M, MH, and H tones are considered to carry M, MH, and H tones, respectively. The following sections, each of which is dedicated to one of the four subsets, show that these broad tonal categories are not fully homogeneous, but they provide a useful framework for presenting the data.

\subsection{L-tone morphemes}
\label{sec:ltoneencliticspluralandassociativeplural}

This subsection examines three morphemes: the \is{postpositions}postposition meaning ‘on; at' and the plural and \is{associative plural}associative clitics. 

The \is{postpositions}postposition //\ipa{bi˩}// ‘on; at' surfaces with L tone after an M-tone noun, e.g.~in /\ipa{gv̩˧mi˧ bi˩}/ ‘on the body'. Additional examples from texts and field notes include disyllabic nouns with L tone, as in /\ipa{ʐæ˩sɯ˩ bi˥}/ ‘on the felt cape' (\textit{Sister.58} \pandoi{0004341\#S58}). %, and with LM tone, as in /\ipa{lo˩qʰwɤ˧ bi˩}/ ‘on the hand'. %% Can't seem to locate it in 2025. Sister3.80 is not an exact match.
To establish a~full dataset, systematic elicitation was conducted, yielding the patterns presented in \tabref{tab:postpositionon}. 


\begin{table}%[t]
	\caption{\label{tab:postpositionon}The behaviour of the L-tone postposition //\ipa{bi˩}// ‘on; at' with {monosyllabic} and disyllabic nouns. There is an additional ‘L \textsc{pro}’ row because L-tone pronouns have exceptional behaviour.}
	\begin{tabularx}{\textwidth}{ Q Q P{30mm} l }
		\lsptoprule
		example & tone & with /\ipa{bi˩}/ & surface tone pattern\\ \midrule
		pig & LM & \ipa{bo˩ bi˥} & L.H\\
		leopard & LH & \ipa{ʐæ˩ bi˥} & L.H\\
		tiger & M & \ipa{lɑ˧ bi˩} & M.L\\
		sheep & L & \ipa{jo˩ bi˩˥} & L.LH\\
		\textsc{2sg} & L \textsc{pro} & \ipa{no˧ bi˩} & M.L\\
		horse & H & \ipa{ʐwæ˧ bi˥} & M.H\\
		deer & MH & \ipa{ʈʂʰæ˧ bi˥} & M.H\\ \addlinespace \hdashline \addlinespace
		fox & M & \ipa{ɖɤ˧mi˧ bi˩} & M.M.L\\
		colt & \#H & \ipa{ʐwæ˧zo˧ bi˥} & M.M.H\\
		cat & MH\# & \ipa{hwɤ˧li˧ bi˥} & M.M.H\\
		she\babelhyphen{nobreak}cat & H\$ & \ipa{hwɤ˧mi˧ bi˥} & M.M.H\\
		dog & L & \ipa{kʰv̩˩mi˩ bi˥} & L.L.H\\
		mule & L\# & \ipa{dɑ˧ʝi˩ bi˩} & M.L.L\\
		wolf & LM+MH\# & \ipa{õ˩dv̩˧ bi˥} & L.M.H\\
		Naxi & LM+\#H & \ipa{nɑ˩hĩ˧ bi˥} & L.M.H\\
		sow & LM & \ipa{bo˩mi˧ bi˩} & L.M.L\\
		boar & LH & \ipa{bo˩ɬɑ˥ bi˩} & L.H.L\\
		rat & H\# & \ipa{hwæ˧tsɯ˥ bi˩} & M.H.L\\
		\lspbottomrule
	\end{tabularx}
\end{table}

The tonal patterns indicated in \tabref{tab:postpositionon} are those observed at the \is{form!surface}surface phonological level, not the underlying tones. In the case of /\ipa{bo˩ bi˥}/ ‘on \mbox{(a/the)} pig' and /\ipa{ʐæ˩ bi˥}/ ‘on \mbox{(a/the)} leopard', it remains uncertain whether the underlying pattern is //L.M// or //L.H//, as both are neutralized at the surface level due to Rule~6 (see \sectref{sec:alistoftonerules}). Consequently,  \tabref{tab:postpositionon} does not reveal whether the tone patterns of ‘on \mbox{(a/the)} pig' and ‘on \mbox{(a/the)} leopard' are underlyingly identical.

Further evidence comes from the plural clitic /\ipa{=ɻæ˩}/ and the \isi{associative plural} clitic /\ipa{=ɻ̩˩}/. As in \ili{Japhug}, where the plural \is{clitics}clitic /\ipa{=ra}/ conveys either plurality or collective meaning, these enclitics are not obligatory for non\babelhyphen{nobreak}singular arguments, even for human referents \citep{jacques2020grammaticalization}. Like the \is{postpositions}postposition /\ipa{bi˩}/ ‘on; at', these two enclitics are analyzed as having L tone on the basis of their tonal behaviour after M-tone nouns. Their tone patterns are identical to those of /\ipa{bi˩}/. 

For the plural and \is{associative plural}associative plural morphemes, an additional diagnostic is available: the {possessive} /\ipa{=bv̩˧}/ can be added to an \textit{N+{plural}} construction to test the \is{form!underlying}underlying tone category, following the method used in the study of the tones of nouns
(Chapter~\ref{chap:thelexicaltonesofnouns}). This test distinguishes //L.M//, which does not depress the tone of the following {possessive}, from //L.H//, which does. \tabref{tab:thetonalbehaviourofplural} presents the
tonal outcomes for N+{\allowbreak}{plural}+{\allowbreak}{possessive}. 

\begin{sidewaystable}[p]
\caption{\label{tab:thetonalbehaviourofplural}The tonal behaviour of nouns followed by the {plural} clitic /\ipa{=ɻæ˩}/ plus the {possessive} suffix /\ipa{=bv̩˧}/.}
\begin{tabularx}{\textwidth}{ l l l Q l }
\lsptoprule
	example & tone & +\textsc{plural} & +\textsc{plural}+\textsc{possessive} & tonal analysis\\ \midrule
	Na (ethnic group) & LM & \ipa{nɑ˩=ɻæ˥} & \ipa{nɑ˩=ɻæ˥=bv̩˩} & LH\\
	daughter & LH & \ipa{mv̩˩=ɻæ˥} & \ipa{mv̩˩=ɻæ˥=bv̩˩} & LH\\
	Han (ethnic group) & M & \ipa{hæ˧=ɻæ˩} & \ipa{hæ˧=ɻæ˩=bv̩˩} & L\#\\
	sheep & L & \ipa{jo˩=ɻæ˩˥} & \ipa{jo˩=ɻæ˩=bv̩˥} & L\\
	person, human being & H & \ipa{hĩ˧=ɻæ˥} & \ipa{hĩ˧=ɻæ˥=bv̩˩ / hĩ˧=ɻæ˧=bv̩˥} & H\# / H\$\\
	deer & MH & \ipa{ʈʂʰæ˧=ɻæ˥} & \ipa{ʈʂʰæ˧=ɻæ˥=bv̩˩} & H\#\\ \addlinespace \hdashline \addlinespace
	aunt & M & \ipa{ə˧mi˧=ɻæ˩} & \ipa{ə˧mi˧=ɻæ˩=bv̩˩} & --L\\
	younger brother & \#H & \ipa{gi˧zɯ˧=ɻæ˥} & \ipa{gi˧zɯ˧=ɻæ˧=bv̩˥ / gi˧zɯ˧=ɻæ˥=bv̩˩} & H\$ / H\#\\
	maternal uncle & MH\# & \ipa{ə˧v̩˧=ɻæ˥} & \ipa{ə˧v̩˧=ɻæ˧=bv̩˥ / ə˧v̩˧=ɻæ˥=bv̩˩} & H\$ / H\#\\
	she\babelhyphen{nobreak}cat & H\$ & \ipa{hwɤ˧mi˧=ɻæ˥} & \ipa{hwɤ˧mi˧=ɻæ˧=bv̩˥ / hwɤ˧mi˧=ɻæ˥=bv̩˩} & H\$ / H\#\\
	woman & L & \ipa{mi˩zɯ˩=ɻæ˥} & \ipa{mi˩zɯ˩=ɻæ˩=bv̩˥} & L+H\#\\
	elder sibling & L\# & \ipa{ə˧mv̩˩=ɻæ˩} & \ipa{ə˧mv̩˩=ɻæ˩=bv̩˩} & L\#--\\
	wolf & LM+MH\# & \ipa{õ˩dv̩˧=ɻæ˥} & \ipa{õ˩dv̩˧=ɻæ˧=bv̩˥ / õ˩dv̩˧=ɻæ˥=bv̩˩} & LM+H\# / LM+H\$\\
	Naxi (ethnic group) & LM+\#H & \ipa{nɑ˩hĩ˧=ɻæ˥} & \ipa{nɑ˩hĩ˧=ɻæ˧=bv̩˥ / nɑ˩hĩ˧=ɻæ˥=bv̩˩} & LM+H\# / LM+H\$\\
	sow & LM & \ipa{bo˩mi˧=ɻæ˩} & \ipa{bo˩mi˧=ɻæ˩=bv̩˩} & LM--L\\
	boar & LH & \ipa{bo˩ɬɑ˥=ɻæ˩} & \ipa{bo˩ɬɑ˥=ɻæ˩=bv̩˩} & LH--\\
	young man & H\# & \ipa{pʰæ˧tɕi˥=ɻæ˩} & \ipa{pʰæ˧tɕi˥=ɻæ˩=bv̩˩} & H\#--\\
\lspbottomrule
\end{tabularx}
\end{sidewaystable}


As in other contexts, expressions involving pronouns do not always follow the same tonal patterns as those containing nouns. The proximal
\is{demonstratives}demonstrative //\ipa{ʈʂʰɯ˥}// and the distal \is{demonstratives}demonstrative
//\ipa{tʰv̩˥}// yield /\ipa{ʈʂʰɯ˧=ɻæ˥\$}/ and /\ipa{tʰv˧{\linebreak}=ɻæ˥\$}/, i.e.\ the same pattern as for {monosyllabic} nouns with H tone. However, the first- and second-person pronouns, //\ipa{njɤ˩}//
and //\ipa{no˩}//, yield /\ipa{njɤ˧=ɻæ˩}/ and /\ipa{no˧=ɻæ˩}/ with the plural, diverging from the pattern observed for L\babelhyphen{nobreak}tone nouns such as /\ipa{jo˩=ɻæ˩˥}/ ‘sheep’. 

\tabref{tab:thetonalbehaviourofplural} brings out tonal differences between nouns and noun-plus-\is{clitics}clitic expressions. For instance, the noun /\ipa{ə˧v̩˧˥}/ ‘uncle’ yields /\ipa{ə˧v̩˧=ɻæ˥}/ ‘the uncles’ and /\ipa{ə˧v̩˧=ɻæ˧=bv̩˥}/ ‘of the uncles’, i.e.\ a~tonal alternation not found among nouns: no tonal category of disyllables ends in H tone \is{form!in isolation}in isolation while also yielding a~final H tone on a~following \isi{possessive}. The H\$ tone category of disyllables followed by a \isi{possessive} yields M.M.M, not M.M.H, as in 
%/\ipa{kv̩˧ʂe˥}/ ‘flea’ and 
/\ipa{kv̩˧ʂe˧=bv̩˧}/ ‘of \mbox{(a/the)} flea’ (see \tabref{tab:thelexicaltonesofdisyllabicnouns}). Thus, the tonal pattern of /\ipa{ə˧v̩˧=ɻæ˥}/ ‘the uncles’ does not strictly correspond to any of the tonal categories of disyllabic nouns. Similarly, the behaviour of /\ipa{hĩ˧=ɻæ˥}/ ‘the people’, which yields /\ipa{hĩ˧=ɻæ˧=bv̩˥}/ ‘of the people’, lacks an exact counterpart among disyllabic nouns. Impressionistically, it is as~if H\$, the \textit{hopping} H tone, exhibited a simplified behaviour when found in an expression containing a~\is{clitics}clitic: the H level tends to attach to the following syllable, even if that syllable is a~\is{suffixes}suffix. This differs from the behaviour of H\$\babelhyphen{nobreak}tone nouns, whose combinations with suffixes of different tone categories yield distinct results. 

The plural clitic /\ipa{=ɻæ˩}/ is frequently used. By contrast, the \isi{associative plural} /\ipa{=ɻ̩}{\kern2pt}/
has a~highly specific meaning, referring to a clan or extended family, and is therefore
mostly restricted to pronouns and family (clan) names, which are few in number. It cannot be used
with kinship terms. However, it is not implausible that the morpheme /\ipa{=ɻ̩˩}/ that partakes in nominalization processes is in fact the \isi{associative plural}. An example is shown in~(\ref{ex:themountainsmakeupacouple}), where the reduplicated verb /\ipa{pʰæ˧{$\sim$}pʰæ˧}/ ‘to attach’ in combination
with /\ipa{=ɻ̩˩}/ conveys the meaning ‘a couple; a~pair; a~set (of things, persons{\dots}) tied
together’. 

\begin{exe}
  \ex
  \label{ex:themountainsmakeupacouple}
  \ipaex{pʰæ˧{$\sim$}pʰæ˧=ɻ̩˩ ɲi˩-kv̩˩ tsɯ˩ {\kern2pt}|{\kern2pt} mv̩˩.}\\
  \gll pʰæ˧\textsubscript{b}	   {$\sim$}	=ɻ̩˩	ɲi˩	-kv̩˧˥		tsɯ˧˥	mv̩˧\\
  to\_tie/fasten	   \textsc{red}	\textsc{associative}	\textsc{cop}	\textsc{abilitive}	\textsc{rep}
  \textsc{affirm}\\
  \glt ‘[The mountains Gemu \ipa{kɤ˧mv̩˧˥} and Aeshae \ipa{æ˧ʂæ˧}] make up a~couple/a pair!’ \textit{(Mountains.99)} \pandoi{0004573\#S99}
\end{exe}

%To avoid unsightly change in interlinear spacing: use of 3\smash{\textsuperscript{rd}}-person pronoun


Not unexpectedly, pronouns followed by the \isi{associative plural} have a~tonal behaviour of their own. The two L-tone pronouns (1\textsc{sg} and
2\textsc{sg}) follow the same tonal pattern with the associative plural as with the plural: /\ipa{njɤ˧=ɻ̩˩}/
and /\ipa{no˧=ɻ̩˩}/. However, the proximal \is{demonstratives}demonstrative (also serving as 3\smash{\textsuperscript{rd}}-person pronoun)
/\ipa{ʈʂʰɯ˥}/ and the distal \is{demonstratives}demonstrative /\ipa{tʰv̩˥}/ yield /\ipa{ʈʂʰɯ˧=ɻ̩˩}/ and
/\ipa{tʰv̩˧=ɻ̩˩}/ with the associative, whereas they yield /\ipa{ʈʂʰɯ˧=ɻæ˥\$}/ and /\ipa{tʰv̩=ɻæ˥\$}/ with the plural. This difference in tone patterns is enough to establish that the plural and associative plural do not always share identical tonal patterns, despite their identical behaviour in almost all cases. 

%It must be kept in mind that, as pointed out at the outset of this section (\sectref{sec:enclitics}), the four broad tonal categories set up here (morphemes with L tone, M tone, MH tone and H tone) are based on one test only: their behaviour after M-tone nouns. These four categories are not homogeneous, and only serve as convenient headings for setting out the data. %%% Cluttery-repetitive paragraph --> commented out in 2025.

\begin{table}[t]
\caption{\label{tab:thetonalbehaviourofassociativeplural}The tonal behaviour of \isi{associative plural} /\ipa{=ɻ̩}{\kern2pt}/.}
\begin{tabularx}{\textwidth}{ l Q l Q l }
\lsptoprule
	example & meaning & tone & with /\ipa{=ɻ̩}{\kern2pt}/ & output\\ \midrule
	– & – & LM & – & –\\
	– & – & LH & – & –\\
	– & – & M & – & –\\
	\ipa{njɤ˩}, \ipa{no˩} & \textsc{1sg}, \textsc{2sg} & L & \ipa{njɤ˧=ɻ̩˩}, \ipa{no=ɻ̩˩}  & L\#\\
	\ipa{ʈʂʰɯ˥}, \ipa{tʰv̩˥} & \textsc{dem}.\textsc{prox}, \textsc{dist} & H & \ipa{ʈʂʰɯ˧=ɻ̩˩}, \ipa{tʰv̩˧=ɻ̩˩} & L\#\\
	– & – & MH & – & –\\ \addlinespace \hdashline \addlinespace
	\ipa{dze˧bo˧} & family name & M & \ipa{dze˧bo˧=ɻ̩˩} & L\#\\
	– & – & \#H & – & –\\
	– & – & MH\# & – & –\\
	\ipa{kv̩˧tsʰɑ˥\$} & family name & H\$ & \ipa{kv̩˧tsʰɑ˧=ɻ̩˥\$} & H\$\\
	\ipa{lɑ˩mɑ˩} & family name & L & \ipa{lɑ˩mɑ˩-ɻ̩˥\$} & L+H\$\\
	\ipa{ə˧ɕjo˩} & family name & L\# & \ipa{ə˧ɕjo˩=ɻ̩˩} & L\#--\\
	– & – & LM+MH\# & – & –\\
	– & – & LM+\#H & – & –\\
	– & – & LM & – & –\\
	– & – & LH & – & –\\
	– & – & H\# & – & –\\
	\ipa{ɖʐɤ˩kɤ˥\$} & family name & LM+H\$ & \ipa{ɖʐɤ˩kɤ˧-ɻ̩˥\$} & LM+H\$\\
\lspbottomrule
\end{tabularx}
\end{table}

The \is{associative plural}associative /\ipa{=ɻ̩˩}/ is presented separately in \tabref{tab:thetonalbehaviourofassociativeplural}. Its low frequency accounts for the numerous gaps in the table.


The Alawua dialect of Yongning Na also has a~dual morpheme /\ipa{=zɯ˩}/ appearing in four pronouns: first-person dual exclusive /\ipa{njæ˧=zɯ˩}/, first-person dual inclusive /\ipa{ə˧=zɯ˩}/, second-person dual /\ipa{no˧=zɯ˩}/, and third-person dual /\ipa{ʈʂʰɯ˧{\allowbreak}=zɯ˩}/. On the basis of these forms, the tone of the dual can be classified as belonging to the same broad tonal category as the plural and associative, namely that of L-tone morphemes~-- keeping in mind that this tonal category serves as a~first\babelhyphen{nobreak}pass label.

\largerpage
The surface phonological patterns for the \is{postpositions}postposition /\ipa{bi˩}/ ‘on; at', shown in \tabref{tab:postpositionon}, are identical in every case to those for the plural \is{clitics}clitic /\ipa{=ɻæ˩}/. The underlying forms for the \is{postpositions}postposition are more difficult to arrive at, for want of a~handy test such as addition of the \isi{possessive} /\ipa{=bv̩˧}/, which works for noun phrases containing the plural \is{clitics}clitic /\ipa{=ɻæ˩}/ but not for locative phrases containing the \is{postpositions}postposition /\ipa{bi˩}/ ‘on; at'. Given the full identity of surface phonological patterns between the \is{clitics}clitic and the \is{postpositions}postposition, it is tempting to extrapolate from the surface phonological forms in \tabref{tab:postpositionon} to the (hypothetical) underlying forms proposed in \tabref{tab:postpositiononsupplemented}. These underlying forms are based on those obtained for the plural \is{clitics}clitic /\ipa{=ɻæ˩}/ (\tabref{tab:thetonalbehaviourofplural}). 



\begin{table}%[t]
	\caption{\label{tab:postpositiononsupplemented}The behaviour of the L-tone postposition //\ipa{bi˩}// ‘on; at' as interpreted on the {analogy} of the {plural} clitic. There is an additional ‘L \textsc{pro}’ row because L-tone pronouns have exceptional behaviour.}
	\begin{tabularx}{\textwidth}{ Q Q P{30mm} l }
		\lsptoprule
		example & tone & with /\ipa{bi˩}/ & underlying tone\\ \midrule
		pig & LM & \ipa{bo˩ bi˧} & LM\\
		leopard & LH & \ipa{ʐæ˩ bi˥} & LH\\
		tiger & M & \ipa{lɑ˧ bi˩} & L\#\\
		sheep & L & \ipa{jo˩ bi˩˥} & L\\
		\textsc{2sg} & L \textsc{pro} & \ipa{no˧ bi˩} & L\#\\
		horse & H & \ipa{ʐwæ˧ bi˥} & H\$ / \#H\\
		deer & MH & \ipa{ʈʂʰæ˧ bi˥} & H\#\\ \addlinespace \hdashline \addlinespace
		fox & M & \ipa{ɖɤ˧mi˧ bi˩} & --L\\
		colt & \#H & \ipa{ʐwæ˧zo˧ bi˥} & H\$\\
		cat & MH\# & \ipa{hwɤ˧li˧ bi˥} & H\$\\
		she\babelhyphen{nobreak}cat & H\$ & \ipa{hwɤ˧mi˧ bi˥} & H\$\\
		dog & L & \ipa{kʰv̩˩mi˩ bi˥} & L+H\$\\
		mule & L\# & \ipa{dɑ˧ʝi˩ bi˩} & L\#--\\
		wolf & LM+MH\# & \ipa{õ˩dv̩˧ bi˥} & LM+H\$ \\
		Naxi & LM+\#H & \ipa{nɑ˩hĩ˧ bi˥} & LM+H\$\\
		sow & LM & \ipa{bo˩mi˧ bi˩} & LM--L\\
		boar & LH & \ipa{bo˩ɬɑ˥ bi˩} & LH--\\
		rat & H\# & \ipa{hwæ˧tsɯ˥ bi˩} & H\#--\\
		\lspbottomrule
	\end{tabularx}
\end{table}


\subsection[M-tone morphemes]{M-tone morphemes: {Agentive}, {dative} and {topic}}
\label{sec:encliticsthatcarrymtonewhenfollowingamtonenoun}

This section presents three morphemes that carry M tone when following an M-tone noun and are therefore analyzed as having an M lexical tone. 

Tables~\ref{tab:topicfull} and \ref{tab:topicabstract} present the tonal behaviour of the agentive /\ipa{ɳɯ˧}/, the dative
/\ipa{-ki˧}/ (whose tonal behaviour is identical with that of the \isi{possessive}, /\ipa{=bv̩˧}/), and the topic
marker /\ipa{ʈʂʰɯ˧}/. This last morpheme can be hypothesized to be an extension of the {third}-person singular \is{pronouns}pronoun /\ipa{ʈʂʰɯ˥}/, which also serves as a~proximal demonstrative, but in view of its tonal behaviour, the topic marker is considered to have M tone, as opposed to H tone for the third-person singular \is{pronouns}pronoun /\ipa{ʈʂʰɯ˥}/. Table~\ref{tab:topicfull} presents examples, and 
Table~\ref{tab:topicabstract} the underlying tone patterns. 




% \label{tab:topic}  %% Commented out on April 30th, 2025: no subtables, only sequentially numbered tables in the entire volume.

\begin{table}[p]
\caption{\label{tab:topicfull}Tone patterns of {agentive} /\ipa{ɳɯ˧}/, {dative} /\ipa{-ki˧}/, and {topic} marker
    /\ipa{ʈʂʰɯ˧}/ with {monosyllabic} and disyllabic nouns: examples in full.}
{\setlength\tabcolsep{5pt}%\renewcommand{\arraystretch}{1.25}
  \begin{tabularx}{\textwidth}{ l l Q P{24mm} P{30mm} }
\lsptoprule
	example & tone & /\ipa{ɳɯ˧}/ & /\ipa{-ki˧}/ & /\ipa{ʈʂʰɯ˧}/\\ \midrule
	pig & LM & \ipa{bo˩ ɳɯ˧} & \ipa{bo˩-ki˧} & \ipa{bo˩ ʈʂʰɯ˧}\\
	leopard & LH & \ipa{ʐæ˩ ɳɯ˥ } & \ipa{ʐæ˩-ki˥} & \ipa{ʐæ˩ ʈʂʰɯ˥}\\
	tiger & M & \ipa{lɑ˧ ɳɯ˧} & \ipa{lɑ˧-ki˧} & \ipa{lɑ˧ ʈʂʰɯ˧}\\
	sheep & L & \ipa{jo˧ ɳɯ˧ / jo˩~ɳɯ˥} & \ipa{jo˧-ki˧ / \mbox{jo˩-ki˥} / \mbox{jo˩-ki˩}\footnote{The third variant was not reported in the first edition of this book.}} & \ipa{jo˧ ʈʂʰɯ˧ / jo˩~ʈʂʰɯ˥}\\
	horse & H & \ipa{ʐwæ˧ ɳɯ˩} & \ipa{ʐwæ˧-ki˧} & \ipa{ʐwæ˧ ʈʂʰɯ˧}\\
	deer & MH & \ipa{ʈʂʰæ˧ ɳɯ˥} & \ipa{ʈʂʰæ˧-ki˥} & \ipa{ʈʂʰæ˧ ʈʂʰɯ˧ / ʈʂʰæ˧~ʈʂʰɯ˥}\\ \addlinespace \hdashline \addlinespace
	fox & M & \ipa{ɖɤ˧mi˧ ɳɯ˧} & \ipa{ɖɤ˧mi˧-ki˧} & \ipa{ɖɤ˧mi˧ ʈʂʰɯ˧}\\
	colt & \#H & \ipa{ʐwæ˧zo˧ ɳɯ˩} & \ipa{ʐwæ˧zo˧-ki˧} & \ipa{ʐwæ˧zo˧ ʈʂʰɯ˧}\\
	cat & MH\# & \ipa{hwɤ˧li˧ ɳɯ˥} & \ipa{hwɤ˧li˧-ki˥} & \ipa{hwɤ˧li˧ ʈʂʰɯ˥}\\
	she\babelhyphen{nobreak}cat & H\$ & \ipa{hwɤ˧mi˥ ɳɯ˩ / hwɤ˧mi˧ ɳɯ˥} & \ipa{hwɤ˧mi˥-ki˩} & \ipa{hwɤ˧mi˧ ʈʂʰɯ˧ / hwɤ˧mi˧ ʈʂʰɯ˥}\\
	dog & L & \ipa{kʰv̩˩mi˩ ɳɯ˥} & \ipa{kʰv̩˩mi˩-ki˥} & \ipa{kʰv̩˩mi˩ ʈʂʰɯ˥}\\
	mule & L\# & \ipa{dɑ˧ʝi˩ ɳɯ˩} & \ipa{dɑ˧ʝi˩-ki˩} & \ipa{dɑ˧ʝi˩ ʈʂʰɯ˩}\\
	wolf & LM+MH\# & \ipa{õ˩dv̩˧ ɳɯ˥} & \ipa{õ˩dv̩˧-ki˥} & \ipa{õ˩dv̩˧ ʈʂʰɯ˧ / õ˩dv̩˧ ʈʂʰɯ˥}\\
	Naxi & LM+\#H & \ipa{nɑ˩hĩ˧ ɳɯ˩} & \ipa{nɑ˩hĩ˧-ki˧} & \ipa{nɑ˩hĩ˧ ʈʂʰɯ˧}\\
	sow & LM & \ipa{bo˩mi˧ ɳɯ˧} & \ipa{bo˩mi˧-ki˧} & \ipa{bo˩mi˧ ʈʂʰɯ˧}\\
	boar & LH & \ipa{bo˩ɬɑ˥ ɳɯ˩} & \ipa{bo˩ɬɑ˥-ki˩} & \ipa{bo˩ɬɑ˥ ʈʂʰɯ˩}\\
	rat & H\# & \ipa{hwæ˧tsɯ˥ ɳɯ˩} & \ipa{hwæ˧tsɯ˥-ki˩} & \ipa{hwæ˧tsɯ˥ ʈʂʰɯ˩}\\
	\lspbottomrule
\end{tabularx}}
\end{table}

\clearpage

\begin{table}%[t]
\caption{\label{tab:topicabstract}Tone patterns of {agentive} /\ipa{ɳɯ˧}/, {dative} /\ipa{-ki˧}/, and {topic} marker
	/\ipa{ʈʂʰɯ˧}/ with {monosyllabic} and disyllabic nouns.}
  \begin{tabularx}{\textwidth}{ Q Q Q Q }
\lsptoprule
	tone of noun & /\ipa{ɳɯ}/ & /\ipa{-ki}/ & /\ipa{ʈʂʰɯ}/\\ \midrule
	LM & L.M & L.M & L.M\\
	LH & L.H & L.H & L.H\\
	M & M.M & M.M & M.M\\
	L & M.M / L.H & M.M / L.H / L.L\footnote{The third variant was not reported in the first edition of this book.} & M.M / L.H\\
	H & M.L & M.M & M.M\\
	MH & M.H & M.H & M.H\\ \addlinespace \hdashline \addlinespace
	M & M.M.M & M.M.M & M.M.M\\
	\#H & M.M.L & M.M.M & M.M.M\\
	MH\# & M.M.H & M.M.H & M.M.H\\
	H\$ & M.H.L / M.M.H & M.M.M & M.M.M\\
	L & L.L.H & L.L.H & L.L.H\\
	L\# & M.L.L & M.L.L & M.L.L\\
	LM+MH\# & L.M.H & L.M.H & L.M.M / L.M.H\\
	LM+\#H & L.M.L & L.M.M & L.M.M\\
	LM & L.M.M & L.M.M & L.M.M\\
	LH & L.H.L & L.H.L & L.H.L\\
	H\# & M.H.L & M.H.L & M.H.L\\
\lspbottomrule
\end{tabularx}
\end{table}


Note that the tones of /\ipa{bo˩ ɳɯ˧}/ and
/\ipa{ʐæ˩ ɳɯ˥}/ are neutralized at the
surface phonological level due to Rule~6 (“In tone\babelhyphen{nobreak}group\babelhyphen{nobreak}final position, H and M are neutralized to H if they follow an L tone”: see \sectref{sec:alistoftonerules}). As elsewhere, variants are separated by a~slash. The difference in tone patterns between these morphemes when associated with `horse' was carefully verified: the tones are M.L in /\ipa{ʐwæ˧ ɳɯ˩}/, as opposed to M.M in /\ipa{ʐwæ˧-ki˧}/ and /\ipa{ʐwæ˧ ʈʂʰɯ˧}/.	


An exceptional pattern is observed for /\ipa{di˩˥}/ ‘earth’: in addition to the expected
/\ipa{di˩ ɳɯ˥}/, attested in (\ref{ex:reward145}), the form /\ipa{di˧ ɳɯ˧}/ is also acceptable (see \ref{ex:reward121}). This \is{variants}variant is not acceptable for other LH\babelhyphen{nobreak}tone nouns; for instance, it is
not possible to say \ipa{$\ddagger${\kern2pt}ʐæ˧ ɳɯ˧} for ‘by \mbox{(a/the)} leopard’.

\newpage
\begin{exe}
	\ex
	\label{ex:reward145}
	\ipaex{di˩ ɳɯ˥ {\kern2pt}|{\kern2pt} ə˧-sɯ˩kv̩˩ li˩-dʑo˩-ɲi˩!}\\
	\gll di˩˥	ɳɯ˧		ə˧-sɯ˩kv̩˩	li˧\textsubscript{a}		-dʑo˧		-ɲi˩\\
	earth		\textsc{a}	\textsc{1pl.incl}	to\_watch		\textsc{prog}			\textsc{certitude}\\
	\glt ‘The Earth is watching us!’ \textit{(Reward.145)} \pandoi{0004446\#S145}
\end{exe}

\begin{exe}
	\ex
	\label{ex:reward121}
	\ipaex{“mv̩˧ ɳɯ˩ {\kern2pt}|{\kern2pt} ki˧! {\kern2pt}|{\kern2pt} di˧ ɳɯ˧ {\kern2pt}|{\kern2pt} ki˧!” {\kern2pt}|{\kern2pt} pi˧-ɲi˥ tsɯ˩ {\kern2pt}|{\kern2pt} mv̩˩.}\\
	\gll mv̩˥	ɳɯ˧		ki˧\textsubscript{a}		di˩˥	ɳɯ˧		ki˧\textsubscript{a}	pi˥		-ɲi˩		tsɯ˧˥	 mv̩˧\\
	sky		\textsc{a}	to\_give					earth	\textsc{a}	to\_give			to\_say			\textsc{certitude}	\textsc{rep}	\textsc{affirm}\\
	\glt ‘“It is a~gift of the Heavens! It is a~gift of the Earth!” he said.’ \textit{(Reward.121)} \pandoi{0004446\#S121}
\end{exe}


Where two variants are possible, \is{stylistics}stylistic nuances can sometimes be pinpointed. One way to investigate such nuances is to combine information about the consultant’s general preference with a detailed examination of examples in context.


Consultant preference is elicited without providing an indication of context. The investigator says one of the two alternatives while raising his right
hand, then the second while raising his left hand. The consultant indicates whether both are acceptable and expresses a~preference, often using an understatement: /\ipa{ʈʂʰɯ˧ bæ˧, {\kern2pt}|{\kern2pt} ɖɯ˧-pi˧˥ {\kern2pt}|{\kern2pt} ho˩˥}/, “This one is \textit{pretty correct}”, meaning “This one is better”.

For instance, ‘to the father’ allows the two variants /\ipa{ə˧dɑ˧-ki˧}/ and /\ipa{ə˧dɑ˥-ki˩}/. Outside context, the former is considered better than the latter. A closer examination of the choice between /\ipa{ə˧dɑ˧-ki˧}/ and /\ipa{ə˧dɑ˥-ki˩}/ in discourse sheds light on subtle links between tone and pragmatic or stylistic effects. In example (\ref{ex:tothefather}), ‘to the father’ was initially transcribed as /\ipa{ə˧dɑ˥-ki˩}/ but was later corrected to /\ipa{ə˧dɑ˧-ki˧}/. 




\begin{exe}
	\ex
	\label{ex:tothefather}
	\ipaex{ə˧dɑ˧-ki˧, {\kern2pt}|{\kern2pt} ɬo˧pv̩˥ ti˩-bi˩-kv̩˩-ze˩ mæ˩!}\\
	\gll ə˧dɑ˥\$    -ki˧   ɬo˧pv̩˥     ti˩\textsubscript{a}    -bi˧    -kv̩˧˥  -ze˧\textsubscript{b}   mæ˧\\
	father		\textsc{dat}	kowtow		to\_hit	  \textsc{imm\_fut}   \textsc{abilitive}      \textsc{pfv}   \textsc{disc.ptcl\_obviousness} \\
	\glt ‘[The young girl] kowtows before her father, of course [just as she did before the members of the household in which she grew up: her mother's household].’ \textit{(ComingOfAge2.18)} \pandoi{0004589\#S18}
\end{exe}

The context of (\ref{ex:tothefather}) is a discussion of the respective roles of the maternal uncle and the father. 

\begin{quotation}
    During the rite of passage to adulthood, the uncle presides over the ceremony. The father, well, we know who he is, but that doesn't really matter: it's the uncle who counts in the family, he’s the one who is treated with the most consideration. When you turn thirteen, the ceremony takes place at home. The uncle is the one who takes care of everything, who keeps the whole household going. \textit{(ComingOfAge2.4-6)} \pandoi{0004589\#S4}
\end{quotation}

In (\ref{ex:tothefather}), the father is mentioned in a somewhat parenthetical manner: on the all-important occasion of a child's coming-of-age ceremony, he is not forgotten, but he is not among the central figures. The visit to the father's family is secondary to the main part of the ceremony, which takes place after all due rites have been completed at home~-- that is, at the mother's home. The choice of /\ipa{ə˧dɑ˧-ki˧}/ in this relatively dispassionate context highlights its linguistic nature as a form exhibiting a high degree of morphosyntactic integration, befitting carefully planned and structured speech. In /\ipa{ə˧dɑ˧-ki˧}/, ‘father’ (//\ipa{ə˧dɑ˥\$}//) is tonally amalgamated with its \is{suffixes}suffix, forming a~neatly packaged unit that can, in turn, be woven into the mesh of a larger syntactic structure. The M tone of the phrase allows for a range of diverse outcomes in \is{morphotonology}morphotonological combinations. 


By contrast, the form /\ipa{ə˧dɑ˥-ki˩}/ places subtle \isi{emphasis} on the noun. Here, ‘father’ surfaces with the same tone it would have \is{form!in isolation}in isolation: /\ipa{ə˧dɑ˥}/ (lexical form: //\ipa{ə˧dɑ˥\$}//). This makes the noun stand out, as if extracting it from the flow of speech. Moreover, from a \is{phonostylistics}phonostylistic point of view, the higher pitch on the syllable meaning ‘father’ (H tone, as opposed to M tone in /\ipa{ə˧dɑ˧-ki˧}/) has \is{iconicity}iconic value: cross\babelhyphen{nobreak}linguistically, raised pitch tends to be associated with \isi{emphasis} in both tonal and non-tonal languages (see \sectref{sec:pragmaticintonation}). 

The High tone has \is{culminativity}culminative value in Yongning Na phonology: it can only be followed by L tones (by rules 4 and 5). This phonological constraint means that /\ipa{ə˧dɑ˥-ki˩}/ precludes any following tones other than L within the tone group. While this restriction is purely phonological in origin, its effect contributes to the \is{phonostylistics}phonostylistic \isi{prominence} of the H-tone syllable, in this case, the main syllable of the word `father'. For these reasons, the variant with a M.H.L tone pattern (/\ipa{ə˧dɑ˥-ki˩}/) is more suited to emotional, emphatic speech and is thus less appropriate in the context of (\ref{ex:tothefather}), which constitutes a low-key, parenthetical statement.  

These reflections are coherent with the speaker's general preference for /\ipa{ə˧dɑ˧-ki˧}/ over /\ipa{ə˧dɑ˥-ki˩}/, explaining why the former is perceived as the “better” form when assessed in isolation, outside a specific discourse context. 

These observations illustrate that phonological \isi{variation} %is not merely a matter of free alternation but 
can reflect subtle distinctions in communicative intent and discourse structure. Tonal variation in Na often serves pragmatic and discourse-related functions. Across the corpus, speakers' phonological choices among possible variants, as well as fine details in the phonetic realization of tones, allow them to convey nuanced effects: to highlight a referent, reinforce contrast, or mark a shift in discourse focus~-- or conversely, to background certain pieces of information. Such cases exemplify, in a language-specific way, broader cross-linguistic patterns in the use of \isi{intonation} for \isi{prominence} (information structuring) and \isi{phrasing} (see \sectref{sec:pragmaticintonation} for further discussion).

%\newpage 
In association with the {agentive} /\ipa{ɳɯ˧}/, the {dative} /\ipa{-ki˧}/, and the {topic} marker
/\ipa{ʈʂʰɯ˧}/, the 1\textsuperscript{st}- and 2\textsuperscript{nd}-person pronouns behave like other L-tone items:
/\ipa{njɤ˧ ɳɯ˧}/, /\ipa{no˧ ɳɯ˧}/. On the other hand, the demonstratives /\ipa{ʈʂʰɯ˥}/ (proximal, also 3\textsuperscript{rd}-person singular) and
/\ipa{tʰv̩˥}/ (distal) pattern differently from other H-tone items. They yield M.M in
association with /\ipa{ɳɯ˧}/: /\ipa{ʈʂʰɯ˧ ɳɯ˧}/ and /\ipa{tʰv̩˧ ɳɯ˧}/ (also /\ipa{ʈʂʰɯ˧ lɑ˧}/,
/\ipa{tʰv̩˧ lɑ˧}/ ‘this/that one too’). 

\subsection[MH-tone morphemes]{The MH-tone morphemes /\ipa{-qɑ˧˥}/ (dative/comitative) and /\ipa{gi˧˥}/ ‘behind’}
\label{sec:mhtoneenclitics}


The morpheme /\ipa{-qɑ˧˥}/ has both dative and comitative functions. It is analyzed as carrying MH tone based on its behaviour after M-tone words, whether {monosyllabic} or disyllabic, as well as after LM\babelhyphen{nobreak}tone
disyllables. \tabref{tab:thetonalbehaviourofthedativecomitativemarkerfollowingnouns} sets out the data. 
%(No recording was conducted.) 
The \is{postpositions}postposition /\ipa{gi˧˥}/
‘behind’ follows an identical pattern.\footnote{By convention, clitics are preceded by an \textit{equals} sign, suffixes are preceded by a~hyphen, prefixes are followed by a hyphen, and postpositions are written as free morphemes.}

\begin{table}%[t]
\caption{\label{tab:thetonalbehaviourofthedativecomitativemarkerfollowingnouns}The tonal behaviour of the dative/comitative marker /\ipa{-qɑ˧˥}/ following nouns.}
%\begin{tabularx}{\textwidth}{ Q Q P{27mm} l }
\begin{tabularx}{\textwidth}{ Q Q P{23.5mm} l }
\lsptoprule
	example & tone & example & tone pattern\\ \midrule
	pig & LM & \ipa{bo˩-qɑ˧} & L.M (neutralized with LH on surface)\\
	leopard & LH & \ipa{ʐæ˩-qɑ˥} & L.H (neutralized with LH on surface)\\
	tiger & M & \ipa{lɑ˧-qɑ˧˥} & M.MH\\
	sheep & L & \ipa{jo˩-qɑ˩˥} & L.LH\\
	horse & H & \ipa{ʐwæ˧-qɑ˩} & M.L\\
	deer & MH & \ipa{ʈʂʰæ˧-qɑ˥} & M.H\\ \addlinespace \hdashline \addlinespace
	fox & M & \ipa{ɖɤ˧mi˧-qɑ˧˥} & M.M.MH\\
	colt & \#H & \ipa{ʐwæ˧zo˧-qɑ˩} & M.M.L\\
	cat & MH\# & \ipa{hwɤ˧li˧-qɑ˥} & M.M.H\\
	she\babelhyphen{nobreak}cat & H\$ & \ipa{hwɤ˧mi˥-qɑ˩} & M.H.L\\
	dog & L & \ipa{kʰv̩˩mi˩-qɑ˥} & L.L.H\\
	mule & L\# & \ipa{dɑ˧ʝi˩-qɑ˩} & M.L.L\\
	wolf & LM+MH\# & \ipa{õ˩dv̩˧-qɑ˥} & L.M.H\\
	Naxi & LM+\#H & \ipa{nɑ˩hĩ˧-qɑ˩} & L.M.L\\
	sow & LM & \ipa{bo˩mi˧-qɑ˧˥} & L.M.MH\\
	boar & LH & \ipa{bo˩ɬɑ˥-qɑ˩} & L.H.L\\
	rat & H\# & \ipa{hwæ˧tsɯ˥-qɑ˩} & M.H.L\\
\lspbottomrule
\end{tabularx}
\end{table}

As in all other morphosyntactic contexts, the //L.H.L// pattern is neutralized with //L.M.L// at
the surface phonological level. Thus, the tone pattern for ‘boar’ could be transcribed as /L.M.L/ in surface phonological representation. However, notation as //L.H.L// is retained here to reflect the analysis that the en\is{clitics}clitic's tone is lowered to L due to the presence of a~preceding H tone~-- specifically, the H component of the LH tone pattern lexically attached to the noun ‘boar’. 

An alternative notation as /L.M.L/ might seem preferable for remaining closer to surface phonological form and limiting abstraction. However, this could create confusion by suggesting that the lexical category \mbox{//LH//} is restructured as \mbox{//LM//}
when the en\is{clitics}clitic is added. To avoid this misleading implication, notation as L.H.L is adopted in \tabref{tab:thetonalbehaviourofthedativecomitativemarkerfollowingnouns}.


\subsection[A~H-tone morpheme]{The H-tone topic marker /\ipa{-dʑo˥}/, with observations about tonal contours in non-final position}
\label{sec:theambivalentbehaviouroftheltonetopicmarker}

The topic marker /\ipa{-dʑo˥}/ can appear after both nouns and verbs. The tone patterns it yields when associated with a~noun are shown in \tabref{tab:thetonalbehaviourofthetopicmarkerfollowingnouns}. Interestingly, an MH\babelhyphen{nobreak}tone noun or verb preceding the topic marker is realized with an MH \is{tonal contour}contour, e.g.~/\ipa{ʈʂʰæ˧˥-dʑo˩}/
‘as for the deer’ (from /\ipa{ʈʂʰæ˧˥}/ ‘deer’) and /\ipa{mɤ˧-lɑ˧˥-dʑo˩}/ ‘as [he/she] did not strike’
(from /\ipa{lɑ˧˥}/ ‘to strike’). This suggests that there is a~\isi{tone group} \is{boundary (between tone groups)}boundary following the noun or
verb, since tonal contours are realized only in tone\babelhyphen{nobreak}group\babelhyphen{nobreak}final position. 

Such behaviour would not be
unparalleled. For instance, the contrastive topic marker /\ipa{-no˧˥}/ and the word /\ipa{tʰi˩˥}/ ‘then’ systematically
mark the beginning of a~new \isi{tone group}. But the topic marker's tonal behaviour when following a noun or verb with a~tone other than MH suggests that, in these cases, it \textit{is} integrated within the same tone
group. For instance, after an M-tone noun or verb, the pattern is (M{\dots})M.H, as in /\ipa{lɑ˧-dʑo˥}/ ‘as for
the tiger’ and /\ipa{mɤ˧-hwæ˧-dʑo˥}/ ‘as [she/he] does not buy’. These expressions clearly form a~single \isi{tone group}. The full dataset is presented in \tabref{tab:thetonalbehaviourofthetopicmarkerfollowingnouns}.\footnote{The data was verified by using additional nouns
  illustrating the tonal categories: /\ipa{zo˥}/ ‘son’ for H tone, /\ipa{õ˧˥}/ ‘oneself’ for MH tone,
  /\ipa{pʰɤ˧bɤ˧}/ ‘gift’ for M tone, and /\ipa{ə˧dɑ˥\$}/ ‘father’ and /\ipa{mv̩˧ʁo˥\$}/ ‘heavens’ for H\$ tone. The results were consistent, including the occurrence of variant patterns: for ‘gift’, two variants were observed,
  /\ipa{pʰɤ˧bɤ˧-dʑo˧}/ and /\ipa{pʰɤ˧bɤ˧-dʑo˥}/, mirroring the variation found with ‘fox’ in \tabref{tab:thetonalbehaviourofthetopicmarkerfollowingnouns}. The only
  unexpected result was with ‘plain’, //\ipa{di˧qo˧}//. On the {analogy} of /\ipa{ɖɤ˧mi˧-dʑo˥}/, one would expect $\dagger${\kern2pt}\ipa{di˧qo˧-dʑo˥}, yet the observed pattern was M.M.M: /\ipa{di˧qo˧-dʑo˧}/. This
  unexpected result, confirmed across elicitation sessions, may have to do with the internal
  structure of this disyllable, which literally means ‘on earth’.} 
  
  These observations are taken up
  in~\sectref{sec:casesofbreachoftonalgroupingandconsequencesforthesystem}, as part of the discussion on breaches of
tonal grouping: how non\babelhyphen{nobreak}final syllables can come to bear a~\is{tonal contour}contour while subsequent syllables become
extrametrical. 
%\clearpage


\begin{table}[t]
\caption{\label{tab:thetonalbehaviourofthetopicmarkerfollowingnouns}The tonal behaviour of the topic marker /\ipa{-dʑo˥}/ following nouns.}
\begin{tabularx}{\textwidth}{ Q Q l l }
\lsptoprule
	example & tone & example & tone pattern\\ \midrule
	pig & LM & \ipa{bo˩˧-dʑo˩} & LM.L\\
	leopard & LH & \ipa{ʐæ˩˥-dʑo˩} & LH.L\\
	tiger & M & \ipa{lɑ˧-dʑo˥} & M.H\\
	daughter & L & \ipa{mv̩˩-dʑo˩˥} & LH.L\\
	horse & H & \ipa{ʐwæ˧-dʑo˩} & M.L\\
	deer & MH & \ipa{ʈʂʰæ˧˥-dʑo˩} & MH.L\\ \addlinespace \hdashline \addlinespace
	fox & M & \ipa{ɖɤ˧mi˧-dʑo˥ / ɖɤ˧mi˧-dʑo˧} & M.M.H / M.M.M\\
	colt & \#H & \ipa{ʐwæ˧zo˧-dʑo˩} & M.M.L\\
	cat & MH\# & \ipa{hwɤ˧li˧˥-dʑo˩} & M.MH.L\\
	she\babelhyphen{nobreak}cat & H\$ & \ipa{hwɤ˧mi˥-dʑo˩} & M.H.L\\
	dog & L & \ipa{kʰv̩˩mi˩˥-dʑo˩} & L.LH.L\\
	mule & L\# & \ipa{dɑ˧ʝi˩-dʑo˩} & M.L.L\\
	wolf & LM+MH\# & \ipa{õ˩dv̩˧˥-dʑo˩} & L.MH.L\\
	Naxi & LM+\#H & \ipa{nɑ˩hĩ˧-dʑo˩} & L.M.L\\
	sow & LM & \ipa{bo˩mi˧-dʑo˥} & L.M.H\\
	boar & LH & \ipa{bo˩ɬɑ˧-dʑo˩} & L.M.L\\
	rat & H\# & \ipa{hwæ˧tsɯ˥-dʑo˩} & M.H.L\\
\lspbottomrule
\end{tabularx}
\end{table}

On the basis of its behaviour after M-tone nouns (and after M-tone verbs, which will be presented in the next chapter), the topic marker is provisionally analyzed as carrying a~lexical H tone. If this tonal identification is confirmed, the morpheme constitutes an extreme case of distance between underlying form and surface forms: in texts, realizations with an L tone outnumber those with an H tone by a~ratio of approximately 10 to 1. (Even more striking cases include the \isi{reported speech} particle /\ipa{tsɯ˧˥}/ and the affirmation particle /\ipa{mv̩˧}/, whose underlying tones hardly ever surface as such: see \ref{sec:implicationsforthetonesofsentenceparticles}.)


\section{Disyllabic postpositions}
\label{sec:spatialpostpositions}

\is{disyllables}Disyllabic locative postpositions include /\ipa{ʁo˧tʰo˩}/ ‘behind’,
/\ipa{ʁo˧dɑ˧}/ ‘in front of’, /\ipa{ɬo˧tɑ˧}/ ‘beside, to the side of’, /\ipa{ʁwæ˧gi\#˥}/ ‘to the
left’, /\ipa{jo˩gi˩}/ ‘to the right’, and /\ipa{-qo˧lo˩}/ ‘inside’. (The {monosyllabic} form /\ipa{-qo˧}/ is also attested, with the same meaning.) \tabref{tab:thetonalbehaviourofthespatialpostpositionsbesidetothesideofandbehind} presents the data for nouns followed by the locative postpositions /\ipa{ɬo˧tɑ˧}/ ‘beside, to
the side of’ and /\ipa{ʁo˧tʰo˩}/ ‘behind, to the back of’. The tonal behaviour of /\ipa{ʁo˧dɑ˧}/ ‘in front of’ is identical with that of /\ipa{ɬo˧tɑ˧}/ ‘beside, to the side of’ and is therefore omitted from the table. Similarly, /\ipa{ʈʰæ˧qo˩}/ ‘under’ behaves identically to /\ipa{ʁo˧tʰo˩}/ ‘behind’. \tabref{tab:thetonalbehaviourofthespatialpostpositionstotheleftandtotheright} provides the data for the locative postpositions /\ipa{ʁwæ˧gi\#˥}/ ‘to the left’ and
/\ipa{jo˩gi˩}/ ‘to the right’. The corresponding recording is: \textit{LocativePostp} \pandoi{0004563}.


\begin{table}[t]
\caption{\label{tab:thetonalbehaviourofthespatialpostpositionsbesidetothesideofandbehind}The tonal behaviour of the locative postpositions /\ipa{ɬo˧tɑ˧}/ ‘beside’ and /\ipa{ʁo˧tʰo˩}/ ‘behind’. There is an additional ‘L \textsc{pro}’ row because L-tone pronouns have exceptional behaviour.}
  \begin{tabularx}{\textwidth}{ l l l Q Q }
\lsptoprule
	tone & example & meaning & beside & behind\\ \midrule
	LM & \ipa{bo˩˧} & pig & \ipa{bo˩ ɬo˧tɑ˧} & \ipa{bo˩ ʁo˥tʰo˩}\\
	LH & \ipa{ʐæ˩˥} & leopard & \ipa{ʐæ˩ ɬo˧tɑ˧} & \ipa{ʐæ˩ ʁo˥tʰo˩}\\
	M & \ipa{lɑ˧} & tiger & \ipa{lɑ˧ ɬo˧tɑ˧} & \ipa{lɑ˧ ʁo˧tʰo˩}\\
	L & \ipa{jo˩} & sheep & \ipa{jo˩ ɬo˩tɑ˩} & \ipa{jo˩ ʁo˩tʰo˥}\\
	L \textsc{pro} & \ipa{no˩} & \textsc{2sg} & \ipa{no˧ ɬo˧tɑ˧ } & \ipa{no˧ ʁo˧tʰo˩}\\
	H & \ipa{ʐwæ˥} & horse & \ipa{ʐwæ˧ ɬo˧tɑ˥} & \ipa{ʐwæ˧ ʁo˧tʰo˥}\\
	MH & \ipa{ʈʂʰæ˧˥} & deer & \ipa{ʈʂʰæ˧ ɬo˧tɑ˥} & \ipa{ʈʂʰæ˧ ʁo˧tʰo˥}\\ \addlinespace \hdashline \addlinespace
	M & \ipa{ɖɤ˧mi˧} & fox & \ipa{ɖɤ˧mi˧ ɬo˧tɑ˧} & \ipa{ɖɤ˧mi˧ ʁo˧tʰo˩}\\
	\#H & \ipa{ʐwæ˧zo\#˥} & colt & \ipa{ʐwæ˧zo˧ ɬo˧tɑ˥} & \ipa{ʐwæ˧zo˧ ʁo˧tʰo˥}\\
	MH\# & \ipa{hwɤ˧li˧˥} & cat & \ipa{hwɤ˧li˧ ɬo˧tɑ˥} & \ipa{hwɤ˧li˧ ʁo˧tʰo˥}\\
	H\$ & \ipa{hwɤ˧mi˥\$} & she\babelhyphen{nobreak}cat & \ipa{hwɤ˧mi˧ ɬo˧tɑ˥} & \ipa{hwɤ˧mi˧ ʁo˧tʰo˥}\\
	L & \ipa{kʰv̩˩mi˩} & dog & \ipa{kʰv̩˩mi˩ ɬo˩tɑ˥} & \ipa{kʰv̩˩mi˩ ʁo˩tʰo˥}\\
	L\# & \ipa{dɑ˧ʝi˩} & mule & \ipa{dɑ˧ʝi˩ ɬo˩tɑ˩} & \ipa{dɑ˧ʝi˩ ʁo˩tʰo˩}\\
	LM+MH\# & \ipa{õ˩dv̩˧˥} & wolf & \ipa{õ˩dv̩˧ ɬo˧tɑ˥} & \ipa{õ˩dv̩˧ ʁo˧tʰo˥}\\
	LM+\#H & \ipa{nɑ˩hĩ\#˥} & Naxi & \ipa{nɑ˩hĩ˧ ɬo˧tɑ˥} & \ipa{nɑ˩hĩ˧ ʁo˧tʰo˥}\\
	LM & \ipa{bo˩mi˧} & sow & \ipa{bo˩mi˧ ɬo˧tɑ˧} & \ipa{bo˩mi˧ ʁo˧tʰo˩}\\
	LH & \ipa{bo˩ɬɑ˥} & boar & \ipa{bo˩ɬɑ˥ ɬo˩tɑ˩} & \ipa{bo˩ɬɑ˧ ʁo˩tʰo˩}\\
	H\# & \ipa{hwæ˧tsɯ˥} & rat & \ipa{hwæ˧tsɯ˥ ɬo˩tɑ˩} & \ipa{hwæ˧tsɯ˥ ʁo˩tʰo˩}\\
\lspbottomrule
\end{tabularx}
\end{table}


\begin{table}[t]
\caption{\label{tab:thetonalbehaviourofthespatialpostpositionstotheleftandtotheright}The tonal behaviour of the locative postpositions /\ipa{ʁwæ˧gi\#˥}/ ‘to the left’ and /\ipa{jo˩gi˩}/ ‘to the right’. There is an additional ‘L \textsc{pro}’ row because L-tone pronouns have exceptional behaviour.}
\begin{tabularx}{\textwidth}{ l l l l Q }
\lsptoprule
	tone & example & meaning & to the left of & to the right of\\ \midrule
	LM & \ipa{bo˩˧} & pig & \ipa{bo˩ ʁwæ˧gi\#˥} & \ipa{bo˩ jo˧gi\#˥}\\
	LH & \ipa{ʐæ˩˥} & leopard & \ipa{ʐæ˩ ʁwæ˧gi\#˥} & \ipa{ʐæ˩ jo˧gi\#˥}\\
	M & \ipa{lɑ˧} & tiger & \ipa{lɑ˧ ʁwæ˧gi\#˥} & \ipa{lɑ˧ jo˩gi˩}\\
	L & \ipa{jo˩} & sheep & \ipa{jo˩ ʁwæ˩gi˩} & \ipa{jo˧ jo˩gi˩}\\
	L \textsc{pro} & \ipa{no˩} & 2\textsc{sg} & \ipa{no˧ ʁwæ˧gi\#˥} & \ipa{no˧ jo˩gi˩}\\
	H & \ipa{ʐwæ˥} & horse & \ipa{ʐwæ˧ ʁwæ˧gi\#˥} & \ipa{ʐwæ˧ jo˥gi˩}\\
	MH & \ipa{ʈʂʰæ˧˥} & deer & \ipa{ʈʂʰæ˧ ʁwæ˧gi\#˥} & \ipa{ʈʂʰæ˧ jo˥gi˩}\\ \addlinespace \hdashline \addlinespace
	M & \ipa{ɖɤ˧mi˧} & fox & \ipa{ɖɤ˧mi˧ ʁwæ˧gi\#˥} & \ipa{ɖɤ˧mi˧ jo˩gi˩}\\
	\#H & \ipa{ʐwæ˧zo\#˥} & colt & \ipa{ʐwæ˧zo˧ ʁwæ˧gi\#˥} & \ipa{ʐwæ˧zo˧ jo˥gi˩}\\
	MH\# & \ipa{hwɤ˧li˧˥} & cat & \ipa{hwɤ˧li˧ ʁwæ˧gi\#˥} & \ipa{hwɤ˧li˧ jo˥gi˩}\\
	H\$ & \ipa{hwɤ˧mi˥\$} & she\babelhyphen{nobreak}cat & \ipa{hwɤ˧mi˧ ʁwæ˧gi\#˥} & \ipa{hwɤ˧mi˧ jo˥gi˩}\\
	L & \ipa{kʰv̩˩mi˩} & dog & \ipa{kʰv̩˩mi˩ ʁwæ˩gi˩} & \ipa{kʰv̩˩mi˩ jo˥gi˩}\\
	L\# & \ipa{dɑ˧ʝi˩} & mule & \ipa{dɑ˧ʝi˩ ʁwæ˩gi˩} & \ipa{dɑ˧ʝi˩ jo˩gi˩}\\
	LM+MH\# & \ipa{õ˩dv̩˧˥} & wolf & \ipa{õ˩dv̩˧ ʁwæ˧gi\#˥} & \ipa{õ˩dv̩˧ jo˥gi˩}\\
	LM+\#H & \ipa{nɑ˩hĩ\#˥} & Naxi & \ipa{nɑ˩hĩ˧ ʁwæ˧gi\#˥} & \ipa{nɑ˩hĩ˧ jo˥gi˩}\\
	LM & \ipa{bo˩mi˧} & sow & \ipa{bo˩mi˧ ʁwæ˧gi\#˥} & \ipa{bo˩mi˧ jo˩gi˩}\\
	LH & \ipa{bo˩ɬɑ˥} & boar & \ipa{bo˩ɬɑ˧ ʁwæ˩gi˩} & \ipa{bo˩ɬɑ˧ jo˩gi˩}\\
	H\# & \ipa{hwæ˧tsɯ˥} & rat & \ipa{hwæ˧tsɯ˥ ʁwæ˩gi˩} & \ipa{hwæ˧tsɯ˥ jo˩gi˩}\\
\lspbottomrule
\end{tabularx}
\end{table}


In both \tabref{tab:thetonalbehaviourofthespatialpostpositionsbesidetothesideofandbehind} and \tabref{tab:thetonalbehaviourofthespatialpostpositionstotheleftandtotheright}, two rows are provided for the L tone because L-tone pronouns, 1\textsc{sg} /\ipa{njɤ˩}/
and 2\textsc{sg} /\ipa{no˩}/, have exceptional tonal behaviour. The difference in tonal output between pronouns and nouns is clear. For instance, with the
M-tone \is{postpositions}postposition ‘beside’, an L-tone \is{pronouns}pronoun yields an M-tone pattern: /\ipa{no˧ ɬo˧tɑ˧}/, whereas \ipa{$\ddagger${\kern2pt}no˩ ɬo˩tɑ˩} is not possible. Conversely, an L-tone noun yields an L-tone pattern:
/\ipa{jo˩ ɬo˩tɑ˩}/, while it is not possible to say \ipa{$\ddagger${\kern2pt}jo˧ ɬo˧tɑ˧}.

The system would look nice and economical if the tonal behaviour of these postpositions were identical with those for
other constructions, such as determinative compounds. This is indeed the case for /\ipa{ʁo˧tʰo˩}/ ‘at the
back’, which patterns tonally in the same way as an L\#-tone head noun in determinative compounds, and for
/\ipa{ɬo˧tɑ˧}/ ‘beside, to the side of’, which behaves like an M-tone noun does in determinative compounds. However, not all locative
postpositions share this behaviour. Compare /\ipa{hwɤ˧li˧-hi˧kʰɯ˧˥}/ ‘cat’s gums (body
part)’ (input tones: MH\# and \#H; output tone: MH\#) and /\ipa{hwɤ˧li˧ ʁwæ˧gi\#˥}/ ‘to the left of
the cat’ (same input; output tone: \#H).

\clearpage
\section{Adverbs}
\label{sec:adverbs}


Despite their name, \textit{adverbs}~-- a loosely defined class of words~-- can occur not only with verbs, but also with nouns and other linguistic units. 

\subsection{The homophonous adverbs /\ipa{lɑ˧}/ ‘only’ and ‘too, and’}
\label{sec:onlyand}


\begin{table}[b]
	\caption{\label{tab:thebehaviourofonlyalsowithmonosyllabicanddisyllabicnouns}The behaviour of /\ipa{lɑ˧}/ ‘only; also’ with {monosyllabic} and disyllabic nouns.}
	\begin{tabularx}{\textwidth}{ Q Q P{30mm} l }
		\lsptoprule
		example & tone & example & abstract tone pattern\\ \midrule
		pig & LM & \ipa{bo˩ lɑ˧} & L.M\\
		leopard & LH & \ipa{ʐæ˩ lɑ˥ } & L.H\\
		tiger & M & \ipa{lɑ˧ lɑ˧} & M.M\\
		sheep & L & \ipa{jo˩ lɑ˥} & L.H\\
		horse & H & \ipa{ʐwæ˧ lɑ˩} & M.L\\
		deer & MH & \ipa{ʈʂʰæ˧ lɑ˥} & M.H\\ \addlinespace \hdashline \addlinespace
		fox & M & \ipa{ɖɤ˧mi˧ lɑ˧} & M.M.M\\
		colt & \#H & \ipa{ʐwæ˧zo˧ lɑ˩} & M.M.L\\
		cat & MH\# & \ipa{hwɤ˧li˧ lɑ˥} & M.M.H\\
		she\babelhyphen{nobreak}cat & H\$ & \ipa{hwɤ˧mi˥ lɑ˩} & M.H.L\\
		dog & L & \ipa{kʰv̩˩mi˩ lɑ˥} & L.L.H\\
		mule & L\# & \ipa{dɑ˧ʝi˩ lɑ˩} & M.L.L\\
		wolf & LM+MH\# & \ipa{õ˩dv̩˧ lɑ˥} & L.M.H\\
		Naxi & LM+\#H & \ipa{nɑ˩hĩ˧ lɑ˩} & L.M.L\\
		sow & LM & \ipa{bo˩mi˧ lɑ˧} & L.M.M\\
		boar & LH & \ipa{bo˩ɬɑ˥ lɑ˩} & L.H.L\\
		rat & H\# & \ipa{hwæ˧tsɯ˥ lɑ˩} & M.H.L\\
		\lspbottomrule
	\end{tabularx}
\end{table}

The tonal behaviour of the two \is{homophony}homophonous adverbs /\ipa{lɑ˧}/ ‘only’ and ‘too, and’ is presented in \tabref{tab:thebehaviourofonlyalsowithmonosyllabicanddisyllabicnouns}. Eliciting expressions consisting of a~noun followed by these adverbs is not easy, as such sequences do not constitute complete utterances. One illustration of the issue is the expression ‘only \mbox{(a/the)} she\babelhyphen{nobreak}cat', which was initially recorded with an M.M.L pattern, /\ipa{hwɤ˧mi˧ lɑ˩}/, and later with an M.M.M pattern, /\ipa{hwɤ˧mi˧ lɑ˧}/. The consultant subsequently pointed out that both were incorrect and that the correct pattern was M.H.L: /\ipa{hwɤ˧mi˥ lɑ˩}/. The homophony between this phrase and ‘to beat \mbox{(a/the)} she\babelhyphen{nobreak}cat' (likewise /\ipa{hwɤ˧mi˥ lɑ˩}/) may have contributed to the difficulty encountered at elicitation.



Comparison with the three M-tone morphemes in Table~\ref{tab:topicabstract} reveals a striking degree of similarity: the tonal behaviour of /\ipa{lɑ˧}/, /\ipa{ɳɯ˧}/, /\ipa{-ki˧}/, and /\ipa{ʈʂʰɯ˧}/ is identical for eleven of the seventeen noun categories. The only observed difference between /\ipa{lɑ˧}/ and /\ipa{ɳɯ˧}/ is in combination with
L-tone monosyllables: /\ipa{lɑ˧}/ follows an L.H pattern, as in (\ref{ex:sheepandgoats}), whereas /\ipa{ɳɯ˧}/ follows an M.M pattern, as in (\ref{ex:bythesheep}). The data was carefully verified across work sessions. (A recording is available: \textit{OnlyAnd} \pandoi{0004569}.)

 \begin{exe}
 	\ex
 	\label{ex:sheepandgoats}
 	\ipaex{jo˩ lɑ˥ {\kern2pt}|{\kern2pt} tsʰɯ˧˥}\\
 	\gll jo˩		lɑ˧		tsʰɯ˧˥\\
 	sheep		and		goat\\
 	\glt ‘sheep and goats’
 \end{exe}
 
 \begin{exe}
 	\ex
 	\label{ex:bythesheep}
 	\ipaex{jo˧ ɳɯ˧}\\
 	\gll jo˩		ɳɯ˧\\
 	sheep		\textsc{a}\\
 	\glt ‘by the sheep’
 \end{exe}

To venture a~hypothesis concerning these partial similarities, it seems plausible that \isi{grammaticalization} is accompanied by a~tonal evolution that departs from the tone of the free root. As noted in \sectref{sec:enclitics}, it may not be coincidental that the dative /\ipa{-ki˧}/ and the \isi{possessive} /\ipa{=bv̩˧}/, which have identical tonal behaviour, also share the morphosyntactic property of being “almost suffixal” \citep[155]{lidz2010}. This distinguishes them from the agentive /\ipa{ɳɯ˧}/, analyzed as a~case adposition. Viewed in this light, the differences in tone patterns may align with differences in morphosyntactic category (part of speech): one pattern is associated with suffixes, another with adpositions, a third with adverbs and conjunctions, and a fourth with discourse particles (in this case, the topic marker). 

\subsection{The adverb /\ipa{pɤ˧to˩}/ ‘even’}
\label{sec:evenplusnoun}

The tonal behaviour of the adverb /\ipa{pɤ˧to˩}/ ‘even’ is presented in \tabref{tab:thetonalbehaviourofeven}; the corresponding recording is \textit{NounsEven} \pandoi{0004565}. This data differs from that of L\#-tone locative postpositions such as /\ipa{ʁo˧tʰo˩}/
‘behind’ and /\ipa{ʈʰæ˧qo˩}/ ‘under’, discussed in \ref{sec:spatialpostpositions}, as well as from that of L\#-tone heads in
\is{compounds}compound nouns.


\begin{table}%[t]
\caption{\label{tab:thetonalbehaviourofeven}The tonal behaviour of /\ipa{pɤ˧to˩}/ ‘even’. There is an additional ‘L \textsc{pro}’ row because L-tone pronouns have exceptional behaviour.}
\begin{tabularx}{\textwidth}{ Q Q Q l }
\lsptoprule
	tone & example & meaning & N+/\ipa{pɤ˧to˩}/ ‘even’\\ \midrule
	LM & \ipa{bo˩˧} & pig & \ipa{bo˩ pɤ˥to˩}\\
	LH & \ipa{ʐæ˩˥} & leopard & \ipa{ʐæ˩ pɤ˥to˩}\\
	M & \ipa{lɑ˧} & tiger & \ipa{lɑ˧ pɤ˧to˩}\\
	L & \ipa{jo˩} & sheep & \ipa{jo˩ pɤ˩to˥}\\
	L \textsc{pro} & \ipa{no˩} & \textsc{2sg} & \ipa{no˧ pɤ˧to˩}\\
	\#H & \ipa{ʐwæ˥} & horse & \ipa{ʐwæ˧ pɤ˧to˩}\\
	MH\# & \ipa{ʈʂʰæ˧˥} & deer & \ipa{ʈʂʰæ˧ pɤ˥to˩}\\ \addlinespace \hdashline \addlinespace
	M & \ipa{ɖɤ˧mi˧} & fox & \ipa{ɖɤ˧mi˧ pɤ˧to˩}\\
	\#H & \ipa{ʐwæ˧zo\#˥} & colt & \ipa{ʐwæ˧zo˧ pɤ˧to˩}\\
	MH\# & \ipa{hwɤ˧li˧˥} & cat & \ipa{hwɤ˧li˧ pɤ˥to˩}\\
	H\$ & \ipa{hwɤ˧mi˥\$} & she\babelhyphen{nobreak}cat & \ipa{hwɤ˧mi˧ pɤ˥to˩}\\
	L & \ipa{kʰv̩˩mi˩} & dog & \ipa{kʰv̩˩mi˩ pɤ˥to˩}\\
	L\# & \ipa{dɑ˧ʝi˩} & mule & \ipa{dɑ˧ʝi˩ pɤ˩to˩}\\
	LM+MH\# & \ipa{õ˩dv̩˧˥} & wolf & \ipa{õ˩dv̩˧ pɤ˥to˩}\\
	LM+\#H & \ipa{nɑ˩hĩ\#˥} & Naxi & \ipa{nɑ˩hĩ˧ pɤ˧to˩}\\
	LM & \ipa{bo˩mi˧} & sow & \ipa{bo˩mi˧ pɤ˧to˩}\\
	LH & \ipa{bo˩ɬɑ˥} & boar & \ipa{bo˩ɬɑ˧ pɤ˧to˩}\\
	H\# & \ipa{hwæ˧tsɯ˥} & rat & \ipa{hwæ˧tsɯ˥ pɤ˩to˩}\\
\lspbottomrule
\end{tabularx}
\end{table}


\section{Concluding note}
\label{sec:asummary}
% The phenomena do not obey general rules such as those found in polar-tone systems, where the tones of suffixes obtain through a~small set of rules.

The phenomena approached in this chapter are more diverse than those examined in the preceding three. Yongning Na exhibits great variety in the tonal behaviour of grammatical elements: the dative \is{suffixes}suffix /\ipa{-ki˧}/ and the \isi{possessive} clitic /\ipa{-bv̩˧}/ behave differently from the agentive adposition /\ipa{ɳɯ˧}/; a~third pattern is found with the \is{conjunctions}conjunction /\ipa{lɑ˧}/ ‘too, and’; and yet another with the topic marker /\ipa{ʈʂʰɯ˧}/. This \is{morphotonology}morphotonological complexity arises from the large number of distinct tonal paradigms rather than from the internal complexity of any single paradigm.
%, none of which is especially complex in itself.

There remains much room for further progress in the description and analysis of this \is{morphotonology}morphotonological archipelago. In particular, systematic study of the lexicon holds potential for uncovering traces of earlier morphological \is{derivation!morphological}derivations, shedding light on historical processes. For instance, an animal \is{suffixes}suffix /\ipa{-li}/ may plausibly be reconstructed in words such as /\ipa{hwɤ˧li˧˥}/ ‘cat' and /\ipa{pʰi˧li˩}/ ‘butterfly', based on cognates in \ili{Naxi} (/\ipa{hwɑ˥le˧}/ and /\ipa{pʰe˧le˩}/). 
