\chapter{Verbs and their combinatorial properties}
\label{chap:verbsandtheircombinatoryproperties}

This chapter discusses the tones of verbs and adjectives, and their combinatorial properties.

\section{The lexical tones of verbs}
\label{sec:thelexicaltonesofverbs}

\subsection{Overview}
\label{sec:overview}

Among \is{monosyllables}monosyllabic verbs, which constitute an overwhelming majority, seven tonal categories have been identified, as shown in
\tabref{tab:thelexicaltonesofverbs}. 

%% Using subtables to obtain numbering as 1a and 1b.
%

\begin{table}
\caption{The seven tonal categories of {monosyllabic} verbs: behaviour in four different contexts.}
\label{tab:thelexicaltonesofverbs}
{\renewcommand{\arraystretch}{1.20}
\begin{tabularx}{\textwidth} { P{25mm} P{20mm} Q Q  P{20mm} }
	% {\textheight}{ l@{\hspace{10mm}} l@{\hspace{10mm}} Q l@{\hspace{10mm}} Q l@{\hspace{10mm}} Q }
\lsptoprule
	 example & \is{form!in isolation}in isolation & \textsc{neg} & \textsc{accomp} & V+‘a~bit’\\ \midrule
	  \ipa{dzɯ} ‘to eat’ & \tikzmark{0a}M & M.H & M.H & \lshadedcell M.M.M\\ 
	 \ipa{hwæ} ‘to buy’ &  \tikzmark{1a} & \tikzmark{1b}M.M & \tikzmark{1c}M.M & \shadedcell M.H.L\\
	 \ipa{tɕʰi} ‘to sell’ & \tikzmark{2a} & \hspace*{\fill}\tikzmark{2b} & \hspace*{\fill}\tikzmark{2c} & \lshadedcell M.M.M\\
	 \ipa{bi} ‘to go’ & \hspace*{\fill}\tikzmark{3a} & \hspace*{\fill}\tikzmark{3b} & \tikzmark{3c}M.L & N/A\\ 
	\ipa{dze} ‘to cut’ & \tikzmark{4a}LH & \tikzmark{4b}M.L & \hspace*{\fill}\tikzmark{4c} & M.M.H\\
	\ipa{ʈʰɯ} ‘to drink’ & \hspace*{\fill}\tikzmark{5a} & \hspace*{\fill}\tikzmark{5b} & \hspace*{\fill}\tikzmark{5c} & M.M.MH\\ 
	\ipa{lɑ} ‘to strike’ & \tikzmark{6a}MH & M.MH & M.MH & \shadedcell M.H.L\\
\lspbottomrule
\end{tabularx}}
\DrawBox{0a}{3a}
\DrawBox{4a}{5a}
\DrawBox{4b}{5b}
\DrawBox{3c}{5c}
\DrawBox{1b}{3b}
\DrawBox{1c}{2c}
\end{table}

%

%The subset of intransitive verbs within the M tone category labelled as ‘M\textsubscript{c}’ will be discussed further below, \sectref{sec:tonemcasubsetoffiveintransitiveverbswithinthemtonecategory}.
The four contexts shown in \tabref{tab:thelexicaltonesofverbs} are: (i)~\is{form!in isolation}in isolation, (ii)~with the \textsc{negation} \is{prefixes}prefix, (iii)~with the \textsc{accomplished} \is{prefixes}prefix, and (iv)~with /\ipa{ɖɯ˧-kʰwɤ˥\$}/ ‘a piece’ or
/\ipa{ɖɯ˧-ʈʰɤ˥\$}/ ‘a drop’ as an object (yielding the meaning ‘to V a~bit’).\footnote{The two {numeral}-plus-classifier phrases ‘a
	piece’ and ‘a drop’ share the same tone: H\$. The tone patterns of
	{numeral}-plus-classifier phrases are analyzed in Chapter~\ref{chap:classifiers}, and the H\$ tone is discussed in \sectref{sec:wordfinalandmorphologicalnucleusfinalHtones}.} These four contexts were chosen because their combination reveals all seven categories, whereas each individual context distinguishes only three or four. Instances of neutralization are reflected in cells that share the same tone pattern within a~given column; adjacent neutralizations are marked by a~dashed box, while nonadjacent ones are indicated by shading. 

\begin{figure}[h!!]
	\includegraphics[width=\textwidth]{figures/ms/1027ABit.jpg}
	\caption{Field notes (2007). At the beginning of a~work session, the consultant cheerfully said /\ipa{wɤ˩˥~| ɖɯ˧-kʰwɤ˧ ʐwɤ˧{$\sim$}ʐwɤ˥}/ ‘Let's have another chat!’ This construction, ‘to V a~bit’, was written down, and subsequently elicited with various verbs. It turned out to be a~useful diagnostic tool for the tones of verbs.}
	\label{fig:msABIT}
\end{figure}


While the data in \tabref{tab:thelexicaltonesofverbs} establishes the existence of seven distinct tonal categories, it allows for several
analytic interpretations. Insights from the study of the tonal categories of nouns (Chapter~\ref{chap:thelexicaltonesofnouns}) suggest that tonal diversity is greatest after an initial M tone, since this position permits the full range of H, M, L, and MH (with only LM and LH excluded by Rule~5, discussed in \sectref{sec:alistoftonerules}). The behaviour of the seven tonal classes of verbs following an M-tone \is{prefixes}prefix thus appeared as a~promising source of evidence to establish their underlying tone. However, a complication arises: not all M-tone prefixes yield the same tonal outcomes when combined with verbs. Specifically, the verb ‘to go’ 
yields /\ipa{mɤ˧-bi˧}/ with the \textsc{negation} \is{prefixes}prefix (M.M) but /\ipa{le˧-bi˩}/ with the \textsc{accomplished} \is{prefixes}prefix (M.L). Examination of a~set of M-tone prefixes, such as the \textsc{prohibitive} /\ipa{tʰɑ˧}-/ and the \textsc{durative} /\ipa{tʰi˧}-/, revealed that the \textsc{accomplished} \is{prefixes}prefix is an outlier in this respect. The \textsc{negation} was selected as the reference context for determining the phonological nature of the tones of verbs. 

Observation of verbs in this context revealed four distinct surface tones: H, M, L, and MH. These were interpreted as four primary tonal categories. Within these, \is{subcategories of lexical tones|textbf}subcategories were identified based on their different behaviour in other contexts. Among M-tone verbs, three subcategories were distinguished: M\textsubscript{a}, M\textsubscript{b}, and M\textsubscript{c}. Similarly, two subcategories were identified within the L-tone category: L\textsubscript{a} and L\textsubscript{b}. Thus, the seven tonal categories represented in \tabref{tab:thelexicaltonesofverbs} are labelled as H, M\textsubscript{a}, M\textsubscript{b}, M\textsubscript{c}, L\textsubscript{a}, L\textsubscript{b}, and MH. In light of this additional layer of analysis, the data in \tabref{tab:thelexicaltonesofverbs} are reproduced in \tabref{tab:Utonesofverbs} with an indication of the underlying tones in its first column.

\begin{table}[h!!]
	\caption{The seven tonal categories of {monosyllabic} verbs: analysis into H, M, L, and LH tones.}
	\label{tab:Utonesofverbs}
	{\renewcommand{\arraystretch}{1.20}
		\begin{tabularx}{\textwidth}{ P{8mm} P{25mm} P{20mm} Q Q  P{20mm} }
			% {\textheight}{ l@{\hspace{10mm}} l@{\hspace{10mm}} Q l@{\hspace{10mm}} Q l@{\hspace{10mm}} Q }
			\lsptoprule
			tone & example & \is{form!in isolation}in isolation & \textsc{neg} & \textsc{accomp} & V+‘a~bit’\\ \midrule
			H &  \ipa{dzɯ˥} ‘to eat’ & \tikzmark{0a}M & M.H & M.H & \lshadedcell M.M.M\\ 
			M\textsubscript{a} & \ipa{hwæ˧\textsubscript{a}} ‘to buy’ &  \tikzmark{1a} & \tikzmark{1b}M.M & \tikzmark{1c}M.M & \shadedcell M.H.L\\
			M\textsubscript{b} & \ipa{tɕʰi˧\textsubscript{b}} ‘to sell’ & \tikzmark{2a} & \hspace*{\fill}\tikzmark{2b} & \hspace*{\fill}\tikzmark{2c} & \lshadedcell M.M.M\\
			M\textsubscript{c}  & \ipa{bi˧\textsubscript{c}} ‘to go’ & \hspace*{\fill}\tikzmark{3a} & \hspace*{\fill}\tikzmark{3b} & \tikzmark{3c}M.L & N/A\\ 
			L\textsubscript{a} & \ipa{dze˩\textsubscript{a}} ‘to cut’ & \tikzmark{4a}LH & \tikzmark{4b}M.L & \hspace*{\fill}\tikzmark{4c} & M.M.H\\
			L\textsubscript{b}  & \ipa{ʈʰɯ˩\textsubscript{b}} ‘to drink’ & \hspace*{\fill}\tikzmark{5a} & \hspace*{\fill}\tikzmark{5b} & \hspace*{\fill}\tikzmark{5c} & M.M.MH\\ 
			MH &  \ipa{lɑ˧˥} ‘to strike’ & \tikzmark{6a}MH & M.MH & M.MH & \shadedcell M.H.L\\
			\lspbottomrule
		\end{tabularx}}
		\DrawBox{0a}{3a}
		\DrawBox{4a}{5a}
		\DrawBox{4b}{5b}
		\DrawBox{3c}{5c}
		\DrawBox{1b}{3b}
		\DrawBox{1c}{2c}
	\end{table}

Realizations \is{form!in isolation}in isolation, which distinguish only three subsets, make sense in light of this analysis. 
\largerpage

\begin{itemize}
    \item H tone is
realized as M due to the \isi{neutralization} of H and M in tone-group-initial position (Rule~3; see the list of phonological tone rules in~\sectref{sec:alistoftonerules}).
    \item  M tones (M\textsubscript{a}, M\textsubscript{b}, and M\textsubscript{c}) are straightforwardly realized as M.
    \item L tones (L\textsubscript{a} and L\textsubscript{b}) are both realized as LH due
to the postlexical addition of an H tone to all-L tone groups (by Rule~7).
    \item MH tone is straightforwardly realized as MH.
\end{itemize}  

L-tone verbs thus behave unlike L-tone nouns, which surface with M tone \is{form!in isolation}in isolation, as explained in Chapter~\ref{chap:thelexicaltonesofnouns}. This is
one of many pieces of evidence showing that the tone system of Yongning Na is not based solely on a~set of
phonological rules that apply uniformly in all contexts but also includes \is{morphotonology}morphotonological rules: \is{tone rules}tone rules that are specific to a~given morphological context.

Tones M\textsubscript{a} and M\textsubscript{b} yield the same
tone pattern in association with the {negation} \is{prefixes}prefix, as do L\textsubscript{a} and L\textsubscript{b}, but they are distinguished in the
fourth context. Conversely, the tone pairs \{M\textsubscript{a}, MH\} and \{M\textsubscript{b}, H\} yield the same tonal pattern when associated
with the object ‘a piece’/‘a drop’ but are distinguished following the {negation} \is{prefixes}prefix. Finally, tone category M\textsubscript{c} behaves differently from M\textsubscript{a} and M\textsubscript{b} after the \textsc{accomplished} \is{prefixes}prefix.

Examples of predicates belonging to the seven categories are presented in \tabref{tab:examplesofthesixcategoriesofverbs}.

\begin{table}%[t]
\caption{\label{tab:examplesofthesixcategoriesofverbs}
	Examples of the seven categories of verbs.}
{\renewcommand{\arraystretch}{1.35}
\begin{tabularx}{\textwidth}{ l Q }
\lsptoprule
	tone & examples\\ \midrule
	H & \ipa{dzɯ˥} ‘to eat’, \ipa{bv̩˥} ‘to divide’, \ipa{ʝi˥} ‘to do’, \ipa{se˥} ‘to walk’, \ipa{ʈʂʰæ˥} ‘to wash’\\
	M\textsubscript{a} & \ipa{hwæ˧\textsubscript{a}} ‘to buy’, \ipa{hõ˧\textsubscript{a}} ‘to go away.\textsc{imp}’, \ipa{ki˧\textsubscript{a}} ‘to give’, \ipa{li˧\textsubscript{a}}~‘to~watch’, \ipa{mæ˧\textsubscript{a}}~‘to~catch hold of’\\
	M\textsubscript{b} & \ipa{tɕʰi˧\textsubscript{b}} ‘to sell’, \ipa{ɖɯ˧\textsubscript{b}} ‘to obtain’,  \ipa{ɖʐæ˧\textsubscript{b}} ‘to ride’, \ipa{pʰæ˧\textsubscript{b}} ‘to fasten’, \ipa{ɲi˧\textsubscript{b}} ‘to need’\\
	M\textsubscript{c} & \ipa{bi˧\textsubscript{c}} ‘to go’, \ipa{hɯ˧\textsubscript{c}} ‘to go.\textsc{pst}’, \ipa{gv̩˧\textsubscript{c}} ‘to go by (of
	time)’, \ipa{ʝi˧\textsubscript{c}} ‘to come’, \ipa{pv̩˧\textsubscript{c}} ‘to chant’, \ipa{pɻ̩˧\textsubscript{c}} ‘to go out, to get out’\\
	L\textsubscript{a} & \ipa{dze˩\textsubscript{a}} ‘to cut’, \ipa{bæ˩\textsubscript{a}} ‘to sweep’, \ipa{ti˩\textsubscript{a}} ‘to hit (gently)’, \ipa{tɕi˩\textsubscript{a}}~‘to write’, \ipa{dzi˩\textsubscript{a}} ‘to sit’\\
	L\textsubscript{b} & \ipa{ʈʰɯ˩\textsubscript{b}} ‘to drink’, \ipa{dɑ˩\textsubscript{b}} ‘to weave’, \ipa{do˩\textsubscript{b}} ‘to see’, \ipa{mɤ˩\textsubscript{b}} ‘to eat food in powder form, typically tsamba (roasted flour)’, \ipa{ʐwɤ˩\textsubscript{b}} ‘to speak’\\
	MH & \ipa{ɕjɤ˧˥} ‘to try; to taste’, \ipa{gɤ˧˥} ‘to carry on one’s shoulder’, \ipa{lɑ˧˥} ‘to strike’, \ipa{tɕɤ˧˥} ‘to boil’, \ipa{ʐv̩˧˥} ‘to sew (clothes)’\\
\lspbottomrule
\end{tabularx}}
\end{table}

The labels used for the sets of categories \{L\textsubscript{a}, L\textsubscript{b}\} and \{M\textsubscript{a}, M\textsubscript{b}, M\textsubscript{c}\} are
deliberately abstract, for want of decisive evidence regarding the phonological nature of the categories
at issue. The letters are assigned on the basis of relative frequency in the lexicon. Among L tones, the ‘to cut’ type is about five times as frequent as the ‘to drink’ type. Among M tones, the ‘to buy’ type is twice as frequent as the ‘to sell’ type, while the ‘to go’ type is infrequent.

 The ‘to cut’ and ‘to drink’ types, labelled here as L\textsubscript{a} and L\textsubscript{b}, are both analyzed as containing an L tone level,
 since they are realized with L tone after the {negation} \is{prefixes}prefix. However, evidence on the
 phonological nature of the distinction between L\textsubscript{a} and L\textsubscript{b} remains limited. A theoretical possibility would be to analyze one of the two as a~simple L tone, and the other as a~\is{tonal contour}contour (LM or LH), on the {analogy} of nouns, but such an analysis would be
 arbitrary, as there is no compelling evidence that either of these categories consists of
 a~\is{tonal contour}contour. The apparent economy gained by adopting labels similar to those used for nouns would come at a high cost in terms of descriptive adequacy. In nouns, the \mbox{//LM//} and \mbox{//LH//} \is{tonal contour}contour tones surface as such \is{form!in isolation}in isolation (where they are neutralized to /LH/), whereas the //L// tone surfaces
 as /M/ \is{form!in isolation}in isolation. By contrast, no such difference is found between the L\textsubscript{a} and L\textsubscript{b} categories of verbs, both of which surface with a~/LH/ \is{tonal contour}contour \is{form!in isolation}in isolation.


\subsection{About subsets of M-tone verbs}
\label{sec:tonemcasubsetoffiveintransitiveverbswithinthemtonecategory}

M-tone verbs, defined as those that are realized with M tone after the {negation} \is{prefixes}prefix, fall into three subsets: M\textsubscript{a}, M\textsubscript{b}, and M\textsubscript{c}. From the standpoint of descriptive economy, it would be appealing to reserve the label M for one of these three and to assign the others labels drawn from the inventory of tone categories of nouns, such as \#H. However, no evidence has been found to support such identifications. Hence, the choice was made to adopt noncommittal abstract labels with subscript letters.

Tone category M\textsubscript{c} encompasses six verbs that behave
like M-tone verbs of the M\textsubscript{a} category except in a~few contexts, such as when preceded by the \textsc{accomplished} \is{prefixes}prefix /\ipa{le˧}-/. These verbs are /\ipa{bi˧\textsubscript{c}}/ ‘to go’ and its past form /\ipa{hɯ˧\textsubscript{c}}/, as well as /\ipa{gv̩˧\textsubscript{c}}/ ‘to go by, to flow, to fly (of time)’, /\ipa{ʝi˧\textsubscript{c}}/ ‘to come’, /\ipa{pv̩˧\textsubscript{c}}/ ‘to chant, to perform (a sacrifice, a~ritual, a~festival)’, and /\ipa{pɻ̩˧\textsubscript{c}}/ ‘to go out, to get out’. With the \textsc{accomplished} \is{prefixes}prefix and the \textsc{perfective} \is{suffixes}suffix, the resulting pattern is M.L.L, as illustrated in (\ref{ex:went})-(\ref{ex:prayed}); the same applies to the other four verbs: /\ipa{le˧-hɯ˩-ze˩}/, /\ipa{le˧-gv̩˩-ze˩}/, /\ipa{le˧-ʝi˩-ze˩}/, and /\ipa{le˧-pɻ̩˩-ze˩}/. This pattern contrasts with that of the other verbs in the M tone category (i.e.\ those of the M\textsubscript{a} and M\textsubscript{b} subtypes), which retain M tone after the
\textsc{accomplished} \is{prefixes}prefix. 

\begin{exe}
	\ex
	\label{ex:went}
	\ipaex{le˧-bi˩-ze˩}\\
	\gll le˧-		bi˧\textsubscript{c}	-ze˧\\
	\textsc{accomp}		to\_go	\textsc{pfv}\\
	\glt ‘[she/he/they{\dots}] went’
\end{exe}



\begin{exe}
	\ex
	\label{ex:prayed}
	\ipaex{le˧-pv̩˩-ze˩}\\
	\gll le˧-		pv̩˧\textsubscript{c}	-ze˧\\
	\textsc{accomp}		to\_chant	\textsc{pfv}\\
	\glt ‘[she/he/they{\dots}] chanted’
\end{exe}

The difference between M\textsubscript{a} and M\textsubscript{c} is not related to verb valency: all M\textsubscript{c}-tone verbs have intransitive uses, but one of the six (‘to chant’) can also be used transitively; conversely, not all intransitive verbs belong to this tonal category, witness ‘to die’, /\ipa{ʂɯ˧\textsubscript{a}}/, which yields /\ipa{le˧-ʂɯ˧}/ (\textit{Sister3.11, 95} \pandoi{0004344\#S11}).

Since these six verbs have the same behaviour as M\textsubscript{a}-tone verbs in most contexts, they could be analyzed as a~subset of the M\textsubscript{a} category. A~further diacritic could be added to their tone label, yielding something such as M\textsubscript{a}’ (M\textsubscript{a} prime). However, it appeared less awkward typographically
to label them as M\textsubscript{c}, a~third \is{subcategories of lexical tones}subcategory within the M tone category. 

The tonal behaviour of M\textsubscript{c} verbs calls for analysis. One observation that may be relevant is that the \textsc{accomplished} \is{prefixes}prefix
/\ipa{le˧}-/ can carry special semantic connotations when associated with these verbs. In examples (\ref{ex:nodinner}) and (\ref{ex:stones}), it is interpreted as ‘back/to return’:
\begin{exe}
  \ex
  \label{ex:nodinner}
  \ipaex{le˧-bi˩-dʑo˩, {\kern2pt}|{\kern2pt} ʈʂʰwɤ˧ {\kern2pt}|{\kern2pt} ɖɯ˧ mɤ˧-kv̩˧ tsɯ˥ {\kern2pt}|{\kern2pt} mv̩˩!}\\
  \gll le˧-	bi˧\textsubscript{c}	-dʑo˥	ʈʂʰwɤ˥	ɖɯ˧	mɤ˧-	-kv̩˧˥	tsɯ˧˥	mv̩˧\\
  \textsc{accomp}	to\_go	\textsc{top}	dinner	to\_get	\textsc{neg}	\textsc{abilitive}
  \textsc{rep}	\textsc{affirm}\\
  \glt ‘If [the daughter] goes back [to her mother’s home after marriage, she] cannot have dinner
  there.’ [She must not stay there for the night; she has to go back to her new home before
    evening.] \textit{(Sister3.116)} \pandoi{0004344\#S116}

  \ex
  \label{ex:stones}
  \ipaex{lv̩˧mi˧ so˩-ʈv̩˩ pɤ˩{$\sim$}pɤ˩! {\kern2pt}|{\kern2pt} le˧-bi˩-ze˩!}\\
  \gll lv̩˧mi˧	so˩-ʈv̩˩	pɤ˧˥	{$\sim$}	le˧-	bi˧\textsubscript{c}	-ze˧\\
  stone	a\_few	to\_carry	\textsc{activity}	\textsc{accomp}	to\_go	\textsc{pfv}\\
  \glt  ‘Tonight, I’ll bring (back) a~couple of stones! I’m going back!’ \textit{Context:} a~man, compelled by his spouse to steal in order to support the family, is about to steal sweetcorn; but he decides to refrain from stealing and instead fills his basket with stones before returning home. (\textit{Reward.77}) \pandoi{0004446\#S77}
\end{exe}

On the analogy of phenomena found in \ili{Naxi}, one might imagine that an additional morpheme is present after the \textsc{accomplished} prefix, its \textit{signified} being the semantic indication `{\dots}~back' and its \textit{signifier} a floating High tone that depresses the verb's tone to L. In \ili{Naxi}, `to go back' is \ipa{le˧-wu˥ bɯ˧}; the second of these three syllables can coalesce with the first, yielding \ipa{le˧˥} \citep{michaudetal2007d,he_naxiyu_2021}. It would not be particularly surprising if the Na cognate for Naxi \ipa{wu˥}, namely \ipa{wo˧˥} `again, back', had undergone a similar type of reduction, leaving only a tonal trace. 

However, this hypothesis does not appear particularly promising, for several reasons. First, in examples (\ref{ex:nodinner}) and (\ref{ex:stones}), it is not possible to insert \ipa{wo˧˥} between the \textsc{accomplished} prefix and the verb, depriving the hypothesis of direct language-internal evidence of the sort available in \ili{Naxi}, where full and reduced forms can be observed side by side. Secondly, instances of syllable reduction and coalescence in Na appear to be relatively rare compared with \ili{Naxi}. The only such case documented in this volume is fairly different: reduction of an L-tone morpheme (the \isi{associative plural}) depresses the tone of a following possessive from M to L (see \sectref{sec:syllablereduction}). Thirdly, the meaning ‘back, again' is not always present when an M\textsubscript{c}-tone verb is preceded by the \textsc{accomplished} prefix, and the tone pattern remains unchanged when the verb simply means ‘to go, to set off’, as in
(\ref{ex:ahihi}), where the motion is clearly away from a~familiar place and towards an unfamiliar one.

\begin{exe}
  \ex
  \label{ex:ahihi}
  \ipaex{“æ.hi.hi!” pi˧, {\kern2pt}|{\kern2pt} le˧-bi˩-zo˩-kv̩˩ {\kern2pt}|{\kern2pt} tsɯ˧˥ {\kern2pt}|{\kern2pt} mv̩˩!}\\
  \gll æ.hi.hi		pi˥	le˧-	bi˧\textsubscript{c}	-zo˧	-kv̩˧˥	tsɯ˧˥	mv̩˧\\
  \textsc{intj}	to\_say	\textsc{accomp}	to\_go	\textsc{obligative}	\textsc{abilitive}
  \textsc{rep}	\textsc{affirm}\\
  \glt ‘[The mother, uncles, aunts and other relatives of the deceased wife’s family shout out a~cry
    of defiance:] “A-hi-hi!” and they set off!’ 
    %[towards the husband’s house]!’ 
    \textit{(Sister1.81)} \pandoi{0004341\#S81}
\end{exe}

To sum up, in the absence of a~principled explanation for the tonal behaviour of these five verbs, it appeared best to set up a~distinct synchronic tone category for them: M\textsubscript{c}.


\subsection{Adjectives as distinct from verbs}
\label{sec:adjectivesasdistinctfromverbs}

\is{adjectives|textbf}
%\is{adjectives|(}
The question of whether adjectives constitute a~distinct part of speech has been raised for various languages of Southeast Asia. For a~review and discussion, with a particular focus on \ili{Tai-Kadai}, see \citet{post2008adjectives}. As \citet[41]{delancey2015adjectival} observes, “[s]ome Tibeto-Burman languages have a definable adjective category, usually only marginally distinguishable from nouns or from verbs”. Yongning Na is one such language: in Na, adjectives behave in most respects like \is{stative verbs|textbf}stative verbs, yet the intensive
/\ipa{ʐwæ˩}/ ‘extremely’ occurs exclusively with adjectives (whereas the intensive /\ipa{ɖwæ˧˥}/ can be used with verbs~-- ‘to V a lot’~-- as well as with adjectives~-- ‘very \textsc{Adj}’). Also, the
copula is not used with adjectives but may be added after any verb to express
certainty \citep[354]{lidz2010}. Moreover, adjectives have some tonal specificities. These pieces of evidence warrant the recognition of adjectives as a~formally distinct class of words. 

Four main tonal categories have been identified for \is{monosyllables}monosyllabic adjectives: L, M, H, and MH. The L tone
category must be further divided into two \is{subcategories of lexical tones}subcategories. The L\textsubscript{b} category only contains two examples: /\ipa{dʑɤ˩\textsubscript{b}}/ ‘good’ and /\ipa{nɑ˩\textsubscript{b}}/ ‘black, dark’. Their behaviour in context is examined in detail later in this chapter.

Importantly, the MH category of \isi{adjectives} and the MH category of verbs do not always have the same
tonal behaviour, as illustrated in \sectref{sec:ahtonesuffixtherelativizernominalizer} below. Likewise, the L\textsubscript{a} and L\textsubscript{b} categories of
adjectives are not fully parallel to the L\textsubscript{a} and L\textsubscript{b} categories of verbs in terms of tonal
behaviour. As for the M category of adjectives, no evidence was found for a~division into \is{subcategories of lexical tones}subcategories analogous to the M\textsubscript{a}, M\textsubscript{b}, and M\textsubscript{c} categories set up for verbs. These discrepancies
between the tone systems of adjectives and verbs exemplify the morphosyntactic
ramifications of tone in Yongning Na.

\is{disyllables}Disyllabic adjectives also exist. Examples include /\ipa{pʰv̩˧ɖɯ˧˥}/ ‘expensive’, from /\ipa{pʰv̩˧}/ ‘price’ and
/\ipa{ɖɯ˩\textsubscript{a}}/ ‘large’, and /\ipa{lo˩ɖɯ˧}/ ‘generous’, from /\ipa{lo˩˧}/ ‘hand’ and, again, /\ipa{ɖɯ˩\textsubscript{a}}/ ‘large’.

In view of the range of tone categories found in nouns, verbs and adjectives in Yongning Na, it is
no wonder that they yield a~wealth of diverse patterns when combined among themselves and in interaction with grammatical morphemes. The structure of the Na verb phrase, as schematized by \citet[350–351]{lidz2010}, consists of the following components: manner
adverb~-- verb complex~-- causative~-- intensifier~-- tense/aspect and modal particles, and auxiliary
verbs~-- quotative evidential. Additionally, the verb may be preceded by spatial indications such as
‘forward’/’backward’ and ‘upward’/‘downward’. The verb complex itself may be a~lexical verb, an \is{existentials}existential
verb, a~\isi{copula}, or a~serial verb construction, and it may take a~verbal \is{prefixes}prefix (or two prefixes, in the
case of the \textsc{durative} \is{prefixes}prefix followed by the {negation} \is{prefixes}prefix). 

The following sections examine the \isi{morphotonology} of the verb phrase in detail. 
%The topic of adverbials is not addressed in this chapter, as adverbials seldom interact with verbs; for a~discussion, see~\sectref{sec:someelementsalwaysconstituteatonegroupontheirown}.
%\is{adjectives|)}

\section{Reduplication}
\label{sec:reduplication}

Reduplication of verbs is a~highly productive process in Yongning Na. From a~semantic point of view, \isi{reduplication} can convey various types of divergence from the prototype of the action or activity referred to by the verb. If the verb refers to an action, the reduplicated form can warp it towards representation as an activity, as illustrated in~(\ref{ex:costume}).  

\begin{exe}
	\ex
	\label{ex:costume}
	\ipaex{jo˧-lo˥dʑo˩ tʰi˩-kʰɯ˩{$\sim$}kʰɯ˩. {\kern2pt}|{\kern2pt} hæ̃˩-lo˩pv̩˩ tʰi˥-kʰɯ˩{$\sim$}kʰɯ˩.}\\
	\gll jo˥	lo˩dʑo˥		tʰi˧-		kʰɯ˧˥	{$\sim$}		hæ̃˩	lo˩pv̩˧˥ 	tʰi˧-		kʰɯ˧˥	{$\sim$}\\
	jade	bracelet	\textsc{dur}	to\_put		\textsc{activity}	gold		ring		\textsc{dur}	to\_put		\textsc{activity}\\
	\glt ‘They adorned her with jade bracelets. They adorned her with gold rings.’ \textit{Context:} a bride is being prepared for a wedding. \textit{(BuriedAlive2.29)} \pandoi{0004536\#S29}
    %\footnote{In view of the ethnological reports about Na family structure reviewed in Appendix~\ref{chap:historyanthropologysociology} (\sectref{sec:anthropologicalresearchthefascinationofnafamilystructure}), it may come as a~surprise that folk stories about weddings exist in Yongning. Arranged marriages and difficult relationships between brides and mothers-in-law are typical of Confucian cultures (see e.g.~\citealt{rileyInterwoven1994, chanviolence2008}), not of pre-1950s Yongning. But the argument of a~story's fit to a~particular culture runs both ways: a~tale can be appealing because it resonates with one's cultural background, or on the contrary, it can gain appeal from exoticism. The 1990s soap opera Ke Wang \zh{渴望} (‘Yearnings’), which was a~roaring success throughout China \citep[a~phenomenon studied by][]{wangkewang1992}, was also popular in Yongning, where the Na gathered at the homes of television owners to watch the story of Liu Huifang, the ideal daughter-in-law (this is narrated by F4 in the document entitled Evenings). Part of the soap opera's appeal to a~Na audience may have been due to the novelty of the social relationships that it stages. A~young {Naxi} woman from the Lijiang plain explained to me in 2010 that she loved watching Chinese soap operas “to see how Han people live”, i.e.\ out of curiosity for cultural habits which she sees as different from {Naxi} custom. A~study in comparative mythology and folklore in this area of the Himalayas would be necessary to trace the origin and development of stories such as that of the unhappy daughter-in-law who gets buried alive, a~story also known to {Laze} consultant F7 (the {Laze} version is also available online \pandoi{0004337}). The thrills and spills of this story make it a~good candidate for adoption by anyone who likes a~good yarn.}
\end{exe}

The \textit{simplex} form /\ipa{jo˧-lo˥dʑo˩ tʰi˩-kʰɯ˩}/ means ‘to put a~jade bracelet (on someone's wrist)’. It depicts a~single, well-delimited event. In the absence of a~distinction between singular and plural, noun phrases tend to be interpreted as singular unless a~\is{numerals}numeral-plus-classifier phrase is added, as in~(\ref{ex:putsimple}).  

\begin{exe}
	\ex
	\label{ex:putsimple}
	\ipaex{jo˧-lo˥dʑo˩{\kern2pt}|{\kern2pt}ɲi˧-ɭɯ˧ tʰi˧-kʰɯ˧˥}\\
	\gll jo˥	lo˩dʑo˥		ɲi˧		ɭɯ˧\textsubscript{b}	tʰi˧-		kʰɯ˧˥\\
	jade	bracelet	two		\textsc{clf}	\textsc{dur}		to\_put\\
	\glt ‘to put two jade bracelets’
\end{exe}

By contrast, (\ref{ex:costume}) means that the bride is adorned with jade bracelets and gold rings by family members. The reduplicated form leads to a~plural interpretation of the nouns, presumably because successive acts of adorning are not performed with the same object: if you adorn someone more that once you would be doing this with additional ornaments, not by taking one off and putting it on again. 

The use of \isi{reduplication} illustrated by~(\ref{ex:costume}) is glossed as \textsc{activity}. It represents the process denoted by the verb as an activity rather than as a~neatly delimited event. In the narrative from which (\ref{ex:costume}) is drawn, verb \isi{reduplication} \is{emphasis}emphasizes the generosity of the bride's family as a~unified social entity. They collectively go through the prescribed steps of the marriage ritual, giving generous measure of the requisite offerings; each family member's individual action partakes in a~collective activity. This builds a~contrast with later events, when the young woman, feeling estranged in her in-laws’ home, becomes oblivious of good manners and lapses into antisocial behaviour (solitary gluttony).

The surface semantic effect of \textsc{activity} is close to that of the {progressive} \is{suffixes}suffix /\ipa{-dʑo˧}/. For instance, the reduplicated form of /\ipa{ʑi˧˥}/ ‘to sleep’ in~(\ref{ex:everyoneasleep}) could be replaced by the \textit{simplex} form followed by the {progressive} \is{suffixes}suffix /\ipa{-dʑo˧}/, as in~(\ref{ex:everyoneasleepprog}). 

\begin{exe}
	\ex
	\label{ex:everyoneasleep}
	\ipaex{hĩ˧ {\kern2pt}|{\kern2pt} ɖɯ˧-tɑ˧˥ {\kern2pt}|{\kern2pt} le˧-ʑi˩{$\sim$}ʑi˩.}\\
	\gll hĩ˥		ɖɯ˧-tɑ˧˥	le˧-			ʑi˧˥	{$\sim$}\\
	person		all		\textsc{accomp}		to\_sleep	\textsc{activity}\\
	\glt ‘Everyone was asleep.’ \textit{(BuriedAlive2.94)} \pandoi{0004536\#S94}
\end{exe}

\begin{exe}
	\ex
	\label{ex:everyoneasleepprog}
	\ipaex{hĩ˧ {\kern2pt}|{\kern2pt} ɖɯ˧-tɑ˧˥ {\kern2pt}|{\kern2pt} le˧-ʑi˧-dʑo˥.}\\
	\gll hĩ˥		ɖɯ˧-tɑ˧˥	le˧-			ʑi˧˥	-dʑo˧\\
	person		all		\textsc{accomp}		to\_sleep	\textsc{prog}\\
	\glt ‘Everyone was asleep.’
\end{exe}

As noted by \citet[373]{lidz2010}, “[w]hen \isi{stative verbs} reduplicate, one gets a~reading of added intensity, while reduplicating non-{stative} verbs gives a~reading of reciprocity of action, or a~semantics of back-and-forth”. An example is shown in~(\ref{ex:likedeachother}), where the reduplicated adjective ‘pleased’ serves as an understatement for ‘in love’ (‘they were in love with each other’).

\begin{exe}
	\ex
	\label{ex:likedeachother}
	\ipaex{ʁo˧dɑ˧, {\kern2pt}|{\kern2pt} no˧bv̩˥-tsʰɯ˩ɻ̩˩ lɑ˩ {\kern2pt}|{\kern2pt} ʈʂʰɯ˧=zɯ˩ | fv̩˧{$\sim$}fv̩˩-ɲi˩!}\\
	\gll ʁo˧dɑ˧		no˧bv̩˥-tsʰɯ˩ɻ̩˩	lɑ˧	ʈʂʰɯ˧=zɯ˩		fv̩˧	{$\sim$} -ɲi˩\\
	before		proper\_name	and		\textsc{3du}	pleased		\textsc{recp}		\textsc{certitude}\\
	\glt ‘Before [my daughter joined your house]{\dots} Nobbu Ci’er and [my daughter]{\dots} they used to like each other!’ \textit{Context:} the unhappy woman's mother explains %to the husband's mother 
    what the matter is with her daughter. \textit{(BuriedAlive2.136)} \pandoi{0004536\#S136}
\end{exe}

The online texts contain more than a~thousand examples of \isi{reduplication}. This volume about morphotonology is not the right place to explore these rich materials to delve into the study of the semantic values 
%of \isi{reduplication} in Yongning Na, to analyze the 
%“paradoxes of fragmentation” \citep{francois2004} 
that arise through divergence from \textit{simplex} meanings, and to look for semantic and \is{stylistics}stylistic common denominators to \isi{reduplication} as it applies to different parts of speech.\footnote{For insights into these topics, see \citet[372-373, 385, 438, 440]{lidz2010}. Additionally, {reduplication} in {Naxi} is discussed in \citet{michaudetal2007c}.} Instead, the present discussion is confined to morphotonological patterns.

\tabref{tab:thetonepatternsofreduplicatedverbsinyongningna} summarizes the tone patterns of reduplicated verbs in Yongning Na. Two of these verbs differ from those in Tables \ref{tab:thelexicaltonesofverbs}-\ref{tab:Utonesofverbs}: ‘to go' does not readily lend itself to \isi{reduplication}, so ‘to chant' was substituted; likewise, ‘to drink' is much less frequently reduplicated than ‘to speak'. The selection of example verbs in this chapter (primarily drawn from \tabref{tab:examplesofthesixcategoriesofverbs}) aims to avoid forms that would be unnatural or jarring to native speakers for semantic, syntactic or pragmatic reasons. Crucially, the seven tonal categories of verbs are internally consistent: within each class, all verbs conform to the same tone patterns. Thus, ‘to chant' always has the same tonal behaviour as ‘to go', and ‘to speak' always has the same behaviour as ‘to drink'. 
%in \tabref{tab:thetonepatternsofreduplicatedverbsinyongningna} exhibit the same behaviourwould behave in the same way as ‘to go' and ‘to drink' if they were in \tabref{tab:Utonesofverbs}, and vice versa. 

The patterns in \tabref{tab:thetonepatternsofreduplicatedverbsinyongningna} hold for all the verbs for which a~reduplicated form could be elicited. The only apparent \is{exceptions}exception is the disyllabic verb //\ipa{wɤ˩}{\allowbreak}\ipa{{$\sim$}wɤ˩}// (surface form:
/\ipa{wɤ˩{$\sim$}wɤ˩˥}/) ‘to detour past, to bypass’, which looks like an unmistakeable case of \isi{reduplication}, but whose L tone does not correspond to any of the patterns found for other verbs. At a~push, this verb can be used in {monosyllabic} form, as in (\ref{ex:bypassed}). 

% 
\begin{exe}
	\ex
	\label{ex:bypassed}
	\ipaex{le˧-wɤ˩-ze˩}\\
	\gll le˧-	??wɤ˩		-ze˧\\
	\textsc{accomp}		to\_bypass	\textsc{pfv}\\
	\glt ‘[She/he/they] bypassed’ (elicited example; the {question} marks ‘??’ signal the problematic status of the {monosyllabic} verb form)
\end{exe}

In view of the M.L.L tone pattern in (\ref{ex:bypassed}), the \is{monosyllables}monosyllabic form must be analyzed as carrying lexical L tone: //\ipa{wɤ˩}//. However, this \is{monosyllables}monosyllabic form appears to result from truncation of the disyllabic form rather than the other way around. It seems reasonable to hypothesize that the disyllabic form //\ipa{wɤ˩{$\sim$}wɤ˩}// ‘to bypass’ is not synchronically \is{derivation!morphological}derived from a~{monosyllabic} root but belongs to the small and heterogeneous set of disyllabic verbs in Yongning Na. Consequently, this verb does not constitute a~\isi{counterexample} to the generalizations shown in \tabref{tab:thetonepatternsofreduplicatedverbsinyongningna}.

\begin{table}%[t]
\caption{\label{tab:thetonepatternsofreduplicatedverbsinyongningna}The tone patterns of reduplicated verbs.}
\begin{tabularx}{\textwidth}{ l@{\hspace{5mm}} l@{\hspace{5mm}} l@{\hspace{5mm}} l@{\hspace{4mm}} l@{\hspace{3mm}} l@{\hspace{3mm}} }
\lsptoprule
	tone & example & meaning & \isi{reduplication} & surface tone & underlying tone\\ \midrule
	H & \ipa{dzɯ˥} & to eat & \ipa{dzɯ˧{$\sim$}dzɯ\#˥} & M.M & \#H\\
	M\textsubscript{a} & \ipa{hwæ˧\textsubscript{a}} & to buy & \ipa{hwæ˥{$\sim$}hwæ˩} & M.L & H--\\
	M\textsubscript{b} & \ipa{tɕʰi˧\textsubscript{b}} & to sell & \ipa{tɕʰi˧{$\sim$}tɕʰi˧} & M.M & M\\
	M\textsubscript{c} & \ipa{pv̩˧\textsubscript{c}} & to chant & \ipa{pv̩˥{$\sim$}pv̩˩} & M.L & H--\\
	L\textsubscript{a} & \ipa{dze˩\textsubscript{a}} & to cut & \ipa{dze˧{$\sim$}dze˥} & M.H & H\#\\
	L\textsubscript{b} & \ipa{ʐwɤ˩\textsubscript{b}} & to speak & \ipa{ʐwɤ˥{$\sim$}ʐwɤ˩} & M.L & H--\\
	MH & \ipa{lɑ˧˥} & to strike & \ipa{lɑ˩{$\sim$}lɑ˧˥} & L.MH & LM+MH\#\\
\lspbottomrule
\end{tabularx}
\end{table}


\begin{table}%[t]
\caption{Surface phonological representation of reduplicated verbs in two different carrier phrases.}
	\label{tab:reduplicatedverbsinacarrierphrase}
	\begin{tabularx}{\textwidth}{ l@{\hspace{6mm}} l@{\hspace{6mm}} l@{\hspace{6mm}} l@{\hspace{6mm}} Q }
		\lsptoprule
		tone & example & meaning & \textsc{accomp}+\textsc{redupl} & ‘to V a~little’\\ \midrule
		H & \ipa{dzɯ˥} & to eat & \ipa{le˧-dzɯ˧{$\sim$}dzɯ˧} & \ipa{ɖɯ˧-dzɯ˧{$\sim$}dzɯ˧-ɻ̩˥}\\
		M\textsubscript{a} & \ipa{hwæ˧\textsubscript{a}} & to buy & \ipa{le˧-hwæ˥{$\sim$}hwæ˩} & \ipa{ɖɯ˧-hwæ˥{$\sim$}hwæ˩-ɻ̩˩}\\
		M\textsubscript{b} & \ipa{tɕʰi˧\textsubscript{b}} & to sell & \ipa{le˧-tɕʰi˧{$\sim$}tɕʰi˧} & \ipa{ɖɯ˧-tɕʰi˧{$\sim$}tɕʰi˧-ɻ̩˩}\\
		M\textsubscript{c} & \ipa{pv̩˧\textsubscript{c}} & to chant & \ipa{le˧-pv̩˥{$\sim$}pv̩˩} & \ipa{ɖɯ˧-pv̩˧{$\sim$}pv̩˥-ɻ̩˩}\\
		L\textsubscript{a} & \ipa{bæ˩\textsubscript{a}} & to sweep & \ipa{le˧-bæ˧{$\sim$}bæ˥} & \ipa{ɖɯ˧-bæ˧{$\sim$}bæ˥-ɻ̩˩}\\
		L\textsubscript{b} & \ipa{ʐwɤ˩\textsubscript{b}} & to speak & \ipa{le˧-ʐwɤ˥{$\sim$}ʐwɤ˩} & \ipa{ɖɯ˧-ʐwɤ˥{$\sim$}ʐwɤ˩-ɻ̩˩}\\
		MH & \ipa{lɑ˧˥} & to strike & \ipa{le˧-lɑ˩{$\sim$}lɑ˩} & \ipa{ɖɯ˧-lɑ˩{$\sim$}lɑ˩-ɻ̩˩}\\
		\lspbottomrule
	\end{tabularx}
\end{table}

The tonal
string that surfaces \is{form!in isolation}in isolation is indicated in the “surface tone” column in \tabref{tab:thetonepatternsofreduplicatedverbsinyongningna}. The underlying
tone (provided in the last column and also used for transcribing the reduplicated expressions in the fourth column) was arrived at by examining the behaviour of reduplicated expressions in different contexts. (The corresponding recordings are \textit{VerbReduplObj} \pandoi{0004526} and
\textit{VerbReduplObj2} \pandoi{0004516}.) \tabref{tab:reduplicatedverbsinacarrierphrase} presents the tone patterns that obtain in frames (\ref{ex:accompred}) and (\ref{ex:delimredincho}). Frame (\ref{ex:delimredincho}) is one of the few contexts where M\textsubscript{a} and M\textsubscript{c} yield different outcomes. 

\begin{exe}
	\ex
	\label{ex:accompred}
	\ipaex{le˧-V{$\sim$}V}\\
	\gll le˧-		V					{$\sim$}\\
	\textsc{accomp}		\textit{target~verb}				\textsc{red}\\
	\glt ‘to V’
\end{exe}

\begin{exe}
	\ex
	\label{ex:delimredincho}
	\ipaex{ɖɯ˧-V{$\sim$}V-ɻ̩˩}\\
	\gll ɖɯ˧-		V					{$\sim$}		-ɻ̩˩\\
	\textsc{delimitative}		\textit{target~verb}				\textsc{red}		\textsc{inchoative}\\
	\glt ‘to V a~little’
\end{exe}

Finally, \tabref{tab:reduplicatedverbswiththehtoneobjectthings} shows the result of associating the object /\ipa{tso˧{$\sim$}tso˧}/ ‘thing’ to a~reduplicated verb. 

\largerpage
Interpretation of the tone pattern for H-tone verbs draws on information from \tabref{tab:reduplicatedverbsinacarrierphrase}. The presence of an H tone on the final syllable of /\ipa{ɖɯ˧-dzɯ˧{$\sim$}dzɯ˧-ɻ̩˥}/ ‘to eat a~little’, from /\ipa{dzɯ˥}/ ‘to eat’, suggests that the reduplicated expression contains an H tone. Since this H tone does not surface when the expression is uttered \is{form!in isolation}in isolation (where the reduplicated verb appears as /\ipa{dzɯ˧{\allowbreak}{$\sim$}dzɯ˧}/), the tonal category cannot be //H--// (an initial H tone), nor \mbox{//H\#//} (a~final H tone), nor \mbox{//H\$//} (a~\textit{hopping} H tone: see \sectref{sec:wordfinalandmorphologicalnucleusfinalHtones}). On the other hand, its behaviour is fully compatible with interpretation as \mbox{//\#H//} (a~\textit{floating} H tone), and this interpretation is therefore adopted in \tabref{tab:thetonepatternsofreduplicatedverbsinyongningna}. 

\begin{table}[t]
	\caption{\label{tab:reduplicatedverbswiththehtoneobjectthings}Reduplicated verbs with the \#H-tone object ‘things’.}
	\begin{tabularx}{\textwidth}{ l@{\hspace{6mm}} l@{\hspace{6mm}} l@{\hspace{6mm}} l@{\hspace{6mm}} Q }
		\lsptoprule
		tone & example & meaning & \isi{reduplication} & ‘things’+reduplicated V\\ \midrule
		H & \ipa{dzɯ˥} & to eat & \ipa{dzɯ˧{$\sim$}dzɯ\#˥} & \ipa{tso˧{$\sim$}tso˧ dzɯ˧{$\sim$}dzɯ˧}\\
		M\textsubscript{a} & \ipa{hwæ˧\textsubscript{a}} & to buy & \ipa{hwæ˥{$\sim$}hwæ˩} & \ipa{tso˧{$\sim$}tso˧ hwæ˧{$\sim$}hwæ˥}\\
		M\textsubscript{b} & \ipa{tɕʰi˧\textsubscript{b}} & to sell & \ipa{tɕʰi˧{$\sim$}tɕʰi˧} & \ipa{tso˧{$\sim$}tso˧ tɕʰi˧{$\sim$}tɕʰi˧}\\
		L\textsubscript{a} & \ipa{bæ˩\textsubscript{a}} & to sweep & \ipa{bæ˧{$\sim$}bæ˥} & \ipa{tso˧{$\sim$}tso˧ bæ˧{$\sim$}bæ˥}\\
		L\textsubscript{b} & \ipa{ʐwɤ˩\textsubscript{b}} & to speak & \ipa{ʐwɤ˥{$\sim$}ʐwɤ˩} & \ipa{tso˧{$\sim$}tso˧ ʐwɤ˧{$\sim$}ʐwɤ˥}\\
		MH & \ipa{lɑ˧˥} & to strike & \ipa{lɑ˩{$\sim$}lɑ˧˥} & \ipa{tso˧{$\sim$}tso˧ lɑ˥{$\sim$}lɑ˩}\\
		\lspbottomrule
	\end{tabularx}
\end{table}


The /M.L/ surface pattern in the \isi{reduplication} of M\textsubscript{a}-tone verbs (e.g.~/\ipa{hwæ˧\textsubscript{a}}/ ‘to buy’ → /\ipa{hwæ˧{$\sim$}hwæ˩}/) could reflect various underlying tones, such as //L\#// (a~final L tone) or //H--// (an H tone associated to the first part of the reduplicated expression, i.e.\ its first syllable). Crucial evidence comes from the contexts documented in \tabref{tab:reduplicatedverbsinacarrierphrase}, where the first syllable of the reduplicated expression carries an H tone, guiding towards an interpretation of the underlying pattern as //H--// and not //L\#//.\footnote{An initial H tone never surfaces as such, due to the {neutralization} of \mbox{//H//} and \mbox{//M//} in tone-group-initial position; this is referred to as Rule~3 (see \sectref{sec:alistoftonerules}).}

Reduplication of M\textsubscript{b}-tone verbs is the simplest case: the reduplicated expression carries M tone \is{form!in isolation}in isolation (e.g.~/\ipa{tɕʰi˧\textsubscript{b}}/ ‘to sell’ → /\ipa{tɕʰi˧{$\sim$}tɕʰi˧}/) and remains unaltered in the frames shown in \tabref{tab:reduplicatedverbsinacarrierphrase} and \tabref{tab:reduplicatedverbswiththehtoneobjectthings}. This strongly suggests that the reduplicated verb is underlyingly M-tone.

Analysis for the remaining three tonal categories is likewise straightforward. L\textsubscript{a}-tone verbs reduplicate into a~pattern with an H tone attached to the final syllable (see \tabref{tab:thetonepatternsofreduplicatedverbsinyongningna}), leading to an analysis as \mbox{//H\#//}. The analysis for the forms issuing from L\textsubscript{b}-tone verbs is identical to that for M\textsubscript{a}-tone verbs set out above. Finally, the underlying representation //LM+MH\#//postulated for the expression reduplicated from an MH-tone verb corresponds to its surface form, without any added complexities.

\largerpage
The relationship between the tones of \is{monosyllables}monosyllabic verbs and those of reduplicated expressions is another matter. Why does reduplication yield a~final H tone for L\textsubscript{a}-tone verbs, for instance? Why do MH-tone verbs get an initial L in their \is{reduplication}reduplicated form? Phonology does not provide answers to these questions: the tone patterns found in reduplicated forms cannot be \is{derivation!tonal}derived from those of the simplex forms through phonological rules. The
correspondences between simple and reduplicated tones need to be learnt; they
constitute one of the components of the tonal grammar of Yongning Na.\footnote{In {Naxi}, {reduplication} similarly lacks a~clear phonological pattern: H reduplicates as H.M, M as M.M, and L as M.L \citep[10-11]{he1987}. These patterns can be analyzed as originating diachronically in total {reduplication}: H → H.H, M → M.M, L → L.L \citep{michaudetal2007c}. In Yongning Na, however, no such simple historical scenario can currently be proposed. This is an area where data from neighbouring dialects appears essential for advancing the {diachronic} analysis.}

As mentioned above, disyllabic verbs are rare and constitute a~heterogeneous set. Only one instance of \isi{reduplication} of a~disyllabic verb was observed. It follows an AABB pattern (repetition of the first syllable of the \textit{simplex} form, then the second): /\ipa{ʂv̩˧{$\sim$}ʂv̩˧ɖv̩˧{$\sim$}ɖv̩˧˥}/
‘pensively; with a~heavy heart’ (\textit{Reward.49} \pandoi{0004447\#S49}), from /\ipa{ʂv̩˧ɖv̩˧}/ ‘to think; to miss’. This is the sole example of
a~{correspondence} between an M tone in the \textit{simplex} form and an MH\# tone in the reduplicated
expression. It looks
like a~unique \is{innovative (phonological form)}innovation rather than the outcome of a~productive pattern.


\section{Prefixes}
\label{sec:prefixes}

\subsection{M-tone prefixes}
\label{sec:mtoneprefixes}

The prefixes described in this section can be analyzed either as M-tone prefixes or as toneless
prefixes that receive M by default. No evidence was found to suggest that they are inherently specified for tone. For convenience, they are referred to here as M-tone prefixes.

The most common M-tone verbal prefixes are the \textsc{negative} /\ipa{mɤ˧}-/, \textsc{prohibi\-tive} /\ipa{tʰɑ˧}-/, \textsc{durative} /\ipa{tʰi˧}-/, and \textsc{accomplished} /\ipa{le˧}-/.

\begin{quotation}
	The \textsc{accomplished} \is{prefixes}prefix \ipa{lə³³-} is used to give a~reading of
	accomplishment to a~verb with lexical aspect of ongoing state, process, or liminality. ({\dots})
	The \textsc{durative} \is{prefixes}prefix \ipa{tʰɯ³³-} [in the Alawua dialect, studied in this volume: /\ipa{tʰi˧}/] is used to give a~reading of
	ongoing action to verbs with lexical aspect of process or liminality. \citep[345]{lidz2010}
\end{quotation}
  
These prefixes all have the same tonal behaviour, with the exception of the \textsc{accomplished} /\ipa{le˧-}/in association with M\textsubscript{c}-tone verbs, as reported in \sectref{sec:tonemcasubsetoffiveintransitiveverbswithinthemtonecategory}. A~less common \is{prefixes}prefix, /\ipa{mv̩˧-}/, conveying \textsc{imminence}, has a~behaviour of its own, described further below.

\tabref{tab:thetonesofverbsinassociationwithamtoneprefix} presents the tonal behaviour of the most common prefixes. To facilitate reference,
redundant data is provided for three prefixes: \textsc{durative}, \textsc{prohibitive}, and \textsc{negation}, all of which follow identical tone patterns. Some of the relevant combinations can be heard in the following online recordings: \textit{VerbProhib} \pandoi{0004522}, \textit{VerbProhib2} \pandoi{0004524}, \textit{VerbDurative} \pandoi{0004520}, and \textit{AccompPfv} \pandoi{0004518}.

\begin{table}%[t]
\caption{\label{tab:thetonesofverbsinassociationwithamtoneprefix}The tones of verbs in association with an M-tone prefix.}
\begin{tabularx}{\textwidth}{ l@{\hspace{7mm}} l@{\hspace{7mm}} l@{\hspace{7mm}} l@{\hspace{7mm}} l@{\hspace{7mm}} l@{\hspace{7mm}} }
\lsptoprule
	tone & example & meaning & \textsc{durative} & \textsc{prohibitive} & \textsc{negation}\\ \midrule
	H & \ipa{dzɯ˥} & to eat & \ipa{tʰi˧-dzɯ˥} & \ipa{tʰɑ˧-dzɯ˥} & \ipa{mɤ˧-dzɯ˥}\\
	M\textsubscript{a} & \ipa{hwæ˧\textsubscript{a}} & to buy & \ipa{tʰi˧-hwæ˧} & \ipa{tʰɑ˧-hwæ˧} & \ipa{mɤ˧-hwæ˧}\\
	M\textsubscript{b} & \ipa{tɕʰi˧\textsubscript{b}} & to sell & \ipa{tʰi˧-tɕʰi˧} & \ipa{tʰɑ˧-tɕʰi˧} & \ipa{mɤ˧-tɕʰi˧}\\
	M\textsubscript{c} & \ipa{pv̩˧\textsubscript{c}} & to chant & \ipa{tʰi˧-pv̩˧} & \ipa{tʰɑ˧-pv̩˧} & \ipa{mɤ˧-pv̩˧}\\
	L\textsubscript{a} & \ipa{bæ˩\textsubscript{a}} & to sweep & \ipa{tʰi˧-bæ˩} & \ipa{tʰɑ˧-bæ˩} & \ipa{mɤ˧-bæ˩}\\
	L\textsubscript{b} & \ipa{ʐwɤ˩\textsubscript{b}} & to speak & \ipa{tʰi˧-ʐwɤ˩} & \ipa{tʰɑ˧-ʐwɤ˩} & \ipa{mɤ˧-ʐwɤ˩}\\
	MH & \ipa{lɑ˧˥} & to strike & \ipa{tʰi˧-lɑ˧˥} & \ipa{tʰɑ˧-lɑ˧˥} & \ipa{mɤ˧-lɑ˧˥}\\
\lspbottomrule
\end{tabularx}
\end{table}

Following an M-tone \is{prefixes}prefix, the lexical tones M, H, L, and MH surface in a~straightforward manner. The
\is{subcategories of lexical tones}subcategories M\textsubscript{a} and M\textsubscript{b} are neutralized in this context, as are L\textsubscript{a} and L\textsubscript{b}.

\tabref{tab:accomplishedpfvcompletion} sets out the tone patterns associated with the \textsc{accomplished} \is{prefixes}prefix /\ipa{le˧}-/ in combination with the \textsc{perfective} \mbox{/\ipa{-ze˧}/} and the morpheme indicating \textsc{completion}, \mbox{/\ipa{-se˩}/}. \tabref{tab:accomplishedcauscertitudecop} lists the patterns that obtain when the verb is followed by the \textsc{causative} /\ipa{-tsæ˧}/ and
\textsc{certitude} morpheme /\ipa{-ɲi˩}/. 

\begin{table}%[t]
\caption{\label{tab:accomplishedpfvcompletion}Tone patterns in constructions consisting of verbs with the \textsc{accomplished} M-tone prefix and a~\textsc{perfective} or \textsc{completion} morpheme.}
\begin{tabularx}{\textwidth}{ l@{\hspace{6mm}} l l l Q Q }
\lsptoprule
	tone & example & meaning & \textsc{accomp} & \textsc{accomp}+V+\textsc{pfv} & \textsc{accomp}+V+{\allowbreak}\textsc{completion}\\ \midrule
	H & \ipa{dzɯ˥} & to eat & \ipa{le˧-dzɯ˥} & \ipa{le˧-dzɯ˥-ze˩} & \ipa{le˧-dzɯ˧-se˥}\\
	M\textsubscript{a} & \ipa{hwæ˧\textsubscript{a}} & to buy & \ipa{le˧-hwæ˧} & \ipa{le˧-hwæ˧-ze˧} & \ipa{le˧-hwæ˧-se˩}\\
	M\textsubscript{b} & \ipa{tɕʰi˧\textsubscript{b}} & to sell & \ipa{le˧-tɕʰi˧} & \ipa{le˧-tɕʰi˧-ze˧} & \ipa{le˧-tɕʰi˧-se˩}\\
	M\textsubscript{c} & \ipa{bi˧\textsubscript{c}} & to go & \ipa{le˧-bi˩} & \ipa{le˧-bi˩-ze˩} & \ipa{le˧-bi˩-se˩}\\
	L\textsubscript{a} & \ipa{bæ˩\textsubscript{a}} & to sweep & \ipa{le˧-bæ˩} & \ipa{le˧-bæ˩-ze˩} & \ipa{le˧-bæ˩-se˩}\\
	L\textsubscript{b} & \ipa{ʐwɤ˩\textsubscript{b}} & to speak & \ipa{le˧-ʐwɤ˩} & \ipa{le˧-ʐwɤ˩-ze˩} & \ipa{le˧-ʐwɤ˩-se˩}\\
	MH & \ipa{lɑ˧˥} & to strike & \ipa{le˧-lɑ˧˥} & \ipa{le˧-lɑ˧-ze˥} & \ipa{le˧-lɑ˧-se˥}\\
\lspbottomrule
\end{tabularx}
\end{table}

\begin{table}%[t]
\caption{\label{tab:accomplishedcauscertitudecop}Tone patterns in constructions consisting of verbs with the \textsc{accomplished} M-tone prefix and \textsc{causative}+\textsc{certitude} morphemes.}
\begin{tabularx}{\textwidth}{ l l l l Q }
\lsptoprule
	tone & example & meaning & \textsc{accomp} & \textsc{accomp}+V+\textsc{caus}+\textsc{certitude}\\ \midrule
	H & \ipa{dzɯ˥} & to eat & \ipa{le˧-dzɯ˥} & \ipa{le˧-dzɯ˧-tsæ˥-ɲi˩}\\
	M\textsubscript{a} & \ipa{hwæ˧\textsubscript{a}} & to buy & \ipa{le˧-hwæ˧} & \ipa{le˧-hwæ˧-tsæ˧-ɲi˩}\\
	M\textsubscript{b} & \ipa{tɕʰi˧\textsubscript{b}} & to sell & \ipa{le˧-tɕʰi˧} & \ipa{le˧-tɕʰi˧-tsæ˧-ɲi˩}\\
	M\textsubscript{c} & \ipa{pv̩˧\textsubscript{c}} & to chant & \ipa{le˧-pv̩˧} & \ipa{le˧-pv̩˧-tsæ˧-ɲi˩}\\
	L\textsubscript{a} & \ipa{bæ˩\textsubscript{a}} & to sweep & \ipa{le˧-bæ˩} & \ipa{le˧-bæ˩-tsæ˩-ɲi˩}\\
	L\textsubscript{b} & \ipa{ʐwɤ˩\textsubscript{b}} & to speak & \ipa{le˧-ʐwɤ˩} & \ipa{le˧-ʐwɤ˩-tsæ˩-ɲi˩}\\
	MH & \ipa{lɑ˧˥} & to strike & \ipa{le˧-lɑ˧˥} & \ipa{le˧-lɑ˧-tsæ˥-ɲi˩}\\
\lspbottomrule
\end{tabularx}
\end{table}

The \textsc{certitude} morpheme is invariantly low in \tabref{tab:accomplishedcauscertitudecop}, in keeping with its lexical L tone. This confirms the absence of a \is{floating tone}floating H tone in any of the preceding expressions, as a floating tone would be expected to land on the \textsc{certitude} morpheme. This is important information for arriving at the underlying tone of expressions with surface M tones, such as /\ipa{le˧-hwæ˧}/ and
/\ipa{le˧-tɕʰi˧}/~-- hypothesizing that if they carried a~\is{floating tone}floating H tone, the following morpheme, /\ipa{-ɲi˩}/, would surface with H
tone. 

The data in \tabref{tab:accomplishedcauscertitudecop} also provides further illustration of the behaviour of MH and H tones, exemplified by
/\ipa{le˧-lɑ˧˥}/ and /\ipa{le˧-dzɯ˥}/, respectively. For MH-tone verbs in this construction, the M part remains on the syllable to which the tone is lexically associated, and the H part is projected onto the next syllable (the {causative}). This is obligatorily followed by L tone on the \textsc{certitude} \is{suffixes}suffix, due to Rule~4: “The syllable following an H-tone syllable receives L tone”. In the case of H-tone verbs, the H tone no longer appears on the morpheme to which it is lexically associated; instead, it associates to the \textsc{causative}, resulting in a surface phonological output that is identical to that for MH tone.

A related set of facts is presented in \tabref{tab:tonepatternsofthevnegvconstruction}: it concerns the V-\textsc{neg}-V construction, as illustrated in example (\ref{ex:whetheritisactuallythecasewedontknow}).\footnote{Two variants are possible: one in which the sequence forms a single {tone group}, as in
(\ref{ex:whetheritisactuallythecasewedontknow}), and another in which it is divided into two groups: /\ipa{ɲi˩˥ {\kern1pt}|{\kern1pt}
mɤ˧-ɲi˩, {\kern1pt}|{\kern1pt} mɤ˧-ɳv̩˥}!/, with the same meaning and morphemic composition as (\ref{ex:whetheritisactuallythecasewedontknow}). In the latter case,
the first syllable has the same tone as in isolation, and the negated form follows the tonal patterns described in
\tabref{tab:thetonesofverbsinassociationwithamtoneprefix} above. These V-\textsc{neg}-V constructions are typically followed by /\ipa{mɤ˧-ɳv̩˥}/
‘[I/we] don’t know’ or /\ipa{mɤ˧-do˩}/ ‘[we] don’t know/can’t see for ourselves’. Sometimes,
the V \textsc{neg}-V portion undergoes {focalization}, as in /\ipa{hwæ˧ mɤ˧-hwæ˧ F {\kern2pt}|{\kern2pt}
mɤ˧-do˩}/ ‘[I/we] don’t know whether [they] bought [it/some] or not’. (For a~discussion of intonational {focalization}, transcribed as ‘F’, see \sectref{sec:focalization}.) In addition to the elicited data in \tabref{tab:tonepatternsofthevnegvconstruction}, further examples appear in texts, e.g.~in
	\textit{BuriedAlive3.133} \pandoi{0004539\#S133}, \textit{Seeds2.85} \pandoi{0004543\#S85}, and \textit{Dog.59} \pandoi{0004442\#S59}.}

\begin{exe}
  \ex
  \label{ex:whetheritisactuallythecasewedontknow}
  \ipaex{ɲi˩ mɤ˥-ɲi˩, {\kern2pt}|{\kern2pt} mɤ˧-ɳv̩˥!}\\
  \gll ɲi˩	mɤ˧-	ɲi˩	mɤ˧-	ɳv̩˥\\
  \textsc{cop}	\textsc{neg}	\textsc{cop}	\textsc{neg}	to\_know/to\_get\_to\_know\\
  \glt ‘Whether it is actually the case{\dots} we don’t know!’ \textit{Context:} two persons discuss what
  a~third person has said.
\end{exe}

% \clearpage % removed on April 11th, 2025
\begin{table}%[t]
\caption{\label{tab:tonepatternsofthevnegvconstruction}Tone patterns of the V-\textsc{neg}-V construction.}
\begin{tabularx}{\textwidth}{ l@{\hspace{7mm}} l@{\hspace{7mm}} l@{\hspace{5mm}} Q Q }
\lsptoprule
	tone & example & meaning & V \textsc{neg} V & V | \textsc{neg} V\\ \midrule
	H & \ipa{dzɯ˥} & to eat & \ipa{dzɯ˧ mɤ˧-dzɯ˧} & \ipa{dzɯ˧ {\kern2pt}|{\kern2pt} mɤ˧-dzɯ˥}\\
	M\textsubscript{a} & \ipa{hwæ˧\textsubscript{a}} & to buy & \ipa{hwæ˧ mɤ˧-hwæ˧} & \ipa{hwæ˧ {\kern2pt}|{\kern2pt} mɤ˧-hwæ˧}\\
	M\textsubscript{b} & \ipa{tɕʰi˧\textsubscript{b}} & to sell & \ipa{tɕʰi˧ mɤ˧-tɕʰi˧} & \ipa{tɕʰi˧ {\kern2pt}|{\kern2pt} mɤ˧-tɕʰi˧}\\
	M\textsubscript{c} & \ipa{pv̩˧\textsubscript{c}} & to chant & \ipa{pv̩˧ mɤ˧-pv̩˧} & \ipa{pv̩˧ {\kern2pt}|{\kern2pt} mɤ˧-pv̩˧}\\
	L\textsubscript{a} & \ipa{gɯ˩\textsubscript{a}} & to be true & \ipa{gɯ˩ mɤ˩-gɯ˥} & \ipa{gɯ˩˥ {\kern2pt}|{\kern2pt} mɤ˧-gɯ˩}\\
	L\textsubscript{b} & \ipa{ʐwɤ˩\textsubscript{b}} & to speak & \ipa{ʐwɤ˩ mɤ˥-ʐwɤ˩} & \ipa{ʐwɤ˩˥ {\kern2pt}|{\kern2pt} mɤ˧-ʐwɤ˩}\\
	MH & \ipa{lɑ˧˥} & to strike & \ipa{lɑ˧ mɤ˥-lɑ˩} & \ipa{lɑ˧˥ {\kern2pt}|{\kern2pt} mɤ˧-lɑ˧˥}\\
\lspbottomrule
\end{tabularx}
\end{table}

  
As noted at the beginning of this section, the \is{prefixes}prefix /\ipa{mv̩˧}-/, conveying \textsc{imminence}, is
infrequent: only one example was found in the first twenty-five transcribed narratives. The
consultant was not comfortable pairing this \is{prefixes}prefix with verbs to form a~disyllabic expression and instead proposed the three constructions (\ref{ex:rightaway}), (\ref{ex:durimmin}), and (\ref{ex:accompneg}). The results are shown in \tabref{tab:tonepatternsfortheprefixandarelatedconstruction}.

\begin{exe}
	\ex
	\label{ex:rightaway}
	\ipaex{mv̩˧-V-bi˧}\\
	\gll mv̩˧-	V		-bi˧\\
	\textsc{imminence}			\textit{target~verb}		\textsc{imm\_fut}\\
	\glt ‘will V right away’
\end{exe}

\begin{exe}
	\ex
	\label{ex:durimmin}
	\ipaex{tʰi˧-mv̩˧-V}\\
	\gll tʰi˧-		mv̩˧-	V\\
	\textsc{dur}		\textsc{imminence}		\textit{target~verb}\\
	\glt ‘is going to V up’
\end{exe}

\begin{exe}
	\ex
	\label{ex:accompneg}
	\ipaex{le˧-mɤ˧-V}\\
	\gll le˧-		mɤ˧-	V\\
	\textsc{accomp}		\textsc{neg}		\textit{target~verb}\\
	\glt ‘does not V / not to V’
\end{exe}

The outcomes for H- and MH-tone verbs are far from trivial. As with other systematically elicited
combinations, it seems prudent to withhold interpretation until further confirmation can be obtained, ideally from narratives.

\begin{table}
  \caption{\label{tab:tonepatternsfortheprefixandarelatedconstruction}Tone patterns of verbs with the prefix /\ipa{mv̩˧}-/, and an antonymic construction.}
{\setlength\tabcolsep{4.5pt}
\begin{tabularx}{\textwidth}{ l l l Q Q Q }
\lsptoprule
	tone & example & meaning & \ipa{mv̩˧-}V\ipa{-bi˧} & \ipa{tʰi˧-mv̩˧-}V & \ipa{le˧-mɤ˧-}V\\ \midrule
	H & \ipa{dzɯ˥} & to eat & \ipa{mv̩˧-dzɯ˧-bi˧} & \ipa{tʰi˧-mv̩˧-dzɯ˧} & \ipa{le˧-mɤ˧-dzɯ˧}\\
	M\textsubscript{a} & \ipa{hwæ˧\textsubscript{a}} & to buy & \ipa{mv̩˧-hwæ˧-bi˧} & \ipa{tʰi˧-mv̩˧-hwæ˧} & \ipa{le˧-mɤ˧-hwæ˧}\\
	M\textsubscript{b} & \ipa{tɕʰi˧\textsubscript{b}} & to sell & \ipa{mv̩˧-tɕʰi˧-bi˧} & \ipa{tʰi˧-mv̩˧-tɕʰi˧} & \ipa{le˧-mɤ˧-tɕʰi˧}\\
	M\textsubscript{c} & \ipa{pv̩˧\textsubscript{c}} & to chant & \ipa{mv̩˧-pv̩˧-bi˧} & \ipa{tʰi˧-mv̩˧-pv̩˧} & \ipa{le˧-mɤ˧-pv̩˧}\\
	L\textsubscript{a} & \ipa{bæ˩\textsubscript{a}} & to sweep & \ipa{mv̩˧-bæ˧-bi˩} & \ipa{tʰi˧-mv̩˧-bæ˩} & \ipa{le˧-mɤ˧-bæ˩}\\
	L\textsubscript{b} & \ipa{ʐwɤ˩\textsubscript{b}} & to speak & \ipa{mv̩˧-ʐwɤ˧-bi˩} & \ipa{tʰi˧-mv̩˧-ʐwɤ˩} & \ipa{le˧-mɤ˧-ʐwɤ˩}\\
	MH & \ipa{lɑ˧˥} & to strike & \ipa{mv̩˧-lɑ˩-bi˩} & \ipa{tʰi˧-mv̩˧-lɑ˧˥} & \ipa{le˧-mɤ˧-lɑ˧˥}\\
\lspbottomrule
\end{tabularx}}
\end{table}


\subsection{The interrogative prefix}
\label{sec:ltoneprefixes}
%

The yes/no interrogative (polar interrogative) is provisionally analyzed as carrying an L tone: /\ipa{ə˩}-/, though not without reservations, as discussed below. This interrogative prefix is
segmentally bleached, consisting of a~neutral vowel that undergoes strong regressive \isi{vowel harmony}. Tonally, on the other hand, it follows distinct patterns, differing from both M-tone prefixes and the L-tone directional prefixes presented in §\ref{sec:themarkingofspatialorientationonverbs}. This suggests that it has a tonal specification of its own. \tabref{tab:thetonesofverbsinassociationwithaltoneprefix} sets out the facts.

\begin{table}
\caption{\label{tab:thetonesofverbsinassociationwithaltoneprefix}The tones of verbs in association with an L-tone prefix.}
\begin{tabularx}{.8\textwidth}{ Q Q Q Q }
\lsptoprule
	tone & example & meaning & interrogative\\ \midrule
	H & \ipa{dzɯ˥} & to eat & \ipa{ə˧-dzɯ˥}\\
	M\textsubscript{a} & \ipa{hwæ˧\textsubscript{a}} & to buy & \ipa{ə˧-hwæ˥}\\
	M\textsubscript{b} & \ipa{tɕʰi˧\textsubscript{b}} & to sell & \ipa{ə˧-tɕʰi˥}\\
	M\textsubscript{c} & \ipa{pv̩˧\textsubscript{c}} & to chant & \ipa{ə˧-pv̩˥}\\
	L\textsubscript{a} & \ipa{bæ˩\textsubscript{a}} & to sweep & \ipa{ə˩-bæ˩}\\
	L\textsubscript{b} & \ipa{ʐwɤ˩\textsubscript{b}} & to speak & \ipa{ə˩-ʐwɤ˩}\\
	MH & \ipa{lɑ˧˥} & to strike & \ipa{ə˧-lɑ˥}\\
\lspbottomrule
\end{tabularx}
\end{table}

In the first edition of this book, the tone patterns for M-tone verbs were misanalyzed as L.M: {\ddagger}\ipa{ə˩-hwæ˧} `buy?', {\ddagger}\ipa{ə˩-tɕʰi˧} `sell?', and {\ddagger}\ipa{ə˩-pv̩˧} `chant?'. Recognition of the M.H pattern was delayed by the hypothesis that the high pitch of the verb in this context was due to an intonational factor. In view of the cross-linguistic tendency for interrogative sentences to be associated with raised pitch, the phonetic realization (phonetic [M.H], in the upper half of the speaker's pitch range) was initially attributed to an intonational modification of an underlying /L.M/ sequence. It was eventually established, however, that the tone patterns for H-tone, MH-tone and M-tone verbs in this context were identical. The main consultant was able to match these patterns with the M.H tone pattern of disyllabic nouns (as in /\ipa{hwæ˧tsɯ˥}/ `rat'), despite the considerable syntactic difference between an interrogative construction and a disyllabic noun.

Under the hypothesis that the interrogative prefix carries a lexical L tone, its realization before L-tone verbs is straightforward but its realization before \mbox{M-,} \mbox{H-,} and MH-tone verbs calls for an explanation: in these contexts, the \is{prefixes}prefix surfaces with an M tone (and the verb carries H). The sequences /L.M/, /L.H/, and /L.MH/ are all phonotactically permissible in Yongning Na, so there is no phonological constraint preventing an L-tone interrogative prefix from surfacing with its expected tone in these contexts. One would thus expect forms such as $\dagger${\kern2pt}\ipa{ə˩-lɑ˧˥} (‘does (s)he strike?’), yet this pattern is not observed. In light of this, the hypothesis that the interrogative prefix carries a lexical L tone is not particularly compelling.
 
Differences in the tonal realization of a prefix depending on the tone of the following verb constitute a~puzzle in view of the scarcity of cases of regressive tonal modification (modification of a~tone by that of a~\textit{following} morpheme) in the Alawua dialect of Yongning Na. In this variety, tone modification is predominantly progressive (unlike vowel harmony, which operates regressively: see Appendix A, \sectref{sec:anoteonvowelharmony}). That said, a~schwa \is{prefixes}prefix is a prime candidate for modification by a following tone, given its phonetic and phonological lightness. The phonetic realization of schwa prefixes warrants an \is{experimental phonetics}experimental phonetic study, focusing on parameters such as \isi{duration}, fundamental frequency, and formant structure. Synchronic patterns of phonetic \isi{variation} may offer clues as to how the current \is{morphotonology}morphotonological patterns of these prefixes developed.


\subsection{The marking of spatial orientation on verbs}
\label{sec:themarkingofspatialorientationonverbs}

Extensive marking of orientation on verbs is found among \ili{Na-Qiangic} languages. In particular, \ili{rGyalrongic} languages “have a~whole array of verbal orientation prefixes, which are
obligatorily present on all perfective and {imperative} verb forms” (\citealt[180]{sun2000a}; see also \citealt{lin2002},
and, on Tangut, \citealt{jacques2011b} and \citealt{Beaudouin2023_Tangut_Horpa}). J.\ Sun describes three distinct pairs of directions in \ili{rGyalrongic}: eastward (i.e.\ in the
direction of the rising sun) vs.\ westward; upstream vs.\ downstream; and uphill (upward) vs.\ downhill
(downward). The system found in \ili{Shixing} (Xumi) comprises two productive pairs of (non-obligatory)
orientation prefixes, only one of which corresponds semantically with the \ili{rGyalrongic} system: upward
vs.\ downward. The other opposition is inward vs.\ outward. \ili{Shixing} also retains traces of a~third pair,
hither vs.\ thither, which appears in a dedicated construction meaning ‘to V back and forth’
\citep{chirkova2009}.

This striking feature is sometimes awarded the status of criterion for phylogenetic classification, e.g.~in
proposals by \citet[105]{matisoff2004}. However, cross-dialect and cross-language variation show that
orientation systems are no less prone to change than other structural features of a~language. If directional prefixes have great historical depth in \il{Sino-Tibetan}Sino-Tibetan,
\ili{Naish} must be hypothesized to have lost them. In \ili{Naish}, topographically-based spatial
deixis is not marked by an obligatory verbal \is{prefixes}prefix. Indications of orientation, such as /\ipa{mv̩˩tɕo˧}/ ‘downward’ in (\ref{ex:downwd}), are better described as orientation adverbials.
 
\begin{exe}
	\ex
	\label{ex:downwd}
	\ipaex{mv̩˩tɕo˧ mɤ˧-hɯ˧}\\
	\gll mv̩˩tɕo˧	mɤ˧-	hɯ˧\textsubscript{c}\\
		downward	\textsc{neg}		to\_go.\textsc{pst}\\
	\glt ‘[The dog, who had come to sit on the wooden platform close to the fire pit, where dogs are not allowed] did not/would not get down!’ (\textit{Context:} 	a~discussion about Sister3.22. Field notes.)
\end{exe}

The only \is{monosyllables}monosyllabic indications of orientation in common use that could plausibly be considered prefixes are /\ipa{gɤ˩}-/ ‘upward’ and /\ipa{mv̩˩}-/ ‘downward’. For instance, /\ipa{ʂo˥}/ ‘to
reap, to gather in’ can combine with /\ipa{gɤ˩}-/ ‘upward’ to mean ‘to reap in, to
bring back to the house and into the granary’: /\ipa{gɤ˩-ʂo˥}/. These \is{monosyllables}monosyllabic prefixes also
occur in set constructions such as /\ipa{gɤ˩-V} {\kern2pt}|{\kern2pt} \ipa{mv̩˩-V}/ ‘to V in all directions’, as illustrated in (\ref{ex:blows}):

\begin{exe}
  \ex
  \label{ex:blows}
  \ipaex{ʈʂʰɯ˧ne˧-ʝi˥ {\kern2pt}|{\kern2pt} gɤ˩-dɑ˧˥, {\kern2pt}|{\kern2pt} mv̩˩-dɑ˧˥, {\kern2pt}|{\kern2pt} gɤ˩-dɑ˧˥, {\kern2pt}|{\kern2pt} mv̩˩-dɑ˧˥, {\kern2pt}|{\kern2pt} ({\dots}) ɖɯ˧-so˩ ʂɯ˩
  ʝi˩ tsɯ˩ {\kern2pt}|{\kern2pt} mv̩˩! {\kern2pt}|}\\
  \gll ʈʂʰɯ˧ne˧-ʝi˥	gɤ˩-	dɑ˧˥	mv̩˩-		dɑ˧˥	ɖɯ˧-so˩ ʂɯ˩		ʝi˥	tsɯ˧˥	mv̩˧\\
  thus		upward	to\_strike	downward	to\_strike	several	times		to\_do	\textsc{rep}
  \textsc{affirm}\\
  \glt ‘He would give blows high and low (= hither and thither), again and again!~/ He would strike
  blows in all directions, again and again!’  \textit{Context:} an exorcist is performing
  a~ritual. \textit{(Healing.38)} \pandoi{0004540\#S38}
\end{exe}

In this context, the prefixes retain some of their literal meaning of ‘upward’ and
‘downward’: the exorcist’s blows with his sword are aimed high, then low (close to the ground), then high again, and so on. Yet in this construction, the spatial indications take on a~broader meaning, summoning
up the swift, dance-like movements of the exorcist as he battles an invisible cohort of demons
surrounding him. Repetition of the prefixed verb (/\ipa{gɤ˩-dɑ˧˥}, {\kern2pt}|{\kern2pt} \ipa{mv̩˩-dɑ˧˥}, {\kern2pt}|{\kern2pt}
\ipa{gɤ˩-dɑ˧˥}, {\kern2pt}|{\kern2pt} \ipa{mv̩˩-dɑ˧˥}/) contributes to weakening the literal indication of
spatial orientation, yielding a~sense of ‘in all directions’ rather than ‘up and down’.

When an orientation \is{prefixes}prefix is separated from the verb by other prefixes, it can form a~tone
group on its own, as in (\ref{ex:afteronehasprayed}). In this example, the \is{prefixes}prefix's //L// tone surfaces as /LH/ due to the prohibition of all-L tone groups. A~{phonological rule} referred to as Rule~7 (see \sectref{sec:asummaryoftonetosyllableassociationrules}) repairs all-L tone groups by addition of a~postlexical H tone to their last syllable. In the first \isi{tone group} of example (\ref{ex:afteronehasprayed}), this last syllable also happens to be the \textit{first} syllable.
%
\begin{exe}
  \ex
  \label{ex:afteronehasprayed}
  \ipaex{gɤ˩˥ {\kern2pt}|{\kern2pt} le˧-ʈʂʰo˧-se˥-dʑo˩ {\kern2pt}|{\kern2pt} tʰi˩˥ {\dots}}\\
  \gll gɤ˩-		le˧-		ʈʂʰo˥	-se˩		-dʑo˥	tʰi˩˥\\
  upward		\textsc{accomp}	to\_pray	\textsc{completion}	\textsc{top}	then\\
  \glt ‘after one has prayed [\textit{literally}: prayed up (to the ancestors)]’ \textit{(Dog2.54)} \pandoi{0004555\#S54}
\end{exe}

Judging from available texts, the {monosyllabic} form /\ipa{gɤ˩}-/ ‘upward’ is more
frequent than /\ipa{mv̩˩}-/ ‘downward’. The two prefixes /\ipa{gɤ˩}-/ ‘upward’ and /\ipa{mv̩˩}-/ ‘downward’ have the same tonal behaviour.
However, the disyllabic expressions /\ipa{gɤ˩tɕo˧}/
‘upward’ and /\ipa{mv̩˩tɕo˧}/ ‘downward’ are generally preferred. Here, the monosyllabic forms are tentatively
classified as prefixes (and transcribed accordingly, with a~following hyphen), while the disyllabic forms
are labelled as adverbials. However, in the absence of clear language-internal criteria distinguishing the two, the divide
between these categories may not be as sharp as this choice of terms suggests. 

From a~tonal point of view, orientation prefixes typically belong to the same \isi{tone group} as the following verb, except in uncommon cases such as (\ref{ex:afteronehasprayed}). On the other hand, orientation adverbials often
constitute a~separate \isi{tone group}, as discussed in~\sectref{sec:someelementsalwaysconstituteatonegroupontheirown}. Tables~\ref{tab:spatialmonoFULL} to \ref{tab:spatialdiFULLLUPDOWN} present cases where the orientation \is{prefixes}prefix or {adverbial} is integrated into the same \isi{tone group} as the verb. The \textsc{perfective} \is{suffixes}suffix /-\ipa{ze˧}/ is used to determine whether the verb carries H tone (which causes the \is{suffixes}suffix to lower to L), M tone (which leaves the \is{suffixes}suffix unaffected), or MH tone (which results in the association of the H part of the \is{tonal contour}contour to the \is{suffixes}suffix). 

For the sake of clarity, the surface forms are provided in full, even though in many cases the tonal behaviour of the \textsc{perfective} \is{suffixes}suffix \mbox{/\ipa{-ze˧}/} follows straightforwardly from the tone pattern of the non-suffixed expression through the application of the seven phonological tone rules recapitulated in \sectref{sec:alistoftonerules}. For
instance, an L.H sequence can only be followed by L, by virtue of Rule~4 (“A syllable following an H-tone syllable receives L tone''), so the L.H expression /\ipa{gɤ˩-se˥}/ ‘to walk up(ward)’ predictably yields L.H+L in (\ref{ex:walkedup}).

\begin{exe}
	\ex
	\label{ex:walkedup}
	\ipaex{gɤ˩-se˥-ze˩}\\
	\gll gɤ˩-		se˥		-ze˧\\
	upward		to\_walk	\textsc{pfv}\\
	\glt ‘walked up(ward)’
\end{exe}

The original data is found in the online document \textit{SpatialOrientation} \pandoi{0004559}. The verb /\ipa{bi˧\textsubscript{c}}/ ‘to go’ was inadvertently omitted in the recording. This verb can combine with the disyllabic
orientation adverbials, but not with {monosyllabic} /\ipa{gɤ˩}-/ ‘upward’ and /\ipa{mv̩˩}-/
‘downward’. The data is shown in (\ref{ex:goforward})-(\ref{ex:goupward}). Note that the \is{variants}variant $\ddagger${\kern2pt}\ipa{jo˩lo˩ bi˥} for (\ref{ex:goright}) was rejected.

\begin{exe}
	\ex
	\label{ex:goforward}
	\ipaex{ʁo˧dɑ˧ bi˧(-ze˧)}\\
	\gll ʁo˧dɑ˧		bi˧\textsubscript{c}		-ze˧\\
	forward		to\_go		\textsc{pfv}\\
	\glt ‘to go forward’
\end{exe}

\begin{exe}
	\ex
	\label{ex:goleft}
	\ipaex{ʁwæ˧gi˧ bi˧(-ze˧)}\\
	\gll ʁwæ˧gi˧		bi˧\textsubscript{c}		-ze˧\\
	leftward		to\_go		\textsc{pfv}\\
	\glt ‘to go to the left’
\end{exe}

\begin{exe}
	\ex
	\label{ex:goright}
	\ipaex{jo˩lo˩ bi˩}\\
	\gll jo˩lo˩		bi˧\textsubscript{c}\\
	rightward		to\_go\\
	\glt ‘to go to the right’
\end{exe}

\begin{exe}
	\ex
	\label{ex:gobackward}
	\ipaex{ʁo˧tʰo˩ bi˩}\\
	\gll ʁo˧tʰo˩	bi˧\textsubscript{c}\\
	backward		to\_go\\
	\glt ‘to go backward’
\end{exe}

\begin{exe}
	\ex
	\label{ex:goupward}
	\ipaex{gɤ˩tɕo˧ bi˧(-ze˧)}\\
	\gll gɤ˩tɕo˧	bi˧\textsubscript{c}		-ze˧\\
	upward		to\_go		\textsc{pfv}\\
	\glt ‘to go upward’
\end{exe}

%‘Leftward’ and ‘rightward’ are expressed by disyllabic /\ipa{ʁwæ˧gi\#˥}/ or /\ipa{ʁwæ˧lo˥}/ ‘to the
%left’, and /\ipa{jo˩gi˩}/ or /\ipa{jo˩lo˩}/ ‘to the right’. 

%The following subtables of \tabref{tab:spatialFULL} set out the data for the adverbials /\ipa{ɬo˧tɑ˧}/ ‘to the side’, /\ipa{ʁwæ˧gi\#˥}/ and /\ipa{ʁwæ˧lo˥}/ ‘to the
%left’, /\ipa{jo˩gi˩}/ and /\ipa{jo˩lo˩}/ ‘to the right’, /\ipa{ʁo˧dɑ˧}/
%‘forward, to the front’ and /\ipa{ʁo˧tʰo˩}/ ‘backward, to the back’.

%The example verbs used are /\ipa{se˥}/ ‘to walk’, /\ipa{li˧\textsubscript{a}}/ ‘to look’, /\ipa{tsi˧\textsubscript{b}}/ ‘to set, to install’, /\ipa{bi˧\textsubscript{c}}/
%‘to go’, /\ipa{kwɤ˩\textsubscript{a}}/ ‘to throw’, /\ipa{ɻ̩˩\textsubscript{b}}/ ‘to turn’, and /\ipa{mi˧˥}/
%‘to push’. 


	% \label{tab:spatialFULL}  %% Commented out on April 30th, 2025: no subtables, only sequentially numbered tables in the entire volume.
	\begin{table}%[h!]
		\caption{\label{tab:spatialmonoFULL}The tonal behaviour of verbs after indications of spatial orientation: {monosyllabic}
			prefixes.}
		\begin{tabularx}{\textwidth}{ l@{\hspace{2mm}} l@{\hspace{2mm}} l@{\hspace{2mm}} Q@{\hspace{2mm}} Q@{\hspace{1mm}} }
		\lsptoprule
		tone & example & meaning & ‘upward’ \is{prefixes}prefix & ‘downward’ \is{prefixes}prefix\\ \midrule
		%Use of kerning to obtain splitting of the contents of the last two columns over two lines: more symmetrical to the eye.
		H & \ipa{se˥} & to walk & \ipa{gɤ˩-se˥, gɤ˩-se˥-ze˩} & \ipa{mv̩˩-se˥, mv̩˩-se˥-ze˩}\\
		M\textsubscript{a} & \ipa{li˧\textsubscript{a}} & to look & \ipa{gɤ˩-li˧, gɤ˩-li˧-ze˧} & \ipa{mv̩˩-li˧, mv̩˩-li˧-ze˧}\\
		M\textsubscript{b} & \ipa{tsi˧\textsubscript{b}} & to set & \ipa{gɤ˩-tsi˧, gɤ˩-tsi˧-ze˧} & \ipa{mv̩˩-tsi˧, mv̩˩-tsi˧-ze˧}\\
		L\textsubscript{a} & \ipa{kwɤ˩\textsubscript{a}} & to throw & \ipa{gɤ˩-kwɤ˥, gɤ˩-kwɤ˥-ze˩} & \ipa{mv̩˩-kwɤ˥, mv̩˩-kwɤ˥-ze˩}\\
		L\textsubscript{b} & \ipa{ɻ̩˩\textsubscript{b}} & to turn & \ipa{gɤ˩-ɻ̩˥, gɤ˩-ɻ̩˥-ze˩} & \ipa{mv̩˩-ɻ̩˥, mv̩˩-ɻ̩˥-ze˩}\\
		MH & \ipa{mi˧˥} & to push & \ipa{gɤ˩-mi˧˥, gɤ˩-mi˧-ze˥} & \ipa{mv̩˩-mi˧˥, mv̩˩-mi˧-ze˥}\\
		\lspbottomrule
		\end{tabularx}
	\end{table}
	
	\begin{table}
		\caption{\label{tab:spatialdiFULLFRONT}The tonal behaviour of verbs after indications of spatial orientation: {adverbial} //\ipa{ɬo˧tɑ˧}// ‘to the side’.}
		\begin{tabularx}{\textwidth}{ l@{\hspace{12mm}} l@{\hspace{12mm}} l@{\hspace{12mm}} Q }
			\lsptoprule
			tone & example & meaning & with {adverbial} ‘to the side’\\ \midrule
			%Use of kerning to obtain splitting of the contents of the last two columns over two lines: more symmetrical to the eye.
			H & \ipa{se˥} & to walk & \ipa{ɬo˧tɑ˧ se˧, ɬo˧tɑ˧ se˧-ze˩}\\
			M\textsubscript{a} & \ipa{li˧\textsubscript{a}} & to look & \ipa{ɬo˧tɑ˧ li˧, ɬo˧tɑ˧ li˧-ze˧}\\
			M\textsubscript{b} & \ipa{tsi˧\textsubscript{b}} & to set & \ipa{ɬo˧tɑ˧ tsi˧, ɬo˧tɑ˧ tsi˧-ze˩}\\
			L\textsubscript{a} & \ipa{kwɤ˩\textsubscript{a}} & to throw & \ipa{ɬo˧tɑ˧ kwɤ˩, ɬo˧tɑ˧ kwɤ˩-ze˩}\\
			L\textsubscript{b} & \ipa{ɻ̩˩\textsubscript{b}} & to turn & \ipa{ɬo˧tɑ˧ ɻ̩˩, ɬo˧tɑ˧ ɻ̩˩-ze˩}\\
			MH & \ipa{mi˧˥} & to push & \ipa{ɬo˧tɑ˧ mi˧˥, ɬo˧tɑ˧ mi˧-ze˥}\\
			\lspbottomrule
		\end{tabularx}
		\end{table}

		\begin{table}%[h!]
		\caption{\label{tab:spatialdiFULLLFRONTBACK}The tonal behaviour of verbs after indications of spatial orientation: adverbials //\ipa{ʁo˧dɑ˧}// ‘forward’ and //\ipa{ʁo˧tʰo˩}// ‘backward’.}
		\begin{tabularx}{\textwidth}{ l@{\hspace{7mm}} l@{\hspace{7mm}} l@{\hspace{7mm}} Q Q }
			\lsptoprule
			tone & example & meaning & ‘forward’ & ‘backward’\\ \midrule
			%Use of kerning to obtain splitting of the contents of the last two columns over two lines: more symmetrical to the eye.
			H & \ipa{se˥} & to walk & \ipa{ʁo˧dɑ˧ se˧,{\kern8pt} ʁo˧dɑ˧ se˧-ze˩} & \ipa{ʁo˧tʰo˩ se˩,{\kern8pt} ʁo˧tʰo˩ se˩-ze˩}\\
			M\textsubscript{a} & \ipa{li˧\textsubscript{a}} & to look & \ipa{ʁo˧dɑ˧ li˧,{\kern8pt} ʁo˧dɑ˧ li˧-ze˧} & \ipa{ʁo˧tʰo˩ li˩,{\kern8pt} ʁo˧tʰo˩ li˩-ze˩}\\
			M\textsubscript{b} & \ipa{tsi˧\textsubscript{b}} & to set & \ipa{ʁo˧dɑ˧ tsi˧,{\kern8pt} ʁo˧dɑ˧ tsi˧-ze˩} & \ipa{ʁo˧tʰo˩ tsi˩,{\kern8pt} ʁo˧tʰo˩ tsi˩-ze˩}\\
			L\textsubscript{a} & \ipa{kwɤ˩\textsubscript{a}} & to throw & \ipa{ʁo˧dɑ˧ kwɤ˩,{\kern8pt} ʁo˧dɑ˧ kwɤ˩-ze˩} & \ipa{ʁo˧tʰo˩ kwɤ˩,{\kern8pt} ʁo˧tʰo˩ kwɤ˩-ze˩}\\
			L\textsubscript{b} & \ipa{ɻ̩˩\textsubscript{b}} & to turn & \ipa{ʁo˧dɑ˧ ɻ̩˩,{\kern14pt} ʁo˧dɑ˧ ɻ̩˩-ze˩} & \ipa{ʁo˧tʰo˩ ɻ̩˩,{\kern14pt} ʁo˧tʰo˩ ɻ̩˩-ze˩}\\
			MH & \ipa{mi˧˥} & to push & \ipa{ʁo˧dɑ˧ mi˧˥,{\kern8pt} ʁo˧dɑ˧ mi˧-ze˥} & \ipa{ʁo˧tʰo˩ mi˩,{\kern8pt} ʁo˧tʰo˩ mi˩-ze˩}\\
			\lspbottomrule
		\end{tabularx}
	\end{table}

	\begin{table}%[h!]
		\caption{\label{tab:spatialdiFULLLEFT}The tonal behaviour of verbs after indications of spatial orientation: //\ipa{ʁwæ˧lo˥}// ‘leftward’ and //\ipa{ʁwæ˧-gi\#˥}// ‘to the left side’.}
		\begin{tabularx}{\textwidth}{ l@{\hspace{7mm}} l@{\hspace{7mm}} l@{\hspace{7mm}} Q Q }
			\lsptoprule
			tone & example & meaning & ‘leftward’  & ‘to the left side’\\ \midrule
			%Use of kerning to obtain splitting of the contents of the last two columns over two lines: more symmetrical to the eye.
			H & \ipa{se˥} & to walk & \ipa{ʁwæ˧lo˥ se˩,{\kern8pt} ʁwæ˧lo˥ se˩-ze˩} & \ipa{ʁwæ˧-gi˧ se˧,{\kern8pt} ʁwæ˧-gi˧ se˧-ze˩}\\
			M\textsubscript{a} & \ipa{li˧\textsubscript{a}} & to look & \ipa{ʁwæ˧lo˥ li˩,{\kern8pt} ʁwæ˧lo˥ li˩-ze˩} & \ipa{ʁwæ˧-gi˧ li˩,{\kern8pt} ʁwæ˧-gi˧ li˩-ze˩}\\
			M\textsubscript{b} & \ipa{tsi˧\textsubscript{b}} & to set & \ipa{ʁwæ˧lo˥ tsi˩,{\kern8pt} ʁwæ˧lo˥ tsi˩-ze˩} & \ipa{ʁwæ˧-gi˧ tsi˧,{\kern8pt} ʁwæ˧-gi˧ tsi˧-ze˩}\\
			L\textsubscript{a} & \ipa{kwɤ˩\textsubscript{a}} & to throw & \ipa{ʁwæ˧lo˥ kwɤ˩,{\kern8pt} ʁwæ˧lo˥ kwɤ˩-ze˩} & \ipa{ʁwæ˧-gi˧ kwɤ˥,{\kern8pt} ʁwæ˧-gi˧ kwɤ˥-ze˩}\\
			L\textsubscript{b} & \ipa{ɻ̩˩\textsubscript{b}} & to turn & \ipa{ʁwæ˧lo˥ ɻ̩˩,{\kern14pt} ʁwæ˧lo˥ ɻ̩˩-ze˩} & \ipa{ʁwæ˧-gi˧ ɻ̩˥,{\kern14pt} ʁwæ˧-gi˧ ɻ̩˥-ze˩}\\
			MH & \ipa{mi˧˥} & to push & \ipa{ʁwæ˧lo˥ mi˩,{\kern8pt} ʁwæ˧lo˥ mi˩-ze˩} & \ipa{ʁwæ˧-gi˧ mi˩,{\kern8pt} ʁwæ˧-gi˧ mi˩-ze˩}\\
			\lspbottomrule
		\end{tabularx}
	\end{table}
	
	\begin{table}%[h!]
		\caption{\label{tab:spatialdiFULLRIGHT}The tonal behaviour of verbs after indications of spatial orientation: //\ipa{jo˩lo˩}// ‘rightward’ and //\ipa{jo˩-gi˩}// ‘to the right side’.}
		\begin{tabularx}{\textwidth}{ l@{\hspace{7mm}} l@{\hspace{7mm}} l@{\hspace{7mm}} Q Q }
			\lsptoprule
			tone & example & meaning & ‘rightward’  & ‘to the right side’\\ \midrule
			%Use of kerning to obtain splitting of the contents of the last two columns over two lines: more symmetrical to the eye.
			H & \ipa{se˥} & to walk & \ipa{jo˩lo˩ se˩˥, {\kern12pt} jo˩lo˩ se˩-ze˥} & \ipa{jo˩-gi˩ se˩˥, {\kern8pt} jo˩-gi˩ se˩-ze˥}\\
			M\textsubscript{a} & \ipa{li˧\textsubscript{a}} & to look & \ipa{jo˩lo˩ li˥,{\kern19pt} jo˩lo˩ li˥-ze˩} & \ipa{jo˩-gi˩ li˥, {\kern14pt} jo˩-gi˩ li˥-ze˩}\\
			M\textsubscript{b} & \ipa{tsi˧\textsubscript{b}} & to set & \ipa{jo˩lo˩ tsi˩˥, jo˩lo˩ tsi˩-ze˥}{\kern2pt}\ipa{≈}{\kern2pt}\ipa{jo˩lo˩ tsi˥, jo˩lo˩ tsi˥-ze˩} & \ipa{jo˩-gi˩ tsi˥, jo˩-gi˩ tsi˥-ze˩}{\kern2pt}\ipa{≈}{\kern2pt}\ipa{jo˩-gi˩ tsi˩˥, jo˩-gi˩ tsi˩-ze˥}\\
			L\textsubscript{a} & \ipa{kwɤ˩\textsubscript{a}} & to throw & \ipa{jo˩lo˩ kwɤ˥,{\kern8pt} jo˩lo˩ kwɤ˥-ze˩} & \ipa{jo˩-gi˩ kwɤ˥,{\kern8pt} jo˩-gi˩ kwɤ˥-ze˩}\\
			L\textsubscript{b} & \ipa{ɻ̩˩\textsubscript{b}} & to turn & \ipa{jo˩lo˩ ɻ̩˥,{\kern22pt} jo˩lo˩ ɻ̩˥-ze˩} & \ipa{jo˩-gi˩ ɻ̩˥,{\kern14pt} jo˩-gi˩ ɻ̩˥-ze˩}\\
			MH & \ipa{mi˧˥} & to push & \ipa{jo˩lo˩ mi˥,{\kern14pt} jo˩lo˩ mi˥-ze˩} & \ipa{jo˩-gi˩ mi˥,{\kern8pt} jo˩-gi˩ mi˥-ze˩}\\
			\lspbottomrule
		\end{tabularx}
	\end{table}
% \clearpage removed on April 11th, 2025

	\begin{table}%[h!]
		\caption{\label{tab:spatialdiFULLLUPDOWN}The tonal behaviour of verbs after indications of spatial orientation: orientation adverbials //\ipa{gɤ˩tɕo˧}// ‘upward’ and //\ipa{mv̩˩tɕo˧}// ‘downward’.}
		\begin{tabularx}{\textwidth}{ l@{\hspace{7mm}} l@{\hspace{7mm}} l@{\hspace{7mm}} Q Q }
			\lsptoprule
			tone & example & meaning & ‘upward’  &  ‘downward’\\ \midrule
			%Use of kerning to obtain splitting of the contents of the last two columns over two lines: more symmetrical to the eye.
			H & \ipa{se˥} & to walk & \ipa{gɤ˩tɕo˧ se˧,{\kern8pt} gɤ˩tɕo˧ se˧-ze˩} & \ipa{mv̩˩tɕo˧ se˧,{\kern8pt} mv̩˩tɕo˧ se˧-ze˩}\\
			M\textsubscript{a} & \ipa{li˧\textsubscript{a}} & to look & \ipa{gɤ˩tɕo˧ li˧, {\kern8pt} gɤ˩tɕo˧ li˧-ze˧} & \ipa{mv̩˩tɕo˧ li˧,{\kern8pt} mv̩˩tɕo˧ li˧-ze˧}\\
			M\textsubscript{b} & \ipa{tsi˧\textsubscript{b}} & to set & \ipa{gɤ˩tɕo˧ tsi˧, {\kern8pt} gɤ˩tɕo˧ tsi˧-ze˧} & \ipa{mv̩˩tɕo˧ tsi˧, {\kern8pt} mv̩˩tɕo˧ tsi˧-ze˧}\\
			L\textsubscript{a} & \ipa{kwɤ˩\textsubscript{a}} & to throw & \ipa{gɤ˩tɕo˧ kwɤ˩ ,{\kern8pt} gɤ˩tɕo˧ kwɤ˩-ze˩} & \ipa{mv̩˩tɕo˧ kwɤ˩, {\kern8pt} mv̩˩tɕo˧ kwɤ˩-ze˩}\\
			L\textsubscript{b} & \ipa{ɻ̩˩\textsubscript{b}} & to turn & \ipa{gɤ˩tɕo˧ ɻ̩˩,{\kern14pt} gɤ˩tɕo˧ ɻ̩˩-ze˩} & \ipa{mv̩˩tɕo˧ ɻ̩˩,{\kern14pt} mv̩˩tɕo˧ ɻ̩˩-ze˩}\\
			MH & \ipa{mi˧˥} & to push & \ipa{gɤ˩tɕo˧ mi˧˥,{\kern8pt} gɤ˩tɕo˧ mi˧-ze˥} & \ipa{mv̩˩tɕo˧ mi˧˥,{\kern8pt} mv̩˩tɕo˧ mi˧-ze˥}\\
			\lspbottomrule
		\end{tabularx}
	\end{table}

% Keep the line below at end of WHOLE SET of subtables with full forms


 
All the data shown in Tables~\ref{tab:spatialmonoFULL} to \ref{tab:spatialdiFULLLUPDOWN} boils down to the surface tone patterns summarized in Tables~\ref{tab:spatialmonoSURFACEOnlyTones} and \ref{tab:spatialdiSURFACEOnlyTones}. The tone indicated after a~‘+’ sign is that carried by the
\textsc{perfective} \mbox{/\ipa{-ze˧}/} as it appears following the directional-plus-verb combination at issue. For example, the notation L.M+H for
the combination of /\ipa{gɤ˩}-/ ‘upward’ with a~M\textsubscript{b}-tone verb indicates the pattern
/\ipa{gɤ˩-tɕi˧}/, /\ipa{gɤ˩-tɕi˧-ze˥}/ ‘to set in an upward direction’. 

An especially interesting aspect of this data is the lowering of the \textsc{perfective} \is{suffixes}suffix (which carries a~lexical M tone: /\mbox{/\ipa{-ze˧}/}/) after expressions such as /\ipa{mv̩˩tɕo˧ se˧}/ ‘to go downward’, yielding /\ipa{mv̩˩tɕo˧ se˧-ze˩}/ ‘(s)he went downward’. This contrasts with expressions such as /\ipa{mv̩˩tɕo˧ li˧}/ ‘to look downward’, after which the \textsc{perfective} retains its lexical M tone: /\ipa{mv̩˩tɕo˧ li˧-ze˧}/ ‘(s)he looked downward’. The lowering influence of an overt H tone on following tones within a~\isi{tone group} is a~hard-and-fast {phonological rule}, but the expression /\ipa{mv̩˩tɕo˧ se˧}/ ‘to go downward’ does not contain an H tone at the surface phonological level. At a~deeper level, referred to in this volume as the \textit{underlying} phonological level, the expressions /\ipa{mv̩˩tɕo˧ se˧}/ ‘to go downward’ and /\ipa{mv̩˩tɕo˧ li˧}/ ‘to look downward’ must be analyzed as carrying different tonal specifications, since they behave differently with the same \is{suffixes}suffix. Since tone lowering is characteristic of H tones, it seems reasonable to posit a~\is{floating tone}floating H in the underlying representation of /\ipa{mv̩˩tɕo˧ se˧}/ ‘to go downward’, thus: //\ipa{mv̩˩tɕo˧ se\#˥}//. 

If the \is{suffixes}suffix /\mbox{/\ipa{-ze˧}/}/ surfaced with H~tone in any of the combinations in Tables~\ref{tab:spatialmonoSURFACEOnlyTones}--\ref{tab:spatialdiSURFACEOnlyTones}, that H~tone would need to be recognized as the manifestation of a~\is{floating tone}floating H tone from the directional-plus-verb expression, which would contradict headlong the interpretation proposed here: that \is{floating tone}floating H tones on such expressions only manifest themselves through \textit{lowering} of a~following tone. Crucially, the \is{suffixes}suffix /\mbox{/\ipa{-ze˧}/}/ does {not}, in fact, surface with H~tone in any of the combinations in Tables~\ref{tab:spatialmonoSURFACEOnlyTones}--\ref{tab:spatialdiSURFACEOnlyTones}. This is taken as confirmation of the present analysis, in light of which the data in Tables~\ref{tab:spatialmonoSURFACEOnlyTones} and \ref{tab:spatialdiSURFACEOnlyTones} is rewritten in Tables~\ref{tab:spatialmonoUNDERL} and \ref{tab:spatialdiUNDERL}, indicating the underlying tone patterns. 

In Tables~\ref{tab:spatialmonoUNDERL} and \ref{tab:spatialdiUNDERL}, reference needs to be made to the \is{juncture (inside a tone group)}juncture between the directional morpheme and the verb (similar to junctures internal to \is{numerals}numeral-plus-classifier phrases and \is{compounds}compound nouns, studied in previous chapters). This morpheme break is indicated by the symbol ‘--’. Thus, the indication ‘L\#--’ refers to a~final L tone (L\#) attaching before the morpheme break. For instance, association of L\#-- to the syllable sequence /\ipa{ʁo.tʰo.li}/ ‘to look back(ward)’ requires identification of the morpheme break following the directional /\ipa{ʁo.tʰo}/ ‘backward’: /\ipa{ʁo.tʰo~-- li}/. The L tone associates to the last syllable of the first part of the expression, yielding /\ipa{ʁo.tʰo˩}/, and spreads (by Rule~1) to the following syllable, hence /\ipa{ʁo.tʰo˩ li˩}/. Finally, the expression's first syllable receives M tone (by Rule~2), yielding /\ipa{ʁo˧tʰo˩ li˩}/. This is illustrated in \figref{fig:lookback}.

\begin{figure}
	\caption{Illustration of the anchoring of tones relative to an internal juncture (notation: ‘--’): representation of association of L\#-- tone to the expression /\ipa{ʁo.tʰo~--~li}/ to yield /\ipa{ʁo˧tʰo˩ li˩}/ ‘to look back'.}
	\begin{tikzpicture}

	
	\node (9) at (1.5,-7) {L\#--};
	
	\node (23) at (0,-8.5) {σ};
	\node (33) at (1,-8.5) {σ};
	\node (3333) at (1.5,-8.5) {--};
	\node (53) at (2,-8.5) {σ};
	\node [anchor=mid] (1553) at (0,-9) {\ipa{ʁo}};
	\node [anchor=mid] (1653) at (1,-9) {\ipa{tʰo˩}};
	\node [anchor=mid] (1753) at (2,-9) {\ipa{li}};
	
	\node[text width=40mm] (s3) at (-3,-7.75) {Stage 1:\\ association of L tone\\ to its specified locus:\\ before the \is{juncture (inside a tone group)}juncture\\between the two parts of the expression};
	
	% arrow from L tone: 
	\draw[decoration={markings,mark=at position 1 with
		{\arrow[scale=2,>=stealth]{>}}},postaction={decorate}] (9) -- (33);
	
	
	\node (44) at (1,-10) {L};
	
	\node (24) at (0,-11.5) {σ};
	\node (34) at (1,-11.5) {σ};
	\node (54) at (2,-11.5) {σ};
	\node [anchor=mid] (1500) at (0,-12) {\ipa{ʁo}};
	\node [anchor=mid] (1600) at (1,-12) {\ipa{tʰo˩}};
	\node [anchor=mid] (1700) at (2,-12) {\ipa{li˩}};
	
	\node[text width=40mm] (s4) at (-3,-10.5) {Stage 2:\\ assignment of L tone\\ by {phonological rule}:\\ L-tone spreading};
	
	\draw (44) -- (34);	
	\draw[decoration={markings,mark=at position 1 with
		{\arrow[scale=2,>=stealth]{>}}},postaction={decorate}] (44) -- (54);	
	
	\node (14) at (0,-13) {M};
	\node (64) at (1,-13) {L};
	\node (44) at (2,-13) {L};
	
	\node (24) at (0,-14.5) {σ};
	\node (34) at (1,-14.5) {σ};
	\node (54) at (2,-14.5) {σ};
	\node [anchor=mid] (2000) at (0,-15) {\ipa{ʁo˧}};
	\node [anchor=mid] (2100) at (1,-15) {\ipa{tʰo˩}};
	\node [anchor=mid] (2200) at (2,-15) {\ipa{li˩}};
	
	\node[text width=40mm] (s4) at (-3,-13.5) {Stage 3:\\ addition of M tone\\ to the remaining\\ toneless syllable};
	
	\draw[decoration={markings,mark=at position 1 with
		{\arrow[scale=2,>=stealth]{>}}},postaction={decorate}] (14) -- (24);	
	\draw (64) -- (34);
	\draw (44) -- (54);
	\end{tikzpicture}
	\label{fig:lookback}
\end{figure}



	% \label{tab:spatialOnlyTones}  %% Commented out on April 30th, 2025: no subtables, only sequentially numbered tables in the entire volume.
	\begin{table}[h!]%[t]
		\caption{\label{tab:spatialmonoSURFACEOnlyTones}The surface tone patterns of verbs after indications of spatial orientation: {monosyllabic}
			prefixes.}
		\begin{tabularx}{\textwidth}{ P{14mm} l@{\hspace{8mm}} l@{\hspace{8mm}} Q Q Q Q }
			\lsptoprule
			\multirow{2}{14mm}{tone of prefix} & \multicolumn{6}{l}{tone of verb}\\ \cmidrule{2-7}
			& H & M\textsubscript{a} & M\textsubscript{b} & L\textsubscript{a} & L\textsubscript{b} & MH\\ \midrule
			L & L.H & L.M+M & L.M+M & L.H & L.H & L.MH\\
			\lspbottomrule
		\end{tabularx}
	\end{table}
	
	\begin{sidewaystable}[p]
		\caption{\label{tab:spatialdiSURFACEOnlyTones}The surface tone patterns of verbs after indications of spatial orientation: disyllabic orientation adverbials.}
		{\renewcommand{\arraystretch}{1.15}  
			\begin{tabularx}{\textheight}{ l Q Q Q Q l@{\hspace{6mm}} l@{\hspace{6mm}} Q }
				\lsptoprule
				\multirow{2}{14mm}{tone of prefix} & \multicolumn{7}{l}{tone of verb}\\ \cmidrule{2-8}
				& H & M\textsubscript{a} & M\textsubscript{b} & M\textsubscript{c} & L\textsubscript{a} &
				L\textsubscript{b} & MH\\ \midrule
				M & M.M.M+L & M.M.M+M & M.M.M+L & M.M.M+M & M.M.L & M.M.L & M.M.MH\\
				\#H & M.M.M+L & M.M.L & M.M.M+L & M.M.M+L & M.M.H & M.M.H & M.M.L\\
				L & L.L.L & L.L.H & L.L.H / L.L.L & L.L.L & L.L.H & L.L.H & L.L.H\\
				L\# & M.L.L & M.L.L & M.L.L & M.L.L & M.L.L & M.L.L & M.L.L\\
				LM & L.M.M+L & L.M.M+M & L.M.M+M & L.M.M+M & L.M.L & L.M.L & L.M.MH\\
				H\# & M.H.L & M.H.L & M.H.L & M.H.L & M.H.L & M.H.L & M.H.L\\
				\lspbottomrule
			\end{tabularx}}
		\end{sidewaystable}

	

% 	\label{tab:spatialUNDERL}
	\begin{table}[h!]%[t]
		\caption{\label{tab:spatialmonoUNDERL}The underlying tone patterns of verbs after indications of spatial orientation: {monosyllabic}
			prefixes.}
		\begin{tabularx}{\textwidth}{ P{14mm} l@{\hspace{8mm}} l@{\hspace{8mm}} Q Q Q Q }
			\lsptoprule
			\multirow{2}{14mm}{tone of prefix} & \multicolumn{6}{l}{tone of verb}\\ \cmidrule{2-7}
			& H & M\textsubscript{a} & M\textsubscript{b} & L\textsubscript{a} & L\textsubscript{b} & MH\\ \midrule
			L & L.H & L.M & L.M & L.H & L.H & L.MH\\
			\lspbottomrule
		\end{tabularx}
	\end{table}
	
	\begin{sidewaystable}[p]
		\caption{\label{tab:spatialdiUNDERL}The underlying tone patterns of verbs after indications of spatial orientation: disyllabic orientation adverbials.}
		{\renewcommand{\arraystretch}{1.15}  
			\begin{tabularx}{\textheight}{ l Q Q Q Q l@{\hspace{6mm}} l@{\hspace{6mm}} Q }
				\lsptoprule
				\multirow{2}{14mm}{tone of prefix} & \multicolumn{7}{l}{tone of verb}\\ \cmidrule{2-8}
				& H & M\textsubscript{a} & M\textsubscript{b} & M\textsubscript{c} & L\textsubscript{a} &
				L\textsubscript{b} & MH\\ \midrule
				M & \tikzmark{A1}\#H & M & \tikzmark{C1}\#H & M & \tikzmark{E1}L\# & \hspace*{\fill}\tikzmark{F1} & MH\#\\
				\#H & \hspace*{\fill}\tikzmark{A2} & L\# & \hspace*{\fill}\tikzmark{C2} & \tikzmark{D2}\#H & \tikzmark{A3}H\# & \hspace*{\fill}\tikzmark{B3} & \#H\\
				L & L & L+H\# & L+H\# / L & L & \tikzmark{E3}L+H\# &  & \hspace*{\fill}\tikzmark{G3}\\
				L\# & \tikzmark{CROCHERON}L\#-- &  &  &  &  &  & \hspace*{\fill}\tikzmark{CROCHE}\\
				LM & LM+\#H & \tikzmark{Y1}LM & & \hspace*{\fill}\tikzmark{Y9} & \tikzmark{LMLdeb}LM+L\# & \hspace*{\fill}\tikzmark{LMLfin} & LM+MH\#\\
				H\# & \tikzmark{Z1}H\#-- & &  &  & & & \hspace*{\fill}\tikzmark{Z9}\\
				\lspbottomrule
			\end{tabularx}}
		\DrawBox[dashed]{A1}{A2}
		\DrawBox[dashed]{A3}{B3}
		\DrawBox[dashed]{E1}{F1}
		\DrawBox[dashed]{E3}{G3}
		\DrawBox[dashed]{C1}{C2}
		\DrawBox[dashed]{CROCHERON}{CROCHE}
		\DrawBox[dashed]{LMLdeb}{LMLfin}
		\DrawBox[dashed]{Y1}{Y9}
		\DrawBox[dashed]{Z1}{Z9}
		\end{sidewaystable}

	
	
A further intricacy is that the behaviour of /\ipa{ɑ˩pʰo˩}/ ‘outside’ is not fully identical with
that of /\ipa{jo˩gi˩}/ and /\ipa{jo˩lo˩}/ ‘to the right’, even though these three spatial expressions share the same
lexical tone. In association with /\ipa{tʰv̩˧\textsubscript{a}}/ ‘come out’, /\ipa{ɑ˩pʰo˩}/ ‘outside’ yields
/\ipa{ɑ˩pʰo˩ tʰv̩˩}/ ‘to go outside, to get outside’, instead of the expected $\dagger${\kern2pt}\ipa{ɑ˩pʰo˩
  tʰv̩˥}, which is not an acceptable \is{variants}variant.

Closer examination of this issue reveals yet another \is{irregularities}{oddity}: different verbs that belong to the same
tonal category, M\textsubscript{a}, display different tone patterns when associated with /\ipa{ɑ˩pʰo˩}/ ‘outside’. ‘To
look outside’ (from /\ipa{li˧\textsubscript{a}}/ ‘to look’) is /\ipa{ɑ˩pʰo˩ li˥}/, and $\ddagger${\kern2pt}\ipa{ɑ˩pʰo˩ li˩} is not
an acceptable \is{variants}variant. The verb /\ipa{tʰv̩˧\textsubscript{a}}/ ‘come out’ is an outlier: it is the only M\textsubscript{a}-tone verb
yielding an L.L.L tone pattern in association with /\ipa{ɑ˩pʰo˩}/ ‘outside’.

Given that L.L.L and L.L.H are both acceptable variants for /\ipa{jo˩gi˩}/ and
/\ipa{jo˩lo˩}/ ‘to the right’ when followed by an M\textsubscript{a}-tone verb, one may speculate that the
same pattern of \isi{variation} once existed for /\ipa{ɑ˩pʰo˩}/ ‘outside’. Under this hypothesis, the
L.L.L \is{variants}variant must have become dominant for ‘to go outside, to get outside’, to the extent that /\ipa{ɑ˩pʰo˩ tʰv̩˩}/ came to be regarded as the only correct form. 
 
But even if one chooses to treat the combination /\ipa{ɑ˩pʰo˩ tʰv̩˩}/ ‘to go outside, to get outside’ as a~lexicalized {oddity}, the behaviour of /\ipa{ɑ˩pʰo˩}/ ‘outside’ still differs from that of /\ipa{jo˩gi˩}/ and
/\ipa{jo˩lo˩}/ ‘to the right’ (see \tabref{tab:thetonalbehaviourofverbsinassociationwithoutside}). Generating the tone patterns requires knowledge of the \is{morphotonology}morphotonological rules specific to the construction at hand. These data provide yet another compelling illustration of morphotonological complexity~-- one that dashes any hope of an exhaustive account relying solely on phonological rules.
 

\begin{table}%[t]
\caption{\label{tab:thetonalbehaviourofverbsinassociationwithoutside}The tonal behaviour of verbs in association with /\ipa{ɑ˩pʰo˩}/ ‘outside’.}
\begin{tabularx}{\textwidth}{ l@{\hspace{8mm}} l@{\hspace{8mm}} l@{\hspace{8mm}} Q }
\lsptoprule
	tone of verb & example & meaning of verb & tone pattern\\ \midrule
	H & \ipa{ɑ˩pʰo˩ se˩} & to walk & L.L.L\\
	M\textsubscript{a} & \ipa{ɑ˩pʰo˩ li˥} & to look & L.L.H\\
	M\textsubscript{a} (exceptional) & \ipa{ɑ˩pʰo˩ tʰv̩˩} & to get/go & L.L.L\\
	M\textsubscript{b} & \ipa{ɑ˩pʰo˩ hõ˩} & to go.\textsc{imperative} & L.L.L\\
	M\textsubscript{c} & \ipa{ɑ˩pʰo˩ bi˩} & to go & L.L.H\\
	L\textsubscript{a} & \ipa{ɑ˩pʰo˩ kwɤ˥} & to throw & L.L.H\\
	L\textsubscript{b} & \ipa{ɑ˩pʰo˩ pʰv̩˥} & to move around & L.L.H\\
	MH & \ipa{ɑ˩pʰo˩ ʑi˥} & to sleep & L.L.H\\
\lspbottomrule
\end{tabularx}
\end{table}


The interrogative /\ipa{zo˩qo˧}/ ‘where’ has the same tonal behaviour as /\ipa{gɤ˩tɕo˧}/ ‘upward’ and
/\ipa{mv̩˩tɕo˧}/ ‘downward’, as shown in \tabref{tab:thetonalbehaviourofverbsinassociationwithwhere}. In combination with /\ipa{zo˩qo˧}/ ‘where’, the verb /\ipa{tʰv̩˧}/, whose association with
/\ipa{ɑ˩pʰo˩}/ ‘outside’ yielded an unexpected pattern (see \tabref{tab:thetonalbehaviourofverbsinassociationwithoutside}), is not any different from the
other M\textsubscript{a}-tone verbs, resulting in /\ipa{zo˩qo˧} \ipa{tʰv̩˧}(\ipa{-ze˧})/.

\begin{table}%[t]
\caption{\label{tab:thetonalbehaviourofverbsinassociationwithwhere}The tonal behaviour of verbs in association with /\ipa{zo˩qo˧}/ ‘where’, with added information about a~following \textsc{perfective} morpheme.}
\begin{tabularx}{\textwidth}{ Q l@{\hspace{8mm}} l@{\hspace{8mm}} l }
\lsptoprule
	tone of verb & example & meaning of verb & tone pattern\\ \midrule
	H & \ipa{zo˩qo˧ se˧(-ze˩)} & to walk & L.M.M+L\\
	M\textsubscript{a} & \ipa{zo˩qo˧ ʂe˧(-ze˧)} & to look for & L.M.M+M\\
	M\textsubscript{b} & \ipa{zo˩qo˧ pʰæ˧(-ze˧)} & to attach, to fasten & L.M.M+M\\
	M\textsubscript{c} & \ipa{zo˩qo˧ hɯ˧(-ze˧)} & to go.\textsc{pst} & L.M.M+M\\
	L\textsubscript{a} & \ipa{zo˩qo˧ dzi˩} & to sit; to live & L.M.L\\
	L\textsubscript{b} & \ipa{zo˩qo˧ ɻ̩˩} & to turn towards & L.M.L\\
	MH & \ipa{zo˩qo˧ lɑ˧˥} & to strike, to hit & L.M.MH\\
\lspbottomrule
\end{tabularx}
\end{table}

 
%\subsubsection{On the morphosyntactic analysis of locative constituents}
 
In principle, the tonal behaviour of locative constituents in Yongning Na could shed light on their morphosyntactic
properties. Cross-linguistically, various configurations are attested. For instance, in Central \ili{Bantu}, verbal agreement reveals a~typologically uncommon pattern: “a locative noun phrase in preverbal position
can be analyzed as the grammatical subject” \citep[34]{creissels2011}. In contrast, in Northern \ili{Sotho}, where such agreement is absent, the construction is better analyzed as “an impersonal construction
with a~preposed locative constituent” \citep{zerbian2006b}. However, the Yongning Na tonal patterns presented above are not fully identical with those of any other construction. In particular, they differ from both subject-verb and object-verb constructions.

Let us now turn from preverbal elements to postverbal ones.


\section{Monosyllabic postverbal morphemes}
\label{sec:verbalsuffixesandverbserializationmonosyllabicelements}

Verbs follow their objects and may be followed in turn by various morphemes, including suffixes,\footnote{\citet[349]{lidz2010}, based on data from the Luoshui dialect, proposes that “[s]uffixation is not attested on verbs in Na”. However, for the dialect under description here, it appears most straightforward to recognize a~few postverbal morphemes as suffixes, such as the \textsc{perfective} \mbox{/\ipa{-ze˧}/} (analyzed by Lidz (\citeyear[424]{lidz2010}) as a~postverbal particle). In transcriptions, a~hyphen is also used in cases where a~verb's \is{grammaticalization}grammaticalized use appears sufficiently distinct from the verb's original meaning to warrant separate recognition, for instance the \textsc{immediate future} /\ipa{-bi˧}/, \is{grammaticalization}grammaticalized from the non-{imperative} form of ‘to go’, /\ipa{bi˧\textsubscript{c}}/. The decision to treat a~morpheme as a~serialized verb or a~{suffix} can be difficult. My use of hyphens fluctuated over the years of preparation of this volume; as of 2025, some inconsistencies remain in the transcriptions of online primary documents in the Pangloss Collection. Consistent implementation of the usage conventions adopted in the dictionary \citep{michaud_et_al_na_dict_2024} in the glossing of the Yongning Na texts is on the to-do list for years to come.} serialized verbs, postpositions, and discourse particles, which will be referred to here under the cover term ‘postverbal morphemes'. 

From the point of view of tone patterns, a~key observation is that not all morphemes following verbs have the same tonal behaviour, suggesting that they have lexical tones of their own. Distributional analysis constitutes a~first step in investigating these tones: finding out how many tonal categories exist for a~given part of speech. Once morphemes have been sorted into tonal sets, the next step consists in analyzing their underlying tones, as was done above for nouns and verbs. 

On the one hand, in Yongning Na, all syllables carry level tones at the surface phonological level: H, M, L, and combinations into low-rising and mid-rising contours, so it makes sense to hypothesize that the lexical tone categories of postverbal morphemes are composed of these level tones and to attempt to pinpoint their phonological nature, as was done for nouns and verbs. On the other hand, one should keep in mind the structural fact that the tone systems of different word classes are not identical. In addition to differences in the number of tones (among monosyllables: seven categories for verbs, five for adjectives, six for free nouns, nine for classifiers{\dots}), tonal behaviour in context also varies across word classes, even between closely related categories such as verbs and adjectives (as pointed out in \sectref{sec:adjectivesasdistinctfromverbs}). 
%and free nouns and nominal classifiers (which are, revealingly, treated in separate chapters: Chapter~\ref{chap:thelexicaltonesofnouns} and Chapter~\ref{chap:classifiers}, respectively). 
Consequently, the tonal analysis of a~verb cannot be assumed to apply straightforwardly to its \is{grammaticalization}grammaticalized counterpart. 

For instance, the morpheme indicating \textsc{completion}, \mbox{/\ipa{-se˩}/}, clearly originates from the verb /\ipa{se˩}/ ‘to finish, to complete'. However, this fact does not by itself justify the assumption that the \is{grammaticalization}grammaticalized morpheme retains the same lexical tone as the verb. Nominal classifiers, studied in Chapter~\ref{chap:classifiers}, are a perfect illustration of the changes in lexical tone that can accompany the process of \isi{grammaticalization}. Returning to postverbal morphemes, cases where a morpheme can clearly be traced back to a verb 
only provide indirect hints on tonal identity. The tonal categories of the grammatical morphemes examined in this chapter are therefore assembled piece by piece, and there remains room for further progress in the analysis. 

The reader will be reminded at various points that a~given lexical-tone label (say, L tone) assigned to different parts of speech (e.g.\ a~verb and a~\is{suffixes}suffix) does not necessarily refer to the exact same phonological entity. Rather, these represent distinct morphophonological categories that appear sufficiently similar~-- at the present stage of analysis~-- to warrant the use of the same label from among the set of phonologically distinctive tonal levels.

Monosyllabic elements will be discussed first, before turning to disyllabic postpositions and combinations of affixes.

\subsection[L tone]{L-tone postverbal morphemes}
\label{sec:ltonesuffixesandserializedverbs}

%\subsubsection{Main facts}
\label{sec:mainfactsaboutlsuffix}

As noted above, the morpheme indicating \textsc{completion}, \mbox{/\ipa{-se˩}/}, is a~\is{grammaticalization}grammaticalized form of the verb /\ipa{se˩}/ ‘to finish, to complete'. This alone does not constitute sufficient evidence that its lexical tone is L, but its tonal behaviour after M-tone verbs (where it surfaces with L tone) points in the same direction; accordingly, it is classified as an L-tone morpheme. Since the tonal behaviour of the {desiderative} morpheme is the same as that of the \textsc{completion} morpheme, the same tonal label is applied, yielding \mbox{/\ipa{-ho˩}/}. 

\tabref{tab:thepatternsofltonetenseaspectmoodsuffixes} presents data illustrating the tonal patterns for these two morphemes, also including the morpheme /\ipa{-sɯ˩}/ ‘yet’ (used in the negative construction ‘not yet’). Other morphemes belonging to the same tonal class (L tone) include the {inchoative}, /\ipa{-ɻ̩˩}/, and the morpheme /\ipa{-dze˩}/ ‘to remain; to be left over’. The latter is only observed in /\ipa{dzɯ˧-dze˥}/ ‘left over after eating’ and /\ipa{ʈʰɯ˩-dze˩}/ ‘left over after drinking (or smoking)’. These \textit{noncompletion resultatives}, which denote the state resulting from an action that was not carried through to completion (the incomplete consumption of an object), have handy equivalents in {Mandarin} Chinese: \textit{chī shèng de} \zh{吃剩的} for ‘left over after eating’ and \textit{hē shèng de} \zh{喝剩的} for ‘left over after drinking’.

\begin{table}%[t]
\caption{\label{tab:thepatternsofltonetenseaspectmoodsuffixes}The patterns of L-tone tense-aspect-modality morphemes.}
\begin{tabularx}{\textwidth}{ l@{\hspace{4mm}} l@{\hspace{4mm}} l@{\hspace{4mm}} l@{\hspace{3mm}} P{25mm}@{\hspace{3mm}} Q@{\hspace{1mm}} }
\lsptoprule
	tone & example & meaning & \textsc{completion} & \textsc{desiderative} & not yet\\ \midrule
	H & \ipa{dzɯ˥} & to eat & \ipa{dzɯ˧-se˥} & \ipa{dzɯ˧-ho˥} & \ipa{mɤ˧-dzɯ˧-sɯ˥}\\
	M\textsubscript{a} & \ipa{hwæ˧\textsubscript{a}} & to buy & \ipa{hwæ˧-se˩} & \ipa{hwæ˧-ho˩} & \ipa{mɤ˧-hwæ˧-sɯ˩}\\
	M\textsubscript{b} & \ipa{tɕʰi˧\textsubscript{b}} & to sell & \ipa{tɕʰi˧-se˩} & \ipa{tɕʰi˧-ho˩} & \ipa{mɤ˧-tɕʰi˧-sɯ˩}\\
	M\textsubscript{c} & \ipa{pv̩˧\textsubscript{c}} & to chant & \ipa{pv̩˧-se˩} & \ipa{pv̩˧-ho˩} & \ipa{mɤ˧-pv̩˧-sɯ˩}\\
	L\textsubscript{a} & \ipa{bæ˩\textsubscript{a}} & to sweep & \ipa{bæ˩-se˩} & \ipa{bæ˩-ho˩} & \ipa{mɤ˧-bæ˩-sɯ˩}\\
	L\textsubscript{b} & \ipa{ʐwɤ˩\textsubscript{b}} & to speak & \ipa{ʐwɤ˩-se˩} & \ipa{ʐwɤ˩-ho˩} & \ipa{mɤ˧-ʐwɤ˩-sɯ˩}\\
	MH & \ipa{lɑ˧˥} & to strike & \ipa{lɑ˧-se˥} & \ipa{lɑ˧-ho˥} & \ipa{mɤ˧-lɑ˧-sɯ˥}\\
\lspbottomrule
\end{tabularx}
\end{table}

The next paragraph is devoted to the nominalizing suffix /\ipa{-di˩}/, which does not fully align with the three morphemes in \tabref{tab:thepatternsofltonetenseaspectmoodsuffixes} in terms of its tonal behaviour. 

\subsubsection*{The nominalizing suffix /\ipa{-di˩}/}
\label{sec:!nominalization}

In her study of the Luoshui dialect of Yongning Na, Liberty Lidz observes that “\ipa{di³³} ‘earth; place’ \is{grammaticalization}grammaticalized into a~locative nominalizer, and then further
grammaticalized into a~purposive nominalizer” \citep[184]{lidz2010}. In the Alawua dialect, the nominalizing morpheme is assigned L tone (hence /\ipa{-di˩}/) on the basis of its behaviour after
M-tone verbs. However, its tone behaviour is distinct from that of the L-tone
morphemes discussed in the previous paragraph. For instance, compare /\ipa{dzɯ˧-di˧˥}/ ‘food, things for eating’ with /\ipa{dzɯ˧-ho˥}/ ‘will eat’, or /\ipa{ʈʰæ˧-di˧˥}/
‘thing to bite (e.g.~a toy given to teething babies)’ with /\ipa{ʈʰæ˧-ho˥}/ ‘will bite’. This observation supports the analysis of the  nominalizing morpheme as a~\is{suffixes}suffix, distinct in its \is{morphotonology}morphotonological properties from serialized verbs. The data is
set out in \tabref{tab:thetonalbehaviourofthenominalizingsuffix}. 

\begin{table}%[t]
	\caption{\label{tab:thetonalbehaviourofthenominalizingsuffix}The tonal behaviour of the nominalizing suffix /\ipa{-di˩}/.}
	\begin{tabularx}{\textwidth}{ l@{\hspace{5mm}} l@{\hspace{5mm}} l@{\hspace{5mm}} l@{\hspace{5mm}} Q }
		\lsptoprule
		tone & example & meaning & nominalizer & meaning\\ \midrule
		H & \ipa{dzɯ˥} & to eat & \ipa{dzɯ˧-di˧˥} & thing to eat, food\\
		M\textsubscript{a} & \ipa{hwæ˧\textsubscript{a}} & to buy & \ipa{hwæ˧-di˩} & thing to buy, product\\
		M\textsubscript{b} & \ipa{tɕʰi˧\textsubscript{b}} & to sell & \ipa{tɕʰi˧-di˩} & thing to sell, commodity\\
		M\textsubscript{c} & \ipa{pv̩˧\textsubscript{c}} & to chant & \ipa{pv̩˧-di˩} & thing to chant, ritual\\
		L\textsubscript{a} & \ipa{dze˩\textsubscript{a}} & to cut & \ipa{dze˩-di˩} & thing to cut, e.g.~knife\\
		L\textsubscript{b} & \ipa{ʈʰɯ˩\textsubscript{b}} & to drink & \ipa{ʈʰɯ˩-di˩} & thing to drink, beverage\\
		MH & \ipa{ʈʰæ˧˥} & to bite & \ipa{ʈʰæ˧-di˧˥} & thing to bite (for infant, dog{\dots})\\
		\lspbottomrule
	\end{tabularx}
\end{table}


As an aside, consultant M21 has a different tone pattern for the L tone: L.H
(e.g.~/\ipa{dze˩-di˥}/ and /\ipa{ʈʰɯ˩-di˥}/) rather than L.L. This could result from \isi{analogy} with
other L-tone morphemes, such as the morpheme indicating \textsc{completion}, \mbox{/\ipa{-se˩}/}, which takes H tone
after an L-tone verb, as discussed in \sectref{sec:mainfactsaboutlsuffix}.  

\subsection[M tone]{M-tone postverbal morphemes}
\label{sec:mtonesuffixes}

The \textsc{imperative},
/\ipa{-hõ˧}/, is \is{grammaticalization}grammaticalized from the {imperative} form of the verb ‘to go’, /\ipa{hõ˧\textsubscript{a}}/, and the \textsc{immediate future}, /\ipa{-bi˧}/, \is{derivation!morphological}derives from the non-{imperative} form of ‘to go’, /\ipa{bi˧\textsubscript{c}}/. This provides an initial indication that these morphemes may belong to an M-tone category. A~stronger argument comes from their behaviour in combination with the seven tonal categories of verbs, shown in \tabref{tab:thetonalbehaviourofmtonesuffixes}. The table also includes data for the experiential /\ipa{-dʑɯ˧}/, which has the same tonal behaviour across all cases. 

The tone patterns observed after verbs with M, L or MH tone suggest that these postverbal morphemes behave as tonally neutral elements: they allow a verb's MH tone to unfold onto them and permit a verb's L tone to spread over them. After an M-tone verb, the surface result is M, consistent with the phonologically inert nature of M tone, already noted in \sectref{sec:analysisofmasadefaulttone}. 
%Further confirmation comes from their behaviour after the {negation} \is{prefixes}prefix, where they surface with M tone. // To preview observations set out in \sectref{sec:suffixesprecededbythenegation}, another test confirms the appropriateness of M tone as a~label for the tonal category to which these morphemes belong: after the negation... 

Other morphemes with the same
tonal behaviour include /\ipa{-ɖo˧}/ ‘must, have to’, the \textsc{volitive} /\ipa{-tso˧}/, the \textsc{obligative}
/\ipa{-zo˧}/, and the \textsc{causative} /\ipa{-tsæ˧}/.

{\setlength\tabcolsep{4.5pt}
\begin{table}%[t]
\caption{\label{tab:thetonalbehaviourofmtonesuffixes}The tonal behaviour of M-tone morphemes.}
\begin{tabularx}{\textwidth}{ l l l l l Q l }
\lsptoprule
	tone & example & meaning & \textsc{imm\_fut} & \textsc{experiential} & \textsc{imperative} & ana\-lysis\\ \midrule
	H & \ipa{dzɯ˥} & to eat & \ipa{dzɯ˧-bi˧} & \ipa{dzɯ˧-dʑɯ˧} & \ipa{dzɯ˧-hõ˧} & M\\
	M\textsubscript{a} & \ipa{hwæ˧\textsubscript{a}} & to buy & \ipa{hwæ˧-bi˧} & \ipa{hwæ˧-dʑɯ˧} & \ipa{hwæ˧-hõ˧} & M\\
	M\textsubscript{b} & \ipa{tɕʰi˧\textsubscript{b}} & to sell & \ipa{tɕʰi˧-bi˧} & \ipa{tɕʰi˧-dʑɯ˧} & \ipa{tɕʰi˧-hõ˧} & M\\
	M\textsubscript{c} & \ipa{pv̩˧\textsubscript{c}} & to chant & \ipa{pv̩˧-bi˧} & \ipa{pv̩˧-dʑɯ˧} & \ipa{pv̩˧-hõ˧} & M\\
	L\textsubscript{a} & \ipa{bæ˩\textsubscript{a}} & to sweep & \ipa{bæ˩-bi˩} & \ipa{bæ˩-dʑɯ˩} & \ipa{bæ˩-hõ˩} & L\\
	L\textsubscript{b} & \ipa{ʐwɤ˩\textsubscript{b}} & to speak & \ipa{ʐwɤ˩-bi˩} & \ipa{ʐwɤ˩-dʑɯ˩} & \ipa{ʐwɤ˩-hõ˩} & L\\
	MH & \ipa{lɑ˧˥} & to strike & \ipa{lɑ˧-bi˥} & \ipa{lɑ˧-dʑɯ˥} & \ipa{lɑ˧-hõ˥} & H\#\\
\lspbottomrule
\end{tabularx}
\end{table}}


In addition to the surface phonological forms, \tabref{tab:thetonalbehaviourofmtonesuffixes} proposes an analysis of the underlying tone in its final column. This analysis is based on the tone patterns when the \textsc{causative} /\ipa{-tsæ˧}/ is
added after the \textsc{immediate future} /\ipa{-bi˧}/, as shown in \tabref{tab:immediatefuturecausativecopula}. The \isi{copula}, in its use to
express {certainty}, was also added, as a~further test to reveal the underlying tonal categories. In every case, the \isi{copula} carries L, i.e.\ its lexical tone. This shows the absence of a~\is{floating tone}floating H tone in any of the verb phrases. The first three expressions are therefore analyzed as having M tone. The underlying tonal category leading to the realizations /\ipa{lɑ˧-bi˥}/ ‘will strike’ and
/\ipa{lɑ˧-bi˧-tsæ˥}/ (not $\ddagger${\kern2pt}\ipa{lɑ˧-bi˥-tsæ˩}) ‘cause to strike’ is interpreted as H\#.


\begin{table}[t]
\caption{\label{tab:immediatefuturecausativecopula}Tone patterns of V+\textsc{immediate future}+\textsc{caus\-ative}\-+\textsc{copula}.}
\begin{tabularx}{\textwidth}{ l@{\hspace{9mm}} Q Q l }
\lsptoprule
	tone & example & meaning & \ipa{-bi˧-tsæ˧-ɲi˩}\\ \midrule
	H & \ipa{dzɯ˥} & to eat & \ipa{dzɯ˧-bi˧-tsæ˧-ɲi˩}\\
	M\textsubscript{a} & \ipa{hwæ˧\textsubscript{a}} & to buy & \ipa{hwæ˧-bi˧-tsæ˧-ɲi˩}\\
	M\textsubscript{b} & \ipa{tɕʰi˧\textsubscript{b}} & to sell & \ipa{tɕʰi˧-bi˧-tsæ˧-ɲi˩}\\
	M\textsubscript{c} & \ipa{pv̩˧\textsubscript{c}} & to chant & \ipa{pv̩˧-bi˧-tsæ˧-ɲi˩}\\
	L\textsubscript{a} & \ipa{bæ˩\textsubscript{a}} & to sweep & \ipa{bæ˩-bi˩-tsæ˥-ɲi˩ ($\ddagger${\kern2pt}bæ˩-bi˩-tsæ˩˥-ɲi˩)}\\
	L\textsubscript{b} & \ipa{ʐwɤ˩\textsubscript{b}} & to speak & \ipa{ʐwɤ˩-bi˩-tsæ˥-ɲi˩ ($\ddagger${\kern2pt}ʐwɤ˩-bi˩-tsæ˩˥-ɲi˩)}\\
	MH & \ipa{lɑ˧˥} & to strike & \ipa{lɑ˧-bi˧-tsæ˥-ɲi˩ ($\ddagger${\kern2pt}lɑ˧-bi˧-tsæ˧˥-ɲi˩)}\\
\lspbottomrule
\end{tabularx}
\end{table}


\begin{table} 
\caption{\label{tab:sametonalcategoryofmorphemesasinprevioustablewithnegation}Tone patterns of V+\textsc{negation} prefix+\textsc{immediate future}.}
\begin{tabularx}{\textwidth}{ l@{\hspace{7mm}} l@{\hspace{7mm}} l@{\hspace{7mm}} l@{\hspace{7mm}} Q }
\lsptoprule
	tone & example & meaning & V \textsc{neg-imm\_fut} & V \textsc{neg-imm\_fut-pfv}\\ \midrule
	H & \ipa{dzɯ˥} & to eat & \ipa{dzɯ˧ mɤ˧-bi˧} & \ipa{dzɯ˧ mɤ˧-bi˧-ze˩}\\
	M\textsubscript{a} & \ipa{hwæ˧\textsubscript{a}} & to buy & \ipa{hwæ˧ mɤ˧-bi˧} & \ipa{hwæ˧ mɤ˧-bi˧-ze˧}\\
	M\textsubscript{b} & \ipa{tɕʰi˧\textsubscript{b}} & to sell & \ipa{tɕʰi˧ mɤ˧-bi˧} & \ipa{tɕʰi˧ mɤ˧-bi˧-ze˧}\\
	M\textsubscript{c} & \ipa{pv̩˧\textsubscript{c}} & to chant & \ipa{pv̩˧ mɤ˧-bi˧} & \ipa{pv̩˧ mɤ˧-bi˧-ze˧}\\
	L\textsubscript{a} & \ipa{bæ˩\textsubscript{a}} & to sweep & \ipa{bæ˩ mɤ˩-bi˩} & \ipa{bæ˩ mɤ˩-bi˩-ze˥}\\
	L\textsubscript{b} & \ipa{ʐwɤ˩\textsubscript{b}} & to speak & \ipa{ʐwɤ˩ mɤ˩-bi˩} & \ipa{ʐwɤ˩ mɤ˩-bi˩-ze˥}\\
	MH & \ipa{lɑ˧˥} & to strike & \ipa{lɑ˧ mɤ˧-bi˧} & \ipa{lɑ˧ mɤ˧-bi˧-ze˩}\\
\lspbottomrule
\end{tabularx}
\end{table}


\begin{table}[b]
	\caption{\label{tab:thebehaviouroftonelesssuffixe}The behaviour of postverbal morphemes belonging to a~second {subcategory} of M tones: M\textsubscript{b}.}
	\begin{tabularx}{\textwidth}{ l@{\hspace{12mm}} Q Q Q Q }
		\lsptoprule
		tone & example & meaning & \textsc{perfective} & to achieve\\ \midrule
		H & \ipa{dzɯ˥} & to eat & \ipa{dzɯ˧-ze˩} & \ipa{dzɯ˧-mæ˩}\\
		M\textsubscript{a} & \ipa{hwæ˧\textsubscript{a}} & to buy & \ipa{hwæ˧-ze˧} & \ipa{hwæ˧-mæ˧}\\
		M\textsubscript{b} & \ipa{tɕʰi˧\textsubscript{b}} & to sell & \ipa{tɕʰi˧-ze˧} & \ipa{tɕʰi˧-mæ˧}\\
		M\textsubscript{c} & \ipa{pv̩˧\textsubscript{c}} & to chant & \ipa{pv̩˧-ze˧} & \ipa{pv̩˧-mæ˧}\\
		L\textsubscript{a} & \ipa{bæ˩\textsubscript{a}} & to sweep & \ipa{bæ˩-ze˩} & \ipa{bæ˩-mæ˩}\\
		L\textsubscript{b} & \ipa{ʐwɤ˩\textsubscript{b}} & to speak & \ipa{ʐwɤ˩-ze˩} & \ipa{ʐwɤ˩-mæ˩}\\
		MH & \ipa{lɑ˧˥} & to strike & \ipa{lɑ˧-ze˥} & \ipa{lɑ˧-mæ˥}\\
		\lspbottomrule
	\end{tabularx}
\end{table}


It is reassuring to note that these M-tone morphemes have the same behaviour when preceded by the {negation} \is{prefixes}prefix,
/\ipa{mɤ˧}-/, e.g.~in /\ipa{V mɤ˧-bi˧}/ ‘is not going to V’, /\ipa{V mɤ˧-ɖo˧}/ ‘ought not to V’, and
/\ipa{V mɤ˧-zo˧}/ ‘must not V’. \tabref{tab:sametonalcategoryofmorphemesasinprevioustablewithnegation} provides examples with /\ipa{V mɤ˧-bi˧}/, also including, in its last column, the same phrases with
an added \textsc{perfective}, \mbox{/\ipa{-ze˧}/}. 

Nonetheless, M-tone morphemes are not without complexities. The verb /\ipa{mæ˧}/ ‘to achieve’ carries M tone, but in serial verb
constructions, where it indicates that an action has achieved its goal, its tonal behaviour is not fully identical to that
of the M-tone morphemes in \tabref{tab:thetonalbehaviourofmtonesuffixes}. The \textsc{perfective} \mbox{/\ipa{-ze˧}/} has the same tonal behaviour as /\ipa{-mæ˧}/. Their tone patterns are shown in \tabref{tab:thebehaviouroftonelesssuffixe}. While their behaviour is in most respects similar to that of the M-tone morphemes in \tabref{tab:thetonalbehaviourofmtonesuffixes}, the pattern following an H-tone verb is
M.L and not M.M. This warrants distinguishing a separate tonal category, leading to the establishment of two M-tone subclasses for postverbal morphemes (M\textsubscript{a} and M\textsubscript{b}). 

For classifiers (discussed in Chapter~\ref{chap:classifiers}), a~total of nine tonal categories has been identified. It would not be particularly surprising if a~comparable number of categories were required for affixes. In view of the greater morphosyntactic diversity of affixes in comparison with classifiers, the number could even be higher. Clearly, the added subscript letters serve merely as classificatory labels: they do not constitute an analysis of the tonal categories. A more thorough analysis would require a~fuller
inventory of morphemes. It should be borne in mind that the current size of the dictionary \citep{michaud_et_al_na_dict_2024} is approximately 3,000 words, i.e.\ only a~small part of the full richness of the language's lexicon. 

In this chapter, postverbal morphemes are simply arranged into broad tonal classes (L, M, H, and MH), while also reporting the internal diversity observed within these classes and setting up provisional subclasses, such as M\textsubscript{a} and M\textsubscript{b}. Subclasses M\textsubscript{a} and M\textsubscript{b} of M-tone postverbal morphemes include the following items: 

\begin{itemize}
	\item{M\textsubscript{a} tone: \textsc{imperative} /\ipa{-hõ˧\textsubscript{a}}/, \textsc{immediate future} /\ipa{-bi˧\textsubscript{a}}/, \textsc{experiential}\Hack{\break} /\ipa{-dʑɯ˧\textsubscript{a}}/, /\ipa{-ɖo˧\textsubscript{a}}/ ‘must, have to’, \textsc{volitive} /\ipa{-tso˧\textsubscript{a}}/, \textsc{obligative}
		/\ipa{-zo˧\textsubscript{a}}/, and \textsc{causative} /\ipa{-tsæ˧\textsubscript{a}}/}
	\item{M\textsubscript{b} tone: \textsc{perfective} /\ipa{-ze˧\textsubscript{b}}/, and /\ipa{mæ˧\textsubscript{b}}/ ‘to achieve’}
\end{itemize}

In the data presented in \tabref{tab:sametonalcategoryofmorphemesasinprevioustablewithnegation}, the H component found in the lexical tone categories H and MH
(illustrated by /\ipa{dzɯ˥}/ ‘to eat’ and /\ipa{lɑ˧˥}/ ‘to strike’, respectively) does not \is{form!surface}surface as such. Instead, it manifests itself through the lowering of the \textsc{perfective} morpheme /\ipa{-ze˧\textsubscript{b}}/, even though this morpheme is no fewer than three syllables away from the
verb. The phrase /\ipa{dzɯ˧ mɤ˧-bi˧-ze˩}/ ‘will not eat anymore’ constitutes a single tone group, and the underlying presence of an H tone within this group becomes apparent through its effect on the \textsc{perfective} morpheme, despite the intervening syllables. The same phenomenon is also observed in the absence of an intervening {negation} \is{prefixes}prefix, as shown in \tabref{tab:withoutintnegation}.

\begin{table}%[t]
	\caption{Same data as in \tabref{tab:sametonalcategoryofmorphemesasinprevioustablewithnegation}, without intervening
		{negation} prefix.}
	\label{tab:withoutintnegation}
	\begin{tabularx}{\textwidth}{ l@{\hspace{9mm}} l@{\hspace{9mm}} l@{\hspace{9mm}} Q }
		\lsptoprule
		tone & example & meaning & V+\textsc{immediate future}+\textsc{perfective}\\ \midrule
		H & \ipa{dzɯ˥} & to eat & \ipa{dzɯ˧ bi˧-ze˩}\\
		M\textsubscript{a} & \ipa{hwæ˧\textsubscript{a}} & to buy & \ipa{hwæ˧ bi˧-ze˧}\\
		M\textsubscript{b} & \ipa{tɕʰi˧\textsubscript{b}} & to sell & \ipa{tɕʰi˧ bi˧-ze˧}\\
		M\textsubscript{c} & \ipa{pv̩˧\textsubscript{c}} & to chant & \ipa{pv̩˧ bi˧-ze˧}\\
		L\textsubscript{a} & \ipa{bæ˩\textsubscript{a}} & to sweep & \ipa{bæ˩ bi˩-ze˥}\\
		L\textsubscript{b} & \ipa{ʐwɤ˩\textsubscript{b}} & to speak & \ipa{ʐwɤ˩ bi˩-ze˥}\\
		MH & \ipa{lɑ˧˥} & to strike & \ipa{lɑ˧ bi˥-ze˩}\\
		\lspbottomrule
	\end{tabularx}
\end{table}

In these contexts, the \textsc{perfective} morpheme //\ipa{-ze˧\textsubscript{b}}// carries different tones depending on the lexical tone of the verb~-- including in the case of H-tone verbs, where the lexical H tone surfaces neither on the verb itself nor on the immediately following morpheme (the \textsc{immediate future}). For H-tone verbs, the \textsc{perfective} morpheme //\ipa{-ze˧\textsubscript{b}}// is consistently lowered to L in each of (\ref{ex:haveeaten}), (\ref{ex:willeat}), and (\ref{ex:willnoteat}).

\begin{exe}
	\ex
	\label{ex:haveeaten}
	\ipaex{dzɯ˧-ze˩}\\
	\gll dzɯ˥		-ze˧\textsubscript{b}\\
	to\_eat		\textsc{pfv}\\
	\glt ‘have eaten’
\end{exe}

\begin{exe}
	\ex
	\label{ex:willeat}
	\ipaex{dzɯ˧-bi˧-ze˩}\\
	\gll dzɯ˥		-bi˧\textsubscript{a}		-ze˧\textsubscript{b}\\
	to\_eat		\textsc{imm.fut}		\textsc{pfv}\\
	\glt ‘will eat’
\end{exe}

\begin{exe}
	\ex
	\label{ex:willnoteat}
	\ipaex{dzɯ˧ mɤ˧-bi˧-ze˩}\\
	\gll dzɯ˥		mɤ˧-	-bi˧\textsubscript{a}		-ze˧\textsubscript{b}\\
	to\_eat		\textsc{neg}	\textsc{imm.fut}		\textsc{pfv}\\
	\glt ‘will not eat’
\end{exe}

Tones that are present \is{form!underlying}underlyingly but only manifest themselves in a~restricted set of contexts constitute a~salient characteristic of the Yongning Na tone system. The surface phonological representations of ‘going to buy’
as /\ipa{hwæ˧-bi˧}/ and ‘going to eat’ as /\ipa{dzɯ˧-bi˧}/, with the same tone pattern, do not
reveal the underlying presence of an H tone in the former phrase. The manifestation of the H tone of
the verb is not straightforward: it does not lower an M-tone postverbal morpheme (as seen in /\ipa{dzɯ˧-bi˧}/, ‘going
to eat’), yet it does induce lowering on the last syllable in (\ref{ex:willeat}). 

The lowering effect of overt H tones is an exceptionless phonological regularity, formalized as Rule~4: “The syllable following an H-tone syllable receives L tone” (\sectref{sec:alistoftonerules}). In certain morphosyntactic contexts, exemplified in \tabref{tab:sametonalcategoryofmorphemesasinprevioustablewithnegation}, H tones exert a~similar influence even though they remain \is{floating tone}floating, i.e.\ unassociated with a~syllable. Such complexities cannot be captured by a~small set of rules. This explains the abundance of tables in this chapter and, more broadly, throughout this volume. 

Further data on M-tone postverbal morphemes is shown in \tabref{tab:anotherpostverbalelementsoverwhichamhtonecannotunfoldthedurative}, illustrating the behaviour of the \textsc{progressive} /\ipa{-dʑo˧}/. The affirmative particle /\ipa{mv̩˧}/ has the same behaviour. An intriguing peculiarity is that an MH \is{tonal contour}contour on the verb does not unfold onto these morphemes. A~MH-tone verb preceding these morphemes retains its MH \is{tonal contour}contour, e.g.~/\ipa{mɤ˧-lɑ˧˥ {\kern2pt}|{\kern2pt} -dʑo˩}/
‘as [they] did not strike{\dots}’. Following the same descriptive approach as before, this difference in tonal behaviour requires setting up a~third \is{subcategories of lexical tones}subcategory of M tones among postverbal morphemes: M\textsubscript{c}.

\begin{table}%[t]
	\caption{Tonal behaviour of the {progressive}, illustrating a~third {subcategory} of M tones among postverbal morphemes: M\textsubscript{c}.}
	\label{tab:anotherpostverbalelementsoverwhichamhtonecannotunfoldthedurative}
	\begin{tabularx}{\textwidth}{ l@{\hspace{9mm}} l@{\hspace{9mm}} l@{\hspace{9mm}} Q }
		\lsptoprule
		tone & example & meaning & \textsc{dur}+V+\textsc{prog}: ‘is currently V-ing’\\ \midrule
		H & \ipa{dzɯ˥} & to eat & \ipa{tʰi˧-dzɯ˥-dʑo˩}\\
		M\textsubscript{a} & \ipa{hwæ˧\textsubscript{a}} & to buy & \ipa{tʰi˧-hwæ˧-dʑo˧}\\
		M\textsubscript{b} & \ipa{tɕʰi˧\textsubscript{b}} & to sell & \ipa{tʰi˧-tɕʰi˧-dʑo˧}\\
		M\textsubscript{c} & \ipa{pv̩˧\textsubscript{c}} & to chant & \ipa{tʰi˧-pv̩˧-dʑo˧}\\
		L\textsubscript{a} & \ipa{bæ˩\textsubscript{a}} & to sweep & \ipa{tʰi˧-bæ˩-dʑo˩}\\
		L\textsubscript{b} & \ipa{ʐwɤ˩\textsubscript{b}} & to speak & \ipa{tʰi˧-ʐwɤ˩-dʑo˩}\\
		MH & \ipa{lɑ˧˥} & to strike & \ipa{tʰi˧-lɑ˧˥-dʑo˩} ($\ddagger${\kern2pt}\ipa{tʰi˧-lɑ˥-dʑo˩})\\
		\lspbottomrule
	\end{tabularx}
\end{table}

\subsection[H tone]{H-tone postverbal morphemes}
\label{sec:ahtonesuffixtherelativizernominalizer}

Two postverbal morphemes are tentatively classified under the heading of “H-tone postverbal morphemes”: the {relativizer/nominalizer} \mbox{//\ipa{-hĩ˥}//} and the {topic} marker //\ipa{-dʑo˥}//. The rationale for this categorization is as follows.

The {relativizer/nominalizer} \mbox{//\ipa{-hĩ˥}//} behaves in many tonal contexts like the M-tone postverbal morphemes described in \sectref{sec:mtonesuffixes}, such as the \textsc{immediate future} \mbox{/\ipa{-bi˧}/.} However, unlike M-tone morphemes, the {relativizer/nominalizer} does not host the H component of
an MH \is{tonal contour}contour from the preceding verb; instead, the outcome is M.M, as exemplified in (\ref{ex:whostrikes}), contrasting with the M.H pattern found for M-tone morphemes, as in (\ref{ex:goingtostrike}). This difference indicates that the {relativizer} belongs to a~tonal category distinct from
those previously identified as L and M. 

\begin{exe}
	\ex
	\label{ex:whostrikes}
	\ipaex{lɑ˧-hĩ˧}\\
	\gll lɑ˧˥		-hĩ˥\\
	to\_strike		\textsc{relativizer/nominalizer}\\
	\glt ‘who strikes’
\end{exe}

\begin{exe}
	\ex
	\label{ex:goingtostrike}
	\ipaex{lɑ˧-bi˥}\\
	\gll lɑ˧˥		-bi˧\\
	to\_strike		\textsc{imm.fut}\\
	\glt ‘is going to strike’
\end{exe}

In assigning labels to tonal categories, I adopt the (debatable) assumption that the tone systems of different parts of speech rest on the same primitives (H, M, and L levels) and on combinations among these primitives (MH, LM, and LH). One argument for classifying the %tonal category of postverbal morphemes exemplified by the  
{relativizer/nominalizer} 
as an H-tone morpheme stems from the hypothesis that it \is{derivation!morphological}derives from the noun meaning ‘person, human being’, which bears H tone: /\ipa{hĩ˥}/ (on the \isi{grammaticalization} of this noun, see Lidz \citeyear[164, 183]{lidz2010}). This argument is not decisive, however. As noted earlier, the process of \isi{grammaticalization} often leads a~morpheme to enter a~tonal subsystem distinct from that of its original morphosyntactic class, sometimes resulting in a~tone that is widely different from that of the source word. 

Another argument for the H-tone classification is that, when combined with the {relativizer/nominalizer} \mbox{//\ipa{-hĩ˥}//}, adjectives in tone category L\textsubscript{b} (such as /\ipa{dʑɤ˩\textsubscript{b}}/
‘good’ and /\ipa{nɑ˩\textsubscript{b}}/ ‘black, dark’) yield an L.H
pattern, as shown in \tabref{tab:thetonalbehaviourofhtonesuffixes}. But this argument is not decisive either: since the rules are \is{morphotonology}morphotonological and not simply phonological, an H tone in the output cannot be taken as direct evidence of an H tone in the input. The existence of an LH tonal category for disyllabic nouns may act as an attractor, funnelling diverse morphological combinations towards an LH output. 

Adjectives and verbs behave quite differently when combined with the {relativizer/nominalizer} \mbox{//\ipa{-hĩ˥}//}. Unexpectedly, the tone patterns of verbs and those of adjectives are also very different: they coincide in only one of five categories (namely L\textsubscript{a}). This comes as a~surprise in view of the fact that, in many other contexts, the tonal behaviour of verbs and adjectives is pretty much the same. This discrepancy may have to do with semantic-syntactic differences in the function of the morpheme: when suffixed to an adjective, \mbox{//\ipa{-hĩ˥}//} functions as a~{nominalizer}, whereas when suffixed to  a~verb, it functions as a~{relativizer}. 

\begin{table}%[t]
	\caption{\label{tab:thetonalbehaviourofhtonesuffixes}The tonal behaviour of the {relativizer{\slash}nominalizer}, analyzed as having a~lexical H tone.}
	\begin{tabularx}{\textwidth}{ l@{\hspace{10mm}} l@{\hspace{10mm}} l@{\hspace{10mm}} l@{\hspace{10mm}} l }
		\lsptoprule
		word class & tone & example & meaning & with \textsc{relativizer}\\ \midrule
		verbs & H & \ipa{dzɯ˥} & to eat & \ipa{dzɯ˧-hĩ˧}\\
		& M\textsubscript{a} & \ipa{hwæ˧\textsubscript{a}} & to buy & \ipa{hwæ˧-hĩ˧}\\
		& M\textsubscript{b} & \ipa{tɕʰi˧\textsubscript{b}} & to sell & \ipa{tɕʰi˧-hĩ˧}\\
		& M\textsubscript{c} & \ipa{bi˧\textsubscript{c}} & to go & \ipa{bi˧-hĩ˧}\\
		& L\textsubscript{a} & \ipa{bæ˩\textsubscript{a}} & to sweep & \ipa{bæ˩-hĩ˩}\\
		& L\textsubscript{b} & \ipa{ʐwɤ˩\textsubscript{b}} & to speak & \ipa{ʐwɤ˩-hĩ˩}\\
		& MH & \ipa{lɑ˧˥} & to strike & \ipa{lɑ˧-hĩ˧}\\ \midrule
		adjectives & H & \ipa{bi˥} & shallow & \ipa{bi˧-hĩ\#˥}\\
		& M & \ipa{tɕi˧} & sour & \ipa{tɕi˧-hĩ\#˥}\\
		& L\textsubscript{a} & \ipa{hṽ̩˩\textsubscript{a}} & red & \ipa{hṽ̩˩-hĩ˩}\\
		& L\textsubscript{b} & \ipa{dʑɤ˩\textsubscript{b}} & good & \ipa{dʑɤ˩-hĩ˥}\\
		& MH & \ipa{tʰɑ˧˥} & sharp & \ipa{tʰɑ˧-hĩ˥\$}\\
		\lspbottomrule
	\end{tabularx}
\end{table}

The \textsc{topic} marker //\ipa{-dʑo˥}// commonly occurs after verb phrases, as well as after noun phrases. On the basis of its tonal behaviour following M-tone verbs (and, similarly, following M-tone nouns, as described in the preceding chapter), this \textsc{topic} marker is (provisionally) analyzed as carrying a~lexical H tone. Data on its behaviour in context is presented in \tabref{tab:thetonalbehaviourofthetopicmarkerwithverbs}. As in \tabref{tab:anotherpostverbalelementsoverwhichamhtonecannotunfoldthedurative}, the MH tone does not unfold onto the following morpheme: it is not correct to say $\ddagger${\kern2pt}\ipa{mɤ˧-lɑ˧-dʑo˥}. 

Importantly, the tone patterns for this \textsc{topic} marker, shown in Tables~\ref{tab:thetonalbehaviourofthetopicmarkerwithverbs} and \ref{tab:thetonalbehaviourofthetopicmarkerwithadjectives}, differ from those of the \textsc{relativizer}, shown in \tabref{tab:thetonalbehaviourofhtonesuffixes}. Instead, the \textsc{topic} marker exhibits stronger surface similarity to the \textsc{progressive} morpheme (documented in \tabref{tab:anotherpostverbalelementsoverwhichamhtonecannotunfoldthedurative}), which is analyzed as carrying lexical M tone. The sole difference between the \textsc{topic} and \textsc{progressive} morphemes lies in their realization following M-tone verbs: the \textsc{topic} marker surfaces with H tone, whereas the \textsc{progressive} retains M tone. 

The logical thing to do would be to recognize two \is{subcategories of lexical tones}subcategories of H-tone postverbal morphemes: H\textsubscript{a}, exemplified by the \textsc{relativizer}, and H\textsubscript{b}, exemplified by the \textsc{topic} marker. But categories containing only a single member are not immensely useful. The classification of postverbal morphemes into L, M, H or MH, as proposed in \sectref{sec:ltonesuffixesandserializedverbs}-\sectref{sec:mhtonesuffixes}, will likely require in-depth revision as a~more comprehensive picture of the tonal behaviour of postverbal elements emerges. At present, it remains premature to rigidify (hypostatize) these newly established tonal categories into a~seemingly neat and tidy system. Yongning Na is replete with \is{morphotonology}morphotonological anfractuosities; the present chapter, which employs “L tone”, “M tone”, “H tone”, and “MH tone” as convenient first-pass labels, only constitutes an initial step towards an orderly inventory and analysis.


%	\label{tab:thetonalbehaviourofthetopicmarker}
	\begin{table}%[t]
		\caption{\label{tab:thetonalbehaviourofthetopicmarkerwithverbs}The tonal behaviour of the \textsc{topic} marker with verbs.}
		\begin{tabularx}{\textwidth}{ l@{\hspace{11mm}} l@{\hspace{11mm}} l@{\hspace{11mm}} l@{\hspace{11mm}} Q }
			\lsptoprule
			tone & example & meaning & V+\textsc{top} & \textsc{neg}+V+\textsc{top}\\ \midrule
			H & \ipa{dzɯ˥} & to eat & \ipa{dzɯ˧-dʑo˩} & \ipa{mɤ˧-dzɯ˥-dʑo˩}\\
			M\textsubscript{a} & \ipa{hwæ˧\textsubscript{a}} & to buy & \ipa{hwæ˧-dʑo˥} & \ipa{mɤ˧-hwæ˧-dʑo˥}\\
			M\textsubscript{b} & \ipa{tɕʰi˧\textsubscript{b}} & to sell & \ipa{tɕʰi˧-dʑo˥} & \ipa{mɤ˧-tɕʰi˧-dʑo˥}\\
			M\textsubscript{c} & \ipa{bi˧\textsubscript{c}} & to go & \ipa{bi˧-dʑo˥} & \ipa{mɤ˧-bi˧-dʑo˥}\\
			L\textsubscript{a} & \ipa{bæ˩\textsubscript{a}} & to sweep & \ipa{bæ˩˥-dʑo˩} & \ipa{mɤ˧-bæ˩-dʑo˩}\\
			L\textsubscript{b} & \ipa{ʐwɤ˩\textsubscript{b}} & to speak & \ipa{ʐwɤ˩˥-dʑo˩} & \ipa{mɤ˧-ʐwɤ˩-dʑo˩}\\
			MH & \ipa{lɑ˧˥} & to strike & \ipa{lɑ˧˥-dʑo˩} & \ipa{mɤ˧-lɑ˧˥-dʑo˩}\\
			\lspbottomrule
		\end{tabularx}
	\end{table}
	
	\begin{table}%[t]
		\caption{\label{tab:thetonalbehaviourofthetopicmarkerwithadjectives}The tonal behaviour of the \textsc{topic} marker with adjectives.}
		\begin{tabularx}{\textwidth}{ l@{\hspace{11mm}} l@{\hspace{11mm}} l@{\hspace{11mm}} l@{\hspace{11mm}} Q }
			\lsptoprule
			tone & example & meaning & \textsc{Adj}+\textsc{top} & \textsc{neg}+\textsc{Adj}+\textsc{top}\\ \midrule
			H & \ipa{bi˥} & shallow & \ipa{bi˧-dʑo˩} & \ipa{mɤ˧-bi˥-dʑo˩}\\
			M & \ipa{tɕi˧} & sour & \ipa{tɕi˧-dʑo˥} & \ipa{mɤ˧-tɕi˧-dʑo˥}\\
			L\textsubscript{a} & \ipa{hṽ̩˩\textsubscript{a}} & red & \ipa{hṽ̩˩˥-dʑo˩} & \ipa{mɤ˧-hṽ̩˩-dʑo˩}\\
			L\textsubscript{b} & \ipa{dʑɤ˩\textsubscript{b}} & good & \ipa{dʑɤ˩˥-dʑo˩} & \ipa{mɤ˧-dʑɤ˩-dʑo˩}\\
			MH & \ipa{tʰɑ˧˥} & sharp & \ipa{tʰɑ˧˥-dʑo˩} & \ipa{mɤ˧-tʰɑ˧˥-dʑo˩}\\
			\lspbottomrule
		\end{tabularx}
	\end{table}


\subsection[MH tone]{MH-tone postverbal morphemes}
\label{sec:mhtonesuffixes}

Following the same exploratory method as outlined above in the discussion of the “L”, “M”, and “H” sets of postverbal morphemes, an “MH” set is postulated, likewise based on fragmentary evidence. The \textsc{abilitive} /\ipa{-kv̩˧˥}/ \is{derivation!morphological}derives from the verb /\ipa{kv̩˧˥}/ ‘to be able to'; this link, together with similarities in tonal behaviour, leads to the adoption of the label “MH” for the tone category of the \textsc{abilitive} and other morphemes that share its tonal behaviour. Three examples are provided in \tabref{tab:thetonalbehaviourofmh}: in addition to the \textsc{abilitive} /\ipa{-kv̩˧˥}/, they are the \textsc{permissive} /\ipa{-tʰɑ˧˥}/ and the \textsc{causative} /\ipa{kʰɯ˧˥}/. Other postverbal elements in this MH tonal class include the reported-speech particle /\ipa{tsɯ˧˥}/. 

The MH tone surfaces as such after M, in keeping with the general phonological tendency whereby the M tone does not interfere with
following tones. The MH tone is lowered to L after H; the interpretation proposed is that the H tone does not surface as such due to the
\isi{neutralization} of M and H in tone-group-initial position (this is formulated in \sectref{sec:alistoftonerules} as Rule~3: “In tone-group-initial position, H and M are neutralized to M''), but that this H tone is present underlyingly and lowers
all following tones to L (through Rules 4 and 5). an L tone on the verb spreads over an MH-tone postverbal morpheme (e.g.\ //\ipa{bæ˩\textsubscript{a}}// → //\ipa{bæ˩-kv̩˩}// ‘is apt to sweep'), as does
the H part of an MH \is{tonal contour}contour (//\ipa{lɑ˧˥}// → //\ipa{lɑ˧-kv̩˥}// ‘is apt to strike'), delinking the MH tone on the postverbal morpheme.

\begin{table}%[t]
 \caption{\label{tab:thetonalbehaviourofmh}The tonal behaviour of MH-tone morphemes after verbs and adjectives.}
{\setlength\tabcolsep{4.5pt}
\begin{tabularx}{\textwidth}{ l l l l l l Q }
\lsptoprule
	word class & tone & example & meaning & \textsc{abilitive} & \textsc{permissive} & \textsc{causative}\\ \midrule
	verbs &	H & \ipa{dzɯ˥} & to eat & \ipa{dzɯ˧-kv̩˩} & \ipa{dzɯ˧-tʰɑ˩} & \ipa{dzɯ˧ kʰɯ˩}\\
	& M\textsubscript{a} & \ipa{hwæ˧\textsubscript{a}} & to buy & \ipa{hwæ˧-kv̩˧˥} & \ipa{hwæ˧-tʰɑ˧˥} & \ipa{hwæ˧ kʰɯ˧˥}\\
	& M\textsubscript{b} & \ipa{tɕʰi˧\textsubscript{b}} & to sell & \ipa{tɕʰi˧-kv̩˧˥} & \ipa{tɕʰi˧-tʰɑ˧˥} & \ipa{tɕʰi˧ kʰɯ˧˥}\\
	& M\textsubscript{c} & \ipa{pv̩˧\textsubscript{c}} & to chant & \ipa{pv̩˧-kv̩˧˥} & \ipa{pv̩˧-tʰɑ˧˥} & \ipa{pv̩˧ kʰɯ˧˥}\\
	& L\textsubscript{a} & \ipa{bæ˩\textsubscript{a}} & to sweep & \ipa{bæ˩-kv̩˩} & \ipa{bæ˩-tʰɑ˩} & \ipa{bæ˩ kʰɯ˩}\\
	& L\textsubscript{b} & \ipa{ʐwɤ˩\textsubscript{b}} & to speak & \ipa{ʐwɤ˩-kv̩˩} & \ipa{ʐwɤ˩-tʰɑ˩} & \ipa{ʐwɤ˩ kʰɯ˩}\\
	& MH & \ipa{lɑ˧˥} & to strike & \ipa{lɑ˧-kv̩˥} & \ipa{lɑ˧-tʰɑ˥} &
   \ipa{lɑ˧ kʰɯ˥}\\ \midrule
	adjectives & H & \ipa{bi˥} & shallow & \ipa{bi˧-kv̩˩} & \ipa{bi˧-tʰɑ˩} & \ipa{bi˧ kʰɯ˩}\\
	& M & \ipa{tsʰi˧} & hot & \ipa{tsʰi˧-kv̩˧˥} & \ipa{tsʰi˧-tʰɑ˧˥} & \ipa{tsʰi˧ kʰɯ˧˥}\\
	& L\textsubscript{a} & \ipa{ɖɯ˩\textsubscript{a}} & large & \ipa{ɖɯ˩-kv̩˩} & \ipa{ɖɯ˩-tʰɑ˩} & \ipa{ɖɯ˩ kʰɯ˩}\\
	& L\textsubscript{b} & \ipa{dʑɤ˩\textsubscript{b}} & good & \ipa{dʑɤ˩-kv̩˧˥} & \ipa{dʑɤ˩-tʰɑ˥} & \ipa{dʑɤ˩ kʰɯ˥}\\
	& MH & \ipa{tʰɑ˧˥} & sharp & \ipa{tʰɑ˧-kv̩˥} & \ipa{tʰɑ˧-tʰɑ˥} & \ipa{tʰɑ˧ kʰɯ˥}\\
\lspbottomrule
\end{tabularx}}
\end{table}

\newpage
The phrase /\ipa{dʑɤ˩ kʰɯ˥}/ (‘good’+\textsc{causative}) is in common use as a~blessing on special
occasions such as the New Year. It could be translated as
‘Best wishes!’ or ‘Let there be good/happiness!’ Its L.H tone pattern is also observed on combinations that do not constitute set phrases, showing that it is not an \is{exceptions}exception. The other phrases in \tabref{tab:thetonalbehaviourofmh}~-- /\ipa{bi˧ kʰɯ˩}/
(‘shallow’+\textsc{causative}), /\ipa{ɖɯ˩ kʰɯ˩˥}/
(‘large’+\textsc{causative}), /\ipa{tsʰi˧ kʰɯ˧˥}/ (‘hot’+\textsc{causative}), and /\ipa{tʰɑ˧ kʰɯ˥}/
(‘sharp’{\linebreak}+\textsc{causative})~-- all have straightforward causative meanings: ‘to make shallow’, e.g.~to
reduce the level of water in a~field by decreasing the flow; ‘to enlarge’, e.g.~to
increase the size of a~farm by adding another building; ‘to heat up’; and ‘to sharpen’.

\largerpage
\tabref{tab:thetonalbehaviourofmh} shows that not all postverbal elements grouped under the provisional heading “MH” have exactly the same tonal behaviour. A~difference in tone pattern between the {causative} and {abilitive} constructions is found in association with L\textsubscript{b}-tone adjectives. (This difference has been verified through elicitation on several occasions; examples are found in \textit{Caravans.203} \pandoi{0004531\#S203} and \textit{ComingOfAge2.61} \pandoi{0004588\#S61}, \textit{72}, \textit{73}, \textit{91}, \textit{92}.) The pattern is L.MH in /\ipa{dʑɤ˩-kv̩˧˥}/ (‘good’+\textsc{abilitive}), and L.H in /\ipa{dʑɤ˩ kʰɯ˥}/ (‘good’+\textsc{causative}). This requires the recognition of at least two \is{subcategories of lexical tones}subcategories of MH-tone postverbal morphemes. As explained above, it appears too early at present to propose a definitive inventory. The labels “L tone”, “M tone”, “H tone”, and “MH tone” are used here as a~first-pass classification.


\section{Disyllabic postverbal morphemes}
\label{sec:disyllabicsuffixes}

\subsection{M.H tone}
\label{sec:mhtone}

A first tonal category of \is{disyllables}disyllabic postverbal elements is illustrated by /\ipa{-kwɤ˧{\allowbreak}tɕɯ˥}/ ‘after; as;
because’. This \is{postpositions}postposition most often appears as part of a~{trisyllabic} expression: /\ipa{-kwɤ˧tɕɯ˥-lɑ˩}/. A~hyphen is used before the syllable /\ipa{-lɑ˩}/ to reflect the fact that this last syllable can be
detached from the other two. In texts, out of 140 examples, ten are without /\ipa{-lɑ˩}/, as in (\ref{ex:lake35354}). (The other examples are \textit{Lake3.54},
\textit{59}, \textit{67}, \textit{Sister.34}, \textit{Sister3.133}, \textit{Caravans.80}, \textit{137}, \textit{Renaming.18}, and
\textit{BuriedAlive2.48}.) 

\begin{exe}
	\ex
	\label{ex:lake35354}
	\ipaex{bo˩-gv̩˥, {\kern2pt}|{\kern2pt} bo˩-hɑ˥ ki˩-hĩ˩=bv̩˩, {\kern2pt}|{\kern2pt} ʈʂʰwæ˩-ne˩˥ {\kern2pt}|{\kern2pt} ɖɯ˧-ɭɯ˧ {\kern2pt}|{\kern2pt} dʑo˧-kwɤ˧tɕɯ˥, {\kern2pt}|{\kern2pt} ʈʂʰɯ˧-qo˧ {\kern2pt}|{\kern2pt} tʰi˧-dzi˩-kwɤ˩tɕɯ˩, {\kern2pt}|{\kern2pt} tɕʰo˩˥ {\kern2pt}|{\kern2pt} ɖɯ˧-nɑ˧ {\kern2pt}|{\kern2pt} tʰi˧-po˧ tsɯ˥ {\kern2pt}|{\kern2pt} mv̩˩.}\\
	\gll bo˩-gv̩˥		bo˩-hɑ\#˥		ki˧\textsubscript{a}	-hĩ˥	=bv̩˧		ʈʂʰwæ˩	-ne		ɖɯ˧-ɭɯ˧		dʑo˧\textsubscript{b}	-kwɤ˧tɕɯ˥		ʈʂʰɯ˧-qo˧	tʰi˧-		dzi˩\textsubscript{a}		-kwɤ˧tɕɯ˥	tɕʰo˩˧		ɖɯ˧-nɑ˧		tʰi˧-	po˧˥	-tsɯ˧˥		mv̩˧\\
	pig\_manger		pig\_feed		to\_give	\textsc{nmlz}	\textsc{poss}		boat~(\textit{loan:~Chinese}~\zh{船})		like	one-\textsc{clf}		\textsc{exist}	as			here		\textsc{dur}	to\_sit		as	ladle	one-\textsc{clf}.tools	\textsc{dur}	to\_bring		\textsc{rep}	\textsc{affirm}\\
	\glt ‘As there was a~pig manger, [you know,] the thing for giving swill, that was like a~boat (=that had the shape of a~boat), as [they] sat [in this manger]{\dots} it is said that they brought a~ladle [with them].’ \textit{(Lake3.53-54)} \pandoi{0004348\#S53}
\end{exe}

Thus, while addition of /\ipa{-lɑ˩}/ (presumably the morpheme //\ipa{lɑ˧}//, meaning ‘and, also’) is
a~well-established habit, the expression can also be used without it. No special nuance of
meaning was identified, except that the formulation without /\ipa{-lɑ˩}/ is perceived as more
pithy and economical. Example (\ref{ex:lake35354}) clarifies that the presence or absence of /\ipa{-lɑ˩}/ is not conditioned by the semantic interpretation of the \is{postpositions}postposition as meaning either ‘when; after’ or ‘because, since’: among the two occurrences in (\ref{ex:lake35354}), both without an accompanying /\ipa{-lɑ˩}/, the \is{postpositions}postposition /\ipa{-kwɤ˧tɕɯ˥}/ has a~causal reading (‘since, because’) in the first instance and a~temporal reading (‘as, when’) in the second. 

The {monosyllabic} form $\dagger${\kern2pt}\ipa{-kwɤ˧} is not attested in this dialect, even though it is reported in another hamlet of the Yongning plain, Walabie (\ipa{ʁwɤ˧lɑ˩-bi˩}; Chinese: \zh{瓦拉片}) (Roselle Dobbs, p.c.\ 2016).

\begin{table}%[t]
\caption{\label{tab:thetonalbehaviourofafterbecause}The tonal behaviour of /\ipa{-kwɤ˧tɕɯ˥}(\ipa{-lɑ˩})/‚ ‘after; because’.}
\begin{tabularx}{\textwidth}{ Q l@{\hspace{8mm}} Q Q P{40mm} }
\lsptoprule
	word class & tone & example & meaning & V+‘after; because’\\ \midrule
	verbs & H & \ipa{dzɯ˥} & to eat & \ipa{dzɯ˧-kwɤ˩tɕɯ˩(-lɑ˩)}\\
	& M\textsubscript{a} & \ipa{hwæ˧\textsubscript{a}} & to buy & \ipa{hwæ˧-kwɤ˧tɕɯ˥(-lɑ˩)}\\
	& M\textsubscript{b} & \ipa{tɕʰi˧\textsubscript{b}} & to sell & \ipa{tɕʰi˧-kwɤ˧tɕɯ˥(-lɑ˩)}\\
	& M\textsubscript{c} & \ipa{bi˧\textsubscript{c}} & to go & \ipa{bi˧-kwɤ˧tɕɯ˥(-lɑ˩)}\\
	& L\textsubscript{a} & \ipa{bæ˩\textsubscript{a}} & to sweep & \ipa{bæ˩-kwɤ˩tɕɯ˥(-lɑ˩)}\\
	& L\textsubscript{b} & \ipa{ʐwɤ˩\textsubscript{b}} & to speak & \ipa{ʐwɤ˩-kwɤ˩tɕɯ˥(-lɑ˩)}\\
	& MH & \ipa{lɑ˧˥} & to strike & \ipa{lɑ˧˥-kwɤ˩tɕɯ˩-(lɑ˩) /
     lɑ˧-kwɤ˥tɕɯ˩-(lɑ˩)}\\ \midrule
	adjectives & H & \ipa{bi˥} & shallow & \ipa{bi˧-kwɤ˩tɕɯ˩(-lɑ˩)}\\
	& M & \ipa{tɕi˧} & sour & \ipa{tɕi˧-kwɤ˧tɕɯ˥(-lɑ˩)}\\
	& L\textsubscript{a} & \ipa{hṽ̩˩\textsubscript{a}} & red & \ipa{hṽ̩˩-kwɤ˩tɕɯ˥(-lɑ˩)}\\
	& L\textsubscript{b} & \ipa{dʑɤ˩\textsubscript{b}} & good & \ipa{dʑɤ˩˥-kwɤ˧tɕɯ˥(-lɑ˩)}\\
	& MH & \ipa{tʰɑ˧˥} & sharp & \ipa{tʰɑ˧˥-kwɤ˩tɕɯ˩(-lɑ˩)}\\
\lspbottomrule
\end{tabularx}
\end{table}

The lexical tone of /\ipa{-kwɤ˧tɕɯ˥}/ is inferred from its behaviour in combination with the adjective
/\ipa{dʑɤ˩\textsubscript{b}}/ ‘good’. The observed pattern is /\ipa{dʑɤ˩˥-kwɤ˧tɕɯ˥-lɑ˩}/, which does not constitute
a~well-formed \isi{tone group} (since it contains two H tones) and must therefore be analyzed as
a~sequence of two tone groups: /\ipa{dʑɤ˩˥ {\kern2pt}|{\kern2pt} -kwɤ˧tɕɯ˥-lɑ˩}/. In this context, the
tones carried by /\ipa{-kwɤ˧tɕɯ˥}/, namely \mbox{/M.H/}, must be supposed to reflect the underlying tone of the expression: a~final H tone, i.e.\ \mbox{//H\#//}. As for the added morpheme, interpreted as //\ipa{lɑ˧}//,  meaning ‘and, also’, it receives L tone through Rule~4 (“A syllable following an H-tone syllable receives L tone”).

The sequence /\ipa{dʑɤ˩˥-kwɤ˧tɕɯ˥-lɑ˩}/ ‘because/since [it is] good’ provides evidence for cases in which a~rising \is{tonal contour}contour is realized on the verb and does not unfold over the
postverbal expression. Another example is with MH-tone verbs and adjectives, e.g.~/\ipa{lɑ˧˥-kwɤ˩tɕɯ˩(-lɑ˩)}/
‘because/since [someone] beat [something]’. Variants in which the MH \is{tonal contour}contour unfolds over the first
syllable of the following morpheme (/\ipa{lɑ˧-kwɤ˥tɕɯ˩(-lɑ˩)}/) are considered acceptable, and one example occurs in a~text (\textit{BuriedAlive2.48} \pandoi{0004536\#S48}), but the predominant pattern is one where the \is{tonal contour}contour
remains on the verb. 

Since contours are only realized tone-group-finally, this suggests that there is a~tone-group \is{boundary (between tone groups)}boundary before the \is{postpositions}postposition //\ipa{-kwɤ˧tɕɯ˥}// ‘as, because’. Such behaviour is not unparalleled in this dialect. For instance, the contrastive topic marker //\ipa{-no˧˥}// and the
word /\ipa{tʰi˩˥}/ ‘then’ always mark the beginning of a~new \isi{tone group}. This interpretation is
consistent with the observation that the syllable preceding //\ipa{-kwɤ˧tɕɯ˥}// ‘as, because’ tends to be \is{lengthening}lengthened~-- a~cue to the presence of an intonational {boundary}.\footnote{On articulatory, acoustic and perceptual cues to intonational boundaries, see \citet{byrdElastic2003}.} In early transcriptions, a~comma was used
to reflect a~perceived pause, transcribing e.g.~/\ipa{le˧-tsɑ˧˥, {\kern2pt}|{\kern2pt} -kwɤ˩tɕɯ˩}/ ‘because [they] rowed’ (\textit{Lake3.59} \pandoi{0004348\#S59}).

But a~difficulty with this analysis is that after certain tonal categories of verbs, the tone patterns that are observed on //\ipa{-kwɤ˧tɕɯ˥}// ‘because’ suggest that
the verb and the postverbal element belong to the same \isi{tone group}. Spreading of an L tone from the verb onto the following syllable demonstrates that the verb and its postverbal element are integrated into the same \isi{tone group}.\footnote{In the case of M-tone verbs, it is impossible to determine the underlying division of the expression into tone groups based on surface tones, because the observed pattern is \mbox{/M.M.H/} in both cases and could correspond to either of the following interpretations: M.M.H and M \ipa{{\kern1pt}|{\kern1pt}} M.H, e.g.~/\ipa{hwæ˧-kwɤ˧tɕɯ˥-lɑ˩}/ or /\ipa{{\kern1pt}|{\kern1pt} hwæ˧ {\kern1pt}|{\kern1pt} -kwɤ˧tɕɯ˥-lɑ˩}/ for ‘because (she/he) buys’. Since the M tone does not affect the following tone, the surface phonological output is the same in both cases.}

% the presence of a~boundary has no effect on the tonal string. The lexical M tone is realized as such on the verb whether followed by a tone-group boundary or not, and the postposition carries an M.H pattern whether it makes up a~tone group of its own or is integrated within the same group as an M-tone verb. The sequences M.M.H and M \ipa{{\kern1pt}|{\kern1pt}} M.H seem to be indistinguishable, e.g.~/\ipa{hwæ˧-kwɤ˧tɕɯ˥-lɑ˩}/ and /\ipa{{\kern1pt}|{\kern1pt} hwæ˧ {\kern1pt}|{\kern1pt} -kwɤ˧tɕɯ˥-lɑ˩}/ for ‘because (she/he) buys’.} 

One possible way of handling this would be to postulate that the division into tone groups is
determined by the lexical tone of the words at issue. There would be a single \isi{tone group} (e.g.~/{\kern2pt}|{\kern2pt}~\ipa{hwæ˧-kwɤ˧tɕɯ˥-lɑ˩}~{\kern2pt}|{\kern2pt}/ ‘when [she/he] buys’) except with an MH-tone verb (or an MH-tone or L\textsubscript{b}-tone adjective), where the division would be as follows: \mbox{/\ipa{{\kern2pt}|{\kern2pt} lɑ˧˥}} \ipa{{\kern2pt}|{\kern2pt} -kwɤ˩tɕɯ˩-lɑ˩ {\kern2pt}|{\kern2pt}}/ ‘when [she/he] strikes’. However, an insuperable difficulty arises: the MH tone permits two variants, one where the two morphemes belong to separate tone groups (no tonal interaction: /\ipa{lɑ˧˥ {\kern2pt}|{\kern2pt} -kwɤ˧tɕɯ˥}/) and another where they are integrated into the same group (the H part of the MH \is{tonal contour}contour associates with the following syllable, triggering a~lowering of the following tone to L, hence /\ipa{lɑ˧-kwɤ˥tɕɯ˩}/). 

A further puzzle is that the behaviour of the MH tone (for verbs and adjectives) differs from that of the L\textsubscript{b} tone (for adjectives). The tones
of the postverbal element are all lowered to L when following an MH-tone verb, whereas they surface unscathed after the rising \is{tonal contour}contour on an L\textsubscript{b}-tone adjective. Lowering to L after
a~preceding H tone level is a~{phonological rule} in the Alawua dialect of Yongning Na (Rule~4: “The syllable following an H-tone syllable receives L tone”; see \sectref{sec:alistoftonerules}); the tone rules operate within the \isi{tone group}, never across
a~tone-group \is{juncture (inside a tone group)}juncture. The fact that /\ipa{kwɤ.tɕɯ-lɑ}/ is lowered to L
after an MH-tone verb (e.g.~/\ipa{lɑ˧˥-kwɤ˩tɕɯ˩-lɑ˩}/ for ‘to strike’) strongly suggests that the
expression /\ipa{kwɤ.tɕɯ-lɑ}/ does not constitute an independent \isi{tone group}. The sequence
\mbox{/\ipa{-kwɤ˩tɕɯ˩-lɑ˩}/} in /\ipa{lɑ˧˥-kwɤ˩tɕɯ˩-lɑ˩}/ ‘when (she/he) strikes’ is not
a~well-formed \isi{tone group}, since it only contains L tones.

This special situation is analyzed as the result of a~\is{stylistics}stylistic process of \isi{emphasis} that has become habitually associated with certain morphemes. The analysis is set out as part of the discussion of the \isi{tone group} as a~key phonological unit in Yongning Na (\sectref{sec:thedivisionofutterancesintotonegroups}).


\subsection{M.L tone and H.L tone}
\label{sec:mltone}

Two categories of disyllabic postverbal elements have a~tantalizingly similar behaviour. They are illustrated
by the \is{postpositions}postposition /\ipa{-ʁo˧to˩}/ ‘on top of; during’ and the adverb /\ipa{pʰæ˥di˩}/ ‘as if/it seems that’, as shown in
\tabref{tab:thebehaviouroftwodisyllabicsuffixesanalyzedascarryingmltone}.\footnote{The adverb ‘as if/it seems that’ was transcribed with an M.L pattern in the first edition of this book: /\ipa{pʰæ˧di˩}/. Since then, it has been recategorized as H.L on the basis of its behaviour after the negation: the outcome is M.H.L. 
%A similar recategorization for /\ipa{-ʁo˧to˩}/ ‘on top of; during’ may well be in order, given the almost complete identity of the tone patterns for the two morphemes and, specifically, the behaviour of ‘on top of’ in combination with the M-tone noun ‘tiger’, likewise M.H.L.
}

Examples from texts are provided below, illustrating cases where /\ipa{pʰæ˥di˩}/ is preceded by a verb with H tone (\ref{ex:shortage68}), L\textsubscript{a} tone (\ref{ex:tiger14}), and L\textsubscript{b} tone (\ref{ex:buriedalive24}), as well as by a~postverbal element carrying MH tone (\ref{ex:dog108}).

 \begin{exe}
 	\ex
 	\label{ex:shortage68}
 	\ipaex{tʰi˩˥ {\kern2pt}|{\kern2pt} hĩ˧=ɻæ˥-dʑo˩ {\kern2pt}|{\kern2pt} wɤ˩˥ {\kern2pt}|{\kern2pt} ʝi˧kʰv̩˥-dʑo˩ | mv̩˧ pʰæ˧di˥!}\\
 	\gll tʰi˩˥		hĩ˥	=ɻæ˩		-dʑo˥			wɤ˩˥		ʝi˧kʰv̩˥	-dʑo˥				mv̩˥					pʰæ˥di˩\\
 	then	people		\textsc{pl}	\textsc{top}	again	some		\textsc{top}	to\_understand		it\_seems\_that\\
 	\glt ‘It seems that some people understood in the end!' \textit{(FoodShortage2.72)} \pandoi{0004657\#S72}
 \end{exe}
 
\begin{exe}
  	\ex
  	\label{ex:tiger14}
  	\ipaex{mv̩˩ ʈʂʰɯ˩-v̩˩˥ {\kern2pt}|{\kern2pt} qʰwɤ˩ pʰæ˩di˥-dʑo˩ {\kern2pt}|{\kern2pt} tʰi˩˥ {\kern2pt}|{\kern2pt} kʰi˧ {\kern2pt}|{\kern2pt} tʰi˧-tv̩˧ tsɯ˥ {\kern2pt}|{\kern2pt} mv̩˩.}\\
  	\gll mv̩˩˥		ʈʂʰɯ˥				v̩˧								qʰwɤ˩\textsubscript{a}	pʰæ˥di˩		-dʑo˥				tʰi˩˥	kʰi˥		tʰi˧-					tv̩˧˥		tsɯ˧˥			mv̩˧\\
  	daughter	\textsc{dem}		\textsc{clf}.individual		clever	it\_seems\_that		\textsc{top}	then	door	\textsc{dur}		to\_support	\textsc{rep}	\textsc{affirm}\\
  	\glt ‘That girl was clever, as it turned out: she propped herself against the door. / That girl reacted smartly: she immediately propped herself against the door [so as to keep the tiger out].' \textit{(Tiger.14)} \pandoi{0004444\#S14}
\end{exe}
  
\begin{exe}
   	\ex
   	\label{ex:buriedalive24}
   	\ipaex{le˧-ʂɯ˧ le˧-nv̩˥-dʑo˩ {\kern2pt}|{\kern2pt} tʰi˩˥ {\kern2pt}|{\kern2pt} hĩ˧ {\kern2pt}|{\kern2pt} ɬo˧tɑ˧ {\kern2pt}|{\kern2pt} wɤ˩˥ {\kern2pt}|{\kern2pt} ɖɯ˧-v̩˧ ɳɯ˧ {\kern2pt}|{\kern2pt} do˩ pʰæ˩di˥ tsɯ˩ {\kern2pt}|{\kern2pt} mv̩˩!}\\
	 \gll le˧-		ʂɯ˧\textsubscript{a}	le˧-	nv̩˥		-dʑo˥		tʰi˩˥		hĩ˥		ɬo˧tɑ˧		wɤ˩˥		ɖɯ˧	v̩˧		ɳɯ˧	do˩\textsubscript{b}	pʰæ˥di˩		 tsɯ˧˥		-mv̩˧\\
   	\textsc{accomp}	to\_die		\textsc{accomp}	to\_bury	\textsc{top}	then	person	to\_the\_side	again	one		\textsc{clf}.individual	\textsc{a}	to\_see			it\_seems\_that			\textsc{rep}	\textsc{affirm}\\
   	\glt ‘When she died and was buried{\dots} apparently, someone close by had seen [what had really happened to her]!' \textit{(BuriedAlive2.24)} \pandoi{0004536\#S24}
\end{exe}
   
\begin{exe}
    	\ex
    	\label{ex:dog108}
    	\ipaex{kʰv̩˩mi˩ lɑ˥ {\kern2pt}|{\kern2pt} ɖʐv̩˧nɑ˥mi˩ ʈʂʰɯ˩-dʑo˩ {\kern2pt}|{\kern2pt} ɖɯ˧-pi˧˥ {\kern2pt}|{\kern2pt} tʰi˧-kv̩˧ pʰæ˥di˩ mæ˩!}\\
    	\gll kʰv̩˩mi˩	lɑ˧	ɖʐv̩˧nɑ˥mi˩		ʈʂʰɯ˧		-dʑo˥		ɖɯ˧-pi˧˥	tʰi˧				-kv̩˧˥		pʰæ˥di˩			mæ˧\\
    	dog			also	heron	\textsc{top}	\textsc{top}	a\_little		skilful		\textsc{abilitive}		it\_seems\_that	\textsc{obviousness}\\
    	\glt ‘The Dog and the Heron were rather talented, it seems!' \textit{(Dog2.108)} \pandoi{0004555\#S108}
\end{exe}

{\setlength\tabcolsep{4.5pt}
\begin{table}%[t]
\caption{\label{tab:thebehaviouroftwodisyllabicsuffixesanalyzedascarryingmltone}The behaviour of a disyllabic postposition analyzed as carrying M.L tone and a disyllabic adverb carrying H.L tone.}
\begin{tabularx}{\textwidth}{ l l l l Q }
\lsptoprule
	tone & example & meaning & V+\ipa{/pʰæ˥di˩/} ‘as if’ & V+\ipa{/-ʁo˧to˩/} ‘during’\\ \midrule
	H & \ipa{dzɯ˥} & to eat & \ipa{dzɯ˧ pʰæ˧di˥} & \ipa{dzɯ˧-ʁo˧to˩}\\
	M\textsubscript{a} & \ipa{hwæ˧\textsubscript{a}} & to buy & \ipa{hwæ˧ pʰæ˧di˩} & \ipa{hwæ˧-ʁo˧to˩}\\
	M\textsubscript{b} & \ipa{tɕʰi˧\textsubscript{b}} & to sell & \ipa{tɕʰi˧ pʰæ˧di˩} & \ipa{tɕʰi˧-ʁo˧to˩}\\
	M\textsubscript{c} & \ipa{pv̩˧\textsubscript{c}} & to chant & \ipa{pv̩˧ pʰæ˧di˩} & \ipa{pv̩˧-ʁo˧to˩}\\
	L\textsubscript{a} & \ipa{bæ˩\textsubscript{a}} & to sweep & \ipa{bæ˩ pʰæ˩di˥} & \ipa{bæ˩-ʁo˩to˥}\\
	L\textsubscript{b} & \ipa{ʐwɤ˩\textsubscript{b}} & to speak & \ipa{ʐwɤ˩ pʰæ˩di˥ } & \ipa{ʐwɤ˩-ʁo˩to˥}\\
	MH & \ipa{lɑ˧˥} & to strike & \ipa{lɑ˧ pʰæ˥di˩ ≈ lɑ˧ pʰæ˧di˥} & \ipa{lɑ˧-ʁo˥to˩}\\
\lspbottomrule
\end{tabularx}
\end{table}}

The only difference in tone patterns between the M.L postposition and the H.L adverb is found in their behaviour when following an H-tone verb. This
difference, however, is sufficient to recognize distinct \is{morphotonology}morphotonological categories for the adverb \ipa{/pʰæ˥di˩/} ‘as if’ and the postposition \ipa{/-ʁo˧to˩/} ‘during’. 

At present, there is no telling whether the different tonal behaviour of these two morphemes stems from their different parts of speech (postpositions vs.\ adverbs), from their internal structure, or from a purely tonal difference, as two categories contrasting with each other within the same paradigm~-- like the various distinctions transcribed in this volume by means of subscript letters added to the tone (e.g.~in the M\textsubscript{a}, M\textsubscript{b}, and M\textsubscript{c} \is{subcategories of lexical tones}subcategories of M-tone verbs). This is one of many issues requiring further investigation.

\section{Combinations of postverbal morphemes}
\label{sec:combinationsbetweenaffixes}

\subsection{Postverbal morphemes preceded by the {negation} prefix}
\label{sec:suffixesprecededbythenegation}

It is common for a~verb to be separated from a~following morpheme by the {negation}, as
in example (\ref{ex:wasnotabletotakeoff}).
\begin{exe}
  \ex
  \label{ex:wasnotabletotakeoff}
  \ipaex{{\dots} pʰv̩˧ mɤ˥-tʰɑ˩-ɲi˩ ho˩ mæ˩!}\\
  \gll pʰv̩˧˥	mɤ˧	tʰɑ˧˥	-ɲi˩	ho˩	mæ˧\\
  to\_take\_off	\textsc{neg}	\textsc{permissive}	\textsc{certitude}	\textsc{desiderative}
  \textsc{obviousness}\\
  \glt ‘[I] was not able to take off [the bracelets]!' \textit{(BuriedAlive2.88)} \pandoi{0004536\#S88}
\end{exe}

One might expect the tonal behaviour of such expressions to be computed progressively
(“left-to-right”) in some simple way. But as is often the case in Yongning Na \isi{morphotonology}, no simple algorithm accounts for all the data. A~telling example is that of MH-tone verbs, such as
/\ipa{pʰv̩˧˥}/ ‘to take off’. In example (\ref{ex:wasnotabletotakeoff}), the H part of the MH \is{tonal contour}contour reassociates to the
{negation} \is{prefixes}prefix, yielding /\ipa{pʰv̩˧-mɤ˥}{\dots}/, which in turn triggers the lowering of all following tones to
L (through Rules 4 and 5). When the postverbal morpheme carries M tone, on the other hand, the H part of the MH
\is{tonal contour}contour is absent from the surface phonological form, e.g.~in /\ipa{pʰv̩˧ mɤ˧-bi˧}/ ‘will
not take off’. In such cases, the MH \is{tonal contour}contour does not unfold ($\ddagger${\kern2pt}\ipa{pʰv̩˧-mɤ˥-bi˩} is not even an acceptable \is{variants}variant); the contour is truncated, as it were, deleting its H component.

The data is therefore presented here in static tabular form rather than as a~set of rules.


\subsubsection*{M-tone postverbal morphemes preceded by the {negation}}

\tabref{tab:thetonepatternsofcombinationsbetweenverbnegationandmtoneortonelesssuffixes} shows combinations of a~verb, a {negation} \is{prefixes}prefix, and a~morpheme carrying M\textsubscript{a} or M\textsubscript{b} tone. The example morphemes are the \textsc{immediate future}, /\ipa{bi˧\textsubscript{a}}/, and /\ipa{mæ˧\textsubscript{b}}/ ‘to succeed, to
achieve’. The verb used to illustrate the M\textsubscript{c} tone category in \tabref{tab:thetonepatternsofcombinationsbetweenverbnegationandmtoneortonelesssuffixes} is /\ipa{ʝi˧\textsubscript{c}}/ ‘to come' rather than
/\ipa{bi˧\textsubscript{c}}/ ‘to go' (standardly used as an example of the M\textsubscript{c} tonal category of verbs), in order to avoid the form /\ipa{bi˧ mɤ˧-bi˧}/ ‘am not going to go'. This expression is well-formed but seriously misleading outside context, as the \isi{homophony} between the verb and the postverbal morpheme makes the entire expression \is{homophony}homophonous with a~V \textsc{neg}-V construction meaning ‘whether [she/he{\dots}] goes or not'.

{\setlength\tabcolsep{4pt}
	\begin{table}%[t]
	\caption{\label{tab:thetonepatternsofcombinationsbetweenverbnegationandmtoneortonelesssuffixes}Combinations of verb, {negation} prefix and M-tone morpheme.}
	\begin{tabularx}{\textwidth}{ l@{\hspace{6mm}} l@{\hspace{6mm}} l@{\hspace{6mm}} l@{\hspace{6mm}} Q }
	\lsptoprule
		tone & example & meaning & ‘am not going to V’  & ‘cannot manage to V’\\
		 &  &  & (tone: M\textsubscript{a}) & (tone: M\textsubscript{b})\\ \midrule
		H & \ipa{dzɯ˥} & to eat & \ipa{dzɯ˧ mɤ˧-bi˧} & \ipa{dzɯ˧ mɤ˧-mæ˩}\\
		M\textsubscript{a} & \ipa{hwæ˧\textsubscript{a}} & to buy & \ipa{hwæ˧ mɤ˧-bi˧} & \ipa{hwæ˧ mɤ˧-mæ˧}\\
		M\textsubscript{b} & \ipa{tɕʰi˧\textsubscript{b}} & to sell & \ipa{tɕʰi˧ mɤ˧-bi˧} & \ipa{tɕʰi˧ mɤ˧-mæ˧}\\
		M\textsubscript{c} & \ipa{ʝi˧\textsubscript{c}} & to come & \ipa{ʝi˧ mɤ˧-bi˧} & \ipa{ʝi˧ mɤ˧-mæ˧}\\
		L\textsubscript{a} & \ipa{dze˩\textsubscript{a}} & to cut & \ipa{dze˩ mɤ˩-bi˩˥} & \ipa{dze˩ mɤ˩-mæ˥}\\
		L\textsubscript{b} & \ipa{ʈʰɯ˩\textsubscript{b}} & to drink & \ipa{ʈʰɯ˩ mɤ˩-bi˩˥} & \ipa{ʈʰɯ˩ mɤ˩-mæ˥}\\
		MH & \ipa{ʈʰæ˧˥} & to bite & \ipa{ʈʰæ˧ mɤ˧-bi˧} & \ipa{ʈʰæ˧ mɤ˥-mæ˩}\\
	\lspbottomrule
	\end{tabularx}
	\end{table}
}


\subsubsection*{MH-tone postverbal morphemes preceded by \textsc{neg} or \textsc{prohib}}

This paragraph examines three constructions that share the same tone patterns: 
(i)~/\ipa{V mɤ˧-tʰɑ˧˥}/, V-\textsc{neg}-\textsc{permissive}:  ‘[one] must not V’; (ii)~/\ipa{V mɤ˧-kʰɯ˧˥}/, V-\textsc{neg}-\textsc{caus}: ‘not to let [someone] V’; and (iii)~/\ipa{V tʰɑ˧-kʰɯ˧˥}/, V-\textsc{prohib}-\textsc{caus}: ‘do not cause to V’, ‘do not let [someone] V’. 


% \begin{itemize}
% 	\item{/\ipa{V mɤ˧-tʰɑ˧˥}/, V-\textsc{neg}-\textsc{permissive}:  ‘[one] must not V’;}
% 	\item{/\ipa{V mɤ-kʰɯ˧˥}/, V-\textsc{neg}-\textsc{caus}: ‘not to let [someone] V’;}
% 	\item{/\ipa{V tʰɑ˧-kʰɯ˧˥}/, V-\textsc{prohib}-\textsc{caus}: ‘do not cause to V’, ‘do not let [someone] V’.}
% \end{itemize}

% Note from 2016: 'Used to be a list. But it took up a lot of space on the page. In case this is restored: to make lines closer to one another: \vspace{-3mm}'
% List format restored in 2025.
%\begin{itemize}
%	\item{/\ipa{V mɤ˧-tʰɑ˧˥}/, V-\textsc{neg}-\textsc{permissive}:  ‘[one] must not V’}
%	\item{/\ipa{V mɤ-kʰɯ˧˥}/, V-\textsc{neg}-\textsc{causative}: ‘not to let [someone] V’}
%	\item{/\ipa{V tʰɑ˧-kʰɯ˧˥}/, V-\textsc{prohibitive}-\textsc{causative}: ‘do not cause to V’, ‘do not let [someone] V’}
%\end{itemize}

Examples from texts include (\ref{ex:whensomeonediesdonotclothe})-(\ref{ex:difflength}).

\begin{exe}
  \ex
  \label{ex:whensomeonediesdonotclothe}
  \ipaex{le˧-ʂɯ˧-dʑo˧, {\kern2pt}|{\kern2pt} dʑi˧hṽ̩˥ {\kern2pt}|{\kern2pt} mv̩˧ mɤ˧-kʰɯ˧˥! {\kern2pt}|{\kern2pt}}\\
  \gll le˧-	ʂɯ˧	-dʑo˧	dʑi˧hṽ̩˥\$	mv̩˧		mɤ˧-	-kʰɯ˧˥\\
  \textsc{accomp}	to\_die	\textsc{prog}	clothes	to\_put\_on	\textsc{neg}	\textsc{caus}\\
  \glt ‘When someone dies, [we] do not clothe [the corpse]!’ \textit{(BuriedAlive3.58)} \pandoi{0004538\#S58}

  \ex
  \label{ex:burncorpse}
  \ipaex{so˩ɲi˩-so˩hɑ̃˥ {\kern2pt}|{\kern2pt} qæ˧˥ {\kern2pt}|{\kern2pt} -dʑo˩ {\kern2pt}|{\kern2pt} tʰi˩˥, {\kern2pt}|{\kern2pt} le˧-qæ˧˥, {\kern2pt}|{\kern2pt} mv̩˩-mɤ˩-tʰɑ˥!}\\
  \gll so˩ɲi˩-so˩hɑ̃˥	qæ˧˥	-dʑo˥	tʰi˩˥	le˧-		qæ˧˥	 mv̩˩			mɤ˧-	-tʰɑ˧˥\\
  3\_days\_and\_nights	to\_burn	\textsc{top}	then	\textsc{accomp}	to\_burn to\_consume/to\_burn\_up
  \textsc{neg}	possible\\
  \glt ‘The corpse was burnt [on the pyre] for three days and three nights, but it was not possible
  to burn it up!’ \textit{(Sister3.93)} \pandoi{0004344\#S93}

  \ex
  \label{ex:difflength}
  \ipaex{ʂæ˧ɖæ˧ {\kern2pt}|{\kern2pt} di˩-tʰɑ˩-kʰɯ˥!}\\
  \gll ʂæ˧ɖæ˧ di˩\textsubscript{a}	-tʰɑ˧˥	-kʰɯ˧˥\\
  differences\_in\_length		\textsc{exist}	\textsc{prohib}	\textsc{caus}\\
  \glt ‘[The tree trunks] must not be different lengths! / There must not be differences in length!’ \textit{Context:} selecting trees that will be felled as lumber for building a~house. \textit{(Housebuilding.19)} \pandoi{0004448\#S19}
\end{exe}

The data for all tone categories of verbs is shown in Tables \ref{tab:mustnotvdonotlet} and \ref{tab:cannotv}, demonstrating the full identity of tone patterns across these three constructions.

\begin{table}%[t]
\caption{\label{tab:mustnotvdonotlet}The tone patterns of /\ipa{V mɤ˧-tʰɑ˧˥}/ ‘[one] must not V’ and /\ipa{V tʰɑ˧-kʰɯ˧˥}/ ‘do not let [someone] V/do not cause to V’.}
\begin{tabularx}{\textwidth}{ l@{\hspace{7mm}} l@{\hspace{7mm}} l@{\hspace{7mm}} l@{\hspace{7mm}} Q }
\lsptoprule
	tone & example & meaning & ‘[one] must not V’ & ‘do not cause to V’\\ \midrule
	H & \ipa{dzɯ˥} & to eat & \ipa{dzɯ˧ mɤ˧-tʰɑ˩} & \ipa{dzɯ˧ tʰɑ˧-kʰɯ˩}\\
	M\textsubscript{a} & \ipa{hwæ˧\textsubscript{a}} & to buy & \ipa{hwæ˧ mɤ˧-tʰɑ˧˥} & \ipa{hwæ˧ tʰɑ˧-kʰɯ˧˥}\\
	M\textsubscript{b} & \ipa{tɕʰi˧\textsubscript{b}} & to sell & \ipa{tɕʰi˧ mɤ˧-tʰɑ˧˥} & \ipa{tɕʰi˧ tʰɑ˧-kʰɯ˧˥}\\
	M\textsubscript{c} & \ipa{bi˧\textsubscript{c}} & to go & \ipa{bi˧ mɤ˧-tʰɑ˧˥} & \ipa{bi˧ tʰɑ˧-kʰɯ˧˥}\\
	L\textsubscript{a} & \ipa{dze˩\textsubscript{a}} & to cut & \ipa{dze˩ mɤ˩-tʰɑ˥} & \ipa{dze˩ tʰɑ˩-kʰɯ˥}\\
	L\textsubscript{b} & \ipa{ʈʰɯ˩\textsubscript{b}} & to drink & \ipa{ʈʰɯ˩ mɤ˩-tʰɑ˥} & \ipa{ʈʰɯ˩ tʰɑ˩-kʰɯ˥}\\
	MH & \ipa{ʈʰæ˧˥} & to bite & \ipa{ʈʰæ˧ mɤ˥-tʰɑ˩} & \ipa{ʈʰæ˧ tʰɑ˥-kʰɯ˩}\\
\lspbottomrule
\end{tabularx}
\end{table}

% The tones for \textsc{V/\textsc{Adj}-neg-abilitive}, /\ipa{V mɤ˧-kv̩˧˥}/ ‘[one] cannot V’, are entirely identical with those for /\ipa{V mɤ˧-tʰɑ˧˥}/ ‘[one] must not V’ and /\ipa{V tʰɑ˧-kʰɯ˧˥}/ ‘do not cause to V’, as shown in \tabref{tab:cannotv}, which also includes adjectives.

\begin{table}%[t]
\caption{\label{tab:cannotv}The tone patterns of V/\textsc{Adj}+\textsc{neg-abilitive}.}
\begin{tabularx}{\textwidth}{ l@{\hspace{9mm}} l@{\hspace{9mm}} l@{\hspace{9mm}} l@{\hspace{9mm}} Q }
\lsptoprule
	 word class & tone & example & meaning & V/\textsc{Adj}+\textsc{neg-abilitive}\\ \midrule
	verbs & H & \ipa{dzɯ˥} & to eat & \ipa{dzɯ˧ mɤ˧-kv̩˩}\\
	 & M\textsubscript{a} & \ipa{hwæ˧\textsubscript{a}} & to buy & \ipa{hwæ˧ mɤ˧-kv̩˧˥}\\
	 & M\textsubscript{b} & \ipa{tɕʰi˧\textsubscript{b}} & to sell & \ipa{tɕʰi˧ mɤ˧-kv̩˧˥}\\
	 & M\textsubscript{c} & \ipa{bi˧\textsubscript{c}} & to go & \ipa{bi˧ mɤ˧-kv̩˧˥}\\
	 & L\textsubscript{a} & \ipa{bæ˩\textsubscript{a}} & to sweep & \ipa{bæ˩ mɤ˩-kv̩˥}\\
	 & L\textsubscript{b} & \ipa{ʐwɤ˩\textsubscript{b}} & to speak & \ipa{ʐwɤ˩ mɤ˩-kv̩˥}\\
	 & MH & \ipa{lɑ˧˥} & to strike & \ipa{lɑ˧ mɤ˥-kv̩˩}\\ \midrule
	adjectives & H & \ipa{bi˥} & shallow & \ipa{bi˧ mɤ˧-kv̩˩}\\
	 & M & \ipa{tsʰi˧} & hot & \ipa{tsʰi˧ mɤ˧-kv̩˧˥}\\
	 & L\textsubscript{a} & \ipa{ɖɯ˩\textsubscript{a}} & large & \ipa{ɖɯ˩ mɤ˩-kv̩˥}\\
	 & L\textsubscript{b} & \ipa{dʑɤ˩\textsubscript{b}} & good & \ipa{dʑɤ˩ mɤ˧-kv̩˧˥}\\
	 & MH & \ipa{tʰɑ˧˥} & sharp & \ipa{tʰɑ˧ mɤ˥-kv̩˩}\\
\lspbottomrule
\end{tabularx}
\end{table}


\subsubsection*{L-tone postverbal morphemes preceded by {negation} prefix}

\begin{table}%[t]
\captionsetup{width=0.85\textwidth}
\caption{\label{tab:willnotv}The tone patterns of the construction /V \ipa{mɤ˧-ho˩}/ ‘will not V’.}
\begin{tabularx}{.75\textwidth}{ l@{\hspace{9mm}} l@{\hspace{9mm}} l@{\hspace{9mm}} Q }
\lsptoprule
	tone & example & meaning & ‘will not V’\\ \midrule
	H & \ipa{dzɯ˥} & to eat & \ipa{dzɯ˧ mɤ˧-ho˥}\\
	M\textsubscript{a} & \ipa{hwæ˧\textsubscript{a}} & to buy & \ipa{hwæ˧ mɤ˧-ho˩}\\
	M\textsubscript{b} & \ipa{tɕʰi˧\textsubscript{b}} & to sell & \ipa{tɕʰi˧ mɤ˧-ho˩}\\
	M\textsubscript{c} & \ipa{bi˧\textsubscript{c}} & to go & \ipa{bi˧ mɤ˧-ho˩}\\
	L\textsubscript{a} & \ipa{dze˩\textsubscript{a}} & to cut & \ipa{dze˩ mɤ˩-ho˥}\\
	L\textsubscript{b} & \ipa{ʈʰɯ˩\textsubscript{b}} & to drink & \ipa{ʈʰɯ˩ mɤ˩-ho˥}\\
	MH & \ipa{ʈʰæ˧˥} & to bite & \ipa{ʈʰæ˧ mɤ˧-ho˥}\\
\lspbottomrule
\end{tabularx}
\end{table}

The tonal behaviour of L-tone postverbal morphemes preceded by the {negation} \is{prefixes}prefix is detailed in \tabref{tab:willnotv}. In these expressions, the verb appears in initial position. Following an exceptionless {phonological rule} (Rule~3, stated in \sectref{sec:alistoftonerules}: “In tone-group-initial position, H and M are neutralized to M”), the verb can only receive one of
two tones: L or M. Thus, L-tone verbs appear with their lexical L tone, while
all others (M, H, and MH) surface with M tone.

Tone assignment on the second and third syllables does not follow a straightforward set of phonological rules. Some regularities can be observed, however. When the verb (the first
syllable) has M tone, the third morpheme (following the {negation} \is{prefixes}prefix)
surfaces with its lexical tone, meaning that the final morpheme's tone remains unchanged. For instance, the MH-tone {abilitive} morpheme /\ipa{-kv̩˧˥}/ retains its MH tone in /\ipa{hwæ˧ mɤ˧-kv̩˧˥}/ ‘cannot buy’, and the L-tone {desiderative} morpheme \mbox{/\ipa{-ho˩}/} retains its L tone in /\ipa{hwæ˧ mɤ˧-ho˩}/ ‘will not buy’. This reflects the tendency for M tone to behave as phonologically neutral: M does not impose any tonal
restriction on the following syllable. 

By contrast, when the initial morpheme (the verb) has H tone, the resulting patterns are remarkably diverse: the MH-tone {abilitive} morpheme /\ipa{-kv̩˧˥}/ is depressed to L, whereas the L-tone {desiderative} morpheme \mbox{/\ipa{-ho˩}/} surfaces with H tone. A fully satisfactory synchronic account of these patterns remains elusive; indeed, it may be illusory to hypothesize that there are clear and cogent synchronic motivations for all patterns. It is nonetheless possible to formulate some tentative remarks about synchronic factors that might be at play. 

One such factor is the \is{culminativity}culminative status of H tone in Yongning Na, reflected in the phonological rule that a tone group can contain at most one H level. 
% ... which will be referred to in Chapter~\ref{chap:toneassignmentrulesandthedivisionoftheutteranceintotonegroups} as the \isi{culminativity} of H tone, 
For overt H tones, culminativity is manifested through the exceptionless levelling of all following tones to L. This massive effect spares no tonal category: not only does it affect tonal categories containing an H component (namely H and MH), but it also extends to M tones, despite their neutral and inert phonological status. However, it seems that in specific contexts, an H tone that does not surface overtly may still exert an effect, preventing the expression of a following H tone. If one wishes to look for a synchronic rationale for the M.M.L pattern of /\ipa{dzɯ˧ mɤ˧-kv̩˩}/ ‘cannot eat’, one might therefore propose that the verb's H tone, although neutralized in tone-group-initial position (in accordance with Rule~3), remains morphophonologically active. 

Under this hypothesis, the effect of the verb's H tone in this context is to induce the levelling of a~subsequent morpheme’s tone to L if that tone contains an H component. Thus, a `covert' H tone would preclude any H tone from appearing inside the tone group, even though it does not itself surface. The observed pattern /\ipa{dzɯ˧ mɤ˧-kv̩˩}/ ‘cannot eat’ (instead of $\ddagger${\kern2pt}\ipa{dzɯ˧ mɤ˧-kv̩˧˥}) would result from a deletion process affecting the MH tone of the third morpheme. This subtle process would specifically target morphemes with H or MH tone, reducing their tone to L, whereas M tones would remain unaffected.%\footnote{If confirmed, this phenomenon would }

Pushing the narrative further, one might attempt to account for the M.M.H pattern of /\ipa{dzɯ˧ mɤ˧-ho˥}/ ‘will not eat’ and /\ipa{ʈʰæ˧ mɤ˧-ho˥}/ ‘will not bite’ by positing that the final H tone \is{derivation!tonal}derives from the verb’s lexical H or MH tone. The idea would be that the H level is, as it were, set afloat and subsequently docks onto the final syllable of the ‘will not V’ construction. From a purely theoretical perspective, this hypothesis is appealing, as such a process would provide additional evidence for the analysis of the mid-rising contour as a sequence of M and H levels (set out in \sectref{sec:contourtonessequencesofleveltonesonthesamesyllable}). Furthermore, it would demonstrate that these two levels are independently accessible to \is{morphotonology}morphotonological processes. 

However, this hypothesis does not appear promising upon closer scrutiny. First, an adequate analysis of the data in \tabref{tab:willnotv} must account for the entire paradigm rather than offering \textit{ad hoc} explanations (`just-so stories') for isolated patterns. In total, three cases yield a final H tone: those in which the verb has H, L or MH tone. At best, the idea that this final H originates in the verb's lexical tone could apply to the H- and MH-tone categories of verbs. 
%(The case of L-tone verbs, discussed below, is also puzzling; progress in its analysis could in turn shed light on the pattern for H- and MH-tone verbs.) 
Secondly, H- and MH-tone verbs do not behave identically across the contexts presented in Tables \ref{tab:mustnotvdonotlet} and \ref{tab:cannotv}. Their similar behaviour in \tabref{tab:willnotv} is therefore likely to be coincidental~-- an identity of surface forms that does not reflect identical underlying mechanisms. 

A final argument against this hypothesis concerns the expected behaviour of MH tones under such a process. If Yongning Na indeed permitted the selective deletion of a specific tonal level within a lexical contour, one would expect the H portion of the MH \is{tonal contour}contour of the third morpheme to be erased, yielding $\dagger${\kern2pt}\ipa{dzɯ˧ mɤ˧-kv̩˧}. However, the data suggests that lexical tones in Yongning Na are treated as {unitary} in this respect: either an MH tone is compatible with the preceding context, in which case it surfaces intact, or it is not, in which case it is lowered to L. There is no such thing as a \is{morphotonology}morphotonological process targeting a specific tonal level within a lexical contour.

For cases where the initial morpheme has L tone, an open question remains as to why the L tone does not spread all the way to the third syllable, which would yield
//$\dagger${\kern2pt}\ipa{dze˩ mɤ˩-tʰɑ˩}// ‘one must not cut’ (an underlying form which, through Rule~7~-- “If a~\isi{tone group} only contains L tones,
a~postlexical H tone is added to its last syllable”~--, would surface as
/$\dagger${\kern2pt}\ipa{dze˩ mɤ˩-tʰɑ˩˥}/). The H
tone observed in /\ipa{dze˩ mɤ˩-tʰɑ˥}/ does not appear to be a truncated remnant of the morpheme's MH lexical tone, as truncation of the initial portion of an MH tone is unattested elsewhere in the language. Rather, it seems that the MH tone of the third syllable is deleted by a~\is{morphotonology}morphotonological rule, leaving the syllable toneless. Phonology then takes over: Rule~7 assigns an H tone to the toneless syllable.


\subsection[Postverbal morphemes preceded by the interrogative]{Postverbal morphemes preceded by the interrogative particle: Tonal oppositions are neutralized}
\label{sec:theneutralizationoftonaloppositionsonmorphemesfollowingtheinterrogativeparticle}


The interrogative particle is analyzed as carrying L tone (see \sectref{sec:ltoneprefixes}). The data in \tabref{tab:thetonepatternforvinterrogativesuffix} reveals a~complete \isi{neutralization} of tonal oppositions on the morphemes following the interrogative particle. The
three morphemes shown in the table have different lexical tones: the \textsc{immediate future} /\ipa{-bi˧\textsubscript{a}}/ has M\textsubscript{a} tone; the \textsc{desiderative} \mbox{/\ipa{-ho˩}/} has L
tone; and the \textsc{permissive} /\ipa{-tʰɑ˧˥}/ has MH tone. The tone patterns for /\ipa{mæ˧\textsubscript{b}}/ ‘to manage to’ are entirely identical to those in the table and are therefore not shown. 

{\setlength\tabcolsep{4.5pt}
\begin{table}%[t]
\captionsetup{width=0.85\textwidth}
\caption{\label{tab:thetonepatternforvinterrogativesuffix}The tone patterns for V+\textsc{interrogative}+postverbal morpheme.}
\begin{tabularx}{\textwidth}{ l l l Q l Q }
\lsptoprule
	tone & example & meaning & ‘going to V?’ & ‘will V?’ & ‘can V?’\\ \midrule
	H & \ipa{dzɯ˥} & to eat & \ipa{dzɯ˧ ə˩-bi˩} & \ipa{dzɯ˧ ə˩-ho˩} & \ipa{dzɯ˧ ə˩-tʰɑ˩}\\
	M\textsubscript{a} & \ipa{hwæ˧\textsubscript{a}} & to buy & \ipa{hwæ˧ ə˧-bi˥} & \ipa{hwæ˧ ə˧-ho˥} & \ipa{hwæ˧ ə˧-tʰɑ˥}\\
	M\textsubscript{b} & \ipa{tɕʰi˧\textsubscript{b}} & to sell & \ipa{tɕʰi˧ ə˧-bi˥} & \ipa{tɕʰi˧ ə˧-ho˥} & \ipa{tɕʰi˧ ə˧-tʰɑ˥}\\
	M\textsubscript{c} & \ipa{bi˧\textsubscript{c}} & to go & \ipa{bi˧ ə˧-bi˥} & \ipa{bi˧ ə˧-ho˥} & \ipa{bi˧ ə˧-tʰɑ˥}\\
	L\textsubscript{a} & \ipa{dze˩\textsubscript{a}} & to cut & \ipa{dze˩ ə˩-bi˥} & \ipa{dze˩ ə˩-ho˥} & \ipa{dze˩ ə˩-tʰɑ˥}\\
	L\textsubscript{b} & \ipa{ʈʰɯ˩\textsubscript{b}} & to drink & \ipa{ʈʰɯ˩ ə˩-bi˥} & \ipa{ʈʰɯ˩ ə˩-ho˥} & \ipa{ʈʰɯ˩ ə˩-tʰɑ˥}\\
	MH & \ipa{ʈʰæ˧˥} & to bite & \ipa{ʈʰæ˧ ə˥-bi˩} & \ipa{ʈʰæ˧ ə˥-ho˩} & \ipa{ʈʰæ˧ ə˥-tʰɑ˩}\\
\lspbottomrule
\end{tabularx}
\end{table}}


\section{Morphemes surrounding adjectives}
\label{sec:combinationsofadjectiveswithgrammaticalmorphemes}

\subsection{Addition of reduplicated suffixes to adjectives}
\label{sec:thereduplicationofadjectives}

Although adjectives behave in many respects like verbs  (i.e.\ as \isi{stative verbs}), as explained in \sectref{sec:adjectivesasdistinctfromverbs},
they do not reduplicate in the same way. Adjectives are intensified through \is{suffixes}suffixation of a~reduplicated
syllable that does not carry any meaning of its own. In some cases, the \is{monosyllables}monosyllabic form of the adjective has fallen into disuse. For instance, /\ipa{bæ˩-lɑ˩{$\sim$}lɑ˥}/ ‘limp, flabby (e.g.~of meat without bones)’ and /\ipa{bæ˩-ʁwæ˩{$\sim$}ʁwæ˥}/ ‘loose (of knot)’ point to a~{monosyllabic} adjective *\ipa{bæ˩} ‘loose, limp’, but in the present state of the language, this root is not attested on its own. 

Another lexical peculiarity is that some such expressions have shorter variants. For instance, ‘short (of persons)’, /\ipa{to˩ʈɯ˩{$\sim$}ʈɯ˥}/, has a~\is{variants}variant /\ipa{to˩ʈɯ˩}/. Comparison with closely related dialects will be necessary to determine whether the disyllable is a~shortened version of the \is{trisyllables}{trisyllable}, or whether a~disyllabic adjective /\ipa{to˩ʈɯ˩}/ underwent \is{expressivity}expressive \isi{reduplication} of its second syllable. Evidence from other dialects would also be required to ascertain the lexical tone of the \is{monosyllables}monosyllabic root from which the longer expressions are \is{derivation!morphological}derived. 

All instances of adjective-plus-reduplicated-\is{suffixes}suffix observed so far carry the same tone pattern (L.L.H). However, the current dataset is too limited to determine whether this is a~\is{morphotonology}morphotonological hallmark of \textit{\textsc{Adj}+reduplicated suffix} expressions or a~mere coincidence.

\subsection{Demonstrative and intensive constructions}
\label{sec:demonstrativeandintensiveconstructions}

In Yongning Na, in addition to the construction involving the {relativizer}
\mbox{/\ipa{-hĩ˥}/}, adjectives frequently appear in \is{demonstratives}demonstrative or intensive constructions, as illustrated in (\ref{ex:thusadj}).

\begin{exe}
	\ex
	\label{ex:thusadj}
	\ipaex{ʈʂʰɯ˧-{\_\_\_\_\_\_\_\_\_}-gv̩˧}\\
	\gll 	ʈʂʰɯ˥					{\_\_\_\_\_\_\_\_\_}			gv̩˧\\
	\textsc{dem.prox}		\textit{{target adjective}}	to\_be/to\_become\\
	\glt ‘thus \textsc{Adj}’ (e.g.~‘thus big’, ‘thus thick’)
\end{exe}

The morpheme /\ipa{gv̩˧}/, \is{grammaticalization}grammaticalized from a~verb meaning ‘to be; to become’, serves to indicate the degree
of a~quality, as exemplified in (\ref{ex:hisnose}).

\begin{exe}
	\ex
	\label{ex:hisnose}
	\ipaex{ʈʂʰɯ˧ {\kern2pt}|{\kern2pt} no˧ {\kern2pt}|{\kern2pt} ɲi˧gɤ˧ {\kern2pt}|{\kern2pt} ʂwæ˧ mɤ˧-gv̩˧.}\\
	\gll 	ʈʂʰɯ˥			no˩				ɲi˧gɤ\#˥	ʂwæ˧	mɤ˧-			gv̩˧\\
			3\textsc{sg}	2\textsc{sg}	nose		tall	\textsc{neg}	to\_be/to\_become\\
	\glt ‘Her nose is not as prominent as yours.’ (Field notes.)
\end{exe}

In the \is{demonstratives}demonstrative
construction ‘thus \textsc{Adj}’, all tonal oppositions on the adjective are neutralized, as shown in \tabref{tab:thisadj}. Examples are found in narratives, for instance in \textit{Sister.28} \pandoi{0004341\#S28} and \textit{BuriedAlive3.144} \pandoi{0004538\#S144}. 

\begin{table}%[t!]
	\caption{\label{tab:thisadj}Demonstrative construction /\ipa{ʈʂʰɯ˧}-\textsc{Adj}-\ipa{gv̩˧}/ ‘thus \textsc{Adj}’.}
	\begin{tabularx}{\textwidth}{ l l l l@{\hspace{4mm}} Q Q }
		\lsptoprule
		tone & example & meaning & ‘thus \textsc{Adj}’ & meaning & tone pattern\\ \midrule
		H & \ipa{ɖæ˥} & short & \ipa{ʈʂʰɯ˧-ɖæ˧-gv̩˧} & thus short & M.M.M\\
		M & \ipa{hwɤ˧} & broad & \ipa{ʈʂʰɯ˧-hwɤ˧-gv̩˧} & thus broad & M.M.M\\
		L\textsubscript{a} & \ipa{tɕi˩\textsubscript{a}} & short & \ipa{ʈʂʰɯ˧-tɕi˧-gv̩˧} & thus short & M.M.M\\
		MH & \ipa{ɬo˧˥} & deep & \ipa{ʈʂʰɯ˧-ɬo˧-gv̩˧} & thus deep & M.M.M\\
		\lspbottomrule
	\end{tabularx}
\end{table}

The same M{\dots}M tone pattern is found in cases where the adjective is reduplicated,
e.g.~/\ipa{ʈʂʰɯ˧-ɖɯ˧{$\sim$}ɖɯ˧}\ipa{-gv̩˧}/ ‘thus big’ (\textit{Agriculture.55} \pandoi{0004440\#S55}), from /\ipa{ɖɯ˩\textsubscript{a}}/ ‘big’ (see
also \textit{FoodShortage.43, 85} \pandoi{0004557\#S43}).\footnote{Related constructions include \textsc{dem}+\textsc{Adj}+\textsc{augmentative} /\ipa{-mi˩}/,
	e.g.~/\ipa{ʈʂʰɯ˧-ʂwæ˧-mi˧-zo˥}/ ‘thus tall’, where /\ipa{-zo}/ is an {adverbializer} (\textit{Dog.12}  \pandoi{0004442\#S12}), and the
	intensive construction /\ipa{qʰɑ˧}- \textsc{Adj} \ipa{-mi˧}/ ‘so \textsc{Adj}’, e.g.~/\ipa{qʰɑ˧-ɖɯ˧-mi˧-hĩ˧}/ ‘thus
	huge, extremely big’ (\textit{Lake3.28}  \pandoi{0004348\#S28}), where \mbox{/\ipa{-hĩ˥}/} is the {relativizer}/{nominalizer}.}

Tonal neutralization also occurs in the construction ‘as \textsc{Adj} (as)’,
/\ipa{tʰɑ˧-}\textsc{redu\-pli\-cat\-ed}~\textsc{Adj}\ipa{-gv̩˩}/, shown in \tabref{tab:asadjas}.

\begin{table}%[t!]
	\caption{\label{tab:asadjas}Tone patterns in the construction ‘as \textsc{Adj} as’.}
	\begin{tabularx}{\textwidth}{ l l l l@{\hspace{3mm}} l l }
		\lsptoprule
		tone & example & meaning & ‘as \textsc{Adj} as’ & meaning & tone pattern\\ \midrule
		H & \ipa{ɖæ˥} & short & \ipa{tʰɑ˧-ɖæ˥{$\sim$}ɖæ˩-gv̩˩} & as short as & M.H.L.L\\
		M & \ipa{hwɤ˧} & broad & \ipa{tʰɑ˧-hwɤ˥{$\sim$}hwɤ˩-gv̩˩} & as broad as & M.H.L.L\\
		L\textsubscript{a} & \ipa{tɕi˩\textsubscript{a}} & short & \ipa{tʰɑ˧-tɕi˥{$\sim$}tɕi˩-gv̩˩} & as short as & M.H.L.L\\
		MH & \ipa{ɬo˧˥} & deep & \ipa{tʰɑ˧-ɬo˥{$\sim$}ɬo˩-gv̩˩} & as deep as & M.H.L.L\\
		\lspbottomrule
	\end{tabularx}
\end{table}

In contrast with the patterns in Tables \ref{tab:thisadj} and \ref{tab:asadjas}, tonal oppositions are not fully neutralized in the construction shown in (\ref{ex:thusadj2}), which expresses ‘thus \textsc{Adj}’, much like (\ref{ex:thusadj}).\footnote{The identification of the morpheme /\ipa{-ɻ̩˩}/ in (\ref{ex:thusadj2}) remains uncertain; it may be the \textsc{inceptive/inchoative} morpheme //\ipa{-ɻ̩˧}//.} As shown in \tabref{tab:thusadj2}, adjectives with MH lexical tone stand out in this construction by their distinct tone pattern: M.M.H, as opposed to M.M.M for all other tonal categories.

\begin{exe}
	\ex
	\label{ex:thusadj2}
	\ipaex{ʈʂʰɯ˧-{\_\_\_\_\_\_\_\_\_}-ɻ̩˧}\\
	\gll 	ʈʂʰɯ˥					{\_\_\_\_\_\_\_\_\_}			ɻ̩˧\\
	\textsc{dem.prox}		\textit{{target adjective}}	\textsc{inceptive?}\\
	\glt ‘thus \textsc{Adj}’ (e.g.~‘thus big’, ‘thus thick’)
\end{exe}

\begin{table}%[t!]
	\caption{\label{tab:thusadj2}Tone patterns in the construction /\ipa{ʈʂʰɯ˧}-\textsc{Adj}-\ipa{ɻ̩˧}/ ‘thus \textsc{Adj}’.}
	\begin{tabularx}{\textwidth}{ l l l l@{\hspace{4mm}} Q Q }
		\lsptoprule
		tone & example & meaning & ‘thus \textsc{Adj}’ & meaning & tone pattern\\ \midrule
		H & \ipa{bv̩˥} & thick & \ipa{ʈʂʰɯ˧-bv̩˧-ɻ̩˧} & thus thick & M.M.M\\
		M & \ipa{fv̩˧} & happy & \ipa{ʈʂʰɯ˧-fv̩˧-ɻ̩˧} & thus happy & M.M.M\\
		L\textsubscript{a} & \ipa{ɖɯ˩\textsubscript{a}} & large & \ipa{ʈʂʰɯ˧-ɖɯ˧-ɻ̩˧} & thus large & M.M.M\\
		MH & \ipa{hæ˧˥} & supple & \ipa{ʈʂʰɯ˧-hæ˧-ɻ̩˥} & thus supple & M.M.H\\
		\lspbottomrule
	\end{tabularx}
\end{table}


\section{Object followed by non-prefixed verb}
\label{sec:objectandnonprefixedverb}

% start of span where 'object' is indexed
\is{object|(}

Cross-linguistically, the object is the nominal argument that has strongest syntactic association to the
verb. This bond is so strong that it has even been used as a~defining criterion for the notion of
object \citep[38]{creissels1991}. From a~\is{morphotonology}morphotonological perspective, it is reasonable to
expect at least as much tonal interaction in object-verb combinations as in subject-verb combinations (which are examined below, in \sectref{sec:subjectandverb}).

In principle, object-verb and subject-verb sequences in Yongning Na could be ambiguous. The language does not have verbal indexing for subjects or objects. Neither is there case marking, or a fixed syntactic position for subjects and objects relative to the verb. The verb follows all noun phrases, except in cases where constituents are tacked on at the end of an utterance as afterthoughts. The \is{postpositions}postposition /\ipa{ɳɯ˧}/, which marks a~noun as an agent, is not
obligatory. Differences in tonal behaviour between S+V and O+V sequences could, in theory, contribute to disambiguation; however, in practice, S+V without intervening morphemes is relatively uncommon, and disambiguation is generally effected through context.

To prevent reinterpretation as subject-verb sequences, object-verb combinations were elicited in contexts designed to clarify their intended structure. For instance, pairing ‘wolf’ with ‘to eat’ naturally suggests agent role for ‘wolf’. To ensure the desired object-verb interpretation, an explicit subject noun phrase was introduced, as shown in (\ref{ex:wolfate}). This additional noun phrase does not influence the tone of the object-verb combination, as it is followed by a tone-group boundary.

\begin{exe}
	\ex
	\label{ex:wolfate}
	\ipaex{lɑ˧ ɳɯ˧ {\kern2pt}|{\kern2pt} õ˩dv̩˧ dzɯ˧-ze˩.}\\
	\gll lɑ˧	ɳɯ˧	õ˩dv̩˧˥	dzɯ˥		-ze˧\textsubscript{b}\\
	tiger		\textsc{a}		wolf	to\_eat		\textsc{pfv}\\
	\glt ‘The tiger ate the wolf.’
\end{exe}

A~sample of the elicited data is shown in \tabref{tab:sampleofofoplusv}, with nouns illustrating the various tone categories as objects of the L\textsubscript{b}-tone verb /\ipa{do˩\textsubscript{b}}/ ‘to see'. The same results obtain when \is{numerals}numeral-plus-classifier phrases are used instead of disyllabic nouns. For instance, /\ipa{ɖɯ˧-bæ˧}/ ‘something’, made up of the \is{numerals}numeral ‘one’ and a~classifier for sorts of things, behaves tonally in the same way as disyllabic lexical units carrying the same tone (M), such as /\ipa{po˧lo˧}/ ‘ram’ and /\ipa{qæ˧do˧}/ ‘timber’. Likewise, combinations with
/\ipa{ɖɯ˧-kʰwɤ˥\$}/ ‘a piece’ yield the same tone patterns as those with /\ipa{dʑi˧hṽ̩˥\$}/ ‘clothes’. 

\begin{table}%[t!]
	\caption{\label{tab:sampleofofoplusv}A sample of object-plus-verb combinations: the verb /\ipa{do˩\textsubscript{b}}/ ‘to see'.}
	\begin{tabularx}{\textwidth}{ l l l l@{\hspace{11mm}} Q }
		\lsptoprule
		tone of N & example & meaning & resulting phrase & tone pattern\\ \midrule
		LM & \ipa{bo˩˧} & pig & \ipa{bo˩ do˧}  & L.M+L\\
		LH & \ipa{ʐæ˩˥} & leopard & \ipa{ʐæ˩ do˥} & L.H\\
		M & \ipa{lɑ˧} & tiger & \ipa{lɑ˧ do˩}  & M.L\\
		L & \ipa{jo˩} & sheep & \ipa{jo˩ do˩˥}  & L.L\\
		H & \ipa{ʐwæ˥} & horse & \ipa{ʐwæ˧ do˧˥}   & M.MH\\
		MH & \ipa{ʈʂʰæ˧˥} & deer & \ipa{ʈʂʰæ˧ do˧˥}   & M.MH\\ \addlinespace \hdashline \addlinespace
		M & \ipa{po˧lo˧} & ram & \ipa{po˧lo˧ do˩}  & M.M.L\\
		\#H & \ipa{ʐwæ˧zo\#˥} & colt & \ipa{ʐwæ˧zo˧ do˧˥}   & M.M.MH\\
		MH\# & \ipa{hwɤ˧li˧˥} & cat & \ipa{hwɤ˧li˧ do˧˥}  & M.M.MH\\
		H\$ & \ipa{kv̩˧ʂe˥\$} & flea & \ipa{kv̩˧ʂe˧ do˧˥}  & M.M.MH\\
		L & \ipa{kʰv̩˩mi˩} & dog & \ipa{kʰv̩˩mi˩ do˩˥}  & L.L.L\\
		L\# & \ipa{dɑ˧ʝi˩} & mule & \ipa{dɑ˧ʝi˩ do˩} & M.L.L\\
		LM+MH\# & \ipa{õ˩dv̩˧˥} & wolf & \ipa{õ˩dv̩˧ do˧˥} & L.M.MH\\
		LM+\#H & \ipa{nv̩˩tɕʰi\#˥} & fine chaff & \ipa{nv̩˩tɕʰi˧ do˧˥}  & L.M.MH\\
		LM & \ipa{bo˩mi˧} & sow & \ipa{bo˩mi˧ do˩}  & L.M.L\\
		LH & \ipa{bo˩ɬɑ˥} & boar & \ipa{bo˩ɬɑ˧ do˩}  & L.H.L\\
		H\# & \ipa{hwæ˧ʈʂæ˥} & squirrel & \ipa{hwæ˧ʈʂæ˥ do˩}  & M.H.L\\
		\lspbottomrule
	\end{tabularx}
	\end{table}

\subsection{The facts}
\label{sec:thefactsobjectandnonprefixedverb}


\tabref{tab:thetonepatternsofobjectverbcombinations} presents the tone rules that apply in object-plus-verb phrases. Recordings
available online include (i)~a selection of combinations among monosyllables: \textit{ObjectVerb} \pandoi{0004472}, (ii)~a~full set,
except for L\textsubscript{b}-tone verbs: \textit{ObjectVerb2} \pandoi{0004474}, and (iii)~L\textsubscript{b}-tone verbs: \textit{ObjectVerb3} \pandoi{0004567}. 

\begin{sidewaystable}[p]
\caption{\label{tab:thetonepatternsofobjectverbcombinations}The tone patterns of object-plus-verb combinations.}
{\renewcommand{\arraystretch}{1.1}
\begin{tabularx}{\textheight}{ l@{\hspace{6mm}} Q l@{\hspace{6mm}} l@{\hspace{6mm}} l@{\hspace{6mm}} l@{\hspace{6mm}} Q }
\lsptoprule
 & tone of verb\\\cmidrule{2-7}
	tone of noun & H & M\textsubscript{a} & M\textsubscript{b} & L\textsubscript{a} & L\textsubscript{b} & MH\\ \midrule
		LM & L.M+L & L.M+M & L.M+M & L.M+L & L.M+L & L.MH\\
	LH & L.L / L.H & L.H & L.L / L.H & L.H & L.H / L.L & L.MH\\
	M & M.M+L & M.M+M & M.M+M & M.L & M.L & M.MH\\
	L & L.L & M.M+M & M.M+M / L.L & L.L & L.L / M.L & L.L\\
	H & M.M+L & M.L & M.M+L & M.H & M.MH & M.L\\
	MH & M.H & M.H & M.H & M.H & M.MH & M.H\\  \addlinespace \hdashline \addlinespace
	M & M.M.M+L & M.M.M+M & M.M.M+M & M.M.L & M.M.L & M.M.MH\\
	\#H & M.M.M+L & M.M.L & M.M.M+L & M.M.H & M.M.MH & M.M.L\\
	MH\# & M.M.MH & M.M.H & M.M.MH & M.M.H & M.M.MH & M.M.H\\
	H\$ & M.M.M+L & M.H.L & M.M.M+L & M.M.H & M.M.MH & M.H.L\\
	L & L.L.L & L.L.H & L.L.L & L.L.H & L.L.L & L.L.H\\
	L\# & M.L.L & M.L.L & M.L.L & M.L.L & M.L.L & M.L.L\\
	LM+MH\# & L.M.M+L & L.M.H & L.M.M+L & L.M.H & L.M.MH & L.M.H\\
	LM+\#H & L.M.M+L & L.M.L & L.M.M+L & L.M.H & L.M.L / L.M.MH & L.M.L\\
	LM & L.M.M+L & L.M.M+M & L.M.M+M & L.M.L & L.M.L & L.M.MH\\
	LH & L.H.L & L.H.L & L.H.L & L.H.L & L.H.L & L.H.L\\
	H\# & M.H.L & M.H.L & M.H.L & M.H.L & M.H.L & M.H.L\\
\lspbottomrule
\end{tabularx}}
\end{sidewaystable}

The notation adopted in \tabref{tab:thetonepatternsofobjectverbcombinations} requires disambiguation for sequences ending in an M tone, namely
M.M and L.M: object-plus-verb phrases realized with one of these patterns \is{form!in isolation}in isolation may have
different underlying tones. Consider (\ref{ex:buytea}) and (\ref{ex:buychicken}): 

\begin{exe}
	\ex
	\label{ex:buytea}
	\ipaex{li˩ hwæ˧}\\
	\gll li˩˥	hwæ˧\textsubscript{a}\\
	tea		to\_buy\\
	\glt ‘to buy tea’ \pandoi{0004473\#W18}
\end{exe}

\begin{exe}
	\ex
	\label{ex:buychicken}
	\ipaex{æ̃˩ hwæ˧}\\
	\gll æ̃˩˧	hwæ˧\textsubscript{a}\\
	chicken		to\_buy\\
	\glt ‘to buy chicken’
\end{exe}

The phrases /\ipa{li˩ hwæ˧}/ ‘to buy tea’ and /\ipa{æ̃˩ hwæ˧}/ ‘to
buy chicken’ both surface with an /L.M/ pattern. But they yield different results when followed by the \textsc{perfective} morpheme /\ipa{-ze˧\textsubscript{b}}/, as shown in (\ref{ex:buyteaPFV})-(\ref{ex:buychickenPFV}).

\begin{exe}
	\ex
	\label{ex:buyteaPFV}
	\ipaex{li˩ hwæ˧-ze˩}\\
	\gll li˩˥	hwæ˧\textsubscript{a}		-ze˧\textsubscript{b}\\
	tea		to\_buy		\textsc{pfv}\\
	\glt ‘bought tea’ \pandoi{0004473\#W17}
\end{exe}

\begin{exe}
	\ex
	\label{ex:buychickenPFV}
	\ipaex{æ̃˩ hwæ˧-ze˧}\\
	\gll æ̃˩˧	hwæ˧\textsubscript{a}		-ze˧\textsubscript{b}\\
	chicken		to\_buy		\textsc{pfv}\\
	\glt ‘bought chicken’
\end{exe}

\tabref{tab:thetonepatternsofobjectverbcombinations} therefore includes information on the tonal
realization of the \textsc{perfective} morpheme /\ipa{-ze˧\textsubscript{b}}/ where relevant. The tone pattern of
‘to buy tea’, which exemplifies the combination of input tones \mbox{//LH//} (on the noun) and \mbox{//M//} (on the verb), is
transcribed in the table as /L.M+L/, whereas that of ‘to buy chicken’ is simply /L.M/. Similarly, the pattern /M.M+L/ is
distinguished from /M.M+M/. The tone on the \textsc{accomplished} \is{prefixes}prefix is preceded by a~‘+’ sign.

In cases where the tone of the postverbal element follows straightforwardly from the
phonological rules of tone association in Yongning Na (as summarized in Chapter~\ref{chap:toneassignmentrulesandthedivisionoftheutteranceintotonegroups}), no further specification is
indicated in the table. All L.H, M.H, L.L.H and M.M.H patterns end in H\#, a final, non-floating, non-`hopping' H tone, and are followed by L tone on the \textsc{perfective} suffix /\ipa{-ze˧\textsubscript{b}}/. Likewise, when the final tone of the object-plus-verb phrase
is L, /\ipa{-ze˧\textsubscript{b}}/ always receives L tone. If the pattern ends in /MH/, then 
/\ipa{-ze˧\textsubscript{b}}/ receives the H part of the \is{tonal contour}contour.

As in the other tables, a~slash separates variants. For instance, for ‘has sold
leopards’ (input: \mbox{//LH//} and \mbox{//M\textsubscript{b}//}), two realizations are acceptable, as shown in (\ref{ex:soldpanther}).

\begin{exe}
	\ex
	\label{ex:soldpanther}
	\ipaex{/ʐæ˩ tɕʰi˥-ze˩/~≈~/ʐæ˩ tɕʰi˩-ze˥/}\\
	\gll ʐæ˩˥ tɕʰi˧\textsubscript{b}		-ze˧\\
	leopard		to\_sell		\textsc{pfv}\\
	\glt ‘sold leopards’
\end{exe}

For //L// and \mbox{//M\textsubscript{b}//}, two patterns are also possible: L.L and M.M+M, as shown in (\ref{ex:soldsheep}).
%There are other examples, such as /\ipa{kʰɯ˩ tɕʰi˩-ze˥}/ ≈ /\ipa{kʰɯ˧ tɕʰi˧-ze˧}/ ‘has sold thread’.

\begin{exe}
	\ex
	\label{ex:soldsheep}
	\ipaex{/jo˩ tɕʰi˩-ze˥/~≈~/jo˧ tɕʰi˧-ze˧/}\\
	\gll jo˩ tɕʰi˧\textsubscript{b}		-ze˧\textsubscript{b}\\
	sheep		to\_sell		\textsc{pfv}\\
	\glt ‘has sold sheep’ \pandoi{0004472\#W97} 
\end{exe}

If the input is \mbox{//LH//} and //H//, two patterns are also possible: L.L and L.H, as illustrated by (\ref{ex:hasdugearth}) and (\ref{ex:haseatenbuff}).

\begin{exe}
	\ex
	\label{ex:hasdugearth}
	\ipaex{/di˩ dv̩˩-ze˥/~≈~/di˩ dv̩˥-ze˩/}\\
	\gll di˩˥	dv̩˥		-ze˧\textsubscript{b}\\
	earth		to\_dig		\textsc{pfv}\\
	\glt ‘has dug earth’ \pandoi{0004473\#W1}
\end{exe}

\begin{exe}
	\ex
	\label{ex:haseatenbuff}
	\ipaex{/(lɑ˧ ɳɯ˧ {\kern2pt}|{\kern2pt}) tʰɑ˩ dzɯ˩-ze˥/~≈~/(lɑ˧ ɳɯ˧ {\kern2pt}|{\kern2pt}) tʰɑ˩ dzɯ˥-ze˩/}\\
	\gll lɑ˧	ɳɯ˧	tʰɑ˩˥	dzɯ˥		-ze˧\textsubscript{b}\\
	tiger		\textsc{a}	buffalo		to\_eat		\textsc{pfv}\\
	\glt ‘(the
	tiger) has eaten (a) buffalo’
\end{exe}

Finally, a~note concerning the L.M.L / L.M.MH output (from input //LM+\#H// and //L\textsubscript{b}//): examples include (\ref{ex:seechaff}), (\ref{ex:seenugg}) and (\ref{ex:seeNaxi}). The consultant
expressed a~preference for the L.M.MH realization.

\begin{exe}
	\ex
	\label{ex:seechaff}
	\ipaex{/nv̩˩tɕʰi˧ do˧˥/~≈~/nv̩˩tɕʰi˧ do˩/}\\
	\gll nv̩˩tɕʰi\#˥ do˩\textsubscript{b}\\
	fine\_chaff		to\_see\\
	\glt ‘to see fine chaff’ \pandoi{0004922\#W18}
\end{exe}

\begin{exe}
	\ex
	\label{ex:seenugg}
	\ipaex{/pi˩ti˧ do˧˥/~≈~/pi˩ti˧ do˩/}\\
	\gll pi˩ti\#˥ do˩\textsubscript{b}\\
	nugget	to\_see\\
	\glt ‘to see nuggets (of silver)’
\end{exe}

\begin{exe}
	\ex
	\label{ex:seeNaxi}
	\ipaex{/nɑ˩hĩ˧ do˧˥/~≈~/nɑ˩hĩ˧ do˩/}\\
	\gll nɑ˩hĩ\#˥ do˩\textsubscript{b}\\
	Naxi	to\_see\\
	\glt ‘to see (the) Naxi’
\end{exe}

The following sections offer observations about the tone patterns of object-plus-verb combinations and their relation to other aspects of the \is{morphotonology}morphotonological system. 


\subsection[Evidence of the opposition between \mbox{//LM//} and \mbox{//LH//}]{Object-plus-verb combinations reveal the tonal opposition between \mbox{//LM//} and \mbox{//LH//} {monosyllabic} nouns}
\label{sec:usefulnessofelicitingobjectverbcombinationstodistinguishtheLMandLHcategoriesofverbs}

The opposition between the \mbox{//LM//} and \mbox{//LH//} categories of nouns (illustrated by //\ipa{bo˩˧}//
‘pig’ and //\ipa{ʐæ˩˥}// ‘leopard’) is neutralized in most contexts. However, among object-plus-verb combinations,
two contexts distinguish them: their association with an M-tone verb, e.g.~/\ipa{bo˩ hwæ˧-ze˧}/ ‘bought pigs’ vs.\ /\ipa{ʐæ˩ hwæ˧-ze˩}/ ‘bought leopards’, and with an H-tone verb,
e.g.~/\ipa{bo˩ dzɯ˧}/ ‘to eat pigs’ vs.\ /\ipa{ɣɯ˩ dzɯ˩˥}/ ‘to eat skin’. (‘Skin' /\ipa{ɣɯ˩˥}/ was chosen as a~semantically more suitable noun than ‘leopard’ as the object of the verb ‘to eat’; nonetheless, /\ipa{ʐæ˩ dzɯ˩˥}/ ‘to eat leopards’ is syntactically well-formed.) 

\subsection{About tonal variants}
\label{sec:abouttonalvariants}

% !! Mistaken passage from the first edition. The tone of 'daughter' is LH, not L. Good progress in the analysis :)
%Any L-tone {monosyllabic} noun functioning as the object of an L\textsubscript{b}-tone verb can yield an //L.L// pattern (surfacing as /L.LH): for instance, /\ipa{ɬi˩ ʑi˩˥}/ ‘to grab/seize
%a~roebuck’, /\ipa{mv̩˩ ʑi˩˥}/ ‘to grab/seize (a/one’s) daughter’, /\ipa{ɬi˩ do˩˥}/ ‘to see
%a~roebuck’, and /\ipa{mv̩˩ do˩˥}/ ‘to see (a/one’s) daughter’. On the other hand, the M.L \is{variants}variant only occurs with certain nouns: it is possible to say /\ipa{ɬi˧ ʑi˩}/ ‘to grab/seize a~roebuck’ and
%/\ipa{ɬi˧ do˩}/ ‘to see a~roebuck’, but not $\ddagger${\kern2pt}\ipa{mv̩˧ ʑi˩} (intended meaning: ‘to grab/seize
%(a/one’s) daughter’) or $\ddagger${\kern2pt}\ipa{mv̩˧ do˩} (intended meaning: ‘to see (a/one’s) daughter’). Considerations of homophony may be at play here. There are twelve /\ipa{mv̩}/ morphemes in the (modest) dictionary of Yongning Na, three of which have M tone, which, in combination with an L\textsubscript{b}-tone verb, yields an /M.L/ pattern. 

There are interesting fine details in the distribution of tonal variants among object-plus-verb combinations. In some cases, it is possible to identify one \is{variants}variant that is more common in frequently occurring word combinations. For instance, with an L-tone object and an LM-tone verb, both M.M+M and L.L patterns are possible, but the former predominates for the expression ‘to drink water’: it is customary to say /\ipa{dʑɯ˧ ʈʰɯ˧}/, while the latter \is{variants}variant, /\ipa{dʑɯ˩ ʈʰɯ˩˥}/, is understandable but sounds unusual. Conversely, in elicitation, ‘to grab sheep’ yielded /\ipa{jo˩ ʑi˩˥}/ rather than the alternative /\ipa{jo˧ ʑi˧}/. This suggests that object-plus-verb combinations with this pair of input tones (L and LM) tend to acquire an M.M+M pattern as a~result of high frequency of occurrence (which amounts to \isi{lexicalization}). If so, the reason why the consultant did not produce ‘to grab sheep’ as /\ipa{jo˧ ʑi˧}/ may be because of this tone pattern's connotation of habitual activity. There are no sheep on the consultant's farm, so sheep-grabbing is indeed not a part of the consultant's farming routine. The M.M+M \is{variants}variant /\ipa{jo˧ ʑi˧}/ could be appropriate in a context such as sheep shearing, where grabbing sheep is a routine action.


\subsection{Exceptional combinations}
\label{sec:exceptionalcombinations}

Two \mbox{//LH//} nouns, ‘leopard’ and ‘monkey’, yield an unexpected L.M+L pattern in combination with an //H//-tone verb,
instead of the expected L.L pattern, as shown in (\ref{ex:haseatenleopards}) and (\ref{ex:haseatenmonkeys}).

\begin{exe}
	\ex
	\label{ex:haseatenleopards}
	\ipaex{ʐæ˩ dzɯ˧-ze˩~~~($\dagger${\kern2pt}ʐæ˩ dzɯ˩-ze˥)}\\
	\gll ʐæ˩˥	dzɯ˥		-ze˧\textsubscript{b}\\
	leopard		to\_eat		\textsc{pfv}\\
	\glt ‘has eaten leopards’
\end{exe}

\begin{exe}
	\ex
	\label{ex:haseatenmonkeys}
	\ipaex{ʑi˩ dzɯ˧-ze˩~~~($\dagger${\kern2pt}ʑi˩ dzɯ˩-ze˥)}\\
	\gll ʑi˩˥	dzɯ˥		-ze˧\textsubscript{b}\\
	monkey		to\_eat		\textsc{pfv}\\
	\glt ‘has eaten monkeys’
\end{exe}

While ‘to eat leopards’ is semantically odd, as
leopards are predators rather than prey, the phrase ‘to eat monkeys’ is unproblematic. The same pattern recurred in several elicitation sessions, suggesting a genuine irregularity. Pending further investigation, this is provisionally treated as an exceptional
pattern requiring individual memorization.

In \tabref{tab:thetonepatternsofobjectverbcombinations}, two variants were indicated for \mbox{//LH//} nouns in combination with \mbox{//M\textsubscript{b}//} verbs: L.L and L.H. Some word combinations allow both: ‘has sold leopards’ may be expressed as either /\ipa{ʐæ˩ tɕʰi˥-ze˩}/, as well as /\ipa{ʐæ˩ tɕʰi˩-ze˥}/. However, others appear to have lexicalized with one tone pattern or the other. For instance, ‘brought in the harvest’ is always /\ipa{bæ˩
  ʂo˥-ze˩}/, never $\dagger${\kern2pt}\ipa{bæ˩ ʂo˩-ze˥}, whereas ‘to eat skin’ (with the same input tones) is consistently
/\ipa{ɣɯ˩ dzɯ˩-ze˥}/, never $\dagger${\kern2pt}\ipa{ɣɯ˩ dzɯ˥-ze˩}. This may indicate a tendency for lexicalized combinations to receive an L.H
pattern. %Sporadic tone change accompanying \isi{lexicalization} is well-attested cross-linguistically. 
Alternatively, L.H may be older (which would explain its higher frequency in lexicalized combinations), whereas L.L is \is{innovative (phonological form)}innovative.

Two further exceptions, both on highly lexicalized items, are /\ipa{kʰv̩˧ ʂæ˧˥}/ ‘to hunt’ (literally ‘to lead a~dog’, from //\ipa{kʰv̩˥}// ‘dog’ and //\ipa{ʂæ˧˥}// ‘to lead along’) and /\ipa{mv̩˧ tsʰi˧˥}/
‘to light a~fire’ (from  //\ipa{mv̩˥}// ‘fire’ and //\ipa{tsʰi˧˥}// ‘to light’). The regular pattern would yield an M.L sequence~-- \mbox{$\dagger${\kern2pt}\ipa{kʰv̩˧}} \ipa{ʂæ˩} and $\dagger${\kern2pt}\ipa{mv̩˧ tsʰi˩}. The observed M.MH tonal string suggests retention of a tone pattern that was once productive but is no longer so.


\subsection[Noun plus copula behaves tonally like O+V]{Noun plus copula behaves tonally like object plus verb}
\label{sec:nounsplusthecopulabehavetonallylikeobjectverbcombinations}

Data on the tonal behaviour of the \isi{copula} //\ipa{ɲi˩\textsubscript{a}}// following nouns of each tonal category were set out in Chapter~\ref{chap:thelexicaltonesofnouns}. The pattern observed is identical to that of L\textsubscript{a}-tone verbs in object-plus-verb combinations, rather than in subject-plus-verb constructions. This perfect match in tone patterns provides strong evidence that a noun preceding the copula is syntactically its object, not its subject. 


\subsection{Interrogative pronoun and verb}
\label{sec:interrogativepronounandverb}

The combination of an interrogative \is{pronouns}pronoun and a~verb is a~special case of the object-plus-verb structure. However, the tonal patterns are not entirely identical, as shown in \tabref{tab:whichwhichplacewhere}. Surprisingly, the three interrogative pronouns in \tabref{tab:whichwhichplacewhere}, which are hypothesized to belong to the
same lexical tone category, namely \mbox{//LM//}, show distinct tonal behaviours. These differences were carefully verified: for instance, it was confirmed that $\ddagger${\kern2pt}\ipa{ze˩bæ˧ lɑ˧˥}
is not an acceptable \is{variants}variant for ‘strike which sort [of things]?’, any more than $\ddagger${\kern2pt}\ipa{ze˩gɤ˧ lɑ˥} is for ‘strike which place?’.

{\setlength\tabcolsep{5.5pt}
\begin{table}%[t]
\caption{\label{tab:whichwhichplacewhere}The tonal behaviour of three interrogative pronouns preceding a~verb: /\ipa{ze˩bæ˧}/ ‘which sort’, /\ipa{ze˩gɤ˧}/ ‘which place’, and /\ipa{zo˩qo˧}/ ‘where’.}
\begin{tabularx}{\textwidth}{ l l l l l Q }
\lsptoprule
	tone & example & meaning & which & which place & where\\ \midrule
	H & \ipa{dzɯ˥} & to eat & \ipa{ze˩bæ˧ dzɯ˧} & \ipa{ze˩gɤ˧ dzɯ˧} & \ipa{zo˩qo˧ dzɯ˧}\\
	M\textsubscript{a} & \ipa{hwæ˧\textsubscript{a}} & to buy & \ipa{ze˩bæ˧ hwæ˩} & \ipa{ze˩gɤ˧ hwæ˧} & \ipa{zo˩qo˧ hwæ˧}\\
	M\textsubscript{b} & \ipa{tɕʰi˧\textsubscript{b}} & to sell & \ipa{ze˩bæ˧ tɕʰi˧} & \ipa{ze˩gɤ˧ tɕʰi˧} & \ipa{zo˩qo˧ tɕʰi˧}\\
	M\textsubscript{c} & \ipa{pv̩˧\textsubscript{c}} & to chant & \ipa{ze˩bæ˧ pv̩˩} & \ipa{ze˩gɤ˧ pv̩˧} & \ipa{zo˩qo˧ pv̩˧}\\
	L\textsubscript{a} & \ipa{bæ˩\textsubscript{a}} & to sweep & \ipa{ze˩bæ˧ bæ˥} & \ipa{ze˩gɤ˧ bæ˩} & \ipa{zo˩qo˧ bæ˩}\\
	L\textsubscript{b} & \ipa{ʈʰɯ˩\textsubscript{b}} & to drink;  & \ipa{ze˩bæ˧ ʈʰɯ˧˥} & \ipa{ze˩gɤ˧ ʈʰɯ˩} & \ipa{zo˩qo˧ ʈʰɯ˩}\\
	 & \ipa{ʐwɤ˩\textsubscript{b}} & to speak & \ipa{ze˩bæ˧ ʐwɤ˧˥} & \ipa{ze˩gɤ˧ ʐwɤ˩} & \ipa{zo˩qo˧ ʐwɤ˩}\\
	MH & \ipa{lɑ˧˥} & to strike & \ipa{ze˩bæ˧ lɑ˥} & \ipa{ze˩gɤ˧ lɑ˧˥} & \ipa{zo˩qo˧ lɑ˧˥}\\
\lspbottomrule
\end{tabularx}
\end{table}}

 

\section{Object and prefixed verb}
\label{sec:objectandprefixedverb}

When an object combines with a~prefixed verb, the two may form a~single \isi{tone group}, as in
(\ref{ex:putonclothes}), or they may be separated by a~\isi{tone group} \is{boundary (between tone groups)}boundary, as in
(\ref{ex:theelderbrotherputoncoarsefeltcloak}).


\begin{exe}
  \ex
  \label{ex:putonclothes}
  \ipaex{bɑ˩lɑ˩ tʰi˥-mv̩˩}\\
  \gll bɑ˩lɑ˩		tʰi˧-	mv̩˧\textsubscript{a}\\
  clothes		\textsc{dur}	to\_put\_on\\
  \glt ‘to put on clothes’ \textit{(ComingOfAge2.37)} \pandoi{0004588\#S37}

  
  \ex
  \label{ex:theelderbrotherputoncoarsefeltcloak}
  \ipaex{ə˧mv̩˧ ɳɯ˥, {\kern2pt}|{\kern2pt} ʐæ˩sɯ˩˥ {\kern2pt}|{\kern2pt} tʰi˧-mv̩˧}\\
  \gll ə˧mv̩˧˥		ɳɯ˧	ʐæ˩sɯ˩		tʰi˧	mv̩˧\textsubscript{a}\\
  elder\_sibling	\textsc{a}	coarse\_felt	\textsc{dur}	to\_put\_on\\
  \glt ‘the elder brother put on [his] coarse felt cloak’ \textit{(Sister3.57)} \pandoi{0004344\#S57}
\end{exe}

The same \is{stylistics}stylistic choice is open for \is{numerals}numeral-plus-classifier phrases in object position. Example (\ref{ex:bowlofwine}) illustrates a~case where the object (‘a~bowl’) and the verb (‘to pour’) are integrated into a single \isi{tone group}. 
The decision to separate the prefixed verb from the object is not dictated by the length of the \isi{tone group}: in (\ref{ex:bowlofwine}), the \isi{tone group} comprises six syllables, whereas in (\ref{ex:theelderbrotherputoncoarsefeltcloak}), which has similar structure, a sequence of just four syllables is split into two tone groups. The placement of tone-group boundaries is discussed in detail in \sectref{sec:thedivisionofutterancesintotonegroups}.

\begin{exe}
  \ex
  \label{ex:bowlofwine}
  \ipaex{ʐɯ˧ {\kern2pt}|{\kern2pt} ɖɯ˧-qʰwɤ˧ tʰi˥-pʰv̩˩ tsɯ˩ {\kern2pt}|{\kern2pt} mv̩˩.}\\
  \gll ʐɯ˧		ɖɯ˧	qʰwɤ˧˥		tʰi˧-	pʰv̩˧˥	  tsɯ˧˥	mv̩˧\\
  liquor/spirits		one	\textsc{clf}.bowls	\textsc{dur}	to\_pour	  \textsc{rep}	\textsc{affirm}\\
  \glt ‘it is said that [she] poured a~bowl of liquor [for her brother].’ \textit{(Sister3.41)} \pandoi{0004344\#S41}
\end{exe}

When the object and prefixed verb are integrated into a single \isi{tone group}, it seems at first glance
as if the tonal adjustments were purely phonological, proceeding
from the beginning (in a “left-to-right” manner). 
In example (\ref{ex:bowlofwine}), for instance, it seems that
the MH \is{tonal contour}contour on the \is{numerals}numeral-plus-classifier unfolds over the prefixed verb: /\ipa{ɖɯ˧-qʰwɤ˧˥}/
‘one bowl(ful)’ plus /\ipa{tʰi˧-pʰv̩˧˥}/ ‘to pour’ would yield /\ipa{ɖɯ˧-qʰwɤ˧
  tʰi˥}{\dots}/ through unfolding of the MH \is{tonal contour}contour, the H component of the MH \is{tonal contour}contour
reassociating to the \textsc{durative} \is{prefixes}prefix //\ipa{tʰi˧}-// (hence /\ipa{tʰi˥}-/). The final output, /\ipa{ɖɯ˧-qʰwɤ˧ tʰi˥-pʰv̩˩}/, would then follow from the application of Rules 4 and 5: “A syllable following an H-tone syllable receives L tone”
and “All syllables following an H.L or M.L sequence receive L tone”. 

A~similar analysis can be applied for almost all cases, but not for quite all of them. Therefore, the process cannot be considered as purely phonological: the association of an object and a~prefixed verb still belongs within morphophonology. However, the system is only a small step away from becoming a purely phonological one: if the handful of non-predictable forms were regularized on the basis of current phonological rules, tonal adjustment in this domain could be fully described in phonological terms.  

The following sections set out the relevant data and discuss these phenomena in detail.

\subsection{The facts}
\label{sec:thefactsobjectandprefixedverb}

The data is arranged by tone. The behaviour of tones M\textsubscript{a}, M\textsubscript{b}, and M\textsubscript{c} is identical, as is that
of tones L\textsubscript{a} and L\textsubscript{b}. Consequently, only one M-tone verb is used as an example in \tabref{tab:objectsplusmtoneverbsprefixedbythedurative},
and only one L-tone verb in \tabref{tab:objectsplusltoneverbsprefixedbythedurative}. Patterns that do not conform with the regularities
discussed below 
%in \sectref{sec:dataanalysisobjectandprefixedverb} 
are shaded in gray.


Attestations in texts are abundant: the \textsc{durative} \is{prefixes}prefix /\ipa{tʰi˧}-/ occurs over 700 times in twenty texts, illustrating a~broad range of combinations. In example (\ref{ex:putashes}), this \is{prefixes}prefix appears after an LH-tone noun, a~context in which it gets an H tone. Other interesting examples are found in \textit{Funeral.69}  \pandoi{0004571\#S69},
\textit{108, 190, 238, 253, Healing.103} \pandoi{0004540\#S103}, \textit{Housebuilding.40} \pandoi{0004448\#S40},\textit{ 110, 121, 217, 239, Mountains.7} \pandoi{0004573\#S7}, \textit{88, 161,
Reward.40} \pandoi{0004446\#S40}, \textit{73, Seeds2.51} \pandoi{0004542\#S51}, \textit{62} and \textit{Sister3.41} \pandoi{0004344\#S41}, \textit{95}.


\begin{exe}
	\ex
	\label{ex:putashes}
	\ipaex{lwɤ˩ tʰi˥-kʰɯ˩ {\kern2pt}|{\kern2pt} tɕɤ˧-kv̩˥ mæ˩, {\kern2pt}|{\kern2pt} ə˧ʝi˧-ʂɯ˥ʝi˩!}\\
	\gll lwɤ˩˥		tʰi˧-			kʰɯ˧˥		tɕɤ˧˥		-kv̩˧˥					mæ˧							ə˧ʝi˧-ʂɯ˥ʝi˩\\
	ashes		\textsc{dur}	to\_put	 	to\_boil	\textsc{abilitive}	\textsc{obviousness}		in\_the\_old\_times\\
	\glt ‘One would add ashes and boil [linen thread], in the old times!’ \textit{(FoodShortage.71)}  \pandoi{0004557\#S71}
\end{exe}

Discussion of the above data will proceed from the most straightforward cases to the more complex ones.
\clearpage


{%\setlength\tabcolsep{4pt}
\begin{table}[t]

\caption{\label{tab:objectsplusmtoneverbsprefixedbythedurative}Objects plus the M-tone verb /\ipa{hwæ˧\textsubscript{a}}/ ‘to buy’ prefixed by the \textsc{durative} /\ipa{tʰi˧}-/.}
\begin{tabularx}{\textwidth}{ l l l Q l }
\lsptoprule
	tone & head & meaning & \ipa{hwæ˧\textsubscript{a}} ‘to buy’ & tone pattern\\ \midrule
	LM & \ipa{bo˩˧} & pig & \ipa{bo˩ tʰi˧-hwæ˧} & L.M.M\\
	LH & \ipa{mv̩˩˥} & daughter & \ipa{mv̩˩ tʰi˥-hwæ˩} & L.H.L\\
	M & \ipa{lɑ˧} & tiger & \ipa{lɑ˧ tʰi˧-hwæ˧} & M.M.M\\
	L & \ipa{jo˩} & sheep & \shadedcell \ipa{jo˧ tʰi˧-hwæ˧} & \shadedcell M.M.M\\
	\#H & \ipa{hĩ˥} & human being & \ipa{hĩ˧ tʰi˩-hwæ˩} & M.L.L\\
	MH\# & \ipa{tsʰɯ˧˥} & goat & \ipa{tsʰɯ˧ tʰi˥-hwæ˩} & M.H.L\\ \addlinespace \hdashline \addlinespace
	M & \ipa{po˧lo˧} & ram & \ipa{po˧lo˧ tʰi˧-hwæ˧} & M.M.M.M\\
	\#H & \ipa{ʐwæ˧zo\#˥} & colt & \ipa{ʐwæ˧zo˧ tʰi˩-hwæ˩} & M.M.L.L\\
	MH\# & \ipa{hwɤ˧li˧˥} & cat & \ipa{hwɤ˧li˧ tʰi˥-hwæ˩} & M.M.H.L\\
	H\$ & \ipa{kv̩˧ʂe˥\$} & flea & \ipa{kv̩˧ʂe˥ tʰi˩-hwæ˩} & M.H.L.L\\
	L & \ipa{kʰv̩˩mi˩} & dog & \ipa{kʰv̩˩mi˩ tʰi˥-hwæ˩} & L.L.H.L\\
	L\# & \ipa{dɑ˧ʝi˩} & mule & \ipa{dɑ˧ʝi˩ tʰi˩-hwæ˩} & M.L.L.L\\
	LM+MH\# & \ipa{v̩˩tsʰɤ˧˥} & vegetables & \ipa{v̩˩tsʰɤ˧ tʰi˥-hwæ˩} & M.M.H.L\\
	LM+\#H & \ipa{ɑ˩mi\#˥} & goose & \ipa{ɑ˩mi˧ tʰi˩-hwæ˩} & L.M.L.L\\
	LM & \ipa{bo˩mi˧} & sow & \ipa{bo˩mi˧ tʰi˧-hwæ˧} & L.M.M.M\\
	LH & \ipa{bo˩ɬɑ˥} & boar & \ipa{bo˩ɬɑ˥ tʰi˩-hwæ˩} & L.H.L.L\\
	H\# & \ipa{kʰv̩˧nɑ˥} & dog & \ipa{kʰv̩˧nɑ˥ tʰi˩-hwæ˩} & M.H.L.L\\
\lspbottomrule
\end{tabularx}
\end{table}}


\begin{table}%[t]
\caption{\label{tab:objectsplushtoneverbsprefixedbythedurative}Objects plus the H-tone verb /\ipa{dzɯ˥}/ ‘to eat’ prefixed by the \textsc{durative} /\ipa{tʰi˧}-/.}
\begin{tabularx}{\textwidth}{ l l l Q l }
\lsptoprule
	tone & head & meaning & \ipa{dzɯ˥} ‘to eat’ & tone pattern\\ \midrule
	LM & \ipa{bo˩˧} & pig & \ipa{bo˩ tʰi˧-dzɯ˥} & L.M.H\\
	LH & \ipa{mv̩˩˥} & daughter & \ipa{mv̩˩ tʰi˥-dzɯ˩} & L.H.L\\
	M & \ipa{lɑ˧} & tiger & \ipa{lɑ˧ tʰi˧-dzɯ˥} & M.M.H\\
	L & \ipa{jo˩} & sheep & \shadedcell \ipa{jo˩ tʰi˩-dzɯ˩˥} & \shadedcell L.L.LH\\
	\#H & \ipa{hĩ˥} & human being & \ipa{hĩ˧ tʰi˩-dzɯ˩} & M.L.L\\
	MH\# & \ipa{tsʰɯ˧˥} & goat & \ipa{tsʰɯ˧ tʰi˥-dzɯ˩} & M.H.L\\ \addlinespace \hdashline \addlinespace
	M & \ipa{po˧lo˧} & ram & \ipa{po˧lo˧ tʰi˧-dzɯ˥} & M.M.M.H\\
	\#H & \ipa{ʐwæ˧zo\#˥} & colt & \ipa{ʐwæ˧zo˧ tʰi˩-dzɯ˩} & M.M.L.L\\
	MH\# & \ipa{hwɤ˧li˧˥} & cat & \ipa{hwɤ˧li˧ tʰi˥-dzɯ˩} & M.M.H.L\\
	H\$ & \ipa{kv̩˧ʂe˥\$} & flea & \ipa{kv̩˧ʂe˥ tʰi˩-dzɯ˩} & M.H.L.L\\
	L & \ipa{kʰv̩˩mi˩} & dog & \ipa{kʰv̩˩mi˩ tʰi˥-dzɯ˩} & L.L.H.L\\
	L\# & \ipa{dɑ˧ʝi˩} & mule & \ipa{dɑ˧ʝi˩ tʰi˩-dzɯ˩} & M.L.L.L\\
	LM+MH\# & \ipa{v̩˩tsʰɤ˧˥} & vegetables & \ipa{v̩˩tsʰɤ˧ tʰi˥-dzɯ˩} & L.M.H.L\\
	LM+\#H & \ipa{ɑ˩mi\#˥} & goose & \shadedcell \ipa{ɑ˩mi˧ tʰi˥-dzɯ˩} & \shadedcell L.M.H.L\\
	LM & \ipa{bo˩mi˧} & sow & \ipa{bo˩mi˧ tʰi˧-dzɯ˥} & L.M.M.H\\
	LH & \ipa{bo˩ɬɑ˥} & boar & \ipa{bo˩ɬɑ˥ tʰi˩-dzɯ˩} & L.H.L.L\\
	H\# & \ipa{kʰv̩˧nɑ˥} & dog & \ipa{kʰv̩˧nɑ˥ tʰi˩-dzɯ˩} & M.H.L.L\\
\lspbottomrule
\end{tabularx}
\end{table}

% \clearpage

\begin{table}%[t]
\caption{\label{tab:objectsplusltoneverbsprefixedbythedurative}Objects plus the L-tone verb /\ipa{di˩\textsubscript{a}}/ ‘to have’ prefixed by the \textsc{durative} /\ipa{tʰi˧}-/.}
\begin{tabularx}{\textwidth}{ l@{\hspace{6mm}} l@{\hspace{6mm}} l@{\hspace{6mm}} Q l }
\lsptoprule
	tone & head & meaning & \ipa{di˩\textsubscript{a}} ‘to have’ & tone pattern\\ \midrule
	LM & \ipa{bo˩˧} & pig & \ipa{bo˩ tʰi˧-di˩} & L.M.L\\
	LH & \ipa{mv̩˩˥} & daughter & \shadedcell \ipa{mv̩˩ tʰi˩-di˥} & \shadedcell L.L.H\\
	M & \ipa{lɑ˧} & tiger & \ipa{lɑ˧ tʰi˧-di˩} & M.M.L\\
	L & \ipa{jo˩} & sheep & \shadedcell \ipa{jo˧ tʰi˧-di˩} & \shadedcell M.M.L\\
	\#H & \ipa{hĩ˥} & human being & \shadedcell \ipa{hĩ˧ tʰi˧-di˥} & \shadedcell M.M.H\\
	MH\# & \ipa{tsʰɯ˧˥} & goat & \shadedcell \ipa{tsʰɯ˧ tʰi˧-di˥} &
   \shadedcell M.M.H\\ \addlinespace \hdashline \addlinespace
	M & \ipa{po˧lo˧} & ram & \ipa{po˧lo˧ tʰi˧-di˩} & M.M.M.L\\
	\#H & \ipa{ʐwæ˧zo\#˥} & colt & \shadedcell \ipa{ʐwæ˧zo˧ tʰi˧-di˥} & \shadedcell M.M.M.H\\
	MH\# & \ipa{hwɤ˧li˧˥} & cat & \shadedcell \ipa{hwɤ˧li˧ tʰi˧-di˥} & \shadedcell M.M.M.H\\
	H\$ & \ipa{kv̩˧ʂe˥\$} & flea & \shadedcell \ipa{kv̩˧ʂe˧ tʰi˧-di˥} & \shadedcell M.M.M.H\\
	L & \ipa{kʰv̩˩mi˩} & dog & \shadedcell \ipa{kʰv̩˩mi˩ tʰi˩-di˥} & \shadedcell L.L.L.H\\
	L\# & \ipa{dɑ˧ʝi˩} & mule & \ipa{dɑ˧ʝi˩tʰi˩-di˩} & M.L.L.L\\
	LM+MH\# & \ipa{v̩˩tsʰɤ˧˥} & vegetables & \shadedcell \ipa{v̩˩tsʰɤ˧ tʰi˧-di˥} & \shadedcell L.M.M.H\\
	LM+\#H & \ipa{ɑ˩mi\#˥} & goose & \shadedcell \ipa{ɑ˩mi˧ tʰi˧-di˥} & \shadedcell L.M.M.H\\
	LM & \ipa{bo˩mi˧} & sow & \ipa{bo˩mi˧ tʰi˧-di˩} & L.M.M.L\\
	LH & \ipa{bo˩ɬɑ˥} & boar & \ipa{bo˩ɬɑ˥ tʰi˩-di˩} & L.H.L.L\\
	H\# & \ipa{kʰv̩˧nɑ˥} & dog & \ipa{kʰv̩˧nɑ˥ tʰi˩-di˩} & M.H.L.L\\
\lspbottomrule
\end{tabularx}
\end{table}

\clearpage

\begin{table}[t]
\caption{\label{tab:objectsplusmhtoneverbsprefixedbythedurative}Objects plus the MH-tone verb /\ipa{ʈʰæ˧˥}/ ‘to bite’ prefixed by the \textsc{durative} /\ipa{tʰi˧}-/.}
\begin{tabularx}{\textwidth}{ l l l@{\hspace{6mm}} Q l }
\lsptoprule
	tone & head & meaning & \ipa{ʈʰæ˧˥} ‘to bite’ & tone pattern\\ \midrule
	LM & \ipa{bo˩˧} & pig & \ipa{bo˩ tʰi˧-ʈʰæ˧˥} & L.M.MH\\
	LH & \ipa{mv̩˩˥} & daughter & \ipa{mv̩˩ tʰi˥-ʈʰæ˩} & L.H.L\\
	M & \ipa{lɑ˧} & tiger & \ipa{lɑ˧ tʰi˧-ʈʰæ˧˥} & M.M.MH\\
	L & \ipa{jo˩} & sheep & \shadedcell \ipa{jo˩ tʰi˩-ʈʰæ˩˥} & \shadedcell L.L.LH\\
	\#H & \ipa{hĩ˥} & human being & \ipa{hĩ˧ tʰi˩-ʈʰæ˩} & M.L.L\\
	MH\# & \ipa{tsʰɯ˧˥} & goat & \ipa{tsʰɯ˧ tʰi˥-ʈʰæ˩} & M.H.L\\ \addlinespace \hdashline \addlinespace
	M & \ipa{po˧lo˧} & ram & \ipa{po˧lo˧ tʰi˧-ʈʰæ˧˥} & M.M.M.MH\\
	\#H & \ipa{ʐwæ˧zo\#˥} & colt & \ipa{ʐwæ˧zo˧ tʰi˩-ʈʰæ˩} & M.M.L.L\\
	MH\# & \ipa{hwɤ˧li˧˥} & cat & \ipa{hwɤ˧li˧ tʰi˥-ʈʰæ˩} & M.M.H.L\\
	H\$ & \ipa{kv̩˧ʂe˥\$} & flea & \ipa{kv̩˧ʂe˥ tʰi˩-ʈʰæ˩} & M.H.L.L\\
	L & \ipa{kʰv̩˩mi˩} & dog & \ipa{kʰv̩˩mi˩ tʰi˥-ʈʰæ˩} & L.L.H.L\\
	L\# & \ipa{dɑ˧ʝi˩} & mule & \ipa{dɑ˧ʝi˩ tʰi˩-ʈʰæ˩} & M.L.L.L\\
	LM+MH\# & \ipa{v̩˩tsʰɤ˧˥} & vegetables & \ipa{v̩˩tsʰɤ˧ tʰi˥-ʈʰæ˩} & L.M.H.L\\
	LM+\#H & \ipa{ɑ˩mi\#˥} & goose & \ipa{ɑ˩mi˧ tʰi˩-ʈʰæ˩} & L.M.L.L\\
	LM & \ipa{bo˩mi˧} & sow & \ipa{bo˩mi˧ tʰi˧-ʈʰæ˧˥} & L.M.M.MH\\
	LH & \ipa{bo˩ɬɑ˥} & boar & \ipa{bo˩ɬɑ˥ tʰi˩-ʈʰæ˩} & L.H.L.L\\
	H\# & \ipa{kʰv̩˧nɑ˥} & dog & \ipa{kʰv̩˧nɑ˥ tʰi˩-ʈʰæ˩} & M.H.L.L\\
\lspbottomrule
\end{tabularx}
\end{table}



%\subsection{Data analysis}
\label{sec:dataanalysisobjectandprefixedverb}

\subsection{The tone of the verb surfaces when the noun phrase has LM or M tone}
\label{sec:thetoneoftheverbexpressesitselfwhenthenounphrasehastonelmorm}

As observed in \sectref{sec:analysisofmasadefaulttone} of Chapter~\ref{chap:thelexicaltonesofnouns}, M tends to behave as an inert tone in Yongning Na: it does not
spread or otherwise affect following tones. This observation extends to the behaviour of M-tone nouns when followed by a~prefixed verb: when the noun phrase has M tone, the prefixed
verb surfaces with the same tonal pattern as \is{form!in isolation}in isolation. The same applies to LM-tone noun phrases.


\subsection{Tonal oppositions on verbs are neutralized after a~disyllabic noun phrase with L\#, LH or H\# tone}
\label{sec:tonaloppositionsonverbsareneutralizedafteradisyllabicnounphrasewithtonellhandh}

When a disyllabic noun phrase bears L\#, LH or H\# tone, it precludes the realization of any tone other than L on following syllables within the \isi{tone group}. This follows from the application of Rules 4 and 5: “A syllable following an H-tone syllable receives L
tone”, and “All syllables following an H.L or M.L sequence receive L tone”. Consequently, all tonal oppositions on verbs are neutralized in this context.

\subsection{Patterns that cannot be fully explained by phonological regularities}
\label{sec:commentsonpatternsthatcannotbefullyexplainedonaphonologicalbasis}

The patterns discussed in the two preceding paragraphs can be fully explained by phonological regularities that apply throughout the system. In total, roughly three of four patterns conform to these regularities, a proportion that appears high enough to support the tentative hypothesis that these tendencies represent the default case, while the remaining cases are likely to be learnt individually. 

The phonological tendencies are as follows:

\begin{enumerate}[label=(\roman*), itemsep=0pt]
\item The \mbox{//LH//} tone on a~{monosyllabic} noun projects its H portion onto the next syllable (in this case the \textsc{durative} \is{prefixes}prefix /\ipa{tʰi˧}/). 
\item The \mbox{//\#H//} and //LM+\#H// tones do not overtly express their H tone, which remains \is{floating tone}floating but is not deleted; this \is{floating tone}floating H tone lowers the tones of the following syllables to L. 
\item The \mbox{//H\$//} tone gets docked on the last syllable of the noun phrase, causing the following syllables to receive L tone, in accordance with Rules 4 and 5: “A syllable following an H-tone syllable receives L tone”, and “All syllables following an H.L or M.L sequence receive L tone”.
\item The \mbox{//MH\#//} and //LM+MH\#// tones project their final H level onto the following syllable. 
\end{enumerate}

From this perspective, combinations that follow these tendencies can be considered regular, while those that do not are irregular. The patterns for //L//-tone verbs are irregular because they contravene tendencies (i), (ii), and (iv). (The behaviour of L-tone verbs is analyzed further in \sectref{sec:thebehaviourofltoneverbsattemptingageneralization}.) The pattern involving a~\mbox{//LM+\#H//} noun and an //H//-tone verb is also irregular: in view of tendency (ii), one would expect L.M.L.L rather than the observed L.M.M.H, as illustrated in (\ref{ex:eatgoose}).

\begin{exe}
	\ex
	\label{ex:eatgoose}
	\ipaex{ɑ˩mi˧ tʰi˧-dzɯ˥~~~~($\dagger${\kern2pt}ɑ˩mi˧ tʰi˩-dzɯ˩)}\\
	\gll ɑ˩mi\#˥			tʰi˧-			dzɯ˥\\
	goose	\textsc{dur}		to\_eat\\
	\glt ‘eating a~goose’
\end{exe}

The issue of regularity and irregularity in \is{morphotonology}morphotonological paradigms will be taken up in the typological discussion in Chapter~\ref{chap:arealandtypologicaldiscussion}. 


\subsection{The behaviour of L-tone verbs: Attempting a~generalization}
\label{sec:thebehaviourofltoneverbsattemptingageneralization}

L-tone verbs exhibit the highest proportion of irregular patterns, i.e. patterns that do
not follow the four phonological tendencies outlined above. Nonetheless, a~generalization
may still be possible. One tentative way to describe what happens in this particular morphosyntactic context is as follows: when the noun phrase contains an H tone that is not unmovably fixed, this H tone shifts to the last syllable of the resulting verb phrase.

This somewhat roundabout generalization takes into account the identical tonal treatment of prefixed verbs following disyllabic nouns with \mbox{//H\#//} or \mbox{//LH//} tone. In a~disyllabic noun, the H part of a~\mbox{//H\#//} or \mbox{//LH//} tone is firmly anchored to the second syllable. By contrast, the tones \mbox{//\#H//}, //LM+\#H//, \mbox{//H\$//}, \mbox{//MH\#//}, and //LM+MH\#// share the property of containing an H tone whose syllabic \is{anchorage}anchoring is context-sensitive.

An apparent \isi{counterexample} to this generalization is the \mbox{//LH//} tone: while disyllabic \mbox{//LH//}-tone nouns yield
L.H.L.L, {monosyllabic} \mbox{//LH//}-tone nouns yield L.L.H, as shown in (\ref{ex:seedaughter}).

\begin{exe}
	\ex
	\label{ex:seedaughter}
	\ipaex{mv̩˩ tʰi˩-do˥}\\
	\gll mv̩˩˥		tʰi˧-			do˩\textsubscript{b}\\
	daughter		\textsc{dur}		to\_see\\
	\glt ‘to see \mbox{(a/the)}
	daughter’ (elicited example)
\end{exe}

At this point, one would be justified in concluding that the \isi{morphotonology} at play is fundamentally irregular and that attempts to impose systematic explanations would be unwarranted. The above generalization would thus have to be abandoned. However, experience in learning and speaking the language suggests otherwise, encouraging further speculation on how the observed patterns relate to one another. 

The different treatment of \mbox{//LH//} tone on {monosyllabic} versus disyllabic nouns could be attributed to differences in syllabic association. In disyllabic nouns, the H part of the \mbox{//LH//} tone is unmovably anchored to the second syllable, making it a straightforward category for \is{language acquisition}language learners. In monosyllabic nouns, on the other hand, the absence of a second syllable creates pressure for the H part to reassociate to
a~later syllable. In various morphosyntactic contexts, the H part of \mbox{//LH//} does not surface on the noun to which it is lexically associated. To sum up, \mbox{//LH//} tone has fixed, hard-and-fast syllabic
\is{anchorage}anchoring in disyllabic nouns but looser \is{anchorage}anchoring in  \is{monosyllables}monosyllables. Seen in this light, it does not come as a~great surprise that \mbox{//LH//} on a~{monosyllabic} noun
should behave as one of the movable H tones (alongside \mbox{//\#H//}, //LM+\#H//, \mbox{//H\$//}, \mbox{//MH\#//}, and //LM+MH\#//) rather than as one of the unmovable H tones.

This generalization is \textit{ad hoc} in that it only concerns a single
morphosyntactic context. However, it does not appear absurd to consider that it has psychological reality, allowing \is{language acquisition}learners to acquire the tonal patterns of \mbox{//\#H//}, //LM+\#H//, \mbox{//H\$//}, \mbox{//MH\#//}, and //LM+MH\#// (as well as \mbox{//LH//} in monosyllables) as a unified set.

Still on a~speculative note, this analysis predicts high cross-dialect \isi{variation} in //L//-tone verbs. Cases that deviate from regularities (i-iv) in \sectref{sec:commentsonpatternsthatcannotbefullyexplainedonaphonologicalbasis} are especially numerous for //L//-tone verbs, and the force of \isi{analogy} would tend to simplify the system by eliminating these irregularities. This is an empirical {question} to be investigated using data from other dialects.

% end of tag indexing 'object' for whole section
\is{object|)}


\section{Subject and verb}
\label{sec:subjectandverb}

Tonal interaction between subject and verb is illustrated by
(\ref{ex:snowrain}--\ref{ex:guestpriest}). The verb ‘to fall’ carries different tones in (\ref{ex:itissnowing}) and (\ref{ex:itisraining}), as does the verb ‘to come’ in (\ref{ex:theguesthasarrived}) and (\ref{ex:thepriesthasarrived}).

\begin{exe}
  \ex
  \label{ex:snowrain}
  \begin{xlist}
    \ex
  \label{ex:itissnowing}
  \ipaex{bi˧ gi˧-ze˩.}\\
  \gll bi˥	gi˥	-ze˧\textsubscript{b}\\
  snow	to\_fall	\textsc{pfv}\\
  \glt 		‘It has snowed.’

  \ex
  \label{ex:itisraining}
  \ipaex{hi˩ gi˩-ze˥.}\\
  \gll hi˩˧	gi˥	-ze˧\textsubscript{b}\\
  rain	to\_fall	\textsc{pfv}\\
  \glt 		‘It has rained.’
  \end{xlist}

  \ex
  \label{ex:guestpriest}
  \begin{xlist}
  \ex
  \label{ex:theguesthasarrived}
  \ipaex{hĩ˧bæ˧ tsʰɯ˧-ze˥.}\\
  \gll hĩ˧bæ\#˥	tsʰɯ˩\textsubscript{a}	-ze˧\textsubscript{b}\\
  guest		to\_come.\textsc{pst}		\textsc{pfv}\\
  \glt 		‘The guest has come. / The guests have come.’

  \ex
  \label{ex:thepriesthasarrived}
  \ipaex{dɑ˧pɤ˧ tsʰɯ˩-ze˩.}\\
  \gll dɑ˧pɤ˧		tsʰɯ˩\textsubscript{a}	-ze˧\textsubscript{b}\\
  priest		to\_come.\textsc{pst}	\textsc{pfv}\\
  \glt 		‘The \textit{Ddabe} priest has come.’
  \end{xlist}
\end{exe}


Subject-plus-verb phrases are relatively infrequent in Yongning Na. This is partly due to the high
frequency of postnominal morphemes and verbal prefixes, which separate the subject from the verb in the linear ordering of the sentence, and, for transitive verbs, to the subject-object-verb \isi{word order}. When the verb is preceded by a~particle, such as the \textsc{accomplished} \is{prefixes}prefix
/\ipa{le˧}-/ or the \textsc{durative} \is{prefixes}prefix /\ipa{tʰi˧}-/, the noun and the verb do not interact directly; they often even belong to two different tone
groups. Compare, for instance, the realizations of the MH-tone verb
/\ipa{qæ˧˥}/ ‘to burn’ in /\ipa{mv̩˧ {\kern2pt}|{\kern2pt} le˧-qæ˧-ze˥}/ ‘the fire burned’ and /\ipa{mv̩˧ qæ˩}/ ‘the fire
burns’. 

The elicited data analyzed in this section is based on intransitive verbs, so as to avoid possible confusions between S+V and O+V constructions. Following the same procedure as for
nouns, two contexts were used to arrive at underlying tone categories: S+V and
S+V+\textsc{perfective}. For instance, (\ref{ex:theguesthasarrived}) without the \textsc{perfective} yields /\ipa{hĩ˧-bæ˧ tsʰɯ˧˥}/ ‘the guests arrive’. The tone pattern for this combination of subject and predicate can therefore be described as /M.M.MH/ and further analyzed as \mbox{//MH\#//}: an MH \is{tonal contour}contour associating with the last syllable.


\subsection{The facts}
\label{sec:thefactssubjectandverb}

For systematic elicitation, the following verbs were used: /\ipa{se˥}/ ‘to walk’, /\ipa{ʂɯ˧\textsubscript{a}}/ ‘to die’, /\ipa{tsʰo˧\textsubscript{b}}/
‘to jump’, /\ipa{tsʰɯ˩\textsubscript{a}}/ ‘to come.\textsc{pst}’, /\ipa{ʐwɤ˩\textsubscript{b}}/ ‘to speak’, and
/\ipa{bæ˧˥}/ ‘to run’. The nouns used were kinship terms and animal names. \tabref{tab:thetonepatternsofsubjectplusverbcombinationsinsurfacephonologicaltranscription} presents the
results. The \mbox{//LM//} and \mbox{//LH//} tone categories of monosyllables always yield the same output, so they are pooled together in the table.

When the subject-plus-verb combination ends in an /H/ or /L/ tone, the \textsc{perfective} morpheme carries /L/ tone. When it ends in an /MH/ tone, the postverbal morpheme receives the /H/ part of the \is{tonal contour}contour. When it ends on
an /M/ tone, the tone of the postverbal morpheme cannot be predicted; in these cases, it is indicated in the table, preceded by a~‘+’
sign.

The recording \textit{SubjectVerb} \pandoi{0004476} contains all of the combinations in \tabref{tab:thetonepatternsofsubjectplusverbcombinationsinsurfacephonologicaltranscription}.


\begin{sidewaystable}[p]
\caption{\label{tab:thetonepatternsofsubjectplusverbcombinationsinsurfacephonologicaltranscription}The tone patterns of subject-plus-verb combinations, in
  surface phonological transcription.}
\begin{tabularx}{\textheight}{ l@{\hspace{6mm}} Q l@{\hspace{6mm}} l@{\hspace{6mm}} l@{\hspace{6mm}} l@{\hspace{6mm}} Q }
\lsptoprule
& tone of verb & & & & &\\ \cmidrule{2-7}	
tone of noun & H & M\textsubscript{a} & M\textsubscript{b} & L\textsubscript{a} & L\textsubscript{b} & MH\\ \midrule
	LM, LH & L.H & L.M+M & L.M+M & L.H & L.H & L.MH\\
	M & M.M+L & M.M+M & M.M+M & M.L & M.L & M.MH\\
	L & M.M+L & L.L  & M.M+M & L.L & L.L~/ M.L & L.L\\
	H & M.M+L & M.M+L & M.M+L & M.MH & M.MH & M.L\\
	MH & M.H & M.H & M.H & M.MH & M.MH & M.H\\ \addlinespace \hdashline \addlinespace
	M & M.M.M+L & M.M.M+M & M.M.M+M & M.M.L & M.M.L & M.M.MH\\
	\#H & M.M.M+L & M.M.M+L & M.M.M+L & M.M.MH & M.M.MH & M.M.L\\
	MH\# & M.M.MH & M.M.MH & M.M.MH & M.M.MH & M.M.MH & M.M.H\\
	H\$ & M.M.M+L & M.M.M+L & M.M.M+L~/ M.M.M+H & M.M.MH & M.M.MH & M.H.L\\
	L & L.L.L & L.L.L & L.L.L & L.L.L & L.L.L & L.L.H\\
	L\# & M.L.L & M.L.L & M.L.L & M.L.L & M.L.L & M.L.L\\
	LM+MH\# & L.M.M+L & L.M.M+L & L.M.M+L & L.M.MH & L.M.MH & L.M.H\\
	LM+\#H & L.M.M+L & L.M.M+M & L.M.M+M & L.M.L & L.M.MH & L.M.MH\\
	LM & L.M.M+L & L.M.M+M & L.M.M+M & L.M.L & L.M.L & L.M.MH\\
	LH & L.H.L & L.H.L & L.H.L & L.H.L & L.H.L & L.H.L\\
	H\# & M.H.L & M.H.L & M.H.L & M.H.L & M.H.L & M.H.L\\
\lspbottomrule
\end{tabularx}
\end{sidewaystable}


About one fourth of the tone patterns for subject-plus-verb phrases differ from the corresponding
object-plus-verb phrases. Among the identical combinations are those where the noun has an //L\#// or \mbox{//H\#//} tone, since these fixed-position tones lower the tones of all following syllables to L (by application of Rules~4 and 5; see \sectref{sec:alistoftonerules}).

As with other types of morphosyntactic combinations, such as \is{numerals}numeral-plus-classifier and object-plus-verb, the tone
patterns in \tabref{tab:thetonepatternsofsubjectplusverbcombinationsinsurfacephonologicaltranscription} cannot be \is{derivation!tonal}derived from a~set of phonological tone rules. Among the more
surprising patterns is \mbox{//LH//} plus \mbox{//LM//}, yielding L.M.M+H, as in /\ipa{bo˩ɬɑ˧ ʈʰɯ˧-ze˥}/ ‘the boar drank’. The noun has an \mbox{//LH//} lexical
tone, which would be expected to cause \isi{neutralization} of all tonal contrasts on the following morpheme (here, on the verb). The five
other combinations involving an \mbox{//LH//}-tone noun do indeed yield L.H.L, but that with a~//L\textsubscript{b}//-tone verb yields L.M.MH. This pattern
is conspicuously unrelated to the phonological tendencies observed in the language.

In subject-verb combinations as in many other morphosyntactic contexts, the //L.M// and //L.H// sequences are neutralized on the
surface. Notation as //L.H// is interchangeable with //L.M+L//. The former was chosen for the sake of
descriptive simplicity: it only requires two tone symbols, and it corresponds to one of the tones
attested on disyllabic nouns. (The argument for using //L.H// rather  than //L.M.L// as a~label for this category of disyllabic nouns was set out in \sectref{sec:twooptionsforanalysislmvslhorlmvslml}.)

An especially interesting issue is how the \textsc{perfective} acquires its tone after an M-tone verb. The interpretation proposed here is that structural \is{gap-filling}gap-filling in the tonal paradigm has taken place, with far-reaching consequences for the system. This topic will be examined in detail in Chapter \ref{chap:yongningnatonesinadynamicsynchronicperspective}, \sectref{sec:howthesuffixacquiresitslmorhtoneafteramtoneverb}, which explores Yongning Na tones from a dynamic perspective. 


\subsection{Variants resulting from a~division into two tone groups}
\label{sec:variantsresultingfromadivisionintotwotonegroups}


Some deviant patterns are observed in recorded data when the \is{morphotonology}morphotonological integration of subject and verb is, as
it were, incomplete: both the subject and the verb retain the tones they would carry in
isolation. For example, ‘the tiger jumped’ can be realized as /\ipa{lɑ˧ {\kern2pt}|{\kern2pt} tsʰo˧-ze˩}/ instead of the expected, regular /\ipa{lɑ˧ tsʰo˧-ze˧}/. This corresponds to a~division into two tone groups, indicated by the vertical bar ‘\ipa{|}’ in the transcription. 

There are, however, borderline cases where the division into
two tone groups is not complete. If the subject and the verb phrase (consisting of the verb and its \is{suffixes}suffix) were truly treated as separate tone groups, a~postlexical H tone would be expected in all-L sequences, such as the subject /\ipa{bo˩}/ ‘pig’ in (\ref{ex:pigjumped}) and the verb phrase
/\ipa{tsʰɯ˩-ze˩}/ ‘came’ in (\ref{ex:sheeparrived}). One is therefore led to conclude that these examples contain a single tone group, hence \ipa{bo˩ tsʰo˧-ze˩} and \ipa{jo˧ tsʰɯ˩-ze˩} rather than \ipa{bo˩ {\kern2pt}|{\kern2pt} tsʰo˧-ze˩} and \ipa{jo˧ {\kern2pt}|{\kern2pt} tsʰɯ˩-ze˩}. 

%These examples are part of a~broader set analyzed here as cases of extrametricality, addressed in \sectref{sec:casesofbreachoftonalgroupingandconsequencesforthesystem}. The device used to distinguish \textit{bona fide} tone-group divisions from junctures between ~diamond symbol `\ipa{◊}' is used instead of the regular tone-group boundary marker `\ipa{|}', 

\begin{exe}
	\ex
	\label{ex:pigjumped}
	\ipaex{bo˩ tsʰo˧-ze˩.}\\
	\gll bo˩˧	tsʰo˩\textsubscript{b}	-ze˧\textsubscript{b}\\
	pig		to\_jump		\textsc{pfv}\\
	\glt ‘The pig jumped.’
\end{exe}

\begin{exe}
	\ex
	\label{ex:sheeparrived}
	\ipaex{jo˧ tsʰɯ˩-ze˩.}\\
	\gll jo˩	tsʰɯ˩\textsubscript{a}	-ze˧\textsubscript{b}\\
	sheep		to\_come.\textsc{pst}		\textsc{pfv}\\
	\glt ‘The sheep have come.’
\end{exe}
		
Such variants exist for all subject-verb combinations. When
the subject and verb are separated in this way, a~pause can be inserted before the verb, with the \is{stylistics}stylistic effect of highlighting the subject. The role of \is{stylistics}stylistic choices in the division of the utterance
into tone groups, which is of great importance in Yongning Na \isi{prosody}, is discussed in Chapter~\ref{chap:toneassignmentrulesandthedivisionoftheutteranceintotonegroups}; however, the deviant tonal patterns in examples (\ref{ex:pigjumped}) and (\ref{ex:sheeparrived}) have a~significance that extends beyond considerations of stylistic choice. Analysis of these two examples will be taken up in \sectref{sec:thesimplificationofmorphosyntactictonerules} as part of the discussion of ongoing structural change in the Na \is{morphotonology}morphotonological system.

When the demonstratives //\ipa{ʈʂʰɯ˥}// and //\ipa{tʰv̩˥}// appear in subject position, they are always separated from the verb by a \isi{tone group} \is{boundary (between tone groups)}boundary, as illustrated by example (\ref{ex:hehasarrived}).

\begin{exe}
	\ex
	\label{ex:hehasarrived}
	\ipaex{ʈʂʰɯ˧ {\kern2pt}|{\kern2pt} tsʰɯ˩-ze˩˥!}\\
	\gll ʈʂʰɯ˥ 	tsʰɯ˩\textsubscript{a}		-ze˧\textsubscript{b}\\
	3\textsc{sg}		to\_come.\textsc{pst}	\textsc{pfv}\\
	\glt ‘(S)he has come!’
\end{exe}

This is unlike H-tone nouns, which tend to be integrated in the same \isi{tone group} as the verb. For instance, /\ipa{ʐwæ˧ tsʰɯ˧˥}/ ‘the horse came’~-- from //\ipa{ʐwæ˥}// ‘horse’ combined with the same verb ‘to come’ as in (\ref{ex:hehasarrived})~-- forms a single tone group. The distinct behaviour of demonstratives in this regard is consistent with other patterns observed in various nooks and crannies of the Yongning Na \is{morphotonology}morphotonological system (see \sectref{sec:encliticsthatcarrymtonewhenfollowingamtonenoun}, in particular).

 
\subsection[Noun plus existential verb behaves tonally like S+V]{Nouns plus the \is{existentials}existential verb /\ipa{dʑo˩\textsubscript{b}}/ behave tonally like subject-verb combinations}
\label{sec:nounsplustheexistentialverbbehavetonallylikesubjectverbcombinations}

The \is{existentials}existential verb /\ipa{dʑo˩\textsubscript{b}}/ follows the same tonal pattern as other L\textsubscript{b}-tone verbs. Tonally, a noun followed by this \is{existentials}existential verb behaves just like a subject-verb combination (documented in \tabref{tab:thetonepatternsofsubjectplusverbcombinationsinsurfacephonologicaltranscription}
above). This parallel is reflected in the data presented in \tabref{tab:thetoneoftheexistentialverbinassociationwithanoun}.

\begin{table}%[t]
\caption{\label{tab:thetoneoftheexistentialverbinassociationwithanoun}The tone of the existential verb /\ipa{dʑo˩\textsubscript{b}}/ in association with a~noun.}
\begin{tabularx}{\textwidth}{ l@{\hspace{7mm}} l@{\hspace{7mm}} l@{\hspace{7mm}} Q l }
\lsptoprule
	example & meaning & tone & with \is{existentials}existential & tone pattern\\ \midrule
	\ipa{bo˩˧} & pig & LM & \ipa{bo˩ dʑo˥(-ze˩)} & L.H\\
	\ipa{ʐæ˩˥} & leopard & LH & \ipa{ʐæ˩ dʑo˥(-ze˩)} & L.H\\
	\ipa{lɑ˧} & tiger & M & \ipa{lɑ˧ dʑo˩} & L.M\\
	\ipa{jo˩} & sheep & L & \ipa{ɬi˧ dʑo˩} & M.L\\
	\ipa{ʐwæ˥} & horse & H & \ipa{ʐwæ˧ dʑo˧˥} & M.MH\\
	\ipa{ʈʂʰæ˧˥} & deer & MH & \ipa{ʈʂʰæ˧ dʑo˧˥} & M.MH\\
	\addlinespace \hdashline \addlinespace
 	\ipa{ɖɤ˧mi˧} & fox & M & \ipa{ɖɤ˧mi˧ dʑo˩} & M.M.L\\
	\ipa{ʐwæ˧zo\#˥} & colt & \#H & \ipa{ʐwæ˧zo˧ dʑo˧˥} & M.M.MH\\
	\ipa{hwɤ˧li˧˥} & cat & MH\# & \ipa{hwɤ˧li˧ dʑo˧˥} & M.M.MH\\
	\ipa{hwɤ˧mi˥\$} & she-cat & H\$ & \ipa{hwɤ˧mi˧ dʑo˧˥} & M.M.MH\\
	\ipa{kʰv̩˩mi˩} & dog & L & \ipa{kʰv̩˩mi˩ dʑo˩˥} & L.L.L\\
	\ipa{dɑ˧ʝi˩} & mule & L\# & \ipa{dɑ˧ʝi˩ dʑo˩} & M.L.L\\
	\ipa{õ˩dv̩˧˥} & wolf & LM+MH\# & \ipa{õ˩dv̩˧ dʑo˧˥} & L.M.MH\\
	\ipa{nɑ˩hĩ\#˥} & Naxi & LM+\#H & \ipa{nɑ˩hĩ˧ dʑo˧˥} & L.M.MH\\
	\ipa{bo˩mi˧} & sow & LM & \ipa{bo˩mi˧ dʑo˩} & L.M.L\\
	\ipa{bo˩ɬɑ˥} & boar & LH & \ipa{bo˩ɬɑ˥ dʑo˩} & L.H.L\\
	\ipa{hwæ˧tsɯ˥} & rat & H\# & \ipa{hwæ˧tsɯ˥ dʑo˩} & M.H.L\\
\lspbottomrule
\end{tabularx}
\end{table}

Yet a subtle difference sets the \is{existentials}existential verb /\ipa{dʑo˩\textsubscript{b}}/ apart from other L\textsubscript{b}-tone verbs. With other verbs, an L-tone subject noun yields an L.L sequence, with an M.L \is{variants}variant for some nouns but not others. In contrast, the \is{existentials}existential verb only allows the M.L pattern for all nouns carrying lexical L tone. This systematic preference is one of the many fine-grained asymmetries that surface upon close examination.
