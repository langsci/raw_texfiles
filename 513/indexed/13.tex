\chapter{Conclusion}
\label{chap:conclusion}

% Add missing right-page heading, with chapter number: 12. (An ugly shortcut, providing the number as text instead of a variable.)
%\rohead{Conclusion\hspace{.5em}13}

\epigraph{When one aims to please others, one may fail, whereas things that we do to please ourselves always have a~chance to interest someone or other.}{Marcel Proust\footnotemark}{}\footnotetext{This sentence is followed by reflections that are close to linguists' concerns: ``No one is unique: our individualities are made out of a~universal fabric; this is what allows for sympathy and understanding, which are such great pleasures in life. If we could analyze the soul as we analyze matter, it would become apparent that below the surface diversity of minds, as under that of material objects, there are but a~few simple substances and irreducible elements; and that what we think of as our personality is made up from elements which are quite common, and which are met again pretty much everywhere in the universe.'' \textit{Original text:} Quand on travaille pour plaire aux autres on peut ne pas réussir, mais les choses qu'on a~faites pour se contenter soi-même ont toujours chance d'intéresser quelqu'un. Il est impossible qu'il n'existe pas de gens qui prennent quelque plaisir à ce qui m'en a~tant donné. Car personne n'est original et fort heureusement pour la sympathie et la compréhension qui sont de si grands plaisirs dans la vie, c'est dans une trame universelle que nos individualités sont taillées. Si l'on savait analyser l'âme comme la matière, on verrait que, sous l'apparente diversité des esprits aussi bien que sous celle des choses, il n'y a~que peu de corps simples et d'éléments irréductibles et qu'il entre dans la composition de ce que nous croyons être notre personnalité, des substances fort communes et qui se retrouvent un peu partout dans l'Univers. (\textit{Pastiches et mélanges}, Paris: Gallimard, 1919, pp. 108--109)}

%Command \noindent added to avoid having a first indent in cases where a paragraph starts after an epigraph without an intervening title.
{\noindent}The Yongning Na tone system comprises (i)~a~set of phonological rules governing tone-to-syllable association, set out in \sectref{sec:asummaryoftonetosyllableassociationrules}, and (ii)~a~host of rules specific to certain morphosyntactic
contexts, detailed in Chapters~\ref{chap:classifiers}-\ref{chap:verbsandtheircombinatoryproperties}. Different rules apply, for instance, to the association of a~verb with a~subject or
an object, the formation of a~{compound} noun, the combination of a~{numeral} and classifier, or the attachment of affixes to a~word. These tonal paradigms constitute the core of Yongning Na \isi{morphotonology} and represent the bulk of what language learners must \is{language acquisition}acquire to master this tone system. 

As a~conclusion, let us return to the initial puzzle presented in the
introduction. It is now possible to set out the mechanisms whereby the surface phonological tone sequences of examples (\ref{ex:ihavetogoandtakemyluggagenowREP}--\ref{ex:ihavetogoimafraidihavetoleaveREP}) obtain. 

\begin{exe}
	\ex %\label{1}
	\begin{xlist}
		\ex
		\label{ex:ihavetogoandtakemyluggagenowREP}
		\gll njɤ˧	ʑi˩	bi˩	-zo˩	-ho˥.\\
		\textsc{1sg}	to\_take	to\_go	\textsc{obligative}	\textsc{desiderative}\\
		\glt ‘I have to go and take [my luggage] now.'
		
		\ex
		\label{ex:ihavetogoimafraidihavetoleaveREP}
		\gll	njɤ˧	bi˧	-zo˧	-ho˩.\\
		\textsc{1sg}	to\_go	\textsc{obligative}	\textsc{desiderative}\\
		\glt ‘I have to go. / I’m afraid I have to leave.' 
	\end{xlist}
\end{exe}

With morpheme-level transcriptions indicating lexical tone in terms of the Yongning Na tone categories, these sentences can be represented as (\ref{ex:ihavetogoandtakemyluggagenow2REP}--\ref{ex:ihavetogoimafraidihavetoleave2REP}):

\begin{exe}
	\ex
	\begin{xlist}
		\ex
		\label{ex:ihavetogoandtakemyluggagenow2REP}
		\ipaex{njɤ˧ {\kern2pt}|{\kern2pt} ʑi˩ bi˩-zo˩-ho˥.}\\
		\gll njɤ˩ 	ʑi˩\textsubscript{a}		bi˧\textsubscript{c}	-zo˧\textsubscript{a}		-ho˩\\
		\textsc{1sg}	to\_take		to\_go	\textsc{obligative}	\textsc{desiderative}\\
		\glt ‘I have to go and take [my luggage] now.'
		
		\ex
		\label{ex:ihavetogoimafraidihavetoleave2REP}
		\ipaex{njɤ˧ {\kern2pt}|{\kern2pt} bi˧-zo˧-ho˩.}\\
		\gll njɤ˩ 	bi˧\textsubscript{c}	-zo˧\textsubscript{a}		-ho˩\\
		\textsc{1sg}	to\_go	\textsc{obligative}	\textsc{desiderative}\\
		\glt ‘I have to go. / I’m afraid I have to leave.'
	\end{xlist}
\end{exe}

A crucial point is the tone-group {boundary} after the \textsc{1sg} subject: these utterances consist of two tone groups, and tonal processes apply independently within each group, as explained in Chapter~\ref{chap:toneassignmentrulesandthedivisionoftheutteranceintotonegroups}.

The first {tone group} consists of a~single syllable whose lexical tone is L. As documented in Chapter~\ref{chap:thelexicaltonesofnouns}, an isolated L-tone syllable is realized as a~level, non-low tone. This accounts for the surface phonological form /\ipa{njɤ˧}/ in (\ref{ex:ihavetogoandtakemyluggagenow2REP}) and (\ref{ex:ihavetogoimafraidihavetoleave2REP}). 

In (\ref{ex:ihavetogoandtakemyluggagenow2REP}), the second {tone group} consists of two serialized verbs followed by two
suffixes. 
%The second verb, `to go', behaves tonally like its
%grammaticalized counterpart, the {immediate future}
%{suffix}. 
Following the \is{morphotonology}morphotonological rules brought out in
Chapter~\ref{chap:verbsandtheircombinatoryproperties}, //\ipa{ʑi˩\textsubscript{a}}// in combination
with //\ipa{bi˧\textsubscript{c}}// `to go' yields //\ipa{ʑi˩-bi˩}//. Addition of the {obligative} //\ipa{-zo˧\textsubscript{a}}// results in //\ipa{ʑi˩-bi˩-zo˩}//. The final {suffix}, the {desiderative}
//\ipa{ho˩}//, carries L tone. The tonal behaviour of an L-tone {suffix}
depends on the length of the preceding expression: 

\begin{itemize}
    \item When attached to an L-tone verb, it retains L tone (e.g.\ //\ipa{ʑi˩-ho˩}// ‘will take’, L.L).
    \item When attached to an L-tone expression of two or more syllables, it carries H tone (e.g.\ //\ipa{ʑi˩-zo˩-ho˥}// `will need to take', L.L.H).
\end{itemize}

Since //\ipa{ʑi˩-bi˩-zo˩}// consists of three syllables, the suffix carries H tone, yielding //\ipa{ʑi˩-bi˩-zo˩-ho˥}//.

Both tone groups, //\ipa{njɤ˧}// and //\ipa{ʑi˩-bi˩-zo˩-ho˥}//,
contain at least one tone other than L. As a~result, the repair rule for all-L tone groups (Rule~7 in \sectref{sec:alistoftonerules}) does not apply. These tone groups thus proceed unaltered to the surface phonological level
as /\ipa{njɤ˧ {\kern2pt}|{\kern2pt} ʑi˩-bi˩-zo˩-ho˥}/.

In (\ref{ex:ihavetogoimafraidihavetoleave2REP}), the second {tone group} consists of a~main verb, `to go', and the same two suffixes as in (\ref{ex:ihavetogoandtakemyluggagenow2REP}). As per the \is{morphotonology}morphotonological rules brought out in
Chapter~\ref{chap:verbsandtheircombinatoryproperties}, the M tone of the main verb does not affect the tones of the suffixes. Similarly, the M tone of the {obligative} suffix //\ipa{-zo˧\textsubscript{a}}// leaves the following morpheme unchanged, so that all three syllables in the tone group surface with their lexical tones.

While it is satisfying to confirm that the \is{morphotonology}morphotonological patterns described in this volume shed light on these and other examples, it must be acknowledged that this book represents only one~step towards an advanced linguistic model of tone in Yongning Na. A~mid- to
long-term objective is the computer-aided analysis of individual utterances based on a~computational model of the grammar (finite-state modelling), following the methodological suggestion of
\citet{karttunen2006}. Achieving this will require (i)~implementing the entire tonal grammar of Yongning Na in computational scripts, (ii)~glossing Yongning Na texts at the morpheme level, establishing a~link from each morpheme to the lexicon, and (iii)~encoding the
morphosyntactic structure of each utterance. The goal will be to generate surface phonological tone patterns for utterances, specifying the full range of possible variants in the division of utterances into tone groups. 

Such a~model would allow for quantitative verification, across the full set of available data, of the
generalizations proposed in this volume.
More ambitiously, it would enable the researcher to compare the tonal patterns observed in a~textual
utterance (a real sentence from a~text) with a~set of possible alternatives. This, in turn, opens new perspectives for appraising the speaker’s stylistic choices in a~narrative, such as the division of the sentence into tone groups and the choice of
one tonal {variant} over another in cases where two or more patterns are acceptable. 
This approach holds promise for identifying the factors underlying so-called nonconditioned {variation}. By modelling the various components of the language’s prosody and morphosyntax, it may become possible to explore the range of stylistic possibilities offered by the linguistic system~-- e.g.~through the choice of congruence vs.\ dissonance between
message and form, or between syntax and prosodic {phrasing}. A~relevant source of inspiration in this regard is Delattre’s work on {French} {intonation} (\citeyear{delattre1966a,delattre1970}).

A second perspective will consist in examining the phonetic implementation of surface phonological tone sequences in relation to intonation. The ultimate aim here is to assess the contribution of various factors to the final phonetic realization of each syllable, disentangling and spelling out the different components of the speech signal and their linguistic interpretation. In this approach, intonational phenomena are defined as encompassing all those aspects of the speech signal that are not predictable from the utterance’s contrastive units~-- namely the sequence of phonemes and the tonal string as parsed into tone groups. Computational modelling may serve as a~tool for identifying intonational phenomena by contrast, bringing them to the fore against the backdrop of predictable elements. 

%\begin{figure}[t]
%	\caption{Working out tone in Yongning Na: field notes from 2007, with comments added in 2008.}
%	\begin{minipage}{.5\textwidth}
%		\centering
%		\includegraphics[width=.95\linewidth]{figures/ms/1027.jpg}
%%		\captionof{figure}{A figure}
%		\label{fig:test1}
%	\end{minipage}%
%	\begin{minipage}{.5\textwidth}
%		\centering
%		\includegraphics[width=.95\linewidth]{figures/ms/1029.jpg}
%%		\captionof{figure}{Another figure}
%		\label{fig:test2}
%	\end{minipage}
%\end{figure}
%

\begin{figure}[h!!]
	\includegraphics[width=.93\textwidth]{figures/ms/1029.jpg}
%	\caption{Working out tone in Yongning Na: field notes from 2007, with comments added in 2008.}
	\caption{Working out tone in Yongning Na: field notes, 2007.}
	\label{fig:ms2}
\end{figure}

% \begin{figure}[h!!]
	% \includegraphics[width=.9\textwidth]{figures/ms/1107Pred.jpg}
	% \caption{Working out tone in Yongning Na: field notes, 2007.}
	% \label{fig:ms3}
% \end{figure}

%\footnote{Marc Brunelle’s work on {Vietnamese} is an~inspirational example of this strand of research \citep{brunelle2015}.}
