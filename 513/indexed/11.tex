\chapter{Yongning Na in dialectal and areal context}
\label{chap:arealperspectives}
\label{sec:acomparisonwithothersinotibetanlanguages}

As a~preliminary to the typological reflections set out in Chapter~\ref{chap:arealandtypologicaldiscussion}, this brief chapter offers a~few tentative observations on Yongning Na in its dialectal and areal
context. %~-- a perspective that is distinct from the typological one adopted in Chapter~\ref{chap:arealandtypologicaldiscussion}.

\section{Na dialectology: A promising field of study}
\label{sec:dialects}
%\section{Comparison within the Naish group of languages}

Since the publication of the first edition of this volume, significant progress has been made in Na dialectology, expanding both the breadth and depth of coverage. In terms of scope, the dialect of Luggu (\ipa{lo˧gv̩˩}; in Chinese: Běiqúbà \zh{北渠坝}), which is not mutually intelligible with that of Yongning, has been documented (\citealt{li_yuanshi_2021};
the fieldwork location is Yanba village, Daxing, Ninglang county: \zh{宁蒗县大兴镇堰坝村}). Closer to Yongning, a paper is dedicated to tone in the dialect of Lomae (\ipa{lo˩mæ˩}; in Chinese: Qiánsuǒ \zh{前所}) \citep{xu2015_tonal}. In terms of depth, the dialect of Lataddi, previously described by \citet{dobbsetal2016}, has been the subject of a Ph.D.\ dissertation \citep{fily_documentation_2022}. These additions contribute to a~growing empirical basis for the comparative study of tone across Na dialects, ultimately aiming at understanding \isi{tonogenesis} in Naic languages~-- a~topic that is recognized as highly complicated \citep[170]{li_laterals_2024}.
%\cites[11]{refA}[33]{refB}.

Beyond Na proper, it appears interesting to touch upon a~few other \il{Sino-Tibetan}Sino-Tibetan languages of the area  with which Na may have been in \is{language contact}contact in the past. The observations presented here are admittedly preliminary and tentative. 


\section{Naxi and Laze: Close relatives, but not part of a~convergence area}
\label{sec:comparisonwithinnaish}

It has been observed that “the same phonological processes such as \isi{tone sandhi} and \isi{lengthening}
obviously make reference to different prosodic domains within the same language family ({Bantu})”
\citep[132]{zerbian2006a}. High diversity is also found within Naish, even though this lower-level grouping is incomparably {\linebreak}smaller than Bantu.

\is{language contact}
Comparison of Na with \ili{Naxi} and \il{Laze|textbf}Laze is fundamental for {diachronic} research, as these two languages are Yongning Na's closest known relatives (see \sectref{sec:thepositionofnaandnaxiwithinsinotibetan}). But from an areal point of view, I have not been able to find evidence of diffusional convergence between Na and its sister languages. It seems that these languages have not formed a~convergence area. On the contrary, their histories have been marked by strong social, political and cultural \textit{divergence} since the 14\textsuperscript{th} century. The Naxi chiefdom of Lijiang underwent increasing Sinicization (and finally came under direct Chinese administration in the 18\textsuperscript{th} century), whereas in Yongning and Muli, the feudal chieftain system persisted until the mid-20\textsuperscript{th} century due to the failure of military campaigns to subdue the Liangshan Yi area, which constitutes the gateway to these peripheral regions (see Appendix~\ref{chap:historyanthropologysociology}, \sectref{sec:historicaloutline}). 

The Laze are a~small group of some four hundred people who migrated from Yanbian to their
current location in Muli towards the end of the 19\textsuperscript{th} century. They are reported to be among the speakers of \ili{Naish} languages who left Yanbian as it became predominantly \ili{Yi}-speaking, but little is known about sociolinguistic configurations prior to the influx of Yi clans. Consequently, I~am not in a~position to assess how areal convergence may have contributed to shaping the prosody of Laze in past centuries. The present-day areal situation is not well-understood either, as the other Naish dialects spoken in the vicinity of the Laze villages remain, to my knowledge, undocumented.

%The Laze who settled in the valley of Xiangjiao \zh{项脚} (in Muli \zh{木里} county) found themselves in an environment where their neighbours in were speakers of other Naish dialects; they developed no bilingualism in languages that are spoken in other areas of Muli, such as Pumi, Tibetan, Shixing (Xumi), Namuyi, or Lizu. This goes a~long way towards explaining why Laze did not undergo convergence towards prosodic culminativity.

From a~prosodic point of view, \ili{Naxi} and \ili{Laze} currently exemplify the \textit{nonrestricted tone} type as defined by \citet{
voorhoeve1973}: tones are assigned to individual syllables without regard to the tonal pattern of the entire word or phrase. In particular, in these 
two languages there is no restriction on the number of successive H tones, unlike in Yongning Na, where H tone is \is{culminativity}culminative (that is, there is 
at most one H tone per tone group). Interestingly, \isi{culminativity} has been proposed as an areal characteristic of the languages that have long 
coexisted in the county of Muli \citep[160]{chirkova2012}.\footnote{Muli was a~semi-independent kingdom ruled by 
Pumi hereditary lama kings until the mid-20\textsuperscript{th} century. It is important to clarify that the claim is not that Muli county as a whole constituted a single convergence area, but rather that convergence took place in certain parts of Muli that were historically multilingual over extended periods. The high mountains and deep valleys of Muli create formidable obstacles to communication, and different parts of the county constitute strikingly different sociolinguistic environments, with different languages in \is{language contact}contact and different relationships of prestige among ethnic groups and 
their languages.} 

In terms of this important property of the prosodic system, Yongning Na does not pattern with Naxi or Laze, but with 
languages that have long been spoken in Muli, such as \ili{Pumi}, \ili{Namuyi}, \ili{Shixing} (Xumi), \ili{Lizu},\footnote{\textit{Lizu} is not to be confused with \textit{Lisu} \zh{傈僳语}. The former is an Ersuish language, spoken by approximately 7,000 
people who reside along the banks of the Yalong \zh{雅砻} River (Tibetan: Nyag chu) \citep{chirkovaetal2012}; the latter is a~Yi (Loloish) language spoken by about 900,000 people in a~wide area that straddles boundaries between China (Yunnan and Sichuan), Thailand, Myanmar, and India.} and the local dialect 
of \ili
{Tibetan}. Among these, Pumi deserves particular attention, as it is also spoken in Yongning.


\section{Comparison with Pumi}
\label{sec:compwithpumi}

The prosodic system of Na is remarkably close to that of \ili{Pumi} (also known as Prinmi), a~neighbouring \il{Sino-Tibetan}Sino-Tibetan language. 

\subsection[The tone group and its ties with information structure]{The tone group and its role in conveying information structure}
\label{sec:thetonegroupasbuildingblockofutterancesanditsroleinconveyinginformationstructure}

In Wadu \zh{瓦都} \ili{Pumi}, as in Yongning Na, utterances are structured into \textit{tone groups}.

\begin{quotation}
	Within a~\isi{tone group}, the underlying tone of one lexical element (usually the left-most element)
	spreads (usually rightwards) to the adjacent morphemes in the same \isi{tone group} ({\dots}). The
	remaining elements in a~\isi{tone group} are assigned default low surface tone. Tone does not
	spread across \isi{tone group} boundaries. \citep[66]{daudey2014}
\end{quotation}

%Command \noindent added to avoid having an indent. Proofreader suggestion: since this sentence continues the argument, it is better not to indent. 
{\noindent}As in Yongning Na, the \isi{tone group} plays a~key role in conveying \isi{information structure}.

\begin{quotation}
	[S]ome elements always combine with others into a~single \isi{tone group}, some elements always form
	a~\isi{tone group} by themselves, and for some elements, speakers can decide to combine or not combine
	them into tone groups. The latter elements are the most interesting, in that they allow the speaker
	to express pragmatic differences through the choice of combining them or not. \citep[68]{daudey2014}
\end{quotation}

%Command \noindent added to avoid having an indent. Proofreader suggestion: since this sentence continues the argument, it is better not to indent. 
{\noindent}The parallel with the observations on Na set out in Chapter~\ref{chap:toneassignmentrulesandthedivisionoftheutteranceintotonegroups} is striking. Such similarities raise the question of whether \isi{language contact} has played a~role. The variety of \ili{Pumi} studied by Henriëtte Daudey (Wadu \ili{Pumi}) is spoken in the plain of Yongning, where the Na and the \ili{Pumi} have coexisted on good terms for at least eight centuries. The similarities could thus stem from \isi{language contact}. The two groups “frequently intermarry and so a~fair amount of \ili{Pumi} speak or understand Yongning Na to some degree. The reverse is not necessarily true” \citep[5]{daudey2014}. However, similar characteristics are also observed in another dialect of \ili{Pumi}, that of Niuwozi \zh{牛窝子}, which is not in contact with Yongning Na. This dialect, spoken near the Ninglang county seat, is in contact with another variety of the Na language (Luggu \ipa{lo˧gv̩˩}; in Chinese: Běiqúbà \zh{北渠坝}), not mutually intelligible with the Yongning variety. (As a~piece of anecdotal evidence illustrating the degree of mutual comprehension: one of the daughters of consultant F4 married a~Na speaker from that area; the differences in dialect led the couple to adopt {Mandarin} as their language of communication.) 

While the linguistic terminology used in the description of Niuwozi \ili{Pumi} differs slightly from that used for Wadu \ili{Pumi} and Yongning Na, the observed prosodic patterns appear highly similar: 

\begin{quotation}
	Under the influence of {intonation}, the underlying H tone of a~phonological word is readily removable when it is situated in the final unit of the clause ({\dots}). When this happens, the phonological word is merged with the other prosodic domain (removal of the original {boundary} of phonological word due to a~loss of H tone in a~following word [{\dots}]). Sometimes, a~series of low tones may appear in the ending syllables of an~utterance after the {boundary} of phonological word is eliminated ({\dots}). \citep[69]{ding2014}
\end{quotation}

%Command \noindent added to avoid having an indent. Proofreader suggestion: since this sentence continues the argument, it is better not to indent. 
{\noindent}Thus, the similarities in prosodic organization between Pumi and Na are not necessarily the outcome of areal convergence. As pointed out in \sectref{sec:concludingremark}, the division of an utterance into intonation phrases plays a central role in conveying \isi{phrasing} and \isi{prominence} in the most diverse languages. Within the Sino-Tibetan family, a~distant parallel to the Na facts is found in \ili{Zhuokeji rGyalrong}, where “clauses that are
juxtaposed without any overt linkage marker that denotes coordination or consecutivization, or any morphosyntactic marking that signals dependency of one clause on the other” can be integrated into one group (called “intonation unit”) to highlight “strong rhetorical links” between these clauses \citep[208]{lin2009}.
% including thoroughly unrelated (and extensively studied) languages such as English.


\subsection{Other similarities}
\label{sec:othersimilarities}

Further similarities emerge in phonological rules. In both Na and \ili{Pumi}, each prosodic domain requires at least one non-L tone; when none is present, an~H tone is added to the final syllable, yielding a~rising tone, LH \citep[60]{ding2014}. Concerning numeral-plus-classifier\is{classifiers} phrases, \citet[69]{ding2014} observes that tone patterns “are not utterly predictable from the tones of the two formatives, as other factors beyond phonology are at work”. This situation is closely parallel to that found in Na (studied in Chapter~\ref{chap:classifiers}), although judging from Picus Ding's description, the degree of complexity would seem to be somewhat lower in that particular variety of \ili{Pumi} than in the Na variety under study.
%(The issue of assessing the degree of complexity of the tone system is addressed in \sectref{sec:morphophonologicalcomplexity}.)

The high number of points of similarity suggests that further comparison of Na and \ili{Pumi} could be highly revealing. %The aim would be to attain the level of analytical depth and precision exemplified by \citet{wagneretal2010} in their comparative study of {English} and French. 



%\subsection{Role of the number of syllables of the words involved a~given tone rule}
%\label{sec:roleofthenumberofsyllablesofthewordsinvolvedagiventonerule}
%
%No tone change takes place in compounding when one of the two input nouns has more than two
%syllables (see~\ref{sec:themainfactscoordinativecompounds}). The influence of the number of syllables on the type of
%phonological processes that take place in compounds is a~point of similarity with other languages of
%the area, such as \ili{Shixing}.

\section{Comparison with Yi}
\label{sec:compwithyi}

\ili{Naish} languages have remarkable typological similarities with languages of the \ili{Yi} (Loloish) branch of {Burmese}-{Yi} ({Burmese}-Lolo), particularly in their morphosyntactic profiles. Tonal alternations arise in comparable contexts, such as compounds and \is{numerals}numeral-plus-classifier phrases. In \ili{Nosu}, for instance, tone \ipa{³³} shifts to a sandhi tone \ipa{⁴⁴} (rendered orthographically with a final \textit{x}) when followed by another \ipa{³³} tone. This dissimilatory change is morphosyntactically conditioned, as illustrated by the tonal contrast between \textit{nga gu} ‘I called (someone)’ and \textit{ngax gu} ‘(Someone) called me’, which reflects a grammatical distinction between {agent} and {patient} \citep[28]{gerner2013}. However, the extent of such \is{morphotonology}morphotonological alternations remains limited: tone sandhi is reported in only eight contexts, described over three pages of a grammar totalling more than five hundred \citep[28--30]{gerner2013}.

At present, the working hypothesis is that such tonal parallels between \ili{Yi} and Naish are instances of \isi{homoplasy}: independent developments arising under similar structural pressures, most notably the drive towards monosyllabicization. This tendency, widely attested across East and Southeast Asia, is generally considered to originate from the influence of Old Chinese, which exerted direct or indirect pressure on languages from several families, including Sino-Tibetan, Tai-Kadai, Hmông-Miên, and Austroasiatic. Yet this process has been more radical in Yi than in Chinese itself, with segmental reduction reaching a more extreme degree (\citealt[125]{michaud2012b}, drawing on \citealt{haudricourt1991}).

%Pending further investigation, the hypothesis concerning tonal similarities between \ili{Yi} and Naish is that they constitute instances of homoplasy: parallel developments, occurring independently in the two branches under similar structural pressures, in particular the pressure towards monosyllabicization. ``While the impetus for monosyllabicization can safely be hypothesized to have come from Old Chinese, which influenced~-- directly or indirectly~-- languages of the Sino-Tibetan, Tai-Kadai, Hmông-Miên and Austroasiatic families, segmental depletion has reached a more extreme development in Yi than within Chinese itself \citep{haudricourt1991}'' \citep[125]{michaud2012b}.

%Earlier formulation: The morphotonology has limited extent, however: in total, there are eight contexts where \isi{tone sandhi} occurs. Their description takes up no more than three pages in a~grammar of half a~thousand pages \citep[28--30]{gerner2013}.

\section[Contact between two-level and three-level tone systems]{A hypothesis about contact between two-level and three-level tone systems}
\label{sec:twolevelsthreelevels}

\is{language contact}Contact between two-level and three-\is{level tones}level tone systems appears as an especially interesting topic for areal studies of tone. \ili{Pumi}, with
which Na has been in at least occasional contact for centuries, has a~two-level system. Among Na dialects, two-level systems are found in Lataddi,
%Wuzhiluo \zh{五指落}, 
on the edge of the swamp area known as the Grass Sea, which forms the
eastern end of Lugu Lake \citep{dobbsetal2016,fily_documentation_2022}, and in Shuiluo \zh{水落}, in the county of Muli \zh{木里} (source: unpublished field notes, 2009). The dialect of Luoshui \zh{落水}, geographically close to the Yongning plain, has three levels, although the highest level has a~relatively marginal status \citep{lidz2010}. Alawua (the dialect studied in this volume) likewise has three levels, but with a~strongly restricted distribution. By contrast, dialects spoken further to the west and northwest, such as Labai \zh{拉柏}, clearly have three levels (L, M, and H). 

\is{language contact}Past contact between two-level and three-\is{level tones}level tone systems may account for some synchronic phenomena found in Alawua, such as the exceptionless {phonological rule} prohibiting H tone in tone-group-initial position, effectively reducing the number of tonal contrasts to two in that position. This rule has far-reaching consequences: at the surface phonological level, H tone (as distinct from M) cannot occur on a~monosyllable pronounced \is{form!in isolation}in isolation, nor can the tonal patterns H.M, H.MH, H.L or
H.H occur on disyllables \is{form!in isolation}in isolation, since an isolated form (often referred to as a~\textit{citation form}) constitutes a~\isi{tone group} in itself. As discussed in \sectref{sec:thecreationoffloatinghtonesaconsequenceofphonotacticconstraints}, correspondences between overt H~tones in the Labai dialect and \is{floating tone}floating H~tones in Alawua suggest that word-initial H~tones in Alawua may have been displaced from their original position on the
first syllable of the word as a~response to the enforcement of the rule prohibiting initial H. If this scenario is correct, the next {question} is why word-initial H tones became phonotactically illicit in the first place. At this stage, one can entertain the hypothesis that \is{language contact}contact with a~two-\is{level tones}level tone system played a~role, as speakers of such a~system might have had special difficulty maintaining a~three-way tonal opposition in initial position.  

Synchronic case studies of \is{language contact}contact between two-level and three-level tone systems provide valuable insights into the types of processes that such contact can induce, as well as their possible consequences for the linguistic systems involved. Ongoing investigations into contact between \ili{Mano} and \ili{Kpelle}, two \ili{Mande} languages that have different tonal systems~-- Kpelle having a~binary tonal contrast (Low vs.\ High) and Mano possessing three contrastive level tones (Low, Mid, and High)~--, are exemplary \citep{konoshenko2024}. Findings from such studies, combined with additional data on present-day dialectal diversity within Na, hold potential for progress in the study of the historical role that \is{language contact}contact may have played in shaping the tonal systems observed today in Na dialects. 

Mention of typological parallels between Naish tone systems and those of \ili{Mande} provides a natural transition to the discussion of typological topics in the next chapter.