\chapter{Vowels and consonants} 
\label{chap:appendixa}
\label{chap:vowelsandconsonants}

% This chapter presents the segmental phonology of Yongning Na; it does not constitute a~necessary preliminary to the discussion of tones, since there are no restrictions on co-occurrence of tones and segments, and there are no special classes of segments triggering synchronic tonal processes (as is the case with depressor consonants in {Bantu}, e.g.~in the Ikalanga language: see \citealt{hymanetal1998}). The information in this chapter is background knowledge, explaining the notation used for vowels and consonants.

% Temporarily widen the epigraph block
{\setlength{\epigraphwidth}{.55\textwidth}

\epigraph{{\dots}~without a~good sense of how languages vary, not only in terms of the symbolic units such as phonemes and allophones but also in the details of their phonetic implementation, we have little hope of understanding the possible range of language. Endangered languages in particular represent an important but often-ignored source of information about what is possible in Language.}{Richard Wright, University of Washington Phonetics Laboratory website, 2014}

} % end of: \setlength{\epigraphwidth}{.53orwhatever\textwidth}

%Command \noindent added to avoid having a first indent in cases where a paragraph starts after an epigraph without an intervening title.
{\noindent}In Yongning Na, there are no restrictions on co-occurrence of tones and segments, nor are there special classes of segments that trigger synchronic tonal processes (as is the case with depressor consonants in \ili{Bantu}, e.g.~in the \ili{Ikalanga} language: see \citealt{hymanetal1998}). So it did not appear appropriate to interpose a~presentation of the language's segmental phonetics and phonology between the reader and the book's central topic~-- tone.

This Appendix provides an opportunity to smuggle a~free-standing overview of the vowels and consonants of Yongning Na into this tonal study. It discusses the choices made in phonemicization, with particular emphasis on areas where phonemic analysis is cracking at the seams. Several of these observations could serve as starting points for \is{experimental phonetics}experimental phonetic/phonological studies, building on the availability of several hours of transcribed and annotated recordings.


%\section[Introductory note]{Introductory note: no phonological alternations; rich {coarticulatory} phenomena}
%\section{Introductory note}
\label{sec:introductionnophonologicalalternationsrichcoarticulatoryphenomena}

The vowels and consonants of Yongning Na are phonologically inert:
they are not involved in synchronic phonological rules and processes. In this respect, Yongning Na
stands at the opposite end of the typological continuum from a~language such as \ili{Kifuliiru} (\ili{Bantu}), which
has (i)~a~range of phonological rules, such as the strengthening of /\ipa{h}/, /\ipa{l}/, and
/\ipa{r}/ to a~plosive when preceded by a~nasal, 
%and the total \isi{assimilation} of a~non-high,
%non-back vowel (/\ipa{e}/ or /\ipa{a}/) to a~following vowel at morpheme boundaries within the
%word, e.g.~/\ipa{a}/+/\ipa{u}/$\rightarrow$/\ipa{uu}/, 
and (ii)~morphological rules, such as the
deletion of final consonants in {resultative} verb forms \citep[37--96]{vanotterloo2011}.

When examining vowels and consonants in Yongning Na, attention is drawn instead to their patterns of \isi{coarticulation}. Due to dramatic \isi{phonological erosion} since proto-\il{Sino-Tibetan}Sino-Tibetan \citep{jacquesetal2011}, syllabic structure in
Yongning Na has collapsed down to (C)(G)V+T, where C is a~consonant, G a~glide~-- with a~severely restricted
distribution~--, V a~syllable nucleus, and T the tone. The brackets indicate that C and G are
optional. 

Coarticulation constitutes a~salient part of a~language’s sound system \citep{keating1990, kuhnertetal1999}. Structural
approaches to phonological systems predict cross-linguistic differences in
\isi{coarticulation} and articulatory reduction. For example, the extent to which high, front vowels exert a~palatalizing
influence depends in part on the number, nature, and
functional yield of existing phonemic oppositions. In Na, which contrasts /\ipa{ki}/ and
/\ipa{tɕi}/, the range of allophonic \isi{variation} of /\ipa{ki}/ can safely be predicted to be narrower
than in \ili{Naxi}, which does not have this contrast.

Phonetic studies confirm that \isi{coarticulation} is language-specific. %It is shaped by factors that include phonological inventory size and the phonological distribution of contrasts \citep[162]{dicanio2012}. 
In the description that follows,
special attention is paid to \isi{coarticulation}, allophonic \isi{variation}, and phenomena of
articulatory reduction. 


\section{Consonant and vowel charts}
\label{sec:ConsonantAndVowelChart}

\tabref{tab:theinitialsofyongningna} presents the consonant inventory, and
\figref{fig:therhymesofyongningna} displays the set of rhymes. %\footnote{This Appendix focuses on the Alawua dialect. A~systematic comparison across dialects is not undertaken here.} 
The chart of rhymes includes the syllabic consonants /\ipa{ɻ̩}{\kern2pt}/ and /\ipa{v̩}/, which are discussed in \sectref{sec:consonantalnuclei}. Their placement in the lower
right-hand and upper right-hand areas of the chart serves as a~rough approximation of their articulatory
characteristics. 

To prevent overcrowding, the following rhymes are not shown in the chart:

\begin{itemize}
	\item Rhymes containing a~glide (discussed in \sectref{sec:apresentationofonglideswithahypothesisaboutadiachroniconsetofhardeningofinitialglides}): /\ipa{wæ}/, /\ipa{wɑ}/, /\ipa{wɤ}/, /\ipa{jæ}/, /\ipa{jɤ}/, and /\ipa{jo}/.
	\item Nasal vowels:
	/\ipa{ĩ}/, /\ipa{ṽ̩}/, /\ipa{w̃ɤ}/,\footnote{In the rhymes /\ipa{w̃ɤ}/ and /\ipa{w̃æ}/, the diacritic for nasality is placed on
		the glide rather than on the final vowel because the degree of phonetic nasalization decreases over
		the course of the rhyme. The nasal rhymes of Yongning Na are discussed in \sectref{sec:nasalrhymes}.} /\ipa{æ̃}/, and /\ipa{ɑ̃}/, all of which occur exclusively after /\ipa{h}/.
	\item The rhymes /\ipa{ɻ̩̃}{\kern2pt}/ and /\ipa{w̃æ}/, which always constitute syllables on their own (i.e.\ they do not combine with an initial consonant).
	\item The nasal vowel /\ipa{õ}/, which occurs either after /\ipa{h}/ or on
	its own, i.e.\ in the syllables /\ipa{hõ}/ and /\ipa{õ}/.
\end{itemize}

{\setlength\tabcolsep{4pt}
	\begin{table}
		\caption{The initials of Yongning Na.}
%		{\fontsize{9}{10.75}\selectfont
		\begin{tabularx}{\textwidth}{ l P{14mm} l P{14mm} l l l l}
			\lsptoprule
			& bilabial~/ labio-dental & dental & alveolo-palatal & retroflex & velar & uvular & glottal\\\midrule
			plosive & \ipa{pʰ p b}  & \ipa{tʰ t d} & & \ipa{ʈʰ ʈ ɖ} & \ipa{kʰ k g} & \ipa{qʰ q} & %\ipa{ʔ} % Glottal stop removed at second edition, to align with analysis of /u/ and /o/ as onsetless: no phonemic glottal stop required.
            \\
			affricate &   & \ipa{tsʰ ts dz} & \ipa{tɕʰ tɕ dʑ} & \ipa{ʈʂʰ ʈʂ ɖʐ}  & & &\\
%			nasal & \ipa{m} & \ipa{n} & \ipa{ɲ} & \ipa{ɳ} & \ipa{ŋ} &\\
			nasal & \ipa{m}  & \ipa{n} &  \ipa{ɲ} &  \ipa{ɳ} & \ipa{ŋ} & &\\
			fricative &  \ipa{f} & \ipa{s z} & \ipa{ɕ ʑ} & \ipa{ʂ ʐ} & & \ipa{ʁ} & \ipa{h}\\
			lateral  &  & \ipa{ɬ l} & & & & &\\
			approximant &  &  &  & \ipa{ɻ} &  &  &\\\lspbottomrule
		\end{tabularx}
%				}
		\label{tab:theinitialsofyongningna}
	\end{table}}
	
	\begin{figure}
		\caption{The rhymes of Yongning Na.}
		\begin{tikzpicture}[scale=1.25]
		\large
		\tikzset{
			vowel/.style={fill=white, anchor=mid, text depth=0ex, text height=1ex},
			dot/.style={circle,fill=black,minimum size=0.6ex,inner sep=0pt,outer sep=-1pt},
		}
		\coordinate (hf) at (0,3); % high front
		\coordinate (hb) at (4,3); % high back
		\coordinate (lf) at (2,0); % low front
		\coordinate (lb) at (4,0); % low back
		\def\V(#1,#2){barycentric cs:hf={(3-#1)*(2-#2)},hb={(3-#1)*#2},lf={#1*(2-#2)},lb={#1*#2}}
		
		% Draw the horizontal lines first.
		\draw (\V(0,0)) -- (\V(0,2));
		\draw (\V(1,0)) -- (\V(1,2));
		\draw (\V(2,0)) -- (\V(2,2));
		\draw (\V(3,0)) -- (\V(3,2));
		
		% Place all the unrounded-rounded pairs next, on top of the horizontal lines.
		\path (\V(0,0))     node[vowel, left] (close) {\ipa{i}} node[vowel, right] {} node[dot] {};
		\path (\V(0,1))     node[vowel, left] {} node[vowel, right] {} node[dot] {};
		\path (\V(0,2))     node[vowel, left] {\ipa{ɯ}} node[vowel, right] {\ipa{u}} node[dot] {};
		\path (\V(0.5,0.4)) node[vowel, left] {} node[vowel, right] {} node[ ] {};
		\path (\V(0.5,1.6)) node[vowel, left] {\ipa{v̩}} node[vowel, right] {} node[ ] {};
		\path (\V(1,0))     node[vowel, left] (closemid) {\ipa{e}} node[vowel, right] {} node[dot] {};
		\path (\V(1,1))     node[vowel, left] {} node[vowel, right] {} node[dot] {};
		\path (\V(1,2))     node[vowel, left] {\ipa{ɤ}} node[vowel, right] {\ipa{o}} node[dot] {};
		\path (\V(2,0))     node[vowel, left] (openmid) {} node[vowel, right] {} node[dot] {};
		\path (\V(2,1))     node[vowel, left] {} node[vowel, right] {} node[dot] {};
		\path (\V(2,2))     node[vowel, left] {} node[vowel, right] {} node[dot] {};
		\path (\V(2.5,0))   node[vowel, left] {\ipa{æ}} node[vowel, right] {} node[   ] {};
		\path (\V(2.5,1.6)) node[vowel, left] {\ipa{ɻ̩}} node[vowel, right] {} node[ ] {};
		\path (\V(3,0))     node[vowel, left] (open) {} node[vowel, right] {} node[dot] {};
		\path (\V(3,2))     node[vowel, left] {\ipa{ɑ}} node[vowel, right] {} node[dot] {};
		
		\node[scale=0.8] [left=of close] {Close};
		\node[scale=0.8] [left=of closemid] {Close-mid};
		\node[scale=0.8] [left=of openmid] {Open-mid};
		\node[scale=0.8] [left=of open] {Open};
		
		\node[scale=0.8] () at (0,3.75) {Front};
		\node[scale=0.8] () at (2,3.75) {Central};
		\node[scale=0.8] () at (4,3.75) {Back};
		
		% Draw the vertical lines.
		\draw (\V(0,0)) -- (\V(3,0));
		\draw (\V(0,1)) -- (\V(3,1));
		\draw (\V(0,2)) -- (\V(3,2));
		
		% Place the unpaired symbols last, on top of the vertical lines.
		\path (\V(1.5,1))   node[vowel]       {\textbf{ə}};
		%\path (\V(2.5,1))   node[vowel]       {};
		\end{tikzpicture}
		\label{fig:therhymesofyongningna}
	\end{figure}
	
%{\largerpage[-1]} % Added on April 21st, 2025
 	
The notation used throughout this book and in the transcription of texts is phonemic in
orientation. However, a~concern for phonetic transparency led to indicate the empty-onset fillers [{\kern1.3pt}\ipa{ʝ}], [\ipa{ɣ}], and [\ipa{w}] in transcriptions, even though they are not phonemically contrastive (as explained in \sectref{sec:smoothphoneticonsets}) and are therefore not included in \tabref{tab:theinitialsofyongningna}. 

\largerpage
Tables \ref{tab:InNucl} and \ref{tab:InNucl2} list all possible vowel nuclei following each initial consonant. In the interest of space, syllables with nasal rhymes are omitted. These tables only offer a~first-pass view of Na phonotactics: they provide no information on lexical frequency, only a~binary indication of whether a~given combination occurs. A~syllable is considered attested if it is firmly established in the Na lexicon, excluding combinations introduced through \is{loanwords}lexical borrowing, \isi{vowel harmony}, or expressive processes, no matter how widely attested these are in the present-day state of the language. 

For instance, combinations of dental stops with the vowel /\ipa{æ}/ are marked as absent because their only attestations occur in words such as /\ipa{tʰæ˩tsɯ˧}/ ‘jar’, \is{loanwords}borrowed from {Mandarin} \textit{tánzi} \zh{坛子}, as well as /\ipa{tæ˧ɻæ˩}/ ‘Adam's apple’ and /\ipa{læ˧dæ˧qæ˥}/ ‘armpit’, where the presence of /\ipa{æ}/ in the first syllable may result from \isi{vowel harmony} with the second syllable. Similarly, combinations of retroflex stops with /\ipa{ɑ}/ are indicated as nonexistent because their only occurrences are in /\ipa{ʈʂʰɑ˧lɑ˧}/ ‘to chat’, /\ipa{qv̩˧ɻ̩˧-ʈʂʰɑ˧nɑ˥\#}/ (the name of a~mountain), and /\ipa{ʈʂɑ˧tɑ˥}/ ‘written sign, tracing’, where the /\ipa{ɑ}/ of the first syllable may likewise result from \isi{vowel harmony}. 

By abstracting away from phenomena that can be identified as {structural} \is{gap-filling}gap-filling, the tables highlight empty slots in the phonotactic inventory and cases of complementary distribution. Comments about the inventory of syllables are provided in \sectref{sec:commentsabouttheinventoryofsyllables}.

%\todo{move split between first and esecond table up}
% Note: \zh{✓} is for ✓ and \ding{55} is for ✗
\begin{table}
	\caption{Inventory of attested combinations of initials and rhymes, leaving aside marginal words (loanwords, expressive words{\dots}). First part.}
	\label{tab:InNucl}
	\begin{tabular}{lllllllllllllllll}
		\lsptoprule
		& \ipa{i} & \ipa{ɯ} & \ipa{u} & \ipa{v̩} & \ipa{ɻ̩} & \ipa{e} & \ipa{ɤ} & \ipa{o} & \ipa{æ} & \ipa{ɑ} & \ipa{wæ} & \ipa{wɑ} & \ipa{wɤ} & \ipa{jæ} & \ipa{jɤ} & \ipa{jo}\\ \midrule
		\ipa{Ø} & \zh{✓} & \zh{✓} & \zh{✓} & \zh{✓} & \zh{✓} & \ding{55} & \ding{55} & \zh{✓} & \zh{✓} & \zh{✓} & \ding{55} & \ding{55} & \zh{✓} & \ding{55} & \zh{✓} & \zh{✓}\\
		\ipa{pʰ} & \zh{✓} & \ding{55} & \ding{55} & \zh{✓} & \zh{✓} & \ding{55}& \zh{✓} & \zh{✓} & \zh{✓} & \ding{55} & \ding{55} & \ding{55} & \ding{55} & \ding{55} & \ding{55} & \ding{55}\\
		\ipa{p} & \zh{✓} & \ding{55} & \ding{55} & \zh{✓} & \zh{✓} & \ding{55} & \zh{✓} & \zh{✓} & \zh{✓} & \ding{55} & \ding{55} & \ding{55} & \ding{55} & \ding{55} & \ding{55} & \ding{55}\\
		\ipa{b} & \zh{✓} & \ding{55} & \ding{55} & \zh{✓} & \zh{✓} & \ding{55} & \zh{✓} & \zh{✓} & \zh{✓} & \ding{55} & \ding{55} & \ding{55} & \ding{55} & \ding{55} & \ding{55} & \ding{55}\\
		\ipa{m} & \zh{✓} & \ding{55} & \ding{55} & \zh{✓} & \ding{55} & \ding{55} & \zh{✓} & \zh{✓} & \zh{✓} & \ding{55} & \ding{55} & \ding{55} & \ding{55} & \ding{55} & \ding{55} & \ding{55}\\
		\ipa{tʰ} & \zh{✓} & \ding{55} & \ding{55} & \zh{✓} & \ding{55} & \ding{55} & \ding{55} & \zh{✓} & \ding{55} & \zh{✓} & \ding{55} & \ding{55} & \ding{55} & \ding{55} & \ding{55} & \ding{55}\\
		\ipa{t} & \zh{✓} & \ding{55} & \ding{55} & \zh{✓} & \ding{55} & \ding{55} & \zh{✓} & \zh{✓} & \ding{55} &  \zh{✓} & \ding{55} & \ding{55} & \ding{55} & \ding{55} & \ding{55} & \ding{55}\\
		\ipa{d} & \zh{✓} & \ding{55} & \ding{55} & \zh{✓} & \ding{55} & \ding{55} & \zh{✓} & \zh{✓} & \ding{55} &  \zh{✓} & \ding{55} & \ding{55} & \ding{55} & \ding{55} & \ding{55} & \ding{55}\\
		\ipa{tsʰ} & \zh{✓} & \zh{✓} & \ding{55} & \ding{55} & \ding{55} & \zh{✓} & \zh{✓} & \zh{✓} & \ding{55} &  \zh{✓} & \ding{55} & \ding{55} & \ding{55} & \ding{55} & \ding{55} & \ding{55}\\
		\ipa{ts} & \zh{✓} & \zh{✓} & \ding{55} & \ding{55} & \ding{55} & \zh{✓} & \zh{✓} & \zh{✓} & \ding{55} &  \zh{✓} & \ding{55} & \ding{55} & \ding{55} & \ding{55} & \ding{55} & \ding{55}\\
		\ipa{dz} & \zh{✓} & \zh{✓} & \ding{55} & \ding{55} & \ding{55} & \zh{✓} & \zh{✓} & \zh{✓} & \ding{55} &  \zh{✓} & \ding{55} & \ding{55} & \ding{55} & \ding{55} & \ding{55} & \ding{55}\\
		\ipa{n} & \zh{✓} & \ding{55} & \ding{55} & \zh{✓} & \ding{55} & \zh{✓} & \ding{55} & \zh{✓} & \ding{55} &  \zh{✓} & \ding{55} & \ding{55} & \ding{55} & \ding{55} & \ding{55} & \ding{55}\\
		\ipa{s} & \zh{✓} & \zh{✓} & \ding{55} & \ding{55} & \ding{55} & \zh{✓} & \zh{✓} & \zh{✓} & \ding{55} &  \zh{✓} & \ding{55} & \ding{55} & \ding{55} & \ding{55} & \ding{55} & \ding{55}\\
		\ipa{z} & \zh{✓} & \zh{✓} & \ding{55} & \ding{55} & \ding{55} & \zh{✓} & \ding{55} & \zh{✓} & \ding{55} &  \zh{✓} & \ding{55} & \ding{55} & \ding{55} & \ding{55} & \ding{55} & \ding{55}\\
		\ipa{ɬ} & \zh{✓} & \ding{55} & \ding{55} & \zh{✓} & \ding{55} & \ding{55} & \ding{55} & \zh{✓} & \ding{55} &  \zh{✓} & \ding{55} & \ding{55} & \ding{55} & \ding{55} & \ding{55} & \ding{55}\\
		\ipa{l} & \zh{✓} & \ding{55} & \ding{55} & \zh{✓} & \ding{55} & \zh{✓} & \ding{55} & \zh{✓} & \ding{55} &  \zh{✓} & \ding{55} & \ding{55} & \ding{55} & \ding{55} & \ding{55} & \ding{55}\\
		\lspbottomrule
	\end{tabular}
\end{table}

\begin{table}
	\caption{Inventory of attested combinations of initials and rhymes, leaving aside marginal words (loanwords, expressive words{\dots}). Second part.}
	\label{tab:InNucl2}
	\begin{tabular}{lllllllllllllllll}
		\lsptoprule
		& \ipa{i} & \ipa{ɯ} & \ipa{u} & \ipa{v̩} & \ipa{ɻ̩} & \ipa{e} & \ipa{ɤ} & \ipa{o} & \ipa{æ} & \ipa{ɑ} & \ipa{wæ} & \ipa{wɑ} & \ipa{wɤ} & \ipa{jæ} & \ipa{jɤ} & \ipa{jo}\\ \midrule
		\ipa{tɕʰ} & \zh{✓} & \zh{✓} & \ding{55} & \ding{55} & \ding{55} & \ding{55} & \zh{✓} & \zh{✓} & \ding{55} & \ding{55} & \ding{55} & \ding{55} & \ding{55} & \ding{55} & \ding{55} & \ding{55}\\
		\ipa{tɕ} & \zh{✓} & \zh{✓} & \ding{55} & \ding{55} & \ding{55} & \ding{55} & \zh{✓} & \zh{✓} & \ding{55} & \ding{55} & \ding{55} & \ding{55} & \ding{55} & \ding{55} & \ding{55} & \ding{55}\\
		\ipa{dʑ} & \zh{✓} & \zh{✓} & \ding{55} & \ding{55} & \ding{55} & \ding{55} & \zh{✓} & \zh{✓} & \ding{55} & \ding{55} & \ding{55} & \ding{55} & \ding{55} & \ding{55} & \ding{55} & \ding{55}\\
		\ipa{ɲ} & \zh{✓} & \ding{55} & \ding{55} & \ding{55} & \ding{55} & \ding{55} & \ding{55} & \ding{55} & \ding{55} & \ding{55} & \ding{55} & \ding{55} & \ding{55} & \ding{55} & \ding{55} & \ding{55}\\
		\ipa{ɕ} & \zh{✓} & \zh{✓} & \ding{55} & \ding{55} & \ding{55} & \ding{55} & \ding{55} & \ding{55} & \ding{55} & \ding{55} & \ding{55} & \ding{55} & \ding{55} & \ding{55} &  \zh{✓} &  \zh{✓}\\
		\ipa{ʑ} & \zh{✓} & \ding{55} & \ding{55} & \ding{55} & \ding{55} & \ding{55} & \ding{55} & \ding{55} & \ding{55} & \ding{55} & \ding{55} & \ding{55} & \ding{55} & \ding{55} & \ding{55} & \ding{55}\\
		\ipa{ʈʰ} & \zh{✓} & \zh{✓} & \ding{55} & \ding{55} & \ding{55} & \ding{55} & \zh{✓} & \ding{55} & \zh{✓} & \ding{55} & \ding{55} & \ding{55} & \ding{55} & \ding{55} & \ding{55} & \ding{55}\\
		\ipa{ʈ} & \zh{✓} & \zh{✓} & \ding{55} & \zh{✓} & \ding{55} & \ding{55} & \zh{✓} & \ding{55} & \zh{✓} & \ding{55} & \ding{55} & \ding{55} & \ding{55} & \ding{55} & \ding{55} & \ding{55}\\
		\ipa{ɖ} & \zh{✓} & \zh{✓} & \ding{55} & \zh{✓} & \ding{55} & \ding{55} & \zh{✓} & \zh{✓} & \zh{✓} & \ding{55} & \ding{55} & \ding{55} & \ding{55} & \ding{55} & \ding{55} & \ding{55}\\
		\ipa{ʈʂʰ} & \zh{✓} & \zh{✓} & \ding{55} & \zh{✓} & \zh{✓} & \zh{✓} & \zh{✓} & \zh{✓} & \zh{✓} & \ding{55} & \ding{55} & \ding{55} & \ding{55} & \ding{55} & \ding{55} & \ding{55}\\
		\ipa{ʈʂ} & \zh{✓} & \zh{✓} & \ding{55} & \zh{✓} & \zh{✓} & \zh{✓} & \zh{✓} & \zh{✓} & \zh{✓} & \ding{55} & \ding{55} & \ding{55} & \ding{55} & \ding{55} & \ding{55} & \ding{55}\\
		\ipa{ɖʐ} & \zh{✓} & \zh{✓} & \ding{55} & \zh{✓} & \zh{✓} & \zh{✓} & \zh{✓} & \zh{✓} & \zh{✓} & \ding{55} & \ding{55} & \ding{55} & \ding{55} & \ding{55} & \ding{55} & \ding{55}\\
		\ipa{ɳ} & \ding{55} & \zh{✓} & \ding{55} & \zh{✓} & \ding{55} & \ding{55} & \ding{55} & \ding{55} & \zh{✓} & \ding{55} & \ding{55} & \ding{55} & \ding{55} & \ding{55} & \ding{55} & \ding{55}\\
		\ipa{ʂ} & \ding{55} & \zh{✓} & \ding{55} & \zh{✓} & \zh{✓} & \zh{✓} & \zh{✓} & \zh{✓} & \zh{✓} & \ding{55} & \zh{✓} & \ding{55} & \zh{✓} & \ding{55} & \ding{55} & \ding{55}\\
		\ipa{ʐ} & \ding{55} & \zh{✓} & \ding{55} & \zh{✓} & \zh{✓} & \zh{✓} & \zh{✓} & \zh{✓} & \zh{✓} & \ding{55} & \zh{✓} & \ding{55} & \zh{✓} & \ding{55} & \ding{55} & \ding{55}\\
		\ipa{ɻ} & \ding{55} & \ding{55} & \ding{55} & \ding{55} & \ding{55} & \ding{55} & \ding{55} & \ding{55} & \zh{✓} & \ding{55} & \zh{✓} & \ding{55} & \ding{55} & \ding{55} & \ding{55} & \ding{55}\\
		\ipa{kʰ} & \zh{✓} & \zh{✓} & \ding{55} & \zh{✓} & \ding{55} & \ding{55} & \zh{✓} & \zh{✓} & \ding{55} & \ding{55} & \ding{55} & \ding{55} & \zh{✓} & \ding{55} & \ding{55} & \ding{55}\\
		\ipa{k} & \zh{✓} & \zh{✓} & \ding{55} & \zh{✓} & \ding{55} & \ding{55} & \zh{✓} & \zh{✓} & \ding{55} & \ding{55} & \ding{55} & \ding{55} & \zh{✓} & \ding{55} & \ding{55} & \ding{55}\\
		\ipa{g} & \zh{✓} & \zh{✓} & \ding{55} & \zh{✓} & \ding{55} & \ding{55} & \zh{✓} & \zh{✓} & \ding{55} & \ding{55} & \ding{55} & \ding{55} & \zh{✓} & \ding{55} & \ding{55} & \ding{55}\\
		\ipa{ŋ} & \ding{55} & \ding{55} & \ding{55} & \zh{✓} & \ding{55} & \ding{55} & \zh{✓} & \ding{55} & \ding{55} & \ding{55} & \ding{55} & \ding{55} & \zh{✓} & \ding{55} & \ding{55} & \ding{55}\\
		\ipa{qʰ} & \ding{55} & \ding{55} & \ding{55} & \zh{✓} & \ding{55} & \ding{55} & \ding{55} & \zh{✓} & \zh{✓} & \zh{✓} & \zh{✓} & \ding{55} & \zh{✓} & \ding{55} & \ding{55} & \ding{55}\\
		\ipa{q} & \zh{✓} & \ding{55} & \ding{55} & \zh{✓} & \ding{55} & \ding{55} & \ding{55} & \zh{✓} & \zh{✓} & \zh{✓} & \zh{✓} & \ding{55} & \zh{✓} & \ding{55} & \ding{55} & \ding{55}\\
		\ipa{ʁ} & \ding{55} & \ding{55} & \ding{55} & \zh{✓} & \ding{55} & \ding{55} & \ding{55} & \zh{✓} & \zh{✓} & \zh{✓} & \zh{✓} & \ding{55} & \zh{✓} & \ding{55} & \ding{55} & \ding{55}\\
		\ipa{h} & \zh{✓} & \zh{✓} & \zh{✓} & \ding{55} & \ding{55} & \ding{55} & \zh{✓} & \zh{✓} & \zh{✓} & \zh{✓} & \zh{✓} & \ding{55} & \zh{✓} & \ding{55} & \ding{55} & \ding{55}\\
		\ipa{f} & \ding{55} & \ding{55} & \ding{55} & \zh{✓} & \ding{55} & \ding{55} & \ding{55} & \ding{55} & \ding{55} & \ding{55} & \ding{55} & \ding{55} & \ding{55} & \ding{55} & \ding{55} & \ding{55}\\
		\lspbottomrule
	\end{tabular}
\end{table}

%\clearpage
%Page typesetting will need verification on proofs to ensure that the tables+figures are in good succession, without too much empty space, and not too much distance between text and tables


	\section{Syllable nuclei: Vowels and syllabic consonants}
	\label{sec:thesyllablenucleivowelsandsyllabicconsonants}
	
	\subsection{Consonantal nuclei}
	\label{sec:consonantalnuclei}
	
{\largerpage[-1]} % Added on April 22nd, 2025

In Na, certain consonantal sounds function as syllable nuclei. In the \isi{International Phonetic Alphabet}, such
	sounds are termed ‘syllabic’. Syllabic fricatives are an areal characteristic, common in neighbouring \ili{Yi} languages, which
	belong to the {Nasoid} subgroup of Loloish (also known as Ngwi; see \citealt[70]{bradley1979}), as well as in \ili{Lizu} \citep{chirkovaetal2012}, \ili{Ersu} \citep{chirkovaersu2015}, and \ili{Pumi} \citep[52]{daudey2014}. Local
	varieties of {Mandarin} also have a~fricative syllable nucleus, /\ipa{fv̩}/, as in \il{Mandarin!Southwestern}Southwestern Mandarin [\ipa{fv̩}] for ‘lake’ (Standard {Mandarin}
	\textit{hú} \zh{湖}; see \citealt{guiyunnanese2001, pinson2008}). 
    
    In the absence of coda consonants in Na, ‘consonantal
	nuclei’ are also referred to in this volume as ‘consonantal rhymes’. The consonantal
	rhymes of Yongning Na are: [\ipa{v̩}], [\ipa{z̩}], [\ipa{ʐ̩}{\kern2pt}], [\ipa{ɻ̩}{\kern2pt}], and [\ipa{ɻ̩̃}{\kern2pt}].
	

	\subsubsection{The voiced fricative /\ipa{v̩}/}
	\label{sec:thevoicedfricative}
	
	The voiced fricative /\ipa{v̩}/ occurs exclusively as a~rhyme, never as an initial. Its
	friction is weaker than in \ili{Naxi}, and there is some degree of formant movement towards
	a~central articulation, tending towards [\ipa{və}]. Since friction noise is slight, even at the beginning of the
	rhyme, this /\ipa{v̩}/ could be described as approaching an approximant, [\ipa{ʋ}]. 
    
    Martinet distinguishes between fricatives proper, which have a~firm articulation and detectable 
	fricative noise, and spirants, which have a~more relaxed articulation, tending towards a~vowel-like
	aperture (\citealt[24]{martinet1956}; \citeyear{martinet1981}). In these terms, the articulation of /\ipa{v̩}/ in
	Yongning Na is spirant rather than fricative. After bilabial initials, /\ipa{v̩}/ tends 
	towards trilling (though less markedly than in \ili{Naxi}): /\ipa{bv̩}/ is realized close to
	[\ipa{ʙ̩}], /\ipa{pv̩}/ to [\ipa{pʙ̩}], and /\ipa{pʰv̩}/ to [\ipa{pʰʙ̩}].
	
	In the absence of strong friction, the rhyme /\ipa{v̩}/ can be difficult to distinguish from the high
	back vowel /\ipa{o}/, especially after consonants that exert similar {coarticulatory} effects on both of these
	rhymes. In particular, uvular stops induce retraction (backing) of both /\ipa{v̩}/ and /\ipa{o}/,
	making the opposition between syllables such as /\ipa{qv̩}/ and /\ipa{qo}/ difficult to perceive at
	first. Minimal pairs such as /\ipa{mæ˧qv̩˩}/ ‘tail’ vs.\ /\ipa{-mæ˧qo˩}/ ‘below; behind’ constitute handy materials for learners to practise this opposition.
	
	From a~\is{comparative method (historical linguistics)}{diachronic} point of view, the rhyme /\ipa{v̩}/ in Yongning Na (as in \ili{Naxi}) is hypothesized to originate in *\ipa{u} \citep{jacquesetal2011}. The process of fricativization is less advanced in Yongning Na than in Lijiang \ili{Naxi}, suggesting a~continuum among \ili{Naish} languages from [\ipa{u}]-like to [\ipa{v̩}]-like realizations, with approximant [\ipa{ʋ̩}] as an intermediate stage. Such cases illustrate that \isi{International Phonetic Alphabet} symbols do not tell the full story about phonological systems, as Martinet emphasized.
	
	\begin{quotation}
		One wonders whether the habit of constantly operating with graphic notations does not make some linguist[s] deaf to the gradual shifts which any painstaking observation can reveal. If one has been taught, not only that phonological systems are made up of discrete units, but also that these units are basically the same in all languages, ({\dots}) one can hardly avoid concluding that no change can take place except by means of jumps from one unit or allophone to another. Only those who know that linguistic identity does not imply physical sameness, can accept the notion that discreteness does not rule out infinite variety and be thus prepared to perceive the gradualness of phonological shifts. \citep[25]{martinet1988}
	\end{quotation}
	

	\subsubsection{Apicalized vowels}
	\label{sec:apicalizedvowels}
	
{\largerpage[-1]} % Added on April 22nd, 2025

	Apicalized vowels are found “on the tip of many tongues” across \il{Sino-Tibetan}Sino-Tibetan \citep{baron1974}, and Na is no \is{exceptions}exception. The vowel  /\ipa{ɯ}/ has fricative allophones after dental and retroflex fricatives and affricates, as in \ili{Naxi}. For instance,  /\ipa{tsʰɯ˧˥}/ ‘goat’
	is realized  [\ipa{tsʰz̩˧˥}], and  /\ipa{ɖʐɯ˥}/ ‘market, city’ as  [\ipa{ɖʐʐ̍˥}]. In phonetic transcription, it is customary in Asian linguistics to use the symbols coined by Chao Yuen-ren:
	[{\kern0.5pt}\ipa{ɿ}{\kern0.5pt}] and [{\kern0.5pt}\ipa{ʅ}{\kern1.5pt}], respectively. \tabref{tab:apicalized} presents Chao's complete set of symbols for apicalized vowels, along with their closest equivalents in the \isi{International Phonetic Alphabet}; symbol descriptions are taken from \citet[80, 89-90]{pullumetal1986}.
	
	\begin{table}[h]
		\caption{Chao Yuen-ren's symbols for apicalized vowels and their closest equivalents in the \isi{International Phonetic Alphabet}.}
		\begin{tabularx}{\textwidth}{ llll }
			\lsptoprule
			sound & symbol & symbol description & IPA\\\midrule
			plain apical vowel & [{\kern0.5pt}\ipa{ɿ}{\kern0.5pt}] & long-leg turned iota & \ipa{z̍}\\
			retroflex apical vowel & [{\kern0.5pt}\ipa{ʅ}{\kern1.5pt}] & right-tail turned iota & \ipa{ʐ̍}\\
			rounded plain
			apical vowel &  [{\kern0.5pt}\ipa{ʮ}{\kern0.5pt}] & curvy turned h & \ipa{z̹̍}\\
			rounded retroflex apical vowel &  [{\kern0.5pt}\ipa{ʯ}{\kern1.5pt}]  & right-tail curvy turned h & \ipa{ʐ̹̍}\\
			\lspbottomrule
		\end{tabularx}
		\label{tab:apicalized}
	\end{table}
	
%\Hack{\newpage}

	Rounded apical vowels are absent in Na for two structural-diachronic reasons: (i)~there is no rounding opposition among front vowels, and hence no source for a~process of apicalization such as *\ipa{y}~> [\ipa{ʮ}]; and (ii)~among back vowels, the close vowel *\ipa{u} underwent fricativization to /\ipa{v̩}/, as described in \sectref{sec:thevoicedfricative}. 
	
	In addition to distinguishing between plain and retroflex apical vowels, Yongning Na exhibits an opposition between /\ipa{i}/ and another high front
	vowel after the alveolopalatal initials /\ipa{dʑ}/, /\ipa{tɕ}/, and /\ipa{tɕʰ}/. The first set of syllables~-- syllables /\ipa{dʑi}/, /\ipa{tɕi}/, and /\ipa{tɕʰi}/~-- has moderate friction during
	the initial, and the rhyme is not strongly apicalized. 

%{\largerpage} % still useful? xyz
	
	The second set consists of syllables with
	an apicalized rhyme and a~slight\-ly more central vowel; these syllables can be approximated as [\ipa{dʑ̍}], [\ipa{tɕʑ̍}], and [\ipa{tɕʰʑ̍}], respectively. If one were to abstract away from the friction on the rhymes, these syllables could be transcribed phonetically as [\ipa{dʑɪ}], [\ipa{tɕɪ}], and [\ipa{tɕʰɪ}]. The phonemic analysis adopted here treats them as allophones of /\ipa{ɯ}/, yielding phonemic representation as /\ipa{dʑɯ}/, /\ipa{tɕɯ}/, and /\ipa{tɕʰɯ}/. This decision is based
	on structural considerations (complementary distribution) rather than phonetic similarity with the canonical [\ipa{ɯ}] vowel, and its degree of
	validity remains to be evaluated experimentally. One possible approach would be to investigate whether
	priming\footnote{Priming is a~memory effect in which exposure to one stimulus (the \textit{prime}) influences the response to another stimulus.} by canonical realizations of the vowel /\ipa{ɯ}/ (as in /\ipa{ʈʰɯ˩}/ ‘to drink’) affects
	reaction times in perceptual tests. Phonetically, the syllables /\ipa{dʑɯ}/, /\ipa{tɕɯ}/, and /\ipa{tɕʰɯ}/ are not easy to distinguish
	from /\ipa{dzɯ}/, /\ipa{tsɯ}/, and /\ipa{tsʰɯ}/, which are shown in \tabref{tab:alveolopal} with their realizations in Chao notation and in \isi{International Phonetic Alphabet}.
	
	\begin{table}
		\caption{\label{tab:alveolopal}Syllables with initial dental affricate and high, unrounded vowel /\ipa{ɯ}/, and their phonetic realizations.}
%		{\renewcommand{\arraystretch}{1.35}	
			\begin{tabularx}{\textwidth}{ P{30mm} l Q }
				\lsptoprule
				phonemic analysis & Chao notation & \isi{International Phonetic Alphabet}\\\midrule
				\ipa{dzɯ} & [\ipa{dzɿ}] & [\ipa{dz̩}]\\
				\ipa{tsɯ} & [\ipa{tsɿ}] & [\ipa{tsz̩}]\\
				\ipa{tsʰɯ} & [\ipa{tsʰɿ}] & [\ipa{tsʰz̩}]\\
				\lspbottomrule
			\end{tabularx}
		\end{table}
		
	An added complexity is related to Voice Onset Time (VOT) oppositions among alveolopalatal
	initials. Syllables with an unvoiced or aspirated initial, /\ipa{tɕ}/ or /\ipa{tɕʰ}/, have stronger
	affrication than those with initial /\ipa{dʑ}/. This difference has {coarticulatory} effects on the
	following vowel, such that both /\ipa{i}/ and /\ipa{ɯ}/ are somewhat less open after /\ipa{tɕ}/ and
	/\ipa{tɕʰ}/ than after /\ipa{dʑ}/.\footnote{A dedicated recording session  (\textit{ApicalizedVowels}  \pandoi{0004470}) was conducted for words containing one of the following syllables:
		/\ipa{dʑi}/, /\ipa{dʑɯ}/, /\ipa{tɕi}/, /\ipa{tɕɯ}/, /\ipa{tɕʰi}/, and /\ipa{tɕʰɯ}/.}
		

	\subsubsection{Rhotic rhymes}
	\label{sec:rhoticrhymes}
	
	In addition to the labiodental fricative and to apicalized vowels, consonantal rhymes include /\ipa{ɻ̩}{\kern2pt}/ and its nasalized counterpart, /\ipa{ɻ̩̃}{\kern2pt}/. From a~phonetic
	point of view, the rhyme /\ipa{ɻ̩}{\kern2pt}/ does not display the considerable lowering of the third formant
	which is the tell-tale characteristic of rhotic vowels \citep[313]{ladefogedetal1996}, such as \ili{Naxi}
	/\ipa{ə˞}{\kern0.7pt}/. The syllabic diacritic serves to distinguish this rhyme from the consonant
	phoneme /\ipa{ɻ}{\kern2pt}/, described in \sectref{sec:lateralsandandtheretroflexapproximant}. As a~rhyme,
	/\smash{\ipa{ɻ̩}}{\kern2pt}/ can occur either on its own or following a~retroflex fricative or affricate or a~bilabial stop.\smash{\footnotemark}\footnotetext{The opposition between /\smash{\ipa{ɻ̩}}{\kern2pt}/ and /\ipa{v̩}{\kern2pt}/ following bilabial stops was not reported in the first edition of this book: it was only recognized in 2024. The lexical distribution of this opposition is documented in the dictionary \citep{michaud_et_al_na_dict_2024} starting at version 2.0 (December 2024).} \tabref{tab:someexamplesillustratingthephonemiccontrastbetweenandafterretroflexfricativesandaffricates} presents examples of
	syllables containing a~/\ipa{ɻ̩}{\kern2pt}/ rhyme, 
    %preceded by a~retroflex affricate,
alongside syllables with the same initial followed by the rhyme /\ipa{v̩}/.
	

	\begin{table}%[t]
		\caption{Examples illustrating the phonemic contrast between  /\ipa{ɻ̩}{\kern2pt}/ and  /\ipa{v̩}/ after
			retroflex fricatives and affricates and bilabial stops.}
		{\renewcommand{\arraystretch}{1.35}
			\begin{tabularx}{.79\textwidth}{ l Q l }
				\lsptoprule
				& \ipa{v̩} & \ipa{ɻ̩}\\\midrule
				
				
				\ipa{ʈʂʰ} & /\ipa{ʈʂʰv̩˧}/ ‘breakfast’ %,  /\ipa{ʈʂʰv̩˧ɻ̩˥\$}/ ‘ant’,  \par /\ipa{bv̩˧ʈʂʰv̩˧}/ ‘cymbals’ 
                & /\ipa{ʈʂʰɻ̩˧˥}/ ‘lungs’\\ 
				\ipa{ʈʂ} & /\ipa{mv̩˧ʈʂv̩˥}/ ‘wrinkles’ & /\ipa{ʈʂɻ̩˥}/ ‘to cough’\\ 
				\ipa{ɖʐ} & /\ipa{ɖʐv̩˥}/ ‘large vein, artery’ & /\ipa{ɖʐɻ̩˥}/ ‘wet, moist’ \\ 
                %& \multicolumn{2}{c}{\textit{no contrasts observed}%\footnote{Roselle Dobbs (p.c.\ 2016) reports that this contrast does exist in the Lataddi dialect. Using her notations, in which /\ipa{u}/ corresponds to Alawua /\ipa{v̩}/, and /\ipa{v̩}/ to Alawua /\ipa{ɻ̩}/, the contrast is exemplified by /\ipa{ʐu˥˩}/ ‘four’ (homophone: ‘delicious’) vs.\ /\ipa{ʐv̩˥˩}/ ‘horse’.}}\\ 
				%\ipa{ʈʰ  ʈ  ɖ  tʰ  t  d} & \multicolumn{2}{c}{\textit{no contrasts observed}}\\ 
				\ipa{ʂ} & /\ipa{ʂv̩˧ɖv̩˧}/ ‘to think’ & /\ipa{ʂɻ̩˧˥}/ ‘full’\\ 
				\ipa{ʐ} & /\ipa{ʐv̩˩\textsubscript{a}}/ ‘delicious’ & /\ipa{ʐɻ̩˩\textsubscript{a}}/ ‘to knead’\\ 
				\ipa{pʰ} & /\ipa{pʰv̩˩\textsubscript{a}}/ \textsc{clf} for fields & /\ipa{pʰɻ̩˩\textsubscript{a}}/ ‘white’\\ 
				\ipa{p} & /\ipa{pv̩˧}/ ‘to perform (a ritual)’ & /\ipa{pɻ̩˧}/ ‘dry’\\ 
				\ipa{b} & /\ipa{bv̩˥}/ ‘thick, coarse’ & /\ipa{bɻ̩˥}/ ‘insect’\\ 
                %\multicolumn{2}{c}{\textit{no contrasts observed}}\\
				\lspbottomrule
			\end{tabularx}}
			\label{tab:someexamplesillustratingthephonemiccontrastbetweenandafterretroflexfricativesandaffricates}
		\end{table}
		
		
		Under the present analysis, the rhyme /\ipa{ɻ̩}{\kern2pt}/ has a~nasalized counterpart, /\ipa{ɻ̩̃}{\kern2pt}/. Only three items
		have been found so far: ‘bone’, /\ipa{ɻ̩̃{\kern0.9pt}˥}/, which also has the meaning of
		‘stem’; ‘helpless, impoverished, troubled’, /\ipa{ɻ̩̃{\kern0.9pt}˥}/; and /\ipa{ɻ̩̃˧ʈʂwæ˩} ‘sambucus, \textit{Toricellia angulata Oliv.}’.
Alternative notations as a~nasal rhotic vowel,
		/\smash{\ipa{ə̃˞}}{\kern1pt}/ or /\smash{\ipa{œ̃˞}}{\kern1pt}/, could also be considered. However, the choice of /\smash{\ipa{ɻ̩̃}}{\kern2pt}/ reflects both phonetic similarity and a~likely structural opposition between /\smash{\ipa{ɻ̩}}{\kern2pt}/ (oral) and /\smash{\ipa{ɻ̩̃}}{\kern2pt}/ (nasal).
        
        The word for ‘bone’, transcribed here as /\smash{\ipa{ɻ̩̃{\kern0.9pt}˥}}/, already puzzled earlier investigators. \citet[25]{fu1983} transcribes it phonetically as [\ipa{ʔɱɹ}]
		and analyzes it phonemically as /\ipa{ŋv̩ɹ}/. In this transcription system, the two-symbol sequence is not to be understood as indicating a~succession of two sounds; rather, Fu Maoji employs the symbol /\ipa{ɹ}/ to denote rhoticity in the vowel. Phonemic analysis as /\ipa{ŋv̩ɹ}/ is an appealing possibility: nasality would be attributed to the initial onset, and rhoticity to a~rhotic fricative rhyme,
		/\ipa{v̩ɹ}/. However, in the attested syllable /\ipa{ŋv̩}/, \isi{coarticulation} between the initial and the 
		rhyme does not result in full coalescence: the syllable retains two distinct parts~-- an initial nasal consonant	[\ipa{ŋ}], with complete oral closure, and a~rhyme [\ipa{ṽ̩}], which is partially nasalized phonetically but without full oral
		closure. A~hypothetical rhotic counterpart to /\ipa{ŋv̩}/ would likewise be
		expected to begin with a~nasal consonant. Since this is not observed, I do not analyze this syllable as a~rhotic counterpart to /\ipa{ŋv̩}/. Instead, I interpret it as a~monophonemic syllable consisting of a~nasalized rhotic rhyme. 
        
        As for whether it should be analyzed as
		/\ipa{ɻ̩̃}{\kern2pt}/ or /\ipa{ṽ̩}/, both options remain open, since there is no opposition between these in
		onsetless syllables. On the basis of phonetic considerations, I find /\ipa{ɻ̩̃ }/ the more adequate notation from a~synchronic perspective.
		
		
		\subsubsection{Potential for the creation of new syllabic consonants and monophonemic syllables}
		\label{sec:potentialforthecreationofnewsyllabicconsonantsandmonophonemicsyllables}
		
		In syllables with a~simple CV (consonant+vowel) segmental structure, the consonant and vowel are subject to strong coarticulation. Their features tend to be realized over the syllable as
		a~whole. %Numerous examples are found in the \ili{Yi} (Lolo) branch of \il{Sino-Tibetan}Sino-Tibetan languages. While the
%		impetus for \isi{monosyllabicization} can safely be hypothesized to have come from Old \il{Sinitic}Chinese, which
%		influenced~-- directly or indirectly~-- languages of the \il{Sino-Tibetan}Sino-Tibetan, \il{Tai-Kadai}Tai-Kadai, Hmông-Miên and
%		Austroasiatic families, segmental depletion has reached a~more extreme stage in \ili{Yi} than within
%		\ili{Sinitic} itself (as pointed out by \citealt{haudricourt1991}). 
		In \ili{Naish} languages, {coarticulation}
		in CV monosyllables tends to create compact units that become less and less tractable to
		straightforward segmentation into distinct phonemes. Over time, such syllables may become monophonemic.
		
		This process is relevant not only to the emergence of apicalized vowels (described in \sectref{sec:apicalizedvowels}) but also to the development of syllabic nasals. “In various Loloish languages some or all of the nasals occur as
		syllabics. In most such cases the \is{comparative method (historical linguistics)}{diachronic} source is syllables with a~nasal initial and a~high
		vowel; sometimes one dialect has nasal syllabics where others have nasals plus a~high vowel. This
		could be called rhyme-gobbling” (\citealt[150]{bradley1989}; see also
		\citealt[8]{bjorverud1998}). 
        
        Yongning Na exhibits an intermediate stage in this development. The syllable /\ipa{mv̩}/ is
		phonetically realized as [\ipa{m̩}] except in careful (hyperarticulated) speech. On the other hand, /\ipa{nv̩}/
		and /\ipa{ŋv̩}/ retain an oral portion
		following the initial nasal: these two syllables are pronounced as [\ipa{nʋ̩}] and [\ipa{ŋʋ̩}], respectively. 
%		Other syllables that tend towards articulation as one single sound include
%		/\ipa{kʰɯ}/, which is often devoiced when it carries an L tone: [\ipa{kʰɯ̥}]. The entire syllable is
%		realized as a~[\ipa{kʰ}] that adopts the lip and tongue configuration of the vowel /\ipa{ɯ}/. 
		
        In the
		present state of the language, these phenomena remain instances of allophonic variation, but they hold potential for further evolution into more advanced stages of \isi{phonological erosion}.
				

		\subsection{Close vowels}
		\label{sec:closevowels}
		
		\tabref{tab:ie} shows that the close vowels /\ipa{i}/ and /\ipa{e}/ 
		contrast after dental fricatives and affricates,
		e.g.~/\ipa{dzi˩}/ ‘to sit’ vs.\ /\ipa{dze˩}/ ‘to fly’, /\ipa{tsi˩}/ ‘to boil’ vs.\ /\ipa{tse˩}/ ‘to
		lock’, /\ipa{tsʰi˥}/ ‘classifier for animal skins’ vs.\ /\ipa{tsʰe˥}/ ‘salt’, and /\ipa{si˥}/ ‘wood’
		vs.\ /\ipa{se˥}/ ‘to walk’. The syllable /\ipa{zi}/ has only been found in /\ipa{tsʰi˧zi\#˥}/ ‘highland
		barley’ and /\ipa{lɑ˧zi˥}/ ‘painter’; instances of /\ipa{ze}/ are more numerous. 
		
		\begin{table}%[t]
			\caption{Distribution of the close vowels /\ipa{i}/ and /\ipa{e}/.}
			\begin{tabularx}{\textwidth}{  l@{\hspace{12mm}} Q Q }
				\lsptoprule
				initial & \ipa{i}  & \ipa{e}\\\midrule
				Ø & \ipa{ʝi˥} ‘ox’ & --\\
				\ipa{b p pʰ} & \ipa{bi˥} ‘snow’ & --\\
				\ipa{m} & \ipa{mi˧} ‘wound’ & --\\
				\ipa{d t tʰ} & \ipa{di˧˥} ‘to hunt’ & --\\
				\ipa{dz} & \ipa{dzi˩} ‘to sit’ & \ipa{dze˩} ‘to fly’\\
				\ipa{ts} & \ipa{tsi˩} ‘to boil’ & \ipa{tse˩} ‘to lock’\\
				\ipa{tsʰ} & \ipa{tsʰi˥} ‘\textsc{clf}.animal skins’ & \ipa{tsʰe˥} ‘salt’\\
				\ipa{s} & \ipa{si˥} ‘wood’ & \ipa{se˥} ‘to walk’\\
				\ipa{z} & \ipa{tsʰi˧zi\#˥} ‘highland barley’ & \ipa{ze˩mi˩} ‘niece’\\
				\ipa{n} & \ipa{ni˥} ‘amaranth’ & \ipa{-ne} ‘as, like’\\
				\ipa{l} & \ipa{li˧\textsubscript{a}} ‘to look’ & \ipa{le˧-} \textsc{accomplished}\\
				\ipa{ɬ} & \ipa{ɬi˥} ‘to rest’ & --\\
				\ipa{dʑ tɕ tɕʰ} & \ipa{tɕi˥} ‘to shake’ & --\\
				\ipa{ɕ ʑ} & \ipa{ɕi˧} ‘rice’ & --\\
				\ipa{ɲ} & \ipa{ɲi˥} ‘to listen’ & --\\
				\ipa{ɖ ʈ ʈʰ} & \ipa{ʈi˩\textsubscript{a}} ‘to get up’ & --\\
				\ipa{ɖʐ ʈʂ ʈʂʰ} & -- &  \ipa{ʈʂe˥} ‘earth’\\
				\ipa{ʐ ʂ} & -- &  \ipa{ʐe˥} ‘arrow’\\
%				\ipa{ɭ} & -- & --\\
				\ipa{ʁ ɻ} & -- & --\\
				\ipa{g k kʰ} & \ipa{gi˥} ‘to owe’ & --\\
				\ipa{q qʰ} & \ipa{qi˧qi˧} ‘originally’ & --\\
				\ipa{h} & \ipa{hi˩˥} ‘rain’ & --\\
				\lspbottomrule
			\end{tabularx}
			\label{tab:ie}
		\end{table}
		
		After the dental nasal /\ipa{n}/ and the dental lateral /\ipa{l}/, the close vowels /\ipa{i}/ and
		/\ipa{e}/ are marginally contrastive. The syllable /\ipa{ne}/ is only found in a~grammatical
		morpheme, /\ipa{-ne}/ ‘as, like’, which appears in the interrogative /\ipa{qʰɑ˩ne˩}/ ‘how’, the
		manner demonstratives /\ipa{ʈʂʰɯ˧ne˧-ʝi˥}/ (proximal) and /\ipa{tʰv̩˧ne˧-ʝi˥}/ (distal), and
		constructions such as /\ipa{tɕʰɤ˧ɲi˧-ne˧-ʝi˥}/ ‘every day; repeatedly, all the time’. Recognition of the morpheme /\ipa{ne}/ was delayed by the fact that the highly frequent expression
		/\ipa{ʈʂʰɯ˧ne˧-ʝi˥}/ ‘thus, in this way’ (occurring over 500 times in 25 texts) is pronounced very close to [\ipa{ʈʂʰɯ˧ni˧˥}], and hence
		was initially transcribed as \ipa{$\ddagger${\kern2pt}ʈʂʰɯ˧ni˧˥}. (This adverb is analyzed further in the section devoted to on-glides, \sectref{sec:smoothphoneticonsets}.) Another marginal case where it appears reasonable to posit an /\ipa{e}/ vowel distinct from /\ipa{i}/
		is in combination with initial /\ipa{l}/. The syllable /\ipa{li}/ is common, appearing in about
		thirty words. A~[\ipa{le}] syllable appears in the \textsc{accomplished} \is{prefixes}prefix /\ipa{le˧-}/, and in
		\is{derivation!morphological}derived items, such as [\ipa{njɤ˧le˧gv̩\#˥}] ‘daytime’, which is perceived by consultant F4 as meaning literally ‘the
		day is flowing/going by’, as shown in (\ref{ex:daytime}).\footnote{Assuming that the etymological glossing in (\ref{ex:daytime}) is correct, the
			change of the first syllable from /{\kern1pt}\ipa{ɲi}/ to /\ipa{njɤ}/ remains to be accounted for.}
		
			\begin{exe}
				\ex
				\label{ex:daytime}
				\ipaex{njɤ˧le˧gv̩\#˥}\\
				\gll ɲi˥ le˧- gv̩˧\textsubscript{c}\\
				day		\textsc{accomp}		to\_flow/to\_go\_by\\
				\glt ‘daytime’, interpreted by consultant F4 as having the literal meaning ‘the
				day is flowing/going by’
			\end{exe}

		In F4's idiolect, the second syllable of ‘daytime’ is perceived as the \textsc{accomplished} \is{prefixes}prefix, and therefore does not constitute evidence for /\ipa{le}/ as
		an attested syllable in nouns. Moreover, the form /\ipa{njɤ˧le˧gv̩\#˥}/ is only one of many avatars of the word for ‘daytime’ among Na dialects, some of which have /\ipa{ɬi}/ and not /\ipa{le}/ as a~middle syllable (\ipa{ɲi33-ɬi31 ku33} in \citealt[297]{lidz2010}). It appeared interesting to dwell on this example nonetheless, as it provides an insight into processes whereby marginal combinations of initials and rhymes can enter the lexicon.

\largerpage
		To sum up, the opposition of /\ipa{e}/ and /\ipa{i}/ is restricted to syllables that have a~dental initial (most examples have a~fricative or affricate). This is one of the many cases
		where a~phonemic opposition is found in highly restricted contexts; in \is{Praguean phonology}Praguean terms, these
		constitute extreme cases of \isi{neutralization}. This issue is taken up in the discussion of the
		inventory of syllables, in \sectref{sec:commentsabouttheinventoryofsyllables}.
		
		Recognition of the phonemic opposition between /\ipa{i}/ and /\ipa{e}/ after dental fricatives and affricates was delayed because I initially failed to notice the existence of /\ipa{i}/ as distinct from /\ipa{ɯ}/ in this context, where both /\ipa{i}/ and /\ipa{ɯ}/ are apicalized, and their phonetic difference is a~very fine
		one. Apicalization is stronger for /\ipa{ɯ}/ than for /\ipa{i}/. Another difference is that the lips
		are stretched for /\ipa{si}/, /\ipa{dzi}/, /\ipa{tsi}/, and /\ipa{tsʰi}/. In the initial
		transcriptions, the contrast between the two apicalized vowels was
		overlooked, leading to erroneous transcriptions of these syllables as $\ddagger${\kern2pt}\ipa{sɯ} (‘wood’),
		$\ddagger${\kern2pt}\ipa{zɯ} (‘barley’), $\ddagger${\kern2pt}\ipa{dzɯ} (‘to sit’), $\ddagger${\kern2pt}\ipa{tsɯ} (‘to boil’), and
		$\ddagger${\kern2pt}\ipa{tsʰɯ} (classifier for animal skins). 
		
		In the many contexts where the opposition
		between /\ipa{i}/ and /\ipa{e}/ is neutralized, the transcription follows their phonetic
		realization, which is closer to [\ipa{e}] after retroflex fricatives and affricates, and to
		[\ipa{i}] in all other contexts (after bilabials, velars, uvulars, retroflex stops and fricatives, laterals,
		nasals, alveolopalatals, and glottals).
		
		As for close back vowels, [\ipa{o}] and [\ipa{u}] are only contrastive after initial /\ipa{h}/. Examples of /\ipa{ho}/ include /\ipa{tɑ˧ho˧}/ ‘together’, /\ipa{ho\#˥}/
		‘porridge, gruel’,	/\ipa{ho\#˥}/ ‘partridge’, /\ipa{qo˩ho˧˥}/ ‘wicker box’, /\ipa{ho˧˥}/ ‘to sip’, /\ipa{ho˩\textsubscript{a}}/ ‘correct’, and \mbox{/\ipa{-ho˩}/} \textsc{desiderative}. Examples of /\ipa{hu}/ are fewer;
		they include /\ipa{hu˧mi˥\$}/ ‘stomach’, /\ipa{hu˥}/ ‘to wait’, and /\ipa{hu˧˥}/ ‘to miss, to long
		for’. Phonetically, there is
		stronger friction in the initial for /\ipa{ho}/ than for /\ipa{hu}/, which could be approximated phonetically as
		[\ipa{hu}] vs.\ [\ipa{χo}]. This suggests an alternative phonemic analysis, dispensing with the opposition between /\ipa{u}/ and /\ipa{o}/,  positing instead an opposition between /\ipa{h}/ and /\ipa{χ}/, and rewriting /\ipa{hu}/ and /\ipa{ho}/ as /\ipa{ho}/ and /\ipa{χo}/. The phonetic difference in vowel quality appears more salient than the difference in the initial, however, hence the choice to interpret the vowel difference as phonemic. 
		
		Another way to reinterpret the syllable [\ipa{hu}] and eliminate the /\ipa{u}/ phoneme altogether would be to grant phonemic status to initial [\ipa{f}], analyzing the syllables [\ipa{fv̩}], [\ipa{hu}] and [\ipa{ho}] as /\ipa{fv̩}/, /\ipa{hv̩}/ and /\ipa{ho}/, respectively (Roselle Dobbs, p.c.\ 2016): see \sectref{sec:theglottalfricativeandthesound}. 
		
		After all initials other than /\ipa{h}/, there is no opposition between close and close-mid rounded back vowels,
		[\ipa{o}] and [\ipa{u}]. Phonetic realizations are close to [\ipa{o}] after dentals, velars and
		uvulars, and more often close to [\ipa{u}] after the other
		consonants. In all the contexts where the opposition is neutralized, the notation chosen is /\ipa{o}/: this appeared less cumbersome than notation as an \isi{archiphoneme} /\ipa{O}/ (or /\ipa{U}/). 
		
		The two sounds [\ipa{o}] and [\ipa{u}] may have stronger phonemic status in the speech of younger
		speakers, whose increasing proficiency in \il{Mandarin!Standard}Standard {Mandarin} makes them familiar with a~phonemic
		/\ipa{u}/ (contrasting with /\ipa{oʷ}/ and /\ipa{ʷo}/; the \textit{Pinyin} transcription of these three
		vowels is: \textit{u}, \textit{ou}, \textit{uo}). \il{Mandarin!Southwestern}Southwestern {Mandarin} does not exert direct
		pressure in this direction, however, since /\ipa{u}/ is fricativized in this dialect of {Mandarin}. For
		instance, the word /\ipa{tsʰu}/ ‘vinegar’ is pronounced [\ipa{tsʰv̩}]. This word, which is in common
		use in Yongning, is accordingly pronounced as [\ipa{tsʰv̩˩˥}] in the speech of the older speaker F4.
		
		
		\subsection{A neutral vowel:  /\ipa{ə}/}
		\label{sec:aneutralvowel}
		
		A~clarification is needed concerning the use of the phonetic symbol /\ipa{ə}/. In their \citeyear{heetal1985} book on \ili{Naxi}, which includes a~word list for Yongning Na, He Jiren and Jiang Zhuyi use /\ipa{ə}/ to represent two different vowels: a~back unrounded vowel, /\ipa{ɤ}/, realized as [\ipa{ɣɤ}] in onsetless syllables, and a~neutral vowel, /\ipa{ə}/, which always forms an independent syllable, harmonizes with the vowel of the following syllable, and is realized with an initial glottal stop \citep[130]{michaud2013b}. This confusion likely stems from the use of /\ipa{ə}/ as the official phonetic equivalent of the letter \textit{e} in the \textit{Pinyin} romanization of Standard {Mandarin}, where it was employed to represent a~back unrounded vowel. In view of its phonetic realizations, which are “\ipa{ɤ}-like or \ipa{ʌ}-like” \citep[42]{association1949}, the ‘ram’s horn’ symbol, /\ipa{ɤ}/, or the turned v, /\ipa{ʌ}/, would have been more felicitous choices than the neutral vowel /\ipa{ə}/. Linguists trained in mainland China in the first decades of the People's Republic of China tended to follow this system, which was reproduced in dictionaries and textbooks until the turn of the twenty-first century. More recent descriptions, such as \citet[110]{waisumetal2003}, have corrected this by distinguishing /\ipa{ɤ}/ from /\ipa{ə}/. 
        
        This misleading notation was adopted within the official phonetic transcription for \ili{Naxi} and remains influential.\footnote{In his manuscript lexicographic notes, He Jiren consistently used the turned v, /\ipa{ʌ}/, and not the neutral vowel, /\ipa{ə}/. However, the symbol /\ipa{ə}/ was reintroduced when the data was edited for publication as a~dictionary \citep{heetal2011}.} 
		
		In the present system, /\ipa{ə}/ is used exclusively for the {interrogative} particle and the initial syllable of certain lexical words,
		where it can plausibly be analyzed as a~\is{prefixes}prefix. In particular, it appears in kinship terms referring to elders, such as
		/\ipa{ə˧mi˧}/ ‘mother’, realized as [\ipa{e˧mi˧}]; /\ipa{ə˧mɑ˧}/ ‘mother (\textsc{vocative})’, realized as 
		[\ipa{ɑ˧mɑ˧}]; and /\ipa{ə˧v̩˧˥}/ ‘uncle’, realized as [\ipa{ɤ˧v̩˧˥}]. The prefix’s realization varies according to the quality of the vowel in the following syllable (the root morpheme): it tends towards [\ipa{æ}]
		before /\ipa{æ}/ and apicalized allophones of /\ipa{ɯ}/, [\ipa{ɛ}] before /\ipa{i}/, /\ipa{ĩ}/,
		and /\ipa{e}/, [\ipa{ɑ}] before /\ipa{ɑ}/ and /\ipa{wɤ}/, and [\ipa{ɤ}] before /\ipa{ɤ}/,
		/\ipa{o}/, /\ipa{ɯ}/, /\ipa{jɤ}/, /\ipa{jo}/, and /\ipa{ɻ̩}{\kern2pt}/. Before /\ipa{v̩}/, its realization depends 
        %the neutral vowel /\ipa{ə}/ harmonizes differently depending 
        on whether an intervening consonant is present: it
		is close to [\ipa{ɤ}] when immediately followed by /\ipa{v̩}/, as in /\ipa{ə˧v̩˧˥}/ ‘uncle’
		(approximately [\ipa{ɤ˧v̩˧˥}]), but shifts towards [\ipa{æ}] when /\ipa{v̩}/ is preceded
		by a~consonant, as in /\ipa{ə˧pʰv̩˧}/ ‘mother’s mother's brother’ and
		/\ipa{ə˧mv̩˩}/ ‘elder sibling’ (phonetic approximation: [\ipa{æ˧pʰv̩˧}] and [\ipa{æ˧mv̩˩}]).\footnote{This phonemic
			analysis of the kinship {prefix} was suggested by Roselle Dobbs.}
		
		Note, however, that phonetic (incomplete) \isi{vowel harmony} is not restricted to the vowel transcribed
		as /\ipa{ə}/. This topic is taken up in \sectref{sec:anoteonvowelharmony}.
		
		No phonetic difference could be detected between the [\ipa{ɑ}] realization of /\ipa{ə}/ before
		/\ipa{ɑ}/ and the realization of the phoneme /\ipa{ɑ}/ itself. For instance, the
		clan name [\ipa{ɑ˧lɑ˧}] is phonemicized as /\ipa{ə˧lɑ˧}/, but an alternative interpretation as /\ipa{ɑ˧lɑ˧}/
		cannot be ruled out. 
		
		\subsection{Nasal rhymes}
		\label{sec:nasalrhymes}
		
		\subsubsection{Nasal rhymes after the glottal /\ipa{h}/}
		\label{sec:nasalrhymesaftertheglottal}
		
		Yongning Na has a~relatively large inventory of nasal rhymes: it comprises /\ipa{ĩ}/, /\ipa{ṽ̩}/,
		/\ipa{õ}/, /\ipa{w̃ɤ}/, /\ipa{æ̃}/ and /\ipa{ɑ̃}/, as well as the syllables /\ipa{ɻ̩̃}{\kern2pt}/ and /\ipa{w̃æ}/. The
		first six occur after /\ipa{h}/, where they contrast neatly with their non-nasal counterparts:  /\ipa{hi}/--/\ipa{hĩ}/,  /\ipa{hv̩}/--/\ipa{hṽ̩}/,  /\ipa{ho}/--/\ipa{hõ}/,  /\ipa{hwɤ}/--/\ipa{hw̃ɤ}/,  /\ipa{hæ}/--/\ipa{hæ̃}/, and  /\ipa{hɑ}/--/\ipa{hɑ̃}/. Examples are provided in \tabref{tab:examplesofinitialsyllablesthatarepartofacorrelationofnasality}.
		
		\begin{table}%[t]
			\caption{Examples of /\ipa{h}/-initial syllables that are part of a~correlation of nasality.}
			{\renewcommand{\arraystretch}{1.35}
				\begin{tabularx}{\textwidth}{ Q P{80mm} }
					\lsptoprule
					oral rhyme & nasal rhyme\\ \midrule
					/\ipa{hi˥}/ ‘tooth’  & /\ipa{hĩ{\kern0.7pt}˥}/ ‘man’; /\ipa{hĩ˧˥}/ ‘to stand’\\ 
					/\ipa{hv̩˧}/ ([\ipa{fv̩˧}]) ‘to like’ & /\ipa{hṽ̩˩}/ ‘red’; /\ipa{hṽ̩˥}/ ‘hair’; /\ipa{nv̩˧hṽ̩˩}/ ‘kidney bean’; /\ipa{dʑi˧hṽ̩˥\$}/ ‘clothes’; /\ipa{hṽ̩˧{$\sim$}hṽ̩˧}/ ‘to stir-fry’\\ 
					/\ipa{ho˧˥}/ ‘to sip’; /\ipa{ho˥}/ ‘to wait’ & /\ipa{hõ˧˥}/ ‘eight’; /\ipa{hõ˧}/ ‘to go (\textsc{imperative})’\\ 
					/\ipa{hwɤ˩}/ ‘to pass over, to hand over’ & /\ipa{hw̃ɤ˩}/ ‘late’\\ 
					/\ipa{hæ˧}/ ‘Chinese’; /\ipa{hæ˧˥}/ ‘lime’; /\ipa{hæ˧pɤ˧}/ ‘plait’  & /\ipa{hæ̃˧}/ ‘wind’; /\ipa{hæ̃˩}/ ‘gold’\\ 
					/\ipa{hɑ˥}/ ‘food’ & /\ipa{hɑ̃˧˥}/ ‘night’\\
					\lspbottomrule
				\end{tabularx}}
				\label{tab:examplesofinitialsyllablesthatarepartofacorrelationofnasality}
			\end{table}
			
			Diachronically, these syllables illustrate a~process whereby nasality is transferred from
			a~syllable-initial consonant cluster to the following vowel. This phenomenon is attested in several
			languages of Asia. In \ili{Kam-Sui} (\il{Tai-Kadai}Tai-Kadai family), Sandong \ili{Sui} lost the stop component of the original
			cluster: the stop+nasal clusters *\ipa{km-}, *\ipa{kn-}, *\ipa{tn-}, and *\ipa{kɲ-} merged with the
			preglottalized initials *\ipa{ˀm-}, *\ipa{ˀn-}, and *\ipa{ˀɲ-}, which remain preserved in \ili{Sui},
			e.g.~/\ipa{ˀma¹}/ ‘vegetables’ and /\ipa{ˀma³}/ ‘flexible’, both corresponding to a~proto-Kam-Sui *\ipa{ˀm}
			initial \citep[251--252]{ferlus1996c}. \ili{Lakkia}, by contrast, retained the initial stop, while the nasal underwent
			lenition, nasalizing the following vowel in the process, e.g.~/\ipa{kũːi}/ ‘bear’, from an earlier stop+nasal cluster *\ipa{km-}. 
            
            Northern \ili{Sui} dialects (Pandong \zh{潘洞}
			and Yang’an \zh{阳安}) illustrate a~possibility for the later evolution of glottal+nasal onsets. Here, distinctive nasality is transferred to the following vowel, and all that remains in onset position is a~glottal sound, yielding
			[\ipa{ʔṼ}] or [\ipa{h̰Ṽ}], with the entire syllable becoming nasalized, including the initial glottal sound
			\citep[176]{haudricourt1967}. This pattern is exactly parallel to what is observed in Yongning Na and other \ili{Naish}
			languages, as brought out in
			\tabref{tab:comparativevocabularyforfivewordsinrgyalrongandinnaxinaandlaze}. %(This table is reproduced from a~cross-linguistic (\is{panchronic phonology}{panchronic}) study of historical transfers of nasality between consonantal onsets and vowels: \citet{michaudetal2012b}.) 
            
            \ili{Japhug}, a~{conservative} \il{Sino-Tibetan}Sino-Tibetan language that preserves a~broad range of initial clusters, provides a valuable point of comparison. Once correspondences among \ili{Naish} languages have been established, comparison with such {conservative} languages suggests hypotheses for fleshing out reconstructions \citep[470-471]{jacquesetal2011}.
			
			\begin{table}%[t]
				\caption{Comparative data pointing to the development of nasality in {Naish} from earlier */\ipa{rN-}/ onsets.}
				\begin{tabularx}{\textwidth}{ l@{\hspace{10mm}} l@{\hspace{10mm}} Q Q l@{\hspace{10mm}} }
					\lsptoprule
					& \ili{Japhug} & Fengke \ili{Naxi} & Yongning Na & \ili{Laze}\\ \midrule
					red & \ipa{ɣɯrni} & \ipa{hỹ˩} & \ipa{hṽ̩˩} &  --\\
					to stand & \ipa{rma} & \ipa{hỹ˩˧} & \ipa{hĩ˧˥} & \ipa{hĩẽ˥}\\
					person & \ipa{tɯ-rme} & \ipa{hĩ˧} & \ipa{hĩ{\kern0.7pt}˥} & \ipa{hĩ˧}\\
					body hair & \ipa{tɤ-rme} & \ipa{hṽ̩˥} & \ipa{hṽ̩˥} & \ipa{hṽ̩˩}\\
					to stir-fry & \ipa{rŋu} & -- & \ipa{hṽ̩˧{$\sim$}hṽ̩˧} & --\\
					two & \ipa{ʁnɯs} & \ipa{ɲi˩˧} & \ipa{ɲi˥} & \ipa{ɲi˧}\\
					\lspbottomrule
				\end{tabularx}
				\label{tab:comparativevocabularyforfivewordsinrgyalrongandinnaxinaandlaze}
			\end{table}
			
			%% %Table 4.
			%% \begin{table}%[t]
			%% \caption{Comparative data pointing to the development of nasality in \ili{Naish} from earlier  /\ipa{*rN-}/ onsets.}
			%%   \begin{tabularx}{\textwidth}{ l l l l l Q Q }
			%% \lsptoprule
			%% 	 & red & to stand & person & body hair & to stir-fry & two\\ \midrule
			%% 	\ili{Japhug} & \ipa{ɣɯrni} & \ipa{rma} & \ipa{tɯ-rme} & \ipa{tɤ-rme} & \ipa{rŋu} & \ipa{ʁnɯs}\\ 
			%% 	Fengke \ili{Naxi} & \ipa{hỹ˩} & \ipa{hỹ˩˧} & \ipa{hĩ˧} & \ipa{hṽ̩˥} & -- & \ipa{ɲi˩˧}\\ 
			%% 	Yongning Na & \ipa{hṽ̩˩} & \ipa{hĩ˧˥} & \ipa{hĩ˥} & \ipa{hṽ̩˥} & \ipa{hṽ̩˧{$\sim$}hṽ̩˧} & \ipa{ɲi˥}\\ 
			%% 	\ili{Laze} & -- & \ipa{hĩẽ˥} & \ipa{hĩ˧} & \ipa{hṽ̩˩} & -- & \ipa{ɲi˧}\\
			%%    \lspbottomrule
			%% \end{tabularx}
			%% \label{tab:comparativevocabularyforfivewordsinrgyalrongandinnaxinaandlaze}
			%% \end{table}
			
			\tabref{tab:comparativevocabularyforfivewordsinrgyalrongandinnaxinaandlaze} brings out
			a~\is{comparative method (historical linguistics)}correspondence between the /\ipa{h̰Ṽ}/ syllables of Yongning Na and etyma with initial
			/\ipa{rm-}/ or /\ipa{rn-}/ in \ili{Japhug}. This leads to the hypothesis that the /\ipa{h̰Ṽ}/ syllables found in \ili{Naish} languages originate in earlier *CNV syllables. 
            
            The hypothesis that nasal vowels
			in some \il{Sino-Tibetan}Sino-Tibetan languages could result from the influence of syllable-initial
			nasals was already expressed by \citet{huang1991a}. On the other hand, no hypotheses had been
			advanced regarding the precise phonemic sequences involved in the change until \citet{michaudetal2012b}.
			
			The final example in \tabref{tab:comparativevocabularyforfivewordsinrgyalrongandinnaxinaandlaze},
			‘two’, illustrates the retention of nasals in \ili{Naxi}, Na, and \ili{Laze} that originate in onsets other
			than */\ipa{rN-}/. This suggests that the *CN- onsets that gave rise to vowel
			nasalization may have all gone through an intermediate */\ipa{sN-}/ stage. (For general phonetic reflections on this topic, see
			\citealt[233]{ohalaetal1993}, who note that “children learning \ili{English} sometimes pronounce target \textit{sm} and
			\textit{sn} clusters as voiceless nasals”.)
			
			Before nasal rhymes, /\ipa{h}/ is itself nasalized; the lowered velum prevents the buildup of
			intra-oral pressure required for strong friction noise. Since the entire syllable is nasalized,
			an alternative phonemic analysis would be to posit a~nasalized glottal fricative,
			/\ipa{h̰}/, contrasting with plain /\ipa{h}/. A~review on “possible and impossible segments” \citep{walkeretal1999} classifies /\ipa{h̰}/ among the set of “possible segments”, as it is firmly attested as a~phoneme contrasting with /\ipa{h}/ in at least two languages: \ili{Kwangali} (\ili{Bantu}; \citealt[132-133]{ladefogedetal1996}) and \ili{Seimat} (Austronesian; \citealt{blust1998}). In the case of Yongning Na, the decision to interpret nasality as a~feature of the vowel rather than the consonant is based on the observation that nasal vowels also occur in a~small set of
			syllables that do not have an initial glottal fricative. These cases, examined in the following section, further illustrate the role of nasality as a distinctive feature in the vowel system of the language.
			
			
			\subsubsection{Onsetless nasal syllables:  /\ipa{õ}/,  /\ipa{æ̃}/ and  /\ipa{ɻ̩̃}/}
			\label{sec:onsetlessnasalsyllables}
			
			Among onsetless nasal syllables, /\ipa{õ}/ contrasts with /\ipa{o}/, /\ipa{æ̃}/ with /\ipa{æ}/, and
			/\ipa{ɻ̩̃}{\kern2pt}/ with /\ipa{ɻ̩}{\kern2pt}/. Examples of /\ipa{o}/ (realized with a~glide onset, [\ipa{wo}]) include
			/\ipa{wo˥}/ ‘hard’; examples of /\ipa{õ}/ include /\ipa{õ˧˥}/ ‘(one)self’, /\ipa{õ˧ʈʂwɤ˧}/
			‘mosquito’, and /\ipa{õ˩dv̩˧˥}/ ‘wolf’. Synchronically, initial glottalization contributes to a~clear
			phonetic contrast between /\ipa{o}/, realized [\ipa{wo}], and /\ipa{õ}/, realized
			[\ipa{ʔõ}]. 
            
            Examples of /\ipa{ɻ̩̃}{\kern2pt}/ and /\ipa{ɻ̩}{\kern2pt}/ were presented earlier
			(\sectref{sec:rhoticrhymes}). The syllable /\ipa{æ̃}/ also has a~glottalized onset, [\ipa{ʔæ̃}]. A
			non-nasalized counterpart, [\ipa{ʔæ}], is found in the system and is analyzed as /\ipa{æ}/,
			i.e.\ recognizing the phonemic status of both /\ipa{æ}/ and /\ipa{æ̃}/. 
            %The syllable /\ipa{æ̃}/ occurs in a relatively large number of words. Among disyllabic words, 
            Examples include /\ipa{æ̃˩zɯ˩}/ ‘agate’ vs.\ /\ipa{æ˩gv̩˩}/ ‘ard’. 
            %\footnote{The ard, also known as scratch plough, is the type of ploughing implement used in Yongning. Unlike the plough, the ard has a~symmetrical share that traces a shallow furrow but does not invert the soil \citep{haudricourtetal1955}.} 
            The words /\ipa{æ̃˩-mi˧}/ ‘hen’
			and /\ipa{æ˩mi˧-ʁwɤ\#˥}/ ‘Aemiwua’ (‘the village of Aemi’) form a~quasi-minimal pair.
			
			From a~diachronic perspective, the glottal onset of [\ipa{ʔõ}] and [\ipa{ʔæ̃}] may result from the
			same phenomenon of empty-onset hardening that led to the presence of an initial [\ipa{ʁ}]
			in words such as /\ipa{ʁwɤ˥}/ ‘village’, corresponding to \ili{Naxi} /\ipa{wɤ˧}/. This phenomenon is discussed in greater detail in \sectref{sec:apresentationofonglideswithahypothesisaboutadiachroniconsetofhardeningofinitialglides}. 
			
			
			\subsubsection{Phonemic analysis of the onsetless nasal syllable [\ipa{w̃æ}]}
			\label{sec:phonemicanalysisoftheonsetlessnasalsyllable}
			
			The syllable realized as [\ipa{ʔw̃æ}] occurs in a~single lexical item: ‘to swell, to
			inflate (e.g.~the belly is swollen)’, [\ipa{ʔw̃æ˧}]. It does not have a~non-nasal counterpart. In
			the absence of an opposition with an oral syllable [\ipa{wæ}], nasality and initial glottalization
			might be considered the product of implementation rules rather than as distinctive features. From this perspective, the underlying form could be analyzed as simply /\ipa{wæ}/. However, additional evidence suggests that this analysis is insufficient. Consultant F4 pronounces the {Mandarin}
			syllables \textit{wa}, \textit{wan}, and \textit{wang} (as in \textit{wáng} \zh{王}, a~common surname) as [\ipa{ʁwæ}], rather than as [\ipa{ʔw̃æ}]. If the underlying form of ‘to swell’ were merely /\ipa{wæ}/, one would expect these {Mandarin}
			forms to pattern with /\ipa{wæ}/ (phonetically [\ipa{ʔw̃æ}]) rather than with /\ipa{ʁwæ}/. This suggests that nasality, which plays a~distinctive role in the vowel system of Yongning Na, should be granted phonemic status, whereas initial glottalization, which is not contrastive, should not. In summary, the syllable /\ipa{w̃æ}/, realized as [\ipa{ʔw̃æ}], contrasts
			with /\ipa{wæ}/, realized as [\ipa{ʁwæ}] (see
			\sectref{sec:theinitialvoiceduvularfricativeasaphonemicizedemptyonsetfiller}).
			
			
			\subsubsection{The syllable /\ipa{ĩ}/}
			\label{sec:thesyllable}
			
			The syllable /\ipa{ĩ}/ occurs in a~single lexical item: the \is{interjections}interjection ‘Yes!’, ‘Okay!’, used in response to an instruction from a~person in a~position of authority, such as one’s elder or another
			person holding authority. The phonemic form is /\ipa{ĩ˧}/, with a~phonetic realization as [\ipa{ʔĩ˧}]. This item is described as
carrying a~Mid tone, but could also be analyzed as a~toneless \is{interjections}interjection generally realized on
			a~level pitch.
			
			
			\subsubsection{The nasal rhyme /\ipa{õ}/ as a~variant in two Tibetan loanwords}
			\label{sec:thenasalrhymeappearsasavariantontwotibetanloanwords}
			
			In two \ili{Tibetan} \isi{loanwords}, the nasal rhyme /\ipa{õ}/ appears as a~variant of its oral counterpart	/\ipa{o}/. The first example is the term for ‘head of
			a~caravan’: /\ipa{tsʰo˧pæ\#˥}/
            %{\kern2pt}
            {\linebreak}\ipa{≈}{\kern2pt}/\ipa{tsʰõ˧{\allowbreak}pæ\#˥}/, corresponding to Written \ili{Tibetan} \textit{tshong} \textit{pa} ‘merchant’. The second case is a~formula of blessing:
			/\ipa{lɑ˧mɑ˧-ko˧ʈʂʰo˧}/{\kern2pt}\ipa{≈}{\kern2pt}/\ipa{lɑ˧mɑ˧-ko˧ʈʂʰõ˧}/. Nathan Hill (p.c.\ 2016) proposes that the Na word /\ipa{ko˧ʈʂʰo˧}/{\kern2pt}\ipa{≈}{\kern2pt}/\ipa{ko˧ʈʂʰõ˧}/ is a~shortened form of \textit{dkon mchog gsum}, \ili{Sanskrit} \textit{triratna}: ‘the Three Jewels: \textit{Buddha}, \textit{dharma} (the teaching), and \textit{saṅgha} (the monastic order or community)’. It is not obvious why the word /\ipa{lɑ˧mɑ˧}/, clearly \is{loanwords}borrowed from \textit{bla ma} ‘spiritual leader, great teacher (at the monastery)’, should be paired with the Three Jewels in this blessing formula, yielding \textit{bla ma dkon mchog gsum}, which could be interpreted as ‘the Lama and the Three Jewels’ or ‘the Lama's Three Jewels’. Nathan Hill suggests that in this context, ‘lama’ may stand in for the Three Roots \textit{(tsa sum)} of the {Tibetan} Buddhist tradition. These Three Roots~-- \textit{lama}, \textit{yidam}, and \textit{khandroma}~-- are more esoteric and recondite than the Three Jewels, making them less familiar to laypeople. 
            %and hence difficult to remember for people who have not received training at a~monastery. 
            Since the first of the Three Roots, the \textit{lama}, is embodied by a~living person (typically a~high-ranking monk), it is more readily understood. This could explain why ‘lama’ appears in this (somewhat garbled) blessing formula.
			
            On the
			other hand, the word for ‘monastery’, /\ipa{go˧bɤ˩}/ (cf.\ Written \ili{Tibetan} \textit{dgon pa}), does not have a~nasalized \is{variants}variant \ipa{$\ddagger${\kern2pt}gõ˧bɤ˩}. These examples suggest that nasalization in \ili{Tibetan} borrowings does not operate in a uniform manner. A~systematic study of \ili{Tibetan} loanwords in Na would be necessary to identify different layers of borrowing and degrees of phonological integration.
			

			\subsection{The open vowels  /\ipa{ɑ}/ and  /\ipa{æ}/, and the vowel /\ipa{ɤ}/}
			\label{sec:theopenvowelsandandthevowel}
			
			Na, \ili{Naxi}, and \ili{Laze} all have a~phonemic opposition between two open vowels. In \ili{Naxi}, different authors have transcribed this contrast in varying ways: some as /\ipa{a}/ vs.\ /\ipa{æ}/ (e.g.\ \citealt{heetal1989}), others as /\ipa{ɑ}/ vs.
			/\ipa{æ}/ \citep{fangetal1995}, and others still as /\ipa{ɑ}/ vs.\ /\ipa{a}/ \citep{fu1981, heetal1985, pinsonetal2012}. From a phonemic point of view, all these notations are valid, as they capture the relevant structural fact: the distinction between two low vowels. However, use of the symbol /\ipa{a}/ can create confusion for linguists consulting data from several sources. To ensure consistency across descriptions of \ili{Naish} languages, I have opted for the notation /\ipa{ɑ}/ vs.\ /\ipa{æ}/, thereby avoiding the symbol /\ipa{a}/.\footnote{For similar reasons of consistency across descriptions of {Naish} languages, the symbol /\ipa{ɯ}/ is used for the high back unrounded vowel in Yongning Na, \ili{Naxi}, and \ili{Laze}, despite slight differences in pronunciation. In Yongning Na, this vowel is articulated less far back than its \ili{Naxi} counterpart, which is transcribed with the same symbol. From a~purely phonetic perspective, one could argue in favour of transcribing
			it as /\ipa{ɨ}/ in Yongning Na.} Phonetically, the vowel transcribed as /\ipa{ɑ}/ is clearly a~back vowel in \ili{Naxi}, close
			to cardinal [\ipa{ɑ}], whereas in Na and \ili{Laze} it is closer to [\ipa{a}]. 
            
            While the vowel transcribed as /\ipa{æ}/ initially seemed to me to be pronounced with less salient differences between these three languages, a~native speaker of Lijiang \ili{Naxi}, Mu Yanjuan \zh{木艳娟}, explained to me (p.c.\ 2016) that in her view the front low vowel in \ili{Naxi} is clearly [\ipa{a}] and notation as [\ipa{æ}] is definitely inappropriate.\footnote{Estimations of formant frequencies (F1, F2, and F3) of Naxi vowels by five Naxi speakers are presented in \citet[475-480]{michaud2005}.}
            %I had not noticed any cross-language difference in the pronunciation of the vowel transcribed as /\ipa{æ}/ until 
			
			In Yongning Na, the sounds [\ipa{ɑ}] and [\ipa{ɤ}] are contrastive in certain contexts, notably after dental affricates and fricatives, e.g.~/\ipa{tsɑ˧}/ ‘busy’ vs.\ /\ipa{tsɤ˧}/ ‘greedy’ and /\ipa{sɑ˥}/ ‘hemp, \textit{Cannabis sativa}’
			vs.\ /\ipa{sɤ˥}/ ‘blood’. %While their present-day distribution suggests that the opposition was historically neutralised in most environments, it has been rAlthough it is clear from their
			The number of contexts in which the opposition is present has probably been increased through processes of \is{gap-filling}gap-filling. (About the notion of structural \is{gap-filling}gap-filling, see \sectref{sec:theinitialvoiceduvularfricativeasaphonemicizedemptyonsetfiller}.)
			
			After labial consonants, only [\ipa{ɤ}] occurs, not [\ipa{ɑ}] (as shown in \tabref{tab:InNucl}). This pattern is consistent throughout the lexicon, with two main exceptions: (i)~in loanwords and (ii)~in items that probably underwent \isi{vowel harmony}. Thus, the sequence /\ipa{mɑ}/, which contrasts with /\ipa{mɤ}/, has been introduced through \ili{Tibetan} borrowings, as seen in names such as /\ipa{ɖɯ˩mɑ\#˥}/, /{\kern1pt}\ipa{ɲi˩mɑ\#˥}/, and /\ipa{gv̩˧mɑ˧}/, as well as in cultural vocabulary, e.g.\ /\ipa{lɑ˧mɑ˧}/ ‘priest, lama’ and
			/\ipa{mɑ˩ɳɯ˧-do˥bv̩˩}/ ‘Mani wall’.
            %, and /\ipa{mɑ˧pʰv̩˧}/ 			‘butter’ (where the second syllable is a~Na adjective meaning ‘white’). %% Correction in 2025: it's in fact /mɤ˧pʰɻ̩˧/.
            %Somewhat paradoxically, the first
			%syllable of /\ipa{mɑ˧pʰv̩˧}/
			%‘butter’ is perceived as semantically and phonemically different from /\ipa{mɤ˩}/
			%‘animal fat’, even though ‘butter tea’ is /\ipa{mɤ˩ɬi˩}/. 
            The sequence /\ipa{mɑ}/ is also found
			in the term of address /\ipa{ə˧mɑ˧}/, ‘Mum, mother’, as well as in the clan name
			/\ipa{lɑ˩mɑ˩}/, which may be of \ili{Tibetan} origin. Two further instances, /\ipa{mɑ˩dzɑ˩}/ ‘solid ink’ and /\ipa{mɑ˧tsɑ˥}/ ‘origin, cause’, could be due to \isi{vowel harmony}.
            
			With bilabial stops, the sequence /\ipa{bɑ˩}/ is found in /\ipa{bɑ˩lɑ˩}/ ‘jacket, upper outer garment’, where it may again be attributable to \isi{vowel harmony}. The final particle /\ipa{bɑ˩˥}/, which serves a~function comparable to that of {question} tags in \ili{English}, also contains this sequence; it belongs to the language's \is{expressivity}expressive margins. Finally, /\ipa{pɑ}/ occurs in /\ipa{pɑ˧tɕɤ˧}/ ‘plantain (a~species of banana)’, a~\is{loanwords}loanword from {Mandarin} (\textit{bājiāo} \zh{芭蕉}).
			
			%On the other hand, /\ipa{ɑ}/ and /\ipa{ɤ}/ are contrastive after affricates and fricatives,
			
			
			Only  [\ipa{ɤ}] occurs after velars, while only [\ipa{ɑ}] is found after uvulars. For example, compare /\ipa{kɤ˧˥}/ ‘to knock (on the door)’ with /\ipa{qɑ˧˥}/ ‘to help’.
			
			After /\ipa{h}/, there is a~phonemic opposition, as illustrated by /\ipa{hɑ˩}/
			‘to open (one’s eyes)’ vs.\ /\ipa{hɤ˩}/ ‘to dry beside or over a~fire’.
			
			\subsection{A note on phonetic diphthongization}
			\label{sec:diphthongization}
			
			In Na, phonetic diphthongization affects the realization of certain phonemes that are written as simple vowels: /\ipa{i}/, /\ipa{e}/, /\ipa{æ}/, /\ipa{ɯ}/, /\ipa{ɤ}/, /\ipa{o}/, and /\ipa{ɑ}/. Some degree of formant movement is also observed in the syllabic consonants /\ipa{ɻ̩}{\kern2pt}/ and /\ipa{v̩}/. In general, formants in Na vowels are less stable than in, say, Northern (Parisian) \ili{French}, whose more {conservative} varieties provide a~canonical illustration of a four-way contrast in vowel openness: /\ipa{i}/, /\ipa{e}/, /\ipa{ɛ}/, and /\ipa{a}/, alongside /\ipa{u}/, /\ipa{o}/, /\ipa{ɔ}/, and /\ipa{ɑ}/. 
			
			Phonetic diphthongization in Na is at its clearest in the vowel /\ipa{e}/, which is realized phonetically close to [\ipa{ej}], as noted by \citet[63, 96]{lidz2010}. This realization of /\ipa{e}/ constitutes a~point of similarity with \ili{Naxi}, a~language in which the simple vowels are otherwise relatively stable. Evidence suggests that this diphthongization dates back at least a~century. Bonin, an explorer of the turn of the twentieth century, transcribed the \ili{Naxi} word /{\kern1pt}\ipa{ɲi˧}/ ‘two’ as \textit{ngié}, and Bacot, another early twentieth-century explorer, rendered the \ili{Naxi} name of the city of Lijiang, /\ipa{ʝi˧gv̩˧dy˩}/, with \textit{yé} for /\ipa{ʝi˧}/ \citep[3]{bacot1913}. Bacot further observed that “each [simple] vowel and its diphthongs [diphthongized variants] are interchangeable” \citep[28]{bacot1913},\footnote{\textit{Original text:} chaque voyelle et ses diphtongues sont interchangeables.} suggesting that diphthongization was not considerable \citep{michaudetal2010}.
			

			\subsection{A note on vowel harmony}
			\label{sec:anoteonvowelharmony}
			
			Anticipatory \isi{vowel harmony} (‘right-to-left’ harmony) is a~salient phonetic tendency in connected
			speech in all the \ili{Naish} languages studied so far. For instance, in \ili{Laze}, /\ipa{ʝi˧dy˧}/ ‘family’ is
			sometimes realized close to [{\kern1.3pt}\ipa{ʝy˧dy˧}]. This is not a~phonological phenomenon in the strict sense: vowel oppositions
			in the first syllable of disyllabic words are not neutralized. However, in certain cases, this phonetic tendency becomes
			lexicalized. For instance, in some \ili{Naxi} dialects, including A-sher, ‘pigswill’ is /\ipa{bu˩-hɑ˧}/ (from /\ipa{bu˩}/ ‘pig’
			and /\ipa{hɑ˧}/ ‘food’), whereas in other dialects, such as
			Nda-le, it has undergone vowel harmony and become /\ipa{bɑ˩-hɑ˧}/. This phenomenon is reported by \citet[11]{he1985}, though without mention of the dialects concerned.
			
			The extent of this phenomenon varies considerably across languages and dialects. Among the three \ili{Naish}
			languages that I studied so far (\ili{Laze}, Na, and \ili{Naxi}), \ili{Naxi} is least prone to the \isi{lexicalization} of such
			phonetic changes, whereas \ili{Laze} exhibits the highest degree of \isi{lexicalization}. The entrenchment of \is{vowel harmony}vowel-harmonized forms sometimes co-occurs with other phonological processes, such as the voicing of intervocalic voiceless consonants. A~typical
			example from \ili{Laze} is /\ipa{ʂie˧-lie˧mie˧}/ ‘seventh month’, from /\ipa{ʂɯ˧}/ ‘seven’ and
			/\ipa{ɬie˧mie˧}/ ‘month’. In this case, the vowel of the first syllable has changed through vowel harmony, and the initial /\ipa{ɬ}/ has undergone voicing to become /\ipa{l}/.
			
			In Na, the phonetic tendency towards regressive \isi{vowel harmony} is especially strong for the vowel
			/\ipa{æ}/. For instance, the combination of /\ipa{ŋwɤ˧}/ ‘five’ with the monetary unit /\ipa{mæ˩\textsubscript{a}}/ is pronounced
			[\ipa{ŋwæ˧-mæ˥}] ‘five yuan’ in casual speech, although the careful (hyperarticulated) pronunciation is [\ipa{ŋwɤ˧-mæ˥}]. \is{function words}Function words, which tend to have a~weaker phonetic realization overall, are especially susceptible to \isi{vowel harmony}. Here are two examples of this tendency:
			
			\begin{enumerate}[label=(\roman*)]
				\item  The
				\textsc{accomplished} \is{prefixes}prefix /\ipa{le˧-}/ is realized close to [\ipa{læ}] when the vowel of the following
				verb is /\ipa{æ}/ or an apical vowel.
				
				\item  The {negation} \is{prefixes}prefix
				/\ipa{mɤ˧-}/ is realized close to [\ipa{mɑ}] when the vowel of the following verb is an apical
				vowel.
				
			\end{enumerate}
			
			\is{vowel harmony} Vowel harmony for these two prefixes is so strong that, for a time (2008--2011), I transcribed them as /\ipa{lə˧-}/ and  /\ipa{mə˧-}/, with a~neutral vowel. However, vowel harmony also affects other morphemes, such as the \textsc{durative} \is{prefixes}prefix \mbox{/\ipa{tʰi˧-}/}, suggesting that this phenomenon is more widespread than initially thought. Pending the results of future phonetic studies of the relative degree of \isi{vowel harmony} across a~broad range of grammatical words, I chose to stop granting special phonemic status to the vowel in the
			\textsc{accomplished} and {negation} prefixes. 
            
            This topic warrants further investigation, particularly in relation to {diachronic} phonology. \is{vowel harmony}Vowel harmony has the potential to introduce new phonotactic combinations into the phonological system, as harmonized phonetic forms occasionally become \is{lexicalization}lexicalized. For
			instance, syllables containing a~dental stop followed by /\ipa{æ}/, such as /\ipa{læ}/, /\ipa{tæ}/,
			and /\ipa{tʰæ}/, are scarce in Na, and most appear to originate from \isi{vowel harmony} (see
			\sectref{sec:retroflexstopsandaffricates}).
			

			\section{Initial consonants}
			\label{sec:consonants}

			\subsection{On-glides}
			\label{sec:smoothphoneticonsets}
			\label{sec:apresentationofonglideswithahypothesisaboutadiachroniconsetofhardeningofinitialglides}
			
			The high vowels /\ipa{i}/, /\ipa{ɯ}/, and /\ipa{o}/ have phonetic on-glides: they are realized as
			[{\kern1.3pt}\ipa{ʝi}], [\ipa{ɣɯ}], and [\ipa{wo}], respectively. 
			
			The degree of friction in the realization of /\ipa{i}/ prompted the transcription of its on-glide as a~fricative, [{\kern1.3pt}\ipa{ʝ}], rather than an approximant, [{\kern0.7pt}\ipa{j}]. The approximant notation is instead used for the complex rhymes /\ipa{jo}/, /\ipa{jɤ}/, and /\ipa{jæ}/, where the initial glide is produced with less friction.
			
			The syllable /\ipa{ɯ}/ could also
			be transcribed with an approximant onset, [\ipa{ɰɯ}]. This is the convention adopted in a~dictionary of
			\ili{Naxi} \citep{pinsonetal2012}, a~language in which the phonemic analysis of this syllable is the same as in Yongning Na and the phonetic
			realization does not sound any different to me.
			
			The phonetic on-glides in [{\kern1.3pt}\ipa{ʝi}], [\ipa{ɣɯ}] and [\ipa{wo}] are salient enough to warrant inclusion in transcriptions, even at the cost of some deviation from notational economy. However, they involve less formant movement than the
			phonemic on-glides of the complex rhymes: /\ipa{jo}/, /\ipa{jɤ}/, /\ipa{jæ}/, /\ipa{wɤ}/, /\ipa{w̃ɤ}/,
			/\ipa{wæ}/, /\ipa{w̃æ}/, /\ipa{wɑ}/, and /\ipa{w̃ɤ}/. 
            
            Onsetless
			syllables tend to coalesce phonetically with the preceding syllable in polysyllabic words and tightly-knit polysyllabic
			expressions. For several years, I mistakenly believed that the manner adverbials ‘in this
			way, thus’ and ‘in that way’ were /\ipa{ʈʂʰɯ˧ni˧˥}/ and /\ipa{tʰv̩˧ni˧˥}/, respectively. In fact, these phrases are /\ipa{ʈʂʰɯ˧ne˧ ʝi˥}/ and /\ipa{tʰv̩˧ne˧ ʝi˥}/, consisting of a~manner {adverbial} (/\ipa{ʈʂʰɯ˧ne˧˥}/, /\ipa{tʰv̩˧ne˧˥}/) followed by the verb ‘to do’, /\ipa{ʝi˥}/ (as discussed in \sectref{sec:closevowels}). The \is{trisyllables}{trisyllabic}
			structure of these phrases did not emerge from phonetic evidence: it was ultimately revealed by their tonal
			behaviour. 
            
            This is illustrated in example (\ref{ex:thisishowtheyusedtosing}), shown here as initially transcribed:
			\begin{exe}
				\ex
				\label{ex:thisishowtheyusedtosing}
				\ipaex{ʈʂʰɯ˧ni˧˥ {\kern2pt}|{\kern2pt} gwɤ˩-ɲi˥ {\kern2pt}◊{\kern2pt} mæ˩!}\\
				\gll ʈʂʰɯ˧ni˧˥ gwɤ˩ -ɲi˩ mæ˧\\
				in\_this\_way to\_sing \textsc{certitude} \textsc{obviousness}\\
				\glt ‘This is how [people] used to sing!’ (\textit{Caravans.51} \pandoi{0004530\#S51}, \textit{53}, \textit{57})
			\end{exe}
			
			If the notation /\ipa{ʈʂʰɯ˧ni˧˥}/ were correct, it should be possible to group the {adverbial} with the following verb phrase into a~single \isi{tone group}, yielding \ipa{$\ddagger${\kern2pt}ʈʂʰɯ˧ni˧ gwɤ˥-ɲi˩-mæ˩}. %\footnote{About the division of the utterance into tone groups, see Chapter~\ref{chap:toneassignmentrulesandthedivisionoftheutteranceintotonegroups}.} 
            But this \is{stylistics}stylistic option is not available: the manner \is{demonstratives}demonstrative cannot project an H tone onto a~following
			syllable, as would be expected if it carried MH\# tone. 
            
            The solution to this puzzle lies in the fact that the {adverbial} phrase ‘thus, in this way’ is \is{trisyllables}{trisyllabic} (/\ipa{ʈʂʰɯ˧ne˧-ʝi˥}/) and remains so despite the strong phonetic {coarticulation} between its final two
			syllables. The correct notation of
			(\ref{ex:thisishowtheyusedtosing}) is provided in (\ref{ex:thisishowtheyusedtosingGOOD}). 
			
\begin{exe}
	\ex
	\label{ex:thisishowtheyusedtosingGOOD}
	\ipaex{ʈʂʰɯ˧ne˧ ʝi˥ {\kern2pt}|{\kern2pt} gwɤ˩-ɲi˥ {\kern2pt}◊{\kern2pt} mæ˩!}\\
	\gll ʈʂʰɯ˧ne˧˥ 		ʝi˥		gwɤ˩ -ɲi˩ mæ˧\\
	in\_this\_way to\_do		to\_sing \textsc{certitude} \textsc{obviousness}\\
	\glt ‘This is how [people] used to sing!’ (\textit{Caravans.51} \pandoi{0004530\#S51}, \textit{53}, \textit{57})
\end{exe}

			Turning now to other vowels, /\ipa{æ}/ and /\ipa{ɑ}/ may be realized either with a~glottal stop (hard phonetic onset) or with soft phonation:
			[\ipa{ɦæ}]{\kern2pt}\ipa{≈}{\kern2pt}[\ipa{ʔæ}], [\ipa{ɦɑ}]{\kern2pt}\ipa{≈}{\kern2pt}[\ipa{ʔɑ}]. Regarding the choice of one onset type over the other, a~study of glottal stops before word-initial vowels in American \ili{English} concludes that “full glottal stops [\ipa{ʔ}] are predicted overwhelmingly by prominence and \isi{phrasing}” \citep[20]{garellek2012}. The study also reviews a~set of factors that would provide a~useful starting point for future phonetic research on this phenomenon in Yongning Na:
			
			\begin{quotation}
				{\dots} the factors that promote the occurrence of word-initial glottalization ({\dots}) may be segmental, lexical, prosodic, or sociolinguistic. In \ili{English}, segmental factors include hiatus (V\#V) environments and word-initial back vowels are found to glottalize more frequently than non-back vowels. As for lexical factors, content words exhibit more frequent glottalization than \isi{function words}. Sociolinguistically, women are known to use glottalization more than men. Prosodically, the presence of stress and/or a~pitch accent on the word-initial vowel, as well as a~larger {juncture} with the preceding word, are known to promote glottalization. Researchers working on other languages have found additional factors that promote the occurrence of word-initial glottalization, including presence of a~preceding pause as well as speech rate and low vowel quality for German. \citep[1-2]{garellek2012} %\footnote{Citations have been removed from this quotation.}
			\end{quotation}
			
			Unlike in \ili{Naxi}, where /\ipa{ɤ}/ is realized as /\ipa{ɣɤ}/, in Yongning Na the vowel /\ipa{ɤ}/ never
			forms a~syllable on its own. As explained in \sectref{sec:theopenvowelsandandthevowel}, the
			opposition between /\ipa{ɤ}/ and /\ipa{ɑ}/ is restricted to a~few contexts; synchronically, it may
			be considered to be neutralized in onsetless syllables.
			
			No distinct onset portion is found for /\ipa{v̩}/, which is realized simply as  [\ipa{v̩}]. Likewise, the rhyme /\ipa{ɻ̩}{\kern2pt}/ is not separated from the preceding rhyme by a~glide or glottal stop. For instance, in /\ipa{bv̩˧ɻ̩\#˥}/
			‘fly (the insect)’ the /\ipa{v̩}/ and /\ipa{ɻ̩}{\kern2pt}/ occur in direct succession: [\ipa{bv̩ɻ̩}{\kern2pt}].
			
			The following paragraph examines syllables that canonically begin with a~glottal stop. 
				
			
			\subsection{Initial glottal stop}
			\label{sec:hardphoneticonsets}
			
			
			The rhymes /\ipa{ə}/ and /\ipa{ɻ̩̃}{\kern2pt}/ begin with a~phonetic glottal stop when pronounced \is{form!in isolation}in isolation, and
			also word-medially in hyperarticulated realizations. For instance, /\smash{\ipa{ʂæ˩ɻ̩̃˩}}/ ‘bone’ is realized as
			[\smash{\ipa{ʂæ˩ʔɻ̩̃˩˥}}] in careful speech, whereas in casual speech there is no glottal stop: [\smash{\ipa{ʂæ˩ɻ̩̃˩˥}}].
			
			The morpheme /\ipa{u˧}/,
            %\rephrase{,}{ is } % Felicitous suggestion by Sebastian Nordhoff, to achieve a typographically balanced line
            an \textsc{associative} first person \is{pronouns}pronoun, appears in the expressions /\ipa{u˧=ɻ̩˩}/ ‘my clan, my people’ and /\ipa{u˧=ɻæ˩}/ ‘us (as opposed to them)’. It
            %\rephrase{,}{.    It }% Felicitous suggestion by Sebastian Nordhoff, to achieve a typographically balanced line
            is pronounced with an initial glottal stop, yielding [\ipa{ʔu˧=ɻ̩˩}] and [\ipa{ʔu˧=ɻæ˩}]. This morpheme constitutes the only documented instance of this syllable type. However, even a~single example suffices to establish it as distinct from the syllable realized as [\ipa{wo}]{\kern2pt}\ipa{≈}{\kern2pt}[\ipa{wu}], which occurs in words such as //\ipa{wo˥}// ‘hard’, //\ipa{wo˩kɤ\#˥}// ‘swing’, and //\ipa{wo˩˥}// ‘turnip leaves’. 
            
            This raises the question of how to phonemicize this contrast. Several solutions are possible, including the three outlined below, which could also be combined in various ways:
			
			\begin{enumerate}[label=(\roman*)]
				\item The syllable [\ipa{ʔu}] could be phonemicized as /\ipa{u}/, and [\ipa{wo}]{\kern2pt}\ipa{≈}{\kern2pt}[\ipa{wu}] as /\ipa{wu}/, thus granting phonemic status to the initial glide. 
				\item The syllable [\ipa{ʔu}] could be phonemicized as /\ipa{ʔu}/, and [\ipa{wo}]{\kern2pt}\ipa{≈}{\kern2pt}[\ipa{wu}] as /\ipa{u}/, thus granting phonemic status to the initial glottal stop. 
				\item The syllable [\ipa{ʔu}] could be phonemicized as /\ipa{u}/, and [\ipa{wo}]{\kern2pt}\ipa{≈}{\kern2pt}[\ipa{wu}] as /\ipa{o}/, ascribing the different initials to phonotactic rules of assignment of empty-onset fillers.
			\end{enumerate}
			
			Option (iii) is adopted here, obviating the need for recognizing the glottal stop as a~phonemic initial. However, from a~perceptual point of view, the most salient cue to the opposition appears to reside in the initial segment. From this perspective, it does not appear unlikely that either the initial glottal stop or the labial-velar approximant could gain phonemic status depending on how new generations of learners interpret expressive forms such as \ipa{ʔo!} (an interjection indicating surprise), dutifully recorded in the dictionary \citep[167]{michaud_et_al_na_dict_2024}.
            
            %The provisional analysis adopted here treats the approximant as an empty-onset filler while recognizing the glottal stop as a~phonemic initial, hence its inclusion in \tabref{tab:theinitialsofyongningna}. 
            Cases of phonemic ambiguity such as this hold potential for changes to the system. A~relevant example is the case of initial /\ipa{ʁ}/, discussed in the next paragraph, which illustrates how new consonants can emerge through the phonemic reinterpretation of empty-onset fillers.
			
			\subsection{Initial /\ipa{ʁ}/ as a~phonemicized empty-onset filler}
			\label{sec:theinitialvoiceduvularfricativeasaphonemicizedemptyonsetfiller}
			
			
			The status of the syllables /\ipa{wɤ}/, /\ipa{wæ}/, /\ipa{o}/, /\ipa{ɑ}/, and /\ipa{æ}/ is particularly interesting. These rhymes can occur with an initial voiced uvular fricative,
			/\ipa{ʁ}/.\footnote{The voiced uvular fricative /\ipa{ʁ}/ is not uncommon in this linguistic area,
				but with a~widely different phonemic status from one language to another. For instance, in {Lizu}
				\citep{chirkovaetal2012}, it is an allophone of the voiced velar fricative /\ipa{ɣ}/ before
				/\ipa{ɐ}/ and /\ipa{wɐ}/, e.g.~/\ipa{ɣɐ˥˩}/ [\ipa{ʁɐ˥˩}] ‘needle’, /\ipa{ɣwɐ˥˩}/ [\ipa{ʁuɐ˥˩}] ‘to
				thunder’.} (Phonetically, this /\ipa{ʁ}/ is weakly articulated and, in
			some hypo-articulated tokens, can be mistaken for /\ipa{w}/.) Some of the words in question \is{derivation!morphological}derive diachronically from
			onsetless syllables. For instance, the word for ‘village’, /\ipa{ʁwɤ˧}/, corresponds to /\ipa{wɤ˧}/ in \ili{Naxi} and
			\ili{Laze}; the \is{homophony}homophonous word for ‘mountain’, also /\ipa{ʁwɤ˧}/, has an equivalent in \ili{Naxi} where it means ‘hill, hillock’ (/\ipa{wɤ˧}/). Other cases reflect developments from syllables with an initial velar or uvular cluster. For example,
the word for ‘sword’, /\ipa{ʁæ˧mi˧}/, corresponds to \ili{Naxi} /\ipa{ŋgæ˩}/.\footnote{A comparable process of onset hardening has been reported in Zeluo \ili{Ersu}, where /\ipa{w}-/ is sometimes realized with frication, as
			[\ipa{ɣʷ-}] \citep{sun1982}. Among the last speakers of the language, there is variation between [\ipa{ʁ}] and [\ipa{w}] \citep{chirkovaduoxu2014}.}
			
			The hardening of soft onsets would, in principle, have led to the systematic absence of syllables realized as
			[\ipa{wɤ}], [\ipa{wæ}], [\ipa{o}], [\ipa{ɑ}], or [\ipa{æ}], since these became [\ipa{ʁwɤ}],
			[\ipa{ʁwæ}], [\ipa{ʁo}], [\ipa{ʁɑ}], and [\ipa{ʁæ}]. This pattern is indeed observed for /\ipa{wæ}/: only [\ipa{ʁwæ}] is
			attested, with no recorded instance of [\ipa{wæ}]. For instance, the word for ‘left, leftward’, /\ipa{ʁwæ˥}/, corresponds to \ili{Naxi} /\ipa{wæ˧}/ and
			\ili{Laze} /\ipa{væ˧}/. For the other syllables, however, there are oppositions between forms with
			and without initial /\ipa{ʁ}/. This suggests that a~process of structural \isi{gap-filling} has taken place, leading to the reintroduction of onsetless syllables into the system. Further progress in the
			etymological analysis of individual examples will be necessary to elucidate the specific mechanisms involved in this development. The hardening of empty-onset fillers
			must date back a~relatively long way, judging from the number of currently attested onsetless syllables. For instance, in the case of /\ipa{wɤ}/, the following examples can be cited: 
            \begin{itemize}
                \item the classifier for loads, /\ipa{wɤ˩\textsubscript{b}}/,
                \item the noun ‘serf, slave’, /\ipa{wɤ˧}/,
                \item the adverb ‘again, anew’, /\ipa{wɤ˩˥}/,
                \item the verbs ‘to depend on, to rely on’,
			/\ipa{wɤ˩\textsubscript{b}}/ and ‘to detour past, to bypass’,
			/\ipa{wɤ˩{$\sim$}wɤ˩}/,
            \item the final exclamative particle /\ipa{wɤ˧}/, which conveys obviousness.
            \end{itemize}
			
			The change from proto-\ili{Naish} *\ipa{j} to Na /\ipa{ʑ}/ may represent another instance of the same hardening process. The verb ‘to
			sleep’ in Na, /\ipa{ʑi˧˥}/, corresponds to \ili{Naxi} /\ipa{ʝi˥}/, which is phonemically a~simple /\ipa{i}/, to which
			an empty-onset filler is added. In \ili{Laze}, the cognate is /\ipa{zi˩}/. The \is{comparative method (historical linguistics)}reconstruction proposed by \citet{jacquesetal2011} is *\ipa{jip}, supporting the hypothesis that the development of /\ipa{ʑ}/ involved a phonemic reinterpretation of an earlier onset filler.
			
			
			\subsection{Velar and uvular stops}
			\label{sec:velaranduvularstops}
			
			
			Velar and uvular stops are in complementary distribution in most environments. However, they contrast before  /\ipa{v̩}/,
			/\ipa{wɤ}/, and  /\ipa{o}/. Examples of the syllables  /\ipa{kv̩}/,
			/\ipa{qv̩}/,  /\ipa{kʰv̩}/,  /\ipa{qʰv̩}/,  /\ipa{ko}/,  /\ipa{qo}/,  /\ipa{kʰo}/, and  /\ipa{qʰo}/ are
			provided in (\ref{ex:velaranduvularstops}). In addition, a~single instance of  /\ipa{qi}/, contrasting with  /\ipa{ki}/, has also
			been recorded.
			
%\Hack{\newpage}

			\begin{exe}
				\ex \label{ex:velaranduvularstops}
				\begin{xlist}
					\extab
					\begin{tabularx}{112mm}{ P{14mm} P{27mm} P{15mm} Q }
						~~~/\ipa{qv̩}/ &  & ~~~/\ipa{kv̩}/ &\\
						\ipa{qv̩˩˥} & handle & \ipa{kv̩˥} & garlic\\ 
						\ipa{qv̩˧˥} & to frighten & \ipa{kv̩˧˥} & to be able to\\ 
						\ipa{qv̩˧ʈʂæ˧˥} & throat & \ipa{kv̩˧ʈʂɯ˥\$} & nail\\ 
						\ipa{mæ˧qv̩˩} & tail & \ipa{kv̩˧dʑɯ˧˥} & tent\\
					\end{tabularx}
					
					\Hack{\vspace*{.5\baselineskip}}  
					\extab
					\begin{tabularx}{112mm}{ P{14mm} P{27mm} P{15mm} Q }
						~~~/\ipa{qʰv̩}/ &  & ~~~/\ipa{kʰv̩}/ &\\
						\ipa{qʰv̩˧˥} & six & \ipa{mv̩˧kʰv̩˧˥} & smoke\\ 
						\ipa{qʰv̩˧} & horn & \ipa{kʰv̩˥} & to cut (grass)\\ 
						\ipa{qʰv̩˧} & hole & \ipa{kʰv̩˥} & dog\\ 
						\ipa{bv̩˩qʰv̩˩} & conch shell & \ipa{kʰv̩˧˥} & year\\
					\end{tabularx}
					
					\Hack{\vspace*{.5\baselineskip}}
					\extab
					\begin{tabularx}{112mm}{ P{14mm} P{27mm} P{15mm} Q }
						~~~/\ipa{qo}/ &  & ~~~/\ipa{ko}/ &\\
						\ipa{qo˥} & to love & \ipa{ko˥} & hill\\ 
						\ipa{qo˩ho˧˥} & bamboo box & \ipa{ko˩} & to bask\\ 
						\ipa{qo˩qɑ˩} & mountain pass & \ipa{ko˩ɖʐo˩} & flail\\ 
						\ipa{-qo˧} & inside & \ipa{mæ˩ko˥} & harness\\
					\end{tabularx}
					
					% \Hack{\newpage}
					\Hack{\vspace*{.5\baselineskip}}
					\extab
					\begin{tabularx}{112mm}{ P{14mm} P{27mm} P{15mm} Q }
						~~~/\ipa{qʰo}/ &  & ~~~/\ipa{kʰo}/ &\\
						\ipa{qʰo˧˥} & to kill & \ipa{kʰo˥} & to spread (e.g.~a~sheet)\\ 
						\ipa{qʰo˧lo˧} & wheel & \ipa{kʰo˧lo˧} & prayer wheel\\ 
						\ipa{qʰo˩tv̩˧˥} & tree stump & \ipa{tse˧kʰo˩} & sanctuary\\ 
						\ipa{qʰo˩mv̩˩} & straw hat & \ipa{hæ̃˧kʰo˧} & princess, young lady\\
					\end{tabularx}
					
					\Hack{\vspace*{.5\baselineskip}}
					\extab
					\begin{tabularx}{112mm}{ P{14mm} P{27mm} P{15mm} Q }
						~~~/\ipa{qi}/ &  & ~~~/\ipa{ki}/ &\\
						\ipa{qi˧qi˧} & originally, at first & \ipa{ki˧} & to give\\
					\end{tabularx}
				\end{xlist}
			\end{exe}
			
			This situation is unsurprising in areal context. In \ili{Lizu}, for example, velar and uvular stops only contrast before
			/\ipa{o}/, as in /\ipa{ko˨˧}/ ‘to beg’ vs.\ /\ipa{qo˥˩}/ ‘hole, pit’ \citep{chirkovaetal2012}. 
			
			From a~\is{comparative method (historical linguistics)}{diachronic} point of view, uvulars may originate from a~variety of historical sources \citep[782-783]{sun2003a}. In the case of Na, cognates with uvular initials in \ili{rGyalrong}~-- a~{conservative} language~-- support the hypothesis that uvulars in Na are of considerable antiquity \citep[492]{jacquesetal2011}.
			
			
			\subsection{Retroflex stops and affricates}
			\label{sec:retroflexstopsandaffricates}
			
			Yongning Na has (i)~an opposition between dental and retroflex affricates, which carries a~high functional
			load, and (ii)~an opposition between dental and retroflex stops and nasals, but only before
			/\ipa{i}/, /\ipa{æ}/, /\ipa{v̩}/ and /\ipa{o}/. Examples include: /\ipa{ʈʰi˩}/ ‘tired’ vs.
			/\ipa{tʰi˩}/ ‘to plane (wood)’; /\ipa{ʈi˩}/ ‘to get up’ vs.\ /\ipa{ti˩}/ ‘to knock, to tap
			(lightly)’; /\ipa{ʈæ˧bɤ˧}/ ‘Buddhist priest’ vs.\ /\ipa{tæ˧pv̩˩}/ ‘thin, skinny’; and /\ipa{ɖo˧\textsubscript{a}}/ ‘to
			allow; ought; to have to’ vs.\ /\ipa{do˥}/ ‘to climb’. After retroflex consonants, the vowel /\ipa{o}/ is realized
			close to [\ipa{u}].
			
			The consonants transcribed here as retroflexes are articulated notably less far back than the canonical
			retroflexes of languages such as \ili{Tamil} or \ili{Nepali} \citep{khatiwada2009}. A~palatographic study may reveal that these sounds are in fact
			postalveolar rather than retroflex. This situation highlights a~shortcoming of the current \isi{International Phonetic Alphabet} system: there is a~need for distinct symbols for postalveolar and dental articulations, as argued by \citet[21-30]{ladefogedetal1996}. This is not merely a~matter of fine phonetic detail to be addressed by means of diacritics. The absence of distinct \is{International Phonetic Alphabet}IPA symbols for postalveolars has led to the widespread adoption of retroflex symbols for non-dental sounds, a~practice that creates no small amount of typological confusion and skews cross-linguistic studies of phonological inventories. 
            
            In the absence of dedicated \is{International Phonetic Alphabet}IPA symbols for postalveolar stops, the provisional solution adopted here is to use the symbols for retroflex sounds. 
			
			
			\subsection{Laterals  /\ipa{l}/ and  /\ipa{ɬ}/, and retroflexes /\ipa{ɭ{\kern2pt}}/ and  /\ipa{ɻ{\kern2pt}}/}
			\label{sec:lateralsandandtheretroflexapproximant}
			
			
			The laterals /\ipa{l}/ and /\ipa{ɬ}/ are contrastive in Yongning Na, as demonstrated by pairs such
			as /\ipa{li˧\textsubscript{a}}/ ‘to look’ vs.\ /\ipa{ɬi˧}/ ‘month’ and /\ipa{lo˧˥}/ ‘thick’ vs.\ /\ipa{ɬo˧˥}/
			‘deep’. Phonetically, the voiced lateral /\ipa{l}/ has a~broad range of allophonic \isi{variation}. It
			is realized as retroflex before /\ipa{ɯ}/, e.g.~in the classifier for round objects (bowls, grains), which also serves as generic classifier: /\ipa{ɭɯ˧\textsubscript{b}}/. The phonemic analysis for this
			syllable was only arrived at after the greatest hesitations. A~wide array of hypotheses was
			explored, including [\ipa{ɻɯ}], [\ipa{ɬɯ}], [\ipa{lv̩}], [\ipa{ly}], [\ipa{ɭ{\kern1pt}y}], and syllabic
			[\ipa{ɭ}{\kern2.5pt}] or [\ipa{ɹ}]. The entire syllable is loosely articulated: the initial is close to
			an approximant, and the vowel quality is not precise, so that the syllable often resembles
			a~monophonemic [\ipa{ɭ}{\kern2.5pt}].
			
			Before /\ipa{v̩}/, the voiced lateral /\ipa{l}/ is also slightly retroflex. In all contexts, its articulation is accompanied by some friction. This characteristic is at its clearest before the high
			front vowel /\ipa{i}/, but it is also observed before open vowels such as /\ipa{ɑ}/ and
			/\ipa{æ}/. Phonetically, this supports a~transcription of this allophone as [\ipa{ɮ}]
			rather than [\ipa{l}]. Phonologically, it would be economical to consider that the two lateral phonemes are
			distinguished solely by voicing, reinforcing the case for a~phonemic analysis as
			/\ipa{ɮ}/.
			
			The compromise adopted here, for the sake of simplicity, consists in using the notation /\ipa{l}/
			rather than /\ipa{ɮ}/. On the other hand, in the syllable [\ipa{ɭɯ}] (phonemically /\ipa{lɯ}/), the use of a~retroflex initial is retained in transcription, to reflect what I perceive as a~significant phonetic departure from the canonical realizations of /\ipa{l}/ in other contexts. This choice aims to keep the
			transcriptions close to surface forms.
			
			These observations on the allophonic \isi{variation} of /\ipa{l}/ shed indirect light on the distribution
			of the retroflex approximant /\ipa{ɻ}{\kern2pt}/: it is plausible that this phoneme originated as an allophone of /\ipa{l}/
			which drifted to such a~phonetic distance that it opened a~structural gap that was subsequently filled
			through borrowing and \is{vowel harmony}vowel-harmony-driven developments. In synchrony, /\ipa{ɻ}{\kern2pt}/ only appears in the syllables /\ipa{ɻæ}/ and /\ipa{ɻwæ}/. These correspond to /\ipa{læ}/ and /\ipa{lwæ}/ in \ili{Laze}: compare Na /\ipa{ɻwæ˥}/ ‘to shout, to cry’ with \ili{Laze} /\ipa{lwæ˧}/, and Na /\ipa{ɻæ˩˥}/ ‘seed’ with \ili{Laze} /\ipa{læ˩}/. Roselle Dobbs (p.c.\ 2013) indicates that in villages of the area referred to in Na as Lataddi \ipa{lɑ˧tʰɑ˧-di˧˥}, located east of Lugu Lake, these items	retain their original lateral initials. 
            
            Synchronically, /\ipa{ɻæ}/ contrasts with /\ipa{læ}/. However, the latter
			only appears (i)~in borrowings such as /\ipa{læ˧tsɯ˥}/ ‘chili peppers’, from \il{Mandarin!Southwestern}Southwestern {Mandarin}
			\zh{辣子} [\ipa{la.tsɿ}], (ii)~in the \textsc{accomplished} \is{prefixes}prefix /\ipa{le˧-}/, whose phonetic
			realizations~-- determined by the vowel of the following syllable~-- include [\ipa{læ}], and (iii)~in forms
			where /\ipa{æ}/ may have arisen through \isi{vowel harmony}, e.g.~/\ipa{læ˧ʁæ˥}/
			‘raven’. (Regressive \isi{vowel harmony}, a~prominent phonetic
			tendency in Na, becomes sporadically lexicalized: see \sectref{sec:anoteonvowelharmony}.) As for /\ipa{ɻwæ}/, no lateral counterpart /\ipa{lwæ}/ is attested.\footnote{A strikingly similar synchronic pattern is found in \ili{Lizu}, where the /\ipa{ɹ}/ phoneme only occurs before
			/\ipa{æ}/, /\ipa{ə}/, and /\ipa{wæ}/, e.g.~/\ipa{ɹæ˥˩}/ ‘yak’, /\ipa{ɹə˨˧}/ ‘to laugh’, and /\ipa{ɹwæ˨˧}/
			‘chicken’ \citep{chirkovaetal2012}. The diachronic origins appear different, however, as comparative data from \ili{Ersu} and \ili{Duoxu} point instead to an earlier *\ipa{r}, as discussed in \citet{yu2012}. Surface similarities in sound patterns tend to arise through areal convergence, as well as from cross-linguistic (\is{panchronic phonology}panchronic) regularities in processes of phonological {erosion}.}
			
			It is therefore plausible to hypothesize that present-day
			/\ipa{ɻæ}/ and /\ipa{ɻwæ}/ \is{comparative method (historical linguistics)}derive from earlier *\ipa{læ} and *\ipa{lwæ}, which underwent phonetic retraction towards [\ipa{ɻæ}] and [\ipa{ɻwæ}], thereby vacating the
			[\ipa{læ}] and [\ipa{lwæ}] phonetic slots. The [\ipa{læ}] slot was subsequently reoccupied by other items. 
			
			These {diachronic} considerations
			do not detract from the synchronic phonemic status of /\ipa{ɻ}{\kern2pt}/.
			
			
			\subsection{The glottal fricative /\ipa{h}/ and the sound  [\ipa{f}]}
			\label{sec:theglottalfricativeandthesound}
			
			Yongning Na has a~glottal fricative /\ipa{h}/. At an earlier stage of the language, the sound [\ipa{f}] can
			be hypothesized to have been entirely absent: early borrowings from \ili{Mandarin} with initial [\ipa{f}]
			in the donor language were reinterpreted with initial /\ipa{h}/. For instance, the word for ‘method, solution’ (\zh{办法}, Standard {Mandarin} \textit{bànfǎ}) was borrowed into Na as /\ipa{pæ˧˥hwɤ˧}/.
			
The sound [\ipa{f}] does appear in more recent layers of borrowings, however: e.g. /\ipa{fæ˧}/ ‘direction’ (\zh{方},
			Standard {Mandarin} \textit{fāng}) and /\ipa{fɑ˩\textsubscript{a}}/ ‘to ferment’ (\zh{发(酵)}{\kern-4pt}, Standard {Mandarin}: \textit{fā}). A~plausible scenario is
			that, at some point in the history of Na, the phoneme /\ipa{h}/ came to be realized with a~friction source located at a~point
			in the vocal tract conditioned by the following vowel. For instance, this could result in palatal realization before /\ipa{i}/ and
			labial-dental realization before /\ipa{v̩}/, yielding forms such as [\ipa{çi}] and [\ipa{fv̩}]. Once the sound [\ipa{f}] had thus been introduced into Na at the phonetic level, the way was paved for [\ipa{f}]-initial
			loanwords to be integrated without difficulty. 
            
            In the present state of the language, speakers of Yongning Na have no difficulty in pronouncing [\ipa{f}] before any rhyme. This can be taken as evidence that [\ipa{f}] is no longer perceived as
			an allophone of /\ipa{h}/. It is well-known that allophones that drift far apart often acquire psychophonetic independence from one another: well-documented cases include \ili{German} [\ipa{ç}]
			and [\ipa{x}], and \il{Mandarin!Standard}Standard {Mandarin} [\ipa{ɕ}] and [\ipa{x}]. As the psychological reality of the
			underlying unity among allophones wanes, the structural resistance to \is{gap-filling}gap-filling likewise decreases.
			
			In view of this situation, and also in order to keep the transcription close to the surface forms,
			the syllable [\ipa{fv̩}] is transcribed as such, rather than pushing phonemicization to an extreme by analyzing it as /\ipa{hv̩}/. Under a~flatly synchronic
			analysis that includes \ili{Mandarin} \is{loanwords}borrowings, the sound [\ipa{f}] needs to be granted phonemic
			status, hence its inclusion in \tabref{tab:theinitialsofyongningna}. 
			
			The choice adopted for the phonemic analysis of [\ipa{fv̩}] has potential implications for the analysis of the syllable [\ipa{hu}]. One possibility would be to phonemicize [\ipa{fv̩}] as /\ipa{fv̩}/, and [\ipa{hu}] as /\ipa{hv̩}/, thereby eliminating the /\ipa{u}/ phoneme altogether (Roselle Dobbs, p.c.\ 2016). This remains one of several areas in Yongning Na phonemics where alternative analyses are defensible.
			
			The syllable [\ipa{çi}], phonemicized as /\ipa{hi}/, contrasts with /\ipa{ɕi}/.
			

			\section{Comments about the inventory of syllables}
			\label{sec:commentsabouttheinventoryofsyllables}
			
			The inventory of attested combinations of initials and rhymes, provided at the outset of this Appendix (in Tables \ref{tab:InNucl} and \ref{tab:InNucl2}), reveals that numerous phonemic oppositions are found in highly restricted contexts
			in Yongning Na. A~similar situation is found in
			\ili{Naxi} \citep{michaud2006c}. The strict application of the principles of
			\is{Praguean phonology}Praguean synchronic description leads to an analysis of these
			cases as extreme instances of \textit{neutralization} of phonemic contrasts. 
            
            For instance, in Yongning Na the opposition between nasal and
			oral vowels is neutralized in all environments except after /\ipa{h}/ and in onsetless syllables. It may appear counter-intuitive to speak of \is{neutralization|textbf}neutralization
			in such cases, since the concept is more commonly associated with situations where a~thoroughgoing contrast disappears in a~restricted environment,
			e.g.~in Trubetzkoy’s classical example: \ili{French} /\ipa{e}/ and /\ipa{ɛ}/ contrast
			only in open syllables, the opposition being neutralized in closed
			syllables.
			\begin{quotation}
				In {French} ({\dots}) an opposition between \ipa{e} and \ipa{ɛ} only occurs word-finally in open syllables, e.g.~\textit{les} ‘definite article.\textsc{pl}’ vs.\ \textit{lait} ‘milk’ and \textit{allez} ‘to go.2\textsc{pl}’ vs.\ \textit{allait} ‘to go.3\textsc{sg.pst}’. In all other positions the occurrence
				of \ipa{e} and \ipa{ɛ} is predictable: \ipa{ɛ} occurs in closed syllables, \ipa{e} in open. These
				two vowels must thus be considered two phonemes in open-syllable-final position, and combinatorial
				variants of a~single phoneme in all other positions. We call such oppositions
				\textit{neutralizable} oppositions, the positions in which the \isi{neutralization} takes place
				\textit{positions of neutralization}, and those positions where the opposition is relevant
				\textit{positions of relevance}.~\citep[78]{trubetzkoy1969}\footnote{Some modifications to the translation were made by Roselle Dobbs.}
                    
    \medskip 
    {\noindent}\textit{Original text:} Im
					Französischen kommen aber \ipa{e} und \ipa{ɛ} nur im offenen Auslaute als Glieder einer
					phonologisch-distinktiven Opposition vor (\textit{les-lait}, \textit{allez-allait}); in den
					übrigen Stellungen ist das Vorkommen von \ipa{e} und \ipa{ɛ} mechanisch geregelt (in gedeckter
					Silbe \ipa{ɛ}, in ungedeckter \ipa{e}), so daß diese zwei Vokale nur im offenen Auslaut as zwei
					Phoneme, in den übrigen Stellungen dagegen als kombinatorische Varianten eines einzigen Phonems
					gewertet werden müssen. Der phonologische Gegensatz ist also im Französischen in gewissen
					Stellungen a{\kern2pt}u{\kern2pt}f{\kern2pt}g{\kern2pt}e{\kern2pt}h{\kern2pt}o{\kern2pt}b{\kern2pt}e{\kern2pt}n. Solche Oppositionen nennen wir
					a{\kern2pt}u{\kern2pt}f{\kern2pt}\-h{\kern2pt}e{\kern2pt}b{\kern2pt}\-b{\kern2pt}a{\kern2pt}r; jene
					Lautstellungen, in denen die Aufhebung erfolgt,
					A{\kern2pt}u{\kern2pt}f{\kern2pt}\-h{\kern2pt}e{\kern2pt}\-b{\kern2pt}u{\kern2pt}n{\kern2pt}g{\kern2pt}s{\kern2pt}\-s{\kern2pt}t{\kern2pt}e{\kern2pt}l{\kern2pt}l{\kern2pt}u{\kern2pt}n{\kern2pt}g{\kern2pt}e{\kern2pt}n,
					jene, wo die Opposition relevant ist,
					R{\kern2pt}e{\kern2pt}l{\kern2pt}e{\kern2pt}v{\kern2pt}a{\kern2pt}n{\kern2pt}z{\kern2pt}\-s{\kern2pt}t{\kern2pt}e{\kern2pt}l{\kern2pt}\-l{\kern2pt}u{\kern2pt}n{\kern2pt}g{\kern2pt}e{\kern2pt}n. \citep[70]{trubetzkoy1939}
			\end{quotation}
			
			Importantly, the term \textit{neutralization} should not be understood in a~dynamic sense, as if the
			opposition once existed and was subsequently neutralized. Rather, its application is static, flatly synchronic (\citealt[257--259]{martinet1969}; \citeyear[87--89]{martinet1970}). It is not unusual for a~synchronic formulation to stand as a~reverse image of a~{diachronic} perspective. For instance, when describing the synchronic stage at which \il{Sinitic}Chinese contrasted three tones (A, B, and C) on non-obstruent-final syllables, it can be said that
			the tonal opposition was neutralized on obstruent-final syllables (classified as belonging in
			a~fourth category, D), even though this opposition never existed on those syllables at any historical stage.
			
			In \is{Praguean phonology}Praguean phonology, phonemes are defined by their relations of opposition to one another, and features are assigned to phonemes on the basis of these relations. If two phonemes share a~set of features not found in any other phoneme, the opposition is bilateral and neutralizable; if not, it is multilateral and non-neutralizable. Stated differently, only where an opposition exists is a~feature phonologically contrastive. An
			opposition is considered neutralized whenever one of its members does not occur in a~particular environment. The output
			of \isi{neutralization} is referred to as an \is{archiphoneme|textbf}archiphoneme. For book-length treatments of the twin notions of \isi{neutralization} and
archiphoneme, see \citet{akamatsu_theory_1988} and \citet{silverman2012}.
			
			Archiphonemic notations are not employed in the present volume, for reasons that have to do with legibility. Devising specific symbols for archiphonemes is easier for \isi{International Phonetic Alphabet} symbols that happen to coincide with Roman letters, as using capitals is an option; %Archiphonemes may be indicated by capital letters, 
            but this won't work for all \isi{International Phonetic Alphabet} symbols, a~number of which do not derive from Roman letters. %Moreover, archiphonemic notations are more abstract than notations
			%containing phonetic symbols: interpreting archiphonemic notations requires a~knowledge of the language's phonotactics. %Concerning the latter, in phonological transcription, representing archiphonemes becomes visually cumbersome in cases where positions of \isi{neutralization} are more numerous than positions of relevance. 
			
			From a~dynamic point of view, gaps in the inventory of syllables can provide structural clues to past
			changes and current tensions within the phonological system.
			

			\subsection{Combinations of a~dental stop and /\ipa{æ}/ vowel seem recent}
			\label{sec:combinationsofadentalstopandvowelareprobablyrecent}
			
			Combinations of a~dental stop followed by /\ipa{æ}/ are scarce. The sole example for
			/\ipa{dæ}/ is /\ipa{læ˧dæ˧qæ˥}/ ‘armpit’. The two examples for /\ipa{tæ}/ are /\ipa{tæ˧ɻæ˩}/ ‘Adam’s
			apple, oesophagus’ and /\ipa{tæ˧pv̩˩}/ ‘thin, skinny’, while the only example for /\ipa{tʰæ}/ is
			/\ipa{tʰæ˧ɻæ˩}/ ‘book’. With the exception of /\ipa{tæ˧pv̩˩}/ ‘thin, skinny’, all of these can plausibly be attributed to \isi{vowel harmony}. The word for ‘book’, /\ipa{tʰæ˧ɻæ˩}/, corresponds to \ili{Laze} /\ipa{tʰɑ˧ɹ˧}/ and \ili{Naxi}
			/\ipa{tʰe˧ɣɯ˧}/; the vowel \is{comparative method (historical linguistics)}correspondence /\ipa{e:æ:ɑ}/ is otherwise unattested, reinforcing the
			hypothesis that \isi{vowel harmony} or some other \is{exceptions}exception-causing factor was at play here. Similarly, Yongning Na
			/\ipa{tæ˧ɻæ˩}/ ‘Adam’s apple, oesophagus’ corresponds to Labai Na /\ipa{tɑ˧ɻ̩˧}/, again presenting an irregular
			\is{comparative method (historical linguistics)}correspondence, as the regular correspondences are simply \ipa{æ::æ} and \ipa{ɑ::ɑ}. 
			
			
			\subsection{A marginal combination: Dental stop plus /\ipa{ɤ}/}
			\label{sec:dentalstopplus}
			
			%Few words contain a~dental stop plus /\ipa{ɤ}/. 
            The only two attested combinations of a dental stop followed by /\ipa{ɤ}/ are /\ipa{dɤ}/ and
			/\ipa{tɤ}/, with no occurrence of /\ipa{tʰɤ}/. Lexical examples are extremely rare. One is the extra-distal \is{demonstratives}demonstrative /\ipa{dɤ˩-qo˧}/ ‘way over there’, analyzed in \sectref{sec:caseswhereintonationinteractswiththetonalstring}. 
            %It is realized as either /\ipa{dɤ˧˥-qo˧}/ or /\ipa{dɤ˥˧-qo˧}/, where the second syllable is the locative /\ipa{qo}/ (also found in /\ipa{ʈʂʰɯ˧-qo˧}/ ‘here’ and /\ipa{tʰv̩˧-qo˧}/ ‘there’). The pitch of the first syllable indicates relative distance: a~mild rise, which could be transcribed as /\ipa{dɤ˧˥-qo˧}/, points to a~less distant place than a~realization with a~super-high, decreasing pitch, which could be transcribed as /\ipa{dɤ˥˧-qo˧}/. The same phenomena are observed for /\ipa{dɤ˧˥tʰv̩˧qo˧}/{\kern2pt}\ipa{≈}{\kern2pt}/\ipa{dɤ˥˧tʰv̩˧-qo˧}/
			% (same meaning, with added distal \is{demonstratives}demonstrative) and
			% /\ipa{dɤ˧˥tʰv̩˧-gi\#˥}/{\kern2pt}\ipa{≈}{\kern2pt}/\ipa{dɤ˥˧tʰv̩˧-gi\#˥}/ ‘that side, way over there’ (for further details, see \sectref{sec:caseswhereintonationinteractswiththetonalstring}). 
            The expressivity
			of extra-distal phrases goes a~great distance towards explaining phonemic and tonal \is{irregularities}oddities in their
			first syllable, where expressivity is strongest.
			
			All other instances of /\ipa{dɤ}/ and /\ipa{tɤ}/ occur before /\ipa{ɻ̩}{\kern2pt}/: in /\ipa{hṽ̩˧-dɤ˧ɻ̩\#˥}/ ‘clumsy’, /\ipa{õ˧-dɤ˧ɻ̩˧}/ ‘fundamentally’ (from /\ipa{õ˧˥}/
			‘(one)self’), /\ipa{ʂɯ˧-tɤ˧ɻ̩˧}/ ‘smooth (e.g.~carefully planed wood)’, and /\ipa{dʑɯ˩-tɤ˩ɻ̩˥}/
			‘humid, moist’ (from /\ipa{dʑɯ˩}/ ‘water’). These words contain a~phonetic sequence resembling a~trilled rhyme. Initially, these were 
			transcribed as /\ipa{dr̩}/ and /\ipa{tr̩}/, adding another rhotic rhyme to the inventory. This analysis appears correct for the phonological system of speaker F5 (F4’s
			daughter-in-law): when repeating the phrase [\ipa{hṽ̩˧dr̩˧{$\sim$}hṽ̩˧dr̩˧-zo˥}] ‘clumsily’ very slowly, she
			still realizes the phonetic sequence [\ipa{dr}] as a~single syllable. By contrast, in the speech of Mrs.\ Latami (F4), the phonetic realization is slightly less packed together, ranging between [\ipa{dər}] and [\ipa{dəɻ}{\kern2pt}] for the syllable with a~voiced initial, and between [\ipa{tər}] and [\ipa{təɻ}{\kern2pt}] for the syllable with an unvoiced initial. 
            
            A~phonological argument demonstrating that there are two syllables, not one, comes from /\ipa{ho˧dʑɯ˧tɤ˥ɻ̩˩}/ ‘paste, starch’, literally
			‘watery gruel’, \is{derivation!morphological}derived from /\ipa{ho˥}/ ‘porridge’ and /\ipa{dʑɯ˩-tɤ˩ɻ̩˥}/ ‘humid, moist’. This noun
			bears an H tone on its penultimate syllable and an L tone on its final syllable. Since no falling tone (HL, HM or ML) ever occurs on a~single syllable in Yongning Na, this tonal pattern supports the conclusion that there are two syllables in this sequence: /\ipa{tɤ.ɻ̩}{\kern2pt}/. 
			
			Interestingly, the two forms containing /\ipa{tɤ}/ have \is{variants}variants with /\ipa{dɤ}/:
			/\ipa{ʂɯ˧-dɤ˧ɻ̩˧}/ for ‘smooth’ and /\ipa{dʑɯ˩-dɤ˩ɻ̩˥}/ for ‘humid, moist’. However, the reverse is not true:
			the two forms containing /\ipa{dɤ}/ do not have \is{variants}variants with /\ipa{tɤ}/~-- \ipa{$\ddagger${\kern2pt}hṽ̩˧-tɤ˧ɻ̩\#˥} for ‘clumsy’ and \ipa{$\ddagger${\kern2pt}õ˧-tɤ˧ɻ̩˧} for ‘fundamentally’ are unacceptable. This suggests
			that /\ipa{-dɤ.ɻ̩}{\kern2pt}/ was once a~\is{suffixes}suffix used to \is{derivation!morphological}derive \is{adjectives}adjectives (also employed adverbially).
			However, it must have ceased to be productive long ago, as two of the four examples
			underwent a~separate phonetic evolution. It may never have been highly productive to begin with.
			

			\subsection{After alveolopalatals, is the rhyme /\ipa{o}/ or /\ipa{jo}/?}
			\label{sec:afteralveolopalatalstwooptionsforanalysisand}
			
			The syllables transcribed as /\ipa{tɕʰo}/, /\ipa{tɕo}/ and /\ipa{dʑo}/ could also be analyzed as
			/\ipa{tɕʰjo}/, /\ipa{tɕjo}/ and /\ipa{dʑjo}/, with a~/\ipa{-jo}/ rhyme. The \is{comparative method (historical linguistics)}correspondence between
			Na /\ipa{dʑo}/ and \ili{Naxi} /\ipa{gy}/ (phonetically: [\ipa{ɟy}]) suggests that the initial in Na became
			palatalized under the influence of a~following high front vowel or glide. From a~synchronic point of view,
			however, it appears more appropriate to transcribe these syllables as consisting of an alveolopalatal
			initial followed by a~back vowel.
			
			
			\subsection{Phonemic status of the~retroflex nasal}
			\label{sec:apossiblereanalysisdispensingwithaphonemicretroflexnasal}
			
			Two instances of the syllable /\ipa{ɳv̩}/ are attested: /\ipa{ɳv̩˥}/ ‘to sniff; to get to know (news)’~-- frequently
			used in the negative, as in /\ipa{mɤ˧-ɳv̩˥}/ ‘[I] don’t know’~-- and /\ipa{ɳv̩˩˧}/ ‘moth'. The syllable /\ipa{ɳv̩}/ contrasts with /\ipa{nv̩}/,
			exemplified by~/\ipa{nv̩˥}/ ‘to bury’. 
            
            An alternative analysis would be as [\ipa{ɳɻ̩}{\kern2pt}], phonemically /\ipa{nɻ̩{\kern2pt}}/. Under this view, there would be no need to posit %in which case one could dispense with positing 
            a~phonemic retroflex nasal /\ipa{ɳ}{\kern1pt}/ contrasting
			with /\ipa{n}/: retroflex realizations would be seen as conditioned by a~following /\ipa{ɯ}/ or /\ipa{ɻ̩}{\kern2pt}/, with 
			the combinations /\ipa{nɯ}/ and /\ipa{nɻ̩}{\kern2pt}/ phonetically realized as [\ipa{ɳɯ}] and [\ipa{ɳɻ̩}{\kern2pt}],
			respectively. In the absence of a~phonemic opposition, and given the phonetic proximity between 
			these two rhymes in a~retroflex context, this interpretation is not absurd. 
            
            Following the same logic, one might further reinterpret /\ipa{ɖv̩}/, /\ipa{ʈv̩}/ and /\ipa{ʈʰv̩}/ as /\ipa{dɻ̩}{\kern2pt}/,
			/\ipa{tɻ̩}{\kern2pt}/ and /\ipa{tʰɻ̩}{\kern2pt}/. This does not, however, yield a~genuine simplification of the system, since a~phonemic opposition between dentals
			and retroflexes is independently attested before other vowels (e.g.~/\ipa{ʈi}/ vs.\ /\ipa{ti}/). 
            
            The choice to transcribe
			these syllables as /\ipa{ɳv̩}/ reflects my auditory impression that, in the current state of the language, the rhyme
			in question is closer to [\ipa{v̩}] than to [\ipa{ɻ̩}{\kern2pt}].
			
			\subsection{The palatal nasal}
			\label{sec:palatalnasal}
			
			The palatal nasal [{\kern2pt}\ipa{ɲ}] only appears in the syllable [{\kern2pt}\ipa{ɲi}], suggesting the possibility of a~reanalysis as an allophone of one of the other nasal initials in the system: /\ipa{m}/, /\ipa{n}/, /\ipa{ɳ{\kern1pt}}/ or /\ipa{ŋ}/. From a~language-independent perspective, the most plausible analysis is phonemicization as /\ipa{ŋi}/, in view of the well-documented palatalizing effects of high front vowels on velar consonants. 
			
			This analysis is possible in principle, given the absence of a~syllable [\ipa{ŋi}] in the syllabic inventory of Yongning Na. In \ili{Naxi}, analysis of palatal initials as allophones of velars is an attractive solution, because it applies throughout the system: [\ipa{cʰi}], [\ipa{ci}], [{\kern2pt}\ipa{ɟi}], [{\kern2pt}\ipa{ɲɟi}] and [{\kern2pt}\ipa{ɲi}] can be analyzed as /\ipa{kʰi}/, /\ipa{ki}/, /\ipa{gi}/, /\ipa{ŋgi}/ and /\ipa{ŋi}/ \citep[14]{michailovskyetal2006}. However, detailed examination of the \ili{Naxi} lexicon shows that this analysis amounts to an internal \is{comparative method (historical linguistics)}\is{reconstruction!internal}reconstruction, rather than a~synchronic phonemic analysis, as expressive coinages have since filled structural gaps that were initially created by the palatalization of velars \citep[7]{michaudetal2015c}. 
            
            In Na, it is less tempting to phonemicize [{\kern2pt}\ipa{ɲi}] as /\ipa{ŋi}/, as phonemic combinations of velar initials with the vowel /\ipa{i}/ do occur and are not strongly palatalized. The transcription adopted here therefore remains close to the surface form: /{\kern1pt}\ipa{ɲi}/.
			
			
			\subsection{Syllables introduced by Mandarin borrowings}
			\label{sec:syllablesintroducedbychineseborrowings}
			
			\ili{Mandarin} borrowings have the potential to bring considerable changes to the phonotactic patterns of Na
			syllables; in particular, they introduce numerous new combinations of vowels with glides. The
			overall situation is comparable to that observed in \ili{Naxi}. A~{Naxi} speaker from Dadong \zh{大东}, He Likun \zh{和丽昆}, compiled
			an inventory of the syllables present in his own speech, and found that recent \ili{Mandarin} \isi{loanwords}
			account for about 150 of the syllables he uses when speaking \ili{Naxi} \citep{michaudetal2015c}. He
			Likun is essentially bilingual in {Mandarin}, a~situation that is common among younger speakers of both {Naxi} and Na. 
            
            By contrast, the main consultant for Yongning Na is thirty-eight years older than He Likun, and
			her knowledge of {Mandarin} is limited. In her speech, there is a~tension between a~general tendency to
			integrate loanwords into the Na phonological system and occasional efforts at getting closer to the “correct” pronunciation in \ili{Mandarin} (either \il{Mandarin!Southwestern}Southwestern
			{Mandarin} or \il{Mandarin!Standard}Standard {Mandarin}, depending on the interlocutor). This source of instability needs to be acknowledged when transcribing {Mandarin} \is{loanwords}borrowings: they
			typically possess both (i)~an adapted form that conforms to Na phonotactics and phonetics, and
			(ii)~forms that are closer to \ili{Mandarin}, and which depart from Na phonotactics and phonetics. 
            
            For
			instance, in the absence of a~rounding opposition among front vowels, \ili{Mandarin} [\ipa{y}] is borrowed as
			[\ipa{i}]: the \ili{Mandarin} \textit{zájūn} \zh{杂菌} [\ipa{tsa.tɕyn}] ‘mixed mushrooms’ is realized as 
			/\ipa{tsɑ˩tɕi˩}/. However, the consultant is aware of the phonetic distance between [\ipa{tɕyn}] and
			[\ipa{tɕi}], and is able to make efforts to round the front vowel, producing forms that get close to
			[\ipa{tɕɥe}] or [\ipa{tɕɥi}]. 
			
			These competing pressures~-- towards adaptation to the Na system, and towards phonetic fidelity to \ili{Mandarin}~-- can sometimes be observed within the lexicon. The \is{loanwords}Chinese word for ‘Westerners, foreigners’, \textit{yáng} \zh{洋}, appears in three forms in Yongning Na: /\ipa{jɤ˩}/, /\ipa{je˩}/ and /\ipa{ʐe˩}/. The most frequent form is /\ipa{jɤ˩}/, as in /\ipa{jɤ˩ho˧}/ ‘matches’ (from \textit{yánghuǒ} \zh{洋火}) and /\ipa{jɤ˩jo\#˥}/ ‘potato’ (from \textit{yángyù} \zh{洋芋}). The second form, /\ipa{je˩}/, occurs in /\ipa{je˩ʐe˧}/, from \textit{yángrén} \zh{洋人}
			‘Westerner’. In turn, this word appears in a~modified form in ‘wild cotton flowers’, /\ipa{ʐe˩ʐe˧-bæ˩bæ˩}/. This is a~distortion of /\ipa{je˩ʐe˧-bæ˩bæ˩}/, literally ‘Westerners’ flower’, which
			remains an acceptable \is{variants}variant. The borrowed syllable /\ipa{je}/ in /\ipa{je˩ʐe˧}/ \zh{洋人}
			‘Westerner’ is Naicized through identification with a~syllable that is well-attested in Na, /\ipa{ʐe}/,
			taking advantage of its presence in the immediate vicinity: as the second syllable of ‘wild cotton flowers’. Unsurprisingly, the \is{stylistics}stylistic effect of Naicization is that /\ipa{ʐe˩ʐe˧-bæ˩bæ˩}/ sounds
			more local, drawing on a~sense of in-group closeness among Na speakers, whereas the more faithful rendering of the \ili{Mandarin} original, /\ipa{je˩ʐe˧-bæ˩bæ˩}/, tends to sound more modern and outward-looking.
			
			
			\section{Articulatory reduction: Reduced forms and their lexicalization}
			\label{sec:articulatoryreductionreducedformsandtheirlexicalization}
			
			
			Phenomena of articulatory reduction pave the way for the \isi{lexicalization} of new forms, sometimes
			resulting in the emergence of new syllabic combinations. Some salient examples are presented below.
			
			The verb /\ipa{ʝi˥}/ ‘to do’ is prone to reduction. In particular, reduction is well on its way to becoming
			\is{lexicalization}lexicalized in the expression /\ipa{gɯ˩ ʝi˥}/ ‘really, truly’ (from /\ipa{gɯ˩}/ ‘authentic, true’), which is realized as
			[\ipa{gi˩˥}] except when hyperarticulated. Phonetic reduction is also strong in the object-verb sequence /\ipa{ə˧tso˧	ʝi˧}/ `to do what?' (\textsc{interrog}:what-to\_do), commonly
			realized in a~hypo-articulated form that can be approximated as [\ipa{ə˧tse˧}]~-- as in /\ipa{no˧ {\kern2pt}|{\kern2pt} ə˧tso˧
				ʝi˧-bi˧}/ ‘What are you going to do?’
			(2\textsc{sg}-\textsc{interrog}:what-to\_do-\textsc{imm.fut}), frequently contracted to \ipa{no˧ {\kern2pt}|{\kern2pt} ə˧tse˧ bi˧}.
            
            A factor preventing lexicalization of the simplified form [\ipa{ə˧tse˧}] may be the quasi-homophony of this reduced form with another interrogative, /\ipa{ə˧tse˥\$}/ `why'. The phonological closeness of /\ipa{ə˧tse˧}/ and /\ipa{ə˧tse˥\$}/ could lead to confusion. This risk of homophony might create pressure towards retarding the lexicalization of a simplified form /\ipa{ə˧tse˧}/ that would replace /\ipa{ə˧tso˧ ʝi˧}/. 
            
            The {relativizer} /\ipa{hĩ{\kern0.7pt}˥}/ is articulated far more weakly than the noun
			/\ipa{hĩ{\kern0.7pt}˥}/ ‘human being, person’. The initial fricative is often strongly reduced, and frequently voiced
			throughout. Before a~voiced stop, realizations as a~nasal consonant (a~nasal stop) are observed, as in
			\figref{fig:anillustrationofthereductionoftherelativizertoanasalconsonant}, which shows a~spectrogram of the utterance in (\ref{ex:reallyhappy}). 
			
						\begin{exe}
							\ex
							\label{ex:reallyhappy}
							\ipaex{ɖwæ˧˥ {\kern2pt}|{\kern2pt} fv̩˧-hĩ˧ ɖɯ˧-v̩˧ ɲi˩!}\\
							\gll ɖwæ˧˥	fv̩˧	-hĩ{\kern0.6pt}˥	ɖɯ˧		v̩˧		ɲi˩\\
							very	happy	\textsc{rel}	one		\textsc{clf}.individual		\textsc{cop}\\
							\glt ‘(S)he is really happy!’ (Field notes.)
						\end{exe}
						
			The spectrogram shows that the sequence /\ipa{hĩ˧ ɖɯ˧}/ is phonetically realized close to [\ipa{nɖɯ˧}], as though the syllable /\ipa{hĩ{\kern0.6pt}˥}/ were realized as a~prenasalization of the following stop. This is not a~categorical process: to claim that /\ipa{hĩ{\kern0.6pt}˥}/ is categorically replaced by /\ipa{n}/ in this context would be untenable, running up against insuperable phonotactic difficulties. Yongning Na does not have a~series of prenasalized stops (and thus has no permissible /\ipa{nɖ}{\kern2pt}/ initial consonant), and /\ipa{n}/ on its own is not a~well-formed syllable, as it lacks both a~rhyme and a~tone. The reduction in question is therefore phonetic in nature. Annotating the reduced realization of /\ipa{hĩ{\kern0.6pt}˥}/ in \figref{fig:anillustrationofthereductionoftherelativizertoanasalconsonant} as [\ipa{n}] is a~convenient shorthand, not a~phonemic reanalysis. Underlying specifications can continue to shape articulatory gestures, even when the primary phonetic targets are no longer being reached \citep[272]{nolan1992}. 
            
            With this caveat in place, it is clear that the reduction is strong. Roselle Dobbs (p.c.\ 2014) notes that some younger speakers are unaware that the
			{relativizer} /\ipa{hĩ{\kern0.6pt}˥}/ is present in contexts such as the one shown in \figref{fig:anillustrationofthereductionoftherelativizertoanasalconsonant}, and tend to omit it altogether.
			
			\begin{figure}%[t]
				\includegraphics[width=.9\textwidth]{figures/fvndeevnhi/fvndeevnhiF5.pdf}
				\caption{An illustration of the reduction of the relativizer /\ipa{hĩ}/ to a~nasal consonant. Top: phonetic transcription; bottom: phonemic transcription. Speaker: F5.}
				\label{fig:anillustrationofthereductionoftherelativizertoanasalconsonant}
			\end{figure}
			
			
			The phrase /\ipa{ʈʰæ˧mi˧-ɳɯ˩}/ ‘really, actually’ is generally reduced to a~\is{monosyllables}monosyllable with a~\is{lengthening}long rhyme, which can be approximated as [\ipa{ʈʰææ̃˧}], occasionally preserving a~trace of the final L tone of the full
			expression: [\ipa{ʈʰæ˧æ̃˩}]. In the absence of a~vowel length opposition in Na, however, the reduced form appears unlikely to become lexicalized.
			
			The disyllabic forms produced by the combination of \isi{demonstratives} with the associative plural \is{clitics}clitic /\ipa{=ɻæ˩}/ are also strongly coalescent. These are /\ipa{ʈʂʰɯ˧{\linebreak}=ɻæ˥\$}/ ‘these things, this sort of things’ 
            %from the proximal \is{demonstratives}demonstrative
            (from proximal /\ipa{ʈʂʰɯ˥}/) and /\ipa{tʰv̩˧=ɻæ˥\$}/ ‘those things’ (from distal /\ipa{tʰv̩˥}/). Regressive vowel
			harmony is strong, giving rise to realizations resembling [\ipa{ʈʂʰæ˧=ɻæ˥}], e.g.~in
			\textit{Caravans.153} \pandoi{0004530\#153} and \textit{Agriculture.109} \pandoi{0004440\#109}. Together with the weakening of the approximant /\ipa{ɻ}{\kern2pt}/ (which is vowel-like to begin with), this often results in forms approaching a~\is{monosyllables}monosyllable, [\ipa{ʈʂʰæ˧˥}] or
			[\ipa{ʈʰæ˧˥}]. Examples include \textit{Caravans.160} \pandoi{0004530\#160},
			\textit{165}; \textit{Mountains.83} \pandoi{0004573\#83}, \textit{109}; \textit{Funeral.190} \pandoi{0004571\#190}; and \textit{BuriedAlive3}.{\linebreak}\textit{50} \pandoi{0004538\#50}.
			
			The exclamative final particle /\ipa{wɤ˧}/, which conveys a~sense of obviousness, tends to fuse with a~preceding /\ipa{-ɲi˩}/ (expressing certitude). In all fourteen occurrences found in F4’s transcribed narratives, the combination /\ipa{-ɲi˩ wɤ˧}/ undergoes tone lowering: the final M tone is depressed to L, in accordance with Rule~5 (see \sectref{sec:alistoftonerules}), and the sequence is realized phonetically close to a~monosyllable:
			\mbox{[\ipa{-ɲo˩}]}. This phenomenon is also highly frequent in the speech of M21.
			

			\section{Expressive coinages and phonostylistic observations}
			\label{sec:expressivecoinagesandmore}
			
% Indexing whole subsection for 'expressive coinages'
\is{expressive coinages|(}

To conclude this appendix on Na phonemes (vowels and consonants), it is worth mentioning expressive coinages, along with some \is{phonostylistics}phonostylistic observations.

Ideophonic and expressive phenomena are attested in languages the world over \citep{marsault2024_ideophones}. The seemingly iconoclastic idea of a reversal of perspective, whereby elements traditionally considered marginal become the focus of attention, offers a~stimulating and heuristic shift in viewpoint. In his work on interjections, Mark Dingemanse refers to ``a~potentially radical reversal: from interjections at ‘the outskirts of real language’ \citep{muller_lectures_1861} to interjections at the heart of language" \citep[258]{dingemanse_interjections_2024}. This perspective is all the more relevant for work on Yongning Na as expressive coinages are reported to be especially abundant in Asian languages:

\begin{quotation}
    The languages of Mainland Southeast Asia are resplendent with elaborate grammatical resources for fashioning elaborative expressions that convey emotions, senses, conditions, and perceptions that enrich discourse -- both everyday and ritualized -- and are grammatical works of art. Over time, a sizeable terminological lexicon has been created to categorize or classify these resources, including echo words, phonaesthetic words, chameleon affixes, chiming derivatives, onomatopoeic forms, ideophones, and most notably expressives. \parencite[1]{williams_aesthetics_2014}
\end{quotation}

Such phenomena have been studied in detail in Vietnamese \parencite{brunelleetal2014,Pham_Alves_Vietnamese_2024} and in Japhug \parencite{jacques2013c}. Their exploration in Na is still at an early stage, but represents a~promising direction for further work.
           
			\subsection{Onomatopoeia and ideophones}
			\label{sec:onomatopoeics}


\is{onomatopoeia|(}% Mark for indexing whole passage 

{\largerpage[-2]} % Added on April 22nd, 2025

			Onomatopoeia constitute one dimension of expressive (phonaesthetic) coinages, %{\linebreak}
            which also include interjections,
			calling sounds, and ideophones. All of these categories have interesting morphological and phonological
			specificities. “Of the 446 known onsets in \ili{Japhug}, forty-five clusters (including thirty-five
			two-consonant and eleven three-consonant clusters) are exclusively attested in ideophones or
			ideophonic verbs” \citep[264]{jacques2013c}. Expressive coinages tend to have a~lilt of their own, but they
			are also subject to a~continuous pressure towards assimilation into the language's phonological system: 
            % tending to their integration into the language's phonological categories: 
            “ideophones fill gaps in the distribution of segments
			within rhymes that have been caused by sound changes” \citep[267]{jacques2013c}. A~classical instance of such structural \is{gap-filling}gap-filling is found in
			\ili{Vietnamese}, where the /\ipa{ɔŋ}/ and /\ipa{oŋ}/
			rhymes underwent an evolution whereby lip rounding was shuffled from the vowel to the consonant, yielding a~final
			labial closure: the resulting forms can be approximated as [\ipa{ʌɔŋ͡m}] and [\ipa{ɤoŋ͡m}]. The phonotactic slots left vacant by this phonetic evolution came to be filled by onomatopoeic coinages, and by loanwords (\citealt{haudricourt1952b}; \citealt[21]{henderson1985}; \citealt[143]{michaud2004a}). 
			
{\largerpage[-2]} % Added on April 22nd, 2025

			He
			Likun, a~native speaker of \ili{Naxi}, examined each cell in a~table of possible combinations of
			initials and rhymes in the Pianding dialect of \ili{Naxi}, assessing (by introspection) whether the
			combination was attested, and if so, in which words. The results were supplemented by sifting through a~word list
			of approximately 3,000 entries. This study identified more than fifteen syllables attested exclusively in
			onomatopoeic words \citep{michaudetal2015c}. 			
			
			Onomatopoeia are no less abundant in Na than in \ili{Naxi}. However, they are scarce in
			the set of transcribed narratives, as might be expected in relatively formal monologic speech. Other data
			collection methods, such as recording lively conversations, will be necessary in order to explore the full range of expressive phenomena found in Yongning Na. Three examples are presented below.
			
			\begin{enumerate}[label=(\roman*)]
				\item The noise produced by a~shock between two hard objects, such as the sound of an axe hitting a~tree trunk (‘Bang!’), is rendered as [\ipa{bõ}]. This syllable contravenes Na phonotactics, as nasal rhymes do not normally occur after stop consonants. 
				\item The onomatopoeia used for rumbling sounds, such as the noise of heavy loads being carried across a~wooden floor, or the noise of lorries, is a~prolonged [\ipa{ʐ}{\kern2pt}].                 
                This form differs in several respects from the lexical syllable /\ipa{ʐɯ}/. The latter surfaces with an apicalized vowel, as [\ipa{ʐʐ̩}{\kern2pt}]: the onset is more consonant-like, while the coda is more vowel-like. It is in view of this asymmetry that Chao Yuen-ren proposed the use of dedicated symbols for apicalized vowels (see \tabref{tab:apicalized}), rather than simply adding a~diacritic to the fricative to indicate its phonological status as syllable nucleus. In his system, the lexical syllable /\ipa{ʐɯ}/ would be transcribed phonetically as [\ipa{ʐʅ}{\kern1.4pt}], rather than [\ipa{ʐʐ̩}{\kern2pt}]. By contrast, in the onomatopoeic form representing a~rumbling sound, friction is sustained from beginning to end, hence transcription as [\ipa{ʐʐʐ}{\kern2pt}].

\item The hissing noise produced when water comes into contact with red-hot metal or incandescent wood (‘Pssshhh!’) is represented as [\ipa{ʈʂʰɻ̩}{\kern2pt}]. While the combination of initial and rhyme used to transcribe this form is also attested in lexical items (e.g.\ /\ipa{ʈʂʰɻ̩˧}{\kern2pt}/ `ploughshare'), its phonetic realization in this onomatopoeic context does not exactly match that of the corresponding syllable /\ipa{ʈʂʰɻ̩}{\kern2pt}/ in the standard lexicon. To reflect its special status, a~possible transcription is [\ipa{ʈʂʰɻɻɻ}{\kern2pt}].
%This sound is unlike the syllable /\ipa{ʐɯ}/. The latter is a~full-fledged syllable, which surfaces with an apicalized vowel, as [\ipa{ʐʐ̩}{\kern2pt}]: the beginning of the syllable is more consonant-like, and the end more vowel-like. This is the reason that Chao Yuen-ren advocated the use of special symbols for apicalized sounds (see \tabref{tab:apicalized}). In his system, the syllable would be transcribed as /\ipa{ʐʅ}{\kern2pt}/. In the onomatopoeic form for rumbling sound, on the other hand, friction is sustained from beginning to end, hence transcription as  [\ipa{ʐʐʐ}{\kern2pt}].

%\item The hissing noise of water that comes in contact with red-hot metal or incandescent wood (‘Pssshhh!’) is [\ipa{ʈʂʰɻ}{\kern2pt}]. The combination of initial and rhyme used to transcribe this onomatopoeia is attested in some lexical items, but its phonetic realization does not exactly match that of the syllable /\ipa{ʈʂʰɻ}{\kern2pt}/ of lexical items. To reflect this special status, a~possible transcription is [\ipa{ʈʂʰɻɻɻ}{\kern2pt}].
			\end{enumerate}
			
\is{onomatopoeia|)}% Mark for indexing whole passage 

%			Because the present volume's focus is on morphotonology, little attention has been devoted to the
%			rich linguistic field of expressives in Yongning Na. But ultimately, this field is not without
%			relevance to tone and \isi{intonation}.
%			
%			Investigation of this topic is
%			envisaged at the stage of experimental study of fine phonetic detail in Na tone and \isi{intonation}~--
%			a~topic about which some initial observations are set out in Chapter~\ref{chap:fromsurfacephonologicalformstophoneticrealizationintonationandtonalimplementation}.
			
			\subsection{Phonostylistic observations}
			\label{sec:phonostylisticobservations}
			\label{sec:liproundingandprotrusionwithdemonstrativeproximalvalue}
			\label{sec:palatalizationconveyingatenderemotion}

{\largerpage[-1]} % Added on April 22nd, 2025

			Expressivity is not confined to specific areas of the lexicon, such as ideophones. The “appeal function” of speech \citep{buhler1934} is constantly at play. The study of this function~-- examining how speakers shape their utterances so as to elicit a~certain response from the hearer~-- figures prominently in the research programme set out by \citet[14]{trubetzkoy1969} (original German edition: \citealt{trubetzkoy1939}), who coined the term
			\is{phonostylistics|textbf}‘phonostylistics’ (for an overview, see \citealt{leon1969}). The term ‘psycho-phonetics’, used by
			\citet{fonagy1983}, is broader and therefore arguably less well suited, although it has the
			advantage of highlighting the considerable breadth of this strand of research: the investigation of how phonetic detail contributes to conveying the communicative intentions of the speaker. If \isi{intonation} is “a symptom of how we
			feel about what we say and how you feel when you say it” \citep[1]{bolinger1989}, then {phonostylistics} is
			part and parcel of \is{intonation}intonational studies.
			
			This fascinating topic is best investigated through experimental phonetic research. As the
			present volume essentially focuses on lexical tone and morphotonology, the approved order of business entails
			postponing the study of expressive phenomena until the more central aspects of
			the language's structure have been elucidated. Detailed discussion is therefore deferred to
			\is{experimental phonetics}experimental phonetic studies to be conducted in the future. For now, let us simply mention two salient cases of phonetic
			modification of vowels and consonants for expressive effect in Na.
			
			
%			\begin{description}
%				\item[Lip rounding and protrusion with \is{demonstratives}demonstrative (proximal) value:] the vowel /\ipa{ɯ}/ has neither lip rounding nor lip protrusion. It acquires lip protrusion when the
%				phrase /\ipa{ʈʂʰɯ˧-ɭɯ˧}/ ‘this one’ (proximal \is{demonstratives}demonstrative plus generic classifier) is used as
%				a~real-world \is{demonstratives}demonstrative, pointing to an object within sight. The speaker’s face points in the
%				direction of the object, and lip protrusion functions as part of the gesture of pointing. It is
%				often accompanied by an upward movement of the chin, further reinforcing the pointing gesture.
%				\item[Palatalization conveying a~tender emotion:] the adjective /\ipa{ɳɯ˧ɕi˩}/ ‘lovely’ can be pronounced close to /\ipa{ni˧ɕi˩}/. This
%				child-speech-like \is{variants}variant has iconic value: palatalization, narrowing the vocal tract, is
%				associated with smallness \citep[22--23]{fonagy1983}. The realization of this cross-linguistic
%				tendency is facilitated in Na by the tendency towards regressive vowel harmony (\sectref{sec:anoteonvowelharmony}).
%			\end{description}
			
The first is \textit{lip rounding and protrusion with \is{demonstratives}demonstrative (proximal) value}. Lip pointing \citep{enfield2001} is used in Na society. When the lip-pointing gesture is employed during speech, lip rounding and protrusion become superimposed onto the speech production gestures. For instance, the vowel /\ipa{ɯ}/, which in its canonical form involves neither lip rounding nor lip protrusion, acquires lip protrusion when the phrase /\ipa{ʈʂʰɯ˧-ɭɯ˧}/ ‘this one’ (proximal \is{demonstratives}demonstrative plus generic classifier) is said while lip-pointing to an object within sight. 
				
The second case is that of \textit{palatalization conveying a~tender emotion}. The adjective /\ipa{ɳɯ˧ɕi˩}/ ‘lovely’ can be pronounced close to /\ipa{ni˧ɕi˩}/, a~\is{variants}variant reminiscent of child speech. This form has \is{iconicity}iconic value: palatalization, which narrows the vocal tract, is widely associated with smallness \citep[22--23]{fonagy1983}. The realization of this cross-linguistic association is facilitated in Na by the tendency towards regressive \isi{vowel harmony} (\sectref{sec:anoteonvowelharmony}).

%			\subsubsection*{Lip rounding and protrusion with \is{demonstratives}demonstrative (proximal) value}
%			\label{sec:liproundingandprotrusionwithdemonstrativeproximalvalue}
%			
%			
%			The vowel /\ipa{ɯ}/ has neither lip rounding nor lip protrusion. It acquires lip protrusion when the
%			phrase /\ipa{ʈʂʰɯ˧-ɭɯ˧}/ ‘this one’ (proximal \is{demonstratives}demonstrative plus generic classifier) is used as
%			a~real-world \is{demonstratives}demonstrative, pointing to an object within sight. The speaker’s face points in the
%			direction of the object, and lip protrusion functions as part of the gesture of pointing. It is
%			often accompanied by an upward movement of the chin, further reinforcing the pointing gesture.
%			
%			
%			
%			%subsec:2-4-6
%			\subsubsection*{Palatalization conveying a~tender emotion}
%			\label{sec:palatalizationconveyingatenderemotion}
%			
%			
%			The adjective /\ipa{ɳɯ˧ɕi˩}/ ‘lovely’ can be pronounced close to /\ipa{ni˧ɕi˩}/. This
%			child-speech-like \is{variants}variant has iconic value: palatalization, narrowing the vocal tract, is
%			associated with smallness \citep[22--23]{fonagy1983}. The realization of this cross-linguistic
%			tendency is facilitated here by the tendency towards regressive vowel \isi{assimilation} found in Na and
%			other \ili{Naish} languages.
%			
%			
			
			\subsection{Expressive uses of reduplication}
			\label{sec:thereduplicationofnonlexicalwords}
			
{\largerpage[-1]} % Added on April 22nd, 2025

			Reduplication serves various grammatical functions in Yongning Na, as it does in \ili{Naxi} \citep[30–33]{heetal1985}. Despite having neatly \is{grammaticalization}grammaticalized uses, such as imparting reciprocal meaning to verbs, it retains an expressive dimension, especially in its sporadic application to parts of speech other than verbs and nouns. This expressive function is reflected in irregular tone patterns~-- and thus this Appendix finally returns to the book's central topic: tone.
			
			The phrase /\ipa{qʰɑ˧~ɲi˧}/ ‘how many days’ reduplicates as /\ipa{qʰɑ˧~ɲi˧{$\sim$}qʰɑ˩~ɲi˩}/ ‘thus and so
			many days’ (\textit{Healing.29} \pandoi{0004540\#29}; the context is the following: a~priest of the Na religion diagnoses the
			number of days of rituals required to cure a~person’s illness). The resulting tone pattern differs from that of \is{numerals}numeral-plus-classifier phrases, where the expected surface form would exhibit M tone
			throughout the phrase (see Chapter~\ref{chap:classifiers}).
			
			The interrogative /\ipa{ə˧tso˧}/ reduplicates as /\ipa{ə˧tso˧{$\sim$}ə˧tso˥}/, again with a~tone pattern that diverges from expectations (\textit{Dog.48} \pandoi{0004442\#S48}).
			
			The word /\ipa{zo˧{$\sim$}zo˧-mv̩˧{$\sim$}mv̩˥}/ ‘thingummy’ looks a~lot like it could be
			the product of \isi{reduplication}, perhaps originating in a~playful manipulation of /\ipa{zo˧mv̩˥}/ ‘child’. A~more
			common word for ‘thing’ is /\ipa{tso˧tso\#˥}/, which may derive from reduplication of the
			nominalizer /\ipa{-tso}/. These two nouns are combined in
			/\ipa{tso˧{$\sim$}tso˧-zo˧{$\sim$}zo˧-mv̩˧{$\sim$}mv̩˥}/ ‘thingummies, stuff’,
			suggesting that they are currently perceived as having a~similar
			internal structure.
			
			The reduplicated form /\ipa{zɯ˧{$\sim$}zɯ˧}/ ‘life, existence’ is more frequent than the 
			{monosyllabic} variant /\ipa{zɯ˧}/, although both are in common use.
			
			The sequence /\ipa{lv̩.lv̩}/ in /\ipa{bi˧-lv̩˧{$\sim$}lv̩˥}/ ‘snowflake’ and
			/\ipa{dzo˧-lv̩˧{$\sim$}lv̩˥}/ ‘hailstone’ looks like a~reduplicated form of the classifier for kernels, /\ipa{lv̩˧}/ \citep[xxxiv]{lidz2010}.
			
			The phrase /\ipa{tɕɤ˧{$\sim$}tɕɤ˧}/ ‘right at the moment that{\dots}’ also has the appearance of a~reduplicated form, although a \textit{simplex} (non-reduplicated) counterpart could not be identified.
			
			A set of four-syllable onomatopoeic expressions of the form ABAB was observed, all displaying an L.L.M.M tone pattern. Examples include:
            
            \begin{itemize}
                \item /\ipa{tsɯ˩qwæ˩{$\sim$}{\allowbreak}tsɯ˧qwæ˧}/ ‘crashing sound, for instance the sound of timber falling down’ (\textit{Housebuilding.243} \pandoi{0004448\#243}),
                \item /\ipa{zɯ˩gɯ˩{$\sim$}{\allowbreak}zɯ˧gɯ˧}/ ‘boom!’ (sound of heavy shock against a~door: \textit{Tiger.15} \pandoi{0004444\#S15}),
                \item /\ipa{ʐɯ˩ʐɤ˩{$\sim$}{\allowbreak}ʐɯ˧ʐɤ˧}/ ‘sound of tearing leaves to pieces’ (\textit{FoodShortage2.37} {\linebreak}\pandoi{0004442\#S37}),
                \item /\ipa{ɕi˩hwɑ˩{$\sim$}{\allowbreak}ɕi˧hwɑ˧}/ and /\ipa{ʐɯ˩ʁæ˩{$\sim$}{\allowbreak}ʐɯ˧ʁæ˧}/, both describing the dizziness of a~character
			under a~dazzling moonlight (\textit{Reward.17} \pandoi{0004446\#S17} and \textit{Reward.68} \pandoi{0004446\#S68}).
\end{itemize}
            
          Among these ABAB expressions, only /\ipa{ʐɯ˩ʁæ˩{$\sim$}{\allowbreak}ʐɯ˧ʁæ˧}/ has an identifiable \textit{simplex} counterpart, namely /\ipa{ʐɯ˧ʁæ˩}/ `drunk'.

            
% End of mark for indexing whole subsection for 'expressive coinages'
\is{expressive coinages|)}

