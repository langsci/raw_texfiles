\chapter{The lexical tones of nouns}
\label{chap:thelexicaltonesofnouns}

% Temporarily widen the epigraph block
{\setlength{\epigraphwidth}{.57\textwidth}

\epigraph{The fundamental problem of a tonological description is to arrive, for each tonal type in the language, at an underlying tonal structure that does not merely record what distinguishes the realizations of the different types in a more or less arbitrarily chosen context, but that makes it possible to predict the full set of combinatorial properties characterizing each type. 

{\medskip}\textit{Original text:} Le problème fondamental d'une description tonologique est de déterminer, pour chaque type tonal existant dans la langue, une structure tonale sous-jacente qui ne se borne pas à enregistrer ce qui différencie la réalisation des différents types que connaît la langue dans un contexte plus ou moins arbitrairement choisi, mais qui permette de prédire l'ensemble des propriétés combinatoires qui caractérisent chaque type.}{\citep[179]{creissels1994}}

{\largerpage}

This chapter examines the lexical tones of nouns. It is customary in tonal studies to proceed “from the tones of nouns to the general organization of the system” (\citealt{riallandetal1989}; see also \citealt[526-527]{hyman2014}). In Yongning Na, tonal oppositions are partially neutralized when words are spoken in isolation.\is{form!in isolation} This casts a~subtle veil over the tonal categories, hiding some of them from casual observation. These cases of \isi{neutralization} make it necessary to consider how tones behave in context, making this chapter a~gateway to the Yongning Na tone system as a~whole.

The chapter follows an analytical progression. It starts out from a~static inventory of tone patterns
across domains of varying lengths and gradually moves towards a~systematic analysis. This approach mirrors, as closely as possible, the course of analysis during fieldwork: building up from surface
data. The aim is to allow the reader to assess the argumentation step by step and to consider
possible alternative analyses, rather than presenting a~complete analysis from a~top-down perspective.


\section{A static inventory of tone patterns}
\label{sec:astaticinventoryoftonepatterns}

Words spoken in isolation\is{form!in isolation} (often termed \textit{citation forms}) provide the starting point in the earliest stages of
fieldwork. \tabref{tab:tonepatternsattestedovermonosyllabicnounsspokeninisolation} presents
an overview of the tone patterns found in \is{monosyllables}monosyllabic nouns spoken \is{form!in isolation}in isolation. Due to the relatively small number of \is{monosyllables}monosyllabic
nouns in the language, a strict minimal set (nouns distinguished solely by tone) could not be identified.

\begin{table}%[t]
  \caption{Tone patterns attested over monosyllabic nouns in isolation.}
\begin{tabularx}{\textwidth}{ Q Q l }
  \lsptoprule
	phonetic realization & preliminary label & example\\\midrule
	non-rising, non-low & M ? H ? HM ? & /\ipa{ʐwæ}/ ‘horse’\\
	low-rising & LM ? LH ? & /\ipa{bo}/ ‘pig’\\
	mid-rising & MH ? & /\ipa{ʈʂʰæ}/ ‘deer’\\
\lspbottomrule
\end{tabularx}
\label{tab:tonepatternsattestedovermonosyllabicnounsspokeninisolation}
\end{table}

At this initial stage, the essential
information is that provided in the leftmost column in \tabref{tab:tonepatternsattestedovermonosyllabicnounsspokeninisolation}, describing the three patterns as
follows: a~non-rising, non-low pattern; a~low-rising pattern; and a~mid-rising pattern.
The second column %of \tabref{tab:tonepatternsattestedovermonosyllabicnounsspokeninisolation} 
offers preliminary labels for these patterns, using level tones:
L(ow), M(id), H(igh), and their combinations. These labels merely suggest, at this point, how the categories might be conceived in terms of %L, M, and H 
phonological levels. Justification for employing a~level-tone analysis
comes from morphophonological alternations in which these tones participate; evidence for this will be provided later on in the course of the analysis. The question marks in the ‘preliminary label’ column emphasize that these possible labels were proposed on a~first pass. 

%Some will be modified later on in the course of the analysis, as will be explained below. 

{\largerpage} % Added on April 27th, 2025

The three surface\is{form!surface} patterns are the same for monosyllables that belong to other word classes, such as verbs. The restrictions on the tones of monosyllables spoken \is{form!in isolation}in isolation are the following. 

\begin{enumerate}[label=(\roman*), itemsep=0pt]
	%\item[(i)] Only two tones contrast on the first syllable: low and non-low. There can be no \is{tonal contour}contour on the first syllable. 
	%\item[(ii)] A~Mid tone cannot be followed by a~low-rising tone.
	%\item[(iii)] A~disyllable cannot be low throughout, any more than a~\is{monosyllables}monosyllable can. 
	%\item[(iv)] There is no contrast between a~low+mid pattern and a~low+high pattern. 
	\item There are no examples of contrastive falling contours.
	\item There is no opposition between a~high tone and a~mid tone:
	only one type of non-low, non-rising tone is observed. Its realizations span the entire upper part
	of the tonal space, varying from mid to high, with a~flat or falling \is{tonal contour}contour. %The choice of the label M (rather than H) for this pattern will be explained further below, at the stage of phonological analysis.
	\item There is only one \is{tonal contour}contour that begins on a~low pitch. In terms of \is{level tones}level-tone
	labels, this observation can be stated as follows: there is no opposition between LM and LH.
	\item There are no examples of low, non-rising tones.
\end{enumerate}

Over disyllabic nouns, seven patterns are observed, as shown in \tabref{tab:tonepatternsattestedoverdisyllabicnounsspokeninisolation}. Since lexical roots are \is{monosyllables}monosyllabic, \isi{disyllables} result from various processes, such as addition of the female gender \is{suffixes}suffix /\ipa{-mi˩}/, found in ‘dog’ and ‘sow’ (this \is{suffixes}suffix is examined in detail in \sectref{sec:thegendersuffixesfacts}). At this point, the aim is to propose a~static inventory covering all attested tonal categories of disyllabic nouns, irrespective of their internal structure. 

\begin{table}%[t]
\caption{Tone patterns attested over disyllabic nouns in isolation.}
\begin{tabularx}{\textwidth}{ l@{\hspace{7mm}} l@{\hspace{7mm}} Q l }
  \lsptoprule
	1\textsuperscript{st} syllable & 2\textsuperscript{nd} syllable & preliminary label & example\\\midrule
	non-low & low & M.L ? & /\ipa{dɑ.ʝi}/ ‘mule’\\
	non-low & low-rising & $\ddagger${\kern2pt}M.LM & --\\
	non-low & mid-rising & M.MH ? & /\ipa{hwɤ.li}/ ‘cat’\\
	non-low & mid & M.M ? & /\ipa{po.lo}/ ‘ram’\\
	non-low & high & M.H ? & /\ipa{hwæ.ʈʂæ}/ ‘squirrel’\\
	low & low & $\ddagger${\kern2pt}L.L & --\\
	low & low-rising & L.LM ? L.LH ? & /\ipa{kʰv̩.mi}/ ‘dog’\\
	low & mid-rising & L.MH ? & /\ipa{õ.dv̩}/ ‘wolf’\\
	low & mid (or high) & L.M ? L.H ? & /\ipa{bo.mi}/ ‘sow’\\
\lspbottomrule
\end{tabularx}
\label{tab:tonepatternsattestedoverdisyllabicnounsspokeninisolation}
\end{table}

{\largerpage} % Added on April 27th, 2025

\tabref{tab:tonepatternsattestedoverdisyllabicnounsspokeninisolation}
includes two unattested combinations, marked with a~double dagger ($\ddagger$) in
the \textit{preliminary label} column. If the tone of the first syllable is non-low, there are four observed tonal patterns on the second syllable: low, mid, high, and mid-rising. Conversely, if the tone of the first syllable is low, three patterns are attested on the second syllable: low-rising, mid, and mid-rising. 

The restrictions on the distribution of tones on disyllables can be described in static terms as follows: 
\begin{enumerate}[label=(\roman*), itemsep=0pt]
%\item[(i)] Only two tones contrast on the first syllable: low and non-low. There can be no \is{tonal contour}contour on the first syllable. 
%\item[(ii)] A~Mid tone cannot be followed by a~low-rising tone.
%\item[(iii)] A~disyllable cannot be low throughout, any more than a~monosyllable can. 
%\item[(iv)] There is no contrast between a~low+mid pattern and a~low+high pattern. 
\item Only two tones contrast on the first syllable: low and non-low. There can be no \is{tonal contour}contour on the first syllable. 
\item A~non-low tone cannot be followed by a~low-rising tone.
\item A~disyllable cannot be low throughout. 
\item There is no contrast between a~low+mid pattern and a~low+high pattern. 
\end{enumerate}

There are also strong limitations on tone patterns in trisyllabic nouns: only twelve patterns are attested. The data in \tabref{tab:tonepatternsattestedovertrisyllabicnounsspokeninisolation} is based on \is{trisyllables}trisyllabic nouns whose degrees of lexical integration vary more widely than those of the disyllables in \tabref{tab:tonepatternsattestedoverdisyllabicnounsspokeninisolation}. These \is{trisyllables}trisyllabic nouns range from transparent compounds~-- such as ‘Year of the Dragon’ and ‘Year of the Snake’~-- to fully indecomposable words, such as ‘lips’. (For decomposable compounds, a hyphen is placed between the two parts.) As with disyllables, the focus here is on a static inventory; the rules relating the tones of compounds to those of their components will be analyzed later (in Chapter~\ref{chap:compoundnouns}).

\begin{table}%[t]
\caption{Tone patterns attested over {trisyllabic} nouns in isolation.}
\begin{tabularx}{\textwidth}{ l l l Q l Q }
  \lsptoprule
	1\textsuperscript{st} σ & 2\textsuperscript{nd} σ & 3\textsuperscript{rd} σ & preliminary label & example & meaning\\\midrule
	non-low & mid & mid & M.M.M ? & \ipa{ɖʐɤ.qʰwɤ.ʈʂe} & awl\\
	non-low & mid & low & M.M.L ? & \ipa{mv̩.gv̩-kʰv̩}̩ & Year of the Dragon\\
	non-low & mid & high & M.M.H ? & \ipa{njo.bi.li} & lips\\
	non-low & mid & mid-rising & M.M.MH ? & \ipa{bv̩.ʐv̩-kʰv̩} & Year of the Snake\\
	non-low & low & low & M.L.L ? & \ipa{mo.jo.mi} & owl\\
	non-low & high & low & M.H.L ? & \ipa{æ.tse.pʰæ} & kneebone\\
	low & low & mid & L.L.M ? & \ipa{tʰo.kʰv̩.mi} & male dog\\
	low & low & low-rising & L.L.LM ? & \ipa{dʑɯ.nɑ.mi} & wilderness\\
	low & mid & mid & L.M.M ? & \ipa{tʰɑ.ʐwæ.mi} & donkey\\
	low & mid & high & L.M.H ? & \ipa{æ.li.pʰæ} & mirror\\
	low & mid & mid-rising & L.M.MH ? & \ipa{bi.pʰv̩-dʑɯ} & flood\\
	low & mid & low & L.M.L ? & \ipa{bæ.bv̩-bv̩} & ladybird\\
\lspbottomrule
\end{tabularx}
\label{tab:tonepatternsattestedovertrisyllabicnounsspokeninisolation}
\end{table}


Since there is a~three-way opposition in tonal levels on the second syllable, these levels are labelled as ‘low’, ‘mid’, and ‘high’, whereas for the first syllable, where there is no opposition between mid and high, the two levels are simply labelled ‘low’ and ‘non-low’.

Based on the information in Tables \ref{tab:tonepatternsattestedovermonosyllabicnounsspokeninisolation}-\ref{tab:tonepatternsattestedovertrisyllabicnounsspokeninisolation}, the following generalizations can be proposed: 
\begin{enumerate}[label=(\roman*), itemsep=0pt]
\item A~non-low tone can be followed by one of four tones: low, mid, high, or mid-rising. 
\item A~low tone can be followed by low, low-rising, mid, or mid-rising. 
\item A~high tone can only be followed by a~low tone.
\item	Non-final syllables never carry a~\is{tonal contour}contour. 
\item	An entire word cannot carry low tone on all of its syllables. L.L.L is not permitted in \is{trisyllables}{trisyllabic} nouns, any more than L.L is found in disyllabic nouns or L in monosyllabic nouns~-- even though L.L is found in syllables 1 and 2 in some \is{trisyllables}{trisyllabic} nouns, and in syllables 2 and 3 in others.
\item	There can never be a~trough: that is, a~tone surrounded by higher tones (for example non-low followed by low followed by mid). 
\end{enumerate}

From the above data alone, it is not yet possible to ascertain whether these generalizations apply at the
level of the word, the phrase, or the entire sentence, since “words produced \is{form!in isolation|textbf}in isolation are minimal
utterances showing both lexical and utterance-level (postlexical) features”
\citep[164]{himmelmann2006}. To preview later analysis, the relevant domain is a~unit between the word and the utterance, which might be termed a~phonological phrase. The term adopted here is \textit{tone group} because the defining characteristic
of this unit is its role as the domain for tonal processes. Tone groups are discussed in detail in Chapter~\ref{chap:toneassignmentrulesandthedivisionoftheutteranceintotonegroups}.

A dynamic approach to the tone categories of nouns sheds light on the above generalizations from static
inventories. 


\section{A dynamic view, bringing out the tonal categories}
\label{sec:dynamicview}

A dynamic perspective brings out six tonal categories for \is{monosyllables}monosyllabic nouns and eleven for \is{disyllables}disyllabic nouns, as the initial patterns branch into subsets when considering contextual variations.

Readers are encouraged to consult Tables~\ref{tab:thelexicaltonesofmonosyllabicnouns} and \ref{tab:thelexicaltonesofdisyllabicnouns} for a~synthetic overview of the noun tone system as it emerges from the analysis. These tables are also reproduced in the ‘Quick reference’ section at the beginning of this volume.

\subsection{Monosyllabic nouns: Six tonal categories}
\label{sec:monosyllabicnouns}

As noted earlier, there are three surface patterns for \is{monosyllables}monosyllables spoken %\is{form!in isolation}
in isolation:
low-rising, non-low, and mid-rising. However, further examination reveals that these sets are not homogeneous. %Thus, examination of the words in various contextsthe set of nouns realized as non-low \is{form!in isolation}in isolation is not
%homogeneous, however. 
Consider, for example, the behaviour of /\ipa{jo}/ ‘sheep’, /\ipa{ʐwæ}/ ‘horse’, and /\ipa{lɑ}/ ‘tiger’, all of
which are realized with the same non-low tone 
%\is{form!in isolation}
in isolation. When combined with the \isi{copula}, they yield three distinct tonal patterns:

\begin{itemize} 
    \item /\ipa{jo˩ ɲi˩˥}/ ‘is \mbox{(a/the)} sheep’, with a low tone on the noun and a rising tone on the \isi{copula};
    \item /\ipa{ʐwæ˧ ɲi˥}/ ‘is (a/the) horse’, with a mid tone on the noun and a high tone on the \isi{copula}; 
    \item /\ipa{lɑ˧ ɲi˩}/ ‘is \mbox{(a/the)} tiger’, with a mid tone on the noun and a low tone on the \isi{copula}. 
\end{itemize}

Since the morphosyntactic context is
the same, 
%\footnote{To preview the results of analyses set out further below, in \sectref{sec:thelexicaltonesofverbs}, the {copula} carries a~lexical L tone. In isolation, it surfaces with a~rising tone, analyzed as LH.} 
these differences indicate that these words belong to three distinct lexical
tone categories, which all \is{neutralization}neutralize to non-low when spoken \is{form!in isolation}in isolation.

Likewise, the set of nouns realized with a low-rising tone \is{form!in isolation}in isolation, such as /\ipa{ʐæ}/ ‘leopard’ and /\ipa{bo}/ ‘pig’, is not homogeneous. Although differences between these nouns do not emerge when they are combined with the \isi{copula} (as they do for non-low nouns), they become apparent in other contexts, 
such as object-plus-verb constructions. For example, ‘has bought leopards’ is realized as /\ipa{ʐæ˩ hwæ˧-ze˩}/, with an L tone on the {perfective} suffix /\ipa{-ze}/,
whereas ‘has bought pigs’ is /\ipa{bo˩ hwæ˧-ze˧}/, with an M tone on the same suffix.

Among the three surface\is{form!surface} patterns observed for monosyllables \is{form!in isolation} spoken in isolation, only one (MH) corresponds to
a~single phonological set: all words realized with MH tone \is{form!in isolation}in isolation have the same tonal
pattern in a~given morphosyntactic context. In contrast, the two others constitute the \isi{neutralization} of
two or more lexical categories: the low-rising \is{tonal contour}contour corresponds to two distinct lexical categories, while the
non-low realization corresponds to three categories (see \tabref{tab:thelexicaltonesofmonosyllabicnouns}). In summary, a~dynamic view reveals six tonal categories for monosyllables.


\subsection{Disyllabic nouns: Eleven tonal categories}
\label{sec:disyllabicnouns}

The same procedure as described above was applied to \is{disyllables}disyllabic nouns, i.e.\ examining the behaviour of
nouns in different morphosyntactic contexts in order to find out how many tone categories need to be
distinguished.

It was discovered that nouns realized with an M.M pattern \is{form!in isolation}in isolation belong to two distinct
categories: one in association with which the \isi{copula} bears an L tone, and one with which the \isi{copula} bears an H tone. One
set is illustrated by /\ipa{po˧lo˧}/ ‘ram’, yielding /\ipa{po˧lo˧ ɲi˩}/ ‘is \mbox{(a/the)} ram’; the other is exemplified by
/\ipa{ʐwæ˧zo˧}/ ‘colt’, yielding /\ipa{ʐwæ˧zo˧ ɲi˥}/ ‘is \mbox{(a/the)} colt’.

Likewise, nouns realized with an M.H pattern \is{form!in isolation}in isolation form two distinct sets: one in association with which the \isi{copula} bears an L tone, and one with which the \isi{copula} bears an H tone. The first set is illustrated by /\ipa{kv̩˧ʂe˥}/ ‘flea’, /\ipa{kv̩˧ʂe˧ ɲi˥}/ ‘is \mbox{(a/the)} flea’, the other by /\ipa{hwæ˧ʈʂæ˥}/ ‘squirrel’,
/\ipa{hwæ˧ʈʂæ˥ ɲi˩}/ ‘is \mbox{(a/the)} squirrel’.

Finally, the nouns realized with an L.M pattern \is{form!in isolation}in isolation (which can also be transcribed as L.H, since there is no opposition between an L.M pattern and an L.H pattern, as noted above) fall into no
fewer than three categories. These three categories are distinguished by intersecting evidence from two
contexts: one in which a~\isi{copula} is added, and one in which a \isi{possessive} is added, as shown in \tabref{tab:examplesillustratingtheexistenceofthreetonecategoriesneutralizedtolminisolation}. The addition of the \isi{copula} sets apart a~category exemplified by the Na word for ‘{Naxi}’, after which the \isi{copula} receives an H
tone. The addition of the \isi{possessive} sets apart a~category exemplified by ‘boar’, which depresses the tone
of the \isi{possessive} to L, in contrast to its realization as M for the other words. Although the evidence
used to bring out these tone categories is morphotonological~-- based on the behaviour of nouns in
context~--, the differences in the
surface phonological\is{form!surface} tone strings shown in \tabref{tab:examplesillustratingtheexistenceofthreetonecategoriesneutralizedtolminisolation} must be ascribed to differences between the
lexical items, and hence to differences in lexical tone category.

In total, this analysis yields eleven tonal categories for disyllabic nouns.

\begin{table}[t]
\caption{Examples illustrating the existence of three tone categories of nouns neutralized to L.H (≈L.M) in isolation. (First-pass tonal notations, to serve as a~basis for further analysis.)}
%\begin{tabularx}{.9\textwidth}{ Q l@{\hspace{2mm}} Q l }
\begin{tabularx}{.9\textwidth}{ Q l Q l }
  \lsptoprule
	in isolation & meaning & with \isi{copula} & with \isi{possessive}\\\midrule
	\ipa{nɑ˩hĩ˥} (≈\ipa{nɑ˩hĩ˧}) & {Naxi} & \ipa{nɑ˩hĩ˧ ɲi˥} & \ipa{nɑ˩hĩ˧=bv̩˧}\\
	\ipa{bo˩mi˥} (≈\ipa{bo˩mi˧}) & sow & \ipa{bo˩mi˧ ɲi˩} & \ipa{bo˩mi˧=bv̩˧}\\
	\ipa{bo˩ɬɑ˥} (≈\ipa{bo˩ɬɑ˧}) & boar & \ipa{bo˩ɬɑ˧ ɲi˩} & \ipa{bo˩ɬɑ˧=bv̩˩}\\
\lspbottomrule
\end{tabularx}
\label{tab:examplesillustratingtheexistenceofthreetonecategoriesneutralizedtolminisolation}
\end{table}

\section[Phonological analysis]{Phonological analysis of the tone categories of nouns}
\label{sec:aphonologicalanalysisofthetonecategoriesofnouns}

As outlined in the preceding paragraphs, a~number of tonal categories were brought out on the basis
of their different behaviour in various morphosyntactic contexts. The phonological analysis of these
categories is up against an issue of circularity, since the tone categories of the simplest units~--
\is{monosyllables}monosyllabic nouns~-- can only be revealed through examination of their combinations with other
morphemes whose tone categories have not yet been independently analyzed. In practice, however,
bootstrapping is often required when analyzing a~new language variety: groping for a~correct
analysis through trial and error.

A step forward in the analysis of the tones of nouns was made possible by making progress in the analysis
of tones on other morphemes. Through an analytic process described below, it was realized
that the \isi{copula} carries a lexical L tone, while the \isi{possessive} carries a lexical M tone. This finding provided a basis for proposing a~phonological analysis
for each of the tone categories of nouns. 

For example, consider the two tonal categories illustrated by /\ipa{lɑ˧ ɲi˩}/ ‘is \mbox{(a/the)} tiger’ and /\ipa{ʐwæ˧ ɲi˥}/ ‘is
(a/the) horse’. In the case of ‘tiger’, the \isi{copula} surfaces with its own
lexical tone, so that ‘tiger’ is analyzed as having a lexical M tone. This is the simplest case: the noun's phonological tone surfaces as such in this context, and does not modify the tone of the copula. (The same analysis is proposed for the disyllabic category exemplified by /\ipa{po˧lo˧}/ ‘ram’.) By contrast, for ‘horse’ the \isi{copula} surfaces with an H tone that must be supposed to be projected onto it by the noun. The noun ‘horse’, therefore, exemplifies a~tone category characterized by an H tone that can only be realized on a~following morpheme~-- a~floating
H tone. 

This phenomenon warrants further explanation. First, the theoretical backdrop needs to be introduced (\sectref{sec:backdropfloating}). On this basis, the synchronic facts about the floating tone in Yongning Na can be set out (\sectref{sec:thesynchronicfacts}). Additionally, \sectref{sec:thefloatinghtoneofyongningnacorrespondstoanoverthtoneinneighbouringdialects} sheds comparative light on this phenomenon, showing that the floating H tone of Alawua corresponds to an overt H tone in two neighbouring dialects, and \sectref{sec:thecreationoffloatinghtonesaconsequenceofphonotacticconstraints} discusses the possibility that floating H tones arose diachronically as a~consequence of phonotactic constraints.

%\subsection{A floating H tone}
\label{sec:afloatinghtonewithcomparativeevidencepointingtoitsorigin}
\subsection{Theoretical backdrop: a~quick introduction to floating tones}
\label{sec:backdropfloating}


% Indexing whole subsection for 'floating tone'
\is{floating tone|(}
% Indexing this page in bold for 'floating tone', as it provides a definition
\is{floating tone|textbf}


Floating tones, sometimes called vowelless tones \citep[84]{goldsmith2002}, are entities postulated to explain categorical modifications in the tonal string. They do not surface directly but affect the tones of neighbouring morphemes. For example, in \ili{Bamana} (a language of the \ili{Mande} subgroup of {Niger-Congo}), the definite article has a~floating L tone as its signifier, as illustrated in (\ref{ex:bamanaSANS}-\ref{ex:bamana}). The examples are reproduced as (\ref{ex:bamanaSANSafr}) and (\ref{ex:bamanaafr}) with tone indicated by accents; equivalents using \isi{tone-letters} are provided as (\ref{ex:bamanaSANSchao}) and (\ref{ex:bamanachao}).

\begin{exe}
  \ex
  \label{ex:bamanaSANS}
  \begin{xlist}
  	\ex
  	\label{ex:bamanaSANSafr}
  	\gll Mùsò tɛ́ yàn\\
  	woman \textsc{neg}.be here\\
  	\glt	‘There is no woman/there are no women here.’
  	\ex
  	\label{ex:bamanaSANSchao}
  	\gll Mu˩so˩ tɛ˥ yan˩\\
  	woman \textsc{neg}.be here\\
  	\glt	‘There is no woman/there are no women here.’
  \end{xlist}	
\end{exe}

\begin{exe}
  \ex
  \label{ex:bamana}
  \begin{xlist}
	\ex
	\label{ex:bamanaafr}
	\gll Mùsó-~{\kern2pt}\char"0300 tɛ́ yàn\\
	woman-\textsc{art} \textsc{neg}.be here\\
	\glt	‘The woman is not here.’
	\ex
	\label{ex:bamanachao}
	\gll Mu˩so˥-~{\kern2pt}˩ tɛ˥ yan˩\\
	woman-\textsc{art} \textsc{neg}.be here\\
	\glt	‘The woman is not here.’
  \end{xlist}	
\end{exe}

%Tests to obtain the symbol for L tone. Much trouble with the grave accent, interpreted by XeLaTeX as a~quotation mark.
%\ipa{Mùsó-~\char"02CB tɛ́ yàn}\\
%\ipa{Mùsó-~\char"0300 tɛ́ yàn}\\
%\ipa{Mùsó-~\char"2035 tɛ́ yàn}\\
%\ipa{Mùsó-~\char"F195 tɛ́ yàn}\\
 
In (\ref{ex:bamana}), there is ample evidence of the presence of a~floating L tone. First, contact between the L tone of the noun ‘woman’ and the floating L tone of the definite article results in the addition of an H tone at the end of the noun, in accordance with a~general rule that inserts a~buffer high tone between two low tones, hence /\ipa{mùsó}/ (L.H) instead of /\ipa{mùsò}/ (L.L). Second, the floating L tone triggers downstep\footnote{\is{downstep|textbf}Downstep refers to a~distinctive lowering of tone. For historical background on this notion, see \citet{rialland1997}. Downstep is not marked in the transcript here, because it is a~phonological consequence of the presence of a~floating L tone. Alternatively, one could provide a surface phonological representation marking {downstep} with the conventional exclamation mark: /\ipa{Mùsó !tɛ́ yàn}/.} because the following morpheme, the negation marker /\ipa{tɛ́}/, carries an H tone.\footnote{An alternative analysis of the Bamana data, following \citet[24-25]{dumestre1987}, posits that the tone change is from L.H.H to L.L.H rather than from L.L.L to L.H.L. This alternative simplifies the analysis of trisyllables by assuming that the category of nouns exemplified by /\ipa{mùsó}/ ‘woman’ has a~lexical LH pattern. Under this assumption, the underlying and surface tones of ‘woman’ in (\ref{ex:bamana}) would be identical (LH), with the change from the underlying form to the surface form occurring in (\ref{ex:bamanaSANS}). However, which of these analytic options is favoured has no bearing on the floating tone analysis: all authors agree that the floating tone triggers {downstep} of the following H in (\ref{ex:bamana}).} 

\is{comparative method (historical linguistics)}Comparative evidence generally indicates that floating tones originate from the reduction (complete segmental ellipsis) of a~syllable. The \ili{Bamana} article is believed to derive from an earlier form *\ipa{-ò}; the full form (comprising both a vowel and a~tone) “is still attested in numerous varieties spoken on the geographic periphery of the \il{Mande}Manding area: \ili{Mandinka}, \ili{Xasonka}, \ili{Worodugukan}, \ili{Marka-Dafin}, some \ili{Kagoro} dialects” \citep{vydrin2016}. 

A~purely tonal morpheme arises when a~morpheme that was originally expressed segmentally (as a~syllable with a~tone) is reduced to a~mere tone, only manifesting itself through its association with another morpheme. This is not the only diachronic scenario, however: a~morpheme can acquire a~floating tone through the loss (complete segmental ellipsis) of one of its syllables. For instance, in earlier stages of the \ili{Igbo} language, disyllabic verbs are postulated; in the present state of the language, “the syllabicity of the final syllables is lost, leaving floating tones, which ({\dots}) shift to the left and knock the first tones off in front of the verb morpheme” \citep[94]{hymanetal1974}. 

Concerning the choice of terms, \citet[424]{voorhoeve1967} initially used the labels “presegmental tonemes” and “postsegmental tonemes”. In his search for an adequate cover term for both sets, he seems to have been aware of the awkwardness of the pleonastic label “nonsegmental tonemes”, which he grazed (but skirted) by referring to “nonsegmental H and L”. He introduced the notion of \textit{floating tones} in \citeyear{voorhoeve1971}. In the early 1970s, tonologists often enclosed this newly coined term in quotation marks at first occurrence, as illustrated by the following commentary about \ili{Fe’Fe’} (\ili{Bantu}, Niger-Congo): 

\begin{quotation}
	[In the word ‘pot’ /\ipa{cὰg~ ́}/] an earlier high tone {suffix} was present. Historically, there was an accompanying vowel, but synchronically, a~mere “{floating}” high tone is posited. ({\dots}) [T]his high tone causes the preceding L tone to raise to a~raised-low tone via the process of low-raising. \citep[86]{hymanetal1974}
\end{quotation}

The term gradually gained acceptance and is now standard in studies of \ili{Bantu} \citep[see e.g.][33]{franichetal2012}, of other languages of Africa (\citealt[102]{idiatov_internal_2020}; \citealt{snider_floating_2021}; \citealt{green_tonal_2022}), and beyond.
%\citep{he_naxiyu_2021}: this study is critical of the notion of 'floating tone' as applied to the Naxi system, where it tends to push away from view the many cases of reduction that does not go all the way to floating tone creation.

%\begin{quotation}
%	In  most  Grassfields languages [a subgroup of \ili{Bantu}], the segmental morphology has been greatly simplified. In the historical development of the Grassfields languages, many of the lost segments and syllables have given rise to what are known in 
%	this  literature  as  “floating  tones”.  The  tones  originally  linked  to  the  lost  syllables  and  segments 
%	persisted  and  morphed  into  the  current  tonal  systems.  In  some  Grassfields  languages,  these  floating 
%	tones  have  largely  been  lost.  However,  in  others,  such  as  Medumba,  they  are  still  robust. \citep[33]{franichetal2012}
%\end{quotation}

The notion of floating tone is here applied to one of the lexical H tone categories of Yongning Na. This is motivated by the observation that the Na tone at issue is never realized on the word to which it is lexically attached: it can only be \is{anchorage}anchored to a~following morpheme. In the present volume, “floating tone” is used as a~synchronic concept, “the concept of \textit{floating} having here the meaning of non-realised tones in an {isolated context}” \citep[61]{some2000}. 

An important caveat is that adopting this concept from \ili{Bantu} tone studies does not imply that the {diachronic} origin is the same. In \ili{Bantu}, all floating tones are assumed to originate from the loss of segmental material~-- the \textit{segmental ellipsis} of a~syllable~--, whereas in Yongning Na, {comparative evidence} (set out in \sectref{sec:thefloatinghtoneofyongningnacorrespondstoanoverthtoneinneighbouringdialects}) suggests that the evolutionary mechanism was different.  

\subsection{The synchronic facts about the floating tone in Yongning Na}
\label{sec:thesynchronicfacts}

{\largerpage} % Added on April 27th, 2025

An example illustrating the floating H tone category in the Alawua dialect of Yongning Na is provided by the monosyllabic form for ‘horse’, realized \is{form!in isolation}in isolation as /\ipa{ʐwæ˧}/. The word for ‘colt’, realized \is{form!in isolation}in isolation as /\ipa{ʐwæ˧zo˧}/, offers a~neat opportunity to extend the analysis to
disyllables: the H tone that appears in /\ipa{ʐwæ˧zo˧ ɲi˥}/ ‘is \mbox{(a/the)} colt’ is interpreted as reflecting
a floating H tone lexically attached to the noun ‘colt’, in a manner exactly parallel to
/\ipa{ʐwæ˧ ɲi˥}/ (‘is \mbox{(a/the)} horse’).\footnote{To preview results to be discussed later: noun-plus-{copula} combinations behave tonally like object-verb combinations, of which a~detailed account is presented in Chapter~\ref{chap:verbsandtheircombinatoryproperties}.
The rules yielding the surface phonological tone pattern are syntactically conditioned: not all combinations of a~\#H tone and an L tone yield an \mbox{/M{\dots}M.H/} sequence.}

Since this floating H tone is the only type of H tone that may be lexically attached to
a~{monosyllable}, it is convenient to transcribe it as a~simple H tone on monosyllabic nouns in the
dictionary \citep{michaud_et_al_na_dict_2024} and in examples within this volume. For example, ‘horse’ is transcribed as //\ipa{ʐwæ˥}//. (The double slashes are used to distinguish underlying phonological forms from surface phonological ones.) For
disyllables, however, an opposition exists between this floating H tone and a~word-final H tone
(as in /\ipa{hwæ˧ʈʂæ˥}/ ‘squirrel’). This complexity of syllabic {anchoring} necessitates the use
a~nonstandard symbol: a~symbol not used in the \isi{International Phonetic Alphabet}. Desperate tones call
for desperate measures: the pound symbol \# was chosen to indicate the end of
a~lexical word, so that the word is notated as /\ipa{ʐwæ˧zo\#˥}/ and the tonal category is designated as \#H.

To illustrate further, consider the \#H-tone word ‘little brother’ and the
M-tone word ‘little sister’, which, when spoken \is{form!in isolation}in isolation,  both carry an M tone on each syllable (/\ipa{gi˧zɯ˧}/ ‘little brother’, /\ipa{go˧mi˧}/ ‘little sister’). In combination with the copula, the former yields /\ipa{gi˧zɯ˧ ɲi˥}/ (‘is little
brother’), with a M.M+H tone sequence, whereas the latter yields /\ipa{go˧mi˧ ɲi˩}/ (‘is little sister’), with a M.M+L tone sequence. The analysis proposed is that ‘little brother’ possesses a~final H tone which remains unassociated
unless it can associate to a~following syllable~-- a floating H tone. 

The association of this floating H tone is conditioned by specific morphosyntactic factors. For
instance, the H tone does not surface when the noun is followed by the {possessive} clitic /\ipa{=bv̩˧}/: thus, \ipa{gi˧zɯ˧=bv̩˧}/ (‘of \mbox{(a/the)} little brother’) is tonally identical to /\ipa{go˧mi˧=bv̩˧}/
(‘of \mbox{(a/the)} little sister’). 
Whether the floating H tone surfaces or not in a~given context is not solely a function of the syntactic class of the added morpheme. To preview data presented in \sectref{sec:ltoneencliticspluralandassociativeplural}, note that another clitic, the \textsc{associative}, {can} host a~floating tone. Such complexities contribute to the necessity for a book-length description and analysis of tone in Yongning Na.


\subsection[The floating tone corresponds to an overt H in neighbouring dialects]{The floating H tone of Alawua corresponds to an overt H tone in two neighbouring dialects}
\label{sec:thefloatinghtoneofyongningnacorrespondstoanoverthtoneinneighbouringdialects}

{\largerpage} % Added on April 27th, 2025

Observations on a~neighbouring dialect, that of the village of Pujjo \ipa{pʰv̩˧dʑo˧} (Labai \zh{拉伯}), offers indirect supporting evidence for the analysis of the Alawua tonal category to which the word ‘horse’ belongs. My only contact with the Labai dialect to date has been through two
work sessions with Mr.\ Lamu Gatusa \mbox{\zh{拉木·}}\zh{嘎吐萨}, a~researcher at the Academy of Social Sciences in Kunming. The monosyllables of Alawua that are analyzed as bearing a~(floating) H tone correspond to monosyllables
with an overt H tone in Labai. Examples are presented in \tabref{tab:hhtonecorrespondence}. When one of the \#H-tone words in \tabref{tab:hhtonecorrespondence} is spoken \is{form!in isolation}in isolation, its H tone does not surface (\mbox{//M//} and \mbox{\mbox{//\#H//}} are neutralized to /M/), whereas in Labai H-tone items are realized as such. The \#H::H \is{comparative method (historical linguistics)}correspondence among monosyllables is the most widely populated. The sole \is{exceptions}exception is the pair \ipa{hæ̃˩::hæ̃˥} 
for ‘gold’ (shown at the bottom of \tabref{tab:hhtonecorrespondence}), for which no explanation can presently be offered. For the sake of completeness, \tabref{tab:mmtonecorrespondence} shows the correspondences for syllables carrying M tone in Labai. (A
recording of some of these words is available online from the Pangloss Collection \pandoi{0004489}.) Importantly, all monosyllables that belong to the M tone category in
Alawua correspond to M-tone monosyllables in Labai. 


	\begin{table}
	  \caption{Tone correspondences between Alawua and Labai for monosyllabic nouns carrying H tone in Labai.}
	\begin{tabularx}{\textwidth}{   l@{\hspace{10mm}} l l   l@{\hspace{10mm}}   l@{\hspace{15mm}} }
	  \lsptoprule
	%	gloss & \pbox{20cm}{Yongning:\\ /\is{form!in isolation}in isolation/} & \pbox{20cm}{Yongning:\\  //underlying form//} & Labai\\
		gloss & 	\multicolumn{2}{c}{Alawua} & Labai & {correspondence}\\
		& /in isolation/ &  //lexical form// & &\\
		\midrule
		earth & \ipa{ʈʂe˧} & \ipa{ʈʂe\#˥} & \ipa{tɕi˥} & \#H::H\\
		hail & \ipa{dzo˧} & \ipa{dzo\#˥} & \ipa{dzo˥} & \#H::H\\
		sky & \ipa{mv̩˧} & \ipa{mv̩\#˥} & \ipa{mv̩˥} & \#H::H\\
		fire & \ipa{mv̩˧} & \ipa{mv̩\#˥} & \ipa{mi˥} & \#H::H\\
		star & \ipa{kɯ˧} & \ipa{kɯ\#˥} & \ipa{kɯ˥} & \#H::H\\
		snow & \ipa{bi˧} & \ipa{bi\#˥} & \ipa{mbi˥} & \#H::H\\
		pond & \ipa{ɖwæ˧} & \ipa{ɖwæ\#˥} & \ipa{ɳɖwæ˥} & \#H::H\\
		canal & \ipa{qʰæ˧} & \ipa{qʰæ\#˥} & \ipa{qʰæ˥} & \#H::H\\
		urine & \ipa{dʑɯ˧} & \ipa{dʑɯ\#˥} & \ipa{ɳɖʐɯ˥} & \#H::H\\
		gall & \ipa{kɯ˧} & \ipa{kɯ\#˥} & \ipa{kɯ˥} & \#H::H\\
		blood & \ipa{sɤ˧} & \ipa{sɤ\#˥} & \ipa{sɤ˥} & \#H::H\\
		head & \ipa{ʁo˧} & \ipa{ʁo\#˥} & \ipa{ʁo˥} & \#H::H\\
		\ili{Pumi} & \ipa{bɤ˧} & \ipa{bɤ\#˥} & \ipa{bɤ˥} & \#H::H\\
		man & \ipa{zo˧} & \ipa{zo\#˥} & \ipa{zo˥} & \#H::H\\
		bronze & \ipa{æ̃˧} & \ipa{æ̃\#˥} & \ipa{æ˥} & \#H::H\\
		salt & \ipa{tsʰe˧} & \ipa{tsʰe\#˥} & \ipa{tsʰe˥} & \#H::H\\
			\midrule
			%			\addlinespace \hdashline \addlinespace
		gold & \ipa{hæ̃˧} & \ipa{hæ̃˩} & \ipa{hæ̃˥} & ~L::H\\ 
	   \lspbottomrule
	\end{tabularx}
	\label{tab:hhtonecorrespondence}
	\end{table}
	
	\begin{table}[t]
		\caption{Tone correspondences between Alawua and Labai for monosyllabic nouns carrying M tone in Labai.}
		\begin{tabularx}{\textwidth}{  l@{\hspace{17mm}} l@{\hspace{10mm}} Q l@{\hspace{15mm}} }
			\lsptoprule
			%	gloss & \pbox{20cm}{Yongning:\\ /\is{form!in isolation}in isolation/} & \pbox{20cm}{Yongning:\\  //underlying form//} & Labai\\
			gloss & Alawua & Labai & tone {correspondence}\\
			\midrule
	%		hail & \ipa{dzo\#˥} & \ipa{dzo˥} & M::M\\
			small dike & \ipa{bo˧} & \ipa{mbu˧} & M::M\\
			intestine & \ipa{bv̩˧} & \ipa{bv̩˧} & M::M\\
			dew & \ipa{ɖʐv̩˧} & \ipa{ɳɖʐɯ˧} & M::M\\
			wind & \ipa{hæ̃˧} & \ipa{hæ̃˧} & M::M\\
			tobacco & \ipa{jɤ˧} & \ipa{jɤ˧} & M::M\\
			field & \ipa{lv̩˧} & \ipa{lv̩˧} & M::M\\
			wound & \ipa{mi˧} & \ipa{mi˧} & M::M\\
			hole & \ipa{qʰv̩˧} & \ipa{qʰɚ˧} & M::M\\
			serf & \ipa{wɤ˧} & \ipa{wɤ˧} & M::M\\
			liquor/spirits & \ipa{ʐɯ˧} & \ipa{ʐɯ˧} & M::M\\
	%		sky & \ipa{mv̩\#˥} & \ipa{mv̩˥} & M::M\\ 
			\midrule
			%			\addlinespace \hdashline \addlinespace
			water & \ipa{dʑɯ˩} & \ipa{ɖʐɯ˧} & L::M\\
			lake & \ipa{hi˩} & \ipa{hɯ˧} & L::M\\
			thread & \ipa{kʰɯ˩} & \ipa{kʰɯ˧} & L::M\\
			silver & \ipa{ŋv̩˩} & \ipa{ŋv̩˧} & L::M\\
			bridge & \ipa{dzo˩} & \ipa{ndzo˧} & L::M\\
			iron & \ipa{ʂe˩} & \ipa{ɕi˧} & L::M\\
			\midrule
%			\addlinespace \hdashline \addlinespace
			bone & \ipa{ɻ̩̃˥} & \ipa{ɚ̃˧} & H::M\\
			pond & \ipa{ɖwæ˥} & \ipa{ɳɖwæ˧} & H::M\\
			rib & \ipa{ɬo˥} & \ipa{hõ˧} & H::M\\ 
			\midrule
			%			\addlinespace \hdashline \addlinespace
			earth & \ipa{di˩˥} & \ipa{di˧} & LH::M\\ 
			rain & \ipa{hi˩˥} & \ipa{hɯ˧} & LH::M\\
	
	%		canal & \ipa{qʰæ\#˥} & \ipa{qʰæ˥}\\
	%		urine & \ipa{dʑɯ\#˥} & \ipa{ɳɖʐɯ˥}\\
	%		gall & \ipa{kɯ\#˥} & \ipa{kɯ˥}\\
	%		blood & \ipa{sɤ\#˥} & \ipa{sɤ˥}\\
	%		head & \ipa{ʁo\#˥} & \ipa{ʁo˥}\\
	%		\ili{Pumi} & \ipa{bɤ\#˥} & \ipa{bɤ˥}\\
	%		man & \ipa{zo\#˥} & \ipa{zo˥}\\
	%		bronze & \ipa{æ̃\#˥} & \ipa{æ˥}\\
	%		salt & \ipa{tsʰe\#˥} & \ipa{tsʰe˥}\\
			\lspbottomrule
		\end{tabularx}
		\label{tab:mmtonecorrespondence}
	\end{table}


On disyllables, the floating H tones of Alawua likewise correspond to H tones in Labai: disyllables that
have a \#H tone in Alawua display an H tone on their second syllable in Labai (while disyllables with M tone in Alawua carry M tone on both syllables in Labai). Some examples of these correspondences are
provided in
\tabref{tab:mmandhmhtonecorrespondences}.


\begin{table}%[p]
\caption{Examples illustrating the M::M and \#H::MH tone correspondences for disyllabic nouns between Alawua and Labai.}
\begin{tabularx}{\textwidth}{ l Q@{\hspace{8mm}} Q Q }
  \lsptoprule
	{correspondence} & meaning & Alawua & Labai\\ \midrule
	M::M & little sister & \ipa{go˧mi˧} & \ipa{go˧mi˧}\\
%	& ancestor & \ipa{ə˧pʰv̩˧} & \ipa{ə˧pʰə˞{\kern1.7pt}˧}\\
	& ancestor & \ipa{ə˧pʰv̩˧} & \ipa{ə˧pʰə˞˧}\\
	& Bai & \ipa{ɬi˧bv̩˧} & \ipa{li˧bv̩˧}\\
	& mother & \ipa{ə˧mi˧} & \ipa{ə˧mi˧}\\
	& body & \ipa{gv̩˧mi˧} & \ipa{gv̩˧mv̩˧}\\
	& heel & \ipa{mv̩˧ʈʰɯ˧} & \ipa{mi˧ʈʰɯ˧}\\
	& thigh & \ipa{do˧bæ˧} & \ipa{do˧bæ˧}\\
	& buttock & \ipa{do˧bv̩˧} & \ipa{do˧bv̩˧}\\
	& nostril & \ipa{ɲi˧qʰv̩˧} & \ipa{ɲi˧qʰə˞˧}\\
	& back & \ipa{gv̩˧dv̩˧} & \ipa{gv̩˧dv̩˧}\\
	& breast & \ipa{ʁɑ˧pv̩˧} & \ipa{ŋgɑ˧pv̩˧}\\
	& belly & \ipa{bi˧mi˧} & \ipa{bi˧mi˧}\\
	& plait & \ipa{hæ̃˧pɤ˧} & \ipa{hæ̃˧pɤ˧}\\
	& sun & \ipa{ɲi˧mi˧} & \ipa{ɲi˧mi˧}\\
	& moon & \ipa{ɬi˧mi˧} & \ipa{li˧mi˧}\\
	& stone & \ipa{lv̩˧mi˧} & \ipa{lv̩˧mi˧}\\
	& powder & \ipa{tsɑ˧bɤ˧} & \ipa{tsɑ˧mbɑ˧}\\
	& hot spring & \ipa{ɻ̩˧qʰv̩˧} & \ipa{ə˞˧qʰə˞˧}\\
	& paddy field & \ipa{ɕi˧lv̩˧} & \ipa{ʂɯ˧lv̩˧}\\ \midrule
	\#H::MH & little brother & \ipa{gi˧zɯ\#˥} & \ipa{gɯ˧zɯ˥}\\
	& grandson & \ipa{ʐv̩˧v̩\#˥} & \ipa{ʐv̩˧v̩˥}\\
	& granddaughter & \ipa{ʐv̩˧mi\#˥} & \ipa{ʐv̩˧mi˥}\\
	& sole & \ipa{mi˧bɤ\#˥} & \ipa{mi˧bɤ˥}\\
	& nose & \ipa{ɲi˧gɤ\#˥} & \ipa{ɲi˧ŋgɤ˥}\\
	& craftsman & \ipa{po˧ɖʐɯ\#˥} & \ipa{po˧ɖʐv̩˥}\\
	& forehead & \ipa{to˧kɤ\#˥} & \ipa{to˧kɤ˥}\\
	& host & \ipa{dɑ˧pv̩\#˥} & \ipa{ndɑ˧pv̩˥}\\
\lspbottomrule
\end{tabularx}
\label{tab:mmandhmhtonecorrespondences}
\end{table}

% %Table 6. in manuscript
% \begin{table}[p]
% \caption{Examples illustrating the M::M and \#H::MH tone correspondences between Yongning and Labai
%   on disyllabic nouns.}

% \vspace*{.5\baselineskip}
% M::M \isi{correspondence}
% \vspace*{.5\baselineskip}

% \begin{tabularx}{.75\textwidth}{ Q@{\hspace{8mm}} Q Q }
%   \lsptoprule
% 	gloss & Yongning & Labai\\ \midrule
% 	little sister & \ipa{go˧mi˧} & \ipa{go˧mi˧}\\
% 	ancestor & \ipa{ə˧pʰv̩˧} & \ipa{ə˧pʰə˞˧}\\
% 	Bai & \ipa{ɬi˧bv̩˧} & \ipa{li˧bv̩˧}\\
% 	mother & \ipa{ə˧mi˧} & \ipa{ə˧mi˧}\\
% 	body & \ipa{gv̩˧mi˧} & \ipa{gv̩˧mv̩˧}\\
% 	heel & \ipa{mv̩˧ʈʰɯ˧} & \ipa{mi˧ʈʰɯ˧}\\
% 	thigh & \ipa{do˧bæ˧} & \ipa{do˧bæ˧}\\
% 	buttock & \ipa{do˧bv̩˧} & \ipa{do˧bv̩˧}\\
% 	nostril & \ipa{ɲi˧qʰv̩˧} & \ipa{ɲi˧qʰə˞˧}\\
% 	back & \ipa{gv̩˧dv̩˧} & \ipa{gv̩˧dv̩˧}\\
% 	breast & \ipa{ʁɑ˧pv̩˧} & \ipa{ŋgɑ˧pv̩˧}\\
% 	belly & \ipa{bi˧mi˧} & \ipa{bi˧mi˧}\\
% 	plait & \ipa{hæ̃˧pɤ˧} & \ipa{hæ̃˧pɤ˧}\\
% 	sun & \ipa{ɲi˧mi˧} & \ipa{ɲi˧mi˧}\\
% 	moon & \ipa{ɬi˧mi˧} & \ipa{li˧mi˧}\\
% 	stone & \ipa{lv̩˧mi˧} & \ipa{lv̩˧mi˧}\\
% 	powder & \ipa{tsɑ˧bɤ˧} & \ipa{tsɑ˧mbɑ˧}\\
% 	hot spring & \ipa{ɻ̩˧qʰv̩˧} & \ipa{ə˞˧qʰə˞˧}\\
% 	paddy field & \ipa{ɕi˧lv̩˧} & \ipa{ʂɯ˧lv̩˧}\\
% \lspbottomrule
% \end{tabularx}

% \vspace*{\baselineskip}
% \#H::MH \isi{correspondence}
% \vspace*{.5\baselineskip}

% \begin{tabularx}{.75\textwidth}{ Q@{\hspace{8mm}} Q Q }
% \lsptoprule
% 	gloss & Yongning & Labai\\\midrule
% 	little brother & \ipa{gi˧zɯ\#˥} & \ipa{gɯ˧zɯ˥}\\
% 	grandson & \ipa{ʐv̩˧v̩\#˥} & \ipa{ʐv̩˧v̩˥}\\
% 	granddaughter & \ipa{ʐv̩˧mi\#˥} & \ipa{ʐv̩˧mi˥}\\
% 	sole & \ipa{mi˧bɤ\#˥} & \ipa{mi˧bɤ˥}\\
% 	nose & \ipa{ɲi˧gɤ\#˥} & \ipa{ɲi˧ŋgɤ˥}\\
% 	craftsman & \ipa{po˧ɖʐɯ\#˥} & \ipa{po˧ɖʐv̩˥}\\
% 	forehead & \ipa{to˧kɤ\#˥} & \ipa{to˧kɤ˥}\\
% 	host & \ipa{dɑ˧pv̩\#˥} & \ipa{ndɑ˧pv̩˥}\\
% \lspbottomrule
% \end{tabularx}
% \label{tab:mmandhmhtonecorrespondences}
% \end{table}

Another neighbouring dialect, that of the village of Wujiao \zh{屋脚} (located just across the border with
the county of Muli in Sichuan, \zh{四川凉山州木里县屋脚乡}), yields similar tonal correspondences. Field notes kindly provided in 2012 by Xu Duoduo \zh{许多多}, then a~graduate student at Tsinghua
University, reveal that words classified in the \#H tone category in Alawua (which neutralize to M \is{form!in isolation}in isolation) are realized with 
an overt H tone in Wujiao. Xu Duoduo transcribed this H tone phonetically as \ipa{⁵³} (high-to-mid)
based on commonly observed patterns in \is{form!in isolation}isolated production; in the absence of a~\ipa{⁵⁵} (high-level)
tone in her notes, this can confidently be interpreted as an H tone. Examples include [\ipa{kʰv̩⁵³}] ‘dog’ (\is{homophony}homophone: ‘to steal’), [\ipa{kv̩⁵³}] ‘garlic’, and [\ipa{hṽ̩⁵³}] ‘hair’. The corresponding words in Alawua are phonologically identical, except that the H tone is floating: //\ipa{kʰv̩\#˥}// ‘dog’, //\ipa{kʰv̩\#˥}// ‘to steal’, //\ipa{kv̩\#˥}// ‘garlic’ and //\ipa{hṽ̩\#˥}// ‘hair’. No {counterexample} has been found.

%\newpage
These comparisons neatly confirm the H tone category independently postulated for the Alawua dialect. Needless to say, the tonal correspondences between
Alawua, Labai and Wujiao warrant a~systematic investigation based on a much larger dataset than that collected so far for Labai and Wujiao. The tone system of Labai calls for an in-depth study in its own
right, with detailed comparisons to Alawua, Lataddi \citep{dobbsetal2016,fily_documentation_2022}, and other dialects. While the tone system of Wujiao appears, at first glance, to be more similar to that of Alawua, it nonetheless exhibits sufficient differences to require its own comprehensive description.


\subsection{The creation of floating H tones: a~consequence of phonotactic constraints?}
\label{sec:thecreationoffloatinghtonesaconsequenceofphonotacticconstraints}

The \is{comparative method (historical linguistics)}correspondence between floating H tones in Alawua and overt H tones in Labai raises the question of
whether an earlier overt H tone became floating in Alawua, or conversely, whether an earlier floating H
tone became overt in Labai. Cross-linguistically, the more common scenario is that overt tones are set afloat by changes in metrical structure. For instance, in \ili{Manding} languages
(a branch of the \ili{Mande} subgroup of Niger-Congo), it has been hypothesized that, at one stage of language history,
heavy syllable rhymes (VV or VN) could carry up to two tonal levels, and that, at
a~later stage exemplified by \ili{Bambara}, the distinction between heavy and light rhymes was
lost, with the second tone on former heavy rhymes becoming floating (see \citealt{creisselsetal1993}; also \citealt{konoshenko2008} on developments in Liberian \ili{Kpelle} concerning the delinking of H in an earlier LH category).

It seems reasonable to hypothesize that in \ili{Naish} languages the {diachronic} change was likewise from overt tones to floating tones.
%, but comparison of Alawua and Labai alone does not constitute a~sufficient basis to shed light on the process. A~highly speculative argument is outlined below; it will be necessary to collect more pieces of the {diachronic} jigsaw puzzle before the origin of the floating H tone in Alawua can be understood clearly. 
A specificity of Alawua, in contrast with Labai, is the prohibition of tone-group-initial H
tone. As a~consequence of this exceptionless rule, it is
impossible at the \is{form!surface}{surface phonological level} to have an H tone on a~{monosyllable} spoken \is{form!in isolation}in isolation. Supposing that there used to be an overt H tone on monosyllables, the opposition between tone categories M and H for monosyllables would be threatened when the number of possible contrasts for a~group-initial syllable
collapsed from three (H, M, L) to two (M, L). Delinking the H tone could then be seen as a~response to this threat, preserving the lexical distinctions.

%\footnote{
The argument that H tones are set afloat \textit{in order to preserve lexical distinctions} must be wielded with great care, since some phonological oppositions are irreversibly lost in the course of language history. “The linguistic system, with its
myriad phonetic and semantic pressures effecting changes simultaneously 
and at times antagonistically, always emerges functionally unscathed, its  
semantic clarity intact” \citep[697]{silverman2015}. For now, I am content to record the fact that the tonal oppositions were maintained; determining \textit{why} is a much more arduous task.
%}

Regarding disyllables, supposing that there once existed nouns carrying a~H.M pattern, their initial H tone would likewise have been affected when the prohibition against tone-group-initial H
tone set in. One might imagine that this initial H tone would be set afloat as a~response, similar to the case in monosyllables. However, this scenario does not take into account the state of the tone system as a~whole. The representation shown in \figref{fig:setafloat} brings out the implausibility of a~change whereby the initial H tone on disyllabic nouns is set afloat while the tonal category characterized by a~final H tone remains unchanged. It does not appear likely that the *HM category could leap over the *MH category, as it were, to yield a~floating tone (scenario 1). Instead, the change affecting word-initial H~tones plausibly initiated a~push chain, represented as scenario 2 in \figref{fig:setafloat}: the \textit{initial} H moves to a final position, and the \textit{final} H is set afloat.\footnote{I am grateful to Henriëtte Daudey and Denis Creissels for closely-argued discussion of this topic.} Scenario 2 is supported by the observed correspondences between Alawua and Labai shown in \tabref{tab:mmandhmhtonecorrespondences}: disyllables with a~floating H tone in Alawua correspond to disyllables with a final H tone in Labai. 
 
\begin{figure}
	\caption{Two scenarios of evolution for *MM, *HM and *MH tone patterns over disyllabic nouns.}
	\begin{tikzpicture}
	\node (12) at (0,-0.5) {*MM};
	\node (42) at (2,-0.5) {*HM};
	\node (4442) at (4,-0.5) {*MH};
	
	\node (22) at (0,-2.5) {~MM};
	\node (32) at (2,-2.5) {};
	\node (52) at (4,-2.5) {~MH};
	\node (5552) at (6,-2.5) {~MM+\#H};
	
	\draw[decoration={markings,mark=at position 1 with
		{\arrow[scale=2,>=stealth]{>}}},postaction={decorate}] (12) -- (22);
	
	\draw[decoration={markings,mark=at position 1 with
		{\arrow[scale=2,>=stealth]{>}}},postaction={decorate}] (42) -- (5552);
	
	\draw[decoration={markings,mark=at position 1 with
		{\arrow[scale=2,>=stealth]{>}}},postaction={decorate}] (4442) -- (52);
	
	\node[text width=40mm] (s2) at (-3,-1.2) {Scenario 1\\ (implausible process):\\ only *HM is affected:\\delinking of {initial} H tone; final H unchanged};

	\node (1) at (0,-4.5) {*MM};
	\node (4) at (2,-4.5) {*HM};
	\node (444) at (4,-4.5) {*MH};
	
	\node (2) at (0,-6.5) {~MM};
	\node (3) at (2,-6.5) {};
	\node (5) at (4,-6.5) {~MH};
	\node (555) at (6,-6.5) {~MM+\#H};
	
	\draw[decoration={markings,mark=at position 1 with
		{\arrow[scale=2,>=stealth]{>}}},postaction={decorate}] (1) -- (2);
	
	\draw[decoration={markings,mark=at position 1 with
		{\arrow[scale=2,>=stealth]{>}}},postaction={decorate}] (4) -- (5);

	\draw[decoration={markings,mark=at position 1 with
		{\arrow[scale=2,>=stealth]{>}}},postaction={decorate}] (444) -- (555);

%	\node [anchor=mid] (s1l) at (0.5,-2) {/\ipa{hwɤ.li}/ ‘cat’};
	%  \node (s1ll) at (0.5,-2.5) {lexical tone: MH\#};
	
%	\node [anchor=mid] (s1lll) at (3,-2) {/\ipa{ɲi}/ \textsc{copula}};
	%  \node (s1llll) at (4,-2.5) {lexical tone: L};
	
	\node[text width=40mm] (s1) at (-3,-5.2) {Scenario 2\\ (hypothesized change): \\push chain:\\ initial H moves\\ to final position;\\ final H is set afloat};

	% Shifting top and bottom parts: 
	%-0.5 -2.5 -1.2
	% -4.5 -6.5 -4.75
	
	\end{tikzpicture}
	\label{fig:setafloat}
\end{figure}

To conclude this brief foray into diachronic territory, a~caveat is in order. The scenarios presented in \figref{fig:setafloat} presuppose that, at the time of the change, the rest of the system was as it is now: specifically,
that the M and H\# categories were already as they are now. They also presuppose that the state of affairs in Labai is more {conservative} than that in Alawua; indeed, the tone patterns observed in Labai are reproduced in \figref{fig:setafloat} with the simple addition of an asterisk, as if they were proto-forms. 
These assumptions are not unreasonable given the close geographic proximity between the two dialects (see Map~\ref{map:1-1}). 
Nonetheless, the history of these dialects is not without complexity, as evidenced by the diversity of tonal correspondences for monosyllables shown in \tabref{tab:hhtonecorrespondence}. 

The logical next step in diachronic analysis is to conduct a comprehensive comparison of the tone systems of Alawua and Labai. As noted in the introduction (\sectref{sec:adilemmabreadthofcoverageofthetonesystemvsbreadthofsociolinguisticcoverage}), the ultimate research goal, conceived as a~collective endeavour, is to document in detail the synchronic tone systems of a number of Naic-speaking villages and gradually reconstruct the evolution from a~non-tonal stage to each
of the present-day varieties. For the present argument, it suffices to conclude that there is solid comparative evidence for a~process of delinking of H tones, which has resulted in the present-day floating H tones of the Alawua dialect. 

\is{floating tone|)}
 
% \largerpage[-1]
\subsection{Word-final H tone and the ‘flea’ H tone category}
\label{sec:wordfinalandmorphologicalnucleusfinalHtones}

It was mentioned above that the words ‘squirrel’ and ‘flea’, realized with an M.H pattern in
isolation (as /\ipa{hwæ˧ʈʂæ˥}/ and /\ipa{kv̩˧ʂe˥}/, respectively), belong to different lexical tone categories.

The word ‘squirrel’ has a~straightforward tonal behaviour: its H tone attaches to the final syllable of the lexical
word. This pattern is consistent across contexts. Under the present analysis, the first syllable receives an M tone by default, yielding a~surface M.H pattern.

%\footnote{
This pattern might alternatively be analyzed as consisting of an MH {contour} associated with the first syllable, with syllable-by-syllable association of successive levels: M on the initial syllable and H on the second. Such an analysis would contrast a word-initial MH tone with another category (MH\#) in which the MH {contour} is associated with the word's
  final syllable (e.g.\ /\ipa{hwɤ˧li˧˥}/ ‘cat’). However, there is no evidence to support the existence of a MH {contour} here. The analysis adopted for the category illustrated by ‘squirrel’ is therefore as a~lexical-word-final
  H tone, transcribed as H\#.
%}

By contrast, the word ‘flea’ proves more elusive. ‘Flea’ is an apt eponym for the H tone category to which this example word belongs, serving as a mnemonic for its propensity to hop around from one host to another. When a~word carrying this tone is pronounced \is{form!in isolation}in isolation, its H tone associates with the final syllable: for instance, /\ipa{kv̩˧ʂe˥}/ ‘flea’ has an M.H sequence on the phonological surface. However, when the \isi{copula} is added, the result is /\ipa{kv̩˧ʂe˧ ɲi˥}/ (‘is \mbox{(a/the)} flea’), with an H tone appearing on the \isi{copula}~-- a surface pattern identical to that observed with the floating H tone, as exemplified by /\ipa{gi˧zɯ˧}/ (‘little
brother’) and /\ipa{gi˧zɯ˧ ɲi˥}/ (‘is \mbox{(a/the)} little
brother’). 
% (tone pattern \is{form!in isolation}in isolation: M.M, as opposed to M.H for ‘flea’). 
Moreover, when the noun is followed by the \isi{possessive} clitic /\ipa{=bv̩˧}/, no H tone reaches the surface: the form is /\ipa{kv̩˧ʂe˧=bv̩˧}/ (‘of \mbox{(a/the)} flea’), with M tone on both the
noun and the clitic. This pattern mirrors that observed with the \is{floating tone}floating H tone in the corresponding ‘little brother’ example (/\ipa{gi˧zɯ˧=bv̩˧}/ ‘of \mbox{(a/the)} little brother’). Consequently, it appears that the ‘flea’ H tone is not anchored on the lexical word's final syllable, nor does it behave in the same manner as the floating H tone in ‘little brother’ (\#H), which never surfaces on the word to which it is lexically attached. 

One possibility entertained in early reflections was that the association of this type of H tone might be specified relative to a~higher prosodic unit. In Na, the syllable is the smallest relevant unit for tonal association, while the \isi{tone group} is the highest relevant unit: successive tone groups are
entirely independent from the point of view of their phonological tones. (On the notion of {tone group}, see Chapter \ref{chap:toneassignmentrulesandthedivisionoftheutteranceintotonegroups}.) Between these extremes, additional levels may be distinguished:

\begin{itemize}
	\item the lexical word, to which tone categories are lexically associated;
	\item the \is{tonal word}tonal word, which is a~combination of lexical words (such as a noun plus a verb in S+V or O+V combinations, or a noun plus a noun in compounds);
	\item the \isi{tonal phrase}, comprising a~tonal word along with any attached clitics and affixes.
\end{itemize}

In an article published in 2015, I proposed that the ‘flea’ H tone attaches to the right edge of the \isi{tonal phrase} \citep{michaud2015b}. Although lexically associated with a~word, this type of H tone ``hops'' to the rightmost position within this higher prosodic domain. In cases where the final syllable of the \isi{tonal phrase} is a~suitable host, the H tone attaches to it; otherwise, as with the {possessive} suffix, the H tone remains unassociated and does not make it to the surface phonological level. 

Upon further reflection, the search for a~single defining characteristic of the ‘flea’ H tone relative to a~specific level in the prosodic hierarchy appears unpromising. While this tone tends to be realized later than the word to which it is lexically associated~-- a behaviour it shares with the \is{floating tone}floating H tone~--, it is not a sufficiently distinctive property to define the category. For example, the main consultant’s family name displays the ‘flea’ H tone: when spoken \is{form!in isolation}in isolation, it is realized as /\ipa{lɑ˧tʰɑ˧mi˥}/; when the \is{associative plural}{associative} \is{clitics}clitic /\ipa{=ɻ̩˩}/ is added, it yields /\ipa{lɑ˧tʰɑ˧mi˧=ɻ̩˥}/ (‘the Latami family; the Latamis’); and with the addition of the {agent} adposition /\ipa{ɳɯ˧}/, it yields /\ipa{lɑ˧tʰɑ˧mi˧=ɻ̩˧-ɳɯ˥}/ (‘by the
Latami family’). Impressionistically, this might suggest that the ‘flea’ tone tends to ``hop'' or ``glide'' to the right edge of the \isi{tonal phrase} as the morphotonological opportunity arises. However, this is not a~defining property, as the \is{floating tone}floating H tone yields the same results with these added syllables. 

The extent to which a~certain lexical tone tends to be realized close to the edge of a~given level in the prosodic hierarchy is a~statistical tendency, not to be taken as a~defining property of that tone. Instead, the various lexical tones must be understood in terms of the full range of their interactions with other morphemes, as summarized in the tables presented in Chapters~\ref{chap:thelexicaltonesofnouns} to \ref{chap:verbsandtheircombinatoryproperties}. 

%metaphorical descriptions of the tones (as semi-personified entities) will not take us very far. It would be misleading to build a~narrative account of the process whereby the noun's H tone would ‘hop’ or ‘glide’ onto the verb. The synchronic reason that the \isi{copula} receives H tone when following the ‘flea’ H tone is because there is a~morphotonological rule to that effect. 

%; they are not transparently analyzable as opportunities for ‘\is{floating tone}floating’, ‘hopping’ or ‘gliding’ of which different lexical tones avail themselves depending on their intrinsic propensity to such or such behaviour. 

Thus, it does not appear feasible to pinpoint the exact phonological nature of the ‘flea’ H tone by proposing a single defining characteristic. Rather, it is more appropriate to regard it as one tone within the system, defined by the set of oppositions in which it participates across the full range of morphotonological contexts. For convenience, a nickname may be applied: the tone of ‘little brother’ (\#H) is referred to as a~\textit{floating} H tone, while the tone of ‘flea’ is labelled a~\textit{hopping} H tone.\footnote{In the first edition of this book, the tone of ‘flea’ was nicknamed the \textit{gliding} H tone. In retrospect, this choice of name appears unfortunate, as `glide' is an established term in phonology, where it refers to semivowels, and in phonetics, where it describes the time course of pitch transitions `gliding' from one target to another. For instance, \citet[61]{hymanetal2000} report the tones of the Hausa words /\ipa{káì}/ ‘head’ and /\ipa{káí}/ ‘you [masculine]’ as being realized as ``an extra-H gliding down to a~raised L”. In this context, the term `gliding' refers to a~gradual pitch movement, not a~change in the syllabic anchoring of a~tone. To avoid potential confusion, all five occurrences of the nickname ``\textit{gliding} H tone" have been replaced with ``\textit{hopping} H tone" in this edition.} This is, however, merely a convenient nickname and should not be mistaken for a~definition. The symbol chosen for transcribing the ‘flea’ H tone is H\$, with the dollar sign added to distinguish it from the other two lexical H tones: H\# (lexical-word-final H tone) and \#H (\is{floating tone}floating H tone). The choice of this arbitrary symbol reflects the abstract nature of this tone category as one of the distinctive tones within the Alawua system. 

This category is relatively small, with only fifty example words in the dictionary \citep{michaud_et_al_na_dict_2024}, compared with 209 examples for the word-final H tone category (H\#) and 368 examples for the \is{floating tone}floating H tone category (\#H). Nevertheless, it is firmly attested, and there are productive rules of compounding, prefixation and suffixation that feed into this category, as will be detailed in later chapters. 

In summary, disyllabic (and polysyllabic) nouns with H tone must be divided into three categories,
labelled H\#, H\$ and \#H. An H tone on the final
syllable of a~disyllabic or polysyllabic noun may have different origins. It may be the realization
of a~High tone that is {anchored} to the final syllable of the lexical word (H\#), or it may be the realization of an H\$ tone. It is
impossible to distinguish these \is{form!in isolation}in isolation. The third category~-- \#H~-- denotes a~noun that carries a~\is{floating tone}floating H tone. To determine the
lexical tone of a word, it must be observed in various contexts. For nouns, telltale environments are: when spoken \is{form!in isolation}in isolation (i.e.\ in tone-group-final position), in tone-group-internal position, and
when followed by a~clitic (e.g.\ the \isi{possessive}). Only by matching the behaviour of a word in these various contexts can its lexical tone be identified with
certainty. 

\subsection[A~postlexical rule for all-L tone groups]{An added complexity concerning L tone: A~postlexical rule for all-L tone groups}
\label{sec:ltonesexistenceofarepairphenomenonforallltonegroups}

{\largerpage[-1]} % Added on April 27th, 2025

‘Sheep’, realized in association with the \isi{copula} as /\ipa{jo˩ ɲi˩˥}/ (i.e.\ with a~low-rising \is{tonal contour}contour on
the verb), is a~case in which the noun’s phonological tone (an L
tone) is hypothesized to surface as such. A~slight complexity is that the \isi{copula} following it surfaces with a~low-rising tone. This is consistent with the exceptionless observation that an utterance cannot carry low tone on all of its
syllables. In other words, sequences such as L+L (monosyllabic noun+\isi{copula}) and L.L+L (disyllabic noun+\isi{copula}) cannot
surface as such, due to a~general prohibition against all-L tone groups in the Alawua dialect of Yongning Na. The \is{tonal contour}contour observed
at the end of a~sequence of L tones is interpreted as resulting from a postlexical addition of
an extra tone. (This \is{tone rules}phonological rule is formulated in \sectref{sec:alistoftonerules} as Rule~7: “If a~\isi{tone group} only contains L tones, a~postlexical H tone is added to its last syllable”.) The same analysis applies to the tonal category of disyllables exemplified by /\ipa{kʰv̩˩mi˩˥}/ ‘dog’.

Concerning the transcription of low-rising contours, one might initially regard the choice between notation as LM or LH as a~nonissue, since there is no surface contrast between LM
and LH contours. However, in the case of /\ipa{jo˩ ɲi˩˥}/ (‘is \mbox{(a/the)} sheep’) or /\ipa{kʰv̩˩mi˩˥}/ (realization of
‘dog’ \is{form!in isolation}in isolation), there is a~language-internal argument for analyzing the endpoint of the
\is{tonal contour}contour (i.e.\ the postlexical tone) as H rather than M. As will be set out in \sectref{sec:analysisofmasadefaulttone}, the M tone in
Na is phonologically inert; if the postlexical tone added to all-L sequences were M, this
would be the only instance of a rule adding an M tone to a~syllable that already hosts another tone. Therefore, the postlexical tone added at the end of a~sequence of L tones is analyzed as H, and the rising \is{tonal contour}contour found in the nouns ‘dog’ and ‘wilderness’, and in the phrase ‘is a~sheep’,
will henceforth be written as LH (hence /\ipa{kʰv̩˩mi˩˥}/, /\ipa{dʑɯ˩nɑ˩mi˩˥}/, and /\ipa{jo˩ ɲi˩˥}/, respectively).

The question of why monosyllabic L-tone nouns, when spoken \is{form!in isolation}in isolation, surface with M tone rather than with an LH \is{tonal contour}contour (comprising their lexical L tone plus a~postlexical H tone) is addressed in \sectref{sec:reflectionsonthestructureofthesystemphonologicalregularitiesandmorphophonologicaloddities}.


\subsection{Tonal contours as sequences of level tones on the same syllable}
\label{sec:contourtonessequencesofleveltonesonthesamesyllable}

As mentioned in the static overview presented earlier, no phonological falling contours occur in
Alawua: no syllable carries HL, HM, or ML contours. Moreover, tone-group-initial H is
never observed.

Rising contours, on the other hand, do occur. They are restricted to the final syllable of a~tone
group. A~rising \is{tonal contour}contour is not found on a~non-group-final syllable, except in some special cases
discussed in~\sectref{sec:casesofbreachoftonalgroupingandconsequencesforthesystem}. The two observed rising contours are M-to-H and L-to-H (with the latter representing the \isi{neutralization} of underlying LM and LH). Unlike the
low-rising \is{tonal contour}contour, the phonological behaviour of the mid-rising \is{tonal contour}contour (MH) is straightforward. When a word
is tone-group-final, the MH \is{tonal contour}contour is realized as such: that is, as a~rising tone beginning at a non-low pitch~-- for instance in /\ipa{ʈʂʰæ˧˥}/ ‘deer’ and /\ipa{hwɤ˧li˧˥}/ ‘cat’. (Note that when a~word is pronounced in
isolation, it forms its own {tone group}, so that its beginning is also the beginning
of the {tone group} and its end is likewise the end of the {tone group}.) When there is
a~following syllable within the {tone group}, the MH \is{tonal contour}contour unfolds, with its H component projecting onto that syllable. Unlike the \is{floating tone}floating High tone (\#H), which cannot attach to a~following \is{clitics}clitic, the MH
\is{tonal contour}contour unfolds over any available syllable. With the \isi{copula}, this produces /\ipa{ʈʂʰæ˧ ɲi˥}/ ‘is \mbox{(a/the)}
deer’ and /\ipa{hwɤ˧li˧ ɲi˥}/ ‘is \mbox{(a/the)} cat’. With the \isi{possessive}, it yields /\ipa{ʈʂʰæ˧=bv̩˥}/ ‘of
(a/the) deer’ and /\ipa{hwɤ˧li˧=bv̩˥}/ ‘of \mbox{(a/the)} cat’. These observations provide strong evidence for analyzing the \is{tonal contour}tonal contours of Yongning Na as sequences of level tones. 

\largerpage
To preview the result of the analysis, low-rising contours can also be decomposed into level components. But they raise some subtle issues for description and
analysis, which are addressed in the following paragraphs.


\subsection[An alternative analysis of \mbox{//LM//} and \mbox{//LH//}]{An alternative analysis of the \mbox{//LM//} and \mbox{//LH//} categories: Could the two terms of the opposition be \mbox{//LM//} and \mbox{//LML//}?}
\label{sec:twooptionsforanalysislmvslhorlmvslml}

Concerning the two categories of tones that neutralize to a~low-rising contour \is{form!in isolation}in isolation, illustrated by
/\ipa{ʐæ˩˥}/ ‘leopard’ and /\ipa{bo˩˥}/ ‘pig’, at least two analytical options are open. Assuming (for reasons which
will be explained below) that the perfective \is{suffixes}suffix normally carries an M tone unless affected by preceding tones, the L tone observed on the \is{suffixes}suffix in (\ref{ex:boughtLEOP}), contrasting with the M tone in (\ref{ex:boughtPIG}), could be put down to
a~\is{floating tone}floating L tone (parallel to the \is{floating tone}floating H tone found in the tone category illustrated by
‘horse’). In this analysis, the floating L tone would remain unassociated when ‘to buy leopards’ is produced without a~\is{suffixes}suffix, and would associate to the \is{suffixes}suffix when one is available. 

\begin{exe}
	\ex
	\ipaex{ʐæ˩ hwæ˧-ze˩}\\
	\label{ex:boughtLEOP}
	\gll ʐæ˩˥	hwæ˧	-ze˧\\
	leopard		to\_buy		\textsc{pfv}\\
	\glt ‘has bought leopards’
\end{exe}

%\Hack{\newpage}

\begin{exe}
	\ex
	\ipaex{bo˩ hwæ˧-ze˧}\\
	\label{ex:boughtPIG}
	\gll bo˩˧	hwæ˧	-ze˧\\
	pig		to\_buy		\textsc{pfv}\\
	\glt ‘has bought pigs’
\end{exe}

The first analytical option is thus to consider that the underlying tone pattern for ‘to buy leopards’
is \mbox{//LM\#L//}~-- that is, an LM tone followed by a floating L tone. The L tone on the suffix in (\ref{ex:boughtLEOP}) would result from the association of the floating L tone. This \mbox{//LM\#L//} pattern contrasts with the simpler underlying \mbox{//LM//} pattern for ‘to buy pigs’. (Recall that double slashes denote underlying phonological tone, whereas simple slashes indicate surface phonological tone.) In this view, the difference
between the two object-plus-verb phrases would be attributed to a \mbox{//LM\#L//} vs.\ \mbox{//LM//} tone contrast on the
noun. %The \mbox{//LML//} sequence could also be transcribed as //LM+\#L//: \mbox{//LM//} followed by a~\is{floating tone}floating L tone.

An alternative analytical option is suggested by the static observation that within a tone group,\footnote{About the key speech unit referred to here as a~\textit{tone group}, see full details in Chapter~\ref{chap:toneassignmentrulesandthedivisionoftheutteranceintotonegroups}.} an H tone is invariably followed by
L tones. In this light, the lowering of the tone of the {perfective} \is{suffixes}suffix 
%\footnote{It may be useful to clarify that the L tone on the {perfective} {suffix} cannot be interpreted as a~phrase-level phenomenon, as shown by the fact that there are contexts where a~clause-final (and utterance-final) {suffix} carries a~tone other than L, e.g.~in (\ref{ex:boughtPIG}) and in /\ipa{æ̃˩ hwæ˧-ze˧}/ ‘bought chicken’.} 
in (\ref{ex:boughtLEOP})
could be ascribed to the influence of a~preceding H tone, which depresses the tones of all following syllables. In this scenario, the underlying form would be //\ipa{ʐæ˩ hwæ˥-ze˩}//, with H tone (and not M tone) on the verb, originating from a lexical LH tone on the noun ‘leopard’. 

Of these two options, the second is currently favoured because there is no evidence elsewhere in the language for \is{floating tone}floating L tones. There are strong reasons to adopt the concept of a \is{floating tone}floating H tone in this dialect's \isi{morphotonology} (including \is{comparative method (historical linguistics)}comparative-{diachronic} evidence presented in \sectref{sec:thecreationoffloatinghtonesaconsequenceofphonotacticconstraints}), whereas positing a~\is{floating tone}floating L
tone would be an \textit{ad hoc} theoretical move. Nonetheless, the evidence is not overwhelming enough to entirely reject the analysis of the ‘leopard’ category as a~sequence of three levels~-- LM plus
a~\is{floating tone}floating L tone (LM+\#L). Such cases of analytical indeterminacy are important for understanding
the system's evolutionary potential, a~topic which will be addressed in Chapter~\ref{chap:yongningnatonesinadynamicsynchronicperspective}.

Under the present analysis, the tone categories \mbox{//LM//} and \mbox{//LH//} contrast not only on monosyllables
but also on disyllables. These two tones surface identically except when the word is followed by
a~\is{clitics}clitic. For instance, //\ipa{bo˩mi˧}// ‘sow, female pig’ and //\ipa{bo˩ɬɑ˥}// ‘boar, male pig’ have the same surface phonological tone pattern both \is{form!in isolation}in isolation and when
followed by the \isi{copula}. In Alawua, L.M.L and L.H.L do not contrast at the surface phonological level, nor do L.M and L.H in final position within a~{tone group}; accordingly, the surface forms may be transcribed interchangeably as /\ipa{bo˩mi˧}/ or /\ipa{bo˩mi˥}/ for ‘sow’ and as /\ipa{bo˩mi˧ ɲi˩}/ or /\ipa{bo˩mi˥ ɲi˩}/ for ‘{is \mbox{(a/the)} sow’; and as /\ipa{bo˩ɬɑ˧}/ or /\ipa{bo˩ɬɑ˥}/ for ‘boar’ and /\ipa{bo˩ɬɑ˧ ɲi˩}/ or /\ipa{bo˩ɬɑ˥ ɲi˩}/ for ‘{is \mbox{(a/the)} boar’. The contexts that allow these two lexical tone categories to be distinguished are exemplified by /\ipa{bo˩mi˧=bv̩˧}/ ‘of \mbox{(a/the)} sow’ vs.\ /\ipa{bo˩ɬɑ˧=bv̩˩}/ (which
could also be transcribed as /\ipa{bo˩ɬɑ˥=bv̩˩}/) ‘of \mbox{(a/the)} boar’: in the latter phrase, the \isi{possessive} \is{clitics}clitic //\ipa{=bv̩˧}// receives L tone, as shown in \tabref{tab:examplesillustratingtheexistenceofthreetonecategoriesneutralizedtolminisolation} above.
%\footnote{Inconsistencies in the tonal behaviour of the word for ‘boar’ in the speech of consultant M21 differences observed across speakers led me to the hypothesis that the word for
%	‘boar’ may be an \is{exceptions}exception \citep[191]{michaud2008c}. But the existence of the opposition was later
%	confirmed in the speech of the consultant of reference, F4, in elicited combinations and also in
%	narratives.}

For the first category (‘sow’), an analysis of the tone pattern as LH is ruled out: if it were transcribed as /\ipa{bo˩mi˥}/, then the form with the \isi{possessive} would have to be
/\ipa{$\ddagger${\kern2pt}bo˩mi˥=bv̩˥}/ (‘of \mbox{(a/the)} sow’), since the \isi{possessive} surfaces with the same tonal level as the noun's second syllable; but the tone sequence H+H is never observed elsewhere
in the Alawua dialect of Yongning Na. By an exceptionless rule, a~syllable following an H-tone syllable receives L tone; this will be referred to in \sectref{sec:alistoftonerules} as “Rule~4”. 
%(the full set of rules is also provided in the ‘Quick reference’ section at the outset of this volume). 
In view of the above argument, ‘of \mbox{(a/the)} sow’ is transcribed as /\ipa{bo˩mi˧=bv̩˧}/, and the category illustrated by ‘sow’ is analyzed as \mbox{//LM//}, hence //\ipa{bo˩mi˧}//.

As for ‘boar’, a~phonological analysis as //\ipa{bo˩ɬɑ˥}// makes good phonological sense, since all following tones are lowered to L, as expected after an H tone. When a~word of this
tonal category is followed by the \isi{possessive} //\ipa{=bv̩˧}//, the latter carries L tone, just as after a~disyllable with H\# tone, as illustrated in~(\ref{ex:oftherat}-\ref{ex:oftheboar}), where notation as LH is adopted for ‘boar’.
\begin{exe}
  \ex \label{boarrat}
  \begin{xlist}
    \ex 
    \label{ex:oftherat}
    \ipaex{hwæ˧ʈʂæ˥=bv̩˩}\\
    \gll hwæ˧ʈʂæ˥	=bv̩˧\\
	rat	\textsc{poss}\\
    \glt ‘of \mbox{(a/the)} rat’

    \ex 
    \label{ex:oftheboar}
    \ipaex{bo˩ɬɑ˥=bv̩˩}\\
    \gll bo˩ɬɑ˥	=bv̩˧\\
    boar	\textsc{poss}\\
    \glt ‘of \mbox{(a/the)} boar’
  \end{xlist}
\end{exe}

Furthermore, all compounds involving the tone category of ‘boar’ (e.g.\ ‘boar’s head’, ‘boar’s blood’, and so on) have the same pattern, which can be described as L followed by H, then a sequence of L tones (L.H.L{\dots}). This
is again parallel to the \mbox{//H\#//} category, where compounds follow a pattern of M followed by H, then a~sequence of
L tones (M.H.L{\dots}).

Under the present analysis, both disyllables and monosyllables undergo \isi{neutralization} of the
\is{form!underlying}underlying \mbox{//LM//} and \mbox{//LH//} categories when realized \is{form!in isolation}in isolation. By convention, the result of this neutralization is transcribed as \mbox{/LH/}.\footnote{In the first edition of this book, \mbox{/LH/} was used for monosyllables~-- hence /\ipa{bo˩˥}/ for `pig' and /\ipa{ʐæ˩˥}/ for `leopard'~-- but \mbox{/LM/} was used for disyllables~-- hence /\ipa{bo˩mi˧}/ for `sow', /\ipa{bo˩ɬɑ˧}/ for `boar', and /\ipa{nɑ˩hĩ˧}/ for `Naxi'. The same treatment is now applied to the LM-vs.-LH opposition across words of one or two syllables, hence /\ipa{bo˩mi˥}/ for `sow', /\ipa{bo˩ɬɑ˥}/ for `boar', and /\ipa{nɑ˩hĩ˥}/ for `Naxi'.}

For disyllables of the ‘sow’ and ‘boar’ types, as for monosyllables of the ‘pig’ and ‘leopard’ types discussed at the outset of this section (\sectref{sec:twooptionsforanalysislmvslhorlmvslml}), an alternative to the {\mbox{//LM//}-vs.-\mbox{//LH//} analysis would be to posit an opposition between \mbox{//LM//} and \mbox{//LML//.} Under this approach, the tone pattern of ‘sow’ would be analyzed as \mbox{//LM//}, and that of ‘boar’ as
\mbox{//LML//}. This analysis equally captures the fact that the tones of syllables following nouns of the ‘boar’ type are lowered to L: by Rule~5 (“All syllables following an H.L or M.L sequence receive L tone”), L.M.L must be followed
by additional L tones. 

This was the analysis I initially adopted, including in the Yongning Na glossary deposited in 2011 in the Sino-Tibetan Etymological Dictionary and Thesaurus (STEDT). At the time, describing this tone category as
a~sequence of three levels did not seem excessively complex, given the presence of other intricate categories such as //L+MH// and //LM+\#H//. However, those two lexical tone
categories are composed of two distinct parts, one associated to the beginning of the word and the other to its
end, whereas \mbox{//LML//} would be the only pattern specifying three consecutive tonal levels. Moreover, it would be the
only category for which an \mbox{//ML//} sequence is posited. For these reasons, notation as \mbox{//LH//} is adopted here. 

That said, an analysis as \mbox{//LML//} remains a viable alternative. Such cases of analytical uncertainty are not merely abstract issues for phonological theorists: they also shed light on the system’s potential for change, as
\is{language acquisition}language learners must navigate these competing analytical options when constructing their own
phonological systems.


\subsection[Cases of neutralization of \mbox{//LM//} and \mbox{//LH//}]{Cases of neutralization of the opposition between \mbox{//LM//} and \mbox{//LH//}: Is the product /LM/ or /LH/?}
\label{sec:neutralizationoflmandlhinisolationistheproductlmorlh}

Monosyllables belonging to the \mbox{//LM//} and \mbox{//LH//} tone categories, such as //\ipa{bo˩˧}// ‘pig’ and //\ipa{ʐæ˩˥}// ‘leopard’,
are realized \is{form!in isolation}in isolation with the same low-rising \is{tonal contour}contour. Phonetically, both a~low-to-mid and a~low-to-high realization are acceptable. My consultants occasionally corrected my
productions of this tone category when the starting point was not low enough, as this could lead to confusion with \mbox{//MH//}; on the other hand, they never corrected me for a~mistaken
endpoint (whether too high or too low).

To investigate whether there was a~preference for a [Mid] or
[High] endpoint in the phonetic realization of this low-rising {contour}, I tried fishing for corrections from
Mrs.\ Latami (consultant F4). On several occasions, I deliberately produced two variants of words such as ‘pig’ and ‘leopard’,
both with a~low starting point: one with a~moderate rise (approximately up to F\textsubscript{0}
mid-range), and another with a~strong, rapid rise towards a~[High] final target. When asked which sounded better, the consultant consistently responded that both were correct (/\ipa{ɲi˧-bæ˧ {\kern2pt}|{\kern2pt} ho˩˥}/: \textit{two-\textsc{clf} correct}).

In this study, I choose to transcribe the product of the \isi{neutralization} of \mbox{//LM//} and \mbox{//LH//} as /LH/
at the surface phonological level. There is no decisive phonetic argument for this notation, however. The
outcome of tone neutralization tends to be phonetically less definite than that of consonantal {neutralization}. For instance, the opposition between coronal and retroflex stops in
Yongning Na is neutralized before /\ipa{ɯ}/, with the resulting sound clearly realized as a~retroflex: [\ipa{ʈʰɯ}] is a~well-formed syllable, whereas [\ipa{tʰɯ}] is not. On the other hand, the product
of the {neutralization} of //H// and \mbox{//M//} \is{form!in isolation}in isolation spans the entire phonetic space corresponding to these two tones: it is a~non-low tone, but it may not be appropriate to assign it a~more precise phonetic label, such as either ‘high’ or ‘mid’. (The topic of phonetic implementation of tone sequences is addressed in Chapter~\ref{chap:fromsurfacephonologicalformstophoneticrealizationintonationandtonalimplementation}.)

\subsection{On the anchoring of tones to word boundaries}
\label{sec:twoparttonecategorieslmhandlmmh}

The M and L tone categories of the Alawua dialect of Yongning Na are hypothesized to associate with the \textit{first} syllable of a~lexical item and to spread from there across the entire word. By contrast, the three categories of H tones have their mode of syllabic association specified respective to the \textit{last} syllable of the lexical item. There are thus tone categories that are \is{anchorage}anchored at the beginning of the word and others at its end. Two tone categories combine both anchorages: they are composed of two parts, with the first \is{anchorage}{anchored} at the beginning of the lexical word, and the second at its end. These are //LM+\#H// and //LM+MH\#//, exemplified in \tabref{tab:thelexicaltonesofdisyllabicnouns} by //\ipa{nɑ˩hĩ\#˥}// ‘Naxi’ and //\ipa{õ˩dv̩˧˥}// ‘wolf’, respectively. 

The ‘+' symbol in //LM+\#H// and //LM+MH\#// denotes the \is{juncture (inside a tone group)}juncture between the first and second parts of these tones. Thus, in the phrase /\ipa{nɑ˩hĩ˧ ɲi˥}/ ‘is \mbox{(a/the)} Naxi’, the H tone is interpreted as the manifestation
of a~\is{floating tone}floating H tone, and the lexical tone of this category is analyzed as //LM+\#H//: a~\mbox{//LM//}
\is{tonal contour}contour followed by a~\is{floating tone}floating H tone, \mbox{//\#H//}. Likewise, //LM+MH\#//, exemplified by //\ipa{õ˩dv̩˧˥}//
‘wolf’, is analyzed as a~tonal category consisting of two parts: a~\mbox{//LM//} tone followed by a~final \mbox{//MH//}
{contour}. In both cases, the pound symbol indicates the syllabic {anchoring} of the second part of
these two-part tone categories: beyond the lexical word boundary in //LM+\#H//, and on the last syllable of the lexical word for \mbox{//LM+MH\#//}. 

%\newpage 
These two tone categories may seem awfully complex, as each consists of two parts that associate at opposite ends of the word. This complexity is real, and probably goes a~long way towards explaining why only two such tone categories exist in Alawua, rather than the full range of theoretically possible combinations: for instance, there is no //LM+H\$// tone. However, seen from within the Na tone system, the behaviour of //LM+\#H// and //LM+MH\#// is not all that complicated, as it follows directly from that of their constituent elements. Each part of the tone behaves as it does when occurring independently, with its own mode of association. The mode of association of the first part (\mbox{//LM//}) is straightforward, as is that of the second part (\mbox{//\#H//} and \mbox{//MH\#//}, respectively). The ‘+' symbol in //LM+\#H// and //LM+MH\#// serves as a notational convention to reflect the fact that these tones consist of two separate components, each of which associates independently. Their complexity stems from the coexistence of these two anchoring patterns, rather than from any additional structural interaction between them.


\section{General observations about the system}
\label{sec:overviewofthesystemandsomereflectionsonitsstructure}

{\largerpage[-1]} % Added on April 27th, 2025

Some generalizations about the Yongning Na tonal system emerge from the observations made above. 

\begin{itemize}
    \item There are three tonal levels: H(igh), M(id) and L(ow).
    \item The tone-bearing unit is the syllable, more specifically the
syllable rhyme. There is no distinction in terms of syllable weight, and thus no need for
a~decomposition into \is{mora|textbf}moras: any syllable rhyme, including syllabic consonants, can function as
a~\is{tone-bearing unit|textbf}tone-bearing unit for one or two tonal levels.
    \item Among the six theoretically possible contours (HM, HL,
MH, ML, LH, and LM), only the three rising contours (MH, LH, and LM) are attested as lexical categories, exemplified in monosyllables by //\ipa{ʈʂʰæ˧˥}//
‘deer’, //\ipa{ʐæ˩˥}// ‘leopard’, and //\ipa{bo˩˧}// ‘pig’.
    \item At the surface phonological level, contours are restricted to tone-group-final position.
    \item At the surface phonological level, LM and LH are neutralized to
LH.
\end{itemize} 

Stated differently, each syllable within a~\isi{tone group} carries one of three levels, H, M, or L, while the final syllable carries one of the following: H, M, L, MH, or LH. There are no phonological
falling contours (HL, HM or ML) on a~single syllable, only rising contours.

The following paragraphs propose an overview of the system of lexical tones for nouns and offer some
reflections on its structure.

\subsection{Usefulness of an autosegmental approach}
\label{sec:autosegmentalphonology}

A first general observation that can safely be made in view of the data presented so far is that tone in Yongning Na is best analyzed within an \is{autosegmental phonology}autosegmental framework: that is, a model in which tones are \textit{auto}nomous from \textit{segment}s (i.e.\ vowels and consonants). Autosegmental models were originally developed for
West African tone systems, but have been convincingly applied to certain languages of the Tibeto-Burman area (see,
in particular, \citealt{hymanetal2002a}). The choice of this framework is motivated by language-internal evidence; it is by no means dictated by
\textit{a~priori} theoretical commitments. 

I have had the opportunity to learn and study two strikingly different tone
systems of Asia: that of Yongning Na, which has phonetically simple yet morphophonologically complex
tones, and that of Northern \ili{Vietnamese}, which has phonetically complex but
morphophonologically inert tones. In my view, it is clear that Yongning Na should be analyzed as a~level-tone system, unlike \ili{Vietnamese}, where “there are no objective reasons to decompose ({\dots}) tone contours into level tones or to reify phonetic properties like high and low pitch
into phonological units such as H and L” (\citealt{brunelle2009c}; see also
\citealt{brunelleetal2010,kirby2010,kirby2011}). This topic is discussed further in \sectref{sec:typologicalbackgroundtotheclassificationofyongningnatonesasleveltones}. 


\subsection{Recapitulation of the lexical tone categories}
\label{sec:overviewofthesystem}

Tables~\ref{tab:thelexicaltonesofmonosyllabicnouns} and \ref{tab:thelexicaltonesofdisyllabicnouns} set out an analysis of the six tone categories of \is{monosyllables}monosyllabic nouns and the eleven tone 
categories of disyllabic nouns. To date, no single morphosyntactic context that brings out all the
tonal contrasts of nouns has been found: each context brings out some oppositions while neutralizing others. 

For instance, addition of the \isi{copula} distinguishes \mbox{//M//} from \mbox{//\#H//}, a contrast that is neutralized \is{form!in isolation}in isolation. 
%(/M+L/ vs.\ /M+H/ for monosyllables, /M.M+L/ vs.\ /M.M+H/ for
%disyllables). This opposition is neutralized to /M/ and /M.M/ respectively \is{form!in isolation}in isolation. 
On the
other hand, addition of the \isi{copula} neutralizes tonal contrasts that are present \is{form!in isolation}in isolation among the disyllabic tone categories \mbox{//\#H//}, \mbox{//MH\#//} and \mbox{//H\$//}. All three yield /M.M+H/ in the \is{copula}copular frame, whereas \is{form!in isolation}in isolation, they
are realized as /M.M/, /M.MH/ and /M.H/, respectively. Thus, to determine the lexical tone of a~word, it is necessary to elicit it in several contexts. 

Tables~\ref{tab:thelexicaltonesofmonosyllabicnouns} and \ref{tab:thelexicaltonesofdisyllabicnouns} provide information on the tone
categories (i)~\is{form!in isolation}in isolation, (ii)~when followed by the \isi{copula} //\ipa{ɲi˩}//, in frame (\ref{ex:carrierthisisathe}), and (iii)~when followed by the \isi{possessive} \is{clitics}clitic //\ipa{=bv̩˧}//.
\begin{exe}
  \ex
  \label{ex:carrierthisisathe}
  \gll ʈʂʰɯ˧ {\_\_\_\_\_\_\_\_\_} ɲi\\
  \textsc{dem.prox} \textit{{target item}}	\textsc{cop}\\
  \glt	‘This is \mbox{(a/the)} \ipa{{\_\_\_\_\_\_\_\_\_}}.’
\end{exe}

A recording of \is{disyllables}disyllabic nouns in frame (\ref{ex:carrierthisisathe}) is available online under the identifier \textit{NounsInFrame} \pandoi{0004528}.

This set of three contexts is sufficient to bring out all tonal oppositions, except for the contrast between \mbox{//LM//}
and \mbox{//LH//} in monosyllables. This opposition only surfaces in highly restricted contexts. As mentioned in
\sectref{sec:disyllabicnouns}, one such context is in combination with the verb ‘to buy’. For instance, the \mbox{//LM//}-tone word
‘pig’ yields /\ipa{bo˩ hwæ˧-ze˧}/ ‘bought pigs’, whereas the \mbox{//LH//}-tone word ‘leopard’
yields /\ipa{ʐæ˩ hwæ˧-ze˩}/ ‘bought leopards’.

The proximal demonstrative //\ipa{ʈʂʰɯ˥}// always carries the same surface tone in (\ref{ex:carrierthisisathe}), regardless of the tone category
of the following noun. Consequently, only the tonal pattern of the target noun and copula is
indicated in Tables~\ref{tab:thelexicaltonesofmonosyllabicnouns} and \ref{tab:thelexicaltonesofdisyllabicnouns}. Conversely, no
tone is indicated for the \isi{copula} in frame (\ref{ex:carrierthisisathe}), as its surface tone varies according to the
tonal category of the noun.

Dots indicate syllable boundaries within the lexical word, while the ‘+’ symbol marks the
\is{juncture (inside a tone group)}juncture between the noun and a~following morpheme. For example, in
\tabref{tab:thelexicaltonesofdisyllabicnouns}, disyllabic L-tone nouns have the pattern L.LH in isolation and L.L+H when followed by either the \isi{copula} or the \isi{possessive} \is{clitics}clitic. The word for ‘dog’ exemplifies this: it appears as
/\ipa{kʰv̩˩mi˩˥}/ \is{form!in isolation}in isolation, yielding /\ipa{kʰv̩˩mi˩ ɲi˥}/ ‘is \mbox{(a/the)} dog’ and
/\ipa{kʰv̩˩mi˩=bv̩˥}/ ‘of \mbox{(a/the)} dog’. 
%The final H tone in /\ipa{kʰv̩˩mi˩˥}/ is due to a~general
%rule, discussed in \sectref{sec:ltonesexistenceofarepairphenomenonforallltonegroups}: tone groups containing only /L/ tones are not allowed by Yongning Na
%phonotactics; if a~\isi{tone group} only has /L/ tones, a~postlexical /H/ tone is added to its last
%syllable.

The leftmost column in the tables (“Analysis”) lists the lexical tone categories. 
%The three subsequent columns provide their surface phonological transcriptions. 
The penultimate column contains example words, transcribed according to
the phonological tone categories, following the conventions set out in \sectref{sec:thenotationoftonalcategoriesinlexicalentries}.


%\label{tab:thelexicaltonesofmonosyllabicanddisyllabicnouns} %% Commented out on April 30th, 2025: no subtables, only sequentially numbered tables in the entire volume.
\begin{table}[t!]
\caption{\label{tab:thelexicaltonesofmonosyllabicnouns}The lexical tone categories of monosyllabic nouns.}
\begin{tabularx}{\textwidth}{ P{20mm} P{19mm} Q Q P{19mm} Q }
  \lsptoprule
analysis & in isolation & +\textsc{cop} & +\textsc{poss} & //example// & meaning\\ \midrule
// LM // & LH & L+H & L+H & \ipa{bo˩˧}  & pig\\
// LH // & LH & L+H & L+H & \ipa{ʐæ˩˥}  & leopard\\
// M // & M & M+L & M+M & \ipa{lɑ˧} & tiger\\
// L // & M & L+LH & L+M & \ipa{jo˩} & sheep\\
// \#H // & M & M+H & M+M & \ipa{ʐwæ˥} & horse\\
// MH\# // & MH & M+H & M+H & \ipa{ʈʂʰæ˧˥} & deer\\
\lspbottomrule
\end{tabularx}
\end{table}

\begin{table}[t!]
\caption{\label{tab:thelexicaltonesofdisyllabicnouns}The lexical tone categories of disyllabic nouns.}
\begin{tabularx}{\textwidth}{ P{21mm} l Q Q P{19mm} Q }
  \lsptoprule
analysis & in isolation & +\textsc{cop} & +\textsc{poss} & //example// & meaning\\ \midrule
// M // & M.M & M.M+L & M.M+M & \ipa{po˧lo˧} & ram\\
// \#H // & M.M & M.M+H & M.M+M & \ipa{ʐwæ˧zo\#˥} & colt\\
// MH\# // & M.MH & M.M+H & M.M+H & \ipa{hwɤ˧li˧˥} & cat\\
// H\$ // & M.H & M.M+H & M.M+M & \ipa{kv̩˧ʂe˥\$} & flea\\
// H\# // & M.H & M.H+L & M.H+L & \ipa{hwæ˧ʈʂæ˥} & squirrel\\
// L // & L.LH & L.L+H & L.L+H & \ipa{kʰv̩˩mi˩} & dog\\
// L\# // & M.L & M.L+L & M.L+L & \ipa{dɑ˧ʝi˩} & mule\\
//LM+MH\#// & L.MH & L.M+H & L.M+H & \ipa{õ˩dv̩˧˥} & wolf\\
//LM+\#H// & L.H & L.M+H & L.M+M & \ipa{nɑ˩hĩ\#˥} & Naxi\\
// LM // & L.H & L.M+L & L.M+M & \ipa{bo˩mi˧} & sow\\
// LH // & L.H & L.H+L & L.H+L & \ipa{bo˩ɬɑ˥} & boar\\
\lspbottomrule
\end{tabularx}
\end{table}



In view of this picture of the noun tone system, the distributional observations made above can be flipped
around. Instead of stating that “a \is{monosyllables}monosyllabic noun that carries an M tone in
isolation may belong to one of three distinct \is{form!underlying}underlying categories”, we can now say that the
three non-\is{tonal contour}contour lexical tones~-- \mbox{//M//}, \mbox{//L//}, and \mbox{//\#H//}~-- all \is{neutralization}neutralize to /M/ when a~\is{monosyllables}monosyllable
is pronounced \is{form!in isolation}in isolation. Among disyllables, \mbox{//M//} and \mbox{//\#H//} \is{neutralization}neutralize to /M.M/; \mbox{//H\$//} and \mbox{//H\#//}
\is{neutralization}neutralize to /M.H/; and \mbox{//LM//}, \mbox{//LH//}, and //LM+\#H// \is{neutralization}neutralize to \mbox{/L.M/}.

When the \isi{possessive} \is{clitics}clitic //\ipa{=bv̩˧}// is added to a~\is{monosyllables}monosyllabic noun, as in /\ipa{ʈʂʰæ˧=bv̩˥}/ ‘of the deer’, tonal contours unfold over the two syllables of the resulting phrase. Specifically, \mbox{//LH//}
yields /L+H/ (as does \mbox{//LM//}, following neutralization with \mbox{//LH//}), and \mbox{//MH//} yields /M+H/. 
%The
%non-\is{tonal contour}contour tones, \mbox{//M//}, //L// and \mbox{//\#H//}, do not affect the \isi{possessive}, which surfaces with
%default /M/.

This last point provides evidence for a distinction between tonal contours (\mbox{//LM//}, \mbox{//LH//}, and
\mbox{//MH\#//}) on the one hand and the \is{floating tone}floating H tone (\mbox{//\#H//}) on the other. The second component of
a~\is{tonal contour}contour is realized on the \isi{possessive}, whereas the \is{floating tone}floating H tone is not.

Why does the {possessive} {clitic} receive an H level when the noun has an MH {contour} but not when the noun has a~floating H tone? One way to think of it~-- admittedly an \textit{ad hoc} hypothesis, based on the investigator's non-native intuition~-- is that anchoring a~\mbox{//\#H//} tone is quite a~different matter from hosting the H level of a~\mbox{//MH\#//} contour that is {anchored} to a~preceding syllable. The \mbox{//MH\#//} tone has a~stable {anchorage} on the noun's last syllable, allowing it to unfold in the usual way: its M part is realized on the noun's last syllable, while its H part lands on the {clitic}. The MH sequence is firmly moored to the syllabic string at one of its ends (the M part), and this mooring suffices for the contour as a whole to hold firm and not lose its second part. In an intuitive sense, the H level, as it were, {protrudes} from the noun, {jutting out} onto the clitic. By contrast, the \mbox{//\#H//} tone does not materialize because the necessary {anchorage} is not established in the first place. In \is{derivation!tonal}derivational terms, a~distinction must be made between \textit{tonal anchoring} and the later stage of \textit{{contour} unfolding}.

A step-by-step schematic representation is provided in \figref{fig:tonereassociation}, taking as an example a~disyllabic noun belonging to the \mbox{//MH\#//} tone category. Stage 1 represents the input: the noun has \mbox{//MH\#//} tone, and the \isi{copula} has //L// tone. %\footnote{For discussion of the {copula}'s tone, see \sectref{sec:thelexicaltonesofverbs}.} 
Stage 2 shows the \is{anchorage}anchoring of the MH tone pattern onto the last syllable of the lexical word, a specification indicated by the \# symbol in the label MH\#. Stage~3 depicts the one-to-one mapping of tonal levels to available syllables. The MH \is{tonal contour}contour associates sequentially, starting from its \isi{anchorage} point: the noun’s last syllable receives M, and the following syllable (the \isi{copula}) receives H. Since no syllable remains available for the \isi{copula}’s lexical L tone, it remains unassociated and does not surface. (Remember that there are no falling contours in Yongning Na: $\ddagger${\kern2pt}HL is not a possible sequence over a monosyllable, any more than $\ddagger${\kern2pt}HM or $\ddagger${\kern2pt}ML.) The noun's first syllable, left toneless in this process, is assigned a default M level in Stage 4. Stage 5 presents the final phonological surface form. 

\begin{figure}[p]
  \caption[{A detailed representation of tone-to-syllable association for ‘is (a/the) cat'.}]{A detailed representation of tone-to-syllable association for /\ipa{hwɤ˧li˧ ɲi˥}/ ‘is \mbox{(a/the)} cat'.}
  \begin{tikzpicture}
  \node (1) at (0.5,-0.5) {MH\#};
  \node (4) at (3,-0.5) {L};
  
  \node (2) at (0,-1.5) {σ};
  \node (3) at (1,-1.5) {σ};
  \node (5) at (3,-1.5) {σ};

  \node [anchor=mid] (s1l) at (0.5,-2) {/\ipa{hwɤ.li}/ ‘cat’};
%  \node (s1ll) at (0.5,-2.5) {lexical tone: MH\#};

  \node [anchor=mid] (s1lll) at (3,-2) {/\ipa{ɲi}/ \textsc{copula}};
%  \node (s1llll) at (4,-2.5) {lexical tone: L};
  
  \node[text width=40mm] (s1) at (-3,-0.75) {Stage 1:\\ input};


  
  \node (12) at (0.5,-4) {MH\#};
  \node (42) at (2,-4) {L};
  
  \node (22) at (0,-5.5) {σ};
  \node (32) at (1,-5.5) {σ};
  \node (52) at (2,-5.5) {σ};

  \node[text width=40mm] (s2) at (-3,-4.75) {Stage 2:\\ \is{anchorage}anchoring of MH\# to\\ its
    phonologically\\ specified locus};

  \draw[decoration={markings,mark=at position 1 with
      {\arrow[scale=2,>=stealth]{>}}},postaction={decorate}] (12) -- (32);
  


  \node (13) at (1,-7) {M};
  \node (63) at (1.5,-7) {H};
  \node (43) at (2,-7) {L};
  
  \node (23) at (0,-8.5) {σ};
  \node (33) at (1,-8.5) {σ};
  \node (53) at (2,-8.5) {σ};

  \node[text width=40mm] (s3) at (-3,-7.75) {Stage 3:\\ one-to-one mapping\\ of levels to available syllables};

  \draw[decoration={markings,mark=at position 1 with
      {\arrow[scale=2,>=stealth]{>}}},postaction={decorate}] (13) -- (33);
  \draw[decoration={markings,mark=at position 1 with
      {\arrow[scale=2,>=stealth]{>}}},postaction={decorate}] (63) -- (53);


  \node (14) at (0,-10) {M};
  \node (64) at (1,-10) {M};
  \node (44) at (2,-10) {H};
  
  \node (24) at (0,-11.5) {σ};
  \node (34) at (1,-11.5) {σ};
  \node (54) at (2,-11.5) {σ};

  \node[text width=40mm] (s4) at (-3,-10.5) {Stage 4:\\ addition of default\\ M tone};

  \draw[decoration={markings,mark=at position 1 with
      {\arrow[scale=2,>=stealth]{>}}},postaction={decorate}] (14) -- (24);
  \draw (64) -- (34);
  \draw (44) -- (54);


  \node (14) at (0,-13) {M};
  \node (64) at (1,-13) {M};
  \node (44) at (2,-13) {H};
  
  \node (24) at (0,-14.5) {σ};
  \node (34) at (1,-14.5) {σ};
  \node (54) at (2,-14.5) {σ};

  \node[text width=40mm] (s4) at (-3,-13.5) {Stage 5:\\ resulting surface\\ phonological tone};

  \draw (14) -- (24);
  \draw (64) -- (34);
  \draw (44) -- (54);
\end{tikzpicture}
\label{fig:tonereassociation}
\end{figure}

\figref{fig:tonereassociationMHclitic} illustrates an analogous process for the \isi{possessive} \is{clitics}clitic //\ipa{=bv̩˧}// in this tonal category of nouns.

\begin{figure}[p]
	\caption[{A detailed representation of tone-to-syllable association for ‘of (a/the) cat'.}]{A detailed representation of tone-to-syllable association for /\ipa{hwɤ˧li˧=bv̩˥}/ ‘of \mbox{(a/the)} cat'.}
	\begin{tikzpicture}
	\node (1) at (0.5,-0.5) {MH\#};
	\node (4) at (3,-0.5) {M};
	
	\node (2) at (0,-1.5) {σ};
	\node (3) at (1,-1.5) {σ};
	\node (5) at (3,-1.5) {σ};
	
	\node [anchor=mid] (s1l) at (0.5,-2) {/\ipa{hwɤ.li}/ ‘cat’};
	%  \node (s1ll) at (0.5,-2.5) {lexical tone: MH\#};
	
	\node [anchor=mid] (s1lll) at (3,-2) {/\ipa{bv̩}/ \textsc{poss}};
	%  \node (s1llll) at (4,-2.5) {lexical tone: L};
	
	\node[text width=40mm] (s1) at (-3,-0.75) {Stage 1:\\ input};
	
	
	
	\node (12) at (0.5,-4) {MH\#};
	\node (42) at (2,-4) {M};
	
	\node (22) at (0,-5.5) {σ};
	\node (32) at (1,-5.5) {σ};
	\node (52) at (2,-5.5) {σ};
	
	\node[text width=40mm] (s2) at (-3,-4.75) {Stage 2:\\ \is{anchorage}anchoring of MH\# to\\ its
		phonologically\\ specified locus};
	
	\draw[decoration={markings,mark=at position 1 with
		{\arrow[scale=2,>=stealth]{>}}},postaction={decorate}] (12) -- (32);
	
	
	
	\node (13) at (1,-7) {M};
	\node (63) at (1.5,-7) {H};
	\node (43) at (2,-7) {M};
	
	\node (23) at (0,-8.5) {σ};
	\node (33) at (1,-8.5) {σ};
	\node (53) at (2,-8.5) {σ};
	
	\node[text width=40mm] (s3) at (-3,-7.75) {Stage 3:\\ one-to-one mapping\\ of levels to available syllables};
	
	\draw[decoration={markings,mark=at position 1 with
		{\arrow[scale=2,>=stealth]{>}}},postaction={decorate}] (13) -- (33);
	\draw[decoration={markings,mark=at position 1 with
		{\arrow[scale=2,>=stealth]{>}}},postaction={decorate}] (63) -- (53);
	
	
	\node (14) at (0,-10) {M};
	\node (64) at (1,-10) {M};
	\node (44) at (2,-10) {H};
	
	\node (24) at (0,-11.5) {σ};
	\node (34) at (1,-11.5) {σ};
	\node (54) at (2,-11.5) {σ};
	
	\node[text width=40mm] (s4) at (-3,-10.5) {Stage 4:\\ addition of default\\ M tone};
	
	\draw[decoration={markings,mark=at position 1 with
		{\arrow[scale=2,>=stealth]{>}}},postaction={decorate}] (14) -- (24);
	\draw (64) -- (34);
	\draw (44) -- (54);
	
	
	\node (14) at (0,-13) {M};
	\node (64) at (1,-13) {M};
	\node (44) at (2,-13) {H};
	
	\node (24) at (0,-14.5) {σ};
	\node (34) at (1,-14.5) {σ};
	\node (54) at (2,-14.5) {σ};
	
	\node[text width=40mm] (s4) at (-3,-13.5) {Stage 5:\\ resulting surface\\ phonological tone};
	
	\draw (14) -- (24);
	\draw (64) -- (34);
	\draw (44) -- (54);
	\end{tikzpicture}
	\label{fig:tonereassociationMHclitic}
\end{figure}

By contrast with MH\#, the \is{floating tone}floating H tone (\#H) does not \is{anchorage}anchor to any of the syllables of the word to which it is lexically attached, and the \isi{possessive} \is{clitics}clitic is unable to provide such \isi{anchorage}. Since this H tone receives no syllabic \isi{anchorage}, either from the word to which it is lexically attached or from the \isi{possessive} \is{clitics}clitic that follows it, it remains unassociated and does not surface at all in this context. This is represented as \figref{fig:tonereassociationfloating}. The figure for the ‘flea' tone, H\$, would be identical with \figref{fig:tonereassociationfloating}: the \isi{possessive} \is{clitics}clitic is not a~suitable host, so the H\$ remains unassociated and does not surface at all in this context. 


\begin{figure}[p]
	\caption[{A detailed representation of tone-to-syllable association for ‘of (a/the) colt'.}]{A detailed representation of tone-to-syllable association for /\ipa{ʐwæ˧zo˧{\allowbreak}=bv̩˧}/ ‘of \mbox{(a/the)} colt'.}
	\begin{tikzpicture}
	\node (1) at (0.5,-0.5) {\#H};
	\node (4) at (3,-0.5) {M};
	
	\node (2) at (0,-1.5) {σ};
	\node (3) at (1,-1.5) {σ};
	\node (5) at (3,-1.5) {σ};
	
	\node [anchor=mid] (s1l) at (0.5,-2) {/\ipa{ʐwæ.zo}/ ‘colt’};
	%  \node (s1ll) at (0.5,-2.5) {lexical tone: MH\#};
	
	\node [anchor=mid] (s1lll) at (3,-2) {/\ipa{bv̩}/ \textsc{poss}};
	%  \node (s1llll) at (4,-2.5) {lexical tone: L};
	
	\node[text width=40mm] (s1) at (-3,-0.75) {Stage 1:\\ input};
	
	
	
	\node (12) at (0.5,-4) {\#H};
	\node (42) at (2,-4) {M};
	
	\node (22) at (0,-5.5) {σ};
	\node (32) at (1,-5.5) {σ};
	\node (52) at (2,-5.5) {σ};
	
	\node[text width=40mm] (s2) at (-3,-4.75) {Stage 2:\\ failure of \#H to get\\ \isi{anchorage}, for want of\\ a~suitable host};
	
%	\draw[decoration={markings,mark=at position 1 with {\arrow[scale=2,>=stealth]{>}}},postaction={decorate}] (12) -- (52);
	
	
	
%	\node (13) at (1,-7) {};
%	\node (63) at (1.5,-7) {};
	\node (43) at (2,-7) {M};
	
	\node (23) at (0,-8.5) {σ};
	\node (33) at (1,-8.5) {σ};
	\node (53) at (2,-8.5) {σ};
	
	\node[text width=40mm] (s3) at (-3,-7.75) {Stage 3:\\ one-to-one mapping\\ of levels to available syllables};
	
	\draw[decoration={markings,mark=at position 1 with
		{\arrow[scale=2,>=stealth]{>}}},postaction={decorate}] (43) -- (53);
	
	
	\node (14) at (0,-10) {M};
	\node (64) at (1,-10) {M};
	\node (44) at (2,-10) {M};
	
	\node (24) at (0,-11.5) {σ};
	\node (34) at (1,-11.5) {σ};
	\node (54) at (2,-11.5) {σ};
	
	\node[text width=40mm] (s4) at (-3,-10.5) {Stage 4:\\ addition of default\\ M tones};
	
	\draw[decoration={markings,mark=at position 1 with
		{\arrow[scale=2,>=stealth]{>}}},postaction={decorate}] (14) -- (24);
	\draw[decoration={markings,mark=at position 1 with
		{\arrow[scale=2,>=stealth]{>}}},postaction={decorate}] (64) -- (34);
	\draw (44) -- (54);
	
	
	\node (14) at (0,-13) {M};
	\node (64) at (1,-13) {M};
	\node (44) at (2,-13) {M};
	
	\node (24) at (0,-14.5) {σ};
	\node (34) at (1,-14.5) {σ};
	\node (54) at (2,-14.5) {σ};
	
	\node[text width=40mm] (s4) at (-3,-13.5) {Stage 5:\\ resulting surface\\ phonological tone};
	
	\draw (14) -- (24);
	\draw (64) -- (34);
	\draw (44) -- (54);
	\end{tikzpicture}
	\label{fig:tonereassociationfloating}
\end{figure}

The behaviour of the word-final H tone (H\#) is shown in \figref{fig:tonereassociationfinal}. In this case, the L tone on the \is{clitics}clitic results from an exceptionless {phonological rule} whereby all tones following H are lowered to L (see \sectref{sec:alistoftonerules}). 

\begin{figure}[p]
	\caption[{A detailed representation of tone-to-syllable association for ‘of (a/the) squirrel'.}]{A detailed representation of tone-to-syllable association for /\ipa{hwæ˧ʈʂæ˥{\allowbreak}=bv̩˩}/ ‘of \mbox{(a/the)} squirrel'.}
	\begin{tikzpicture}
		\node (1) at (0.5,-0.5) {H\#};
		\node (4) at (3.5,-0.5) {M};
		
		\node (2) at (0,-1.5) {σ};
		\node (3) at (1,-1.5) {σ};
		\node (5) at (3.5,-1.5) {σ};
		
		\node [anchor=mid] (s1l) at (0.5,-2) {/\ipa{hwæ.ʈʂæ}/ ‘squirrel’};
		%  \node (s1ll) at (0.5,-2.5) {lexical tone: MH\#};
		
		\node [anchor=mid] (s1lll) at (3.5,-2) {/\ipa{bv̩}/ \textsc{poss}};
	%	\node [anchor=mid] (s1lll) at (3,-2) {/\ipa{bv̩}/ \textsc{poss}};
		%  \node (s1llll) at (4,-2.5) {lexical tone: L};
		
		\node[text width=55mm] (s1) at (-3,-0.75) {Stage 1:\\ input};
		
		
		
		\node (12) at (1.5,-4) {H\#};
		\node (42) at (3,-4) {M};
		
		\node (22) at (1,-5.5) {σ};
		\node (32) at (2,-5.5) {σ};
		\node (52) at (3,-5.5) {σ};
		
		\node[text width=55mm] (s2) at (-3,-4.75) {Stage 2:\\ \is{anchorage}anchoring of H\# to its
			phonologically specified locus};
		
		\draw[decoration={markings,mark=at position 1 with
			{\arrow[scale=2,>=stealth]{>}}},postaction={decorate}] (12) -- (32);
		
	%	\draw[decoration={markings,mark=at position 1 with {\arrow[scale=2,>=stealth]{>}}},postaction={decorate}] (42) -- (52);
		
		
		
		\node (13) at (2,-7) {H};
	%	\node (63) at (1.5,-7) {H};
		\node (43) at (3,-7) {M};
		
		\node (23) at (1,-8.5) {σ};
		\node (33) at (2,-8.5) {σ};
		\node (53) at (3,-8.5) {σ};
		
		\node[text width=55mm] (s3) at (-3,-7.75) {Stage 3:\\ syllabic anchoring \\ of the H level};
		
		\draw[decoration={markings,mark=at position 1 with
			{\arrow[scale=2,>=stealth]{>}}},postaction={decorate}] (13) -- (33);
	%	\draw[decoration={markings,mark=at position 1 with {\arrow[scale=2,>=stealth]{>}}},postaction={decorate}] (43) -- (53);
		
		
		%\node (14) at (0,-10) {M};
		\node (64) at (2,-10) {H};
		\node (44) at (3,-10) {L};
		
		\node (24) at (1,-11.5) {σ};
		\node (34) at (2,-11.5) {σ};
		\node (54) at (3,-11.5) {σ};
		
		\node[text width=55mm] (s4) at (-3,-10.75) {Stage 4: assignment of final L by {phonological rule}: H can only be followed by L. The suffix's lexical M is deleted.};
		
		% \draw (14) -- (24);
		\draw (64) -- (34);
		%	\draw (44) -- (54);
		\draw[decoration={markings,mark=at position 1 with
			{\arrow[scale=2,>=stealth]{>}}},postaction={decorate}] (44) -- (54);

		% beginning of stage 5
		\node (14) at (1,-13) {M};
		\node (64) at (2,-13) {H};
		\node (44) at (3,-13) {L};
		
		\node (24) at (1,-14.5) {σ};
		\node (34) at (2,-14.5) {σ};
		\node (54) at (3,-14.5) {σ};
		
		\node[text width=55mm] (s4) at (-3,-13.75) {Stage 5:\\ addition of default M tone to the syllable that remained toneless};
		
		\draw[decoration={markings,mark=at position 1 with
			{\arrow[scale=2,>=stealth]{>}}},postaction={decorate}] (14) -- (24);
		\draw (64) -- (34);
		\draw (44) -- (54);
	\end{tikzpicture}
	\label{fig:tonereassociationfinal}
\end{figure}

These step-by-step representations provide more detail than tonologists with experience in level tones may find necessary. It does not appear indispensable to draw similar figures for the other lexical categories of Yongning Na, though this could be offered as an exercise for an introductory phonology class. 
%(If you are reading this during my lifetime, you are welcome to get in touch to discuss issues of Yongning Na tonology and tonological representations.) 

%\Hack{\newpage}

\subsection{M as a~default tone}
\label{sec:analysisofmasadefaulttone}

The analysis set out above assumes that M serves as a~default tone: syllables that are not specified for
tone receive M. For instance, a~\mbox{//\#H//} tone carried by a~disyllabic noun can only manifest itself
on a~following word: the H tone, though lexically attached to the noun, never appears on the noun
itself. Both syllables of the noun receive /M/ tone in the surface phonological form. Under the
present analysis, this is understood as default tone assignment. Likewise, /M.L/ is observed as
a~surface pattern on disyllabic and polysyllabic words, such as /\ipa{dɑ˧ʝi˩}/ ‘mule’, but this
pattern is analyzed as the manifestation of a~lexical-word-final L tone (notation: //L\#//): the /M/
tone on the first syllable is analyzed as a~default tone, not a~lexically specified one. Evidence for this
analysis will be presented in the course of the discussion, drawing on the combinatorial properties
of tones, such as the tone rules that apply in compounding.

One may be tempted to push this analysis further and attempt to eliminate the specification of M tone altogether in
the inventory of lexical tones, for instance dispensing with the M component in the analysis of the \mbox{//LM//} tone category, illustrated by 
% As stated above, disyllables such as /\ipa{dɑ˧ʝi˩}/ ‘mule’, with
% a~surface phonological M.L pattern, are analyzed as having a~final L (notation: //L\#//), the M tone
% on the first syllable being a~default tone. By contrast, 
disyllables such as /\ipa{bo˩mi˧}/, with surface
L.M. There is, however, a~motivation for considering the M tone of
their second syllable as phonologically specified at the underlying lexical level: this M tone blocks L-tone \is{tone spreading}spreading.
% ... are analyzed as having a~phonological \mbox{//LM//} tone: at the underlying phonological level, the M tone of
% their second syllable is specified. 

The background to this analysis is the observation that L tone spreads progressively
(“left-to-right”) onto syllables that are unspecified for tone (this is referred to as Rule~1; see
\sectref{sec:asummaryoftonetosyllableassociationrules} for further detail). In \tabref{tab:thelexicaltonesofdisyllabicnouns}, disyllables that carry L tone on both of their syllables
are accordingly analyzed as possessing a~simple lexical L tone. In transcriptions, L tone is
indicated on both syllables by convention (e.g.~the word for ‘dog’ is written //\ipa{kʰv̩˩mi˩}//), so as to stay
close to \is{form!surface}surface forms, but at a~lexical level, these words are analyzed as carrying a~simple \mbox{//L//}
tone. Disyllabic nouns that have a~\mbox{/L.M/} pattern (L on the first syllable, M on the second) are
analyzed as possessing a~phonological \mbox{/M/} that blocks L-tone \is{tone spreading}spreading.

Under an approach that dispenses with M at the lexical level, the syllabic \is{anchorage}anchoring of all L tones
would need to be specified. This would, to some extent, parallel H tones, which fall into three categories with different modes of
association to the syllabic string: H\#, \#H, and H\$. However, if the L.M pattern were
reanalyzed as a~word-initial L tone, it would be necessary to specify that it does not spread,
unlike other L tones. Reanalyzing the \mbox{//LM//} category as a~non-\is{tone spreading}spreading L, contrasting with
a~\is{tone spreading}spreading L, is a~theoretical possibility, but one which (at present) appears to me to be less consistent with the rest of the description than positing a~combination of two levels as the underlying lexical tone. 

%{\largerpage} % Added on April 21st, 2025

Yet another alternative would be to analyze the /L.H/, /M.L/, and /L.L/ surface patterns as the
realization of initial //L//, final //L//, and //L.L// (with L tone specified on both syllables),
respectively, thereby avoiding any reference to L-tone \is{tone spreading}spreading. However, given that L-tone \is{tone spreading}spreading is such
a~commonly attested process in the Alawua dialect of Yongning Na, this does not appear to be a~promising line of analysis. 

Moreover, any description that avoids M tones at the lexical level would also require an alternative means of accounting for MH \is{tonal contour}contour tones. It would then be necessary to posit a~separate type of H tones:
a~{contour}-creating H tone, in addition to the three types recognized so far. This is hardly an appealing analytical option.

For these various reasons, it seems reasonable to adopt a~model in which M is included in the lexical
specification of certain tonal categories.


\subsection{The notation of tonal categories in lexical entries}
%\largerpage
\label{sec:thenotationoftonalcategoriesinlexicalentries}

{\largerpage[-1]} % Added on April 27th, 2025

This section explains the choices made for the notation of tonal categories in lexical entries
(as headwords in dictionary entries and in the interlinear glossing of texts), as exemplified for nouns in Tables~\ref{tab:thelexicaltonesofmonosyllabicnouns} and \ref{tab:thelexicaltonesofdisyllabicnouns}.

One typographical option would be to indicate the phonological category in superscript at the beginning or end of
the word, e.g.~//\ipa{ʐwæ\textsuperscript{\#H}}// for ‘horse’ and //\ipa{õ.dv̩\textsuperscript{LM+MH\#}}// for ‘wolf’. This notation, which
separates tone from vowels and consonants, makes good sense in view of the analysis proposed here, according to which tone in Na is lexically associated with entire lexemes,
not individual syllables. However, working out the tone-to-syllable mapping requires
complete familiarity with the tonal rules of Yongning Na. A~transcription
that more closely resembles \is{form!surface}surface phonology therefore seemed preferable, marking tone at the end of each syllable. \is{International Phonetic Alphabet|textbf}International Phonetic Alphabet tone-letters \citep{chao1930} were chosen: \ipa{˥} for
High, \ipa{˧} for Mid, \ipa{˩} for Low, \ipa{˩˧} for Low-to-Mid, \ipa{˩˥} for Low-to-High, and \ipa{˧˥} for Mid-to-High.

This system is strictly equivalent to the Africanist notation that uses accents. For instance, //\ipa{bo˩˧}// ‘pig’
could be written as //\ipa{bo᷄}// in Africanist notation, and //\ipa{ʐæ˩˥}// ‘leopard’ as //\ipa{ʐæ̌}//. \is{tone-letters|textbf}Tone-letters are preferred over accents for want
of a~satisfactory solution to the typographic issue of diacritic combinations, such as how to
indicate a~rising \is{tonal contour}contour on a~syllable like /\ipa{ɻ̩̃}{\kern2pt}/.

In China, numbers rather than 
%\is{tone-letters}
tone-letters have become standard in phonetic notation. The strict numerical equivalents are as follows: \ipa{˥} corresponds to \ipa{⁵}, \ipa{˧} to \ipa{³}, \ipa{˩} to \ipa{¹}, \ipa{˩˧} to \ipa{¹³}, \ipa{˩˥} to \ipa{¹⁵}, and \ipa{˧˥} to
\ipa{³⁵}. However, this is not how the numerical system is actually used in China. In order to provide the necessary background and clarify the complexities associated with the Chinese use of the numerical system, a detour through the history of Chao Yuen-ren's initial proposal and its adoption in China is in order.

In his initial proposal to the International Phonetic Association, Chao Yuen-ren introduced constraints to limit the full combinatorial potential of the five-point scale. “In order not to make
distinctions too fine, points 2 and 4 are used either alone or with each other, but not in combination
with 1, 3, or 5” \citep[25]{chao1930}. This restriction reduced the number of proposed tone-letters to thirteen (rather than twenty-five) for ``straight tones''~-- tones composed of two pitch targets, which may be identical or different. Additional restrictions applied to ``circumflex tones'', which involve three pitch targets. Not only were the \{1, 3, 5\} and \{2, 4\} sets of levels never allowed to mix, but certain sequences were quietly excluded, presumably on grounds of phonetic implausibility. These include \ipa{˩˥˩} \textit{(bottom-top-bottom)}, \ipa{˩˥˥} \textit{(bottom-top-top)}, and their mirror images \ipa{˥˩˥} \textit{(top-bottom-top)}, \ipa{˥˩˩} \textit{(top-bottom-bottom)}, \ipa{˥˥˩} \textit{(top-top-bottom)}, and \ipa{˩˩˥} \textit{(bottom-bottom-top)}. \tabref{tab:ChaoLetters1930} reproduces the full set from the original publication. % (with English text converted from IPA to orthography).

\begin{table}
\caption{Full set of tone-letters proposed in \citet[25]{chao1930}.}
\begin{tabularx}{\textwidth}{@{} X X @{\hspace{1em}} X X @{\hspace{1em}} X X @{}}
\lsptoprule
    \multicolumn{2}{l}{straight tones} & \multicolumn{2}{l}{circumflex tones} & \multicolumn{2}{l}{short tones}\\
   \cmidrule(r){1-2} \cmidrule(r){3-4} \cmidrule(r){5-6}
    tone-letter & name & tone-letter & name & tone-letter & name \\
   \cmidrule(r){1-2} \cmidrule(r){3-4} \cmidrule(r){5-6}
    \ipa{˩˩} & 11 & \ipa{˩˧˩} & 131 & \ipa{˩} & 1 \\
    \ipa{˩˧} & 13 & \ipa{˩˥˧} & 153 & \ipa{˨} & 2 \\
    \ipa{˩˥} & 15 & \ipa{˨˦˨} & 242 & \ipa{˧} & 3 \\
    \ipa{˨˨} & 22 & \ipa{˧˩˧} & 313 & \ipa{˦} & 4 \\
    \ipa{˨˦} & 24 & \ipa{˧˩˥} & 315 & \ipa{˥} & 5 \\
    \ipa{˧˩} & 31 & \ipa{˧˥˩} & 351 & & \\
    \ipa{˧˧} & 33 & \ipa{˧˥˧} & 353 & & \\
    \ipa{˧˥} & 35 & \ipa{˦˨˦} & 424 & & \\
    \ipa{˦˨} & 42 & \ipa{˥˩˧} & 513 & & \\
    \ipa{˦˦} & 44 & \ipa{˥˧˥} & 535 & & \\
    \ipa{˥˩} & 51 & & & & \\
    \ipa{˥˧} & 53 & & & & \\
    \ipa{˥˥} & 55 & & & & \\
\lspbottomrule
\end{tabularx}
\label{tab:ChaoLetters1930}
\end{table}

% \begin{table}
% \caption{Full set of tone-letters proposed in \citet[25]{chao1930}.}
% \begin{tabularx}{\textwidth}{ l l P{27mm} l l Q }
% \lsptoprule
% 	\multicolumn{2}{l}{straight tones} & \multicolumn{2}{l}{circumflex tones} & \multicolumn{2}{l}{short tones}\\
%    \cmidrule(r){1-2} \cmidrule(r){3-4} \cmidrule(l){5-6}
% 	tone-letter & name & tone-letter & name & tone-letter & name \\
%    \cmidrule(r){1-2} \cmidrule(r){3-4} \cmidrule(l){5-6}
% 	\ipa{˩˩} & 11 & \ipa{˩˧˩} & 131 & \ipa{˩} & 1 \\
% 	\ipa{˩˧} & 13 & \ipa{˩˥˧} & 153 & \ipa{˨} & 2 \\
% 	\ipa{˩˥} & 15 & \ipa{˨˦˨} & 242 & \ipa{˧} & 3 \\
% 	\ipa{˨˨} & 22 & \ipa{˧˩˧} & 313 & \ipa{˦} & 4 \\
% 	\ipa{˨˦} & 24 & \ipa{˧˩˥} & 315 & \ipa{˥} & 5 \\
% 	\ipa{˧˩} & 31 & \ipa{˧˥˩} & 351 & & \\
% 	\ipa{˧˧} & 33 & \ipa{˧˥˧} & 353 & & \\
% 	\ipa{˧˥} & 35 & \ipa{˦˨˦} & 424 & & \\
% 	\ipa{˦˨} & 42 & \ipa{˥˩˧} & 513 & & \\
% 	\ipa{˦˦} & 44 & \ipa{˥˧˥} & 535 & & \\
% 	\ipa{˥˩} & 51 & & & & \\
% 	\ipa{˥˧} & 53 & & & & \\
% 	\ipa{˥˥} & 55 & & & & \\
% \lspbottomrule
% \end{tabularx}
% \label{tab:ChaoLetters1930}
% \end{table}

The inventory in \tabref{tab:ChaoLetters1930} was never implemented with full consistency, however. In \textit{A grammar of Modern Chinese} \citep[]{chao1968}, Chao Yuen-ren himself deviated from his earlier principle of not blending the \{1, 3, 5\} and \{2, 4\} levels by proposing the notation \ipa{²¹⁴} (\ipa{˨˩˦}) for the third tone of Beijing Mandarin~-- a convention that later became standard.\footnote{\citet[110]{waisumetal2003} propose \ipa{²¹³} (\ipa{˨˩˧}) instead of \ipa{²¹⁴} (\ipa{˨˩˦}), similarly disregarding the earlier restriction on mixing the \{1, 3, 5\} and \{2, 4\} subsets.} The lack of an explicit, principled basis for the original inventory ultimately limited its practical applicability. 

Once these restrictions were relaxed, however, the use of tone-letters became impractical. A~set consisting of five short tones, twenty-five ``linear'' tones and a hundred and twenty-five ``circumflex'' tones was manageable neither in handwritten form nor in the pre-digital, pre-Unicode typographical world. 
% the tone-letters became really difficult to interpret visually and to handle typographically. 
This led to the widespread adoption of numerical notation, particularly in large-scale dialect surveys \citep[45-58]{diaocha_1957}. 
% Compounded with the issue of typesetting special characters in pre-Unicode days, it led to uniform adoption of numerical notation. 
In the process, however, the original distinction between short, straight, and circumflex tones was lost. 
%But in the process of turning to numerical notation, the three-way distinction made in \tabref{tab:ChaoLetters1930} between short tones, straight tones and circumflex tones was lost. 

The convention that emerged in China required at least two numbers for each tone: one for its beginning and one for its endpoint. This practice is implicitly based on the notation adopted for Chinese tones, whereby the four-way tonal contrast in Beijing Mandarin is transcribed using the labels \ipa{⁵⁵}, \ipa{²⁴}, \ipa{²¹⁴}, and \ipa{⁵¹}. Such a convention creates a difficulty when transcribing level-tone languages such as \ili{Japanese}, \ili{Pumi} or Yongning Na. To reflect the insight that there is no phonologically relevant pitch movement in the course of the H, M, and L tones in Yongning Na, one might consider doubling the number indicating each tone's relative pitch level, transcribing the H tone as \ipa{⁵⁵}, the M tone as \ipa{³³}, and the L tone as \ipa{¹¹}. But this notation would be incongruent with the conventions used in China, where the numbers are selected to reflect pitch curves, not abstract phonological entities. The systematic use of at least two numbers in the standard five-point tonal notation encourages linguists to pay attention to any differences in pitch between a~tone's beginning and endpoint~-- differences that are crucial for systems where tones do not simply consist of sequences of level pitches (see \sectref{sec:typologicalbackgroundtotheclassificationofyongningnatonesasleveltones}), but irrelevant for level-tone systems. For instance, since an L tone is often realized phonetically (in Yongning Na, as also in other level-tone languages) as a~pitch fall rather than a~flat, sustained low pitch, it is typically transcribed by {Chinese} linguists as \ipa{²¹} or \ipa{³¹}. (Thus, the L tone of \ili{Naxi} is transcribed as \ipa{³¹} by \citealt{heetal1985}.) Notation as \ipa{¹¹} would not accurately reflect the phonetic realization of the Low tone and would thus be counterintuitive for users accustomed to this system. On the other hand, notation as \ipa{³¹} would obscure the phonological nature of the L tone, misleadingly suggesting that the tones transcribed as \ipa{³¹} (L in the present description) and \ipa{¹³} (LM) are mirror images of each other.\footnote{The present reflections on notational choices for level tones and their influence on the phonological analysis of tone in {Naish} languages build on a~talk given at the {International} {Conference} on {Phonetics} of the {Languages} in {China}  \citep{michaud2013c}. The same point is made by Jackson Sun about \ili{Va}, a three-tone language of the Austroasiatic family: ``Va distinguishes three tonal levels: high, mid, and low. What matters in this system
is relative pitch register, as the actual pitch contours vary according to syllable type
and the position of a syllable inside a word. ({\dots}) It is therefore misleading
to assign steady tone letters ({\dots}) or tone numerals ({\dots})
to underlying tones in Va" \citep[139]{sun2018_va}.}

For tonologists accustomed to the {Chinese}-style numerical notation, I therefore strongly recommend the following set of equivalences: \ipa{⁵} for \ipa{˥}, \ipa{³} for \ipa{˧}, \ipa{¹} for \ipa{˩}, \ipa{¹³} for \ipa{˩˧}, \ipa{¹⁵} for \ipa{˩˥}, and \ipa{³⁵} for \ipa{˧˥}. Tables~\ref{tab:thelexicaltonesofmonosyllabicnounsInToneNumbers} and \ref{tab:thelexicaltonesofdisyllabicnounsInToneNumbers} spell out these equivalences for the lexical tone categories of nouns, taking up the same examples as in Tables~\ref{tab:thelexicaltonesofmonosyllabicnouns} and \ref{tab:thelexicaltonesofdisyllabicnouns}.
These equivalences may initially seem counterintuitive, but they minimize the risk of phonological misinterpretation. 


% \label{tab:thelexicaltonesInToneNumbers}  %% Commented out on April 30th, 2025: no subtables, only sequentially numbered tables in the entire volume.
\begin{table}[t!]
\caption{\label{tab:thelexicaltonesofmonosyllabicnounsInToneNumbers}Equivalences between tone-letters and tone digits for transcribing the lexical tone categories of monosyllabic nouns.}
\begin{tabularx}{\textwidth}{ P{15mm} P{17mm} Q Q P{15mm} }
  \lsptoprule
analysis & with tone-letters & with tone digits (underlying form) & with tone digits (surface form) & meaning\\ \midrule
// LM //  & \ipa{bo˩˧} & \ipa{bo¹³} & \ipa{bo¹⁵}  & pig\\
// LH // & \ipa{ʐæ˩˥}  & \ipa{ʐæ¹⁵}  & \ipa{ʐæ¹⁵}  & leopard\\
// M //  & \ipa{lɑ˧} &  \ipa{lɑ³}&  \ipa{lɑ³} & tiger\\
// L //  & \ipa{jo˩} & \ipa{jo¹} & \ipa{jo³} & sheep\\
// \#H //  & \ipa{ʐwæ˥} & \ipa{ʐwæ⁵} & \ipa{ʐwæ³} & horse\\
// MH\# // & \ipa{ʈʂʰæ˧˥}  & \ipa{ʈʂʰæ³⁵}  & \ipa{ʈʂʰæ³⁵}  & deer\\
\lspbottomrule
\end{tabularx}
\end{table}

\begin{table}[t!]
\caption{\label{tab:thelexicaltonesofdisyllabicnounsInToneNumbers}Equivalences between tone-letters and tone digits for transcribing the lexical tone categories of disyllabic nouns.}
%\begin{tabularx}{\textwidth}{ P{21mm} l Q Q P{14mm} }
\begin{tabularx}{\textwidth}{ P{21mm} P{17mm} Q Q P{14mm} }
  \lsptoprule
analysis & with tone-letters & with tone digits (underlying) & with tone digits (surface) & meaning\\ \midrule
%\mbox{(surface} \mbox{form)}
// M //  & \ipa{po˧lo˧} & \ipa{po³lo³} & \ipa{po³lo³} & ram\\
// \#H //   & \ipa{ʐwæ˧zo\#˥} & \ipa{ʐwæ³zo\#⁵} & \ipa{ʐwæ³zo³} & colt\\
// MH\# //   & \ipa{hwɤ˧li˧˥} & \ipa{hwɤ³li³⁵} & \ipa{hwɤ³li³⁵} & cat\\
// H\$ //   & \ipa{kv̩˧ʂe˥\$} & \ipa{kv̩³ʂe⁵\$} & \ipa{kv̩³ʂe⁵} & flea\\
// H\# //   & \ipa{hwæ˧ʈʂæ˥} & \ipa{hwæ³ʈʂæ⁵} & \ipa{hwæ³ʈʂæ⁵} & squirrel\\
// L //   & \ipa{kʰv̩˩mi˩} & \ipa{kʰv̩¹mi¹} & \ipa{kʰv̩¹mi¹⁵} & dog\\
// L\# //   & \ipa{dɑ˧ʝi˩} & \ipa{dɑ³ʝi¹} & \ipa{dɑ³ʝi¹} & mule\\
//LM+MH\#//   & \ipa{õ˩dv̩˧˥} & \ipa{õ¹dv̩³⁵} & \ipa{õ¹dv̩³⁵} & wolf\\
//LM+\#H// &   \ipa{nɑ˩hĩ\#˥} & \ipa{nɑ¹hĩ\#⁵} & \ipa{nɑ¹hĩ⁵} & Naxi\\
// LM //   & \ipa{bo˩mi˧} & \ipa{bo¹mi³} & \ipa{bo¹mi⁵} & sow\\
// LH //   & \ipa{bo˩ɬɑ˥} & \ipa{bo¹ɬɑ⁵} & \ipa{bo¹ɬɑ⁵} & boar\\
\lspbottomrule
\end{tabularx}
\end{table}



In the process of mapping tonal categories onto syllables transcribed in the \isi{International Phonetic Alphabet}, some cases are straightforward. For
instance, the LM tone category can be represented by associating both levels with a~{monosyllable}, as in //\ipa{bo˩˧}// ‘pig’, and distributing them over the two syllables of a~disyllable, as in //\ipa{bo˩mi˧}// ‘sow’. However, not all cases are this simple, necessitating detailed explanations of the notational choices made here.

The first decision concerns the transcription of the \is{floating tone}floating H tone of monosyllabic nouns. Here, it is represented as a~simple H tone, without an explicit indication that it is \is{floating tone}floating. Thus, ‘horse’ is transcribed simply as //\ipa{ʐwæ˥}// rather than //\ipa{ʐwæ\#˥}//. The rationale for this choice is that, on monosyllables, there is only one type of H tone~-- namely, this \is{floating tone}floating H, which can only surface after the syllable to which it is lexically attached. Since no contrast exists among different types of H tones in this context, it was deemed economical to dispense with
the pound symbol ‘\ipa{\#}’ (an explicit indicator of the tone's mode of syllabic association). 

For nouns of two or more syllables, on the other hand, three types of H tones must be distinguished: H\#, \#H, and H\$. In these cases, some indication of syllabic \isi{anchorage} is indispensable in lexical transcriptions. At
least two diacritic marks are required to differentiate among H\#, \#H and H\$. The
first of these three tones is transcribed with a~simple H-tone mark \ipa{˥} on the last syllable, as its mode of {anchoring} appears as
phonologically simplest: it remains fixed on the final syllable and never shifts. Hence, ‘squirrel’ is transcribed as //\ipa{hwæ˧ʈʂæ˥}// (rather than the fully explicit //\ipa{hwæ˧ʈʂæ˥\#}//). 

The decision to mark a~Mid tone \ipa{˧} on the first syllable of such words follows from the principle of keeping lexical forms close to the surface phonological forms. When a~disyllable with lexical H\#, \#H or H\$ tone is produced \is{form!in isolation}in isolation, its first syllable carries M tone, which is analyzed here as a~default tone (assigned to syllables lexically unspecified for tone; see \sectref{sec:analysisofmasadefaulttone}). In the transcription of lexical forms, it thus appears preferable to provide a~tone for each syllable. The same principle underpins the transcription of ‘colt’ as //\ipa{ʐwæ˧zo\#˥}// rather than //\ipa{ʐwæ.zo\#˥}//, ‘flea’ as //\ipa{kv̩˧ʂe˥\$}// rather than //\ipa{kv.ʂe˥\$}//, and ‘cat’ as //\ipa{hwɤ˧li˧˥}// rather than //\ipa{hwɤ.li˧˥}//.

In the case of L-tone words, the L tone is indicated on both
syllables, introducing some redundancy. For instance, ‘dog’ is transcribed
as //\ipa{kʰv̩˩mi˩}// rather than //\ipa{kʰv̩˩mi}//.

Conversely, in disyllabic words where the tone pattern is analyzed as LM+\#H, the lexical transcription omits the M tone, by convention. This is to avoid stacking two tone marks on the same syllable, which would misleadingly suggest the presence of a \is{tonal contour}contour. The lexical transcription of tone therefore consists simply of L on the first syllable and \#H on the second. Thus, ‘{Naxi} (person)’ is written as //\ipa{nɑ˩hĩ\#˥}// rather than //\ipa{nɑ˩˧hĩ\#˥}// or //\ipa{nɑ˩hĩ˧\#˥}//.

Since is not easy to master all these conventions at once, the dictionary \citep{michaud_et_al_na_dict_2024} provides several layers of tonal information to facilitate gradual familiarization with the Yongning Na tone system and enhance accessibility to the online linguistic materials. Specifically, each entry includes (i)~a~bespoke field indicating the lexical tone, (ii)~the complete underlying (lexical) form, in which tonal specifications are encoded in the way explained above, and (iii)~the surface form, which reflects the actual pronunciation in isolation. For instance, the entry for `dog' clarifies that the lexical tone is Low (``Tone: L") and provides the underlying form \ipa{kʰv̩˩mi˩}, where L tone is indicated on both syllables, alongside the surface form \ipa{kʰv̩˩mi˩˥}, which is how the word in pronounced. 

\subsection{Attested and unattested lexical tones}
\label{sec:attestedandunattestedlexicaltones}

The static regularities identified in \sectref{sec:astaticinventoryoftonepatterns} can be reformulated in dynamic terms, as the outcome of a~set of phonological \is{tone rules}tone rules. These rules will be set out in \sectref{sec:alistoftonerules}, and discussed in detail throughout Chapter~\ref{chap:toneassignmentrulesandthedivisionoftheutteranceintotonegroups}, which also discusses the
phonological unit within which they apply: the {tone group}. The following is a~preview of the full set of rules, to which the reader will need to refer frequently in subsequent chapters. 

\begin{enumerate}[leftmargin=2cm, itemsep=0pt, labelwidth=\widthof{Rule~1:}]%[topsep=12pt, partopsep=0pt]
%\begin{enumerate}[leftmargin=!,labelwidth=\widthof{Rule~1:}]
	\item[Rule~1:] L tone spreads progressively (“left-to-right”) onto syllables unspecified for tone.
	\item[Rule~2:] Syllables that remain unspecified for tone after the application of Rule~1 receive M tone.
	\item[Rule~3:] In tone-group-initial position, H and M are neutralized to M.
	\item[Rule~4:] The syllable following an H-tone syllable receives L tone.
	\item[Rule~5:] All syllables following an H.L or M.L sequence receive L tone.
	\item[Rule~6:] In tone-group-final position, H and M are neutralized to H if they follow an L tone.
	\item[Rule~7:] If a~\isi{tone group} only contains L tones, a~postlexical H tone is added to its final syllable.
\end{enumerate}

Three key observations are particularly relevant here: (i)~L tone spreads progressively (“left-to-right”); (ii)~all tones following H are lowered
 to L; and (iii)~H and M are neutralized to M in tone-group-initial position. These generalizations, together with the fact that no falling contours occur on a~single
syllable, account for the absence of all the unattested lexical tone patterns in monosyllables. They also explain most of the unattested
patterns for disyllables, such as $\ddagger${\kern2pt}H.L, $\ddagger${\kern2pt}H.M, $\ddagger${\kern2pt}M.LM, and $\ddagger${\kern2pt}ML.M. 

However, one theoretically possible combination is unattested despite being compatible with the language's phonotactics: there is no //LM\mbox{+H\$//} lexical tone category for disyllables, even though the categories //LM+MH\#// and //LM+\#H// are attested. (As for \mbox{//LM}+\mbox{H\#//,} it is indistinguishable from
\mbox{//LH//}, since both lead to the same tonal association in a disyllable: L on the first
syllable, and H on the second.) This gap can be interpreted as evidence that \mbox{//H\$//} is not only less widespread in the lexicon than \#H and H\#~-- as was pointed out in \sectref{sec:wordfinalandmorphologicalnucleusfinalHtones}~-- but is also less deeply integrated into the system.


\subsection{Phonological regularities and morphotonological oddities}
\label{sec:reflectionsonthestructureofthesystemphonologicalregularitiesandmorphophonologicaloddities}

Looking back at the data in Tables~\ref{tab:thelexicaltonesofmonosyllabicnouns} and \ref{tab:thelexicaltonesofdisyllabicnouns}, it is tempting to seek phonological \is{irregularities}regularities that capture all the observed patterns. However, such a~search soon encounters facts that resist phonological generalizations. 
%The search for phonological regularities is soon up against sets
%of facts that resist phonological generalizations, however. 

For instance, there is no obvious reason
why L should surface as M \is{form!in isolation}in isolation. This may be linked to the prohibition of all-L tone
groups (see \sectref{sec:ltonesexistenceofarepairphenomenonforallltonegroups}), and \textit{a~fortiori} of all-L utterances. However, for verbs, this state of affairs is repaired by the addition of a~postlexical final H tone, so that verbs with lexical \mbox{//L//} tone surface with
\mbox{/LH/} \is{tonal contour}contours \is{form!in isolation}in isolation (see \tabref{tab:Utonesofverbs}). If the tone system were governed solely by
phonological rules applying uniformly across all morphosyntactic contexts, one would expect lexical //L// on
a~noun to surface as /LH/, not as /M/. 

A~similarly puzzling case concerns the
//L// tone on disyllabic nouns. A~word such as //\ipa{kʰv̩˩mi˩}// ‘dog’ yields /\ipa{kʰv̩˩mi˩˥}/ in
isolation, as expected. However, when followed by the \isi{copula}, it surfaces as /\ipa{kʰv̩˩mi˩ ɲi˥}/ ‘is \mbox{(a/the)}
dog’: the \isi{copula} loses its lexical //L// tone. There is no obvious reason why this should be so: one might instead expect a~/L.L.L/ sequence, //$\dagger${\kern2pt}\ipa{kʰv̩˩mi˩ ɲi˩}//, which, following the addition of
a~postlexical H tone to avoid an all-L \isi{tone group}, would be realized as /$\dagger${\kern2pt}\ipa{kʰv̩˩mi˩ ɲi˩˥}/.

This asymmetry in the tonal behaviour of the \isi{copula} after a~//L//-tone noun, depending on the number
of syllables in the noun, highlights a~crucial aspect of Yongning Na tone: many tone rules have
narrowly restricted domains of application. They operate within highly specific morphosyntactic contexts and are sensitive to both the number of syllables and the internal structure of the morphemes concerned.

These reflections on the overall outlook of the Yongning Na tone system will be taken up in \sectref{sec:morphophonologicalcomplexity}, in light of the detailed morphotonological account to which we now turn (Chapters~\ref{chap:classifiers}-\ref{chap:verbsandtheircombinatoryproperties}).
