\chapter{Typological perspectives}
\label{chap:arealandtypologicaldiscussion}

%\subsection{Typology and universals}
%\label{sec:typologyanduniversals}
%
%Linguistic typology is central to the series ``Studies in diversity linguistics'' which hosts this volume. The ``Aims and scope'' of the series fit snugly with the approach adopted here: 
%
%\begin{quotation} 
%This book series will publish book-length studies on individual less-widely studied languages (especially, but not only reference grammars), as well as works in broadly comparative typological linguistics that takes into account the world-wide diversity of human languages. Work on individual languages and broadly comparative work is of a~different nature, but this book series sees the two as closely related: comparative studies need in-depth work on individual languages from around the world to build on, and descriptive work is done best in a~comparative perspective.
%\end{quotation}

The following statement sets a~possible stage for typological inquiry:

\begin{quotation} [L]anguages may differ at virtually all levels in their process of categorisation~-- not only in
	how they group sounds into emic categories (phonemes) but also in the way their particular
	constraints group these phonemes into meta-categories (classes of phonemes). These constraints, in
	turn, have to be defined system-internally, even when they derive from such supposedly universal
	parameters as sonority. \citet[129]{haspelmath2007} reminds us that “structural categories of language
	are language-particular, and we cannot take pre-established, \textit{a~priori} categories for
	granted”. Such a~stance does not rule out the possibility of universal generalisations, but
	entails that they can only be based on the empirical study of language-internal structures, and
	the acknowledgment of cross-linguistic diversity.~\citep[429]{francois2010}
\end{quotation}

%Command \noindent added to avoid having an indent. Proofreader suggestion: since this sentence continues the argument, it is better not to indent. 
{\noindent}In this passage, François emphasizes that every language has its own emic categories, which can only be discovered through an in-depth, language-internal analysis. Universals should not be assumed \textit{a priori}: they are to be investigated empirically and explored through cross-linguistic comparison. This stance does not amount to a~relativistic view in which each language requires an entirely distinct conceptual framework. Instead, it opens up a~comparative research programme that highlights functional similarities across languages. 

% looking for similarities in evolutionary potential, rather than static characteristics. 


\section{Tonal typology}
\label{sec:typologicalperspectives}
\largerpage
This section clarifies the typological distinction between “level tones” and “complex tones”, which constitutes the background to the classification of Yongning Na tones as “level tones” (\sectref{sec:typologicalbackgroundtotheclassificationofyongningnatonesasleveltones}). Some reflections are then set out (in \sectref{sec:tonalrules}) concerning the typological profile of Na \isi{prosody}, shaped by the tone rules described in the preceding chapters. 

\subsection{Level tones and phonetically complex tones}
\label{sec:typologicalbackgroundtotheclassificationofyongningnatonesasleveltones}

\is{level tones|textbf}
\is{level tones|(}
\is{complex tones|(}

In what can broadly be termed as “Africanist” usage, the term “\is{level tones}level tone”
refers to \textit{a~tone defined solely by a~discrete level of relative pitch}. Level tones (a~phrase used interchangeably with “tonal levels”) are monodimensional: they are defined along a~single parameter, F\textsubscript{0}. This differs from segmental phonology, as vowels and consonants are defined along intersecting parameters, such as voicing, nasality, and place of articulation. Since level tones do not freely combine multiple properties, they are not analyzed further in terms of features: level tones constitute phonological primitives (\citealt[20]{clementsetal2011}; \citealt{hyman2011c}).

Level-tone systems have two to five levels
of relative pitch: L vs.\ H; L vs.\ M vs.\ H; L vs.\ M vs.\ H vs T(op); or B(ottom) vs.\ L vs.\ M vs.\ H
vs.\ T(op). Systems with more than three levels are relatively uncommon (e.g.\ \ili{Bariba}:
\citealt{welmers1952}, and \ili{Bench}, also known as \ili{Gimira}: \citealt{wedekind1983,wedekind1985}). One single case
of a~six-level system has been reported: \ili{Chori} \citep{dihoff1977}, for which a~reanalysis is possible
\citep{odden1995}. Bariba, Bench, and Chori are spoken in Sub-Saharan Africa, an area where level tones
are especially common. However, level-tone representations have proved useful beyond the Sub-Saharan
area, for which they were initially developed (on languages of the Americas:
\citealt{gomezimbert2001}; \citealt{hargusetal2005}; \citealt{gironhiguitaetal2007};
\citealt{michael2010}; on languages of Asia: \citealt{ding2001};
\citealt{hymanetal2002a,hymanetal2004}; \citealt{donohue2003,donohue2005}; \citealt{evans2008a};
\citealt{jacques2011a}). 

In Yongning Na, the \is{morphotonology}morphotonological alternations studied in
the preceding chapters provide overwhelming evidence for a~level-tone analysis. \is{tonal contour|textbf}In level-tone
systems, a~phonetic {contour} is \textit{the realization of two or more level tones on a~single
  syllable}. These contours are phonologically decomposable: the observed F\textsubscript{0} movement results from interpolation between successive tonal levels.

There are some languages for which attempts to decompose contours into level tones have been
less successful, however, to the point of casting doubt on the relevance of such an approach for these
languages. In the \ili{Austroasiatic} and \ili{Tai-Kadai} language families, no convincing evidence has been found for
the decomposition of contours into simpler units \citep[e.g.][639]{morey2014}. 
%%Quotation not fully appropriate here because it focuses on BINARY tone.
%\begin{quotation}
%	I do not find the idea of binary features necessary or helpful in analysing the languages that I
%	have worked on. In these languages, I do not believe that reducing the analysis of tones
%	to a~binary choice of H and L will assist in the understanding of the tonal system.
%\end{quotation}


In the field of \ili{Sinitic} languages (\il{Sinitic}Chinese dialects), Chao Yuen-ren’s early 20\textsuperscript{th}-century work on \ili{Mandarin} \citep{chao1929,chao1933} brought to light the complexities
of its tone system. Following sustained exchanges with Chao, Kenneth Pike proposed
a~typological divide between (i)~register-tones, defined solely in terms of
discrete pitch levels, and (ii)~{contour}-tones, about which he concluded: “the glides of a~{contour}
system must be treated as unitary tonemes and cannot be broken down into end points which constitute
lexically significant contrastive pitches” \citep[10]{pike1948}. 

Later studies have emphasized the
role of phonation-type characteristics. In some prosodic systems,
phonation types are merely low-level phonetic characteristics that occasionally accompany tone (see, for example, the investigation of the effect of
creaky voice on \ili{Cantonese} tonal perception by \citealt{yuetal2014}). In others, phonation type constitutes a~distinctive feature which is phonologically independent of tone (\textit{orthogonal} to tone, to use a~term that has long been popular in the speech sciences: see e.g.\ \citealt[636]{Fant1967Features}; \citealt{ohala1978b_phonological}), as in the \ili{Oto-Manguean} languages \ili{Mazatec}
\citep{garelleketal2011} and \ili{Trique} \citep{dicanio2012}. Finally, in a~third type of system,
phonation-type characteristics are part and parcel of the definition of tones. Experimental studies
of this third type of tone system include \citet{rose1982,rose1989a,rose1990} for the Wu branch of
\ili{Sinitic}, \citet{edmondsonetal2001} for \ili{Yi} and \ili{Bai}, \citet{mazaudonetal2008} for \ili{Tamang}, and
\citet{andruskietal2000}, \citet{andruskietal2004} and \citet{kuang2013} for \ili{Hmong}. 

\begin{quotation}
	Languages such as \il{Hmong}{Black Miao} and \ili{Vietnamese} highlight the difficulty of drawing a~line in 
	the sand separating ‘tone’ languages from ‘register’ languages. This problem is even more 
	strikingly illustrated by \ili{Burmese} ({\dots}), which has been described both as a~register system 
	\citep[e.g.][]{bradley1982, jones1986} and as a~tone language \citep[e.g.][]{watkins2001a, gruber2011} ({\dots}). \citet{gruber2011} has shown that glottalisation, creakiness and the presence of a~high pitch target are all important perceptual cues, thus demonstrating that, much like \ili{Vietnamese} or \il{Hmong}Black Miao, \ili{Burmese} should not be analyzed in terms of pitch or \is{phonation types}phonation type alone, nor is it straightforward to decide which property is the primary acoustic cue to the contrast. \citep[194]{brunelleetal2016}
\end{quotation}


Pike’s two-way typology of tone,
while it emphasizes typologically relevant properties of the languages which he was able to take
into consideration, has some limitations when applied to cases such as \ili{Vietnamese}. In the \ili{Vietnamese} system, the tones contrast with one another through a~set of characteristics that include specific phonation types in addition to
the time course of F\textsubscript{0}
\citep{alves1995,mixdorffetal2003,brunelleetal2010,nguyenetal2013,macetal2015,ta_register_2023}. For such tones, the label “{contour} tones” appears underspecific.
For this reason, the term “complex tones” is used here in preference to Pike’s “{contour} tones”. 

To
recapitulate the terminology used in the present discussion: \textit{complex-tone systems} are distinguished from \textit{level-tone systems} (which are based on
discrete levels of relative pitch). \is{complex tones|textbf}Complex tones include
Pike’s category of “{contour} tones”, with the explicit addition of tones that comprise phonation-type
characteristics.\footnote{This definition differs from that used in the \textit{World Atlas of
  Language Structures}, where “complex” refers to the number of oppositions, not to the nature of the
  tones: “[t]he languages with tones are divided into those with a~simple tone system~-- essentially
  those with only a~two-way basic contrast, usually between high and low levels~-- and those with
  a~more complex set of contrasts” \citep{maddieson2011}.}

Under this set of definitions, “{contour}” refers to a~unitary {contour}: a~tone that is phonologically defined in terms of an overall template specifying the time course of F\textsubscript{0} over the \isi{tone-bearing unit}. Phonologically unitary {contour} tones are encoded as overall shapes: “there are no objective
reasons to decompose \ili{Vietnamese} tone contours into level tones or to reify phonetic properties like
high and low pitch into phonological units such as H and L” (\citealt[94]{brunelle2009a}; see also
\citealt{brunelleetal2010,kirby2010}). In this type of system, the term “\is{level tones}level tone” is used to
refer to \textit{a tone that does not exhibit any salient fluctuations in F\textsubscript{0}}. For instance,
\ili{Mandarin} tone 1 and \ili{Vietnamese} tone 1 (orthographic \textit{ngang}) can be referred to as “level
tones” because, unlike the other tones in these languages, their F\textsubscript{0} curve remains relatively
stable over the course of the syllable. This does not entail that they are phonologically defined by
a~discrete level of relative pitch (on \ili{Mandarin}, see \citealt{xuetal2001}).

The two types of phonological \is{tonal contour}contour tones~-- sequences of levels, on the one hand, and unitary contours, on the other~-- can be \is{phonetic realization of tones}
phonetically indistinguishable, making it necessary to relate phonetic observation to functional-structural levels of description. The evidence for distinguishing these two types of
contours is \is{morphotonology}morphotonological. A~rising {contour} in a~\ili{Bantu} language will typically exhibit
phonological behaviour showing that it consists of a~low \is{level tones}level tone followed by a~high \is{level tones}level tone
\citep{clementsetal1984,clementsetal2007}. In Yongning Na, too, there is a~wealth of evidence supporting the analysis of \is{tonal contour}contour tones as sequences of level
tones. From a~typological point of view, rather than assuming that all tones can be decomposed into
level components, it is at least as reasonable to adopt the opposite standpoint: viewing contours as
nondecomposable units unless there is positive evidence to the contrary (Nick Clements, p.c.\ 2008).

Tonal systems based on levels (tone heights) are not
unheard of in \il{Sino-Tibetan}Sino-Ti\-betan. Examples include \ili{Pumi} \citep{matisoff1997a,ding2006,jacques2011a,daudey2014}, Cone \ili{Tibetan} \citep{sun2003b,jacques2014b}, Mianchi Qiang (\ili{Rma})
\citep{evans2008a}, \ili{Shixing} \citep{chirkovaetal2009}, \ili{Hakha Lai} \citep{hymanetal2002a},
\ili{Kuki-Thaadow} \citep{hyman2010b}, and the Lataddi dialect of Na \citep{dobbsetal2016}. 
 
The distinction
between level tones and complex tones is proposed here as a~rule-of-thumb distinction; it aims to
bring attention to a~substantial body of Asian data that may not be readily visible to some prosodic typologists and which deserves to be more widely appreciated. Needless to say, the two-way distinction between level tones and complex tones is by no means
water-tight: there are borderline situations. 
\is{level tones|)}
\is{complex tones|)}

Among other research perspectives opened up by the typological distinction between level tones and phonetically complex tones, comparison between the intonation systems found alongside these two types of tonal organization appear well worth carrying out, hypothesizing that there may be different trends between these two broad types.

\subsection{Intonation in level-tone systems and complex-tone systems}
\label{sec:IntonationInLevelTonesAndComplexTones}

In his study of tone and intonation in Lingala, mentioned earlier (\sectref{sec:IssuesWithATonalDescriptionOfIntonation}), Guthrie suggests that the tonal range is set at the sentence level. This arguably reflects a~characteristic of \ili{Lingala}, where successive tones are much more cohesive than in \is{complex tones}complex-tone systems. 
%\footnote{On the typological distinction between level-tone systems and complex-tone systems, see \sectref{sec:typologicalbackgroundtotheclassificationofyongningnatonesasleveltones}.} 
In complex-tone systems, attention is drawn to \textit{local} phenomena of F\textsubscript{0} \isi{range expansion} or compression. 
%, which do not alter the phonological nature of the syllables' tones. 

To venture an impressionistic description of the contrast between the two types of systems: level tones exist as part of a~sequence, whereas phonetically \isi{complex tones} have a~stronger individual identity. Level tones are subject to a~range of categorical processes that modify the tonal string, such as tone \is{tone spreading}spreading (where an H or L tone is copied onto successive syllables). By contrast, \isi{complex tones} are less prone to categorical change and more prone to noncategorical, local intonational modifications conveying emphasis or \isi{phrasing}. 

This does not imply that successive complex tones are fully independent. For instance, in \ili{Mandarin}, focus on one syllable affects neighbouring syllables, particularly those that follow: “focus is usually related to F\textsubscript{0}-range-expansion of focused words that are not in the final position of an utterance and F\textsubscript{0}-range-suppression of post-focus words'' \citep[449]{zhangetal2004}. Moreover, tonal \isi{coarticulation} effects in \ili{Sinitic}, \ili{Vietnamese}, and \ili{Thai} are strong and tend to solidify diachronically into \is{tone sandhi}sandhi patterns \citep{abramson1979a, gandouretal1992, brunelle2003, brunelle2009b, zhangetal2011}. Thus, it would be misguided to view the noncategorical intonational modification of complex tones as purely local, or conversely, to assume that level tones are entirely exempt from local, noncategorical intonational variation. Nevertheless, the following generalization seems to hold: in the study of \isi{intonation}, Bantuists' attention is regularly drawn to sentence-level phenomena rather than to local pragmatic emphasis. This suggests that local variations of the kind observed in complex-tone systems~-- as exemplified by \ili{Vietnamese}, \ili{Thai}, and \ili{Sinitic}~--, where they do not alter the phonological nature of the tones, are less salient in level-tone systems such as those found in \ili{Bantu} languages. In \ili{Bantu} \isi{prosody}, scholarly attention is drawn instead to \textit{categorical}  %local 
changes (modifications that alter the phonological tonal sequence) and, secondarily, to sentence-level phenomena. Echoing Guthrie's study of \ili{Lingala}, a~study of \ili{Chichewa} \isi{intonation} likewise focuses on sentence mode, specifically on differences in F\textsubscript{0} between questions and statements \citep{myers1996}.

In light of the above typological perspectives, the following subsection attempts to convey a~feel for the organization of the Na prosodic system by pointing out some of the consequences of its tone rules for the outlook of this level-tone system.


\subsection[Typological profile of Na prosody]{Typological profile of Na prosody as shaped by the tone rules}
\label{sec:tonalrules}

\is{tone rules}

One of the salient characteristics of Yongning Na is the partial \isi{neutralization} of lexical oppositions when words are said \is{form!in isolation}in isolation. However, this does not appear to have far-reaching implications for the organization of the tonal system as a~whole. Such \isi{neutralization} is found in prosodic systems that differ widely from one another, such as \ili{Japanese} \citep{kubozono1993}, San Juan Quiahije \ili{Chatino}
(\ili{Oto-Manguean} family) \citep[91]{cruz2011}, and \ili{Sotho} and \ili{Tswana} (\ili{Bantu}) \citep{creisselsetal1997, zerbianetal2010b, zerbian2016}. 

On the other hand, the levelling rules of Yongning Na (Rules 4 and 5), which lower
all tones following an H tone or an M.L sequence to L, have far-reaching consequences for \is{form!surface}surface phonological tone sequences. These two rules lead to the \isi{neutralization} of tonal oppositions over large portions of tone groups~-- a~massive levelling effect reminiscent of stress systems in which all syllables following a~major stress are de-stressed. The culminative nature of its H tone makes Yongning Na strikingly different from the extensive set of systems known as `terraced-\is{level tones}level tone languages' \citep{armstrong1968}. 

The concept of `terracing' refers to categorical shifts in register affecting all subsequent tones: \textit{downstep}, a~distinctive lowering, and \textit{upstep}, a~distinctive raising. An important consequence of terracing is that tones within the same `terrace' hang together more closely than successive tones in a~language like Na, without \isi{downstep} or upstep. To use an image from weaving, tones in the same terrace are tightly knit together; to use an image from woodwork, they are pegged together. This intuition is reflected in Nick Clements's proposed treatment of `terracing' in terms of a~``tone level frame''. 

``Within this framework, terracing is seen to be the result of ({\dots}) processes applying to the tone level frame itself, rather than directly to individual tones'' \citep[538]{clements1979}. `Terracing' places constraints on the range of fundamental frequency within which the tonal levels are realized, since shifts in register are distinctive. It plays a~major role in shaping surface phonological tone sequences and their phonetic realization. A~key property of terracing, as Clements notes, is that categorical register shifts can take place more than one time within a~\isi{tone group}. 

\begin{quotation}
	[An] important feature of tone terracing, at least in the case of \isi{downstep}, is that there is no limit on the number of register lowerings that may occur within a~\isi{tone group}; the only limit is the external one imposed by the lexical, grammatical, or phonological factors that govern the occurrence of \isi{downstep}. \citep[540]{clements1979}
\end{quotation}

%Command \noindent added to avoid having an indent. Proofreader suggestion: since this sentence continues the argument, it is better not to indent. 
{\noindent}This type of tonal organization requires preplanning strategies, which can get highly elaborate. Brilliant speakers anticipate the amount of downsteps required in a~long utterance, raise the initial pitch accordingly, and manage successive lowerings throughout. Less talented orators need to reset their F\textsubscript{0} when successive downsteps bring them to the lower limit before they reach the end of a~\isi{tone group} \citep{rialland2001}.

In Yongning Na, by contrast, there is no \isi{downstep}, and hence no need for such long-distance preplanning. Whenever a~\isi{tone group} contains an H tone, this tone serves as the \isi{tone group}'s climax. From an information-processing perspective, once an H tone is identified, the remainder of the \isi{tone group} contain no further tonal contrasts: there is nothing to expect but a~sequence of phonological L tones. Phonetically, the pitch gradually descends towards its floor value; the realization of the F\textsubscript{0} curve on this portion of the \isi{tone group} is free from the trammels of phonological distinctiveness.

To summarize, the `tone level frame' (Nick Clements's term for the tone space at a~given point in an utterance) is subject to far fewer constraints in Yongning Na than in `terracing' tone languages. The absence of \isi{downstep} or upstep, combined with the \is{culminativity}culminative nature of H tones, go a~long way towards explaining the different feel of its \isi{prosody} compared to that of `terracing' tone languages. There are simply fewer possible tone sequences: for instance, while both Yongning Na and \ili{Yala} (Ikom) have H, M, and L tones, Yala also has downstepped counterparts (!H, !M, and !L) and allows their full range of combinations \citep{armstrong1968}. Moreover, in Na, there is no contrast between a~fall from M to L and one from H to L, allowing for greater phonetic freedom than in systems where the tonal space is more crowded. (This phonetic flexibility is exploited for intonational purposes, as discussed in \sectref{sec:pragmaticintonation}.)


Like \isi{downstep}, \textit{downdrift}~-- the gradual phonetic lowering of phonologically identical tones separated by a~lower tone~-- is absent in Yongning Na, for the same reason: the two possible sequences of a~higher tone and a~lower one are H.L and M.L, both of which constitute a~descent to the lowest phonological level. By Rule~5, these sequences can only be followed by L tones, prohibiting configurations such as $\ddagger${\kern2pt}M.L.M, $\ddagger${\kern2pt}H.L.H or $\ddagger${\kern2pt}H.M.H. This is another salient trait of the typological profile of Yongning Na.


\section{Assessing the complexity of the Na tone system}
\label{sec:morphophonologicalcomplexity}

\is{complexity|textbf}

``Language complexity is a notion that has hovered in linguistics for more than
a century" \citep[789]{oh2023}, as a~scale in language description (with one feature
being considered as more complex than another) and as a~general backdrop against which to evaluate differences among languages.
An article discussing methods for measuring the degree of complexity of tone systems \citep{konoshenko2014} proposes the number of tonal contrasts and the number of tonal rules as the two main dimensions of tonal complexity. In terms of these two variables, the tone system of Yongning Na is more complex than that of \ili{Naxi} and \ili{Laze}, its immediate siblings in the \ili{Naish} subgroup of \il{Sino-Tibetan}Sino-Tibetan (\sectref{sec:thepositionofnaandnaxiwithinsinotibetan}): Na has more lexical tone categories, and richer \isi{morphotonology}. 
A~task for the future will be to assess this complexity by modelling regularities and irregularities in the paradigms that make up Na morphotonology. %In this endeavour, it will be possible to build on advances in computational tools (for an example in the field of segmental morphology: \citealt{sagotetal2013}). 
As a~stepping-stone towards this mid- and long-term goal, some dimensions of this complexity are recapitulated below and compared with other languages. The comparisons are based not on phylogenetic or areal closeness but on synchronic typological similarities.

\subsection{Partially regular morphotonology}
\label{sec:partlyregularmorphotonology}

Partially regular morphological paradigms are cross-linguistically widespread. Examples include the inflection of transitive verbs in \ili{Dinka} \citep[8]{andersen1993}, where certain inflections are marked by a~particular toneme for all
verbs alike, while others are specific to particular classes of verbs, and the inflection of interrogative pronouns in the Australian language \il{Anguthimri}Angu\-thimri \citep[172]{crowley1981}. 

In the Na tone system, \is{numerals}numeral-plus-classifier\is{classifiers} phrases
constitute an area where tone patterns have proliferated. As shown in Chapter~\ref{chap:classifiers}, no fewer than nine tonal categories exist for {monosyllabic} classifiers, compared to five for {monosyllabic} nouns. Furthermore, the tonal patterns of these nine classifier categories in combination with numerals need to be learnt individually, as they do not follow from synchronically regular rules. While this complexity is not as spectacular as that of \ili{Ahmao} (\ili{Hmong}-Mien family), where classifiers have “12
basic forms, each displaying a~complex cluster of meanings” \citep{gerneretal2009}, the Na data nonetheless contributes to typological generalizations, demonstrating that the tones of classifiers can be more complex than those of nouns. 


\subsection{Nouns and verbs: A~comparable degree of complexity?}
\label{sec:limitationsontonaloppositions}

Keeping in mind that certain types of nouns (especially classifiers) are more complex than others in terms of their tone categories, it seems that there is no conspicuous imbalance between Na nouns and verbs in terms of tonal complexity. This differs from tonal systems in \ili{Bantu} and, more broadly, in the Niger-Congo family, where verbs generally display less tonal diversity than nouns. Many \ili{Bantu} languages only have two tonal types of verbs (irrespective of their number of syllables), versus three or more tonal categories for {monosyllabic} nouns and even more for disyllabic and longer nouns \citep[183]{creissels1994}. In some languages, such as \ili{Gbeya}, \ili{Kissi}, \ili{Baoulé}, and \ili{Urhobo}, there are no tonal oppositions among verbs at all \citep[184]{creissels1994}.

\subsection{More progressive spreading than regressive spreading: A~typologically common pattern}
\label{sec:propagationanticipation}

Under the present analysis, Yongning Na has a~phonological tone rule (Rule~1) whereby L tone spreads progressively onto syllables that are unspecified for tone. In contrast, the H tone does not spread in the sense of associating to several syllables in a~row; there can only be one syllable carrying H tone in a~\isi{tone group}. Despite this, an H tone influences the following tones within the group, causing them all to lower to L. 

While the \is{morphotonology}morphotonological rules are context-specific and cannot be reduced to a~set of purely phonological rules, they also reveal an overall tendency towards progressive tone \is{tone spreading}spreading, rather than the other way round. For instance, averaging over the entire set of tone rules applying in determinative compounds (\sectref{sec:determinativecompoundnounsII}), the determiner (which appears first) makes a~larger contribution than the head to the tone pattern of the entire \is{compounds}compound. Seen in this light, Na exhibits a~clear predominance of progressive over regressive \is{tone spreading}spreading. 

This pattern is typologically unsurprising. Regressive \is{tone spreading}spreading is firmly attested, for instance in \ili{Tswana}, Odienné \ili{Dioula}, and \ili{Kikwere} (as analyzed by \citealt[177-178]{odden1998b}), but progressive \is{tone spreading}spreading is more common cross-linguistically \citep[206-207]{creissels1994}. 

A separate but not wholly unrelated observation is that H tone in Yongning Na has a~tendency to occur late within a~\isi{tone group}. an H tone never associates to the first syllable within a~\isi{tone group}. The \mbox{//H\#//} tone associates to the last syllable of the root; so does \mbox{//H\$//}, but it typically `hops' from this position to a~later syllable (\sectref{sec:wordfinalandmorphologicalnucleusfinalHtones}); and \mbox{//\#H//} never associates to the word to which it is lexically attached, only to a~later syllable, if available (\sectref{sec:afloatinghtonewithcomparativeevidencepointingtoitsorigin}). This overall tendency is relatively common cross-linguistically. Late realization of H tone targets is more common than early realization: ``perseverative tone \is{tone spreading}spreading phonologizes the tendency of tone targets to be realized late'' \citep[19]{hyman2007d}. The {diachronic} developments that have led to the present diversity of final H tones in Yongning Na appear to align with well-attested cross-linguistic tendencies. This is a~case where the synchronic complexity is not particularly surprising when considered from the perspective of the typology of language change, that is, from a~\is{panchronic phonology}\textit{panchronic} perspective (\sectref{sec:theoreticalbackdrop}).

\subsection{The dual status of the M tone is not a~typological rarity}
\label{sec:thestatusofmtone}

In the present analysis, the M tone in Yongning Na has two facets. On the one hand, it functions as a~fully specified phonological tone: the M component in categories such as LM and MH cannot be omitted, as LM contrasts with L, and MH\# with H\#. On the other hand, M tone serves as a~default tone: by Rule~2, M tone is assigned to syllables that remain toneless after the application of Rule~1 (whereby L tone \is{tone spreading}spreads onto toneless syllables). 

Na is not unique in this respect. In
\ili{Yorùbá}, too, the M tone has a~dual status. It is not lexically specified~-- the only two lexical tones are L and H~--, but after being introduced through default-tone assignment rules, M behaves as a~specified
tone in \is{derivation!tonal}tonal derivations \citep{akinlabi1985}. 




