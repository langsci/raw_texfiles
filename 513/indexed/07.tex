\chapter{Tonal association rules and the division of utterances into tone groups}
\label{chap:toneassignmentrulesandthedivisionoftheutteranceintotonegroups}

\is{tone group|(}

This chapter sets out and discusses (i)~the rules of tonal association, %{\linebreak} xyz still useful?
whereby \is{form!surface}surface phonological tones are \is{derivation!tonal}derived from the 
\is{form!underlying}
underlying tones, and (ii)~the principles underlying the division of the utterance into tone groups, a~key unit for tonal processes. Several readers of this book remarked that, for them, the really engaging part of the study lay in this chapter. Here, the archipelago of tables from the earlier chapters gives way to linguistic analysis in the full sense: conveying a~feel for the \is{stylistics}stylistic choices available to speakers (\sectref{sec:thedivisionofutterancesintotonegroups}-\ref{sec:casesofbreachoftonalgroupingandconsequencesforthesystem}) and thereby shedding light on Yongning Na morphotonology in use. 

The unit within which tonal processes apply in Yongning Na is
referred to here as a~\is{tone group|textbf}\textit{tone group}. In transcriptions, its boundaries are indicated by the International Phonetic
Alphabet symbol ‘\ipa{|}’. Successive tone groups are entirely independent in tonological terms: tones never spread or otherwise influence one another across tone-group
boundaries. In other words, tonal computation takes place separately for
each tone group.

From a~phonetic point of view, successive tone groups are linked to one another through a~variety of phenomena, ranging from local effects such as tonal {coarticulation} to more global patterns, in particular \textit{declination} at the level of utterances and higher-level discourse units. However, these intonational
phenomena, discussed in Chapter~\ref{chap:fromsurfacephonologicalformstophoneticrealizationintonationandtonalimplementation}, do not affect the phonological tonal string of the utterance. Conceptually, {intonation} must be distinguished from the processes by which the surface phonological string of a~tone group obtains. 

In prosodic hierarchy models such as the universal framework proposed by \citet{selkirk1986} and
\citet{nesporetal1986}, made up of utterance phrase ⊃
%{$\supset$} 
intonational phrase ⊃ phonological phrase ⊃ phonological word ⊃ foot ⊃ syllable ⊃ mora, tone groups may be considered as corresponding to \textit{phonological phrases}. Nonetheless, the term “tone group” is
      preferred here over “phonological phrase” for several reasons. 
      
First and foremost, the defining characteristic of this phonological unit is its role as the domain of tonal processes. The constituency test for tone groups is whether tonal rules and processes apply: a~tone-group boundary is where tonal computation starts anew. The emphasis laid on constituency tests aligns with the perspective set out by \citet[2]{Tallman2024}: carrying out language description and language comparison ``in terms of constituency test results themselves, rather than only abstract constituency structures proposed in the linguistics literature". 

Na does not have segmental rules
such as the lenition of word-medial consonants found in other languages of the area, such as Qiang (\ili{Rma})
(\citealt[31–32]{lapollaetal2003a}; \citealt[35–42]{sims2014_Yonghe}) and \ili{Shixing} \citep[12–13]{chirkova2009}, which provide evidence
for the phonological word as a~prosodic domain. 
      
Another reason for favouring “tone group” over “phonological phrase” is that, in Yongning Na, another level is also a~plausible candidate for identification as a~“phonological phrase”, namely the \textit{tonal phrase}, discussed further below.

 In Yongning Na, the tone group is the highest unit for tonal computation, while the syllable is the smallest: the \is{tone-bearing unit|textbf}tone-bearing unit at the surface phonological level. Between these two extremes, I~propose the following additional levels:
 
 \begin{itemize}
 	\item{\is{lexical word|textbf}\textit{the lexical word}, to which tone categories are lexically associated;}
 	\item{\is{tonal word|textbf}\textit{the tonal word}, a~combination of lexical words, such as noun plus verb in S+V or O+V combinations, and noun plus noun in compounds;}
 	\item{\is{tonal phrase|textbf}\textit{the tonal phrase}, a~tonal word plus any added clitics and affixes.}
 \end{itemize}

For illustration, consider the lexical word //\ipa{kʰv̩˩mi˩}// ‘dog'. A~tonal word is exemplified in (\ref{ex:hitdog}): the object-plus-verb combination ‘to hit a~dog'. A~tonal phrase is shown in (\ref{ex:hitdogPFV}): it consists of the tonal word (\ref{ex:hitdog}) augmented by the \textsc{perfective} suffix //\ipa{-ze˧\textsubscript{b}}//.

\begin{exe}
	\ex
	\label{ex:hitdog}
	\ipaex{kʰv̩˩mi˩ ti˥}\\
	\gll kʰv̩˩mi˩				ti˩\textsubscript{a}\\
	dog		to\_tap/to\_hit\_gently\\
	\glt ‘to hit a~dog (gently)'
\end{exe}

\begin{exe}
	\ex
	\label{ex:hitdogPFV}
	\ipaex{kʰv̩˩mi˩ ti˥-ze˩}\\
	\gll kʰv̩˩mi˩		ti˩\textsubscript{a}		-ze˧\textsubscript{b}\\
	dog		to\_tap/to\_hit\_gently			\textsc{pfv}\\
	\glt ‘has hit a~dog (gently)'
\end{exe}

An alternative approach would be to use “prosodic stem” instead of \textit{tonal word} and “prosodic word” instead of \textit{tonal phrase}, thereby allowing the use of “phonological phrase” in place of \textit{tone group}. However, a~difficulty with this alternative is that the \is{tonal word}tonal word, as defined here, tends to encompass more material than the ``prosodic stem'', which ``usually coincides with the morphological stem'' (\citealt[17]{vandevelde2008b}; see also \citealt{downing2015}). For this reason, the terms “prosodic stem”, “prosodic word”, and “phonological phrase” are not adopted in the present volume. The hope is that the present description, structured in terms of \textit{lexical words}, \textit{tonal words}, \textit{tonal phrases}, and \textit{tone groups}, is explicit enough to be fully intelligible and transposable into various theoretical frameworks.

This chapter begins by addressing the phonological aspects of the system, starting with tone-to-syllable association rules (\sectref{sec:asummaryoftonetosyllableassociationrules}). It then moves on to the division of the utterance into tone groups, a~topic which links up with \is{stylistics}stylistic considerations (\sectref{sec:thedivisionofutterancesintotonegroups}).

\section{A summary of tone-to-syllable association rules}

\label{sec:asummaryoftonetosyllableassociationrules}

\is{tone rules}

This section provides a summary of the tone-to-syllable association rules in Yongning Na. 

The description is framed in terms of \is{derivation!tonal|textbf}derivation from an \is{form!underlying|textbf}underlying level to
a~surface phonological 
\is{form!surface|textbf} 
level. The notion of {derivation} from underlying to surface phonological representations is a~great help in phonological analysis, to the point that it may not seem to require special justification. However, it has come under criticism from various quarters. Some scholars argue that it is more satisfactory to conceptualize phonology in terms of sets of related surface forms rather than derivational processes. \citet{hyman2015}, reviewing this issue, ultimately advocates for retaining underlying representations as a~central analytical tool. Of course, this does not imply that underlying representations are suited to all questions of linguistic inquiry or that they constitute the final word in phonological description. The ultimate goal is to approach the actual processes at work in the speaker’s brain. In formulating generalizations~-- such as the tone rules proposed below~-- efforts were made to keep in mind psychological (cognitive) plausibility. Mid- and long-term perspectives for modelling with more elaborate tools are briefly addressed in this volume's conclusion (Chapter~\ref{chap:conclusion}). 

One illustration of how the distinction between underlying and surface levels proves useful is provided by the LM and LH lexical tone categories: these are distinct at the underlying level, 
%As an example of the usefulness of the distinction between underlying and surface levels, the LM and LH lexical tone categories are distinct at the underlying level, 
but this opposition is neutralized at the surface phonological level when words are spoken \is{form!in isolation}in isolation. This is due to a~contextual \isi{neutralization} of the M and H levels: these tones do not contrast in tone-group-final position when preceded by an L tone. To avoid ambiguity in contexts where the underlying and surface phonological levels might be confused, double slashes are used for transcriptions at the //underlying 
%\is{form!underlying} 
phonological
level//, as against simple slashes for the /surface 
%\is{form!surface} 
phonological level/. This may not be visually elegant, but desperate tones call for desperate measures.

 
\subsection{The phonological tone rules}
\label{sec:alistoftonerules}
\largerpage
\is{tone rules|textbf}


The tone-to-syllable association rules determine the surface phonological tones of a~given tone group based on its underlying tones. Seven phonological tone rules have been identified in Yongning Na. For ease of reference, these are first listed below before their analysis and discussion.

	\begin{enumerate}[leftmargin=2cm, itemsep=0pt, labelwidth=\widthof{Rule~1:}]%[topsep=12pt, partopsep=0pt]
		\item[Rule~1:] L tone spreads progressively (“left-to-right”) onto syllables unspecified for tone.
		\item[Rule~2:] Syllables that remain unspecified for tone after the application of Rule~1 receive M tone.
		\item[Rule~3:] In tone-group-initial position, H and M are neutralized to M.
		\item[Rule~4:] The syllable following an H-tone syllable receives L tone.
		\item[Rule~5:] All syllables following an H.L or M.L sequence receive L tone.
		\item[Rule~6:] In tone-group-final position, H and M are neutralized to H if they follow an L tone.
		\item[Rule~7:] If a~tone group only contains L tones, a~postlexical H tone is added to its last syllable.
	\end{enumerate}

%%Version below: flush left at margin, with broad vertical spacing between lines.
%\begin{enumerate}[leftmargin=!,labelwidth=\widthof{Rule~1:}]
%	\item[Rule~1:] L tone spreads progressively (‘left-to-right’) onto syllables that are unspecified for tone.
%	\item[Rule~2:] Syllables that remain unspecified for tone after the application of Rule~1 receive M tone.
%	\item[Rule~3:] In tone-group-initial position, H and M are neutralized to M.
%	\item[Rule~4:] The syllable following an H-tone syllable receives L tone.
%	\item[Rule~5:] All syllables following an H.L or M.L sequence receive L tone.
%	\item[Rule~6:] In tone-group-final position, H and M are neutralized to H if they follow an L tone.
%	\item[Rule~7:] If a~tone group contains only L tones, a~postlexical H tone is added to its last syllable.
%\end{enumerate}

The following paragraphs explain the motivation for positing these seven {\linebreak}phonological rules.

The tone association rules for each of the lexical tone categories have been set out in the course of Chapter~\ref{chap:thelexicaltonesofnouns}, including discussions in \sectref{sec:afloatinghtonewithcomparativeevidencepointingtoitsorigin} and \sectref{sec:wordfinalandmorphologicalnucleusfinalHtones} on the association of tone to syllables for the three types of High tones:
H\#, \#H, and H\$. The same rules apply to the tones of more complex entities, referred to here as \is{tonal word}\textit{tonal words}, such as \is{compounds}compound
nouns or \is{numerals}numeral-plus-classifier phrases. Unless otherwise specified, the tone pattern associates to the tonal word syllable by syllable
(“left-to-right”), one tone level at a~time. When there are fewer syllables than tone levels, two levels associate with the last syllable.

Thus, L tone associates with the first syllable of the \is{tonal word}tonal word. Likewise for M tone. For LM tone, the first
syllable receives L, and the second receives M. For LH, the first syllable likewise receives L, and
the second receives H. These four tonal categories (L, M, LM, and LH) are the simplest in terms of
tone-to-syllable association. The other tone categories (\#H, MH\#, H\$, L\#, LM+MH\#, LM+\#H, and
H\#) each have a~specific syllabic \is{anchorage}anchoring, described in \sectref{sec:aphonologicalanalysisofthetonecategoriesofnouns}, and reflected in the special symbols used in the present
transcription. 

In the case of the mixed tone categories LM+MH\# and LM+\#H, the first part (LM) associates with syllables following the usual rules (like for a~simple LM tone), while the second part (MH\#
or \#H, respectively) associates as indicated by the added symbol (\#). Thus, for LM+MH\# tone, the
first syllable receives L, the second receives M, and the last carries an MH \is{tonal contour}contour. If only two syllables are available, the MH \is{tonal contour}contour associates with the second syllable,
overriding its M tone. For LM+\#H, the first syllable likewise receives L, and the second M, while the H level associates with the first syllable following the \isi{tonal phrase}, provided that a~suitable carrier is available. (These two tone categories never associate with a~\is{monosyllables}monosyllable.)


At this stage of tone-to-syllable mapping, some syllables remain toneless. For instance,
the lexical disyllable //\ipa{dɑ.ʝi˩}// ‘mule’, the \is{compounds}compound //\ipa{po.{\allowbreak}lo-ɬi.pi˩}//{\linebreak} ‘ram’s ear’, the
\is{numerals}numeral-plus-classifier phrase //\ipa{gv̩-ʂɯ˩}// ‘nine times’, and the object-plus-verb combination
//\ipa{ɖɤ.mi ʑi˩}// ‘to grab a~fox’, all of which have L\# tone, are specified for tone only on their
last syllable. Tonal nuclei carrying L tone constitute a mirror image of this situation: they are specified for tone only on their \textit{first} syllable. For instance, the noun //\ipa{v̩˩dze}// ‘bird’, the \is{compounds}compound //\ipa{kʰv̩˩mi-hṽ̩}// ‘dog’s hair’, the \is{numerals}numeral-plus-classifier
phrase //\ipa{so˩-dze}// ‘three pairs’, and the object-plus-verb combination //\ipa{li˩ tɕʰi}// ‘to sell
tea’ all carry L tone, which associates with their first syllable. Their remaining syllables receive a~surface phonological tone through phonological rules.

First, L tone spreads, and toneless syllables receive M tone by default.

\begin{enumerate}[leftmargin=2cm, itemsep=0pt, labelwidth=\widthof{Rule~1:}]%[topsep=12pt, partopsep=0pt]
%\begin{enumerate}[leftmargin=!,labelwidth=\widthof{Rule~1:}]
\item[Rule~1:] L tone spreads progressively (“left-to-right”) onto syllables unspecified for tone.
\item[Rule~2:] Syllables that remain unspecified for tone after the application of Rule~1 receive M tone.
\end{enumerate}

The {phrasing} of Rule~2 makes explicit that these rules must be applied in order: if Rule~2 were to apply
before Rule~1, no tonally unspecified syllables would remain for L tone to spread over.

\largerpage
Applying Rule~1 to the examples given above yields the following results:

\begin{enumerate}[itemsep=-1mm]
	\item[] //\ipa{v̩˩dze}// → /\ipa{v̩˩dze˩}/ ‘bird’ 
	\item[] //\ipa{kʰv̩˩mi-hṽ̩}// → /\ipa{kʰv̩˩mi˩-hṽ̩˩}/ ‘dog’s hair’
	\item[] //\ipa{so˩-dze}// → /\ipa{so˩-dze˩}/ ‘three
	pairs’ 
	\item[] //\ipa{li˩ tɕʰi}// → /\ipa{li˩ tɕʰi˩}/ ‘to sell tea’
\end{enumerate}


Subsequently applying Rule~2 yields: 

\begin{enumerate}[itemsep=-1mm]
	\item[] //\ipa{dɑ.ʝi˩}// → /\ipa{dɑ˧ʝi˩}/ ‘mule’
	\item[] //\ipa{po.lo-ɬi.pi˩}// → /\ipa{po˧lo˧-ɬi˧pi˩}/ ‘ram’s
	ear’
	\item[] //\ipa{gv̩-ʂɯ˩}// → /\ipa{gv̩˧-ʂɯ˩}/ ‘nine times’
	\item[] //\ipa{ɖɤ.mi ʑi˩}// → /\ipa{ɖɤ˧mi˧ ʑi˩}/ ‘to
	grab a~fox’
\end{enumerate}

%Application of Rule~1 yields:\\
%//\ipa{v̩˩dze}// > /\ipa{v̩˩dze˩}/ ‘bird’~~~~~~~~~~~~~~~~~~~~//\ipa{kʰv̩˩mi-hṽ̩}// > /\ipa{kʰv̩˩mi˩-hṽ̩˩}/ ‘dog’s hair’\\
%//\ipa{so˩-dze}// > /\ipa{so˩-dze˩}/ ‘three
%pairs’~~~{\kern1pt}//\ipa{li˩ tɕʰi}// > /\ipa{li˩ tɕʰi˩}/ ‘to sell tea’
%
%Application of Rule~2 yields:\\
%//\ipa{dɑ.ʝi˩}// > /\ipa{dɑ˧ʝi˩}/ ‘mule’~~~~~~~~~~~~~~~~~~//\ipa{po.lo-ɬi.pi˩}// > /\ipa{po˧lo˧-ɬi˧pi˩}/ ‘ram’s
%ear’\\
%//\ipa{gv̩-ʂɯ˩}// > /\ipa{gv̩˧-ʂɯ˩}/ ‘nine times’~~~{\kern1pt}//\ipa{ɖɤ.mi ʑi˩}// > /\ipa{ɖɤ˧mi˧ ʑi˩}/ ‘to
%grab a~fox’

After the application of Rules~1 and 2, no toneless syllables remain. (In Yongning Na, it is an exceptionless observation that every syllable carries tone at the surface phonological level.) The rules that follow pertain to tone-group boundaries: tonal oppositions are neutralized in certain positions within the tone group (Rules 3-6), and a~repair rule adds an H tone on the last syllable if the whole group only contains L tones (Rule~7).

\begin{enumerate}[leftmargin=2cm, itemsep=0pt, labelwidth=\widthof{Rule~1:}]%[topsep=12pt, partopsep=0pt]
%\begin{enumerate}[leftmargin=!,labelwidth=\widthof{Rule~1:}]
	\item[Rule~3:] In tone-group-initial position, H and M are neutralized to M.
	\item[Rule~4:] The syllable following an H-tone syllable receives L tone.
	\item[Rule~5:] All syllables following an H.L or M.L sequence receive L tone.
	\item[Rule~6:] In tone-group-final position, H and M are neutralized to H if they follow an L tone.
	\item[Rule~7:] If a~tone group only contains L tones, a~postlexical H tone is added to its last syllable.
\end{enumerate}

Rule~6 is posited on the basis of the observation that there is no opposition between H and M on a~tone-group-final syllable when the preceding
syllables carry L tone. This also applies to tonal contours: LH and LM contours are neutralized to LH in
tone-group-final position. Thus, a~tone-group-final syllable following an L-tone syllable can only
bear one of the following tones at the surface phonological level: L, H, LH or MH. There is no opposition between L{\dots}L.LH and
L{\dots}L.LM, any more than between L{\dots}L.H and L{\dots}L.M. In transcriptions, it appeared advisable to adhere to the principle of
providing a~\is{form!surface}surface transcription of tone, with no more tonal oppositions than those that are actually present at
the phonological surface. This required choosing between two alternatives: transcribing the result of \isi{neutralization} of M and H as M or as H. A phonological reason for adopting the latter choice was that it appeared more appropriate to represent a~two-term opposition by means of the two extreme values of the tone scale (L and H).\footnote{At one stage, the added tone was analyzed as M \citep{michaud2008c}. The choice between M and H may seem to be
		a~non-issue, since LM and LH are neutralized in this position. However, the M tone in Yongning Na is not
		phonologically active: it does not spread, float, or reassociate (see \sectref{sec:analysisofmasadefaulttone}). A~tone rule\is{tone rules} affecting all-L tone groups is therefore far more likely to involve H tone than M tone. In two-tone systems, addition of a~final H tone in domains having only L tone is common: it is attested in Lhasa \ili{Tibetan} \citep[498-499]{sun1997}, \ili{Japanese} \citep[19]{haraguchi1999}, the \ili{Bantu} languages \ili{Matengo} and \ili{Kimatuumbi} \citep[415]{odden2005}, and
		\ili{Shixing} \citep{chirkovaetal2009}. In a~three-tone system, describing the added tone as M could create confusion in cross-linguistic comparisons. For these reasons, the postlexical tone of Yongning Na is analyzed here as~H.} 

An unfortunate consequence is that, in transcriptions, the same word may appear with an L.M pattern in some positions and with L.H in others, even though from a~phonetic point of view the realizations may be indistinguishable. This paradox is taken to an extreme in (\ref{ex:mushcome}).

\begin{exe}
	\ex
	\label{ex:mushcome}
	\ipaex{mo˩kv̩˧ tʰv̩˧-kv̩˩-ze˩. {\kern2pt}|{\kern2pt} mo˩kv̩˥!}\\
	\gll mo˩kv̩\#˥	tʰv̩˧\textsubscript{a}	-kv̩˧˥	 -ze˧\textsubscript{b}		mo˩kv̩\#˥\\
	meadow\_mushroom	to\_grow	\textsc{abilitive}	\textsc{pfv}	meadow\_mushroom\\
	\glt ‘[Starting in the third month,] meadow mushrooms grow. Meadow mushrooms!’ \textit{(Mushrooms.149)} \pandoi{0004616\#S149}
\end{exe}

In example (\ref{ex:mushcome}), the second occurrence of the word //\ipa{mo˩kv̩\#˥}// is transcribed as /\ipa{mo˩kv̩˥}/, with an H tone on the last syllable due to its tone-group-final position, whereas the first occurrence is transcribed as /\ipa{mo˩kv̩˧}/. Phonetically, on the other hand, the second occurrence is realized \textit{lower} than the first, due to utterance-\isi{final lowering}.\footnote{For a~discussion of intonational factors that come into play in the phonetic realization of tones, see Chapter~\ref{chap:fromsurfacephonologicalformstophoneticrealizationintonationandtonalimplementation}.} I can't say that I am perfectly happy with this transcription, but it is principled, systematic and unambiguous. Readers are invited to think of more elegant solutions. 

Rule~7 (“If a~tone group only contains L tones, a~postlexical H tone is added to its last syllable”) is based on the observation that no tone groups contain solely L tones. It is through Rule~7 that L-tone expressions acquire a~final rising \is{tonal contour}contour when spoken \is{form!in isolation}in isolation: when the postlexical H tone is added to a~syllable that carries an L tone, this results in an LH \is{tonal contour}contour on that syllable. Applying Rule~7 to the same examples as
above yields the following: 

\begin{enumerate}[itemsep=-1mm]
	\item[] /\ipa{v̩˩dze˩}/ → /\ipa{v̩˩dze˩˥}/ ‘bird’ 
	\item[] /\ipa{kʰv̩˩mi˩-hṽ̩˩}/ → /\ipa{kʰv̩˩mi˩-hṽ̩˩˥}/ ‘dog’s hair’
	\item[] /\ipa{so˩-dze˩}/ → /\ipa{so˩-dze˩˥}/ ‘three
	pairs’ 
	\item[] /\ipa{li˩ tɕʰi˩}/ → /\ipa{li˩ tɕʰi˩˥}/ ‘to sell tea’
\end{enumerate}

\subsection{About the ordering of rules}
\label{sec:abouttheorderingofrules}

As mentioned above, the rules must be applied in a specific order. If Rule~7, which adds an H tone to all-L sequences,
were to apply before Rule~2, which assigns M tone to toneless syllables, a~sequence such as ‘has not come’, made up of the negation \is{prefixes}prefix \mbox{/\ipa{mɤ˧-}/} and the verb /\ipa{tsʰɯ˩\textsubscript{a}}/ ‘to come.\textsc{pst}’, would only have L tones (/\ipa{mɤ-tsʰɯ˩}/) at the point when Rule~7 applied. It would thus receive a~final H tone
($\ddagger${\kern2pt}\ipa{mɤ-tsʰɯ˩˥}) before undergoing the assignment of an M tone to its first syllable ($\ddagger${\kern2pt}\ipa{mɤ˧-tsʰɯ˩˥}). However, its actual realization is /\ipa{mɤ˧-tsʰɯ˩}/, demonstrating that Rule~7 applies only after all the other rules have taken effect. 

%Rule~2 also applies before Rule~5, as shown by the sequence
%/\ipa{mɤ˧-tsʰɯ˩-sɯ˩}/ ‘has not come yet’, made up of the negation \is{prefixes}prefix, /\ipa{mɤ˧-}/, the verb
%/\ipa{tsʰɯ˩\textsubscript{a}}/ ‘to come.\textsc{pst}’, and /\ipa{sɯ˧}/ ‘yet; first’. The levelling-down of the M tone of
%/\ipa{sɯ˧}/ is a~result of the application of Rule~5, “All syllables following an H.L or M.L sequence receive L tone”; at the point where it applies, the negation \is{prefixes}prefix must therefore be supposed to bear an M
%tone.

Rules 3 and 4 (“H and M are neutralized to M in tone-group-initial position” and “A syllable following an H-tone syllable receives L tone”) likewise require a~specific ordering. If Rule~4 applied first,
an underlying sequence such as //\ipa{dzɯ˥-bi˧}// ‘will eat’ would have its second syllable lowered
to L, yielding $\ddagger${\kern2pt}\ipa{dzɯ˥-bi˩}, and would then undergo Rule~3, resulting in $\ddagger${\kern2pt}\ipa{dzɯ˧-bi˩}. The
observed surface phonological pattern, however, is /\ipa{dzɯ˧-bi˧}/, with M tone on the second
syllable rather than L tone, confirming that Rule~3 must precede Rule~4 in the sequence of application.


\subsection{A discussion of alternative formulations}
\label{sec:adiscussionofalternativeformulations}

The generalizations formulated here in the form of Rules~1-7 could also be captured through other formulations. For instance, it might seem simpler to collapse Rules 4 and 5 into a single rule stating that “H tone can only be followed by L tones”. However, a~separate rule would then be needed specifically for the M.L sequence: “All syllables following an M.L sequence receive L tone”. The decision to adopt the present formulation for Rule~5 (“All syllables following an H.L or M.L sequence receive L tone”) is based on the intuition that a common mechanism underlies both cases: H.L and M.L are
both stepping-down sequences, moving from a~higher tone level to a~lower one. The generalization is that
stepping-down sequences can only be followed by L tones.

This observation could also be phrased as a~static constraint: “There can be no trough within a~tone group” or
“A tone cannot be surrounded by higher tones within a~tone group”. Such a constraint would rule out sequences
such as $\ddagger${\kern2pt}MLH, $\ddagger${\kern2pt}MLM, or $\ddagger${\kern2pt}MHLM. However, while this formulation captures an important restriction, it does not specify how the offending sequences are avoided or repaired in Yongning Na. In contrast, Rule~5 provides an explicit mechanism: the tones that would otherwise result in such sequences are systematically lowered to L.

Rule~4 precludes {\dots}H.H{\dots} sequences, meaning that the H tone is \is{culminativity}culminative. In this
light, it might be possible to analyze H as an HL \is{tonal contour}contour. During the early stages of analysis of the Yongning Na tone system, I~attempted an account in which the underlying phonological entities were not tones but rather stepwise movements along the three-level tonal scale: an upward step (from L to M or from M to H) or a downward step (from H or M to L). Proposals in this vein have been made for \il{Japanese|textbf}Japanese, where the fall from H to L can be treated as a~single phonological entity~-- termed a~\is{tonal accent|textbf}\textit{tonal accent}~-- rather than as a~succession of two distinct tones \citep[1399]{kubozono2012}. Under a~strictly tonal account, the {tonal accent}s of \ili{Japanese} dialects must be represented as sequences of two tones, which is less economical. A~“dynamic treatment of tone” was also attempted for \ili{Igbo} (Niger-Congo family): such is the
title of Mary Clark’s dissertation \citep{clark1976}. 

For Yongning Na, an HL sequence might be reinterpreted as a~“high fall”, ML as a~“low fall”, LM as a~“low rise”, and MH as a~“high rise”. However, the presence of contour tones on individual syllables strongly supports a~strictly tonal
analysis of the system. The dynamic approach, which was ultimately abandoned for Igbo, does not
appear promising for Yongning Na either. (This is of course not to say that it may not prove useful
for other languages.)

Rule~4 also precludes {\dots}H.M{\dots} sequences. This can be described as a~\isi{neutralization}
(to H.L) of the contrast between H.M and H.L.


\subsection{Implications for the tones of sentence-final particles}
\label{sec:implicationsforthetonesofsentenceparticles}

In Yongning Na, as in many East Asian languages, sentence-final particles play a~major role in
conveying evidentiality and speaker attitude. In tonal languages, these particles may carry lexical tone (e.g.~in \ili{Vietnamese}), but there is a~cross-linguistic tendency for an evolution towards reduced tonal distinctions or tonelessness (e.g.~in \ili{Mandarin}). Their position at the end of a~sentence subjects them to strong intonational effects. Moreover, in Yongning Na, sentence-final position implies tone-group-final position, which exerts a~strong influence on phonological tone patterns: the tone of the last syllable in a tone group is often conditioned by preceding tones. If the tone group contains an H tone or an M.L
sequence, all following tones are lowered to L by Rules 4 and 5. For instance, the lexical M tone of 
final particle //\ipa{mæ˧}// (`obviousness') is lowered to L in (\ref{ex:fertil}) because the preceding M.L sequence /\ipa{le˧-ʁæ˩}/ imposes L on all subsequent tones via Rule~5. 

\begin{exe}
	\ex
	\label{ex:fertil}
	\ipaex{ʈʂe˧ ʈʂʰɯ˧ {\kern2pt}|{\kern2pt} le˧-ʁæ˩-ɲi˩ mæ˩.}\\
	\gll ʈʂe˥	ʈʂʰɯ˧	le˧-	ʁæ˩\textsubscript{a}		-ɲi˩		mæ˧\\
	earth	\textsc{top}	\textsc{accomp}		to\_melt/to\_fall\_apart	\textsc{certitude}		\textsc{obviousness}\\
	\glt ‘The clods of earth fall apart / the clods of earth melt [into the water].’ \textit{(Agriculture.54)} \pandoi{0004440\#S54}
\end{exe}

Example (\ref{ex:fertil}) illustrates the majority case. The next tonal pattern in descending order of frequency is that the final particle receives an H tone projected by a~\is{tonal contour}contour
tone lexically attached to the preceding syllable, as in (\ref{ex:wearfelt}). Here, the lexical M tone of the
affirmative particle //\ipa{mæ˧}// is replaced by an H tone (hence /\ipa{mæ˥}/) through reassociation of the H component of the MH \is{tonal contour}contour of the
\textsc{abilitive} //\ipa{-kv̩˧˥}//.

\begin{exe}
	\ex
	\label{ex:wearfelt}
	\ipaex{ʐæ˩sɯ˩˥ {\kern2pt}|{\kern2pt} -dʑo˩ {\kern2pt}|{\kern2pt} ʈʂʰɯ˧ne˧-ʝi˥ {\kern2pt}|{\kern2pt}
		tʰi˧-mv̩˧-kv̩˧ mæ˥.}\\
	\gll ʐæ˩sɯ˩		dʑo˥		ʈʂʰɯ˧ne˧-ʝi˥	tʰi˧-		mv̩˧\textsubscript{a}		-kv̩˧˥		mæ˧\\
	felt	\textsc{top}	thus		\textsc{dur}		to\_put\_on		\textsc{abilitive}		\textsc{obviousness}\\
	\glt ‘This is how we used to wear felt.’ \textit{(Sister3.74)} \pandoi{0004344\#S74}
\end{exe}

Only after transcribing ten texts, containing over 150 instances of //\ipa{mæ˧}//, was
this particle finally observed in a~context where preceding syllables did not impose a~tone
on it: 
%example (\ref{ex:medicinesstom}).

\begin{exe}
	\ex
	\label{ex:medicinesstom}
	\ipaex{hu˧mi˧-ʈʂʰæ˧ɣɯ˧ ʈʰɯ˧˥ {\kern2pt}|{\kern2pt} le˧-qʰwɤ˧-ze˧-mæ˧!{\kern2pt}|{\kern2pt}}\\
	\gll hu˧mi˥\$		ʈʂʰæ˧ɣɯ\#˥	ʈʰɯ˩\textsubscript{b}	le˧-	qʰwɤ˧\textsubscript{b}	-ze˧\textsubscript{b}		mæ˧\\
	stomach		medicine		to\_drink	\textsc{accomp}		to\_heal		\textsc{pfv}		\textsc{obviousness}\\
	\glt ‘[nowadays, the diseased person] drinks
	medicines for the stomach, and [they] are healed, aren’t they!’ \textit{(Healing.66)} \pandoi{0004540\#S66}
\end{exe}

The phrase
/\ipa{le˧-qʰwɤ˧-ze˧}/ ({\textsc{accomp}}-to\_heal-{\textsc{pfv}}) contains neither a~\is{tonal contour}contour nor a~\is{floating tone}floating tone that could associate to a~following syllable. Nor does it contain an H tone
or an M.L sequence that would impose an L tone on subsequent syllables. In this context, which allows expression of the particle's lexical tone, it surfaces with an M. This establishes that the particle does not carry
a~lexical //L//, //H// or \mbox{//MH//} tone, and should be analyzed as carrying underlying \mbox{//M//} tone: //\ipa{mæ˧}//. 

Similarly, the tone of the \is{reported speech}reported-speech particle //\ipa{tsɯ˧˥}// is affected by the preceding tonal environment in the vast majority of cases. Across ten narratives, it
surfaces with MH tone in only eight instances out of more than three hundred. 
%As for the affirmative final particle //\ipa{mv̩˧}//, in almost all examples it follows the
%  reported-speech particle //\ipa{tsɯ˧˥}//, which determines its surface phonological tone.

To determine the lexical tones of sentence-final particles, elicitation proved a~valuable complement to text-based observations. One such case concerns the final particle /\ipa{mo˩}/,
which conveys invitation. This particle cannot appear immediately after a verb, in a~$\ddagger${\kern2pt}V+\ipa{mo˩}
construction. Instead, invitation is expressed as in (\ref{ex:pleaseV}), illustrated with the verb ‘to eat’ in (\ref{ex:pleaseeat}). 

\begin{exe}
	\ex
	\label{ex:pleaseV}
	\ipaex{ɖɯ˧-V-ɻ̩˥ mo˩}\\
	\gll ɖɯ˧-		V		-ɻ̩˥	mo˩\\
	\textsc{delimitative}		{\textit{target verb}}	\textsc{inchoative}		\textsc{disc.ptcl:invitation}\\
	\glt ‘Please go ahead and V a~little!’
\end{exe}

\begin{exe}
	\ex
	\label{ex:pleaseeat}
	\ipaex{ɖɯ˧-dzɯ˧-ɻ̩˥ mo˩!}\\
	\gll ɖɯ˧-		dzɯ˥		-ɻ̩˥	mo˩\\
	\textsc{delimitative}		to\_eat	\textsc{inchoative}		\textsc{disc.ptcl:invitation}\\
	\glt ‘Please have some/please eat some [of it]!’
\end{exe}

\tabref{tab:thetonepatternsoftheconstruction} presents a~set of elicited data, showing that the particle /\ipa{mo˩}/ carries L tone in all six cases. With verbs carrying H, L, and MH tones, this L tone \is{derivation!tonal}derives phonologically from the preceding tonal sequence. Within
a~tone group, a~syllable following an H tone can only have L tone (as observed after both H and MH
tones); likewise, a~syllable following an M.L sequence may only carry L tone. In the case of M tones,
on the other hand, the preceding tonal sequence (M.M.M) does not impose such a~constraint: it does not preclude any of L, M, H or MH on the final syllable. The L tone observed on the surface must therefore be attributed to the lexical specification of the particle, hence its analysis as //\ipa{mo˩}//, with lexical L tone.

\begin{table}%[t]
\caption{\label{tab:thetonepatternsoftheconstruction}The tone patterns of the /\ipa{ɖɯ˧-V-ɻ̩˥ mo˩}/ construction.}
\begin{tabularx}{\textwidth}{ l@{\hspace{40pt}} Q Q l@{\hspace{40pt}} }
  \lsptoprule
	tone & example & meaning & \ipa{/ɖɯ˧-V-ɻ̩˥ mo˩/}\\\midrule
	H & \ipa{dzɯ˥} & to eat & \ipa{ɖɯ˧-dzɯ˧-ɻ̩˥ mo˩}\\ 
	M\textsubscript{a} & \ipa{hwæ˧\textsubscript{a}} & to buy & \ipa{ɖɯ˧-hwæ˧-ɻ̩˧ mo˩}\\ 
	M\textsubscript{b} & \ipa{tɕʰi˧\textsubscript{b}} & to sell & \ipa{ɖɯ˧-tɕʰi˧-ɻ̩˧ mo˩}\\ 
	L\textsubscript{a} & \ipa{bæ˩\textsubscript{a}} & to sweep & \ipa{ɖɯ˧-bæ˩-ɻ̩˩ mo˩}\\ 
	L\textsubscript{b} & \ipa{ʐwɤ˩\textsubscript{b}} & to speak & \ipa{ɖɯ˧-ʐwɤ˩-ɻ̩˩ mo˩}\\ 
	MH & \ipa{lɑ˧˥} & to strike & \ipa{ɖɯ˧-lɑ˧-ɻ̩˥ mo˩}\\
   \lspbottomrule
\end{tabularx}
\end{table}





%%%subsec:8-1-3
%\subsection[Illustration: deriving surface tone for isolated words]{A simple illustration: deriving the surface tone pattern of words spoken in isolation}
%\label{sec:asimpleapplicationderivingthesurfacetonepatternofwordsspokeninisolation}
%
%
%A simple illustration of the tone rules consists in deriving the tones of words spoken \is{form!in isolation}in isolation from their underlying tone category. 




\section{The division of utterances into tone groups}
\label{sec:thedivisionofutterancesintotonegroups}

\is{boundary (between tone groups)|textbf}

The division of an utterance into tone groups is a~central part of Na \isi{prosody}. While there are some general tendencies and
a~few rigid rules in the division of utterances into tone groups, speakers often have several options, with different groupings affecting the relative informational \isi{prominence} of the various
components. \is{prominence}Prominence (linked to \isi{information structure}) and \isi{phrasing} (reflecting syntactic
structure) interact in the division of an utterance into tone groups. There is therefore no
one-to-one {correspondence} between syntactic structure and tone-group division.

Tone groups can have highly diverse syntactic compositions. A~tone group may consist of a~single syllable: \is{monosyllables}monosyllabic nouns and verbs spoken \is{form!in isolation}in isolation constitute a~tone group on their own. Personal pronouns can associate with other words but often appear in an independent tone group, as in (\ref{ex:iwillbuildabridge}).

\begin{exe}
	\ex
	\label{ex:iwillbuildabridge}
	\ipaex{njɤ˧ {\kern2pt}|{\kern2pt} tso˩-bi˩-zo˩-ʝi˩˥.}\\
	\gll njɤ˩	tso˩\textsubscript{a}	bi˧\textsubscript{c}		-zo˧-ʝi˧\\
	1\textsc{sg}	to\_build	to\_go	to\_have\_to\\
	\glt ‘I shall go and build [a bridge].’ \textit{(Renaming.13)} \pandoi{0004534\#S13}
\end{exe}

More commonly, tone groups are longer, encompassing, for example: 

\begin{itemize}
    \item a \is{compounds}compound noun and a \is{numerals}numeral-plus-classifier phrase, as in (\ref{ex:themotherandthedaughterthetwoofthem});
    \item a noun phrase and a verb with its affixes, as in (\ref{ex:therewasnofood});
    \item a \is{numerals}numeral-plus-classifier phrase and a verb with its affixes and particles, as in (\ref{ex:willyoubuildabridge}).
\end{itemize}
  
\begin{exe}
  \ex
  \label{ex:themotherandthedaughterthetwoofthem}
  \ipaex{ə˧mi˧-mv̩˩ ɲi˩-kv̩˩}\\
  \gll ə˧mi˧	mv̩˩ 		ɲi˧-kv̩˧˥\\
  mother	daughter	two-\textsc{clf}.persons\\
  \glt ‘the mother and the daughter, the two of them’ \textit{(Lake4.93)} \pandoi{0004350\#S93}

  \ex
  \label{ex:therewasnofood}
  \ipaex{dzɯ˧-di˧ mɤ˧-dʑo˧˥}\\
  \gll dzɯ˥	-di	mɤ˧	dʑo˧\\
  to\_eat	\textsc{nmlz}	\textsc{neg}	\textsc{exist}\\
  \glt ‘there was no food’ \textit{(Seeds2.69)} \pandoi{0004542\#S69}

  \ex
  \label{ex:willyoubuildabridge}
  \ipaex{dzo˧ {\kern2pt}|{\kern2pt} ɖɯ˧-pɤ˩ tso˩ ə˩-bi˩?}\\
  \gll dzo˩	ɖɯ˧-pɤ˩	tso˩\textsubscript{a} 		ə-˩		-bi˧\\
  bridge	one-\textsc{clf}		to\_build		\textsc{interrog}	\textsc{imm\_fut}\\
  \glt ‘will [you] build a~bridge?’ \textit{(Renaming.10)} \pandoi{0004534\#S10}
\end{exe}

This section begins with the simplest case: morphemes that consistently constitute a~tone group on their own.

\subsection[Morphemes that constitute a~tone group on their own]{Morphemes that always constitute a~tone group on their own}
\label{sec:someelementsalwaysconstituteatonegroupontheirown}

Some morphemes always constitute a~tone group on their own. They could be referred to as \is{tonal standalones|textbf}\textit{tonal standalones}. These include the gap-filler /\ipa{tʰi˩˥}/
‘(and) so, (and) then’; the
contrastive topic marker /\ipa{-no˧˥}/; /\ipa{wɤ˩˥}/ ‘again; also’, which, in quite a~few cases, does not have its full
lexical meaning and functions as a~mere gap-filler; and the \is{intensifiers}intensifier /\ipa{ɖwæ˧˥}/ ‘very’. The first three happen to appear in succession in (\ref{ex:Housebuilding144}): /\ipa{tʰi˩˥ | -no˧˥ | wɤ˩˥}/.

\begin{exe}
	\ex
	\label{ex:Housebuilding144}
	\ipaex{tʰi˧-gv̩˩-se˩-dʑo˩ | tʰi˩˥ | -no˧˥ | wɤ˩˥ | qwɤ˧ tʰi˧-gv̩˩.}\\
	\gll tʰi˧-		gv̩˩\textsubscript{a}	-se˩	-dʑo˥	tʰi˩˥	-no˧˥	wɤ˩˥		qwɤ˧	tʰi˧-	gv̩˩\textsubscript{a}\\
	\textsc{dur}		to\_make/to\_build	\textsc{completion}			\textsc{top}	then	\textsc{cntr.top}		again/also	fire\_pit		\textsc{dur}	to\_make/to\_build\\
	\glt ‘After one has finished to build [the cupboard], well, one builds the fire pit!’ \textit{(Housebuilding.144)} \pandoi{0004448\#S144}
\end{exe}

The gap-fillers /\ipa{tʰi˩˥}/ ‘(and) so, (and) then’ and /\ipa{wɤ˩˥}/ ‘again’ have high textual frequency. The former appears in most sentences in the narratives told by Mrs.\ Latami (consultant F4): over 1,500 occurrences among twenty narratives. The latter appears more than 120 times in the
same twenty narratives. One might speculate that these two items owe part of their conspicuous success to their properties with regard to tone-group divisions: since they always constitute a~tone group
on their own, they demarcate tone groups clearly. They place a final boundary on the preceding tonal group; their own tonal realization is straightforward; and morphotonological computation subsequently starts fresh in a new tone group.

But one may just as well hypothesize an inverse causal link: that these words initially tended to be set apart due to their function as gap-fillers, and ultimately acquired the property of constituting independent tone groups. Support for this hypothesis comes from items that appear to be in the process of becoming \textit{tonal standalones}. The adverb //\ipa{ɖɯ˧ njɤ˧}// ‘continuously, ceaselessly’ is a~case in point. It was elicited in
association with verbs exemplifying the six tone categories of verbs, yielding the results shown in
\tabref{tab:continuously}. In narratives, however, the adverb is always followed by a~tone-group \is{boundary (between tone groups)}boundary, as illustrated in (\ref{ex:theelderswouldalwayssay}), where
//\ipa{ɖɯ˧ njɤ˧}// ‘continuously, ceaselessly’ and //\ipa{ʐwɤ˩}\textsubscript{b}// ‘to say’ remain in separate tone groups. (There are more than thirty examples in the first twenty texts recorded.)

\begin{exe}
	\ex
	\label{ex:theelderswouldalwayssay}
	\ipaex{ hĩ˧mo˥=ɻæ˩ ɳɯ˩   {\kern2pt}|{\kern2pt} ɖɯ˧-njɤ˧ {\kern2pt}|{\kern2pt} ʐwɤ˩-kv̩˩˥ {\kern2pt}|{\kern2pt} mæ˩ ({\dots})}\\
	\gll hĩ˧mo˥=ɻæ˩	ɳɯ˧	ɖɯ˧-njɤ˧	ʐwɤ˩\textsubscript{a}	-kv̩˧˥		mæ˧\\
	elders=\textsc{pl} \textsc{a}		constantly	to\_say	\textsc{abilitive}	\textsc{obviousness}\\
	\glt ‘The elders would always say{\dots}’ \textit{(Dog2.32)} \pandoi{0004555\#S32}
\end{exe}


\begin{table}%[t]
\caption{\label{tab:continuously}The tone patterns of phrases made up of the adverb //\ipa{ɖɯ˧ njɤ˧}// ‘continuously, ceaselessly’ followed by a~verb.}
\begin{tabularx}{\textwidth}{ l@{\hspace{30pt}} Q Q l@{\hspace{30pt}} Q }
  \lsptoprule
	tone & example & meaning & result & tone pattern\\\midrule
	H & \ipa{dzɯ˥} & to eat & \ipa{ɖɯ˧-njɤ˧ dzɯ˧} & M.M.M\\ 
	M\textsubscript{a} & \ipa{hwæ˧\textsubscript{a}} & to buy & \ipa{ɖɯ˧-njɤ˧ hwæ˩} & M.M.L\\ 
	M\textsubscript{b} & \ipa{tɕʰi˧\textsubscript{b}} & to sell & \ipa{ɖɯ˧-njɤ˧ tɕʰi˧} & M.M.M\\ 
	L\textsubscript{a} & \ipa{dze˩\textsubscript{a}} & to cut & \ipa{ɖɯ˧-njɤ˧ dze˧˥} & M.M.MH\\ 
	L\textsubscript{b} & \ipa{ʐwɤ˩\textsubscript{b}} & to speak & \ipa{ɖɯ˧-njɤ˧ ʐwɤ˧˥} & M.M.MH\\ 
	MH & \ipa{lɑ˧˥} & to strike & \ipa{ɖɯ˧-njɤ˧ lɑ˧˥} & M.M.MH\\ 
\lspbottomrule
\end{tabularx}
\end{table}


Using the context of this narrative, an attempt was made to combine the adverb with the verb, but the
consultant judged this incorrect, even when the sentence was truncated after the main verb: $\ddagger${\kern2pt}\ipa{ɖɯ˧ njɤ˧
  ʐwɤ˧˥}. This judgment underscores the fact that the data in \tabref{tab:continuously} was elicited at a~push: in
the present state of the language, such constructions verge on the unacceptable, and the adverb is well advanced on
its path towards becoming a \is{tonal standalones}\textit{tonal standalone}. This example illustrates how easily different data
collection methods can lead to divergent conclusions. The combination of several types of data, gathered with suitable precautions, is indispensable for cumulative progress in
research.\footnote{See the set of themed articles \textit{How to study a~tone language}, edited by Steven Bird and Larry Hyman, in volume 8 of the journal
  \textit{Language Documentation and Conservation} (2014). A~case of diverging notations in
  a~level-tone language is analyzed by \citet{roux2003}, leading to similar recommendations for methodological precautions. Additional perspectives on this issue are presented in
  \citet{niebuhretal2015}.}

A discourse factor that arguably plays a~leading role in the evolution of the adverb //\ipa{ɖɯ˧
  njɤ˧}// ‘continuously, ceaselessly’ is the \isi{emphasis} associated with
it from a~semantic-pragmatic point of view. In narratives, this adverb sometimes carries
\isi{emphatic stress}. The scenario would thus be one of generalization (\isi{lexicalization}) of intonational \isi{emphasis}, a~phenomenon analyzed in \sectref{sec:acaseofhabitualassociationofintonationalfocalizationtoaphraseillustratingtheaffinitiesbetweenfocalizationandnegation}.


\subsection{Topicalized constituents always end a~tone group}
\label{sec:atonegroupboundaryisalwaysfoundaftertopicalizedphrases}

A~tone group \is{boundary (between tone groups)}boundary invariably follows topicalized constituents. More specifically: 
\begin{itemize}
\item The topic marker /\ipa{-dʑo˥}/ always terminates a~tone group. No \is{exceptions}exception has been found among 2,000 examples from narratives.
\item The topic marker /\ipa{ʈʂʰɯ˧}/ likewise marks the end of a~tone group, except in cases where it is immediately followed by the topic marker just mentioned: /\ipa{-dʑo˥}/.
\item The contrastive topic marker /\ipa{-no˧˥}/ invariably constitutes a~tone group on its own, as noted above and illustrated by (\ref{ex:asforthecatithasalifespanoffourfiveyears}).
\end{itemize}

\begin{exe}
  \ex
  \label{ex:asforthecatithasalifespanoffourfiveyears}
  \ipaex{hwɤ˧li˧˥ {\kern2pt}|{\kern2pt} -no˧˥, {\kern2pt}|{\kern2pt} ʐv̩˧kʰv̩˩-ŋwɤ˩kʰv̩˩. {\kern2pt}|{\kern2pt}}\\
  \gll hwɤ˧li˧˥	-no˧˥	ʐv̩˧kʰv̩˩		ŋwɤ˧kʰv̩˩\\
  cat		\textsc{cntr.top}	four.years	five.years\\
  \glt ‘As for the cat, [it has a~lifespan of] four or five years.’ \textit{Context:} the previous
  discussion concerns the dog’s lifespan, and the speaker now shifts the focus to cats. \textit{(Dog2.84)} \pandoi{0004555\#S84}
\end{exe}


\subsection[Flexibility in the division into tone groups]{Flexibility in the division into tone groups}
\label{sec:optionslefttothespeakerinthedivisionintotonegroups}

Apart from the cases presented in \sectref{sec:someelementsalwaysconstituteatonegroupontheirown}--\ref{sec:atonegroupboundaryisalwaysfoundaftertopicalizedphrases}, speakers generally have several options. They may choose to integrate large stretches of speech into
a~single tone group, or they may divide the utterance into multiple tone groups, with the \is{stylistics}stylistic effect of highlighting these individual
components one after the other. This parallels observations about the
\isi{intonation} of numerous languages. For instance, \citet[204]{karcevskij1931} remarks on \ili{Russian} and \ili{German}:
“Within certain limits, it is possible to change the position of the rhythmic breaks that separate a~sentence into
parts”.\footnote{\textit{Original text:} Dans certaines limites, nous
  pouvons déplacer les anti-cadences séparant les membres de la
  phrase.} 
  
An interesting characteristic of Yongning Na is that this
division exerts a~strong influence on tone, as tonal
processes never apply across tone-group junctures.

For instance, /\ipa{dzɯ˧-di˧˥}/ ‘things to eat; food’, from /\ipa{dzɯ˥}/ ‘to eat’ and the
{nominalizer} /\ipa{-di˩}/, can combine with /\ipa{mɤ˧-dʑo˧}/ ‘there isn’t’ to express ‘there isn’t any
food, there is nothing to eat’. The noun and the negated verb may either be integrated into a~single tone group, yielding /\ipa{dzɯ˧-di˧ mɤ˧-dʑo˧˥}/, or separated: /\ipa{dzɯ˧-di˧˥ {\kern2pt}|{\kern2pt} mɤ˧-dʑo˧}/. The latter
option is illustrated in (\ref{ex:therewasnothingtoeatandnothingtodrink}):

\begin{exe}
  \ex
  \label{ex:therewasnothingtoeatandnothingtodrink}
  \ipaex{dzɯ˧-di˧˥   {\kern2pt}|{\kern2pt}   mɤ˧-dʑo˧,   {\kern2pt}|{\kern2pt}   ʈʰɯ˩-di˩˥  {\kern2pt}|{\kern2pt}  mɤ˧-dʑo˧!}\\
  \gll dzɯ˥	-di˩	mɤ˧-	dʑo˧\textsubscript{b}	ʈʰɯ˩\textsubscript{b}	-di˩	mɤ˧-	dʑo˧\textsubscript{b}\\
  to\_eat	\textsc{nmlz}	\textsc{neg}	\textsc{exist}	to\_drink	\textsc{nmlz}	\textsc{neg}
  \textsc{exist}\\
  \glt ‘[Before mankind had learnt to grow crops], there was nothing to eat and nothing to drink!’
  \textit{(Seeds2.67)} \pandoi{0004542\#S67}
\end{exe}

In
(\ref{ex:therewasnothingtoeatandnothingtodrink}), separating the noun phrase /\ipa{dzɯ˧-di˧˥}/ ‘food’ from the negated \is{existentials}existential verb /\ipa{mɤ˧-dʑo˧}/ ‘there
isn’t’, creating two tone groups, has the effect of \is{emphasis}emphasizing the two noun
phrases, /\ipa{dzɯ˧-di˧˥}/ ‘food; things to eat’ and /\ipa{ʈʰɯ˩-di˩}/ ‘drink; beverage’. 
%This could
%be analyzed as a~case of \isi{focalization}, and transcribed as /\ipa{dzɯ˧-di˧˥ F {\kern2pt}|{\kern2pt} mɤ˧-dʑo˧, {\kern2pt}|{\kern2pt} ʈʰɯ˩-di˩˥
%  F {\kern2pt}|{\kern2pt} mɤ˧-dʑo˧}/, where the symbol ‘F’ indicates intonational \isi{focalization}. The presence of
%a~tone-group \is{boundary (between tone groups)}boundary before the negation \is{prefixes}prefix could then be interpreted as a~consequence of
%\isi{focalization}.

The following sentence in the story reiterates the statement ‘There was no food’, continuing the same
strategy of highlighting the noun phrase ‘food’, this time adding the topic marker /\ipa{-dʑo˥}/ (\ref{ex:nofood}). The effect is to emphasize how dire the situation was getting.

\begin{exe}
	\ex
	\label{ex:nofood}
	\ipaex{dzɯ˧-di˧˥ {\kern2pt}|{\kern2pt} -dʑo˩, {\kern2pt}|{\kern2pt} mɤ˧-dʑo˧-ɲi˥ tsɯ˩ {\kern2pt}|{\kern2pt} mv̩˩!}\\
	\gll dzɯ˥	-di˩				-dʑo˥				mɤ˧-			dʑo˧\textsubscript{b}			-ɲi˩							tsɯ˧˥				mv̩˧\\
		to\_eat		\textsc{nmlz}	\textsc{top}	\textsc{neg}	\textsc{exist}	\textsc{certitude}	\textsc{rep}	\textsc{affirm}\\
	\glt  ‘As for food, it’s said that there was none!’	\textit{(Seeds2.68)} \pandoi{0004542\#S68}
\end{exe}

The narrator then recapitulates in (\ref{ex:astherewasnothingtoeat}):

\begin{exe}
  \ex
  \label{ex:astherewasnothingtoeat}
  \ipaex{dzɯ˧-di˧ mɤ˧-dʑo˧˥  {\kern2pt}|{\kern2pt}  -dʑo˩  {\kern2pt}|{\kern2pt}  tʰi˩˥ {\dots}}\\
  \gll dzɯ˥	-di˩	mɤ˧-	dʑo˧\textsubscript{b}	-dʑo˥	tʰi˩˥\\
  to\_eat	\textsc{nmlz}	\textsc{neg}	\textsc{exist}	\textsc{top}	so/then\\
  \glt  ‘As there was nothing to eat, {\dots}’ (the narrative moves on to: ‘there were some
  exceptional, smart people, who stood up and did something about it’) \textit{(Seeds2.69)} \pandoi{0004542\#S69}
\end{exe}

At this juncture, ‘there was no food’ is integrated into a single tone group and followed by the
topic marker /\ipa{-dʑo˥}/. This provides an exemplary illustration of how larger
chunks of information become incorporated into one tone group as they shift from new to
old and backgrounded information.

Long tone groups, within which phonological and morphosyntactic tone rules operate freely, yield a~\is{stylistics}stylistic effect of carefully constructed,
poised, stately speech. Conversely, in lively discourse, pragmatic phenomena of \isi{emphasis} take centre stage, and tone-group boundaries are inserted at various points to draw attention to the preceding word or
phrase. Even \isi{function words} can be highlighted in this manner, as in (\ref{ex:itissaidthatonthatoccasionthewholefamilywillkowtow}).
\begin{exe}
  \ex
  \label{ex:itissaidthatonthatoccasionthewholefamilywillkowtow}
  \ipaex{ɬo˧pv̩˥ ti˩-kv̩˩ {\kern2pt}|{\kern2pt} tsɯ˧˥ {\kern2pt}|{\kern2pt} mv̩˩!}\\
  \gll ɬo˧pv̩˥	ti˩\textsubscript{a}	-kv̩˧˥		tsɯ˧˥	mv̩˧\\
  kow-tow	to\_hit	\textsc{abilitive}	\textsc{rep}	\textsc{affirm}\\
  \glt ‘It is said that [on that occasion, the whole family] will kow-tow!’ \textit{(Sister3.138)} \pandoi{0004344\#S138}
\end{exe}

A simpler formulation would be /\ipa{ɬo˧pv̩˥ ti˩-kv̩˩ tsɯ˩ {\kern2pt}|{\kern2pt} -mv̩˩}/. The formulation in
(\ref{ex:itissaidthatonthatoccasionthewholefamilywillkowtow}) \is{emphasis}emphasizes the \is{reported speech}reported-speech
particle /\ipa{tsɯ˧˥}/. This evidential particle is used whenever the speaker only has indirect knowledge of an event, and hence appears repeatedly in narratives. But in the
context of (\ref{ex:itissaidthatonthatoccasionthewholefamilywillkowtow}), the particle takes on its full significance: the narrator never witnessed the ritual she describes. The \isi{emphasis} placed on the evidential particle in this sentence exemplifies the speaker's 
commitment to truthfulness and precision.

%\subsubsection{The role of the morphological complexity of constituents}
%\label{sec:theroleofthemorphologicalcomplexityofconstituents}

%The degree of internal complexity of the successive constituents of an utterance is among the parameters that influence its division into tone groups. A~verb without prefixes or suffixes is typically monosyllabic and readily associates with a~preceding adverb or noun. 


% Removed: 1st edition: "whose output tone is determined by the tone rules that apply in subject-plus-verb phrases (see~\sectref{sec:subjectandverb})". Wrong: the regular pattern (whether for O+V, the more likely interpretation, or for S+V) is M.M.MH. 
When a~directional adverb is inserted between a~noun and a~verb, it can be integrated into the tone group, as in (\ref{ex:stretchdown}), where the noun, the adverb //\ipa{mv̩˩-tɕo˧}//
‘downward’, and the verb form a~single tone group, just like the simple object-plus-verb combination /\ipa{kʰɯ˧tsʰɤ˧ ʈʂʰe˧}/ ‘to stretch out [one’s] legs’ (from /\ipa{kʰɯ˧tsʰɤ˧˥}/
‘leg’ and /\ipa{ʈʂʰe˧\textsubscript{b}}/ ‘to stretch out’).
\begin{exe}
	\ex
	\label{ex:stretchdown}
	\ipaex{kʰɯ˧tsʰɤ˧
		mv̩˥-tɕo˩ ʈʂʰe˩}\\
	\gll kʰɯ˧tsʰɤ˧˥		mv̩˩-tɕo˧		ʈʂʰe˧\textsubscript{b}\\
	leg		downward	to\_stretch\\
	\glt ‘to stretch (one’s) leg downward’
\end{exe}

However, the adverb often marks the beginning of a~new tone group, as in (\ref{ex:stretchdown2}), where the noun is in a~separate tone group from the adverb and verb.

\begin{exe}
	\ex
	\label{ex:stretchdown2}
	\ipaex{kʰɯ˧tsʰɤ˧˥ {\kern2pt}|{\kern2pt} mv̩˩tɕo˧ ʈʂʰe˧}\\
	\gll kʰɯ˧tsʰɤ˧˥		mv̩˩tɕo˧		ʈʂʰe˧\textsubscript{b}\\
	leg		downward	to\_stretch\\
	\glt ‘to stretch (one’s) leg downward’
\end{exe}

Like directional adverbs, \is{numerals}numeral-plus-classifier phrases often introduce a~new tone group, as in (\ref{ex:sheetofpaper}). But
they can also be integrated into a~single tone group with a~preceding noun, as in (\ref{ex:motheranddaughter}). 

\begin{exe}
	\ex
	\label{ex:sheetofpaper}
	\ipaex{ʂv̩˧{$\sim$}ʂv̩˧˥ {\kern2pt}|{\kern2pt} ɖɯ˧-pʰæ˧˥}\\
	\gll ʂv̩˧{$\sim$}ʂv̩˧˥	ɖɯ˧-pʰæ˧˥\\
	paper		one-\textsc{clf}.flat\_objects\\
	\glt ‘a sheet of paper’
\end{exe}

\begin{exe}
	\ex
	\label{ex:motheranddaughter}
	\ipaex{ə˧mi˧-mv̩˩ ɲi˩-kv̩˩}\\
	\gll ə˧mi˧	mv̩˩˥		ɲi˧-kv̩˧˥\\
	mother		daughter	two-\textsc{clf}.persons\\
	\glt  ‘the mother and her daughter’ \textit{Literally:} ‘mother and
		daughter, the two’ (\textit{Tiger.11} \pandoi{0004444\#S11}, \textit{51}, and \textit{Lake4.93} \pandoi{0004350\#S93}\textit{, 96-98, 125})
\end{exe}

The expression in (\ref{ex:motheranddaughter}) is a~special case: /\ipa{ə˧mi˧-mv̩˩}/ is a~coordinative compound meaning ‘mother and daughter’. 
The \is{numerals}numeral-plus-classifier
phrase is clearly not intended to count mother-and-daughter pairs~-- otherwise, the classifier would be that for pairs rather than persons. Instead, the expression can be paraphrased as ‘these two: the mother and the daughter’. 
% {\linebreak}: xyz useful?

Demonstrative-plus-classifier\is{classifiers} phrases are commonly integrated with a~preceding noun. For instance,
in the first version of the \textit{Lake} story, the same two characters, a~mother and her daughter, are
referred to as /\ipa{ə˧mi˧ ʈʂʰɯ˧-v̩˧ lɑ˩ {\kern2pt}|{\kern2pt} mv̩˩ ʈʂʰɯ˩-v̩˩˥}/,
‘that mother and that daughter’: see (\ref{ex:thatmotherandthatdaughter}). 
%(On this topic, see the discussion of classifiers in Chapter~\ref{chap:classifiers}.)

\begin{exe}
  \ex
  \label{ex:thatmotherandthatdaughter}
  \ipaex{ə˧mi˧ ʈʂʰɯ˧-v̩˧ lɑ˩ {\kern2pt}|{\kern2pt} mv̩˩ ʈʂʰɯ˩-v̩˩˥}\\
  \gll ə˧mi˧ 	ʈʂʰɯ˧-v̩˧		lɑ˧	mv̩˩˥		ʈʂʰɯ˧-v̩˧\\
  mother 	\textsc{dem}-\textsc{clf}.individual	and	daughter	\textsc{dem}-\textsc{clf}\\
  \glt ‘that mother and that daughter’ \textit{(Lake3.52)} \pandoi{0004348\#S52}
\end{exe}

The choices made by a~speaker in placing tone-groupe boundaries have to do with considerations of informational \isi{prominence}, as outlined below. 


%\subsection[Some general tendencies]{Some general tendencies in the division into tone groups}
%\label{sec:somegeneraltendenciesinthedivisionintotonegroups}

\subsection{The role of information structure: Considerations of prominence}
\label{sec:theroleofinformationstructureconsiderationsofprominence}

As a first illustration of how \isi{information structure} influences the division into tone groups, consider (\ref{ex:itssaidthatonemustnteatdogmeat}): 

\begin{exe}
  \ex
  \label{ex:itssaidthatonemustnteatdogmeat}
  \ipaex{kʰv̩˩mi˩-ʂe˩˥, {\kern2pt}|{\kern2pt} dzɯ˧ mɤ˧-ɖo˧ pi˧-zo˥!}\\
  \gll kʰv̩˩mi˩-ʂe˩	dzɯ˥	mɤ˧-	ɖo˧\textsubscript{a}		pi˥	-zo\\
  dog-meat	to\_eat	\textsc{neg}	ought\_to	to\_say	\textsc{advb}\\
  \glt ‘It’s said that one mustn’t eat dog meat! / It’s said that dog meat is something one must not
  eat!’ \textit{(Dog2.37)} \pandoi{0004555\#S37}
\end{exe}

In (\ref{ex:itssaidthatonemustnteatdogmeat}), the noun phrase ‘dog meat’ is set into relief by
forming a~tone group on its own. Despite the absence of a~morphemic indication of
topicalization, 
% such as a~topic marker, 
%/\ipa{ʈʂʰɯ˧}/ or /\ipa{-dʑo˥}/, 
it clearly functions as the topic of the utterance.
In this context, tonal integration with the following verb would not be stylistically
appropriate.

Likewise, in (\ref{ex:intheoldtimesonewouldntusuallyletdogsgooutside}), the {adverbial} ‘outside’,
/\ipa{ə˩pʰo˩}/, constitutes a~tone group on its own. An alternative option would be to integrate it tonally
with the following verb. (The tonal paradigms for \textit{spatial adverbial}+\textit{verb} combinations
are set out in Chapter~\ref{chap:verbsandtheircombinatoryproperties}.) In this context, integration into a~single tone group would be 
\is{stylistics}stylistically acceptable. Separation
into two tone groups, however, has the effect of providing information incrementally, giving the impression
that the speaker is constructing the utterance as she speaks, rather than delivering preplanned chunks of speech.

\begin{exe}
  \ex
  \label{ex:intheoldtimesonewouldntusuallyletdogsgooutside}
  \ipaex{ə˧ʝi˧-ʂɯ˥ʝi˩-dʑo˩, {\kern2pt}|{\kern2pt} kʰv̩˧ ʈʂʰɯ˧-dʑo˩, {\kern2pt}|{\kern2pt} dʑɤ˩˥ {\kern2pt}|{\kern2pt} ə˩pʰo˩˥ {\kern2pt}|{\kern2pt} kʰɯ˧ mɤ˥-kv̩˩!}\\
  \gll ə˧ʝi˧-ʂɯ˥ʝi˩	dʑo˥	kʰv̩˥	ʈʂʰɯ˧	-dʑo˥	dʑɤ˩˥		ə˩pʰo˩ kʰɯ˧˥	mɤ˧-	-kv̩˧˥\\
  in\_the\_past	\textsc{top}	dog	\textsc{top}	\textsc{top}	\textsc{ints}	outside
  to\_let	\textsc{neg}	\textsc{abilitive}\\
  \glt ‘In the old times, one wouldn’t usually let dogs go outside! / In the old times, dogs weren’t usually allowed to leave the house!’ \textit{(Dog2.75)} \pandoi{0004555\#S75}
\end{exe}

It is uncommon for a~verb preceded by the {accomplished} \is{prefixes}prefix /\ipa{le˧-}/ to interact tonally with
a~preceding noun phrase. In \textit{Caravans.191} \pandoi{0004530\#S191}, for instance, ‘the uncle comes back’ is realized as
/\ipa{ə˧v̩˧˥ {\kern2pt}|{\kern2pt} le˧-tsʰɯ˩}/ rather than /\ipa{ə˧v̩˧ le˥-tsʰɯ˩}/, although the latter form is also
acceptable. Cases where tonal interaction does take place are characterized by a~high degree of
semantic givenness, as illustrated in (\ref{ex:therearrivedonepersonthentwothenthree}):
\begin{exe}
  \ex
  \label{ex:therearrivedonepersonthentwothenthree}
  \ipaex{ɖɯ˧-v̩˧ le˧-tsʰɯ˩, {\kern2pt}|{\kern2pt} ɲi˧-kv̩˧ le˧-tsʰɯ˧˥, {\kern2pt}|{\kern2pt} so˩-kv̩˩ le˩-tsʰɯ˩˥.}\\
  \gll ɖɯ˧-v̩˧	le˧-	tsʰɯ˩\textsubscript{a}	ɲi˧-kv̩˧˥	so˩-kv̩˩\\
  one-\textsc{clf}.individual	\textsc{accomp}	to\_come.\textsc{pst}	two-\textsc{clf}	three-\textsc{clf}\\
  \glt ‘One person came, then two, then three.’ \textit{Context:} explanation proposed by
  consultant F4 during a~discussion of \textit{Lake4.126} \pandoi{0004350\#S126}. (Field notes.)
\end{exe}
 
It would not be incorrect to say /\ipa{ɖɯ˧-v̩˧ {\kern2pt}|{\kern2pt} le˧-tsʰɯ˩, {\kern2pt}|{\kern2pt} ɲi˧-kv̩˧˥ {\kern2pt}|{\kern2pt} le˧-tsʰɯ˩, {\kern2pt}|{\kern2pt} so˩-kv̩˩˥ {\kern2pt}|{\kern2pt}
  le˧-tsʰɯ˩}/, but this would be inappropriate in a~context where the \isi{emphasis} is on the count, not on the verb. %Since there is no reason to There would be no point in setting the subject apart from the verb, hence the division into three tone groups, rather than six.

When an explanation is added as an afterthought, a~relatively long sequence of syllables can be
integrated into a single tone group, as in (\ref{ex:fromchengduinthepastsilk}), where the final tone
group contains ten syllables.
\begin{exe}
  \ex
  \label{ex:fromchengduinthepastsilk}
  \ipaex{jɤ˧ŋɤ˧-dʑo˧, {\kern2pt}|{\kern2pt} ə˧ʝi˧-ʂɯ˥ʝi˩, {\kern2pt}|{\kern2pt} hæ̃˩-bɑ˥lɑ˩! {\kern2pt}|{\kern2pt} hæ̃˩-bɑ˥lɑ˩-bɑ˩lɑ˩ le˩-po˩ jo˩-kv̩˩ mæ˩!}\\
  \gll jɤ˧ŋɤ˧		-dʑo˥	ə˧ʝi˧-ʂɯ˥ʝi˩	hæ̃˩-bɑ˥lɑ˩	hæ̃˩-bɑ˥lɑ˩-bɑ˩lɑ˩  le˧-		po˧˥		jo˩		-kv̩˧˥
  mæ˧\\
  Chengdu	\textsc{top}	in\_the\_past	silk		silk\_clothes \textsc{accomp}	to\_bring	to\_come
  \textsc{abilitive}	\textsc{obviousness}\\
  \glt ‘From Chengdu, in the past{\dots} Silk!! [The people who went on caravans] would bring back
  silk clothing [from their journeys to Chengdu]!’ \textit{(Caravans.104-105)} \pandoi{0004530\#S104}
\end{exe}

\largerpage
Here, the word ‘silk’ conveys the essential information and stands as a~tone group of its own. The final tone group serves as a~backgrounded explanation, a~status reflected in the levelling down of all the tones in /\ipa{le˩-po˩-jo˩-kv̩˩-mæ˩}/ ‘[they] would bring back’.

As many as twelve syllables are bunched together in (\ref{ex:elders3}):

\begin{exe}
	\ex
	\label{ex:elders3}
	\ipaex{ə˧ʑi˧˥, {\kern2pt}|{\kern2pt} ɖɯ˩mɑ˧-ɬɑ˩tsʰo˩
		pi˩-hĩ˩ ɖɯ˩-v̩˩ dʑo˩-ɲi˩ tsɯ˩ {\kern2pt}|{\kern2pt} mv̩˩.}\\
	\gll ə˧ʑi˧˥		ɖɯ˩mɑ˧-ɬɑ˩tsʰo˩		pi˥		-hĩ˥ ɖɯ˧	v̩˧		dʑo˧\textsubscript{a}	-ɲi˩		tsɯ˧˥		mv̩˧\\
	grandmother		\textsc{given\_name}	to\_say		\textsc{rel/nmlz}	one	\textsc{clf}.individual		\textsc{exist}	\textsc{certitude}	\textsc{rep}	\textsc{affirm}\\
	\glt ‘[Among] women elders, it is said that there was one by the
	name of Ddeema Lhaco.’ \textit{(Elders3.11)} \pandoi{0004532\#S11}
\end{exe}

The speaker places considerable \isi{emphasis} on the person’s name,
\textit{Ddeema Lha\-co}, while the rest of the sentence is strongly backgrounded. Phonologically, the name and all that follows are integrated into
a single tone group, causing all the syllables from the third to the twelfth and last
to be lowered to L.

As a~last example, consider (\ref{ex:youCAME}).

\begin{exe}
	\ex
	\label{ex:youCAME}
	\ipaex{dzo˧ {\kern2pt}|{\kern2pt} le˧-gv̩˩ {\kern2pt}|{\kern2pt} tʰi˧-tɕɯ˥ {\kern2pt}|{\kern2pt} tʰi˩˥ {\kern2pt}|{\kern2pt} no˧ {\kern2pt}|{\kern2pt} le˧-tsʰɯ˩-ɲi˩-ze˩-mæ˩, {\kern2pt}|{\kern2pt} ə˩-gi˩! {\kern2pt}|{\kern2pt} hĩ˧ ɖɯ˧-v̩˧ mɤ˧-tsʰɯ˩! {\kern2pt}|{\kern2pt} no˩ le˩-tsʰɯ˩-ɲi˥-ze˩ mæ˩!}\\
	\gll dzo˩		le˧-	gv̩˩\textsubscript{b}	tʰi˧-	tɕɯ˥	tʰi˩˥	no˩	le˧-	tsʰɯ˩\textsubscript{a}	-ɲi˩		-ze˧\textsubscript{b}		mæ˧	ə˩-gi˩		hĩ˥	ɖɯ˧		v̩˧		mɤ˧-		tsʰɯ˩\textsubscript{a}	no˩	le˧-	tsʰɯ˩\textsubscript{a}	-ɲi˩		-ze˧\textsubscript{b}		mæ˧\\
	bridge		\textsc{accomp}	to\_build		\textsc{dur}	to\_put		then 2\textsc{sg}	\textsc{accomp}		to\_come	\textsc{certitude}	\textsc{pfv}	\textsc{affirm}		isn't\_it	person	one	\textsc{clf}.individual		\textsc{neg}	to\_come	2\textsc{sg}	\textsc{accomp}		to\_come	\textsc{certitude}	\textsc{pfv}	\textsc{affirm}\\
	\glt ‘After the bridge is built, and left there [=and the person who built it waits for someone to cross], you come along! [=someone comes along: you, for instance!] [For a long time] nobody comes, [but at last] you come along!’ \textit{(Renaming.17)} \pandoi{0004534\#S17}
\end{exe}

The same syntactic structure, ‘you come along’, is realized as
two tone groups: /\ipa{no˧ {\kern2pt}|{\kern2pt} le˧-tsʰɯ˩ ɲi˩-ze˩ mæ˩}/, then repeated as a~single tone group:
/\ipa{no˩ le˩-tsʰɯ˩ ɲi˥-ze˩ mæ˩}/. This provides an exemplary illustration of how tone groups tend to be longer when the speaker assumes that the semantic content is already familiar
to the listener.


\subsection{Extreme cases of tonal integration: Set phrases and proverbs}
%[Extreme cases of tonal integration]
\label{sec:extremecasesoftonalintegrationsetphrasesandproverbs}

%\subsubsection{Tonal integration in set phrases}
\label{sec:tonalintegrationinsetphrases}

Set phrases constitute a~typical case of integration. For instance, formulaic expressions recapitulate affinities among animals symbolizing the twelve Terrestrial Branches. These subsets play a role in fortune-telling, where the year of birth serves as a~basis for predicting whether or not an individual will be able to relate harmoniously with another. Within the twelve-year cycle, the twelve animals are grouped into four sets of three, shown in (\ref{ex:SerpentOxRooster})-(\ref{ex:TigerHorseDog}).

\begin{exe}
	\ex
	\label{ex:SerpentOxRooster}
	\ipaex{bv̩˧ʐv̩˧ ʝi˧ {\kern2pt}|{\kern2pt} æ̃˩ so˥-kʰv̩˩}\\
	\gll bv̩˧ʐv̩˧	ʝi˥	æ̃˩˧	so˩		kʰv̩˧˥\textsubscript{a}\\
	snake		ox	chicken	three	\textsc{clf}.years\\
	\glt ‘the three years of the Snake, the Ox, and the Rooster’
\end{exe}

\begin{exe}
	\ex
	\label{ex:DragonApeRat}
	\ipaex{mv̩˧gv̩˧ ʑi˧˥ {\kern2pt}|{\kern2pt} hwɤ˧ so˧-kʰv̩˥}\\
	\gll mv̩˧gv̩˧	ʑi˩˥	hwɤ˧	so˩		kʰv̩˧˥\textsubscript{a}\\
	dragon		ape		rat		three	\textsc{clf}.years\\
	\glt ‘the three years of the Dragon, the Ape, and the Rat’
	%\footnote{The Na twelve-year cycle has the Cat instead of the Rat. Various cases of replacement are observed in the history of the cycle's geographical travels, such as a~change from buffalo to ox when the cycle was borrowed into \il{Sinitic}Chinese culture from an Austroasiatic origin, i.e.\ from a~more Southern environment to a~more Northern environment; this change was later reverted (from ox to buffalo) when the \il{Sinitic}Chinese cycle was borrowed into \ili{Vietnamese}, and thence into Khmer \citep{coedes1935,ferlus2010,ferlus2013b}.}
\end{exe}

\begin{exe}
	\ex
	\label{ex:RabbitPigSheep}
	\ipaex{tʰo˧li˧ bo˩ {\kern2pt}|{\kern2pt} jo˩ so˩-kʰv̩˩˥}\\
	\gll tʰo˧li˧	bo˩˧		jo˩		so˩		kʰv̩˧˥\textsubscript{a}\\
	rabbit		pig			sheep	three	\textsc{clf}.years\\
	\glt ‘the three years of the Rabbit, the Pig, and the Sheep’
\end{exe}

\begin{exe}
	\ex
	\label{ex:TigerHorseDog}
	\ipaex{lɑ˧ {\kern2pt}|{\kern2pt} ʐwæ˧ {\kern2pt}|{\kern2pt} kʰv̩˧ {\kern2pt}|{\kern2pt} so˩-kʰv̩˩˥}\\
	\gll lɑ˧		ʐwæ˥	kʰv̩˥		so˩		kʰv̩˧˥\textsubscript{a}\\
	tiger		horse	dog	three	\textsc{clf}.years\\
	\glt ‘the three years of the Tiger, the Horse, and the Dog’
\end{exe}


%\begin{itemize}
%\item /\ipa{bv̩˧ʐv̩˧ {\kern2pt}|{\kern2pt} ʝi˧ {\kern2pt}|{\kern2pt} æ˩˥}/ are grouped as /\ipa{bv̩˧ʐv̩˧, ʝi˧, {\kern2pt}|{\kern2pt} æ˩-so˥-kʰv̩˩}/ ‘the three years of the Serpent, the Ox, and the Rooster’;
%\item /\ipa{mv̩˧gv̩˧ {\kern2pt}|{\kern2pt} ʑi˩˥ {\kern2pt}|{\kern2pt} hwɤ˧˥}/ are grouped as /\ipa{mv̩˧gv̩˧ ʑi˧˥ {\kern2pt}|{\kern2pt} hwɤ˧ so˧-kʰv̩˥}/ ‘the three years of the Dragon, the Ape, and the Rat’; 
%\item /\ipa{tʰo˧li˧ {\kern2pt}|{\kern2pt} bo˩˥ {\kern2pt}|{\kern2pt} jo˩˥}/ are grouped as /\ipa{tʰo˧li˧-bo˩ {\kern2pt}|{\kern2pt} ʝo˩-so˩kʰv̩˩˥}/ ‘the three years of the Rabbit, the Pig, and the Sheep’;
%\item /\ipa{lɑ˧ {\kern2pt}|{\kern2pt} ʐwæ˧ {\kern2pt}|{\kern2pt} kʰv̩˧}/ are grouped as /\ipa{lɑ˧, {\kern2pt}|{\kern2pt} ʐwæ˧, {\kern2pt}|{\kern2pt} kʰv̩˧ {\kern2pt}|{\kern2pt} so˩-kʰv̩˩˥}/ ‘the three years of the Tiger, the Horse, and the Dog’.
%\end{itemize}


The tonal grouping in these four phrases is not entirely uniform. In (\ref{ex:SerpentOxRooster}), (\ref{ex:DragonApeRat}), and (\ref{ex:RabbitPigSheep}), the phrase is divided into two tone groups, instead of the four tone groups that would obtain if each animal name were said separately. This results in a stronger degree of integration than in a~list where each item forms an independent tone group. The first tone group comprises the names of two animals, while the second contains the third animal name along with the phrase ‘three years’. Each of these two tone groups consists of three syllables, creating a~\is{rhythm}rhythmic balance. The tone that arises from the association of two animal names conforms to the patterns observed in coordinative compounds (as detailed in \sectref{sec:coordinativecompounds}): a disyllabic M-tone noun plus a {monosyllabic} H-tone noun yields \mbox{//\#H//} (e.g.\ //\ipa{bv̩˧ʐv̩˧ ʝi\#˥}//), which surfaces as M tone: /\ipa{bv̩˧ʐv̩˧ ʝi˧}/.

For ease of comparison, the forms that would obtain if each noun stood in a~tone group of its own are shown as (\ref{ex:SerpentOxRoosterCUT}), (\ref{ex:DragonApeRatCUT}), and (\ref{ex:RabbitPigSheepCUT}). 

\begin{exe}
	\ex
	\label{ex:SerpentOxRoosterCUT}
	\ipaex{bv̩˧ʐv̩˧ {\kern2pt}|{\kern2pt} ʝi˧ {\kern2pt}|{\kern2pt} æ̃˩˥ {\kern2pt}|{\kern2pt} so˩-kʰv̩˩˥}\\
	\gll bv̩˧ʐv̩˧	ʝi˥	æ̃˩˧	so˩		kʰv̩˧˥\textsubscript{a}\\
	snake		ox	chicken	three	\textsc{clf}.years\\
	\glt \textit{modified from (\ref{ex:SerpentOxRooster}) by placing each noun in a separate tone group}
\end{exe}

\begin{exe}
	\ex
	\label{ex:DragonApeRatCUT}
	\ipaex{mv̩˧gv̩˧ {\kern2pt}|{\kern2pt} ʑi˩˥ {\kern2pt}|{\kern2pt} hwɤ˧ {\kern2pt}|{\kern2pt} so˩-kʰv̩˩˥}\\
	\gll mv̩˧gv̩˧	ʑi˩˥	hwɤ˧	so˩		kʰv̩˧˥\textsubscript{a}\\
	dragon		ape		rat		three	\textsc{clf}.years\\
	\glt \textit{modified from (\ref{ex:DragonApeRat}) by placing each noun in a separate tone group}
\end{exe}

\begin{exe}
	\ex
	\label{ex:RabbitPigSheepCUT}
	\ipaex{tʰo˧li˧ {\kern2pt}|{\kern2pt} bo˩˥ {\kern2pt}|{\kern2pt} jo˩˥ {\kern2pt}|{\kern2pt} so˩-kʰv̩˩˥}\\
	\gll tʰo˧li˧	bo˩˧		jo˩		so˩		kʰv̩˧˥\textsubscript{a}\\
	rabbit		pig			sheep	three	\textsc{clf}.years\\
	\glt \textit{modified from (\ref{ex:RabbitPigSheep}) by placing each noun in a separate tone group}
\end{exe}

By contrast, in (\ref{ex:TigerHorseDog}), the presence of an odd number of syllables precludes the formation of two tone groups of equal length. As a result, the symmetrical structure observed in (\ref{ex:SerpentOxRooster}), (\ref{ex:DragonApeRat}), and (\ref{ex:RabbitPigSheep}) cannot be achieved. Instead, each animal name is assigned to a separate tone group, yielding a total of four tone groups. This illustrates the influence of \is{rhythm}rhythmic factors in tonal organization. (For an introduction to the notoriously complex domain of linguistic \is{rhythm}rhythm, see \citealt{niebuhr2009b, cummins2012, house2012}.)


% \subsubsection{Tonal integration in proverbs} %%% Removed at 2nd edition to avoid a 4th-level heading
\label{sec:tonalintegrationinproverbs}

Proverbs provide another example of tightly-knit tonal integration. An example is shown in
(\ref{ex:thepoormustnotborrowmoneytheshinbonemustnotreceivewounds}).
\begin{exe}
  \ex
  \label{ex:thepoormustnotborrowmoneytheshinbonemustnotreceivewounds}
  \ipaex{hĩ˧dzɑ˧ {\kern2pt}|{\kern2pt} ɖʐe˧ {\kern2pt}|{\kern2pt} tʰɑ˧-ʝi˥, {\kern2pt}|{\kern2pt} ɻ̩̃˧ko˩ mi˩ tʰɑ˩-tʰv̩˩.}\\
  \gll hĩ˥	dzɑ˥	ɖʐe˧	tʰɑ˧-	ʝi˥	ɻ̩̃˧ko˩	mi˧	tʰɑ˧-	tʰv̩˧\textsubscript{a}\\
  person	poor	money	\textsc{proh}	to\_borrow	shinbone	wound	\textsc{proh}	to\_get\\
  \glt ‘The poor must not borrow money; the shinbone must not receive wounds.’ (Field notes.)
\end{exe}

The proverb warns about vulnerability: one must beware of avoiding hits to fragile spots. The listener is expected to
know that a~blow to the shin is especially painful and, by {analogy}, to imagine how hard it is for
a~poor person to repay a~loan with added interest. The sequence /\ipa{ɻ̩̃˧ko˩ mi˩ tʰɑ˩-tʰv̩˩}/
‘the shinbone must not receive wounds’ forms a single tone group, with the \is{stylistics}stylistic
effect of presenting it as a~self-evident truth~-- an established fact rather than a~statement coined on the
fly by the speaker. If produced as an on-the-spot utterance, the division into tone groups would instead be /\ipa{ɻ̩̃˧ko˩ {\kern2pt}|{\kern2pt} mi˧ tʰɑ˧-tʰv̩˧}/ or /\ipa{ɻ̩̃˧ko˩ {\kern2pt}|{\kern2pt} mi˧ {\kern2pt}|{\kern2pt} tʰɑ˧-tʰv̩˧}/.


Importantly, even in proverbs and set phrases, speakers retain a~latitude of choice in their division of the utterance into tone groups. A comparison of different
versions of the same story by the same speaker reveals numerous instances of such variation. For instance, the saying ‘If you see a~tiger, it means your father is going
to die; if you see a~panther, it means your mother is going to die’, which is at the heart of the
narrative \textit{Tiger}, is divided into four tone groups in (\ref{ex:seetigerdie4}), three in (\ref{ex:seetigerdie3}), and just two in (\ref{ex:seetigerdie}). 

\begin{exe}
	\ex
	\label{ex:seetigerdie4}
	\ipaex{ʐæ˩ do˥ {\kern2pt}|{\kern2pt} ə˧mi˧ ʂɯ˧; {\kern2pt}|{\kern2pt} lɑ˧ do˩, {\kern2pt}|{\kern2pt} ə˧dɑ˧ ʂɯ˧.}\\
	\gll ʐæ˩˥			do˩\textsubscript{b}			ə˧mi˧		ʂɯ˧\textsubscript{a}		lɑ˧		do˩\textsubscript{b}		ə˧dɑ˥\$		ʂɯ˧\textsubscript{a}\\
	panther		to\_see	mother		to\_die		tiger			to\_see					father		to\_die\\
	\glt ‘If you see a~panther, it means your mother is going to die; if you see a~tiger, it means your father is going
	to die.’ (\textit{Tiger.10} \pandoi{0004444\#S10}; \textit{Tiger2.5} \pandoi{0004545\#S5}, \textit{13})
\end{exe}

\begin{exe}
	\ex
	\label{ex:seetigerdie3}
	\ipaex{ʐæ˩ do˥ {\kern2pt}|{\kern2pt} ə˧mi˧ ʂɯ˩, {\kern2pt}|{\kern2pt} lɑ˧ do˩ ə˩dɑ˩ ʂɯ˩.}\\
	\gll ʐæ˩˥			do˩\textsubscript{b}			ə˧mi˧		ʂɯ˧\textsubscript{a}		lɑ˧		do˩\textsubscript{b}		ə˧dɑ˥\$		ʂɯ˧\textsubscript{a}\\
	panther		to\_see	mother		to\_die		tiger			to\_see					father		to\_die\\
	\glt ‘If you see a~panther, it means your mother is going to die; if you see a~tiger, it means your father is going
	to die.’ \textit{(Tiger.50)} \pandoi{0004444\#S50}
\end{exe}

\begin{exe}
	\ex
	\label{ex:seetigerdie}
	\ipaex{lɑ˧ do˩ ə˩dɑ˩ ʂɯ˩, {\kern2pt}|{\kern2pt} ʐæ˩ do˥ ə˩mi˩ ʂɯ˩.}\\
	\gll lɑ˧		do˩\textsubscript{b}		ə˧dɑ˥\$		ʂɯ˧\textsubscript{a}		ʐæ˩˥			do˩\textsubscript{b}			ə˧mi˧		ʂɯ˧\textsubscript{a}\\
	tiger			to\_see					father		to\_die		panther		to\_see	mother		to\_die\\
	\glt ‘If you see a~tiger, it means your father is going
	to die; if you see a~panther, it means your mother is going to die.’ (\textit{Tiger.31}  \pandoi{0004444\#S31}; \textit{Tiger2.111} \pandoi{0004545\#S111})
\end{exe}

As in examples (\ref{ex:therewasnothingtoeatandnothingtodrink})--(\ref{ex:itissaidthatonthatoccasionthewholefamilywillkowtow}) and (\ref{ex:therearrivedonepersonthentwothenthree})--(\ref{ex:youCAME}), the \is{stylistics}stylistic nuance is
that a greater number of tone groups directs attention to the individual
components of the sentence. In \textit{Tiger}, the {phrasing} in (\ref{ex:seetigerdie4}) is found at first occurrence, at the beginning of the story, while later occurrences tend to have fewer tone groups, as in (\ref{ex:seetigerdie3}) and (\ref{ex:seetigerdie}). Example (\ref{ex:seetigerdie3}) is an interesting intermediate case in which the ‘panther/mother’ and ‘tiger/father’ pairs are not treated symmetrically. The context is a~journey through the mountains, where a~mother and daughter encounter a~tiger; the father is not present. This asymmetry serves to foreground the omen concerning the mother. The reference to the father is retained to ensure the proverb remains recognizable but is backgrounded through its integration into a single tone group.\footnote{The order of the two clauses~-- ‘panther/mother + tiger/father’ versus ‘tiger/father + panther/mother’~-- does not directly affect the division into tone groups. Both orders are acceptable: the ordering in  (\ref{ex:seetigerdie4})-(\ref{ex:seetigerdie3}) is not necessarily a~departure from a~canonical form represented by (\ref{ex:seetigerdie}), or vice versa.}

As a~final example, let us examine (\ref{ex:heavensee}).

\begin{exe}
	\ex
	\label{ex:heavensee}
	\ipaex{hĩ˧ ɳɯ˩ mɤ˩-do˩, {\kern2pt}|{\kern2pt} mv̩˧ ɳɯ˩ {\kern2pt}|{\kern2pt} do˩˥!}\\
	\gll hĩ˥	ɳɯ˧	mɤ˧-	do˩\textsubscript{b}		mv̩˥	ɳɯ˧		do˩\textsubscript{b}\\
	person/human	\textsc{a}		\textsc{neg}	to\_see			sky/heavens		\textsc{a}		to\_see\\
	\glt ‘People may not see, but the \textit{heavens} see!’
\end{exe}

This saying serves as a~reminder that other people’s gaze is not the touchstone of good conduct,
and that one’s actions should be guided by the same principles whether seen or unseen. The most common
realization of this saying is (\ref{ex:heavensee}), in which the first part
(‘What people do not see’) is integrated into a single tone group, whereas the second part is divided
into two. This structure places particular \isi{emphasis} on the verb /\ipa{do˩\textsubscript{b}}/ ‘to see, to observe’, which, occurring alone in its tone group, receives a~final H tone and is realized with a~rising contour \mbox{(/LH/)}, following Rule
7: “If a~tone group only contains L tones, a~postlexical H tone is added to its last
syllable”. (Variants are found in the narrative \textit{Reward}, at sentences \textit{28} \pandoi{0004446\#S28}, \textit{36}, \textit{62}, and \textit{114}.)


\subsection{Two cases of resistance to tonal integration}
\label{sec:acaseofresistancetotonalintegration}

Some expressions resist integration into
a single tone group. As noted in~\sectref{sec:theroleofthenumberofsyllables}, tonal changes in compounding occur only when the second element (the
head) has fewer than three syllables. A~typical example is the proper name “Lugu Lake”, shown in example (\ref{ex:lugulake}) of Chapter~\ref{chap:compoundnouns}. 

Similarly, the phrase /\ipa{sɑ˧ {\kern2pt}|{\kern2pt} zo˩bv̩˥ɭɯ˩}/, meaning ‘the universe, the whole world’ (\textit{Mountains.69} \pandoi{0004574\#S69}), is also perceived as composed of two distinct parts,
/\ipa{sɑ˧}/ and /\ipa{zo˩bv̩˥ɭɯ˩}/, even though the first syllable, /\ipa{sɑ˧}/, is no longer
independently intelligible and does not appear on its own. The \is{trisyllables}trisyllable /\ipa{zo˩bv̩˥ɭɯ˩}/ can be used alone as a synonym for the four-syllable expression. The existence of
this \is{trisyllables}{trisyllabic} form may contribute to the resistance of /\ipa{sɑ˧ {\kern2pt}|{\kern2pt} zo˩bv̩˥ɭɯ˩}/ to tonal integration. If such integration were to occur, the expected output would be $\dagger${\kern2pt}\ipa{sɑ˧-zo˩bv̩˩ɭɯ˩} (M.L.L.L), following Rule~5: “All syllables following an H.L or M.L sequence receive L tone”.

By contrast, the same argument cannot be invoked to explain the absence of tonal integration in the Na term for ‘field penny-cress’ (\textit{Thlaspi
  arvense}), a~foetid plant with round, flat pods. The plant's name, /\ipa{ʁv̩˧=bv̩˥ {\kern2pt}|{\kern2pt} v̩˩tsʰɤ˩˥}/, literally ‘the crane’s vegetable’, would be expected to form a~single tone group, as it constitutes a lexical item. However, its transparent structure as a~\isi{possessive} construction probably contributes to slowing down its phonological integration. Moreover, this plant is not commonly used, and the word's low frequency in discourse reduces the pressure towards phonetic-phonological \isi{simplification}.


\subsection{Illustration: Sample derivations}
\label{sec:samplederivation}

This section recapitulates some of the mechanisms described above by presenting sample \is{derivation!tonal}derivations for two sentences from transcribed texts. 

\begin{exe}
	\ex
	\label{ex:ahornofwater}
	\ipaex{dʑɯ˧ {\kern2pt}|{\kern2pt} ɖɯ˧-qʰv̩˧tʰv̩˧ tʰi˩-kʰɯ˩.}\\
	\gll dʑɯ˩	ɖɯ˧-qʰv̩˧tʰv̩\#˥	tʰi˧-	kʰɯ˧˥\\
	water	one-\textsc{clf}.hornful	\textsc{dur}	to\_put\\
	\glt ‘[People who travel all day] put a~hornful of water [in their bag, so as to have something to
	drink].’ \textit{(Tiger2.51)} \pandoi{0004545\#S51}
\end{exe}

In example (\ref{ex:ahornofwater}), the noun ‘water’ forms a~tone group on its own and is realized with M tone, following the regular pattern shown in \tabref{tab:thelexicaltonesofmonosyllabicnouns}. The second tone group
comprises five syllables: /\ipa{ɖɯ˧-qʰv̩˧tʰv̩˧ tʰi˩-kʰɯ˩}/. 

First, the tone of the
\is{numerals}numeral-plus-classifier phrase is determined via the table-lookup rules set out in Chapter~\ref{chap:classifiers} (\tabref{tab:hornfuls}), yielding //\ipa{ɖɯ˧-qʰv̩˧tʰv̩\#˥}// ‘one hornful’. The combination of this noun phrase with the prefixed verb //\ipa{tʰi˧-kʰɯ˧˥}// ‘to put (into)’ also follows a table-lookup process (the tone patterns for object-plus-verb phrases are set out in Chapter~\ref{chap:verbsandtheircombinatoryproperties}). The resulting surface tone sequence is M.M.M.L.L. Thus, knowledge of a~tone group's internal morphosyntactic structure, along with the lexical tones of its components, suffices to arrive at the surface tone sequence.

%%subsubsec:8-2-7-2
%\subsubsection{Movement and unfolding of the MH tone}
%\label{sec:movementandunfoldingofthemhtone}
%
%Example (\ref{ex:itissaidthatshespatouttheegg}) contains two tone groups. 
%
%\begin{exe}
%  \ex
%  \label{ex:itissaidthatshespatouttheegg}
%  \ipaex{æ̃˩ʁv̩˩˥ {\kern2pt}|{\kern2pt} gɤ˩-pʰi˧ le˧-tsʰɯ˧-tsɯ˥}\\
%  \gll æ̃˩ʁv̩˩	gɤ˩-	pʰi˧˥	le˧-	tsʰɯ˩	tsɯ˧˥\\
%  egg	upward	to\_spit	\textsc{accomp}	to\_come	\textsc{rep}\\
%  \glt  ‘It is said that [she] spat out the egg!’ (BuriedAlive2.143)
%\end{exe}
%
%The derivation of the second of these tone groups, /\ipa{gɤ˩-pʰi˧ le˧-tsʰɯ˧-tsɯ˥}/, can be presented as follows: 
%\begin{enumerate}
%	\item[(i)] Lexical tones: /\ipa{gɤ˩- pʰi˧˥ le˧- tsʰɯ˩ -tsɯ˧˥}/
%	\item[(ii)] First level of grouping: /\ipa{{gɤ˩-pʰi˧˥} {le˧-tsʰɯ˩} -tsɯ˧˥}/
%	\item[(iii)] Left-to-right computation: the MH pattern of /\ipa{pʰi˧˥}/ spreads rightwards, up to the syllable /\ipa{tsʰɯ˩}/: /\ipa{{gɤ˩-pʰi˧ le˧-tsʰɯ˧˥} -tsɯ˧˥}/
%	\item[(iv)]	Unfolding of the MH \is{tonal contour}contour, overriding (replacing) the lexical tone of the final particle: /\ipa{gɤ˩-pʰi˧ le˧-tsʰɯ˧-tsɯ˥}/
%\end{enumerate}
%
%
As a~second sample \is{derivation!tonal}derivation, let us consider a~slightly more complex example, shown in (\ref{ex:whenthebigbrothercamebacktheyoungersisterdidntrecognizehim}).

%\subsubsection{Derivation of an entire sentence: Sister.49}
\label{sec:anentiresentencesister49}

\begin{exe}
  \ex
  \label{ex:whenthebigbrothercamebacktheyoungersisterdidntrecognizehim}
  \ipaex{ə˧mv̩˧˥ {\kern2pt}|{\kern2pt} le˧-tsʰɯ˩ {\kern2pt}|{\kern2pt} tʰi˩˥, {\kern2pt}|{\kern2pt} go˧mi˧ ɳɯ˧ {\kern2pt}|{\kern2pt} ə˧mv̩˧˥ {\kern2pt}|{\kern2pt} mɤ˧-sɯ˥ tsɯ˩ {\kern2pt}|{\kern2pt} mv̩˩!}\\
  \gll ə˧mv̩˧˥		le˧-		tsʰɯ˩\textsubscript{a}	tʰi˩˥		go˧mi˧ ɳɯ˧	ə˧mv̩˧˥		mɤ˧-	sɯ˥		tsɯ˧˥	mv̩˧\\
  elder\_sibling	\textsc{accomp}	to\_come.\textsc{pst}	gap\_filler:well	younger\_sister
  \textsc{a}/\textsc{top}
  elder\_sibling	\textsc{neg}	to\_know	\textsc{rep}	\textsc{affirm}\\
  \glt ‘[When] the big brother came back, the younger sister didn’t recognize him!’ \textit{(Sister.49)} \pandoi{0004341\#S49}
\end{exe}

In (\ref{ex:whenthebigbrothercamebacktheyoungersisterdidntrecognizehim}), the group \ipa{{\kern2pt}|{\kern2pt} ə˧mv̩˧˥
{\kern2pt}|{\kern2pt}} ‘elder sibling’ simply consists of a~noun, so that the \isi{tonal word}
and tone group coincide. Since there are no suffixes or final particles, the noun's lexical MH tone is
realized on the last syllable of the \isi{tonal word}, which is also the last syllable of the tone
group.
%In the group {\kern2pt}|{\kern2pt} ə˧mv̩˧-ɳɯ˥ {\kern2pt}|{\kern2pt} ‘by the elder sibling’, the rising \is{tonal contour}contour (MH) of /\ipa{ə˧mv̩˧˥}/ ‘elder sibling’ projects its H part onto the \is{suffixes}suffix.

The tone pattern of \ipa{{\kern2pt}|{\kern2pt} le˧-tsʰɯ˩ {\kern2pt}|{\kern2pt}} ‘came back’ obtains as described in \tabref{tab:accomplishedpfvcompletion}.
The word /\ipa{tʰi˩˥}/ ‘(and) then, (and) so’ always forms an independent tone group, as discussed in \sectref{sec:someelementsalwaysconstituteatonegroupontheirown}.
The tone pattern of \ipa{{\kern2pt}|{\kern2pt}go˧mi˧ ɳɯ˧{\kern2pt}|{\kern2pt}} ‘by the sister’ follows the pattern set out in \tabref{tab:topicfull}.

In \ipa{{\kern2pt}|{\kern2pt} mɤ˧-sɯ˥ tsɯ˩ {\kern2pt}|{\kern2pt}}, the verb ‘to know, to recognize’,
//\ipa{sɯ˥}//, is flanked by the {negation} \is{prefixes}prefix //\ipa{mɤ˧-}//
and the sentence-final particle //\ipa{tsɯ˧˥}// ({reported speech}). The {negation} \is{prefixes}prefix surfaces with its lexical M tone and the verb retains its H tone (in keeping with the general pattern documented in \tabref{tab:thelexicaltonesofverbs}), and the
sentence-final particle, being preceded by an H tone, receives L through Rule~4.

The final particle //\ipa{mv̩˧}// ({affirmative}) was already encountered above in various examples in which it appears after a~tone-group boundary and carries L tone: \mbox{/\ipa{{\kern2pt}|{\kern2pt} mv̩˩}/.} It does not constitute a~well-formed tone group, since a~tone group cannot contain L tones only. The special status of this particle, analyzed here as \textit{extrametrical}, brings us to the topic of breaches of tonal grouping, which introduce extrametrical syllables into the system.


\section[Cases of breach of tonal grouping and their consequences]{Cases of breach of tonal grouping and their consequences for the system}
\label{sec:casesofbreachoftonalgroupingandconsequencesforthesystem}

\is{extrametricality|textbf}

A breach of tonal grouping occurs when a~non-final syllable comes to carry a~\is{tonal contour}contour. This causes the following syllables to become extrametrical, with repercussions for the tonal system as a~whole. Considerations of analytical consistency lead to positing extrametrical syllables even in some contexts where they are not preceded by a~tonal \is{tonal contour}contour. This is the case, for instance, of the affirmative final particle /\ipa{{\kern2pt}|{\kern2pt}mv̩˩}/ in example (\ref{ex:whenthebigbrothercamebacktheyoungersisterdidntrecognizehim}).

\subsection[Non-final contours as a~stylistic option]{The stylistic option of realizing a~contour on a~word in non-final position }
\label{sec:thestylisticoptionofrealizingacontouronawordinnonfinalposition}

Syllables that are not in final position within a tone group cannot carry a~\is{tonal contour}contour (\mbox{//MH//}, \mbox{//LM//} or \mbox{//LH//}). This restriction is a~core property of the tone group as a~phonological unit. But this rule is at odds with a~\is{stylistics}stylistic
device whereby a~word is \is{emphasis}emphasized by ending the tone group immediately after that word. This
device interrupts tonal computation, allowing the realization of a~\is{tonal contour}contour on the emphasized word,
as illustrated in (\ref{ex:byrowingrowingrowingtheyescapedtheymanagedtoescape}).
\begin{exe}
  \ex
  \label{ex:byrowingrowingrowingtheyescapedtheymanagedtoescape}
  \ipaex{le˧-tsɑ˧˥, {\kern2pt}|{\kern2pt} le˧-tsɑ˧˥, {\kern2pt}|{\kern2pt} le˧-tsɑ˧˥ {\kern2pt}|{\kern2pt} -kwɤ˩tɕɯ˩, {\kern2pt}|{\kern2pt} le˧-lv̩˧˥!}\\
  \gll le˧-		tsɑ˧˥		-kwɤ˧tɕɯ˥	le˧-		lv̩˧˥\\
  \textsc{accomp}	to\_row		because	\textsc{accomp}	to\_escape\\
  \glt ‘By rowing, rowing, rowing, they managed to escape!’ \textit{(Lake3.59)} \pandoi{0004348\#S59}
\end{exe}

This example occurs in a~highly emotional context: a~mother and her daughter are rowing for their lives, struggling against the flood that has suddenly transformed their homeland into a~vast lake. The verb ‘to row’ is repeated, and the sentence is chopped into short tone groups, as if mimicking the oars chopping into the water in rapid, successive strokes. Phonetically, the
verb is articulated strongly, each time bearing its lexical rising tone. The \is{conjunctions}conjunction
/\ipa{-kwɤ˩tɕɯ˩}/ is tacked on at the end as if it were an afterthought. 
%(The hyphen after the tone group
%\is{boundary (between tone groups)}boundary (/\ipa{{\dots} {\kern2pt}|{\kern2pt} -kwɤ˩tɕɯ˩}/) serves as an indication that the syllables at issue are
%extrametrical, and do not constitute a~full-fledged tone group on their own.) 
A~more structured alternative would be to integrate the \is{conjunctions}conjunction with the preceding tone group, as in (\ref{ex:byrowingMODIFIED}). This would yield the standard tonal pattern, with the MH \is{tonal contour}contour unfolding over the verb and the first syllable of the \is{conjunctions}conjunction. But this
deliberate, neatly structured \is{variants}variant would be stylistically inappropriate in this context.

\begin{exe}
	\ex
	\label{ex:byrowingMODIFIED}
	\ipaex{le˧-tsɑ˧˥, {\kern2pt}|{\kern2pt} le˧-tsɑ˧˥, {\kern2pt}|{\kern2pt} le˧-tsɑ˧-kwɤ˥tɕɯ˩, {\kern2pt}|{\kern2pt} le˧-lv̩˧˥!}\\
	\gll le˧-		tsɑ˧˥		-kwɤ˧tɕɯ˥	le˧-		lv̩˧˥\\
	\textsc{accomp}	to\_row		because	\textsc{accomp}	to\_escape\\
	\glt ‘By rowing, rowing, rowing, they managed to escape!’ \textit{Modified from} (\ref{ex:byrowingrowingrowingtheyescapedtheymanagedtoescape})
\end{exe}

An example using the same \is{conjunctions}conjunction as in (\ref{ex:byrowingrowingrowingtheyescapedtheymanagedtoescape}), but where the expected division into tone groups is
respected and the expected process of unfolding of a~\is{tonal contour}contour tone takes place, is found in (\ref{ex:buuurp}).
\begin{exe}
  \ex
  \label{ex:buuurp} 
  \ipaex{lo˩dʑo˥ {\kern2pt}|{\kern2pt} ʈʂʰɯ˧ne˧-ʝi˥ {\kern2pt}|{\kern2pt} mv̩˩tɕo˧ pʰv̩˧-kwɤ˥tɕɯ˩-ɳɯ˩, {\kern2pt}|{\kern2pt} ``qʰʰʰ{\dots}ə!” {\kern2pt}|{\kern2pt} pi˧ tsɯ˩ {\kern2pt}|{\kern2pt} mv̩˩.~{\kern2pt}|{\kern2pt}}\\
  \gll lo˩dʑo˥	ʈʂʰɯ˧ne˧-ʝi˥	mv̩˩tɕo˧	pʰv̩˧˥	-kwɤ˧tɕɯ˥	-ɳɯ qʰʰʰ{\dots}ə!			pi˥	tsɯ˧˥	mv̩˧\\
  bracelet	thus	downward	to\_take\_off	because		\textsc{top} onomatopoeia:burp!		to\_say
  \textsc{rep}	\textsc{affirm}\\
  \glt ‘When [the man] took off [the buried woman’s bracelets], like this, [the corpse made
    a~gurgling sound]: Buuurp!’ \textit{(BuriedAlive2.48)} \pandoi{0004536\#S48}
\end{exe}

This is the only example found so far where the H part of a~verb’s MH \is{tonal contour}contour reassociates to the
\is{conjunctions}conjunction //\ipa{-kwɤ˧tɕɯ˥}//, in contrast to numerous examples where this \is{tonal contour}contour surfaces as such on the
verb preceding this \is{conjunctions}conjunction (for instance in \textit{Dog.49} \pandoi{0004442\#S49}, \textit{Tiger.46} \pandoi{0004444\#S46}, \textit{BuriedAlive3.65} \pandoi{0004538\#S65}, \textit{Caravans.80} \pandoi{0004530\#S80}, \textit{Sister.50} \pandoi{0004341\#S50},
\textit{Sister3.133} {\linebreak}\pandoi{0004344\#S133}, \textit{Seeds2.34} \pandoi{0004542\#S34}, \textit{Renaming.18} \pandoi{0004534\#S18}, and \textit{Fu\-ner\-al.51} \pandoi{0004571\#S51}). Example (\ref{ex:buuurp}) provides just enough evidence
to show that realization with \is{tonal contour}contour \is{tone spreading}spreading is possible. Contour unfolding may once have
been the norm, while realization with a~\is{tonal contour}contour on the verb may originally have constituted a~conspicuous \is{stylistics}stylistic effect. Be that as it may, the latter is currently much more common than the former, to the point
that realization with \is{tonal contour}contour unfolding is now a~stylistically marked option.

Realizations of contours in non-final position are not uncommon,
%(e.g.~Housebuilding.71 , 98, 100), %%% La référence ne paraît pas bonne
each instance carrying a~distinct \is{stylistics}stylistic twist. For instance, in (\ref{ex:bite}), avoidance of \is{tonal contour}contour unfolding has an \is{emphasis}emphatic effect, making the statement more eloquent by reinforcing the sense of {certitude} conveyed by the particle /\ipa{-ɲi˩}/. Contour unfolding (/\ipa{le˧-ʈʰæ˧-ɲi˥}/) would be less forceful in this context. 

\begin{exe}
	\ex
	\label{ex:bite} 
	\ipaex{“mɤ˧-dʑɤ˩ ɲi˩, {\kern2pt}|{\kern2pt} ə˧-sɯ˥! {\kern2pt}|{\kern2pt} ə˧ʝi˧-ʂɯ˥ʝi˩, {\kern2pt}|{\kern2pt} ə˧mi˧-mv̩˩ ɲi˩-kv̩˩, {\kern2pt}|{\kern2pt} zo˩no˥, {\kern2pt}|{\kern2pt} ɬi˧di˩-di˩mi˩ qo˩ dzi˩, {\kern2pt}|{\kern2pt} le˧-wo˥ {\kern2pt}|{\kern2pt} le˧-hɯ˩-zo˩, {\kern2pt}|{\kern2pt} ə˧ʑi˧-ə˧pʰv̩˧-ki˥ {\kern2pt}|{\kern2pt} le˧-hɯ˩-dʑo˩, {\kern2pt}|{\kern2pt} ʈʂʰɯ˧ne˧-ʝi˥, {\kern2pt}|{\kern2pt} ʐæ˩ ɳɯ˥ {\kern2pt}|{\kern2pt} le˧-ʈʰæ˧˥ {\kern2pt}|{\kern2pt} lɑ˧ ɳɯ˧ {\kern2pt}|{\kern2pt} \textbf{le˧-ʈʰæ˧˥ {\kern2pt}|{\kern2pt} -ɲi˩}” {\kern2pt}|{\kern2pt} pi˧-zo˩!}\\
	\gll mɤ˧-	dʑɤ˩\textsubscript{b}	ɲi˩		ə˩-		sɯ˥		ə˧ʝi˧-ʂɯ˥ʝi˩									ə˧mi˧		mv̩˩˥	ɲi˧-kv̩˧˥		zo˩no˥		ɬi˧di˩-di˩mi˩		qo˧		dzi˩\textsubscript{a}		le˧-wo˥le˧-hɯ˩		-zo		ə˧ʑi˧˥		ə˧pʰv̩˧		-ki˧	le˧-		hɯ˧\textsubscript{c}		-dʑo˥			ʈʂʰɯ˧ne˧-ʝi˥	ʐæ˩˥		ɳɯ˧	le˧-		ʈʰæ˧˥		lɑ˧		ɳɯ˧		\textbf{le˧-}	\textbf{ʈʰæ˧˥}		\textbf{-ɲi˩}		pi˥		-zo\\
	\textsc{neg}	good	\textsc{cop}	\textsc{interrog}	to\_know		once\_upon\_a\_time		mother		daughter	two-\textsc{clf}.persons	well	Yongning\_plain		inside		to\_dwell		went\_back		\textsc{advb}	grandmother		grandmother’s\_brother		 \textsc{all}		\textsc{accomp}		to\_go.\textsc{pst}		\textsc{top}	thus		panther	\textsc{a}	\textsc{accomp}			to\_bite 		tiger		\textsc{a}	\textsc{accomp}			\textbf{to\_bite}		\textbf{\textsc{certitude}}		to\_say		\textsc{advb}\\
	\glt ‘[Elders would tell stories about the tiger eating people,] saying: “That is very bad, you know! Once upon a time, a~mother and her daughter who lived in the Yongning plain went to see the grandmother and her brothers; and they were actually attacked by the big cats!”’ \textit{Literally:} ‘they were bitten by the panther, by the tiger’ \textit{(Tiger.51)} \pandoi{0004444\#S51}
\end{exe}


A~second \is{stylistics}stylistic consequence of the absence of \is{tonal contour}contour unfolding in (\ref{ex:bite}) is the symmetry between the clauses /\ipa{ʐæ˩ ɳɯ˥ | le˧-ʈʰæ˧˥}/ ‘(a/the) panther bit’ and /\ipa{lɑ˧ ɳɯ˧ | le˧-ʈʰæ˧˥}/ ‘(a/the) tiger bit’, which would be diminished if the tone patterns differed (/\ipa{le˧-ʈʰæ˧˥}/ in the first case, and /\ipa{le˧-ʈʰæ˧-ɲi˥}/ in the second). This symmetry is crucial in linking the story to the Na saying ‘If you see a~tiger, it means your father is going to die; if you see a~panther, it means your mother is going to die’ (example (\ref{ex:seetigerdie}) above). The story is about a~tiger killing a young woman's mother, so there is an apparent mismatch with the saying (which associates \textit{panther} with \textit{mother} and \textit{tiger} with \textit{father}). Mention of both a~panther and a~tiger in (\ref{ex:bite}) is not to be taken literally, as there is no panther in the story. Instead, this parallel mention clarifies that the seeming contrast between \textit{panther} and \textit{tiger} is irrelevant, guiding towards a~broader interpretation: that encountering a~big cat is an ill omen, foreboding the death of a~parent. Example (\ref{ex:bite}) thus provides a clear hint that the panther and tiger in the saying are no more distinct from each other than \textit{cats} and \textit{dogs} in the {English} expression \textit{it's raining cats and dogs}.

 
Another construction for which the set of narratives contains examples without the unfolding of
a~rising \is{tonal contour}contour is /\ipa{ʈʂʰɯ˧ne˧ gv̩˧˥}/, which combines the adverb ‘thus, in this way’ with the verb ‘to
take place, to occur’, as illustrated in (\ref{ex:mygrandmotherknewabouteverything}).
\begin{exe}
  \ex
  \label{ex:mygrandmotherknewabouteverything}
  \ipaex{ʈʂʰɯ˧-ʑi˧˥ {\kern2pt}|{\kern2pt} -dʑo˩, {\kern2pt}|{\kern2pt} ʈʂʰɯ˧ne˧ gv̩˧˥ {\kern2pt}|{\kern2pt}
  -ɲi˩! {\kern2pt}|{\kern2pt} tʰv̩˧-ʑi˧˥ {\kern2pt}|{\kern2pt} -dʑo˩, {\kern2pt}|{\kern2pt} ʈʂʰɯ˧ne˧ gv̩˧˥ {\kern2pt}|{\kern2pt} -ɲi˩!}\\
  \gll ʈʂʰɯ˧-ʑi˧˥	-dʑo˥	ʈʂʰɯ˧ne˧	gv̩˧\textsubscript{c}		-ɲi˩		tʰv̩˧-ʑi˧˥\\
  \textsc{dem.prox}-\textsc{clf}.households		\textsc{top}	thus to\_take\_place		\textsc{certitude}	 \textsc{dem.dist}-\textsc{clf}.households\\
  \glt ‘This is what happened to this household! And this is what happened to that household!' \textit{Context:} the narrator reports how her grandmother used to teach children the proper way to behave by drawing on real-life examples of events that had occurred in the neighbourhood. \textit{(Elders3.44)} \pandoi{0004532\#S44} 
\end{exe}

In this example, a~tone-group \is{boundary (between tone groups)}boundary follows both occurrences of the phrase /\ipa{ʈʂʰɯ˧ne˧
  gv̩˧˥}/, setting it into relief. The following morpheme also stands out by not being incorporated within the same tone group. This morpheme, /\ipa{-ɲi˩}/, is a~\is{grammaticalization}grammaticalized form of the \isi{copula}, used to convey “an epistemic strategy that marks a~high degree of certitude” \citep[497]{lidz2010}. The context clarifies what happens here: this passage is in \isi{reported speech}, where the narrator adopts the tone of voice of her mother's mother, whom she regards as the highest authority on traditional lore. 

\begin{quotation}
	My grandmother knew about everything! In the old days, she would also tell us stories about people in the village and what we
	must learn from them: “This household, this is what happened to them! That household, this is
	what happened to them! One must develop habits of doing
	good! One mustn’t do wrong!” (\textit{Elders3.44-45} \pandoi{0004533\#S44}) 
\end{quotation}

The \is{stylistics}stylistic choice of inserting a~tone-group \is{boundary (between tone groups)}boundary before the morpheme indicating {certitude} conveys the assertiveness of the character to whom this passage in \isi{reported speech} is assigned. Realization as /\ipa{ʈʂʰɯ˧ne˧ gv̩˧˥ {\kern2pt}|{\kern2pt} -ɲi˩}/ rather than /\ipa{ʈʂʰɯ˧ne˧ gv̩˧-ɲi˥}/
conveys an authoritative stance. On the other hand, when the phrase ‘This is how it happened’ is used more casually, as
an introduction to a~narrative, it forms a single tone group, as seen in (\ref{ex:howithappened}). A~related {phrasing}, which likewise constitutes a~set phrase and is therefore integrated into
one tone group, is presented in (\ref{ex:howitwouldhappen}).

\begin{exe}
	\ex
	\label{ex:howithappened}
	\ipaex{ʈʂʰɯ˧ne˧ gv̩˧-ɲi˥ tsɯ˩ mv̩˩.}\\
	\gll ʈʂʰɯ˧ne˧	gv̩˧\textsubscript{c}		-ɲi˩		tsɯ˧˥	mv̩˧\\
	thus		to\_take\_place		\textsc{certitude}	 \textsc{rep}		\textsc{affirm}\\
	\glt ‘This is how it happened.’ \textit{(BuriedAlive2.1)} \pandoi{0004536\#S1}
\end{exe}

\begin{exe}
	\ex
	\label{ex:howitwouldhappen}
	\ipaex{ʈʂʰɯ˧ne˧ gv̩˧-kv̩˥.}\\
	\gll ʈʂʰɯ˧ne˧	gv̩˧\textsubscript{c}		-kv̩˧˥\\
	thus		to\_take\_place		\textsc{abilitive}\\
	\glt ‘This is how it would happen.’ (\textit{Sister3.149} \pandoi{0004344\#S149}, \textit{Tiger.2} \pandoi{0004444\#S2})
\end{exe}

Example (\ref{ex:afterherushedawayhisyoungerdaughtercriedhereyesout}) illustrates that the \is{stylistics}stylistic device whereby a~tone group is cut short after a~certain word can be applied as early as the first syllable of a~sentence. A more strongly integrated formulation would be /\ipa{pʰo˩-hɯ˩-kwɤ˩tɕɯ˥-lɑ˩}/, without any special
\isi{emphasis} on the verb ‘to flee/to rush’.

\begin{exe}
	\ex
	\label{ex:afterherushedawayhisyoungerdaughtercriedhereyesout}
	\ipaex{pʰo˩˥ {\kern2pt}|{\kern2pt} hɯ˧-kwɤ˧tɕɯ˥-lɑ˩ {\kern2pt}|{\kern2pt} tʰi˩˥, {\kern2pt}|{\kern2pt} go˧mi˧ ʈʂʰɯ˧-v̩˧-dʑo˩, {\kern2pt}|{\kern2pt} le˧-ŋv̩˩, {\kern2pt}|{\kern2pt} le˧-ŋv̩˩, {\kern2pt}|{\kern2pt} le˧-ŋv̩˩, {\kern2pt}|{\kern2pt}
		le˧-ŋv̩˩, {\kern2pt}|{\kern2pt} le˧-ŋv̩˩-zo˩!}\\
	\gll pʰo˩\textsubscript{a}		hɯ˧\textsubscript{c}		-kwɤ˧tɕɯ˥-lɑ˩		tʰi˩˥ go˧mi˧	ʈʂʰɯ˥	v̩˧	-dʑo˥	le˧-	ŋv̩˩	-zo\\
	to\_flee/to\_rush	to\_go.\textsc{pst}	after		then younger\_sister
	\textsc{dem}.\textsc{prox}
	\textsc{clf}	\textsc{top}	\textsc{accomp}	to\_cry	\textsc{advb}\\
	\glt ‘After he rushed away, [his] younger daughter cried her eyes out!’ \textit{(Sister3.68)} \pandoi{0004344\#S68}
\end{exe}


\subsection[The emergence of extrametrical syllables]{Consequences for the tone system: The emergence of extrametrical syllables}
\label{sec:consequencesforthetonesystemtheemergenceofextrametricalsyllables}

\is{extrametricality}

The phenomenon whereby a~tone group is cut short after a~certain word (noun or verb) has implications for the overall architecture of the tone system. In cases where the portion of the tone group that is cut off can stand on its own as a~tone group, the tenets of the system
remain unaffected, as illustrated in example (\ref{ex:thechinesehaneatdogmeat}).

\begin{exe}
  \ex
  \label{ex:thechinesehaneatdogmeat}
  \ipaex{hæ˧, {\kern2pt}|{\kern2pt} kʰv̩˩mi˩-ʂe˩˥ {\kern2pt}|{\kern2pt} dzɯ˧-kv̩˩!}\\
  \gll hæ˧		kʰv̩˩mi˩-ʂe˩	dzɯ˥		-kv̩˧˥\\
  Chinese	dog\_meat	to\_eat		\textsc{abilitive}\\
  \glt ‘The Chinese (Han) eat dog meat!’ (Field notes, 2012.)
\end{exe}

Example (\ref{ex:thechinesehaneatdogmeat}) lays \isi{emphasis} on ‘dog meat’. In the Na world view, dogs
and men are close friends: according to myth, the dog agreed to exchange its original sixty-year lifespan for the thirteen-year
lifespan initially granted to man (see the narrative \textit{Dog}). Eating dog meat is
therefore taboo among the Na, and the realization that some other ethnic groups do consume it comes as a~shock. An unmarked {phrasing} of (\ref{ex:thechinesehaneatdogmeat}) would be /\ipa{hæ˧ {\kern2pt}|{\kern2pt}
  kʰv̩˩mi˩-ʂe˩ dzɯ˩-kv̩˥}/, in which a~single tone group spans both the object and the verb, allowing for regular tonal
computation.

By contrast, when particles or conjunctions are left stranded, as in
(\ref{ex:byrowingrowingrowingtheyescapedtheymanagedtoescape}), they do not constitute a~well-formed tone group
on their own. The rules recapitulated in \sectref{sec:alistoftonerules}, such as the addition of a~final H tone to all-L
sequences, do not apply to them~-- otherwise, one would expect a~final rising \is{tonal contour}contour in (\ref{ex:byrowingrowingrowingtheyescapedtheymanagedtoescape}):
$\dagger${\kern2pt}\ipa{le˧-tsɑ˧˥ {\kern2pt}|{\kern2pt} -kwɤ˩tɕɯ˩˥}. These stranded syllables are neither assigned a~final H tone nor integrated into the following
tone group.

Several analytical options are open here. One possibility is to consider that, at some
phonological level, the division into tone groups remains unchanged. This would imply that
a~\is{tonal contour}contour can be realized in a~non-final position within a~tone group~-- an implication which contradicts headlong the definition of the tone group adopted in the present analysis. A~preferred option is to consider that the \isi{emphasis} laid on a~word,
and the consequent realization of a~\is{tonal contour}contour on that word, modifies the division of the utterance into
tone groups, leaving certain syllables strand\-ed. These syllables acquire \is{extrametricality}extrametrical status. The notion of
\isi{extrametricality} rescues the general rule that serves as one of the key criteria for defining the tone group as a~phonological unit, namely that tonal contours appear only in
tone-group-final position. 

Consider example (\ref{ex:pluckbutton}):
\begin{exe}
  \ex
  \label{ex:pluckbutton}
  \ipaex{pv̩˩ɭɯ˥ {\kern2pt}|{\kern2pt} ɖʐɤ˧˥ {\kern2pt}|{\kern2pt} ki˩ tsɯ˩ {\kern2pt}|{\kern2pt} mv̩˩.}\\
  \gll pv̩˩ɭɯ˥	ɖʐɤ˧˥		ki˧		tsɯ˧˥	mv̩˧\\
  button	to\_pluck	to\_give		\textsc{rep}	\textsc{affirm}\\
  \glt ‘It is said that [he] plucked a~[button from his jacket] and gave it [to the child].~/ He plucked a~button and gave it [to the child].’ \textit{(Renaming.23)} \pandoi{0004534\#S23}
\end{exe}

At least three \is{stylistics}stylistic options are open here. The most tightly-knit configuration involves a~single tone group:
/\ipa{ɖʐɤ˧ ki˥ tsɯ˩}/.\footnote{For the sake of simplicity, the {affirmative} final particle is omitted from the present analysis; it will be discussed in \sectref{sec:furtherexamplesofextrametricalelements}.} The most analytic alternative consists of two full-fledged tone groups: /\ipa{ɖʐɤ˧˥
  {\kern2pt}|{\kern2pt} ki˧ tsɯ˧˥}/. The third possibility, found in (\ref{ex:pluckbutton}), is intermediate between the previous two: the verb /\ipa{ɖʐɤ˧˥}/ ‘to pluck’ is realized with its lexical MH
\is{tonal contour}contour, as if it were tone-group-final, while the following syllables are all lowered to L, as if
they still belonged to the preceding tone group. The syllables /\ipa{ki˩ tsɯ˩}/  are \is{extrametricality}extrametrical: they do not constitute a~full-fledged tone group on their own.

This range of \is{stylistics}stylistic \isi{variation} is a~salient characteristic of Yongning Na. 
% \Hack{\break} xyz Is this still necessary?
Among other potential
consequences for the evolution of the tone system, \is{extrametricality}extrametrical syllables at the end of a~tone
group may, in some cases, come to be reanalyzed as part of 
%tend to become affiliated to 
the following tone group~-- provided that the sequence of (surface)\is{form!surface} tones allows for such reinterpretation. A~case in point is the highly frequent sequence consisting of the topic marker //\ipa{-dʑo˥}// and the discourse marker //\ipa{tʰi˩˥}// ‘so, then’. The latter systematically forms an independent tone group, as noted in \sectref{sec:someelementsalwaysconstituteatonegroupontheirown}. However, in the narratives
recorded by Mrs.\ Latami (consultant F4), no pause precedes it, whereas there tends to be a~pause before the topic marker. As a result, the two morphemes are pronounced
in rapid succession. 

This situation creates a~discrepancy between two levels: the division into tone
groups, on the one hand, and linguistic \is{rhythm}rhythm, on the other. Now, the topic marker //\ipa{-dʑo˥}// most often surfaces as /\ipa{-dʑo˩}/, due to the presence of an H tone earlier in the tone group, and the sequence /\ipa{-dʑo˩}/ + /\ipa{tʰi˩˥}/ would constitute a~well-formed tone group. The tone sequence L.LH can be the surface realization of either underlying //L.LH// or underlying //L//. One may speculate that the
high discourse frequency of the sequence /\ipa{-dʑo˩ tʰi˩˥}/, which on the phonological surface resembles a~tightly-integrated tone group, favours its reinterpretation as a single tone group. Such a~reinterpretation appears particularly likely in contexts such as (\ref{ex:actual fact}).

\begin{exe}
	\ex
	\label{ex:actual fact}
	\ipaex{{\dots} gɯ˩-ʝi˥ {\kern2pt}|{\kern2pt}
		-dʑo˩ {\kern2pt}|{\kern2pt} tʰi˩˥ {\dots}}\\
	\gll gɯ˩-ʝi˥	-dʑo˥		tʰi˩˥\\
	really		\textsc{top}		then\\
	\glt ‘{\dots} in actual fact, {\dots}’ \textit{(Mountains.58)} \pandoi{0004573\#S58}
\end{exe}

As explained in Appendix A (\sectref{sec:articulatoryreductionreducedformsandtheirlexicalization}), the expression /\ipa{gɯ˩-ʝi˥}/ ‘really, truly’ (from /\ipa{gɯ˩}/ ‘authentic, true’) is well on its way towards reduction to a~\is{monosyllables}monosyllable: to my ears, it sounds like [\ipa{gi˩˥}] except when hyperarticulated. In (\ref{ex:actual fact}), for instance, /\ipa{gɯ˩-ʝi˥}/ sounds very much like a~\is{monosyllables}monosyllable carrying a~rising \is{tonal contour}contour: [\ipa{gi˩˥}]. Now, a~rising \is{tonal contour}contour signals the end of a~tone group. While the topic marker that follows, /\ipa{-dʑo˩}/, does not constitute a~well-formed tone group, the sequence /\ipa{-dʑo˩ tʰi˩˥}/ does. To labour the point: although in data from speaker F4 the division into tone groups is clearly /\ipa{{\dots}-dʑo˩ {\kern2pt}|{\kern2pt} tʰi˩˥} {\kern2pt}|{\kern2pt}/, this sequence could easily be interpreted by a~\is{language acquisition}language learner as an L-tone group: /\ipa{{\kern2pt}|{\kern2pt} -dʑo˩ tʰi˩˥ {\kern2pt}|{\kern2pt}}/.

 
\subsection{Further examples of extrametrical elements}
\label{sec:furtherexamplesofextrametricalelements}

Additional language-internal evidence for resorting to the concept of
\isi{extrametricality} in describing the Na tone system comes from the {affirmative} particle //\ipa{-mv̩˧}// and the expression /\ipa{ə˩-gi˩}/ ‘isn’t it!’, ‘right!’

The
{affirmative} particle //\ipa{-mv̩˧}// cannot host an H tone from the preceding
{reported-speech} particle //\ipa{tsɯ˧˥}//: the sequence is realized as /\ipa{tsɯ˧˥ mv̩˩}/, not
\ipa{$\ddagger${\kern2pt}tsɯ˧ mv̩˥} or \ipa{$\ddagger${\kern2pt}tsɯ˧˥ {\kern2pt}|{\kern2pt} mv̩˧}. This case appears to be best analyzed in terms of \isi{extrametricality}.

The expression /\ipa{ə˩-gi˩}/ ‘isn’t it!’, ‘right!’ is commonly appended at the end of an utterance. Two observations suggest that it constitutes a~tone group on
its own. First, a~preceding LH or MH \is{tonal contour}contour does not unfold over it, as would be expected inside a~single tone group
(\textit{Caravans.257} \pandoi{0004530\#S257}, \textit{287}; \textit{Housebuilding.113} \pandoi{0004448\#S113}; \textit{Mountains.159} \pandoi{0004573\#S159}; \textit{Sister3.86} \pandoi{0004344\#S86}). Secondly, the expression /\ipa{ə˩-gi˩}/ is often preceded by a~short
(perceived) pause. On the other hand, the fact that it only contains L tones implies that it does \textit{not} constitute a~tone group on its own, otherwise it would be realized as
/\ipa{ə˩-gi˩˥}/ (following Rule~7). The latter form, /\ipa{ə˩-gi˩˥}/, is perfectly acceptable and attested in the narratives, but it is a~full-fledged {question} (‘Is it true?’), whereas /\ipa{ə˩-gi˩}/ is more phatic, almost a~gap-filler. For these reasons, /\ipa{ə˩-gi˩}/ is here analyzed as \is{extrametricality}extrametrical. 

To sum up, the tone group may be interrupted after the last syllable of
a~word (generally, though not exclusively, a~verb or noun), leaving some syllables strand\-ed. They are described as having
\is{extrametricality}extrametrical status. Such syllables do not constitute a full-fledged tone group: specifically, Rule~7 does not apply to them. Yet these \is{extrametricality}extrametrical syllables are still subject to the lowering influence of the preceding tones: Rules 4 and 5 apply to them as if they were inside the tone group. 

In the transcriptions, \is{extrametricality}extrametrical syllables will henceforth be preceded by a diamond symbol `\ipa{◊}', introduced in 2018 (after the publication of the first edition of this book) as a~juncture marker distinct from the pipe symbol `\ipa{|}', which is used for a~standard tone-group \is{boundary (between tone groups)}boundary. The diamond 
%indicates the end of a~tone group while 
signals that the syllables following this mark, up to the next tone-group boundary, are \is{extrametricality}extrametrical. Thus, example (\ref{ex:pluckbutton}) is rewritten as (\ref{ex:pluckbuttonNEW}).

\begin{exe}
  \ex
  \label{ex:pluckbuttonNEW}
  \ipaex{pv̩˩ɭɯ˥ {\kern2pt}|{\kern2pt} ɖʐɤ˧˥ {\kern2pt}◊{\kern2pt} ki˩ tsɯ˩ mv̩˩.}\\
  \gll pv̩˩ɭɯ˥	ɖʐɤ˧˥		ki˧		tsɯ˧˥	mv̩˧\\
  button	to\_pluck	to\_give		\textsc{rep}	\textsc{affirm}\\
  \glt ‘It is said that [he] plucked a~[button from his jacket] and gave it [to the child].~/ He plucked a~button and gave it [to the child].’ \textit{(Renaming.23)} \pandoi{0004534\#S23}
\end{exe}
%"A~shared property of the `\ipa{◊}' and `\ipa{|}' boundaries is that no tonal interaction occurs across them." --> that is not technically correct in that lowering of tones to L can 


\subsection[Deviant tone patterns and Mandarin loanwords]{Deviant tone patterns and Mandarin loanwords: Does the existence of extrametrical syllables facilitate the introduction of loanwords with a~non-final rising tone?}
\label{sec:extrametricalconsolidates}

The phenomena described above in terms of \textit{\isi{extrametricality}} are clearly marginal. Yet they pave the way for increasingly significant changes to the tone system as a~whole. They introduce unusual tone patterns that may become consolidated through loanwords: once a~pattern exists in a~language, however peripheral it may be, it becomes available for accommodating foreign sound combinations. 

 
For instance, the gap-filler \textit{jiùshi} \zh{就是}, ‘quite right; exactly, precisely, just’ is borrowed as
/\ipa{tɕo˧˥ʂɯ˩}/, with a~word-internal MH \is{tonal contour}contour.\footnote{Note that the \is{loanwords}borrowing is from \il{Mandarin!Southwestern|textbf}Southwestern Mandarin, where the syllable /\ipa{tɕo}/ \zh{就} carries a rising tone, unlike the falling tone it has in \il{Mandarin!Standard}Standard Mandarin \citep[on Southwestern Mandarin:][]{guiyunnanese2001, pinson2008}.} At first blush, this appears to contravene a~basic phonotactic rule of Na: the restriction of contours to tone-group-final position. However, the process of \isi{emphasis} described in \sectref{sec:thestylisticoptionofrealizingacontouronawordinnonfinalposition} introduces tone-group-internal contours, which have now become habitually associated with certain morphemes. The existence of these rising contours arguably facilitated the adoption of {Mandarin} loanwords with similar tone patterns. In turn, loanwords contribute to the gradual spread of what was previously a~deviant phonotactic configuration.

Pushing the argument one step further, if the \is{emphasis}emphatic function of tone-group-internal contours predates the \is{loanwords}borrowing of the gap-filler \textit{jiùshi} \zh{就是} ‘exactly’ as /\ipa{tɕo˧˥ʂɯ˩}/, this \is{expressivity}expressive dimension may well have favoured the retention of the rising tone at \is{loanwords}borrowing. Emphasis is particularly well-suited to this item: a~hint of insistence is not out of place for a~gap-filler, as it can help signal to the addressee that the speaker wishes to keep their~speech turn open. The word /\ipa{tɕo˧˥ʂɯ˩}/ also serves as a~rejoinder (‘Exactly!’), in Yongning Na as in \ili{Mandarin}. In this function too, a~touch of emphasis is welcome, reinforcing the intended message of convergence of viewpoints between interlocutors. In other words, the item's lexical tone in \ili{Mandarin} is congruent with its \is{expressivity}expressive interpretation within the Na \is{prosody}prosodic system.

A~cross-linguistic analogue to this situation can be found in the success of the \ili{Vietnamese} \is{loanwords}loanword \textit{nhà quê} in \ili{French}. The original \ili{Vietnamese} expression is a~derogatory term meaning ‘yokel, hayseed, country bumpkin, backwoods person’. It was borrowed into \ili{French} as \textit{niakoué} (also spelt \textit{niacoué}), initially as a~pejorative designation for the \ili{Vietnamese}, and later for the Chinese as well. Lexical tones were lost in the process of \is{loanwords}borrowing, but the vowels and consonants match exactly: \ili{Vietnamese} /\ipa{ɲa.kwe}/ was borrowed as /\ipa{ɲa.kwe}/. Initial \ipa{ɲ} carries expressive connotations in \ili{French}. It is almost only found in slang words: \textit{gnan-gnan} ‘mawkish, mushy’, \textit{gniaf} ‘cobbler’, \textit{gn(i)ard} ‘child’, \textit{gn(i)ouf} ‘prison’, \textit{gnognot(t)e} ‘worthless stuff’, \textit{gnolle} ‘futile person’, \textit{gnôle} ‘alcohol, hard stuff’, \textit{gnon} ‘blow’, \textit{niaque (gnaque)} ‘combativeness’. The villain in Lyon's puppet theatre is tellingly named \textit{Gnafron}. The only \is{exceptions}exception is an Italian \is{loanwords}loanword, \textit{gnocchi}, which apparently managed to gain integration despite the slangy overtones of its initial \ipa{ɲ}. Seen in this light, the adoption of \ili{Vietnamese} \textit{nhà quê} /\ipa{ɲa.kwe}/ into \ili{French} was presumably facilitated by the expressive undertones of its initial consonant to French ears.

Returning to Yongning Na, word-initial contours in Chinese words used by Mrs.\ Latami (consultant F4) are not restricted to expressively loaded items. The word for ‘television’ is a~case in point.\footnote{Remember that these borrowings are from Southwestern {Mandarin}, where tone values are nearly the reverse of those in Standard (Beijing) Mandarin, so that the syllable \textit{diàn} \zh{电} in \textit{diànshì} \zh{电视} ‘television’ carries a rising tone, not a falling tone as in Standard Mandarin.} Realizations \is{form!in isolation}in isolation fluctuate between /\ipa{tjɤ˩˥ʂɯ˧}/ (with a~tone pattern unattested in Yongning Na, apart from Chinese borrowings), /\ipa{tjɤ˧ʂɯ˧}/ (suggesting an underlying M or \#H tone), and /\ipa{tjɤ˩ʂɯ˧˥}/ (pointing unambiguously to an underlying LM+MH\# tone). The noun's tonal behaviour in context is similarly variable. The tones of /\ipa{tjɤ˩ʂɯ˧ li˥}/ ‘to watch television’, /\ipa{tjɤ˩ʂɯ˧-qo˥}/ ‘on TV’, and /\ipa{tjɤ˩ʂɯ˧ ɲi˥}/ ‘is \mbox{(a/the)} TV’ suggest that the noun carries either LM+\#H tone or LM+MH\# tone (in light of the data presented in \sectref{sec:encliticsthatcarrymtonewhenfollowingamtonenoun}, \sectref{sec:objectandnonprefixedverb}, and \sectref{sec:overviewofthesystem}, respectively), but a~\is{tonal contour}contour is sometimes realized on the first syllable, as in the variant /\ipa{tjɤ˩˥ʂɯ˧ ɲi˥}/ for ‘is \mbox{(a/the)} TV’.

This variability reflects a~broader state of flux affecting recent Chinese \isi{loanwords}, also evident in the fluctuating realization of the rhyme of ‘television’~-- alternating between /\ipa{jɤ}/, which conforms to Na phonotactics, and /\ipa{je}/, which does not but is phonetically closer to the Chinese model. These words remain perceived by consultant F4 as Chinese rather than fully integrated Na items and have yet to acquire a~stable Na form. Nonetheless, their presence paves the way for gradual changes in the \is{morphotonology}morphotonological system.

\section{Concluding note}
\label{sec:concludingremark}

The division of an utterance into tone groups plays a~central role in conveying \isi{phrasing} and
\isi{prominence}. In this respect, the Na facts appear closely parallel to the division of
sentences into intonational groups in \ili{English} (or \ili{French})~-- extensively studied languages for which a~wealth of references is available (on \ili{French}, see for instance \citealt{vaissiere1975,dicristo1998,rossi1999,martin2015}). A striking characteristic of Na is the constant interaction between these intonational choices and the language’s tonal processes. The notion of tone groups in Na is deeply intertwined with \is{morphotonology}morphotonology. Boundaries between tone groups not only demarcate intonational units but also impose a~strict constraint on tonal interactions, shaping the realization of tones at several levels. 

The existence of \is{extrametricality}extrametrical syllables further complicates this picture, loosening constraints at the margins of the system and thereby providing a~path for the introduction of atypical tonal configurations. Although these syllables remain peripheral to the core system, they nonetheless create a space in which new tonal patterns can emerge and stabilize, particularly in the context of \is{loanwords}loanword adaptation. The interaction of extrametricality with the incorporation of present-day \ili{Mandarin} borrowings is especially revealing. While the emergence of tone-group-internal rising contours seems to have been initially tied to emphasis marking, their consolidation through loanwords constitutes an ongoing process, whereby peripheral phonotactic combinations are progressively lexicalized and integrated into the language’s tonal system. This reinforces the idea that intonational structures, even when seemingly marginal, can play a~role in shaping a language’s phonological evolution.

The following chapter further explores the complex interplay between \isi{morphotonology} and intonational organization in Yongning Na, a~key issue in tonal studies. 

\is{tone group|)}
