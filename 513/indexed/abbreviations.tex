\addchap{Abbreviations and conventions}
% \addchap{Abbreviations and symbols}

% Add missing right-page heading
\rohead{Abbreviations and conventions}

%\begin{refsection} % Commented out so as to get the references cited in this chapter included in the general bibliography.
	
\section*{Interlinear glosses}
\label{sec:glosses}

Glosses follow the Leipzig Glossing Rules \citep{comrieetal}, with some additions.

\begin{table}[H]
{%\renewcommand{\arraystretch}{1.35}
\begin{tabularx}{\textwidth}{ @{}l Q } 
	\textsc{a} & agent marking (adposition)\\
	\textsc{abilitive} & abilitive (suffix)\\
	\textsc{abl} & ablative (adposition)\\
	\textsc{Adj} & adjective\\
	\textsc{advb} & adverbializer (suffix)\\
	\textsc{affirm} & affirmative (particle)\\
	\textsc{all} & allative\\
	\textsc{associative} & associative plural\\
	\textsc{aug} & augmentative\\
	\textsc{caus} & causative\\
	\textsc{certitude} & \parbox[t]{\linewidth}{a use of the copula described by Lidz (2010:497) as ``an epistemic strategy that marks a high degree of certitude"}\\
	\textsc{clf} & classifier\\
	\textsc{cntr} & contrastive\\
	\textsc{com} & comitative\\
	\textsc{completion} & completion (suffix)\\
	\textsc{cop} & copula\\
   \textsc{dat} & dative\\
   \textsc{dem} & demonstrative\\
	\textsc{desiderative} & desiderative (suffix)\\
	\textsc{dim} & diminutive\\
	\textsc{disc.ptcl} & discourse particle\\
	\textsc{dist} & distal\\
	\textsc{du} & dual\\
	\textsc{dur} & durative\\
	\textsc{excl} & exclusive\\
	\textsc{exist} & existential (verb)\\
	\end{tabularx}}
\end{table}

\begin{table}[H]
\begin{tabularx}{\textwidth}{ @{}l Q }
	\textsc{fut} & future\\
	\textsc{imm.fut} & immediate future\\
	\textsc{imminence} & imminence (prefix): the event is imminent\\
	\textsc{incl} & inclusive\\
	\textsc{interrog} & interrogative (particle or pronoun)\\
	\textsc{intj} & interjection\\
	\textsc{ints} & intensifier\\
	\textsc{N} & noun\\
	\textsc{neg} & negation\\
	\textsc{nmlz} & nominalizer\\
	\textsc{num} & numeral\\
	\textsc{O} & object\\
	\textsc{obligative} & obligative (suffix)\\
	\textsc{permissive} & permissive\\
	\textsc{pfv} & perfective\\
	\textsc{pl} & plural\\
	\textsc{poss} & possessive\\
	\textsc{prog} & progressive\\
	\textsc{proh} & prohibitive\\
	\textsc{prox} & proximal\\
	\textsc{pst} & past\\
	\textsc{recp} & reciprocal\\
	\textsc{redupl} & reduplication\\
	\textsc{rel} & relativizer\\
	\textsc{rep} & reported-speech particle\\
	\textsc{S} & subject\\
	\textsc{sg} & singular\\
	\textsc{top} & topic marker (suffix)\\
	\textsc{V} & verb\\
	1 & first person\\
	2 & second person\\
	3 & third person\\
\end{tabularx}
\end{table}

Verbs are glossed in the infinitive, as ‘to fly’, ‘to say’, ‘to lead’, ‘to go’, and so on, in order to clarify that the intended English gloss is the verb, not the noun. 

\section*{Other abbreviations and symbols}
\label{sec:othersymbols}

\begin{table}[H]
{\renewcommand{\arraystretch}{1.35}
\begin{tabularx}{\textwidth}{ @{}l Q } 
 cs & centisecond: a~unit of time equal to 0.01 seconds\\
 F & \parbox[t]{\linewidth}{focalization of the word that precedes (through local intonational modification of tone: see~\sectref{sec:focalization})}\\
 F\textsubscript{0} & fundamental frequency (a standard abbreviation in phonetics)\\
 F1, F2{\dots} & \parbox[t]{\linewidth}{Female language consultant number 1, 2{\dots} (this is a~standard convention in phonetics; the numbering is chronological, referring to the set of Naish recordings that have been collected since 2002)}\\
 H & High tone\\
 L & Low tone\\
 M & Mid tone\\
 M1, M2{\dots} & Male language consultant number 1, 2{\dots}\\
 N/A & no answer: used in tables for combinations that are not possible \\
 p.c. & personal communication\\
= & clitic boundary\\
- & affix boundary\\
. & syllable boundary\\
σ & syllable (a standard convention in phonology)\\
\ipa{|} & tone group boundary (see Chapter~\ref{chap:toneassignmentrulesandthedivisionoftheutteranceintotonegroups})\\
\ipa{◊} & boundary preceding extrametrical syllables (see \sectref{sec:casesofbreachoftonalgroupingandconsequencesforthesystem})\\
\ipa{{$\sim$}} & reduplication (example: /\ipa{wɤ˩{$\sim$}wɤ˩˥}/)\\
\ipa{≈} & \parbox[t]{\linewidth}{free variation: variation that is not conditioned by phonological or morphosyntactic parameters (example: [\ipa{tɕɥe}]{\kern2pt}\ipa{≈}{\kern2pt}[\ipa{tɕɥi}]). In this volume, the tilde {$\sim$}, commonly used in linguistics as a~symbol for free variation, is reserved for reduplication.}\\
\ipa{↑} & emphatic stress on syllable that follows (see~\sectref{sec:emphaticstressanditstoneddownavatars})\\ 
\# & \parbox[t]{\linewidth}{word boundary (see Chapter~\ref{chap:thelexicaltonesofnouns}); by extension: the boundary of the entire expression to which a~tone pattern is associated (see Chapters~\ref{chap:classifiers}-\ref{chap:verbsandtheircombinatoryproperties})}\\
\$ & a symbol used in H\$, one of the lexical categories of H tones\\
\end{tabularx}}
\end{table}

\begin{table}[H]
{\renewcommand{\arraystretch}{1.35}
\begin{tabularx}{\textwidth}{ @{}l Q } 
-- & \parbox[t]{\linewidth}{morpheme break, used in the representation of the tone pattern of an expression made up of two (or more) morphemes. Thus, for a dimorphemic compound noun, --L refers to an~L tone that attaches after the morpheme break (i.e.\ on the second noun: the head noun), and LM--L indicates that the first noun gets LM tone and the second gets L. (In my publications on Yongning Na before 2016, the symbol used was a~superscript circle~$^{\circ}$, but this use of the symbol conflicted headlong with Africanist usage, in which L$^{\circ}$ refers to a~\textit{level} L tone: an~L tone followed by a floating H tone which results in the absence of a~phonetic falling contour on the L tone. See e.g.\ \citealt[84]{hyman1993} and \citealt[4]{akumbu2016featural}.)}\\
//\ipa{ʐwæ˥}// & \parbox[t]{\linewidth}{underlying phonological form (a~vertical bar is more usual, but in this volume the vertical bar is used for tone group boundaries)}\\
/\ipa{ʐwæ˧}/ & surface phonological form\\
{[\ipa{ʐwæ˧}]} & phonetic realization\\
*\ipa{ʐwæ˧} & a reconstructed form\\
$\dagger${\kern2pt}\ipa{ʐwæ˧} & a form that is predicted on the basis of regular rules, but unattested\\
$\ddagger${\kern2pt}\ipa{ʐwæ˧} & \parbox[t]{\linewidth}{incorrect (ungrammatical) form; note that the asterisk is not used, to preclude confusion with reconstructed forms}\\
:: & \parbox[t]{\linewidth}{phonological correspondence between two languages or dialects (a~standard convention in historical linguistics)}\\
\end{tabularx}}
\end{table}

\section*{References to texts, and links to online annotated recordings of Yongning Na}
\label{sec:refstotxts}

Examples from the online texts are referred to in the following
format: text identifier followed by a dot followed by the sentence
number. For instance, \textit{Sister.27} refers to sentence 27 in the
narrative referred to for short as ‘Sister'. For some of the stories, several versions were recorded. In these cases, the version number is indicated after the text identifier, without a separator: thus, \textit{Dog2.35} refers to sentence 35 in the second version of the narrative ‘Dog'.

One-click links from examples cited in this book to the audio recordings available online (about which see \sectref{sec:onlinematerials}) have been added for the present (second) edition. These hyperlinks use the Digital Object Identifiers assigned to all the resources of the Pangloss Collection since 2020 \citep{vasile2020doi}. The link leads straight to the relevant sentence (in texts) or word (in word lists) inside the document. 

The typographical conventions for inserting these links into the PDF document follow the example set by Guillaume Jacques's grammar of Japhug \citep[3]{jacques2021}. Only the informative part of the URL (Uniform Resource Locator) is visible, omitting the prefix shared by all the documents in the Pangloss Collection, namely \mbox{https://doi.org/10.24397/pangloss-}; thus, \pandoi{0004531\#S119} links to sentence 119 inside the text that bears number 0004531 in the Pangloss Collection. The embedded link contains the complete URL: \url{https://doi.org/10.24397/pangloss-0004531\#S119}. 

Here is a list of texts, providing the correspondences between
identifiers and full titles. Links to the online texts are also provided for the convenience of readers of the electronic version.

%\begin{table}[H]
%\setlength{\parindent}{0pt}
%{\renewcommand{\arraystretch}{1.35}
%\begin{tabularx}{\textwidth}{ @{}l Q }
%\end{tabularx}}
%\end{table}

%\begin{table}[H]
%{\renewcommand{\arraystretch}{1.35}
%\begin{tabularx}{\textwidth}{ @{}l Q }
%\end{tabularx}}
%\end{table}
  
\begin{table}[H]
 	{\renewcommand{\arraystretch}{1.35}
 		\begin{tabularx}{\textwidth}{ @{}l Q >{\raggedleft\arraybackslash}P{20mm}}
  \textbf{Short title} & \textbf{Title} & \multicolumn{1}{c}{\textbf{Link}} \\  
  \textit{Agriculture} & Agriculture: agricultural activities over the course of the year & ~~\pandoi{0004440} \\
  \textit{Benevolence} & Benevolence: about attitudes that were valued among the Na & ~~\pandoi{0004582}\\
  \textit{BuriedAlive} & Buried alive: how a young woman ran into great trouble because of her greed & 1: \pandoi{0004536} 2: \pandoi{0004538}\\
  \textit{Caravans} & Caravans: about the trade which flourished in the area in the second quarter of the twentieth century & ~~\pandoi{0004530}\\
  \textit{ComingOfAge} & Coming of age: the ritual performed at age thirteen & 2: \pandoi{0004588}\\
  \textit{Dog} & Dog: how dog and man exchanged their lifespans & 1: \pandoi{0004442} 2: \pandoi{0004555}\\
  \textit{Elders} & Elders: elders and ancestors & 1: \pandoi{0004442} 2: \pandoi{0004532} \\
  \textit{FoodShortage} & Food shortage: how parents set out to sell children, and then changed their mind & 1: \pandoi{0004557} 2: \pandoi{0004442} \\
  \textit{Funeral} & Funeral: how funeral rites used to be conducted & ~~\pandoi{0004571} \\
  \textit{Healing} & Healing: how diseases used to be treated through rituals & ~~\pandoi{0004540} \\
\end{tabularx}}
\end{table}

\begin{table}[H]
 	{\renewcommand{\arraystretch}{1.35}
 		\begin{tabularx}{\textwidth}{ @{}l Q >{\raggedleft\arraybackslash}P{20mm}}

  \textbf{Short title} & \textbf{Title} & \multicolumn{1}{c}{\textbf{Link}} \\  
  \textit{Housebuilding} & Housebuilding: the process of building a house & 1: \pandoi{0004448} 2: \pandoi{0004549} \\
  \textit{Lake} & Lake: how the Lake was created & 3: \pandoi{0004348} 4: \pandoi{0004350} \\
  \textit{Mountains} & Mountains: some beliefs associated to the mountains around Yongning & ~~\pandoi{0004573} \\
  \textit{Mushrooms} & Mushrooms: which ones are collected for cooking and for medicine & ~~\pandoi{0004616} \\
  \textit{Renaming} & Renaming: how one used to change a child's name
  to give it a happier start in life & ~~\pandoi{0004534} \\
  \textit{Reward} & Reward: how the heavens rewarded an honest man & ~~\pandoi{0004446} \\
  \textit{Seeds} & Seeds: how mankind obtained seeds and learnt to grow crops & 1: \pandoi{0004547} 2: \pandoi{0004542} \\
  \textit{Sister} & Sister: the sister's wedding & 1: \pandoi{0004341} 3: \pandoi{0004344}  \\
  \textit{Tiger} & Tiger: how the tiger attacked a woman and her daughter & 1: \pandoi{0004444} 2: \pandoi{0004545} \\
  \textit{TraderAndHisSon} & Trader and his son: how a trader taught his son how to handle the ups and downs of commerce & ~~\pandoi{0004632} \\

   
\end{tabularx}}
\end{table}

%\begin{table}[H]
%	{\renewcommand{\arraystretch}{1.35}
%		\begin{tabularx}{\textwidth}{ @{}l Q }
%\end{tabularx}}
%\end{table}

%\clearpage

For elicited phonological and morphotonological data, the correspondences between identifiers and full titles are as follows.

\begin{table}[H]
  {\renewcommand{\arraystretch}{1.35}
    \begin{tabularx}{\textwidth}{ @{}l Q >{\raggedleft\arraybackslash}P{20mm}}
  \textbf{Short title} & \textbf{Title} & \multicolumn{1}{c}{\textbf{Link}} \\  
      \textit{AccompPfv} & Verbs illustrating the various tone categories, in the frame \textsc{accomplished+verb+perfective} & \pandoi{0004518} \\
      \textit{ApicalizedVowels} & Words illustrating the opposition between two apicalized high front vowels following alveolo-palatal initials & \pandoi{0004470} \\
    \end{tabularx}}
\end{table}

\begin{table}[H]
  {\renewcommand{\arraystretch}{1.35}
    \begin{tabularx}{\textwidth}{ @{}l Q >{\raggedleft\arraybackslash}P{20mm}}
  \textbf{Short title} & \textbf{Title} & \multicolumn{1}{c}{\textbf{Link}} \\  
      \textit{CoordCompounds} & Coordinative compounds, 1 & \pandoi{0004561} \\
      \textit{CoordCompounds2} & Coordinative compounds, 2: pairs of numerals in association with ‘year', ‘month' or ‘day' & \pandoi{0004662} \\
      \textit{CoordCompounds3} & Coordinative compounds, 3: pairs of numerals in association with ‘year', ‘month' or ‘day' & \pandoi{0004869} \\
      \textit{DemClf} & Demonstrative-plus-classifier phrases, 1 & \pandoi{0004830} \\
      \textit{DemClf2} & Demonstrative-plus-classifier phrases, 2 & \pandoi{0004832} \\
      \textit{DemClf3} & Demonstrative-plus-classifier phrases, 3 & \pandoi{0004834} \\
      \textit{DetermCompounds1to4} & Determinative compound nouns, 1-4: body parts of animals & \pandoi{0004450} \\
      \textit{DetermCompounds5} & Determinative compound nouns, 5: body parts of animals (verifications) & \pandoi{0004452} \\
      \textit{DetermCompounds6} & Determinative compound nouns, 6: body parts of animals (verifications) & \pandoi{0004454} \\
      \textit{DetermCompounds7} & Determinative compound nouns, 7: body parts of animals (extensive set) & \pandoi{0004456} \\
      \textit{DetermCompounds8to10} & Determinative compound nouns, 8-10: body parts of animals (supplements) & \pandoi{0004458} \\
      \textit{DetermCompounds11} & Determinative compound nouns, 11: body parts of animals (a few verifications) & \pandoi{0004460} \\
      \textit{DetermCompounds12} & Determinative compound nouns, 12: body parts of animals & \pandoi{0004462} \\
      \textit{DetermCompounds13} & Determinative compound nouns, 13: body parts of animals (compounds with the noun ‘sheep') & \pandoi{0004464} \\
    \end{tabularx}}
  \end{table}
      
\begin{table}[H]
  {\renewcommand{\arraystretch}{1.35}
    \begin{tabularx}{\textwidth}{ @{}l Q >{\raggedleft\arraybackslash}P{20mm}}
  \textbf{Short title} & \textbf{Title} & \multicolumn{1}{c}{\textbf{Link}} \\  
      \textit{DetermCompounds14} & Determinative compound nouns, 14: body parts of animals in possessive constructions & \pandoi{0004466} \\
      \textit{DetermCompounds15} & Determinative compound nouns, 15: body parts of animals (a few compounds with the noun ‘cat') & \pandoi{0004468} \\
      \textit{DetermCompounds16} & Determinative compound nouns, 16: cultural artefacts and peoples & \pandoi{0004491} \\
      \textit{LocativePostp} & Nouns followed by locative (spatial) postpositions: ‘beside', ‘behind', ‘to the left', ‘to the right' & \pandoi{0004563} \\
      \textit{NounsEven} & Nouns followed by ‘even' & \pandoi{0004565} \\
      \textit{NounsInFrame} & Disyllabic nouns placed in a carrier sentence: ‘This is \mbox{(a/the)} N', in order to bring out their tone patterns & \pandoi{0004529} \\
      \textit{NumClf} (41 documents) & The titles of all 41 documents follow
      the same format: “Numeral-plus-classifier phrases. Tone:
      \textit{T}. Classifier: \textit{C}. Range: \textit{n} to \textit{m}'', where \textit{T} is the lexical tone, \textit{C} the class of objects to which the classifier applies, \textit{n} the start of range (either 1 or 30), and \textit{m} the end of range (either 30 or 100). & \pandoi{0004357} \pandoi{0004357} \pandoi{0004361} \pandoi{0004365} \pandoi{0004371} \pandoi{0004377} \pandoi{0004379} \pandoi{0004387} \textit{+33 more} \\
      \textit{ObjectVerb} & Data illustrating the tone patterns of object-plus-verb combinations, 1 & \pandoi{0004472} \\
      \textit{ObjectVerb2} & Data illustrating the tone patterns of object-plus-verb combinations, 2 & \pandoi{0004474} \\
      \textit{ObjectVerb3} & Data illustrating the tone patterns of object-plus-verb combinations, 3 & \pandoi{0004567} \\
      \textit{OnlyAnd} & Nouns followed by the morpheme ‘only' (homophone: ‘and') & \pandoi{0004569} \\
    \end{tabularx}}
	\end{table}

\begin{table}[H]
  {\renewcommand{\arraystretch}{1.35}
    \begin{tabularx}{\textwidth}{ @{}l Q >{\raggedleft\arraybackslash}P{20mm}}
  \textbf{Short title} & \textbf{Title} & \multicolumn{1}{c}{\textbf{Link}} \\  
      \textit{PossessPro} & Possessive constructions with pronouns, without an intervening particle & \pandoi{0004567} \\
      \textit{SpatialOrientation} & Spatial orientation: combinations of verbs with prefixes (or adverbials) indicating spatial orientation & \pandoi{0004559} \\ % OK
      \textit{SubjectVerb} & Data illustrating the tone patterns of subject-plus-verb combinations & \pandoi{0004476} \\
%     \end{tabularx}}
% 	\end{table}

% \begin{table}[H]
%   {\renewcommand{\arraystretch}{1.35}
%     \begin{tabularx}{\textwidth}{ @{}l Q }
		\textit{VerbDurative} & Verbs illustrating the various tone categories, in the frame \textsc{durative}+V+\textsc{progressive} & \pandoi{0004520} \\
		\textit{VerbProhib} & Verbs of all tone categories preceded by the \textsc{prohibitive}, 1 & \pandoi{0004522} \\
		\textit{VerbProhib2} & Verbs of all tone categories preceded by the \textsc{prohibitive}, 2 & \pandoi{0004524} \\
		\textit{VerbReduplObj} & Reduplicated verbs (tones: M, H, L, and MH) preceded by an object & \pandoi{0004526} \\
		\textit{VerbReduplObj2} & Reduplicated verbs of all tone categories preceded by an object & \pandoi{0004516} \\
      \end{tabularx}}
	\end{table}

%\begin{table}[H]
%	{\renewcommand{\arraystretch}{1.35}
%		\begin{tabularx}{\textwidth}{ @{}l Q }
%		\end{tabularx}}
%	\end{table}
%\begin{table}[H]
%  {\renewcommand{\arraystretch}{1.35}
%    \begin{tabularx}{\textwidth}{ @{}l Q }
%      \end{tabularx}}
%	\end{table}

%\begin{table}[H]
%  {\renewcommand{\arraystretch}{1.35}
%    \begin{tabularx}{\textwidth}{ @{}l Q }
%		\end{tabularx}}
%	\end{table}

\section*{Photographs and figures}
\label{sec:images}

All photographs and figures are my own. 

\section*{Translation of citations}
\label{sec:translofcit}

Unless a~reference to a~published translation is provided, English translations of citations are my own. 

%\end{refsection}