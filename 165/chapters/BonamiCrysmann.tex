\documentclass[output=paper]{langsci/langscibook}
\ChapterDOI{10.5281/zenodo.1407001}

\author{Olivier Bonami\affiliation{Laboratoire de linguistique    formelle, Université Paris Diderot}\lastand  Berthold Crysmann\affiliation{Laboratoire de linguistique    formelle, CNRS}}



\title{Lexeme and flexeme in a formal theory of grammar}

\abstract{This paper deals with the role played by the notion of a lexeme in a   constraint-based lexicalist theory of grammar such as Head-driven  Phrase Structure Grammar. Adopting a Word and Paradigm view of   inflection, we   show how the distinction  between lexemes, individuated by their lexical semantics, and   flexemes, individuated by their inflectional paradigm, can   fruitfully be integrated in such a framework. This allows us to present  an integrated analysis of stem spaces, inflection classes, heteroclisis   and overabundance.
}

\maketitle

\newcommand{\DIM}[1]{\uppercase{#1}}
\newcommand{\DIMBOX}[1]{\fbox{\DIM{#1}}}

\avmoptions{center}
\avmfont{\normalfont\scshape}
\avmvalfont{\normalfont\itshape}
\avmsortfont{\normalfont\itshape}

\newcommand{\BC}[2]{#2}


\newcommand{\phon}[1]{\mbox{\textsf{#1}}}



\begin{document}
\selectlanguage{english}


\is{Head-driven Phrase Structure Grammar|(}
\is{morphology!Information-based Morphology|(}

It is often observed by morphologists that contemporary work in
theoretical morphology has little impact on formal theories of
grammar, which on average are content with a view of morphology quite
close to that of offered by the post-Bloomfieldian morphemic
toolkit.
A notable exception to this
situation is the pervasive use in Head-driven Phrase Structure Grammar
(henceforth HPSG) of the distinction between \textsc{words} and
\textsc{lexemes} familiar from Word and Paradigm approaches to
morphology \citep[see among many
others][]{Robins59,Hockett67,Matthews72,Zwicky85,Anderson92,Aronoff94,Stump01,Blevins14}. In
this paper we reevaluate the role of the lexeme in HPSG in the light
of 20 years of research, and in particular of recent attempts to
integrate a truly realisational theory of inflection within the HPSG
framework \citep{Crysmann14}. We conclude that current theorizing
conflates two distinct notions of an abstract lexical object: lexemes,
which are characterised in terms of their syntax and semantics, and
\textsc{flexemes} \citep{Fradin03b}, which are characterised in terms
of their inflectional paradigm. We propose distinct formal
representations for lexemes and flexemes, and explore the benefits of
the distinction for a formally explicit theory of morphology and the
morphology-syntax interface.

The structure of the paper is as follows. In Section~\ref{sec:1}, we present the
standard view of the lexeme in contemporary HPSG, and show that
lexemes are given a dual representation, as a distinct type of signs
and as the value of the feature \textsc{lid}. In Section~\ref{sec:2}, we present
Information-based Morphology (IbM),%
\footnote{The framework is presented and elaborated in
  \citet{Bonami13d,Bonami15b,Crysmann:14:OUP,Crysmann14}. The name is
  intended as a reference to \citeauthor{Pollard87}'s (1987)
  \emph{Information-based Syntax and Semantics}. \BC{}{In IbM, the
    notion of information in the sense of feature logic plays a
    central role in determining morphological wellformedness, defined
    in terms of exhaustive expression of morphosyntactic
    properties. Furthermore, IbM implements Paninian competition on
    the basis of subsumption, a measure of informativity in feature
    logic.} } an HPSG-compatible realisational approach to inflection,
and show that lexemes-as-signs have no role to play in an HPSG using
IbM as its inflectional component. In Section~\ref{sec:3} we discuss Fradin and
Kerleroux's distinction between lexemes and flexemes, and argue that
this should be encoded by distinguishing a feature \textsc{lid} and
the values it can take from \textit{pid} objects: while the former
reside in syntactic/semantic representations, the latter are found in
inflection proper. Finally in section~\ref{sec:4} we discuss the consequences of
the distinction between \textsc{lid} and \textit{pid} for the
modelling of heteroclisis and overabundance.


\section{The lexeme in standard HPSG}
\label{sec:1}

\subsection{Lexemes as a distinct type of lexical signs}

Most current work in Head-driven Phrase Structure Grammar
\citep[henceforth HPSG;][]{Pollard94,Ginzburg00,Sag03} and its variant
Sign-Based Construction Grammar \citep[henceforth SBCG;][]{Boas12}
embraces the notion of a lexeme, familiar from Word-and-Paradigm
approaches to morphology. Under this view, a lexeme is an abstract
lexical object encapsulating what is common to the collection of words
belonging to the same inflectional paradigm. Although the details are
complex and disputed, it is uncontroversial enough to assume that a lexeme
may be comprised of some amount of phonological information (in the
form of a stem, a collection of stem alternants, a consonantal
pattern, etc.), morphological information (e.g. inflection class
information), syntactic information (at the very least part of speech
and valence information), and semantic information corresponding to a
notion of `lexical meaning' (plus linking of semantic roles to
syntactic dependents). Inflection is then concerned with the relation
between (abstract) lexemes and (concrete)
words,\footnote{Alternatively, within an \emph{abstractive}
  conceptualisation of morphology \citep{Blevins06}, where words are
  seen as primitives rather than derived objects, inflection is
  concerned with the relation between words in a paradigm, and the
  abstract notion of a lexeme captures what is common between these
  words.} while `word formation', more adequately called
\textsc{lexeme formation} \citep{Aronoff94}, is concerned with
morphological relations between lexemes.

Since the late 1990s a growing consensus has emerged within HPSG that
lexemes should be treated as signs on a par with words.\footnote{See \citet{Bonami15b} for a thorough overview of work on morphology in HPSG.} That is, the \is{type hierarchy}
hierarchy of linguistic objects includes the subhierarchy in
Figure~\ref{fig:BonamiCrysmann:signs}. Syntactic rules may form phrases by combining
signs of type \emph{syn-sign}, while rules of morphology manipulate
only signs of type \emph{lex-sign}.

\begin{figure}
\centering
\smaller
\normalfont\itshape
\setlength{\tabcolsep}{0pt}
\begin{tabular}{ccccc}
& & \rnode{a}{\psframebox[linecolor=white,framesep=1pt]{sign}}\\ \\
& \rnode{b}{syn-sign} & & \rnode{c}{lex-sign}\\ \\
\rnode{d}{phrase} & & \rnode{e}{word} & & \rnode{f}{lexeme}\\
\psset{linewidth=.6pt,angleA=-90,angleB=90,arm=0}
\ncdiag{a}{b}
\ncdiag{a}{c}
\ncdiag{b}{d}
\ncdiag{b}{e}
\ncdiag{c}{e}
\ncdiag{c}{f}
\end{tabular}
\normalfont
\caption{\label{fig:BonamiCrysmann:signs} A standard HPSG subhierarchy of signs}
\end{figure}

This is intended to implement the notion of strong lexicalism. First,
words constitute the interface of morphology and syntax, since they
belong to both types. Second, morphology and syntax are discrete
components of grammar inasmuch as some aspects of the feature geometry
of signs will be specific to phrases or lexemes; likewise, this
architecture allows for the possibility that the kind of combinatory
rules relating phrases to their component parts be very different from
the kind of combinatory rules relating words to their component parts.
% As a case in point, consider briefly the view of inflectional
% morphology defended by \cite{Crysmann15}. Under this view, rules of
%  inflection take the form of statements linking a set of $n$ exponents
% (bits of phonology indexed for position) to a set of $m$
% morphosyntactic properties. Inflectional well-formedness amounts to
% verifying that there is a set of rules licensing the presence of
% exactly that set of exponents to realise the relevant morphosyntactic
% information. Under that view, morphological structure is nothing like
% syntactic phrase structure: morphological representations are strings,
% not trees, and their licensing does not amount to parsing these
% strings into contiguous subunits. While Crysmann and Bonami's views
% may be disputable, the very fact that such a view is entirely
% compatible with the architecture in Figure~\ref{fig:BonamiCrysmann:signs} highlights
% how different morphology and syntax may be taken to be.

Although this is by no means an obligation, as we will see below,
standard practice in HPSG and SBCG in the past two decades has been to
assume an Item and Process view of morphology
\citep{Orgun96,Riehemann98,Koenig99,Muller02,Sag03,Sag12}, where the
word-lexeme opposition captures the difference between inflection and
lexeme formation. Rules of inflection map a lexeme to a word, rules of
derivation map a lexeme to a lexeme, rules of composition map two
lexemes to a lexeme. The three toy rules in Figures~\ref{fig:BonamiCrysmann:plural},
\ref{fig:BonamiCrysmann:agent} and~\ref{fig:BonamiCrysmann:compound} illustrate the basic
architecture.

\begin{figure}
\centering
\smaller
\begin{avm}
\[ \asort{word}
   phon & \normalfont\@1+/z/\\
   synsem & \@2 \[head & noun\\ num & pl\]\\
   m-dtrs & \< \[\asort{lexeme} phon & \@1\\ synsem & \@2\] \>
\]
\end{avm}

\caption{Simplified rule of regular English plural
  formation.\label{fig:BonamiCrysmann:plural}}
\end{figure}

\begin{figure}
\centering\smaller
\begin{avm}
\[ \asort{lexeme}
   phon & \normalfont \@1+/ɚ/\\
   synsem & \[head & noun\]\\
   m-dtrs & \<\[\asort{lexeme} phon & \@1\\ ss$|$hd & verb\]\>
\]
\end{avm}
\caption{Simplified rule of English Agent noun
  formation.\label{fig:BonamiCrysmann:agent} }
\end{figure}

\begin{figure}
\centering\smaller
\begin{avm}
\[ \asort{lexeme}
   phon & \@1+\@2\\
   synsem & \[head & noun\]\\
   m-dtrs & \<\[\asort{lexeme} phon & \@1\\ ss$|$hd & noun\],\,\[\asort{lexeme} phon & \@2\\ ss$|$hd & noun\]\>
\]
\end{avm}
\caption{Simplified rule of English noun-noun compound
  formation\label{fig:BonamiCrysmann:compound} }
\end{figure}

Formally, morphological rules are modeled on a par with
phrase-structure rules, except for the fact that, in inflectional and
derivational rules, the relation between the phonology of the mother
(the output lexical sign) and the phonology of the daughter (the input
lexical sign) is specified syncategorematically: affixes are not
signs, but bits of phonology added by rule.\footnote{For a dissenting
  view see \citet{Emerson15}.} The main difference between inflection
and lexeme formation rules lies in the fact that inflection does not
modify the \textsc{synsem} value, but merely expresses some of its
aspects. The main specificity of composition is that the input (the
daughter signs) consists of two lexemes rather than
one. Figures~\ref{fig:BonamiCrysmann:lovers} and~\ref{fig:BonamiCrysmann:birdwatchers} illustrate
typical morphological analyses within such a framework.


\begin{figure}[htb]
\centering
\smaller
\begin{avm}
\[	\asort{word}
	ph & \normalfont/lʌvɚz/\\
	ss|hd & \@1 \[num & pl\]\\
	m-dtrs & \<\[	\asort{lexeme}
					ph & \normalfont/lʌvɚ/\\
					ss|hd & \@1 noun\\
					m-dtrs & \<\[	\asort{lexeme}
									ph & \normalfont/lʌv/\\
									ss|hd & verb
							  \]\>
			 \]\>
\]\end{avm}

%\br{}{
%\lf{\begin{avm}\[\asort{lexeme} ph & /bɝd/\\ ss|hd & noun\]\end{avm}}
%\br{\[\asort{lexeme} ph & /wɑtʃɚ/\\ ss|hd & \@1 noun\]}{
%\lf{\begin{avm}\[\asort{lexeme} ph & /wɑtʃ/\\ ss|hd & verb\]\end{avm}}
%}}}
%\end{avmtree}

%\begin{avmtree}
%\psset{linewidth=.6pt,levelsep=12ex}
%\br{\[\asort{word} ph & /lʌvɚz/\\ ss|hd & \@1 \[num & pl\]\]}{
%\br{\[\asort{lexeme} ph & /lʌvɚ/\\ ss|hd & \@1 noun\]}{
%\lf{\begin{avm}\[\asort{lexeme} ph & /lʌv/\\ ss|hd & verb\]\end{avm}}
%}}
%\br{\[\asort{word} ph & /bɝdwɑtʃɚz/\\ ss|hd & \@1 \[num & pl\]\]}{
%\br{\[\asort{lexeme} ph & /bɝdwɑtʃɚ/\\ ss|hd & \@1 noun\]}{
%\lf{\begin{avm}\[\asort{lexeme} ph & /bɝd/\\ ss|hd & noun\]\end{avm}}
%\br{\[\asort{lexeme} ph & /wɑtʃɚ/\\ ss|hd & \@1 noun\]}{
%\lf{\begin{avm}\[\asort{lexeme} ph & /wɑtʃ/\\ ss|hd & verb\]\end{avm}}
%}}}
%\end{avmtree}
\caption{Analysis of the noun \emph{lovers} under an  Item-and-Process view of morphology \label{fig:BonamiCrysmann:lovers}}
\end{figure}

\begin{figure}[htb]
\centering
\smaller
\begin{avm}
\[	\asort{word}
	ph & \normalfont/bɝdwɑtʃɚz/\\
	ss|hd & \@1 \[num & pl\]\\
	m-dtrs & \<\[	\asort{lexeme}
					ph & \normalfont/bɝdwɑtʃɚ/\\
					ss|hd & \@1 noun\\
					m-dtrs & \<\[	\asort{lexeme}
									ph & \normalfont/bɝd/\\
									ss|hd & noun
							   \],
							   \[	\asort{lexeme}
									ph & \normalfont/wɑtʃɚ/\\
									ss|hd & \@1 noun\\
									m-dtrs & \<\[\asort{lexeme}
							        			 ph & \normalfont/wɑtʃ/\\
							        			 ss|hd & verb
										     \]\>
							  \]\>
			\]\>
\]
\end{avm}
\caption{Analysis of the noun \emph{birdwatchers} under an
  Item-and-Process view of morphology \label{fig:BonamiCrysmann:birdwatchers}}
\end{figure}

\subsection{Lexeme identifiers}

It is sometimes necessary for a lexical entry or syntactic
construction to be able to select a particular lexical item in its
environment. One clear case of this is that of flexible
idioms. Consider the idiom \emph{pull strings} `try something'. As the
examples in~(\ref{ex:BonamiCrysmann:strings}) make clear, while the idiomatic meaning
is present only when the object of \emph{pull} is headed by the lexeme
\emph{strings}, the noun may occur in either singular or plural form,
and combine with a variety of determiners and modifiers
\citep{Bargmann17}.

\begin{exe}
  \ex \label{ex:BonamiCrysmann:strings} \begin{xlist} \ex There I learned whom [\emph{sic}] my
    secret advocate was, the man who had pulled strings to get me the
    teaching job in the midst of a terrible economy, and who had
    pulled more strings to allow me to keep it, and who had then
    pulled even more strings to have my commission assigned to the
    Abwehr.\footnote{K. Ryan, \emph{The Somnambulist}, New York:
      iUniverse, 2006.}

    \ex You'll never know the trouble I had, and the strings I had to
    pull to get you back from Berlin.\footnote{K. McDermott, \emph{The
        time of the corncrake}, Victoria: Trafford, 2004.}

    \ex We have to remember that Jacob was at their wedding. Just how
    many strings did he
    pull?\footnote{\url{http://www.losttv-forum.com/forum/showthread.php?t=65542}. Accessed
      on November 26, 2016.}

    \ex So I didn't pull any string. Didn't need
    to.\footnote{\url{http://www.justusboys.com/forum/archive/index.php/t-437037.html}. Accessed
      on November 26, 2016.}

    \ex When I got the job, I thought to myself, ``Someone upstairs
    finally pulled a string for
    me''.\footnote{\url{http://ultraphrenia.com/2016/10/02/a-cigarette-break-behind-heavens-gate}. Accessed
      on November 13, 2016.}

    \ex No string was pulled, it was based on
    merit.\footnote{\url{http://obafemayor02.blogspot.fr/2013_03_24_archive.html}. Accessed
      on November 26, 2016.}
\end{xlist}
\end{exe}

This type of situation motivated the introduction of the feature
\textsc{lid} (or \textsc{Lexeme IDentifier}) as a head feature
projecting to phrasal level information as to which lexeme heads a
phrase \citep{Sag07,Sag12}.\footnote{Note that a very similar role is
  played by the feature \textsc{listeme} in \citet{Soehn06} and
  \citet{Richter09}. % See also the use of \textsc{head|keys|keyrel} in
  % the English Resource Grammar \citep{lingoerg}, and, more generally
  % in Minimal Recursion Semantics \citep{Copestake06}.
} Simplifying
matters considerably, one can see the constructions above as licensed
by the two idiomatic lexical entries in
Figure~\ref{fig:BonamiCrysmann:pull-strings:LI}, which contrast with the two
ordinary entries in Figure~\ref{fig:BonamiCrysmann:ordinary:LI}: a special lexical
entry of \emph{pull} with idiomatic meaning selects specifically for
an object headed by a form of \emph{strings} with idiomatic
meaning. The postulation of a specific \textsc{lid} value for
idiomatic \emph{string} allows idiomatic \emph{pull} to select for a
specific combination of an inflectional paradigm with an idiomatic
meaning, while abstracting away from inflectional and syntactic
variability in the makeup of the object of \emph{pull}.


\begin{figure}
\centering
\smaller
\begin{avm}
	\[	\asort{lexeme}
	    phon & \normalfont {/pʊl/}\\
	    head & \[lid & pull-lid\]\\
	    val & \<NP$_i$,NP$_j$\>\\
	    cont & \[\asort{pull-rel} act & i\\ und & j\]
	\]
	\end{avm}
~~~~~~\begin{avm}
	\[	\asort{lexeme}
	    phon & \normalfont{/stɹɪŋ/}\\
	    head & \[lid & string-lid\]\\
	    cont & string-rel
	\]
	\end{avm}

        \caption{\label{fig:BonamiCrysmann:ordinary:LI} Ordinary lexical entries for
          \emph{pull} and \emph{strings}}
\end{figure}
\begin{figure}
\centering
\smaller
\begin{avm}
	\[	\asort{lexeme}
	    phon & \normalfont{/pʊl/}\\
	    head & \[lid & idiomatic-pull-lid\]\\
	    val & \<NP$_i$,NP$_j$[\textsc{lid}~{\upshape\itshape idiomatic-string-lid}]\>\\
	    cont & \[\asort{use-rel} act & i\\ und & j\]
	\]
	\end{avm}
~~~~~~\begin{avm}
	\[	\asort{lexeme}
	    phon & \normalfont{/stɹɪŋ/}\\
	    head & \[lid & idiomatic-string-lid\]\\
	    cont & influence-rel
	\]
	\end{avm}
\caption{\label{fig:BonamiCrysmann:pull-strings:LI} Idiomatic lexical entries for \emph{pull} and \emph{strings}}
\end{figure}



The feature \textsc{lid} provides a useful mechanism for spreading
lexical information in syntactic structures that has  been
used since in the analysis of complex predicates \citep{Muller10} and
periphrastic inflection
\citep{Bonami13,Bonami15,bonami2015,Bonami16b}. It also provides a
direct encoding of lexemic identity. Since \textsc{lid} is a head
feature, and inflected words share the \textsc{head} value of the
lexeme they are derived from, all inflected forms of a lexeme will
have the same \textsc{lid}. Under the natural assumption that all
lexemes have a distinct \textsc{lid} value, whether two words
instantiate the same lexeme can thus be deduced by inspection of their
\textsc{lid} values, without examining their derivation history.


\section{The lexeme in a Word and Paradigm version of HPSG}
\label{sec:2}

\subsection{Going Word and Paradigm}



While an Item and Process view of morphology has been dominant in the
HPSG literature, over the last 20 years a number of authors have
become more vocal in advocating the incorporation into HPSG of a Word
and Paradigm view of inflection (see
among others \citealt{Erjavec94,Miller97,Ackerman98,Crysmann02,Bonami06,Bonami13,bonami2015,Bonami15,Crysmann14}). Under
such a view, rules of inflection do not incrementally specify how a
basic sign is augmented with morphosyntactic information and
phonological exponents; rather, a full morphosyntactic specification
of the word is given as input to a system of rules of exponence
indicating how such a specification is partially realised by exponents
in various positions with respect to the basic stem. The arguments in
favour of such a move are the usual ones
\citep{Matthews74,Zwicky85,Anderson92,Stump01,Brown12}: systems of
exponence depart too strongly from a one-to-one correspondence between
form and content for the Item and Process view to make sense in the
general case. We will not rehearse these arguments here, and simply
make the sociological observation that Word and Paradigm approaches
have over the last two decades become the \emph{de facto} standard for
theoretical and typological reasoning on inflection systems.

Recent attempts at implementing Word and Paradigm inflection in HPSG
come in two flavors. One the one hand,
\citet{Bonami13,bonami2015,Bonami15} explicitly interface Paradigm
Function Morphology \citep{Stump01,Stump16} with HPSG through a set of
relational constraints. On the other hand, \citet{Crysmann14} design a
realisational framework for inflection native to the HPSG
architecture, \emph{Information-based Morphology} (IbM), making heavy
use of the \isi{underspecification} techniques provided by a typed feature
structure formalism.

Figure~\ref{fig:BonamiCrysmann:buvions} illustrates the main features of IbM by way
of the analysis of a rather simple inflected word, the French verb
\emph{buvions} `we drank'.
\begin{figure}
\centering\smaller
\begin{avm}
\[ \asort{word}
phon & \normalfont/byvjɔ̃/\\
mph & \{\,\rnode{aa}{\[ph &  \normalfont/byv/\\ pc & 0\]} , \rnode{ab}{\[ph &  \normalfont/j/\\ pc & 2\]} , \rnode{ac}{\[ph &  \normalfont/ɔ̃/\\ pc & 3\]}\,\}\\ \\
   rr & \{\,\[	mph & \{\,\rnode{ba}{\[ph &  \@1 \normalfont/byv/\\ pc & 0\]}\,\}\\
   				mud & \{\,\rnode{ca}{\[lid & \[\asort{drink-lid} stems & \q<\@1,\ldots\q>\]\]}\,\}
			\] ,
			\[	mph & \{\,\rnode{bb}{\[ph &  \normalfont/j/\\ pc & 2\]}\,\}\\
   				mud & \{\,\rnode{cb}{\[tns & pst\\ asp & ipfv\]}\,\}\\
%				ms & \{\[\asort{subj} per & 12\\ num & pl\]\}
			\]
			,
			\[	mph & \{\,\rnode{bc}{\[ph &  \normalfont/ɔ̃/\\ pc & 3\]}\,\}\\
   				mud & \{\,\rnode{cc}{\[\asort{subj} per & 1\\ num & pl\]}\,\}\\
			\]

		\}	\\ \\
   ms & \{\,\rnode{da}{\[lid & \[\asort{drink-lid} stems & \<\normalfont/byv/,/bwav/,/bwa/\>\]\]} , \rnode{db}{\[tns & pst\\ asp & ipfv\]},\rnode{dc}{\[\asort{subj} per & 1\\ num & pl\]}\,\}\\
   synsem & \[
              head & \[lid & drink-lid\\ tns & pst\\ asp & ipfv\]\\
              arg-st & \<NP$_{\text{\itshape 1pl}}$, NP\>
            \]
\]
\end{avm}
%\psset{arrows=<->,linecolor=red,angleA=-90,angleB=90}
%\nccurve{aa}{ba}
%\nccurve{ab}{bb}
%\nccurve[angleA=45]{ac}{bc}
%\nccurve{ca}{da}
%\nccurve{cb}{db}
%\nccurve{cc}{dc}

\caption{A sample IbM analysis: the French \textsc{ipfv.2pl} word
  \emph{buvions} `we drank'\label{fig:BonamiCrysmann:buvions}}
\end{figure}
IbM specifies the inflectional system of a language as a set
of constraints relating a word's \textsc{synsem} value to its
\textsc{phon}ology. In the present example, a word realising the past
imperfective of the verb \textsc{boire} in the context of a
\textsc{1pl} subject is constrained to have the string /byvjɔ̃/
as its phonological realisation. The specification of these
constraints makes use of three intermediate, strictly morphological,
representations. The feature \textsc{ms} (standing for
`morphosyntactic properties') encodes those syntactic and semantic
properties of the word that are relevant to inflection, in a format
suitable for the expression of constraints on exponence. The feature
\textsc{mph} (standing for `morphs') indicates the set of morphs
making up the word, indexed for their position within the word
(\textsc{pc}, standing for `position class'). Finally, the feature
\textsc{rr} (standing for `realisation rules') indicates which
generalisations on the relationship between morphosyntactic properties
and morphs license the particular association between form and content
instantiated in that word. Importantly, realisation rules relate a
\emph{set} of morphosyntactic properties (listed under \textsc{mud},
standing for `morphology under discussion') to a \emph{set} of morphs
(listed under \textsc{mph}). Thus, while in this simple example, there
is a one-to-one mapping between properties and morphs, IbM realisation
rules can just as easily accommodate cumulative exponence
($m\, \text{properties}:1\, \text{morph}$), extended exponence
($1\, \text{property}:n\, \text{morphs}$), overlapping exponence
($m\, \text{properties}:n\, \text{morphs}$), and zero exponence
($m\, \text{properties}:0\, \text{morphs}$).

The relationship between the various features is regulated by a set of
general principles that we will  only state in prose here; we refer the
reader to \citet{Bonami13d} or \citet{Crysmann14} for a more explicit
formulation.
Let us start with the relationship between the
\textsc{synsem} and \textsc{ms} values. This is regulated by a set of
language-specific constraints, since which aspects of syntax and
semantics are realised by inflection is a highly parochial matter. Two
features of this interface are worth noting. First, lexeme-specific
information on inflection class and stem alternants is included in
\textsc{ms} inside the \textit{lid} value. In particular, a
list-valued feature \textsc{stems} provides an indexed set of stem
alternants, also known as a \isi{stem space} \citep{Bonami06}.%
\footnote{\citet{Bonami06} argue that the French stem space has 12
coordinates. for simplicity we show only 3 in the example in
Figure~\ref{fig:BonamiCrysmann:buvions}.}
The choice of
a particular stem is then effected by a realisation rule of \emph{stem
  selection} \citep{Stump01}, picking out the appropriate value in this
list, depending on the morphosyntactic context; In the present instance,
the default of picking the first stem applies. In other words, in IbM,
even the stem is taken to be the realisation of some word-level
information, namely lexical identity. Second, \textsc{ms} values are
relatively flat in comparison to \textsc{synsem} values, consisting of
a set of small feature structures, rather than one large, deeply
recursive feature structure. This is necessitated by the different
demands of morphological and syntactic combination.\footnote{The
  distinction between \textsc{synsem} and \textsc{morsyn} may also be
  used to account for mismatches between content and form at the
  morphology-syntax interface, as variously captured in the literature
  by distinguishing syntactic and morphological features
  \citep{Sadler:Spencer01,Corbett06,bonami2015} or content and form paradigms
  \citep{Stump06,Stump16}.}

\newpage   
We may now turn to the relationship between \textsc{ms} and
\textsc{rr}. This is regulated by a principle of morphological
wellformedness: the \textsc{ms} value of a word must be identical to
the disjoint union of the \textsc{mud}s of the realisation rules. In
other words, each morphosyntactic property must be realised by exactly
one rule, although a single rule may realise multiple properties at
once.\footnote{Implicit here are two assumptions familiar from
  Paradigm Function Morphology: (i) if two realisation rules are
  appropriate in some context, only the rule realising more content
  may apply (Panini's Principle); and (ii) there exists a universal
  rule of default non-realisation, ensuring that a property set
  remains unrealised if and only if the inflection system provides no
  other rule for its realisation.}

Finally, the relation between \textsc{rr}, \textsc{mph} and
\textsc{phon} is rather straightforward. First, the \textsc{mph} value
of a word is the union of the \textsc{mph} values of its realisation
rules: in other words, every morph must be licensed by at least one
realisation rule, although a realisation rule may license more than
one morph (extended or overlapping exponence), or even no morph at all
(zero exponence). Second, a word's phonology is determined by
appending the phonology of its morphs in accordance with the linear
sequence of position class indices. Note that, although the system of
position class indices encodes the notion of a morphotactic template,
it does so with appropriate flexibility. There is no notion of an
`empty position' in the template: position class indices regulate the
relative order of morphs, but morph ordering is not effected by
putting bits of phonology in slots, just by appending bits of
phonology in order. More importantly, realisation rules may partially
underspecify the position they assign morphs to, allowing one to
capture an unprecedented set of situations of variable
morphotactics. Note also that, although a realisation rule may encode
zero exponence, it is not equivalent to a zero morpheme: having no
morph as one's exponent is not the same thing as having a morph with
no phonological realisation. In particular, since no empty morphs are
postulated, no sybilline decisions need to be taken as to the
positioning of  inaudible elements.

\subsection{The role of the lexeme in IbM}

Now that we have outlined the main features of IbM, let us consider
the role of the lexeme in such a framework. Remember that in classical
HPSG, inflection rules take the form of unary rules relating an
abstract sign, the lexeme, to a surface sign, the inflected word. IbM
has no use for such a notion of inflection rule, since inflection is
stated directly as a relation between content and form at the word
level. On the other hand, IbM makes crucial use of the notion of a
lexeme identifier to state lexeme-specific phonological and
morphological information; and the word/lexeme opposition is still
a useful way of capturing the relationship between lexical entries
and inflected words, and making a clear distinction between lexeme
formation and inflection.

We thus assume that, while there are no inflectional lexical rules,
there is a general constraint on objects of type \emph{word} to the
effect that they are the realisation of a lexeme, as indicated in~(\ref{ex:BonamiCrysmann:lexp}).
This constraint enforces the monotonic character of inflection: unlike derivation, inflection does not modify syntax or semantics but merely realises whatever features are made available by paradigm structure and compatible with the syntactic context. This is enforced by the identity of \textsc{synsem} values at the \textit{lexeme} and \textit{word} levels.

\begin{exe}
\ex\label{ex:BonamiCrysmann:lexp}\small
\textit{word} $\rightarrow$
\begin{avm}
\[  synsem & \@1\\
	m-dtrs & \< \[ \asort{lexeme}
                   synsem & \@1
                \] \>
\]
\end{avm}
\end{exe}

As a consequence of~(\ref{ex:BonamiCrysmann:lexp}), an inflected word will inherit any constraint imposed by the lexeme's lexical entry within \textsc{synsem}, including, crucially, lexical identity and stem alternants as specified through the \textsc{lid} feature. Note that we assume the \textsc{phon} attribute to be appropriate only for \textit{syn-sign} objects (that is, words and phrases): lexemes constrain the phonology of their inflected forms through the \textsc{stems} feature instead \citep{Bonami06}. The inflection-specific features \textsc{mph}, \textsc{rr} and \textsc{sc} are appropriate for \textit{word}s only. The format of lexical entries and lexeme formation rules is essentially unchanged.


%In effect, the notion of a lexeme as a member of the \emph{sign}
%hierarchy is made entirely redundant in IbM, and can be
%eliminated. Lexical relatedness between members of a paradigm amounts
%to identity of \textsc{lid}. Lexical entries can be stated as
%underspecified descriptions of words that leave to the inflectional
%component the specification of the word's \textsc{phon} value as a
%function of its \textsc{lid} and the syntactic and semantic context in
%which it is used. While these correspond to lexeme signs in
%Item-and-Process HPSG, in IbM they do not constitute a distinct entity
%type from words.
%
%Lexeme formation processes can be modeled as relating partial
%descriptions of words, rather than lexemes. For instance, the
%derivation rule in Figure~\ref{fig:BonamiCrysmann:agent} could be reimplemented along
%the lines shown in Figure~\ref{fig:BonamiCrysmann:agent:IbM}.
%
%\begin{figure}[htb]
%\centering
%\smaller
%\begin{avm}
%\[ \asort{word}
%    ss & \[cat|hd & \[\asort{noun}
%                      lid & \[stems & \<\normalfont \@1+/ɚ/\>\]
%                    \]\\
%%           sem & \[ind & \@{x}\\
%%                   restr & \{\[\asort{habitual-rel} soa & \@2\]\}
%%                 \]
%          \]\\
%   m-dtrs & \<\[\asort{word}
%                ss & \[cat|hd & \[\asort{verb}
%                                  lid & \[stems & \<\normalfont \@1\>\]
%                                \]\\
%%                       sem|restr & \{\@2 \[\asort{agt-rel} act & \@{x}\]\}
%                     \]
%              \]\>
%\]
%\end{avm}
%\caption{Simplified rule of English Agent noun formation in
%  IbM.\label{fig:BonamiCrysmann:agent:IbM} }
%\end{figure}
%
%
%This rule relates a word to a word by specifying how the
%\textsc{lid} value of the output is derived from
%the \textsc{lid} value of the input.
%Since the rule says nothing about number or the
%\textsc{phon} value, its output is compatible with whichever
%relationship between \textsc{synsem} and \textsc{phon} values are
%allowed by the inflectional morphology of the language. For instance,
%under simple assumptions, the rule is compatible with the two
%instantiations shown in Figure~\ref{fig:BonamiCrysmann:derivations:IbM}, where the
%base sign is taken to be the verb \textsc{love} and the two distinct
%ways of inflecting the noun \textsc{lover} depending on number are
%shown.\footnote{The second rule in \textsc{rr} in the analysis of
%  \emph{lover} is the default non-realisation rule, stating that in
%  the absence of a more specific rule, any property may be unrealised,
%  i.e. associated with an empty set of morphs.}
%
%
%\begin{figure}[htb]
%\centering\smaller
%\avmoptions{top}
%\begin{avm}
%\[	   \asort{word}
%       phon & \normalfont/lʌvɚ/\\
%       mph & \{\@5 \[ph & \@1 \\ pc & 0\]\}\\
%	   rr & \{\[mph & \{\@5\}\\ mud & \{\@3\}\],
%	          \[mph & \{~\}\\ mud & \{\@4\}\]
%	        \}\\
%	   ms & \{\@3 \[lid & \@2\], \@4 \[num & sg\]\}\\
%       ss|hd & \[\asort{noun} lid & \@2 \[\asort{lover-lid} stems & \<\normalfont \@1 /lʌvɚ/\>\]\\ num & sg\]\\
%       m-dtrs & \<\[\asort{word} ss|hd & \[\asort{verb} lid & \[\asort{love-lid} stems & \<\normalfont /lʌv/\>\]\]\]
%       \>\]
%\end{avm}%
%~%
%\begin{avm}
%\[	   \asort{word}
%       phon & \normalfont/lʌvɚz/\\
%       mph & \{\@5 \[ph & \@1 \\ pc & 0\], \@6 \[ph & /z/\\ pc & 1\]\}\\
%	   rr & \{\[mph & \{\@5\}\\ mud & \{\@3\}\],
%	          \[mph & \{\@6\}\\ mud & \{\@4\}\]
%	        \}\\
%	   ms & \{\@3 \[lid & \@2\], \@4 \[num & pl\]\}\\
%       ss|hd & \[\asort{noun} lid & \@2 \[\asort{lover-lid} stems & \<\normalfont \@1 /lʌvɚ/\>\]\\ num & pl\]\\
%       m-dtrs & \<\[\asort{word} ss|hd & \[\asort{verb} lid & \[\asort{love-lid} stems & \<\normalfont /lʌv/\>\]\]\]
%       \>\]
%\end{avm}
%
%\caption{Analyses of the nouns \emph{lover} and \emph{lovers} under an
%  IbM view of morphology \label{fig:BonamiCrysmann:derivations:IbM}}
%\end{figure}
%
%
%To sum up, it is now clear that, under a Word and Paradigm view of
%morphology (as exemplified by IbM), the classical HPSG distinction
%between two kinds of lexical signs, lexemes and words, becomes
%redundant. The lexicon and lexeme formation components provide various
%\textsc{lid} values; the inflectional component determines what
%phonology words with such \textsc{lid} values should have in different
%morphosyntactic contexts.

\section{Lexemes and flexemes}
\label{sec:3}

\is{flexeme|(}

In this section we build on the general architecture just presented and argue that a distinction between two notions of lexical identity needs to be made.

\subsection{Introducing the flexeme}

Up to now, we have assumed a simple relationship between lexemes and
inflectional paradigms: the value of the same feature \textsc{lid} is
used for purposes of lexeme selection and for purposes of
individuating inflectional paradigms. In doing so we have been
following standard practice in realisational morphology, where
paradigm functions take `lexemes'
\citep{Stump01,Stump16} or equivalently a `lexemic index'
\citep{Spencer13} as an argument.

In an important but rarely cited paper, \citet{Fradin03b} note that
matters are not so simple, for reasons having to do with lexical
ambiguity and the division of labour between inflection and lexeme
formation.\footnote{We purposefully use the general term `lexical
  ambiguity' because whether the relevant examples are instances of
  polysemy or homonymy does not affect the argument.} Rules of
inflection are not generally
% there's e.g. hanged and hung
concerned with matters of lexical ambiguity: from the point of view of
inflection, the  two French verbs \textsc{devoir$_1$} `must' and
\textsc{devoir$_2$} `owe' are indistiguishable, as they have the same
(highly irregular) inflectional paradigm. From the point of view of
derivation, however, things are different. Derived lexemes normally
relate to one sense of their base: for instance, while the French noun
\textsc{fille} is ambiguous between two readings \textsc{fille$_1$}
`girl' and \textsc{fille$_2$} `daughter', the diminutive
\textsc{fillette} `small girl' only relates to the
first.\footnote{This very short summary does not do justice to
  Fradin and Kerleroux's insights, which build on an examination of
  the compatibility of various lexeme formation rules in French
  \citep{Fradin03b} with various families of meanings. See also
  \citet{Fradin09} for more discussion.}  \citet{Fradin03b} argue that
this warrants a distinction between two kinds of abstract lexical
objects: \emph{lexemes} and \emph{flexemes}. Inflection is about
flexemes, while derivation is about lexemes. Because of the pervasive nature
of lexical ambiguity, a single flexeme often corresponds to multiple
  lexemes.

In the remainder we follow \citet{Walther13} in assuming that inflection is strictly concerned with flexemes, and propose an implementation of the  lexeme-flexeme distinction in IbM.

\subsection{\textsc{lid} and \textsc{pid}}

Within an HPSG view of the world, it is tempting to capture the
relationship between lexemes and flexemes in terms of
\isi{underspecification} in an inheritance hierarchy\is{type hierarchy}: flexemes would then be
abstract groupings of lexemes. Suppose for concreteness a hierarchical
organisation of \textsc{lid} values such as that indicated in
Figure~\ref{fig:BonamiCrysmann:devoir}. Rules of inflection can then be stated in
terms of the supertype \emph{fille}, while lexemes are properly
individuated in terms of the subtypes; and hence \textsc{fillette} can
be uniquely related to the lexeme whose \textsc{lid} value is
\emph{fille$_1$}.

\begin{figure}[htb]
\centering
\smaller
\itshape
\begin{tree}
\psset{linewidth=.6pt,linestyle=dashed,levelsep=8ex,treesep=6em}
\br{\itshape lid}{
\br{\begin{avm}\[\asort{fille-lid} stems &  \q\<\normalfont/fij/\q\>\]\end{avm}}{\psset{linestyle=solid}\lf{fille-1-lid}\lf{fille-2-lid}}}
\end{tree}
\caption{\normalfont\label{fig:BonamiCrysmann:devoir} A first pass at flexemes in HPSG: flexemes as underspecified \textsc{lid} values}
\end{figure}


While this is technically feasible, such an approach only obscures the
orthogonal roles played by the two notions. As illustrated above, IbM
\textsc{lid} values are structured objects, which encompass all
lexically-specified information relevant to inflection, including most
notably stem alternants and inflection class. Such information is
clearly irrelevant to syntax, although it is an indispensable
component of inflection.  On the other hand, studies that use
\textsc{lid} for purposes of syntactic selection presuppose a tight
correspondence between \textsc{lid} values and lexical semantic
identity, and have no use for purely morphological information on stem
alternants or inflection classes. In particular, \citealt{Sag12}
argues that \textsc{lid} values are to be identified with the main semantic
predicate associated with a lexeme. One clear advantage of this
convention is avoidance of redundancy in lexical entries: it is not
necessary to postulate a new symbol as the \textsc{lid} value of each
lexeme, since such a symbol is already present in the lexical entry as
the constant designating the lexeme's main semantic predicate.


We now propose to clarify the situation by adopting Sag's view of
\textsc{lid}. This entails that, for purposes of inflection, a
separate index must be posited that individuates words according to
which flexeme they instantiate. We call this index \textsc{pid}, standing
for `paradigm identifier'. While \textsc{lid} resides in \textsc{head}
and is thus available for selection in idioms, complex predicate
constructions, or periphrastic constructions, \textsc{pid} is a top-level
feature carried by signs of type \textit{lexeme} only. As such it can be specified by lexical entries or manipulated by lexeme formation rules. In addition, it is universally constrained to be present among the features realised by inflection through inclusion in \textsc{ms}, as indicated in~(\ref{ex:BonamiCrysmann:pid}). This is crucial to ensuring that inflection is always concerned with the realisation of lexical identity.

\begin{exe}
\ex\label{ex:BonamiCrysmann:pid}\small
\textit{word} $\rightarrow$
\begin{avm}
\[  ms & \{\,\@1, \ldots\,\}\\
	m-dtrs & \< \[ \asort{lexeme}
                   pid & \@1
                \] \>
\]
\end{avm}
\end{exe}

In this architecture then, lexical entries need to specify both an \textsc{lid} and a \textsc{pid} value. To elaborate on the same example, an appropriate analysis of
\textsc{fille} would posit two lexical entries sharing the same
\textsc{pid} object while having different \textsc{lid} values, as
indicated in Figure~\ref{fig:BonamiCrysmann:fille:two}.

\begin{figure}[htb]
\centering\smaller

\begin{avm}
\[ \asort{lexeme} ss & \[cat|hd|lid & \@1 girl-rel\\
                       sem|restr & \{\,\@1\,\}
                     \]\\
   pid &  fille-pid%\[\asort{pid} stems & \<\normalfont /fij/\>\]
\]
~~~~~
\[ \asort{lexeme} ss & \[cat|hd|lid & \@1 daughter-rel\\
                       sem|restr & \{\,\@1\,\}
                     \]\\
   pid & fille-pid%\[\asort{pid} stems & \<\normalfont /fij/\>\]
\]
\end{avm}
\caption{Proposed lexical entries for the two lexemes
  \textsc{fille}.\label{fig:BonamiCrysmann:fille:two}}
\end{figure}

Under this analysis, the two lexemes \textsc{fille} are related by
virtue of having indistinguishable \textsc{pid}s, but they are still distinguishable in terms of \textsc{lid}. Hence, as indicated in the lexical entry in
Figure~\ref{fig:BonamiCrysmann:fillette}, the derived noun \textsc{fillette} adds diminutive semantics (\emph{dim-rel}) to the semantics of its base which is constrained to be that lexeme with \textsc{lid} \emph{girl-rel}, i.e., the left-hand lexeme in Figure~\ref{fig:BonamiCrysmann:fille:two}. This captures the notion of formal lexical identity at the level of \textsc{pid} while implementing Fradin and
Kerleroux's insight that derivational morphology operates on fully
specific rather than underspecified lexemes.

\begin{figure}[htb]
\centering\smaller
\begin{avm}
\[ \asort{lexeme} ss & \[cat|hd|lid & \@1 \[\asort{dim-rel} inst & \@{x}\]\\
                       sem|restr & \{\,\@1, \@2\,\}
                     \]\\
   pid & fillette-pid\\%\[\asort{pid} stems & \<\normalfont/fijɛt/\>\]\\
   m-dtrs & \<\[ \asort{lexeme}
                 ss & \[cat|hd|lid & \@2 \[\asort{girl-rel} inst & \@{x}\]\\
                        sem|restr & \{\,\@2\,\}
                      \]\\
                 pid & fille-pid\\%\[\asort{pid} stems & \<\normalfont/fij/\>\]
              \]\>
\]
\end{avm}
\caption{Proposed lexical entry for the lexeme \textsc{fillette}
  `small girl'.\label{fig:BonamiCrysmann:fillette}}
\end{figure}


\subsection{Individuating flexemes: stem spaces}
\is{stem!stem space|(}
\il{French|(}

We now turn to the nature of \textit{pid} objects. Evidently, there
should be enough distinct \textsc{pid} values to be able to distinguish
each flexeme from one another; that is necessary and sufficient to
capture the notion of a flexeme.  In the context of a typed-feature
structure ontology, however, it is very natural to use \textsc{pid} to
capture all aspects of inflectional identity. We thus take
\textsc{pid}s to be structured objects providing enough phonological
and inflectional information to deduce a whole paradigm with minimal
redundancy: Hence, at the very least, for the simplest inflectional
systems, a basic stem. For systems of any complexity, this basic
information needs to be supplemented with inflection class information
(if there is more than one inflectional strategy) and information on
stem alternants (if there are unpredictable stem alternations).

We illustrate a simple approach to the encoding of stem alternations
by adapting the HPSG analysis of French conjugation presented in
\citet{Bonami06}. French verbs exhibit pervasive stem alternations,
illustrated in Table~\ref{tab:BonamiCrysmann:fr-vs} in the indicative present
subparadigms. Regular verbs from the first conjugation use a uniform
stem in the present, and regular verbs from the second conjugation use
an augmented stem in /-s/ in the plural. In addition to these two
patterns, however, there are hundreds of irregular verbs instantiating
others, which can be grouped into three types: either there is one stem
for the singular and one for the plural, or the same stem is used for
the singular and for the third plural, or three different stems are
used following the pattern illustrated by
\textsc{boire}.\footnote{\citet{Bonami06} deliberately set apart a
  handful of highly irregular and very frequent verbs instantiating an
  unpredictable form in the \textsc{1sg}, \textsc{1pl} or
  \textsc{2pl}.}


\begin{table}[htb]
\fittable{
\begin{tabular}{llllllll}
\lsptoprule
& \scshape 1sg & \scshape 2sg & \scshape 3sg & \scshape 1pl & \scshape 2pl & \scshape 3pl\\
\midrule
\textsc{laver} `wash' & lav & lav & lav & lav-ɔ̃ & lav-e & lav & (1st conjugation)\\
\textsc{finir} `finish' & fini & fini & fini & finis-ɔ̃ & finis-e & finis & (2nd conjugation)\\
\midrule
\textsc{envoyer} `send' & ɑ̃vwa & ɑ̃vwa & ɑ̃vwa & ɑ̃vwaj-ɔ̃ & ɑ̃vwaj-e & ɑ̃vwa\\
\textsc{joindre} `join' & ʒwɛ̃ & ʒwɛ̃ & ʒwɛ̃ & ʒwaɲ-ɔ̃ & ʒwaɲ-e  & ʒwaɲ & (other patterns)\\
\textsc{boire} `drink' & bwa & bwa & bwa & byv-ɔ̃ & byv-e & bwav\\
\lspbottomrule
\end{tabular}
}
\caption{Sample French present indicative paradigms illustrating recurrent stem alternation patterns\label{tab:BonamiCrysmann:fr-vs}}
\end{table}

Given the pervasive nature of these alternations and the general
unpredictability of the shapes of the alternants, \citet{Bonami03a}
build on previous work by
\citet{Aronoff94,Brown98,Hippisley98}, and \citet{Stump01}, and posit that each lexeme
is associated with a \textsc{stem space}, a vector of phonological
shapes indicating the shape of the stem used in some zone of the
paradigm. Limiting attention again to the stems found in the indicative
present, the stem space of the verbs under consideration is indicated
in Table~\ref{tab:BonamiCrysmann:ss}: Stem 1 the default stem, Stem 2 is used in the
\textsc{3pl}, and Stem 3 is used in the singular.

\begin{table}[htb]

\begin{tabular}{lllllll}
\lsptoprule
& Stem 1 & Stem 2 & Stem 3\\
\midrule
\textsc{laver} `wash' & lav & lav & lav\\
\textsc{finir} `finish' & finis & finis & fini\\
\midrule
\textsc{envoyer} `send' & ɑ̃vwaj & ɑ̃vwa & ɑ̃vwa\\
\textsc{joindre} `join' & ʒwaɲ & ʒwaɲ& ʒwɛ̃\\
\textsc{boire} `drink' & byv & bwav & bwa\\
\lspbottomrule
\end{tabular}
\caption{Stem spaces for a sample of French verbs in the present indicative\label{tab:BonamiCrysmann:ss}}
\end{table}

\newpage 
In the context of an Item-and-Process view of inflection,
\citet{Bonami06} propose to encode stem spaces as the value of a
feature carried by lexemes, and posit a hierarchy\is{type hierarchy} of stem space types
capturing different patterns of identity among coordinates in the stem
space. This analysis can be readily adapted to the current framework by
assuming that stem spaces are represented inside \textit{pid} objects
using a list-valued feature \textsc{stems}. Let us first consider the
lexical entry of \textsc{boire} `drink'. This needs to list three
unpredictable stems, as indicated in Figure~\ref{fig:BonamiCrysmann:boire}.

\begin{figure}[htb]
\centering\smaller
\begin{avm}
\[\asort{lexeme}
\avmspan{ss|cat|hd\;\[\asort{verb} lid & drink-rel\]}\\
  pid & \[\asort{boire-pid} stems & \<\normalfont/byv/,/bwav/,/bwa/\>\]
\]
\end{avm}
\caption{Lexical entry for \textsc{boire} `drink'\label{fig:BonamiCrysmann:boire}}
\end{figure}

The grammar then needs to specify in which context each element in
\textsc{stems} is to be used. Following insights from
\citet[chap.~6]{Stump01}, we assume that this is effected by
\textsc{stem selection rules}, a special kind of realisation rule that
selects a stem alternant for insertion. The relevant rules are
presented in Figure~\ref{fig:BonamiCrysmann:ssr}.\footnote{We use the em dash (`—')
  to denote an unconstrained string of segments. `—' in a
  \textsc{stems} value thus indicates that the shape of that stem is
  not constrained by the rule, type, or lexical entry under
  consideration.}


\begin{figure}[htb]
\smaller\centering
\begin{avm}
\[	mph & \{\,\[ph & \@1\\ pc & 0\]\,\}\\
    mud & \{\,\[\asort{pid} stems & \<\@1,\ldots\>\]\,\}
\]~~~~%
\[	mph & \{\,\[ph & \@1\\ pc & 0\]\,\}\\
    mud & \{\,\[\asort{pid} stems & \<—,\@1,—\>\]\,\}\\
    ms & \{\,\[per & 3\\ num & pl\],\ldots\,\}
\]~~~~%
\[	mph & \{\,\[ph & \@1\\ pc & 0\]\,\}\\
    mud & \{\,\[\asort{pid} stems & \<—,—,\@1\>\]\,\}\\
    ms & \{\,\[num & sg\],\ldots\,\}
\]
\end{avm}
\caption{Stem selection rules for French  present indicative\label{fig:BonamiCrysmann:ssr}}
\end{figure}

The first rule states that, by default, lexical identity
(i.e.~\textsc{pid}) is realised by inserting the first element on the
\textsc{stems} list as a morph in position 0.\footnote{This rule can
  be thought of as capturing an inflectional universal, as it simply
  states that some stem must be provided for every word. In systems
  without unpredictable stem allomorphy, this will be the sole element
  on the \textsc{stems} list. In systems with stem allomorphy, by
  convention, we place the default stem alternant first.} The two
other rules add some allomorphic conditioning: the second element is
only used if the morphosyntactic context is that of a \textsc{3pl}
subject, while the third is used when it is that of a \textsc{sg}
subject.

\largerpage
Note that the stem selection rules are in no way sensitive to
inflection class. This is in keeping with Bonami and Boyé's (\citeyear{Bonami2003,
Bonami06}) analysis, which starts from the assumption that all variation in
French conjugation originates in differential distributions of
alternants in the stem space. That being said, it is useful to
characterise classes of flexemes in terms of the patterns of identity
they instantiate. In the present context, such a classification can be
stated in the form of a type hierarchy of  \textit{pid} objects, as
indicated in Figure~\ref{fig:BonamiCrysmann:hier:fr}.


\begin{figure}[htb]
\centering\smaller
\avmoptions{top}
\begin{avmtree}
\itshape
\psset{linewidth=.6pt}
\br{\upshape\itshape pid}{
\br{\rnode{y}{\[\asort{AAx-pid} stems & \<\@1,\@1,—\>\]}}{\lf{AAB-pid}
\lf{\begin{avm}\[\asort{reg-II-pid} stems & \<\normalfont\@1+/is/,\@1+/is/,\@1+/i/\>\]\end{avm}}}
\lf{full-irreg-pid}
\br{\[\asort{xBB-pid} stems & \<—,\@2,\@2\>\]}{\lf{\rnode{x}{reg-I-pid}}
	\lf{ABB-pid}
	}
}
\ncdiag[arm=0pt,nodesep=1pt,angleA=-90,angleB=90]{y}{x}
\end{avmtree}
\caption{Hierarchy of \emph{pid} subtypes capturing aspects of the French verbal stem space\label{fig:BonamiCrysmann:hier:fr}}
\end{figure}



The hierarchy\is{type hierarchy} of \textit{pid} objects highlights the structure of the
system, and allows the grammar writer to minimise redundancy in the
stamement of lexical entries. In particular, all regular verbs can be
described with mention of the first stem only, while different types
of irregulars necessitate information on two or more stems in
different coordinates of the stem space. More sample lexical entries
are provided in Figure~\ref{fig:BonamiCrysmann:more:le} for illustration. Note that
the lexical entry for \textsc{boire} of Figure~\ref{fig:BonamiCrysmann:boire} does
not need to mention  a subtype of \emph{pid} explicitly, since
\emph{full-irreg-pid} is the only subtype compatible with the listing
of three distinct stems.

\begin{figure}[htb]
\centering\smaller
\begin{tabular}{cc}
\begin{avm}
\[\asort{lexeme}\avmspan{ss|cat|hd\;\[\asort{verb} lid & wash-rel\]}\\
  pid & \[\asort{reg-I-pid} stems & \<\normalfont/lav/,–,–\>\]
\]
\end{avm}
&
\begin{avm}
\[\asort{lexeme}\avmspan{ss|cat|hd\;\[\asort{verb} lid & finish-rel\]}\\
  pid & \[\asort{reg-II-pid} stems & \<\normalfont/finis/,–,–\>\]
\]
\end{avm}
\\ \\
\begin{avm}
\[\asort{lexeme}\avmspan{ss|cat|hd\;\[\asort{verb} lid & send-rel\]}\\
  pid & \[\asort{ABB-pid} stems & \<\normalfont/ɑ̃vwaj/,/ɑ̃vwa/,–\>\]
\]
\end{avm}
&
\begin{avm}
\[\asort{lexeme}\avmspan{ss|cat|hd\;\[\asort{verb} lid & join-rel\]}\\
  pid & \[\asort{AAB-pid} stems & \<\normalfont/ʒwaɲ/,–,/ʒwɛ̃/\>\]
\]
\end{avm}
\end{tabular}
\caption{Lexical entries for  a sample of French verbs\label{fig:BonamiCrysmann:more:le}}
\end{figure}

\largerpage 
Finally, the distinction between \textsc{pid} types and stem inventories provides a simple account of situations where two verbs belonging to different stem alternation types have the same basic stem, as is the case e.g. with \textsc{tapir} `hide' and \textsc{tapisser} `paper', wich have both have a basic stem /tapis/, witness the ambiguous \textsc{prs.1pl} /tapisɔ̃/ `we hide'/`we paper'. Figure~\ref{fig:BonamiCrysmann:tapis} shows the relevant lexical entries.
\newpage 

\begin{figure}[t]
\centering\smaller
\begin{tabular}{cc}
\begin{avm}
\[\asort{lexeme}\avmspan{ss|cat|hd\;\[\asort{verb} lid & paper-rel\]}\\
  pid & \[\asort{reg-I-pid} stems & \<\normalfont/tapis/,–,–\>\]
\]
\end{avm}
&
\begin{avm}
\[\asort{lexeme}\avmspan{ss|cat|hd\;\[\asort{verb} lid & hide-rel\]}\\
  pid & \[\asort{reg-II-pid} stems & \<\normalfont/tapis/,–,–\>\]
\]
\end{avm}
\end{tabular}
\caption{Lexical entries for two French verbs with homophonous basic stems\label{fig:BonamiCrysmann:tapis}}
\end{figure}


  
To sum up then, \textsc{pid} provides a natural locus for the
representation of lexical information on stem alternations, and allows
for a natural encoding of Bonami and Boyé's notion of a stem space. In
addition, in a system where (by hypothesis) all variation in
inflection is located in the stems, the indication of a specific
vector of stem alternants is sufficient to fully individuate
flexemes. In such a system, the hierarchy of \textit{pid} values is
merely used to limit the statement of redundant information in lexical
entries.
\is{stem!stem space|)}
\il{French|)}

\subsection{Individuating flexemes: affixal inflection classes}
\label{sec:3:4}
\is{inflection!inflection class|(}
\il{Czech|(}

We now turn to the role of \textsc{pid}  in a system with
nontrivial affixal inflection classes. As an illustration, let us
examine a subset of the Czech nominal declension
system. Table~\ref{tab:BonamiCrysmann:CzechNoun} provides partial paradigms for four
nouns belonging to four of the major inflection classes of masculine
inanimate and neuter nouns.


\begin{table}[htb]
\centering\smaller
%\begin{tabular}{ll|ll|ll|ll}
%\lsptoprule
%& & \multicolumn{2}{c}{\sc masculine} & \multicolumn{2}{c}{\sc feminine} & \multicolumn{2}{c}{\sc neuter}\\
%& & hard &  soft & hard &  soft & hard &  soft\\
%\midrule
%\multirow{4}{*}{\sc sg} & \sc nom & most & pokoj & žen-a & růž-e &  měst-o & moř-e\\
%& \sc gen & most-u &  pokoj-e & žen-y & růž-e	& měst-a &  moř-e\\
%& \sc dat & most-u &  pokoj-i & žen-ě & růž-i & měst-u &  moř-i\\
%& \sc acc & most &  pokoj & žen-u & růž-i & měst-o &  moř-e\\
%\midrule
%\multirow{4}{*}{\sc pl} & \sc nom & most-y &  pokoj-e & žen-y & růž-e &  měst-a &  moř-e\\
%& \sc gen & most-ů  & pokoj-ů & žen & růž-í& měst & moř-í\\
%& \sc dat & most-ům &  pokoj-ům & žen-ám & růž-ím & měst-ům &  moř-ím\\
%& \sc acc & most-y &  pokoj-e & žen-y & růž-e & měst-a &  moř-e\\
%\midrule
%& & `bridge' &  `room' & `woman' & `rose' & `town' &  `sea'\\
%\lspbottomrule
%\end{tabular}
\begin{tabular}{ll|ll|ll}
\lsptoprule
& & \multicolumn{2}{c}{\scshape masculine} &  \multicolumn{2}{c}{\scshape neuter}\\
& & hard &   soft & hard &  soft\\
\midrule
\multirow{4}{*}{\scshape sg} & \scshape nom & most & pokoj  &  měst-o & moř-e\\
& \scshape gen & most-u &  pokoj-e &  měst-a &  moř-e\\
& \scshape dat & most-u &  pokoj-i & měst-u &  moř-i\\
& \scshape acc & most &  pokoj &  měst-o &  moř-e\\
& \scshape voc & most-e & pokoj-i & měst-o & moř-e\\
& \scshape loc & most-ě & pokoj-i & měst-ě & moř-i\\
& \scshape ins & most-em & pokoj-em & měst-em & moř-em\\
\midrule
\multirow{4}{*}{\scshape pl} & \scshape nom & most-y &  pokoj-e &   měst-a &  moř-e\\
& \scshape gen & most-ů  & pokoj-ů & měst & moř-í\\
& \scshape dat & most-ům &  pokoj-ům &  měst-ům &  moř-ím\\
& \scshape acc & most-y &  pokoj-e &  měst-a &  moř-e\\
& \scshape voc & most-y & pokoj-e & měst-a & moř-e\\
& \scshape loc & most-ech & pokoj-ích & měst-ech &moř-ích\\
& \scshape ins & most-y & pokoji & měst-y & moř-i\\
\midrule
& & `bridge' &  `room' &  `town' &  `sea'\\
\lspbottomrule
\end{tabular}
\caption{Partial declension for the four inflection classes of Czech inanimate nouns}
\label{tab:BonamiCrysmann:CzechNoun}
\end{table}

The distinction between hard and soft declension is correlated with
the phonological properties of the stem-final consonant; however, it is
not in general possible to categorically predict whether a noun will
belong to a hard or soft declension on the basis of the phonological
shape of its stem.  Groups of declensions do share characteristics of exponence; in particular, it is evident from the table that some
exponent strategies are common to the soft declensions (e.g. \emph{-e}
marking the \textsc{gen.sg}), to the masculine declensions
(e.g. \emph{-ů} in the \textsc{gen.pl}), or to larger groups of
declensions (e.g. \emph{-ům} is used in the \textsc{dat.pl} accross
the declensions shown here, except in the soft neuter). These observations
motivate arranging flexemes in a hierarchy\is{type hierarchy} of classes, so that the
application of rules of exponence can be restricted to arbitrary
collections of declension classes. We thus propose a simpler hierarchy\is{type hierarchy}
of \textit{pid} objects reflecting the distinction between hard and
soft declensions, as indicated in Figure~\ref{fig:BonamiCrysmann:cz:hier}.

\begin{figure}[htb]
\centering\smaller
\avmoptions{top}
\begin{tree}
\itshape
\psset{linewidth=.6pt,arm=0pt,nodesep=1pt,angleA=-90,angleB=90,treesep=5em}
\br{pid}{
	\lf{\rnode{h}{hard-pid}}
	\lf{\rnode{s}{soft-pid}}
}
\end{tree}
\caption{Premiminary hierarchy of \emph{pid} subtypes for Czech declension\label{fig:BonamiCrysmann:cz:hier}}
\end{figure}

In addition, we propose that, since gender is inherent for nouns (in
contrast to agreement gender) yet still conditions inflectional
realisation, it should be represented as part of \textsc{pid}. Hence
the lexical entries of the 4 nouns under consideration are as
indicated in Figure~\ref{fig:BonamiCrysmann:cz:le}. Note that traditional declensions
correspond to a combination of a \textit{pid} subtype and a gender
value.\footnote{This bidimensional representation of declension
  classes is possible because gender is a strict predictor of
  inflection class in Czech: all members of each declension class
  belong to the same gender. Some declension classes corresponding to
  different genders are very similar, but always differ in at least
  one paradigm cell: e.g. masculine \textsc{táta} `dad' inflects like a
  feminine hard noun in only about half of its paradigm cells. Also
  note that a full description of the system would require more
  subtypes of \emph{pid}, as there are more than two classes per
  gender, and hence organizing the \emph{pid} hierarchy\is{type hierarchy} as a dense
  semi-lattice of inflection class groupings \citep{Beniamine16}.}


\begin{figure}[htb]
\centering\smaller
\begin{tabular}{cc}
\begin{avm}
\[\asort{lexeme} \avmspan{ss|cat|hd\;\[\asort{noun} lid & bridge-rel\]}\\
  pid & \[\asort{hard-pid} gen & mas\\ stems & \<\normalfont/most/\>\]
\]
\end{avm}
&
\begin{avm}
\[\asort{lexeme} \avmspan{ss|cat|hd\;\[\asort{noun} lid & room-rel\]}\\
  pid & \[\asort{soft-pid} gen &mas\\ stems & \<\normalfont/pokoj/\>\]
\]
\end{avm}
\\ \\
\begin{avm}
\[\asort{lexeme} \avmspan{ss|cat|hd\;\[\asort{noun} lid & town-rel\]}\\
  pid & \[\asort{hard-pid} gen &neu\\ stems & \<\normalfont/měst/\>\]
\]
\end{avm}
&
\begin{avm}
\[\asort{lexeme} \avmspan{ss|cat|hd\;\[\asort{noun} lid & sea-rel\]}\\
  pid & \[\asort{soft-pid} gen &neu\\ stems & \<\normalfont/moř/\>\]
\]
\end{avm}
\end{tabular}
\caption{Preliminary lexical  entries for  a sample of Czech nouns\label{fig:BonamiCrysmann:cz:le}}
\end{figure}









\newpage
To see how this hierarchy\is{type hierarchy} helps in capturing the distribution of
exponents in Czech, consider the partial hierarchy\is{type hierarchy} of rules of exponence for the
expression of \textsc{gen.sg} in Figure~\ref{fig:BonamiCrysmann:cz:re}.  The three
rules have the same general structure: they associate a specific
phonological shape with the expression (through the \textsc{mud}
value) of the \textsc{gen.sg}, but place a condition on that
expression by restricting the \textsc{ms} value to contain specific
information in its \textit{pid} value. That is, they limit the use of
an exponent to flexemes belonging to a particular inflection class or
group of inflection classes. The first two rules express the
conditioning in terms of both a type in the \textit{pid} hierarchy\is{type hierarchy}
 and a gender value.  The third one, however, does not mention
gender, and hence can apply both in the case of masculine and neuter
soft nouns.


\begin{figure}[t]
\centering\smaller
\avmoptions{top}
\begin{avmtree}
\itshape
\psset{linewidth=.6pt,arm=0pt,nodesep=1pt,angleA=-90,angleB=90,treesep=3em,levelsep=12ex}
\br{\upshape\itshape rln-rule}{\psset{levelsep=22ex}
	\br{\[	\asort{gs-rule}
			mph & \{\,\[pc & 2\]\,\}\\
            mud & \{\,\[case & gen\\ num & sg\]\,\}
        \]
        }{
		\lf{\begin{avm}
			\[	mph & \{\,\[ph & \<\normalfont/u/\>\]\,\}\\
    			ms & \{\,\[\asort{hard-pid} gen & mas\], \ldots\,\}
			\]\end{avm}}
		\lf{\begin{avm}
			\[	mph & \{\,\[ph & \<\normalfont/a/\>\]\,\}\\
    			ms & \{\,\[\asort{hard-pid} gen & neu\], \ldots\,\}
			\]\end{avm}}
		\lf{\begin{avm}
			\[	mph & \{\,\[ph & \<\normalfont/e/\>\]\,\}\\
    			ms & \{\,\upshape\itshape soft-pid, \ldots\,\}
			\]\end{avm}}
		}
	}
\end{avmtree}
\caption{Preliminary realisation rules for Czech \textsc{gen.sg}\label{fig:BonamiCrysmann:cz:re}}
\end{figure}


This simple example illustrates how the typed feature structure
architecture allows for a straightforward statement of generalisations
on exponence across declension types by locating inherent
inflectional information in \textsc{pid} values and conditioning the
application of rules of exponence to families of possible \textsc{pid}
values.
%\textcolor{red}{In addition, this architecture provides a natural way of
%accounting for situations where two flexemes share the same stem
%inventory but have distinct paradigms. As a case in point, the two masculine Czech nouns \textsc{u\v{r}ad} `administration' and \textsc{u\v{r}ada} `administrator' have the same stem \emph{u\v{r}ad-} but belong to distinct inflection classes, specifically two subtypes of the hard masculine declension. This can be captured by assigning \textsc{u\v{r}ad} the same \textsc{pid} type as \textsc{most} `bridge' and \textsc{u\v{r}ada} the same \textsc{pid} type as \textsc{táta} `daddy'. More generally, any difference in inflection
%that is not reducible to the shape of a stem or a vector of stem
%alternants can be captured by making some distinction on the type of
%\textsc{pid}.}

We conclude this section by noting that the use of stem spaces,
inherent features such as gender, and type of \textsc{pid} does not
necessarily exhaust the inventory of relevant information that should
be coded inside \textsc{pid} for the languages of the world.  For
instance, \citet{Bonami11e} proposed that lexical information on
thematic suffixes in the conjugation of the Kartvelian language Laz
should be stored as the value of a dedicated feature inside the
\textsc{pid}, since information on the shape of the thematic affix
needs to be lexically stipulated but the affix is neither always
present nor always contiguous to the root; and \citet{Crysmann17} propose a concrete implementation of that idea in the context of Estonian declension.
Our general claim is that
\textsc{pid} should be the sole locus of lexically stipulated
information on inflection.
\is{inflection!inflection class|)}
\il{French|)}

\section{Flexemes and overabundance}
\label{sec:4}

\is{overabundance|(}
\is{heteroclisis|(}

In previous sections we have justified the distinction between lexemes
and flexemes by arguing that a single flexeme (characterised by a
single inflectional paradigm) may correspond to multiple lexemes
(characterised by different lexical semantic and/or syntactic
properties). In this final section we explore situations where one may
want to argue the opposite: multiple flexemes corresponding to a
single lexeme.

Although we have not made use of it yet, the analytic scheme defined
in the previous section certainly leaves room for such a
possibility. Both for French verbs and Czech nouns, we have proposed
that \textit{pid} objects be organised in a hierarchy\is{type hierarchy}, capturing
families of inflectional behavior. The lexical entries used thus far
all introduce a \textsc{pid} value corresponding to a specific leaf
type in the hierarchy: hence one flexeme for each lexeme. However, if
some lexical entries were to refer to some \textit{pid} supertype,
this would authorise multiple inflectional behaviours for the same
lexeme~--~hence, in a sense, multiple flexemes for one lexeme.

As a matter of fact, both French conjugation and Czech declension
provide examples of phenomena that are insightfully analysed in this
fashion. The phenomena at hand fall under the general heading of
\textsc{overabundance} \citep{Thornton11,Thornton12,Thornton17}, that
is, of situations where a single lexeme has multiple realisations for
the same set of morphosyntactic properties.


\is{stem!stem space|(}
First consider the French verb \textsc{asseoir}. There is considerable
variation in the realisation of different paradigm cells of this verb,
leading to free variation at least for some paradigm cells in some
varieties \citep{Bonami10}. Limiting ourselves again to the indicative
present, there seem to be two equally felicitous forms for each
person-number combination in Standard French, as indicated in
Table~\ref{tab:BonamiCrysmann:asseoir}.

\begin{table}[htb]

\begin{tabular}{lllllll}
\lsptoprule
\scshape 1sg & \scshape 2sg & \scshape 3sg & \scshape 1pl & \scshape 2pl & \scshape 3pl\\
\midrule
aswa & aswa & aswa & aswaj-ɔ̃ & aswaj-e & aswa\\
asje & asje & asje & asɛj-ɔ̃ & asɛj-e & asɛj\\
\lspbottomrule
\end{tabular}
\caption{The two main indicative present subparadigm of \textsc{asseoir} `sit'\label{tab:BonamiCrysmann:asseoir}}
\end{table}

Although this situation could be described in terms of overabundance
in individual paradigm cells, such an approach would not capture the
fact that the forms seem to be organised in two distinct paradigms,
each with two stem alternants, and each instantiating a different but
familiar pattern of stem allomorphy: the /aswa/~/aswaj/ contrast
follows an ABB pattern similar to that of \textsc{envoyer} (see
Table~\ref{tab:BonamiCrysmann:fr-vs}), while the /asje/~/asɛj/ contrast follows an AAB
pattern similar to that of \textsc{joindre}. It is thus more
perspicuous to describe this case of overabundance as involving two
different stem spaces, and hence two different \textsc{pid} values,
rather than variation in individual paradigm
cells. Figure~\ref{fig:BonamiCrysmann:asseoir:LE} shows two appropriate lexical
entries corresponding to the two paradigms of \textsc{asseoir} that
readily integrate with the analysis presented in Section~\ref{sec:3} and account
for overabundance directly.

\begin{figure}[htb]
\smaller\centering
\begin{avm}
\[\asort{lexeme} \avmspan{ss|cat|hd\;\[\asort{verb} lid & sit-rel\]}\\
  pid & \[\asort{AAB-pid} stems & \<\normalfont/asej/,—,/asje/\>\]
\]
\end{avm}
~~~
\begin{avm}
\[\asort{lexeme} \avmspan{ss|cat|hd\;\[\asort{verb} lid & sit-rel\]}\\
  pid & \[\asort{ABB-pid} stems & \<\normalfont/aswaj/,/aswa/,—\>\]
\]
\end{avm}
\caption{Lexical entries for two variants of the verb \textsc{asseoir}
  `sit'\label{fig:BonamiCrysmann:asseoir:LE}}
\end{figure}

The French verb \textsc{asseoir} exemplifies a case of stem-based
overabundance, which is readily accommodated by having two stem spaces
for a single lexeme.  \is{stem!stem space|)} Let us now turn to Czech and discuss a situation
of exponent-based overabundance.\il{Czech|(}
% This
% contrasts more naturally with stem

\is{inflection!inflection class|(}
In Section~\ref{sec:3:4} we discussed the fact that the Czech inflection system
distinguishes `hard' and `soft' declensions. As it happens, some
lexemes follow a hybrid or `mixed' pattern that does not clearly fall
into one type or the other, but rather makes use of both hard and soft
exponents. However, this has different manifestations for neuter and
masculine inanimate nouns, as evidenced by the examples in
Table~\ref{tab:BonamiCrysmann:cz:ov}.



\begin{table}[htb] 
\setlength{\tabcolsep}{4pt}
\fittable{
\begin{tabular}{ll@{\qquad}ll@{$\sim$}ll@{\qquad\qquad}lll}
\lsptoprule
& &&  \multicolumn{2}{c}{\scshape masculine} &&& \multicolumn{1}{l}{\scshape neuter}&\\
& & hard & \multicolumn{2}{c}{mixed} & soft & hard & mixed & soft\\
\midrule
\multirow{4}{*}{\scshape sg} & \scshape nom & most & \multicolumn{2}{c}{pramen} & pokoj & měst-o & kuř-e & moř-e\\
& \scshape gen & most-u & pramen-u&pramen-e & pokoj-e & měst-a & kuř-et-e & moř-e\\
& \scshape dat & most-u & pramen-u&pramen-i & pokoj-i & měst-u & kuř-et-i & moř-i\\
& \scshape acc & most & \multicolumn{2}{c}{pramen} & pokoj & měst-o & kuř-e & moř-e\\
& \scshape voc & most-e & pramen-e & pramen-i & pokoj-i & měst-o & kuř-e & moř-e\\
& \scshape loc & most-ě & pramen-u & pramen-i & pokoj-i & měst-ě & kuř-et-i & moř-i\\
& \scshape ins & most-em & \multicolumn{2}{c}{pramen-em} & pokoj-em & měst-em & kuř-et-em & moř-em\\
\midrule
\multirow{4}{*}{\scshape pl} & \scshape nom & most-y & \multicolumn{2}{c}{pramen-y} & pokoj-e & měst-a & kuř-at-a & moř-e\\
& \scshape gen & most-ů & \multicolumn{2}{c}{pramen-ů} & pokoj-ů & měst & kuř-at & moř-í\\
& \scshape dat & most-ům & \multicolumn{2}{c}{pramen-ům} & pokoj-ům & měst-ům & kuř-at-ům & moř-ím\\
& \scshape acc & most-y & \multicolumn{2}{c}{pramen-y} & pokoj-e & měst-a & kuř-at-a & moř-e\\
& \scshape voc & most-y & \multicolumn{2}{c}{pramen-y} & pokoj-e & měst-a & kuř-at-a & moř-e\\
& \scshape loc & most-ech & \multicolumn{2}{c}{pramen-ech} & pokoj-ích & měst-ech & kuř-at-ech & moř-ích\\
& \scshape ins & most-y & \multicolumn{2}{c}{pramen-y} & pokoji & měst-y & kuř-at-y & moř-i\\
\midrule
& & `bridge' & \multicolumn{2}{c}{`spring'} & `room' & `town' & `chicken' & `sea'\\
\lspbottomrule
\end{tabular}
}
\caption{Overabundance and Heteroclisis in Czech declension}
\label{tab:BonamiCrysmann:cz:ov}
\end{table}

The paradigm of the mixed neuter noun \textsc{kuře} `chicken' exhibits
\textsc{heteroclisis} \citep{Stump06}: \textsc{kuře} inflects like a
soft noun in the singular, but like a hard noun in the plural. By
contrast, the paradigm of the mixed masculine noun \textsc{pramen}
`spring' exhibits a combination of heteroclisis and partial
overabundance. In the plural, \textsc{pramen} inflects like a hard
noun; in the singular, it may inflect either like a hard noun or like
a soft noun. Correctly capturing the difference between these two
types of mixed inflectional behaviour is a serious challenge for any
theory of inflection.

Both behaviours are readily accomodated in the present framework,
using a more refined hierarchy\is{type hierarchy} of \textsc{pid} values. The crucial
insight is that overabundance amounts to ambiguity, i.e. disjunctive
membership of two inflection classes, whereas heteroclisis involves
simultaneous membership of two classes: while the former is modelled
straightforwardly by means of \isi{underspecification}, corresponding to the
\textsc{join} in the semi-lattice of \textit{pid} types, the latter
can be captured by overspecification, i.e.  the \textsc{meet}, as
shown by the type hierarchy\is{type hierarchy} in Figure~\ref{fig:BonamiCrysmann:cz:hier:bis}.


\begin{figure}[t]
\centering\smaller
\itshape

\rnode{s}{pid}

\vspace*{1.5\baselineskip}
\rnode{nh}{hard-pid}~~~~~~~~~~~
\rnode{ns}{soft-pid}\ncdiag{n}{ns}

\vspace*{1.5\baselineskip}
\rnode{a}{\itshape strict-hard-pid}~~~~%
\rnode{b}{\itshape mixed-pid}~~~~%
\rnode{c}{\itshape strict-soft-pid}
\ncdiag[arm=0pt,nodesep=1pt,angleA=-90,angleB=90]{s}{nh}\ncdiag[arm=0pt,nodesep=1pt,angleA=-90,angleB=90]{s}{ns}
\ncdiag[arm=0pt,nodesep=1pt,angleA=-90,angleB=90]{nh}{a}
\ncdiag[arm=0pt,nodesep=1pt,angleA=-90,angleB=90]{ns}{c}
\ncdiag[arm=0pt,nodesep=1pt,angleA=-90,angleB=90]{nh}{b}
\ncdiag[arm=0pt,nodesep=1pt,angleA=-90,angleB=90]{ns}{b}
\normalfont
\caption{Improved hierarchy of \emph{pid} subtypes capturing heteroclite Czech declension classes\label{fig:BonamiCrysmann:cz:hier:bis}}
\end{figure}


Figure~\ref{schema} shows schematically to which \textsc{pid} value each
noun is assigned, and Figure~\ref{schemabis} which \textsc{pid} value
rules of exponence for the \textsc{gen.sg} (left hand side) and
\textsc{nom.pl} (right hand side) are restricted to. More detailed
lexical entries and rules of exponence are presented below in
Figures~\ref{fig:BonamiCrysmann:cz:LE:bis} and~\ref{fig:BonamiCrysmann:cz:rr:bis}. Any noun can be
inflected using a realisation rule declared with a compatible
\textsc{pid} value. That is, any point in the hierarchy\is{type hierarchy} that is
identical to that of the noun, dominates it, or is dominated by
it.

\begin{figure}[p]
\centering\smaller
\itshape
\newcommand{\lxnode}[2]{\rnode{#1}{\psframebox[linestyle=dashed,framearc=0.2,dash = 1pt 1pt,linecolor=gray]{\upshape\scshape#2}}}

\begin{tabular}[t]{c}
~\\
\\
\lxnode{pramen}{pramen}
\\[3ex]
\lxnode{most}{most}
\\[3ex]
\lxnode{mesto}{město}
\end{tabular}
\hfill
\begin{tabular}[t]{@{}c@{}c@{}c@{}c@{}c@{}}
& & \rnode{s}{pid}\\ \\
& \rnode{nh}{hard-pid} & & \rnode{ns}{soft-pid}\\ \\
\rnode{a}{strict-hard-pid} && \rnode{b}{mixed-pid} & & \rnode{c}{strict-soft-pid}\\
\end{tabular}
\hfill
\begin{tabular}[t]{c}
~\\
\\
\lxnode{pokoj}{pokoj}
\\[3ex]
\lxnode{more}{moře}
\\[3ex]
\lxnode{kure}{kuře}
\end{tabular}
\ncdiag[arm=0pt,nodesep=1pt,angleA=-90,angleB=90]{s}{nh}\ncdiag[arm=0pt,nodesep=1pt,angleA=-90,angleB=90]{s}{ns}
\ncdiag[arm=0pt,nodesep=1pt,angleA=-90,angleB=90]{nh}{a}
\ncdiag[arm=0pt,nodesep=1pt,angleA=-90,angleB=90]{ns}{c}
\ncdiag[arm=0pt,nodesep=1pt,angleA=-90,angleB=90]{nh}{b}
\ncdiag[arm=0pt,nodesep=1pt,angleA=-90,angleB=90]{ns}{b}
\nccurve[angleA=-20,angleB=200,arrows=<->,linecolor=gray,linestyle=dashed,dash = 2pt 1pt]{pramen}{nh}
\nccurve[angleA=-20,angleB=200,arrows=<->,linecolor=gray,linestyle=dashed,dash = 2pt 1pt]{most}{a}
\nccurve[angleA=-20,angleB=225,arrows=<->,linecolor=gray,linestyle=dashed,dash = 2pt 1pt]{mesto}{a}
\nccurve[angleA=-10,angleB=190,arrows=<->,linecolor=gray,linestyle=dashed,dash = 2pt 1pt]{c}{pokoj}
\nccurve[angleA=-20,angleB=200,arrows=<->,linecolor=gray,linestyle=dashed,dash = 2pt 1pt]{c}{more}
\nccurve[angleA=-20,angleB=200,arrows=<->,linecolor=gray,linestyle=dashed,dash = 2pt 1pt]{b}{kure}


\normalfont
\caption{Schematic representation of inflection class assignment for Czech nouns\label{schema}}
\end{figure}

\begin{figure}[p] 
\centering\smaller
\itshape
\newcommand{\lxnode}[2]{\rnode{#1}{\psframebox[linestyle=dashed,framearc=0.2,dash = 1pt 1pt,linecolor=gray]{\upshape\scshape#2}}}

\begin{tabular}[t]{c}
\lxnode{y}{\textsc{m.nom.pl}~:~\upshape\itshape -y}
\\[3ex]
\lxnode{anom}{\textsc{n.nom.pl}~:~\upshape\itshape -a}
\\[3ex]
\lxnode{u}{\textsc{m.gen.sg}~:~\upshape\itshape -u}
\\[3ex]
\lxnode{agen}{\textsc{n.gen.sg}~:~\upshape\itshape -a}
\end{tabular}
\hfill
\begin{tabular}[t]{@{}c@{}c@{}c@{}c@{}c@{}}
& & \rnode{s}{pid}\\ \\
& \rnode{nh}{hard-pid} & & \rnode{ns}{soft-pid}\\ \\
\rnode{a}{strict-hard-pid} && \rnode{b}{mixed-pid} & & \rnode{c}{strict-soft-pid}\\
\end{tabular}
\hfill
\begin{tabular}[t]{c}
~\\ \\
\lxnode{egen}{\textsc{gen.sg}~:~\upshape\itshape -e}
\\[3ex]
\lxnode{enom}{\textsc{nom.pl}~:~\upshape\itshape -e}
\end{tabular}
\ncdiag[arm=0pt,nodesep=1pt,angleA=-90,angleB=90]{s}{nh}\ncdiag[arm=0pt,nodesep=1pt,angleA=-90,angleB=90]{s}{ns}
\ncdiag[arm=0pt,nodesep=1pt,angleA=-90,angleB=90]{nh}{a}
\ncdiag[arm=0pt,nodesep=1pt,angleA=-90,angleB=90]{ns}{c}
\ncdiag[arm=0pt,nodesep=1pt,angleA=-90,angleB=90]{nh}{b}
\ncdiag[arm=0pt,nodesep=1pt,angleA=-90,angleB=90]{ns}{b}
\nccurve[angleA=0,angleB=150,arrows=<->,linecolor=gray,linestyle=dashed,dash = 2pt 1pt]{y}{nh}
\nccurve[angleA=0,angleB=180,arrows=<->,linecolor=gray,linestyle=dashed,dash = 2pt 1pt]{anom}{nh}
\nccurve[angleA=0,angleB=190,arrows=<->,linecolor=gray,linestyle=dashed,dash = 2pt 1pt]{u}{nh}
\nccurve[angleA=0,angleB=210,arrows=<->,linecolor=gray,linestyle=dashed,dash = 2pt 1pt]{agen}{a}
\nccurve[angleA=-20,angleB=200,arrows=<->,linecolor=gray,linestyle=dashed,dash = 2pt 1pt]{ns}{egen}
\nccurve[angleA=-20,angleB=200,arrows=<->,linecolor=gray,linestyle=dashed,dash = 2pt 1pt]{c}{enom}

%\hfill\smaller
%\newcommand{\stack}[1]{\begin{tabular}[t]{@{}c@{}}#1\end{tabular}}
%
%\begin{minipage}{.3\textwidth}
%\centering
%\rnode{xs}{\itshape pid}
%
%\vspace*{1.5\baselineskip}
%\rnode{xnh}{\stack{\itshape hard-pid\\\textsc{m.gen.sg}~:~\itshape -u}}~~~~~~~~~~~
%\rnode{xns}{\stack{\itshape soft-pid\\\textsc{gen.sg}~:~\itshape -e}}\ncdiag{n}{ns}
%
%\vspace*{1.5\baselineskip}
%\rnode{xa}{\stack{\itshape strict-hard-pid\\\textsc{n.gen.sg}~:~\itshape -a}}~~~~%
%\rnode{xb}{\stack{\itshape mixed-pid}}~~~~%
%\rnode{xc}{\stack{\itshape strict-soft-pid}}
%\ncdiag[arm=0pt,nodesep=1pt,angleA=-90,angleB=90]{xs}{xnh}\ncdiag[arm=0pt,nodesep=1pt,angleA=-90,angleB=90]{xs}{xns}
%\ncdiag[arm=0pt,nodesep=1pt,angleA=-90,angleB=90]{xnh}{xa}
%\ncdiag[arm=0pt,nodesep=1pt,angleA=-90,angleB=90]{xns}{xc}
%\ncdiag[arm=0pt,nodesep=1pt,angleA=-90,angleB=90]{xnh}{xb}
%\ncdiag[arm=0pt,nodesep=1pt,angleA=-90,angleB=90]{xns}{xb}
%\end{minipage}
%\hfill
%\begin{minipage}{.4\textwidth}
%\centering
%\rnode{s}{\itshape pid}
%
%\vspace*{1.5\baselineskip}
%\rnode{nh}{\stack{\itshape hard-pid\\\textsc{m.nom.pl}~:~\itshape -y\\\textsc{n.nom.pl}~:~\itshape -a}}~~~~~~~~~~~
%\rnode{ns}{\stack{\itshape soft-pid}}\ncdiag{n}{ns}
%
%\vspace*{1.5\baselineskip}
%\rnode{a}{\stack{\itshape strict-hard-pid}}~~~~\rnode{b}{\stack{\itshape mixed-pid}}~~~~%
%\rnode{c}{\stack{\itshape strict-soft-pid\\\textsc{nom.pl}~:~\itshape -e}}
%
%\ncdiag[arm=0pt,nodesep=1pt,angleA=-90,angleB=90]{s}{nh}\ncdiag[arm=0pt,nodesep=1pt,angleA=-90,angleB=90]{s}{ns}
%\ncdiag[arm=0pt,nodesep=1pt,angleA=-90,angleB=90]{nh}{a}
%\ncdiag[arm=0pt,nodesep=1pt,angleA=-90,angleB=90]{ns}{c}
%\ncdiag[arm=0pt,nodesep=1pt,angleA=-90,angleB=90]{nh}{b}
%\ncdiag[arm=0pt,nodesep=1pt,angleA=-90,angleB=90]{ns}{b}
%\end{minipage}
%\hfill

\normalfont
\caption{Schematic representation of the scope of rules of exponence for Czech nouns\label{schemabis}}
\end{figure}




\begin{figure}[p]
\smaller\centering
\begin{tabular}{ccc}
\begin{avm}
\[\asort{lexeme} \avmspan{ss|cat|hd\;\[\asort{noun} lid & bridge-rel\]}\\
  pid & \[\asort{strict-hard-pid} gen & mas\\ stems & \<\normalfont/most/\>\]
\]
\end{avm}
&
\begin{avm}
\[\asort{lexeme} \avmspan{ss|cat|hd\;\[\asort{noun} lid & spring-rel\]}\\
  pid & \[\asort{hard-pid} gen & mas\\ stems & \<\normalfont/pramen/\>\]
\]
\end{avm}
&
\begin{avm}
\[\asort{lexeme} \avmspan{ss|cat|hd\;\[\asort{noun} lid & room-rel\]}\\
  pid & \[\asort{strict-soft-pid} gen & mas\\ stems & \<\normalfont/pokoj/\>\]
\]
\end{avm}
\\ \\
\begin{avm}
\[\asort{lexeme} \avmspan{ss|cat|hd\;\[\asort{noun} lid & city-rel\]}\\
  pid & \[\asort{strict-hard-pid} gen & neu\\ stems & \<\normalfont/město/\>\]
\]
\end{avm}
&
\begin{avm}
\[\asort{lexeme} \avmspan{ss|cat|hd\;\[\asort{noun} lid & chicken-rel\]}\\
  pid & \[\asort{mixed-pid} gen & neu\\ stems & \<\normalfont/kuře/\>\]
\]
\end{avm}
&
\begin{avm}
\[\asort{lexeme} \avmspan{ss|cat|hd\;\[\asort{noun} lid & sea-rel\]}\\
  pid & \[\asort{soft-pid} gen & mas\\ stems & \<\normalfont/moře/\>\]
\]
\end{avm}
\end{tabular}

\caption{Lexical entries for six Czech nouns\label{fig:BonamiCrysmann:cz:LE:bis}}
\end{figure}

\begin{sidewaysfigure}
\centering
\avmoptions{top}
\resizebox{\textwidth}{!}{
\begin{avmtree}
\itshape
\psset{linewidth=.6pt,arm=0pt,nodesep=1pt,angleA=-90,angleB=90,treesep=1em,levelsep=12ex}
\br{\upshape\itshape rln-rule}{\psset{levelsep=22ex}
	\br{\[	\asort{decl-rule}
			mph & \{\,\[pc & 2\]\,\}\\
        \]
        }{
	\br{\[	\asort{gs-rule}
            mud & \{\,\[case & gen\\ num & sg\]\,\}
        \]
        }{
		\lf{\begin{avm}
			\[	mph & \{\,\[ph & \<\normalfont/u/\>\]\,\}\\
    			ms & \{\,\[\asort{hard-pid} gen & mas\], \ldots\,\}
			\]\end{avm}}
		\lf{\begin{avm}
			\[	mph & \{\,\[ph & \<\normalfont/a/\>\]\,\}\\
    			ms & \{\,\[\asort{strict-hard-pid} gen & neu\], \ldots\,\}
			\]\end{avm}}
		\lf{\begin{avm}
			\[	mph & \{\,\[ph & \<\normalfont/e/\>\]\,\}\\
    			ms & \{\,\upshape\itshape soft-pid, \ldots\,\}
			\]\end{avm}}
		}
	\br{\[	\asort{np-rule}
            mud & \{\,\[case & nom\\ num & pl\]\,\}
        \]
        }{
		\lf{\begin{avm}
			\[	mph & \{\,\[ph & \<\normalfont/y/\>\]\,\}\\
    			ms & \{\,\[\asort{hard-pid} gen & mas\], \ldots\,\}
			\]\end{avm}}
		\lf{\begin{avm}
			\[	mph & \{\,\[ph & \<\normalfont/a/\>\]\,\}\\
    			ms & \{\,\[\asort{hard-pid} gen & neu\], \ldots\,\}
			\]\end{avm}}
		\lf{\begin{avm}
			\[	mph & \{\,\[ph & \<\normalfont/e/\>\]\,\}\\
    			ms & \{\,\upshape\itshape strict-soft-pid, \ldots\,\}
			\]\end{avm}}
		}
	}
	}
\end{avmtree}}
\caption{Realisation rules for Czech \textsc{gen.sg} and \textsc{nom.pl}  nouns\label{fig:BonamiCrysmann:cz:rr:bis}}
\end{sidewaysfigure}

%\begin{avm}
%\[	mph & \{\[ph & \<\normalfont/y/\>\\ pc & 2\]\}\\
%    mud & \{\[case & nom\\ num & pl\]\}\\
%    ms & \{\[\asort{hard-pid} gen & mas\],\ldots\}
%\]
%\end{avm}
%&
%\begin{avm}
%\[	mph & \{\[ph & \<\normalfont/e/\>\\ pc & 2\]\}\\
%    mud & \{\[case & nom\\ num & pl\\\]\}\\
%    ms & \{\normalfont\itshape{strict-soft-pid},\ldots\}
%\]
%\end{avm}
%&
%\begin{avm}
%\[	mph & \{\[ph & \<\normalfont/a/\>\\ pc & 2\]\}\\
%    mud & \{\[case & nom\\ num & pl\]\}\\
%    ms & \{\[\asort{hard-pid} gen & neu\],\ldots\}
%\]
%\end{avm}
%\end{tabular}
%\caption{Realisation rules for Czech \textsc{gen.sg} and \textsc{nom.pl}  nouns\label{fig:BonamiCrysmann:cz:rr:bis}}
%\end{figure}

As shown in Figure~\ref{schema}, nouns belonging to non-mixed
declensions are assigned to either of the two simple leaf types
\emph{strict-hard-pid} (\textsc{most, město}) and
\emph{strict-soft-pid} (\textsc{pokoj, moře}). The heteroclite noun
\textsc{kuře} is assigned to \emph{mixed-pid}, and hence may inflect
using either \emph{hard} or \emph{soft} exponents, but not
\emph{strict-hard} or \emph{strict-soft} ones. The assignment of
exponents to \emph{pid} values (shown in Figure~\ref{schemabis})  ensures that it must use soft exponents in the
  singular, yet hard exponents in the plural. By contrast, the
overabundant noun \textsc{pramen} is assigned to an underspecified
inflection class, namely \emph{hard-pid}. As such it may use any
one of \emph{hard-pid}, \emph{strict-hard-pid},
\emph{mixed-pid}, or \emph{soft-pid} exponents, but, crucially, not
\emph{strict-soft} exponents. This accounts pretty concisely
for its contrasting behaviour in the singular and the plural: since
the \textsc{gen.sg} exponent \emph{-e} is only \emph{soft-pid}, there
are two \textsc{gen.sg} exponents available for \textsc{pramen}, which
is thus overabundant: inflection with \textit{-e} by resolving
\textit{soft-pid} demanded by the rule and \textit{hard-pid} demanded
by \textsc{pramen} to the hetoroclite type \textit{mixed-pid}, or else
with \textit{-u}, by the sheer fact that this is the exponent
available for all \textit{hard-pid} words, whether strict or
heteroclite. By contrast, the \textsc{nom.pl} exponent \emph{-e} is
constrained to \emph{strict-soft}. As such, it is inaccessible to
\textsc{pramen}, which hence behaves like a simple hard masculine noun
in the plural.



We have thus established that mixed overabundant declensions can be
accommodated by assigning a lexeme to a supertype in the \textit{pid}
hierarchy\is{type hierarchy}, while mixed heteroclite declensions can be accommodated by
introducing a subtype intermediate between the hard and soft
declensions.

\is{inflection!inflection class|)}
\il{Czech|)}



 The discussion in this section has exhibited the benefits of
associating multiple \textit{pid} objects with a single \textsc{lid}
value to address some situations of overabundance; which amounts to
positing that a single lexeme may correspond to multiple flexemes.  We
by no means claim that all overabundance phenomena are best thought of
in such terms; See Thornton~(this volume) for relevant
discussion. Rather, we suggest that, where overabundance results from
a lexeme being ambiguous between two classes of paradigms, lexically
underspecified \textit{pid}s make good sense of the situation.

\is{overabundance|)}
\is{heteroclisis|)}


\section{Conclusions}

% In this paper we have addressed two issues. The first issue pertains
% to the architecture of HPSG. We have argued that, while the notion of
% a lexeme is a valuable analytic tool that can and should be given an
% explicit representation, the lexemic index or \textsc{lid} is
% sufficient for that purpose. The assumption that, in addition, there
% is a type of sign \emph{lexeme} distinct from the \emph{word}, has
% become irrelevant, in a context where \textsc{lid} accounts for
% lexemic identity and inflectional morphology is realisational. We have
% shown in particular that Information-based Morphology has no need
% for the type \emph{lexeme}. %%% BC: to be discussed

In this paper we have addressed the representation of lexical identity
in morphology. Following \citet{Fradin03b}, we have argued that a
distinction should be made between lexemes, individuated in terms of
lexical semantics, and \textsc{flexemes}, individuated in terms of
inflectional paradigms. We have then shown that lexemes and flexemes
stand in a many-to-many relation: in cases of lexical ambiguity, one
flexeme realises multiple lexemes; in at least some situations of
overabundance, multiple flexemes realise the same lexeme. We have
shown how this distinction can be integrated into Information-based
Morphology by providing words with two independent indices:
\textsc{lid} and \textsc{pid}.

The distinction between \textsc{lid} and \textsc{pid} clarifies the role
of lexical identity at the interface between inflectional morphology
and syntax: syntax cares about lexemes, but not flexemes; inflectional
morphology cares about flexemes, but not about lexemes. In the present framework, this is captured by the fact that \textsc{lid} is not
represented in \textsc{ms}, the input to rules of
inflection. Arguably, the distinction is also useful to clarify the
role of lexical identity in lexeme formation. Recent work on French
lexeme formation has highlighted the many-to-many nature of lexeme
formation rules (see \citealt[§3.1]{Bonami15b} and references
cited therein): typically, a single formal process may be associated
with multiple meanings, and the same type of meaning may be realised
by multiple processes. \citet{Bonami12t} and \citet{Tribout14} explore how the
\textsc{lid}/\emph{pid} can be used to make sense of that distinction.
In their analytic scheme, lexeme formation rules are organised in a
bidimensional multiple inheritance hierarchy\is{type hierarchy}, with one dimension
laying out formal strategies, and the other dimension describing a
syntactic/semantic operation.  Formal strategies determine a new
\emph{pid} from that of the base, while syntactic/semantic operations
amount to constructing a new \textsc{lid} from that of the base.

  
More work is needed to integrate Bonami and Tribout's insights into IbM,
but this integration paves the way towards a general, \isi{underspecification}-based
framework for morphological analysis.

\is{Head-driven Phrase Structure Grammar|)}
    
\is{morphology!Information-based Morphology|)}
\is{flexeme|)}


\section*{Acknowledgments}

We thank  Gilles Boyé and Jana Strnadová for their comments. This  work was partially supported  by a public grant
    overseen by the French National Research Agency (ANR) as part of
    the ``Investissements d'Avenir'' program (reference:
    ANR-10-LABX-0083).


{\sloppy
    \printbibliography[heading=subbibliography,notkeyword=this]
}

\end{document}

%%% Local Variables:
%%% mode: latex
%%% TeX-master: ''../main''
%%% End:
