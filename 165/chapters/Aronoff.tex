\documentclass[output=paper]{langsci/langscibook}
\ChapterDOI{10.5281/zenodo.1406987}
\title{Morphology and words: A memoir}

\author{Mark Aronoff\affiliation{Stony Brook University}}
\abstract{Lexicographers\is{lexicographer} agree with Saussure that the basic units of language are not morphemes but words, or more precisely lexemes. Here I describe my early journey from the former to the latter, driven by a love of words, a belief that every word has its own properties, and a lack of enthusiasm for either phonology or syntax, the only areas available to me as a student. The greatest influences on this development were \citeauthor{Chomsky1970}'s \emph{Remarks on Nominalization}, in which it was shown that not all morphologically complex words are compositional, and research on English \isi{word-formation} that grew out of the European philological tradition, especially the work of Hans Marchand. The combination leads to a panchronic analysis of \isi{word-formation} that remains incompatible with modern linguistic theories.}

\maketitle

\begin{document}
\selectlanguage{english}

Since the end of the nineteenth century, most academic linguistic
theories have described the internal structure of words in terms of the
concept of the \emph{morpheme}, a term first coined and defined by
%
Baudouin 
%de Courtenay (1895/1972, p. 153)
\citeauthor{Baudouin1895} (\citeyear{Baudouin1895}/1972, p. 153)%
%Baudouin de Courtenay
%
:

\begin{quote}
that part of a word which is endowed with psychological autonomy and is
for the very same reason not further divisible. It consequently subsumes
such concepts as the root (radix), all possible affixes, (suffixes,
prefixes), endings which are exponents of syntactic relationships, and
the like.
\end{quote}

This is not the traditional view of lexicographers\is{lexicographer} or lexicologists or,
surprising to many, Saussure, as %
%Anderson (2015) 
\citet{Anderson15} %
%Anderson
%
has reminded us. Since
people have written down lexicons, these lexicons have been lists of
words. The earliest known ordered word list is Egyptian and dates from
about 1500 BCE (Haring 2015). In the last half century, linguists have
distinguished different sorts of words. Those that constitute dictionary
entries are usually called \emph{lexemes}. Since the theme of this
volume is the lexeme, I thought that it might be useful to describe my
own academic journey from morphemes to lexemes. Certainly, when I began
this journey, the morpheme, both the term and the notion, seemed so
modern, so scientific, while the word was out of fashion and undefined.
Morphemes were, after all, atomic units in a way that words could never
be, and if linguistics were to have any hope of being a science, it
needed atomic units.

I grew up with morphemes. The structuralist phoneme may have fallen
victim to the generative weapons of the 1960s, but no one questioned
the validity of morphemes at MIT. They were needed to construct the
beautiful syntactic war machines that drove all before them, beginning
with the analysis of English verbs in \emph{Syntactic Structures}, which
featured such stunners as the morpheme S, which ``is singular for verbs
and plural for nouns (`comes', `boys')'' and ∅, ``the morpheme which is
singular for nouns and plural for verbs, (`boy', `come')'' %
%(Chomsky 1957,
%p. 29, fn. 3)
\citep[29, fn. 3]{Chomsky1957}%
%Chomsky
%
.

Aside from brief mentions here and there in \emph{Syntactic Structures}
and the cogent but little noted discussion at the end of Chomsky's other
masterwork, \emph{Aspects} %
%(Chomsky 1965)
\citep{Chomsky1965}%
%Chomsky
%
, by the time I arrived at MIT
as a graduate student in 1970 there was no talk of morphology; the place
was all about phonology and syntax. These two engines, which everyone
was hard at work constructing, would undoubtedly handle everything in
language worth thinking about. My problem was that I very quickly
discovered that I had little taste for either of the choices, phonology
or syntax. It was like having a taste for neither poppy seed bagels nor
sesame seed bagels, and having no other variety available at the best
bagel bakery in the world, but still wanting a bagel. This had never
happened to me before, and not just with bagels. Maybe I should go to
another store, but I liked the atmosphere in this one a lot and, like
the St. Viateur bagel shop, famous to this day
(\href{http://www.stviateurbagel.com}{www.stviateurbagel.com}), it was
acknowledged to be the best in the world.

What I did love was words. I had purchased a copy of the two-volume
compact edition of the \emph{Oxford English Dictionary} (OED) as soon as
I could scrape together the money to buy one, even though reading the
microform-formatted pages of the dictionary required a magnifying glass.
I also owned a copy of \emph{Webster's III}. I kept these dictionaries
at home, not at my desk in the department. Dictionaries and the words
they contained were my dark secret. Why should I tell anyone I owned
them? These dictionaries served no purpose in our education, where the
meanings of individual words were seldom of much use, though we did talk
a lot about the word classes that were relevant to syntax: \emph{raising
verbs, psych verbs, ditransitive verbs}. The only dictionary we ever
used in our courses was \emph{Walker's Rhyming Dictionary}, a
reverse-alphabetical dictionary of English, first published in 1775. Its
main value, as Walker had noted in his original preface, was ``the
information, as to the structure of our language, that might be derived
from the juxtaposition of words of similar terminations.'' 
%Chomsky and Halle 
\citeauthor{Chomsky68} %
%
%
had mined it extensively in their research for \emph{The Sound
Pattern of English} and it was to prove invaluable in my work on English
suffixes, though I did not know it at first.

The 1960's had seen the brief flowering of \isi{ordinary language philosophy},
whose proponents, beginning with the very late %
%Wittgenstein (1953)
\citet{Wittgenstein1953}%
%Wittgenstein
%
, were
most interested in how individual everyday words were used, in
opposition to the logical project of Wittgenstein's early work. Despite
the popularity of such works as %
%Austin (1962) 
\citet{Austin1962} %
%Austin
%
and %
%Searle (1969)
\citet{Searle69}%
%Searle
%
,
\isi{ordinary language philosophy} never went very far, at least in part
because its proponents never developed more than anecdotal methods of
mining the idiosyncratic subtleties of usage of individual words. But
there was no contradicting the view that every word is a mysterious
object with its own singular properties, a fact that most of my
colleagues willfully ignored, in their search for the beautiful
generality of rules. The question for me was and remains how to balance
the two, words and rules.

\newpage 
Morris Halle had given a course on morphology in the spring of 1972, in
preparation for his presentation at the International Congress of
Linguists in the summer. Noam Chomsky had published a paper on derived
nominal two years before, in 1970, which, though it was directed at
syntacticians, provided a different kind of legitimation for the study
of the individual words that my beloved dictionaries held. Maybe I could
find something there, I said to myself with faint hope, though the
approach that Halle had outlined did not open a clear path for me and I
knew that I was not a syntactician, so Chomsky's framework did not
appear at first to provide much hope, despite his attention to words.

Beginning in early 1972, I spent close to a year reading everything I
could lay my hands on that had anything to do with morphology. I started
with Bloomfield and the classic American Structuralist\is{American Structuralists} works of the
1950s that had been collected in Martin %
%Joos's (1958) 
\citepos{Joos58} %
%Joos
%
\emph{Readings in
Linguistics}, almost all of which dealt with inflection. Though I
learned a lot, I couldn't find much of anything in that literature to
connect with the sort of work that was going on in the department or in
generative linguistics more broadly at the time.

In the end, I did find something to study in morphology, though not in
generative linguistics. I have come back to this topic, English word
formation, again and again ever since, but only now am I beginning to
gain some real grasp of how it works. The seeds of my understanding were
sown in my earliest work on the topic but they lay dormant for decades,
until they fell on fertile ground, far outside conventional linguistic
tradition. And though again I did not come to understand it for decades,
\isi{word-formation} was also a fine fit for the Boasian approach that I had
learned to love in my first undergraduate linguistics training, in which
the most interesting generalizations are often emergent, rather than
following from a theory. Also, the nature of the system in morphology,
and especially \isi{word-formation}, is much better suited to someone of my
intellectual predilections. This is an area of research in which regular
patterns can best be understood in their interplay with irregular
phenomena. I enjoy this kind of play.

Word-formation\is{word-formation} and morphology in general had had an odd history within
the short history of generative linguistics before 1972, generously
twenty years. One of the best-known early generative works was about
\isi{word-formation}, Robert 
%Lees's 
\citeauthor{Lees60}'s 
immensely successful \emph{Grammar of
English Nominalizations} (1960). This book, though, despite its title,
dealt mostly with compounds and not nominalizations, using purely
syntactic mechanisms to derive compounds from sentences, seemingly
modeled on the method of \emph{Syntactic Structures}.\footnote{Lees's
  book went through five printings between 1960 and 1968, extraordinary
  for a technical monograph that was first published as a supplement to
  a journal and then reissued by a university research center.} Lees's
book directly inspired very little research on \isi{word-formation} in its
wake, though the idea of trying to derive words from syntactic
structures has surfaced regularly ever since %
%(Marchand 1969; Hale \&
%Keyser 1993; Pesetsky 1995)
\citep{Marchand1969,Hale1993,Pesetsky.1995}%
%Marchand;Hale-Keyser;Pesetsky
%
.

%Chomsky's
\citeauthor{Chomsky1970}'s
 \citeyear{Chomsky1970} ``Remarks on nominalization'' (henceforth Remarks) echoed
Lees's book in title only. It was in fact its complete opposite in
spirit, method and conclusions, although Chomsky never said so. After
all, he owed Lees a great personal debt. Lees had played a large role in
making Chomsky famous with his (1957) review in \emph{Language} of
%
%Chomsky (1957)
\citet{Chomsky1957}%
%Chomsky
%
. Remarks injected for the first time into generative
circles the observation that some linguist units, in this case derived
words, are semantically idiosyncratic and not derivable in syntax
(unless one is willing to give up on the bedrock principle of semantic
\isi{compositionality}). Word-formation\is{word-formation}, it turns out, is centered on the
interplay between the idiosyncrasies of individual words that Chomsky
noted and the regular sorts of phenomena that are enshrined in the rules
of grammar.

My first excursion into original morphological research took place in
the fall and winter of 1972--73, a time when I was entirely adrift. I had
begun to read widely and desperately on morphology early in 1972, hoping
it might save me from myself, but had not yet lit on any phenomenon that
held the faintest glimmer of real promise. This is the lifelong agony of
an academic: the struggle to find something that is both new and of
sufficient current interest for others to give it more than a passing
glance. For some reason, I embarked on a study of Latinate verbs\is{verb!Latinate verb} in
English and their derivative nouns and adjectives, verbs like
\emph{permit} and \emph{repel}, and their derivatives: \emph{permission}
and \emph{permissive}; \emph{repulsion} and \emph{repulsive,} which
contained a Latin prefix followed by a Latin root that did not occur
independently in English. All the verbs had been borrowed into English
and I can't recall for the life of me what led me to study this peculiar
class of words.

What I first noticed about these verbs and their derivatives was that
the individual roots very nicely determined the forms of the nouns and
adjectives from the verb by affixation. Each individual root such as
\emph{pel} generally set the form of the following noun suffix (always
-\emph{ion} after \emph{pel}). Also, a given root often also had an
idiosyncratic form (here \emph{puls-}) before both the noun and
adjective suffix: \emph{com\textbf{puls}ion},
\emph{com\textbf{puls}ive}; \emph{ex\textbf{puls}ion},
\emph{ex\textbf{puls}ive}; and so on for all verbs containing this
Latinate root\is{root!Latinate root}. With a very small number of exceptions, the pattern of
root and suffix forms was entirely systematic for any given root but
idiosyncratic to it, and therefore predictable for many hundreds of
English verbs, nouns, and adjectives. The whole system was also
obviously entirely morphological. And best of all, no one had noticed it
before. I had discovered something new in morphology and I quickly
outlined my findings in by far the longest paper that I had ever
written, almost fifty pages, filled with typos, which I completed in
April 1973.

The central results of this first work were entirely empirically driven.
I have prized empirical findings above all other aspects of research
ever since, because these findings don't change with the theoretical
wind. The generalizations I found are as true today as they were in
1973. In this emphasis on factual generalization I differ from most of
my linguist colleagues. Of the empirical discoveries that I have made
over the years, I am proudest of three: this one, the morphome, and the
morphological stem.

It wasn't long before I realized that Latinate roots\is{root!Latinate root} presented a
fundamental problem for standard structural linguistic theories of
morphology. All of these theories were --and many still are --based on
the still unproven assumption that Baudouin de Courtenay had first made
explicit almost a century before in linguistics, that all complex
linguistic units could be broken down exhaustively into indivisible
meaningful units, which were reassembled compositionally (in a
completely rule-bound manner) to make up utterances.\footnote{The idea
  that morphology and syntax are both compositional was simply assumed
  from the beginning, though it should be noted that Baudouin's work
  predates Frege's discussion of \isi{compositionality}.} The problem was
that, although these Latinate roots\is{root!Latinate root} could not be said to have constant
meaning, or in some cases any meaning at all that could be generalized
over all their occurrences, they had constant morphological properties.
The English verbs \emph{ad\textbf{mit,} com\textbf{mit}},
\emph{e\textbf{mit}}, \emph{o\textbf{mit}}, \emph{per\textbf{mit}},
\emph{re\textbf{mit}}, \emph{sub\textbf{mit}}, \emph{trans\textbf{mit}},
and so on, do not share any common meaning. What they do share are the
morphological peculiarities of the root \emph{mit}. The classical Latin
verb \emph{mittere} meant `send' and the prefixed Latin verbs to which
the English verbs are traceable may have had something to do with this
meaning in the deep historical past of Latin, but even in classical
times the prefixed verbs had begun to diverge semantically from their
base and from each other. What ties them so closely together in English
is only the structural fact that, without exception, they share the
alternant \emph{miss} before the noun suffix -\emph{ion} and the
adjective suffix \emph{-ive}, and that the form of the noun suffix that
they take is similarly always \emph{-ion}, and not -\emph{ation} or
-\emph{ition}.

The verb root \emph{mit/miss} has very consistent, unmistakable, and
idiosyncratic morphological properties in English today. Unless we
choose to disregard them, these properties must be part of the
morphology of the language. But the root has no meaning, so it can't be
a morpheme in the standard sense. How can we make sense of this apparent
paradox?
 
The answer \rephrase{lies}{is found} in the empirical observation that formed the core of %
%Chomsky
\citeauthor{Chomsky1970}%
's Remarks: derived words are not always semantically
compositional. This observation, which Chomsky called the
\emph{lexicalist hypothesis}\is{lexicalist hypothesis}, is the single greatest legacy of
Remarks. It is far from original; only its audience is new.
Jespersen, for example, writing about compound words, had pointed out
many times over several decades that the relations between the members
of a compound are so various as to defy any semantically predictive
analysis. Jespersen concluded that the possible relations between the
two members of a compound are innumerable:

\begin{quote}
Compounds express a relation between two objects or notions, but say
nothing of the way in which the relation is to be understood. That must
be inferred from the context or otherwise. Theoretically, this leaves
room for a large number of different interpretations of one and the same
compound [\ldots] On account of all this it is difficult to find a
satisfactory classification of all the logical relations that may be
encountered in compounds. In many case the relation is hard to define
accurately [\ldots] The analysis of the possible sense-relations can never
be exhaustive. %
%(Jespersen 1954, pp. 137-138)
\citep[137-138]{Jespersen54}%
%Jespersen
%

\end{quote}

The purpose of Remarks had been tactical. As %
%Harris (1993) 
\citet{Harris1993} %
%Harris
%
recounts
in detail, at the time of writing the article, Chomsky was locked in
fierce combat with a resurgent group of younger colleagues, the
generative semanticists, who sought to ground all of syntax in
semantics. Syntax at the time was assumed to encompass \isi{word-formation},
though in truth almost no work had been done on \isi{word-formation} besides
%
%Lees (1960)
\citet{Lees60}%
%Lees
%
. Reminding everyone in the room that at least some word-formation was 
not compositional, a purely empirical observation, cut the
legs out from under generative semantics in a single stroke from which
the movement never recovered. More importantly, although Chomsky never
mentioned it and may not have realized it, the demonstration that some
complex words are not semantically compositional also destroyed
Baudouin's traditional morpheme and lent support to Saussure's sign
theory of words. The non-compositional complex words at the core of
Remarks lie within the class of what %
%Jespersen (1954) 
\citet{Jespersen54} %
%Jespersen
%
called
\emph{naked words}: uninflected words. Complex naked words are formed by
derivational morphology and compounding. Inflected forms, by contrast,
are always compositional, because they realize cells in the
morphosyntactic paradigm of the naked word. Their properties are
accidental, in the traditional grammatical sense of the term, not
essential.

What I had learned from Remarks about \isi{compositionality} within words,
combined with my discoveries about meaningless Latinate roots\is{root!Latinate root}, led me to
realize that \is{word-formation} word-forma\-tion needed to be studied in a way that was free
from Baudouin's axiom, an axiom that had held sway for over a century:
that complex words can be broken down exhaustively into meaningful
morphemes. Although I was entirely unaware of the consequence at the
time, and remained unaware of it for decades, this discovery freed me to
do linguistics in the way I loved to, not deductively as I had been
taught to do at MIT, following some current theory where it led, and not
inductively, but by working towards what the great Barbara McClintock
had called ``a feeling for the organism'' %
%(Keller 1983)
\citep{Keller83}%
%Keller
%
. My first two
years at MIT had taught me that the theory and deduction game held
little charm for me. Perhaps that's because I wasn't very good at it.
Working on my own terms made me feel better about myself than I had for
the entire preceding two years. I could stop worrying whether I was as
smart as all those other people. It turned out I didn't have to be
smart. Common sense was at least as valuable, and much rarer in those
circles.

English had been an exotic object of inquiry for American linguistics
from the start. The first American Structuralists\is{American Structuralists} were anthropological
field workers who confined themselves deliberately to the native
languages of North America. Only in his very last years did Edward Sapir
turn to English. Bloomfield discussed English in his \emph{Language}
(1933), presumably to engage a broad readership, but in his technical
writing he too dealt mostly with languages of North America on which he
did original fieldwork. Bloomfield's successors, notably %
%Trager and Smith (1951)
\citet{Trager51} %
%
%
 did important work on English, but they were in a decided
minority.

\largerpage
Generative grammar was different. The vast bulk of research in the first
two decades, beginning with %
%Chomsky, Halle \& Lukoff (1956)
\citet{Chomsky56}%
%Chomsky-Halle-Lukoff
%
, had been on
English. This English bias was especially true of generative syntax,
whose success was due in no small part to the analyst being able to come
up with novel sentences on the fly that the grammar could label as
either grammatical or ungrammatical. Only a native English speaker could
have come up with the most important sentence in the history of
linguistics, Chomsky's \emph{colorless green ideas sleep
furiously}.\footnote{All the data in the most important American
  structuralist\is{American Structuralists} work on syntax before \emph{Syntactic Structures}, 
%Wells (1947)
\citet{Wells1947}%
, is from English, except for one small example from Japanese.}
Even in generative phonology, whose earliest works, %
%Chomsky (1951) 
\citet{Chomsky51} %
%Chomsky
%
on
Modern Hebrew and %
%Halle (1959) 
\citet{Halle1959} %
%?Halle
%
on Russian had dealt with other
languages, the high-water mark of this tradition was an analysis of
English, \emph{The Sound Pattern of English}. It was therefore not
entirely unexpected that I should turn my attention to English word
formation. Even my earliest excursion into morphology had dealt with
English, albeit Latin roots that had been borrowed into English. It
would be a decade before I looked seriously at \isi{word-formation} in other
languages %
%(Aronoff \& Sridhar 1984)
\citep{Aronoff84}%
%Aronoff-Sridhar
%
.

  
American linguists had not written much about \isi{word-formation} in the
preceding quarter century. The great Structuralists from Bloomfield to
Hockett had done seminal work on morphology. Much of it was collected in Martin %
%Joos's 1958
\citepos{Joos58}
 \emph{Readings in Linguistics}, which I read
carefully, along with the chapters on morphology in Bloomfield's
\emph{Language} (1933). But the Structuralists had dealt almost
exclusively with inflection. I could find almost nothing on uninflected
words. There was %
%Lees's (1960) 
\citepos{Lees60} %
%Lees
%
monograph, but his approach was not
useful in a post-Remarks environment, and besides, he mostly dealt
with compounds.

The most notable exception of the previous decade had been Karl Zimmer's
monograph on English negative prefixes %
%(Zimmer 1964)
\citep{Zimmer64}%
%?Zimmer
%
. This book opened
up an entirely new world for me, the tradition of English linguistics.
This world had existed for a century and more, parallel to the one I
inhabited but completely unknown to us, and it was one in which the
study of \isi{word-formation} had always occupied an important place.

English linguistics had emerged in departments of English language and
literature, where in the 1970s it still retained the connections to
philology that most of the rest of the field had left behind in the
19\textsuperscript{th} century. To this day, it is much more rooted in
texts than other kinds of linguistics, because of its closeness to
literature. Much of English linguistics was historically oriented, but
in a very different way from the comparative historical linguistics that
lay at the root of modern structural linguistics. Its focus was on the
linguistic history of a single language, the record of English since its
emergence as a distinct written language around 800 CE. The connection
to philology lay in this shared basis of written texts, though
philologists were much more literarily oriented. People who read Beowulf
and Chaucer and Shakespeare had to know something about the language
these people were writing in and English linguistics served this
purpose.

Every undergraduate English major---and there were many more in those
days---had to take a course on the history of the English language. For
the same reasons, English linguistics had sister disciplines in the
other major standard European languages and language families: French,
German, Italian, Spanish, Romance, Scandinavian, etc. As I learned much
later, the OED was the greatest monument of this tradition of English
linguistics, but much of the best work had been done on the European
continent, especially in German departments of \emph{Anglistik}. The
best-known exponent of this tradition was a Dane, Otto Jespersen.

Hans Mar\-chand reviewed Zimmer's monograph in \emph{Language} in 1966.
Mar\-chand had fled from Germany to Istanbul in 1934 as a Catholic
political refugee with the help of his mentor, the Jewish Romance
philologist Leo Spitzer. He gradually turned towards the study of
language rather than literature, remaining in Istanbul until 1953.
Marchand returned to Germany in 1957, after a stint in the United
States, to teach \emph{Anglistik} at the University of Tuebingen. His
book, \emph{The Categories and Types of Present-Day English
Word-Formation}, published in 1960 and greatly revised in 1969, has
remained the authoritative description of English \isi{word-formation} since
its first publication. Remarkably, Marchand had written most of the book
while in internal exile in Turkey in an Anatolian village from 1944 to
1945, under threat of repatriation to Germany, which had drafted him
into the military in absentia in 1944. He had sought unsuccessfully for
years to publish this early version while still in Turkey.

\newpage 
Marchand and Zimmer follow very similar approaches, quite
different from that of American structural linguistics\is{American Structuralists}. They ask what a
given derivational affix meant (what Zimmer calls its ``semantic
content''), what it applied to, and what it produced. The prefix
\emph{un-} that most occupies Zimmer's mind, for example, is negative in
meaning and derives adjectives from adjectives.\footnote{\emph{Un-} also
  attaches to verbs and has the sense of undoing the action of the verb.
  Whether these two are one and the same affix has been much discussed
%(Hornxx)
\citep{Horn84}%
.} This is all very traditional and in line with the treatment
of derivational affixes in the OED, which contained entries for
derivational affixes from the beginning, though not for inflectional
affixes. The adjectival negative prefix \emph{un-} has a very extensive
entry in OED, with many observations similar to those of Marchand and
Zimmer, and hundreds of examples (my favorite being
\emph{unpolicemanly}). The OED even notes the morphological environments
in which a given derivational affix is particularly productive, which
was of special importance to Zimmer and to my own work. For \mbox{\emph{un-},}
the OED notes that it is especially common with adjectives ending in
\emph{-able}: ``In the modern period the examples become too numerous
for illustration; in addition to those entered as main words, those
given below will serve as specimens of the freedom with which new
formations are created.''

This traditional approach to \isi{word-formation} provided an intuitively
satisfying solution to the problem of the morpheme that my work on
Latinate roots\is{root!Latinate root} had uncovered. If derivation is not a matter of combining
morphemes but of attaching affixes to words, then we don't need all the
morpheme components of words to be meaningful and we don't need the
internal semantics of words to be compositionally derived from these
components. All we need is for words to be meaningful. We don't need to
worry about morphemes at all, only words and what the derivational
affixes do with them.

This traditional approach circumvented the problem of meaningless
morphemes for a simple reason: it predated the notion of the morpheme.
The earliest citation in OED by far for any sense of the word
\emph{derivation} equates it with \emph{formation.} It comes from
Palsgrave's 1530 English-language grammar of French,
\emph{L'esclarcissement de la langue françoyse}, the first known grammar
of French ever written in any language: ``1530 J. Palsgrave
\emph{Lesclarcissement} 68 Derivatyon or formation, that is to saye,
substantyves somtyme be fourmed of other substantyves.'' This has become
my favorite citation of the words derivation and (word) formation and,
though I did not know it at first, it encompasses the claim that words
are formed from words; my observation that words are formed from words
merely updates Palsgrave's remark. This claim is the essence of the
traditional treatment of \isi{word-formation} and it is the motto that I
adopted, elevating the observation to a principle.\footnote{The idea
  that words are formed from words may ultimately be traceable to the
  Greek and Latin grammatical traditions, which were entirely
  word-based\is{morphology!word-based}, even at the level of inflection %
%(Robins 1959)
\citep{Robins59}%
%Robins
%
.}

\largerpage
In my dissertation and subsequent monograph, I took complete credit for
the axiom that morphology was \is{morphology!word-based}word-based. Even decades later, when I
clarified the terminology and called it \emph{lexeme-based morphology}\is{morphology!lexeme-based},
I did not provide any direct attribution to the tradition of English
\isi{word-formation} studies. My only defense is that neither Marchand nor
Zimmer ever stated what for them was simply an unspoken assumption. All
I did was to make this assumption clear as an axiom. I can therefore at
least take credit for the realization that this was a useful axiom on
which to base the analysis of \isi{word-formation}.

\newpage 
Notation meant everything in those days. %
%Chomsky and Halle (1968) 
\citet{Chomsky68} %
%Chomsky-Halle
%
had
even gone so far as to extoll the explanatory power of parentheses. My
most important task was therefore to create a simple notation in which
traditional OED-style generalizations about \isi{word-formation} could be
stated in a way that generative linguists might understand. This was the
\isi{word-formation} rule (WFR). It bore close resemblance in form to the
rewrite rules that were standard in generative grammar. A WFR took a
word from one of the three major lexical categories (Noun, Verb, or
Adjective) and mapped it onto a lexical category (the same or another),
usually adding an affix, and making another word. The rule of \emph{un-}
prefixation, for example, could be written as {[}X{]}\textsubscript{A} $\rightarrow$
{[}un-{[}X{]}\textsubscript{A}{]}\textsubscript{A} or it could be
written simply as the output
{[}un-{[}X{]}\textsubscript{A}{]}\textsubscript{A}. This notation was
transparent and made generative linguists, myself included, think that
this way of dealing with \isi{word-formation} could be easily assimilated into
their way of thinking. The acronym WFR added a nice touch. The title of
the published version of my dissertation, \emph{Word Formation in
Generative Grammar} %
%(Aronoff 1976) 
\citep{Aronoff1976} %
%Aronoff
%
was suggested by S. Jay Keyser, the
editor of the series of which this would be the inaugural monograph. It
only served to strengthen the impression that I had integrated the study
of \isi{word-formation} into generative grammar. The monograph was a great
success, thanks in no small part to its title, and most accounts treat
the book as central to the treatment of morphology and \isi{word-formation}
within generative grammar.

Nothing could be further from the truth. The title of the monograph was
deeply deceptive and in agreeing to it I was also deceiving myself. Word
formation rules, as conceived of and discussed in that monograph, are
incompatible with generative grammar or with any grammar-based
linguistic framework, because, like the tradition they encode, these
rules cross the synchronic-diachronic boundary that is central to all
post-Saussurean structural linguistics. I have only recently come to
appreciate this fact. I certainly believed at the time that I was doing
generative grammar, as have most of the book's readers since. What is
true is that I was a member of a social community self-organized around
generative grammar. I did my work on \isi{word-formation} within that
community and it was accepted as legitimate almost entirely on those
social grounds.

In his great posthumous work, %
%Saussure (1916/1959) 
\citealt{Saussure16}/1959 %
%Saussure
%
set up a distinction
that has been accepted throughout the field ever since, between
\emph{synchronic} and \emph{diachronic} linguistics. Synchronic
linguistics deals with a single state of a language---the
present---while diachronic linguistics deals with successive
states---history. Generative grammar seeks to provide a theory of what
is a possible synchronic grammar of a language, the basic idea being
that the grammar generates the language %
%(Chomsky 1957)
\citep{Chomsky1957}%
%Chomsky
%
. The theory is
also supposed to mirror the innate capacity that a child brings to the
task of constructing a grammar for the input that the child receives
%
%(Chomsky 1965)
\citep{Chomsky1965}%
%Chomsky
%
. But traditional research on \isi{word-formation}, which
preceded Saussure in its origins, is neither synchronic nor diachronic:
it is about how new derived words accumulate in a language \textbf{over
time}. That is why Mar\-chand gave his \emph{magnum opus} the subtitle ``A
Synchronic-Diachronic Approach'' and why Jespersen called his monumental
six-volume life's work \emph{A Modern English Grammar on Historical
Principles}, both titles in direct contradiction of the Saussurean
split, both by scholars working within the tradition of English
linguistics. In truth, Marchand's approach was neither synchronic nor
diachronic, in spite of its fashionable title, because the study of word
formation lends itself to neither synchrony nor diachrony: the word
formation system of the language at any given moment can only be
understood through the historical accumulation of the lexicon. The study
of \isi{word-formation} is concerned at its core with how words are created,
how they are formed, and how they are added to the language. Unlike
sentences, words, once formed, accumulate, and this accumulated
storehouse has an effect on new words. Words accumulate both in the
mental lexicon of an individual speaker and in the collective lexicon of
the larger linguistic community.

This brings us back to Chomsky's lexicalist hypothesis\is{lexicalist hypothesis}. To understand
this hypothesis, we need to clarify two distinct senses of the word
\emph{lexical}\is{lexical} %
%(Aronoff 1988)
\citep{Aronoff88}%
%Aronoff
%
. One is Bloomfield's lexicon, the list of
what DiSciullo and %
%Williams (1987) 
\citet{Di-Sciullo87} %
%Williams
%
later so nicely called the
``unruly.'' The other encompasses the \isi{word-formation} rules themselves
and maybe all morphology including inflection too. The term
\emph{lexical component} is usually meant to include both the rules of
morphology and the lexicon. Chomsky's original lexicalist hypothesis\is{lexicalist hypothesis}
says no more than that the lexical component is responsible for forming
and storing some of the complex words of the language, in addition to
the simple monomorphemic words that have always been thought of as
arbitrary signs stored in the lexicon. His major criterion for
distinguishing lexically from `transformationally' derived words is
semantic predictability or \isi{compositionality} (lexically derived words are
not compositional) though most later lexicalist theorists used others as
well %
%(Aronoff 1994, Pesetsky 1995)
\citep{Aronoff94,Pesetsky.1995}%
%Aronoff;Pesetsky
%
.

%
%Halle's (1973) 
\citepos{Halle73} %
%Halle
%
lexicon, which he described as ``a special filter through
which the words have to pass after they have been generated by the word
formation rules'' (p. 5), is a Bloomfieldian list of words, separate
from the morphological rules. Halle suggested that ``the list of
morphemes together with the rules of \isi{word-formation} define the set of
\emph{potential} words\is{word!potential} of the language. It is the filter and the
information that is contained therein which turn this larger set into
the smaller subset of \emph{actual} words\is{word!actual}'' (p. 6). This way of looking
at the relation between \isi{word-formation} and the lexicon appears to permit
us to include \isi{word-formation} in a synchronic grammar: the morphemes and
the abstract rules of \isi{word-formation} will be part of the grammar, not
the lexicon, while the actual results of the application of the rules to
the morphemes, which can be quite messy and idiosyncratic, as Chomsky
had already emphasized, will be housed outside the grammar in the
Bloomfieldian lexicon. Words will be formed by rules in the grammar,
just as sentences are, though perhaps by a distinct lexical component,
along the lines of the theory of Remarks. On this story, though,
once words are formed they are stored in the lexicon and should
accordingly have no further interaction with the grammar or the rules.

Over the years, this general strategy of strictly separating the rules
from the unruly in order to better assimilate \isi{word-formation} to syntax,
what Marantz much later called the \emph{single engine hypothesis}\is{single engine hypothesis}
%
%(Marantz 2005) 
\citep{Marantz2005} % ???
%?Marantz
%
has faced a number of problems, all of which are
traceable to the fact that the strategy allows for no interaction
between the rules (and the morphemes they operate on) and the set of
words formed by the rules, which are stored in the lexicon. The
insulation of the rules from the lexicon makes it impossible to ask many
interesting questions with even more interesting answers. I will discuss
briefly here only the two most important ones, morphological
productivity\is{productivity} and \isi{blocking}.

Unlike most rules of syntax, rules of \isi{word-formation} vary widely in
their \isi{productivity}. A standard example is the trio of suffixes
-\emph{ness}, -\emph{ity}, and -\emph{th}, all of which form nouns from
adjectives in English. of the three, -\emph{th} is the least productive;
only a handful of words end in this suffix. The only one I can identify
as having been added to the language in the last couple of centuries is
\emph{illth}, which was coined on purpose by John Ruskin in 1862 to
denote the opposite of \emph{wealth}. The word is almost never used
today, except in close proximity to \emph{wealth} or \emph{health}.
Speakers of English know that new or infrequent words in -\emph{th} have
an odd flavor about them. The OED remarks about the word \emph{coolth},
for example, that it is ``Now chiefly literary, arch{[}aic{]}, or
humorous.''

The suffix -\emph{ity} is more productive, but limited in the morphology
of what it can attach to. The OED lists approximately 2400 nouns in
current use ending in the letter sequence \textless{}ity\textgreater{},
most of which contain the suffix, compared with about 3600 ending in the
letters \textless{}ness\textgreater{}. But a closer look reveals that
\textless{}ity\textgreater{} is much more likely to appear after a
select set of suffixes. With -\emph{ic} it is preferred by a ratio of
almost 7/1 over \emph{-ness}. This preference is reflected in speakers'
judgments and in the relative frequency of members of individual pairs.
The word \emph{automaticity} feels much more natural than
\emph{automaticness} and a simple Google search shows 109,000 ``hits''
for \emph{automaticity} but only 242 for \emph{automaticness}. Even for
very rare words, the same pattern emerges. While \emph{oceanicity}, a
word I have never heard of, gets only 762 hits, its counterpart,
\emph{oceanicness}, gets only 5!

Once we leave the few affixes that \emph{-ity} is attracted to, though,
\emph{-ness} is ascendant. \emph{Greenness} outnumbers \emph{greenity}
1000/1. Google even thinks that you have made a mistake when you search
for \emph{greenity and} asks: ``Did you mean: greenify?'' A similar
pattern of results is found for all the other color words. In the same
vein, we can find examples of humorous uses of words like \emph{sillity}
or \emph{slowity} in the Urban Dictionary, but not in many other places
on the Web.

There are numerous ways of distinguishing the \isi{productivity} of these
three suffixes, but \isi{productivity} is clearly related to the number of
words that are already present in the language: the more you have, the
more you get. Productivity\is{productivity} depends on the accumulation of words. It is a
dance between the lexicon and the grammar. If we try to make a strict
separation between the two, we will never understand how the dance
works. Both Marchand and Zimmer knew about the nuances of \isi{productivity}.
Marchand closes his review of Zimmer's book with the following somewhat
backhanded compliment: ``Zimmer's investigation is a valuable
contribution not to the study of semantic universals, which it planned
to be, but to the problem of \isi{productivity} in \isi{word-formation}'' \citep[142]{Marchand66}.

The other problem that \isi{productivity} poses for modern linguistics is that
it is variable. Mainstream formal linguistics, with its roots in the
triumphal 19\textsuperscript{th} century neo-grammarian slogan that
sound change laws have no exceptions %
%(Paul 1880) 
\citep{Paul1880} %
%Paul
%
has never dealt well
with variation. If anything, formal linguists continue to be blind to
the fact that variation is a part of language (I-language). One response
to variability is simply to deny that a phenomenon like \isi{productivity}
exists. Another is to admit that it exists, but to deny that the
phenomenon is variable, claiming instead that it is all or none. That is
what Marchand does. Referring to \citet[225]{Harris1951}, Marchand notes disapprovingly that ``a descriptivist
like Zellig S. Harris maintained that `the methods of descriptive
linguistics cannot treat of the degree of \isi{productivity} of elements{'}''
\citep[141]{Marchand66}
%
%(Harris 1951, p. 225)
%
%Harris
%
. But he himself only dichotomizes \isi{word-formation}
rules into those that are productive and those that are, in his words,
restricted:

\begin{quote}
Zimmer's merit is to have seen an important problem in word-formation,
that of \isi{productivity}. . . . Zimmer's study . . . calls our attention to
the fact that what seems to be the same type of combination, viz.
derivation by means of a negative prefix, is in reality split up into
two groups, one of restricted \isi{productivity} (instanced by \emph{unkind})
and another, deverbal group (instanced by \emph{unread}) which is of
more or less unrestricted \isi{productivity} \citep[141]{Marchand66}.
\end{quote}

Even here, Marchand is not talking about one productive rule vs. a
different unproductive rule, but rather a single rule, which is more
productive in one environment (with past participles and \emph{-able}
derivatives, both of which have a passive reading) and less productive
in another (with underived adjectives like \emph{kind}). As Zimmer
demonstrates, there is not in fact a dichotomy, but rather a cline in
\isi{productivity} that depends on both environments and rules. In the half
century since, the nondiscrete nature of \isi{productivity} has been
demonstrated time and again, most definitively in 
\citet{Bauer01}.

Productivity\is{productivity} is a question of fecundity, how many words there can be and
how easily they can be created. A pattern is highly productive if there
can be many new words in that pattern. It is unproductive if there can
be only a few new words. When we say that the English nominal suffix
\emph{-ness} is highly productive we mean that the pattern can form many
nouns from adjectives; when we say that the suffix -\emph{th}, which
also derives nouns from adjectives, is unproductive, we mean that it
cannot. And because words are formed from words, there is a direct
relation between how easy it is to form words in a pattern and how many
already exist in that pattern, in either  the mind of a speaker or the
language of a community. As we have just seen, there are many
-\emph{ness} nouns in English. The OED lists over 4000 nouns ending in
the letters \textless{}ness\textgreater{}, the great majority of them
containing the suffix. There are no more than a handful of -\emph{th}
nouns derived from adjectives. If how many words there can be of a given
type depends on a combination of how many words there are already of
this type and how many there are for the type to feed on, then words
differ sharply from sentences. For starters, it makes little sense to
even ask how many sentences there are of a given type. Sentences are not
stored, they are produced and then vanish.

Blocking\is{blocking} is the second phenomenon that demonstrates how the formation of
individual words depends intimately on the words we already know. For
four decades, since the moment that I first stumbled on this phenomenon,
it has been clear to me that \isi{blocking} is a real empirical phenomenon and
that it is just what I first defined it to be: ``the nonoccurrence of
one form due to the simple existence of another'' \citep[43]{Aronoff1976}. A few pages later, I made an explicit connection to synonymy:
``Blocking\is{blocking} is basically a constraint against listing synonyms in a given
stem'' \citep[55]{Aronoff1976}. And on the same page I wrote: ``To exclude having
two words with the same meaning is to exclude synonymy, and that is
ill-advised.'' A few pages later, I referred to ``the \isi{blocking} rule.''
Clearly, I had no idea precisely what \isi{blocking} was, beyond an empirical
phenomenon. Only now, though, do I understand why my empirical
observation might be true: the avoidance of synonymy in general and
\isi{blocking} in particular are the result of competition, a topic I have
spent the last half decade investigating.

The tradition of word-based morphology\is{morphology!word-based} dates to the first grammarians,
although it was eclipsed for much of the twentieth century by the rise
of synchronic linguistics. In Cambridge, Massachusetts one didn't learn
much about what was happening in Cambridge, England, but soon after
leaving for Stony Brook I learned that word-based morphology\is{morphology!word-based} had been
revived in England in the decade or so before my own research, notably
by R. H. %
%Robins (1959) 
\citet{Robins59} %
%Robins
%
and Peter %
%Matthews (1965, 1972)
\citet{Matthews65,Matthews72}%
%Matthews;Matthews
%
. This line of
research, especially in derivational morphology\is{derivation}, has grown in the
decades since, notably in France, led by Danielle %
%Corbin (1987)
\citet{Corbin87}%
%Corbin
%
,
Françoise %
%Kerleroux (1996)
\citet{kerleroux96}%
%Kerleroux
%
, and Bernard %
%Fradin (2003)
\citet{Fradin03}%
%Fradin
%
. Together, they
created a new thriving research community, of which I am proud to be a
member.

















%\nocite{Anderson15,
%Aronoff1976,
%Aronoff84,
%Aronoff88,
%Austin1962,
%Bauer01,
%Baudouin1895,
%Bloomfield1933,
%Chomsky51,
%Chomsky56,
%Chomsky68,
%Chomsky1957,
%Chomsky1965,
%Chomsky1970,
%Corbin87,
%Di-Sciullo87,
%Fradin03,
%Hale1993,
%Halle73,
%Harris1993,
%Harris1951,
%Jespersen54,
%Joos58,
%Keller83,
%kerleroux96,
%Lees57,
%Lees60,
%Marchand66,
%Marchand1969,
%Matthews65,
%Paul1880,
%Pesetsky.1995,
%Robins59,
%Saussure16,
%Searle69,
%Trager51,
%Wells1947,
%Wittgenstein1953,
%Zimmer64%
%}

{\sloppy
    \printbibliography[heading=subbibliography,notkeyword=this]
}


\end{document}
