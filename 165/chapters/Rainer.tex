\documentclass[output=paper]{langsci/langscibook}
\ChapterDOI{10.5281/zenodo.1406991}
\title{\texorpdfstring{Word formation and word history:\\ The case of 
 \textsc{capitalist} and \textsc{capitalism}}{Word formation and word history:  CAPITALIST and 
CAPITALISM}} 

\shorttitlerunninghead{Word formation and word history:  CAPITALIST and 
CAPITALISM} 
\renewcommand{\lsCollectionPaperFooterTitle}{Word formation and word history: The case of  \noexpand\textsc{capitalist} and \noexpand\textsc{capitalism}}
\author{Franz Rainer \affiliation{WU Vienna}}

% \chapterDOI{} %will be filled in at production 
\abstract{The treatment of the history of modern vocabulary in historical and etymological dictionaries is generally disappointing, especially with respect to the processes by which the words came into being. The \emph{TLFi}\footnote{\emph{Trésor de la langue française informatisé}, available at \url{http://atilf.atilf.fr/}.} only provides the following information concerning the history of French \emph{capitalisme} and \emph{capitaliste}: ``\textbf{Capitalisme} [\ldots] Dér. de \emph{capital}\textsuperscript{2}*; suff. \emph{-isme}*'', ``\textbf{Capitaliste} [\ldots] Dér. de \emph{capital}*; suff. \emph{-iste}*''. Such a treatment, which is inadequate even from a synchronic point of view (in the sense `a supporter of capitalism', \emph{capitaliste} is derived from \emph{capitalisme} by affix substitution\is{affixation!affix substitution}), does not do justice to the manifold relationships that have developed between these two words and their common base \emph{capital} in the course of the 300 years since the creation of \ili{Dutch} \emph{Capitalist} in 1621. The present paper retraces in detail the many steps of the unfolding of these two words in French. It is shown that each of their many senses constitutes a separate lexeme and must be provided with an etymology of its own. Particular attention is dedicated to the identification of the exact mechanism (\isi{borrowing}, semantic extension, word formation) that was at work at each step.}

\maketitle

\begin{document}
\selectlanguage{english}
\il{French|(}
\is{word-formation|(}
\is{word history|(}
\is{capitalist|(}
\is{capitalism|(}



\section{In the beginning was the lexeme}

Right from the beginning of the study of the internal structure of
complex words, scholars have been divided between those who tried to put
complex words together from smaller pieces in a bottom-up fashion (the
Pāṇinian tradition) and those who tried to account for the internal
structure by mapping words onto other words (the Greco-Roman tradition,
based on \isi{analogy}). This fundamental divide is still with us, in the form
of an opposition between what we now call ``morpheme-based\is{morphology!morpheme-based}''
and
``word-based\is{morphology!word-based}'' (or ``lexeme-based\is{morphology!lexeme-based}'') approaches
to morphology %
%(cf.
%Aronoff 2007)
\citep[see][]{Aronoff2007}%
%Aronoff
%
. In the French linguistic landscape, the morpheme-based
approach held some sway before the turn of the millennium due to having been embraced by Danielle Corbin %
%(cf. Corbin 1987)
\citep[see][]{Corbin87}%
%Corbin
%
, who played an
important role in the renewal of the study of word formation in France.
But more recently most French morphologists seem to be quite unanimous
in preferring the lexeme-based approach, not least due to the forceful
argumentation in its favour in %
%Fradin (2003)
\citet{Fradin03}%
%Fradin
%
.

In my contribution, I would like to pour more water on the lexeme-based
mill by looking in some detail into the history of the two words
\textsc{capitalist} and \textsc{capitalism}, in which semantic change,
calques and word formation ‒ suffixation, conversion, but also suffix
substitution, a notorious conundrum for morpheme-based approaches ‒
have interacted in a complex manner. It will become apparent that these
changes find a natural explanation within a lexeme-based framework,
while they seem to be difficult to accommodate without contortions in a
morpheme-based one. However, the chapter is meant to be of interest not
only to morphologists or lexicologists, who constitute the main intended
readership. Both words treated are key concepts of present-day
intellectual vocabulary and as such have attracted considerable
attention from scholars from other disciplines,
mostly historians such as Fernand Braudel, Lucien Febvre, Henri Hauser
or Edmond Silberner in France, or Richard Passow, Marie-Elisabeth Hilger
and Annette Höfer in Germany. For such readers, the linguistic arguments
of this contribution may sometimes seem to be a little far-fetched,
while they would probably here and there like to receive more abundant
encyclopedic information. This latter type of information, however, must
be kept to a minimum here, providing just what is necessary for
underpinning the linguistic argumentation. Even so, non-linguists will
hopefully appreciate the new facets of the history of these two words,
which I was able to add to the existing documentation due to the
abundance of new material that we can now dip into thanks to Google
Books and Gallica.\footnote{On the history of \textsc{capitaliste}, see %
%Rainer (1998)
\citet{Rainer98}%
%Rainer
%
. A short, updated entry on the  history of French
  \emph{capitaliste}, written together with Jean-Paul Chauveau, can be
  found on  \emph{TLF-Étym}, an etymological online dictionary that
  can be consulted at \url{http://www.atilf.fr/tlf-etym/}. The
  corresponding entry on French \emph{capitalisme} can be consulted on the same
site.}

In order to avoid misunderstandings, one formal proviso is in order
before we start our investigation. It is established practice in
linguistics to write lexemes in small caps. In this tradition, the
English lexeme \textsc{capitalist} would represent the set of English
word forms \{ \emph{capitalist}, \emph{capitalists} \}. I will not follow
this usage here, but use small caps instead whenever referring to a word
independently from its exact formal realization in individual European
languages. Throughout this text, \textsc{capitalist} therefore
represents the set \{English \emph{capitalist}, German
\emph{Kapitalist}, French \emph{capitaliste}, etc.\}, and similarly for
other words in small caps.

\section{\textsc{capital}, \textsc{capitalist} and \textsc{capitalism} in
synchrony and diachrony}

For present-day speakers of European languages, \textsc{capitalism}
refers to a specific kind of economic system and is undoubtedly felt to
be based somehow on \textsc{capital}, though many speakers will be
hard-pressed to specify the exact semantic relationship between base and
derivative or will construe it in different ways. This indeterminacy is
mainly due to the fact that the word \textsc{capital} itself has various
senses, not all of them equally familiar to non-economists, and that it
is not obvious which sense is the relevant one for the construal of the
meaning of \textsc{capitalism}. The \emph{Free
Dictionary},\footnote{http://www.thefreedictionary.com/capitalism.}
for example, manages to define \textsc{capitalism} without recourse to
\textsc{capital}: ``An economic system in which the means of production
and distribution are privately or corporately owned and development
occurs through the accumulation and reinvestment of profits gained in a
free market.'' \textsc{Capitalist}, on the contrary, will most often be
spontaneously analyzed as based on \textsc{capitalism}, referring to a
supporter of the particular kind of economic system denoted by this
word. `A supporter of capitalism', in fact, is the first sense in the
online dictionary quoted above, which adds two more senses that seem to
be less prominent today: 2. `An investor of capital in business,
especially one having a major financial interest in an important
enterprise'; 3. `A person of great wealth'. The foregoing remarks seem
to be valid for European languages in general. In other respects,
however, individual languages differ, for example, with respect to
whether they tolerate the adjectival usage of \textsc{capitalist},
possible in French and English, but not in German. The connotations of
the members of this word family will also differ, depending
on the stance that a speaker or speech community takes with respect to
the economic system called \textsc{capitalism}.

The etymological treatment of \textsc{capitalism} and
\textsc{capitalist} in historical dictionaries seems to have been
inspired by and large by such intuitions about the synchronic
relationship between \textsc{capital}, \textsc{capitalist} and
\textsc{capitalism}. The \emph{TLFi}, for example, writes:

\begin{quote}
\textbf{Capitalisme} \emph{subst}. \emph{masc}. {[}\ldots{}{]} Dér. de
\emph{capital}\textsuperscript{2}*; suff. \emph{-isme}*.

\textbf{Capitaliste} \emph{adj}. et \emph{subst}. {[}\ldots{}{]} Dér. de
\emph{capital}*; suff. \emph{-iste}*. L'hyp. d'un empr. au néerl.
\emph{kapitalist} (BL.-W.\textsuperscript{5}) ne semble pas justifiée.
Le corresp. all. \emph{Kapitalist} « possesseur d'un capital » est
attesté dep. 1694 (WEIGAND).\footnote{{[}\textbf{Capitalisme}
  \emph{masc}. \emph{noun} {[}\ldots{}{]} Derived from
  \emph{capital}\textsuperscript{2}*; suffix \emph{-isme}*. /
  \textbf{Capitaliste} \emph{adj}. and \emph{noun} {[}\ldots{}{]}
  Derived from \emph{capital}*; suffix \emph{-iste}*. The hypothesis
  that it be a loan from Dutch \emph{kapitalist}
  (BL.-W.\textsuperscript{5}) does not seem to be justified. The
  corresponding German word \emph{Kapitalist} `owner of capital' has
  been attested since 1694 (WEIGAND).{]}}
\end{quote}

As we will see, this kind of analysis in no way does justice to the
complex interrelationships that have developed over time among the three
words of this word family, nor to the inter-European relationships that
link corresponding members in different European languages. I will now
 describe these relationships by following the evolutions of
the individual words step by step from the 17\textsuperscript{th} century up
to the present time.

\section{The evolution of the noun \textsc{capitalist} from the
17\textsuperscript{th} to the 19\textsuperscript{th} century}

\subsection{\textsc{Capital}}

This is not the place to take up the complex history of \textsc{capital}
at full length. Suffice it to say that by the time that the first
derivative, \textsc{capitalist}, appeared, \textsc{capital} generally
referred to the property, not necessarily only money, that a rich person
owned. In double-entry bookkeeping, the term referred to the net worth
owned by the merchant after taking away the liabilities from the assets.
Towards the end of the 18\textsuperscript{th} century economists
extended the meaning of the term to include the means of production
(buildings, machines, tools) used in agriculture or industry, what is
now called \textsc{physical capital}. This more technical sense still
has not really penetrated into common language, but it did play a role
in the history of \textsc{capitalist} and \textsc{capitalism}, as we
will see. More recent extensions of the concept, by contrast, such as
\textsc{human capital} or \textsc{social capital}, had no influence.

\subsection{\textsc{Capitalist}: the Dutch origins}

As we saw in Section 2, the \emph{TLFi} rejected the hypothesis of a
Dutch origin of the French noun \emph{capitaliste}, which had first been
put forward by %
%Barbier (1944-52, nr. XXV)
\citet[nr. XXV]{Barbier44}%
%Barbier
%
. This decision was
ill-advised, since the noun \emph{Capitalist} was indeed coined in the
Netherlands (then: ``United Provinces'') back in 1621 by tax authorities
in order to designate a wealthy citizen who possessed 2,000 guilders or
more:


\begin{quote}
Special registers distinguished the taxpayers into two categories: those
owning more than 2,000 guilders were called `capitalists' (from 1621),
and those owning 1,000 to 2,000 guilders were the so-called `half
capitalists' (from 1625). People owning less than 1,000 guilders were
fully exempt from extraordinary property taxes. A proposal from 1641 to
introduce another level, from 20,000 or 30,000 upwards, was not
accepted. The word `capitalist', here used in its earliest meaning,
clearly designated someone owning property.
\citep[122--123]{Hart93}
\end{quote}

Dutch \emph{Capitalist} was derived from \emph{Capitaal} `capital' and
followed the pattern of formations in -\textsc{ist} that designated
persons engaged in some activity, not the supporter pattern, both of
which were already well established at that time %
%(cf. Wolf 1972)
\citep[see][]{wolf72}%
%Wolf
%
. In
order to understand the choice of suffix, we probably have to assume
that the coiner conceived of a \emph{Capitalist} as a money-lender or
investor, not as a passive possessor of a huge sum of money or property.
Dutch \emph{Capitalist} was a complex concept, designating at the same
time a wealthy person, mostly engaged in money-lending or investment
activities, as well as a category of the tax authorities. Since both these
facets were linked by mutual inference, we should view them as part of
one and the same concept, not as two independent concepts, very much
like \emph{book} can designate at the same time the object on the table
and its content. It is also highly probable that the precise original
definition of \emph{Capitalist} on the part of the tax authorities (`a
person worth 2,000 guilders or more') was relaxed in common parlance to
refer simply to very rich individuals in general.

\largerpage[-2]
The 17\textsuperscript{th} century is called the ``Golden Age'' in Dutch
historiography, because the United Provinces at that time were at the
forefront of trade, military, science and art. This background,
especially their eminent position in international finance, explains how
a Dutch neologism could spread abroad and start an a\-stounding
international career. Already by the end of the 17\textsuperscript{th}
century, we find loan translations in German and French. German
\emph{Capitalist} (today written \emph{Kapitalist}) appears as early as
1671 in a document on the financial system of the United Provinces,
where, due to its novelty, it is glossed as `money-lender' %
%(Rainer 1998:
%10)
\citep[10]{Rainer98}%
%Rainer
%
. The German word, as far as I can see, had no influence on French,
which will be the focus of the rest of this paper.

\subsection{French \emph{capitaliste}: its semantic evolution until the
Physiocrats}

\largerpage[-2]

There can be no doubt about the Dutch origin of the French noun
\emph{capitaliste}. The oldest example, in fact, comes from the
\emph{Mercure Hollandois} of 1678, p. 13 and clearly refers to the very
special fiscal meaning which the term had at that time in the United
Provinces: ``Pour cet effet {[}i.e. to put up an army of 100,000 men in a
fortnight{]} ils posoient qu'il y avoit dans la Province de Hollande 65
500 Capitalistes, qui étoient taxés sur les Cahiers de l'Etat à
2.4.6.10.20. \& 80 000 livres.''\footnote{{[}To that effect they assumed
  that there were in the province of Holland 65,500 capitalists, whose
  tax charge according to the state's tax lists was 2, 4, 6, 10, 20 or
  80 thousand pounds.{]}} The few examples that we find in French until
the middle of the 18\textsuperscript{th} century (quoted under II.A in
the corresponding \emph{TLF-Étym} entry) refer to that same Dutch
reality. In the second half of the 18\textsuperscript{th} century,
however, the noun \emph{capitaliste} firmly established itself in French
with a reference independent from the Dutch context. Here is a quote
from the \emph{Dictionnaire domestique portatif} (Paris: Vincent 1765),
vol. 3, p. 505: ``RENTIERS; ce terme est synonyme à \emph{capitaliste},
c'est-à-dire, à celui qui fait valoir son argent, en le disposant
suivant le cours de la place, \& qui vit de ses rentes.''\footnote{{[}RENTIERS~;
  this term means the same as \emph{capitalist}, that is, one who
  invests his money according to the evolution of rates on the market
  and lives off his private income.{]}}

The diffusion of the term among a wider public was furthered by its
adoption by the Physiocrats, an economic school that began holding much
sway at that time, in France and abroad. The following example from
Turgot's \emph{Réflexions sur la formation et la distribution des
richesses} illustrates the meaning that will be the
dominant one throughout the 19\textsuperscript{th} century:

\begin{quote}
§ XCIII

\emph{Le \textbf{capitaliste} prêteur d'argent appartient, quant à sa
personne, à la classe disponible.}

Nous avons vu que tout homme riche est nécessairement possesseur ou d'un
capital en richesses mobilieres, ou d'un fonds équivalent à un capital.
Tout fonds de terre équivaut à un capital~; ainsi tout propriétaire est
\textbf{capitaliste}, mais tout \textbf{capitaliste} n'est pas
propriétaire d'un bien fonds~; et le propriétaire d'un capital mobilier
a le choix, ou de l'employer à acquérir des fonds, ou de les faire
valoir dans des entreprises de la classe cultivatrice ou de la classe
industrieuse. Le \textbf{capitaliste}, devenu entrepreneur de culture ou
d'industrie, n'est pas plus disponible, ni lui ni ses profits, que le
simple ouvrier de ces deux classes~; tous deux sont affectés à la
continuation de leurs entreprises.
  (Turgot, \emph{Réflexions sur la formation et la distribution des
richesses}, s.l. 1788, p. 125)%
\footnote{{[}§ XCIII / \emph{The
  money-lending capitalist is part of the available class} / We have
  seen that any monied man necessarily owns either capital constituted
  of transferable riches or a property equivalent to capital. Landed
  properties are always equivalent to capital; therefore all land\-owners
  are capitalists, but not all capitalists own property; and the owner
  of transferable capital can choose to use it to buy property or to
  invest it in enterprises of the agricultural or industrial class. The
  capitalist who has become entrepreneur in agriculture or industry is
  no more available, neither he himself nor his profits, than the simple
  worker of these two classes; both are engaged in the continuation of
  their enterprises.{]}}
\end{quote}

As one can see, the term is now completely detached from its original
fiscal context and simply refers to wealthy individuals who try to
increase their capital by either lending money at interest or investing
it in productive enterprises (directly, or on the stock market). The
meaning, therefore, roughly corresponded to both  senses 2 and 3 of
the \emph{Free Dictionary} quoted in Section 2. It was not really a
French innovation: already the Dutch capitalists typically engaged in
precisely these two activities. What is new is that the word could now
be used without reference to the particular Dutch context and that the
fiscal perspective to which the Dutch term was originally tied had sunk
into oblivion. By the same token, the original concept was simplified,
being stripped of its fiscal facet.

\subsection{\textsc{Capitalist} spilling over to the Anglo-Saxon world}

Nowadays we strongly associate capitalism with the Anglo-Saxon world,
but the truth is that Great Britain and the United States were the last
among the big, developed nations to take up the word \textsc{capitalist}.
In English, \emph{capitalist} does not make its appearance before 1787,
when the following example is attested in Madison's writings (\emph{The
Writings of James Madison}, ed. G. Hunt. New York/London: Putnam's Sons
1903, vol. 4, p. 123):\footnote{The first attestation given in the \emph{OED}
  is from 1792.} ``In other Countries this dependence results in some
from the relations between Landlords and Tenants in others both from
that source and from the relations between wealthy \textbf{capitalists}
and indigent labourers.'' Four years later, the word is used in England
by Edmund Burke:

\begin{quote}
On the policy of that transfer I shall trouble you with a few thoughts.
In every prosperous community something more is produced than goes to
the immediate support of the producer. This surplus forms the income of
the landed \textbf{capitalist}. It will be spent by a proprietor who
does not labour. (Edmund Burke, \emph{The Political Magazine} 21, 1791, p. 75)
\end{quote}

Up to that moment, capitalists were generally referred to as
\emph{monied men} in English, an expression that rapidly succumbed to
the prestige of the newcomer, but not before giving rise, for a short
period of time, to the blend \emph{monied capitalists}. There can be no
doubt that French was the donor language for the English calque.

\subsection{The capitalist as entrepreneur}

From the 17\textsuperscript{th} century to the
19\textsuperscript{th} century, the dominant meaning of
\textsc{capitalist} in all European languages was that of a wealthy
person who made his capital ``work'' by lending it at interest, buying
bonds or shares, or investing it in productive activities. In this last
case, a capitalist could easily become an entrepreneur himself,
directly engaged in the management of the firm he owned or of which he
was an associate. By shifting the attention from the `monied man' sense
to this latter facet of the complex concept `capitalist', the word
eventually also became established  in the new sense of `entrepreneur',
defined in the \emph{Free Dictionary} as `a person who organizes,
operates, and assumes the risk for a business venture'. As already
observed by %
%Passow (1927: 109-111)
\citet[109--111]{Passow27}%
%Passow
%
, this shift in meaning  occurred first
in English:

\begin{quote}
When the manufacturing \textbf{capitalist} of Europe shall advert to the
many important advantages, which have been intimated, in the course of
this report, he cannot but perceive very powerful inducements to a
transfer of himself and his capital to the United States. (\emph{The
American Museum}, Philadelphia: Carey 1792, Part I, from January to
June, Appendix II, p. 19)
\end{quote}

\begin{quote}
All the laws connected with our manufacturing system, appear to be
foun\-ded on one erroneous principle, that the \textbf{capitalists} or
masters are the only part to be protected against combination and
injustice, though the artizans or workmen have~an equal right to be
protected in their property or skill {[}\ldots{}{]}.
(\href{https://books.google.co.uk/books?id=f1dFAAAAYAAJ\&pg=PT537\&dq=\%
22capitalists\%22\&hl=de\&sa=X\&ved=0ahUKEwjI9r33rZbOAhUEuRQKHYi9BFo4ChDoAQgbMAA
}{\emph{The
Parliamentary Debates from the Year 1803 to the Present Time}}. Vol.
23. London: Longman 1812. July 21, 1812 -- column 1165)
\end{quote}

\begin{quote}
The small farmer has disappeared, and the smaller manufacturers are
superseded by large \textbf{capitalists}, who alone can afford to
purchase expensive machinery.
(\href{https://books.google.co.uk/books?id=oUlBAAAAYAAJ\&pg=PA6\&dq=\%
22capitalists\%22\&hl=de\&sa=X\&ved=0ahUKEwiK88jqr5bOAhUOrRQKHTg6DU04HhDoAQhVMAg
}{\emph{Remarks
on the Practicability of Mr. Robert Owen's Plan to Improve the Condition
of the Lower Classes.}} London: Leigh 1819, p. 6)
\end{quote}


The new sense may have arisen in English at that time due to the lack of 
specific word for `entrepreneur'
(\emph{entrepreneur} in the relevant sense dates from the
mid-19\textsuperscript{th} century). What is more surprising is that
this English usage should be taken over by French, where the word
\emph{entrepreneur}, which English was to borrow a few decades later,
was already well established. One precocious example which, at least at
first sight, seems relevant in our context is the following  from
Charles Caseaux' \emph{Considérations sur les effets de l'impôt dans les
différens modes de taxation}:\footnote{Note
  that Caseaux lived in London at that time.}

\begin{quote}
{[}\ldots{}{]} on doit toujours distinguer avec le même soin deux
espèces de \textbf{capitalistes} ou propriétaires~; j'appelle les uns
\emph{capitalistes de la terre}, et les autres \emph{capitalistes de
l'industrie}~: ---les capitalistes de la terre ou territoriaux, sont
non-seulement les propriétaires du grand capital de la terre mais ceux
de toutes les espèces de capitaux nécessaires pour tirer du grand
capital, tout le produit dont il est susceptible~: ---les capitalistes
industriels, ou de l'industrie, sont les différens propriétaires
non-seulement du capital en argent qui met journellement le travailleur
en action dans l'industrie comme il le met sur la terre, mais de tous
ces autres capitaux appelés bâtimens, ustensiles, machines,
\emph{crédit} même etc.
  (Charles Caseaux, \emph{Considérations sur les effets de l'impôt dans les
différens modes de taxation}, London: Spilsbury 1794, p. 98)%
\footnote{{[}one always has to distinguish
  carefully two types of capitalists or owners; I call the first one
  \emph{landed capitalists}, and the other \emph{manufacturing
  capitalists}: ---the landed capitalists are not only the owners of the
  important capital of the land but also of all kinds of capital
  necessary for deriving from the land all the produce it can yield:
  ---the manufacturing capitalists are owners not only of the money that
  makes workers become active in the factory as it does on the land, but
  of all the other capitals called buildings, tools, machines, even
  \emph{loans}, etc.{]}}
\end{quote}


This use of \emph{capitaliste} by Caseaux straightforwardly ties in with
his Physiocratic background: the capitalist, for him, is not simply a
money-lender but the person who provides capital in the broad sense of
the word, that is, including both fixed (land, buildings, machinery,
tools) and circulating (intermediate goods, operating expenses) capital.
Jean Baptiste Say, in the fourth edition of his \emph{Traité d'économie
politique}, is well aware
of the potential dangers of the polysemy of the term \textsc{capitalist}
and therefore carefully demarcates the concept `capitalist' from that of
`entrepreneur':

\begin{quote}
\textsc{Capitaliste}~; est celui qui possède un \emph{capital} et qui le
fait valoir par lui-même, ou bien le prête, moyennant un \emph{intérêt},
à l'\emph{entrepreneur d'industrie} qui le fait valoir, et dès lors en
\emph{consomme} le \emph{service} et en retire les \emph{profits}.
{[}\ldots{}{]} Un entrepreneur d'\emph{industrie agricole} est
\emph{cultivateur} lorsque la terre lui appartient~; \emph{fermier}
lorsqu'il la loue. Un entrepreneur d'\emph{industrie manufacturière} est
un \emph{manufacturier}. Un entrepreneur d'\emph{industrie commerciale}
est un \emph{négociant}. Ils ne sont \emph{\textbf{capitalistes}} que
lorsque le \emph{capital}, ou une portion du capital dont ils se
servent, leur appartient ; ils sont alors à la fois \emph{capitalistes}
et \emph{entrepreneurs}.
  (Jean Baptiste Say, \emph{Traité d'économie
politique}, 4th edition, Paris: Deterville 1819, vol. 2, pp. 456, 469)%
\footnote{{[}\textsc{Capitalist}: one who
  possesses capital and puts it to use himself or lends it to an
  entrepreneur on interest who then consumes its service and reaps the
  profit made. {[}\ldots{}{]} An entrepreneur in \emph{agriculture} is
  called a \emph{farmer} if he owns the land, a \emph{tenant} if he
  rents it. An entrepreneur in \emph{industry} is called a
  \emph{manufacturer}. An entrepreneur in trade is a \emph{merchant}.
  They are only \emph{capitalists} if they own the \emph{capital}, or
  part of the capital they use; in that case, they are at the same time
  \emph{capitalists} and \emph{entrepreneurs}.{]}}
\end{quote}

Despite Say's efforts at clarifying the meaning of \textsc{capitalist},
some of his French compatriots yielded to the new English semantics,
using \emph{capitaliste} in lieu of \emph{entrepreneur} or \emph{patron}
`master' and opposing it with \emph{ouvrier} or \emph{travailleur}
`worker'. The English usage may have crept into the French language
through translations such as the following:

\begin{quote}
Nouveau système d'association entre les petits \textbf{capitaliste}s et
les ouvriers, proposé par l'auteur (Babbage, Charles \emph{Traité sur
l'économie des machines et des manufactures.} Traduit de l'anglais par
Éd. Biot. Paris~: Bachelier 1833, p. xiv)

Les marchandises étant le produit du capital et du travail, sont la
propriété commune du \textbf{capitaliste} et du travailleur (ici
ouvrier).
(\href{https://books.google.at/books?id=o55UAAAAcAAJ\&pg=PA179\&dq=\%
22capitaliste\%22\&hl=de\&sa=X\&ved=0ahUKEwj7haSz_ZXOAhUG7BQKHTW2D7k4FBDoAQgjMAE
}{\emph{Contes
de Miss Harriet Martineau sur l'économie politique. Traduits de
l'anglais par B. Maurice.}} La Haye~: Vervloet 1834, p. 179)
\end{quote}

From the mid-1830s onwards, this new usage also became quite frequent  in
texts written by French authors and was to establish itself alongside the
more restrictive traditional use (respectively senses II.A and II.B in
the \emph{TLFi}):

\begin{quote}
Maintenant cherchons la loi qui détermine le taux des profits. Cette loi
devra avoir un rapport intime avec celles des salaires, car le
\textbf{capitaliste} et le travailleur se partagent le même produit.
(\emph{Journal général de l'Instruction publique}, nouvelle série, vol.
7, nr. 95 (1838), p. 1005 {[}Pellegrino Rossi{]})

Il s'agissait de la grande question de la lutte établie entre le
\textbf{capitaliste} et le salarié, entre l'entrepreneur et l'ouvrier,
de la question du paupérisme enfin. (\emph{Mélanges Religieux}, vol. 1,
nr. 21, 11 juin 1841, p. 331)

Malheureusement la question du salaire se compliqua de celle de la
jouissance de la case et du terrain en dépendant, et, ainsi
enchevêtrées, elles donnèrent lieu aux plus grandes difficultés entre le
\textbf{capitaliste} et le travailleur. (Milliroux, Félix
\href{https://books.google.at/books?id=FwsUAAAAIAAJ\&pg=PA31\&dq=\%22capitaliste
\%22\&hl=de\&sa=X\&ved=0ahUKEwjT0cOwhpbOAhXGvhQKHUD0Cx84KBDoAQgoMAI}{\emph{
Demerary,
transition de l'esclavage à la liberté. }}Paris: Fournier 1843, p. 31)
\end{quote}

It is easy to see that the rise of the `entrepreneur' sense of
\textsc{capitalist} goes hand in hand with the progress made by the
Industrial Revolution, where France followed England with a certain time
lag. It was the Industrial Revolution that provided capitalists with new
opportunities to put their wealth to use by engaging in industrial
activities, instead of lending money on interest or speculating on
sovereign debt or the shares of trading companies. This new opposition
between capitalists and workers will be of crucial importance for the
further fate of our word family from the mid-19\textsuperscript{th}
century onwards.

From a linguistic point of view, this second semantic change of
\textsc{capitalist} is another example of a shift of emphasis that took
place within a complex concept, mirroring changes that had previously
occurred in the extra-linguistic world. Examples such as these make it
clear that what we call ``semantic'' change in historical linguistics
cannot be described on the basis of a minimalist semantics as conceived
by the structuralists and other semanticists, but needs to take into
account concepts in all their encyclopedic richness. It should also be
mentioned here that the rise of the `entrepreneur' sense led to a
decrease in transparency of \textsc{capitalist}, since the new technical
sense of \textsc{capital} on which it was based, introduced by the
Physiocrats and focusing on land, buildings, machinery, raw materials
and intermediate goods more than solely money, had not become familiar
to the speech community at large. The relationship between base and
derivative, which had been quite transparent in the `monied man' sense,
thereby became somewhat obscured for ordinary speakers.

\section{\textsc{capitalism}}

Throughout the 17\textsuperscript{th} century and most of the
18\textsuperscript{th} century, the noun \textsc{capitalist} was an
``only child'', pertaining to a word family with only two members,
\textsc{capital} and \textsc{capitalist}. At the beginning of the
19\textsuperscript{th} century, however, this nuclear family started
expanding in several directions. With \textsc{capitalism}, a little
brother was born, and \textsc{capitalist} itself brought into the world
an adjectival progeny, as we will see in Section 5. At the same time,
complex incestuous relations developed between \textsc{capitalism} and
\textsc{capitalist}, both in their nominal and  adjectival uses. In
this section, we will follow the development of \textsc{capitalism} from
its obscure beginnings to its establishment as one of the key notions
of modern economic and political discourse in the mid-19\textsuperscript{th} century.

\subsection{\textsc{Capitalism} `condition of being rich' (1753): a ghost word?}

%
%Dauzat (1972) 
\citet{Dauzat72} %
%Dauzat
%
claimed that French \emph{capitalisme} was used as early
as 1753 in the \emph{Encyclopédie} with the meaning `état de celui qui
est riche'.\footnote{{[}condition of being rich{]}} He was followed on
this point by the \emph{TLFi}, while %
%Braudel
\citeauthor{Braudel79}%
%
's search for the text
alluded to by Dauzat yielded no result: ``Le texte invoqué reste
introuvable.''\footnote{{[}The text alluded to is nowhere to be found{]}}
(1979, vol. 2, p. 205). I could not find it either in the electronic
version of the \emph{Encyclopédie} that we have at our disposal
nowadays.\footnote{Cf. http://portail.atilf.fr/encyclopedie/.} It is
difficult to imagine that Dauzat should have invented his early first
attestation, but something must have gone wrong. In fact, neither can the French word be found with Google Books in the entire second half of the  18\textsuperscript{th} century.

However, this latter source provides one isolated early attestation of
German \emph{Kapitalismus}, a clearly jocular occasionalism from
Itzehoe's \emph{Komische Romane} (Göttingen: Dieterich 1787, vol. 4, p.
304), in a text full of somewhat contrived neologisms. It seems to
express very much the same sense as the one indicated by Dauzat for
French: ``Der Redakteur dieser Papiere, der, wie aus allen seinen
Schreibereyen hervorgeht, sich voll tiefer Ehrerbietung gegen jegliches
Menschengesicht fühlt, das nur halbwege mit dem Stempel der
Vornehmigkeit und des \textbf{Kapitalismus} gemarket ist, sieht sich
hier in großer Verlegenheit.''\footnote{{[}The editor of these papers
  who, as can be seen in all his writings, feels deference for any human
  face that somehow expresses high rank and capitalism, faces great
  embarrassment here.{]}} Since there are no other examples for German
either until around 1840, it is best to leave this potential proto-use
of \textsc{capitalism} as a riddle for future research and turn to its first appearance
 in the 19\textsuperscript{th} century.

\subsection{\textsc{Capitalisme} `high finance' (\emph{ca.} 1810)}

At the time of the French Revolution, the noun \emph{capitaliste} had
acquired distinctly negative overtones, referring to individuals who had
enriched themselves in the political and economic turmoil of those
years, to the detriment of the general good %
%(cf. Höfer 1986)
\citep[see][]{Hoefer86}%
%Höfer
%
. We should keep this
background  in mind in order to understand the
following passage, written at the moment when Napoleon had reached the
climax of his power (around 1810) and drawn from a letter addressed to a
statesman by an ``agent observateur'' whose name is not disclosed:

\begin{quote}
Mais qui {[}sic{]} dire de cette puissance nouvelle du
\textbf{Capitalisme}, qui née du commerce qu'elle ruine, a succédé avec
toute son immoralité, à la puissance si morale de la fructification du
sol qu'elle opprime en détournant ses capitaux~? de cette puissance qui
sacrifie l'avenir au présent, et le présent à l'individualité, cette
lèpre contemporaine. Cette puissance égoïste, cosmopolite, qui s'empare
de tout, ne produit rien et n'est infiniment liée qu'à elle-même~;
souveraine des souverains qui ne peuvent sans elle ni faire la guerre ni
demeurer en paix~; et qui s'enrichit également de leur prospérité et de
leur ruine, des biens du peuple qu'elle partage, de leurs maux qu'elle
accroît~?
  (Alphonse de Beauchamp,  \emph{Mémoires tirés des papiers d'un homme
d'État}. Paris: Michaud 1836, vol. 11, p. 46)%
\footnote{{[}What should one say about this new power of
  capitalism, which arose from the commerce that it ruins and with all
  its immorality succeeded the highly moral power of agriculture that it
  oppresses by diverting its capital? About this power which sacrifices
  the future to the present, and the present to individualism, the
  leprosy of our days. This egoistical, cosmopolitan power that grabs
  everything, does not produce anything and is only infinitely tied to
  itself; sovereign of sovereigns, who cannot without it make war nor
  remain in peace; and that enriches itself both by their prosperity and
  their ruin, at the expense of the goods of the people that it divides
  up, of their troubles that it increases?{]}}
\end{quote}

The `high finance' sense of \emph{capitalisme}, however, does not seem
to have had a wide circulation. We meet it again in 1822 in Georges
Laurent Aubert du Petit-Thouars' \emph{Toujours la guerre au cadastre
français}, where it is used as antonym of
\emph{propriété}, designating a society dominated by rentiers rather
than the landowning class:

\begin{quote}
Deux individus, l'un capitaliste et l'autre propriétaire, ont chacun
vingt-cinq mille livres de rente~; {[}\ldots{}{]}. Ainsi la propriété,
seul et véritable soutien des monarchies, perd tous les jours en France
de son ascendant au profit du \emph{capitalisme} qui de sa nature tend
toujours au républicanisme~: chaque jour nous le prouve.
(Georges
Laurent Aubert du Petit-Thouars, \emph{Toujours la guerre au cadastre
français}, Paris: Trouvé 1822, p. 42)%
\footnote{{[}Two
  individuals, a capitalist and a landowner, both have an income of
  25,000 pounds; {[}\ldots{}{]} In that way ownership, the only true
  support of monarchies, loses influence day by day to the benefit of
  \emph{capitalism}, which by its very nature tends towards
  republicanism: each day proves this to be the case.{]}}
\end{quote}

Significantly, the word is written in italics in order to highlight its
novelty. Our third example appears three years after the publication of
Beauchamp's work, in which the anonymous observer's invective quoted
above had been made public, in Pons Louis François de Villeneuve's
\emph{De l'agonie de la France}:

\begin{quote}
Avec le malaise ou l'instabilité de la fortune privée, concorde le
malaise encore plus pénétrant de la fortune sociale~: et un mal nouveau,
le \textbf{capitalisme}, insinuant et dangereux serpent, étouffe en ses
plis et replis l'une et l'autre. {[}\ldots{}{]} Autre et plus féconde
proie est pour le \textbf{capitalisme} la fortune publique. Il en pompe
les budgets par la rente~; il fait comme à son gré la paix ou la
guerre.
(Pons Louis François de Villeneuve,
\emph{De l'agonie de la France}, Paris: Perisse 1839, pp. 139-140)%
\footnote{{[}The difficulties and instability of private fortunes
  matches the even greater difficulties of public fortune: and a new
  evil, capitalism, this insinuating and dangerous snake, suffocates in
  its folds the one and the other. (\ldots{}) Another, even more fertile
  prey for capitalism is the public fortune. It sucks the budgets by
  means of government bonds; it makes war and peace as it pleases.{]}}
\end{quote}

These three examples of \emph{capitalisme} are still transparently tied
to the old sense of the word \emph{capitaliste}, referring to a very
wealthy individual lending his money at interest or placing it in bonds
or shares. What is less immediately obvious is the patterns of word formation 
by means of which  this word came into being. Was it derived from
\emph{capitaliste} by affix substitution? Was it an independent
derivation from \emph{capital}? Nouns in -\emph{isme}, at any rate, were
already in use at that time for designating economic systems, witness
\emph{colbertisme} (1775, \emph{TLF-Étym}) and \emph{mercantilisme}
(1809, \emph{TLF-Étym}).\footnote{Similar formations from outside the
  economic sphere were already older; see \emph{marianisme}
  (1665, \emph{TLF-Étym}), \emph{spinozisme} (1685, \emph{TLF-Étym}),
  etc.} Thus, from a chronological perspective, these words could have
served as models for \emph{capitalisme}. The corresponding nouns in
-\emph{iste}, \emph{colbertiste} and \emph{mercantiliste}, designated
the supporters of the respective doctrine. Since \emph{capitaliste} did
not refer to a supporter, but to a profession or occupation,
\emph{capitalisme}, for semantic reasons, could not be derived by affix
substitution according to a proportional analogy of the kind
\emph{colbertiste}~:~\emph{colbertisme}~= \emph{capitaliste}~:~\emph{x}.
The more plausible solution, therefore, is to consider \emph{capitalisme}
to have been an independent derivation on the basis of \emph{capital},
following the general pattern noun + -\emph{isme} `(economic) system
somehow related to N'.

\subsection{\textsc{Capitalism} as the antonym of \textsc{socialism}}
\largerpage[-1]
As we saw in Section 3.5, \textsc{capitalist} acquired the sense
`entrepreneur' after having crossed the Channel (and the Atlantic), a
sense that migrated back to France from the 1830s onwards, where it has
cohabitated with the original sense ever since. \emph{Capitaliste}, in
that way, became the antonym of \emph{ouvrier}, \emph{travailleur} (both
`worker') and \emph{prolétaire} `proletarian', just like \emph{capital}
`capital' had become the antonym of \emph{travail} `work'. This lexical
opposition simply reflected an extra-linguistic phenomenon,
namely the well-known social divide created by the Industrial Revolution.
In the 1840s, French \emph{capitalisme} was also attracted by this
lexical field and thereby was converted into the standard designation of the
new economic system characterized by the exploitation of workers in
factories owned and often run by a small group of
capitalists/entrepreneurs. Here are some of the first examples of this
new sense, which are probably attributable to Louis Blanc:\footnote{See already%
%Silberner (1940)
\citet{Silberner1940}%
%
%%
.}

\begin{quote}
Une lutte récemment engagée entre Lamartine et L. Blanc a donné
naissance à un nouveau mot~; le capitalisme. Ce n'est pas au capital,
s'écrie ce dernier, que nous avons déclaré la guerre, mais au
capitalisme~; c'est-à-dire, sans doute, aux capitalistes. (\emph{Mémoires de l'Académie royale des sciences,
belles-lettres et arts de Lyon} 1, 1845, p. 282, n. 1)%
\footnote{{[}A
  quarrel that recently opposed Lamartine to L. Blanc has given rise to
  a new word, capitalism. It is not to capital, claims the latter, that
  we have declared war, but to capitalism; that is, no doubt, to
  capitalists.{]}}

L'orateur compare la féodalité ancienne avec le capitalisme actuel. La
féodalité protégeait du moins l'exploitation de la terre, et par
conséquent le travail de l'ouvrier, tandis que le capitalisme exploite
l'ouvrier lui-même. (\emph{L'Ami de la religion} 138,
1848, p. 621)%
\footnote{{[}The speaker compares feudalism with
  present-day capitalism. Feodalism at least protected the exploitation
  of the land, and hence the activity of the worker, while capitalism
  exploits the worker himself.{]}}
\end{quote}

In this new `economic system' sense, \emph{capitalisme} became the
antonym of an alternative system where the workers themselves would own
the capital that forms the basis of their activity. Avril, V.
\emph{Histoire philosophique du crédit} (Paris: Guillaumin 1849, vol.
1, p. 153) already explicitly opposed \textsc{capitalism} and
\textsc{socialism}: ``la différence radicale qui sépare le capitalisme du
socialisme''.\footnote{{[}the radical difference that opposes capitalism
  and socialism{]}} \emph{Socialisme} (1831, \emph{TLFi}) had already
been in use for more than a decade when \emph{capitalisme} in this new
sense appeared, and \emph{communisme} (1840, \emph{TLFi}) for a few
years. Both may well have served as its immediate models.

The case of \emph{capitalisme} in the sense discussed here aptly
illustrates the complex factors that come into play in the creation and
diffusion of a neologism. The \emph{TLFi}'s statement that it is
composed of a base \emph{capital} and a suffix -\emph{isme} is
acceptable as a synchronic, though not particularly revealing,
description of the word's internal makeup, but hardly qualifies as an
etymology doing justice to the circumstances of the word's creation. At
the outset, we have to admit that the lack of documentation does not yet
allow us to gain full certainty about how it came into being, the most
plausible scenario being the following: Assuming that the `high
finance' sense was known to the coiner, which seems likely, we should
consider the process as one of semantic change, a conceptual adaptation
of the `high finance' sense to the new situation of capitalists acting
themselves as entrepreneurs, and not just as financiers. From that
perspective, the new lexical opposition with \emph{socialisme} and
\emph{communisme} could be viewed either as a consequence of this
conceptual change, or as its trigger. In fact, the relevant meaning of
these two terms, namely an `economic system where the means of production
pertains to the workers or to society as a whole', called for a
designation for the opposite concept of an economic system where the
means of production was concentrated in the hands of a small group of
wealthy individuals. Since this means of production was referred to
technically as \emph{capital} and the entrepreneurs had come to be
called \emph{capitalistes}, \emph{capitalisme} was a natural choice.
This reconstruction of the word's origin also neatly explains why the
word was used with negative connotations right from the beginning: it
was launched by the opponents of capitalism, while capitalists
themselves and circles close to them used to call the then prevailing
economic system \emph{libéralisme} (\emph{économie de marché} `market
economy' is of much more recent vintage). The transition from the `high
finance' sense to the `economic system' sense was therefore essentially
a process of conceptual rearrangement within an existing lexeme.
Nevertheless, word formation also came into play, namely by licensing the
pattern noun + -\emph{isme} with the overall meaning `system somehow
related to N' (note that both \emph{socialisme} and \emph{communisme}
have adjectival bases; therefore, strict proportional analogy with these two words
 would not suffice).

\subsection{The further fate of \textsc{capitalism}}
\largerpage
The French neologism \emph{capitalisme} in its `economic system' sense
had an immediate and resounding international success in the wake of the
1848 revolution. I will not describe here the diffusion of the
term in different European languages,\footnote{For German, see %
%Hilger (1982)
\citet{Hilger82}%
%
.} but concentrate instead on its further development in French.

By a simple metonymic process, designations of systems and similar
abstract entities are routinely taken to refer to the persons who
represent or support the system. Such was also the case with
\emph{capitalisme}. The first example of Section 4.3 could already be
interpreted in that sense. Here is a later and clearer example of this
collective sense (Burg, Joseph \emph{De la} \emph{vie sociale}\ldots{}
Rixheim: Sutter 1885, p. 739): ``Le capitalisme, dur et arrogant,
coudoie le paupérisme, exaspéré et découragé.''\footnote{{[}Capitalism,
  hard and arrogant, rubs shoulders with pauperism, exasperated and
  discouraged.{]}}

A more interesting conceptual change occurred at the beginning of the
20\textsuperscript{th} century. At that time, academic circles began
using the term not only to refer to the contemporary economic system,
what we now call \textsc{industrial capitalism}, but also to economic
systems of past times that, in their opinion, presented sufficient
similarities with the contemporary system to be called
\textsc{capitalism}. Proto-capitalism was located in the Renaissance, in
the Middle Ages, or even in Antiquity. This conceptual change, which was
the result of conscious conceptual manipulation for scientific purposes,
resulted in a more abstract concept of capitalism, freed from some of
the more contingent aspects of 19\textsuperscript{th} century industrial
capitalism, as well as its negative overtones. In France, the historian
Henri Hauser was the first to deal with the origins of capitalism in
\emph{Les Origines du capitalisme moderne en France} (Paris: Larose) in
1902. However, the international success of this scientific sense was
certainly due to the publication, some months before, of Werner
Sombart's monumental \emph{Der modern Kapitalismus} (Leipzig: Duncker \&
Humblot 1902). If Hauser had been inspired by Sombart, the new sense
would have to be classified as a calque.

\section{\textsc{Capitalist} going adjectival}

\textsc{Capitalist}, as we saw in Section 3, started out as a noun, and
it remained exclusively nominal until the end of the
18\textsuperscript{th} century. It is at that time when French
\emph{capitaliste} developed adjectival uses that are still parts of the
language. Three different adjectival senses must be distinguished: 1.
`owning (a huge amount of) capital', 2. `of capitalists', and 3. `of
capitalism'.

\subsection{\emph{Capitaliste} adj. `owning (a huge amount of) capital'}

As early as 1790,
\href{https://www.google.at/search?hl=de\&tbo=p\&tbm=bks\&q=inauthor:\%22Charles
-Nicolas+Ducloz-Dufresnoy\%22}{
Charles-Nicolas Ducloz-Dufresnoy}, in his \emph{Observations sur
l'état des finances}, quotes a
``publiciste'' called Cerruti who wrote:

\begin{quotation}
On ne peut appauvrir la Capitale sans appauvrir les Provinces dont elle
assemble, grossit, répartit et multiplie les richesses territoriales et
industrielles.

Voilà la véritable idée d'une Capitale.

Voilà la véritable idée des Capitalistes.

Le \textbf{peuple Capitaliste} est composé de tous ceux qui par leur
économie ou par leur activité, ont formé des trésors disponibles prêts à
circuler, prêts à se reposer, prêts à se transformer en papier, prêts à
se réaliser en terres.
  (Charles-Nicolas Ducloz-Dufresnoy, \emph{Observations sur
l'état des finances}, Paris: Clousier 1790, pp. 14-15)%
\footnote{{[}One cannot make the capital poorer
  without making poorer the provinces whose agricultural and industrial
  wealth it assembles, increases, distributes and multiplies. / This is
  the true idea of a capital. / This is the true idea of capitalists. /
  The capitalist people is composed of all those who through their
  savings and activity have formed treasures ready to circulate, ready
  to lie idle, ready to be transformed into paper, ready to be realized
  as landed property.{]}}
\end{quotation}

In the first half of the 19\textsuperscript{th} century this possessive
use of \emph{capitaliste} established itself in wider circles, as the
following examples show:

\begin{quote}
l'aristocratie territoriale adoucit vis-à-vis des campagnes
l'aristocratie \textbf{capitaliste} (Laborde, Alexandre de \emph{Des
aristocraties représentatives}. Paris: Le Normant 1814, p. 96)\footnote{{[}the
  landed aristocracy makes the capitalist aristocracy more acceptable
  for the countryside{]}}

comme s'il ne suffisait pas {[}\ldots{}{]} d'un imprimeur
\textbf{capitaliste} ou laborieux pour multiplier ces produits
(\href{https://books.google.at/books?id=Pz9BAAAAcAAJ\&pg=RA1-PA452\&dq=\%
22capitaliste\%22\&hl=de\&sa=X\&ved=0ahUKEwjqzZ6N_5XOAhVCchQKHe7-DsM4MhDoAQgiMAE
}{\emph{Revue
encyclopédique}}, t. 49, janvier-mars 1831, ‎p. 452)\footnote{{[}as if
  it were not enough {[}\ldots{}{]} to have a well-capitalized or
  hard-working type-setter in order to multiply these products{]}}
\end{quote}

\begin{quote}
{[}la législation des Émigrés{]} a rendu le peuple propriétaire et la
noblesse \textbf{capitaliste} (Lahaye de Cormenin, Louis-Marie de
\href{https://books.google.at/books?id=P8tIAAAAcAAJ\&pg=PR37\&dq=\%22capitaliste
\%22\&hl=de\&sa=X\&ved=0ahUKEwiK1NTVgpbOAhUIGhQKHTlEAIc4ggEQ6AEIXzAJ}{\emph{
Droit
administratif. Paris: Thoral 1840, t. 1, p. xxxvii}})\footnote{{[}{[}the
  legislation on emigrants{]} has turned the people into owners and the
  aristocracy into capitalists{]}}
\end{quote}

\begin{quote}
La bourgeoisie moderne {[}\ldots{}{]} forme une espèce d'aristocratie
\textbf{capitaliste} et foncière, {[}\ldots{}{]}. (Proudhon,
Pierre-Joseph
\href{https://books.google.at/books?id=uzk-AAAAcAAJ\&pg=RA2-PA21\&dq=
aristocratie+capitaliste\&hl=de\&sa=X\&ved=
0ahUKEwj_4aPTz5jOAhVIvxQKHbFUAPwQ6AEIPDAE}{\emph{Organisation
du crédit et de la circulation.}} Paris: Garnier 1848, p. 21)\footnote{{[}The
  modern bourgeoisie {[}\ldots{}{]} forms a kind of capitalist and
  landed aristocracy{]}}
\end{quote}

\begin{quote}
Ce n'est pas la bourgeoisie qui est boursière, c'est la société tutta
quanta qui veut être \textbf{capitaliste} en exploitant les éventualités
des échanges. (Bianchini, Lodovico \emph{La science du bien-être
social}. Bruxelles: Librairie universelle 1857, p. 351)\footnote{{[}It is not
the bourgeoisie who is crazy about the stock market, it is the entire society that wants to be capitalist by taking advantage of the opportunities of
trading.{]}}


\end{quote}

From a linguistic point of view, the meaning `owning (a huge amount of)
capital' constitutes a case of noun-adjective conversion, the base being
constituted by the noun \emph{capitaliste} with the meaning `person
owning (a huge amount of) capital'. This conversion pattern does not
seem to have had any direct model among words in -\emph{iste}, none of
which had a possessive meaning, by the way, if we exclude obsolete
\emph{actioniste} `shareholder', which was also of Dutch origin. As
argued in Section 3.2, \emph{capitaliste} should be classified as a
marginal member of the agentive niche represented by words such as
\emph{aubergiste} `innkeeper', \emph{copiste} `copyist', \emph{ébéniste}
`cabinetmaker', \emph{latiniste} `Latin scholar or student',
\emph{psalmist} `psalmist'. Such nouns, however, do not seem to
have developed adjectival uses (of the relevant kind), according to the
information provided by the \emph{TLFi}.\footnote{Appositions such as
  \emph{rabbin cabaliste} `cabalist rabbi', \emph{moine copiste} `monk
  copyist', \emph{ouvrier ébéniste} `cabinet worker', etc. are
  classified as adjectival in the \emph{TLFi}, but this is highly
  questionable. Some of the nouns quoted are indeed used as adjectives,
  but in a relational sense (e.g. \emph{la tradition ébéniste} `the
  tradition of cabinet-making', etc.).} The model must therefore be
sought outside derivative patterns in -\emph{iste}.

\subsection{\emph{Capitaliste} adj. `of capitalists'}

The second adjectival sense ‒ which, incidentally, the \emph{TLFi} fails
to mention ‒ corresponds to a relational use referring to the
corresponding noun \emph{capitaliste}. Again, we find one early outlier
in 1791, this time in a translation of Adam Smith's \emph{Inquiry into
the Nature and Causes of the Wealth of Nations} :

\begin{quote}
Lorsque ces compagnies {[}\ldots{}{]} commercent avec des capitaux
réunis, et que chacun des membres a sa part dans le bénéfice commun ou
dans la perte commune, en proportion des fonds qu'il y a mis~; on les
appelle compagnies \textbf{\textsc{capitalistes}}.
(Adam Smith, \emph{Recherches sur la
nature et les causes de la richesse des nations}, translated by J. A.
Roucher, Paris: Buisson 1791, vol. 4, p. 90)
\end{quote}

This passage translates the following one from Smith's original (I 
quote here from the 9\textsuperscript{th} edition, where, as we can see, \emph{joint stock company}
corresponds to the translator's \emph{compagnie capitaliste}).

\begin{quote}
When they trade upon a joint stock, each member sharing in the common
profit or loss in a proportion to his share in this stock, they are
called joint stock companies.
(Adam Smith, \emph{Inquiry into
the Nature and Causes of the Wealth of Nations}, 9\textsuperscript{th} edition, London: Strahan 1799, vol.
3, p. 110)

\end{quote}

\emph{Compagnie capitaliste} must therefore be considered to be a
neologism created by the translator. The only other example provided by
Google Books until the mid-19\textsuperscript{th} century is the
following, which is obviously inspired by the example just quoted:

\begin{quote}
La confection ou entretien d'un canal navigable qui ne peuvent guère
être exécutés que par des \textbf{compagnies} \textbf{capitalistes},
sont des entreprises qui~portent avec elles le privilège qui garantit
aux entrepreneurs le bénéfice qu'ils doivent en retirer. (Roux, Vital
\href{https://books.google.at/books?id=H6xaAAAAYAAJ\&pg=PA257\&dq=\%
22capitalistes\%22\&hl=de\&sa=X\&ved=
0ahUKEwij7Ou8xpTOAhUrCsAKHS4aBdk4FBDoAQhEMAU}{\emph{De
l'influence du gouvernement sur la prospérité du commerce}}. Paris:
Fayolle 1800, p. 257)\footnote{{[}The building and maintenance of a
  shipping canal, which can hardly be undertaken but by a capitalist
  company, are enterprises that come with a privilege that guarantees
  the entrepreneurs the profit they can make on it.{]}}
\end{quote}

Overall, however, Rouchet's neologism did not catch on. The more common
way throughout the 19\textsuperscript{th} century of denominating a
company composed of various capitalists in French was \emph{compagnie de
capitalistes} `company of capitalists' or \emph{société de capitalistes}
`society of capitalists', both amply attested since the time of the
French Revolution.

On a larger scale, the relational sense `of capitalists' only appears
from the second half of the 19\textsuperscript{th} century onwards.
These examples, it seems, were independent from the use of
\emph{capitaliste} by Roucher in 1791 in the term \emph{compagnie
capitaliste}. It is not always easy to distinguish the relational sense
`of capitalists' from the sense `of capitalism', since
\emph{capitalisme} can also be understood metonymically as the totality
of capitalists. In the following list, I have chosen examples
where reference to capitalists seems more plausible than to capitalism
as an economic system.

\begin{quote}
{[}\ldots{}{]} à fin de se délivrer de l'exploitation
\textbf{capitaliste} et usuraire, comme ils se sont délivrés de la
tyrannie monarchique et jesuitique (Eugène Sue, \emph{Mystères du
peuple}, 1851, vol. 2, p. 90, quoted in:
\href{https://books.google.at/books?id=F1BfAAAAcAAJ\&pg=RA1-PA57\&dq=\%
22exploitation+capitaliste\%22\&hl=de\&sa=X\&ved=
0ahUKEwj2z7q8_pjOAhWlJsAKHRGiDDgQ6AEILDAC}{
\emph{Archiv des Criminalrechts.}} Neue Folge. Jahrgang 1851, p.
57)\footnote{{[}in order to free themselves from capitalist and usurious
  exploitation, as they had freed themselves from monarchic and jesuitic
  tyranny{]}}

Comme nous le disions hier, la conjuration \textbf{capitaliste},
l'alliance offensive et défensive du privilége contre le prolétariat est
formée ; il y a entente cordiale entre tous ces hommes que nous
supposions ennemis : {[}\ldots{}{]}. (Proudhon, P.-J.
\href{https://books.google.at/books?id=OdvlnzjmD9EC\&pg=PA229\&dq=\%
22conjuration+capitaliste\%22\&hl=de\&sa=X\&ved=
0ahUKEwj1_qG9tZnOAhVTrRQKHZGtAsMQ6AEIMzAD}{\emph{Mélanges.
Articles de journaux 1848-1852.}} Premier volume. Paris: Lacroix,
Verboeckhoven \& Cie 1868, p. 229)\footnote{{[}As we said yesterday, the
  capitalist conspiracy, the offensive and defensive alliance of the
  privilege against the proletariat already exists; there is an entente
  cordiale between all these men that we deemed ennemies{]}}

la tyrannie \textbf{capitaliste} et mercantile (Colins, Jean Guillaume
\href{https://books.google.at/books?id=Gu0paho7pGYC\&pg=RA1-PA56\&dq=\%
22tyrannie+capitaliste\%22\&hl=de\&sa=X\&ved=0ahUKEwjdz-
vSwJnOAhUB2RoKHRjmBX8Q6AEIHDAA}{\emph{L'économie
politique source des révolutions et des utopies prétendues socialistes.
Paris:}} Librairie générale 1856, p. 56)\footnote{{[}the capitalist and
  mercantile tyranny{]}}

Ce sera donc bien une association ouvrière. --- Ce sera une association
\textbf{capitaliste} où {[}\ldots{}{]} le travail sera subordonné au
capital.
(\href{https://books.google.at/books?id=qgEcAQAAIAAJ\&pg=PA172\&dq=\%
22association+capitaliste\%22\&hl=de\&sa=X\&ved=
0ahUKEwiE9Zb8rpnOAhUHORoKHWm_CtoQ6AEIHjAA}{\emph{Journal
des économistes, t. 15, juillet à septembre }} 1869, p. 172)\footnote{{[}This
  will therefore indeed be an association of workers. --- This will
  therefore indeed be a capitalist association where work will be
  subordinated to capital{]}}
\end{quote}

\begin{quote}
la classe \textbf{capitaliste} et la classe ouvrière {[}\ldots{}{]} dans
le milieu \textbf{capitaliste} (Marx, Karl
\href{https://books.google.at/books?id=7ZcBAAAAQAAJ\&pg=PA248\&dq=\%22classe+
capitaliste\%22\&hl=de\&sa=X\&ved=0ahUKEwiS1abDuZnOAhVD7hoKHf02A7IQ6AEIHjAA}{\
emph{Le
capital}. Tr. de J. Roy revisée par l'auteur.} Paris: Lachatre 1872,
pp. 248, 285)\footnote{{[}the capitalist class and the working class
  (\ldots{}) in capitalist circles{]}}
\end{quote}

\begin{quote}
Marx est donc bien loin d'appeler subjectivement le profit
\textbf{capitaliste} un vol
(\href{https://books.google.at/books?id=DLTNAAAAMAAJ\&q=\%22profit+capitaliste\%
22\&dq=\%22profit+capitaliste\%22\&hl=de\&sa=X\&ved=
0ahUKEwjiwdeshJnOAhUJ2BoKHZgCCFkQ6AEINjAC}{\emph{Revue
internationale du socialisme rationnel}, t. 8,} 1883, p. 147)\footnote{{[}Marx
  is therefore far from subjectively calling capitalist profit theft{]}}

le député Rasseneur parlerait de ``l'oppression \textbf{capitaliste} et
de la revanche prolétarienne'' (Bonnetain, Paul
\href{https://books.google.at/books?id=nN3KG9DsCLgC\&pg=PA581\&dq=\%
22capitaliste\%22\&hl=de\&sa=X\&ved=0ahUKEwjnvfbYsZnOAhVEHxoKHSHiDOQQ6AEIUjAI}{\
emph{L'Opium}}.
Paris: Charpentier 1886, p. 581)\footnote{{[}MP Rasseneur was said to
  speak about ``capitalist oppression and proletarian revenge''{]}}

l'avidité \textbf{capitaliste} contraint les mécaniciens des chemins de
fer à effectuer des journées de travail de dix-huit et vingt heures
(\href{https://books.google.at/books?id=wSkrAAAAYAAJ\&q=\%22avidité+capitaliste\%22\&dq=\%22avidité+capitaliste\%22\&hl=de\&sa=X\&ved=
0ahUKEwjUz7j2hpnOAhUFrxoKHQnhAT4Q6AEIHDAA}{\emph{La
Revue socialiste}, t. 10, }1889, p. 685)\footnote{{[}capitalist greed
  obliges the train drivers to work for 18 or 20 hours{]}}

la moyenne de la vie ouvrière est inférieure à la moyenne de la vie
\textbf{capitaliste} (\emph{La Réforme sociale}, t. 25, 1893, p.
467)\footnote{{[}the lifetime of a worker on average is shorter than a
  capitalist's lifetime{]}}

incapables {[}\ldots{}{]} d'opposer aux \textbf{exigences capitalistes}
une résistance efficace
(\href{https://books.google.at/books?id=LU05AQAAMAAJ\&q=\%22exigences+
capitalistes\%22\&dq=\%22exigences+capitalistes\%22\&hl=de\&sa=X\&ved=
0ahUKEwj8yImlvpnOAhXJVRoKHWH4Bo8Q6AEIKDAC}{\emph{La
Société nouvelle}}, t. 2, 1894, p. 448)\footnote{{[}unable to counter
  the demands of capitalists with an efficient opposition{]}}
\end{quote}

These examples should suffice to prove the existence of the relational
sense `of capitalists' from the mid-19\textsuperscript{th} century
onwards. This relational use followed a pattern of conversion turning
personal nouns into relational adjectives that was already quite well
established by the middle of the 19\textsuperscript{th} century, even
with nouns in -\emph{iste} %
%(cf. Rainer 2017)
\citep[see][]{Rainer17}%
%Rainer
%
. Outside nouns in
-\emph{iste}, we find the relational use of \emph{ouvrier} in
collocations such as \emph{association ouvrière} `workers' association'
and \emph{classe ouvrière} `working class; lit. workers' class' as early
as 1802 in the \emph{TLFi}. The same relational sense is also attested
in the \emph{TLFi} for \emph{prolétaire} (in the example from Bonnetain
above, though, the synonymous suffixal derivative \emph{proletarien} is
used). Since the noun \emph{capitaliste} by the
mid-19\textsuperscript{th} century had become the antonym of
\emph{ouvrier} and \emph{prolétaire}, it could well be that its
relational use was induced by the relational use of these two antonyms.
There is no need to choose between these two hypotheses: the influence
of \emph{ouvrier} and \emph{prolétaire} may well have worked in tandem
with the pattern converting nouns in -\emph{iste} into relational
adjectives.

\subsection{\emph{Capitaliste} adj. `of capitalism'}

The relational sense `of capitalism' was also established in the
French language in the middle of the 19\textsuperscript{th} century. As
we saw in Section 4, \emph{capitalisme} in the relevant sense was itself
a neologism at that time. Here are some early examples in which the
sense `of capitalists' definitively seems less adequate than the sense
`of capitalism'.

\begin{quote}
Le système \textbf{capitaliste} a été établi en France sous des
conditions bien moins propices (Sagra, Ramon de la \emph{Révolution
économique}. Paris: Capelle 1849, p. 81)\footnote{{[}The capitalist
  system has been established in France under much less favourable
  conditions{]}}
\end{quote}

\begin{quote}
le plus grand écrivain de vos théories \textbf{capitalistes} (Avril, V.
\href{https://books.google.at/books?id=oghGAQAAMAAJ\&pg=PA69\&dq=\%22théories+
capitalistes\%22\&hl=de\&sa=X\&ved=0ahUKEwjD8cL4v5nOAhXJyRoKHcMSD8kQ6AEIHjAA}{\emph{Histoire
philo\-sophique du crédit.}} Paris: Guillaumin 1849, p. 69)\footnote{{[}the
  greatest writer on your capitalist theories{]}}
\end{quote}

\begin{quote}
la négation du \textbf{régime capitaliste}, agioteur et gouvernemental,
qu'a laissé après elle la première révolution
(Proudhon,
Pierre-Joseph \href{https://www.google.at/search?biw=1440\&bih=808\&tbs=cdr:1,cd_min:01.01.
1850,cd_max:31.12.1870\&tbm=bks\&tbm=bks\&q=inauthor:\%22Pierre-Joseph+Proudhon\%22\&sa=X\&ved=0ahUKEwiXz5DoiJbOAhXIXhQKHVRtDM4Q9AgIPDAD}{\emph{Idée générale de la révolution au XIXe siecle}}.
Paris: Garnier 1851, p. 107)\footnote{{[}the negation of the capitalist,
  speculative and governmental regime left over from the first
  revolution{]}}
\end{quote}

\begin{quote}
Le résultat sera donc un accroissement de population dans le
\textbf{pays capitaliste} B. (De Laveleye, Emile
\href{https://books.google.at/books?id=tIIPAAAAQAAJ\&pg=PA88\&dq=\%22pays+
capitaliste\%22\&hl=de\&sa=X\&ved=0ahUKEwj2kuiEzpjOAhUNrRQKHQsTAk0Q6AEIHjAA}{%
\emph{Etudes
historiques et critiques sur le principe et les conséquences de la
liberté du commerce international.}} Paris: Guillaumin 1857, p.
88)\footnote{{[}The result will therefore be an increase in population
  in the capitalist country B.{]}}
\end{quote}

From a present-day perspective, this usage seems straightforward, since
most nouns in -\emph{isme} referring to ideologies and similar notions
are flanked by a relational adjective in -\emph{iste}:
\emph{marxisme}/\emph{marxiste}, \emph{racisme}/\emph{raciste}, etc.
Morphologically, the relationship between such pairs is one of affix
substitution. What is crucial in our context is whether this relation of
affix substitution was already operative in the middle of the
19\textsuperscript{th} century. The \emph{TLFi} does not provide
reliable evidence bearing on this question, since in most entries a date
of first attestation is only given for the nominal use of -\emph{iste}.
However, relevant examples are not difficult to come by. In many cases,
one may waver between the interpretations `of Xists' and `of Xism':
``mouvement anarchiste'' (d'Ivernois, Francis \emph{Les cinq promesses}.
Londres: Cox 1802, p. 149), for example, could be glossed equally
naturally as `movement of anarchists' and `movement inspired by
anarchism', ``journal légitimiste'' (\emph{Procès de M. Gisquet contre}
Le Messager. Paris: Pagnerre 1839, p. 1) as `newspaper of/for
legitimists' and `newspaper inspired by/defending legitimism'. In ``une
thèse matérialiste'' (Gibon, H. \emph{Fragments philosophiques}. Paris:
Hachette 1836, p. 69), however, `a dissertation inspired by materialism'
would seem to be the only reasonable gloss.

We can therefore safely assume that the `of capitalism' sense could be
derived, by the middle of the 19\textsuperscript{th} century, from
\emph{capitalisme} by means of affix substitution. For the sake of completeness,
however, let us still check an alternative possibility which some might
wish to entertain. As we have seen, \textsc{capitalist} already spilled
over to the Anglo-Saxon world at the end of the 18\textsuperscript{th}
century and since then it has been a much-used term in the English
language. Could it not be, therefore, that the relational sense in
question was simply due to a calque from English? In order to answer
this question, let us observe the dates of first
attestation\footnote{Using the first book allowing a full view of the
  text, front matter included, in Google Books.} of the English
collocations corresponding to those quoted above for French:
\emph{capitalist country} (1861), \emph{capitalist system} (1862),
\emph{capitalist regime} (1863), \emph{capitalist theories} (after
1900). As we can see, the English collocations follow the French ones by
a lapse of time of some 10 years. It may therefore safely be assumed
that English imitated French, not vice versa.

\section{A 20\textsuperscript{th}-century codicil: \textsc{capitalist}
`supporter of capitalism'}

As we saw in
Section 3, n the middle of the 19\textsuperscript{th} century, the `entrepreneur' sense had been added to the `monied man'
sense. In the 20\textsuperscript{th} century, a third sense was added to
these two, namely that of `supporter of capitalism', which has largely
superseded the other two. In the second half of the
19\textsuperscript{th} century, \textsc{capitalism} had evolved from a
name characterizing an economic system to that of an ideology.
Especially after the international success of Marxism,
\textsc{capitalism} became the antonym of \textsc{communism}, which
could also denote both an economic system and an ideology.
Due to this status of \textsc{capitalism} as an antonym of
\textsc{communism}, \textsc{capitalist} followed \textsc{communist} in
designating a person that embraced the ideology expressed by the
corresponding word in -\textsc{ism}. The following example illustrates
this last transformation of \textsc{capitalist} with French
\emph{capitaliste}: ``Outre la question de l'attitude du Chrétien, un
point irrite particulièrement André Gide~; c'est le reproche qui lui est
fait d'être à la fois \textbf{capitaliste} et communiste et il s'ingénie
à retourner l'accusation contre les chrétiens.''\footnote{{[}Apart from
  the question of the attitude of the Christian, one point in particular
  irritates André Gide: the reproach that is addressed to him of
  embracing at the same time the ideology of capitalism and communism,
  and he is at pains to return the charge against the Christians.{]}}
(Fillon, Amélie \emph{François Mauriac}. Paris: Société Française
d'Éditions Littéraires et Techniques 1936, p. 330). What the author 
wanted to say here is that Gide was accused of having embraced 
the ideologies of capitalism and communism at the same
time, not that he was a
financier, investor, or entrepreneur. In this latest sense one can even
be a capitalist without possessing any money or property.

From a linguistic point of view, this last transformation of
\textsc{capitalist} is to be regarded as a case of affix substitution on
the basis of \textsc{capitalism}, as the gloss `supporter of capitalism'
suggests. What is less easy to tell is whether this affix substitution
first took place in French or in some other European language, notably
English or German. The question is almost impossible to answer since at
that time these three languages were already in perfect harmony
concerning \textsc{capitalist} and \textsc{capitalism} as well as the
-\textsc{ism}/-\textsc{ist} pattern. In French, for example, this kind
of affix substitution could base itself on a sizeable number of
potential models: an \emph{anarchiste} was a supporter of
\emph{anarchisme}, a \emph{communiste} a supporter of \emph{communisme},
etc. It is worth mentioning that, from a historical perspective, the
derivative in -\emph{iste} tended to occur earlier than that in
-\emph{isme}, but at some point in time the names of the supporters came
to be reinterpreted as dependent on the names of the doctrines.

\section{Conclusion}

After having accompanied \textsc{capitalist} and \textsc{capitalism} in
their unfolding since the 17\textsuperscript{th} century, it is time to
draw some general conclusions about the relationship between word
history and word formation and to highlight the role of the lexeme in
this affair.

As we have seen, semantic change, borrowing and word formation have all
\rephrase{contributed substantially}{substantially contributed} to the evolution of these two key words of our
politico-economic vocabulary. And in each of these three modes of
lexical enrichment the lexeme has been seen to play a key role. What is
traditionally called \emph{semantic change} in reality should better be
called \emph{conceptual change}, as Andreas %
%Blank 
\citeauthor{Blank97} %
%
convincingly argues in
his %
%1997 
\citeyear{Blank97} %
%
book. The semantic changes observed in the history of
\textsc{capitalist} and \textsc{capitalism} affected holistic concepts
tied to lexemes, in close interaction with changes in extra-linguistic
reality, not affixes or roots. Borrowing also repeatedly played a role:
in the migration of \textsc{capitalist} from the United Provinces to
France, from France to the Anglo-Saxon world and back again, to mention
just those involving French. Now, calquing is a process that is also
located at the level of the lexeme. It can be conceived of as an
analogical process where model and copy are located in different
languages (though in the same speaker's mind). If seen in this light,
calquing is close to word formation, which is also best conceived of as
an analogical, pattern-based process. This is particularly obvious in
the case of affix substitution, which played a prominent role in
derivatives with -\textsc{ist} and -\textsc{ism}.

We have also seen that a full understanding of the evolution of our two
words requires taking into consideration the structure of the lexicon at
the relevant points in time. A lacuna in the lexicon may induce semantic
change, as Passow already surmised in relation to the rise of the
`entrepreneur' sense of English \emph{capitalist}. The absence of a
specific word for `entrepreneur' around 1800 may have prompted the
English speakers to adapt the meaning of \emph{capitalist},
originally referring to a rich money lender or investor, in order to
fill this empty slot. Another case in point may have been the
introduction of the `economic system' sense of French \emph{capitalisme}
in the 1840s, which filled the need for an antonym of \emph{socialisme}
and \emph{communisme}. Similarly, the specific configuration of a
semantic field may induce change, as we have seen in the case of the
opposition `entrepreneur' vs. `worker', which may have helped
to establish the relational use of French \emph{capitaliste} in the `of
capitalists' sense, providing a ready counterpart for the already
established relational use of \emph{ouvrier} and \emph{prolétaire}. The
same search for formal/semantic parallelism was probably also operative
in the rise of the `supporter' sense of French \emph{capitaliste} in the
20\textsuperscript{th} century. These latter processes can be accounted
for straightforwardly as proportional analogies.

At many points in our discussion we have seen that the French historical
dictionaries that we have at our disposal, notably the \emph{TLFi}, only
provide a shaky basis for detailed investigations into the history of
word-formation patterns in post-Renaissance French. In some sense, the
\emph{TLFi} is a marvel of a dictionary, second probably only to the \emph{OED}. Nevertheless, it is obvious in many entries that
the lexicographers where overwhelmed by the wealth of raw data at their
disposal and hampered by the lack of a sound theory of word formation
(or an inconsistent application of the theory, if they had one). The
relationship between words in -\emph{isme} and the corresponding
relational adjectives in -\emph{iste}, for example, is not given a
separate etymological treatment but identified with that of nouns in
-\emph{iste}, which are themselves handled in different ways in
different entries:

\begin{quote}
\emph{Anarchiste}: ``Dér. du rad. de \emph{anarchie}*; suff.
\emph{-iste}*''

\emph{Animiste}: ``Dér. du rad. du lat. \emph{anima (âme}*\emph{)};
suff. \emph{-iste}*''

\emph{Colbertiste}: ``du rad. de \emph{colbertisme,} suff.
\emph{-iste}*''

\emph{Cubiste}: ``Dér. de \emph{cube}*; suff. \emph{-iste}*''

\emph{Fétichiste}: ``Dér. de \emph{fétiche}* formé sur le modèle de
\emph{fétichisme}*; suff. \emph{-iste}*''

\emph{Piétisme}: ``Dér. de \emph{piétiste*}; suff. \emph{-isme*}''

\emph{Quiétiste}: ``Dér. de \emph{quietisme}* par substitution du suff.
\emph{-iste}* à \emph{-isme}''
\end{quote}

In a proper etymological treatment, each step in the history of a word,
which roughly corresponds to a word's subentries in a well-ordered
dictionary, must be provided with a separate etymological explanation,
and each explanation should explicitly name the change according to a
catalogue of standard mechanisms of lexical change. In the case of
semantic change and borrowing, a list of universal mechanisms such as
\emph{calque}, metaphor, and metonymy will generally be sufficient,
though some of these mechanisms also show language-specific
patterns that should then be named explicitly.\footnote{For example, in French or Spanish the name of the
  central product can be used to designate the respective economic
  sector or activity, while this is not an established metonymic pattern
  in German or English, witness \emph{travailler dans la tomate} /
  \emph{trabajar en el tomate} vs. *\emph{in der Tomate arbeiten} /
  *\emph{to work in the tomato}.}
For word formation, by contrast, it is vital to make sure that the
pattern alluded to in a certain etymological explanation was productive
at the moment in question.

The rather glaring shortcomings of the \emph{TLFi} in that respect are
now being emended by the \emph{TLF-Étym} project, to which I am happy to
contribute from time to time. Word histories in the \emph{TLF-Étym}
style are a necessary prerequisite for a history of word formation in
modern French,\footnote{On French -\emph{isme}, see %
%Roché (2007)
\citet{roche2007.isme-decembrettes}%
%Roché
%
. For
  Spanish, %
%Muñoz Armijo (2012)
\citet{Munoz12}%
%Muñoz Armijo
%
.} which constitutes a great desideratum. 
At the same time, detailed studies on the history of single
word-formation patterns would yield important contributions to
historical lexicography. The two fields are so intimately intertwined,
that they of necessity must evolve in tandem.



\nocite{Aronoff2007}
\nocite{Barbier44}
\nocite{Blank97}
\nocite{Braudel79}
\nocite{Corbin87}
\nocite{Dauzat72}
\nocite{Febvre39}
\nocite{Fradin03}
\nocite{Hilger82}
\nocite{Hoefer86}
\nocite{Munoz12}
\nocite{OED}
\nocite{Passow27}
\nocite{Rainer98}
\nocite{Rainer17}
\nocite{roche2007.isme-decembrettes}
\nocite{Hart93}
\nocite{Thiele74}
\nocite{wolf72}




{\sloppy
    \printbibliography[heading=subbibliography,notkeyword=this]
}
\end{document}
