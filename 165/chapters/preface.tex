\documentclass[output=paper]{langsci/langscibook} 
\ChapterDOI{10.5281/zenodo.1406985}
% \addchap{Introduction}
\author{Olivier Bonami\affiliation{Université Paris Diderot}\and Gilles Boyé\affiliation{Université Bordeaux-Montaigne}\and Georgette Dal\affiliation{Université de Lille}\and Hélène Giraudo\affiliation{CLLE, Université de Toulouse, CNRS, Toulouse}\lastand Fiammetta Namer\affiliation{Université de Lorraine}}
 
\selectlanguage{english}

\title{Introduction} 
\rohead{Introduction} 
%
\abstract{\noabstract}
\maketitle


\begin{document}

 \section{Introducing the lexeme}


It is customary (see for instance \citealt[4]{Aronoff94}) to associate the
notion of a lexeme with Peter H. \citet{Matthews65,Matthews72,Matthews74,Matthews91}.\footnote{\citet[160]{Matthews72} himself notes that the use of
  the word \emph{lexeme} in this sense originates in \citet{Lyons63}, and
  that his understanding of the lexeme is very close to that of the
  \emph{semanteme} in \citet[287]{bally}. See also \citet[28]{Trnka49}. On
  the other hand, the use of \emph{lexeme} in the tradition starting
  with Matthews has little to do with Martinet's \emph{lexème} (e.g.
  \citealt{Martinet60}), which designates what in the English-speaking world
  would be called a morpheme with lexical meaning, or a root.} \citet[160-161]{Matthews72}
 contrasts three uses of the term \emph{word} that may be
differentiated as follows. 

\begin{itemize}
\item
  The term \emph{word} may denote a certain type of syntactic constituent. In this sense, the
  term unambiguously designates a kind of Saussurean sign, possibly
  complex: it associates a phonological representation with a meaning.
  \newpage 
\item
  The term \emph{word} is used to denote the phonological sequence that is the shape of a word in
  the first sense. Matthews coins the term \emph{wordform} to designate
  this and avoid ambiguity.\footnote{\citet{Lyons68} and some more recent
    authors use \emph{phonological word} instead of \emph{wordform}.
    This is problematic, ``phonological word'' being standardly used to
    denote a particular type of prosodic constituent, which may or may
    not be coextensive with a wordform. Matthews is explicit on the
    difference between wordforms and phonological words, both in
    \citet[2, 96, 161]{Matthews72} and in the second edition of his
    textbook \citep[42, 216]{Matthews91}. Unfortunately, the first edition
    was somewhat confusing on this particular issue \citep[32-33, 35]{Matthews74}. Adding to the confusion, \citet{melcuk93} and
    \citet{Fradin03} use the French term \emph{mot-forme} (litteraly,
    ``word-form'') to denote what Matthews, and after him the whole
    English-speaking literature, simply calls  \emph{word}.}
\item
  The term \emph{word} often denotes that lexical object dictionaries talk about: an item
  characterized by a stable lexical meaning and a set of syntactic
  properties, but that abstracts away from inflection. This unit is what
  Matthews calls the \emph{lexeme}.
\end{itemize}

One may illustrate these definitions by saying that the French lexeme
\textsc{vieux} `old' is associated with four words filling the four
cells in its paradigm: \textsc{m.sg} \emph{vieux}, \textsc{f.sg} \emph{vieille}, \textsc{m.pl}
\emph{vieux}, and \textsc{f.pl} \emph{vieilles}. To these four words correspond
only three wordforms, since the \textsc{m.sg} and \textsc{m.pl} are phonologically
identical. This characterization of the lexeme is deliberately silent on
phonology: the lexeme is defined in terms of the syntactic and semantic
cohesion of a family of words, ignoring phonology. Literature from the
1990s was not so prudent, and presented the lexeme as an underspecified
sign. The following quote is representative of the dominant view:

\begin{modquote}
Each lexeme can be viewed as a set of properties, which will in some
sense be present in all occurrences of the lexeme. These crucially
include some semantic properties, some phonological properties
{[}\ldots{}{]}, and some syntactic properties. \citep[333]{Zwicky92}
\end{modquote}

Such a definition is obviously not adequate if one wants to be able to
take into account the full spectrum of stem allomorphy, including
suppletion. In some cases, there is no phonological property that is
shared by all forms of the lexeme; e.g. there is nothing common between
the 3sg forms of the French lexeme \textsc{aller} `go' in the imperfect
(\emph{allait}), present (\emph{va}) and future (\emph{ira}). This
example shows that lexemes are ineffable: one can't utter a lexeme, but
only one of its forms. It also highlights the importance of cleanly
distinguishing lexemes from their \textsc{citation form.}\footnote{The
  unfortunate use of the term \emph{lemma} in many discussions in
  psycholinguistics and Natural Language Processing rests on such a
  confusion between lexeme and citation form.} The French grammatical
tradition happens to use infinitives as citation forms, and the
infinitive of \textsc{aller} happens to use the \emph{al-} stem. From
this, no conclusion can be drawn as to \emph{al-} being a more
reflective of the fundamental phonological identity of that lexeme: if
French grammarians had kept the Latin tradition of using the present
\textsc{1sg} as a citation form, we would call the lexeme \textsc{vais},
and the \emph{v-} stem would seem crucial.

Because the definition of a lexeme derives from that of an inflectional
paradigm (lexemes abstract away from inflection), using the notion
commits one to a particular view of morphology. It presupposes the
existence of a split between inflectional and derivational morphology
(\citealt[140, note 4]{Matthews65}; \citealt{Anderson82}; \citealt{Perlmutter88}).
Delineating the sets of words instantiating the same lexeme, such as the
one shown in~(1a), requires one to distinguish it from a set of words
that merely belong to the same morphological family, as the one in~(1b).

\begin{exe}
\ex\begin{xlist}
\ex  \{ \emph{vieux} `old' \textsc{m.sg}, \emph{vieille} `old' \textsc{f.sg},
\emph{vieux} `old' \textsc{m.pl}, \emph{vieilles} `old' \textsc{f.pl} \}
\ex 
\{ \emph{vieux} `old' \textsc{m.sg}, \emph{vieillard} `old man' \textsc{sg},
\emph{vieillesse} `old age' \textsc{sg} \}
\end{xlist}
\end{exe}

As characterised above, the lexeme is a descriptive category. As such it
is compatible with diverse models of morphology, as long as they
implement a notion of structured paradigms and split morphology. In
practice, however, the notion of a lexeme is mainly used within
theoretical frameworks that adopt a constructive view of morphology
\citep{Blevins06} and use the lexeme as the pivot of the theory, linking
inflection and derivation. Following \citet{Fradin03}, we may call this
family of frameworks \textsc{lexemic morphology}, and assume that they
rely on the series of key hypotheses in~(2). The wording is deliberately
noncommittal as to how inflection is to be modeled, since proponents of
lexemic morphology have assumed either \emph{Item and Process} or
\emph{Word and Paradigm} approaches \citep{Hockett54}.

\begin{exe}
\ex\begin{xlist}
\ex  Atoms of morphological description are \textsc{simple lexemes}.
\ex
\textsc{Lexeme formation rules} predict the possibility of
\textsc{complex lexemes}  from either a single pre-established
lexeme (\textsc{derivation}) or a pair of pre-established lexemes
(\textsc{composition}).
\ex Inflectional morphology deduces, for each lexeme, the set of words
constituting its inflected forms.
\end{xlist}
\end{exe}

It is noteworthy that such a conception of morphology predates the
coining of the term \emph{lexeme}. It is very clearly outlined by
\citet{Kurylowicz}, where \emph{theme} plays a role analogous to
\emph{lexeme} as used by lexemic morphology:

\begin{quote}
When we say that \emph{lupulus} is derived from \emph{lupus}, or, more
precisely, that the theme \emph{lup-ul-} is derived from the theme
\emph{lup-}, this means that the \emph{paradigm} of \emph{lupulus}~is
derived from the \emph{paradigm} of \emph{lupus}.

{[}\ldots{}{]}

The derivation process for \emph{lupulus} takes the following concrete
form:

\begin{center}
\begin{tabular}{cllcl}
\it lupus & \it -i, -o, -um, -orum, -is, -os &  ou & \it lup- & (\textit{-us}, \textit{-i}, \textit{-o}, etc.)\\
$\downarrow$ & & & $\downarrow$\\
\it lupulus & \it -i, -o, -um, -orum, -is, -os &   & \it lupul- & (\textit{-us},  
\textit{-i}, \textit{-o}, etc.)\\
\end{tabular}
\end{center}
\hfill \citep[p.~123; my translation]{Kurylowicz} 
\end{quote}

\clearpage 
\section{Morpheme, lexeme, and the recent history of morphology}

The notion of a \emph{morpheme} is without doubt the most popular
theoretical innovation of 20\textsuperscript{th} century
morphology.\footnote{Although the term \emph{morpheme} was coined by
  Baudouin de Courtenay in 1895 with a meaning close to the contemporary
  one, its widespread usage with that meaning can be traced back to
  \citet{Bloomfield1933} and his immediate readers. See \citet{Anderson15} and
  \citet{Blevins14} for relevant discussion of the history of the morpheme.}
Although questions about its usefulness were raised from the 1950s, most
notably by \citet{Hockett54, Hockett67}, \citet{Robins59}, \citet{Chomsky1965} and
\citet{Matthews65}, morphemic analysis firmly occupied the center of the
stage until the 1990s. Accordingly, the notion of a lexeme barely
figured in discussions of morphology. For example, although he adopts a
word-based (vs. morpheme-based) approach of morphology, \citet{Aronoff1976}
claims in his preface that he has ``avoided the term \emph{lexeme}
{[}instead of \emph{word}{]} for personal reasons'' and used ``the term
\emph{morpheme} in the American structuralist sense, which means that a
morpheme must have phonological substance and cannot be simply a unit of
meaning''.

In the 1980s, most generative morphologists \citep{Lieber81, Williams81, Selkirk82} explicitly reject word-based models and assume that
the traditional morpheme is a legitimate unit of analysis \citep{Lieber.2015}. \citet{Aronoff2007} claims that the classical lexicalist hypothesis
\citep{Chomsky1970} holds instead that the central basic meaningful
constituents of language are not morphemes but lexemes. However, even
among supporters of the lexicalist hypothesis, things are not so clear.
Some of them, such as \citet{Halle73}, explicitly adopt a so-called
\emph{Item-and-Arrangement} (IA) model while others, such as \citet{Jackendoff75}, adopt a so-called \emph{Item-and-Process} (IP) model. \citet{Hockett54} coined these two terms \emph{IA} and \emph{IP} to refer to two
different views of mapping between phonological form and morphosyntactic
and semantic information. In IA models, complex words are viewed as
arrangements of lexical and derivational morphemes; in IP models, they
are viewed as the result of an operation, called a Word Formation Rule
\citep{Aronoff1976}, applying to a root paired with a set of morphosyntactic
features and possibly modifying its phonological form. In such models, a
complex word is not a concatenation of morphemes but is considered as a
single piece. IA models clearly reject lexemes as a pertinent unity. IP
models are not so consensual and hesitate between morpheme-based and
word (or lexeme)-based theory, and some of them continue to involve
morphemes. Corbin's position illustrates this hesitation. While adopting
the lexicalist hypothesis, \citet{Corbin87} never uses the term
\emph{lexeme}: she claims ``une morphologie du morphème (\ldots{}) ou
plus exactement une morphologie du morphème-mot'' (p. 183) and treats
affixes as morphemes (p. 285).

Indeed, ``this conflict between morpheme-based and lexeme-based theories
has haunted generative grammar ever since'' %
%(Lieber, 2015)%
\citep{Lieber2015}%
.

The work collected in this volume is representative of the growingly
dominant view that the lexeme is an unavoidable component of useful
morphological descriptions as well as theorizing. The high number of
French scholars represented in the volume reflects  the importance
that the notion of a lexeme has played for that community for the past
twenty years, mostly under the impulsion of Bernard %
%Fradin (1993, 2003)
\citet{Fradin1993,Fradin2003}%
%?Fradin;Fradin
%
,
and the group of researchers involved in the CNRS cooperation network
\emph{Groupe de Recherche Description et modélisation en morphologie} he
coordinated between 2000 and 2007. We are happy to dedicate this volume
to him.

\section{Presentation of the volume}

While the notion of lexeme is in widespread use in contemporary
descriptive and theoretical morphology, many questions remain
unresolved. Among others: what is exactly a lexeme: a theoretical
description or an object manipulated by rules? Is the difference between
lexemes and word-forms as clear as in Matthews' definition? Are lexemes
and Lexeme Formation Rules (LFR) always sufficient to explain the
formation of lexicon? Do LFR always apply to lexemes?

The twenty papers collected in this volume address the previous
questions and some others. They are organized in four sections:

\subsection{Lexemes in standard descriptive and theoretical lexeme-based
morphology}

Three papers centrally deal with this first theme.

In his atypical but stimulating contribution based on his own
intellectual biography, Aronoff traces the emergence of lexeme in
descriptive and theoretical morphology since the 1960's in Generative
Grammar.

In his paper, Boyé focuses on French cardinals and their place in Word
and Paradigm models. He argues that, like simple French cardinals, complex
cardinals are lexemes, and that their phonological idiosyncrasies can
better be modeled in a morpholexical system than in syntax.

Rainer studies the linguistic history of two keywords of economics and
politics, viz. \textsc{capitalist} and \textsc{capitalism}, in which semantic change,
calques and word formation ‒ suffixation, conversion, suffix
substitution ‒ interacted in a complex manner. He argues that, within a
morpheme-based model, it would not be possible to account for this
history, which, consequently, supports the hypothesis of a lexeme-based
conception of the word.

\subsection{Lexeme Formation Rules}

Lexeme Formation Rules (LFRs) are the main theme of four contributions.

Amiot \& Tribout deal with the category of outputs of French
suffixation(s) in -\emph{iste}: are they basically adjectives, nouns,
lexically underspecified or do we need two different suffixations to
account for data-observation? Their proposal is the last one. They
consider that, categorically and semantically, the French morphological
system contains two suffixations: one of them forms basically
professional nouns, the other basically adjective meaning ``in relation
to (a practice, an ideology, an activity, a behavior)''. They argue that, because 
such properties can apply to humans, these adjective can easily
converted in nouns.

In her contribution, Dal addresses the status of French adverbs in
-\emph{ment}. Although they are usually considered derivational, she
shows that this status is highly questionable. For her, neither inputs nor
outputs respect undoubtedly constraints imposed by a LFR and her
conclusion is that they can be regarded as word-forms belonging to the
paradigm of adjectives.

Villoing \& Deglas focus two morphological patterns in Creole
languages based on nouns to form verbs: suffixation N-é and
parasynthetic verbs dé-N-é. The hypothesis is that these two patterns
emerged following the reanalysis of converted and prefixed French verbs.

Strictly speaking, clipping of deverbal nouns is not a standard LFR.
However, the treatment proposed in \v{S}tichauer's paper, which applies
Fradin \& Kerleroux's (\citeyear{Fradin03}) Hypothesis of a Maximal (Semantic)
Specification,  conforms to standard conception of LFRs: in case of
polysemous lexemes, clipping applies to specific semantic features of
lexeme-bases, and outputs inherit these features, without being
synonymous to the full parental form.

\subsection{Troubles with lexemes}
\largerpage
Six of the contributions centrally address the issue of the definition
of lexeme and its use in morphological theories.

Bonami \& Crysman's contribution reevaluates the role of the lexeme in
recent Head-Driven Phrase Structure Grammar (HPSG) integrating a truly
realisational theory of inflection within the HPSG frameword %
%(Crysmann
%\& Bonami 2016)
\citep{Bonami.2016}%
%Crysmann-Bonami
%
. After having distinguished two notions of an abstract
lexical object: lexemes, which are characterized in terms of their
syntax and semantics, and flexemes (\citealt[159]{Fradin03}; \citealt{Fradin03b}), which are characterized in terms of their inflectional
paradigm, they show how the two notions interact to capture various inflectional phenomena, most prominently heteroclisis and overabundance.

Cruz \& Stump deal with essence predicates in San Juan Quiahije
Chatino: do they fall in the domain of morphology or in the domain of
syntax? Their conclusion is that, even though their structure comprises
a predicate base and a nominal component, their inflectional morphology
differs from that of simple lexemes.

In his paper on traces of feminine agreement within complex words in
Norwegian and Istro-Romanian, Enger tries to overcome troubles with
lexemes. He combines a modified version of the Agreement Hierarchy
\citep{Corbett79} and grammaticalisation to explain what he considers as
intra-morphological meaning.

Kihm examines the realization of the copula in Haitian Creole,
suggesting that the absence of an overt copula in some contexts should
be modeled by postulating an empty stem alternant. He outlines a formal
account based on Crysmann \& Bonami's (\citeyear{Crysmann14}) Information-based Morphology,
but extending that framework to the analysis of periphrastic inflection.

Spencer questioned whether lexemes are abstract representations of
properties unifying a set of inflected word-forms or objects manipulated
by rules. Using the architecture of his model of lexical relatedness
Generalized Paradigm Function Morphology (GPFM) \citep{Spencer13}, he
proposes an answer to verb-to-adjective transpositions (participles),
which can be seen as lexemes-within-lexemes according to their double
status of word-forms in relation to verbs, and lexemes in relation to
their adjective properties. His proposal is that a lexeme is not a
theoretical observation but is best regarded as a maximally
underspecified object, bearing all and only those properties which are
not predictable from default specification.

Flexemes are also the central issue of Thornton's paper. After reviewing
the development of this notion since \citet{Fradin03} and \citet{Fradin03b}, she focuses on the concept of overabundance in
inflectional paradigms and presents data illustrating cases in which a
single lexeme maps to two distinct flexemes.

\subsection{Troubles with Lexeme Formation Rules}
\largerpage

LFRs are questioned in seven papers.

In their study on reduplication in Mandarin Chinese where difference
between lexemes and word-forms is less apparent than in languages with
clear inflection, Basciano \& Melloni claim that the domain of
application of reduplication is below the level of the word, or below X°
in the standard X-bar approach: for them, in Mandarin Chinese, base
units do not have a lexical category and should be vague enough to make them
compatible with nominal, verbal and adjectival meanings.

Hathout \& Namer explore limits of LFRs to explain and predict the
formation of the lexicon. They confront parasynthetics lexemes, in other
words complex lexemes that apparently result from simultaneous
application of a prefixation and a suffixation, with different
hypothesis. This recurrent theme leads them to propose the system ParaDis
(for: Paradigms and Discrepancies). ParaDis is a model particularly
useful to analyze, explain and predict noncanonical 
formations \citep{Corbett10}. It is lexeme-based and combines independency of the three
dimensions of LFRs \citep{Fradin03} and constraints on outputs founded on
derivational families and derivational series \citep{Hathout11,Blevins14}.

Giraudo validates this double view of complex words articulating
syntagmatic and paradigmatic dimensions, from a psycholinguistic
perspective. She identifies two levels in processing of complex lexemes:
the first decomposes complex lexemes into pieces called ``morcemes''; the
second deals with the internal structure of words according to LFRs and
contains lexemes. Her model poses family clustering as an organizational
principle of the mental lexicon. She argues that, during language
acquisition, growing of family size consecutively continually strengthens
links between complex lexemes.

Montermini is devoted to variation of derivational exponents. Adapting
the frame developed in \citet{Plenat-Roche2014} and \citet{Roche14,Roche16}, he argues that this variation obeys to the same constraints as
those which explain forms of complex lexemes.

Plag, Andreou \& Kawaletz tackle a recurrent and central problem with
LFRs: polysemy. They rely frame semantics (\citealt{Barsalou.1992a,Barsalou.1992b}; \citealt{Lobner13}), an approach to lexical semantics based on elaborate structured representations modelling mental representations of
concepts. They hypothesize that the semantics of a derivational process
can be described as its potential to perform certain operations on the
frames of the bases to which they apply.

Schwarze deals also with the semantic outputs of LFRs. His hypothesis is
they are semantically underspecified. The model he proposes is
multilayered: it comprises four layers of representation: phonology,
constituent structure, functional feature structure and lexical
semantics. The meaning of complex words is treated in the framework of
two-level semantics. It is assumed that LFRs derive underspecified
semantic forms, parting from which the actual meanings are construed by
recourse to conceptual structure. Three morphological processes are
studied: French \emph{é}- prefixation, Italian denominal verbs of
removal, and French noun-to-verb conversion.

Strnadová addresses the issue of apparent rivalry between French
denominal adjectives and prepositional phrases in \emph{de}+N where N is the
lexeme-base of the adjective (or in relation to it). She discusses some
motivations explaining the choice between the former and the latter
strategy, and shows that they usually do not have  the same distribution
and, therefore, are not interchangeable.

\section*{Acknowledgements}

\sloppy
We thank Sacha Beniamine for his extensive work on the preparation of the {\LaTeX} manu\-script for this book, and Sebastian Nordhoff for continuous support and help. 
The production of this book was partially supported  by a public grant
    overseen by the French National Research Agency (ANR) as part of
    the ``Investissements d'Avenir'' program (reference:
    ANR-10-LABX-0083).



{\sloppy
    \printbibliography[heading=subbibliography,notkeyword=this]
}


\end{document}
