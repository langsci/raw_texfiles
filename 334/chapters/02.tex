\documentclass[output=paper]{langscibook} 
\ChapterDOI{10.5281/zenodo.6620107}
\author{Pierluigi {Basso Fossali}\affiliation{UMR 5161 ICAR; Centre National de la Recherche Scientifique; Ecole Normale Supérieure; Université Lumière Lyon 2}}
\title[Semiotic mediations and complexity management]
      {Semiotic mediations and complexity management: Paradoxes and regulative principles}
\abstract{Semiotic mediations have a number of roles in managing complexity. We review how they lead to paradoxes, but also how regulative principles emerge from them. We propose three complexity principles that researchers can successively apply when they must establish through language games how a new research domain will function: 1) the analogical approach, 2) the recursivity of distinctions, and 3) emergentism. We discuss tensions between open and closed systems, individual identities and cultural norms, and between characteristics of linguistic complexity. Using a language games lens, double binds and paradoxes are presented in regards to communication, language structure, and the theory-practice relation. Nevertheless, we argue that semiotic mediations are the solution for exploring relations between complex systems and the environment despite two noteworthy systemic ambitions of language.}

\begin{document}
\maketitle 

\section{Wise decomplexification} \label{sec:2:1}

All things considered, a theory of complexity involves a commitment to reductionism. This starting point may be unexpected, but rather cautious and modest, compared to what is sometimes erroneously claimed in popular science literature. As a matter of fact, even before negotiating its epistemological relevance, each scientific approach should present itself as a theoretical practice that must necessarily “decomplexify” relations with its reference environment in order to display its ambitions of reorganising a restricted domain. We know that the \textit{horizon of} \textit{testability} of a theory sets restrictive conditions for its falsifiability \citep{Popper1972}; this precaution openly denounces a kind of lateral “blindness” with respect to the relevance of the theoretical view. By following the Kantian lesson, a theoretical view can discover \textit{founding principles of} the observed domain, but its perceptive experience must take gaps into account (out of field, out of size, etc.), which requires the elaboration of \textit{regulative principles} (by \textit{analogy}, one acts \textit{as if} the environment that has not yet been mapped is composed similarly to its already “territorialised” parts).\largerpage[1.5]

The progressive precision of the \textit{horizon of testability} of scientific approaches correlates with an increase in their number – hence the fragmentation of specialisms and the epistemological complexification of a field of study. This leads to a multiplication of descriptive re-commitments on the same subject-area, which makes it increasingly difficult to adopt an integrative point of view. Complexification is the progressive adaptation of a view to its local limits (a sort of semantic “jurisdiction”), which prevents any \textit{totalisation of} the frame and makes value conversions necessary. This is the first role that can be given to semiotic mediations: the interconnection of fluctuations in the values\footnote{\Citet[333]{Saussure2002} had already pointed out that through the comparison between linguistic value and economic value, one can grasp the correlation between the fluctuations of two counter-values (referential counterparts): the quantity of gold that corresponds to the nominal value of a currency can vary in the stock market independently from fluctuations in the value of a commodity in a competitive product market, but the two dynamics are sensitive to their correlation (purchasing power). In a parallel manner, the semantic autonomy of a language is then subject to fluctuations in its own intentional valencies, but these will not be assessed independently from fluctuations in the valencies of extensional domains (\citealt{BassoFossali2007}: 49--50).} of two or more domains investigated\footnote{A \textit{domain} is a frame for the \textit{life form of} a system, which cannot “live” just with its internal organisation: the system depends on one or more couplings with other systems, and therefore with an environment. Every life form necessarily involves an identity in a critical state (productive imbalance).}  and the ensuring of some form of translatability between them within a \textit{confidence interval}.\footnote{Basically, data interpretation practices must take responsibility for appreciating and assuming margins of error (the range of fluctuation of external variables is known to be uncontrollable). At the social level, the attribution of semantic values can be thought of as a statistical distribution that requires us to evaluate locally the confidence interval between a canonical or idiolectal meaning, a literal or rhetorical use, a \textit{ratio facilis} (conventional) or a \textit{ratio difficilis} (creative) \citep[cf.][133]{Eco1986}.} Languages allow for the transformation and differentiation of the “cultivation” domains of values (and thus the correlative loss of generality\footnote{The differentiation of social domains allows an autonomisation in the treatment of values (scientific, artistic, religious, etc.) but it also implies a loss of generality (science does not explain art or religion jurisprudence). Secondly, translations between domains are agreements on an equivalence of meaning that is not valid beyond the socio-cultural framework that justifies it. This is why there is always the need for new translations of the classics.}) into \textit{federation}. The latter exploits semiotic mediations in order to:

\begin{itemize}
\item Ensure that all the members of the federation are sensitive to the initial conditions of any locally emerging variation (\textit{contingency networking});
\item Reorganise the distinctive and translational retroactions of domains that are in interaction into a redefining circuit for the identities and purposes of federation members. Thanks to languages, \textit{circular causality} rises to the level of reciprocal reinterpretation for systems and their couplings. For example, government policies regarding the university, fund emerging research projects that particular academic circles of universities themselves have helped to select by influencing the way the criteria for institutional recognition of scientific value have been written.
\end{itemize}

To summarise, in every epistemological approach, one can notice a negative thematisation of complexity, as a series of hypothetical sketches of what goes unnoticed for the promoted theoretical view. This is why, from a scientific point of view, \textit{decomplexification is acceptable only if it is made suitable for a type of complexity that was never seen before}. As a result, complexity is pluralistic in terms of approaches, and requires both the federating of trans-disciplinarity and the recognition of the endpoints of the description languages and models used.

This invites clarifications regarding the roles of semiotic mediations in the structuring of federations constituted by domains of research (that stand in opposition to generalities and prior hierarchies between type and occurrence, \textit{de jure} and \textit{de facto} models, theoretical approaches and field approaches). Their purpose of achieving correlation between value domains with autonomous organisations and dynamics does not entail that they have the privilege of maintaining a stable and neutral descriptive organisation, as a kind of all-purpose representational language. On the contrary, complexity is a principle that must “fit” into the very purpose of semiotic mediations, so that each descriptive gesture should preserve the distinction between its metalinguistic organisation and its protention towards the \textit{characterisation} of a local model (the “labour” of a language is to spend its structural potentialities to catalyse the recognition of other organisations). Thus, conceptualisation corresponds to the management of the distance between the language of the theory and the semiotic life form of the object under study, in order to build a movable hinge (familiarisation) between two systems with complex behaviours, which are sensitive to their future divergence. Complexity theory only knows recalcitrant objects\footnote{The object must be conceived as an alterity endowed with a complexity not yet fully described. In this sense, the object resists the projection of the theoretical model and it is from this resistance that the research advances, refines its hypotheses, understands that it must not only interpret the reactions of the object but reconceptualise the positions of observation and attestation of the phenomena.}, and its heuristic grasp emerges from the dissociations recorded in relation to the models inherent to the description language, which invites reconceptualisation. Thus, the scope of the \textit{semantic} dimension reveals itself as the impossibility to continue the syntactic computation on the same computation field.\footnote{A classic way to present this problem is to insist on the difference between map and territory. The issue that Bateson has highlighted is not only of an ontological nature, because this would still allow, once an accurate denotative plan has been established, to describe equivalent syntactical operations on the map, without controlling the behaviours that can actually be implemented in the territory. Bateson's problem is that there is still “play” possible between the signs of both the map and the territory, and therefore also the margins of play in the translation between map and territory. The value of the signs in the respective domains is susceptible to autonomous semantic interpretations: “In the Andaman Islands, peace is concluded after each side has been given ceremonial freedom to strike the other. This example, however, also illustrates the labile nature of the frame \textit{This is play} or \textit{This is ritual}. The discrimination between map and territory is always liable to break down, and the ritual blows of peace-making are always liable to be mistaken for the ``real'' blows of combat. In this event, the peace-making ceremony becomes a battle” (\citealt{Bateson1971}: 187--188).}

\section{Complexity principles}\largerpage %2. /

The autonomy of a domain implies the establishment of its own language games, its own semiotic mediations. Every disciplinary doubling of a description plan (\textit{map}) requires a new semiotisation of the scientific field (\textit{territory}) and indicates qualitative crossings in the observed, split domains; in short, a “decomplexifying” tactic must be used when faced with an order of complexity that is higher than expected. This has led to the identification of a certain number of guiding principles and, for our purposes, it is important to specify at least three of them, already mentioned in the first paragraph: 

\begin{itemize}\sloppy
\item the \textit{analogical approach}, which explains how the internal mereological economy of a system (self-similarity\footnote{Self-similarity is a reverse recursivity where the global mereology begins to reorganise contained mereologies.}) can, in the autonomising tension of the latter (autopoiesis), rise to a general \textit{configurational paradigm} according to the adjustment of proportions and by diagrammatic projection from one domain to another;\footnote{The example of the interpretative play between territory and map can still well illustrate the issue of proportionality and diagrammatic projection.}

\item the \textit{recursivity} of distinctions and in particular the \textit{re-entry}\footnote{The notion of \textit{re-entry} corresponds to the reintroduction of the founding distinctions within the system, which has already benefited from them with the paradoxical qualitative result of a distancing. Instead of blocking the rule, its reapplication to what one might think has already been settled causes a kind of insolvency of the meaning, which was seemingly already guaranteed.}  \textit{phenomenon} which enables self-reflexive organisational phenomena\footnote{A typical example of re-entry is the application of the right vs. wrong category to the local application of the category itself (is it right to apply the law in an exceptional situation?). Less trivial is the phenomenon of the re-entry of consciousness into the perceptual data that it itself has elaborated within what William James has called the specious present.};

\item \textit{Emergentism} \citep{Alexander1920}, as it is envisaged within the \textit{order from noise} paradigm (\citealt{Foerster1960}), which enables to design interactional organisations based on the convergence of contingent lines of transformation.
\end{itemize}



The sequence in which these principles are presented is not innocent; we consider that the quality of each passage (from \textit{i} to \textit{ii} and to \textit{ii} to \textit{iii}) should be characterised according to two movements: 

\begin{enumerate}
\item[(A)] a movement of openness that goes (a1) from territorialisation (a2) to the awareness of edges to finally arrive (a3) at the inter-subjectivity of scientific challenges and discoveries; 

\item[(B)] a circular movement that explains how (b1) from the application of internal functions (\textit{exercises}) (b2) one moves to functional differentiation to respond to dysfunctional external influences (\textit{irritations}), (b3) to finally achieve a restructuring of the coupling with the environment that takes advantage of emerging trend lines (\textit{modal reloading}\footnote{The restructuring of the coupling with the environment is an opportunity to restart the modal charges (\textit{epistemic, alethic, deontic,} etc.), i.e. the forms of involvement of the system in the environment (desire, knowledge, capacity, etc.). For example, the new coupling of the art world with the economic market has restructured the powers of the artist and the objects in which his or her practices can find an end.}).
\end{enumerate} 


The two movements draw, through (A) \textit{social complexity} and (B) \textit{structural complexity}, a dynamic model for redefining the identity of all the instances involved in the domain, a model which is sensitive to space (\textit{coupling through symbolisation)} and of circular causality that is sensitive to time (\textit{coupling through retroaction}). On the one hand, systems are interrogated by \textit{serial couplings -} the operations of one are restricted to the operations of the other -- on the other hand, systems must recognise a common ecology and therefore \textit{parallel couplings}.\footnote{The distinctions we use for coupling are inspired by \citet{Walliser1977}.} The combination of these types of couplings gives rise not only to \textit{non-linear operations}, but also to a tension of cultural systems that work on \textit{gaps}\footnote{The term “gap” is preferred to “difference” because in social domains and between them one has to work on distances of position and value which also determines the meaning of confrontation and dialogue.} and promote inventions  as a response to the initial conditions for progressive emancipation. It is from this responsive attitude that complexity stratifies its own history of emancipation in relation to the previous “order/random” balances. Cultural stratification implies a diachronic differentiation of organisational structures and a synchronic coexistence of axiological and semiotic paradigms; the emancipatory tension explains the non-causal arrangement of cultural solutions to historically encountered problems.

Thus, despite the existence of prior semiotic organisations (the origin is already the fruit of a complex reconstruction that cannot be detached from the self-referential definition of systems), complexity theory seems to encourage the recognition of the unsuspected emergence of inventive solutions to the downgrading of certain meaning institutions and the protention towards a space that has not yet been territorialised. Every mapping, every model proposed can only be incomplete, but cultural complexity also signals that every meaning project is “unfinished” and non “resettable”:\footnote{This interpretation of the meaning project comes into critical tension with Grice's intentionalistic perspective: “Grice's meaning project was to explain what ``non-natural'' meaning is by defining expression meaning in terms of speaker meaning and speaker meaning in terms intention” \citep[41]{Davis2007}.} the dual foundation of the historicity of systems lies in the fact that there is a coupling of memories. Sometimes the environment reminds us, with the traces left over time, of our interpretations concerning the domain, despite false leads or lost trails. Essentially, the \textit{extended mind} is simply the other side of the memory of a coupling between system and environment.

\section{Complexity and semiotics of cultures} %3. /

One can recognise at least three different orders of complexity in a semiotics of cultures: (${\alpha}$) the systemic complexity that concerns all forms of dynamic organisation; (${\beta}$) the complexity of cultural systems as such (which therefore characterises anthropogenic life forms); (${\gamma}$) the complexity of semiotic mediations, which characterises language games.

Despite their qualitative differences, systemic, anthropogenic and linguistic complexities are organised in a series of embedded homologies from one order of relevance to another. This requires some coordination for definitions: 

\begin{enumerate}\sloppy
\item[(${\alpha}$)] After a long debate on whether systems should be conceptualised as either \textit{open} or \textit{closed}, it has become increasingly apparent that there is a tension, in biological life forms, and even more so in cultural life forms, between: (${\alpha_1}$) a physical opening due to border permeability and inter-organic collusion and (${\alpha_2}$) the closing of internal self-structuring codes. 

Every living heritage reproduces in itself this contradiction between tensions at the opening and tensions at the closing, according to unstable mereological frames (content and container, integrated totality and partitive totality, etc.). This contradiction stems from the tensive articulation between the appropriation of a subsistence domain and the adaptation of interactions to the forms of distinctions and correlations that are coded within the system. A language tries to constitute itself as a system of internal difference (semantic autonomy) but its historical evolution is solicited by contacts with other languages; that is why it can only try to adapt translations and imports to the formats of its own structures.

\item[(${\beta}$)] With regard to the complexity of cultural systems, the non-trivial behaviours of cultural systems are characterised by the reinterrogation of the identities of subjects and objects; in this sense, symbolisation is a form of constant feedback from the linguistic figure projected in discourse (i.e. a pronoun) on the actor of utterance (i.e. the identity of the social actor). This reinterrogation is carried out from a shared environment, from agreement on a specific language game, but it bears non-trivial fruits because it accepts exposure to violations of norms and competitive inventions. Basically, intuitions or analogical reasoning redraw the purposes of the systems themselves.

\item[(${\gamma}$)] With regard to the specific characteristics of linguistic complexity, attention should be drawn to three fundamental factors: ($\gamma_1$) the \textit{multimodality of} expression, given that speakers’ bodies are involved, which implies the management of all their expressive potentialities beyond intentional aspects; (${\gamma_2}$) \textit{second order observation on meaning}, given that one must always assume an asymmetry of competence between speaker and interlocutor, and an autonomous semantisation of linguistic productions; (${\gamma_3}$) the conquest of a \textit{distal dimension} \citep{Rastier2001}, namely the use of linguistic mediations enables to return the basis of negotiated values to the institutions of meaning that lay the foundations and ensure the regulatory effectiveness of language games beyond their structural defects or voids.
\end{enumerate}

\section{Duplicating language games} %4. /

Multimodality, second-order observation and the conquest of \textit{distality} can only contribute to the complexification of frames for the use of linguistic resources. Yet, this complexification is expressed above all as the appealing to several language games at the same time. The simplest case is apparently that of \textit{mise en abyme}, where there is an interpretative dialectics between the “framing” and the “framed” language game (e.g. a parable within an academic essay or a legal trial within a play). The problem is that, in a cultural space with a symbolic purpose, nesting cannot be read as a hierarchisation prejudice, and so even the smallest possible world (an anecdote in a novel, a blazon in a painting) can rise to the level of a key that interprets an entire life form.

More generally, communications that establish and support social domains always involve “double play” from their participants, meaning that interaction takes place on several fields of play at the same time. The first consequence of a semiotic regime that is always split according to several competitions played at the same time is the non-linear (expression plane) and non-consequential (content plane) nature of the semiotic choices made in a praxeological frame. The second consequence is that the strategic articulation between several discursive instances and different organisational frames can only lead to the local emergence of new “remedial” forms of organisation. 

The systematic nature of interactions probably feeds on prior organisations (frameworks), but it is still the product of modal asymmetries (e.g., different desires) that motivate communication. It is therefore the product of a \textit{double contingency} \citep{Luhmann1995Social}. While one can invent games that regulate other games (a “meta-game”), there is never a meta-rule to choose the order in which the games should be embedded for interpretation. The design of a model is in fact the first demonstration of a hierarchical reversion between the global frame and the local scenario of meaning issues (see \sectref{sec:2:1}). Furthermore, one must consider that there is always some form of “play” in language games, and therefore random factors and flexibilities, which makes exchanges fascinating.

\section{Double binds and paradoxes} %5. /

Doubles in games also give rise to double constraints on communication, which inevitably ends in paradoxes. The pragmatic \textit{double bind of} making oneself understood, for the maxim of \textit{manner,} and to say at the same time what the reader has not yet understood, for the maxims of \textit{quantity} and \textit{relation} (content plan), or the double constraint of simplifying pronunciation in front of a foreign interlocutor and at the same time ensuring the distinctive perceptibility of phonetic features (expression plane), raise the question of whether it is simply necessary to find a balance or whether one should accept the paradoxes created by inscribing them in a “reframing” capable of mitigating the contradiction. With the notion of \textit{simplexity}, Alain \citet{Berthoz2009} has shown that simplification is not necessary, but that “it is necessary to present issues in a different way”\footnote{Personal translation.} through the restructuring of relevance and therefore of the conditions of observation. 

In cultural sciences, epistemological reflection has enabled the recognition of insoluble dialectical tensions exemplified by the organisational shape of languages. Thus, the competition between semiotic density (the constitution of the substance of linguistic expression tends to be similar to a perceptual inquiry) and notation (language emancipated from the material conditions of inscription and from the contingencies of a sensitive approach\footnote{On the notion of \textit{density} and on the theory of \textit{notation} (allographic languages), see \citet{Goodman1968}, while the concept of “substance of expression” refers to Hjelmslev's work.}) has been projected on two epistemological dualities strictly associated: between \textit{emic} and \textit{etic} et between characterisation through \textit{thick description} and generalisation through grammar extraction. 

Theory cannot emancipate itself from the paradoxical conditions of practice. Moreover, it is also a victim of circular causality since it must take into account, in its practice strategies, the disturbance created by its observation activity in the field under analysis (i.e. the challenges of participant observation in ethnographic research are linked to the potential influence of the researcher on the data because of its anomalous presence in the foreign cultural context, its means of recording practices, etc.). Furthermore, the complexity of scientific culture concerns a disproportion of the research horizon in relation to the process of operations that can already be applied in a series according to an approved procedural syntax; reconceptualisations are required, even in the most formal theories.

More generally, one can observe the continuous deparadoxalisation of every culture, through (i) situated decomplexifications that go from the global to the local, according to a principle of meaning compartmentalisation;\footnote{The court system has to handle general principles about many areas of law, but it is not capable of resolving disputes in specific topics because these are not sensitive to the general criteria. So, jurisprudence sets up administrative tribunals to make less formal decisions that are reasonably expected to not be in opposition with the “common” law.} (ii) agentive decomplexifications that begin with local initiative and lead to the homogeneous configuration of an entire scenario, according to a principle of unilateral proceduralisation of meaning;\footnote{This is the typical case of emergency procedures where any discussion of principles and rules is no longer compatible with the need to act immediately as a compact and supportive community.} (iii) the taking into account of hyper-complexity linked to the effects of observation itself on the field serving as a frame for the ongoing operations. 

Beyond the aporetic recursivity of distinctions (“is the distinction between inclusive and exclusive inclusive?”), one discovers that, for a semiotics of cultures, there are only regulative principles: we act \textit{as if there were} non-contradictory rules but in reality, there are only \textit{imbalances provided with regular decomplexed responses} in relation to persistent double binds. 

In this respect, \textit{metastability} – the struggle of systems against entropy conducted by constantly new means and measures – cannot be described as the discrete transition from one balance to another (i.e. the paradoxical coexistence of liberty and law). On the one hand, it should be added that it is always shared by one or more couplings and, therefore, cannot be only examined at the self-referential level; on the other hand, the oscillations between balances invite the formulation of a description of emerging forms according to an oscillation range between the phases of \textit{fluidisation} and \textit{coagulation}. The two remarks are expressed in the idea that there is a reciprocal moulding between forms promoted by the system and forms promoted by the environment, and that formative initiatives are in competition. Local coagulation occurs when one tends to prevail over the other. That said, fluidisations do not only take precedence in transitions; they even persist in conditions favourable to coagulation as inhibiting factors. Thus, breaks in symmetry in couplings give rise to resistant flows capable of opposing the crystallisation of relationships between structures and functions in correlated systems. On the one hand, it is clear that this dialectics between coagulation and fluidisation can explain the reciprocal determinations between the \textit{global} and the \textit{local}: the semantic consistency of a sentence – attributability of actantial roles – does not prevent the spread of remote semantic values in the discursive \textit{co-text}\footnote{\citet[215]{Eco1994} has “reserved the name of \textit{co-text}  for the actual environment of an expression in the course on an actual process of communication”.} nor, on the contrary, the reception of distant \textit{semes} in an utterance that is apparently already semantically saturated. On the other hand, this dialectics seems to respect \textit{imbalance as a} promoter of renewed order, which is expressed through the tensive relations between \textit{flow}, \textit{function} and \textit{structure} \citep{Prigogine1983}. We propose to assign flow to \textit{couplings} in their reversible polarisations, function to interacting \textit{devices}, and structure to \textit{systems}. The three instances concerned are mutually influenced by evolving \textit{feedbacks} and \textit{feedforwards}. For example, the prolonged application of a function – driving a car – releases energies, thanks to the flexibility gradually acquired, in order to take into account fluctuations in values (flow) in a new field of operation – conversation with a passenger –, and thus to consider new tasks (bifurcation of functions), which promotes the ambition of restructuring the practice – prove to be a good cicerone -- and a restructuring of ambitions (non-linear qualitative leap) – playing the role of Don Juan.

\section{Between concessivity and limitation: The semiotic proportion}\largerpage[2] %6. /

Although \textit{complicated items} are normally the result of an over-codification that seems to defy interpretation, they can still potentially be analysed, given the commensurable nature of assessment parameters (\textit{problem solving}). On the other hand, \textit{complex items} display a partial indeterminacy and/or an irresolvable, unintegrable heterogeneity that challenges us to change the conditions of observation: decomplexification is a point of arrival in the management of relations that are not yet coded, a kind of scale abduction that allows us to later surmount a complexity that remains out of reach. 

Compared to previous paradigms, complexity theory does not rely on a conceptual architecture that is capable of branching out infinitely, according to a “constructionist” perspective on knowledge. Every advancement corresponds on the other hand to the acknowledgment of the “concessivity” of an exploration field that remains partly impenetrable and obscure. The conceptual distance between creativity and discovery is thus filled by the reciprocal mouldings between gesture and hosting space, by the ecological niches where the positive form of the semiotic construction is given together to the negative form of the space that remains secluded behind the concessive nature of its response to our investigation. Thus, the \textit{redundancy of} organisational constraints and the recursivity of gnoseological attacks{\interfootnotelinepenalty=10000\footnote{The term “attack” is employed here as it is used in music, i.e. to indicate the initial run-up of musical gesture, the start of a given note or of a solo. Every silence dramatises the subsequent musical attack.}} can only lead to local reductions in complexity. The latter indicates at the same time the negative condition of immediately available meaning: science’s fascination paradoxically constitutes the continuous “creation” of the “yet unknown”. 

Convergent determinations of instrumental reason (optimisation of calculations) are surpassed by the negotiation of a sensible coalescence of eccentric options (lateral thinking), which can characterise the significance of choices. In fact, even from an ergonomic point of view, an adequate hesitation system\footnote{A hesitation system is qualified by the idea to leave some “game” between the bolt and the nut, a mechanism that conceives some room for action. At the same time, it obliges the user to interpret his conditional freedom, which is a time to hesitate before deciding on the mode of “attack”. \citet[305]{Leroi-Gourhan1993} has always stressed the importance of aesthetic “approximation”  in culture, that involves a certain freedom in the interpretation of the relationship between form and function.} must be left between the prosthesis\footnote{“In a strict sense, a prosthesis is an apparatus replacing a missing organ (an artificial limb, a denture); but, in a broader sense, it is any apparatus extending the range of action of an organ. This is why we can also consider hearing aids, megaphones, stilts, magnifying lenses, periscopes as prostheses” (\citealt{Eco1994}:  208).} (free modal fusion between subject and instrument) and the interface (strict modal coding), in order to envisage a creativity/discovery capable of restructuring the identities of the subject and the object and the space relevant for their coordinated actions.

An isolated system, which is omnipotent in its environment, can only experience modal vertigo, and thus evolve according to a progressive increase in the insignificance of these acts, while at the same time seeking a definitive stabilisation of its connections with its surroundings, i.e. absolute power. Admitting randomness is an apotropaic solution to the obsession to always present a one-sided complexity. Thus, the subjects of an interaction are reciprocal “black boxes” that begin to include each other's selections as restrictions on the entropic vertigo of solipsism. In this sense, conflictuality is already a form of modal restriction and channelling. That said, coordination remains uncertain, and indeterminacy will affect all attempts at code-based stabilisation. 

\begin{sloppypar}
As for possible “neuroses” when confronted to heterogeneous or self-generated complexity, social stakeholders find a compensatory balance in the impossibility of observing all relations between systems and the environment, which means that the purposes remain open with respect to the functions already coded. But this balance and this search for further purposes can only be promoted through semiotic mediations, where the relations between expression and content are a dialectical laboratory between memory and innovation. Languages are our environment and the \textit{semiosphere} \citep{Lotman2005} perfectly reproduces this reciprocal moulding between creation and discovery, scale abduction and response from a cultural space of reception. 
\end{sloppypar}

Of course, one cannot erase the systemic ambitions of language as an institution of meaning. Two concluding remarks are thus necessary: 

\begin{itemize}\sloppy
\item there is a shift from quantitative complexity to qualitative complexity through: (i) the offer of non-redundant varieties of semiotic features (extension of choices); (ii) the emergence of innovative combinatorial potentialities (actualisation of the reconfiguration possibilities of the system); (iii) the plurality, the translatability and the syncretic use of available linguistic systems; (iv) the constitution of institutions of meaning (domains) without prior hierarchy (\textit{heterarchy)} and whose interconnection density gives rise to a network of mutual influences and non-linear causes;

\item the linguistic environment cannot dissociate \textit{langue} (language as a system) and \textit{parole} (speech acts), which means that the system emerges, in synchrony, as the latest reciprocal moulding between discourse practices and the responding \textit{semiosphere}; and in diachrony, as a partial crystallisation of linguistic habits that are regulatively elevated to the level of \textit{norms}.
\end{itemize}

\section*{Acknowledgements}
The author would like to thank Kris Lund for her critical remarks and suggestions. The author is grateful to the ASLAN project (ANR-10-LABX-0081) of the Université de Lyon, for its financial support within the French program “Investments for the Future” operated by the National Research Agency (ANR).

{\sloppy\printbibliography[heading=subbibliography,notkeyword=this]}
\end{document} 
