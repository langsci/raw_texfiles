\documentclass[output=paper]{langscibook} 
\ChapterDOI{10.5281/zenodo.6620127}
\author{Elizaveta Chernyshova\affiliation{Skilder}
        and Vanessa Piccoli\affiliation{UMR 5161 ICAR; Centre National de la Recherche Scientifique; Ecole Normale Supérieure; Université Lumière Lyon 2}
        and Biagio Ursi\affiliation{UMR 5161 ICAR; Centre National de la Recherche Scientifique; Ecole Normale Supérieure; Université Lumière Lyon 2}}
\title[Multimodal conversational routines]
      {Multimodal conversational routines: Talk-in-interaction through the prism of complexity}
\abstract{\sloppy Analyses of multimodality within human interaction showcase adaptivity and emergence in that the nature of talk is both context-shaped and context-renewing. While recurring structures and patterns illustrate order in natural conversation, unpredictable elements point to the importance of the particular setting of the conversation under study. In this chapter we discuss the balance between such expectations and emergent practices and argue for talk-in-interaction as a complex adaptive system. In two separate everyday contexts (doctor’s office and grocer’s), we show how a multimodal conversational routine emerges and plays different roles while satisfying the criteria of being variable, emergent, collaborative, and recognisable by the participants.}

\begin{document}
\maketitle 

\section{Introduction}

From its very beginnings, conversation analysis (CA) has been interested in discovering the unwritten laws of social interaction, in order to show how participants handle the delicate task of communicating with each other. Since the seminal work of \citet{SacksEtAl1974}, many studies have contributed to show that in natural conversation, much like in every other social activity, there is “order at all points” \citep[22]{Sacks1984}. Individual actions, social meaning, and activities are co-constructed and recognised through the use of recurring structures and observable interactional patterns and designs. As a matter of fact, several mechanisms underlying talk-in-interaction have been shown: among others, turn-taking (\citealt{SacksEtAl1974}) as the universal infrastructure for conversation, repair (\citealt{SchegloffEtAl1977}) and overlap onset \citep{Jefferson1984} as pervasive processes for the management of troubles. However, these mechanisms can only be observed in unique instances, since every stretch of talk features emerging elements, situated in a context (see \citealt{Heritage1984}: 242, on the “doubly contextual” nature of talk in being both “context-shaped” and “context-renewing”). There is a part of social interaction that cannot be described or predicted using universal patterns, and is solely attributable to a particular setting or framework. The tension between the regular and the unpredictable is what seems to be the driving force of social interaction.

The aim of this chapter is to discuss this delicate balance between what is contextually expected on the one hand, and what emerges given the elements of a particular setting and framework, on the other. In order to underline this twofold feature, we will focus on the local and situated co-construction of a conversational routine\footnote{Here, the expression “conversational routine” is to be intended as an emergent and shared array of verbal, vocal and multimodal resources, which are recurrently mobilised for practical purposes, and not necessarily as “conventionalised prepatterned expressions” in interaction \citep[2--3]{Coulmas1981}.} in different contexts of everyday life. Within the CA framework, we offer a qualitative and fine-grained multimodal analysis\footnote{For methodological challenges of interactional multimodal analysis, see \citet{Mondada2018}.} of two instances of the conversational practices of multimodal repetition. The analysis provides evidence for the crucial role of collaboration and emergence in talk-in-interaction, two features that allow us to consider conversation as a complex adaptive system.

\section{Complexity in the study of language}\label{sec:12:1}\largerpage

In the last decades, several studies have applied the notion of complexity to the study of language(s). However, as \citet{MufweneEtAl2016} notice, few studies have questioned the nature of what is called \textit{complexity}. A landscape of approaches to complexity is proposed by \citet{Edmonds1999}, who shows that five main concepts are traditionally named using complexity as a paradigm:
(1) the size of a system in terms of composing elements,
(2) the degree of ignorance of a system,
(3) the minimum description size of a system (the Kolmogorov complexity),
(4) variety and variations present in the system, and
(5) its hybrid organisational nature. As Edmond puts it, “[c]omplexity is sometimes posited as a mid-point between order and disorder” (\citeyear{Edmond1999}: 5), this point interestingly echoes with our general observations on conversation above.

Some studies in linguistics have adopted a more quantitative approach, by estimating the complexity of the different components of a language in terms of its size, variation and minimum description size (e.g., \citealt{NewmeyerPreston2004}, \citealt{AckermanMalouf2013}, \citealt{Bisang2014}). Others have focused on the evolving nature of complex systems to describe how speakers learn second and foreign languages (\citealt{Oxford2017}, \citealt{OxfordEtAl2018}), how languages change over time and space (\citealt{Kusters2008}, \citealt{DeGroot2008}), how language choices evolve in society (\citealt{Loureiro-PortoSanMiguel2016}). Complexity theory has been applied to many different sub-fields of research in linguistics and these studies have shown that all these different aspects are connected. For example, when studying the emergence of linguistic patterns “we cannot understand these phenomena unless we understand their interplay” (\citealt{BecknerEtAl2009}: 18).

Studying talk-in-interaction as a complex system has to include a systematic account of sequentially positioned conversational turns, as well as relevant temporalities of multimodal resources in context (\citealt{DeppermannGünthner2015}). This analytic dimension remains primordial both for the observation of the multimodal organisation (including gestures, gaze, body movements, facial expressions) and for the semiotic complexity of talk-in-interaction. It is relevant for practical purposes and the accountability of social actions, routinised practices, and recognisable emerging \textit{gestalts} grounding on the domain of intersubjectivity and co-operative engagement (\citealt{Levinson2013}, \citealt{Goodwin2018}).

\subsection{Talk-in-interaction as a complex adaptive system}

Despite a growing interest for complexity in language in the last two decades, to this date, little or no work has been done on naturally occurring talk-in-in\-ter\-ac\-tion through the prism of complexity. By referring to the key components proposed by \citet{Johnson2007} to describe a complex system, we argue that:

\begin{itemize}
\item Talk-in-interaction features a collection of many interacting agents, i.e. participants that are competing for a limited resource, the floor of conversation.\largerpage

\item Participants' behaviour is affected by their memory, i.e. their interactional history.\footnote{The term of \textit{interactional history} has been discussed by \citet{Deppermann2018} referring to the establishment of a \textit{common ground}, while the interaction between participants unfolds through time (within one single encounter or over several encounters).} Therefore, participants can adapt their strategies according to their history. 

\item Talk-in-interaction exhibits emergent phenomena that arise from the participants' situated actions and are not prewritten by an “invisible hand”.

\item Talk-in-interaction shows a mix of ordered and disordered behaviour.
\end{itemize}

The paradigm of complexity appears to be relevant when describing conversational phenomena, since the central concepts related to the paradigm according to \citet{Edmonds1999} can be considered:


\begin{itemize}
\item The size of interaction as a system is hardly definable in terms of composing elements.

\item To this day, no formal model of talk-in-interaction based on naturally occurring data\footnote{The regularity of certain conversational mechanisms allows for providing models of talk-in-interaction, such as it is done within CA (e.g., the infrastructure of the turn-taking machinery). However, these models only account for a limited number of dimensions of conversation (see \sectref{sec:12:1}). In order to provide a formal and predictive model of conversation, the CA framework is necessary but not sufficient for methodological reasons. Interesting work has been done in the domain of dialogue modelling; for example, the information-state based approach to dialogue developed by \citet{TraumLarsson2003} has been discussed with insights from CA by \citet{Ginzburg2012}. Being profoundly multidimensional (see, for example, the role of syntactic structures, prosodic clues and visual-embodied resources in action formation), modelling talk-in-interaction is a real challenge that, to our knowledge, has not yet been overcome.}  has been elaborated. Hence, it is difficult to estimate the minimum description size of such model.

\item The great number of variations present from one conversational setting to another, as well as from one singular interaction to another in a same setting, makes it difficult to grasp exhaustively stable and unstable elements.
\end{itemize}

\begin{sloppypar}
Furthermore, studies in CA show that conversational resources are profoundly adaptable\label{Ch12:profoundlyadaptable} “because of the reflexive relationship between action and context” (\citealt{HeritageClayman2010}: 21). As a matter of fact, each “conversational move” impacts and supplies the interactional context. It is thus of a particular interest to consider how conversational practices emerge and are locally and interactively achieved.
\end{sloppypar}

\subsection{The collaborative nature of talk-in-interaction}

The collaborative nature of talk-in-interaction has been accounted for through a variety of concepts: the cooperative principle \citep{Grice1975}, interactional synchrony (\citealt{CondonOgston1966}), joint actions and common ground \citep{Clark1996}, and, more generally, co-operative engagements relying on a large variety of available semiotic resources that are collaboratively indexed by participants for practical purposes \citep{Goodwin2018}. Studies in CA have been particularly focusing on the microanalysis of the turn-taking system, which is at the heart of the conversational device. Studying the human \textit{interaction engine}, \citet{Levinson2006} argued that talk-in-interaction is essentially cooperative, since participants intend their actions to be interpretable and to contribute to some larger joint venture (e.g. having a conversation). 

Moreover, in talk-in-interaction, participants continuously adjust to one another, adapting their actions/turns to the other and to the interactional contingencies. Through these adjustments, participants can either \textit{align} or \textit{disalign} with the others' actions/turns \citep{Stivers2008}. 

\section{The collaborative construction of multimodal conversational routines}\largerpage

As mentioned before, the conversational practices are on the one hand recognizable and, on the other hand, unique and displayed in singular contexts. It thus seems particularly interesting to consider moments in interaction where participants focus on a new element and turn it into a shared resource within their interaction. 

In what follows, we focus on multimodal accomplishments of the conversational practice of repetition (previously described as a verbal practice by \citealt{Traverso2005}, \citealt{Schegloff2007}, \citealt{Bazzanella2011}, among others). Among the very large variety of functions that repetition can implement in talk-in-interaction, it has been shown that it can express both agreement and disagreement \citep{Traverso2012}. In both cases, repetition contributes to the overall coordination of the conversation, since it operates on the common ground (\citealt{ClarkBernicot2008}). The practice under study here is a particular kind of multimodal repetition: 

\begin{enumerate}

\item A participant produces a verbal item in combination with some specific multimodal features – as a \textit{multimodal gestalt} (\citealt{Mondada2014Local, Mondada2018});

\item Other participants reuse the same multimodal configuration in the following exchanges.

\end{enumerate}

Our analysis shows how, through this repetition, the item acquires a special situated meaning that is shared by all participants. The practice of repeating the verbal item together with multimodal elements contributes to the constitution of its locally shared meaning, and illustrates the emergence of a conversational routine.


\subsection{Data}

The examined data consists of video-recorded naturally occurring interactions in French. For the present paper, two examples have been chosen, representing two situations of everyday life: a medical consultation \REF{ex:12:1} and a greengrocer's shop \REF{ex:12:2}. All data have been transcribed using ICOR conventions.\footnote{\url{http://icar.cnrs.fr/documents/2013_Conv_ICOR_250313.pdf}. For multimodal transcription, we use the conventions developed by Mondada: \url{https://www.lorenzamondada.net/multimodal-transcription}. See also the transcription conventions at the end of the chapter.} Multimodal annotations and screenshots have been added in order to account for all practices relevant in interaction, with an emic perspective (i.e. from the perspective of the participants themselves, \citealt{Pike1954}). All data have been anonymised.

\subsection{Analyses}\largerpage[2]

In the following excerpt, Vera and her son, Anton, talk about the health problem of the latter with a physician in France. The two visitors are Albanian, and do not speak French fluently; having lived in Italy, they speak Italian.

\ea\label{ex:12:1} (1) \textit{atchoum} [Corpus REMILAS\footnote{See REMILAS Project
(\url{http://www.icar.cnrs.fr/sites/projet-remilas}).} (see Figures 1 and 2)]\z

\begin{figure}[p]
\caption{Multimodal transcript, \textit{atchoum} (part 1)\label{fig:12:1}}
\includegraphics[width=\textwidth]{figures/a3-1.pdf}
\end{figure}

\begin{figure}[p]
\caption{Multimodal transcript, \textit{atchoum} (part 2)\label{fig:12:2}}
\includegraphics[width=\textwidth]{figures/a3-2.pdf}
\end{figure}

In this excerpt, a multimodal conversational routine emerges. It involves the onomatopoeia \textit{atchoum} and the combined movements of the head (down) and the hand (up, towards the head) acting out a sneeze. This configuration is at first carried out as an adequate response (l. 06--07) to the doctor's question, which is initially addressed to the boy (l. 01). The mother solicits a reactive move from her son (l. 03); following his hesitation (\textit{bah::}, l. 04), she self-selects herself as a responder (im. 1, open-palm gesture) and produces repeatedly the multimodal configuration, echoing the verbal repetition (\textit{beaucoup}, l. 07). This configuration is accomplished in line 07 through four instances of a head tilt and a hand movement down (im. 05--08), after \textit{atchoum}. Missing the lexical item that designates Anton's symptom, Vera illustrates it through a corporeal manifestation, which becomes a recurrent resource exploited by both the mother and the physician.

In line 11, the doctor formulates a question about the presence of the symptom in the past, when the family lived in Italy. The onomatopoeic \textit{atchoum}, accompanied by the multimodal configuration, is used as a turn-expansion (im. 10), in order to secure the questioned referent, with a gestural and verbal framing (\textit{en Italie}, im. 9 and 11). The negative answer is verbally produced by the mother (l. 14), and multimodally by her son (l. 15). In line 25, the doctor reuses this multimodal configuration for asking confirmation about the presence of \textit{atchoum} as the only symptom that Anton presents (\textit{juste (.) atchoum}, im. 12--13).

In this excerpt, the embodied accomplishement of \textit{atchoum} becomes a multimodal conversational routine that participants use throughout the whole exchange to refer to a precise symptom. Participants mobilise it as a shared resource for preserving the successful nature of the communication, ensuring mutual understanding. Significantly, the accomplishement of the multimodal configuration that accompanies the onomatopoeic item is not exactly the same across the two participants: even though the bodily resources mobilised in a multimodal conversational routine are subject to individual implementation, participants keep a recognisable shape of these mobilisations as part of the social construction of actions. The illustrative character of the sneezing gesture is evident at the end of the excerpt, when the mother introduces it by an explanatory device: \textit{comme ça} (l. 29).

In another setting, multimodal conversational routines can punctuate some context-dependent phases of the interaction. The following excerpt is issued from an exchange recorded in a greengrocer’s shop (see also \citealt{Traverso2016}).

\ea\label{ex:12:2} (2) \textit{tadam} [Corpus Primeur] (see Figures 3 and 4)\z


\begin{figure}[p]
\includegraphics[width=\textwidth]{figures/a3-3.pdf}
\caption{Multimodal transcript, \textit{tadam} (part 1)\label{fig:12:2}}
\end{figure}

 
\begin{figure}
\includegraphics[width=\textwidth]{figures/a3-4.pdf}
\caption{Multimodal transcript, \textit{tadam} (part 2)\label{fig:12:3}}
\end{figure}

Here, \textit{tadam} does not convey a specific meaning (unlike the onomatopoeic item \textit{atchoum}). Instead, it embodies a practice that is proper to the commercial setting, i.e. handing money before the leave-taking. More precisely, the seller stretches out his arm and says \textit{tadam} (l. 03, im. 2) in order to take the money from the hand of the client, who has previously responded to the announcement of the payment by stretching out her arm with the money (l. 02, im. 01). She then reproduces the vocal accomplishement (l. 04, im. 03), but it is once again the seller who utters this item while giving the change to the client (im. 04--05). He produces \textit{tadam} three times, corresponding to the actions of leaving the money and the banknote in her hand.

In this case, the multimodal conversational routine is mobilised by one participant, i.e. the seller, and is repeated by the client, who recognises this emergent configuration as a resource to highlight the money exchange within the context of a commercial transaction. Similarly to the previous case, the multimodal conversational routine presents a variety of accomplishements. Namely, the seller produces the last three occurrences (l. 06 and 09) with a falling intonation contour \textit{versus} the rising intonation of the previous one (l. 03). Here, the falling intonation signals the closing projection of the routine, which is followed by the closing statement of the seller (\textit{tout est là}, `everything is in here', l. 13) and the leave-taking.

In this excerpt, the participants orient to the activity of money exchange as an accountable activity, which can be delimited within the unfolding of the interaction: “doing being exchanging money” (cf. \citealt{Sacks1984}).

\section{Multimodal conversational routines as an illustration of interaction complexity}\largerpage

In this paper, the CA theoretical and analytical standpoints have led us to propose a new look on talk-in-interaction through the prism of complexity. We have offered an analysis of the conversational practice of reproducing a multimodal \textit{gestalt} by different participants within the same interaction; thanks to its repetition, it acquires a situated meaning. We have called this practice a \textit{multimodal conversational routine}. The practice is relevant in order to illustrate talk-in-interaction as a complex adaptive system. As a matter of fact, our analyses of naturally occurring data show that these routines feature both recognisability and uniqueness at each instance: even though the verbal element of the repetition is accurately reproduced, multimodal accomplishements may slightly vary from one occurrence to another, from one speaker to another. 

This reproducibility allows for example overcoming interactional troubles. In the first case, \textit{atchoum} accompanied by hand gestures and head movements is a resource to achieve mutual understanding between speakers who do not share the same language (or, at least, do not have a sufficient competence of one language). Participants crystallise and express a referent, not relying on lexical resources: the potential referential trouble is thus solved and the linguistic gap between speakers is filled. 

Moreover, the context matters: in medical settings the participants have urgent practical purposes, i.e. communicating symptoms in order to let the physician express the diagnostic outcome. In the second case, \textit{tadam} is an interjection that is accompanied by transactional gestures (giving money, taking money, money exchanging), it punctuates the phases preceding the closing of the interaction. Once this resource is mobilised by one participant, the co-participant suddenly aligns with him. This case is interesting for its unexpected and emergent character, as opposed to the first example, illustrating the exploitation of a multidimensional accomplishement of an onomatopoeic item, which is issued from a shared linguistic (but non-lexical) \textit{repertoire}. There is no contextual pressure in the second example; the practical purpose for using the conversational routine could be the playful nuance that is added on a specific phase of the transactional script in a commercial setting.\largerpage

More generally, the mobilisation of these lesser known conversational objects is treated as unproblematic by all participants, who can reuse them with a certain degree of variability (amplitude of gestures, additional facial expressions, specific prosodic contours of the verbal item, etc.). Thus, there is a degree of recognisability for multimodal conversational routines implemented by participants as vernacular forms for interaction in given contexts, to which participants can easily adapt their interactional style. All the features that we have highlighted for the characterisation of multimodal conversational routines – recognisability, variability, emergence, collaboration – illustrate successfully the multidimensional and complex nature of interaction.\footnote{This is our main contribution to the plea in favour of overcoming reductionist views on language complexity (see \citetv{chapters/00}).}

As prospective research, it would be interesting to track the evolution over time of the \textit{gestalts} that are used as multimodal conversational routines, i.e. style specificities and the appropriation of other's way for indexing referents and moments in interaction (through bodily conduct and verbal resources). Once defined as an analytic category, multimodal conversational routines would be retrieved and mapped, across languages and interaction types, as a key for exploring interaction as a complex adaptive system, and a locus of creativity and intersubjectivity.

\section*{Acknowledgements}
The authors are grateful to the ASLAN project (ANR-10-LABX-0081) of the Université de Lyon, for its financial support within the French program “Investments for the Future” operated by the National Research Agency (ANR).

\section*{Transcription conventions}

\begin{tabbing}
((laughter))\hspace{1ex}\=Delimitation of described phenomenon\kill
[oui]    \>  Overlapping talk                                       \\
/   \    \>  Rising or falling intonation of the prior segment    \\
°oui °   \>  Lower voice                                        \\
:        \>  Prolongation of the prior sound                       \\
p`tit    \>  Elision                                                \\
trouv-   \>  Truncation of a word                                   \\
=        \>  Latching, turn continues on new line                  \\
(peut-être)   \>  Uncertain transcription                        \\
((laughter))  \>  Comment, transcriber's description             \\
<(2.4)((laughter))> \hspace{.5ex} Delimitation of described phenomenon       \\
\&       \>  Turn of the same speaker interrupted by an overlap     \\
(.)      \>  Micro-pause (<0.2 s)                                 \\
(0.6)    \>  Timed pause in seconds and tenths of second          \\
**       \>  delimit gestures done by VER                         \\ 
@@       \>   delimit gestures done by DOA                        \\
+ +      \> delimit gestures done by CLI                          \\
\% \%    \> delimit gestures done by SEL                         \\
\#       \>  indicates the exact point where a screen shot (image) \\
        \> has been taken within a turn or a time measure        \\
*- - ->    \> gesture continues across subsequent lines   \\
*- - ->>   \> gesture continues after the excerpt's end   \\
- - ->*    \> gesture continues until the same symbol is reached \\
. . . . \> gesture's preparation                       \\
- - - - \> gesture's apex is reached and maintained    \\
, , , ,  \> gesture's retraction
\end{tabbing}

{\sloppy\printbibliography[heading=subbibliography,notkeyword=this]}
\end{document}
