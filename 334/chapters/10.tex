\documentclass[output=paper]{langscibook} 
\ChapterDOI{10.5281/zenodo.6620123}
\author{Audrey Mazur\affiliation{UMR 5161 ICAR; Centre National de la Recherche Scientifique; Ecole Normale Supérieure; Université Lumière Lyon 2}
        and Véronique Traverso\affiliation{UMR 5161 ICAR; Centre National de la Recherche Scientifique; Ecole Normale Supérieure; Université Lumière Lyon 2}}
\title{Introduction to complexity, interaction and multimodality}
\abstract{The third and final section of this book is based on the observation that interactive processes involve a large number of elements, related to language, but also to other semiotic fields. Interactions are embedded in contexts and contain activities and actions. Participants are characterised in a multitude of ways and their objectives unfold over time. These heterogeneous elements are also linked by multiple relationships that may also evolve thus satisfying one criterion of a complex system. Emergence is also key as researchers describe unexpected or unpredictable phenomena, either fleeting or that provoke a complete restructuring of the interaction. The chapters in this section focus on a variety of interactional and linguistic analyses on empirical data that question models or will otherwise elaborate on the theme of complexity, interaction, and multimodality.}
\IfFileExists{../localcommands.tex}{
  \addbibresource{localbibliography.bib}
  \usepackage{langsci-optional}
\usepackage{langsci-gb4e}
\usepackage{langsci-lgr}

\usepackage{listings}
\lstset{basicstyle=\ttfamily,tabsize=2,breaklines=true}

%added by author
% \usepackage{tipa}
\usepackage{multirow}
\graphicspath{{figures/}}
\usepackage{langsci-branding}

  
\newcommand{\sent}{\enumsentence}
\newcommand{\sents}{\eenumsentence}
\let\citeasnoun\citet

\renewcommand{\lsCoverTitleFont}[1]{\sffamily\addfontfeatures{Scale=MatchUppercase}\fontsize{44pt}{16mm}\selectfont #1}
   
  %% hyphenation points for line breaks
%% Normally, automatic hyphenation in LaTeX is very good
%% If a word is mis-hyphenated, add it to this file
%%
%% add information to TeX file before \begin{document} with:
%% %% hyphenation points for line breaks
%% Normally, automatic hyphenation in LaTeX is very good
%% If a word is mis-hyphenated, add it to this file
%%
%% add information to TeX file before \begin{document} with:
%% %% hyphenation points for line breaks
%% Normally, automatic hyphenation in LaTeX is very good
%% If a word is mis-hyphenated, add it to this file
%%
%% add information to TeX file before \begin{document} with:
%% \include{localhyphenation}
\hyphenation{
affri-ca-te
affri-ca-tes
an-no-tated
com-ple-ments
com-po-si-tio-na-li-ty
non-com-po-si-tio-na-li-ty
Gon-zá-lez
out-side
Ri-chárd
se-man-tics
STREU-SLE
Tie-de-mann
}
\hyphenation{
affri-ca-te
affri-ca-tes
an-no-tated
com-ple-ments
com-po-si-tio-na-li-ty
non-com-po-si-tio-na-li-ty
Gon-zá-lez
out-side
Ri-chárd
se-man-tics
STREU-SLE
Tie-de-mann
}
\hyphenation{
affri-ca-te
affri-ca-tes
an-no-tated
com-ple-ments
com-po-si-tio-na-li-ty
non-com-po-si-tio-na-li-ty
Gon-zá-lez
out-side
Ri-chárd
se-man-tics
STREU-SLE
Tie-de-mann
} 
  \togglepaper[1]%%chapternumber
}{}

\begin{document}
\maketitle 
%\shorttitlerunninghead{}%%use this for an abridged title in the page headers


The chapters in this section demonstrate above all that interaction, whatever its nature, context, objectives, or role in a broader whole, constitutes, as such, a complex process. In simple terms (if we dare in this context), interactive processes are complex first and foremost because, if considered as composite systems, they involve a very large number of elements: resources related to languages (syntax, lexicon, words, sounds, etc.) as well as to other semiotic fields (gestures, gaze, face expression, manipulation of objects and artefacts); different senses (sight, hearing, touch, smell); contexts, activities and actions; objectives (that can be local, global, and that evolve as the exchanges unfold); stakes of different levels; participants, to whom are attached numerous possible characterisations, such as ongoing social relations, identities, cultures, emotions, etc.

This Prévert-style inventory, though partial, highlights the heterogeneity of the elements which an attempt to describe the functioning of interaction should, at one time or another, consider. Indeed, the number and heterogeneity of elements that play a role in interaction are not sufficient to speak of complexity (see \textcitetv{chapters/00}). However, other characteristics make it possible to attribute to interaction this qualification. Foremost is the fact that these elements are linked by multiple relationships, which are forms of organisations (“gestalts” as conversation analysts sometimes say), but which are not necessarily stable. To echo \citet{Genelot2014} taking over Morin's view on complexity, it can be said that these elements do not add up (talking about “everything” and “parts” does not make sense), but are organised in different ways, which – and this is another reason for complexity – change throughout the temporal course of the interaction. We can also add that these organisations do not necessarily include all the elements that could possibly be involved, and that from one organisation to another, and from one moment to another of the interaction, it is necessarily the same elements that combine in a relevant way in order to understand the interactional process.

In terms of temporality, interaction as a process is characterised by the fact that each new action carried out is likely, either to continue the current trajectory, or to redefine it and therefore to requalify the previously constructed whole.

These organisations are describable and recognisable, but they remain unstable in the sense that they also give way to local emergences, unexpected follow-up conducts or unpredictable phenomena, which may be only fleeting or, on the contrary, provoke complete restructuring.

Over the past 50 years, research on interaction has grown increasingly, and much has been done to understand these processes, shedding light on their hitherto unsuspected complexity, on the verbal, vocal and gestural levels, as well as in the fine-grained weaving of these different dimensions. Some scholars have tried to establish a general model of human interaction (see for example in France \citealt{Roulet1991}, \citealt{Trognon2003}). Others have sought to highlight local orders at certain levels of interactive functioning (starting with \citealt{SacksEtAl1974} on the turn-taking system). Many others have worked on descriptions of an extremely fine degree of granularity for highlighting local orders that characterise certain levels of interactional functioning over a short temporal span (an example would be Goodwin’s descriptions of interactions with an aphasic man, \citealt{Goodwin2018, Goodwin2003}).

\begin{sloppypar}
The chapters in this part of the book, which refer to different fields of research and approaches (discourse-in-interaction analysis (DIA), \citealt{Kerbrat-Orecchioni2005}; argumentation studies, \citealt{Plantin2016}, \citealt{Doury2016}; conversation analysis and interactional linguistics, \citealt{SidnellStivers2012,Traverso2016}; second language acquisition, \citealt{CoyleEtAl2010,EllisLarsenFreeman2009}; multimodality, \citealt{StreeckEtAl2011,Mondada2014Corps}; research in education, \citealt{Kress2001}), position themselves in different ways in relation to the proposals referred to above. They give more or less space to a reference model and to predetermined categories. Most of them on the other hand develop a mixed approach (be it cross-disciplinary, multidisciplinary or interdisciplinary, \citealt{Falk-Krzesinski2016,Narcy-Combes2018}), considering that only hybrid theoretical and methodological frameworks are able to help us understand and conceptualise complexity in human activities \citep{SuthersEtAl2013}. It is therefore necessary to examine the conditions under which decision-making trajectories from science, practitioners, politicians and other actors can be carried out together \citep{Scholz2017}, leading to cross-disciplinary or even transdisciplinary projects \citep{Mazur-PalandreEtAl2019}.
\end{sloppypar}

All the chapters are based centrally on empirical data in which recurrences are sought, which will confirm or question the chosen models, or which will lead to an elaboration on the theme: complexity, interaction and multimodality.

The paper by \textcitetv{chapters/11} deals with “interaction and complexity” through the issue of how to assess the quality of students' reasoning in the specific context of a “scientific  café” at school. The authors choose a mixed methodology, articulating argumentation studies, discourse-in-interaction analysis, and research in education. Their analysis of a stretch of debate among students leads them to argue that grasping the quality of argumentation in dialogue necessitates multiple units of analysis. 

\textcitetv{chapters/14} address the issue of “interaction and complexity” in the field of second language acquisition, on the basis of a view of language development as a co-adaptive and iterative process, the study of which interaction is the natural site of observation. They examine how “constructions”, as language resources, go through the process of consolidation and generalisation of language through their recycling in classroom interaction. The main objective of the paper is to contribute to a better understanding of the link between patterns of verbal interaction in this classroom setting and the development of second language competence. The analysis, which includes multimodality, mixes a qualitative and quantitative approach to the data, and shows how task-based approaches integrating language and content learning in an immersion classroom setting creates good conditions for a second language acquisition and development.

The paper by \textcitetv{chapters/12} deals with the issue of “interaction and complexity” through discussing the balance between “regularity” and “recognisability” on the one hand, and “irregularity” and “unexpectedness” on the other hand. The paper questions two different types of routinised turns, contrasting a well-known and expected one with an emerging one. Through examining the repeats of these routines in interaction, it shows that a reverse process is at work in the two cases: the “regular” routine becomes used unexpectedly (both syntactically and actionally); and the emergent routine ends up being a stabilised turn format shared by the participants in the given interaction. Using conversation analysis methods, the authors show that conversational routines can be a resource for mutual understanding between speakers from different cultures with different languages, in addition to constituting vernacular forms facilitating interaction. 

\textcitetv{chapters/13} focus on participation frameworks in a particular situation of multiactivity: the interaction of five members of a family during a card game. Four extracts are analysed using interactional linguistic methods. The analysis of this very common and apparently simple context (card game in family) highlights how the collaborative and temporally organised character of interaction might be a good indicator of complexity and contribute, on a broader level, to the thinking about language complexity.

This third part of the book offers an interesting panel of interactional and linguistic contexts analysed with different types of tools, via different scientific methods. This richness permits the authors to discuss the theoretical, methodological and analytical conditions in which interactional practices, including dimensions, can be observed and interpreted in their complexity, and to propose ways to better understand how interactional complexity works in various contexts.

\begin{center}
\parbox{\textwidth-2cm}{%
\textit{Our warm thoughts go to Peter Griggs who has just passed away at the moment we edit this manuscript. We won’t forget Peter’s passion for research, his rigor and the finesse he brought to analyses. He always had a kind word for others, and the quality of his presence facilitated collaborative work.} }
\end{center} 

\section*{Acknowledgements}\largerpage
The authors are grateful to the ASLAN project (ANR-10-LABX-0081) of the Université de Lyon for its financial support within the French program “Investments for the Future” operated by the National Research Agency (ANR).

{\sloppy\printbibliography[heading=subbibliography,notkeyword=this]}
\end{document} 
