\documentclass[output=paper]{langscibook} 
\ChapterDOI{10.5281/zenodo.6620129}
\author{Heike Baldauf-Quilliatre\affiliation{UMR 5161 ICAR; Centre National de la Recherche Scientifique; Ecole Normale Supérieure; Université Lumière Lyon 2}
        and Isabel {Colon de Carvajal}\affiliation{UMR 5161 ICAR; Centre National de la Recherche Scientifique; Ecole Normale Supérieure; Université Lumière Lyon 2}}
\title{Multimodal practice of participation in a complex and dynamic framework}
\abstract{Based on the argument that the resources we use in conversation are highly adaptable, we focus on the collaborative and temporally organised character of interaction in order to describe how speakers display participation. Drawing especially on Goffman's and Goodwin's work, we analyse how players of family board and card games use different resources to indicate shifts in participation frameworks. Moreover, our sequential analysis of four excerpts focusing on describing talk, gaze, body posture, gesture, and the handling of artefacts reveal an overlapping of participation frameworks. Participants orient to gaming as an overall structuring activity, but they will simultaneously accomplish individual actions and form ephemeral groups in order to co-construct specific action trajectories that will nevertheless interact with the overarching activity. Our results illustrate both the patterns in interaction as well as its non-linear and emergent character.}
\IfFileExists{../localcommands.tex}{
  \addbibresource{localbibliography.bib}
  % add all extra packages you need to load to this file

\usepackage{tabularx,multicol}
\usepackage{url}
\urlstyle{same}

\usepackage{listings}
\lstset{basicstyle=\ttfamily,tabsize=2,breaklines=true}

\usepackage{langsci-basic}
\usepackage{langsci-optional}
\usepackage{langsci-lgr}
\usepackage{langsci-osl}
% \usepackage{./langsci/styles/langsci-lgr}
% \usepackage{./langsci/styles/langsci-osl}
% \usepackage{langsci-gb4e}

\usepackage{tikz}
\usetikzlibrary{patterns,calc}
\pgfdeclarepatternformonly{south east lines}{\pgfqpoint{-0pt}{-0pt}}{\pgfqpoint{3pt}{3pt}}{\pgfqpoint{3pt}{3pt}}{
    \pgfsetlinewidth{0.6pt}
    \pgfpathmoveto{\pgfqpoint{0pt}{3pt}}
    \pgfpathlineto{\pgfqpoint{3pt}{0pt}}
    \pgfpathmoveto{\pgfqpoint{.2pt}{-.2pt}}
    \pgfpathlineto{\pgfqpoint{-.2pt}{.2pt}}
    \pgfpathmoveto{\pgfqpoint{3.2pt}{2.8pt}}
    \pgfpathlineto{\pgfqpoint{2.8pt}{3.2pt}}
    \pgfusepath{stroke}}
    
\usepackage{stmaryrd}
\usepackage{wasysym}
\usepackage{multirow}
\usepackage{caption}
\usepackage{subcaption}
\usepackage{mathrsfs}
\usepackage{qtree}

\usepackage{linguex}


  %pminos do not split footnotes
% \interfootnotelinepenalty=10000 %Footnote in Laporte chapters has to be split SN


%\DeclareIndexNameFormat{default}{%
%\nameparts{#1}%
%\usebibmacro{index:name}%
%{\index[names]}%
%{\namepartfamily}%
%{\namepartgiveni}%
% {}% L1
% {}% L2
%{\namepartprefix}% generates spurious space L3
%{\namepartsuffix}% generates spurious space L4
%}

%  {\DeclareIndexNameFormat{default}{%
%     \usebibmacro{index:name}{\index[names]}{#1}{#3}{#5}{#7}}}

%\DeclareIndexNameFormat{default}{%
%  \usebibmacro{index:name}{\sindex[nom]}{#1}{#3}{#5}{#7}}

%\DeclareIndexNameFormat{default}{%
%  \usebibmacro{index:name}{\sindex[person]}{#1}{#3}{#5}{#7}}
%\DeclareIndexNameFormat{default}{%
%\nameparts{#1} \usebibmacro{index:name}{\sindex[person]]}{\namepartfamily}{‌​\namepartgiven}{\nam‌​epartprefix}{\namepa‌​rtsuffix}}

%\newcommand{\smiley}{:)}

%\renewbibmacro*{index:name}[5]{%
%\usebibmacro{index:entry}{#1}%
%{\iffieldundef{usera}{}{\thefield{usera}\actualoperator}\mkbibindexname{#2}{#3}{#4}{#5}}}

% \newcommand{\noop}[1]{}

%remove for final
%\overfullrule=1mm

\newcommand{\tobi}[2]}}
\renewcommand{\S}[1]{\tobi{#1}{\textsc{*}}}

% this volume references
% puts: [this volume]
% already defined: \citetv
%\newcommand{\citepv}[1]{(\citeauthor{#1} \citeyear*{#1} [this volume])}
\newcommand{\citealtv}[1]{\citeauthor{#1} \citeyear*{#1} [this volume]}

%parentheses around example number
\newcommand{\pref}[1]{(\ref{#1})}

% in-text examples

\newcommand{\lnex}[1]{\textit{#1}} %target lang word
\newcommand{\lnlit}[1]{(lit.: `#1')} %literal reading
\newcommand{\lnlat}[1]{(#1)} % latinization
\newcommand{\lntrans}[1]{`#1'} %translation
\newcommand{\lnexl}[2]%
{\lnex{#1}{} \lnlat{#2}} % ex with latinization
\newcommand{\lnexlat}[3]{\lnex{#1}{} \lnlat{#2}{} \lntrans{#3}} % ex with latinization and tranl.

%ch01
\newcommand{\co}[1]{\mbox{\textbf{#1}}}

%ch09

\newcommand{\cyrbulg}[1]{\begin{otherlanguage*}{bulgarian}#1\end{otherlanguage*}}


%ch10
\newcommand{\nlp}{{\small NLP}}
\newcommand{\mwe}{{\small MWE}}
\newcommand{\rae}{{\small RAE}}
\newcommand{\lvc}{{\small LVC}}
\newcommand{\pos}{{\small P}o{\small S}}
%\newcommand{\todo}[1]{ \textcolor{red}{#1} }

%\renewcommand{\labelenumi}{\theenumi}
%\ainamefmt{{vv}{ll}{, ff}{, jj}} % fullname

\newcommand{\biberror}[1]{{\color{red}#1}}

\newcommand{\osenovaitem}{--~} 
  %% hyphenation points for line breaks
%% Normally, automatic hyphenation in LaTeX is very good
%% If a word is mis-hyphenated, add it to this file
%%
%% add information to TeX file before \begin{document} with:
%% %% hyphenation points for line breaks
%% Normally, automatic hyphenation in LaTeX is very good
%% If a word is mis-hyphenated, add it to this file
%%
%% add information to TeX file before \begin{document} with:
%% %% hyphenation points for line breaks
%% Normally, automatic hyphenation in LaTeX is very good
%% If a word is mis-hyphenated, add it to this file
%%
%% add information to TeX file before \begin{document} with:
%% \include{localhyphenation}
\hyphenation{
    Beck-man
    Ngu-yen
    back-chan-nel
    back-chan-nels
    mo-not-o-nous
    ste-reo-typ-i-cal
}

\hyphenation{
    Beck-man
    Ngu-yen
    back-chan-nel
    back-chan-nels
    mo-not-o-nous
    ste-reo-typ-i-cal
}

\hyphenation{
    Beck-man
    Ngu-yen
    back-chan-nel
    back-chan-nels
    mo-not-o-nous
    ste-reo-typ-i-cal
}
 
  \togglepaper[1]%%chapternumber
}{}

\begin{document}
\maketitle 

\section{Introduction}


In their paper \textit{Language is a complex adaptive system} (CAS), \citet[1]{BecknerEtAl2009} state that “language has a fundamentally social function”, a claim shared by other methodological approaches such as conversation analysis and interactional linguistics which show that language has to be considered as \textit{talk-in-interaction}, as \textit{situated} and \textit{embodied}. \textcitetv{chapters/12} argue from a conversation-analytic point of view that not only language, but \textit{interaction} can be described within the framework of complexity and adaptability since conversational resources are “profoundly \textit{adaptable}” (p.~\pageref{Ch12:profoundlyadaptable}). Based on this argumentation, we show through a detailed interactional analysis how different features described with regard to the complexity of language coincide with key concepts of conversation analysis (CA), and how a consideration of the collaborative and temporally organised character of interaction might be fruitful for further reflexions on complexity.

Our paper deals with the way speakers display participation in interactions. As was shown very early in interactional research (by \citealt{Goffman1981}), gatherings do not become automatically social encounters and speakers can participate in different ways in interaction. Drawing on Goffman’s observations, \citet{GoodwinGoodwin2004} show very convincingly that these different types of participation are not defined categories, which draw on the speakers’ \textit{footings.} Speakers and hearers continuously display their footings and how they participate in the sequential construction of interaction (see also \textcitetv{chapters/11}). Participation is considered as \textit{practice} “through which different kinds of parties build action together by participating in structured ways in the events that constitute a state of talk” (\citeyear{GoodwinGoodwin2004}: 225); in other words, a participation framework (\citealt{GoodwinGoodwin1992}) is strongly related to the sequential organisation of activities. The growing interest in multiactivity (e.g. \citealt{HaddingtonEtAl2014}) has further highlighted the \textit{dynamics} of a participation framework: accomplishing simultaneously different actions and being engaged simultaneously in different activities means also participating simultaneously in different courses of (inter)action and therefore displaying different ways of participating.

We focus on the participation framework in a particular situation of multiactivity: the interaction of five members of a family during a card game. After presenting our data and the methodology we adopt (\sectref{sec:13:2}), we show a case analysis in which we point to different practices of participation (\sectref{sec:13:3}). We then relate our findings to several features presented in the CAS approach (\sectref{sec:13:4}). 

\section{Data and methodology}\label{sec:13:2}

Our study follows an emic approach to complexity: we are interested in the way in which interaction is constructed by the simultaneous and sequential emergence of multiple elements on different levels and in the way these elements are related to and respond to each other (\citealt{Mondada2019}, \citealt{Keevalik2018}, \citealt{ImoLanwer2016}). We consider interaction as ingeniously multimodal and we are interested in the multimodal practices used to indicate shifts in participation framework (\citealt{CobelasCartagenaPriego-Vazquez2019}). 

The paper draws on seven hours of board and card game data, recorded in 2015 within the project \textit{JouEs!: Playing Together – Interactional practices of gaming}, funded by the Laboratory of Excellence ASLAN (Advanced Studies on Language Complexity). It shows excerpts of one gaming situation\footnote{For a detailed literature review of gaming from an interactional perspective see \citet{HofstetterRobles2018}.}  where the parents AMA and NOE play together with their three daughters ELI, JEN and LEA different board games, sitting around a large table in the living room. The first game they choose is \textit{Mistiboo}, an Old Maid card game where the odd card is a cat (the \textit{Mistigri}). Though all participants are engaged in the gaming activity, the interaction is not straightforward: different ephemeral groups occur, sequences overlap and the players switch constantly between individual actions and interactions to co-construct the gaming situation. 

\section{Case analysis}\label{sec:13:3}
We propose a sequential analysis of four excerpts from the very beginning of the game, adopting an interactional approach and focusing on the emergent construction of turns and (action) sequences. All data have been transcribed using ICOR conventions,\footnote{\url{http://icar.cnrs.fr/documents/2013_Conv_ICOR_250313.pdf}. For multimodal transcription, we use the conventions developed by Mondada: \url{https://www.lorenzamondada.net/multimodal-transcription},} but see our specific transcription conventions at the end of this chapter, before the references.

Once the game is chosen and the cards distributed, JEN, ELI and NOE (who forms a team with LEA) take their cards and arrange them in their hands. Meanwhile AMA reads the rules on the rule card. After having collaboratively identified LEA as the youngest player who has to start the game, AMA finally also takes her cards and NOE explains to LEA how to arrange them best in her hands. The start of the game is slightly deferred while ELI orients to another pre-gaming activity: searching matching pairs. All of the participants show that they are looking for matching pairs and announce that they don’t have any, except AMA who silently puts two matching cards on the table. At this moment, all participants are engaged individually in their own activity. Although these individual activities are part of the game and thus “authorised”, the participants verbalise what they are doing and indicate the end of their search. These very minimal turns maintain the joint focus on the gaming interaction as a shared and collaborative co-construction and orient to the participation framework of \textit{gaming} as a general and overarching activity.

The activity of searching is overlapped by different sequences involving different participants: 

\begin{enumerate}
\item NOE twice instructs ELI, who sits beside him not to show her cards (and thus to change her posture). 
\item AMA retakes the rule card and reads the rules partly silently and partly aloud. She structures the ongoing game and is answered by other participants who interpret this reading aloud differently. 
\item LEA responds to some of the turns addressed to all participants but sometimes with a delay or by answering only parts of the turn. She keeps going on sequences that have already been closed. These overlapping sequences are generally very short and respond to different micro-problems which have emerged locally. They involve different participants and form different ephemeral groups. 
\end{enumerate}

In our analysis, we focus on two aspects concerning the complexity and the entanglement of this overlapping participation framework. We first show how participants use different modalities to interact within different participation frameworks. Then we describe how the categorisation and construction of groups emerge in this interaction. 

The first extract (\figref{fig:13:extract1}) starts with AMA, NOE, JEN and ELI, who are searching for matching pairs. LEA is first looking out of the window but soon comes back to the cards, too. AMA, ELI and JEN announce what they are doing in different ways (l.28, 29, 30), while NOE searches silently. 


\begin{figure}[p]
\caption{Extract 1\label{fig:13:extract1}}
\includegraphics[height=.95\textheight]{figures/a4BaldaufQuilliatreandColondeCarvajalfinal-img001.pdf}
\end{figure}

While looking at his cards and thus being involved in an individual gaming activity, NOE addresses an instruction to ELI (l.31): “don’t show me your cards”. The turn is accompanied by an arm gesture in ELI’s direction, marking a border between the two of them. The instruction is followed by an account, explaining how to carry out the instructed action (“turn away a bit”, l.32). Simultaneously, ELI effectively turns her upper body and accompliswhat ELI does by an evaluative post-completion musing \citep{Schegloff2007}, indicating that ELI’s change in posture has been understood as a responding action to his instruction and therefore closes the sequence (“that’s it”, l.32). 

NOE’s verbal turn and his arm gesture together construct the instruction turn. They are produced simultaneously and amplify each other: the gesture makes the space relevant; the verbal turn specifies it as part of an instruction. ELI’s response is completely nonverbal \citep{Keevalik2018}: she interprets the turn as an instruction to change posture and accomplishes the instructed action. However, during the whole sequence NOE and ELI do not gaze at each other; they continue looking at their cards, indicating that they are still acting within the actual gaming activity: searching for matching pairs. Gaze is therefore used in a competitive sense with regard to gesture and language. The extract shows how different elements can be used together in order to amplify an action as well as how they can be used in a more conflictive way to accomplish different things (in other words, it shows its complexity): while gaze indicates the continuity with the on-going participation framework, language and gesture make it possible to open a new sequence in overlay with the ongoing activity which involves only two of the players.

This kind of conflictive use of different elements is rather frequent in our corpus since it allows a response to very local problems (“local sensitivity”, \citealt{Bergmann1990}) by continuing the ongoing gaming activity. Extract 2 (\figref{fig:13:extract2}) which occurs shortly after is very similar.

\begin{figure}[p]
\caption{Extract 2\label{fig:13:extract2}}
\includegraphics[height=.85\textheight]{figures/a4BaldaufQuilliatreandColondeCarvajalfinal-img002.pdf}
\end{figure}

AMA is reading out loud one of the rules of the game. She hereby closes the searching sequence and opens the following activity: “the game can then begin” (l.45). The beginning has already been anticipated by ELI (l.41): she leans forward so that LEA, the youngest player and identified as the one who starts, may easily take one of her cards. ELI’s change of posture does not request a (verbal) action of another participant. Nevertheless, NOE responds to this change: he produces an instruction and simultaneously points to ELI. Again, the use of different modalities makes it possible to amplify and to single out the instruction as relevant at this moment while the gaze at the cards shows the continuity with the on-going gaming activity. ELI responds to NOE’s turn by silently accomplishing the instructed action: she leans backwards so that her cards become invisible for NOE.

The third extract goes back to the beginning of the game and occurs directly before Extract 1. All players are looking at their cards, except AMA who is reading aloud the rules of the game. The extract starts when AMA reads the rule that the youngest player begins (l.21). 

\begin{figure}[p]
\caption{Extract 3\label{fig:13:extract3}}
\includegraphics[height=.85\textheight]{figures/a4BaldaufQuilliatreandColondeCarvajalfinal-img003.pdf}
\end{figure}

 
ELI is the first participant answering AMA’s turn (l.22; “it is lea”); she considers the reading as an interactive turn which requires an answer and orients to the ongoing game. Meanwhile NOE turns to LEA who is looking at the cards in her hands. He initiates a new sequence (l.25; “you've to put together what goes together”) and a new action trajectory explaining to LEA how to arrange cards. Though LEA does not respond, NOE’s turn initiates a new activity in the game in which all players are involved (“searching matching pairs”, see Extract 1). But simultaneously, in l.26, ELI seems to initiate a repair, starting a turn which uses the same syntactic structure as in l.22 (“it is lea”, l.22 vs. “it is uh::”, l.26) and therefore going back to an unsolved problem – her orientation to the start of the game has not been ratified. 

The reaction of LEA in lines 26--27 (see Extract 3) is particularly interesting with regard to two elements of her turns and to the adaptability of semiotic and interactional resources. First, she orients her gaze towards ELI immediately at the end of the repair, foreshadowing her identification as the one who starts (l.26). Her gaze precedes a change in posture in order to draw the first card. Second, LEA answers verbally “we it is w- it’s us” (l.27) and thereby shows a categorisation of herself as “us” (LEA and NOE). She reorients the category “youngest player” towards her team and to that extent re-negotiates the category of what is meant by “youngest player”.

The negotiation of categories \citep{Stokoe2012}, e.g. “speaking as an individual player” or “speaking as member of a team”, is rather frequent. Shortly after the previous excerpt, NOE initiates a question–answer sequence in Extract 4, asking LEA if she sees any identical cards (l.35). His question is accompanied by a gesture separating the cards in LEA’s hand in order to compare them more easily.   

\begin{figure}[p]
\caption{Extract 4\label{fig:13:extract4}}
\includegraphics[height=.95\textheight]{figures/a4BaldaufQuilliatreandColondeCarvajalfinal-img004.pdf}
\end{figure}
 

At this moment, LEA starts to gaze successively at the cards in her and NOE’s hands. She does not answer a polar question but produces a responsive action to a request (searching silently for matching pairs). Meanwhile, JEN and ELI announce that they don’t have any matching pairs. In l.39, NOE pursues this announcement series, in overlap with ELI. Without waiting for LEA to answer his question, he responds as an individual (“ well I don't think”) for their team (“we have any either”). AMA then terminates the “searching for matching pairs” activity by reading aloud the rule (see Extract 1).

During all this time, LEA continues gazing at her and NOE’s cards. She then produces a responsive turn in overlap with AMA (l.42). This turn takes into account elements of the previous turns and seems to be a post-completion of the sequence. She positions herself as somebody who also “knows how to make matching pairs”. Finally, in l.48, after the instruction sequence between NOE and ELI (analysed in Extract 2), LEA confirms NOE’s announcement concerning the matching pairs in their team (“we don’t have any”) and hereby finally indirectly answers NOE’s polar question (l.35). Her categorisation as a team is simultaneously illustrated by NOE assembling all their cards in his hands.

\section{Discussion}\label{sec:13:4}

The extracts have shown some aspects which seem to be particularly interesting with regard to the complexity of interaction and the adaptability of semiotic and conversation resources: on the one hand, a very common and apparently simple scene of life reveals a dynamic and fluctuating participation framework with different ephemeral groups, overlapping sequences and multiple trajectories of action and interaction. On the other hand, different types of resources are used in different ways and allow participants to interact simultaneously in different participation frameworks. 

When \citet{BecknerEtAl2009} characterise language as a CAS, they draw mostly on seven different arguments. They argue for instance that language structures are intertwined with and dependent on local particularities. This approach seems to be complementary to general claims and findings of CA, such as the conception of interaction as co-construction of the participants, the emergence of structures in the process of interaction or the multiplicity of interconnected elements and modes participating in the organisation of interaction.

According to CA, interaction is organised in sequences, whether they are constituted of verbal turns or verbal and nonverbal ones \citep{Keevallik2018}. Different sequences can show an \textit{overall structural organisation} \citep{Robinson2013} and constitute a more or less extended activity (\citealt{HeritageSorjonen1994}). In our example, the opening phase of the card game is constituted by different \textit{sequences of sequences} \citep{Schegloff2007} and \textit{sequences of actions}. Some of them are \textit{a priori} done individually (e.g. searching for matching pairs), others have to be accomplished interactionally (e.g. establishing the person who starts playing). The players do not always progress in the same way, since sequences as well as \textit{sequences of sequences} may overlap. At the same time, extra-linguistic elements or events can locally become important and relevant for interaction. Changes in posture, for instance, can be identified as strategically positive or negative and to that extent are addressed as interactionally relevant. Participants, then, have to deal with different overlapping trajectories and with a non-linear interactional structure by constantly adapting their practices and the resources they use. 

The emergent character of structures is closely related to the co-construction of turns, sequences and \textit{sequences of sequences}: each turn has to take into account prior turns and activities, the (multiple) participation framework, and the constraints and opportunities of the interactional site as well as the speaker’s entitlement to take the turn and the type of action they are asked to produce. While AMA and ELI focus on the identification of the one who starts the game (Extract 3, l.21--22), NOE opens another sequence, including LEA and himself as members of a team (l.25). His turn treats the previous sequence as closed and goes on with other gaming activities. ELI and LEA nevertheless come back to the previous sequence and show that it has not been closed yet.

Since turns emerge in interaction and draw attention to what happened previously, sequences and activities do not always proceed in linear ways: ELI’s and LEA’s turns draw attention to AMA’s reading aloud, while NOE’s turn draws attention to the way LEA arranges the cards in her hand. The “searching for matching pairs” activity shows how participants nevertheless orient to progressivity (\citealt{StiversRobinson2006}), not in a linear way, but by constantly paying attention to local changes. 

\textit{Sequence} organisation, as well as \textit{sequential} organisation, are closely related to the projection and ascription of actions: turns and actions are constantly adapted to local micro-problems. AMA for instance is reading aloud the game rules, gazing at and focusing on the rule card. While the reading in Extract 3, l.21--22 (“the youngest player starts”), is interpreted as an interactive turn and answered by ELI, the following rule read by AMA l.23 (“we go clockwise”, not reproduced in the transcript) is not being answered at all; the other players are engaged in other, individual activities, looking at their cards and arranging them in their hands. AMA’s reading has been considered as reviewing the game rules on her own.

The analyses have also oriented to the importance of considering interaction as a \textit{multimodal} achievement. \textit{Multimodality} here includes the elements -- other than speech -- produced during the turns or during a silence: gaze, gesture, non-vocal actions, but also elements related to speech such as prosody or voice quality. When \citet[16]{BecknerEtAl2009} argue that “multiple interacting elements, […] may amplify and/or compete with one another’s effects”, interactional analysis can show not only how this is carried out in concrete interactions, but also how it is achieved collaboratively. When AMA, in Extract 2, begins to read aloud the rules of the game, her reading is aligned with her gaze on the rule card and with her posture, holding the card in her hand: different resources are used to construct the turn as “reading aloud”. In extract 4 (l.35 “do you see some which are identical”), different resources design the turn as particular action addressed to one particular participant (and therefore construct an ephemeral group): while NOE simultaneously to his question separates the cards in LEA’s hands, LEA alternates her gaze between her and NOE’s cards. Gesture and gaze specify NOE’s turn as a question addressed to LEA which implies a search that LEA accomplishes. NOE’s turn takes into account LEA’s actions and LEA responds to NOE, even if her responsive actions are not verbal ones.

Whereas in these examples the different resources specify or amplify each other, other examples show that they can also be used competitively, for instance to interact simultaneously in different participation frameworks: in Extract 1, NOE points his arm towards ELI and produces a verbal turn, advising her not to show her cards (and forming an ephemeral group NOE\slash ELI). His gaze remains focused on his cards (and maintains the participation framework he forms with his team member LEA). In a similar way, in Extract 4, NOE advises ELI in a verbal turn not to show her cards (forming an ephemeral group NOE\slash ELI), but at the same time he gazes at his cards and continues collecting all the cards in his hand (acting within the NOE\slash LEA team). Such examples show not only the importance of considering language as one resource among others, but also the complex ways in which these resources work together, particularly in terms of sequence organisation (overlapping, non-linearity, local vs. global organisation, emergence of ephemeral groups, etc.). 

\section{Conclusion}\label{sec:13:5}
\begin{sloppypar}
While \citetv{chapters/12} draw on conversational routines and multimodal \textit{gestalts} to illustrate the complex nature of interaction through a CA analysis, we discussed the notion of complexity in interaction with regard to multimodal practices of constructing a dynamic and fluctuating participation framework. Our paper showed that the concept of \textit{complex adaptive system} and the arguments developed by \citet{BecknerEtAl2009} can be useful for describing the complexity of interaction and that a conversation analytic approach can complete and specify these arguments.
\end{sloppypar}

Our analysis has focused on the construction of participation frameworks: whereas participants orient to \textit{gaming} as an overall structuring activity, they interact in different overlapping sequences, accomplish individual actions, follow different action trajectories and form different ephemeral groups. They hereby point to interaction as non-linear and turns as emergent though well organised.

Our study is based on a sequential and multimodal analysis and considers language as an important but not the only resource for interaction. It therefore aims to highlight how the notions of sequentiality, temporality and multimodality can be used to complete the CAS approach. By treating turn and sequence production as inherently emergent and as a joint achievement in interaction, conversation analysis adds another perspective to an emic approach of complexity. 

\section*{Acknowledgements}
The authors are grateful to the ASLAN project (ANR-10-LABX-0081) of the Université de Lyon for its financial support within the French program “Investments for the Future” operated by the National Research Agency (ANR).

\section*{Transcription conventions}

\begin{tabular}{@{}>{\ttfamily}ll@{}}
[oui]    &  Overlapping talk                                       \\
/   \    &  Rising or falling intonation of the prior segment    \\
°oui°    &  Lower voice                                            \\
:        &  Prolongation of the prior sound                       \\
p`tit    &  Elision                                                \\
trouv-   &  Truncation of a word                                   \\
xxx      &  Incomprehensible syllable                              \\
=        &  Latching, turn continues on a new line                 \\
( )    &  Uncertain transcription                      \\
((laughter))  &  Comment, transcriber's description          \\
\&       &  Turn of the same speaker interrupted by an overlap     \\
(.)      &  Micro-pause ($<0.2\text{s}$)                                 \\
(2.6)    &  Timed pause Timed pause in seconds and tenths of second\\
*- - ->12    & gesture continues until line 12   \\
*- - ->{}>   & gesture continues after the excerpt's end   \\
££        &  delimit LEA’s gestures and gazes                     \\
**        &  delimit ELI’s gestures and gazes                  \\
\$\$      &  delimit AMA’s gestures and gazes                   \\
@@        &  delimit NOE’s gestures and gazes                   \\
\#       & indicates the exact point where a screen shot (image) has been \\ 
         &  taken within a turn or a time measure \\
\end{tabular}

{\sloppy\printbibliography[heading=subbibliography,notkeyword=this]}
\end{document}
