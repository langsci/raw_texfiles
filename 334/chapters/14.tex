\documentclass[output=paper]{langscibook} 
\ChapterDOI{10.5281/zenodo.6620131}
\author{Peter Griggs\affiliation{UMR 5161 ICAR; Centre National de la Recherche Scientifique; Ecole Normale Supérieure; Université Lumière Lyon 2}
        and Nathalie Blanc\affiliation{UMR 5161 ICAR; Centre National de la Recherche Scientifique; Ecole Normale Supérieure; Université Lumière Lyon 2; Université Claud Bernard Lyon 1}}
\title[Second language use and development in an immersion class]
      {Second language use and development in an immersion class considered as a complex adaptive process}
\abstract{We used a mixed methods approach in this chapter to show the strong relation between the development of second language competence and how language is (re)used and analysed within various discourse settings. Building on theory that explains learning as a progression from the social and interactive planes to the cognitive, the term “proceduralisation” is proposed to denote the gradual development of language learning as a cyclical process. Results show that the trajectories of language use and learning are closely intertwined. On the one hand, language constructions are consolidated and generalised through their use in different contexts. On the other hand, the regulation and structuring of language use is mediated by the multimodal resources of the classroom.}

\epigram{\textit{We fondly remember Peter Griggs who passed as we edit this manuscript. Peter never counted the time he spent with his students and he was devoted to helping teachers build a strong, research-based practice.}}

\IfFileExists{../localcommands.tex}{
  \addbibresource{localbibliography.bib}
  \usepackage{langsci-optional}
\usepackage{langsci-gb4e}
\usepackage{langsci-lgr}

\usepackage{listings}
\lstset{basicstyle=\ttfamily,tabsize=2,breaklines=true}

%added by author
% \usepackage{tipa}
\usepackage{multirow}
\graphicspath{{figures/}}
\usepackage{langsci-branding}

  
\newcommand{\sent}{\enumsentence}
\newcommand{\sents}{\eenumsentence}
\let\citeasnoun\citet

\renewcommand{\lsCoverTitleFont}[1]{\sffamily\addfontfeatures{Scale=MatchUppercase}\fontsize{44pt}{16mm}\selectfont #1}
   
  %% hyphenation points for line breaks
%% Normally, automatic hyphenation in LaTeX is very good
%% If a word is mis-hyphenated, add it to this file
%%
%% add information to TeX file before \begin{document} with:
%% %% hyphenation points for line breaks
%% Normally, automatic hyphenation in LaTeX is very good
%% If a word is mis-hyphenated, add it to this file
%%
%% add information to TeX file before \begin{document} with:
%% %% hyphenation points for line breaks
%% Normally, automatic hyphenation in LaTeX is very good
%% If a word is mis-hyphenated, add it to this file
%%
%% add information to TeX file before \begin{document} with:
%% \include{localhyphenation}
\hyphenation{
affri-ca-te
affri-ca-tes
an-no-tated
com-ple-ments
com-po-si-tio-na-li-ty
non-com-po-si-tio-na-li-ty
Gon-zá-lez
out-side
Ri-chárd
se-man-tics
STREU-SLE
Tie-de-mann
}
\hyphenation{
affri-ca-te
affri-ca-tes
an-no-tated
com-ple-ments
com-po-si-tio-na-li-ty
non-com-po-si-tio-na-li-ty
Gon-zá-lez
out-side
Ri-chárd
se-man-tics
STREU-SLE
Tie-de-mann
}
\hyphenation{
affri-ca-te
affri-ca-tes
an-no-tated
com-ple-ments
com-po-si-tio-na-li-ty
non-com-po-si-tio-na-li-ty
Gon-zá-lez
out-side
Ri-chárd
se-man-tics
STREU-SLE
Tie-de-mann
} 
  \togglepaper[1]%%chapternumber
}{}

\begin{document}
\maketitle 

\section{Introduction}

This study explores second language acquisition in a fourth year science class in a French immersion primary school in Minnesota composed of pupils (8--9 years old) from principally monolingual American families. Recent research in this field \citep{Carol2005,CoyleEtAl2010,Dalton-Puffer2007,Lyster2007,SerraSteffen2010} emphasises the importance of relating the development of language competence to the way in which language is used, analysed and recycled in different discourse settings in the course of content learning. The usage-based perspective adopted in the present study regards language not as a static system composed of top-down rules and principles but as a complex adaptive system emerging bottom-up from the interplay between multiple components of the discourse context \citep{EllisLarsen-Freeman2009}.

\section{Usage-based and complexity theories of second language learning} %2. /
\begin{sloppypar}
Usage-based theories consider language to be a dynamic set of patterns (or “chunks”, “constructions”) emerging from use, those that saliently, frequently and reliably occur stabilising over time in a complex system \citep{Ellis2001,MacWhinney1987,Tomasello2003}. Complexity theory, as it is presented by \citet{EllisLarsen-Freeman2009}, focuses more particularly on the social and interactive dimensions of usage-based theory. Language development is considered to be a co-adaptive and iterative process whereby language resources are dynamically altered through learners interacting with one another and repeatedly revisiting similar discourse domains. Drawing on the work of \citet{Anderson1996}, \citet{GriggsEtAl2002} and \citet{Griggs2007}, we propose to use the term “proceduralisation” to denote the gradual development of second language competence through its use as a communicative tool in different discourse contexts. Proceduralisation is considered to be based on such processes as generalisation, discrimination, consolidation, and automatisation of language knowledge and the tuning of this knowledge to the norms of the target language.
\end{sloppypar}
\section{A praxeological and multimodal context of second language use} %3. /

Such a perspective needs also to include the whole ecological dimension of the classroom context by integrating into the analysis of language production the praxeological and multimodal aspects of classroom interaction. Much research has shown how joint social actions are built and interpreted on the basis of the simultaneous use of a heterogeneous collection of semiotic resources (eg. \citealt{Goodwin1981,StreeckEtAl2011,Kress2001,Kress2019}). Other researchers working in a sociocultural approach \citep{Lantolf2011,McCafferty2002,NegruelaLantolf2008}, have developed on the Vygotskyan notion of “internalisation” \citep{Galperin1967} to explain learning processes in terms of a progression from a social, interactive plane to a cognitive plane – the multimodal resources available in the learning context serving as mediation tools to control and regulate cognitive functioning.

\section{Aims of the study and methods of analysis} %4. /

The objective of this study is to explore the interplay between the discourse features of the pedagogical tasks, the verbal output of the learners and their exploitation of multimodal resources and to envisage these factors as a dynamic source of proceduralisation and internalisation of second language skills. The method used consists essentially in relating a finely grained analysis of multimodal interactions occurring locally to a more global analysis of the way emerging phenomena evolve on a larger temporal scale. We consider that articulating these two levels of activity should allow us to comprehend more clearly the links between patterns of verbal interaction in this classroom setting and the development of second language competence.

The corpus is based on a video recording of the last two lessons (about one and a half hours) of an eight-hour teaching module.\footnote{For more methodological details see \citet{BlancGriggs2015}.} One camera recorded without interruption the different phases of classroom activity, focusing primarily on a group of three pupils (S, G and A). 

An initial transcription was typed into the software package TRANSANA\footnote{A software package for analysing video data developed by the Wisconsin Centre for Educational Research University of Madison (EU): \href{http://www.transana.org}{www.transana.org}.} and precoded according to the criterion of “discourse genre” which, in this classroom setting, corresponds to Dalton Puffer’s definition (\citeyear{Dalton-Puffer2007}: 40) of a “sequence of discourse defined by a predominant communicative objective”.

We defined five categories:

\begin{enumerate}
\item \textit{Summary:} describe, discuss or evaluate previous activities;
\item \textit{Instruction:} give, reformulate or discuss instructions;
\item \textit{Explanation:} explain, or clarify knowledge, actions or diagrams;
\item \textit{Commentary:} describe or justify actions in process;
\item \textit{Negotiation:} discuss and argue proposals and decisions.
\end{enumerate}

Using these criteria, we first identified interactive sequences representing important phases in the evolution of the scientific project. A qualitative study of these sequences was carried out on the basis of a more detailed transcription combining verbal and non-verbal signs and drawing on methods of conversational analysis. Quantitative analyses were then done on a more global scale, concerning the frequency and the evolution of the phenomena that occur.

\section{Qualitative analysis of the group production of a diagram} %5. /

The lessons are composed of a series of intermediary tasks, performed in semi-autonomous groups and scaffolded by teacher interventions, instructions, work cards and collective debriefing sessions, involving the production of a diagram and a model of an electric circuit. These tasks are at the service of a pedagogical objective which is to construct scientific knowledge and, incidentally, second language competence. In this first extract, the task in which a group of three students is engaged consists in drawing collectively the diagram of an electric circuit from rough drafts that they have produced individually. 



\begin{longtable}{llp{.66\textwidth}}
\caption{Extract 1 (see page~\pageref{14:conventions} for a list of transcription conventions)\label{tab:14:1}}\\
\lsptoprule {Turn}  & {Speaker} & {Verbal} {and} {non} {verbal} {productions}\\\midrule\endfirsthead
\midrule {Turn}  & {Speaker} & {Verbal} {and} {non} {verbal} {productions}\\\midrule\endhead
\endfoot\lspbottomrule\endlastfoot
{\bfseries \textmd{1}} & {\bfseries \textmd{G}} & Premier\slash ou la liste des mots \textbf{{\textbackslash}}\\
{\bfseries \textmd{2}} & {\bfseries \textmd{S}} & \textbf{on} \textbf{va} \textbf{fai::re} 

☻le dessin en premier 0 

☻\textbf{\textit{deictic} \textbf{gesture}} \textit{: S points at G’s drawing with her pencil} 

\textit{G sits up and positions herself in front of the sheet of paper} 

☻parce que on n’est pas 

☻\textit{S leans towards the sheet of paper} 

☻sûres

☻\textbf{\textit{beat} \textbf{gesture}}\textit{: S punctuates the word with her two hands} \\
{\bfseries \textmd{3}} & {\bfseries \textmd{S}} & \textit{S draws} 

maintenant:: (…) \\
{\bfseries \textmd{4}} & {\bfseries \textmd{G}} & ☻(……………………………………………)

☻ G \textit{moves her finger quickly towards the pencil} \\
{\bfseries \textmd{5}} & {\bfseries \textmd{S}} & oui \\
{\bfseries \textmd{6}} & {\bfseries \textmd{G}} & \textit{G takes the pencil}

[est-ce qu’on va::\\
{\bfseries \textmd{7}} & {\bfseries \textmd{S}} & [maintenant \textbf{on} \textbf{a} besoin 

☻d’une ampoule 

☻\textbf{\textit{deictic} \textbf{gesture}}\textit{: S points her finger towards the paper}\\
{\bfseries \textmd{8}} & {\bfseries \textmd{G}} & \textit{G starts to draw}

attends \textbf{je} \textbf{peux} \textbf{pas} (…)

\textit{G takes the sheet on her knees and continues to draw} 

\textit{S snaps her fingerss}

☺et on besoin des euh

☺ \textit{S takes the sheet and the pencil}

☻piles

☻\textbf{\textit{beat} \textbf{gesture}}: \textit{G punctuates the word with her finger}\\
{\bfseries \textmd{9}} & {\bfseries \textmd{S}} & et \textbf{on} \textbf{va} \textbf{connecter} à la base \textbf{{\textbackslash}}  

☻maintenant on a besoin d’un::e  

☻\textit{S leans over and draws}

☻et maintenan\textbf{t} \textbf{on} \textbf{va} \textbf{connecter} 0 \textbf{je} \textbf{vais} \textbf{faire} \textbf{deux} \textbf{lignes}

☻\textit{S traces a first line then a second}\\
{\bfseries \textmd{10}} & {\bfseries \textmd{G}} & \textit{G takes the pen from S}\\

{\bfseries \textmd{11}} & {\bfseries \textmd{S}} & \textit{with her right hand S traces 2 other connections to make in the diagram} \\
& {\bfseries \textmd{G}} & \textit{G takes the paper from her}

☻donc ça à ça

☻\textit{she redraws the line with the pencil}\\
{\bfseries \textmd{13}} & {\bfseries \textmd{S}} & ☻et 0 ça à ça et ça à ça

☻\textbf{\textit{deictic} \textbf{gesture}}: \textit{S points her finger at the connections to make in the diagram} \\
{\bfseries \textmd{14}} & {\bfseries \textmd{G}} & \textit{G draws the lines in the diagram}\\

{\bfseries \textmd{15}} & {\bfseries \textmd{A}} & Je pensais qu’on a deu(x) ampoules ou deux ampoules\\
{\bfseries \textmd{16}} & {\bfseries \textmd{G}} & \textit{G stops drawing}

\textit{G raises her head and looks at} S

\textit{S raises her head at the same time} \\
{\bfseries \textmd{17}} & {\bfseries \textmd{S}} & ☻\textbf{on} \textbf{peut} \textbf{avoir} \textbf{deux} \textbf{ampoules} 

☻\textit{S shakes her head}\\
{\bfseries \textmd{18}} & {\bfseries \textmd{G}} & ☻\textbf{tu} \textbf{veux} \textbf{deux} \textbf{ampoules} 

☻\textit{G looks at S}\\
{\bfseries \textmd{19}} & {\bfseries \textmd{S}} & ☻\textbf{on} \textbf{va} \textbf{tester} \textbf{avec} \textbf{une} \textbf{ampoule} 

☻ \textbf{\textit{deictic} \textbf{gesture:}} \textit{S shakes her finger towards a bulb}

 et après c’est::\\
{\bfseries \textmd{20}} & {\bfseries \textmd{G}} & ☻\textbf{on} \textbf{peut} \textbf{ajouter} \textbf{une} \textbf{ampoule} \textbf{ici}

☻\textit{G looks at S and}\textbf{ \textbf{(deictic} \textbf{gesture}}\textit{) points the pencil at her paper}\\
{\bfseries \textmd{21}} & {\bfseries \textmd{S}} & ☻d’accord  

☻ \textit{S nods in agreement}\\
\end{longtable}

In this joint drawing activity, language has essentially an interactive function of task management. G and S take turns drawing the diagram, passing the pencil and paper from one to another, while A looks on. Speech production during the drawing task allows the pupils to make and justify decisions about the activity in progress and to confirm mutual understanding. It also serves to define roles, coordinate individual participation in the joint activity, consolidate links between different levels of the task and negotiate decision-making. Our analysis shows that the pupils perform these diverse interactive functions smoothly and effectively on the basis of the reiterative use of a set of 9 analogous language constructions (underlined in the transcription).

\section{Quantitative analysis of recurrent construction} %6. /

Following our analysis of this and other extracts, we elaborated from the constructions isolated in the extract a more generalised, prototypical construction which we fed into the textometric software package TXM\footnote{Developed by the research laboratory UMR 5191 ICAR, cf. presentation: \url{http://textometrie.ens-lyon.fr/spip.php?rubrique96}.} in order to investigate its frequency:

\ea PRONOUN+( )+\textit{vouloir/pouvoir/devoir/aller}+( )+ VERB/NOUN\z

The method adopted involved, first of all, carrying out a quantitative analysis of the frequency with which the construction was used by the teacher and the pupils, and then charting the trajectory of use of the construction in the two lessons and the way it was recycled in the five pre-classed discourse genres (see Tables~\ref{tab:14:1} and~\ref{tab:14:2}).

\begin{table}
\caption{Frequency of construction in teacher and pupil ouput}
\label{tab:14:2}

\begin{tabular}{l rrrr} 
\lsptoprule
& {$n$ words} & {$n$ construction}  & \multicolumn{2}{c}{Words in constr.}\\\cmidrule(lr){4-5}
&                   &                             & $n$ & \%\\\midrule
{Teacher ouput} & 3061 & 102 & 398 & 13.0\\
{Native assistant output} & 84 & 1 & 3 & 3.6\\
{Pupil output} & 3105 & 111 & 446 & 14.4\\
{Whole corpus} & 6250 & 214 & 847 & 13.6\\
\lspbottomrule
\end{tabular}
\end{table}

The first column in \tabref{tab:14:1} gives the total number of words used in the whole corpus and their distribution into three categories – teacher output, pupil output and native assistant output – and shows approximately the same amount of teacher and pupil output during the three lessons.

The second column concerns the target construction, showing that it is used with a similar frequency by the teacher and by the pupils.

The third and fourth columns calculate the frequency of the construction in terms of the number of words used and in terms of a percentage of the total number of words used in the whole corpus. We postulate that these percentages (teacher 13\%; pupils 14.4\%; corpus 13.6\%) are very high. The only measure of comparison we could carry out using TXM in order to verify this hypothesis was that of the frequency of the same construction in the corpus of a native language science class in a secondary school in France. This calculation revealed a rate of frequency of 0.09\%.

Another interesting result was that the rate of use of the construction was about the same for the teacher as for the pupils. 

\tabref{tab:14:2} shows the distribution of the same structure in the different discourse genres.

\begin{table}
\caption{Frequency of the construction according to discourse genre.\label{tab:14:3}} 
\begin{tabular}{l rrrr} 
\lsptoprule
& {$n$ words} & {$n$ construction}  & \multicolumn{2}{c}{Words in constr.}\\\cmidrule(lr){4-5}
&                   &                             & $n$ & \%\\\midrule
{Summary} & 1146 & 28 & 104 & 9.0\\
{Instruction} & 1596 & 62 & 242 & 15.2\\
{Explanation} & 1079 & 37 & 161 & 14.9\\
{Commentary} & 1029 & 24 & 80 & 7.8\\
{Negotiation} & 1400 & 63 & 260 & 18.6\\\midrule 
{Total} & 6250 & 214 & 847 & 13.6\\
\lspbottomrule
\end{tabular}
\end{table}


The results show that the frequency of use is relatively high in all five discourse genres, varying nevertheless between 18.6\% for the “negotiation” category and 7.8\% and 9\% respectively for “commentary” and “summary”. The variance in rate of frequency between genre categories can be explained by the impact of the respective form-function patterns specific to the discourse of each category. 


This analysis indicates, first of all, that there is a tendency for the pupils in this semi-autonomous immersive context to rely on a limited nucleus of language constructions in order to produce the speech acts necessary to carry out the scientific project. The quantitative analysis also reveals that the rate of use of the constructions by the teacher is similar to that of the pupils. The fact that the teacher functions on the basis of a reduced linguistic repertoire underlines the weight of the impact of the institutional context and the pedagogical format on teacher-pupil and pupil-pupil interactions. Finally, the results show that the high rate of frequency of the construction applies across the different genres, despite the effect of the specificity of the discourse related to each genre. The recycling and manipulation of the construction in different discourse contexts is thus optimised, which according to the usage-based perspective presented in the first part of this study should favour the processes of proceduralisation.

\section{Regulation of language output through co-verbal gesture} %7. /

In the second part of this study, which integrates a multimodal dimension into the analysis of classroom interaction, we envisage internalisation as a parallel and analogical process which interacts with proceduralisation in the development of second language competence. In the following analysis, we relate co-verbal gestures to the discourse context in which they occur. 

\subsection{Deictic gestures in commentary contexts} %7.1. /

If we turn back to Extract 1, we can note that most of the gestures in this sequence are of a deictic nature \citep{McNeill1992}, which can be explained by the fact that the interaction is geared towards the manipulation of objects in order to produce a diagram. Finger gestures reinforce speech acts by pointing at objects (e.g. T2: “on va faire le dessin en premier”) or trace processes represented in the diagram (eg. T9 “on va connecter à la base”) or complement speech acts by accompanying deictic expressions (eg. T13 “ça à ça et ça à ça”). In the corpus as a whole a large proportion of the gestures (68 out of 128) can be categorised as deictic, and this is particularly true in the commentary and explanation discourse contexts where the pupils are working autonomously in groups in the presence of pedagogical artefacts. Of the 68 deictic gestures observed in the corpus, 18 coincide with deictic expressions and 40 accompany specialised vocabulary introduced previously during the teaching module. We consider therefore that these deictic gestures have a dual role in this immersive context of both reinforcing the coordination of the joint activity and scaffolding second language production.

\subsection{Beat, iconic and metaphoric gestures in summary contexts} %7.2. /

During summary phases in which pupils report back on the scientific tasks that they have carried out in groups, it is essentially iconic, metaphoric and beat gestures that accompany their second language discourse. The absence of deictic gesture can be explained by the simple fact that during these phases the pupils’ attention is not focused on the manipulation of objects. 

The two following examples illustrate how gesture is regularly used by pupils to facilitate their production of lexical items that are in the process of proceduralisation. 

In Extract 2 (\tabref{tab:14:4}), which takes place at the beginning of the recording, the class is summing up collectively the content of the previous lesson.


\begin{table}
\caption{Extract 2\label{tab:14:4}}

\begin{tabularx}{\textwidth}{llQ}

\lsptoprule
{Turn}  & {Speaker} & {Verbal} {and} {non} {verbal} {productions}\\
\midrule 
1 & T & Oui tu peux ajouter G\\
2 & G & On a parlé de le processus d’ingénierie et on a créé

☻un petit 0

\textbf{\textit{☻iconic} \textbf{gesture}}

☻chose de balance 

\textbf{\textit{☻iconic} \textbf{gesture} \textbf{(repeated)}}

et 0 ça c’est tout\\
3 & T & alors vous avez créé la balance et hier vous avez commencé à écrire à dessiner  \\
\lspbottomrule
\end{tabularx}

\end{table}

G’s use of  \textit{chose de} to qualify ‘balance’ is a way of objectifying for better control a word which remains relatively unfamiliar to her. Her difficulty in retrieving the lexical item is indicated by the fact that her iconic gesture, in the form of a pivoting motion of her two hands, precedes its production and is then repeated after a slight pause at the same time as the word is pronounced. In the corpus as a whole, iconic gestures often precede pupils’ productions of difficult words.

In another example occurring during a group monitoring phase (\tabref{ex:14:5}), the lexical search takes on a more interactive dimension.  



\begin{table}
\caption{Extract 3}
\label{ex:14:5}
\begin{tabularx}{\textwidth}{llQ}

\lsptoprule
{Turn}  & {Speaker} & {Verbal} {and} {non} {verbal} {productions}\\
\midrule 
1 & G & On juste on fait fait

☻le dessin

\textbf{\textit{☻beat} \textbf{gesture}}

maintenant on marque les :

\textit{turns her head towards the teacher}

☻mince

☻\textbf{\textit{metaphoric} \textbf{gesture}} + \textit{turns her head towards the drawing}\\
2 & T & matériels\\
3 & G & On a 

☻combiné 

\textbf{\textit{☻iconic} \textbf{gesture}}

☻eh :

\textbf{\textit{☻iconic} \textbf{gesture}}\\
4 & T & ok\\
\lspbottomrule
\end{tabularx}

\end{table}

First, a beat gesture, accompanying the word \textit{dessin} after the repetition of \textit{fait} indicates a certain cognitive effort in producing the targeted lexical item. Then G solicits the help of the teacher in order to find the word \textit{matériels} with the utterance of \textit{mince} accompanied by paralinguistic and kinetic markers. Next, G executes a clear iconic gesture to scaffold the production of the accompanying word \textit{combiné}. Finally, she sketches another vaguer gesture seeming to announce the end of the sentence which she is manifestly unable to complete. This time the teacher does not offer help but simply closes the sequence with a ratifying \textit{ok}.

Of the 60 examples of beat, iconic and metaphorical gesture in the corpus, 28 occur during summary phases of which 22 are linked to specialised vocabulary targeted in the science project.

\subsection{Beat, iconic and metaphoric gestures in negotiation contexts}

Of the remaining 32 examples of beat, iconic and metaphorical gestures, 22 occur in negotiation phases in which students argue and discuss aspects of the task in progress. In the following sequence (\tabref{tab:14:6}), G and S discuss the functioning of the electric circuit and, more specifically, the capacity of the black rock in their bag of materials to conduct an electric current. 

\begin{longtable}{llp{.66\textwidth}}
\caption{Extract 4\label{tab:14:6}}\\
\lsptoprule {Turn}  & {Speaker} & {Verbal} {and} {non} {verbal} {productions}\\\midrule\endfirsthead
\midrule  {Turn}  & {Speaker} & {Verbal} {and} {non} {verbal} {productions}\\\midrule\endhead
\endfoot\lspbottomrule\endlastfoot
1 & S & et on est CERTAINES\\
2 & G & oui je suis certaine parce que 0

☻j’ai utilisé ça

\textbf{\textit{☻deictic} \textbf{gesture}}

une fois et

☻l’ampoule a s’allumé

\textbf{\textit{☻iconic} \textbf{gesture}}

☻donc ça a du fer dedans

\textbf{\textit{☻} \textbf{beat} \textbf{gesture}}\\
3 & S & mais on ne va pas avoir

☻la MÊME SAC que:  0 

\textbf{\textit{☻beat} \textbf{gesture}}\\
4 & G & oui je sais mais le roche noir 0 c’est toujours\\
5 & S & oui pas

☻TOUJOURS

\textbf{\textit{☻beat} \textbf{gesture}}

mais si on est 

☻CERTAINES

\textbf{\textit{☻beat} \textbf{gesture}}

 que il y a du fer dedans

☻ce ROCHE

\textbf{\textit{☻deictic} \textbf{gesture}}\\
7 & G & Oui Mademoiselle a dit\\
8 & S & ☻donc tous les roches

\textbf{\textit{☻deictic} \textbf{gesture}}\\
9 & G & pas tout 0 juste 

☻UN

\textbf{\textit{☻iconic} \textbf{gesture}}

des roches a du fer et 

☻les autres n’ont pas

\textbf{\textit{☻metaphorical} \textbf{gesture}}\\
11 & S & Mais on ne sait pas QUEL\\
12 & G & oui c’est différent

☻tu peux voir

☻\textbf{\textit{metaphorical} \textbf{gesture}}

comme c’est pas pas 0 c’est

☻le seul roche

\textbf{\textit{☻beat} \textbf{gesture}}

qui est 

☻noir dedans

\textbf{\textit{☻beat} \textbf{gesture}}\\
\end{longtable}

Despite the relative simplicity of the language they use, the two pupils manipulate effectively causal and logical connectors (e.g. \textit{parce que}, \textit{comme}, \textit{donc}, \textit{mais}) and modal markers (e.g. \textit{certaine}, \textit{toujours}) to structure their discourse and use stress to highlight strategically  important words (eg. \textit{MÊME SAC}, \textit{ROCHE}, \textit{UN}, \textit{QUEL}) in order to give more force to their arguments. In this context, different gestures function, therefore, primarily on a discourse level, serving to punctuate and thus reinforce articulations within the argument structure and to draw attention to operative words. The tendency for gesture to be used in these negotiation phases more as a means of structuring discourse than of facilitating lexical retrieval is indicated by the fact that in these contexts the majority of gestures (14 out of 22) focus on language items other than the potentially difficult targeted specialised vocabulary.

\begin{sloppypar}
This analysis of co-verbal gesture illustrates the different ways in which multimodal resources fulfill the function of regulating and structuring language production in these three recurrent types of discourse context. In the first context (commentary, explanation) involving the performance of concrete tasks in groups, deictic gestures combine with pedagogical artefacts both to coordinate the joint activity and to scaffold second language production. In the second context (summary), in which pupils report back to the whole class on the work they have accomplished, without having recourse to pedagogical artefacts, iconic gestures compensate for the lack of concrete objects in the immediate discourse environment, replacing deictic gesture by evoking a physical representation of the lexical production they are scaffolding. In the third context (negotiation), a more diversified repertoire of gestures (beat, deictic, iconic, metaphorical) is used to structure language production on a higher discourse level. Such modulations between more or less concrete and abstract planes of second language activity can be theorised, we believe, in terms of Gal’perin’s notion of internalisation, with multimodal resources contributing to this process in different forms and to variable degrees. However, while \citet{Galperin1967} envisages internalisation as a gradual progression from a concrete to an abstract level of cognitive functioning, it presents itself in this institutional immersive setting rather as a cyclical process, structured according to the nature of the speech act and the discourse context in which it is performed in the course of the execution of the scientific project.\end{sloppypar}

\section{Conclusion} %8. /

This study has shown how a task-based approach integrating language and content learning in an immersion classroom setting creates conditions favouring second language development as it is envisaged in a complexity theory of learning. According to this perspective, language use and language learning are considered to be interlinked, with language structure based on form-meaning mappings emerging in a multimodal context from an interplay between patterns of social interaction and cognitive mechanisms. Adopting the notions of proceduralisation and internalisation, we have theorised and explored the trajectory of language use and development in terms of two parallel and interrelated processes: the consolidation and generalisation of language constructions through their recycling in different discourse contexts; the regulation and structuring of language use, at varying levels of abstraction, mediated by the multimodal resources of the immersion classroom setting.  

\section*{Acknowledgements}
The authors are grateful to the ASLAN project (ANR-10-LABX-0081) of the Université de Lyon for its financial support within the French program “Investments for the Future” operated by the National Research Agency (ANR).

\section*{Transcription conventions}\label{14:conventions}

\begin{tabularx}{\textwidth}{@{}Ql@{}}
1.&turn\\
0&short pause\\
00&long pause\\\relax
[  ]&overlap\\
/ {\textbackslash}&rising/falling intonation\\
(…)&inaudible passage\\
TOUJOURS&stressed word\\
et: &lengthened syllable\\
☻ deictic gesture  & coverbal kinetic or paralinguistic behaviour\\
☺ turns her head towards the teacher & autonomous kinetic or paralinguistic behaviour\\
\end{tabularx}

{\sloppy\printbibliography[heading=subbibliography,notkeyword=this]}
\end{document}
