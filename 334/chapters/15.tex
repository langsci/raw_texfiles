\documentclass[output=paper]{langscibook} 
\ChapterDOI{10.5281/zenodo.6620135}
\author{Kristine Lund\affiliation{UMR 5161 ICAR; Centre National de la Recherche Scientifique; Ecole Normale Supérieure; Université Lumière Lyon 2}
        and Pierluigi {Basso Fossali}\affiliation{UMR 5161 ICAR; Centre National de la Recherche Scientifique; Ecole Normale Supérieure; Université Lumière Lyon 2}
        and Audrey Mazur\affiliation{UMR 5161 ICAR; Centre National de la Recherche Scientifique; Ecole Normale Supérieure; Université Lumière Lyon 2}
        and Magali Ollagnier-Beldame\affiliation{UMR 5161 ICAR; Centre National de la Recherche Scientifique; Ecole Normale Supérieure; Université Lumière Lyon 2}}
\title[Multiple vantage points]
      {Considering the complex adaptive system from multiple vantage points}
\abstract{In this final chapter, our title plays on how vantage points highlight different visions of what occurs in a complex adaptive system. We look back on the ways in which this book gives credence to the view that language is such a system. We zoom out in order to situate our work in relation to a selection of focused work on complexity. We zoom in for specific results, taking each chapter in turn. We then zoom out again in order to address the current societal impacts of our work. Next, we summarise our contributions and, finally we turn to future research avenues.}
\IfFileExists{../localcommands.tex}{
  \addbibresource{localbibliography.bib}
  \usepackage{langsci-optional}
\usepackage{langsci-gb4e}
\usepackage{langsci-lgr}

\usepackage{listings}
\lstset{basicstyle=\ttfamily,tabsize=2,breaklines=true}

%added by author
% \usepackage{tipa}
\usepackage{multirow}
\graphicspath{{figures/}}
\usepackage{langsci-branding}

  
\newcommand{\sent}{\enumsentence}
\newcommand{\sents}{\eenumsentence}
\let\citeasnoun\citet

\renewcommand{\lsCoverTitleFont}[1]{\sffamily\addfontfeatures{Scale=MatchUppercase}\fontsize{44pt}{16mm}\selectfont #1}
   
  %% hyphenation points for line breaks
%% Normally, automatic hyphenation in LaTeX is very good
%% If a word is mis-hyphenated, add it to this file
%%
%% add information to TeX file before \begin{document} with:
%% %% hyphenation points for line breaks
%% Normally, automatic hyphenation in LaTeX is very good
%% If a word is mis-hyphenated, add it to this file
%%
%% add information to TeX file before \begin{document} with:
%% %% hyphenation points for line breaks
%% Normally, automatic hyphenation in LaTeX is very good
%% If a word is mis-hyphenated, add it to this file
%%
%% add information to TeX file before \begin{document} with:
%% \include{localhyphenation}
\hyphenation{
affri-ca-te
affri-ca-tes
an-no-tated
com-ple-ments
com-po-si-tio-na-li-ty
non-com-po-si-tio-na-li-ty
Gon-zá-lez
out-side
Ri-chárd
se-man-tics
STREU-SLE
Tie-de-mann
}
\hyphenation{
affri-ca-te
affri-ca-tes
an-no-tated
com-ple-ments
com-po-si-tio-na-li-ty
non-com-po-si-tio-na-li-ty
Gon-zá-lez
out-side
Ri-chárd
se-man-tics
STREU-SLE
Tie-de-mann
}
\hyphenation{
affri-ca-te
affri-ca-tes
an-no-tated
com-ple-ments
com-po-si-tio-na-li-ty
non-com-po-si-tio-na-li-ty
Gon-zá-lez
out-side
Ri-chárd
se-man-tics
STREU-SLE
Tie-de-mann
} 
  \togglepaper[1]%%chapternumber
}{}

\begin{document}
\lehead{Kristine Lund et al.}
\maketitle 

%\section{Introduction}

%In this final chapter, we look back on the ways in which this book gives credence to the view that language is a complex adaptive system. However, our gaze is also positioned from additional vantage points. We first zoom out in order to situate our work in relation to a selection of focused work on language complexity. We then zoom in for specific results, taking each chapter in turn. At this point, we zoom out again in order to address the various societal impacts of our work. Next, we summarise our contributions and, finally we turn to future research avenues.

\section{Complexity in language sciences: The current state of the art}

In addition to being analysed from within different disciplinary frameworks (biological, cognitive, social, cultural), language is fundamentally dynamic -- regardless of the angle of analysis -- and given this, lends itself quite well to conceptual constructs that are usually brought to bear in the science of complexity. In \citetitle{Kretzschmar2015}, \citet{Kretzschmar2015} mobilises constructs such as
1) continuing dynamic activity in the system,
2) random interaction of large numbers of components,
3) exchange of information with feedback,
4) reinforcement of behaviours, and
5) emergence of stable patterns without central control. \citet{Massip-Bonet2013} notes the following as characteristics of language being a complex adaptive system:
1) distributed control and collective emergence,
2) intrinsic diversity,
3) perpetual dynamics,
4) adaptation through factors of amplification and competition,
5) non-linearity and phase transitions, and 6) sensitive dependence on network structure.

While mobilising these concepts in some way, research on complexity and language has often taken approaches at the crossroads of linguistics and cognitive science where models are conceptualised to analyse different phenomena. The special issue of \textit{Language Learning} is a case in point \citep{BecknerEtAl2009}, and it is mainly in relation to this work that the current volume took position. \textit{Emergence} is a powerful construct and through this angle, papers in this special issue analysed emergent linguistic constructions during second language learning \citep{EllisLarsenFreeman2009}, emerging regularities during artificial language evolution \citep{CornishEtAl2009}, specific types of emergent grammar \citep{BecknerBybee2009}, language change that emerges from language behaviour \citep{BlytheCroft2009}, and the emergence of an individual’s personalised meaning potential \citep{Matthiessen2009}. \textit{Interplay} as a catalyst for change within a complex system is another angle of analysis, for example between language, agent, and environment in the language acquisition process \citep{Dörnyei2009}, the processes by which speakers acquire mappings between semantics and syntactic devices such as word order \citep{BoydEtAl2009}, or more broadly, between culture and biology, giving rise to language itself \citep{Schoenemann2009}.

Indeed, emergence is a strong and useful conceptual construct for applied linguistics \citep{EllisLarsen-Freeman2006}, but also interactional linguistics (e.g. \citealt{Mondada2009,DeppermannEtAl2010}), although in sociology, it has been used in contradictory ways \citep{KeithSawyer2001}. In the introduction to their special issue on \textit{Language emergence: Implications for applied linguistics}, \citet{EllisLarsen-Freeman2006} describe research in 12 areas of applied linguistics, each time gleaning what they call a moral to the story (or lesson learned) that is a characteristic of an emergent system. For example, if language acquisition and language representation is exemplar based (e.g. based on memories of previously experienced utterances), then the moral is that regularities are emergent, growth is non-linear, and cognition is adaptive.\largerpage

This volume has contributed in two ways to understanding language as a complex adaptive system. First, we have extended the field of application of some of the above conceptual constructs. This means that we have described new instantiations, for example of how language displays distributed control and collective emergence or how competition of factors contributes to the adaptation of language. Second, we have added a new conceptual construct that, to our knowledge, has not been named in the literature. In the following sections we give details on these two types of contributions. 

\section{Zooming in for specific results}\largerpage

This book has three main sections:
1) Epistemological views on complexity,
2) Complexity pragmatics and discourse, and
3) Complexity interaction and multimodality.
In what follows, each section and chapter is briefly discussed in relation to the conceptual constructs they give rise to, where these constructs find an echo in other literature on complexity.

\subsection{Epistemological views on complexity}

Epistemological debates show how assumptions collide. Some of the collisions between theories and frameworks in the language sciences can be reconciled (e.g.~accepting a way to frame cognition that is compatible with conversation analysis) and some seem incommensurable if one or the other must be chosen (e.g. internalising social norms versus co-construction of interaction as an explanation for linguistic phenomena). In \textit{Language and complex systems}, \citet{Kretzschmar2015} takes the recognition of emergence in language as a reason to put aside universal grammar in favour of a functionalist approach where “order in language emerges from the linguistic interactions of speakers, agents using speech” \citep[2]{Kretzschmar2015}. Here, emergence –~a fundamental construct in complex systems~– helps us to decide between competing theories that explain the origin of order in language.

\subsubsection{Semiotic mediations and complexity management: Paradoxes and regulative principles}

It is well known that people confuse the map with the territory. In other words, they mistake a model for reality. Given that each disciplinary framework pays attention to different aspects of a phenomenon, it follows that researchers will make a different map of the territory, or come up with a different model for reality. \textcitetv{chapters/02} proposes that superposing the maps or combining the models is a way to operationalise interdisciplinarity. That said, whether a researcher considers the system studied as open or closed depends on their epistemological position and therefore affects attempts at combining maps or combining models. One of the paradoxes he points out is that, given the perpetual evolution of culture, any attempt at meaning-making can never be finished. However, despite these difficulties, language as a system combined with talk-in-interaction succeeds in producing practices that indeed become norms.

\subsubsection{What knowledge owes to experience: Complexity and first-person epistemology}

In a third-person approach, the researcher relies on external observables that reflect the subject’s activity, such as digital traces of screen-based tasks, or recordings of pupil activity. In a second-person approach, the subject communicates his or her experience to the researcher by a variety of means, such as pressing a button in order to signal perception of a particular stimulus. \textcitetv{chapters/03} works on a first-person approach, based on micro-phenomenological interviews, the objective of which is to reintegrate the subjective experience into ways of knowing. Discourse analysis of such interviews allows for the discovery of subjective invariants that can then be compared to objective invariants from third-person data, leading to a more complete picture of the human experience.

\subsubsection{Modelling the co-elaboration of knowledge: Connecting cognitive, linguistic, social and interactional systems}

Any object of study is always approached from a particular vantage point, given the assumptions of the researcher, and the co-elaboration of knowledge is no different. \textcitetv{chapters/04} argues for an interdisciplinary model that proposes a system of interrelated subsystems where disciplinary views from language sciences, education, and psychology can be combined to describe and predict the co-elaboration of knowledge, especially regarding collective emergence and perpetual dynamics, both for individuals and groups. In addition, instances of proposed unidirectional causality and bi-directional causality illustrate how different vantage points may be connected through intermediate variables. Combining these varied views on the same object broadens understanding and increases explanatory power.

\subsubsection{Epistemological contributions}
\begin{sloppypar}
The three contributions in this section of the volume  (Basso Fossali, Ollagnier-Beldame, \& Lund) all plead for epistemological plurality. Basso Fossali proposes to use semiotic mediations in order to interconnect the fluctuating values within the differing domains under study and aim for translatability between frameworks. Ollagnier-Beldame argues for surpassing the dichotomy of objective versus subjective methodologies in cognitive sciences, proposing to connect invariants from first-person and third-person approaches to ways of knowing. Finally, Lund argues for using intermediate variables as a way to connect between particular vantage points that stem from different disciplinary frameworks and as a consequence, focus on different aspects of the co-elaboration of knowledge. Within a complexity science view of language sciences, these are all ways of increasing connections – be it between models, approaches, or vantage points – and thus all build a more comprehensive understanding of the phenomena under investigation. That being said, some frameworks are incommensurable for epistemological reasons, and this must be recognised as a limit to the goal of connecting across conceptual constructs.
\end{sloppypar}
\subsection{Complexity, pragmatics and discourse} 

Discourse practices show convincing evidence of their complexity at different levels of description. Language itself provides a reference environment for discourse, but people mobilise the expressive resources of language and of other semiotic systems in local contexts. These contexts are in turn characterised by interlocutors’ perception, memory, and their understanding of culture. The ways in which people use language in context shows us how speakers and listeners restrain indeterminacy and interweave points of view. 

\subsubsection{A proposal for a simplex account of discourse complexity using the pragmatic-enunciative theory of points of view}

A complex system can be accounted for in an integrated and economical manner, deemed simplex. \textcitetv{chapters/06} proposes a pragmatic-enunciative theory on points of view that uses an empathy angle to consider how others perceive, feel, think, say, and do, both from their physical position, and in relation to their value system. A person speaking first on his or her own behalf and then on behalf of another person changes enunciative stance. Changing the place from which we speak and taking the stance of another allows us to understand the world from new points of view.

\subsubsection{The morphogenesis of language action: Complexity and rhythmic synchronisation of enunciation}

\textcitetv{chapters/07} describes the complexity involved in the rhythmic synchronisation of enunciation. He combines biophysical vocal production with thought, memory, language resources, and the needs of communication between hearers and speakers. In the biophysical view, many muscles drive an internal coordination involving breath, sound, and articulation. But these phenomena are also coordinated with our thoughts and how they are situated in memory, the ways in which we use language to convey such thoughts, and the interactional constraints of our communication contexts. Painting this picture of enunciation in the mind gives the reader a new appreciation for visualising a model that is capable of accounting for these connections.

\subsubsection{Dialogism for daily interaction}

Dialogism is the constitutive orientation of discourse, whether it is dialogical (e.g. debate, interview) or monological (e.g. newspaper article, novel). This orientation is toward other discourse in the form of internal dialogue. \textcitetv{chapters/08} illustrate three types of dialogism with examples from everyday life. First, interdiscursive dialogism is oriented toward discourse that has already taken place regarding the same object, but with different interlocutors. Second, interlocutive dialogism is thinking what the person with whom we are speaking is saying and imagining what we are going to say next. Third, intralocutive dialogism occurs when the speaker orients her discourse to herself. Speakers must manage all of these dialogisms in daily life while in interaction with others.

\subsubsection{Modalities in written chat interactions: A complex system}

Incorporating emojis into text messages or on-line chats gives new possibilities for expressing modality in the form of a judgement such as likelihood, ability, permission, request, capacity, suggestions, order, obligation, or advice. Modality expression is already a complex system composed of heterogeneous elements and subsystems, and \textcitetv{chapters/09} studies the role of emojis as a written form of gesture and/or facial mimicry. He illustrates how this new expression of modality interacts with conversation to form a framework that is constantly changing. Halte argues that emojis allow us to track modality expression at the level of conversational exchange.

\subsubsection{Contributions in complexity, pragmatics and discourse}

The four contributions in this section of the volume (Rabatel, Bondì, Nowakowska \& Constantin de Chanay, and Halté) all show how changing one’s vantage point changes the salience of our world, both for researchers of language and in everyday human interaction. In addition, the methodological choice of thick description (versus a hypothetico-deductive approach) forces choices of focus and decisions about what to include as greater context. In terms of complexity, coordination between heterogeneous elements is primordial and orientation to interlocutors, both present and absent, is varied and changing and described by enunciative stance. Finally, communications technology (e.g. emoticons) has added a new layer to modality expression thus illustrating the pertinence of a complexity science framework.

\subsection{Complexity, interaction and multimodality}

The multimodal expression of human interaction seems especially adapted to a complexity science framework, given that such expression is typically the focus of micro-analyses in the form of thick description (e.g. syntax, lexicon, prosody, gesture, gaze, body posture, etc.), but that human interaction is also part of composite systems at the local level (e.g. artefact manipulation within a task, interactional role, physiological reactions) and at a more global level (e.g. institutional role, relational history, culture). Indeed, there are different ways of interweaving between these levels of analyses, given one’s vantage point and the extent to which different frameworks and disciplines are brought to bear on what is studied.  

\subsubsection{Collective reasoning as the alignment of self-identity footings}

Polo and colleagues propose to model collective reasoning on the one hand as the alignment of \textit{footings} (reflecting the posture that a person assumes during a debate) and on the other hand by taking into account \textit{facework}, (involving discursive elaborations that aim to preserve the face of oneself and others). The politeness norm is often in tension with a debate situation. In the socio-scientific debates Polo et al. analysed, they saw three types of group discussion:
1) cumulative, in which \textit{footing} is consensual and \textit{facework} does not allow disagreement to be expressed,
2) exploratory, in which \textit{footing} is critical, but constructive and \textit{facework} focuses on the success of the group and finally
3) disputational, in which \textit{footing} is competitive and \textit{facework} aims to win over others. Studying how these different types of group discussion can intertwine renews the theorisation of group reasoning.

\subsubsection{Multimodal conversational routines: Talk-in-interaction through the prism of complexity}

Multimodal conversational routines are recognised parts of our repertoire, but they can also emerge. Chernyshova and colleagues show how there is a tension -- a delicate balance -- between the expectations we have when talking with others and the emergence of the unexpected. The authors describe how a well-known conversational routine is used unexpectedly by one interlocutor, how that new usage is adopted by the other and how it thus emerges in the interaction with a modified, localised meaning that is now shared between the two speakers. 

\subsubsection{Multimodal practice of participation in a complex and dynamic framework}

Baldauf-Quilliatre \& Colon de Carvajal describe what is called the participatory framework in terms of multimodal practice. Using audio-visual data from a family card game with five players, their analyses show the formation of different ephemeral subgroups, overlapping sequences and multiple trajectories of action and interaction. Different types of resources are used, depending on the context, and the same interaction is shown to exist simultaneously in different participatory frameworks. 

\subsubsection{Second language use and development in an immersion class considered as a complex adaptive process}

Language learning in an immersion class is considered by Griggs \& Blanc to be an iterative process, where their analyses show a co-influence between the patterns of verbal action in the classroom and the development of competence in the target language. They mobilise Vygotsky’s \textit{zone of proximal development}, where an individual first develops her competence in interaction with more competent peers and then she internalises the competence with the help of mediation. This mediation is described as the construction and consolidation of language resources through repeated use in the classroom.

\subsubsection{Contributions in complexity, interaction and multimodality}

All of the contributions in this section of the volume (Polo, Lund, Plantin \& Niccolai, Chernyshova, Piccoli \& Ursi, Baldauf-Quilliatre \& Colon de Carvajal, and Griggs \& Blanc) highlight the power of the temporal lens for analysing human interaction. Each of the sections focuses on collaborative, emergent phenomena, while highlighting different tensions: between the regular and the unpredictable, between the contextually expected and the emerging, between the context-shaping and the context-renewing. Human interaction is multidimensional, characterised by interconnected elements that describe change and development. Finally, implementing a complexity approach that connects new descriptive elements can renew theorising.

\section{Zooming out for societal impact}

The laboratory of excellence ASLAN (Advanced Studies on Language Complexity) has a specific approach to giving societal value to our research. Traditionally, such value is thought of as solely economic. In other words, how many companies have you created? Or how many patents do you hold? Although ASLAN has created a number of start-ups and works regularly with companies, we have a broader approach. First, we put in place tools to raise awareness of the culture of giving societal value to our work. For example, this involves inviting researchers who work for companies to present their trajectory during a seminar, which has led to collaborations.  Second, a research project can be developed in response to societal needs. ASLAN is often solicited by both the public and private sectors, by associations, health professionals and families. Developing a project together with stakeholders outside of academia -- as real partners -- often leads to a redefinition of the boundaries of the themes on which the researcher is working. Third, our projects target outreach for the general public, and many of the projects in this volume have been presented during such events. Fourth, we create training programs for different audiences such as teachers or healthcare professionals.

\section{Summarising our contributions}

\citet{BecknerEtAl2009} consider seven characteristics of language as a complex system (distributed control and collective emergence, intrinsic diversity, perpetual dynamics, adaptation, non-linearity and phase transitions, sensitivity to and dependence on network structure, and locality of change). If the complex system can be considered as a model, we make two types of contributions. First, we extend in a number of ways the field of application of the model’s elements. Second, we add a new element to the model.

Taking the second contribution first, we add collective control and distributed emergence (in addition to the current model element of distributed control and collective emergence). This phenomenon occurs in the work of \textcitetv{chapters/14} in the language immersion class where a relation is explored between patterns of verbal and multimodal interaction in the group and the development of individual language competence. That said, if one questions the vantage point of observation, this is likely to be a case of reciprocal causation (cf.~\citealt{Bandura1986}).

\begin{sloppypar}
In the second type of contribution, we extend the field of application through new examples of emergence, interplay of heterogeneous elements, intrinsic diversity, feedback, novelty, self-organisation, adaptation, multidimensionality, and indeterminism. Although not all of these concepts are part of the seven characteristics of language, as cited by \citet{BecknerEtAl2009}, they are constructs cited by either \citet{Kretzschmar2015} or \citet{Massip-Bonet2013} in relation to language.
\end{sloppypar}

Emergence allows for two ways to extend the field of application. Self-identity footings emerge \parencitetv{chapters/11}, as do conversational routines \parencitetv{chapters/12}, and they both involve the interplay of heterogeneous elements. In the former, interplay involves facework and argumentative interactions during debate and in the latter interplay involves talk, gesture, and gaze in an everyday commercial transaction. The data on conversational routines also illustrate novelty in human interaction.

It is argued that adaptation works through amplification and competition of factors. \textcitetv{chapters/11} show the tension between politeness and argumentation, in other words, between the social and the cognitive. Another tension is shown by \textcitetv{chapters/12} between how we expect an interaction to play out and the novelty that can emerge. The examples discussed in these chapters illustrate how adaptation occurs through these forces.

Forms of coordination during enunciation \parencitetv{chapters/07} show both intrinsic diversity and self-organisation between the physical speed of the production system, language resources, thought, memory and communication context. Such is the case as well for the fluctuating participation frameworks and the use of multidimensional resources that allow their forming and un-forming \parencitetv{chapters/13}.

Forms of dialogism \parencitetv{chapters/08} and points of view based on empathy of physical location and value systems \textcitetv{chapters/06} show intrinsic diversity, depending on the meaning-making the interlocutors engage in as the interaction evolves.

Building on distributed control and collective emergence, \textcitetv{chapters/09} shows how the expression of modality is changed by emojis and that such expression is now possible at the level of a conversational exchange. This example also illustrates the notion of feedback in that expressive needs have brought about new emojis which have then modified modality and allowed other needs to emerge.

\textcitetv{chapters/03}, \textcitetv{chapters/04}, and \textcitetv{chapters/02} propose to create spaces where epistemologies and explanations can be compared, choosing methodologies based on the research questions posed and not the reverse. Their underlying motivation is to surpass dichotomies and dogma. Models are useful, but so is thick description and bridging between world views is the ultimate goal. Theoretical and methodological integration is pursued, but only insofar as it helps to reach objectives. Although not in the epistemology section, \citeauthor{chapters/06}’s chapter (this volume) argues for such integration, proposing a paradigm involving enunciation and argumentation that links to other paradigms, using a simplex approach. He illustrates enunciators' and co-enunciators’ intentions and interprets texts from linguistic, cognitive, and practical points of view.

\section{Conclusions and perspectives}

In conclusion, it is no longer questioned that a complexity science framework is a useful way to conceptualise language use, acquisition, and change. \citet{BecknerEtAl2009} have argued that perhaps the search for top-down linguistic universals is stagnating. We also know that behaviour can be described from two different standpoints, leading to results that shade into one another. “The etic viewpoint studies behavior as from outside of a particular system, and as an essential initial approach to an alien system. The emic viewpoint results from studying behavior as from inside the system” \citep[37]{Pike1967}.

Maybe top-down linguistic universals are stagnating because a \textit{change in vantage point} is necessary. This volume shows many candidates for potential \textit{bottom-up} linguistic universals, but how is the universal quality measured in this case?  In pursuing this thought, results in this manuscript lead us to consider the extent to which the bottom-up patterns we have uncovered may lead to more global changes at the group and community levels.

\section*{Acknowledgements}
The authors are grateful to the ASLAN project (ANR-10-LABX-0081) of the Université de Lyon, for its financial support within the French program “Investments for the Future” operated by the National Research Agency (ANR).

{\sloppy\printbibliography[heading=subbibliography,notkeyword=this]}
\end{document} 
