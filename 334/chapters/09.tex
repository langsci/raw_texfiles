\documentclass[output=paper]{langscibook} 
\ChapterDOI{10.5281/zenodo.6620121}
\author{Pierre Halté\affiliation{Université de Paris}}
\title{{M}odalities in written chat interactions: A complex system}
\abstract{Expressing judgement about what one says is accomplished with modalities. Even if modalities are considered as a primarily linguistic phenomenon (e.g. semantically inscribed in particular lexemes or inferred from uttered propositions), the parameters required for determining modality cover all traditional fields of linguistics and can easily be considered to be a complex system.  In this chapter, we reveal aspects of this system that have not yet been taken into account but that indicate a speaker’s emotion and thus bear modality, such as the iconic pictograms that represent mimics or gestures in the case of chat conversations over written, (a)synchronous, on-line communication. I argue that in addition to utterance level and speech turn modalities, linguists should include modalities at the level of exchanges. I illustrate a complex system that is a non-linear and unpredictable succession of (in)validations of ex/implicit contents, marked by different semiotic systems that relate to each other while determining subsequent turns of speech.}

\IfFileExists{../localcommands.tex}{
  \addbibresource{localbibliography.bib}
  % add all extra packages you need to load to this file

\usepackage{tabularx,multicol}
\usepackage{url}
\urlstyle{same}

\usepackage{listings}
\lstset{basicstyle=\ttfamily,tabsize=2,breaklines=true}

\usepackage{langsci-basic}
\usepackage{langsci-optional}
\usepackage{langsci-lgr}
\usepackage{langsci-osl}
% \usepackage{./langsci/styles/langsci-lgr}
% \usepackage{./langsci/styles/langsci-osl}
% \usepackage{langsci-gb4e}

\usepackage{tikz}
\usetikzlibrary{patterns,calc}
\pgfdeclarepatternformonly{south east lines}{\pgfqpoint{-0pt}{-0pt}}{\pgfqpoint{3pt}{3pt}}{\pgfqpoint{3pt}{3pt}}{
    \pgfsetlinewidth{0.6pt}
    \pgfpathmoveto{\pgfqpoint{0pt}{3pt}}
    \pgfpathlineto{\pgfqpoint{3pt}{0pt}}
    \pgfpathmoveto{\pgfqpoint{.2pt}{-.2pt}}
    \pgfpathlineto{\pgfqpoint{-.2pt}{.2pt}}
    \pgfpathmoveto{\pgfqpoint{3.2pt}{2.8pt}}
    \pgfpathlineto{\pgfqpoint{2.8pt}{3.2pt}}
    \pgfusepath{stroke}}
    
\usepackage{stmaryrd}
\usepackage{wasysym}
\usepackage{multirow}
\usepackage{caption}
\usepackage{subcaption}
\usepackage{mathrsfs}
\usepackage{qtree}

\usepackage{linguex}


  %pminos do not split footnotes
% \interfootnotelinepenalty=10000 %Footnote in Laporte chapters has to be split SN


%\DeclareIndexNameFormat{default}{%
%\nameparts{#1}%
%\usebibmacro{index:name}%
%{\index[names]}%
%{\namepartfamily}%
%{\namepartgiveni}%
% {}% L1
% {}% L2
%{\namepartprefix}% generates spurious space L3
%{\namepartsuffix}% generates spurious space L4
%}

%  {\DeclareIndexNameFormat{default}{%
%     \usebibmacro{index:name}{\index[names]}{#1}{#3}{#5}{#7}}}

%\DeclareIndexNameFormat{default}{%
%  \usebibmacro{index:name}{\sindex[nom]}{#1}{#3}{#5}{#7}}

%\DeclareIndexNameFormat{default}{%
%  \usebibmacro{index:name}{\sindex[person]}{#1}{#3}{#5}{#7}}
%\DeclareIndexNameFormat{default}{%
%\nameparts{#1} \usebibmacro{index:name}{\sindex[person]]}{\namepartfamily}{‌​\namepartgiven}{\nam‌​epartprefix}{\namepa‌​rtsuffix}}

%\newcommand{\smiley}{:)}

%\renewbibmacro*{index:name}[5]{%
%\usebibmacro{index:entry}{#1}%
%{\iffieldundef{usera}{}{\thefield{usera}\actualoperator}\mkbibindexname{#2}{#3}{#4}{#5}}}

% \newcommand{\noop}[1]{}

%remove for final
%\overfullrule=1mm

\newcommand{\tobi}[2]}}
\renewcommand{\S}[1]{\tobi{#1}{\textsc{*}}}

% this volume references
% puts: [this volume]
% already defined: \citetv
%\newcommand{\citepv}[1]{(\citeauthor{#1} \citeyear*{#1} [this volume])}
\newcommand{\citealtv}[1]{\citeauthor{#1} \citeyear*{#1} [this volume]}

%parentheses around example number
\newcommand{\pref}[1]{(\ref{#1})}

% in-text examples

\newcommand{\lnex}[1]{\textit{#1}} %target lang word
\newcommand{\lnlit}[1]{(lit.: `#1')} %literal reading
\newcommand{\lnlat}[1]{(#1)} % latinization
\newcommand{\lntrans}[1]{`#1'} %translation
\newcommand{\lnexl}[2]%
{\lnex{#1}{} \lnlat{#2}} % ex with latinization
\newcommand{\lnexlat}[3]{\lnex{#1}{} \lnlat{#2}{} \lntrans{#3}} % ex with latinization and tranl.

%ch01
\newcommand{\co}[1]{\mbox{\textbf{#1}}}

%ch09

\newcommand{\cyrbulg}[1]{\begin{otherlanguage*}{bulgarian}#1\end{otherlanguage*}}


%ch10
\newcommand{\nlp}{{\small NLP}}
\newcommand{\mwe}{{\small MWE}}
\newcommand{\rae}{{\small RAE}}
\newcommand{\lvc}{{\small LVC}}
\newcommand{\pos}{{\small P}o{\small S}}
%\newcommand{\todo}[1]{ \textcolor{red}{#1} }

%\renewcommand{\labelenumi}{\theenumi}
%\ainamefmt{{vv}{ll}{, ff}{, jj}} % fullname

\newcommand{\biberror}[1]{{\color{red}#1}}

\newcommand{\osenovaitem}{--~} 
  %% hyphenation points for line breaks
%% Normally, automatic hyphenation in LaTeX is very good
%% If a word is mis-hyphenated, add it to this file
%%
%% add information to TeX file before \begin{document} with:
%% %% hyphenation points for line breaks
%% Normally, automatic hyphenation in LaTeX is very good
%% If a word is mis-hyphenated, add it to this file
%%
%% add information to TeX file before \begin{document} with:
%% %% hyphenation points for line breaks
%% Normally, automatic hyphenation in LaTeX is very good
%% If a word is mis-hyphenated, add it to this file
%%
%% add information to TeX file before \begin{document} with:
%% \include{localhyphenation}
\hyphenation{
    Beck-man
    Ngu-yen
    back-chan-nel
    back-chan-nels
    mo-not-o-nous
    ste-reo-typ-i-cal
}

\hyphenation{
    Beck-man
    Ngu-yen
    back-chan-nel
    back-chan-nels
    mo-not-o-nous
    ste-reo-typ-i-cal
}

\hyphenation{
    Beck-man
    Ngu-yen
    back-chan-nel
    back-chan-nels
    mo-not-o-nous
    ste-reo-typ-i-cal
}
 
  \togglepaper[1]%%chapternumber
}{}

\begin{document}
\AffiliationsWithoutIndexing{}
\maketitle 

\section{Introduction} 

Modalities, in linguistics, are ways to express the speaker’s judgment about what she is saying. Whether in formal semantics \citep{Portner2009} or in \citegen{Gosselin2010} cognitive semantics, they are most often considered at the level of utterances, or even lexemes. Furthermore, they are systematically considered as a primarily linguistic phenomenon, whether they are semantically inscribed in certain lexemes (such as the verbs \textit{must}, \textit{can}, etc.), or inferred from uttered propositions. At this level of analysis, the complexity of modalities is already evident: the parameters required for its determination are numerous, cover all the traditional fields of linguistics, and are determined by different elements that interact with each other in the sentence. They are part of what can be called a “system”, following \citet{Larsen-FreemanCameron2008}:

\begin{quote}
In the abstract and as a broad definition, a system is produced by a set of components that interact in particular ways to produce some overall state or form at a particular point in time. (p. 26)
\end{quote}

My purpose here is to explore the complexity of this system, and to reveal aspects of it that have not yet been sufficiently taken into account in standard analyses. A system can be considered “complex” when it meets several criteria. The first criterion is the presence and interaction of agents which are heterogeneous by their nature, within the system: 

\begin{quote}
One way in which complex systems often differ from simple systems is in having many different types of elements or agents: i.e. they are “heterogeneous”. […] The complexity of a complex system arises from components and subsystems being interdependent and interacting with each other in a variety of different ways. \citep[26]{Larsen-FreemanCameron2008}
\end{quote}

The second criterion is as follows: to be considered complex, a system must have an evolutionary and dynamic nature. It is characterised by a change comparable to a flow.

\begin{quote}
In the type of complex systems that we are concerned with, everything is dynamic: not only do the component elements and agents change with time, giving rise to changing states of the system, but the ways in which components interact with each other also change with time. \citep[29]{Larsen-FreemanCameron2008}
\end{quote}

Finally, the third and last criterion states that within a complex system, this constant change is not linear, due to the heterogeneous nature of its components and the interactions that bind them: 

\begin{quote}
In a non-linear system, the elements or agents are not independent, and relations or interactions between elements are not fixed but may themselves change. \citep[31]{Larsen-FreemanCameron2008}
\end{quote}

I will work on a short conversation held via chat, a form of written, synchronous, and online communication: users are simultaneously present in a virtual chat room and produce written utterances that constitute their {turns of speech}. This corpus will be explored from the three angles mentioned above. First, I will show that modality is a complex system composed of heterogeneous subsystems and elements. Chat conversations, which first appeared in the 1970s, make the aforementioned heterogeneity clearly visible, and certainly to compensate for the lack of a posturo-mimo-gestural system in writing, they are filled with iconic pictograms representing mimics or gestures. They are intended to indicate the speaker's emotion\footnote{The speaker is the instance that takes responsibility for the propositional content of utterances. This instance is built in and by utterances and is committed to the “truth” of what is said (the term “truth” should not be understood here in its usual sense in formal logic, but rather as an implicit intersubjective agreement concerning the truth values of the proposition uttered -- on this subject see the work of \citealt{Berrendonner1981} or \citealt{Anscombre2005}). The speaker is the source of a “voice” \citep{Perrin2009} and takes responsibility of the propositional content \citep{Rabatel2009}.} , and thus bear modality. I will therefore present in more detail what I mean by “modality” in the first part, before describing this iconic semiotic subsystem which bears modalities and does not belong to a linguistic system per se. Then, we will study the evolutionary dynamics of modalities as the conversation progresses and show, amongst other things, that modalities interact with each other to form a global and ever-changing frame.

\section{Modalities as a heterogeneous system}
\subsection{What I mean by “modality”}\label{sec:09:2.1}

There are several ways of conceiving the notion of modality, which has its source in Aristotle's early work on logic and is historically linked to the notions of possibility and necessity. Nowadays, two major approaches address this notion. The first one, formal semantics and the semantics of possible worlds (see for example \citealt{Portner2009,Portner2018}), is based on a purely descriptivist approach to linguistic meaning: in this context, a linguistic utterance always bears, at a more or less deep level, an absolute and verifiable truth about the world. This position is directly derived from analytical philosophy and formal logic: behind every linguistic utterance, there is somewhere a proposition or a system of propositions (called “conversational background” in \citealt{Kratzer1977}, for example), which can be said to be true or false. In this perspective, modality is understood as “the linguistic phenomenon whereby grammar allows one to say things about, or on the basis of, situations which need not be real” (\citealt{Portner2009}: 1). This approach aims to describe everything related to the expression of possibility and necessity, especially the tense and mode systems of verbs (see for example \citealt{Mari2015}).

For our part, we follow the second approach, which adopts a “broad” interpretation of modality, as defined, for example, by \citet{Gosselin2010} who states that modality is the “validation [or invalidation] of a predicated representation” (\citealt{Gosselin2010}: 50, author’s translation). This approach is in line with the work of \citet{Bally1944} and \citet[59]{Berrendonner1981}. In addition to the famous “modus\slash dictum” distinction (the \textit{dictum} being the “representational” part of the utterance – i.e. its propositional content, and the \textit{modus} the judgement of the speaker who relates to it, whether it is semantically conveyed or pragmatically inferred), \citet{Bally1944} introduces the notion of modal subject, an instance that reacts to the enunciation of a truth-conditional representation: when I say something, I automatically position myself in a certain way with respect to the truth of what I say. As for \citet[59]{Berrendonner1981}, he proposes to redefine what is meant by “truth” when analysing linguistic utterances. In particular, he shows that even assertions are based on an intersubjective truth, shared by a community of speakers (this is what he calls “on-vérité” or “we-truth”). In this context, he considers that it is more appropriate to speak about the “validation” or “invalidation” of a content than about “truth” and “falseness”. Modality is therefore understood as the positioning of an enunciative instance (the modal subject) with respect to the relative “truth” of the utterance. This second view of modalities is much broader than the first one, and is not limited to the analysis of phenomena explicitly encoded in the language system, such as temporal or modal morphemes. In this perspective, modalities can be defined according to nine parameters: “validation instance, direction of adjustment, strength of validation, level in the syntactic hierarchy, scope in the logical structure, enunciative commitment, relativity, temporality, marking” (\citealt[60]{Gosselin2010}, author’s translation). These parameters concern most fields of classical linguistics: syntax, semantics, pragmatics, and enable us to define the traditional categories of modality.

\subsection{Different subsystems to express modalities}

Below is an exchange, taken from a chat corpus. The speakers (<L1> and <L2>) mention a poker game. The ten turns of speech are indicated using T1, T2, etc.

\ea 

\ttfamily
\parbox{30mm}{T1 [14:54] <L1>}alors t as gagné ?\\
\parbox{30mm}{~}\textsl{so you won?}\medskip

\parbox{30mm}{T1 [14:54] <L1>}:p\medskip 

\parbox{30mm}{T2 [14:55] <L2>}pfff m'en parles pas\\
\parbox{30mm}{~}\textsl{pfff tell me about it}\medskip

\parbox{30mm}{T3 [14:58] <L1>}:o\\
\parbox{30mm}{T3 [14:58] <L1>}t'as perdu combien ?\\
\parbox{30mm}{~}\textsl{how much did you lose?}\medskip

\parbox{30mm}{T3 [14:59] <L1>}épanche-toi mon petit\\
\parbox{30mm}{~}\textsl{spill it out, kid}\medskip

\parbox{30mm}{T4 [14:59] <L2>}bah je dois en être à -100\$\\
\parbox{30mm}{~}\textsl{well I must be at -100\$}\medskip

\parbox{30mm}{T5 [14:59] <L1>}ah ça va encore\\
\parbox{30mm}{~}\textsl{ah it's still okay}\medskip

\parbox{30mm}{T5 [14:59] <L1>}je pensais que ça se chiffrait en milliers\\
\parbox{30mm}{~}\textsl{I thought it was thousands}\medskip

\parbox{30mm}{T6 [14:59] <L2>}mais bon c'est que des gains que j'ai perdu :)\\
\parbox{30mm}{~}\textsl{but I only lost money I had won :)}\\

\parbox{30mm}{T7 [14:59] <L1>}ah ok\medskip

\parbox{30mm}{T8 [15:00] <L2>}ca va ca vient...\\
\parbox{30mm}{~}\textsl{it comes and goes}\medskip

\parbox{30mm}{T8 [15:00] <L2>}mais bon en ce moment ca vient pas trop :S\\
\parbox{30mm}{~}\textsl{well right now it's not coming too much :S}\medskip

\parbox{30mm}{T9 [15:00] <L1>}:(\medskip 

\parbox{30mm}{T10 [15:03] <L2>}non pas que je joue mal, mais j'ai pas de\\
\parbox{30mm}{~}{chance, je perds souvent avec le meilleur jeu}\\
\parbox{30mm}{~}\textsl{not that I play badly, but I have bad luck, I\\
\parbox{30mm}{~}often lose with the best game}\\

\z 

I will not analyse here all the modalities of the excerpt above, due to a lack of space, but I will mention some observations that will allow us to further broaden the notion of modality. First, it is possible to identify two ways of expressing modalities, as already described elsewhere (see \citealt{Gosselin2010}). Some modalities are marked: they are semantically coded and integrated in certain linguistic forms, for example verbs. Thus, in T4, the verb \textit{devoir} ‘must’ semantically bears an epistemic modality: the speaker indicates that he is not sure that he is “at $-$100\$” but that it is probable. Other modalities are rather inferred and implicit: in T2, an idiomatic expression such as \textit{m'en parles pas} ‘tell me about it’ triggers an inferential process that leads to the conclusion that L2 has lost, but also that he is disappointed (appreciative modality). In addition to these two systems of meaning -- semantics and pragmatics, which are well described in the literature on modalities (see for example \citealt{Portner2009,Gosselin2010}, etc.) there are other lesser-studied elements in digital writings: pictograms, here emoticons (T1, T3, T8 and T9), which also bear modalities (on this subject see \citealt{Halté2018,Halté2019}). They constitute a new “subsystem” for the expression of modality. Indeed, we believe that beyond linguistic phenomena, other utterances belonging to various semiotic systems allow the speaker to validate or invalidate predicated representations. This has already been shown by a number of publications on speech/gestures interactions, such as \citet{McNeill2005} or, more recently, by \citet{RoseanoEtAl2016}, who deal with the coding of speech and gesture patterns in face-to-face interaction regarding the expression of epistemicity and evidentiality. We dare to go a bit further by considering that pictograms in digital writings can be seen as forms of written gestures, see \citet{Halte2019gestes}.

\subsection{Iconic modalities} 

In addition to the already complex system of expressed or inferred linguistic modalities, \textit{iconic} modalities are therefore at work in digital interactions in writing. These modalities are identified and interpreted according to an \textit{iconeme}: a minimal unit of iconic meaning, based on the recognition of the resemblance to an object. The emoticons of this excerpt\footnote{To be read by tilting your head to the left, to recognise: 1) :p sticking one’s tongue out, 2) :o a mimic of surprise, 3) :) a smile, 4) :S a mimic of disgust, and 5) :( a sad mimic.} are composed of two minimal units: first, the colon represents the eyes and constitutes what can be called a “positional” iconeme (its only purpose is to help understand how the icon is spatially organised); second, the shapes of the mouths, which are modal iconemes since they are the sole bearers of modality. This can be illustrated with the following test, which simply consists in switching these “iconemes”: 


\ea 
\label{key}

\ttfamily
\parbox{30mm}{T8 [15:00] <L2>}mais bon en ce moment ça vient pas trop :S
\parbox{30mm}{~}\textsl{but right now it's not coming too much :S}

\z 


Replacing the “S”, an “iconeme” representing a mouth twisted with disgust, by a closing parenthesis changes the orientation of the modality (here appreciative), which shifts from undesirable to desirable: 

\ea 
\ttfamily
\parbox[t]{36mm}{(1’) T8[15:00] <L2>}mais bon en ce moment ça vient pas trop :) 
\parbox[t]{36mm}{~}but right now it's not coming too much :)
\z 

The modalities borne by “iconemes” interact with the modalities borne by the linguistic part of utterances, and produce effects (irony, emphasis, empathy, etc.) depending on whether they are in opposition or in agreement with the latter (see \citealt{Yus2011}).

To sum up: the interpretation of linguistic modality is based on nine parameters (seen in \sectref{sec:09:2.1}); modalities can be marked or inferred; and finally, modalities can be borne by different semiotic systems interacting more or less linearly with each other as the conversation unfolds. Rather than using the term modality, I propose that from now on we talk about “complex modal system”. Here, we will try to describe its evolution over the course of a conversation. Indeed, it seems plausible to imagine that all these elements would constitute a global modal configuration, which would condition the uttered contents and the successive behaviours adopted by speakers as their conversation unfolds.

\section{Dynamics of modalities in conversation}
\subsection{Monological modal sequences}

We will defend here the idea that modality is not only the reaction of a modal subject to a content that he/she utters, but rather a co-construction based on conversational parameters: sequences of turns of speech, enunciation situation, knowledge shared by the interlocutors, etc. From this perspective, following \citet{BresNowakowska2006} in particular, I consider that modality can relate not only to a content uttered by the speaker (monological modality), but also to a content uttered by the interlocutor (dialogical modality). Our purpose here is to describe sequences of monological modalities, then of dialogical ones, and finally that of an exchange, which will help to show the dynamic, changing and non-linear aspect of the complex modal system that I am trying to characterise here.

At a first level, modalities can be distributed and they interact within an utterance produced by the speaker. It is sometimes possible to identify modalities locally, at the level of one or more lexemes, but these local modalities generally contribute to the construction of a global modality, which is interpreted at the level of utterances. This is the case, for example, in T4, which is globally uttered in an epistemic manner: 

\ea 

\ttfamily
\parbox{30mm}{T4 [14:59] <L2>}bah je dois en être à -100\$\\
\parbox{30mm}{~}\textsl{well I must be at -100\$}

\z 

Here, the epistemic modality, which is semantically borne by the verb \textit{devoir} ‘must’, is combined with an appreciative modality expressed by \textit{bah} ‘well’ (which has also an epistemic value here). In T6, the speaker combines an alethic modality, by presenting a content as an objective truth using an assertion centred around the verb \textit{être} ‘to be’, and an appreciative modality borne by the emoticon: 

\ea 

\ttfamily
\parbox{30mm}{T6 [14:59] <L2>}mais bon c'est que des gains que j'ai perdu :)
\parbox{30mm}{~}\textsl{well I only lost money I had won :)}

\z 

The appreciative modality, unlike other modalities, can easily be combined with all the other modalities. 

Modalities can also explicitly show their fundamentally interactional nature. Thus, in T1, the question mark, with an epistemic modality, and the “sticking one's tongue out” emoticon indicating that the utterance previously produced is a provocation have one thing in common: they make explicit the fact that these modalities are addressed to others.

\ea 

\ttfamily
\parbox{30mm}{T1 [14:54] <L1>}alors t as gagné ?\\
\parbox{30mm}{~}\textsl{so you won?}\\
\parbox{30mm}{T1 [14:54] <L1>}:p\\

\z 

The question mark indicates a real question, asked to the interlocutor; and the “sticking one's tongue out”, which relates to the previously uttered proposition (it indicates that, in some way, the speaker expects a negative answer to the question asked), can only be explicitly addressed to the interlocutor. The epistemic modality, which expresses doubt or questioning, necessarily triggers an attempt by the latter to answer this doubt or question, whether it is explicitly (as is the case here) or implicitly addressed to the interlocutor. These sequences of monological modalities come into contact, at a second level, with modalities assumed by the interlocutor: they thus become dialogical modalities. 

\subsection{Dialogical modal sequences}

Let's look at an example taken from our conversation: 

\ea 
\ttfamily

\parbox{30mm}{T8 [15:00] <L2>}mais bon en ce moment ça vient pas trop :S\\
\parbox{30mm}{~}\textsl{well right now it's not coming too much :S}\medskip

\parbox{30mm}{T9 [15:00] <L1>}:(\\
 
\z 

In T8, L2 produces an utterance with a modality that gives an appreciation (positive or negative), materialised by the emoticon “:S”, which indicates disgust. T9 is produced by L1 and only consists of an emoticon representing a mimic of sadness. It is clear that the modality expressed by L2 in T8, which is an appreciative modality oriented towards the “undesirable” pole (see \citealt{Gosselin2010} for more data on appreciative modalities), determines which modalities can be used in T9. The question is how. Two hypotheses can be suggested: 

\begin{itemize}

\item By producing T9, L1 implicitly takes up the content (or \textit{dictum}, which is therefore not modalised) produced by L2 in T8 and modalises it in turn; 

\item By producing T9, L1 assigns an appreciative modality to a complete and already modalised utterance, produced in T8 by L2.

\end{itemize}

We believe that the second option is the least costly and most defensible. It is difficult to imagine that the emoticon in T9 would relate to anything other than T8, that is to say, a complete utterance, which is therefore already modalised by L2. We agree here with Bres \& Nowakowska’s proposition when they define dialogism in the following way:

\begin{quote}
(a) I will consider as dialogical an utterance (or fragment of an utterance) in which the modalisation of E1 [enunciator n°1] applies to a dictum presented as already having the status of utterance, that is, a dictum that has been modalised by another enunciator, whom I refer to as e1 (\citealt[72]{Bres1999}, author’s translation).
\end{quote}

\begin{quote}
(b) We assume that dialogical utterances differ from monological utterances in the following way: in monological utterances, deictic and modal actualisation applies on a dictum; in dialogical utterances, this operation is not carried out on a dictum, but on (what is presented as) an already actualised utterance (\citealt[29]{BresNowakowska2006}, author’s translation).
\end{quote}

The framework set by Bres \& Nowakowska changes the scope of application of modalities, as defined by \citet{Gosselin2010}. Indeed, this view on dialogism presupposes that it is possible to have modalities on something other than a predicated representation: an already modalised utterance. Monological modality thus sets a kind of modal background, to which other modalities can be added – dialogical ones – which are expressed later. These dialogical modalities, which apply to a previously fixed modal domain, have effects: it is because L1 expresses in T9 an affective modality along the same lines as that expressed in T8 that T9 is interpreted as an expression of empathy.

Furthermore, it is possible not only to infer modalities, but also to make implicit contents bear modalities, still within the context of dialogue. Thus, in the sequence from T1 to T3: 

\ea 
\label{ex:key:6}

\ttfamily
\parbox{30mm}{T1 [14:54] <L1>}alors t as gagné ?\\
\parbox{30mm}{~}\textsl{so you won}\\
\parbox{30mm}{T1 [14:54] <L1>}:p\medskip

\parbox{30mm}{T2 [14:55] <L2>}pfff m'en parles pas\\
\parbox{30mm}{~}\textsl{pfff tell me about it}\medskip

\parbox{30mm}{T3 [14:58] <L1>}:oY\\
\parbox{30mm}{T3 [14:58] <L1>}t'as perdu combien ?\\
\parbox{30mm}{~}\textsl{how much did you lose’}\medskip
 
\parbox{30mm}{T3 [14:59] <L1>}épanche-toi mon petit\\
\parbox{30mm}{~}\textsl{spill it out, kid}\\

\z 

The surprise emoticon that appears at the very beginning of T3 brings an epistemic modality to an implicit content. Indeed, in T2, L2 produces \textit{p{f}{f}{f}{f} m'en parles pas} (`p{f}{f}{f} tell me about it'). This utterance, consisting of an idiomatic expression preceded by an onomatopoeic interjection\footnote{Interjections are discursive markers that have a deictic meaning and have specific functions in interactions (see for example \citet{Baldauf-Quilliatre2016}, who deals specifically with \textit{pff} in French). From a semiotic point of view, they are often divided in two groups that have strong semiotic differences: primary (or onomatopoeic) interjections (such as \textit{ouf!}, for example), which come from sounds; and secondary interjections, which derive from symbolic (in the peircian frame) words (such as \textit{Mon Dieu!}, or \textit{Merde!}). For a definition of interjections and their semiotic categorisation, see for example \citealt{Swiatkowska2006}, \citealt{Kleiber2006}, \citealt{Halté2018}.} indicating disappointment, cannot be interpreted literally, but it triggers an inferential process leading to a conclusion that can be glossed as follows: `I have lost and I am disappointed'. The epistemic modality, borne by the surprise emoticon, relates to this implicit conclusion and not to the literally uttered proposition \textit{ne m’en parle pas} (lit: `don’t talk about it to me'). Subsequently, the conversation develops on the basis of this epistemic modality, and in T3 a request for explanation is indeed provided by L2.

\section{Conclusion}

While the definition of modality as the “validation of a predicated representation” appears as quite simple and easily isolable in the context of monological utterances, the analysis of dialogical utterances shows that the notion needs to be extended, or at least to be considered at different levels of the conversation: we are dealing with a complex modal system, which is rather a frame where predicated, and sometimes already modalised representations, appear. We propose that, in addition to the modalities studied at the level of utterances or turns of speech, research should also include the study of modalities at the level of exchanges, as a complex system: a non-linear and unpredictable succession of validations or invalidations of explicit and implicit contents and/or already modalised contents, marked by signs belonging to different semiotic systems (not only linguistic but also posturo-mimo-gestural), interacting with each other, which determine subsequent turns of speech. 


\section*{Acknowledgements}
The author is grateful to the ASLAN project (ANR-10-LABX-0081) of the Université de Lyon, for its financial support within the French program “Investments for the Future” operated by the National Research Agency (ANR).

{\sloppy\printbibliography[heading=subbibliography,notkeyword=this]}
\end{document} 
