\documentclass[output=paper]{langscibook} 
\ChapterDOI{10.5281/zenodo.6620119}
\author{Aleksandra Nowakowska\affiliation{UMR 5267 Praxiling; Université Paul Valéry Montpellier 3} 
        and Hughes {Constantin de Chanay}\affiliation{UMR 5161 ICAR; Centre National de la Recherche Scientifique; Ecole Normale Supérieure; Université Lumière Lyon 2}}
\title{Dialogism for daily interaction}
\abstract{We build upon dialogism as practiced by praxematics and upon analysis of verbal interactions of multimodal corpora in order to propose a toolkit that is equipped to describe enunciative heterogeneity in verbal interaction. First, we present the principles and concepts of dialogism. Then, through a set of example enunciations taken from a recording of an aperitif between friends, we examine the extent to which these concepts function on selected instances, and then we discuss how they function on the discourses that are attached to these instances. The specificity of verbal interaction implies distinguishing between speaker and enunciator, defining interlocutive and anticipative interlocutive dialogism, and accounting for dialogic markers such as prosody, multimodality, but also for prototypic syntactic markers. Our analyses argue that although dialogism was conceptually developed on non-interactive discourse, specific adjustments show that it is also pertinent for describing interactive discourse.}
\IfFileExists{../localcommands.tex}{
  \addbibresource{localbibliography.bib}
  \usepackage{langsci-optional}
\usepackage{langsci-gb4e}
\usepackage{langsci-lgr}

\usepackage{listings}
\lstset{basicstyle=\ttfamily,tabsize=2,breaklines=true}

%added by author
% \usepackage{tipa}
\usepackage{multirow}
\graphicspath{{figures/}}
\usepackage{langsci-branding}

  
\newcommand{\sent}{\enumsentence}
\newcommand{\sents}{\eenumsentence}
\let\citeasnoun\citet

\renewcommand{\lsCoverTitleFont}[1]{\sffamily\addfontfeatures{Scale=MatchUppercase}\fontsize{44pt}{16mm}\selectfont #1}
   
  %% hyphenation points for line breaks
%% Normally, automatic hyphenation in LaTeX is very good
%% If a word is mis-hyphenated, add it to this file
%%
%% add information to TeX file before \begin{document} with:
%% %% hyphenation points for line breaks
%% Normally, automatic hyphenation in LaTeX is very good
%% If a word is mis-hyphenated, add it to this file
%%
%% add information to TeX file before \begin{document} with:
%% %% hyphenation points for line breaks
%% Normally, automatic hyphenation in LaTeX is very good
%% If a word is mis-hyphenated, add it to this file
%%
%% add information to TeX file before \begin{document} with:
%% \include{localhyphenation}
\hyphenation{
affri-ca-te
affri-ca-tes
an-no-tated
com-ple-ments
com-po-si-tio-na-li-ty
non-com-po-si-tio-na-li-ty
Gon-zá-lez
out-side
Ri-chárd
se-man-tics
STREU-SLE
Tie-de-mann
}
\hyphenation{
affri-ca-te
affri-ca-tes
an-no-tated
com-ple-ments
com-po-si-tio-na-li-ty
non-com-po-si-tio-na-li-ty
Gon-zá-lez
out-side
Ri-chárd
se-man-tics
STREU-SLE
Tie-de-mann
}
\hyphenation{
affri-ca-te
affri-ca-tes
an-no-tated
com-ple-ments
com-po-si-tio-na-li-ty
non-com-po-si-tio-na-li-ty
Gon-zá-lez
out-side
Ri-chárd
se-man-tics
STREU-SLE
Tie-de-mann
} 
  \togglepaper[1]%%chapternumber
}{}

\begin{document}
\maketitle 
%\shorttitlerunninghead{}%%use this for an abridged title in the page headers

{We would like to propose the draft of a “toolkit” that is adapted to the description of enunciative heterogeneity in verbal interaction, and which is compatible with two theoretical approaches: that of dialogism as practiced by \textit{praxematics} within the Praxiling research team (UMR 5267); and that of the analysis of verbal interactions on multimodal corpora as practiced at ICAR}{ (UMR 5191).}

Their theoretical compatibility is enabled by:

\begin{itemize}

\item the primacy given to observable forms in the analysis;
\item the central position given to exchange in the study of discursive phenomena.

\end{itemize}


This study is based on an interactional corpus consisting in the recording of an aperitif between friends.\footnote{Entitled “Apéritif entre amis -- rupture” freely available from the CLAPI database at \url{http://clapi.icar.cnrs.fr}.}

The first part of this chapter is dedicated to the presentation of the fundamental principles and concepts of dialogism; then we will examine to what extent these concepts can be functional in order to study the test-corpus, first by focusing on discursive instances, and then on the discourses attached to these instances.

\section{Dialogism: The praxematic approach in the wake of Bakhtine}

\textit{Dialogism} (\citealt{Bakhtine1970,Bakhtine1984}) is the constitutive \textit{orientation of} all discourses, regardless of their genre: \textit{dialogal}\footnote{Praxematics distinguishes “dialogal” (which has no correspondent term in English) to characterise the interactions between participants who are connected to each other, in other words the dialogue in its external form, marked by the alternation of speech turns produced by two or several speakers.} (interview, debate, etc.) or {\textit{monologic}} (press article, novel, etc.) towards other types of discourse, in the form of \textit{internal dialogue}. This orientation towards other discourses is achieved in three ways: 

\begin{enumerate}

\item through {\textit{interdiscursive dialogism,} }{i.e. towards previous discourse produced by third parties on the same subject;} 

\item through {\textit{interlocutive}}{} {\textit{dialogism,} }{i.e}{.}{ towards the previous turn of speech of the addressee in dialogical genres and towards the “response” of the recipient that is anticipated in discourse;} 

\item  through {\textit{intralocutive dialogism,} }{i.e. towards the speaker himself, as he/she is his/her first addressee.} 

\end{enumerate}

{We intend to grasp dialogism during verbal interaction in order to show its complexity, and to propose an adaptation of the descriptive principles that were elaborated based on monologic texts. The notion of interlocutive dialogism directly articulates dialogism and verbal interaction.}

\subsection{Interlocutive dialogism} 

\textit{Interlocutive dialogism} indicates the fact that speakers adjust their discourse according to their interlocutor, the knowledge and discourse that they attribute to him or her, etc. These facts have been described by rhetoric, sociolinguistics, pragmatics or conversational analysis through the notions of \textit{recipient design} \citep{SacksEtAl1974} or \textit{audience design} \citep{ClarkMurphy1982,Bell1984}. The notion of interlocutive dialogism also reflects the following two facts, which are not taken into consideration by the notions of \textit{recipient design}  or  \textit{audience design}:

\begin{itemize}
\item the syntax of utterances can be analysed in its dialogical dimension. 
\end{itemize}

\citet{DuBois2014} shows how a speaker uses his or her interlocutors’ previous utterances to construct his/her own utterances (“syntactic parallelism”).\footnote{The notion of \textit{parallelism} is particularly explored in J. W. Du Bois' recent research (2007--2014) on dialogical syntax in verbal interaction: “when speakers selectively reproduce aspects of prior utterances, and when recipients recognise the resulting parallelisms and draw inferences from them” (\citeyear[359]{DuBois2014}).} Parallelism can be observed in the case of “diaphonic” recovery \citep{RouletEtAl1985}:

\ea \label{ex:8:1}
{(e.g. 1, 00 : 01 : 01 : 28-00 : 01 : 39) The sequence focuses on the Moroccan slippers that Justine (JUS) shows to her guests Arnaud (ARN) and Albine (ALB)}

\upshape\ttfamily
\parbox{8mm}{JUS}du coup vous avez fait quoi aujourd'hui/t' as vu mes \\
\parbox{8mm}{~}supers babouches marocaines/\\
\parbox{8mm}{~}‘So what did you do today/you saw my great Moroccan\\ 
\parbox{8mm}{~} slippers/’\medskip

\parbox{8mm}{ARN}\upshape\ttfamily {ah elles sont coo:ls}\\
\parbox{8mm}{~}{‘ah they're coo:l’}\medskip

\parbox{8mm}{JUS} \textit{elles sont belles} \textsl{hein}\footnote{{ The parts in italics are indicated by us. We italicise the element that is studied in terms of enunciative heterogeneity, here the diaphonic repetition.}}{/}\\
\parbox{8mm}{~}{‘\textsl{they're beautiful, huh}?’}\medskip

\parbox{8mm}{ARN}{tu les sors: d'où à part du maroc évidemment/}\\
\parbox{8mm}{~}{‘you got them from where apart from Morocco of course/’}\medskip

\parbox{8mm}{JUS}{ben \textsl{du maroc}   [((rit))}\\
\parbox{8mm}{~}{‘well \textsl{from Morocco} [  ((laughs))}’\largerpage[-1]\pagebreak

\parbox{8mm}{ARN}{[ouais mai:s tu r{\textasciigrave}viens pas du maroc que j' sache/]}\\
\parbox{8mm}{~}{‘[yeah, but you're not coming back from Morocco as far as I 
\parbox{8mm}{~}know.] ‘}\medskip

\parbox{8mm}{JUS}{nan c'est une copine [qui::  ]} \\
\parbox{8mm}{~}{‘no it’s a friend [who::: ]‘}\medskip

\z 

In this excerpt partial diaphonic repetitions of the speaker's turn follow one another with a change of modality (exclamation\slash question\slash assertion) and/or tone (seriousness\slash laughter);


\begin{itemize}
\item speakers interact with their interlocutor’s previous utterances and also with the potential subsequent response that they attribute to the latter, which they constantly anticipate (\citealt[303]{Bakhtine1984}). This second aspect is addressed using the notion of {\textit{anticipative interlocutive dialogism} (\citealt{BresEtAl2016})}{.} 
\end{itemize}
 


\subsection{Anticipative interlocutive dialogism}

The prepositional phrase \textit{à part du maroc évidemment} `apart from Morocco, of course' combines the anticipated response of the interlocutor JUS (they come from Morocco, as their name suggests) with the reaction of the speaker ARN (\textit{à part évidemment} `apart, of course') to this offbeat response. ARN knows that JUS has not been to Morocco, and since their manufacturing and commercialisation locations may be different, he would like to know the latter. This spoken anticipation, based on the speaker's experiential knowledge, is precisely at the origin of the humorous nature of the exchange following it. ARN's turn closely intertwines the response attributed to JUS in answer to his question and ARN’s reaction to that response.

\section{Instances: The enunciative apparatus for the description of dialogism}
\subsection{Double enunciation}

Any dialogic utterance implies enunciative discordance, i.e. two distinct enunciations that are not always consistent with respect to the situation. These enunciations are in a relation of incorporation: embedding enunciation [E] and embedded enunciation [e]. In other words, dialogical utterances are structured around an internal micro-dialogue: they are the result of the interaction of an enunciation act [E]\footnote{[E] indicates the enunciation representation, and (E) the utterance that is produced.} spoken by the enunciator with another enunciation act [e], resulting in their enunciative heterogeneity. The reported speech (which would correspond to \textit{ton rapport d’étonnement est super} ‘your astonishment report is great’ in the example below) provides the emblematic structure for the representation of enunciative heterogeneity:

\ea 
\label{ex:8:2}

{(e.g. 2, 00: 03: 43-00: 03: 56) The sequence is based on the astonishment report prepared by Albine}

\upshape\ttfamily
\parbox{8mm}{ALB}{~~~ah si j'étais contente parce que \textsl{mon père il m'a: \\
\parbox{8mm}{~} il m'a DIT que: mon rapport d'étonnement était super   \\
\parbox{8mm}{~}[il a dit qu'il] était passionné et tout par c{'} que\\
\parbox{8mm}{~} j'ai dit:}}\\
\parbox{8mm}{~}{‘ah yes I was happy because \textsl{my father he told me: \\
\parbox{8mm}{~} he TOLD me that: my astonishment report was great  \\
\parbox{8mm}{~}[he said he] was fascinated and everything about \\
\parbox{8mm}{~}what I said:}’} \medskip

\parbox{8mm}{JUS}{[parce qu'il a lu/]}\\
\parbox{8mm}{~}{                 ‘[because he read/]’}\medskip

\parbox{8mm}{JUS}{~~~donc c'est quoi en fait/rapport [d'étonnement/]}\\
\parbox{8mm}{~}{‘so what is it in fact/report of [astonishment/]‘}\medskip

\parbox{8mm}{ALB}{~~~~~~~~~~~~~~~~~~~~~~[rapport]     d'étonnement ouais}\\
\parbox{8mm}{~}{~~~~~~~~~~~~~~~~~~~~~~~~[report] of   astonishment yeah’}\medskip

\parbox{8mm}{ALB}~~~c'est:: comme un rapport de stage mais c'est plus toutes \\
\parbox{8mm}{~} les questions que tu peux t{'} poser par rapport à c' qui \\
\parbox{8mm}{~} {s'est passé:}
\parbox{8mm}{~}‘it’s:: like an internship report but it's  \\
\parbox{8mm}{~} more all the questions you can ask yourself about what 
\parbox{8mm}{~} happened:’\medskip

\parbox{8mm}{JUS}{~~~donc pourquoi il appelle ça rapport d'étonnement/}\\
\parbox{8mm}{~}{‘so why does he call it astonishment report/’}

\parbox{8mm}{ALB}{~~~bah parce que c'est d{'} l'étonnement{\textbackslash}}\\
\parbox{8mm}{~}{‘well because it's astonishment{\textbackslash}’}

\parbox{8mm}{JUS}{~~~d'accord{\textbackslash}}
\parbox{8mm}{~}~{okay{\textbackslash}}

\z 

The parameters of the embedded enunciation [e] are dependent on those of [E] (imperfect of \textit{était} ‘was’, 1st person deictics). The participant ALB can be identified by [L\textsubscript{ALB}] (which is therefore distinct from [L\textsubscript{JUS}] and [L\textsubscript{ARN}]). This [L\textsubscript{ALB}] is by default co-referential with the main enunciator E\textsubscript{1} in the dialogical utterance, who is therefore called E\textsubscript{1ALB.} E\textsubscript{1ALB} corresponds by default “to the actualisation instance of the utterance in its lexico-semantic, deictic, syntactic and modal dimensions” (\textit{ibid}.); in other words, he is presumed to be intentionally responsible for making sensible choices with respect to the (lexical and grammatical) morphemes of the utterance, including the sequence presented as reported (\textit{mon rapport d’étonnement était super}) ‘my astonishment report was great’). He is also presented as the linguistic equivalent -- and therefore as the relay to [L\textsubscript{ALB}], which can only be represented in the utterance by E\textsubscript{1ALB} -- of the 1st person marker in the utterance (\textit{j'}, \textit{je} ‘I’, \textit{mon} ‘my’, \textit{m'} ‘me’), including those in “indirect style” (IRS), and of the present of enunciation (here implicit) in relation to which the imperfect and perfect past tenses are defined. He is also the one who actualises the utterance as an assertion in the indicative mode and is made responsible for the indirect mode of the report. In this case, we can also identify a source for the speech reported in indirect style, here the enunciator e\textsubscript{1} who corresponds to the father of the speaker-enunciator [L\textsubscript{ALB-}E\textsubscript{1ALB}]. The lowercase letter indicates the hierarchy of enunciators in the dialogic utterance. One can also speak in this case of “enunciative recursivity” \citep{Rosier2008}: X said that Y said, etc., concerning the enunciator\textsubscript{} ${\sum}$\textsubscript{1} of\textsubscript{} the reported speech (${\sum}$), ${\sum}$\textsubscript{1} corresponding in our case to the speaker-enunciator [L\textsubscript{ALB-} E\textsubscript{1ALB}]. In the turn of speech produced by the speaker [L\textsubscript{ALB}], we outline the dialogic utterance (E) in italics. The instance assuming the enunciative responsibility for (E) is [L\textsubscript{ALB-}E\textsubscript{1ALB}]. In this dialogical utterance, we identify the reported speech in indirect style, indicated as (e), which is referred to {the}{ enunciator e\textsubscript{1} who}{} {reports himself} {in}{ indirect style the utterance (${\sum}$) of an enunciator ${\sum}$\textsubscript{1}. Let us assume the embedding of utterances in the dialogical utterance (E\textsuperscript{(e(${\sum}$)}), or by specifying the instances (E\textsuperscript{(e}[e\textsubscript{1p}\textsubscript{è}\textsubscript{re}]\textsuperscript{ (${\sum}$}[${\sum}$\textsubscript{1fille}\textsuperscript{)} \textsubscript{[LALB-E1ALB]}). Finally, let us specify that (e), in IRS, is a representation in (E), at a time T\textsubscript{0} of what was actually produced by e\textsubscript{1} at time t\textsubscript{0}, when he had the status of speaker.}

\subsection{Instances, interaction and multimodality}

{The interactive and multimodal nature of verbal interaction corpora encourages us to propose theoretical adjustments that take into account their relation to real participants.}

\begin{enumerate}

\item It is necessary to take into account the discursive nature of corpora: whatever the definition of utterances, most of the time, there will be several of them in the discourse of a given participant. In order not to multiply the number of speakers, which would be counter-intuitive, it is necessary to postulate a federating instance such as the “textual speaker” of the ScaPoLine \citep{NølkeEtAl2004}, better designated as “discourse speaker” by H. \citet[128]{Kronning2014}.

\item As \citet{DendaleColtier2005} point out, this instance corresponds to the lambda speaker of Ducrot, “the speaker as a being of the world”, who considers by default that the interactant is the source of his representations. Since interactions are most often polylogic, it is necessary to plan as many speakers in discourse as interactive spaces, which correspond to the identities of the interactants (specification with indexes above).

\item Since the data are multimodal, one may wonder whether it is necessary to provide special instances for the correspondents of the speaker’s verbal speech: postures, gestures, mimics; that is, in parallel with speaker (L-E), P, G, M instances. But if we assume that these various instances are convergent and associated in a single discourse by default, we can postulate a unifying instance D (as “discourse”; the notion of “speaker” targets linguistic discourse in priority) which takes into account the multimodal aspect of the discourse. Multimodality also leads to connecting the lambda speaker to the real participant, i.e. to the “being of the world” outside linguistic discourse, because he or she is really perceptible, through hearing (voice) and sight (postures, gestures, facial expressions). In other words, we would reintroduce in linguistics the notion of “speaking subject” that has long been excluded from the field in order to establish the specificity and relevance of enunciative subjects.

\item Ultimately, interactive corpora make it necessary to “untie” \citep{Rabatel2010} the speaker and enunciator -- which are “matched” to a speaker, but can extend beyond their enuniciative scope. This fact was pointed out by \citep{Perrin2021} about what he calls “monologic dialogue” (two speakers, one enunciator), for example during two-way narratives. Our corpus contains many examples of “shared enunciators”, who co-construct utterances, propositional contents, even a sequence forming a narrative (here through couple complicity). The sequences in italics below correspond to an enunciator shared by ARN and ALB:\largerpage[2]

\ea 
\label{ex:8:3}
{(e.g. 3, 00 : 09 : 09 : 46-00 : 09 : 52)}

\upshape\ttfamily
\parbox{8mm}{JUS}{ça y est vous avez décidé/} \\
\parbox{8mm}{~}{‘that's it,   you've decided/’} \medskip

\parbox{8mm}{ARN}{\textsl{ouais mercredi on a même une date de:   [de rupture]}}\\
\parbox{8mm}{~}{{‘yeah Wednesday we even have a date of:   [of rupture]’}}\\
\parbox{8mm}{JUS}{~~ ~~~~~~~~~~~~~~~~~~~~~~~~~~~~~~~~~~~~~~[de rupture/]}\\
\parbox{8mm}{~}{ \textsl{‘}[of rupture/]’}\medskip

\parbox{8mm}{ALB}{~\textsl{ouais}} \\
\parbox{8mm}{~}{\textsl{‘yeah’}}\medskip

\parbox{8mm}{JUS}{c'est quand/}\\ 
\parbox{8mm}{~}{‘when is it/’}\\
\parbox{8mm}{~}{(0.5)} \medskip

\parbox{8mm}{ARN}{\textsl{nan c'est secret ça par contre/}}\\
\parbox{8mm}{~}{    ‘\textsl{No, that's secret, though/’}}\\

\z 
\end{enumerate}

{We will discuss more precisely the relevance of this theoretical adjustments in a future study.}

\section{Dialogical markers and the complexity of verbal interaction}

We will address different markers of dialogism, whether they belong to linguistics or not, by focusing on enunciative instances and their discourse. 

\subsection{Echo}

By echo, the speaker-enunciator L\textsubscript{1}{}-E\textsubscript{1} reuses the utterance, or most often a part of the utterance (e), of the previous conversational turn of another enunciator, either to simply acknowledge receipt of what was just said or to comment, question, make fun of, even ridicule the echoed segment. In oral language, speakers can use a specific prosody that functions as the evaluative commentary. In the following example:\largerpage[2]

\ea 
\label{ex:8:4}
{(e.g. 3, 00 : 05 : 05 : 17- 00 : : 1805)}

\upshape\ttfamily
\parbox{8mm}{ALB}{~~ ~euh oui j{\textasciigrave} leur ai dit: euh: c'est pas l{\textasciigrave} jour de l'an}\\
\parbox{8mm}{~}{‘uh yes I told them: uh: it's not New Year's Day’}\medskip

\parbox{8mm}{ALB}{~~ ~puisque: toi t{\textasciigrave} as [t{\textasciigrave} es pas là]}\\
\parbox{8mm}{~}{~~~~~~~~~~~~~~~~~~~~~~~~‘since: you have [you're not here]’}\medskip
	
\parbox{8mm}{ARN}{~~~~~~~~~~~~~~~~~~~~~~~~~~~~~~~~~~~~~~~[c'est pas LE] jour de l'an mais c'est pareil:}\\ 
\parbox{8mm}{~}{~~~~~~~~~~~~~~~~~~~~~~~~~~~~~~~~~~~~~~~~‘[it's not THE] New Year's Day but it's the same:’}\\

\z 

{ARN marks an enunciative disassociation on the echoed sequence (\textit{c’est pas LE jour de l’an} ‘it’s not THE New Year's Day’) by the prosody, stress given on LE (THE), and by taking a enunciative non-engagement posture (looking straight in front of him outside the interaction, he moves his neck forward), and he keeps his postural disengagement, but not the forward position of his neck, on the corrective commentary \textit{mais c’est pareil} ‘but it is the same’.}

\subsection{Syntactic marking: left dislocation of an adjectival phrase in the superlative of superiority}\largerpage

Many syntactic markers of dialogism (cleft and pseudo-cleft sentences, concession, negation, interrogation, dislocation, subordination, comparison, etc.) are described in \citet{BresEtAl2019}. Dislocation is the most frequent syntactic marker in the corpus. Let us examine a case of left-dislocation of an adjectival phrase in the superlative of superiority. This syntactic structure is usually analysed (\citealt{Nowakowska2009,BresEtAl2019}) as a marker of anticipatory interlocutive dialogism, based on monologic writings; however, its functioning is more complex in verbal interactions:

\ea 

\label{ex:8:5}

{(e.g. 4, 00 : 14 : 43-00 : 15 : 01) Arnaud and Albine are in a relationship, at least for the moment, Albine acknowledges calling her companion by the name of a mutual friend Sébastien, Arnaud then confesses that this confusion goes further than a slip of the tongue on the name} 

\upshape\ttfamily
\parbox{8mm}{ALB}{  j{\textasciigrave} dis toujours euh sébastien à la place de: d'arnaud}\\
\parbox{8mm}{~}{  ‘I always say uh sebastian instead of: arnaud’}\medskip

\parbox{8mm}{ARN}{  ouais c'est: ça finit par être emmerdant parce qu'elle veut qu{\textasciigrave} je mettre son parfum/(.) elle elle veut qu{\textasciigrave} j{\textasciigrave} mette les mêmes chaussures}\\ 
\parbox{8mm}{~}{  ‘yeah it is: it ends up being annoying because she wants me to wear his perfume/(...) she wants me to wear the same shoes’}\medskip

\parbox{8mm}{ALB}{hein/}\\ 
\parbox{8mm}{~}{  ‘right/’}\medskip

\parbox{8mm}{ARN}{[elle m'appelle sébastien]}\\
\parbox{8mm}{~}{  ‘[she calls me sebastian]’}\medskip

\parbox{8mm}{JUS}{[les mêmes chaussures/]}\\
\parbox{8mm}{~}{  ‘[the same shoes/]’}

\parbox{8mm}{ALB}{  n'im:porte quoi{\textbackslash}}\\
\parbox{8mm}{~}{  ‘nonsense{\textbackslash}’}

\parbox{8mm}{ARN}{oh j'en rajoute un peu mais c'est pour (inaud.) le truc tu vois .h mais elle m'appelle sébastien nan mais \textsl{le pire l- le plus chiant c'est l{\textasciigrave} parfum quoi} (.) sébastien son parfum {\textasciigrave}fin moi dans ma tête il est vachement associé à lui le: hm le récent là la nuit de l'homme/}\\
\parbox{8mm}{~}{‘I’m exaggerating a little but it's for (inaud.) the thing you see.h but she calls me sebastian no but \textsl{the worst l- the most annoying is the perfume you know} (..) sebastian his perfume {\textasciigrave}well me in my head it's really associated with him the: hm the recent one there la nuit de l'homme ((name of the perfume))/’}\medskip

\z 

{This dislocation can be analysed as follows:}

\begin{enumerate}

\item \textit{interaction with the previous turn of the same speaker: ça finit par être emmerdant} ‘it ends up being annoying’

The dislocated element on the left (\textit{le pire le plus chiant} ‘the worst the most annoying’) establishes a relation of superiority comparison with the adjective “annoying”. Dislocation is used, from the point of view of intralocutive dialogism, to establish a relation of comparison and hierarchy between two elements enunciated by the same speaker in two distant turns of speech in the same interactional thematic sequence;

\item \textit{interaction with the anticipated  reaction of the interlocutor} 

The left-dislocation of the adjective establishes a relation of superiority comparison with the evaluation attributed to the interlocutor: [c’est terrible ‘it is terrible’]{, in} {reaction to \textit{elle m'appelle sébastien} ‘she calls me Sebastian’ above.}

Arnaud's turn is simultaneously in a dialogic relation with his previous discourse and with the anticipated reaction-response of the interlocutor. Turns of speech are not only produced in relation to one’s or others’ previous turns, they are sometimes simultaneously oriented, through the same marker, towards the expected discourse that they constantly anticipate.

\end{enumerate}


\subsection{Prosodic and verbal marking}


\ea 
\label{ex:8:6}

{Let us consider the case of reformulation when it is introduced by an intra-turn pause:}
{(e.g. 5,} 00 : 00 : 00 :12-00 : 00 : 13{)}{ Sequence of interaction opening} \textbf{}

\upshape\ttfamily
\parbox{8mm}{JUS}{<((très aigu)) coucou::>}\\
\parbox{8mm}{~}{  ‘<((very high-pitched)) hi::>’}\medskip

\parbox{8mm}{ALB}{bon:jou::r ((rires))}\\
\parbox{8mm}{~}{  ‘he:llo:: ((laughs))’}\medskip

\parbox{8mm}{JUS}{ [ça a été pour l'étage/]} {\textsl{(.)  tu t'es pas plantée}}{/}\\
\parbox{8mm}{~}{  ‘[was it okay for the floor/] (.) \textsl{you didn't get it wrong/}’}{}\\

\z 

The question \textit{tu t’es pas plantée/} ‘you didn't get it wrong/’, after a short intra-turn pause, is a specifying self-reformulation of the previous question. This reformulation anticipates and responds to a potential request for an explanation from the addressee [i.e. what do you mean\slash can you explain] following the imprecise wording of the first question.

\begin{sloppypar}
The intra-turn pause indicates the cognitive activity corresponding to the speaker's consideration and treatment of the discourse attributed to the interlocutor, to whom she responds using reformulation.
\end{sloppypar}

\section{Conclusion}

{For the description of dialogue, the specificity of verbal interaction implies:} 

\begin{itemize}
\item {An adaptation of the enunciative frame: the distinction between speaker and enunciator is of crucial importance in oral language to differentiate uttering instances from instances of deictic, syntactic and modal actualisation, which enables to analyse the purely spoken dimension of speech in verbal interaction and its dialogical functioning;}

\item {The study of specific enunciative features: interlocutive and anticipative interlocutive dialogism is inherent to the cohesion of discourse in verbal interactions; the enunciative level is also the level at} which{ associations or dissociations between participants are linguistically made (sharing of enunciators);}

\item {The specificity of dialogic markers: on the one hand, there are markers specific to the resources available for verbal interaction (prosody, multimodality of markers) and, on the other hand, the more complex functioning of prototypic syntactic markers, often described on the basis of monologic writing, in the case of dislocation.} 

\end{itemize}

{It can be assumed that speakers have an open-ended range of discursive skills, including an enunciative competence with respect to the discursive and non-situational functioning. The dialogic enunciative level has mainly been conceptually developed based on non-interactive discourses, but with some adjustments to redefine the concept, dialogism is obviously adapted to describe the interactions from which it is derived.}

\section*{Acknowledgements}
The authors are grateful to the ASLAN project (ANR-10-LABX-0081) of the Université de Lyon, for its financial support within the French program “Investments for the Future” operated by the National Research Agency (ANR).

{\sloppy\printbibliography[heading=subbibliography,notkeyword=this]}
\end{document}
