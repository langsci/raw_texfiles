\documentclass[output=paper]{langscibook} 
\ChapterDOI{10.5281/zenodo.6620105}
\author{Magali Ollagnier-Beldame\affiliation{UMR 5161 ICAR; Centre National de la Recherche Scientifique; Ecole Normale Supérieure; Université Lumière Lyon 2}}
\title{Introduction to epistemological views on complexity}
\abstract{In this first section of the book, the authors examine a selection of epistemological approaches provided by a complexity framework. They touch on the notions of boundaries between systems and ways in which systems interact, on the implications of different scales of analysis, and on the limits of the descriptions and models used in order to study phenomena. The authors argue for a plurality of approaches, in particular those that allow for integrating across theories and across methods within a transdisciplinary context or those that accept coexistence with knowledge other than the formal, with dimensions of history, and with subjectivity.}

\begin{document}
\maketitle 
%\shorttitlerunninghead{}%%use this for an abridged title in the page headers

\noindent The notion of complexity has not a precise and a formal definition, it is rather drawn from everyday language, even if the use of this notion is more and more frequent in research. According to \citet{Israel2005}, complexity is particularly resistant to a precise definition~because 1) it is often confused with the word “complication”, and 2) both terms are mostly used to mean the opposite of “simple”. The “incompressibility” of complex systems implies that they cannot be simplified, i.e. the representation of a complex system is as complex as the system itself \citep{Partanen2015}. Thus, for \textcite{Israel2005}, the notion of complexity is related to something possessing characteristics that are resistant to any attempt at simplification, and the topics of complexity are related to “a style of theorizing” rather than to “a specific subject matter”.

For \textcite{Partanen2015}, complex systems are by definition constituted through a large number of non-linear interactions and cannot be separated from their environment, this usually leading to a holistic approach of the studied phenomena (as also mentioned by \citeauthor{Israel2005}). This holistic approach to phenomena is the one Magali Ollagnier-Beldame claims to be the case in her text entitled \citetitle{chapters/03}. She defends the idea that for the study of the ways of knowing, a complex approach is necessary to better understand knowledge processes as it attempts to integrate and to go beyond the dichotomies that are most often used in cognitive sciences. She also supports the idea that such an approach must take into account subjective experience as a prism to challenge these dichotomies.

Complexity science provides a global philosophical foundation and combines different epistemological views.  It provides a shared framework for both objective and subjective positions in knowledge production. This project of the complex approach, to consider in a distinct but unique manner the objective and subjective positions in knowledge production, is pointed out in Kristine Lund’s text, entitled \citetitle{chapters/04}. In her text, the author explains how constructing knowledge with others is fundamental to all human activity. She presents the “Multi-grain” knowledge-building model, proposing a framework that allows systems within different disciplines to “speak” to each other and defines the space in which explanatory models can be proposed about the different forms of knowledge co-construction. In her paper, methods used for study include both emic and etic approaches \citep{Headland1990,Pike1967} and illustrate how they can be combined through the articulation of different levels of analysis. The study proposed by Lund mobilizes intermediate variables for the study of multifactorial phenomena appearing within cognitive, linguistic, interactional, and social systems, thus additionally arguing for viewing complex behaviour as a system of interrelated systems \citep{Levinson2005}. This raises questions of scale in the study of phenomena, which is an important issue in the study of complex systems. Indeed, \citet{Partanen2015} reminds us that, since there are no absolute boundaries in the universe (except for perhaps some fundamental strings), the question to derive knowledge – and especially from different scales -- of a system if no actual “system” exists is a crucial one. Despite the non-existence of boundaries, this author claims that we can assume that certain relatively resilient and stable temporary structures or patterns emerge. These can be treated as if they had a “limited existence”, as if they almost existed \citep{Richardson2005}, or were substantially real \citep{Partanen2015}. In practice a certain reduction (temporary “closing” of the system) is often needed to enable any research. The system must be defined, or framed for description – “separated” temporarily from the environment which it is inherently a part of. Indeed, as mentioned by Pierluigi Basso Fossali in his chapter entitled \citetitle{chapters/02}, a theory of complexity is a kind of assumed reductionism, each theoretical practice necessarily having to “decomplex” relations with its environment to enable research. These issues concerning boundaries and limits raise some questions of empirical data analysis, and can be partially solved by software (including visualization and simulation), playing the role of macroscope not producing knowledge as such, since for data to become knowledge, meaning given by a human is required.

The question of passages, related to the notion of scaling, is also mentioned by Basso Fossali. In his text, he shows that the crucial epistemological problem is not the study of complex systems, but that of complex behaviours \citep{Prigogine1983} -- as Lund also points out in her proposal -- which forces us to recognize the existence of “qualitative passages” that require a conceptualization not only beyond the description languages ​​already mastered, but transversally to the hierarchies of relevance plans. For Basso Fossali, this requires clarifying the roles of semiotic mediations, whose purpose is to ensure the correlation of value domains with autonomous organizations and dynamism, in the same way as the “multi-grain” knowledge-building model presented by Lund. He proposes a series of guiding principles for these qualitative passages (analogical approach, recursivity, emergentism) as well as two possible movements for these passages: an opening movement and a circular movement. This focus on complex behaviours echoes the work of \citet{BecknerEtAl2009} who present the language as a complex adaptive system, using conventions -- as regularities of behaviours -- to support communication as a joint action, and allowing the building of human culture.

As Basso Fossali reminds us in his text, complexity is pluralizing in terms of approaches, asking for transdisciplinarity and recognition of the limit-points of the languages ​​of descriptions and models used. Thus, typically mixed methods are used in the study of complex systems, implying a combination of quantitative and qualitative approaches from different epistemological foundations. Faced with the pluralization of approaches, the science of complexity, to get stronger,  should aim at adopting an integrative point of view and thus accept the coexistence and collaboration with other forms of knowledge having a different nature to formal knowledge, and that are essential, especially in order to account for the dimension of historical time and subjectivity \citep{Israel2005}. Taking into account the experiential dimension of phenomena is defended by Ollagnier-Beldame in her text, which proposes an approach considering the experiential dimension of the act of knowing, both as being where we go from and what we have to bond to in return \citep[26]{Varela2017}.

\section*{Acknowledgements}
The authors are grateful to the ASLAN project (ANR-10-LABX-0081) of the Université de Lyon, for its financial support within the French program “Investments for the Future” operated by the National Research Agency (ANR).

{\sloppy\printbibliography[heading=subbibliography,notkeyword=this]}
\end{document} 
