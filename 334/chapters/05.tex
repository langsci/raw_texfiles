\documentclass[output=paper]{langscibook} 
\ChapterDOI{10.5281/zenodo.6620113}
\author{Pierluigi {Basso Fossali}\affiliation{UMR 5161 ICAR; Centre National de la Recherche Scientifique; Ecole Normale Supérieure; Université Lumière Lyon 2}}
\title{Introduction to complexity, pragmatics and discourse}
\abstract{This second part of the book is introduced by the notion of cultural semiospheres and how they lead to a system of double binds that inscribe paradoxes in the society in which culture is expressed. In the linguistic field, paradoxes are illustrated in paradigmatic, syntagmatic, and semantic terms. All identified forms of complexity occur in discourse practices, for example: structural, mereological, stochastic, and complexity in terms of scale (prospective or even political). Complexity is a mapping of the lines of resistance and of the reception boundaries of a cultural community coupled with its semiosphere. Aligning with this characterisation of discourse, the authors of this part propose a pragma-enunciative theory of points of view with a simplex approach, rhythmic synchronisation of enunciation as a complex system, dialogism for daily interaction, and a complex system as modalities within a written interaction.}
\IfFileExists{../localcommands.tex}{
  \addbibresource{localbibliography.bib}
  \usepackage{langsci-optional}
\usepackage{langsci-gb4e}
\usepackage{langsci-lgr}

\usepackage{listings}
\lstset{basicstyle=\ttfamily,tabsize=2,breaklines=true}

%added by author
% \usepackage{tipa}
\usepackage{multirow}
\graphicspath{{figures/}}
\usepackage{langsci-branding}

  
\newcommand{\sent}{\enumsentence}
\newcommand{\sents}{\eenumsentence}
\let\citeasnoun\citet

\renewcommand{\lsCoverTitleFont}[1]{\sffamily\addfontfeatures{Scale=MatchUppercase}\fontsize{44pt}{16mm}\selectfont #1}
   
  %% hyphenation points for line breaks
%% Normally, automatic hyphenation in LaTeX is very good
%% If a word is mis-hyphenated, add it to this file
%%
%% add information to TeX file before \begin{document} with:
%% %% hyphenation points for line breaks
%% Normally, automatic hyphenation in LaTeX is very good
%% If a word is mis-hyphenated, add it to this file
%%
%% add information to TeX file before \begin{document} with:
%% %% hyphenation points for line breaks
%% Normally, automatic hyphenation in LaTeX is very good
%% If a word is mis-hyphenated, add it to this file
%%
%% add information to TeX file before \begin{document} with:
%% \include{localhyphenation}
\hyphenation{
affri-ca-te
affri-ca-tes
an-no-tated
com-ple-ments
com-po-si-tio-na-li-ty
non-com-po-si-tio-na-li-ty
Gon-zá-lez
out-side
Ri-chárd
se-man-tics
STREU-SLE
Tie-de-mann
}
\hyphenation{
affri-ca-te
affri-ca-tes
an-no-tated
com-ple-ments
com-po-si-tio-na-li-ty
non-com-po-si-tio-na-li-ty
Gon-zá-lez
out-side
Ri-chárd
se-man-tics
STREU-SLE
Tie-de-mann
}
\hyphenation{
affri-ca-te
affri-ca-tes
an-no-tated
com-ple-ments
com-po-si-tio-na-li-ty
non-com-po-si-tio-na-li-ty
Gon-zá-lez
out-side
Ri-chárd
se-man-tics
STREU-SLE
Tie-de-mann
} 
  \togglepaper[1]%%chapternumber
}{}

\begin{document}
\maketitle 
%\shorttitlerunninghead{}%%use this for an abridged title in the page headers


The Russian semiotician Yuri Lotman was one of the first to integrate a systems theory into language sciences \citep[215]{Lotman2005}. He considered each cultural \textit{semiosphere} as a dialectics between differentiation and integration; “each of its parts creates its own whole, isolated in its structural independence. Its connections with other parts are complex and are characterised by a high level of deautomatisation”.

Indeed, the problem on which complexity theory focuses is the non-trivial behaviour of systems (deautomatisation). On the one hand, a culture commits to promoting devices; on the other hand, it must continue to ensure forms of emancipation from them. This double commitment can be claimed because in most cases discursive practices are communications that seek to mediate contradictory constraints arising from “totalities” which are certainly independent but nevertheless connected by recursive inter-observation. In fact, these totalities are institutions of meaning that observe each other while they observe the others' operations, which brings out the interpenetration of their fields of action, despite the attempt to define distinct meaning jurisdictions. Thus, culture reveals to be a system of double binds that inscribe paradoxes in the society in which it is expressed. In the linguistic field, this has repercussions, on the double requirement of guaranteeing \textit{varieties} and \textit{redundancies} in paradigmatic terms \citep{Atlan1972}, on the constant compromise between \textit{agreement} and \textit{discontinuity} in syntagmatic terms (cf. \citealt{Harris1945,Missire2010}), and on the tension between \textit{implication} and \textit{concession} in semantic terms \citep{Zilberberg2012}.

On the other hand, the inter-observation of totalities, the overlapping of meaning jurisdictions, and conflictual or alternative norms are general issues which have specific repercussions on discourse practices, even if the latter seek to reduce this overabundant complexity. Knowing how to dose complexity constitutes the abstract prototype of virtues, which are normally acquired through training. Indeed, manageable complexity only reflects a reasonable compartmentalisation of the major complexity of the reference environment. A proportion is sought between complexity within the system and environmental complexity, but eventually complexity managed through acquired ease becomes transparent while complexity that is unrelated to the adopted linguistic techniques manifests itself as a negative scope of our hope to find later order. Thus, interactional complexity has the means to develop a double reduction of \textit{both} the prophecy of order in the system (omnipotent unilateralism) \textit{and} the disproportionate nature of the environment’s random complexity (force exerted by local contingencies). In this modal frame, which is built by the reciprocal delimitation of ambitions and prostrations, interaction frees the meaning from such neuroses in order to restore an open significance within linguistic games that allow us to assume or claim the values negotiated during the communicative exchange. Deautomatisation, in relation to laws and environmental dependencies, builds an inter-world \textit{of} exchanges (self-referential\slash hetero-referential communications) and \textit{for} exchanges (import\slash export). 

\begin{sloppypar}
Semiotic mediations should simultaneously ensure metastable balances, which are internal to the systems, and imbalances that are profitable when passing through the environment. But this double performance can only be ensured through the reproduction of the same dualistic logic, where languages must appear both as systems and as an environment. It can be hypothesised that this double performance is only achieved through the shift from uniqueness to plurality and from self-reference to interpenetration; in short, linguistic culture would be a kind of internal dialectics between singular languages and semiospheres. The heterogeneity of languages would then correspond to the maximal tension of culture in its mimesis of the environment’s phenomenal complexity; this mimesis is reductionist and multi-criterial but it is at least regulatively shareable (all linguistic items are anthropic, all discursive items are potential communication, even in the case of the unavailability of translation manuals). 
\end{sloppypar}

This mimesis, which enables the modelling of the phenomenality of environmental complexity, is the paradoxical result of anti-economy in languages (Babel as the mapping of \textit{alternities of being} -- to use George Steiner's expression). This symbolic expenditure (multiplication of linguistic mediations) allows discursive “postures” (self-descriptions) to be at the same time an enunciative stability (point of view) and a mimesis of overexposure to the environment (improbability of inter-subjective interpretation). Every semiotic mimesis of complexity is characterised by this in-between feature, which leaves speakers in the double, albeit contradictory, requirement to take some distance (to engage in discourse through avatars, which allows, for example, for irony) and to make themselves available to become the responsible and sensitive embodiment of meaning management (for example, laughing after the ironic assertion so as to reintroduce the proper body).

Discourse practices involve different levels of language organisation, in a proportion that is adapted to the regime of interaction. This is not limited to a simple use of expressive resources: on the one hand, \textit{language} also constitutes a reference environment for the singular organisation of discourse that is locally established; on the other hand, it can recognise other semiotic systems (syncretisms) and other sources of values (perception, memory, institution, etc.). The interpenetration of the systemic horizons of reference, the change in the statutes of the organisations used and the impossibility of crystallising hierarchies and embedding, can already be seen as convincing clues of the complexity of discursive practices. If the anticipation of an enunciative production is sufficiently questionable, despite the norms and routines that characterise discursive praxis, predictions about its interpretation and its perlocutionary effects are all the more unlikely.

In short, in spontaneous and largely improvised interactions, two instances confront each other, which connect various reference spaces/times, appeal to different language games, and add to socialised scores (interaction frameworks) random complements or totally unpredictable detours from norms. All the forms of complexity identified are attested in discourse practices: deterministic (structural) complexity, which starts from a junction of rules and ends up reaching discontinuity points; aggregation (mereological) complexity, through the plurality of enunciative instances that are formed and dissolved; random (stochastic) complexity, for the qualitative divergence between the organisations used and the unpredictable emergence of forms without a strategic paternity; complexity in terms of  (prospective, or even political) scale, because discourses are always disproportionate to their ambitions and negotiate a purpose that is finally acceptable, even favourable. These four dimensions, although not exhaustive, illustrate that complexity is a mapping of the lines of resistance and of the reception boundaries of a cultural community coupled with its semiosphere. 

It is no surprise that complexity theory is associated with the ecological paradigm of culture (cf. \citetv{chapters/07}); indeed, ecology is rooted in a rather basic aspect of experience: perception, enunciation and interpretation can only select a limited number of values (economy) in relation to the density of the (respectively) sensitive environment, the historical-social frame and the textual space. However, the intended meaning must also meet all the demands of the space in which it is included (dependency extends the grasping of values). Ecology shows a kind of double constraint on meaning: choosing to make a difference and accepting to be overwhelmed by the complexity of the environment, which is how the coupling between subjects and their ecosystem is qualified. 

Speakers are also confronted with another environment: their proprioception and psychology. Language sciences have increasingly reintroduced the notion of body among the fundamental instances involved in the constitution of linguistic forms, the exercise of semiosis, and more generally in discourse practices. The problem is the same: directing meaning, imposing an ecological behaviour knowing that other competing inner “voices” will be listened to.\largerpage

Taking complexity into account may however be a risk as it can lead to presenting an unmanageable framework for theoretical practices, which have difficulty dealing with indeterminate aspects and cannot be satisfied with solutions whose relative effectiveness remains unclear. This is why the enumeration of interconnections, recursive phenomena and gaps in meaning must be counterbalanced by a strategy enabling theoreticians to \textit{simplexly account for complex} facts (cf. \citetv{chapters/06}). This strategy, suggested by Alain Berthoz's research, seems to be well suited to language sciences as soon as the field accepts the need to position itself at the level of discursive practices themselves. Linguistic mediations, like the enunciative building of points of view, are an example of the management of self-ascribed or attributed modalities, which determine an inter-subjective and simplex frame of reference that goes beyond the clouds of cognitive and affective clues as well as the impenetrability of psychological states.

Modalities emerge both as an epicentre of complexity requiring here to to be managed discursively, and as an exemplary field for the observation of simplex mediations. On the one hand, modalities open up possible worlds; on the other hand, they find verbal or iconic manifestations that paradoxically become clearer and more stable thanks to their dialogical dimension (cf. \citetv{chapters/09}) and to the modalities that are already operational in a given institutional space for the interacting parties. 

The complexity of exchanges emerges as non-linearity in the process of interlocution, where turns of speech constantly redesign the treatment of values and modal implications in concentrated or diffuse formats, in individualising or collectivising regimes, in retrospective or anticipatory postures (\citetv{chapters/08}). This non-linearity leads to a kind of “dissociative complementarity” of contributions to the conversational texture, in which one can distinguish different critical positions on the common and strategic background of an interlocutory dialogue; on the other hand, non-linearity brings out, through the encounter between several contingencies, true forms of exchange, with a complex organisation in terms of the distribution and interdependence of roles, the rate of turns of speech, and the management of conversational pressures and thematic focus. The bonding of expectations (anticipations) and the rhythmic pattern of dissociative movements (choices) must lead to the emergence of a reliable treatment of indeterminacy: (i) an institutionalised organisation, (ii) trust based on a relational history, (iii) a belief that is freed from contingencies, (iv) a plastic and extemporaneous harmonisation.

Non-linearity is strongly related to the circularity of double contingency (\citealt{Luhmann1984}: §3), because the treatment of indeterminacy, which has provided a stable regime to a discursive practice, becomes the element that must be confirmed in a more effective and conclusive version, which can only make it subject to risks of failure (e.g., can an initial belief become loyalty or a concessive harmonisation become a sustainable agreement?).

Complexity shows us which road was taken by societies that do not accept to be confined in coded systems, therefore meaning cannot be equivalent to law. On the contrary, the latter is a necessary organising principle in order to deal with the desire to leave the doors open to contingency: a contingency given by the presence of otherness, of the foreigner, by some possibilities that are not yet mapped by the systems used by the actors involved in the encounter. Possibility must move from system (virtuality) to environment, where contingency factors can bring out other sensible reasons for developing values and destinies.\largerpage

Although a very wide range of language games are known in advance, the field of language practice is in constant building because comprehension cannot be reduced to the categorisation and subsumption of occurrences (\textit{tokens)} within standardised classes (\textit{types}). This is why an epistemology of language sciences can give importance to the notion of narrativity (\citetv{chapters/07}), which is ultimately the framework, full of explanatory heterogeneities and gaps in meaning, of a series of \textit{impossible successful communications}. Forms of life are determined through local acquired skills and unpredictable encounters, and semiotic mediations are used to weave the experience of contingency, which explains the dynamisation of meaning. In this regard, each account or reformulation, with the recursive introduction of implicit or explicit enunciators and phrasal embedding (\citetv{chapters/08}), is a search for new distinctive determinations or productive improbabilities. Despite their baroque cycles, complex things can confine meaning to tautology; on the other hand, complexity even redetermines the reasons for restraining indeterminacy throughout the organisation.

Ratifying a conversation after someone's opening gesture is a way to both speaker and listener chances to reconsider identities, positions, ambitions, and vulnerabilities. At the same time, dialogue with a stranger is feared because it is well known that the productivity of the conversation is not predictable (what connection? what implication? what conclusion?) and that the motivations for conversational incursion may not be proportional to their effects.

Empathy, mediated through the discursive interweaving of points of view (\citetv{chapters/06}), is an enunciative tension faced with an improbable grasp (penetrating the perception, thought, experience of others); but at the same time, it is the perfect example of a retroaction circuit that tests how each projection of affective and cognitive simulacrum influences itself: the conditions of my empathetic disposition are redesigned by the others’ sensitivity to my sensitisation, by affecting it later or cooling it.

Fundamentally, the symbolic intervenes both as a treatment of this circular causality (on the reflexive level) and as a treatment of causes that cannot be isolated in a context composed of a series of other innumerable causes, and which are sometimes of an indeterminate and random nature (on the transitive level). But it is clear that the result of symbolic remediation only corresponds to the dramatisation of a responsible complexity (subjective system) and of a stochastic complexity (environment); this is why analogy, iconic transpositions, and games of forms can be used as an attempt to reduce the distance between these two types of complexity in order to consider the \textit{re-entry} of one over the other. Paraphrasing Prigogine's lesson, Lucien Sève described this performance as that of a complexity theory applying its stakes to itself: “On the one hand, [...] there is \textit{no chance without law}: contingency is permeated with necessity to its core. On the other hand, there is \textit{no law without chance}: necessity only manifests itself within contingency” (\citealt{Sève2005}:  186, personal translation).

\section*{Acknowledgements}
The author is grateful to the ASLAN project (ANR-10-LABX-0081) of the Université de Lyon, for its financial support within the French program “Investments for the Future” operated by the National Research Agency (ANR).

{\sloppy\printbibliography[heading=subbibliography,notkeyword=this]}
\end{document}
