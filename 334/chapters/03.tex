\documentclass[output=paper]{langscibook} 
\ChapterDOI{10.5281/zenodo.6620109}
\author{Magali Ollagnier-Beldame\affiliation{UMR 5161 ICAR; Centre National de la Recherche Scientifique; Ecole Normale Supérieure; Université Lumière Lyon 2}}
\title[What knowledge owes to experience]
      {What knowledge owes to experience: Complexity and first-person epistemology}
\abstract{Definitions of “complexity” or “complex systems” are numerous, and research concerns both modelling and empirical works. Our proposal is part reflexive in which the complexity sciences question their foundations and their methodologies. We postulate that a complex approach is necessary to better understand knowledge processes as it attempts to integrate and to go beyond the dichotomies that are most often used in cognitive sciences. We also support the idea that such an approach must take into account the experiential dimension of the act of knowing, considered as a starting and an ending point. \citep{Varela2017}.}
\IfFileExists{../localcommands.tex}{
  \addbibresource{localbibliography.bib}
  \usepackage{langsci-optional}
\usepackage{langsci-gb4e}
\usepackage{langsci-lgr}

\usepackage{listings}
\lstset{basicstyle=\ttfamily,tabsize=2,breaklines=true}

%added by author
% \usepackage{tipa}
\usepackage{multirow}
\graphicspath{{figures/}}
\usepackage{langsci-branding}

  
\newcommand{\sent}{\enumsentence}
\newcommand{\sents}{\eenumsentence}
\let\citeasnoun\citet

\renewcommand{\lsCoverTitleFont}[1]{\sffamily\addfontfeatures{Scale=MatchUppercase}\fontsize{44pt}{16mm}\selectfont #1}
   
  %% hyphenation points for line breaks
%% Normally, automatic hyphenation in LaTeX is very good
%% If a word is mis-hyphenated, add it to this file
%%
%% add information to TeX file before \begin{document} with:
%% %% hyphenation points for line breaks
%% Normally, automatic hyphenation in LaTeX is very good
%% If a word is mis-hyphenated, add it to this file
%%
%% add information to TeX file before \begin{document} with:
%% %% hyphenation points for line breaks
%% Normally, automatic hyphenation in LaTeX is very good
%% If a word is mis-hyphenated, add it to this file
%%
%% add information to TeX file before \begin{document} with:
%% \include{localhyphenation}
\hyphenation{
affri-ca-te
affri-ca-tes
an-no-tated
com-ple-ments
com-po-si-tio-na-li-ty
non-com-po-si-tio-na-li-ty
Gon-zá-lez
out-side
Ri-chárd
se-man-tics
STREU-SLE
Tie-de-mann
}
\hyphenation{
affri-ca-te
affri-ca-tes
an-no-tated
com-ple-ments
com-po-si-tio-na-li-ty
non-com-po-si-tio-na-li-ty
Gon-zá-lez
out-side
Ri-chárd
se-man-tics
STREU-SLE
Tie-de-mann
}
\hyphenation{
affri-ca-te
affri-ca-tes
an-no-tated
com-ple-ments
com-po-si-tio-na-li-ty
non-com-po-si-tio-na-li-ty
Gon-zá-lez
out-side
Ri-chárd
se-man-tics
STREU-SLE
Tie-de-mann
} 
  \togglepaper[1]%%chapternumber
}{}

\begin{document}
\maketitle 

\section{Introduction}

Research in the field of “complex systems” is vast and it concerns as much modelling as empirical studies. \citet{BertinEtAl2011} define the complexity sciences not by their objects but by their way of questioning objects. They associate complexity with the observer’s perspective on her object, classifying complexity research in three categories: 

\begin{itemize}\sloppy
\item Studies of specific systems within a discipline or at the intersection of several disciplines
\item Transverse theoretical studies on the generic characteristics of certain classes of systems
\item Reflexive works in which the complexity sciences question their foundations and their methodologies. 
\end{itemize}

Our proposal is part of the third group. In order to integrate and to go beyond the dichotomies most often used in cognitive sciences, we claim a complex approach is necessary to better understand knowledge processes as they unfold. We also support the idea that such an approach must take into account the experiential dimension of the act of knowing, both as 

\begin{quote}
being where we go from and what we have to bond to in return \citep[26]{Varela2017}.
\end{quote}

\section{Ways of knowing and the paradigm of 4-E cognition}

Over the last thirty years, research within the cognitive sciences has massively grown in the field of the Ways of Knowing (\citealt{Suchman1987}, \citealt{VarelaEtAl1991}, \citealt{Hutchins1995}, \citealt{Clark1997}). In recent years, the paradigm of 4-E cognition (for embodied, embedded, enactive and extended cognition) suggests that cognition involves the whole body, as well as the situation of the body in the environment \citep{NewendeBruinGallagher2018}. The term “embodied” is actually the most general term, encompassing the other three. Indeed, the idea is to take into account the way the body contributes to cognitive processes, body being always located in a physical, social and cultural environment (embedded), body allowing us to perceive our environment according to its “affordances” (enactive), and body mobilising objects and instruments of the environment (extended). Studies within this paradigm rely on different methodologies but they all reject or radically reconfigure traditional cognitivism considering cognition as a manipulation of representations in the brain. The principle of cognition rooted in an embodied subject, situated into a particular setting, is the foundation of the theory of enaction \citep{VarelaEtAl1991}. This theory considers cognition as an “embodied action”, i.e. as a phenomenon rooted in the constant interactions between the subject and her environment and by which these two realities co-occur. And while the aim of research within the 4-E cognition paradigm is to understand Human “in situation”, it mobilises various dualisms (subject vs. object, action vs. cognition, interior vs. exterior, body vs. psyche, innate vs. acquired) -- often seen as contradictions and considered by science as antagonisms or aporias -- with a view to their integration.

\section{Complexity and first-person science}

According to \citet{Morin1990Introduction}, the non-integration of these contradictions leads to a “blind intelligence”, a simplifying scientific thought that disfigures and mutilates reality through disjunctions (separating what is linked), reductions (unifying what is diverse) and abstractions (isolating objects from their environment). According to him, on the contrary, it is necessary to move on to a complex generalised thought -- in the etymological sense of \textit{complexus}: “what is woven together” -- integrating these contradictions and linking what seems to be opposed. In the same vein, \citet{VarelaEtAl1991} and \citet{DeprazEtAl2003} argue for taking into account these dualities, without seeking to go beyond them by synthesising them, but recognising them the possibility to co-occur. For them, subjective experience, as a research material, is the possible place where these dualities can be integrated because, owing to its “intermediate” nature,  experience questions the relevance of these dualities. In this direction, we defend the need for a new paradigm for the study of the Ways of Knowing rehabilitating and reintegrating lived experience at the core of the process (\citealt{PetitmenginEtAl2015}).

\begin{sloppypar}
Indeed, the irreducible complexity of cognitive processes needs an open interrogation method, considering their phenomenal manifestations, i.e. the way they are from a “first-person” -- subjective -- perspective. Phenomenological approaches differentiate first-person, second-person and third-person points of view distinguishing the perspective of the subject living the experience from that of another subject, e.g. the researcher (\citealt{DeprazEtAl2003}). The first-person point of view deals with the experience as it is accessed by the subject. In the case where the researcher collects data about her own experience, the “radically first-person” point of view denotes the idea that the data are drawn from the researcher’s own lived experience. The second-person perspective implies 
\end{sloppypar}
\begin{quote}
enabling the gathering of “first-person” data, i.e., data that express the viewpoint of the subject herself, in the grammatical form “I...”. But since the data have been gathered through another person (a “You”), the method has been dubbed “second-person” \citep[230--231]{Petitmengin2006}.
\end{quote}

This perspective suggests a combination of empathic resonance and heterophenomenological observation \citep{Depraz2012}\footnote{According to \citet[419]{Depraz2012}, the heterophenomenological observation is that of the researcher who studies the experience of another subject without identifying with it.}, i.e. an inference from behaviours – language, gestures or other forms of semiosis. The idea is that a second-person perspective is an indirect point of view on the subjective perspective. The first-person and second-person points of view rely on a first-person epistemology that considers subjectivity as it is experienced by the subject herself (\citealt{VarelaShear1999}, \citealt{Depraz2014}).\footnote{They are defined this way as opposed to the third-person point of view that does not allow studying the experience as it focuses on behaviours and examines them according to predefined categories. This point of view implies a third-person epistemology in which subjectivity and lived experience are generally viewed as epiphenomena or as being beyond the reach of science \citep{Vermersch2000a}.} Such an epistemology is often undervalued in comparison to third-person approaches, on the assumption that an external point of view offers greater objectivity. The limits of this last statement have however been stressed, and the epistemic validity of first-person approaches has been analysed in detail (\citealt{PetitmenginBitbol2009}). Especially, claims denying subjects’ introspective abilities (\citealt{NisbettWilson1977}) have been rebutted (\citealt{PetitmenginBitbol2009, PetitmenginEtAl2013}). Moreover, external observations based on third-person epistemology leave aside entire facets of the studied phenomenon, which simply cannot be accessed since they occur “within” individuals, “behind” physical movements and “in front” of patterns from neuronal imagery. These classically inaccessible facets of the subject’s activity, her experience, can, however, be reported by her; hence, the benefits of accompanying her to do so with a particular method. As for reducing introspection to solipsism \citep[10]{Zahavi2017}, we contend that gathering authentic descriptions of lived experiences is the first and necessary step to ground our research in the things themselves and access the invariant structure of experience (\citealt[36]{BitbolPetitmengin2011}). As we wrote in \citet{PetitmenginEtAl2015}, once a corpus of singular descriptions of experiences has been collected, a whole work of reorganisation, analysis and formalisation is necessary in order 1. to identify the possible structure of the described experiences, i.e. a network of relationships between descriptive categories, independently from the experiential content, and 2. to detect any generic structures, progressively extracted from the initial descriptions thanks to a succession of operations of abstraction (\citealt{Ollagnier-BeldameCoupe2019}, \citealt{PetitmenginEtAl2018}, \citealt{ValenzuelaRosati2019}).

\section{Micro-phenomenology as a way to explore the richness of experience}

At the heart of first-person epistemology, micro-phenomenology (\citealt{Petitmengin2006}, \citealt{BitbolPetitmengin2016}) is based on enaction and neurophenomenology \citep{Varela1996}. It is close to Morin's co-constructivist vision of “human facts” \citep{Morin1990a} for which the subject is constructed by the outer world at the same time as the outer world is constructed by the subject, by a recursive loop. Interested in “what it is like to be” \citep{Nagel1974}, the micro-phenomenology seeks precisely to understand the complexity of the human experience with an emic approach, i.e. using data from the “discourse” of subjects as opposed to an etic approach, i.e. using observational data (\citealt{OlivierdeSardan1998}).

This discourse is not a narration but relies on what \citet{Vermersch1994} calls the “embodied posture of speech” (EPS), or “evocation”, which is the verbalisation of experience in close contact with it. The “level of experiencing” \citep{Hendricks2001} is the degree of connection between what a person is saying and her experience when she says it. It is a quantifiable first-person process: there are low, medium and high levels of experiencing. The micro{}-phenomenological interview aims at a high level of experiencing in order to facilitate the experiential description. The experiencing scale \citep{Hendricks2009}, which measures this process, is the third-person evaluation of a first-person process, based on specific linguistic and somatic indicators.  

The micro-phenomenological interview relies on the explicitation interview, developed by \citet{Vermersch2012} and \citet{Petitmengin2006}.\footnote{It has been supplemented by a method of data analysis and validation \citep{PetitmenginEtAl2015,PetitmenginEtAl2018}.} It consists in “guided retrospective introspections”, aiming at accompanying an interviewee in recalling a past situation. It does not, however, guide the subject on the content she verbalises, which comes to her consciousness through a movement of letting go. This is possible thanks to a specific posture from the interviewer guiding the interviewee’s attention with open and non-inductive questions but never inducing the content of what the latter says. During this movement, the interviewee is accompanied by the interviewer to suspend her judgment – the Husserlian \textit{epoché} (\citealt{DeprazEtAl2003})\footnote{According to (\citealt{DeprazEtAl2003}: 25), one accomplishes the \textit{epoché} in three principal phases: A0: Suspending your “realist” prejudice that what appears to you is truly the state of the world; this is the only way you can change the way you pay attention to your own lived experience; in other words, you must break with the “natural attitude.” A1: Redirecting your attention from the “exterior” to the “interior.”
A2: Letting-go or accepting your experience.} –, which allows her to access her past lived experience. The main characteristics of the explicitation interview are: 

\begin{enumerate}

\item The EPS within the interviewee, allowing her to initiate and to maintain an intimate contact with the evoked past situation; 

\item The concept of “satellites of action”, to help the interviewer be aware of the area of verbalisation to which the interviewee is referring to and to drive the interviewee's attention according to these areas; 

\item The contact with a singular past situation (unique in time and space) in order to collect specific descriptions rather than generalisations (such as know-how or habits); 

\item The holistic description of the lived experience in the entanglement of its cognitive and bodily dimensions, beyond the division; 

\item The precise use of perlocutionary effects \citep{Austin1962};

\item The consideration of the temporality of the experience, carefully explored by the interviewer who is guiding the fragmentation of the interviewee experience into a series of very detailed phenomena through specific questions. 

\end{enumerate}

Like the explicitation interview, the micro-phenomenological interview aims to describe the experience in an intuitive mode (as opposed to a signitive, purely conceptual mode), i.e. based on a presentification of the past moment.

It is important to bear in mind that first-person epistemology is not an epistemology of immediacy since experience, although lived by the subject, is not immediately known by her, despite its apparent transparency and familiarity. Experience is not directly accessible to the subject and the first-person perspective should not be confused with immediate donation, i.e., for the subject, a sudden, clear and distinct illumination \citep{Vermersch2000}. 

\begin{quote}Indeed, being epistemically related to facts about oneself is not a sufficient condition for first-person perspective taking: You can also have an objective, third-person view on your headache. […] What is needed is a difference not in terms of the epistemic object but, rather, in terms of epistemic access – even if it may turn out to be necessary to refer to specific epistemic objects in order to clarify what the specific kind of access is. The decisive point seems to be that there are certain features of oneself that do require a specific kind of epistemic access \citep[37--38]{Pauen2012}.
\end{quote}

It is precisely an epistemic access to experience that is both unusual and privileged that the micro-phenomenological interview offers.


\section{Towards a dialogical integration of emic and etic data}

Micro-phenomenological descriptions can be enriched with third-person data, for instance as \citet{DeprazEtAl2017} did, crossing micro-phenomenological data with third-person physiological data, in their study about the experience of surprise and depression. Precisely, it seems that a science questionning the cognitive processes with a complex approach must consider the relation between etic and emic data according to a dialogic mode. This is what \citet{Varela2017} proposes when he distinguishes two main classes of scientific invariants. First, he specifies objective invariants (derived from third-person or etic research), which can be treated as if they were separable from the variety of experiences of which they are the focus\footnote{Regarding this point, Varela warns about the widespread temptation to forget the status of scientific objects, which is that of inter-situational invariants, as well as their origin, which is part of a situated experience. In a 1996 article, he had already stressed this temptation and the need to preserve it by recalling that third-person researches, as much as first-person ones, are made by people who are embodied in their social and natural world.}. He also characterises intersubjective invariants (coming from first-person or emic research), which are different from objective invariants, by the fact that it is generally impossible to separate them from the situations and concrete people they coordinate \citep{Varela2017}. The project of neurophenomenology (ibid.), which micro-phenomenology joins, is precisely to define “generative mutual constraints” between objective and intersubjective invariants. From these two classes of invariants, the idea is to go back and forth between third and first person-person approaches, experience being from where one leaves and to which everything must be connected in return. This project is based on methods of guiding research towards invariants belonging to one of the two classes, relying on the invariants of the other class. 

\section{Conclusion}\largerpage

A first-person approach as described above aims to enrich the understanding of the Ways of Knowing in their complexity -- particularly by seeking to re-question classical dichotomies -- through the reintegration of subjective experience. It is in line with \citet{BerthozPetit2014} for whom the modelling of reality complexity is always far from the complexity as it is lived in flesh and bone, and for whom 

\begin{quote}
the Living will always have the priority by to all models because it is the one who lives in the immanence of this real complexity (ibid.: 37). 
\end{quote}

\begin{sloppypar}
Combined with a third-person approach, it can allow for the construction of new models of cognitive processes, more operative than models made solely from third-person data, which can not grasp the phenomenological dimension in which the living “dwells” and “decides” within the concret and embodied complexity.
\end{sloppypar}

Finally, we claim a holistic approach to cognitive processes fully considering human subjectivity through experience, beyond the normativity of subjectivity. To achieve this goal, it seems essential to broaden our conception of the subject and the world and to accept to be obedient to the authority of the Living in its dynamic and processual dimensions, and in its contradictions. The challenge of such a proposal is not only epistemological, but also ethical and societal. Indeed, what gives the person the feeling of being the person she is, if not her experience? Is not experience “the only tangible reality of the person, at the forefront of the living being alive” \citep{Lamboy2003}? Leaving aside lived experience, does not that mean depriving oneself of a considerable source of knowledge?\footnote{“Indeed nothing is as debilitating as a confused and distant functioning of experiencing” \citep[15--16]{Gendlin1962}.} Reintegrating the Living into the study of cognitive processes would help develop an ecosystemic vision of the person as a complex unit that shapes in the relationship between her organism and the environment. This vision also fosters an understanding of the relationship between humans and nature, allowing for both non-dissociation and differentiation, thus bringing together the seemingly contradictory notions of belonging and autonomy. Could we consider that the rehabilitation of subjective experience, as a material for research, could be a form of simplexity \citep{Perrier2014, BerthozPetit2014}?\footnote{Inasmuch as it is one of the fundamental properties of the Living to be able to invent simple answers to problems that the complexity of reality poses for its survival.}

\section*{Acknowledgements}
The author is grateful to the ASLAN project (ANR-10-LABX-0081) of the Université de Lyon, for its financial support within the French program “Investments for the Future” operated by the National Research Agency (ANR).

{\sloppy\printbibliography[heading=subbibliography, notkeyword=this]}
\end{document}
