\documentclass[output=paper]{langscibook} 
\ChapterDOI{10.5281/zenodo.6620125}
\author{Claire Polo\affiliation{Laboratoire Education Culture Politiques – ECP; Université Lumière Lyon 2}
        and Kristine Lund\affiliation{UMR 5161 ICAR; Centre National de la Recherche Scientifique; Ecole Normale Supérieure; Université Lumière Lyon 2}
        and Christian Plantin\affiliation{UMR 5161 ICAR; Centre National de la Recherche Scientifique; Ecole Normale Supérieure; Université Lumière Lyon 2}
        and Gerald P. Niccolai\affiliation{UMR 5161 ICAR; Centre National de la Recherche Scientifique; Ecole Normale Supérieure; Université Lumière Lyon 2}}
\title{Collective reasoning as the alignment of self-identity footings}
\abstract{Assessing student interactions during socio-scientific debates requires an interdisciplinary theoretical background involving linguistics, argumentation, and collaborative learning. Such controversies involve techno-scientific knowledge, but also values and emotions and therefore there is no one correct answer. In this paper we revisit three types of talk (exploratory, disputational, and cumulative) used to assess the quality of students' argumentation in small groups. Authors have mainly used this typology in a mutually exclusive way in problem-solving contexts and our contribution is to show how the three intertwine in authentic interactions, focusing on the construction of complex dialogic arguments. We argue that group talk is a dynamic construct resulting from individuals unceasingly adjusting to each other. We thus propose a theoretical interpretation of how group talk is shaped by and reciprocally shapes individual communicative behaviour, through a process of (non)alignment where self-identity footings are imbricated with face-work within either the ordinary or argumentative politeness system.}
\IfFileExists{../localcommands.tex}{
  \addbibresource{localbibliography.bib}
  \usepackage{langsci-optional}
\usepackage{langsci-gb4e}
\usepackage{langsci-lgr}

\usepackage{listings}
\lstset{basicstyle=\ttfamily,tabsize=2,breaklines=true}

%added by author
% \usepackage{tipa}
\usepackage{multirow}
\graphicspath{{figures/}}
\usepackage{langsci-branding}

  
\newcommand{\sent}{\enumsentence}
\newcommand{\sents}{\eenumsentence}
\let\citeasnoun\citet

\renewcommand{\lsCoverTitleFont}[1]{\sffamily\addfontfeatures{Scale=MatchUppercase}\fontsize{44pt}{16mm}\selectfont #1}
   
  %% hyphenation points for line breaks
%% Normally, automatic hyphenation in LaTeX is very good
%% If a word is mis-hyphenated, add it to this file
%%
%% add information to TeX file before \begin{document} with:
%% %% hyphenation points for line breaks
%% Normally, automatic hyphenation in LaTeX is very good
%% If a word is mis-hyphenated, add it to this file
%%
%% add information to TeX file before \begin{document} with:
%% %% hyphenation points for line breaks
%% Normally, automatic hyphenation in LaTeX is very good
%% If a word is mis-hyphenated, add it to this file
%%
%% add information to TeX file before \begin{document} with:
%% \include{localhyphenation}
\hyphenation{
affri-ca-te
affri-ca-tes
an-no-tated
com-ple-ments
com-po-si-tio-na-li-ty
non-com-po-si-tio-na-li-ty
Gon-zá-lez
out-side
Ri-chárd
se-man-tics
STREU-SLE
Tie-de-mann
}
\hyphenation{
affri-ca-te
affri-ca-tes
an-no-tated
com-ple-ments
com-po-si-tio-na-li-ty
non-com-po-si-tio-na-li-ty
Gon-zá-lez
out-side
Ri-chárd
se-man-tics
STREU-SLE
Tie-de-mann
}
\hyphenation{
affri-ca-te
affri-ca-tes
an-no-tated
com-ple-ments
com-po-si-tio-na-li-ty
non-com-po-si-tio-na-li-ty
Gon-zá-lez
out-side
Ri-chárd
se-man-tics
STREU-SLE
Tie-de-mann
} 
  \togglepaper[1]%%chapternumber
}{}

\begin{document}
\maketitle 

\section{Introduction: Linguistics applied to group reasoning} 

The aim of this study is the assessment of the educational quality of student-student interactions during a “scientific café” at school. Our corpus allowed us to consider 76 group dialogues about water management, videotaped in 4 schools of 3 countries, in 2011--2012. In particular, we present an in-depth analysis of a complex dialogue among French students, making sense of its apparent disorganisation thanks to an interactional linguistic approach. Such perspective help us build a new analytical toolkit including a thinner analytical grain, and taking an emic standpoint, trying to understand how each student makes sense of the ongoing activity. Going beyond the description of this specific dialogue, we propose a renewed theoretical view on group reasoning.


\section{Theoretical background}
\subsection{Assessing student-student interactions in socio-scientific debates} 

Literature in education research raises the role of argumentation both for learning new concepts and as skills to develop (e.g. \citealt{AndriessenEtAl2003}). In the science classroom, argumentative tasks are often associated with solving scientific controversies and the introduction of frontier science topics and socio-scientific issues (e. g. \citealt{DriverEtAl2000}).

Such issues can be defined as social controversies involving techno-scientific knowledge as well as values, emotions and stakes (e.g. \citealt{Albe2006,Legardez2006,OultonEtAl2004,SimonneauxSimonneaux2009}). They challenge typical science classroom practices due to four specific features: interdisciplinarity, use of information of diverse epistemic status, inclusion of subjectivity, and controversy. Students are expected to argue, but in a different way than they do in traditional problem-solving tasks, since they do not necessarily reach one single answer and may experience strong disagreement.

Nevertheless, little is known about how to assess the quality of student debates about socio-scientific issues. Most studies agree that four factors influence the quality of debates about a socio-scientific issue: students’ knowledge about the topic (e.g. \citealt{LewisLeach2006}), their understanding of the controversial and interdisciplinary nature of the issue (e.g. \citealt{DriverEtAl1996}), their epistemic values (e.g. \citealt{DésautelsLarochelle1998,Sandoval2005}), and the quality of students’ interactions during group debates (e.g. \citealt{Albe2006,Mercer1996}). This paper aims at contributing to a better understanding of this last, interactional factor.

\subsection{Face-work and the specificities of argumentative interactions}

Insights from linguistics help us understand students’ interactions. One key result of interactional linguistics is showing that politeness rules largely structure interactions. Politeness is defined as anything that someone does to make sure to preserve his own and others’ face, or positive social value \citep{Goffman1967}. The term “face-work” stands for the discursive elaborations produced during interactions that embody this concern for politeness. It consists in avoiding speech acts that might be face-threatening (FTAs), and, when they cannot be avoided, in softening them \citep{BrownLevinson1988}. Speech acts conforming to the politeness code are therefore considered as “preferred moves” while FTA, for instance, are “dispreferred”. In ordinary conversation, linguistic politeness implies a preference for agreement over disagreement \citep{Pomerantz1984}.

Argumentative interactions, however, are characterised by the explicit expression of disagreement. It has been claimed that disagreement then does not correspond to a breaking of the general face-work rules, but rather fit as a normal communicative act in another politeness system specific to argumentative contexts \citep{Plantin2018}.

\subsection{Collaborative learning \& group talk}

Research on collaborative learning has studied group argumentation as a way to learn in a diversity of settings (school, workplace, informal education, etc). Such a focus led to considering the group as a cognitive unit (e. g. \citealt{Stahl2006}), and to analysing discussion features as markers of collective reasoning (e. g. \citealt{OsborneEtAl2004}). Mercer and Wegerif, in the context of the mathematics classroom, defined a specific talk valuable for learning referred to as exploratory talk (\citealt{Mercer1996,WegerifMercer1997}):

\begin{quote}
First it is talk in which partners present ideas as clearly and as explicitly as necessary for them to become shared and jointly evaluated. Second, it is talk in which partners reason together – problems are jointly analysed, possible explanations are compared, joint decisions are reached. \citep[363]{Mercer1996}
\end{quote}

Later on, the concept of “exploratory talk” was adapted to analyse argumentation about socio-scientific issues. Then, the focus is not on consensus building but rather on students’ understanding of alternative viewpoints, building up complex dialogic arguments (\citealt{Albe2006,LewisLeach2006}). Exploratory talk was distinguished from both cumulative talk and disputational talk, considered of less educational value. Disputational talk is “characterised by disagreement and individualised decision making” and “short exchanges consisting of assertions and counter-assertions”; whereas in cumulative talk “speakers build positively but uncritically on what the other has said” \citep[369]{Mercer1996}. Each type of talk is related to a specific social recognition, associated to the self (disputational talk), the group (cumulative talk) or more balanced (exploratory talk) (\citealt{WegerifMercer1997}: 54--56). So far, authors only used this typology in a mutually exclusive way, at the scale of a whole small-group dialogue, identified as corresponding to either exploratory, cumulative or disputational talk and no previous study has addressed whether or how the three could intertwine in authentic interactions.

\section{Methodology}
\subsection{Pedagogical situation and corpus} 

Our data consist in videotaped scientific cafés about drinking water management implemented in Mexico, the US and France \citep{Polo2014}. After giving an overview of the analysis of group talk among the 76 student-student dialogues of our corpus, this paper focuses on a debate among French high school students.

The students are in groups of 3 or 4 around a table. The activity (110 minutes long) is organised around a multiple-choice questionnaire and oriented towards a main question. The students are first asked to answer it individually with anonymous electronic devices. Then, three subtopics are explored, providing students with basic information through quiz-type questions. Each subtopic ends with a socio-scientific type question, called an “opinion question”, that the students must discuss in group, arriving at a common answer. This collective vote is made public and a classroom debate begins, ending with an individual electronic survey. At the end of the activity, the main question is asked again and treated as an opinion question. During the class debate, the students can freely defend their group answer or another viewpoint, or even change their mind. 

\subsection{Indicators of the quality of group talk}

We used five indicators, which, when all positive, define exploratory talk.

\begin{enumerate}
\item \textit{Justification of opinions}: Any assertion accepting or rejecting a proposition is supported by a justification, either produced by the student who makes it or by another group member, spontaneously or after the idea was challenged. As in \citegen{Toulmin1958} pattern of an argument, a diversity of linguistic forms can embody justifications, often but not always introduced by causal connectors such as “because”.

\item \textit{Topical alignment}: Do the students elaborate on the argumentative content of previous turns? Linguistic markers of topical alignment are typically referential verbal or gestural repetitions.

\item \textit{Critical examination}: Exploratory talk requires all the ideas to be truly investigated, and critically but constructively appraised, even the ones that end up discarded. In some cases, such examination relies on the \textit{ad hoc} establishment of a specific discussion procedure consisting in going through the six options as they appear on the slide. In other groups, the dialogue focuses on the ones pre-selected by its members as they introduce them into the discussion (either directly or after each opinion was expressed). Still, this indicator does not only apply to the options of an answer: critical examination of any spoken alternative idea is expected in exploratory talk, such as competing justifications for a single option.

\item \textit{Cooperative decision-making process}: When they are really engaged in a high-quality cognitive collaboration, the students try to have every member of the group agree on the collective vote, even if there is no consensus. For instance, a joint decision that is fairly common in our corpus is to overpass the exercise’s rules and to display two group answers instead of one.

\item \textit{Dialogic strengthening of arguments}: Do the individual contributions gradually integrate the rest of the group’s arguments? Such an indicator relates to the extent to which the whole group feels responsible for the decision made. In the context of our study, this means that during the class debate, any of the group members can bring up any argument developed during the group discussion, and not only his own initial ideas. This last indicator emphasises the role of multivocality in the elaboration of an argumentative discourse.
\end{enumerate}

\section{Empirical studies and interpretation}
\subsection{Global inventory and typical cases}

\tabref{tab:11:1} presents the results of the global inventory of exploratory, cumulative and disputational cases in our corpus.

\begin{table}
\caption{Inventory of the type of talk among our 76 small-group debates about socio-scientific issues related to water management.\label{tab:11:1}}
\begin{tabular}{lrrrrr}
\lsptoprule School type & Expl. & Cum. & Disp. & Hybrid & Total\\\midrule
US public  & 4 & 6 & 0 & 8 & {18}\\
FR public  & 4 & 0 & 3 & 11 & {18}\\
MX public rural  & 3 & 3 & 0 & 14 & {20}\\
MX private urban  & 8 & 2 & 0 & 10 & {20}\\\midrule
Total & {19} & {11} & {3} & {43} & {76}\\
\lspbottomrule
\end{tabular}
\end{table}


Since the indicators allowed us to identify exploratory talk sequences among Mexican, American and French students (\citealt{Polo2014}: 123--155), we believe that those indicators are little dependent on culture and language. Surprisingly, we found no cases of cumulative talk in the French data while cases of disputational talk were only identified among French students. Nevertheless, excerpts of some French dialogues were similar to cumulative talk, and in each school, some cases comprised parts of dialogues with the characteristics of disputational talk. These findings led us to distinguish between “typical” cases corresponding as a whole to one category and “hybrid” cases, in which students seemed to alternate between different types of talk or develop intermediate, hybrid talk. Such hybrid cases prevailed (43/76).

We have no space here to detail the typical cases studied (\citealt{Polo2014}: 156--198), but a few comments on emblematic cases of cumulative and disputational talk are necessary in order to better understand how they differ from exploratory talk. Such differences are specified along the five indicators of the quality of talk in \tabref{tab:11:2}. Cumulative talk is characterised by a negative third indicator: the lack of critical appraisal of arguments, resulting in a partial exploration of the space of debate, limited to its uncontroversial side. The typical disputational case investigated shows
(1) repetitions rather than justifications of opinions;
(2) limited topical alignment;
(3) rejection of others’ ideas without true examination and
(4) individual decision-making. In both disputational and cumulative talk, the 5th indicator is negative: in the end, the students repeat their own initial ideas instead of building more complex arguments integrating the diverse perspectives.


\begin{table}
\caption{Characteristics of the type of group talk: five indicators.\label{tab:11:2}}

\begin{tabularx}{\textwidth}{cQQQ}

\lsptoprule

 ID & Exploratory & Disputational & Cumulative\\
 
 \midrule
 
 1 & Justification of opinions & Repetition instead of justifications & Justification of opinions\\\tablevspace
 2 & Topical alignment & Limited & Topical alignment\\\tablevspace
 3 & Critical examination & Rejection without examination & Acceptance without examination\\\tablevspace
 4 & Cooperative decision-making & Individual decision-making & Cooperative decision-making\\\tablevspace
 5 & Dialogic strengthening of arguments & Absent & Absent\\
\lspbottomrule

\end{tabularx}
\end{table}

Interestingly, the group engaged in the typical cumulative dialogue (about opinion question 1) proves capable, later on, to display rich exploratory talk during opinion question 3. Such a fact shows that engaging in a specific group talk is not only a matter of cognitive skills, but also of contextual relevancy. During the first discussion, the students might have understood the task as a display of their knowledge about the environment, and adopted the corresponding consensual self-identity attitude. When they came to opinion question 3, they had realised that the task consisted in challenging each other’s ideas for the sake of group achievement. 

\subsection{Hybrid case \& the topical and sequential nature of group talk}

Three French students, Jérémie, Julie and Laurent, are discussing the main question, namely what would access to drinking water in the future depend on, displaying hybrid talk. Most of their dialogue is transcribed and translated below. 

\ea 
\label{ex:11:1}

\ttfamily
\parbox{14mm}{1  JUL}euh: i found F xxx but i don’t remember what it \\
\parbox{14mm}{~}is\medskip

\parbox{14mm}{2  JER }i am sorry but it’s going to be A{\textbackslash}\medskip

\parbox{14mm}{3  JUL}  no it’s [C{\textbackslash}\medskip

\parbox{14mm}{4  JER}       [à because: nowadays it’s based on A\medskip

\parbox{14mm}{5  LAU}  yeah because water is gonna become more and more \\
\parbox{14mm}{~}expensive\medskip

\parbox{14mm}{6  JER}  it’s gonna become more and more expensive and the \\
\parbox{14mm}{~}people are\\
\parbox{14mm}{~} capitalists and it won’t change it has always been like \\ 
\parbox{14mm}{~}that and it will always be like that=\medskip

\parbox{14mm}{7  LAU}  =water has nothing to do with capitalism{\textbackslash}=\medskip

\parbox{14mm}{8  JER}  =yeah because [it’s: the people when)\medskip

\parbox{14mm}{9  LAU}      [because water is vital so it’s gonna automatically 
\parbox{14mm}{~}become more expensive [even\medskip

\parbox{14mm}{10  JER}  [it’s vit- it’s vital\medskip

\parbox{14mm}{11  LAU}  [would they be communist or whatever it’d be the same\medskip

\parbox{14mm}{12  JER}  it wouldn’t{\textbackslash}\medskip

\parbox{14mm}{13  LAU}   it would{\textbackslash}\medskip

\parbox{14mm}{14  JER}  no{\textbackslash}\medskip

\parbox{14mm}{15  LAU}  water would become expensive anyway\medskip

\parbox{14mm}{16  JER}  no{\textbackslash}\medskip

\parbox{14mm}{17  LAU}  sure it would how would it work otherwise/\medskip

\parbox{14mm}{18  JER}  because\medskip

\parbox{14mm}{19  LAU}  the less there is the the scarcer it becomes and the
 \parbox{14mm}{~} more expensive it becomes that’s logical{\textbackslash}\medskip

\parbox{14mm}{20  JER}  of course the less there is the more expensive it it \\
\parbox{14mm}{~}becomes\medskip

\parbox{14mm}{21  LAU}  what{\textbackslash}\medskip

\parbox{14mm}{22  JER}  yeah but there’ll always be the same amount of: \medskip

\parbox{14mm}{23  JUL}  of water\medskip

\parbox{14mm}{24  JER}  of water{\textbackslash}=\medskip

\parbox{14mm}{25  JUL}  =[but after yeah but it=\medskip

\parbox{14mm}{26  LAU}   =[yeah but after ya gotta find ways exactly for uh: for \\
\parbox{14mm}{~}uh:\medskip

\parbox{14mm}{27  JUL}  to make it [clean/=\medskip

\parbox{14mm}{28  LAU}  =get the water the water from the sea and all that\medskip

\parbox{14mm}{29  JER}  sure yeah and the ways what are they it’s cash\medskip

\parbox{14mm}{30  JUL}  no{\textbackslash} it’s scientific [progress\medskip

\parbox{14mm}{31  LAU}           [yeah it’s cash{\textbackslash})\medskip

\parbox{14mm}{32  JER}  and how do you make scientific progress [how/\medskip

\parbox{14mm}{33  LAU}               [no but there’s no need{\textbackslash}= \medskip

\parbox{14mm}{34  JUL}   [it’s not gonna work if you put bills on the waterfront\\
\parbox{14mm}{~}scientific progress is needed\medskip

\parbox{14mm}{35  JER}        [cash is needed{\textbackslash} cash is needed so [it’s A:\medskip

\parbox{14mm}{36  LAU}                   [no but we they already know how to do you know uuhh \\ \parbox{14mm}{~}unsalt the water desalinate [the water{\textbackslash}\medskip

\parbox{14mm}{37  JER}           [but it’s expensive{\textbackslash}\medskip

\parbox{14mm}{38  LAU}  yes but it’s also vital so you don’t give a shit about\\ \parbox{14mm}{~}[money{\textbackslash}\medskip

\parbox{14mm}{39  JER}          [((pretending to count bills))\medskip

\parbox{14mm}{40  JUL}                       [if you do scientific inventions in a few years you \\  \parbox{14mm}{~}find a cheap way to euh to:\medskip

\parbox{14mm}{41  JER}  you find a way/ go ahead find one{\textbackslash}\medskip

\parbox{14mm}{42  JUL}  no but i’m not a [scientist thanks anyway{\textbackslash}\\

\parbox{14mm}{~}(...)\medskip

\parbox{14mm}{98  JUL}  cash AND scientists are needed but scientists are also\\
\parbox{14mm}{~}needed\medskip

\parbox{14mm}{99  JER}  but cash is needed\medskip

\parbox{14mm}{100 JUL}  [yes but if you have scientists it’s euh logical they\\ 
\parbox{14mm}{~}must be paid\medskip

\parbox{14mm}{101 LAU}  [yes but the scientists don’t worry they’re relaxed\medskip

\parbox{14mm}{102 JUL}  yeah but they won’t pay out of their own pockets euh to \medskip
\parbox{14mm}{~}euh:


\parbox{14mm}{103 LAU}              [and euh it’s okay yeah{\textbackslash} no but the scientists i think \parbox{14mm}{~}they earn enough not to break our butts)\medskip

\parbox{14mm}{104 JER}   [for su:re)\largerpage\medskip

\parbox{14mm}{105 LAU}  [they’re not gonna stop working oh shit it’s a shame \\ 
\parbox{14mm}{~}i’m not paid bad luck everybody dies{\textbackslash}\medskip

\parbox{14mm}{106 JER}  ((puts letter A on the stand))\medskip

\parbox{14mm}{107 MO1} on three you put it up one two three go ahead\medskip

\parbox{14mm}{108 LAU}  ((takes letter A off))\medskip

\parbox{14mm}{109 JER}  ((takes it from him and puts it on the stand again))\medskip

\parbox{14mm}{110 LAU}  °yeah we have to put this{\textbackslash}°\medskip

\parbox{14mm}{111 JER}   °but:°\medskip

\parbox{14mm}{112 LAU}  °we have to put it right now°\medskip

\parbox{14mm}{113 JER}  °what do YOU wanna put/°\medskip

\parbox{14mm}{114 LAU}  °A{\textbackslash}°\medskip

\parbox{14mm}{115 JUL}  °no but it doesn’t matter{\textbackslash}°\medskip

\parbox{14mm}{116 JER}  °the scientists°\medskip

\parbox{14mm}{117 LAU}  NO we have to put A dude two of us said so{\textbackslash}\medskip 

\parbox{14mm}{118 JER}  [we put both <((putting F next to A)) we put both:>) \\
\parbox{14mm}{~}(...)\medskip 

\parbox{14mm}{124 JUL}  °i promise it’s okay and i don’t wanna talk anyway{\textbackslash}°

\z 

We first qualified their discussion as alternating between disputational and exploratory talk, through 6 distinct episodes. The students are engaged in disputational talk at turns 1--4: they oppose each other with non-supported assertions that they keep on repeating. At turn 2, Jérémie shows no concern for what the others think of option A (access will depend on economic income), which he raises as a definitive choice. Indicators 1 to 4 are negative. Jérémie and Laurent are also conducting disputational talk at turns 11--16, even if they seem to agree on option A: the disagreement is about the reasons supporting their choice. Turns 29--31 show similar patterns, but on the opposition of A to F (access will depend on scientific progress). Similarly, we identified 3 episodes of exploratory talk. At turns 4--10, Jérémie and Laurent are deeply collaborating. Laurent formulates two arguments supporting Jérémie’s assertion, which he immediately repeats: water is going to become more expensive and is “vital”. Indicators 1 and 2 are positive. Indicator 3 is partly positive: there is a critical look at the other’s argument but no justification of the counter-assertion. The exploration of Jérémie’s reason only starts at turn 17 when Laurent asks him for more explanation, beginning another episode of exploratory talk. Laurent also develops one of the reasons he gave before about water scarcity (turn 19). This idea is co-criticised by Julie and Jérémie, who reject the premise with the argument that there will always be the same amount of water on Earth (turns 22--24). Julie then reintroduces option F in the discussion, with Laurent’s help. Indicators 1, 2 and 3 are positive. From turn 32 onwards, the students are also engaged in exploratory talk. Jérémie uses a causal argument to justify that money precedes scientific progress, to which Julie opposes first a refutation by absurdity (turn 34) and then a refutation of the direction of causality (turn 40). Finally, Laurent goes beyond this materialistic opposition by referring to the fundamental norm that “you can’t put a price on life” (turns 38, 105). Indicators 1, 2 and 3 are positive. The 4\textsuperscript{th} indicator is also completed: Jérémie shows great concern for taking everybody’s vote into account (turns 110, 113, 115), even if Julie chooses not to make her opinion visible (turn 124).

Nevertheless, considering the 5\textsuperscript{th} indicator led us to proposes a deeper, alternative interpretation of this dialogue. The contributions to the class debate reveal which content from the table debate was actually shared by all the group members. This content-orientated criterion prevents quick judgment based on style effects, for instance considering that a confrontational rhetorical stance necessarily indicates disputational talk. Members of the studied group contribute to the class-debate twice. In the first contribution, the moderator asks Julie to justify her opinion. She seems pretty confused at speaking in front of everybody and cannot even remember the letter of the option that she defended. Jérémie and Laurent then help her, rephrasing an argument that she used during the group debate. Julie’s main contribution is based on their previous collaboration: she concedes that money is necessary to do science, but not sufficient. A few minutes later, Laurent contributes to the discussion, then only reporting on his own idea, showing that in the end, when discussing the reasons for A, the students were not effectively cooperating.

On the contrary, the first contribution reveals that the students were actually debating efficiently when discussion centered on A vs F. Therefore, all the turns concerning this issue can be interpreted as a global topical sequence of exploratory talk. The two parts of disputational talk opposing options A and F, occurring at the very beginning of the discussion, and when the topic is re-introduced (respectively at turns 1--2 and 30--32) can be understood as opening sub-sequences, serving a function of exhibiting the different viewpoints before exploration. On the contrary, we can reinterpret part of the dialogue as a global topical sequence of disputational talk, still embedding two collaborative sub-sequences. The first one (turns 4--10) is an opening sub-sequence that allows Laurent and Jérémie to understand what they agree and disagree on before disputing. The second one (turns 18--29) can be qualified as a transition sub-sequence during which each participant goes back to a behaviour that makes possible the following exploratory sequence. Notably, during this transitional sub-sequence, Julie steps into the dialogue again, reintroducing option F, and therefore producing a topical shift back to the A vs F debate. Such deeper analysis emphasises the topical and sequential nature of group talk. Figures 1 and 2 summarise our final interpretation of this dialogue.

\begin{figure}
% \includegraphics[width=\textwidth]{figures/a2PoloLundPlantinandNiccolaifinal-img001.tif} 
\begin{tabular}{|c>{\raggedright\arraybackslash}p{10cm}|}\hline
                                 1--3  & \multicolumn{1}{>{\raggedright\arraybackslash}p{10cm}|}{Disputational > Expository pre-sequence about the competing options (A vs F)}\\\hline
 \multicolumn{1}{c}{\textcolor{gray}{4--11}} & \multicolumn{1}{>{\raggedright\arraybackslash}p{10cm}}{\textcolor{gray}{Exploratory > Expository pre-sequence about what they disagree on about reasons for A}}\\
\multicolumn{1}{c}{\color{gray}11--16} & \multicolumn{1}{>{\raggedright\arraybackslash}p{10cm}}{\textcolor{gray}{Disputational, about reasons for A}}\\\hline
                              {17--28} & Embedded transitional sub-sequence, gradual alignment to constructively critical footing\\
                              {29--31} & Disputational > Expository pre-sequence of still competing options (A vs F)\\
                              {32--60} & Exploratory, about A vs F\\
                  \textcolor{gray}{61--90} & \textcolor{gray}{Embedded disputational sequence, about reasons for A}\\
                             {91--110} & {Exploratory, about A vs F}\\
                            {110--146} & {Closure, with gradual disalignment of individual footings}\\\hline
\end{tabular}
\caption{Exploratory topical sequence concerning the opposition between options A and F.}  
\end{figure}


\begin{figure}

\begin{tabular}{>{\centering}p{.115\textwidth}>{\raggedright\arraybackslash}p{.8\textwidth}}
{\color{gray}{1--3}} & {\color{gray}{Disputational > Expository pre-sequence about the competing options (A vs F)}}\\
\end{tabular}\\
\begin{tabular}{|>{\centering}p{.115\textwidth}>{\raggedright\arraybackslash}p{.8\textwidth}|}\hline
 4--11 & Exploratory > Expository pre-sequence about what they disagree on about reasons for A\\
  11--16 & Disputational, about reasons for A\\
    \color{gray}{17--28} & \color{gray}{Embedded transitional sub-sequence, gradual alignment to constructively critical footing}\\
    \color{gray}{29--31} & \color{gray}{Disputational > Expository pre-sequence of still competing options (A vs F)}\\
    \color{gray}{32--60} & \color{gray}{Exploratory, about A vs F}\\\hline
\end{tabular}\\
\begin{tabular}{>{\centering}p{.115\textwidth}>{\raggedright\arraybackslash}p{.8\textwidth}}
{61--90} & {Embedded disputational sequence, about reasons for A}\\
{\color{gray}{91--110}} & {\color{gray}{Exploratory, about A vs F}}\\
{\color{gray}{110--146}} & {\color{gray}{Closure, with gradual disalignment of individual footings}}\\
\end{tabular}
% % \includegraphics[width=\textwidth]{figures/a2PoloLundPlantinandNiccolaifinal-img002.tif}
\caption{Disputational topical sequence about reasons for A.}
\end{figure}


\section{Conclusion}

In the hybrid dialogue analysed, we identified both sequences and sub-sequences of typical talk playing specific functions and a transitional sub-sequence of hybrid talk. The latter reveals that the process of engaging, as a group, in a typical form of talk may take time and work. Those findings raise two methodological concerns: the need for multiplying the units of analysis and the importance of considering group talk as a dynamic construct resulting from individuals unceasingly adjusting to each other. These methodological concerns call for adequate conceptual categories.

We propose to use of the terms “dialogue”, “sequence” and “sub-sequence” to address the first point. “Dialogue” corresponds to an actual interaction starting with the students being prompted by the moderator to carry out discussions in small groups, and ending when they are asked to stop. A “sequence” encompasses a topical unit of talk within a dialogue. A whole dialogue may consist in a single sequence, as in the first two cases. But a dialogue may also consist in several sequences of different group talk. Getting to a finer analytical grain, sub-sequences are parts of sequences that serve specific interactional functions, and therefore may differ from the rest of the sequence. Sub-sequences generally comprise several speech turns. 

To make sense of hybrid talk, we also propose a theoretical interpretation of how group talk is shaped by and reciprocally shapes individual communicative behaviour, through a process of alignment. As the emotions associated with face preservation play a crucial role in group reasoning \citep{PoloEtAl2016}, we hypothesise that politeness strongly affects the perceived social relevancy of communicative behaviour, and therefore the type of talk developed. In small-group settings, the students must choose the appropriate politeness system, whether ordinary or argumentative. Therefore, they ensure face preservation by choosing a specific self-identity footing. When individuals display different self-identity footings, they fall into hybrid group talk. When all the members of a group are aligned on the same self-identity footing, they may engage in any of the three types of talk (\tabref{tab:11:3}). In cumulative talk, face preservation relies on displaying a consensual footing, as in ordinary conversation. In disputational talk, face is attached to individuals’ own ideas, everyone displays a competitive footing. What is very special to exploratory talk is that the matter of face preservation is transferred to the group level, face being associated with group achievement. This need for recognition so satisfied, individuals can use a constructively critical footing, shifting from the relational to the cognitive dimension of the interaction.

\begin{table}
\caption{Self-identity footing and face preservation system associated with each type of group talk.\label{tab:11:3}} 
\begin{tabularx}{\textwidth}{XQQQ}

\lsptoprule 

~ & \multicolumn{3}{c}{Group talk}\\
\midrule
& Cumulative & Exploratory & Disputational\\
\midrule
Self-identity footing & consensual & constructively critical & competitive\\\tablevspace
Face-preservation system & preserving consensus by not expressing disagreement & focusing on group achievement & searching for victory of one's own ideas upon others'\\
\lspbottomrule
\end{tabularx}
\end{table}

\section*{Acknowledgements}
The authors are grateful to the ASLAN project (ANR-10-LABX-0081) of the Université de Lyon, for its financial support within the French program “Investments for the Future” operated by the National Research Agency (ANR).

{\sloppy\printbibliography[heading=subbibliography,notkeyword=this]}
\end{document}
