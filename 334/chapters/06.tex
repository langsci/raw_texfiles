\documentclass[output=paper]{langscibook} 
\ChapterDOI{10.5281/zenodo.6620115}
\author{Alain Rabatel\affiliation{UMR 5161 ICAR; Centre National de la Recherche Scientifique; Ecole Normale Supérieure; Université Lumière Lyon 2; Université Claud Bernard Lyon 1}}
\title[Proposal for a simplex account of discourse complexity]
      {Proposal for a simplex account of discourse complexity using the pragma-enunciative theory of points of view}
\abstract{This chapter argues for an alternative to the difficulties that arise when studying interacting subsystems in isolation while using a complex systems approach. Conceiving complexity in terms of simplexity allows one to account for complexity through procedures which are integrated and economic because they are based on principles that simplify yet also account for the past and anticipate the future. Empathic complexity is analysed through points of view (POV) and used to illustrate a simplex account of what is complex. Four examples, from both literary contexts and recorded talk, are analysed to show the different POVs possible (speaker/enunciator, secondary speakers/enunciators, secondary non-speaker/enunciators), whether these POVs are explicitly or implicitly expressed, and whether they are with or without opinion or judgement. Such an analysis reveals the strategies that may lurk behind seemingly objective statements while forming the basis of a simplex account of relations between the primary speaker and others and between his or her POV and those of others, all within an empathetic, polyphonic and dialogical framework.}
\IfFileExists{../localcommands.tex}{
  \addbibresource{localbibliography.bib}
  % add all extra packages you need to load to this file

\usepackage{tabularx,multicol}
\usepackage{url}
\urlstyle{same}

\usepackage{listings}
\lstset{basicstyle=\ttfamily,tabsize=2,breaklines=true}

\usepackage{langsci-basic}
\usepackage{langsci-optional}
\usepackage{langsci-lgr}
\usepackage{langsci-osl}
% \usepackage{./langsci/styles/langsci-lgr}
% \usepackage{./langsci/styles/langsci-osl}
% \usepackage{langsci-gb4e}

\usepackage{tikz}
\usetikzlibrary{patterns,calc}
\pgfdeclarepatternformonly{south east lines}{\pgfqpoint{-0pt}{-0pt}}{\pgfqpoint{3pt}{3pt}}{\pgfqpoint{3pt}{3pt}}{
    \pgfsetlinewidth{0.6pt}
    \pgfpathmoveto{\pgfqpoint{0pt}{3pt}}
    \pgfpathlineto{\pgfqpoint{3pt}{0pt}}
    \pgfpathmoveto{\pgfqpoint{.2pt}{-.2pt}}
    \pgfpathlineto{\pgfqpoint{-.2pt}{.2pt}}
    \pgfpathmoveto{\pgfqpoint{3.2pt}{2.8pt}}
    \pgfpathlineto{\pgfqpoint{2.8pt}{3.2pt}}
    \pgfusepath{stroke}}
    
\usepackage{stmaryrd}
\usepackage{wasysym}
\usepackage{multirow}
\usepackage{caption}
\usepackage{subcaption}
\usepackage{mathrsfs}
\usepackage{qtree}

\usepackage{linguex}


  %pminos do not split footnotes
% \interfootnotelinepenalty=10000 %Footnote in Laporte chapters has to be split SN


%\DeclareIndexNameFormat{default}{%
%\nameparts{#1}%
%\usebibmacro{index:name}%
%{\index[names]}%
%{\namepartfamily}%
%{\namepartgiveni}%
% {}% L1
% {}% L2
%{\namepartprefix}% generates spurious space L3
%{\namepartsuffix}% generates spurious space L4
%}

%  {\DeclareIndexNameFormat{default}{%
%     \usebibmacro{index:name}{\index[names]}{#1}{#3}{#5}{#7}}}

%\DeclareIndexNameFormat{default}{%
%  \usebibmacro{index:name}{\sindex[nom]}{#1}{#3}{#5}{#7}}

%\DeclareIndexNameFormat{default}{%
%  \usebibmacro{index:name}{\sindex[person]}{#1}{#3}{#5}{#7}}
%\DeclareIndexNameFormat{default}{%
%\nameparts{#1} \usebibmacro{index:name}{\sindex[person]]}{\namepartfamily}{‌​\namepartgiven}{\nam‌​epartprefix}{\namepa‌​rtsuffix}}

%\newcommand{\smiley}{:)}

%\renewbibmacro*{index:name}[5]{%
%\usebibmacro{index:entry}{#1}%
%{\iffieldundef{usera}{}{\thefield{usera}\actualoperator}\mkbibindexname{#2}{#3}{#4}{#5}}}

% \newcommand{\noop}[1]{}

%remove for final
%\overfullrule=1mm

\newcommand{\tobi}[2]}}
\renewcommand{\S}[1]{\tobi{#1}{\textsc{*}}}

% this volume references
% puts: [this volume]
% already defined: \citetv
%\newcommand{\citepv}[1]{(\citeauthor{#1} \citeyear*{#1} [this volume])}
\newcommand{\citealtv}[1]{\citeauthor{#1} \citeyear*{#1} [this volume]}

%parentheses around example number
\newcommand{\pref}[1]{(\ref{#1})}

% in-text examples

\newcommand{\lnex}[1]{\textit{#1}} %target lang word
\newcommand{\lnlit}[1]{(lit.: `#1')} %literal reading
\newcommand{\lnlat}[1]{(#1)} % latinization
\newcommand{\lntrans}[1]{`#1'} %translation
\newcommand{\lnexl}[2]%
{\lnex{#1}{} \lnlat{#2}} % ex with latinization
\newcommand{\lnexlat}[3]{\lnex{#1}{} \lnlat{#2}{} \lntrans{#3}} % ex with latinization and tranl.

%ch01
\newcommand{\co}[1]{\mbox{\textbf{#1}}}

%ch09

\newcommand{\cyrbulg}[1]{\begin{otherlanguage*}{bulgarian}#1\end{otherlanguage*}}


%ch10
\newcommand{\nlp}{{\small NLP}}
\newcommand{\mwe}{{\small MWE}}
\newcommand{\rae}{{\small RAE}}
\newcommand{\lvc}{{\small LVC}}
\newcommand{\pos}{{\small P}o{\small S}}
%\newcommand{\todo}[1]{ \textcolor{red}{#1} }

%\renewcommand{\labelenumi}{\theenumi}
%\ainamefmt{{vv}{ll}{, ff}{, jj}} % fullname

\newcommand{\biberror}[1]{{\color{red}#1}}

\newcommand{\osenovaitem}{--~} 
  %% hyphenation points for line breaks
%% Normally, automatic hyphenation in LaTeX is very good
%% If a word is mis-hyphenated, add it to this file
%%
%% add information to TeX file before \begin{document} with:
%% %% hyphenation points for line breaks
%% Normally, automatic hyphenation in LaTeX is very good
%% If a word is mis-hyphenated, add it to this file
%%
%% add information to TeX file before \begin{document} with:
%% %% hyphenation points for line breaks
%% Normally, automatic hyphenation in LaTeX is very good
%% If a word is mis-hyphenated, add it to this file
%%
%% add information to TeX file before \begin{document} with:
%% \include{localhyphenation}
\hyphenation{
    Beck-man
    Ngu-yen
    back-chan-nel
    back-chan-nels
    mo-not-o-nous
    ste-reo-typ-i-cal
}

\hyphenation{
    Beck-man
    Ngu-yen
    back-chan-nel
    back-chan-nels
    mo-not-o-nous
    ste-reo-typ-i-cal
}

\hyphenation{
    Beck-man
    Ngu-yen
    back-chan-nel
    back-chan-nels
    mo-not-o-nous
    ste-reo-typ-i-cal
}
 
  \togglepaper[1]%%chapternumber
}{}

\begin{document}
\maketitle 

Most of the work on complexity highlights the fact that multiple level interactions between subsystems make it impossible to study one of these subsystems independently of other interaction and determination mechanisms, without taking into account self-organisation, non-linearity, regulation by feedback, and proaction aimed at accomplishing a particular task. Those studies generally consider that, if a particular function is not performed by a particular biological organ or through a particular display (concerning \textit{language} -- with words of a particular word class, according to the rules of the language in question and this is probably even more true at the level of discourse), then the function can be performed by others, in virtue of a principle of vicariance \citep{Berthoz2013}.

But this research program faces difficulties in capturing the multiple determinations that come into play in a system, and, above all, in articulating them. This is a challenge for holistic approaches and perhaps even more so for emerging theories:\footnote{Due to space constraints, I refer to the references given in the introduction.} where should the search for determinations stop, between accidental and constitutive ones? There are many other difficulties, as \citet[8]{Berthoz2009} points out ironically, as in our societies, “we are overwhelmed by complexity”. 

Therefore I propose the complementary (and perhaps alternative) hypothesis of conceiving complexity... in terms of simplexity, in order to set the objective of accounting for complexity through procedures, which are not \textit{simple} but \textit{simplex}, because they are integrated and economic. We will therefore agree that simplexity is not synonymous with simplicity, just like complexity is not synonymous with complicated. \citet{Berthoz2009} stresses that “despite the complexity of natural processes, the brain must find solutions, and these solutions are based on simplifying principles. They allow for a quick, elegant and efficient processing of complex situations, taking into account past experience and anticipating the future. In intersubjectivity, they facilitate the understanding of other people’s intentions. They maintain or favour `meaning'. They are neither caricatures, nor shortcuts, nor summaries. They can involve detours, an apparent complexity, but they pose problems in an original way.” (\citeyear[7; personal translation]{Berthoz2009})


\section{Looking through a simplexity prism from the point of view of enunciation in order to manage empathic complexity}


This research program proposes to explore this hypothesis with an approach based on empathy, through the study of points of view (POV), and, on this basis, to show that the translinguistic issue of POV corresponds to a simplex approach to complexity, by proposing a global approach to a semantic issue that can be broken into several modules. To this end, my approach to empathy integrates primary somesthesic data into secondary empathy at the hinge of the sensory, sensitive, cognitive, and projective areas. This choice corresponds to the fact that empathy is not reduced to mirror neurons, or to proprioceptive phenomena, I am rather interested in more elaborate simulation movements, such as those mentioned below by Jouvent, which establish a continuum between the (micro) “events” to which we are exposed and the intentional, reflexive mechanisms, reflecting (on) the past and projecting towards the future:


\begin{itemize}
\item “When an animal is cold, it can only curl up, bristle its fur and seek shelter. We humans, in the same circumstances, have a multitude of alternatives.
\item Because I am physically feeling cold, I can say that I am cold, and find some relief, or even comfort, in saying that, or I can say that I like this cold which stimulates me. I can decide, even under the cold, that what I feel is called “hot”, or “warm”. I can make fun of my sensitiveness to the cold, or conversely be proud of my resistance.
\item To the first feeling of cold, I can associate memories:

\begin{itemize}
\item pleasant: it reminds me of winter at my grandparents' house during the Christmas holidays;
\item or on the contrary traumatic: “It was on a day like this that my father had his accident”.
\end{itemize}

\item I can anticipate by saying that, as soon as I finish my trip, I will drink a large bowl of hot chocolate. I can add that next time I will cover up more warmly. 

\item This psychic dressing of reality can be turned into a game for two. If my friend is by my side, I can tell him: “It's not warm today”, “You look frozen”, or “I don't feel the cold” with a smile. I can put myself in his shoes: “I imagine, you who don't like the cold, what you can feel”. I can say, in a serious way: “I'm too hot”.

\item In this game with the physical reality of the world, I do not only have language and mental images: I can connote all these maxims with a gesture, mimic that I shiver, whether or not I match my words with my gestures, mimic the other person... I can make a movement, a pout, imitating a third party, or imagine that I am the other person making this movement, this pout. I can play a character, imitate his peculiarities, I can play a multitude of roles.” (\citealt{Jouvent2013}: 12--13, personal translation)
\end{itemize}


Having insisted on the linguistic continuum between perception, thought and action, through the problem of (pre-)reflexivity (\citealt{Rabatel2008b}: 417--420, 440--449, 464--469), I find it very interesting to observe that psychologists make a similar observation, even if they express it without insisting on the role of language in this process, which is obvious, in the examples mentioned by Jouvent. However, relating these psychological mechanisms to language does not imply that the linguist adopts de facto a subjectivist position that would make language a transparent means of expression serving a will to say that is external and prior to it. For the linguist such as myself, intentionality is essentially analysed based on the organisation of discourse, even if he/she cannot rule out the idea of articulating what is said with the will to say or with many other co-textual or contextual elements essential to the production and understanding of meaning \citep[211--213]{Rabatel2014}.

Since my goal is not to discuss the notion of empathy from a psychological or philosophical point of view, but to carry out a linguistic analysis of the phenomenon, I propose an explicit reformulation of the common language definition of empathy (“putting oneself in the place of others”) that is exploitable for the (enunciativist) linguist, based on my conception of points of view (POV). If we try to define the cognitive-linguistic components of this decentralisation, we will say that empathy, from the perspective of linguistics that I defend, consists in putting oneself in the place of others, in considering from their point of view what they \textit{perceive} (from the place where they are), \textit{feel}, \textit{think}, \textit{say}, \textit{do}.... We hereby consider that points of view first correspond to a given spatio-temporal location, then to a much more abstract one, consisting in considering things according to a given notional framework, with given values and concerns, without necessarily making explicit judgments.

\begin{itemize}
\item This very general conception of empathy is constantly at the intersection with my theory of POV, because it immediately provides linguistic concepts to account for how a speaker is able, in discourse, to imagine, in a global and articulated way, what another person can “do\footnote{I intentionally use this term to mean that doing is not limited to the action-oriented module; it is rather the horizon of the other modules.}”, that is to say:
\begin{itemize}

\item \textit{perceive}: hence the linguistic study of the various forms of perception (to see, hear, feel, touch...), of their relationships;

\item \textit{feeling}: hence the linguistic analysis of affects, emotions, feelings; 

\item \textit{think, say}: hence the articulated study of the expression of thoughts, words, their relationships, especially internal wording, the various forms of represented hybrid discourses that cannot be reduced to reported discourses;

\item \textit{to do}: hence the linguistic expression of action, its values, its motivations, and therefore its intentionality. 

\end{itemize}

\item The tools\footnote{These tools are linguistic \textit{observables}, composed of discourse markers or non-grammaticalised markers, the re-occurrence and the co-occurrence of which contribute to sense making.} used are largely of an enunciative nature, since POV are a projective phenomenon; the speaker imagines situations with the eyes, sensitivity, values, knowledge, and needs of others. Points of view are instantiated in predications, which enable to hear the POV of an enunciator (the one which is in syncretism with the speaker, or another speaker) on the subject denoted by the discourse, by the choice of words, and their order, regardless of the explicit presence of a judgment (\citealt{Rabatel2009, Rabatel2016}). In other words, a POV occurs when the referenciation\footnote{\citet[34]{Danon-Boileau1982} refers to the linguistic construction of referents. The link between referencing and enunciation is consubstantial, because the choice of words and of their order gives information about the PDV of the enunciator concerning the object-of-speech.} denotes the discourse object(s) while providing information about the enunciator's point of view on the same object(s). 

\begin{itemize}

\item First, the tools are based on the locutor/enunciator distinction.\footnote{The locutor is at the origin of utterances, their pronouncing/writing; the enunciator is at the origin of the POV. These two parts often go hand in hand, but not always \citep{Rabatel2009}. Many secondary enunciators can be involved in L1/E1's discourse, and can refer to anonymous, indefinite (doxal or not) or clearly identified sources. The enunciator, at the origin of the POV, has therefore nothing to do with the notion of enunciator at the origin of the production of an utterance, which is a parasynonym of the notion of speaker.} Indeed, the attentive reader can ask: but who does this “other person” correspond to, from a linguistic point of view, in the above statement, “the way in which a speaker is able to imagine what this other person can”? In terms of linguistic instances, this \textit{other person} may be a secondary \textit{speaker or locutor} in a hetero-dialogical context\footnote{I will come back to the auto-dialogical context, when the speaker plays with his own POV, in the conclusion.}, but only if he speaks. On the other hand, if the aim is to account for the enunciator’s POV through his/her perceptions, feelings, thoughts, or actions, it can only concern a non-speaking source, in{\textmd{ \textup{sentences without words, as \citet{Banfield1995} says}}}: a \textit{secondary enunciator}, corresponding to intratextual actors, at the origin of POV \citep{Ducrot1984}. 

\item Secondly, the preferred tool concerns everything that enables to account for the subjectivity of the POV, in particular the modal values, which are essential in the perspective of intentionality analysis. Subjectivity is understood as a subjective source on the one hand, and as a manifestation of subjectivity on the other. These explicit linguistic manifestations operate through diverse expressions of subjectivity, or \textit{subjectivemes} \citep{Kerbrat-Orecchioni1980}, which can be centred on the enunciator or on the object. However, even if they are centred on the object, they refer to the enunciator's POV on the object. The analysis of subjectivity must integrate the phenomena of enunciative self-effacement and objectification, because despite the absence of \textit{subjectivemes}, subjectivity can be seen through communication strategies: thus, a speaker/enunciator may have an interest in expressing his/her POV in a generic manner that makes it less questionable.

\end{itemize}

\end{itemize}


\section{Analysis}

I illustrate this approach with hetero-dialogical examples, which I will use as a base to draw some conclusions regarding simplexity. 

\eanoraggedright(1 Samuel, 17, 42--43)\label{ex:06:1}\\
Le philistin regarda et, quand il aperçut David, il le méprisa~: \textit{c’était un gamin au teint clair et à la jolie figure}. “Suis-je un chien pour que tu viennes à moi armé de bâtons ?”
\glt `The Philistine looked and, when he saw David, he despised him: \textit{he was a light-skinned kid with a pretty face}. \textbf{“Am I a dog so that you will come to me armed with sticks?”}'.
\z 


In this excerpt\footnote{The analysis provided here is based on the French version of the verses.} three forms of POV (see below) follow one another from the character's point of view, i.e. Goliath (= le Philistin)  as it is indeed his contemptuous perspective that the narrator adopts, which cannot be shared by the first speaker/enunciator, in syncretism with the prophet Samuel who views David as the symbol of the rebirth of Jewish royalty\footnote{The hypothesis is not only based on knowledge external to the extract, it is based on the fact that L1/E1 names David by his first name, while Goliath is only mentioned by the name of the people of whom he is the champion, thus demonstrating a distance \textit{vs} proximity opposition.}. At first, Goliath is only a secondary enunciator at the origin of a point of view, in verse 42, before becoming a secondary speaker/enunciator with the direct speech in verse 43. The represented POV, in italics, clearly emerges through its opposition to the embryonic PDV (underlined), with which it is in strong contrast, because of the opposition between the foreground, with the prototypical tense of the \textit{passé simple}, and the \textit{imparfait}, in the background. The part in italics, announced by the colon and the presentative \textit{c’était}, expands the overall impression of contempt by detailing the characteristics that justify this judgment on David. This explanatory commentary, which has a secant aim, presents an internalised perception that combines apparently objective perceptual elements and judgments attached to these descriptions despite the absence of explicit judgment. The term \textit{gamin} ‘kid’ pejoratively expresses Goliath's contempt for an opponent who is so far from his expectations. As an aggravating circumstance, he is \textit{un gamin au teint clair et à la jolie figure} ‘a light-skinned kid with a pretty face’: these stereotypical qualifiers, ameliorative when they describe women, see their polarity reversed if they describe a man.

The asserted POV, in bold, corresponds to direct speech, which expresses an explicit opinion, referred to a source of speech (a secondary speaker). This POV verbalises in a rise of anger all of Goliath's indignation, offended because his opponent is not worthy of his rank. On that basis, we can consider, using backward chaining, that the indignation that explodes with strong reflexivity in the asserted POV is already announced in a minor way in the represented POV. This indignation is also implicit in the embryonic POV. Contrary to what \citet[240--241]{Benveniste1966} may have said about its objective character, the first sentence of the excerpt, which is linked to the \textit{passé simple} {and to narratives in the third person, also contains traces of subjectivity, because it describes the Philistine’s movements by associating them with thoughts and by bringing out their immediate, almost reflexive character: the} \textit{quand} ‘when’ could be replaced by ({\textit{aus}})\textit{sitôt que,}\textit{dès que} ‘as soon as, as’ in accordance with the original. This first sentence is already subjective because it renders an \textit{immediate} reaction of contempt, which unfolds in the represented POV before being expressed with Direct Speech and then with the action.

Thus, the theory of POV enables to account, in a unitary approach, for the fact that the POV can correspond either to that of the speaker/enunciator (cases of conjunction of instances) or to those of secondary speakers/enunciators, or to those of secondary non-speaker enunciators (by a distinction of enunciative instances); and it also enables to account for POV which are explicitly or implicitly expressed, with or without opinions or judgments. The advantage is therefore to read, behind what might appear to be objective statements by the narrator, the strategies by which the primary speaker/enunciator can adopt the POV of internal enunciators, and also the degrees of expression of their POV, depending on the form of the POV and the nature of the markers.

All things being equal, the same is true in \REF{ex:06:2}: the superintendent commits to the elliptical link established between the decided (\textit{franche ‘}straightforward’) handshake and the gaze, the blue eyes (= \textit{franche comme son regard bleu glacier} ‘as straightforward as his steel blue eyes’):



\ea (Quadruppani, \textit{The Sudden Disappearance of the Worker Bees}, Gallimard, Folio Noir, [2011] 2013: 144)\label{ex:06:2}\\
La commissaire lui tendit la main. La personnalité, ici, c’est vous il me semble, dit-elle [elle = la commissaire Tavianello]. La poignée de l’homme était franche comme son regard bleu glacier.\\
\glt `The superintendent held out her hand to him. “It seems to me that the personality here  is you,” she said [she = the superintendent Tavianello]. The man's handshake was as straightforward as his steel blue eyes.'
\z 


One could argue that this interpretation is forced, and that in reality, the text merely talks about the superintendent without adopting her POV, and therefore that the POV that connects straightforwardness, handshake and gaze belongs to L1/E1. The problem is that extract \REF{ex:06:2} is not complete, and that the rest of the text, quoted in \REF{ex:06:3}, completely invalidates this hypothesis: 

\ea  (\textit{Ibid}.)\label{ex:06:3}
 La commissaire lui tendit la main. La personnalité, ici, c’est vous il me semble, dit-elle [elle = la commissaire Tavianello]. La poignée de l’homme était franche comme son regard bleu glacier. (et ne critique pas cette phrase de roman sentimental, cher lecteur, car elle convient parfaitement à la douceur ingénue qui serra soudain la gorge de Simona [Simona = la commissaire Simona Tavianello]).\\
 \glt `The superintendent held out her hand to him. “It seems to me that the personality here is you,” she said [she = the superintendent Tavianello]. The man's handshake was as straightforward as his steel blue eyes. (and do not criticise this sentence coming from a sentimental novel, dear reader, because it perfectly suits the ingenious sweetness that suddenly appeared as a lump in Simona's throat [Simona = the superintendent Simona Tavianello]).'
\z


The meta-enunciative commentary in brackets shows that the narrator is playing with the beliefs of readers unfamiliar with the cunning of the POV, who believe that speakers are at the origin of all the POV that they express, especially when these POV are not part of the reported discourse. This transparent and non-problem-oriented way of reading is usual for those who underestimate, or even ignore, the theories of enunciation, or only relate them to the speaker who initiated an enunciation act. I deliberately quote a non-literary example, in oral form, to show that the previous issue is not reduced to narrative literary writings.


\ea (Bernard Joly, author of \textit{A history of alchemy}, France Culture, 11/09/2013)\label{ex:06:4}\\
      Les mathématiciens, les physiciens, ils n’ont pas besoin de laboratoire, ils théorisent, ça leur suffit (pause), disent-ils\\
\glt `Mathematicians, physicists, they do not need a laboratory, they theorise, it is enough for them (pause), so they say'
\z

The extract is delivered by a researcher, in the context of a scientific programme, and the judgment, \textit{ça leur suffit} ‘it is enough for them’, does not correspond to his judgment: it is the POV of mathematicians and physicists, as confirmed by the \textit{verbum dicendi}, after the pause. But one must admit that we are not dealing with  “classical” Direct Speech (“it is enough for us”), nor with equally “classical” Indirect Speech (“they say it is enough for them”). Nevertheless, this empathic reconstruction does correspond to a shift from the primary speaker towards the secondary enunciators’ POV. The question as to whether or not the speaker agrees with this POV – which the extract does not allow to decide – can only be asked after the attribution of this POV to the secondary enunciators. 

\section{Conclusion}

These four examples all illustrate the interest of the locutor/enunciator distinction; they also illustrate to varying degrees the fact that the sensorial is associated with what is sensitive, to affects, thoughts, possibly language and action, without it being necessary to summon them all in all situations, by virtue of a principle of reality that is primarily scriptural, and which results from the stakes and needs of enunciators in this kind of situation, and with this kind of co-text. Finally, they also illustrate the fact that this referenciation, carried out by the primary speaker, is able to verbalise, and even show, multimodal attitudes and interactions, which contributes to an additional level of complexity, that of represented and shown complexity in discourse \citep{Rabatel2013}. 

In reality, \textit{if one agrees that there is a continuum between pre-reflective and reflective, between sensory and sensitive, between intelligibility (what can be imagined, consented, said) and praxis, this amounts to thinking in terms of complexity. Accounting for this continuum using the theory of POV is a way of giving a simplex account of what is complex:} indeed, based on the speaker/enunciator distinction, one can simplexly account for a set of complex facts which are articulated within the framework of a unitary approach. In other words, my POV approach positions the locutor and enunciator in relation to complex and heterogeneous data, that is sensorial, emotion-based, reasoned, and praxeological, resulting in a method that is unitary and economical:

\begin{itemize}\sloppy
\item The fact that a primary locutor/enunciator can change his or her enunciative position, change place, temporality, theoretical framework, or value system to comprehend objects of the world, through their discursive construction and thus the referentiation operations, on behalf of others, whether they speak or not, formulate opinions or judgments or not.

\item The fact that the inferential analysis of referentiation enables the discrimination of the source of the POV and the nature of the relationships, the path between the various modules (perceptual, sensitive, etc.) and, ultimately, the nature of the intentional link within them.

\item The fact that this same referentiation, which combines truth value, modal values and direct or indirect language acts, refers to two levels of modality and modalisation\footnote{It follows that modality is not reduced to the modus, it also crosses the dictum, as clearly shown by \citealt{Ducrot1993}.}, the first one being concerned with the primary locutor/enunciator, the second one with the secondary locutor/enunciators or the secondary non-speaker enunciators -- this second level is largely underestimated, but partially taken into account by \citet{Gosselin2010}. 

\item The fact that, in parallel with this phenomenon of modal diffraction, one must consider an equal phenomenon of commitment and quasi-commitment for secondary enunciators’ POV, for which one presupposes that it took place before the scene of the text \citep{Rabatel2009}.
\end{itemize}

It has yet to be shown that empathic dialectics works on the relationships between otherness and identity, without reducing otherness to the het\-ero-dia\-log\-ical persons \textit{that are not} oneself, but by opening up to other auto-dialogical aspects \textit{of} oneself \citep{Rabatel2016}, which would imply the study of utterances in the first person. At the same time, simplexity would open up a new field, that of relations between the primary/speaker and others, between his/her POV and those of others, in a polyphonic and dialogical framework (\citealt{Rabatel2008a}: 361--380). But it seemed strategic to start by highlighting utterances in the third person, which may seem objective in terms of denotation and subjective in accordance with the disjunction of the modal values they express and enable to infer.

\section*{Acknowledgements}
The author is grateful to the ASLAN project (ANR-10-LABX-0081) of the Université de Lyon, for its financial support within the French program “Investments for the Future” operated by the National Research Agency (ANR).

{\sloppy\printbibliography[heading=subbibliography,notkeyword=this]}
\end{document} 
