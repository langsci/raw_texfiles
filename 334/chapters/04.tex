\documentclass[output=paper]{langscibook} 
\ChapterDOI{10.5281/zenodo.6620111}
\author{Kristine Lund\affiliation{UMR 5161 ICAR; Centre National de la Recherche Scientifique; Ecole Normale Supérieure; Université Lumière Lyon 2}}
\title[{Modelling} the co-elaboration of knowledge]
      {{Modelling} the co-elaboration of knowledge: Connecting cognitive, linguistic, social and interactional systems}
\abstract{\sloppy Language and multimodal human interaction in context are central in modelling knowledge co-elaboration. Cognitive, linguistic, social, and emotional aspects are closely imbricated within such interaction, and analyses can target individual, small group, organisational, or cultural level phenomena. Given that many research questions in this area lead to crossing boundaries, I argue for both an interdisciplinary and a complex systems approach in constructing a new descriptive model entitled the MULTi-theoretical and Interdisciplinary model of the GRoup And Individual (MULTi-GRAIN).  This model allows for the study of different types of unidirectional and bi-directional causality and is meant as a guide for setting up empirical work where emergence is studied in systems of different orders.}
\IfFileExists{../localcommands.tex}{
  \addbibresource{localbibliography.bib}
  \usepackage{langsci-optional}
\usepackage{langsci-gb4e}
\usepackage{langsci-lgr}

\usepackage{listings}
\lstset{basicstyle=\ttfamily,tabsize=2,breaklines=true}

%added by author
% \usepackage{tipa}
\usepackage{multirow}
\graphicspath{{figures/}}
\usepackage{langsci-branding}

  
\newcommand{\sent}{\enumsentence}
\newcommand{\sents}{\eenumsentence}
\let\citeasnoun\citet

\renewcommand{\lsCoverTitleFont}[1]{\sffamily\addfontfeatures{Scale=MatchUppercase}\fontsize{44pt}{16mm}\selectfont #1}
   
  %% hyphenation points for line breaks
%% Normally, automatic hyphenation in LaTeX is very good
%% If a word is mis-hyphenated, add it to this file
%%
%% add information to TeX file before \begin{document} with:
%% %% hyphenation points for line breaks
%% Normally, automatic hyphenation in LaTeX is very good
%% If a word is mis-hyphenated, add it to this file
%%
%% add information to TeX file before \begin{document} with:
%% %% hyphenation points for line breaks
%% Normally, automatic hyphenation in LaTeX is very good
%% If a word is mis-hyphenated, add it to this file
%%
%% add information to TeX file before \begin{document} with:
%% \include{localhyphenation}
\hyphenation{
affri-ca-te
affri-ca-tes
an-no-tated
com-ple-ments
com-po-si-tio-na-li-ty
non-com-po-si-tio-na-li-ty
Gon-zá-lez
out-side
Ri-chárd
se-man-tics
STREU-SLE
Tie-de-mann
}
\hyphenation{
affri-ca-te
affri-ca-tes
an-no-tated
com-ple-ments
com-po-si-tio-na-li-ty
non-com-po-si-tio-na-li-ty
Gon-zá-lez
out-side
Ri-chárd
se-man-tics
STREU-SLE
Tie-de-mann
}
\hyphenation{
affri-ca-te
affri-ca-tes
an-no-tated
com-ple-ments
com-po-si-tio-na-li-ty
non-com-po-si-tio-na-li-ty
Gon-zá-lez
out-side
Ri-chárd
se-man-tics
STREU-SLE
Tie-de-mann
} 
  \togglepaper[1]%%chapternumber
}{}

\begin{document}
\maketitle 

\section{Elaborating knowledge with others}
\subsection{Why use an interdisciplinary approach to modelling knowledge co-elaboration?}

We elaborate knowledge with others in all areas of our lives. These others are people with experiences and views that are often different than our own. In the same way that accounting for a variety of perspectives leads to a broader understanding and perhaps better solving of the tasks we are confronted with in our lives, it makes sense to call upon different disciplinary views when we want to study and understand such knowledge elaboration.

Many disciplines have sought to understand how the individual, other people, and the context, all influence collaborative knowledge elaboration, be it individual or group knowledge. It is outside the scope of this chapter to review such a vast body of research, but see §5, \textit{A cross disciplinary analysis of the individual versus the group in learning contexts}, in \citet{Lund2016} regarding this topic at the frontiers of sociology and language sciences, and within psychology. This section also performs a meta-review of numerous studies of learning (assimilated to co-elaboration of knowledge\footnote{Learning has many different definitions: a physical response to a stimulus, or a mental process in the brain, possibly due to interactions with the environment. The definitions that can be more easily compared to co-elaboration of knowledge include learning as an interaction between a child’s individual maturation and a system of symbolic tools and activities that the child appropriates from his or her sociocultural environment or learning as a set of processes at the small group level that take place through the weaving of semantic and indexical references within a group discourse.}) regarding the individual, the small group, or a community during short, mid, and long-term timescales. These studies take place within behaviourist, cognitivist, sociocognitivist, or sociocultural paradigms and despite this variety, it is possible to pinpoint how disciplinary approaches may complement each other. For example, interdisciplinary work would be useful between conversation analysis and psychology in the sociocultural paradigm in order to combine analyses of interaction-in-context with characteristics of individuals. In the sociocognitive paradigm, work in organisational learning could be combined with microsociological studies of situated social practice that zoom in at different time periods. In the first example, the way a phenomenon is embedded in a process is combined with the study of individual behaviour as explained by these individuals’ characteristics. In the second, different levels of analysis are combined (i.e. macro descriptions of organisational change and micro descriptions of localised human interactions). Achieving such a broad view requires a particular approach, developed in the next section.

\subsection{Why use a complex systems approach to modelling knowledge co-elaboration?}

Researchers may focus on particular and narrowed aspects of co-elaboration of knowledge due to disciplinary boundaries, epistemological assumptions, methodological approaches or even societal impact objectives. But in order to better understand a phenomenon of interest, its various characteristics need to be studied in concert. Multiple levels of analysis need to be taken into account (individual, small group, organisation/community/culture) and multiple aspects (e.g. cognitive, linguistic, social and interactional) of the phenomenon need to be studied. 

A complex systems approach allows researchers to consider these aspects as systems in and of themselves, yet also to unite them together. One definition of complexity puts it this way:

\begin{quote}
Let us go back to the Latin root complexus, which means “entwined” or “embraced”. This can be interpreted in the following way: in order to have a complex you need: 1) two or more distinct parts, 2) that are joined in such a way that it is difficult to separate them. Here we find the basic duality between parts which are at the same time distinct and connected. Therefore, the analytical method alone won’t allow us to understand a complex, as by taking apart the components it will destroy their connections (\citealt{GershensonHeylighen2005}:  48).
\end{quote}

Many disciplines frame their empirical objects in terms of systems. There are linguistic systems \citep{Grinevald2001}, cognitive systems \citep{Woods1986}, social systems \citep{Parsons1951} and interaction systems \citep{VissersEtAl2016}, just to name a few. Each of these systems has both distinct and connected parts, according to the researchers who study them. And in many cases, research questions arguably stay within the boundaries of the defined system. For example, in the case of linguistic systems, one could ask: do all languages in the same family have similar proportions of lexical meaning versus units that have grammatical meaning? Or given a cognitive system, one could ask: how does one model changes in a belief system when the initial state contains opposing, yet co-occurring beliefs?\largerpage

That said, given the complex phenomena researchers are often interested in, it is usually difficult to stick to one system, and many questions lead to the consideration of hybrid systems such as socio-cognitive systems \citep{NoriegaEtAl2014}, sociolinguistic systems \citep{Hymes1967}, socio-emotional systems \citep{PankseppEtAl2002} and to attempts to integrate new elements into a single system.\footnote{See the book \textit{Action and Language Integration in Cognitive Systems}, edited by Angelo Cangelosi.} Questions posed at the boundaries of systems include: how does the structure of an organisation change the way newcomers understand the way it functions? Or how is second language learning influenced by the professional expertise of immigrant groups?

It is therefore not surprising to argue that in order to broadly describe and understand the co-elaboration of knowledge in different contexts, versions of at least the above systems are necessary (though biological, neurological and other systems also come to mind). These separate systems need to be connected in order to meaningfully understand the forces behind knowledge co-elaboration. Our empirical object thus becomes a system of interrelated systems, but if such a complex object gives new opportunities for broader understanding, it also poses serious methodological challenges. 


 \subsubsection{Methodological challenges of studying complex systems}

A major risk of studying systems of systems is \textit{conceptual chaos}. One definition of this is making errors during agglomeration of results. This latter can happen when two researchers use the same category name, but put different elements into it or when the same element is put into different categories. Both are a common occurrence in interdisciplinary work, but cause problems during agglomeration of results. \citet{Longino2013} gives an example concerning the category “environment” for studying what affects a fetus in the prenatal stage. In the behavior genetic approach, what happens in the uterus (the “uterine effect”) is independent of environment after birth, of neurological factors, and of rearing children. Yet it is still the environment because the uterus is a context within the body. But in the social environment approach, any effect of the uterus is rather a part of the biological category because their main search for causality focuses on social aspects of the environment. These researchers would not place the “uterine effect” into the causal space of the environment. So when another researcher is attempting to agglomerate results from studies of different disciplines, she must be careful to respect how the causal space was parsed in each study and not add effects to the category “environment” that were not meant to be there originally.

A similar agglomeration danger around the term “multimodal” is present in work at the frontier of language sciences, computer science, and discourse analysis. In multimodal discourse analysis \citep{Paltridge2012}, much of which is based on work from \citet{Halliday1978} in social semiotics, words combine with pictures, film, video, images, and sound in order to make meaning. Computer scientists refer to human-computer interaction modalities and also include video, images, talk, and text, but add speech recognition, vision-based gesture recognition, eye-tracking, electroencephalographs, touchpad pointing, and pen use (\citealt{Oviatt2002,SharmaEtAl1998}), including not only interaction, but interpretations and analyses of modes of human computer interaction. Interactional linguistics uses the term “multimodal resources” to include gesture, gaze, and talk (and its characteristics of prosody, lexis and grammar), but also entire bodies as well as multiple bodies interacting within a material and spatial environment \citep{Mondada2016}. She argues for a view of modalities as “constitutively intertwined and language as integrated within this plurality as one among other resources, without any a priori hierarchy” \parencites[338]{Mondada2016}, a view compatible with a science of complexity approach. Yet, the term multimodal is used to cover a large variety of phenomena, varying with the discipline, so a similar danger is present during the agglomeration of results.

Another way to court conceptual chaos is to switch haphazardly between levels (individual and groups of different sizes) and aspects (e.g. linguistic, cognitive, interactional, social) of analysis during exploration of causality.  The elaboration of a framework for the study of individual and group knowledge elaboration can take a lesson from sociology: 

\begin{quote}
One important measurement problem in sociology concerns the two levels on which sociologists must work: the level of the individual and that of the group. We have observations at two levels, concepts at two levels, and relationships at two levels. Furthermore, it is necessary to shift back and forth: measuring group-level concepts from individual data; or inferring individual relationships from group-level relations (\citealt[84]{Coleman1964}, cited by \citealt[141]{Singer1968}).
\end{quote}

This chapter proposes a complexity framework where shifting between levels and aspects is controlled by an intermediate variable, defined by the researcher, thus allowing for the description and the \textit{prediction} of the connection between aspects and between levels of analysis, according to the researcher’s objectives and assumptions.

Assuming that referring to systems of systems is a fruitful way for studying the co-elab\-o\-ration of knowledge, this paves the way for the study of emergence, one of the central tenets of complexity theory. This is the claim that particular kinds of systems are capable of giving rise to radically new properties not present in the components of the system \citep{BechtelRichardson2010}, a phenomenon that is inherent in human interaction. Another central tenet of complexity theory is that multiple, simultaneous, non-linear interactions can take place between components. How can these be studied? Some disciplinary areas have a variable-oriented view on causality while others have a process-oriented view. In the first view, one searches for the consequences attributable to deliberately varying a treatment. The second is operationalised by “clarifying the mechanisms through which and the conditions under which the causal relationship holds” (\citealt{ShadishEtAl2002}: 9).  Decomposition and localisation \citep{BechtelRichardson2010} can help pull apart causality in either case. This includes differentiating component parts, identifying component operations and linking the operations with the parts (\citealt{BechtelAbrahamsen2005}).

\section{MULTi: Theoretical and interdisciplinary model of the group and individual} 
\subsection{The grains of a tree in the forest: A metaphor}

The MULTi-GRAIN model for the co-elaboration of knowledge stands for MULTi: Theoretical and Interdisciplinary model of the Group And Individual (\citealt{Lund2016,Lund2019}). In a context where the objective is to model how individuals and groups co-elaborate knowledge together, the cognitive, linguistic, interactive, and social systems that compose this activity can be compared to interwoven grains in the bark or leaves of a tree (see \figref{fig:4:1a}). According to recent research on tree ecosystems \citep{BaderLeuzinger2019}, trees exchange water, carbon, mineral nutrients, and microorganisms through their roots, often keeping neighboring tree stumps alive or helping trees that are struggling (see \figref{fig:4:1b}). Such resources are both used to nourish the individual tree as well as the surrounding trees that are part of the larger forest.  So if I push the metaphor further, if the tree is an individual and the forest the group, then the cognitive, linguistic, interactive, and social aspects of co-elaboration of knowledge are interwoven together in the fabric of the individual (i.e. in the wood or leaf grains of the tree), but also in the other trees in the forest. These interwoven aspects are mutually influenced by the exchanges between trees which can be compared at a large grain level to the constructs I have labelled as intermediate variables, in collaborative work on the study of knowledge co-elaboration, such as semiotic bundle \citep{LundBécu-Robinault2013}, procedural explanation \citep{Mazur-PalandreEtAl2014}, overall emotional framing of a debate \citep{PoloEtAl2016}, and level of collaboration in a community of practice \citep{EberleEtAl2013}. But at a finer grain level, the exchanges between trees can also be compared to the various facets of human interaction, as described in the next section. Following \citet{Coleman1964}, the relationships we observe and the concepts we define occur both at the tree and forest level. But I propose to differentiate the component parts, identify the operations and link them to the parts, whether they be at the tree or forest level.

\begin{figure}
\subfigure[~]{\label{fig:4:1a}\includegraphics[width=.4\textwidth]{figures/a4Lundfinal-img001.jpg}}
\subfigure[~]{\label{fig:4:1b}\includegraphics[width=.4\textwidth]{figures/a4Lundfinal-img002.jpg}}
\caption{Figure 1a on the left is tree bark, the interwoven grains of which are a metaphor for the cognitive, linguistic, interactive, and social aspects of co-elaboration of knowledge. On the right, the forest is a metaphor for how these aspects of co-elaboration are instantiated at the group level.}
\end{figure}


Given that the MULTi-GRAIN model is completely open-ended, resources (or intermediate variables and facets of human interaction) vary with the contexts studied and are chosen according to researchers’ objectives, worldview and focus. Having up to this point argued for the interest of such a model, the next section presents the model’s structure and its possibilities for interdisciplinary research that broadens the understanding of the co-elaboration of knowledge.

\subsection{Structure of the MULTi-GRAIN model}

In his book \emph{Monde Pluriel} (A Plural World), the French sociologist Bernard Lahire (2012) argues that although it is true that the diversity in the human and social sciences has part of its origin in the way that researchers construct their objects of study, this is not the only reason for the scattered and dissipated nature of the research in this field. It also is a result of the social division of scientific work into disciplines (e.g. the sciences of “language”, “psyche” or “society”) and further into specialties within disciplines. Such a division means that researchers of different ilk separately study each domain of practice or sector of social life and form parallel theories of the actor. Lahire asks three questions that emanate from this state of affairs (\citealt{Lahire2012}: 11, my translation):

\begin{enumerate}
\item How can we obtain a global view of the social world if each researcher must keep his or her nose glued to the functioning of his or her small world parcel?
\item How can we conserve a complex conception of individuals in society when disciplinary boundaries and within those, internal specialties constrain researchers to work on the dimensions that are particular to narrow practices?
\item How can we maintain a high level of scientific creativity when a narrow vision of professional research leads to hyper specialisation and a normalisation of research and researchers?
\end{enumerate}

When Lahire asks – rhetorically – if it is possible to understand the invention of the economic market without taking into account how economy relates to law, religion, politics and culture, I take a similar stance and ask how it is possible to obtain a broad understanding of knowledge co-construction while considering only one specific discipline. These are approaches that take the stance of interdisciplinarity as types of integration between separate disciplines \citep{Klein1990}.

In his own academic context centered in sociology, Lahire’s goal is to obtain a global view of the social world and in order to do so, he asks the following question: why do individuals do what they do, think what they think, feel what they feel and say what they say? He works to answer this question by attempting at the origin, a combination of different research foci in sociology – those focused on actors’ inherent proprieties and those who focused on context. He uses a “formula” to describe his approach, evolved from a criticism he makes of the one proposed by \citet{Bourdieu1979}: habitus\,+\,field = practices: Incorporated past of the actor (dispositions or competencies)\,+\,context of the present action = observable practices.

\begin{sloppypar}
This formula corresponds to the interdisciplinary analytical model called MULTi-GRAIN that I propose below, standing for MULTi-theoretical and Interdisciplinary model of the GRoup And Individual. I built this model from work carried out over many years with colleagues in language sciences, education and psychology \citep{Lund2016}. The human interactions I observed and analysed originally with colleagues were the result of both dispositions and competencies and the context of the action participants carried out their activities. The elements of this formula are found in various forms in the wider literature that treats the co-construction of knowledge.
\end{sloppypar}

\begin{figure}
\includegraphics[width=0.66\textwidth]{figures/lund.pdf}
\caption{The structure of the model MULTi-theoretical and Interdisciplinary model of the GRoup And Individual\label{fig:4:2}}
\end{figure}


The intermediate variable is at the center of this model (see \figref{fig:4:2}) and researchers define them according to their goals and focus. The criterion for an intermediate variable is that it can be qualified within at least two different systems in a way that connects them.

Researchers also define facets according to the goals and foci of their research questions. The collaborative work used to elaborate the MULTi-GRAIN model included the study of modes of expression such as talk and gesture, drawing, or manipulation of experimental apparatus in physics labwork, choice of argumentative claim during socio-scientific debates, emotional positioning of such claims, self-identity footing (e.g. consensual, competitive), group talk type (following \citealt{WegerifMercer1997}: disputational, cumulative, or exploratory) schematisation \citep{Grize1996}, tonality of discourse objects, facework \citep{Goffman1967}, subjects’ expressed feelings, propositions, lexical information, common ground, shared understanding, pragmatic competence, knowledge about an academic community, community participation, and duration of face-to-face interaction. These facets were all operationalised in research questions concerning the above intermediate variables and the co-elaboration of knowledge in human interaction \citep{Lund2016,LundBécu-Robinault2013,PoloEtAl2016,Mazur-PalandreEtAl2014,EberleEtAl2021}.

Both intermediate variables and the facets that compose them belong to one or more complex systems. The systems I consider are linguistic, cognitive, interactional, and social. Systems may overlap, depending on how researchers define them. For example, interacting facets can belong to the same system (e.g. talk and gesture belong both to the linguistic system and to the interactional system).

The MULTi-GRAIN model allows for the study of four types of unidirectional causality and two types of bi-directional causality as shown in \figref{fig:4:2}. The first of the unidirectionals is when one facet influences another through an intermediate variable (e.g. pragmatic competence influences shared understanding through explanation). The second is where an intermediate variable influences a facet (e.g. the quality of an explanation influences shared understanding). The third is where a facet influences an intermediate variable (e.g. the completeness of pertinent lexical information influences the quality of an explanation). The fourth is where one facet directly influences another (e.g. reformulating content during problem solving across modalities (talk, gesture) influences success of manipulation of experimental apparatus). Bi-directional causality is where two facets directly influence each other (e.g. facework influences which of the subject’s feelings will be expressed in the group) or where two facets influence each other through an intermediate variable (e.g. development of the pragmatic competence of making sure a listener is following and shared understanding are reciprocally influenced through explanation). Using the MULTi-GRAIN model as a guide, empirical work can be set up to study the extent to which these relationships are correlational or causal, and how they develop over time, both for the individual and group.

\subsection{How the MULTi-GRAIN model informs thinking on language complexity}

Taking the example of interactive finalised procedural explanation \citep{Mazur-PalandreEtAl2014}, the bi-directional causality of the co-construction of talk and gesture, be it individual or within interaction, is analysed from three system perspectives (linguistic, cognitive, and interactional) and we track the development of these three competencies in three different age groups.

In terms of modelling individual and group knowledge co-elaboration, the analysis of young children’s talk and gesture showed that they had difficulties in managing the interaction with their peers when they gave an instructional explanation for a finalised task. For example, they rarely asked if their interlocutor understood the explanation, if he or she was paying attention, or if he or she had any questions about the rules. In more recent work, we illustrated \citep{Mazur-PalandreEtAl2019} that both completeness of a procedural explanation for playing a collaborative game and information content conveyed in a multimodal manner increased with age, as did number of non-verbal verifications of shared understanding. In general, this approach proves useful for better understanding the feedback processes present during the emergence of cognitive and interactional competencies in procedural explanations as children’s language develops.

\figref{fig:4:3} illustrates how both speaker and listener behaviour is reciprocally influenced by perceptual, social, and cognitive constraints, as well as by social and cognitive motivations. In addition, such behaviour is reciprocally influenced by past speaker and hearer interactions, they being of course not equal \citep{BecknerEtAl2009}. This description closely echoes \citet{Lahire2012} in terms of competencies, context, and observable practices, thus conforming to the MULTi-Grain model.

\begin{figure}
\begin{tabular}{@{}>{\centering}p{3cm}r c l>{\centering\arraybackslash}p{3cm}@{}}
\multicolumn{2}{c}{Past interactions of \tikz [remember picture, baseline=(04Speaker.base)] \node [inner sep=0pt] (04Speaker) {speaker};} & & \multicolumn{2}{c}{\tikz [remember picture, baseline=(04PastInt.base)] \node [inner sep=0pt] (04PastInt) {Past}; interactions of hearer}\\
Perceptual, cognitive, and \tikz [remember picture, baseline=(04SocConstr.base)] \node [inner sep=0pt] (04SocConstr) {social constraints}; &  & \multirow{3}{*}{\Large ≠} &  & Social and cognitive \tikz [remember picture, baseline=(04Mot.base)] \node [inner sep=0pt] (04Mot){motivations};\\
\multicolumn{5}{c}{\tikz [remember picture, baseline=(04Speaker2.base)] \node [inner sep=0pt] (04Speaker2) {Speaker}; and \tikz [remember picture, baseline=(04Listener.base)] \node [inner sep=0pt] (04Listener) {listener};}
\begin{tikzpicture}[remember picture, >={Triangle[]}]
\draw[overlay, <->] (04Listener) -| (04Mot);
\draw[overlay, <->] (04Speaker2) -| (04SocConstr);
\draw[overlay, <->] (04Listener) -- ++(0,1.66cm);
\draw[overlay, <->] (04Speaker2) -- ++(0,1.66cm);
\end{tikzpicture}
\end{tabular}
\caption{A representation of the complexity of speaker and listener behaviour including the possibilities for intermediate variables and facets of human interactions.}
\label{fig:4:3}
\end{figure}

\section{Conclusions and further work}

In conclusion, the MULTi-theoretical and Interdisciplinary model of the GRoup And INdividual was developed as a general framework for understanding the co-elaboration of knowledge at the individual and group level. It is also adaptable to other topics of research that can be framed as a complex adaptive system and studied at both the individual and group level. The MULTi-GRAIN model takes on a complexity sciences framework where intermediate variables – chosen by researchers as a function of their assumptions and interests – connect systems of different orders: linguistic, cognitive, social, and interactional systems. These intermediate variables connect facets of human interaction, also defined by researchers as a function of their assumptions and interests. The MULTi-GRAIN model allows for the study of individual and collective emergence as well as perpetual dynamics within a system of systems of different orders. 

Future work will focus on extending the reach of the MULTi-GRAIN model both by applying it to other datasets on knowledge co-elaboration in order to deepen understanding and to other types of human interaction in order to extend the model’s application. In both cases, it is possible to pull apart causality by differentiating component parts, identifying component operations and linking the operations with the parts. In contexts where development is studied, the MULTi-GRAIN model can pinpoint emergence of aspects of human interaction within a system of perpetual dynamics.

\section*{Acknowledgements}
The author is grateful to the ASLAN project (ANR-10-LABX-0081) of the Université de Lyon, for its financial support within the French program “Investments for the Future” operated by the National Research Agency (ANR).

{\sloppy\printbibliography[heading=subbibliography,notkeyword=this]}
\end{document} 
