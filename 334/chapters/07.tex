\documentclass[output=paper]{langscibook} 
\ChapterDOI{10.5281/zenodo.6620117}
\author{Antonino Bondì\affiliation{Università di Catania}}
\title[The morphogenesis of language action]
      {The morphogenesis of language action: Complexity and rhythmic synchronisation of enunciation}
\abstract{In this chapter, we will discuss two consequences of the fact that speech is both an object of study and a theoretical concept. First, observing its functioning leads to contemplating the limits of linguistic knowledge as necessary elements fall outside of the dimension of language. Secondly, questioning speech as a complex spatiotemporal phenomenon or event raises questions about what is indeed knowable within the frame of a linguistic description and theory. Our focus on two orientations of the scientific description of linguistics allows us to take enunciation as an example and use it to argue for epistemological reductionism, and a morphodynamic approach that promotes an intersection between descriptive and explanatory levels of analysis. This discussion allows us to propose language action as an instance of a complex microsystem that can be modelled and the example of enunciation as a complex synchronisation.}
\IfFileExists{../localcommands.tex}{
  \addbibresource{localbibliography.bib}
  \usepackage{langsci-optional}
\usepackage{langsci-gb4e}
\usepackage{langsci-lgr}

\usepackage{listings}
\lstset{basicstyle=\ttfamily,tabsize=2,breaklines=true}

%added by author
% \usepackage{tipa}
\usepackage{multirow}
\graphicspath{{figures/}}
\usepackage{langsci-branding}

  
\newcommand{\sent}{\enumsentence}
\newcommand{\sents}{\eenumsentence}
\let\citeasnoun\citet

\renewcommand{\lsCoverTitleFont}[1]{\sffamily\addfontfeatures{Scale=MatchUppercase}\fontsize{44pt}{16mm}\selectfont #1}
   
  %% hyphenation points for line breaks
%% Normally, automatic hyphenation in LaTeX is very good
%% If a word is mis-hyphenated, add it to this file
%%
%% add information to TeX file before \begin{document} with:
%% %% hyphenation points for line breaks
%% Normally, automatic hyphenation in LaTeX is very good
%% If a word is mis-hyphenated, add it to this file
%%
%% add information to TeX file before \begin{document} with:
%% %% hyphenation points for line breaks
%% Normally, automatic hyphenation in LaTeX is very good
%% If a word is mis-hyphenated, add it to this file
%%
%% add information to TeX file before \begin{document} with:
%% \include{localhyphenation}
\hyphenation{
affri-ca-te
affri-ca-tes
an-no-tated
com-ple-ments
com-po-si-tio-na-li-ty
non-com-po-si-tio-na-li-ty
Gon-zá-lez
out-side
Ri-chárd
se-man-tics
STREU-SLE
Tie-de-mann
}
\hyphenation{
affri-ca-te
affri-ca-tes
an-no-tated
com-ple-ments
com-po-si-tio-na-li-ty
non-com-po-si-tio-na-li-ty
Gon-zá-lez
out-side
Ri-chárd
se-man-tics
STREU-SLE
Tie-de-mann
}
\hyphenation{
affri-ca-te
affri-ca-tes
an-no-tated
com-ple-ments
com-po-si-tio-na-li-ty
non-com-po-si-tio-na-li-ty
Gon-zá-lez
out-side
Ri-chárd
se-man-tics
STREU-SLE
Tie-de-mann
} 
  \togglepaper[1]%%chapternumber
}{}

\begin{document}
\AffiliationsWithoutIndexing{}
\maketitle 

\section{The problem of complexity in linguistics between \textit{knowable} and \textit{unknowable items}}

In his \textit{Introduction à la littérature fantastique}, the linguist and narratologist Tzvetan \citet{Todorov1970} reflects upon the epistemological issues of semiolinguistic disciplines when dealing with an apparently common phenomenon such as the exercise of speech. For Todorov, the study of speech in practice forces linguistic theories to ask two major questions. On the one hand, since speech is both an object of study and a theoretical concept, the observation of its functioning, as well as the study of its formation and constitution phases, entails as a consequence an investigation on the limits, or better on the perimeter itself, of linguistic knowledge. Todorov noted that speech would not only enable to make a series of useful epistemological distinctions in order to capture the constant and invariant dimensions proper to the activity of speaking. It would also be able, as a theoretical concept, to play an important role in the design of an observation and analysis method, which could go beyond the simple observation of the phenomenological density of enunciation and the complexity of its manifestation in the concrete occurrence of an utterance. Consequently, the analysis of speech in practice would enable to discern the features exhibiting a regularity dimension, in relation to the other parameters which, even though they are necessary to the instantiation of the enunciative scene, do not fall within the language dimension. On the other hand, the questioning of speech as a complex spatiotemporal phenomenon or event also raises questions about the status of what is actually knowable, within the frame of a linguistic description and theory. 

Language activity, as reconstructed from the praxis of interlocution, can be characterised in the first instance as a connection between two or more participants in the game of interlocution. This game cannot be captured and described independently of the praxeological frame by which it is oriented. Such a connection is also a way to set the context: its deployment takes place in one or more phases and within one or more spaces. The enunciation process consequently consists in the construction of a complex web of linguistic elements alluding to these two parameters (space and time), as well as to other modalities that organise the enunciative scene. Todorov draws an instructive conclusion from this, when he writes:

\begin{quote}
The exercise of speech is not an individual and chaotic activity, and hence unknowable. There is an irreducible part in enunciation, but next to it there are others which can be conceived as repetition, play, convention. Our object is therefore constituted by the rules of enunciation and the different scopes of their application. (\citealt{Todorov1970}: 3, personal translation)
\end{quote}


Linguistic theories have always favoured the investigation and description of forms whose nature is phenomenological \citep{Piotrowski2018}: they are perceived and distinguished by the speakers themselves, in the spontaneous course of action, within the semiolinguistic use, which is concrete, transmitted and socialised. This epistemological preference is possible provided that these forms can acquire the rank of object. In other words, such forms must be objectivised and, by virtue of this gnoseological operation, assume the status of a fully knowable object. Consequently, the scientific description of linguistics could have been oriented towards at least two directions: 


\begin{itemize}
\item predict the potential for the usage, updating and use of such object-forms, with the belief that \textit{all occurrences}\footnote{It is the structuralist dream of glossematics. See \citet{Bondi2012}.} can be explained and described;

\item collect the diversity of uses and variations, according to observable occurrences.

\end{itemize}


Only then can a second-order observer – such as the linguist – engage in interpretative hypotheses and refute others considered as less plausible.


\section{On the need for epistemological reductionism}

Like any scientific approach, linguistics has understood the advantages of introducing a set of epistemological reduction operations: first, the identification of the levels in which linguistic objects and language operations are organised. A cutting operation for analytical purposes of this type\footnote{For a discussion on the semiotic epistemology of \textit{cutting}, see \citet{Paolucci2010,Paolucci2017,Bondi2011}.} is based on the idea that a phenomenon such as enunciation is organized around two typologies of components, which govern its construction and emergence:


\begin{itemize}
\item elements that can be reduced to regular forms (or formants), which can in turn be objectivised and delimited;

\item dynamics of constitution which, even if they are essential for the semiological life of the interactive and praxeological games, remain unknowable from the linguistic point of view, because of their psychological (or even subjective) and chaotic (i.e. \textit{non-deterministic}) nature.

\end{itemize}

\begin{sloppypar}
The above-mentioned remark of Todorov explains the reductionist\footnote{This attitude could be defined as \textit{necessarily reductionist}.} attitude that is necessarily assumed when confronted with the phenomenological complexity of semiolinguistic action. Through this, the distinction is made between linguistic traces -- and the symbols with which they are carried and transmitted -- and the spontaneous phases (or moments) of creation and constitution of the language action itself. From Todorov’s\footnote{The remark is here caricaturised as an example; it is not a specific interpretation of Todorov's position, which is in fact much finer.}  point of view, only the former would deserve to be analysed, because they are regular, social and conventional elements, whereas construction phases would constitute, to use Jean \citeauthor{Petitot-Cocorda1985}'s (\citeyear{Petitot-Cocorda1985,Petitot-Cocorda1992}) famous formula, the \textit{partie maudite} of the language experience. These dynamic or chaotic phases are related to the spontaneous morphogenesis of linguistic action, i.e. the putting into words of a network of communicative intentions. These are the general aspects of the implementation of enunciation, which are rooted in the nature of the various substrates that constitute the dynamic matter for language formation. Such a distinction between the forms of objects and the forces that affect their constitution reflects that enunciation is conceived as an exercise or application of pre-established \citep{Ingold2013} rules. Incidentally, in the history of linguistic theories, enunciation theory has advocated since Benveniste (and at least up to Culioli) an almost absolute distinction between the formal apparatus of enunciation and communicative agreements, where the chaotic dimensions of meaning and its expressiveness overflow (see \citealt{Bondi2016,Ducard2012,Longhi2012}).

Therefore, such a separation is certainly justified from an epistemological point of view; but, when the phenomenological complexity of a language act in the ordinary experience of one or more speakers is pointed at, two theoretical problems appear. First, it is necessary to specify and define what is meant by complexity. Are we targeting linguistic forms (and their stratified organisation)? Or is our interest more focused on social behaviour? The two questions open onto two different conceptions of semiotic and linguistic complexity. The first approach focuses on forms and forces in terms of dynamic objects, which can be modelled by using an imaginary world that is derived from the theories of complex systems, while the second one focuses on taking into account the heterogeneous domains that explain any given semiotic behaviour.\footnote{On the first option, which brings together theory of dynamic systems and semiolinguistic theory, see \citet{Bondi2015,Bondi2017,Piotrowski2018}. The second option, which rather belongs to semiotics, is discussed in \citet{BassoFossali2017}.} The first interpretation will be favoured here.
\end{sloppypar}

\section{The morphodynamics of spontaneous speech: Epistemological remarks} 

A perspective oriented towards morphodynamics focuses on the semiogenetic phases of language action. According to the linguist and semiotician Wolfgang Wildgen, this perspective questions the distinction between the recognisable dimension of enunciation (regular, social and conventional), and the more subjective and chaotic one, which, on the contrary, would remain unknowable, because it is at stake in the spontaneous morphogenesis of speech. Two arguments support this idea. 

First, the morphodynamic approach emphasises the necessity of the intersection between analysis levels and object sizes, in order to hold together the study of forms and also the forces that organise them. This intersection should take place at both the description and explanatory modelling levels, as objects are constantly differentiated – by size, construction rate, etc. – as the thematic rises and the wording progresses. From this point of view, the morphodynamic epistemological orientation guarantees the possibility of echoing a number of purely quantitative approaches with qualitative studies. While the former focus their attention on the more regular aspects of enunciative construction (such as nominal schematisation, index construction, modalisation, etc.), qualitative approaches are more oriented towards the description of the processes of morphological emergence and stabilisation (such as illocutionary and perlocutionary forces, motor or rhythmic coordination, and various forms of synchronisation). As Wildgen says:

\begin{quote}\sloppy
Some problems, such as the production of utterances and language choices, can be continued by making accurate observations (even quantitative and statistical) and by building mathematical models that will then be evaluated using quantitative results. Other problems, such as perlocution effects, rhetorical functions, and index dynamics, require qualitative research. In all cases, the approach must implement multidisciplinary methods (...). Morphodynamic modelling, which is immediately transdisciplinary, allows us to move in the right direction without having to leave the scope of applicability of morphodynamic concepts. (\citealt{Wildgen1999}: 295, personal translation)
\end{quote}

Morphology is understood here as one or more linguistic and socio-cultural forms or \textit{gestalts} (i.e. patterns), which are stabilised as the speech and the conflicts of manifestation progress. The main characteristic of semiotic and linguistic morphology is to be immediately prominent and/or significant for a speaker (or for a social group). A form is prominent when it has the property (or dynamic capacity) of triggering (without any particular mediation) two things: (i) an economy of morphological and cultural values at stake, which direct the attention and movements of subjects; (ii) a proto-actantial distribution of syntactic roles, which enable the orienting and guiding of the perception and movements of speakers, as well as the socialisation of the forms themselves. Morphology is therefore prominent because it is penetrated by forces. It is a complex and dynamic organisation, which becomes a cognitive material of motivation and reuse in order to constantly generate new linguistic forms. 

But what does complex mean? To answer this question, one must put forward the second argument, which is still of epistemological nature, concerning the knowable nature of the chaotic aspects of enunciation. They can be understood by models directly inspired by theories of complex systems. Indeed, enunciative action is no longer merely analysed in terms of rule application or of the actualisation of a predetermined potential, as is still the case in some linguistic approaches. On the contrary, (\citealt{Wildgen1999}: 295, personal translation) states that “the constraints of the spontaneous genesis of enunciation that define the frame within which the rules of the game are formed and interpreted in usage”. The theory of complex systems appears here because the chaotic conditions of spontaneous enunciation become known within this frame. They also acquire the status of a space that may be controlled, i.e. a (a priori \textit{unstable}) parameter that guarantees and regulates the emergence and temporary stabilisation of a given enunciative form, as well as its reformulation and solicitation systems, its reorganisation and even its possible disappearance. In order to characterise the morphogenesis of linguistic action in terms of a complex system, one must specify that the latter is a “system of interacting forces, each consisting of several interacting factors whose identity is not known initially and whose values may continuously change while the states of the global system remain globally constant” (\citealt{Virole2019}: 91, personal translation).

\begin{quote}
According to the generalised model of catastrophe theory, this system manifests itself in apparent states. It is assumed that an internal dynamic exists within this system, which is also unknown and unobservable, and which defines the states that this system can hold in a stable way. These states (...) are considered as system attractors. These states virtualise each other. They do not exist in isolation. They are bound by mutual determination relationships and thus fulfil the conditions of a structural system in the sense of Deleuze. (\citealt{Virole2019}: 91, personal translation)
\end{quote}

To describe language action as a case of a complex microsystem that can be modelled, it is first necessary to identify the multiple factors that control the system, by continuously changing in an external space in which observable qualities are manifested. An utterance can therefore no longer be taken into consideration only in terms of a space in which it may manifest or of the actualisation of a series of given possibilities in language. A morphodynamic analysis approaches all enunciative structures as control spaces for a series of variable dynamics, which are dedicated to the temporary stabilisation of interactions. From this point of view, while these dynamics are at the same time corporeal, psycho-physiological, psycho-social, and also phenomenological, rhythmic, semantic, pragmatic, etc., utterances are only temporarily stabilised dynamic fragments, destined to be deformed: 

\begin{quote}
When a complex dynamic system, driven by its need for self-organisation, is led to achieve its effects using the elements received in input, it ultimately develops a process of self-regulation that requires the emergence of an internal representation of itself and leads to a complete reorganisation of its structure. (\citealt{Virole2019}: 91--92, personal translation)
\end{quote}


\section{A temporary conclusion: Enunciation as a complex synchronisation} 

While for Benveniste enunciation was an appropriation process submitted to an accomplished and previously given form, namely language (which the speaker assimilates during elocution), for morphodynamic theory enunciation is, on the contrary, a transition between the non-linguistic and the linguistic areas. The consequences of such a theoretical and epistemological difference also concern the representation models and the description of the phenomena taken into consideration. 

In particular, the morphodynamic approach conceives utterances as an epiphenomenon or manifestation of a \textit{set of dynamic coordinations}. Instead of simply describing linguistic traces that are organised in corpora, focus is made on the temporal dimensions organising the life of an utterance, that is, the internal rhythms of the life of a form. But what kinds of coordination are we talking about? Wildgen proposes a list in his study. These coordination types contribute, at different and sometimes heterogeneous levels, to the enunciation process as a transition from the linguistic to the non-linguistic area, i.e. as an action that modifies the mental space of the subjects participating in the event, the cultural space of groups and the real space of environments and habitats. For the semiotician, there are five rhythmic coordination types: 

\begin{itemize}
\item coordination between speaker and addressee;

\item coordination between language resources and communication needs

\item coordination, within the speaker, between memory processes (long-term, short-term), thought and transposition into words;

\item coordination during vocal production between different rhythms (breathing, sound production, articulation);

\item coordination in the articulation between different muscle groups.

\end{itemize}

We do not have time to go into depth into Wildgen's phenomenological description and mathematical semiotic model, nor can we propose a case study. Suffice it to say as a conclusion, that the morphodynamic perspective intersects with linguistic approaches that are currently interested in the phenomenon of \textit{languaging}, the psycho-physiological and social coordination process that drives language experience \citep{Bottineau2017}. In this contribution we modestly aimed to give an insight into the potential and relevance of connecting linguistic theory and the theory of complex systems. Further ad hoc research in this direction is necessary in order to propose a semiolinguistic approach to complexity. 

\section*{Acknowledgements}
The author is grateful to the ASLAN project (ANR-10-LABX-0081) of the Université de Lyon, for its financial support within the French program “Investments for the Future” operated by the National Research Agency (ANR).

{\sloppy\printbibliography[heading=subbibliography,notkeyword=this]}
\end{document} 
