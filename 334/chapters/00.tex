\documentclass[output=paper]{langscibook} 
\ChapterDOI{10.5281/zenodo.6620103}
\author{Pierluigi {Basso Fossali}\affiliation{UMR 5161 ICAR; Centre National de la Recherche Scientifique; Ecole Normale Supérieure; Université Lumière Lyon 2}
        and Kristine Lund\affiliation{UMR 5161 ICAR; Centre National de la Recherche Scientifique; Ecole Normale Supérieure; Université Lumière Lyon 2}}
\title{Introduction to language as a complex adaptive system}
\abstract{In this introductory chapter to the book, we argue that the notion of complexity has not reached its full potential in reconceptualising the field of language sciences.  Others have argued that language use and language acquisition are in a mutually influencing relationship, but additional empirical evidence from a variety of authentic human interactions is needed. Indeed, the structural complexity of a language cannot be immediately matched to its learning complexity, or to the complexity of its use in context. We discuss alternative -- yet compatible -- complexity explanations for the self-organisation of languages, the first pitting redundancy against economy, and the second focusing on environmental variables. The book examines the place of language in relation to interactive, pragmatic, multimodal discourse processes, but also in relation to cognition, argumentation and meaning-making, and to social structures and education. In doing so, our goal is to illustrate how complexity emerges as a network of functions which are organised differently and rooted in variable ways and which question each other.}

\begin{document}
\maketitle 

\noindent The notion of complexity was introduced in linguistics a long time ago, but generally from an autonomous reflection compared to the theoretical developments that have characterised other scientific fields and have transformed complexity into a new research paradigm with a transdisciplinary vocation \citep{Piaget1972,Gray2012,RichardsonEtAl2014}. The imperviousness of this paradigm merits some apparently contradictory remarks: on the one hand, one can appreciate the refusal to import a conceptual elaboration rooted in other fields of investigation, such as biology, in order to avoid metaphorical uses; on the other hand, one can seriously doubt that the contribution of a general epistemological change based on complexity theory has so far been fully beneficial for an internal critical reflection in the field of linguistics, in order to reconceptualise, at least in part, its ambitions, its objects and its methodological approaches (cf. \citealt{LaMantiaEtAl2016}). 

Positioning themselves from a language learning perspective, the “Five Graces Group” \citep{BecknerEtAl2009} have argued that patterns of language use on the one hand and language acquisition, its use and development over time on the other, are in a mutually influencing relationship.  But since then, this view has not been extensively supported in the literature. It is our aim in this volume to explore and give evidence for such a notion of complexity. We will examine evidence for the set of characteristics of language as a complex adaptive system that these authors put forth, such as distributed control and collective emergence, intrinsic diversity, perpetual dynamics, adaptation through amplification and competition of factors, non-linearity and phase transitions, and finally sensitivity to and dependence on network structure. These arguments will be brought together in the conclusion of this volume.

Instead of this broader view, that can extend into frontiers with other disciplines, linguistics has up to now primarily conceived complexity as a factor that unifies or differentiates languages as far as the organisation of their morphologies and syntaxes is concerned. In this sense, complexity has been used as a distinctive quality of the object under study, which has created ambiguity between, on the one hand, the scientific observation (typological view) of a factor of complexity that can be common or divergent between one language and another and, on the other hand, ethical implication (suspicion of ethnocentrism). Typological studies have addressed the question of the more or less complex nature of the languages that were studied and compared, yet the epistemological issue of complexity emerges every time that the diachronic approach is adopted, thus imposing a specific questioning on the reasons leading to the complexification or simplification of languages. 

While complexity is a historical factor, the assumption that the evolution of languages, which are subject to practices, is based on simplification (cf. \citealt{Culicover2013}) cannot explain the initial establishment of complexity and even less its resistance to time. The latter appears as an almost enigmatic phenomenon and is clearly contrary to an apparently “normal” historical evolution. 

Nowadays one can notice that the asynchronous evolution of languages can only call into question the \textit{equi-complexity of} languages \citep[9]{HadermannEtAl2017}; but also that the presence of the grammatical complexity of languages can be conceived “as the outcome of natural processes of self-organisation whose motivation is largely or entirely system-internal” (\citealt[33]{Gil2009}). 

Two requirements seem to emerge: (i) the establishment of parameters to determine the different complexities of languages; (ii) a reconsideration of the role historically played by complexity, which can be neither a general quality nor the result of a single evolutionary factor.

\begin{sloppypar}
This necessarily implies a reconceptualisation of the notion of \textit{complexity}. Moreover, the structural complexity of a language cannot be immediately matched to its learning complexity, or to the complexity of its use in context. Each of these specific notions of complexity call for establishing relations with other disciplines: for example, structural complexity with the ontologies of computer science, learning complexity with educational sciences and didactics, and complexity of use with at least sociology and anthropology. Keeping to a strict linguistics view, it can seem that simplifications lead to more language learning difficulties. For example, apparent simplifications in syntax, such as allowing simple juxtapositions (verbal serialisation in a number of Asian languages), can affect the learning difficulty of managing potentially long verbal sequences. Moreover, the difference between \textit{overt complexity} and \textit{hidden complexity} must be considered (\citealt[48]{Bisang2009}), since the former is grammaticalised, or at least explicit in discourse, and the latter, on the other hand, concerns implicit forms and invites interactants to make sophisticated pragmatic inferences.\end{sloppypar}

It should be noted that \textit{overt} complexity is also the manifestation of a meaningful tension that seeks to fill or hide the \textit{blanks} that each enunciation inevitably leaves, given that it is impossible to saturate all the internal links and to make explicit all the external conditions for it to assume its linguistic functions. Complexity emerges according to a negative profile, as a frame for raising awareness of the limits of meaning management, which is the most authentic characteristic of this notion from an epistemological point of view.\largerpage

The self-organisation of languages in the development (or reduction) of their complexity raises questions for linguistics because two heuristic formats are possible: 

\begin{enumerate}
\item the “internalist” explanation can envisage research on the balance that would characterise each language between two divergent tensions: redundancy, as a factor that reduces indeterminacy, and economy, as a factor that reduces the exploitation of resources;
\item the “externalist” explanation can consider that complexity is strictly related to environmental variables, particularly societal ones. In this sense, the complexity of a language can be correlated to the internal differentiation of society into autonomous domains, the density and frequency of communications, relations with other linguistic communities which are close or share the same territory,\footnote{\citet[98]{Trudgill2009} considers that there are social factors that need to be examined in order to understand their correlation with trends leading to the complexification or simplification of a language: (a) degree of contact vs. isolation, (b) denseness vs. looseness of social networks, (c)~small vs. large community size.} etc.
\end{enumerate}

The two explanations cannot be considered to be in total opposition, since redundancy and economy are pragmatic factors and are therefore linked to communicational action, and societal pressure cannot influence language without translating its dynamics into tensions within the structural organisation of a semiotic system. While the two explanations are assumed as two sides of the same heuristic hypothesis, it is clear that the relevance of a theory of complex systems becomes not only relevant, but necessary. This volume intends to contribute to the ways in which complexity theory is relevant to the language sciences. 

The way in which the language system is viewed is at stake, its coupling with the environment, its dynamics related to the absorption of external irritations through internal rearrangements, and its capacity to offer symbolic forms in interaction domains that seem to lack sufficient organisation. Such a description sets the stage for examining the place of language in relation to interactive, pragmatic, multimodal discourse processes, but also in relation to cognition, argumentation and meaning-making, and to social structures and education. That said, adopting a theory of complexity that was developed outside the linguistic domain cannot erase the problem of clarifying and detecting the linguistic clues and rules of this complexity. This is why one should emphasise the efforts of current linguistics to overcome reductionist views on complexity. In this regard, it is worth quoting the study by \citet{BiberGray2016}, which aims to challenge the idea that grammatical complexity is identifiable with the extensive use of subordinate clauses. On the contrary, one should see alternative complexity in “maximally compressing structures” (ibid, p. 18). In these compressed structures, there is no explicit indication of the intended semantic relations between the different elements of the sentence. The complexity of subordination and thus of embedding (\textit{clausal complexity features)} is indeed recognised, however on the other hand, embedding within the sentence itself (\textit{phrasal complexity features}\footnote{According to \citet[246]{BiberGray2016}, the figures of complexity are: (1) pre-modifying nouns (e.g. \textit{cell membrane}), (2) attributive adjectives (e.g. \textit{preparative treatement}), (3) prepositional phrases as post-nominal modifiers (e.g. \textit{a basis for the interpretation}), (4) appositive noun phrases (e.g. \textit{the strongly oxidising fixatives: osmium tetroxide and potassium permanganate}).}) can also be observed.

In fact, \citet{Givón2009} had already attributed two dimensions to structural complexity: it can be \textit{recursive}, with the addition of subsequent constituents or syntactic levels, or \textit{condensatory}, in which case it aims to make syntactic connections implicit, up to the point of maximum integration (desententialisation of subordinates\footnote{See \cites[]{Lehmann1989}[87]{Havu2017}.}). 

Complexity asserts itself as a “generative power” that can no longer be described through the classic principle of \textit{compositionality}, since the interplay between several strategies of use and the local adaptation to the environment of use leads to an increase in the linguistic system’s internal entropy and, consequently, to a certain degree of unpredictability concerning enunciative realisations and their normative stabilisation. It is no longer possible to observe “transparency” in the derivational and inflectional declension of a language through the “historical accidents” feeding the complexification of languages: semantic continuity in the generativity of a lexical or verbal paradigm is no longer accompanied by an “iconic” (diagrammatic) morphotactic relation highlighting the family resemblance between forms belonging to the same paradigm. There is therefore a rupture, a “strong substitution” \citep{Dressler1985} of a new form with respect to the initial morphological base (e.g., “ponots” or “aniciens” as demonyms designating the people of Le Puy en Velay). 

As a result of the pressure of change in social areas, the paradigmatic organisation of languages must constantly adapt to an environment full of heterogeneous rationalities in order to master the extensional scope of words, although this does not prevent resistance or the mobilisation of proper morphosyntactic resources. The unpredictability of the evolution of languages is due to the multiplicity of their criteria and to the suppletive alternation of the rationales used.

Beyond biological metaphors, the life of complex systems cannot be linear and their reciprocal interpenetration forces language sciences to step aside from the dichotomy between “internalist” and “externalist” approaches. Complexity theory encourages the recognition of a dialectics between internal orders and indeterminacy factors, and external orders and indeterminacy factors. The paradoxical internal tension may find metastability through the linking, or even coupling, with an environment full of systems characterised by the same internal laceration. The interpenetration between different instances of enunciation as well as the interpenetration between languages in contact show the dialectic emergence of value forms that are for example, valid in a given ecological situation. Normally, the “art of bringing together” (\textit{dialektikḗ}) indicates not only the search for homologies (\textit{logoi}) between the internal and the interactional management of the instances concerned, but also the complementarity of different approaches and parameters. That being said, complexity theory no longer allows for any \textit{Aufhebung}, any surpassing towards a higher integration. Managing complexity means:\pagebreak

\begin{enumerate}
\item on the one hand, becoming familiar with the boundary intervals between order performances (structuring) and the acceptance of external irritations arising from competitive orders; 
\item on the other hand, establishing areas where endogenous and exogenous indetermination factors can give rise to other forms. 
\end{enumerate}

The encounter between two systems, which are at the same time organisations aiming at an optimal internal order and epicentres of indeterminacy in inter-observation, creates a productive \textit{double} \textit{contingency}, i.e. beams of co-de\-ter\-mi\-na\-tion that impose coordinated, albeit controversial, limits and orientations on each other. Finally, complexity theory can assert itself as a commensurability of functional deficiencies and structural fragilities between languages, which explains at the same time the unresolved dualities of their internal criteria (dialectics), the confrontation of singular solutions (translations) and the application of regulatory principles in the management of interaction (maxims). 

The complexity of a system is not directly related to the internal presence of complex solutions. This can be demonstrated by the fact that a language in limited contact with other language communities tends to preserve its original complex solutions \citep[9]{HadermannEtAl2017}. Moreover, if its society is solid and exclusive, the spoken varieties of this language subsequently begin to reinforce the instability and heterogeneity of enunciative choices, but probably to reassert the common language as a stabilising element of belonging. 

Complexity is part of the strategy of engaging difference in order to enhance a possible stability, even if such a stability has been implicitly attacked by its shortcomings or inaccuracies through its varieties. In short, complexity emerges as the negative trace of complex solutions adopted to remedy internal paradoxes and as the positive opening towards metastability based on external couplings and supports. 

The overload of syntagmatic structures is the symptom of an uncontrolled complexity, \textit{complex elements} being only iconised by phrasal nesting or accumulations. On the other hand, true \textit{complexity} can emerge when blanks in propositional connections immediately show the support of local elements on the global structure and when interpretative stability indicates a diastematic search, i.e., an informed observation of deviations and intervals. 

Research in linguistics on factors of complexity (syntactic structures, grammatical markers, etc.) and the focus on language learning also suggest the appealing to a related notion, that of \textit{complexification}. But once again, if we take the example of Michel \citegen{Pierrard1988} research on \textit{free relative clauses} (without an antecedent) we can see that complexity does not concern syntactic overload, but a qualitative aspect: the search for an antecedent must be performed in another plane of consistency (e.g. a pragmatic situation where the indefinite pronouns of the relative clause without antecedent finally become assignable, but always thanks to a paradigmatic class of competing scenarios). 

The interpenetration of systems can only be achieved in the perspective of a tensive complementarity without a definitive integration (totalisation), where the evaluation of differences (intervals) becomes a diastemic area of deviations that locally require a value according to a dialectical and tactical future. Language itself must be conceived as a “dress made of patches” \citep[132]{Saussure1907}.

This may help to explain why a language system is at the same time a principle of order and a principle of variation, a rational economy and a “clinamen” of apparently supernumerary forms. Contacts between languages and, more generally, interactions call for an adaptation of the order promoted to the situation, which can take internal blanks as a primary measure of the distances between proper forms and “other” forms. 

Thus, the evolutionary complementarity between order and variants\footnote{See \citet{GershersonFernandez2012}.}  is not expressed in terms of historical accidents, but in terms of ad-hoc catalysts of\footnote{As \citet[814]{Barthes1971} pointed out, “there is no (structural) way to finish a sentence. All completed catalysis narratives reveal voids, zero-\textit{signifiers} that still go through the plane of discursive manifestation. Explicit catalyses are only the paradoxical attempt to hide the voids that affect language games, even if they do try to saturate the combinatorial potentialities. Catalysis thus becomes a meta-observation factor: the sign of signs that are missing, the sign of an implication-squared or of an element that needs to be translated and that reveals a lack in the target language” (cf. \citealt{BassoFossali2016}).}  their potential, triggered by encounters with other systems. The systems use the differences exemplified by their internal variants as the possible redefinition of a suffered competition and thus of blanks that are “vital” for their existence in time. 

The tendency to select less complex and more general solutions can only be contrasted by exogenous principles, even though the latter may in many respects be the \textit{raison d'être} of a linguistic heritage as symbolic field. In our opinion, complexity is thus the very resistance to decomplexification, a dynamics that has no internal “reasons”. If each deviation entails a cost with regards to the pressure of change towards simplification, social stakeholders are the only ones who perceive the reasons for a contrary tension, a preservation of, or even an increase in the critical gap between order and variance. 

\citet{Culicover2013}, on the other hand, considers that the co-option of more complex forms can occur locally, only on the condition that the internal complexity of the system has been reduced more globally in grammar (ibid, p. 207). Thus, “such increases in complexity are the by-product of changes that reduce complexity elsewhere” (ivi, p. 209).

One may wonder whether this vision of complexity is not linked to the idea of a progressive saturation of language and its pragmatic functions, which contrasts with other theoretical positions, such as the one stated by \citet[32]{Gil2009}: “Language is hugely dysfunctional [...], comes remotely close to providing the necessary expressive tool [...] and forces us to say things that we don't want to say”. 

Complexity theory suggests that linguistic organisations are looking for subsequent functions, which avoids the need to conceive of their prior and fixed, or even universal, encoding, which is reflected in the very morphology of a language. “Purpose is a property `revealed' by the behaviour of the system” (\citealt[93]{Dauphiné2003}), given the retroactions from the \textit{global level} to the \textit{local level} and vice versa. In fact, from a praxeological perspective, the global level is \textit{opened up} by the very fact that linguistic practices have other organisational systems which guide the course of action. Linguistic action thus transits through the environment and finds areas that are modalised according to heteronomous trends. As a result, complexity emerges as a network of functions which are organised differently and rooted in variable ways and which question each other. 

As soon as a multitude of individuals assimilate the language, immediate asymmetries appear in terms of competence and purpose, which is incidentally the symbolic conversion of the very reason for communication: negotiating identities and the degree of involvement in the management of values. In interaction, competition and cooperation are co-present, which already illustrates the paradoxical interweaving of the ways in which stakeholders get involved in social complexity. In addition, attractor-values are innumerable and heterogeneous; thus, choices can no longer identify a balanced assortment, which generates bifurcations, or dramatisations concerning the difference between life forms, or diverging destinies. This is why the choice of a word can appear to have important consequences and immediately shape a vision of the world and the future (a kind of \textit{butterfly effect} applied to the semiosphere). This inference based on a detail is admittedly unbridled, but it shows how complexity is also experienced at the epistemic level as a constant suspicion, even a sceptical attitude.\footnote{Scepticism turns out to be the willingness to know one's own aporia and blindness. More generally, proactive scepticism avoids prophecies and conspirational reconstructions by accepting an \textit{ecology of non-knowledge} (Luhmann).} The interweaving of the reference spaces of linguistic practices, where self-communication, interaction and the involvement of instances that are not immediately questionable (e.g., institutions) constantly alternate a role of pre-eminence over each other, can only generate complexity.

These musings guide our exploration within this volume of the following sections. \partref{part:1} is introduced by Ollagnier-Beldame and proposes three epistemological views on complexity. Whereas Basso Fossali focuses on the contribution of semiotics to complexity theory, Ollagnier-Beldame takes a phenomenological view of the role of experience in knowledge. Lund connects systems of different orders in a model of the co-elaboration of knowledge. This section addresses the interpenetration of systems. The notion that non-linear interactions cannot be separated from their environments is brought into tension with the notion that a temporary separation may be needed in order to do initial research.  Finally, complex systems lead to a description of complex behaviours, also rendering necessary a transdisciplinary approach that will include different epistemological foundations. 

\partref{part:2} is introduced by Basso Fossali and deals with complexity, pragmatics, and discourse. In this section, Rabatel proposes a simplex account of discourse complexity using the pragma enonciative theory of points of view, Bondì focuses on the morphogenesis of language action and defines complexity in relation to the rhythmic synchronisation of enunciation. Nowakowska \& Constantin De Chanay write about dialogism for daily interaction, and Halté proposes a complex system of the modalities within a written interaction. In each of these chapters, the complexity of discourse practices is shown through different levels of organisation that they involve, touching on contingent and dynamic meaning, the restraining of indeterminacy and interweaving points of view.

\partref{part:3} views complexity through interaction and multimodality and is introduced by Mazur and Traverso. Polo, Lund, Plantin, and Niccolai describe collective reasoning as the alignment of self-identity footings. Chernyshova, Piccoli, and Ursi highlight multimodality within interaction to discuss adaptivity and emergence. Baldauf-Quilliatre and Colon de Carvajal focus on the multimodal practice of participation within a dynamic framework. And Griggs and Blanc consider second language use and development in an immersion class as a complex adaptive process. In all of these chapters, multiple relationships are noted between elements of language and the ways in which they change over time are described, often giving rise to emergent interactive phenomena, sometimes unpredicted and unexpected.  

Finally, Lund, Basso Fossali, Mazur, and Ollagnier-Beldame revisit the volume, draw conclusions on advancements that can be claimed in relation to language as a complex adaptive system, and plan for future initiatives.

\section*{Acknowledgements}
The authors are grateful to the ASLAN project (ANR-10-LABX-0081) of the Université de Lyon, for its financial support within the French program “Investments for the Future” operated by the National Research Agency (ANR).

{\sloppy\printbibliography[heading=subbibliography,notkeyword=this]}
\end{document}
