\message{ !name(habil-master_rev-1.tex)}% Started in April 2013
% 2013-05-08 Merged with habil-structure.tex
\documentclass[a4paper,12pt,oneside]{book}
\usepackage{graphicx}  %%% for including graphics
\usepackage{url}       %%% for including URLs
\usepackage{times}
\usepackage{natbib}
\usepackage{paralist}
\usepackage{tabulary}
\usepackage{multicol}
	\setlength{\multicolsep}{0pt}
\usepackage{longtable}
\usepackage{booktabs}
\usepackage{float}
\usepackage[margin=25mm]{geometry}

\usepackage[utf8]{inputenc}
\usepackage[safe]{tipa}
\usepackage{linguex}
\usepackage{setspace}
\usepackage{tikz}
\usetikzlibrary{positioning,decorations.pathreplacing,calc,arrows}
% for 3d graphics:
\usepackage{3dplot} %requires 3dplot.sty to be in same directory, or in your LaTeX installation
\usepackage[nottoc]{tocbibind}
% \usepackage[active,tightpage]{preview}  %generates a tightly fitting border around the work
% \PreviewEnvironment{tikzpicture}
% \setlength\PreviewBorder{2mm}


\usepackage{delarray}
\newcommand{\Env}{\rule{2em}{.5pt}}%Defines the phonological environment line
\usepackage{amsmath}
\usepackage{bm}
\usepackage[normalem]{ulem}
\usepackage{avm}
\usepackage[english,ngerman]{babel}
\usepackage{caption}
\usepackage[toc,page]{appendix}
\setcounter{secnumdepth}{3}
\newcommand{\ger}{\selectlanguage{ngerman}}
\newcommand{\noger}{\selectlanguage{english}}

\addto\captionsenglish{\renewcommand{\bibname}{References}} %changes
                                %bibliography to references

% \renewcommand\bibsection{\section*{References}}


\includeonly{
introduction,
semTranPsycho,
semTranTheo,
semantics,
modellingPrevious,
empirical-1,
empirical-2,
conclusion,
appendices,
% appendix-wordnet-senses
}

\begin{document}

\message{ !name(empirical-1.tex) !offset(635) }
). In \Next, A stands for the first  constituent of a complex
nominal, and B for the second constituent. 
% This representation is one common
% way to introduce underspecified semantic representations, cf. the
% discussion of compositionality in section 

% \ex. $\lambda$ y $\lambda$ x [A(x) \& R(x,y) \& B(y)]
\ex. \label{ex:underspecified}
$\lambda$B $\lambda$A $\lambda$y $\lambda$ x [A(x) \& R(x,y)
\& B(y)]

% The annotation used can be seen as an extension of Levi's system to a
% wider range of cases, especially to cases including meaning shifts
% like metaphor or metonomy.
We assume that an underspecified relation R links the denotations of A
and B in a given construction.
\is{underspecification!{in scheme for compound combinatorics}}
% The denotations that A and B predicate over are linked by an underspecified
% relation R. 
% In order to be able to classify compounds where constituents or the
% whole compound have undergone meaning shifts, we assume that the
% predicates themselves can be metaphorically or metonymically shifted.
Based on this, we developed the scheme given in \figref{fig:AN_combinatorics}, where, for reasons
of perspicuity, we omitted the arguments of the predicates. Shifted
predicates are followed by an apostrophe.
%  (note that in a
% full model, they are needed, because they can be shifted independently from
% the predicates).


\begin{figure}[h]
  \centering
\begin{tikzpicture}[>=stealth]
\node [shape= rectangle, draw] (context) {context/world knowledge};
% \node (context) {};
\node (placeholder_firstline) [below=of context]  {};
\draw (0,-1) node (label_context-R)  {specifies};
\node (relationR) [below=of placeholder_firstline]  {R};
\node (placeholder_thirdline) [below=of relationR]  {};
% \node (compoundpartA) [left=of placeholder_firstline]{A(x)};
\node (compoundpartA) [left=of placeholder_firstline]{A};
\node (compoundpartB) [right=of placeholder_firstline] {B};
% \node (compoundpartB) [right=of placeholder_firstline] {B(y)};
\draw (4.8,-1) node (label_west)  {initiates shifts};
\draw (-4.8,-1) node (label_east)  {initiates shifts};
\node (shiftedcompoundpartB) [right=of placeholder_thirdline] {B'};
\node (shiftedcompoundpartA) [left=of placeholder_thirdline] {A'};
% \node (shiftedcompoundpartB) [right=of placeholder_thirdline] {B'(y)};
% \node (shiftedcompoundpartA) [left=of placeholder_thirdline] {A'(x)};
\node (placeholder_secondlinewest) [left=of relationR]  {};
\node (placeholder_secondlineeast) [right=of relationR]  {};
\draw (0,-5.95) node (shiftedAB) {(AB)'};

\draw [->,thick] (compoundpartA) -- (shiftedcompoundpartA);
\draw [->,thick] (compoundpartB) -- (shiftedcompoundpartB);
\draw [-] (compoundpartA.east) -- (relationR.west);
\draw [-] (compoundpartB.west) -- (relationR.east);
\draw [-] (shiftedcompoundpartA.east) -- (relationR.west);
\draw [-] (shiftedcompoundpartB.west) -- (relationR.east);
\draw[thick,dotted] [-] (context.south) to  (label_context-R);
\draw[thick,dotted] [->] (label_context-R.south) to  (relationR);
\draw[densely dashed] [->] (context.east) to [bend left=45]
(placeholder_secondlineeast);
\draw[densely dashed] [->,thick] (context.east) to [bend left=70]
(2.5,-5.25);
\draw[densely dashed] [->,thick] (context.west) to [bend right=70]
(-2.5,-5.25);

\draw[densely dashed] [->,thick] (context.west) to [bend right=45] (placeholder_secondlinewest);
\draw[densely dashed] [->,thick] (0,-4.9) to  (0,-5.5);

\draw [decoration={brace,mirror,amplitude=0.5em},decorate,thick]
 (-2,-4.75) -- (2,-4.75);
% ($(right)!(2nd.north)!($(right)-(0,1)$)$) --  ($(right)!(4th.south)!($(right)-(0,1)$)$); 

\end{tikzpicture}  

  \caption{Scheme for A B combinatorics}
  \label{fig:AN_combinatorics}
\end{figure}
%%%%%%%%%%%%%%%%%%%%%%%%%%%%%%%%%%%%%%%%%%%%%%%%%%%%%%%%%%%%%%%%%%%%
%%    FRITZ-THYSSEN/TOULOUSE VERSION!!
%%%%%%%%%%%%%%%%%%%%%%%%%%%%%%%%%%%%%%%%%%%%%%%%%%%%%%%%%%%%%%%%%%%
% \begin{figure}[h]
%   \centering
% \begin{tikzpicture}
% \node [shape= rectangle, draw] (context) {context/world knowledge};
% \node (placeholder_firstline) [below=of context]  {};
% \draw (0,-1) node (label_context-R)  {specifies};
% \node (relationR) [below=of placeholder_firstline]  {R};
% \node (placeholder_thirdline) [below=of relationR]  {};
% \node (compoundpartA) [left=of placeholder_firstline]{A};
% \node (compoundpartB) [right=of placeholder_firstline] {B};
% % \node (label_east) [right=of compoundpartB] {initiates shift};
% \draw (3.5,-1) node (label_west)  {initiates shift};
% \draw (-3.5,-1) node (label_east)  {initiates shift};
% \node (shiftedcompoundpartB) [right=of placeholder_thirdline] {B'};
% \node (shiftedcompoundpartA) [left=of placeholder_thirdline] {A'};
% \node (placeholder_secondlinewest) [left=of relationR]  {};
% \node (placeholder_secondlineeast) [right=of relationR]  {};
% \draw [->] (compoundpartA) -- (shiftedcompoundpartA);
% \draw [->] (compoundpartB) -- (shiftedcompoundpartB);
% \draw [-] (compoundpartA.east) -- (relationR.west);
% \draw [-] (compoundpartB.west) -- (relationR.east);
% \draw [-] (shiftedcompoundpartA.east) -- (relationR.west);
% \draw [-] (shiftedcompoundpartB.west) -- (relationR.east);
% \draw[densely dashed] [-] (context.south) to  (label_context-R);
% \draw[densely dashed] [->] (label_context-R.south) to  (relationR);
% % \draw [->] (context.south) to  (compoundpartA);
% % \draw [->] (context.south) to  (compoundpartB);
% \draw[densely dashed] [->] (context.east) to [bend left=45]
% (placeholder_secondlineeast);
% % node[below,text width=2cm, align=center, midway] {initiates shift};
% \draw[densely dashed] [->] (context.west) to [bend right=45] (placeholder_secondlinewest);
% % \draw   [->] (0,0) node[below] {R} -- (3,-0.0) node[below,text width=3cm,text centered]
% % {B};
% % \draw   [->] (0,0)  -- (-3,-0.0) node[below,text width=3cm,text centered]
% % {A};
% % % \draw [->] (-1.5,-0.3) -- (1.7,-0.3);
% \end{tikzpicture}  
%   \caption{Scheme for A B combinatorics}
%   \label{fig:AN_combinatorics}
% \end{figure}
% AUCH LINK VON A DIREKT ZU B!
As the scheme indicates, we assume that context and world knowledge are
responsible for any further specification of the meaning of an AB combination. 
\is{meaning shifts|(}
Specifically, we assume that
A as well as B can be shifted from their literal meaning to a
secondary meaning, labeled A' and B'. \is{literal meaning!{meaning shifts and}}
Metaphors and metonyms present
types of well-known shifts, other candidates would be e.g. the process
of \isi{meaning differentiation}, cf. \citet{Bierwisch:1982}.\is{metaphor}\is{metonymy}
% confined
% ourselves to metaphorical and  metonymical in nature. \textbf{[Rev2: are there
%   additional shifts that might be predictors?]}
However, even after a shift,
they are still linked to the other part of the construction via the R
relation. This kind of semantics for A B combinations therefore clearly falls
into the category of radically underspecified approaches (cf. the
characterization in \citealt[128]{Blutner:1998}, and Section 
\message{ !name(habil-master_rev-1.tex) !offset(-9) }

\end{document}
