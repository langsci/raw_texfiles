\addchap{Acknowledgments}
\begin{refsection}

My first thanks go to Barbara Schlücker. She initiated my
work on compounds by suggesting, quite insistently, that I should submit an abstract to her and
Matthias Hüning's Naming Strategies workshop in 2008. Likewise, it was
her initiative which brought me to a workshop on Meaning and
Lexicalization of Word Formation at the 14th International Morphology
Meeting, Budapest, where I first met Sabine Arndt-Lappe. Both have
been the best of colleagues, providing not only linguistic feedback, but
also all-purpose advice and motivation. 

In 2011 I first met Melanie Bell when we both gave talks on English
compounds at the 4th International Conference on the Linguistics of
Contemporary English at the Universität Osnabrück. That was the
starting point of a still ongoing collaboration between the two of
us, and our discussions and work together crucially shaped my thinking about semantic transparency and
compounds. Apart from that, it was also a lot of fun, and overall a
surprisingly and overwhelmingly fulfilling experience in a world of
academia that I had almost come to see exclusively as a cynical caricature of its
original purpose. Thank you Melanie!

Preliminary versions of the material in this book were
presented at numerous conferences and talks, and I thank all the
audiences for their feedback. Special thanks go to Ingo Plag, Carla
Umbach, and Thomas Weskott.

Turning to my actual place of work, the English department of the
University Jena, I would like to thank all my colleagues there,
especially Volker Gast, Florian Haas, Karsten Schmidtke-Bode and Holger Dießel, who witnessed
the whole developmental progress of this work and provided feedback
and encouragment throughout. Very special thanks go to my office
mate Christoph Rzymski. He was my main statistics and \textsf{R}
advisor, and also carefully read and helpfully commented on the manuscript
before I submitted it. Quite over and above that, he also provided the
office with much-needed \isi{Supertee}, and generally made office life most enjoyable.

\enlargethispage{1\baselineskip}
This work is the revised version of my Habilitationsschrift, which was
accepted in 2017 by the Philosophische Fakultät of the
Friedrich-Schiller-Universität Jena. I thank the original reviewers of
the Habilitationsschrift, Sabine Arnd-Lappe, Holger Dießel, and Volker
Gast as well as the anonymous referee for Language Science Press for
their many helpful comments and suggestions.

Speaking of Language Science Press: many thanks to Sebastian Nordhoff, who made working with them a very pleasant experience.

The work by Melanie Bell and me presented in this book was partially supported by
three short visit grants from the European Science Foundation through
NetWordS—The European Network on Word Structure (grants 4677, 6520 and
7027). The corpus frequencies for our analyses presented in Chapter
\ref{cha:empirical-2} were gratefully provided by Cyrus Shaoul and
Gero Kunter. 


% The code for figure \ref{fig:RumelhartMcClelland} originally comes from ...
% Author: Robert Felty
% Source: http://blog.robfelty.com/2007/02/14/pgf-gallery
% Model structure from Rumelhart \& McClelland (1986, p .222)%

\printbibliography[heading=subbibliography]
\end{refsection}

