 
\chapter[The semantic analysis of compounds]{Compounds and the semantic analysis of complex nominals}
\label{cha:semantics}

Complex nominals, that is modifier-head combinations with a noun as
their head, are traditionally distinguished into compounds and phrasal
constructions. Compounds, because they are morphological units, are
treated by morphologists, phrasal constructions are not.\footnote{Unless the phrasal constructions are themselves embedded in words, cf. \citet{TripsandKornfilt:2017} for a recent edited volume on phrasal compounding.} At the same
time, in works on noun noun compounds, one often finds reference to
``phrase-like" noun noun combinations
(cf. \citealt[8]{Giegerich:2009}), or ``phrase-like semantics" \is{phrase-like semantics} of noun
noun combinations (cf. \citealt[48]{Bell:2012}). While these
formulations imply that there is a specific semantic analysis for
phrasal modifier head constructions, a considerable body of work in
formal semantics on the semantics of phrasal modifier head
constructions, especially adjective noun constructions, has
shown that this is not the case. In contrast, with notable exceptions
like \citet{Fanselow:1981} and \citet{Meyer:1993}, formal semantic treatments of compounds
are rare.
% , while compounds
% are usually not considered. \citet{Levi:1978} is one of the very few authors
% who started with a wider perspective from the beginning. 
% One distinct characteristic of the literature on the semantics of complex
% nominals is the treatment of specific constructions in different
% schools. Thus, in  In the more traditional morphological literature, but also
% in some rare cases like \citet{Fanselow:1981}, the focus has been firmly on
% compounds and compounds only. 

Due to these differences in focus, the resulting analyses of complex
nominals from the formal semantics and the morphological  traditions also show major
differences. In particular, early
 analyses in formal semantics are based on set-theoretic properties, while morphological analyses focus on
relations. Newer approaches, in contrast, mix ideas from
these 2 approaches.
In this chapter, I will start by sketching the main ideas behind the
set-theoretic and the relation-based approaches, and then introduce some mixed
approaches. 

While giving an overview of possible approaches to compound
semantics, this chapter will also show why compound classifications
based on the Levi system of classification are still so useful.
% Finally, I will also discuss in how far the current approaches
% can be fruitfully used for investigating the whole variety of complex
% nominals, or whether in fact some approaches are better suited to
% constructions traditionally analyzed as compounds, and other
% approaches better suited to constructions that have been considered to
% be phrasal.


% \begin{itemize}
% \item formal semantics: much on AN phrases, little on compounds
% \item morphology: much on compounds, little on phrases (`phrase-like semantics')
% \end{itemize}

\section[Set-theoretic approaches]{Set-theoretic approaches: the
  semantics of adjective noun combinations}

\is{set-theoretic approaches|(}
\is{adjective noun constructions!{formal semantic analysis}|(}
As mentioned in the introduction, formal semantics has focused on the
analysis of phrasal constructions, and, when it comes to nominals,
especially on set-theoretic analyses of adjective noun combinations.
The set-theoretic approaches  usually start from classifications for
adjectives, which are differentiated into intersective and
non-intersective adjectives, the latter set again being 
differentiated into subsective and non-subsective adjectives, cf. \citet{Partee:1995}.
Here, I follow this tradition by illustrating intersective,
subsective, and non-subsective modification with the help of
adjective noun constructions. % containing intersective, subsective, and non-intersective adjectives . 
In addition, I give pointers to similar behavior within the class of
noun noun constructions.

\subsection{Intersective modification}
\label{sec:intersectives}
\is{modification!{intersective}|(}
\is{adjective!{intersective modification}|(}
Intersective modification refers to combinations of modifier and
modified that can semantically be analyzed as the intersection of the
2 sets denoted by modifier and modified respectively. The class of
intersective adjectives is defined by its participation in the
respective intersective modification patterns, illustrated below for the
adjective \emph{radioactive} in \Next.

% \ex. The superconductive coil, which must be chilled to a few degrees
%  kelvin, is a solenoid whose field confines the superheated plasma. COCA
% Date 	2001 (Dec)
% Publication information 	Dec2001, Vol. 123 Issue 12, p51, 5p, 1 diagram, 2c
% Title 	Basic drives.
% Author 	Hutchinson, Harry
% Source 	Mechanical Engineering

\ex. \label{ex:radioactive_bumper-cars}
\textbf{Radioactive bumper cars} lie silent in the abandoned city of
Priypat near the Chernobyl reactor. COCA
% Date 	2012
% Publication information 	Mar/Apr2012, Vol. 46 Issue 2, p17-21, 5p, 2 Color Photographs
% Title 	Nuclear Power's Unsettled Future.
% Author 	Zehner, Ozzie OzzieZehner@berkeley.edu
% Source 	Futurist

Assuming that \emph{bumper car} denotes a set of
individuals, that is, the set of bumper cars, and that
\emph{radioactive} likewise denotes a set of individuals, namely the
set of radioactive things, the denotation of the combination of the
2 strings can be analyzed as the intersection of the 2
sets, cf. \Next (this representation format is directly adapted from
\citealt{KampandPartee:1995}, cf. also \citealt{Partee:1995}).

%% \ex. \emph{superconductive cable}\\
%% $\begin{array}[H]{l@{\;=\;}l}
%% [\![\mbox{superconductive}]\!] & \{x|\mbox{x is superconductive}\} \tabularnewline
%% {}[\![\mbox{cable}]\!] & \{x|\mbox{x is a cable}\}\\
%% {}[\![\mbox{superconductive cable}]\!]&[\![\mbox{superconductive}]\!]\cap [\![\mbox{cable}]\!]\\
%% &\{x|\mbox{x \mbox{is superconductive and} x \mbox{is a cable}}\}%  & 
%% \end{array}$


\ex. \emph{radioactive bumper car}\\
$\begin{array}[H]{l@{\;=\;}l}
[\![\mbox{radioactive}]\!] & \{x|\mbox{x is radioactive}\} \tabularnewline
{}[\![\mbox{bumper car}]\!] & \{x|\mbox{x is a bumper car}\}\\
{}[\![\mbox{radioactive bumper car}]\!]&[\![\mbox{radioactive}]\!]\cap
[\![\mbox{bumper car}]\!]\\
&\{x|\mbox{x \mbox{is radioactive and} x \mbox{is a bumper car}}\}%  & 
\end{array}$

Intersective adjectives therefore allow the inference patterns given
in \Next and \NNext.\enlargethispage{1.5\baselineskip}

\ex. \label{ex:radioactive-bumpercar} This is a bumper car.\\
This is radioactive.\vspace{-0.2cm}\\
\rule{3.3cm}{.3mm}\\
% -----------------------------------\\
$\rightarrow$ This is a radioactive bumper car.

\ex. \label{ex:radioactive-bumpercar-2} This is a radioactive bumper
car.\vspace{-0.2cm}\\
\rule{5.2cm}{.3mm}\\
%----------------------------------\\
$\rightarrow$ This is radioactive.\\
$\rightarrow$ This is a bumper car.

\citet[124]{Kamp:1975} refers to these adjectives as \emph{predicative}, and he
mentions that technical and scientific adjectives like \emph{endocrine,
  differentiable} and \emph{superconductive} constitute typical
examples. \is{adjective!{predicative}}\is{adjective!{technical and scientific}}
\citet[124]{KeenanandFaltz:1985} name \emph{male, female} and
\emph{Albanian} as examples; 
\citet{KampandPartee:1995} and \citet{Partee:1995} use
\emph{carnivorous} as their example.


% Partee 2001 ist erschienen als Partee 2010, 

% \ex. \emph{two-legged}, \emph{radioactive}, \emph{sick}, \emph{red}, \emph{superconductive},
% \emph{German}
% % carnivorous, leaden, , , , 

\is{noun!{material}|(}\is{modification!{with material modifier}}
While the discussion revolves around adjectives, it is easy to come up
with examples of noun noun combinations that behave similarly. In
particular, material nouns like \emph{plastic}, \emph{nylon}, or
\emph{silk} give rise to similar inference patterns, cf.
\emph{silk shirt} in \Next. 

\ex. 
% HA4 424 	She wore a black sacklike dress, a large silver medallion on a chain, black \textbf{nylon stockings} and flat-heeled shoes. 
% FRC 2070 	Melanie never found out to whom the \textbf{plastic toy} she discovered in the bath on her first morning belonged. BNC
% AC3 2081 	
He wore his best suit, a clean \textbf{silk shirt} and shaved
extra close.\\
BNC/AC3 2081 	

Clearly, the same inference pattern arises here:

\ex. \label{ex:silk-shirt} This is a silk shirt.\vspace{-0.2cm}\\
\rule{3cm}{.3mm}\\
%----------------------------------\\
$\rightarrow$ This is silk.\\
$\rightarrow$ This is a shirt.

Note that the material nouns are typically mass nouns, and that,
presumably due to this inference pattern, a standard dictionary
practice is to simply assign them double class membership as nouns and
adjectives (e.g., the noun sense of \emph{silk} is the fiber, and the
adjective sense is `composed of or similar to silk', cf. the
entry in the American Heritage College Dictionary \citeyear{AmericanHeritageDictionary3rd}). \is{noun!{mass}}\is{noun!{material}|)}

Another class of noun noun combinations that allows this inference are
so-called copulative compounds, e.g. \emph{singer-songwriter}, see
also the remarks in Section \ref{sec:fanselows_basic_relations}.\is{compound!{copulative}}
\is{adjective!{intersective modification}|)}
\is{modification!{intersective}|)}

\subsection{Subsective modification}
\label{sec:subsective_adjectives}
\is{modification!{subsective}|(}
\is{adjective!{subsective modification}|(}
Subsective modification differs from intersective modification in that
the combination of modifier and modified results in a subset of only the
set denoted by the modified. Importantly, the denotation of
the modifier by itself does not yield a single independent set
denotation, because it is always relative to some scale or measure provided
either by the linguistic or the extra-linguistic context. The class of
subsective adjectives is defined by its participation in subsective
modification patterns.
% differ from
% intersective adjectives in that their combination with a head noun
% does not yield an intersection, but rather only a subset of the set
% denoted by the head noun.
Classic examples for this class are
dimensional adjectives like \emph{big} or \emph{small}, cf. \Next for
2 combinations with \emph{big}.

\ex. \label{ex:big_mouse}
\a. A rat is not just a \textbf{big mouse}. COCA
% Date 	2004 (20040409)
% Title 	Interview: Richard Gibbs discusses nearing the completion of the mapping of the rat genome
% Author 	IRA FLATOW
% Source 	NPR_Science
\b.  \label{ex:big_snake}
There I was, face to face with a \textbf{big snake}, getting over my fears. COCA
% Date 	2001 (Sep/Oct)
% Publication information 	Vol. 34, Iss. 5; pg. 36, 5 pgs
% Title 	See Jane run
% Author 	Lybi Ma
% Source 	Psychology Today

\is{context sensitivity|(}
\is{vagueness|(}
The denotation of \emph{big mouse} is not the intersection of the set
of big things and the set of mice, and without further qualification, the inference patterns
discussed for the intersective adjectives in the previous section are
not available. In particular, a snake that is as big as a big mouse
is not a big snake, and big mice are mice, but mice as a class
are typically counted among the small things. The most obvious feature
of these adjectives is thus that they display a certain
context sensitivity or \isi{vagueness}, cf. \citet{Kamp:1975}, \citet{Partee:1995}, \citet{HeimandKratzer:1998} and
\citet{Chierchiaetal:2000}. Note that this context sensitivity is
not only influenced by the choice of the head noun. This is very
convincingly demonstrated by \citet{Partee:1995} with \Next, her (17). 

%%   cannot simply let the size standard be determined by the noun that
%%   is modified is illustrated by examples such as \Next.
\ex. \a. My 2-year-old son built a really \textbf{tall snowman} yesterday.
\b. The D.U. fraternity brothers built a really \textbf{tall snowman} last weekend.
% \z. = (17) in \citet{Partee:1995}

Although both sentences talk about tall snowmen, the size standards used
to evaluate the adjective differ: One expects the snowman
built by the 2-year old to be far smaller than the one built by the fraternity.
In a similar way, information from previous utterances can influence
which size standard is used in evaluation.

Note that once the context sensitivity is taken into account and the
correct size standard has been chosen and is then fixed,
subsective adjectives behave technically like intersective adjectives,
cf. \citet[330--336]{Partee:1995}.

There are various technical solutions on how \isi{vagueness} can be accounted for. One
popular implementation is \citet{Kennedy:2007}, who analyzes gradable adjectives as
functions from individuals to degrees. The degrees, in turn, constitute a
scale, that is, a total ordering of the degree with respect to some dimension.
 The semantics of the
positive form morpheme \emph{pos} handles the vagueness, cf. \Next, his (27).

\ex. $[\![\;\; [_{\text{Deg}} pos]\;\;]\!] = \lambda g \lambda x.g(x) \geq \text{\textbf{s}}(g)$
% \\ = (27) in \citeasnoun{Kennedy:2007}

Here, ``[\dots ] \textbf{s} is a context-sensitive function that chooses a
standard of comparison in such a way as to ensure that the objects
that the positive form is true of `stand out' in the context of
utterance, relative to the kind of measurement that the adjective
encodes'' \citep[17]{Kennedy:2007}.


\citet[6]{Kennedy:2007} points out that \isi{vagueness} needs to be distinguished from
\isi{indeterminacy},\is{vagueness!vs. indeterminacy} ``the possibility of associating a single lexical item with
several distinct but related measure functions''. Thus, he argues that his example (4a),
\emph{Chicago is larger than Rome}, is ambiguous with regard to the exact
measure function used; one could at least refer to either population or
sprawl. \citet[6]{Kennedy:2007} views adjectives like \emph{skillful} and
\emph{clever} as extreme examples for this kind of indeterminacy, because
they are ``highly underspecified for the precise feature being
measured''. However, whether vague or indeterminate, both types of
adjectives lead to the same pattern of subsective modification (note
that the examples for indeterminacy given here are also vague and require a standard of comparison once a measure function is selected). 

Parallels to the behavior of subsective adjectives in the domain of
noun noun
constructions are not so obvious. However, \emph{star} in the 2
examples in \Next parallels the behavior of indeterminate adjectives. 

\ex. % \a. Alaska Lawmakers Hire Star Lawyers To Block Obamacare Medicaid
% Expansion WEB 
% http://www.forbes.com/sites/theapothecary/2015/08/25/ak-lawmakers-hire-star-lawyers-to-block-obamacare-medicaid-expansion/
% Accessed on 2015-10-21, 11:22
% Published Aug 25, 2015 @ 06:15 AM By Jonathan Ingram, Nic Horton and Josh Archambault—Mr. Ingram is Research Director, Mr. Archambault is a Senior Fellow, and Mr. Horton is Policy Impact Specialist, at the Foundation for Government Accountability.
\a. Antonio was a \textbf{star dancer} and he could not take an objective view
of the whole. BNC/A12 1732 	
\b.   	As he was the NME 's \textbf{star writer} I guess Malcolm realised that
once the band really started to get going, Nick would be able to help
us out — whether he knew it or not. BNC/A6E 908

That is, after selecting a domain, here either the domain of dancing
or writing, the modifier \emph{star} is evaluated relative to the
scale for this domain. And arguably, \emph{star} also gives rise to the
typical patterns for vague modification, since standards of starhood
differ, consider the star writer of a high-school yearbook as opposed
to the NME star writer in \Last[b].
\is{adjective!{subsective modification}|)}
\is{modification!{subsective}|)}
\is{context sensitivity|)}
\is{vagueness|)}


\subsection{Non-subsective modification}
\label{sec:nonintersective_adjectives}

\is{modification!{non-subsective}|(}
\is{adjective!{non-subsective modification}|(}
Non-subsective modification refers to cases where the denotations of
the modifier and the modified do not intersect. %\is{adjective!non-intersective modification}
Classic examples of adjectives that are analyzed as non-subsective are e.g. \emph{former} in
\Next and \emph{alleged} in \NNext.

\ex. \label{ex:former_employees}
These deaths occurred primarily among \textbf{former employees}. COCA 
% Date 	2012
% Publication information 	Jan2012, Vol. 120 Issue 1, p44-49, 6p, 5 Charts
% Title 	Radiographic Evidence of Nonoccupational Asbestos Exposure from Processing Libby Vermiculite in Minneapolis, Minnesota.
% Source 	Environmental Health Perspectives


\ex. \label{ex:alleged_vandals}
 A fight ensued, and one of the \textbf{alleged vandals} was stabbed with
a kitchen knife. COCA
% Date 	2012 (120614)
% Publication information 	Metro; Pg. T21
% Title 	Alexandria and Arlington crime report Alexandria and Arlington crime report
% Source 	Chicago Sun-Times

\pagebreak[4]
What happens in the case of \emph{former} is that any overlap with the current
denotation of the head noun is excluded, that is, the set of people in
the denotation of \emph{former employee} does not overlap with the set
of people in the current denotation of \emph{employee}, cf. \Next. 

\ex. \emph{former employee}\\
$[\![\mbox{former employee}]\!] \neq [\![\mbox{former}]\!]\cap [\![\mbox{employee}]\!]$

The case of \emph{alleged} is a bit more complicated, because an
overlap is not excluded. 
Both adjectives are also different from the adjectives discussed so
far in that they require a more complicated semantic analysis in any
case and cannot fruitfully be understood as one place
predicates of alleged or former things respectively. This property is
reflected in their inability to occur in predicative position.

The \emph{former}-type adjectives are also referred to as privative
adjectives, cf. \citet[325]{Partee:1995}. There, she gives \emph{counterfeit}
as an additional example.\is{adjective!{privative}}

% Note that actual usage sometimes deviates from the expected behavior,
% cf. \Next.

Within the group of constructions traditionally labeled as compounds,
non-subsective usages can also be found. Thus, we have formations like
\emph{nonentity} in \Next: % comes closest:

\ex. `Imagine them not even getting his name right, Weasley, it's almost
as though he's  a complete \textbf{nonentity}, isn't it?' he crowed.\\
J. K. Rowling, Harry Potter and the Goblet of Fire, Chapter 13, Mad-eye Moody
% , chapter 13, Mad-eye Moody
  
Another example is \emph{shadow cabinet} in \Next, for additional examples
from German cf. \ref{fanselow:abNOTb}. 

\ex. Mr Prescott is unquestionably closer to a
         large swathe of the rank and file than most other members of
         Labour's \textbf{Shadow Cabinet}. BNC/A1J 588 	

\emph{Shadow} in \emph{shadow cabinet} seems slightly similar to
\emph{former}, pointing to a virtual cabinet that might become the actual cabinet at a later point on the time axis.
\is{modification!{non-subsective}|)}
\is{adjective!{non-subsective modification}|)}
\subsection{Problems for a set-theoretic classification of adjectives}
\label{sec:problems_basic}

While the main differences between the 3 different types of
modification are clear, it is not so clear whether adjectives
can be classified with the help of these classes, or whether or not
all adjectives are more or less subsective. For intersective
adjectives, it has been the class of
color adjectives which
led to principled discussion of the question of intersectivity. 

Another set of observations concerning the combinatorics of adjectives and
nouns that is not accounted for by the set-theoretic approach is
discussed in the section on pragmatic anomaly.

The non-intersectivity of specific adjectives is also sometimes
questioned. \citet[325]{Partee:1995} discusses the adjective \emph{fake} as a problematic
candidate for the class of privative adjectives, pointing to questions like
\emph{Is that gun real or fake?} as rather suggesting otherwise. \is{adjective!{privative}}

\subsubsection{Color adjectives}
\label{sec:color_adjectives}

\is{adjective!{color}|(} \is{adjective!{intersective modification}}
Color adjectives are typically taken to be good examples for the class
of intersective adjectives, and combinations of color-adjective noun
are often used to illustrate the expected inference pattern
(cf. textbook discussions, e.g. \citealt[62--70]{HeimandKratzer:1998}
on \emph{gray cats} and \citealt[459--461]{Chierchiaetal:2000} on
\emph{pink tadpoles}, but also
\citealt[43]{FodorandPylyshyn:1988}\footnote{\citet[43]{FodorandPylyshyn:1988}
  do not use the term `intersective', but their example is clear
  enough: ``Consider predicates like `\dots is a brown cow'. This
  expression bears a straightforward semantical relation to the
  predicates `\dots is a cow' and `\dots is brown'; viz. that the
  first predicate is true of a thing if and only if both of the other
  are.''}). Many examples confirm this expectation, cf. \emph{green
  chair} in \Next, which gives rise to the 2 inferences in \Next[a]
and \Next[b] and whose main clause, likewise, should be deducable from
\Next[a] and \Next[b] treated as its 2 premises.

\ex. \label{ex:green_chair}
%Gem sunk into the green chair. COCA
% Date 	2005 (Spring)
% Publication information 	# . Vol. 28, Iss. 2; pg. 382, 7 pgs
% Title 	SALT
% Author 	Rosanna Armendariz
% Source 	Callaloo
He sinks into a \textbf{green chair}, though James has not invited him to sit. COCA
% Date 	1990
% Publication information 	# Summer90, Vol. 26 Issue 3, p604, 9p
% Title 	High--Rise.
% Author 	Brown, Suzanne Hunter
% Source 	Southern Review
\a. He sinks into something green.
\b. He sinks into a chair.

However, even here the situation is not always so
straightforward. Consider the 2 occurrences of \emph{blue
  wall} in \Next and \NNext.

\ex. \label{ex:blue_wall}
So, if you wouldn't mind just standing over here against the \textbf{blue
wall}. COCA 
% Date 	2011 (110325)
% Title 	BURNING BED;
% ROLLER-COASTER ROMANCE ENDS IN FLAMES
% Source 	20/20 10:00 PM EST

\enlargethispage{1\baselineskip}
\ex. \label{ex:blue_police_wall}
\a.[BRADLEY: ]I -- i -- is there a reluctance on the part of police officers to talk about other
police officers and what some of them may have done? 
\b.[SCHWARZ: ]Are you
referring to, like, a \textbf{blue wall}, what everybody else refers to? No, absolutely
not. 
\c.[BRADLEY: ]There is no \textbf{blue wall}? 
\d.[SCHWARZ: ]No. \hfill COCA
% Source information:
% .
% Date 	1997 (19970824)
% Title 	OFFICER; OFFICER CHARLES SCHWARZ CLAIMS HE WAS MISTAKENLY IDENTIFIED AS BEING INVOLVED IN THE ALLEGED POLICE BRUTALITY AND SODOMY AGAINST ABNER LOUIMA AT THE 70TH PRECINCT IN NEW YORK CITY
% Source 	CBS_Sixty
% Expanded context:
% Mr-WORTH: That's right. BRADLEY: There's an appearance there that he's got
% something to hide. Mr-WORTH: I -- I'm aware of that appearance, and I
% understand that. And you want to know what? If people want to think he has
% something to hide about it, they're entitled to their opinion. BRADLEY: I -- i
% -- is there a reluctance on the part of police officers to talk about other
% police officers and what some of them may have done? Off-SCHWARZ: Are you
% referring to, like, a blue wall, what everybody else refers to? No, absolutely
% not. BRADLEY: There is no blue wall? Off-SCHWARZ: No. BRADLEY: If the federal
% authorities prosecute this case, you could face up to life in prison. Have you
% thought about that? Off-SCHWARZ: I -- I know there's a severe penalty. But,
% you know, I'm confident I'm going to be -- I'm going to be vindicated in the
% end. I know the truth. The truth is I was  

In \LLast, meaning composition for \emph{blue wall} follows the
intersective pattern, i.e., the denotation of
\emph{blue wall} is the intersection of the set of walls with the set of blue
things, and its meaning can likewise be seen as the addition of the 2
meanings of \emph{blue} and \emph{wall}. In contrast, \emph{blue wall} in \ref{ex:blue_police_wall} clearly is used with another meaning, referring to
the \emph{blue wall of silence}, a euphemism for the police practice of
stonewalling investigations into police misbehavior. While this usage
of \emph{blue} involves a clear meaning shift, the existence of true
intersectivity has also been questioned for usages not involving
obvious meaning shifts.

Some remarks by \citet{Quine:1960} throw first doubts on 
% the straightforwardness of 
an intersective analysis for color adjectives. First, he points out that \emph{red
  wine} can be treated as a compound mass term where
``[r]ed wine is that part of the world's wine which is also part of the
world's red stuff'' \citep[104]{Quine:1960}.  In contrast, ``[r]ed houses and
red apples overlap the red substance of the world in only the most superficial
sort of way, being red only outside'' \citep[104]{Quine:1960}. Secondly, \citet[132--133]{Quine:1960}
mentions a suggestion by Jakobson\ia{Jakobson, Roman} to him, according to which, based on
examples like \emph{black bread}, \emph{white wine} and \emph{white man}, \emph{white} and
\emph{black} should be construed as comparative adjectives (that is, along the
lines of \emph{white X} being interpreted as X is more white than the average
X) due to the fact that ``no wine is white stuff and no men are white
things'' \citep[133]{Quine:1960}. In a tradition dating back to
\citet{Partee:1984}, \citet{Quine:1960} is attributed with the contrasting
pair \emph{red apple} vs. \emph{pink grapefruit}, with a red apple being red
only outside (see above), and the pink grapefruit only being pink inside. 

\citet{Lahav:1989} even uses the color adjective \emph{red} to make a
forceful attempt against the whole idea of compositionality. \is{compositionality!{Lahav on}}
His
exercise on what it means to be a red noun is worth citing in its
entirety:

\begin{quotation}
Consider the adjective `red'. What it is for a bird to count as red is
not the same as what it is for other kinds of objects to count as
red. For a bird to be red (in the normal case), it should have most of
the surface of its body red, though not its beak, legs, eyes, and of
course its inner organs. Furthermore, the red color should be the
bird's natural color, since we normally regard a bird as being
`really' red even if it is painted white all over. A kitchen table, on
the other hand, is red even if it is only painted red, and even if its
`natural' color underneath the paint is, say, white. Moreover, for a
table to be red only its upper surface needs to be red, but not
necessarily its legs and its bottom surface. Similarly, a red apple,
as Quine pointed out, needs to be red only on the outside, but a red
hat needs to be red only in its external upper surface, a red crystal
is red both inside and outside, and a red watermelon is red only
inside. For a book to be red is for its cover but not necessarily for
its inner pages to be mostly red, while for a newspaper to be red is
for all of its pages to be red. For a house to be red is for its
outside walls, but not necessarily its roof (and windows and door) to
be mostly red, while a red car must be red in its external surface
including its roof (but not its windows, wheels, bumper, etc.). A red
star only needs to appear red from the earth, a red glaze needs to be
red only after it is fired, and a red mist or a red powder are red not
simply inside or outside. A red pen need not even have any red part
(the ink may turn red only when in contact with the paper). In short,
what counts for one type of thing to be red is not what counts for
another. Of course, there is a feature that is common to all the things
which count (non-metaphorically) as red, namely, that some part of
them, or some item related to them, must appear wholly and literally
redish. But that is only a very general necessary condition, and is
far from being sufficient for a given object to count as red.\\
\citep[264]{Lahav:1989}
  \end{quotation}
The same point is taken up again in \citet{Lahav:1993}, cf. especially
\citet[76]{Lahav:1993}.
 


\citet{Blutner:1998}, in discussing these data, also points to the phenomenon of lexical blocking. \is{adjective!{lexical blocking}}\is{lexical blocking}
Lexical blocking in the case
of color adjectives concerns for example the contrast between \emph{pale
  green/blue/yellow} vs. \emph{pale red}. Due to the availability of
the word \emph{pink}, the combination \emph{pale red} is anomalous for some speakers, for others
its domain is restricted to only the non-pink sub-part of the domain
of pale red (\citealt[123]{Blutner:1998} attributes this observation to \citealt{Householder:1971}). 

\citet{Travis:2000} also discusses some examples containing color
adjectives in connection with the notion of occasion-sensitivity. He
writes:
\begin{quotation}
The English sentence `It's blue' represents (that is, is a
means of representing) some contextually definite object as blue. That
form, as produced in different surroundings, in different speakings of
those words (of a given object at a given time) might engage with the
world in any of indefinitely many ways. One might, in so producing
it, say any of many different things to be so. For there are
indefinitely many and various \emph{possible} understandings of an
object's being blue. \citep[200, his emphasis]{Travis:2000} \is{context sensitivity}
\end{quotation}
As a consequence, the
only rule for the predicate \emph{blue} is ``it is correctly used on
an occasion only to describe what then \emph{counts} as blue''
\citep[213, his emphasis]{Travis:2000}. Again, his examples include
the search for \emph{blue ink} at a stationer, where on most
occasions, \emph{ink} will count as \emph{blue ink} if it produces
blue writing, and on these occasions, \emph{ink} that looks blue but
writes black will not count as \emph{blue ink} (though on other
occasions it perfectly well might count as blue ink).
\is{adjective!{color}|)}
% \textbf{CHECK: \citet{Ziff:1960} for `good' relative to a salient interest;
%   \citet{Keenan:1974} for abolishment of simplistic intersection view}

\subsubsection{Pragmatic anomaly of adjectives}
\label{sec:pragmatic_anomaly}
\is{adjective!{pragmatic anomaly}}
\citet[123]{Blutner:1998} uses the data in \Next, his (5), to illustrate what
he calls the pragmatic anomaly of adjectives:

\ex. 
\a. The tractor is red.
\b. The tractor is defective.
\c. The tractor is loud.
\d. The tractor is gassed up.
\d. \label{ex:tractor_pumped_up}
?The tractor is pumped up.
\d. \label{ex:tractor_sweet}
?The tractor is sweet.
\d. *The tractor is pregnant.
\d. *The tractor is bald-headed.
% \z. = (5) in \citet[123]{Blutner:1998}

\citeauthor{Blutner:1998} argues that pregnant and bald-headed tractors are simple cases of category
violations, whereas the combinations in \ref{ex:tractor_pumped_up} and
\ref{ex:tractor_sweet} are cases of pragmatic anomaly. Or, as Blutner
writes, ``[t]hat \emph{sweet} is not an appropriate attribute of
\emph{tractors} can't be explained on grounds of an ontological
category violation. A tractor \emph{can} be sweet, by the way. Taste
one: it might surprise you'' \citep[123, his
emphasis]{Blutner:1998}. \citet[265--266]{Lahav:1989} comes to the same
conclusion, when he discusses the fact ``that many adjectives do
not apply to many objects at all'' \citep[265]{Lahav:1989}, pointing to
cases like \emph{a straight house}, \emph{a soft car}, or \emph{a
  quiet stone}, or even \emph{gradual rats} and \emph{intense trees}. He continues:
\begin{quotation}
Notice, that the point is not that
houses are never straight or that trees are never intense in the same
way that trees never breath or talk. Rather, we have no agreed upon
conception of what it would be for a house to count -- or to fail to
count -- as straight, [\dots] \citep[265]{Lahav:1989} % or for a rat to be gradual.''
\end{quotation}
 
\is{set-theoretic approaches|)}
\is{adjective noun constructions!{formal semantic analysis}|)}
\section{Relation-based approaches: the semantics of compounds}
\label{sec:relation-based-approaches}
There is a considerable number of compound classifications that are in one way
or another relation based. My aim in this section is not so much to compare
all these approaches, instead, I want to focus on 2 important works from the same period, namely \citet{Levi:1978} and \citet{Fanselow:1981}. I
will start with a more detailed description of Levi's work, because her
classification system or adaptions thereof are still used widely
today. This holds both for psycholinguistic approaches
(cf. especially the discussion of the work relating to conceptual
combination in Chapters \ref{cha:semTranPsycho} and
\ref{cha:modPrevious}), as well as for work in computational
linguistics (cf. \citealt{Oseaghdha:2008}, who starts from
Levi's proposal in order to arrive at a new annotation scheme). In addition, Levi already
includes more than traditional compounds in her analysis, and, as pointed out in
chapter \ref{cha:intro}, Levi's approach and usage of the term
\emph{complex nominal} is the starting point for my own, extended
usage of the term.

For earlier work on semantic relations, a good starting point is the overview in
\citet[77]{Levi:1978} which lists the traditional names of her
relational predicates and points to relevant earlier literature. In
  particular, she refers to
  \citet{Koziol:1937}, \citet{Jespersen:1942}, \citet{Hatcher:1960},
      \citet{Brekle:1970}, and \citet{Adams:1973} for
  English, and to \citet{Li:1971} for Chinese and \citet{Motsch:1970} for German.
\is{semantic relations!{pre-1978 approaches}}\il{Chinese!{Li on semantic relations}}\il{German!{Motsch on semantic relations}}

\section{\citet{Levi:1978}}
\label{sec:levi1978}
\subsection{Levi's complex nominals}
\label{sec:levi_complex_nominals}

\citet[1--2]{Levi:1978} introduces the term `complex nominals' in order to
cover 3 sets of expressions ``which have generally been called
`nominal compounds', `nominalizations', and `noun phrases with nonpredicating
adjectives'\,'' \citet[1]{Levi:1978}. Examples for each group,
chosen from her
original examples (1.1)--(1.3), are given in \Next.

\ex. 
\begin{tabular}[t]{lp{4cm}p{5.6cm}}
  a.&nominal compounds: &\emph{apple cake, windmill}\\
b.&nominalizations:& \emph{presidential refusal, dream analysis}\\
c.&noun phrases with nonpredicating adjectives:&\raisebox{-2ex}{\emph{electrical conductor, musical talent}}
\end{tabular}

Why does she treat these 3 distinct groups as one? The main reason, stated
in \citet[4--5]{Levi:1978}, is the
observation that the third group, the noun phrases with nonpredicating
adjectives, are very similar to noun noun constructions as far as their syntax
and semantics are concerned, which leads Levi to the hypothesis that
these adjectives are
derived from underlying nouns. Following this hypothesis, she identifies
complex nominals as a group encompassing the 3 subgroups mentioned above.

While Levi's understanding of complex nominals is thus wider than the
traditional class of compounds, it nevertheless does not equate to a
consideration of all sorts of traditional phrasal constructions. This can be
seen very clearly when looking at the kind of data she considers as evidence
for the introduction of her new class. Of the 6 properties \citet[19]{Levi:1978} proposes, 3
are particularly interesting, namely nondegreeness, conjunction
behavior, and case relations. 
\is{complex nominals!{nondegreeness}|(}
\is{complex nominals!{conjunction behavior}|(}
\is{complex nominals!{case relations}|(}
The first 2 of them are reminiscent
of traditional compound tests, cf. the remarks in Chapter
\ref{cha:intro}, Section \ref{sec:intro-complex-nominals}.

\citet[19]{Levi:1978} exemplifies the property of nondegreeness with
the help of the following examples, cf. her (2.4).

\ex. \a. *very urban riots
\b. *very bodily injury
\c. *a very electrical conductor
\d. *very automotive emissions.
% \d. Cf. (2.4) in \citet[22]{Levi:1978}

This property can also be found in items traditionally considered as compounds, cf. \emph{*very
  blackbird} or \emph{*very blackboard}.
\is{complex nominals!{nondegreeness}|)}

The conjunction behavior of interest is illustrated in \Next and \NNext, her (2.6)
and (2.7): As \Next illustrates, nonpredicating adjectives can be
conjoined with common nouns. In contrast, they cannot be conjoined
with true adjectives, that is, prototypical attributive adjectives, cf. \NNext.

% \newpage
\ex. nonpredicating adjectives conjoined with nouns:
\a. electrical and mining engineers
\b. a corporate and divorce lawyer
\c. solar and gas heating
\d. electrical and water services
\d. domestic and farm animals
% \d. Cf. (2.6) in \citet[19]{Levi:1978}

\ex. nonpredicating adjectives conjoin only with nonpredicating adjectives, not with
true adjectives
\a. a civil and mechanical/*rude engineer
\b. anthropological and ethnographic/*respected journals
\c. continental and oceanic/*expensive studies
\d. literary and musical/*bitter criticism
% \d. Cf. (2.7) in \citet[23]{Levi:1978}

While the co-ordination criterion also plays a role in the compound
vs. phrase debate (cf. \citealt[74--76]{Bauer:1998}, who
discusses this issue extensively), the main point here is that the
nonpredicating adjectives follow the pattern of the nouns and not the
pattern of the other, more prototypical adjectives. \is{compound vs. phrase debate}
\is{complex nominals!{conjunction behavior}|)}

The data that \citet[27--28]{Levi:1978} discusses under the heading of case
relations concerns the observation that one can attribute the semantic
relations of agent, object, location, dative/possessive, and instrument
to nonpredicating adjectives. Her `agentive'
category is illustrated in \Next, cf. her (2.12).

\ex. \a. presidential refusal
\b. editorial comment
\c. revisionist betrayals
\d. senatorial investigations
\d. national exports
% \z. Cf. (2.12) in \citet[27]{Levi:1978}

\is{complex nominals!{case relations}|)}
% In this work, I use the term complex nominals in a wider sense, but I will
% come back to the scope of Levi's proposal in section \textbf{BLABLA}. 
Levi
distinguishes between 2 distinct analyses (or, in her understanding,
derivational pathways) for complex nominals. In
Section \ref{sec:levi_predicates_overview}, I give an overview of the first
approach, the recoverably deletable
predicates. Section \ref{sec:levi_predicate_nominalization} discusses her second
approach, which involves predicate nominalizations.

\subsection{Levi's recoverably deletable predicates }
\label{sec:levi_predicates_overview}

\is{semantic relations!{recoverably deletable predicates}}
\citet[75--80]{Levi:1978} introduces 9 types of \emph{recoverably deletable predicates}:
\textsc{cause}, \textsc{have}, \textsc{make}, \textsc{use}, \textsc{be},
\textsc{in}, \textsc{for}, \textsc{from}, and \textsc{about}. The first 3,
\textsc{cause}, \textsc{have}, and \textsc{make}, come in 2 different
versions.

The basic idea behind her analysis is that a construction like \emph{tear gas}
can be derived via an underlying relative clause in which the respective
predicates serve as main verbs. Thus, \emph{tear gas} is derived
from \emph{gas that causes tears}, and so on. Below, I give 2 of her examples
for each predicate, one containing what is traditionally considered a compound
noun, the other a phrase containing a nonpredicating adjective (cf. Table
4.1 in \citealt[76--77]{Levi:1978}).\footnote{Due to zero occurrences in the
  COCA and the BNC, I replaced \emph{nasal mist} with \emph{nasal spray}. Likewise,
  \emph{rural visitors} was replaced by \emph{rural
    lawmakers}, \emph{linguistic lecture} with \emph{linguistic theory}, and
  \emph{professorial friends} with \emph{professorial staff}.} The complex nominals
are embedded in sentences retrieved via COCA. Behind the examples, I added paraphrases
which make the intended interpretation clear. Note that since not
all recoverably deletable predicates are verbs, the actual derivation
pathways that Levi suggests are rather complex, cf. \citet[4.2,
Derivations][118--153]{Levi:1978} for the details. Here, I will ignore
this aspect of her work, focusing on the resulting semantic
classification of complex nominals. 

\ex. \textsc{cause}
\a. \textsc{cause1} [N2 causes N1]
\a. \label{ex:disease_germ}
You can no more deal with them in good faith than you can with a--a \textbf{disease
germ}. COCA\\
% Date 	2001 (Oct)
% Publication information 	# . Vol. 121, Iss. 10; pg. 8, 39 pgs
% Title 	Pele
% Author 	Poul Anderson
% Source 	Analog Science Fiction & Fact
%disease germ 
`germ that causes a disease'
\b. \label{ex:traumatic_event}
The 9/11 attacks was a deeply \textbf{traumatic event} for our country. COCA\\
% 17 	2010 	SPOK 	CNN_News 
% Quellenanzeige funktioniert nicht richtig!
% traumatic event 
`event that causes a trauma'
\z.
\b. \textsc{cause2} [N1 causes N2]
\a. \label{ex:drug_deaths}% drug deaths 
As we have been reporting, \textbf{drug deaths} in Mexico skyrocketed. COCA
% Date 	2009 (090415)
% Title 	U.S. War Against Pirates; Obamas Release Tax Returns
% Source 	CNN_Situation
\\`deaths that drugs cause'
\b. \label{ex:viral_infection}% viral infection 
Disease detectives are taking a serious look at the emerging link between
\textbf{viral infection} during pregnancy and the later development of mental
impairment in the fetus. COCA
% Date 	2011
% Publication information 	Sep/Oct 2011
% Title 	A Viral Link to Mental Illness
% Author 	Begley, Sharon
% Source 	The Saturday Evening Post
\\ `infection that viruses cause'

\ex. \textsc{have}
\a. \textsc{have1} [N2 has N1]
\a. \label{ex:picture_book}% picture book 
 The children narrated a wordless \textbf{picture book}. COCA
% Date 	2012
% Publication information 	Apr2012, Vol. 43 Issue 2, p205-221, 17p
% Title 	The Narrative Language Performance of Three Types of At-Risk First-Grade Readers.
% Author 	Allen, Melissa M. 1 mallen20@uwyo.edu Ukrainetz, Teresa A. 1 Carswell, Alisa L. 2
% Source 	Language, Speech & Hearing Services in Schools
\\ `book that has pictures'
\b. \label{ex:industrial_area }%industrial area 
One teacher described the immediate area around the school as an \textbf{industrial
area} with no houses and several major intersections. COCA
% Date 	2011
% Publication information 	Apr2011, Vol. 81 Issue 4, p194-201, 8p, 4 Charts
% Title 	School Climate Factors Contributing to Student and Faculty Perceptions of Safety in Select Arizona Schools.
% Author 	BOSWORTH, KRIS 1 boswortk@email.arizona.edu FORD, LYSBETH 2 lford@email.arizona.edu HERNANDAZ, DILEY 3 dyla@email.arizona.edu
% Source 	Journal of School Health
\\`area that has industry'
\z.
\b. \textsc{have2} [N1 has N2]
\a. \label{ex:government_land}% government land 
 Instead, it has issued demolition notices throughout the slum, which sits
 illegally on \textbf{government land}. COCA
% Date 	2010 (101030)
% Publication information 	BUSINESS NEWS
% Title 	India: Land of many cell phones, fewer toilets
% Author 	By RAVI NESSMAN, The Associated Press
% Source 	Associated Press
\\
`land that the government has'
\b. \label{ex:feminine_intuition}%feminine intuition 
Her \textbf{feminine intuition} told her that he was very definitely attracted to
women, but she was pretty sure that he did not permit himself to cross the
line that separated physical satisfaction from mind-spinning passion. COCA
% Date 	2004
% Publication information 	Waterville, Me. : Wheeler Pub., Edition: Large print ed.
% Title 	Dawn in Eclipse Bay /
% Author 	Krentz, Jayne Ann.
% Source 	Dawn in Eclipse Bay
\\`intuition that females have'
\enlargethispage{1\baselineskip}

\ex. \textsc{make}
\a. \textsc{make1} [N2 makes N1]
\a. \label{ex:silk_worm}% silkworm 
The town had a large-scale \textbf{silkworm} cultivation and many factories employed
Korean workers. COCA
% Date 	2003 (Fall)
% Publication information 	Fall2003, Vol. 76 Issue 4, p731-748, 18p
% Title 	The Great Kanto Earthquake and the Massacre of Koreans in 1923: Notes on Japan's Modern National Sovereignty.
% Source 	Anthropological Quarterly
\\`worm that makes silk'
\b. \label{ex:musical_clock }% musical clock 
A digital clock on the computer screen starts to tick down
from sixty seconds, and a \textbf{musical clock} starts to sound too -- something like
the ``Jeopardy" theme. COCA
%  Date 	1993
% Title 	Jurassic Park
\\`clock that makes music'
\z.\pagebreak[4]
\b. \textsc{make2} [N1 makes N2]
\a. \label{ex:daisy_chains}% daisy chains 
 % Instead Tootle took off into a bright green meadow to make a daisy chain and
 % play with butterflies and dragonflies, dragging his coal car behind him. 
 % Date 	2000
% Publication information 	# 2000, Vol. 85 Issue 2, p290, 23p
% Title 	Melville's House (Short story).
% Author 	Furman, Laura
% Source 	Southwest Review
 ``I taught her how to make \textbf{daisy chains}," Essa said from the doorway. COCA
% Date 	1996
% Publication information 	# Dec96, Vol. 91 Issue 6, p13, 27p, 1bw
% Title 	Out of the Mouths.
% Author 	Finch, Sheila
% Source 	Fantasy & Science Fiction
\\
`chains that daisies make'
\b. \label{ex:molecular_chains}
The atmospheric reactions can create \textbf{molecular chains} heavy
enough to rain out on Titan's surface. COCA
% Date 	1997 (Jul)
% Publication information 	Vol. 94, Iss. 1; pg. 42, 6 pgs
% Title 	Life: A cosmic imperative?
% Author 	Yvonne J Pendleton
% Source 	Sky and Telescope
\\ `configurations that molecules make'
% stellar configurations ``configurations that stars make'

\ex. \textsc{use} [N2 uses N1]
\a. \label{ex:steam_iron}%steam iron  
If you need to press the felt, use a \textbf{steam iron} or damp cloth. COCA
% Date 	1994 (Oct)
% Publication information 	Vol. 11, Iss. 7; pg. 84
% Title 	A homemade holiday: sew-easy costumes that are more treat than trick
% Source 	Todays Parent
\\`iron that uses steam'
\b. \label{ex:manual_labor}
It's hot, it's dirty, and it's undoubtedly \textbf{manual labor}. COCA
% Date 	2012 (120617)
% Publication information 	Food; Pg. G1
% Title 	A do-it-yourself harvest;
% U-pick farms offer exceptional produce at reasonable prices
% Author 	Amanda Gold, Chronicle Staff Writer
% Source 	San Francisco Chronicle
%manual labor 
\\`labor that uses hands'


\ex. \textsc{be} [N2 is N1]
\a. \label{ex:target_structure}
Grammar boxes -- the \textbf{target structure} explained and exemplified for
clarification and for reference. BNC/CLL 2985 
% Date 	(1985-1994)
% Title 	[Selection of OUP English Language Teaching promotional leaflets]. Oxford: OUP, 1992, pp. ??. 3163 s-units.  
%target structure 
\\`structure that is a target'
\b. \label{ex:professorial_staff}
Setzler had done graduate work at the University of Chicago, and he
maintained strong ties with the \textbf{professorial staff} there. COCA
% Date 	2004 (Jun/Oct)
% Publication information 	Jun/Oct2004, Vol. 42 Issue 2/3, p86-117, 32p
% Title 	Capturing the Public Imagination: The Social and Professional Place of Public History.
% Source 	American Studies International
\\`staff that are professors'
% professorial friends ``friends that are professors'

\ex. \textsc{in} [N2 is in N1]
\a. \label{ex:morning_prayers}
He hops out of the truck and goes inside to quickly say his \textbf{morning
prayers}. COCA
\\ `prayers that are in the morning'
% Date 	2006 (20061221)
% Publication information 	WORLD
% Title 	The new walls of Jerusalem: Part 3 * From the West Bank, a circuitous road to market
% Author 	Ilene R. Prusher Staff writer of The Christian Science Monitor
% Source 	Christian Science Monitor
% \a.  %field mouse 
% Far below, a gray field mouse scurries through the grass. COCA
% % Date 	2009 (Sep 2009)
% % Publication information 	. Vol. 9, Iss. 1; pg. 3, 7 pgs
% % Title 	SEEING EYE TO EYE
% % Author 	Leslie Hall
% % Source 	National Geographic
% \\``mouse that is in a field'
\b. \label{ex:marital_sex }% marital sex 
In addition, it should be noted that great \textbf{marital sex} is good for your
health, in addition to the glow it puts on your face and the spirit it puts in
your step. COCA
% Date 	1993 (Aug)
% Publication information 	Vol. 48, Iss. 10; pg. 32, 3 pgs
% Title 	10 secrets to a happy marriage
% Author 	Norment, Lynn
% Source 	Ebony
\\`sex that is in a marriage'

% \newpage
\ex. \textsc{for} [N2 is for N1]
\a.  \label{ex:horse_doctor}
Kirghiz, the bay gelding, needs the \textbf{horse doctor}. COCA
% Date 	2005 (Winter)
% Publication information 	# . Vol. 27, Iss. 1; pg. 1, 11 pgs
% Title 	THE PIGEON
% Author 	William Boyd
% Source 	The Kenyon Review
% horse doctor 
\\
`doctor that is for horses'
\b. \label{ex:nasal_spray}
Retrieving a \textbf{nasal spray} from an inner pocket of his waistcoat, he assumed
a thoughtful expression: COCA
% Date 	2006 (2006)
% Publication information 	# . , Iss. 125; pg. 44, 29 pgs
% Title 	Train Delayed Due to Horrible, Horrible Accident
% Author 	Brian Booker
% Source 	Triquarterly
\\
% nasal mist ``mist that is for the nose'
`spray that is for the nose'


\ex. \textsc{from} [N2 is from N1]
\a. \label{ex:olive_oil}
Stir in the \textbf{olive oil}; it does not need to emulsify. COCA
% Date 	2012 (120226)
% Publication information 	Food; Pg. M1
% Title 	The power of sour;
% COOKING;
% Put leftover wine to good use by making your own vinegar
% Author 	Lynne Char Bennett
% Source 	San Francisco Chronicle
\\
%olive oil  
`oil that is from olives'
\b. \label{ex:rural_lawmakers}
Despite strong opposition from \textbf{rural lawmakers}, the bill passed
the GOP-led House of Delegates with support from Democratic and Republican
lawmakers throughout the urban crescent. COCA
% Date 	2012 (120227)
% Publication information 	METRO; Pg. B01
% Title 	Fairfax frustrated by lack of urban coalition
% Author 	Fredrick Kunkle
% Source 	Washington Post
\\
`lawmaker that are from the countryside'
% rural visitors ``visitors that are from the countryside'

\ex. \textsc{about} [N2 is about N1]
\a.  \label{ex:tax_law}
This has been \textbf{tax law} in, in America for almost 10 years now, existing \textbf{tax law}. COCA
% Date 	2010 (100822)
% Title 	Senate Minority Leader Source
%tax law 
\\`law that is about tax'
\b. \label{ex:linguistic_theory}
%linguistic lecture ``lecture that is about linguistics'
He believed that an adequate \textbf{linguistic theory} should include not
only just linguistic competence, but also the social-cultural aspects, which
are ``so salient" in any linguistics proper. COCA
% Date 	2001 (Winter)
% Publication information 	Winter2001, Vol. 122 Issue 2, p326, 11p
% Title 	SECOND LANGUAGE LEARNING AND THE TEACHING OF GRAMMAR [1].
% Author 	Zhongganggao, DR. Carl
% Source 	Education
\\
`theory about linguistics'

Levi's study contains extensive commentary on these different classes. Here, I
will only point to some of her remarks and findings that are particularly
important as far as their interaction with or contribution to semantic
transparency is concerned.

\citet[85--86]{Levi:1978} notes that her 9 recoverably deletable predicates
are quite different in terms of their productivity (note, though, that
from extensively studying \citealt{Levi:1978} it has not become clear
to me what exactly the data is that she uses to draw these conclusions). \is{semantic relations!{productivity}}
According to her, \textsc{have1}, \textsc{cause}, \textsc{make} and
\textsc{from} are least productive (cf. \emph{picture book}, \emph{disease
  germ/drug deaths}, \emph{silkworm/daisy chains} and \emph{olive oil}
above). Moderately productive are \textsc{use}, \textsc{be}, and
\textsc{about}. Finally, \textsc{for}, \textsc{in}, and \textsc{have2}
(cf. \emph{government land} above) are
most productive. In addition, for all 3 predicates with 2 configurations
she finds a skew in her data towards those derived from passivized
verbs \citep[86]{Levi:1978}. Interestingly, she mentions that she finds a
`surprisingly' similar distribution in an early study on Modern Hebrew, cf. \citet{Levi:1976}.\il{Modern Hebrew!{semantic relations in}} 

\is{semantic relations!{analytic indeterminacy}|(}
\is{analytic indeterminacy|(}
Levi's system in many cases allows for multiple alternative analyses of one and
the same complex nominal. \citet{Levi:1978} points out that a
particular subgroup of those nominals that can be analyzed via the
\textsc{make2} predicate have an alternative analysis via \textsc{be}, again
corresponding to a specific subset of complex nominals that fall under this predicate, cf. the
examples in \Next.

\ex. \label{ex:levi_4.11_make2-have-be-doubles} \emph{landmass, chocolate bar, stone wall, sugar cube, bronze statue}\\
{}[modifier denotes a unit, head denotes a configuration]

In general, \textsc{make2} nominals are derived from sources with what
\citet[90]{Levi:1978} calls a `compositional reading', corresponding to
\emph{make up of/made out of}, as opposed to \textsc{make1}, where
\citet[90]{Levi:1978} diagnoses ``a sense of `physically producing, causing to
come into existence'''. In \Last, there are ``head nouns that denote
either a mass or an artifact of some sort, and modifiers that describe its
constituent material''.  On the \textsc{make2} analysis, a complex nominal
like \emph{chocolate bar} is derived from \emph{a bar that chocolate makes},
that is, \emph{a bar made from/made of chocolate}. On the \textsc{be}
analysis, it is derived from an underlying \emph{a bar that is
  chocolate}. Crucially, these 2 analyses are available for the same
compound reading, that is, the compound itself is not
ambiguous.\is{ambiguity!{vs. analytic indeterminacy}}
\begin{sloppypar}
Other instances of `analytic indeterminacy' \citep[90]{Levi:1978}
occur between \textsc{make1} and \textsc{for} (e.g. \emph{musical clock,
  music box, sweat/sebaceous/salivary glands}, the
analysis of \emph{suspense film} (\emph{film that causes suspense/has
  suspense}), and \emph{job tension} (\emph{tension caused by the job, tension
that the job has, tension on the job}),  cf. \citet[91]{Levi:1978} for all examples. 
\end{sloppypar}
\citet[262--269]{Levi:1978} raises a number of issues connected with analytic indeterminacy. First, she notes that in many cases the
analytic indeterminacy may in fact be regular and predictable,
pointing to the \textsc{make2}/\textsc{be} pattern illustrated in
\ref{ex:levi_4.11_make2-have-be-doubles}. \is{semantic relations!analytic indeterminacy}
Secondly, in some cases analytic
indeterminacy may in fact be non-existent on an ideolectal level. People
may agree on the denotatum of a complex nominal, but nevertheless
disagree if explicitly asked why a given complex nominal is called that
way. \citet[265]{Levi:1978} mentions that an example like \emph{tidal
  wave} is explained by some by \emph{because it is caused by the
  tide}, others explain it by \emph{because it sweeps in
like the tide, only it's more powerful}. An important point
that she makes in this context is that this kind of intersubject variation is
not bound to high frequency complex nominals, but can also be expected for new
nominals, exemplifying this by her first encounter with \emph{athletic
  charges}, where even when being offered an explicit explanation she could either
assign a \textsc{for} deletion or a nominalization based analysis (for the
curious: ``students on athletic scholarships had their book bills
charged to the Athletic department'' \citealt[265]{Levi:1978}). % p.265 athletic charges example
\is{semantic relations!{analytic indeterminacy}|)}
\is{analytic indeterminacy|)}

Analytic indeterminacy, especially the case in which several
non-conflicting analyses are held simultaneously, is also discussed by
\citet[427--428]{Jackendoff:2010}, who proposes to label the words in
question as promiscuous, cf. the discussion in Chapter \ref{cha:empirical-1}, Section \ref{sec:con-wordclass}.\is{promiscuity}

\subsection{Predicate nominalization}
\label{sec:levi_predicate_nominalization}

\is{nominalization|(}
A predicate nominalization analysis is only relevant for nominals with
deverbal nouns as heads. \is{noun!deverbal} Again, the focus will be on the resulting
classification rather than on the derivational system Levi
introduces. \citet[167--174]{Levi:1978} works with 2 different axes
of classification, one involving the type of nominalization, the other
involving the syntactic status of the first, premodifying element, in
the assumed underlying structure.

As far as nominalization types are concerned, she distinguishes between act,
product, agent and patient nominalizations. Examples along with
illustrating paraphrases, drawn from her (5.1) and (5.2),
cf. \citet[168--169]{Levi:1978}, and enriched with actual corpus occurrences and some
additional explanatory paraphrases are given in \Next.

% \newpage
\ex. \a. act nominalizations
\a. \label{ex:dream_analysis}
McPhee acknowledges that \textbf{dream analysis} isn't a highly respected element
in psychology. COCA
% Date 	2002 (20021215)
% Publication information 	Features
% Title 	What your dreams may mean
% Author 	RODNEY HO
% Source 	Atlanta Journal Constitution
\\`act of analyzing dreams'
\b. \label{ex:musical_criticism}% \emph{musical criticism}
Until now the pan-German press had, however thinly, veiled its attacks in the
rhetoric of \textbf{musical criticism}, but now they savaged Anna with unrestrained
glee. COCA
% Date 	2003 (2003)
% Publication information 	# . Vol. 88, Iss. 1; pg. 123
% Title 	Fantasy for eleven fingers
% Author 	Ben Fountain III
% Source 	Southwest Review
\\ `act of criticizing music'
\z. 
\b. product nominalizations
\a. \label{ex:human_error}% \emph{human error} 
Cognitive science is a young, changing discipline subject to \textbf{human error} and
ambition; only recently, a Harvard evolutionary biologist has been accused of
fabricating data about animal cognition. COCA
% Date 	2011
% Publication information 	Mar2011, Vol. 131 Issue 3, p69-72, 4p
% Title 	Cogs in the Machine.
% Author 	HYMOWITZ, KAY1
% Source 	Commentary
\\`that which is produced by (the act of) humans erring'
\b. \label{ex:musical_critique}%\emph{musical critique}
Ten most common misconceptions regarding \textbf{musical critique}. WEB
% http://www.lastfm.pl/user/0k0k0k0/journal/2008/09/06/25g6hz_ten_most_common_misconceptions_regarding_musical_critique. Accessed 2013-11-26
\\ `that which is produced by (the act of)
criticizing music'
\z. 
\c. agent nominalizations
\a. \label{ex:mail_sorter}% \emph{mail sorter} 
 My father worked in the post office, first as a \textbf{mail sorter} and then as
 station manager. COCA
%  Date 	2001 (0101)
% Title 	TRANSITION IN WASHINGTON; Excerpts From Judge's Testimony at Ashcroft Confirmation Hearing
% Source 	New York Times
\\`x such that x sorts mail'
\b. \label{ex:film_cutter}% \emph{film cutter} 
 He was a successful Hollywood attorney; she was a \textbf{film cutter} for Hollywood
 movies. COCA
% Date 	2011 (110528)
% Title 	For May 28, 2011, CBS
% Source 	CBS_48Hours
\\`x such that x cuts film'
\z. 
\d. patient nominalizations
\a. \label{ex:student_invention}% \emph{student invention} 
\textbf{Student invention} could save kids in overheated cars WEB
% \url{http://hub.jhu.edu/gazette/2013/june/news-child-motion-detector-student-invention},
% accessed 2013-11-26
\\`y such that students invent y' %no COCA HIT
\b. \label{ex:presidential_appointee}% \emph{presidential appointee} 
 He has also served as a \textbf{presidential appointee} to the National Museum and
 Library Services Board since 2006. COCA 
% Date 	2011
% Publication information 	Nov2011, Vol. 133 Issue 11, p51-75, 20p
% Title 	A Celebration of Engineering ASME 2011 Honors.
% Source 	Mechanical Engineering
\\`y such that presidents appoint y'
\z. % = (5.2) in \citet[169]{Levi:1978}

For the second axis of classification, Levi distinguishes between subjective,
objective, and multi-modifier nominals. In the case of subjective
nominalizations, the premodifier is analyzed as the subject in the
corresponding derivational source, for the objective nominalizations, it is
the object, and in the case of multi-modifier nominalizations, both subject
and object of the underlying forms are realized as premodifiers. These 3
types are illustrated in \Next, drawn from
her (5.6--5.8), cf. \citet[173--174]{Levi:1978}.

\enlargethispage{1\baselineskip}
\ex. \a. subjective: 
\a.  \label{ex:parental_refusal}%\emph{parental refusal} 
\textbf{parental refusal} to allow initiative and creativity; COCA
%  Date 	1998 (Winter)
% Publication information 	Winter98/99, Vol. 17 Issue 4, p313, 20p, 1 chart
% Title 	The Defense Mechanisms Inventory: Theoretical and Psychometric Implications.
% Author 	Juni, Samuel
% Source 	Current Psychology
% \emph{parental refusal}, \emph{clerical error}, \emph{royal gifts} 
\\`act of parents refusing'
\b. \label{ex:clerical_error}
Anna Chau's 2010 Fulton County and Johns Creek tax bills were sent
to the wrong address because of a \textbf{clerical error}. COCA
%  Date 	2011 (110227)
% Publication information 	NEWS; Pg. 1A
% Title 	In tax lien limbo;
% Property owners caught in middle of policies.
% Author 	M.B. Pell; Staff
% Source 	Atlanta Journal Constitution
\\ `that which clerics making errors produce'
\c. \label{ex:royal_gifts}
Then Surapati and his men left Kartasura, reportedly with\\ some of
the Susuhunan's horses and fine firearms as \textbf{royal gifts}. COCA
% Date 	1990 (Nov)
% Publication information 	Vol. 40 Issue 11, p40, 7p, 2c, 7bw
% Title 	Balance and military innovation in 17th-century Java.
% Author 	Ricklefs, Merle
% Source 	History Today
\\`that which royals give as gifts'
\z. 
\b. objective:
\a. \label{ex:birth_control}
What kind of access should women have to \textbf{birth control}? COCA
% Date 	2012 (120315)
% Title 	Secret of Assad Regime Revealed; Contraception Controversy
% Source 	CNN_Cooper
\\`the act of controlling birth'
\b. \label{ex:tuition_subsidies}
Her income, from welfare, food stamps, rent and \textbf{tuition subsidies}
and a \$3,000 gift from her mother, puts Ms. Owens, a single mother,
and her three children just above the official poverty line. COCA
% Date 	1999 (19991018)
% Title 	DEVISING NEW MATH TO DEFINE POVERTY
% Author 	 By LOUIS UCHITELLE 
% Source 	New York Times
\\`that which subsidizes the tuitions'
\c. \label{ex:acoustic_amplifier}
Privacy, hah! I slipped the \textbf{acoustic amplifier} out of my desk
drawer and stuck it on the wall that my office shared with Sam's. COCA
% Date 	2003 (Jun)
% Publication information 	# . Vol. 123, Iss. 6; pg. 114
% Title 	Sam and the Flying Dutchman
% Author 	Ben Bova
% Source 	Analog Science Fiction & Fact
\\`that which amplifies the acoustics'
\z.
% \emph{birth control}, \emph{tuition subsidies}, \emph{acoustic amplifier}
\c. multi-modifier:
\a.  \label{ex:industrial_water_pollution}
In the mid-1980s, the Indian government began an ambitious effort
to clean up municipal and \textbf{industrial water pollution} in the Ganges
River, where most of the 1.4 billion liters of sewage generated every
day by cities and towns along the river is dumped without treatment. COCA 
% Date 	1995 (Jul/Aug)
% Publication information 	Jul/Aug95, Vol. 37 Issue 6, p6, 16p, 1 graph, 1c, 8bw
% Title 	The future of populous economies China and India shape their destinies. (cover story)
% Author 	Livernash, Robert
% Source 	Environment
\\`the industry's act of polluting water'
\b. \label{ex:government_price_supports}
Why should there be \textbf{government price supports} for sugar? COCA
% Date 	1997 (19970821)
% Title 	FREELOADERS
% Source 	ABC_Special
\\'the products of governments supporting the price'
\z.
% \emph{industrial water pollution}, \emph{government price supports}
% = (5.6-5.8) in \citet[173]{Levi:1978}

As can be seen from the examples, subjective constructions can be found with
act, product, and patient nominalizations, objective constructions with act,
product, and agent nominalizations, and multi-modifier constructions only with
act and product nominalizations.
\is{nominalization|)}

\subsubsection{Scope restrictions of Levi's analysis}
\label{sec:levi_scope}
\citet{Levi:1978} aims ``to demonstrate
the pervasive regularities that may be discerned in the area of
CN [complex nominal]
formation'' \citep[269]{Levi:1978}. To this end, she excludes certain sets of data
from her analysis.

First of all, she is only interested in \isi{endocentric} formations, that
is, ``those CN [complex nominals] whose referents
  constitute a subset of the set of objects denoted by the head
  noun'' \citep[6]{Levi:1978}. With this, she in particular excludes the
  3 groups illustrated in \Next, cf. \citet[6]{Levi:1978}.

\ex. \a. metaphorical names, e.g.:\\ the usage of \emph{ladyfinger} for
a type of pastry, of \emph{tobaccobox} for a sunfish, of
\emph{silverfish} for an insect, of \emph{foxglove} for a flower.
\b. synecdochical reference (using a part to present the whole), e.g.:\\ \emph{peg leg, blockhead, birdbrain,
      eagle-eyes} in reference to people, or \emph{razorback,
      glasseye, hammerhead, cottontail}  in reference to animals. 
\c. coordinate structures ``such that neither noun may be taken as
head", e.g.:\\ 
\emph{speaker-listener},
      \emph{participant-observer}, \emph{player-coach}, \emph{secretary-trea\-sur\-er},
      \emph{screwdriver-hammer}, \emph{sofa-bed}, \emph{library-guestroom}
% \z. cf. \citet[6]{Levi:1978}

Secondly, she excludes proper nouns that resemble complex nominals in form but contain a first element
  used primarily to name a single and definite referent, e.g.
  \emph{Kennedy Library} or \emph{Sheridan
    Road}.\is{noun!{proper}}\is{proper name}
 \citet[7]{Levi:1978}
  notes that these usually denote places or businesses. Thirdly, she excludes
  constructions which contain non-predicating adjectives that, in her opinion,  are derived from underlying
  adverbs, cf. the examples in \Next, from her
  (1.9).\is{adjective!{non-predicating}}

\enlargethispage{1\baselineskip}
\ex. \a. potential enemy \b. occasional visitor \c. former roommate
\d. alleged attacks

Paraphrase possibilities like those in \Next, cf. (1.10) in
\citet[8]{Levi:1978}, are taken by her as suggestive evidence
for an underlying derivation from adverbs.
\ex. \a. They are all potential enemies/potentially enemies.
\b. She is a former roommate/was formerly a roommate.
% \z. cf. (1.10) in \citet[8]{Levi:1978}

Finally, she wants her theory to be a theory about productive processes and
therefore excludes
  metaphorical, lexicalized, or idiomatic meanings. She
  distinguishes between lexicalized meaning and idiomatic meanings as follows:
  lexicalized meanings are meanings of complex nominals that have
  idiosyncratic meaning added on to a
  predicted literal reading, cf. her example \emph{ball park}, which
  is predicted to have the meaning `park for ball' but has developed
  the lexicalized meaning `park or stadium designed for people to play
  baseball in [rather than football, basketball, or handball]',
  cf. (1.16) in \citet[10]{Levi:1978}. In contrast, idiomatic meanings are those meanings
  where the choice of the specific 
  constituents is `more or less' irrelevant. \is{lexicalized
    compounds!{vs. idiomatic}}\is{idiomatic meaning}
Thus, she considers \emph{fiddlesticks, horsefeather}, and \emph{bullshit} with their meaning
  `nonsense' as fully idiomatic, cf. (1.20) in \citet[12]{Levi:1978}
  for more examples. \is{literal meaning!vs. lexicalized meaning}
\is{literal meaning!vs. idiomatic meaning}
% \emph{duck soup,
%     honeymoon, fiddlesticks, horsefeather, bullshit, swan song, soap
%     opera} as fully idiomatic, 
Complex nominals are also excluded if only one of the constituents is
idiomatic. \citet[12]{Levi:1978} illustrates these constructions with
complex nominals containing an idiomatic prenominal modifier, e.g.
\emph{polka dot} as the name for a dot-based pattern, or \emph{cottage
  cheese} as the name for a type of cheese. 
% and constructions like \emph{polka dot, cottage cheese, cathouse, banana
%     republic, handbook, penknife, station wagon, rock music} would be examples
%   for complex nominals containing just one idiomatic element, here the
%   prenominal constituent.

\subsection{Evaluating Levi's approach}
\label{sec:levi_eval}

Levi's approach has been much discussed, starting with
\citet{Downing:1977} (she discusses
\citealt{Levi:1975}, Levi's dissertation which forms the basis of the 1978
book). 

\citet[827]{Downing:1977} points out that when reducing the semantics of a
compound to the formulas proposed by Levi, ``it is unclear how much of
essential semantic content of the item is lost". In addition, she
points out that her experimental results and some of the attested novel
compounds ``would be very difficult to reduce to any of these
categories", illustrating this claim with the examples in \Next,
cf. her (14).

\ex. \label{Downing_1977_14}
\a. interpretations of novel compounds from a context free
interpretation task:
\a. \emph{cow-tree}: a tree that cows like to rub up against
\b. \emph{egg-bird}: a bird that steals other birds' egg
\c. \emph{pea-princess}: a genuine princess, who passes the test of a
pea under 20 mattresses
\z. 
\b. rankings of novel compounds (rating of given interpretations as
\\`likely', `possible', `impossible')
\a. \emph{pumpkin-bus}: `a bus that turns into a pumpkin at night' one
likely, 6 possible, one impossible
\b. \emph{oil-bowl}: `a bowl designed to hold oil or syrup' 3 likely,
5 possible, zero impossible
\z.
\c. attested compounds (from a scene-description task, a
newspaper, and 2 novels, cf. \citealt[817]{Downing:1977} for
details): \emph{thalidomide parent}; \emph{cranberry morpheme};
\emph{pancake-stomach} `a stomach full of pancakes' \emph{plate-length}
`what your hair is when it drags in your food'

However, \citet[828]{Downing:1977} acknowledges that the fact that
many of the compound taxonomies proposed in the earlier literature are
reducible to Levi's, and that, in addition, her own novel compounds
are also reducable to a limited set of basic semantic categories akin
to Levi's suggests that ``these lists are something less than
arbitrary". \citet[828--829]{Downing:1977} also points out that at
least in her data there is a link between the semantic class of the
head of the compound and the resulting preferred interpretation of the
compound.\is{semantic relations!{semantic class of the head and}}

More principled criticism comes from \citet[151--154]{Fanselow:1981}
who sees Levi's work in the tradition of \citet{Motsch:1970}. He sees
the resulting ambiguities in the classifications of individual
compounds as problematic. Further, he doubts that the number of predicates
can be kept as low as the respective authors assume, questioning the
appropriateness of an analysis of, e.g., \emph{Polizeihund} `police dog'
as `dog that the police uses', and pointing to the exploitation of
polysemies in the analyses of the different authors as indicative of
this problem, citing Levi's analysis of both \emph{cell block} as well
as \emph{silk worm} as formed with \textsc{make} as an example (note
that this criticism holds although the 2 are distinguished as
\textsc{make2} and \textsc{make1} respectively). In general, he
questions the usefulness of the resulting classification, which can
only be a classification of those \emph{Verrichtungen} `doings' which
dominate in our society, not a classification of possible ways of
forming compounds. Finally, the system does not allow one to test whether a
given classification is correct, since the final step from the
predicates to the relation needed for the specific compounds is
missing. If one acknowledges that the specific compound's constituents
are responsible for the concrete specification of the relation in
question, that is, when the constituents themselves allow to deduce
the relation, then the underlying predicates are superfluous.

\citet{DevereuxandCostello:2005} experimentally investigate the issue
of analytic indeterminacy in a Levi-derived classification system, investigating
the system used in establishing the CARIN model in
\citet{GagneandShoben:1997} (for this system and the relations used, cf. Chapter\ \ref{cha:semTranPsycho}, Section \ref{sec:carin}).\footnote{For the
  purpose of their experiment, they add 2 additional relations,
  modifier \textsc{is} head, and modifier \textsc{makes} head,
  cf. \citet[495]{DevereuxandCostello:2005}.} 
\is{semantic relations!{analytic indeterminacy}}\is{analytic
  indeterminacy!{experiment on}}
In their experiments,
subjects could choose from 18 relations in classifying 60 compounds,
and they were allowed to choose as many relations as fit. Although the
compounds were presented along with an interpretation, subjects on
average suggested 3.23 relations for every compound. One of the
compounds where participants consistently chose several relations was \emph{job anxiety}, where `MODIFIER causes HEAD' and `HEAD about MODIFIER' occur most
often, with 13 (!) of the 18 relations chosen at least once.

% \textbf{ADD: Pius p. 4-5 for criticism: \citet{TenHacken:2016a}}
\subsection{Conclusion: the enduring appeal of Levi's system}
\label{sec:semrel-compound-classification}

% As the criticism in the previous section 
As shown in the previous section, Levi's system has a
number of weaknesses. However, if one is interested in compound
formation from a cognitive point of view, one would like to be able to
assess the productivity of different kinds of compounds. To assess the
productivity, in turn, one needs to have access to some kind of
frequency data. Here, the approaches to compound semantics that rely
on a categorization in terms of different relations are in widespread
use, and within these, the relations proposed by \citet{Levi:1978}
have proven hugely influential. The reason for the success of her
system is succinctly summed up in the following quote: ``[\dots]
Levi’s proposals are informed by linguistic theory and by empirical
observations, and they intuitively seem to comprise the right kind of
relations for capturing compound semantics"
\citep[27]{Oseaghdha:2008}. Often, the Levi-relations provide the
starting point for classifications and are enriched with additional
relations as needed (cf. especially the discussion of the works using
the CARIN or RICE models of conceptual combination in Chapter
\ref{cha:semTranPsycho} and Chapter \ref{cha:modPrevious}). While these
classifications are still very similar to Levi's original
classification, other reworkings include a number of greater
changes. 
The proposal by \citet{Oseaghdha:2008}, who dubs Levi's and similar systems as
inventory-style approaches \citep[17]{Oseaghdha:2008}, is a good example for a very extensive and careful reworking of Levi's system.
% , provides what is perhaps the
% most careful reworking. 
Starting with her original 9 relations, he
points out 4 main problems with her classification
(cf. \citealt[30--31]{Oseaghdha:2008}): 
\begin{compactenum}% \enlargethispage{1\baselineskip}
\item The \textsc{cause} relation is very infrequent.
\item The \textsc{make1} relation is also infrequent; in addition, alternative
  relations are possible for `most, if not all' examples that Levi
  gives for this relation. 
\item Nominalizations and recoverably deletable predicates %(see the discussion in section \ref{sec})
  are treated apart.
\item Levi does not provide explicit annotation guidelines and is
  unconcerned with regards to overlapping categorization or vague
  boundaries between categories.
\end{compactenum}
\citet[31]{Oseaghdha:2008} singles out the overlapping categorization
as the most critical of the problems, using the compound \emph{car factory} to
illustrate: whether categorized as \textsc{for} (factory for producing cars), \textsc{cause}
(factory that causes cars to be created), \textsc{in} (factory in which cars
are produced), or \textsc{from} (factory from which cars originate), all 4
categories still describe the very same meaning (cf. also
\citet{DevereuxandCostello:2005} discussed in section
\ref{sec:levi_eval} above).

\citeauthor{Oseaghdha:2008} works both aspects of Levi's analysis into
one consistent annotation system, and adds detailed annotation
guidelines for his system. Perhaps the most important changes are the
removal of \textsc{make1}, \textsc{cause}, \textsc{use}, and \textsc{for}, and the introduction of 2 new
categories \textsc{actor} and \textsc{instrument}. 
% I will come back to his annotation
% guidelines and how it compares to the original Levi taxonomy in
% chapter \ref{}. \marginpar{TODO: correct reference}   

% Despite these criticisms, Levi-like classification have proven
% popular, because the resulting classifications have been shown to have
% considerable usefulness in psycholinguistic modelling . Similarly, these relation have proven their
% usefulness in psycholinguistic modelling, cf. the discussion in .

% \citet{Oseaghdha:2008}



\section{\citet{Fanselow:1981}}
\label{sec:fanselow_1981}

\il{German!{Fanselow on German compounds}|(}
Working on German, \citet{Fanselow:1981} distinguishes between 2 major groups of
compounds, \emph{nominale Rektionskomposita} `nominal relational compounds' and 
\emph{Determinativkomposita} `determinative compounds'. 
% namely those involving only common nouns, and those
% involving not only common nouns.
% \emph{nominale Rektionskomposita} and
% 
Following the structure of his work, I
will start with the former in the next section and then discuss the
\emph{Determinativkomposita} in Section
\ref{sec:fanselow-determinativ}, cf. Part II and Part III in \citet{Fanselow:1981} respectively. Since Fanselow's approach is only
published in German, I present his ideas here somewhat more extensively.


\subsection{Compounds involving relational nouns}
\label{sec:fanselow_relational_nouns}

\is{noun!relational}
% \is{compound!{with relational nouns}|(}
\is{compound!{with deverbal head}|(}
Fanselow not only discusses compounds with deverbal heads, but also
 other relational heads in some detail. Examples are words
like \emph{Sozialdemokratenfan} `fan of the social democrats', \emph{Professorenkomplize} `professor accomplice' and
\emph{Kanzlerbruder} `chancellor brother', cf. \citet[81]{Fanselow:1981}. What they have in
common with deverbal heads is that the noun in the head
position has an open position for a term. Within this group, Fanselow
distinguishes a number of subgroups. Here, I am not going to discuss
these subgroups in any detail but simply point to several interesting
observations in his work. Thus, he notes that for the deverbal cases,
especially those formed with agent nominalizations like
\emph{LKW-Fahrer} `truck driver', either a habitual or a non-habitual reading is
possible. Corpus examples illustrating these 2 usages are given in \Next, where \emph{LKW-Fahrer} in \Next[a] clearly requires a habitual
interpretation, because it describes a specific function in a company. In
contrast, in \Next[b] the driver does not need to have been a habitual truck driver.

\ex. \label{ex:LKW_Fahrer}
\a. \label{ex:LKW_Fahrer1}
Als die Familie Bauer Trans OG ihre Firma 2001 gründete, arbeitete Oswald
Bauer noch bei der Firma Köck als \textbf{Lkw Fahrer}. DeReKo\\
% NON13/MAI.00132 Niederösterreichische Nachrichten, 02.05.2013, Ressort: Lokales; Lkw als Bonus
`When family Bauer started their business, Trans OG, in 2001, Oswald Bauer was still working as a truck driver for Köck.'
\b. \label{ex:LKW_Fahrer2} 	
% Watenstedt. Ein 43-jähriger Autofahrer verletzte sich leicht bei einem
% Verkehrsunfall am Dienstag gegen 16 Uhr auf der Industriestraße Mitte in
% Watenstedt. Der 43-Jährige fuhr mit seinem PKW aus Richtung Lebenstedt auf dem
% rechten Fahrstreifen. Er wechselte dann in Höhe MAN auf die linke Fahrspur und
% musste abbremsen, weil die Ampel rot zeigte. 
Ein dahinter fahrender 29-jähriger \textbf{LKW- Fahrer} konnte nicht mehr rechtzeitig
bremsen und fuhr mit seinem LKW auf, teilt die Polizei mit. DeReKo 
% An beiden Fahrzeugen entstand ein Gesamtschaden von ungefähr 9000 Euro. 
% BRZ13/MAR.05371 Braunschweiger Zeitung, 14.03.2013, Ressort: 1SZ-Lok; 9000 Euro Schaden bei Verkehrsunfall
\\
`The driver behind them, a 29-year old truck driver, didn't manage to stop in time and drove his truck into the preceding vehicle.' 
% % NON13/JUL.00004 Niederösterreichische Nachrichten, 04.07.2013, Ressort: Lokales; Hitzköpfe hinter
% Auch als Lkw Fahrer erlebt man so manche Überraschung, wenn man auf den
% Straßen unterwegs ist. DeReKo
% % Dieter Gansterer von Erdbau Transporte in Kirchberg meint darüber: „Leider sind die Leute immer mehr unter Druck und das macht sich dann auch beim Autofahren bemerkbar. Ein bestimmter Autotyp ist mir noch nicht aufgefallen, aber die Städter sind am ärgsten, was die Aggressivität betrifft. Es wird geschnitten, der Vorrang erzwungen oder auf unübersichtlichen Straßenstücken überholt. Die Radfahrer sollten möglichst von der Straße weg auf den Radweg verfrachtet werden, auch zu ihrer eigenen Sicherheit.“ 

In contrast to compounds with non-derived relational heads, compounds
with deverbal heads also allow local interpretations of their first
constituent. \is{compound!{with deverbal head}}\is{compound!{with relational head}}
Thus, while both \emph{Zeitungsverteiler}  `newspaper distributor (=newspaper boy)' and
\emph{Hochschul\-lehrer}  `university teacher (=teacher at a university)' have deverbal heads, the modifier fills an argument position in the former case, but requires a location interpretation in the latter case, cf. \citet[93--94]{Fanselow:1981}.
%  illustrates with the
% contrast between 
% \begin{quote}
%   ``Ein Zeitungsverteiler ist jemand, der Zeitungen verteilt, aber
%   ein Straßenverteiler verteilt keine Straßen, sondern etwas auf der
%   Straße. Genauso lehrt ein Mathematiklehrer Mathematik, aber ein
%   Hochschullehrer lehrt nicht `Hochschule' oder bringt einer
%   Hochschule etwas bei, sondern er lehrt an einer Hochschule. Im
%   Gegensatz zu den nicht-derivierten relationalen Nomina wie
%   \emph{Komplize}, \emph{Vorsitzender} lassen die derivierten
%   relationalen Nomina offenbar auch Kompositumsbildung mit lokaler
%   Interpretation zu.'' \citet[93--94]{Fanselow:1981}
% \end{quote}

\noger
\is{compound!{with deverbal head}|)}
Besides their relational reading, compounds with a non-deverbal relational head like
\emph{Richterfreund} `judge friend' might also have a simple coordinated reading,
cf. \Next.

\ex. \label{ex:richterfreund}
\ger
% Richterfreund
% 0 Treffer in Cosmas!
Der von Renaud D\'ely gekürte \glqq Mann der Woche \grqq {} ist Pierre
Estoup, der heimliche \textbf{Richterfreund} von Bernard Tapie. WEB\\
`The man of the week chosen by Renaud D\'ely is Pierre Estoup, the
secret judge friend of Bernard Tapie.'
% {http://www.arte.tv/de/homo-ehe-das-erste-jawort-managergehaelter/7534620,CmC=7534622.html}, accessed on 2013-11-23
%  3. Juni um um 4.05 Uhr - 04/06/13
% Homo-Ehe: Das erste Jawort! / Managergehälter
% Mit Jean-Michel Ribes, Titiou Lecoq, Dominique Seux und Riss 
\noger

\emph{Heimlicher Richterfreund} `secret judge friend ' in \Last needs
to be interpreted as `secret friend and judge'.

% Jetzt paragraph 10
While in the examples so far the relational noun is always the head, Fanselow also presents
numerous cases where a relational noun constitutes the first part of a
compound. \is{compound!{with relational modifier}|(}
Among Fanselow's initial examples are 
% \emph{Chefputzfrau} `',
\emph{Mitgliedsbuch} member:book `party book', \emph{Freundeskreis} friend:circle `circle of friends', and
\emph{Lieblings\-politiker} `favourite politician'. Generally, these relational nouns yield
com\-pounds that are themselves relational. However, as Fanselow makes clear, there are very
few general rules for these compounds,
cf. e.g. \emph{Lieblingspolitiker} vs. \emph{Traumpolitiker} `dream politician'. In
passing, he notes a number of examples where the AB is not a B,
cf. \Next, \citet[104]{Fanselow:1981}. 

\ex. \label{fanselow:abNOTb} 
\a. \label{fanselow:abNOTb:scheingefecht}
\gll
\emph{Scheingefecht}\\
appearance:battle\\ 
`mock battle'
\b. \gll
\emph{Kunsthonig}\\
art:honey\\ 
`fake honey'
\c. \gll 
\emph{Schattenkanzler}\\
shadow:chancellor\\ 
`shadow chancellor'
\d. \gll 
\emph{Ehrenpräsident}\\
honor:president\\
`honorary president'
\d. \gll 
\emph{Falschgeld}\\
wrong:money\\
`counterfeit money'
\d. \gll 
\emph{Pseudocleft\-konstruktion}\footnotemark \\
pseudocleft:construction\\ 
`pseudo-cleft construction' 

\footnotetext{Note that the bracketing
  that Fanselow must have in mind here is [Pseudo[cleftkonstruktion]],
  contrasting with the bracketing suggested in the English translation
equivalent. On the latter bracketing, the compound appears to be a
regular determinative compound.}
% \ex. \label{fanselow:abNOTb} 
% \begin{tabular}[t]{lp{3.5cm}p{5cm}}
% a.&Scheingefecht&appearance.battle `mock battle'\\
% b.&Kunsthonig&art.honey `fake honey'\\
% c.&Schattenkanzler&shadow.chancellor `shadow chancellor'\\
% d.&Ehrenpräsident&honor.president `honorary president'\\
% e.&Falschgeld&wrong.money `counterfeit money'\\
% f.&Pseudocleft\-konstruktion\footnote{Note that the bracketing
%   that Fanselow must have in mind here is [Pseudo[cleftkonstruktion]],
%   contrasting with the bracketing suggested in the English translation
% equivalent. On the latter bracketing, the compound appears to be a
% regular determinative compound.}&pseudo.cleft.construction `pseudo-cleft construction' 
% \end{tabular}
For these cases, \citet[105]{Fanselow:1981} assumes an analysis where
the first element operates on the intension of the second argument.
% Jetzt paragraph 11
In other cases with a relational first element, \citet[107]{Fanselow:1981} notes an asymmetry
with regard to the examples in \Next as opposed to those in \NNext.

\ex. \a. \gll
\emph{*Fanprofessor}\\
fan:professor\\
`fan professor' (intended
reading: professor of which somebody is the fan)
\b. \gll
\emph{*Enkellinguist}\\
grandchild:linguist\\
`grandchild linguist' [intented reading: linguist
who is the grandfather of someone]

\ex. \label{ex:fanselow:anfangskapitel} \a. \gll
\emph{Anfangskapitel}\\
begin:chapter\\
`first chapter'
\b. \gll
\emph{Schlußstein}\\
end:stone\\
`keystone'

Note that
Fanselow is explicitly exluding readings where the first noun is not
used as a relational noun. That is, \citet[107]{Fanselow:1981} acknowledges that
\emph{Vorstandspartei} % steeringcommitee.party
`steering
committee party' could actually denote a party that only consists of
the steering commitee or a party that supports the steering
commitee. Given that, note that while he is correct that the
first constituents in \emph{Anfangskapitel} `first chapter' and
\emph{Schlußstein} `keystone' in \ref{ex:fanselow:anfangskapitel} both still receive a relational interpretation,
the relational argument place is not filled by the second noun, but is
filled by something outside of the compound, as e.g. in
\emph{Anfangskapitel des Buches} `first chapter of the book' or \emph{Schlußstein des
  Gewölbes} `keystone of the vault'. That is, the asymmetry is tied to whether the resulting
compound is still relational, with the relationality deriving from the
first noun. For compounds consisting
of 2 relational nouns, he also distinguishes between those that yield
relational nouns, e.g. \emph{Zweigstellenleiter} `branch manager', and those that do not,
e.g. \emph{Rektorentochter}  `headmaster daughter (=daughter of the
headmaster)'. And again, there are sometimes ambiguities due to   
different readings of the compound constituents. Thus, \emph{Kind} `child' in
\emph{Kindsmörder} can either be taken as a common noun or as a relational
noun, leading either to a reading `set of persons who killed a child' or
`set of persons who killed their own child', cf. \citet[114]{Fanselow:1981}.
\is{compound!{with relational modifier}|)}

Perhaps Fanselow's most important finding with regard to the compounds involving
relational nouns is that deverbal heads are not so very
special, since
non-deverbal heads often function very similarly.\is{compound!{with
    deverbal head}}
% \is{compound!{with relational nouns}|)}

\subsection[Determinative compounds]{Common nouns with common nouns: \emph{Determinativkomposita} `determinative compounds'}
\label{sec:fanselow-determinativ}
\is{common noun}
\is{noun!common}
\is{compound!{determinative}|(}
\is{compound!{common nouns in}|(}

\subsubsection{Restrictions and the question of subsectivity}
\label{sec:fanselow_restrictions}

\citet[130]{Fanselow:1981} begins his treatment of determinative compounds by discussing the question whether there are any general
restrictions on this type of compounds, starting with the categories that,
according to \citet{Brekle:1970}, play no role for compounds: quantification,
tense, assertion, mode and negation. For negation, \citet{Fanselow:1981}
agrees that this relation is in fact non-existent for compounds (contra
\citealt{Downing:1977}, cf. also \emph{nonentity} discussed in Section \ref{sec:nonintersective_adjectives}). As for quantification, he agrees with
\citet{Brekle:1973} that it is usually indefinite (e.g., \emph{car engine}
$\approx$ engine of a car), and he likewise agrees with
Brekle that the category of \emph{assertion} is irrelevant for
compounds, insofar as it has nothing to do with compound formation in particular. For
mode, \citet[139]{Fanselow:1981} argues that it is needed, giving \emph{Ziegellehm} `brick clay' and
\emph{Kuchenmehl} `cake flour' as examples: brick clay can be used to make bricks, and cake flour can be used to make
cakes. On tense, \citet[133--139]{Fanselow:1981} argues that it is
needed, but essentially restricted to perfect and co-temporality. 
% restricted to what he calls non-deictic tempora, that is, tempora that
% require a certain \emph{Betrachtzeit}. 
A relevant pair of examples is \emph{eine
    Nagelfabrik} `a nail factory (= a factory that produces nails)' %\emph{Fabrik, die Nägel herstellt} 
vs. \emph{ein
    Fabriknagel} `a factory nail (= a nail that has been produced in a
  factory)'% \emph{Ein Nagel, den man in der Fabrik hergestellt hat.} `'
  , illustrating  co-temporality and perfect respectively.  In this
  connection, he also points again to the different readings due to habitual
  vs. at least once interpretations, cf. \emph{LKW-Fahrer} `truck driver' in \ref{ex:LKW_Fahrer} above,
  and also notes their complementarity with what will be discussed in Section \ref{sec:fanselow:two-patterns} as
  basic relations (cf. \citealt[138--139]{Fanselow:1981}). 

In a next step, Fanselow considers the similarity between the first nominal
element in these compounds and adjectives: ``We can view the first constituent of a compound as a very
  special adjective with complicated semantics'' [my translation]
  \citep[142]{Fanselow:1981}, and addresses to what extent they are
  subsective. \is{modification!{subsective}|(}
He distinguishes 5 types of deviations from
  subsectivity. The first type subsumes combinations that do not fall
  into the common noun - common noun
  category, e.g. combinations of proper nouns like \emph{Baden-Württemberg}
  `Baden Württemberg', or combinations with a relational noun as the second
  element. The second type are compounds where either the first or the second element contain a
  `regelmäßig bedeutungsverschiebende[n] Faktor', a regularly meaning-shifting
  element. An example is \emph{Scheingefecht} `mock
  battle', cf. \ref{fanselow:abNOTb:scheingefecht} above, further examples are given in \Next.

  \ex. \a. \gll
  \emph{Saufbruder}\\
  drinking:brother\\
  `drinking companion'
  \b. \gll
  \emph{Ehrenjungfrau}\\
  honor:maiden\\
  `lady of honor' 
  \c. \gll
  \emph{Boykott\-brüder}\\
  boycott:brothers\\
  `guys involved in a boycott'

For these examples, compare also the comments following
\ref{fanselow:abNOTb} above. \enlargethispage{\baselineskip}
Thirdly, he
excludes so-called bahuvrihis, cf. the examples in \Next,  due to their low productivity and
unsystematicity.

\ex. \a. \gll
\emph{Dummkopf}\\
stupid:head\\
`idiot'
\b. \gll
\emph{Blaustrumpf}\\
blue:sock\\
`bluestocking'
\c.  \gll
\emph{Einhorn}\\
one:horn\\
`unicorn'

Fourthly, some compounds (1) do not contain a regularly meaning-shifting element,
(2) can be understood context-free, and (3) are not bahuvrihis. His
examples are repeated in \Next and his further sub-classification is given in \NNext, cf. (4) in \citet[143]{Fanselow:1981}. 


\ex. % Deviations from 'AB is contained in B':
% compounds that a) do not contain a regularly meaning-shifting factor,
% b) can be understood context-free and c) are not bahuvrihis:
\a. \gll 
\emph{Kindergeld}\\
child:money\\
 `child benefit'
\b. \gll 
\emph{Spielgeld}\\
play:money\\
 `toy money'
\c. \gll 
\emph{Stoffhund}\\
cloth:dog\\ 
`stuffed dog'
\d. \gll 
\emph{Bronzegott}\\
bronze:god\\
`bronze god'
\d. \gll 
\emph{Holzgewehr}\\
wood:rifle\\
`wooden rifle'
\d. \gll 
\emph{Spielzeugauto}\\
toy:car\\ 
`toy car'
\d. \gll 
\emph{Schokoladenzigarette}\\
chocolate:cigarette\\ 
`chocolate cigarette'

% \ex. % Deviations from 'AB is contained in B':
% % compounds that a) do not contain a regularly meaning-shifting factor,
% % b) can be understood context-free and c) are not bahuvrihis:
% \begin{tabular}[t]{lll}
% a.&\emph{Kindergeld}&child.money `child benefit'\\
% b.&\emph{Spielgeld}&play.money `toy money'\\
% c.&\emph{Stoffhund}&cloth.dog `stuffed dog'\\
% d.&\emph{Bronzegott}&bronze.god `bronze god'\\
% e.&\emph{Holzgewehr}&wood.rifle `wooden rifle'\\
% f.&\emph{Spielzeugauto}&toy.car `toy car'\\
% g.&\emph{Schokoladenzigarette}&chocolate.cigarette `chocolate cigarette'
% \end{tabular}

\ex. 
% A rough subclassification is possible:
\a. A stands for a material from which the objects denotated by B
cannot be made/consist of (for functional or other reasons):\\
\emph{Schokoladenzigarette} `chocolate cigarette'
\b. A stands for a function that normally is not the function of the
B-objects:\\ \emph{Spielzeugauto} `toy car'
\c. A is normally not a participant in the activity associated with
B:\\ \emph{Kindergeld} `child benefit'
\z.

Finally, the fifth type of deviation subsumes compounds that would not receive the indicated interpretation
without a specific context, \citet[143]{Fanselow:1981} gives the 3
examples in \Next, corresponding to his (5). 

\ex. % Deviations from 'AB is contained in B':\\
Compounds that would not receive the indicated interpretation
without a specific context:
\a. \gll
\emph{Fahrradbaby}\\
bicycle:baby\\
`somebody who just learned how to bicycle'  
\b. \gll
\emph{Tribünen\-sportler}\\
tribune:sportsman\\
`somebody who likes to watch sports from the grandstand'
\c. \gll
\emph{Juso-Oma}\\
Juso-grandmother \\
`somebody who supports Juso-aims but is,
in the view of the speaker, already too old' [Juso: a youth organization of the social democratic party] 
\z.
% \a.
% \begin{tabular}[t]{p{1.8cm}p{2cm}p{5.6cm}}
% \emph{Fahrradbaby}& `bicycle baby'& somebody who just learned how to bicycle  
% \end{tabular}
% \b. \begin{tabular}[t]{p{1.8cm}p{2cm}p{5.6cm}}
% \emph{Tribünen\-sportler}& `tribune sportsman'& somebody who likes to
% watch sports from the grandstand
% \end{tabular}
% \c. \begin{tabular}[t]{p{1.8cm}p{2cm}p{5.6cm}}
% \emph{Juso-Oma}& `Juso grandmother'& [Juso: a youth organization of
% the social democratic party] somebody who supports Juso-aims but is,
% in the view of the speaker, already to old
% \end{tabular}
% \z.

\citet[144]{Fanselow:1981} points out that the interpretation of a compound like \emph{Bronze\-löwe} `bronze lion'
does not present a compound-specific problem, as can be seen when looking at corresponding phrasal variants, cf. \Next.

\ex. \a. \gll
\emph{bronzener} \emph{Löwe}\\
bronze$_{\text{ADJ}}$ lion\\
`bronze lion'
\b. \gll
\emph{Löwe} \emph{aus} \emph{Bronze}\\
lion from bronze\\
`bronze lion'

Further, \citet[144]{Fanselow:1981} points out that the required meaning shifts are not in any
way different from shifts that are already needed for simplicia. This
latter point can be easily demonstrated by passages like the one in \Next:
\pagebreak[4]

\ex. \label{ex:loewe}
Der andere \textbf{Löwe} stammt von der einstigen Landesrechtspartei im Herzogtum, die
sich seit langem um die Wiedereinsetzung eines Welfen in Braunschweig bemüht
hatte. Auch dieser \textbf{Löwe} erinnert an das welfische Wappentier. Nur hockt der
\textbf{Löwe} auf einem Granitsockel, reißt das Maul auf und legt die Pranke auf das
Wappenschild des Herzogtums. So wacht er symbolisch über das Land Ernst
Augusts und Victoria Luises.  DeReKo
% BRZ13/MAI.11357 Braunschweiger Zeitung, 31.05.2013, Ressort: 1BS-Lok; Braunschweigische Löwenals Hochzeitsgeschenke
\\[0.4ex]
`The other {lion} comes from the former right-of-the-land party in the
dukedom [\dots] This {lion}, too, reminds one of the Welfian heraldic
animal. But the {lion} crouches on a granite pedestal, yanks open his
mouth and has his paw on the coat of arms of the dukedom. [\dots]'

In addition, \citet[145]{Fanselow:1981} notes that even for adjective noun
combinations, one finds either reinterpretations of the noun (as in
e.g. \emph{bronzener Löwe} `bronze lion') or reinterpretations of the adjectives (as in
\emph{scharfer Hund}  `sharp (= aggressive) dog'). In general, ``[w]hat exactly is re-interpreted is determined by rules that
  crucially rely on questions of psychology and the state of things in
  the world'' [my translation] \citep[147]{Fanselow:1981}.

\citeauthor{Fanselow:1981} comes to the following conclusion:
\begin{quotation}
We have thus reached a point where the compositional semantics
  must be silent and is allowed to assume that \emph{Bronzelöwen}
  `bronze lions' are indeed lions, and a \emph{Juso-Oma} `Juso grandmother' is indeed a
  grandmother, but \emph{Löwe} `lion' and \emph{Oma} `grandmother' understood with a
  re-interpreted denotation. Thus the first constituents [\emph{Vorderglieder}] are
  subsective. That is, every semantic rule should be of the kind:
  $\lambda x\; (R(A,B))(x)\; \&\; B(x))$. That the first constituents are subsective
  is also the only implication relationship [\emph{Folgerungsbeziehung}] that we can defend for
  compounds consisting of 2 common nouns'' [my translation] \citep[147]{Fanselow:1981}.\is{compositionality!{Fanselow on}}
  \end{quotation} % checked the quote
\is{compound!{common nouns in}|)}
\is{modification!{subsective}|)}
\subsubsection{Two patterns for compounds: stereotypes or basic relations}
\label{sec:fanselow:two-patterns}
\is{basic relations!{Fanselow's system}|(}
\is{stereotypes|(}
Fanselow assumes that in many cases the relation that is not explicitly expressed in a
  compound can be derived from the meaning of one of its 2 parts. The
  general idea is best illustrated with his examples,
  cf. \Next.

  \ex. \a. \gll
  \emph{Zeitungsfrau}\\
  newspaper:woman \\
  `woman who delivers the newspaper'
  \b. \gll
  \emph{Buchgeschäft}\\
  book:store\\
  `book store'

According to Fanselow, the meaning of \emph{Zeitung} `newspaper' is the source of the inferred relation \emph{zustellen} `deliver' in \Last[a], and the meaning of \emph{Geschäft} `store'  is responsible for the
  inferred relation \emph{verkaufen} `sell' in \Last[b]. Some further examples
  from \citet[156]{Fanselow:1981} are reproduced in \Next, with the compound part responsible for the inference set in boldface. The inferred relation itself is made explicit and set in boldface in the free paraphraes.

  \ex. \label{ex:fanselow_156_stereotpyes}
  \a. \gll
  \emph{\textbf{Taschen}messer}\\
  pocket:knife\\
  `knife \textbf{carried in} one's pocket' %[carried in]
  \b. \gll
  \emph{\textbf{Fabrik}geige}\\
  factory:violin\\
  violin \textbf{made in} a factory
  \c. \gll
  \emph{\textbf{Zug}passagier}\\
  train:passenger\\
  `paassenger \textbf{riding} a train'
  \d. \gll
  \emph{\textbf{Garten}blume}\\
  garden:flower\\
  `flower \textbf{growing in} gardens'
  \d. \gll
  \emph{\textbf{Düsen}jäger}\\
  jet:hunter\\
  `jet fighter, i.e. fighter \textbf{powered by} jets'
  \d. \gll
  \emph{Roß\textbf{arzt}}\\
  horse:doctor\\
  `doctor \textbf{treating} horses'
  \d. \gll
  \emph{Tag\textbf{falter}}\\
  day:butterfly\\
  `butterfly \textbf{flying} during the day'
  \d. \gll
  \emph{Zuhälter\textbf{mercedes}}\\
  pimp:Mercedes\\
  `mercedes that pimps \textbf{drive in}'
  \d. \gll
  \emph{Sarg\textbf{nagel}}\\
  coffin:nail\\
  `nail for \textbf{pounding into} coffins' %\\ % einschlagen in]\\
  \d. \gll
  \emph{Sekt\textbf{flasche}}\\
  champagne:bottle\\
  `bottle \textbf{containing} champagne'
  
% \begin{tabular}[t]{lll}
%   \emph{\textbf{Taschen}messer}&&\\pocket:knife&[carried in]&\\
% \textbf{Fabrik}geige&factory:violin&[made in]\\
% \textbf{Zug}passagier&train:passenger&[drive]\\
% \textbf{Garten}blume&garden:flower&[grow in]\\
% \textbf{Düsen}jäger&jet:hunter `jet fighter'&[to power]\\
% Roß\textbf{arzt}&horse:doctor&[treat]\\
% Tag\textbf{falter}&day:butterfly&[fly during]\\
% Zuhälter\textbf{mercedes}&pimp:Mercedes&[drive in]\\
% Sarg\textbf{nagel}&coffin:nail&[pound into]\\ % einschlagen in]\\
% Sekt\textbf{flasche}&champagne:bottle&[contain]\\
% \end{tabular}
% \item seems possible from a learner's point of view
% \item some examples: 
% \item p. 156: list of further examples where always one part lets one
%   infer the correct relation
\citet[156]{Fanselow:1981} argues that not all compounds fall under
this generalization, compare \emph{Politiker-Komponist} `politician-composer', 
  \emph{Juso-Student} `Juso student',  or \emph{Küstenstadt} `coast
  town (= coastal town)', where Fanselow thinks that it would be strange to argue that the
  \emph{and} relation in \emph{Politiker-Komponist} `politician-composer' or the \emph{located by} relation
  in \emph{Küstenstadt} `coastal town' are linked to the meanings of either
  \emph{politician} or \emph{town}. Therefore, he argues that 2 classes of
  compounds need to be distinguished: A first, smaller class, whose members
  can be generated with the help of the 5 basic relations
  \emph{und} \textsc{and}, \emph{gemacht aus} \textsc{made of}, \emph{ähnelt}
  \textsc{similar to}, \emph{ist teil von} \textsc{part of}, and 
      \emph{ist lokalisiert bezüglich} \textsc{located relative to}. %p. 156
And a second, larger class, where the  meaning is derived from the
meaning of its constituents.
\is{semantic relations!inferred relations}

\citet[157]{Fanselow:1981} proposes the following operational distinction
between the set of basic relations and inferred relations:\is{semantic
relations!basic relations}\is{semantic
relations!inferred relations}
\begin{quotation}
If the most explicit paraphrase
    of the compound AB contains nothing that has to do either with the
    meaning of A or B, then the relation is a basic relation. \footnote{``Wenn die expliziteste Paraphrase des
    Kompositums \uline{AB} nichts enthält, was mit der Bedeutung von A oder B
    in Zusammnhang stünde, so liegt eine Grundrelation vor.''}
    [my translation] \citep[157]{Fanselow:1981}
\end{quotation}
 He motivates  his
 operationalization by using the compound \emph{Kinder\-zimmer} `nursery', cf. \Next.

 \ex. \gll
 \emph{Kinder\-zimmer} \\
children:room\\
`nursery'

One could propose a reading à la \emph{room meant for
    children}, and, based on this reading, establish a basic relation \textsc{meant
  for}. However, there is a more explicit paraphrase for this compound, namely \emph{room
    in which usually the children live}. And since \emph{to live in} can be
  related to \emph{room}, no basic relation is needed for a successful interpretation. 
According to Fanselow, all basic relations (except \textsc{be similar to}) can be linked to basic principles of the
  organization of the lexicon, e.g. hyponymy, partonomy, and local
  inclusion (here Fanselow refers to \citealt[79]{Miller:1978}). In addition, Fanselow states:
  \begin{quotation}
If one learns the meaning of, e.g., \emph{hammer}, then one has to
  learn that it holds of things with a specific form and function, if
  one learns the meaning of \emph{nail}, that these are things to be
  hammered into walls etc. But one does not need to learn that their
  denotations are located somewhere, that they can belong to other
  denotations, or that they are made out of something.

While therefore the inferred relations are something that
  needs to be learned when learning the meanings of the words, the
  basic relations are organizational principles of perception or of
  semantic classification that are constituted independent of the
  meanings of individual words. [my translation] \citep[158]{Fanselow:1981}
    
  \end{quotation}
\is{basic relations!{Fanselow's system}|)}
\is{stereotypes|)}

\subsubsection{More on compound interpretations based on stereotypes}
\label{sec:stereotypes}
\is{stereotypes|(}
For Fanselow, ``[a] stereotype A$_i$ of a word A is a typical property
  of things that fall under A. Its semantic type is
  therefore necessarily the same as the translation of A'' [my
  translation] \citep[169]{Fanselow:1981}. In many cases, the
  stereotypes of 2 compound constituents will not generate the
  specific meaning of the compound. A case in point is his example
  \emph{Taschenmesser} `pocket knife'. \emph{Tasche} `bag, pocket' has
  the stereotype \emph{carry in}, and one can therefore generate the
  meaning `knife that can be carried in a bag'. In contrast, the more
  specific meaning, e.g. `a small knife with one or more blades that
  fold into the handle', is due to the compound developing its own stereotype.

\citet[168]{Fanselow:1981} mentions the categories introduced in \citet{Shaw:1978}, that is
\emph{vollmotiviert} `completely motivated', % e.g. \emph{Nagelfabrik} `nail factory',
\emph{ teilmotiviert} `partially motivated', % e.g.  \emph{Steinpilz} stone.mushroom `Boletus edulis (penny bun/cep)' and \emph{Butterblume} butter.flower `buttercup'
\emph{unmotiviert} `not motivated', %, e.g. \emph{Hahnenfuß} cock.foot `buttercup/crowfoot'
and ponders the introduction of 2 categories building on these ideas:
(1) \emph{systemmotiviert}  `motivated by the system',
% e.g. \emph{Taschenmesser} `pocket knife' and \emph{Blutbuche} `blood beech':
where the production system almost, but not quite, yields
  the full meaning of the compound, and (2) \emph{motiviert im engern
    Sinne} `motivated in a strict sense', where 
  % e.g. \emph{Staudacherbruder} `Staudacher's brother' and \emph{Nagelfabrik} `nail factory':
  the production system yields
  the full meaning of the compound. Shaw's categories are illustrated in \Next, Fanselow's 2 categories in \NNext, cf. \citet[168]{Fanselow:1981}.

\enlargethispage{1\baselineskip}
\ex. \a. completely motivated
  \gll \emph{Nagelfabrik}\\
nail:factory\\
  `nail factory'
  \b. partially motivated
  \a. \gll
  \emph{Steinpilz}\\
  stone:mushroom\\
  `Boletus edulis (penny bun)'
  \b. \gll
  \emph{Butterblume}\\
  butter:flower\\
  `buttercup'
\z.
\c. not motivated
\gll
\emph{Hahnenfuß}\\
cock:foot\\
`buttercup/crowfoot'

\ex. 
\a. motivated by the system
\a. 
\gll
\emph{Taschenmesser}\\
pocket:knife\\
`pocket knife'
\b. \gll
\emph{Blutbuche}\\
blood:beech\\
`blood beech'
\z. 
\b. motivated in a strict sense
\a. \emph{Staudacherbruder}\\
Staudacher:brother\\
`Staudacher's brother'
\b. \emph{Nagelfabrik}\\
nail:factory\\
`nail factory'
\z.
  
\citet[\S 18]{Fanselow:1981} discusses how best to formulate the
stereotypes. Again, I report here only some of his observations. In general,
if the inferred relation is  stative, cf. e.g. \emph{Rheinbrücke}
`Rhine bridge (= bridge over the Rhine)' or
\emph{Kandidatenplakat}  `candidate poster (= poster for a candidate)' (cf. \citealt[157]{Fanselow:1981}), there is no
ambiguity between habitual and instantaneous [`instantiell'] readings.\is{ambiguity!{habitual vs. instantaneous readings}}\is{compound!{ambiguity}!{habitual vs. instantaneous readings}}
 In all other cases, one
finds an ambiguity, with the habitual reading being the preferred
one. \citet[192]{Fanselow:1981} assumes that stereotypes are generally
habitual, but allow the derivation of instantaneous readings. The relations
inferred via the first constituent might be either efficient (as in
\emph{Fabriknagel} `factory nail' $\rightarrow$ produce) or afficient (as in
\emph{Raketenbasis} `missile base' $\rightarrow$ fire); among other things,
this will influence the choice of tense. Words typically have several
stereotypes that are relevant for compound composition, \emph{Milch}
`milk' allows one to
infer \emph{drink in} in \emph{Schulmilch} `school milk' but \emph{given by}
in \emph{Kuhmilch} `cow milk'. As already shown in the
examples in \ref{ex:fanselow_156_stereotpyes}, some of the stereotypes contain
a local relation, e.g. \emph{Schulmilch} `school milk' and \emph{Teehaus} `teahouse', which both use
the relation \emph{drink in}. \enlargethispage{\baselineskip}
Stereotypes come with constraints on their
argument places, see \emph{Professorenfabrik} `factory for professors' in \Next, his
(9).

\ex. \gll Die Uni Konstanz ist eine richtige \textbf{Professorenfabrik}.\\
The Uni Konstanz is a right professor.\textsc{pl}:factory\\
`The University of Konstanz is a right factory for professors.' 
% \\ = (9) in \citet[194]{Fanselow:1981}

Factories, according to Fanselow, produce inanimate things. Since professors are animate, the compound
must be understood metaphorically. 
% I think one can even be more precise and
% say that only \emph{Fabrik} `factory' must be understood metaphorically. 

According to \citet[201]{Fanselow:1981}, the frequency
adverbials that are most likely to modify a given inferred relation
seem to be determined by the relevant stereotypes (or, as he puts it,
the words determine this for their stereotypes). To illustrate this, Fanselow contrasts
\emph{Raketenbasis} `base for firing missile in case of need for doing
so' with \emph{Gartenblume} `flower that usually grows in a
garden'. In addition, he notes that compounds like \emph{Nagelfabrik}
`nail factory', `a factory that usually produces nails', can be
given a more precise paraphrase, e.g. `usually, if this factory
  produces sth., it produces nails'. A similar step, according to Fanselow, is
possible for \emph{Gartenblume} `garden flower' (`usually, if
  this flower grows somewhere, it grows in the garden') and
\emph{Sektflasche} `champagne bottle' (`usually, if this bottle
  contains something, it contains champagne'). This is, however, not
possible across the board, as he illustrates with
\emph{Silberbergwerk} `silver mine', which not only excavates silver
(technically, it excavates ore which contains silver, for one thing),
and \emph{Zuhältermercedes} `pimp mercedes', which is not only driven
by pimps. Finally, in a few cases, stereotypes of both compound parts
are involved, see his example \emph{Teehaus} `teahouse', where, according to
\citet[202]{Fanselow:1981}, the inferred relation \emph{drink in} is
due to the contribution of both parts, \emph{tea} being responsible
for the \emph{to drink} relation, and \emph{house} being responsible
for the \emph{in} (as opposed to the local relation to be inferred for \emph{Thekenbier}
`counter beer', `a beer that is drunken at the counter').% \emph{drink at}).
\is{stereotypes|)}

\subsubsection{More on Fanselow's basic relations}
\label{sec:fanselows_basic_relations}

\is{basic relations!{Fanselow's system}|(}
\citet[\S 17]{Fanselow:1981} discusses the basic relations in more detail and
gives further examples. For the combination of 2 common nouns, the
first basic relation, \emph{und} `and', is analyzed as intersection, that is,
for a compound AB, we have $\lambda$ x (a'(x) \&  b'(x)), where a' is the
semantic translation of A and b' that of B. Relevant examples are \emph{Eichbaum} `oak tree',
    \emph{Juso-Student} `Juso-student', \emph{Hausboot} `house boat',
    \emph{Radio-Uhr} `radio clock', \emph{Negerfrau}
    `negro woman'\footnote{And yes, this is considered to be politically
      incorrect in German nowadays, too.}, and \emph{Juso-Oma} `Juso-grandmother'. Other
  examples, like \emph{Mördergeneral} `murderer general', are already more complex;
% However, even for formations based on \emph{and}, 
he speculates:
\begin{quotation}
One might wonder whether words
  like \emph{Mördergeneral} `murderer general (= general who is a murderer)', \emph{Mörderpolizist}
  `murderer police man', \emph{Mörder\-kanzler} `murderer chancelor', and
    \emph{Mörderpräsident} `murderer president'
   are all \emph{Analogiebildungen} `analogical formations'. Because it
  further seems to hold that the interpretation of compounds via
  stereotypes constitutes an explication of that which is usually seen
  as \emph{Analogiebildung} `analogical formation', we can view
  \emph{Mördergeneral}`murderer general' and \emph{Einbrecherpolizist} `thief policeman' simply as a very
  specific case of a general word formation possibility. [my translation]
  \citep[176]{Fanselow:1981}  
\end{quotation}

Note that the difference between the 2
  groups observed by Fanselow seems to correspond to compounds that,
  in Levi's system, would be analyzed with the \textsc{be} relation, and, on
  the other hand, those combinations she refers to as coordinate
  structures and excludes from her analysis. In
  \citet[479--480]{Baueretal:2013}, these 2 types are both subtypes
  of coordinative compounds, referred to as
  appositive and additive respectively.

The \textsc{made of} relation is essentially treated by Fanselow as an extension of the
\textsc{and}-relation, so that e.g. an x is a \emph{Roggenbrot}
`rye bread' iff x is bread
and has been rye, and there is a process that caused the rye to be bread
afterwards, cf. \citet[180]{Fanselow:1981}.  

For the basic relation \textsc{part of} \citet[184--185]{Fanselow:1981}
gives the examples \emph{Autokotflügel}  `car mudguard' and \emph{Kammzinke}
`comb tooth'. Interestingly, he points out that ``[w]e cannot simply translate \emph{Autokotflügel} `car mudguard'
  into: an x, that is a mudguard and part of a car. The mudguard can
  be dismantled, the corresponding car does not need to continue to
  exist, nor does there need to be a car at all in order for a thing
  to be a car mudguard.''[my translation]
  \citet[184]{Fanselow:1981}. Cf. \nocite{Bell:2012} Bell's entailment
  criterion discussed in Section \ref{sec:semTran-Stress}, Chapter
  \ref{cha:theo} for a similar point. \is{semantic relations!entailment criterion}



\citet[185--186]{Fanselow:1981} believes that the location-relation has fewer usages than commonly
assumed; he takes \emph{Küstenstraße} `coast road (=coastal road)',
\emph{Ha\-fen\-stadt} `harbor town', and
\emph{Bergdenkmal} `mountain monument (= monument in the mountains)' to be clear examples. In contrast, compounds like
\emph{Nachtarbeiter}\linebreak[4] `night worker', \emph{Automotor}
  `car engine' and  \emph{Rheinbrücke} `Rhine bridge' need ste\-reo\-types for
  their correct interpretation. In trying to find a good formal spell-out for a
  location relation, he mentions an observation by \citet{Warren:1978} for English
  as support for what is essentially an underspecified localization
  relation: the conjunction with \emph{und} `and' is only possible, if the
  underlying relation is the same, cf. combinations like \emph{Sekt-
    und Weingläser}. Since one can form combinations like  \emph{die Gruben- und Landarbeiter
    Boliviens} `mine and farm laborers of Bolivia', with the mine laborers being in the
  mine and the farm laborers being on the farm land, this can be taken as
  support for an underspecified locative relation. \is{semantic
    relations!underspecified locative relation}\is{underspecification!{locative relation and}}
However, he also notes that
  this idea does not generalize to all cases, as examples like \emph{das
    Münchener Schnell- und Untergrundbahnsystem} `the Munich express and
  subway railsystem' and \emph{Schnell-
    und Güterzüge} `express and freight trains' contain coordination with \emph{and} even though
  the relationship between the head and the first elements cannot be the same,
  cf. \citet[Footnote 10]{Fanselow:1981}. A second rule to introduce location
  relations is needed for compounds like \emph{Denkmalsberg} `monument mountain' and
  \emph{Stadtküste} `town coast', which, according to Fanselow, are converses of
  the location relation as seen in the examples above. However, I
  think that instead of introducing a converse location relation one
  could also argue for other relations here, e.g. \textsc{part of} or \textsc{and}.

The last basic relation to be discussed is \textsc{be similar to},
which comes in many forms. \Next gives a few examples from his
overview.\footnote{\citet[Footnote 13, p.~188]{Fanselow:1981} points to a similar overview for English in
  \citet{Warren:1978}.}

\ex. \a. B has the form of A: 
\a. \gll
\emph{Flammenschwert}\\
flame:sword\\
`flame-bladed sword'
\b. 
\emph{Einhornplastik}\\
`unicorn sculpture'
\z.
\b. B has the color of A:
\a. \emph{Blutbuche}\\
`blood beech' 
\b. \gll
\emph{Silberpappel}\\
silver:poplar\\
`white poplar' 
\c. \gll
\emph{Laubfrosch}\\
foliage:frog\\
`European tree frog'
\d. \emph{Milchglas}\\
`milk glass'
% \c. B has the consistency of A \emph{Wasserglas} form Blatz 1895
% p. 742 \textbf{strange, I don't understand this example}
% \d. \dots
\z.

Fanselow's basic observations is that the exact type of the
  \emph{being-similar-to} relation is determined via stereotypes, and he
  exploits this via a semantic rule that makes use of stereotypes. As a
  result, in his system compounds like \emph{Blutbuche} `blood beech'
  are explained with the help of a semantic rule, but a compound like
  \emph{Bronzelöwe} with the help of a
  pragmatic rule, which \citet[191]{Fanselow:1981} justifies by
  arguing that these semantic rules follow a clear pattern, whereas
  the shifts of the head
  % Hinterglieder-shifts
  seen in \emph{Bronzelöwe} seem to behave more
    unruly. In addition, in his view explicit rules are always better than pragmatic explanations.
\is{basic relations!{Fanselow's system}|)}
\subsubsection{Context dependency and ambiguity in Fanselow's system}
\label{sec:fanselow_big_picture}

\is{compound!{context sensitivity}|(}
\is{context sensitivity|(}
Fanselow argues that his system can derive most readings that compounds have.
However, his system also generates ambiguity.\is{ambiguity!{in Fanselow's system}} 
In addition, nothing in his
system as it stands is able to deal with context sensitivity.
 Thus, Fanselow
mentions that combinations like \emph{Taschenmesser} `pocket knife'
and
\emph{Fabrikgeige} `factory violin' could also be used to mean `knife
      to cut pockets with', or `violin for usage in a factory' respectively.

He thinks that a hierarchy like \Next, cf. \citet[215]{Fanselow:1981},  is likely to be in place, and proposes
the general hypothesis for compound interpretation in \NNext.
\ex. Hierarchy for compound rules:
\a. stereotypes
\b. and-rules
\c. location-rules
\d. similarity-rule
% \z. Cf. \citet[215]{Fanselow:1981}

\ex. \label{ex:hypothesis_of_compound_composition_second_version}
  Hypothesis for the interpretation of nominal compounds\\
  In the interpretation of an AB compound, that relation R holds between A
  and B which is the most prominent relation in a given context among the relations whose
  linguistic realization occurs most often in sentences between A and
  B so that the compound interpreted with that relation R makes sense
  in the given context. [my translation] \citet[215]{Fanselow:1981}\footnotemark

\footnotetext{``Hypothese zur Interpretation der Nominalkomposita\\
Bei der Interpretation eines Kompositum AB tritt genau die semantische
Beziehung R zwischen A und B, die unter den Beziehungen, deren sprachliche
Realisation mit großer Häufigkeit in Sätzen zwischen A und B tritt, im
jeweiligen Kontext die prominenteste ist, so daß das Kompositum interpretiert
mit R im Kontext sinnvoll ist" \citep[215]{Fanselow:1981}.}



Based on this, \citet[215--216]{Fanselow:1981} thinks that it is not the case that every interpretation of a compound is
    possible in the respective contexts, and that the number of possible readings is smaller than
    the number of possible relations that are technically available.

He concludes his work by pointing out that it is still an open question which factors determine the most
  prominent relations/stereotypes in a given context, and illustrates the
  problem with the 2 examples in \Next and \NNext, examples (1) and
  (2) in \citet[221]{Fanselow:1981}, where, according to Fanselow, in
  neither case the interpretation suggested in the sentence preceding
  the sentence containing the compound is able to win over a compound interpretation based on
  stereotypes.

\ex. Ich schlug einen Nagel in einer Fabrik ein. In dieser \textbf{Nagelfabrik} war es kalt.\\
`I pounded a nail into the factory. It was cold in this nail factory.'
% (1) in \citet[221]{Fanselow:1981}

\ex. Hedwig ist eine Lehrerin, die sich sehr für Geschichte
interessiert. Diese \textbf{Geschichts\-lehrerin} treffe ich leider zu selten in der `Schwedenkugel'.\\
`Hedwig is a teacher who is really into history. Unfortunately, I meet this history teacher all too seldomly in the `Schwedenkugel'.\,'
% (2) in \citet[221]{Fanselow:1981}

That is, \emph{Nagelfabrik} in \LLast is not interpreted as `factory
into which I pounded a nail' but is still a factory that produces
nails, and likewise the \emph{Geschichtslehrerin} `history teacher' in
\Last is still somebody who teaches history, not a person who is
interested in history and at the same time a teacher.
 
In general, he assumes
  that ``the further one moves away from whatever one can call the
  pragmatically normal relation, the stronger the contextual marking needs to
  be'' [my translation] \citet[221]{Fanselow:1981}.\footnote{``Je weiter man sich von dem fortbewegt, was man als die pragmatisch
  normale Relation bezeichnen kann, desto stärker muß die kontextuelle
  Markierung sein'' \citep[221]{Fanselow:1981}.} 
\is{pragmatics!{normal interpretation}}
In addition, he assumes that
  the problem of selecting the pragmatically normal interpretation crops up
  in a similar way when it comes to the interpretation of genitives or
  attribute phrases.
\is{compound!{determinative}|)}
\is{compound!{context sensitivity}|)}
\is{context sensitivity|)}


\subsection{Evaluating Fanselow's approach}

% \subsubsection{Fanselow and Levi compared}
% \marginpar{\textbf{ADD: Link between Fanselow-like and Levi-like discussion}}
% Check: basic relation/map on the Levi relations
% Check: does Fanselow brake down Levi's huge for class?
Fanselow's work continues to impress through his analytic clarity and
wide scope. The idea that stereotypes associated with the individual
compound constituents play a major role in arriving at the most
specific interpretation of a given compound is particular
attractive. While this view of compound semantics might at first sight
seem very different from Levi's 9 predicate system, aspects of both systems can be fruitfully combined in an approach in which
the semantic relations are seen as tied to specific concepts. If the relations are not seen as independently existing
objects but only relative to specific concepts, like
in the conceptual combination approach pursued by Gagné and
collaborators (cf. Chapter \ref{cha:semTranPsycho}, Section \ref{sec:conceptual_combination}), then categorizing
compounds into these relations quite naturally leads to a localization
of the set of Levi-relations to specific concepts. I will come back to
this issue in the description of the coding done for the analysis presented
in Chapter \ref{cha:empirical-2}, cf. especially Section \ref{sec:methodsSemCoding}.
\il{German!{Fanselow on German compounds}|)}

\section{Mixed approaches}
\label{sec:mixed_approaches}

Levi's and Fanselow's approaches have been
discussed in detail because they show the range of possibilities in
approaching compound semantics. 
The aim of this final section is not to give an overview of everything
that followed but rather to
introduce 2 further well-known strands of approaches. They are mixed in the sense that
they introduce means that allow them to integrate knowledge that is
not traditionally seen as belonging to the lexical meaning of
words in the building of their semantic representations. The next section introduces Pustejovsky's generative lexicon
and is followed by 2 sections discussing proposals that extend these
ideas to compounds. The final section introduces approaches based on
underspecification. On deciding to focus on these works, much other
work is necessarily left aside, notably
\citet{Meyer:1993}. 
Focusing on German novel compounds, \citet{Meyer:1993} provides a theory of compound
comprehension based on the 2-level semantics of
\citet{Bierwisch:1989} and adapting Discourse Representation Theory
\citep{Kamp:1981} to
represent lexical meaning and to model the process of arriving at an
utterance meaning of new compounds. In addition, \citet[12--38]{Meyer:1993} provides a very useful overview
of theories on compound semantics bridging the gap between \citet{Fanselow:1981} and the
early 1990s. 
%  allowing him to integrate a wide range of knowledge sources in  

\subsection{\citet{Pustejovsky:1995}}
\label{sec:mixed_approaches:Pustejovsky}



\is{adjective noun constructions!{Pustejovsky's examples}|(}
\citet{Pustejovsky:1995} introduces a general approach to lexical semantics
with no specific focus on compounds or complex nominals. However, as adjective noun
constructions constitute one major class of his examples, and his approach is
used elsewhere for the analysis of traditional compounds (cf. especially
\citealt{Jackendoff:2009} and \citealt{Asher:2011}, see also the discussion
below), I will discuss his approach here, focusing especially on his
discussion of adjective noun combinations. 

\citet[32]{Pustejovsky:1995}, pointing to earlier works by \citet{Katz:1964} and \citet{Vendler:1963}, notes that the adjective \emph{good}
occurs with multiple different senses, depending on which noun it modifies, cf. the examples in \Next, his (23) .

\ex. \a. a good car
\b. a good meal
\c. a good knife

Typical interpretations for \emph{good} in \Last[a] and \Last[c] might
be `of high quality', while \Last[b] is typically interpreted as
`delicious'. Illustrating this point further, he gives the following
examples and paraphrases for
\emph{fast}, cf. his examples (7--14)
\citep[44--45]{Pustejovsky:1995}.\footnote{For the corpus sources of
  Pustejovky's examples, cf. \citet[244, Endnote 2]{Pustejovsky:1995}.}

\ex. The island authorities sent out a fast little government boat, the
Culpeper, to welcome us.\\
\emph{a boat driven quickly} or \emph{a boat that is inherently fast}
% \\ = (7) in \citet[44]{Pustejovsky:1995}

% \newpage
\ex. a \textbf{fast typist}\\
\emph{a person who performs the act of typing quickly} 
% \\ = (8) in \citet[44]{Pustejovsky:1995}

\ex. Rackets is a \textbf{fast game}.\\
\emph{the motions involved in the game are rapid and swift} 
% \\ = (9) in \citet[44]{Pustejovsky:1995}

\ex. a \textbf{fast book}\\
\emph{one that can be read in a short time} 
% \\ = (10) in \citet[44]{Pustejovsky:1995}

\ex. My friend is a \textbf{fast driver} and a constant worry to her cautious husband.\\
\emph{one who drives quickly} 
% \\ = (11) in \citet[44]{Pustejovsky:1995}

\ex. You may decide that a man will be able to make the \textbf{fast, difficult, decisions}.\\
\emph{a process which takes a short amount of time} 
% \\ = (12) in \citet[44]{Pustejovsky:1995}

\ex. The Autobahn is the \textbf{fastest motorway} in Germany.\\
\emph{a motorway that allows vehicles to sustain high speed} 
% \\ Cf. (13a) in \citet[45]{Pustejovsky:1995}

\ex. I need a \textbf{fast garage} for my car, since we leave on Saturday.\\
\emph{a garage that takes little time to repair cars} 
% \\ Cf. (13b) in \citet[45]{Pustejovsky:1995}

\ex. The \textbf{fastest road} to school this time of day would be Lexington Street.\\
\emph{a road that can be quickly traversed} 
% \\ Cf. (14) in \citet[45]{Pustejovsky:1995}

As \citet[44--45]{Pustejovsky:1995} points out, these readings allow to
distinguish between at least 4 distinct senses of \emph{fast}, cf. \Next. 
\ex. 
\a. \texttt{fast(1)}: to move quickly
\b. \texttt{fast(2)}: to perform some act quickly
\c. \texttt{fast(3)}: to do something that takes little time
\d. \texttt{fast(4)}: to enable fast movement
% \z. Cf. \citet[45-46]{Pustejovsky:1995}

In addition, \citet[46]{Pustejovsky:1995} assumes blended senses for
\emph{fast garage} (blends senses 2 and 3) and \emph{fast route} (blends senses 3 and
4). One important point is that there is no principled limit to new
senses or new blends of previous senses. Therefore, any system based
on what Pustejovsky calls a sense enumeration lexicon is bound to
fail. Instead, he proposes the system of the generative lexicon, which
I will introduce in the next section.
\is{adjective noun constructions!{Pustejovsky's examples}|)}


\subsubsection{The generative lexicon}
\label{sec:puste_gen}

Pustejovsky assumes that one can distinguish between different levels
of semantic representation, and that one needs a set of generative
devices that can be used to create new senses.

The 4 levels of representation he assumes are:
  \begin{inparaenum}[(1)]
  \item argument structure,
  \item event structure,
  \item qualia structure, and
  \item lexical inheritance structure.
  \end{inparaenum}

The set of generative devices includes the following semantic transformations:
  \begin{inparaenum}[(1)]
  \item Type coercion, that is, the semantic types can be shifted so that they
    match the type required by the functor they are to combine with.
  \item Selective binding, that is, specific senses can be tied to
    specific aspects of meaning.
  \item Co-composition, that is, information from both functor and argument is
    responsible for the creation of new senses.
  \end{inparaenum}

This allows rich meta entries, and in consequence a reduced size of the
  lexicon.  The meta entries are called \emph{lexical conceptual
    paradigms} (lcps). 

I will not go into details of the presentation of argument and event
structure, but instead focus here on the qualia structure, in my view
the element of Pustejovsky's approach that is usually considered to
be its main innovation. Since I am interested in nominals, I will
concentrate on them in the discussion of qualia structure, too.

\subsubsection{Qualia structure}
\label{sec:qualia-structure}

% \subsection{Qualia structure [76-81]}
The \emph{qualia structure} of a lexical item is intended to be ``the
structured representation which gives the relational force of a
lexical item'' \citep[76]{Pustejovsky:1995}. 
The 4 essential aspects of a word's qualia structure are listed by
\citet[76]{Pustejovsky:1995} as follows:
  \begin{inparaenum}[(1)]
  \item the constitutive aspect: ``the relation between an object and its constituent parts'';
  \item the formal aspect: ``that which distinguishes it within a larger domain'';
  \item the telic aspect: ``its purpose and function''; and
  \item the agentive aspect: ``factors involved in its origin or `bringing it about'\,''.
  \end{inparaenum}
Importantly, ``every [grammatical] category expresses a qualia
structure'', but ``[n]ot all lexical items carry a value for each qualia role'' \citep[76]{Pustejovsky:1995}.
Qualia values themselves come with their own types and relational
structures, cf. his example for \emph{novel} in \Next, reproducing (35) in \citet[78]{Pustejovsky:1995}. 

\ex. 
\begin{avm}
  \[ \textbf{novel} \\
\dots \\
QUALIA = \[ FORMAL = book(x)\\
TELIC = read(y,x)\\
\dots
\]
\]
\end{avm}

Specifically for nominals, he introduces the concepts of dotted types
in order to deal with cases like \emph{door, book, newspaper},  and \emph{window}, that is, cases
of what he calls `logical polysemy'. The intuition behind this
terminology is that for a noun like \emph{door} (at least) 2
word senses (the physical object and the corresponding aperture) can
be distinguished, with both senses being related since both are
arguments of the meaning of the the noun. The byword `logical' seems
to be used to indicate that this `inherently relational'
(\citealt[91]{Pustejovsky:1995}) characteristic of the nominal is
located at the level of lexical semantics, as opposed to the
level of concepts (cf. also \citealt{PustejovskyandAnick:1988}).  
 According to
\citet[92]{Pustejovsky:1995}, logical polysemy occurs in a number of
nominal alternations. \Next, his (11), reproduces his list of alternations along with 
examples.

\ex. \a.
\begin{tabular}[t]{p{6cm}p{6cm}}
count/mass alternations& \emph{lamb}  
\end{tabular}
\b. \begin{tabular}[t]{p{6cm}p{6cm}}
container/containee alternations& \emph{bottle}
\end{tabular}
\c. \begin{tabular}[t]{p{6cm}p{6cm}}
figure/ground reversals& \emph{door}, \emph{window}
\end{tabular}
\d. \begin{tabular}[t]{p{6cm}p{6cm}}
product/producer diathesis& \emph{newspaper, Honda}
\end{tabular}
\d. \begin{tabular}[t]{p{6cm}p{6cm}}
plant/food alternation& \emph{fig, apple}
\end{tabular}
\d. \begin{tabular}[t]{p{6cm}p{6cm}}
process/result diathesis& \emph{examination, merger}
\end{tabular}
\d. \begin{tabular}[t]{p{6cm}p{6cm}}
place/people diathesis& \emph{city, New York}
\end{tabular}
% \z. = (11) in \citet[92]{Pustejovsky:1995}

Each argument is of a specific type. The dotted types are the results
of combining the 2 types to form a complex type. That it is possible to
distinguish between the 2 senses in \Last and reference to a sense
corresponding to the resulting complex
type, or dot object, is illustrated by \citet[94]{Pustejovsky:1995}
with the help of the 3 occurrences of \emph{construction} in
\Next, his (17):

\ex. \a. The house's \textbf{construction} was finished in two months.
\b. The \textbf{construction} was interrupted during the rains.
\c. The \textbf{construction} is standing on the next street.
% \z. = (17) in \citet[94]{Pustejovsky:1995}

According to Pustejovsky, \emph{construction} in \Last[b] refers to the
process, while it refers to the result in \Last[c] and to the entire
dotted type in \Last[a].

\subsubsection{Adjective noun combinations and qualia structure}
\label{sec:adj_nouns_pustejovsky}

\is{adjective noun constructions!{qualia structure}|(}
Returning to adjective modification in a system with qualia structure,
Pustejovsky comments on \Next and \NNext:
\ex. \label{ex:pust1995:p89:ex6}
\a. a bright bulb
\b. an opaque bulb

\ex. \label{ex:pust1995:p89:ex7}
\a. a fast typist
\b. a male typist

According to \citet[89]{Pustejovsky:1995}, the 2 adjectives
\emph{bright/fast} in \LLast and \Last are event predicates. The event
they predicate over must in some way be associated with the qualia
structure of the noun (\emph{bulb} with the telic role of
illumination, \emph{typist} with the telic role making reference to
the process of typing). In contrast, 
\emph{opaque/male} access the  formal role of their respective heads.
\is{adjective noun constructions!{qualia structure}|)}
% [If we come back to the examples with \emph{fast} that where introduced
% above, how could they be modeled? ADD/LEAVE TILL LATER?]

\subsection{Extending the analysis to compounds 1: \citet{Jackendoff:2010}}
\label{sec:jackendoff}

\citet[442--445]{Jackendoff:2010}\footnote{Cf. also \citet{Jackendoff:2009},
  an earlier, shorter version of the same article.}, in a section
explicitly entitled `Using material from the meanings of N1 and N2', gives a number of
examples for which he assumes an analysis that is based on
Pustejovskian co-composition. His first example is \emph{water
  fountain}, ``a fountain that water flows out
of'' \citep[443]{Jackendoff:2010}. Because it is the proper function of
a fountain that liquid flows out of it, and because water is a
liquid, water can fill this spot in the telic role of
fountain. Jackendoff assumes a similar process for \emph{coal mine},
\emph{gas pipe}, \emph{Charles River bridge}, and \emph{toe-web}, and
mentions several larger families (below always illustrated with one of
his examples): N2 is a container (cf.
\emph{fishtank}), N2 is a vehicle (cf. \emph{oil truck}),
N2 is an article of clothing (cf. \emph{ankle bracelet}), N2 is
a location (cf. \emph{liquor store}), and N2 is the
incipient stage of something else (cf. \emph{dinosaur egg}). In addition, there are also cases
where N1 gives the topic of N2 (cf., \emph{research paper}), or
N2 is an agent or causer (cf., \emph{pork butcher}), or an
artifact (cf. \emph{steak knife}). Likewise, he discusses cases where the proper function
that drives the interpretation comes from the N1, like \emph{cannonball}. 

\citet[443, Footnote 22]{Jackendoff:2010} points to \citet{Brekle:1986},
who discusses these types of compound under the heading of `stereotype
compounds' (cf. \citealt[42, Section 2.2]{Brekle:1986}). \citeauthor{Brekle:1986}, in turn,
refers to the analysis based on stereotype relations from
\citet{Fanselow:1981}, cf. the discussion in Section \ref{sec:stereotypes}. I will come back to the connection
between the Pustejovskian approach and stereotypes below. % \marginpar{\textbf{TODO}}
For an analysis of these types of compound in terms of Pustejovsky's
(1995) qualia structure, Jackendoff points to
\citet{Bassac:2006}. % \marginpar{\textbf{Unable to find that paper!}}

\subsection{Extending the analysis to compounds 2: \citet{Asher:2011}}
\label{sec:asher}

\citet{Asher:2011} is not an extension of Pustejovsky's analysis to
compound nouns, but rather provides an alternative framework to
Pustejovsky's approach, which \citeauthor{Asher:2011} calls Type Compositional Logic or TCL. Nevertheless, for the data that concerns
compounding in particular, the discussion and analysis can be seen as
one way of spelling out the Pustejovskian approach. 
\citet[301--305]{Asher:2011} focuses on material modifiers, that is,
adjectives like \emph{wooden}, or nouns like \emph{glass},
\emph{stone}, etc. 
\is{noun!{material}|(}\is{modification!{with material modifier}|(}
These, as the examples in \Next, his (11.1) illustrate, ``supply the
material constitution of objects that satisfy the nouns these
expressions combine with" \citep[301]{Asher:2011}. 

\ex.
\[ \left.
\begin{array}{r}
\text{\emph{glass}}\\\text{\emph{wooden}}\\\text{\emph{stone}}\\\text{\emph{metal}}\\\text{\emph{tin}}\\\text{\emph{steel}}\\\text{\emph{copper}}
\end{array}\right\}  \text{\emph{bowl}}\]
% \{glass/wooden/stone/metal/tin/steel/copper\ bowl
% \z. 11.1 in \citet[301]{Asher:2011}

\citet[302]{Asher:2011} notes that material modifiers are particular
in being able to affect the typing of the head noun, cf. the examples in \Next,
his (11.2). % \is{modifier!material noun}
% \is{noun!material noun as modifier}

\ex. \label{ex:asher_2011_p301_ex11.2}
\a. stone lion (vs. actual lion)
\b. paper tiger (vs. actual tiger)
\c. paper airplane
\d. sand castle
\d. wooden nutmeg
%  \item type combinatorics (Asher)

\citet[301]{Asher:2011} notes that these constructions support
different inferences, cf. \Next, his (11.3).

\ex. \label{ex:asher_2011_p302_ex11.3}
\a. A \textbf{stone lion} is not a lion (a real lion), but it looks like one.
\b. A \textbf{stone jar} is a jar. \label{ex:asher_2011_p302_ex11.3b}
\c. ?A \textbf{paper airplane} is an airplane. \label{ex:asher_2011_p302_ex11.3c}

\pagebreak[4]
Crucially, \emph{stone} in \Last[a] does not allow the typical inference
pattern known from intersective and subjective adjectives, while the
very same noun in \Last[b] allows the standard interference pattern
expected for intersective and subsective modification. 
The question mark for \Last[c] is explained by Asher as follows:
\begin{quotation}
I am not sure whether a paper airplane is an airplane. If one thinks of
airplanes as having certain necessary parts like an engine or on board
means of locomotion, then most paper airplanes aren't airplanes. On
the other hand, many people tell me that their intuitions go the other
way.\\\citep[302]{Asher:2011}. 
\end{quotation}
Continuing his exploration of
\emph{paper plane}, he furthermore notes that it apparently gives rise
to a similar bridging inference as \emph{airplane}, cf. \Next, his
(11.4).

\ex. \label{ex:asher_2011_p302_ex11.4}
\a. John closed the door to the \textbf{airplane}. The engine started smoothly.
\b. John made a \textbf{paper airplane} in class. The engine started smoothly.

Here, according to Asher, \emph{paper} behaves more like an intersective
modifier. This raises the question whether in the respective
combinations the modifier itself is also having an effect. Asher then
presents a formal analysis in which it is in fact the modifier that
 changes the typing of the head noun by specifying the ``matter of
the satisfier of the noun'' \citep[304]{Asher:2011}. In addition,
Asher assumes that the matter which may constitute an object is also
specified in his type compositional logic. In this way, \emph{stone
  jar} will, without type conflict, be interpreted as a jar made out
of stone, allowing the inference in
\ref{ex:asher_2011_p302_ex11.3b}. In contrast, the combination of
\emph{paper} and \emph{plane} will lead to a type conflict, and the
corresponding inference is not available, explaining the question mark
on \ref{ex:asher_2011_p302_ex11.3c}. That \emph{paper airplane} is
nevertheless interpretable is due to reinterpretation processes that
are available for ``predications that don't literally work"
\citet[305]{Asher:2011}. How these kind of predications are to be
handled in his framework is discussed by him in his section on
\emph{Loose Talk}, cf. \citet[305--309]{Asher:2011}. The general idea
is that ``[l]oose talk relies on a set of distinctive and contingent
characteristics associated with the typical satisfier of a predicate"
\citep[308]{Asher:2011}. Whether a predicate P applies loosely or not
then depends on whether the relevant object is closer to the elements
that fall under the predicate P than it is to other relevant
alternatives to P, cf.  \citet[308]{Asher:2011} for a formal spell-out
of the relevant conditions (however, also note that the idea of loose
talk is again based on there being a clear notion of 
literal meaning, something that itself is very questionable,
cf. the previous discussions of literal meaning in Chapter
\ref{cha:theo}, Section \ref{sec:literality}).\is{literal
  meaning!{loose talk}}
\is{noun!{material}|)}\is{modification!{with material modifier}|)}

What Asher shows us, then, is one way to spell out in detail how
semantic characteristics inherent to compound constituents can be made
to work in yielding an appropriate compositional meaning of a
compound. Note that these semantic characteristics are very similar to
Fanselow's stereotype relations.

\subsection{Approaches using underspecification}
\label{sec:other_solutions}
% \begin{itemize}
% \item Levinson
% \item radical underspecification (Blutner)
% \end{itemize}
\is{underspecification!{analysis of complex nominals}|(}
What all the approaches discussed so far have in common is the general
aim of providing a semantic analysis of complex nominals which in all
cases also involved considerable parts that are independent of world
knowledge and therefore truly semantic in nature. However, one can also
find accounts where, apart from
the acknowledgement that some
interpretation exists between 2 compound constituents,
the burden of interpretation is placed squarely on
the pragmatic apparatus.\is{pragmatics}
% \footnote{In a way, placing the burden of
%   interpretation somewhere is already a good thing; note that one also finds accounts
%   that are content with statements to the effect that most compound
%   interpretations are simply of no linguistic interest, cf.,
%   e.g. \citet[25]{Selkirk:1982} who writes: "I would argue that it is
%   a mistake to attempt to characterize the grammar of the semantics of
%   nonverbal compounds in any way. [\dots] The only compounds whose
%   interpretation appears to be of linguistic interest, in the strict
%   sense, are the verbal compound, \dots"} 
\citet[45--46]{Bauer:1979}, for example, argues that there is
just one abstract `pro-verb' that needs to be deleted in the
generation of compounds, with the meaning of this proverb being
something like `there is a connection between'
\citep[46]{Bauer:1979}. Thus, in order to arrive at at the actual
interpretation of a compound, pragmatic knowledge is always needed.
Similar points are made by \citet[23]{Selkirk:1982} and
\citet[49]{Lieber:2004}, both explicitly
addressing non-deverbal compounds.  
% A similar
% sentiment is uttered in \citet[53]{Lieber:2004}, who "would argue that
% everything that goes on in arriving at a semantic interpretation of a
% root compound except for its referential properties and the semantic
% property of headedness involves context and encyclopedic
% knowledge".\footnote{Root compounds are simply those compounds whose
%   second stem is not derived from a verbal root,
%   cf. \citet[46]{Lieber:2004}.} 

However, nothing is said about how the pragmatic apparatus would come
to an interpretation. \citet{Levinson:2000} makes the following
proposal: ``Nominal compounds in English, and in many languages, have
an unmarked N-N form. Assuming that the semantic relation between the
nouns is no more than an existentially quantified variable over
relations, the exact relation must be inferred"
\citep[147]{Levinson:2000}. To infer the exact relation,
conversational implicatures are used. The general idea becomes clear
when looking at \Next, cf. (47) \citet[117]{Levinson:2000}, where the
symbol ``$+\!>$'' is used to mark conversational implicatures.

\ex.  noun-noun compounds (NN-relations)\footnotemark \\
   \emph{The oil compressor gauge.}\\
   $+\!>$ 'The gauge that measures the state of the compressor that
       compresses the oil.' 

\footnotetext{\citealt[117]{Levinson:2000} here refers to
  \citealt{Hobbsetal:1993}, cf. comments on that paper below.}
In Levinson's system, the inference here is done via an I-implicature,
that is, via an implicature to the most specific interpretation
possible. An ``I-induced interpretation, [\dots], is usually to a
rich relationship between the nouns, as fits most plausibly with
stereotypical assumptions'' \citep[147]{Levinson:2000}. 

\citealt{Hobbsetal:1993}, who are credited by
\citet[117]{Levinson:2000} with bringing the resolution of NN relations
into pragmatics, start with the same observation, stating that ``[t]o
resolve the reference of the noun phrase `lube-oil alarm', we need to
find to entities \emph{o} and \emph{a} with the appropriate
properties. The entity \emph{o} must be lube oil, \emph{a} must be an
alarm, and there must be an implicit relation between
them" \citep{Hobbsetal:1993}. This implicit relation is treated as a
predicate variable by them, following \citet{Downing:1977} in assuming
that any relation is possible here. Interestingly, in their actual
implementation they treat the relation as a predicate constant,
encoding the most common possible relations, i.e., the Levi-relations,
in axioms. 
\enlargethispage{1\baselineskip}
This, in turn, is just one small aspect of the general
system of weighted abduction they introduce in that paper (see also
\citealt{Blutner:1998} for another abduction-based approach to lexical
semantics). 
\is{underspecification!{analysis of complex nominals}|)}


%  TODO:
% Gleitman and Gleitman, 1970; Phrase and Paraphrase: Some Innovative
% Uses of Language. Norton, New York.
% nicht gefunden
% \citet{Selkirk:1982}
% \citet{Lieber:2004}
% Bauer, 1979; Selkirk, 1982; Lieber, 2004


% \subsubsection{Leftovers/TODO}

% \begin{itemize}
% \item Add Meyer/compound interpretation in context and isolation
% \end{itemize}

\section{Conclusion}
\label{sec:sem-conclusion}

This chapter gave an overview of semantic analyses for compounds,
including a discussion of approaches that either are originally focused
on phrasal structures, like the set-theoretic approaches from formal
semantics, or include both traditional compounds as well as a subset
of phrasal constructions, like Levi's approach. The following points
can be singled out as the most important ones:
\begin{compactenum}
\item There is no clear-cut difference between compounds and phrasal
  constructions in terms of the semantic analyses they can be
  subjected to. Some compounds can fruitfully be analyzed by using
  the set-theoretic classification originally developed for adjective
  noun combinations in formal semantics.
\item Compounds can in many cases be successfully classified using a
  relatively small number of categories, comprised of semantic
  relations and different nominalization patterns.
\item These classifications, originally meant to constitute semantic
  analyses, are not able to predict the final meanings of compounds,
  but seem to represent useful generalizations.
\item Stereotypes and analogies associated with specific concepts play
  a huge role in eventually arriving at a compound's correct interpretation.
\item Constituent specific information might be internally represented
  in different ways; proposals range from conceptual to semantic
  information to full underspecification. 
\end{compactenum}
% I will come back to most of these issues when discussing the semantic
% annotation scheme used for a large compound database presented in
% chapter \ref{cha:empirical-2}, section \ref{sec:methodsSemCoding}.


% \section[Conspectus]{Conspectus: proposals for phrases and compounds: continuity or not?}
% \label{sec:conspectus}

% \textbf{GET RID OF THIS SECTION!!!}

% If we consider the data and the different approaches to the data discussed in
% this section, I think we can single out three points as justifying an approach
% that includes not only noun noun constructions, but also A N
% constructions. These three points are
% \begin{enumerate}
% \item a single meaning can sometimes be described by either an adjective noun
%   or a noun noun combination (cf. e.g. \emph{atom/atomic bomb} or
%   \emph{bronzener Löwe/Bronzelöwe})
% \item noun noun combinations can have an intersective semantics, often seen as
%   a hallmark of adjective noun combinations (cf. e.g. \emph{silk shirt} or
%   \emph{stone lion})
% \item adjective noun combinations can have clearly non-intersective semantics
%   whose correct analysis seems to require the same ingredients as are needed
%   for an analysis of noun noun constructions (cf. \emph{presidential failure}
%   or \emph{fast highway})
% \end{enumerate}


%%% Local Variables: 
%%% mode: latex
%%% TeX-master: "habil-master_rev-1"
%%% End: 
