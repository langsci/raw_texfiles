% \chapter{Wordnet sense coding}
% \section{Wordnet sense coding}
\section{N1 families}
\label{sec:n1-wordnet-senses}
%  [1] acid        agony       application balance     bank        blame      
%  [7] brass       brick       call        car         case        cash       
% [13] chain       cheat       china       climate     cloud       cocktail   
% [19] couch       crash       credit      crocodile   cutting     diamond    
% [25] end         engine      eye         face        fashion     fine       
% [31] firing      flea        front       game        gold        graduate   
% [37] grandfather graveyard   gravy       ground      guilt       head       
% [43] health      human       interest    ivory       kangaroo    law        
% [49] lip         lotus       mailing     melting     memory      monkey     
% [55] nest        night       number      panda       parking     pecking    
% [61] polo        public      radio       rat         research    rocket     
% [67] rock        role        rush        sacred      search      shrinking  
% [73] silver      sitting     smoking     snail       snake       speed      
% [79] spelling    spinning    swan        swimming    think       video      
% [85] web         zebra      

\subsection{acid (22 compound types)}
Rated compound: \emph{acid test}

\vspace*{1ex}

\noindent
\begin{longtable}{c>{\raggedright\arraybackslash}p{5.5cm}rc>{\raggedright\arraybackslash}p{2cm}}\lsptoprule
{\small wnSense}&WordNet description&types&class&example\\\midrule
{1}&{any of various water-soluble compounds having a sour taste and capable of turning litmus red and reacting with a base to form a salt)}&{17}&{n}&{acid alkali}\\\tablevspace 
2&street name for lysergic acid diethylamide&2&n&acid experience \\\tablevspace
3&harsh or corrosive in tone&2&adj&acid tongue\\\tablevspace
4& being sour to the taste&1&adj&acid tang\\\lspbottomrule
\end{longtable}
% \vspace*{1em}

\noindent
Notes:\\
Senses 3 and 4 are adjective synsets 1 and 2 in WordNet.
% Keine Weiteren Anmerkungen in coding notes KWA
% - wnSense3 adjective: harsh or corrosive in tone
% - did use wnSense4 for acid tang

\subsection{agony (3 compound types)}
Rated compound: \emph{agony aunt}

\vspace*{1ex}

\noindent
\begin{longtable}{c>{\raggedright\arraybackslash}p{5cm}rc>{\raggedright\arraybackslash}p{2cm}}\lsptoprule
{\small wnSense}&WordNet description&types&class&example\\\midrule
{1}&{intense feelings of suffering; acute mental or physical pain}&{3}&n&agony column\\\lspbottomrule
\end{longtable}
% vspace*{1em}

\noindent
Notes:\\ No distinction was made between sense 1 and the second WordNet sense (a state of acute pain).

%KWA
% ** DONE agony-n1 
% -both WordNet senses indistinguishable

\subsection{application (42 compound types)}
Rated compound: \emph{application form}

% Noun

    % S: (n) application, practical application (the act of bringing something to bear; using it for a particular purpose) "he advocated the application of statistics to the problem"; "a novel application of electronics to medical diagnosis"
    % S: (n) application (a verbal or written request for assistance or employment or admission to a school) "December 31 is the deadline for applications"
    % S: (n) application, coating, covering (the work of applying something) "the doctor prescribed a topical application of iodine"; "a complete bleach requires several applications"; "the surface was ready for a coating of paint"
    % S: (n) application, application program, applications programme (a program that gives a computer instructions that provide the user with tools to accomplish a task) "he has tried several different word processing applications"
    % S: (n) lotion, application (liquid preparation having a soothing or antiseptic or medicinal action when applied to the skin) "a lotion for dry skin"
    % S: (n) application, diligence (a diligent effort) "it is a job requiring serious application"
    % S: (n) application (the action of putting something into operation) "the application of maximum thrust"; "massage has far-reaching medical applications"; "the application of indexes to tables of data"
\vspace*{1ex}

\noindent
\begin{longtable}{c>{\raggedright\arraybackslash}p{5cm}rc>{\raggedright\arraybackslash}p{2cm}}\lsptoprule
{\small wnSense}&WordNet description&types&class&example\\\midrule
1& (the act of bringing something to bear; using it for a particular purpose)&1&n&application area\\\tablevspace
{2}&{a verbal or written request for assistance or employment or admission to a school}&{20}&{n}&application document\\\tablevspace
3& (the work of applying something)&1&n&application technique\\\tablevspace
4&(a program that gives a computer instructions that provide the user with tools to accomplish a task)&20&n&application server\\\lspbottomrule
\end{longtable}
% vspace*{1em}

\noindent
Notes:\\
\emph{Application list} occurs in the BNC with 2 meanings (chance discovery), bound to 2 different WordNet senses (in the coding, WordNet sense 2 and relation \textsc{for} is used):
\ex. \a.  BNN 221 	Much praised at the Evian conference for a bold offer to absorb up to 100,000 refugees, the Dominican Republic had closed its application list at 2000.
\b. JXG 19 	Use the cursor keys to select "BASIC" from the application list. 

% Reddy item is application form
% - applications-X included + consolidated  
% - application year [BNC: in the application year ending 31 March 1980,] classified as \textsc{for}, but a bit strange, almost more like a name
% - application language in BNC is for computer applications, but could also be real applications
% - application list in BNC with 2 meanings (chance discovery) bound to 2 different wnSenses [used wnSense2 \textsc{for}]:  
%  BNN 221 	Much praised at the Evian conference for a bold offer to absorb up to 100,000 refugees, the Dominican Republic had closed its application list at 2000.
%  JXG 19 	Use the cursor keys to select "BASIC" from the application list. 
%  - most with computer meanings

% KWA

\pagebreak[4]
\subsection{balance (12 compound types)}
Rated compound: \emph{balance sheet}

% Reddy: balance sheet
% 1        1      balance_beam    1
% 2        1 balance_condition    2
% 3        1 balance_mechanism    1
% 4        1     balance_point    1
% 5        1      balance_tank    1
% 6        1    balance_theory    1
% 7       10  balance_engineer    1
% 8       11     balance_wheel    1
% 9        2    balance_change    1
% 10       2   balance_problem    1
% 11       2     balance_sheet    2
% 12       9     balance_value    1
\vspace*{1ex}

\noindent
\begin{longtable}{c>{\raggedright\arraybackslash}p{5cm}rc>{\raggedright\arraybackslash}p{2cm}}\lsptoprule
{\small wnSense}&WordNet description&types&class&example\\\midrule
1& a state of equilibrium&6&n&balance beam\\\tablevspace
{2}&{equality between the totals of the credit and debit sides of an account}&{3}&{n}&balance sheet\\\tablevspace
9&(mathematics) an attribute of a shape or relation; exact reflection of form on opposite sides of a dividing line or plane) &1&n&balance value\\\tablevspace
10&a weight that balances another weight&1&n&balance engineer\\\tablevspace
11&(a wheel that regulates the rate of movement in a machine; especially a wheel oscillating against the hairspring of a timepiece to regulate its beat)&10&n&balance wheel\\\lspbottomrule
\end{longtable}
% vspace*{1em}

\noindent
Notes:\\ WordNet sense 11 contains the type: balance wheel, balance (a wheel that regulates the rate of movement in a machine; especially a wheel oscillating against the hairspring of a timepiece to regulate its beat). Group is small but contains 3 single occurrence WordNet senses. 
% - small group with some single specialized wnSenses

\noindent
Mistakes:\\
\emph{Balance problem} is coded as WordNet sense 2, while in the BNC
it only occurs in the context of guitar design and music recording, corresponding to
WordNet sense 1 or 10, or, for the latter, 9. \emph{Balance change}
does not occur as a compound in the BNC.
\subsection{bank (61 compound types)}
Rated compound: \emph{bank account}
\vspace*{-.2cm}

% 1    1    5
% 2   10    1
% 3    2   52
% 4    4    2
% 5    5    1

% \vspace*{.5ex}

\noindent
\begin{longtable}{c>{\raggedright\arraybackslash}p{5.5cm}rc>{\raggedright\arraybackslash}p{1.5cm}}\lsptoprule
{\small wnSense}&WordNet description&types&class&example\\\midrule
1&sloping land (especially the slope beside a body of water)&5&n&bank barn\\\tablevspace
2&a financial institution that accepts deposits and channels the money into lending activities&52&n&bank job\\\tablevspace
4&an arrangement of similar objects in a row or in tiers&2&n&bank switch\\\tablevspace
5&a supply or stock held in reserve for future use (especially in emergencies)&1&n&bank nurse\\\tablevspace
10&a flight maneuver; aircraft tips laterally about its longitudinal axis (especially in turning)&1&n&bank angle\\\lspbottomrule
\end{longtable}
\vspace*{-.2cm}

\noindent
Notes:\\
Within WordNet sense 2, people working in a bank were either classified with the \textsc{have2} or the \textsc{in} relation (cf. \emph{bank boss} and \emph{bank chief} vs. \emph{bank clerk} and \emph{bank cashier}).

\vspace*{-.2cm}
\subsection{blame (1 compound types)}
Rated compound: \emph{blame game}
\vspace*{-.2cm}

\noindent
\begin{longtable}{c>{\raggedright\arraybackslash}p{5cm}rc>{\raggedright\arraybackslash}p{2cm}}\lsptoprule
{\small wnSense}&WordNet description&types&class&example\\\midrule
1&an accusation that you are responsible for some lapse or misdeed&1&n&blame game\\\lspbottomrule
\end{longtable}
\vspace*{-.2cm}


\noindent
Notes:\\\enlargethispage{1\baselineskip}There is only one second WordNet sense \emph{a reproach for some
  lapse or misdeed}; the compound meaning can also be construed from this meaning.

\subsection{brass (49 compound types)}
Rated compound: \emph{brass ring}

\vspace*{1ex}

\noindent
\begin{longtable}{c>{\raggedright\arraybackslash}p{5cm}rc>{\raggedright\arraybackslash}p{2cm}}\lsptoprule
{\small wnSense}&WordNet description&types&class&example\\\midrule
1&an alloy of copper and zinc)&43&n&brass foundry\\\tablevspace
2&a wind instrument that consists of a brass tube (usually of variable length) that is blown by means of a cup-shaped or funnel-shaped mouthpiece)&5&n&brass ensemble\\\tablevspace
3&the persons (or committees or departments etc.) who make up a body for the purpose of administering something)&1&n&brass hat\\\lspbottomrule
\end{longtable}
% \vspace*{1em}

% \noindent
% ** brass-n1
% most items wnSense1, even brass nerve/neck, where the single wnSenses do not fit for the whole word
% nb: in any event, brass neck also has just the literal meaning

\subsection{brick (26 compound types)}
Rated compound: \emph{brick wall}

\vspace*{1ex}

\noindent
\begin{longtable}{c>{\raggedright\arraybackslash}p{5cm}rc>{\raggedright\arraybackslash}p{2cm}}\lsptoprule
{\small wnSense}&WordNet description&types&class&example\\\midrule
1&rectangular block of clay baked by the sun or in a kiln; used as a building or paving material&26&n&brick terrace\\\lspbottomrule
\end{longtable}
% vspace*{1em}

\noindent
Notes:\\
\emph{Brick yard} occurs in the BNC in both the \textsc{for} and the \textsc{make2} construal, cf. \Next[a] and \Next[b]. Note that \emph{brickyard} comes with its own WordNet entry,`S: (n) brickyard, brickfield (a place where bricks are made and sold)', and occurs in the OED as a sub-entry of \emph{brick} n$^1$ with the meaning ` brickyard  n. a place where bricks are made, a brickfield.'
% ** brick-n1 
% brick yard in BNC either \textsc{make2} or \textsc{for}
\enlargethispage{1\baselineskip}
\ex. \a. B0A 1380 Boulton and Watt beam engines pumped out water at both ends, and a brick yard was set up to make the bricks near the site — seven million were used.
\b. CA0 47 Outside they admired a pink brick yard for twenty ponies, which looked like three sides of a Battenberg cake, and an indoor school, completely walled with bullet-proof mirrors. 

\subsection{call (17 compound types)}
Rated compound: \emph{call centre}

% 1    1    9
% 2   10    2
% 3   13    4
% 4    7    1
% 5    9    1

\vspace*{1ex}

\noindent
\begin{longtable}{c>{\raggedright\arraybackslash}p{5cm}rc>{\raggedright\arraybackslash}p{2cm}}\lsptoprule
{\small wnSense}&WordNet description&types&class&example\\\midrule
1&a telephone connection&9&n&call log\\\tablevspace
7&a demand by a broker that a customer deposit enough to bring his margin up to the minimum requirement&1&n&call loan\\\tablevspace
9&a request&1&n&call button\\\tablevspace
10&an instruction that interrupts the program being executed&2&n&call instruction\\\tablevspace
13&the option to buy a given stock (or stock index or commodity future) at a given price before a given date&4&n&call option\\\lspbottomrule
\end{longtable}
% \vspace*{1em}

% \noindent
% \newpage
\subsection{car (109 compound types)}
Rated compound: \emph{car park}

\vspace*{1ex}

\noindent
\begin{longtable}{c>{\raggedright\arraybackslash}p{5cm}rc>{\raggedright\arraybackslash}p{2cm}}\lsptoprule
{\small wnSense}&WordNet description&types&class&example\\\midrule
1&a motor vehicle with four wheels; usually propelled by an internal combustion engine&108&n& car commercial\\\tablevspace
2&a wheeled vehicle adapted to the rails of railroad&1&n&car flat\\\lspbottomrule
\end{longtable}
% vspace*{1em}

\noindent
Notes:\\ Alternative construal for \emph{car flat} compound is `flat railroad car
for cars'. This construal better fits the actual BNC context, while
the coded construal corresponds to \emph{flat car} or \emph{flat
  wagon}. 
% ** car-n1
% - new wnSense6: Car pn in Car dyke
% This does not occur in the final selection!
\subsection{case (37 compound types)}
Rated compound: \emph{case study}

% 1   11    4
% 2   17    2
% 3    3   11
% 4    5    2
% 5    6   18
\vspace*{.0ex}

\noindent
\begin{longtable}{c>{\raggedright\arraybackslash}p{5cm}rc>{\raggedright\arraybackslash}p{2cm}}\lsptoprule
{\small wnSense}&WordNet description&types&class&example\\\midrule
3&a comprehensive term for any proceeding in a court of law whereby an
individual seeks a legal remedy&11&n&case file\\\tablevspace
5&a portable container for carrying several objects)&2&n&case lid\\\tablevspace
6&a person requiring professional services&18&n&case conference\\\tablevspace
11&nouns or pronouns or adjectives (often marked by inflection)
related in some way to other words in a sentence&4&n&case ending\\\tablevspace
17&the enclosing frame around a door or window opening&2&n&case base\\\lspbottomrule
\end{longtable}
% vspace*{1em}

\noindent
Notes:\\
Sometimes strings occurred as compounds and as non-compounds in the BNC, e.g. \emph{case slots}. Due to the lexical ambiguity of \emph{slots} (plural form of the noun or singular form of the verb), there are compound occurrences, cf. \Next[a], as well as non-compound occurrences, cf. \Next[b].  
\ex. \a. EES 521 Typically, the program would look at the words from left to right, and test whether each word in the sentence was a likely candidate for the case slots of the main verb.\pagebreak[4]
\b. HAC 5089 Fitting a drive to the 2000 series can be tricky because of the way the case slots together and the need to use a 3.5" drive and mount. 

While \Last contains an identical string, \Next illustrates a
structural ambiguity that, at least in the BNC, is tied to the number
of the first noun. When the first noun is in the singular, the string occurs as a compound, cf. \Next[a]. When the first noun is in the plural, the nouns in the string are part of 2 different phrases, cf. \Next[b].  
% , or \NNext, with a
% Similar: \emph{case restrictions}
% - similar: case restrictions: in  some cases restrictions placed
% - and mishits that where either N N but not compound or N V: case benefits
\ex. \a. EES 482 For a verb such as’ collide ’, all that is specified by the case restrictions is that the object case can be any inanimate entity. 
\b. K94 2424 In some cases restrictions placed against imports may take the form of complex (and unnecessarily prohibitive) safety or packaging regulations. 


\subsection{cash (78 compound types)}
Rated compound: \emph{cash cow}

\vspace*{1ex}

\noindent
\begin{longtable}{c>{\raggedright\arraybackslash}p{5cm}rc>{\raggedright\arraybackslash}p{2cm}}\lsptoprule
{\small wnSense}&WordNet description&types&class&example\\\midrule
1&money in the form of bills or coins&77&n&cash point\\\tablevspace
3&United States country music singer and songwriter (1932-2003)&1&n&cash brother\\\lspbottomrule
\end{longtable}
% vspace*{1em}

\noindent
Notes:\\ WordNet sense 3 used for any personal name usage of \emph{cash}
(it is not the singer's name in the BNC context). 

\pagebreak[4]
\subsection{chain  (31 compound types)}      
Rated compound: \emph{chain reaction}

% 1 1    7
% 2 2    9
% 3 3   13
% 4 4    1
% 5 9    1
\vspace*{1ex}

\noindent
\begin{longtable}{c>{\raggedright\arraybackslash}p{5cm}rc>{\raggedright\arraybackslash}p{2cm}}\lsptoprule
{\small wnSense}&WordNet description&types&class&example\\\midrule
1&a series of things depending on each other as if linked together&7&n&chain letter\\\tablevspace
2&(chemistry) a series of linked atoms (generally in an organic molecule)&9&n&chain configuration\\\tablevspace
3&a series of (usually metal) rings or links fitted into one another to make a flexible ligament&13&n&chain brake\\\tablevspace
4&(business) a number of similar establishments (stores or restaurants or banks or hotels or theaters) under one ownership&1&n&chain store\\\tablevspace
9&a linked or connected series of objects&1&n&chain stitch\\\lspbottomrule
\end{longtable}
% ** chain-n1-final-BrSgIdentifier-2015-clean
\noindent
Notes:\\ \emph{Chain reaction} is coded as \textsc{verb}, modeled on \emph{chain
  smoker}. The latter, though, did not make it into the final selection.
% - considered changing "chain reaction" from \textsc{verb} to BE, following the
%   analysis in the reaction file; however, left it as is, because of
%   some similarity to chain smoker, which is clearly not BE  


\subsection{cheat  (2 compound types)}      
Rated compound: \emph{cheat sheet}

\vspace*{1ex}

\noindent
\begin{longtable}{c>{\raggedright\arraybackslash}p{5cm}rc>{\raggedright\arraybackslash}p{2cm}}\lsptoprule
{\small wnSense}&WordNet description&types&class&example\\\midrule
5&a deception for profit to yourself&2&n&cheat mode\\\lspbottomrule
\end{longtable}
\pagebreak[3]
\subsection{china        (27 compound types)}
Rated compound: \emph{china clay}

\vspace*{1ex}

\noindent
\begin{longtable}{c>{\raggedright\arraybackslash}p{5cm}rc>{\raggedright\arraybackslash}p{2cm}}\lsptoprule
{\small wnSense}&WordNet description&types&class&example\\\midrule
1&a communist nation that covers a vast territory in eastern Asia; the most populous country in the world&17&n&china lobby\\\tablevspace
2&high quality porcelain originally made only in China&10&n&china plate\\\lspbottomrule
\end{longtable}
% vspace*{1em}

\noindent
Notes:\\ WordNet sense 4 \emph{dishware made of high quality porcelain}
can alternatively be used for some items, e.g. \emph{china bowl} or
\emph{china plate}.


\subsection{climate      (11 compound types)}
Rated compound: \emph{climate change}

\vspace*{1ex}

\noindent
\begin{longtable}{c>{\raggedright\arraybackslash}p{5cm}rc>{\raggedright\arraybackslash}p{2cm}}\lsptoprule
{\small wnSense}&WordNet description&types&class&example\\\midrule
1&the weather in some location averaged over some long period of time&11&n&climate model\\\lspbottomrule
\end{longtable}
% ** climate-n1-final-BrSgIdentifier-2015-clean
% - changed climate shift/change from \textsc{verb} to \textsc{in}; this is in conformity
%   with the coding of climated change in the change family and gives a
%   better analysis for the climate family, where now climate
%   controller and these 2 items are distinguished

\subsection{cloud        (20 compound types)}
Rated compound: \emph{cloud nine}

\vspace*{1ex}

\noindent
\begin{longtable}{c>{\raggedright\arraybackslash}p{5cm}rc>{\raggedright\arraybackslash}p{2cm}}\lsptoprule
{\small wnSense}&WordNet description&types&class&example\\\midrule
2&a visible mass of water or ice particles suspended at a considerable
altitude&20&n&cloud bank\\\lspbottomrule
\end{longtable}
% ** cloud-n1
% -cloud nine not in data

\subsection{cocktail    (11 compound types)}
Rated compound: \emph{cocktail dress}


\vspace*{1ex}

\noindent
\begin{longtable}{c>{\raggedright\arraybackslash}p{5cm}rc>{\raggedright\arraybackslash}p{2cm}}\lsptoprule
{\small wnSense}&WordNet description&types&class&example\\\midrule
1&a short mixed drink&11&n&cocktail lounge\\\lspbottomrule
\end{longtable}

\subsection{couch        (2 compound types)}
Rated compound: \emph{couch potato}


\vspace*{1ex}

\noindent
\begin{longtable}{c>{\raggedright\arraybackslash}p{5cm}rc>{\raggedright\arraybackslash}p{2cm}}\lsptoprule
{\small wnSense}&WordNet description&types&class&example\\\midrule
1&an upholstered seat for more than one person&2&n&couch grass\\\lspbottomrule
\end{longtable}

\subsection{crash       (19 compound types)}
Rated compound: \emph{crash course}

% 1 18    1
% 2 19    2
% 3  2   15
% 4  3    1

\vspace*{1ex}

\noindent
\begin{longtable}{c>{\raggedright\arraybackslash}p{5cm}rc>{\raggedright\arraybackslash}p{2cm}}\lsptoprule
{\small wnSense}&WordNet description&types&class&example\\\midrule
2&a serious accident (usually involving one or more
vehicles)&15&n&crash barrier\\\tablevspace
3&a sudden large decline of business or the prices of stocks (especially one that causes additional failures)&1&n&crash period\\\tablevspace
18&sleep in a convenient place&1&n&crash pad\\\tablevspace
19&{}[very fast, rapid]&2&{}[adj]&crash dive\\\lspbottomrule
\end{longtable}
% \vspace*{1em}

\noindent
Notes:\\ WordNet sense 19 added to cover the almost adjectival usage
('very fast, rapid').
% ** crash-n1 
%  - new wnSense19 for crash programme/crash course

\subsection{credit       (63 compound types)}
Rated compound: \emph{credit card}

%      x freq
% 1    1    1
% 2    2   60
% 3    8    1
% 4    9    1




\vspace*{1ex}

\noindent
\begin{longtable}{c>{\raggedright\arraybackslash}p{5cm}rc>{\raggedright\arraybackslash}p{2cm}}\lsptoprule
{\small wnSense}&WordNet description&types&class&example\\\midrule
1&approval&1&n&credit side\\\tablevspace
2&money available for a client to borrow&60&n&credit agreement\\\tablevspace
8&an entry on a list of persons who contributed to a film or written work&1&n&credit sequence\\\tablevspace
9&an estimate, based on previous dealings, of a person's or an organization's ability to fulfill their financial commitments&1&n&credit rating\\\lspbottomrule
\end{longtable}
% vspace*{1em}

\noindent
Notes:\\WordNet sense 9 contains coded compound in its entry: `credit rating,
credit (an estimate, based on previous dealings, of a person's or an
organization's ability to fulfill their financial
commitments)'. Since it was impossible to clearly distinguish WordNet sense 2
and 5 (`arrangement for deferred payment for goods and services'), only
WordNet sense 2 was used.
% ** credit-n1
% - unable to clearly distinguish wnSense2 and 5, collapsed into
%   wnSense 2:
%  - 2: (money available for a client to borrow) 
%  - 5:  deferred payment (arrangement for deferred payment for goods and services) 

\subsection{crocodile    (5 compound types)}
Rated compound: \emph{crocodile tears}


\vspace*{1ex}

\noindent
\begin{longtable}{c>{\raggedright\arraybackslash}p{5cm}rc>{\raggedright\arraybackslash}p{2cm}}\lsptoprule
{\small wnSense}&WordNet description&types&class&example\\\midrule
1&large voracious aquatic reptile having a long snout with massive jaws and sharp teeth and a body covered with bony plates; of sluggish tropical waters&1&n&crocodile farm\\\lspbottomrule
\end{longtable}

% ** crocodile-n1
% - coded as BE; there is only one wnSense;
% - general problem: our shift rating is not in sync with the WordNet
%   senses, which sometimes do, sometimes don't contain shifts

\subsection{cutting      (34 compound types)}
Rated compound: \emph{cutting edge}

% 1    1    2
% 2    3   26
% 3    5    2
% 4    9    1
% 5   A1    2
% 6   A3    1
\begin{longtable}{c>{\raggedright\arraybackslash}p{5cm}rc>{\raggedright\arraybackslash}p{2cm}}\lsptoprule
{\small wnSense}&WordNet description&types&class&example\\\midrule
1&the activity of selecting the scenes to be shown and putting them together to create a film&2&n&cutting room\\\tablevspace
3&the act of cutting something into parts&26&n&cutting knife\\\tablevspace
5&an excerpt cut from a newspaper or magazine&2&n&cutting book\\\tablevspace
9&{}[An open, trench-like excavation through a piece of ground that rises above the level of a canal, railway, or road which has to be taken across it.]&1&n&cutting side\\\tablevspace
A1&(of speech) harsh or hurtful in tone or character&1&adj&cutting disdain\\\tablevspace
A3&painful as if caused by a sharp instrument&1&adj&cutting wind\\\lspbottomrule
\end{longtable}
% \vspace*{1em}

\noindent
Notes:\\
WordNet sense 9 manually added. The description is a verbatim quote from the OED (cutting, n. 8.).

% ** cutting-n1
%  - cutting ice discarded because gerund
%  - new wnSense A4 for cutting edge/end
%  - in the end, got rid of A4 sense, because I could not see a reason not to use \textsc{for},3. Also, if coded as A4, it would be the noun-compound interpretation
% - cutting edge highly dominant token: use in coding?

\pagebreak[4]
\subsection{diamond     (20 compound types)}
Rated compound: \emph{diamond wedding}
\vspace*{1ex}

\noindent

\begin{longtable}{c>{\raggedright\arraybackslash}p{5cm}rc>{\raggedright\arraybackslash}p{2cm}}\lsptoprule
{\small wnSense}&WordNet description&types&class&example\\\midrule
1&a transparent piece of diamond that has been cut and polished and is valued as a precious gem&17&n&diamond industry\\\tablevspace
3&a parallelogram with four equal sides; an oblique-angled equilateral parallelogram&2&n&diamond logo\\\tablevspace
4&a playing card in the minor suit that has one or more red rhombuses on it&1&n&diamond lead\\\lspbottomrule
\end{longtable}
% ** diamond-n1
% -diamond anniversary as BE, wnSense1

\subsection{end          (86 compound types)}
Rated compound: \emph{end user}

% 1    1   49
% 2    2   16
% 3    3   16
% 4    4    5
\vspace*{.5ex}

\noindent
\begin{longtable}{c>{\raggedright\arraybackslash}p{5cm}rc>{\raggedright\arraybackslash}p{2cm}}\lsptoprule
{\small wnSense}&WordNet description&types&class&example\\\midrule
1&either extremity of something that has length&49&n&end tent\\\tablevspace
2&the point in time at which something ends&16&n&end product\\\tablevspace
3&the final stage or concluding parts of an event or occurrence)&16&n&end game\\\tablevspace
4&the state of affairs that a plan is intended to achieve and that (when achieved) terminates behavior intended to achieve it&5&n& end objective\\\lspbottomrule
\end{longtable}
% \vspace*{1em}
\enlargethispage{1\baselineskip}
\noindent
Notes:\\
Very many mishits due to \emph{in the end} and similar constructions (e.g.,
\emph{end families} resolves to \emph{In the end families \dots}). For
the first 3 WordNet senses, the decision between \textsc{be}/\textsc{in} proofed
difficult. Decided to go with \textsc{be} for wnSense 1, while for WordNet
sense 2 \textsc{in} was used (except 1 usage of \textsc{be} for \emph{end date}). For
WordNet sense 3, both were used (cf. \textsc{be} for \emph{end section} 
vs. \textsc{in} for \emph{end game}). Due to both factors, coding this family
took forever and almost every string was manually checked in the BNC. 
% end wall
% ** end-n1
% - again, many wrong hits due to in the end etc; e.g. end families in
%   the end families or West End families
% - difficult to decide \textsc{be}/\textsc{in} for the 3 first WordNet sense
%   [spatial/temporal point/temporal final stage]: 
%    - decided to go with mainly \textsc{be} for wnSense1/spatial: end wall;
%      except end fitting/cap \textsc{for}
%    - wnSense2 point in time: end value etc./wnSense2 \textsc{in} (the value is
%      not the end, but the end wall; \textsc{be} for end date
%    - wnSense3: final period: \textsc{be} end stage, \textsc{in} end titles/game
%    - wnSense4: end objective/purpose etc.
% - only 81 left of 219
% - took forever to do; virtually looked up almost every word

\subsection{engine       (54 compound types)}
Rated compound: \emph{engine room}
\vspace*{1ex}

\noindent

\begin{longtable}{c>{\raggedright\arraybackslash}p{5cm}rc>{\raggedright\arraybackslash}p{2cm}}\lsptoprule
{\small wnSense}&WordNet description&types&class&example\\\midrule
1&motor that converts thermal energy to mechanical work&52&n&engine mounting\\\tablevspace
3&a wheeled vehicle consisting of a self-propelled engine that is used to draw trains along railway tracks&2&n&engine driver\\\lspbottomrule
\end{longtable}
% \vspace*{1em}

\noindent
Notes: \\Very clear differentiation between \textsc{have2}/\textsc{for}, cf. e.g. \emph{engine
  size} vs. \emph{engine oil}.
% ** engine-n1
%  -clear \textsc{have2}/\textsc{for} difference: engine size vs. engine oil

\subsection{eye          (43 compound types)}
Rated compound: \emph{eye candy}

\vspace*{1ex}

\noindent
\begin{longtable}{c>{\raggedright\arraybackslash}p{5cm}rc>{\raggedright\arraybackslash}p{2cm}}\lsptoprule
{\small wnSense}&WordNet description&types&class&example\\\midrule
1&the organ of sight&43&n&eye patch\\\lspbottomrule
\end{longtable}
% \vspace*{1em}

\noindent
Notes:\\
\emph{Eye witness} coded as ABOUT, because all the other relations did
not fit, and neither \textsc{verb}
  nor \textsc{idiom} were really appropriate either.

% ** eye-n1
% - eye witness as ABOUT, because all the others did not fit, and \textsc{verb}
%   or \textsc{idiom} also not really appropriate
\pagebreak[4]
\subsection{face         (20 compound types)}
Rated compound: \emph{face value}

\vspace*{1ex}

\noindent
\begin{longtable}{c>{\raggedright\arraybackslash}p{5cm}rc>{\raggedright\arraybackslash}p{2cm}}\lsptoprule
{\small wnSense}&WordNet description&types&class&example\\\midrule
1&the front of the human head from the forehead to the chin and ear to ear&11&n&face cream\\\tablevspace
8&the side upon which the use of a thing depends (usually the most prominent surface of an object)&9&n&face sheet\\\lspbottomrule
\end{longtable}
% NOTE: both face pace and face vowel are not in our final data
% ** face-n1
% - face pace: coded as ABOUT: 
%   1	 A0E 560 	The face pace of CHANCER was
%   another leading ingredient moving the story along briskly in the
%   best of popular drama traditions. 
%   Context: discussion of the apparently popular TV-series CHANCER
% - face vowel: added new wnSense 14 for face as the word 'face'

\subsection{fashion      (45 compound types)}
Rated compound: \emph{fashion plate}

\vspace*{1ex}

\noindent
\begin{longtable}{c>{\raggedright\arraybackslash}p{5cm}rc>{\raggedright\arraybackslash}p{2cm}}\lsptoprule
{\small wnSense}&WordNet description&types&class&example\\\midrule
3&the latest and most admired style in clothes and cosmetics and behavior&45&n&fashion trend\\\lspbottomrule
\end{longtable}
\noindent
Notes:\\
Only WordNet senses 3 and 4 occur in our data. However, because 4
(`consumer goods (especially clothing) in the current mode') and 3
were 
mostly indistinguishable in the compounds, they were collapsed into 3.
% from 3
% ** fashion-n1
% - only wnSense 3-4 relevant, but collapsed because mostly
%   indistinguishable [e.g. fashion advertisement]

\subsection{fine        (1 compound types)}
Rated compound: \emph{fine line}

\vspace*{1ex}

\noindent
\begin{longtable}{c>{\raggedright\arraybackslash}p{5cm}rc>{\raggedright\arraybackslash}p{2cm}}\lsptoprule
{\small wnSense}&WordNet description&types&class&example\\\midrule
1&minutely precise especially in differences in meaning&1&adj&fine line\\\lspbottomrule
\end{longtable}

\subsection{firing       ( compound types)}
Rated compound: \emph{firing line}

\vspace*{1ex}

\noindent
\begin{longtable}{c>{\raggedright\arraybackslash}p{5cm}rc>{\raggedright\arraybackslash}p{2cm}}\lsptoprule
{\small wnSense}&WordNet description&types&class&example\\\midrule
1&the act of firing weapons or artillery at an enemy&14&n&firing pattern\\\tablevspace
3&the act of setting something on fire&4&n&firing process\\\lspbottomrule
\end{longtable}

\subsection{flea         (6 compound types)}
Rated compound: \emph{flea market}

\vspace*{1ex}

\noindent
\begin{longtable}{c>{\raggedright\arraybackslash}p{5cm}rc>{\raggedright\arraybackslash}p{2cm}}\lsptoprule
{\small wnSense}&WordNet description&types&class&example\\\midrule
1&any wingless bloodsucking parasitic insect noted for ability to leap&6&n&flea bite\\\lspbottomrule
\end{longtable}

\subsection{front        (5 compound types)}
Rated compound: \emph{front runner}

\vspace*{1ex}

\noindent
\begin{longtable}{c>{\raggedright\arraybackslash}p{5cm}rc>{\raggedright\arraybackslash}p{2cm}}\lsptoprule
{\small wnSense}&WordNet description&types&class&example\\\midrule
1&the side that is forward or prominent&5&n&front entrance\\\lspbottomrule
\end{longtable}

\newpage
\subsection{game         (60 compound types)}
Rated compound: \emph{game plan}

% 1    1   17
% 2   12    1
% 3    2    5
% 4    3   13
% 5    4   21
% 6    8    1
% 7    9    2
\vspace*{1ex}

\noindent
\begin{longtable}{c>{\raggedright\arraybackslash}p{5cm}rc>{\raggedright\arraybackslash}p{2cm}}\lsptoprule
{\small wnSense}&WordNet description&types&class&example\\\midrule
1&a contest with rules to determine a winner&17&n&game theory\\\tablevspace
2&a single play of a sport or other contest&5&n&game teacher\\\tablevspace
3&an amusement or pastime&13&n&game experience\\\tablevspace
4&animal hunted for food or sport&21&n&game bird\\\tablevspace
8&a secret scheme to do something (especially something underhand or
illegal))&1&n&game plan\\\tablevspace
9&the game equipment needed in order to play a particular
game&1&n&game fair\\\tablevspace
12&{}[proper name]&1&n&game stock\\\lspbottomrule
\end{longtable}
% vspace*{1em}

\noindent
Notes:\\ 
WordNet sense 12 manually added (shares by a company named Game). 
Used WordNet Sense 1 as super-set in cases of doubt; WordNet senses 1-3 not
  always clearly distinguishable; WordNet Sense 2 used for physical
  education/games involving sports in school etc.\\
Mistakes:\\
The coding for \emph{game stock} is wrong, as the BNC context makes it clear that WordNet sense 4 is meant. This makes the added sense 12 superfluous.
% ** game-n1 
% - Merged the plural n1 for game-n1-consolidated and
%   game-n1-FamAna
%  - game leg with new wnSense 12 
%  - wnSense 1 used as superset in cases of doubt; wnSense 1-3 not
%   always clearly distinguishable, wnSense 2 used for physical
%   education for games involving sports in school etc.
%  - quite a few game=hunting thingies
% - new wnSense game12 for company named game

\newpage
\subsection{gold         (102 compound types)}
Rated compound: \emph{gold mine}

% 1    1    2
% 2    2   38
% 3    3   54
% 4    5    5
% 5    6    3
\vspace*{1ex}

\noindent
\begin{longtable}{c>{\raggedright\arraybackslash}p{5cm}rc>{\raggedright\arraybackslash}p{2cm}}\lsptoprule
{\small wnSense}&WordNet description&types&class&example\\\midrule
1&coins made of gold&2&n&gold stater\\\tablevspace
2&a deep yellow color&38&n&gold embroidery\\\tablevspace
 3& a soft yellow malleable ductile (trivalent and univalent) metallic
element; occurs mainly as nuggets in rocks and alluvial deposits; does
not react with most chemicals but is attacked by chlorine and aqua
regia& 54& n& gold mine\\\tablevspace
5&something likened to the metal in brightness or preciousness or
superiority etc.&5&n&gold card\\\tablevspace
6&{}[gold medal/medallist]&3&n&gold winner\\\lspbottomrule
\end{longtable}
%\vspace*{1em}

\noindent
Notes:\\ Introduced WordNet Sense 6 for gold medal/medallist, in analogy to the
WordNet entry for \emph{silver}.

% ** gold-n1
% - wnSense 5 is shifted, however, relations to not change (e.g. gold
% coast would be \textsc{be} in any case)
% - wnSense 1 \textsc{be}, otherwise \textsc{make2} (see in contrast silverware and silver dollar,
% also \textsc{be})
% - introduced wnSense 6 for gold medal/medallist in analogy to silver

\subsection{graduate    (9 compound types)}
Rated compound: \emph{graduate student}


\vspace*{1ex}

\noindent
\begin{longtable}{c>{\raggedright\arraybackslash}p{5cm}rc>{\raggedright\arraybackslash}p{2cm}}\lsptoprule
{\small wnSense}&WordNet description&types&class&example\\\midrule
1&a person who has received a degree from a school (high school or college or university)&9&n& graduate association\\\lspbottomrule
\end{longtable}

% ** graduate-n1 
% - merged graduates association

\subsection{grandfather  (1 compound types)}
Rated compound: \emph{grandfather clock}


\vspace*{1ex}

\noindent
\begin{longtable}{c>{\raggedright\arraybackslash}p{5cm}rc>{\raggedright\arraybackslash}p{2cm}}\lsptoprule
{\small wnSense}&WordNet description&types&class&example\\\midrule
1&the father of your father or mother&1&n&grandfather clock\\\lspbottomrule
\end{longtable}

\subsection{graveyard (1 compound types)}
Rated compound: \emph{graveyard shift}

\vspace*{1ex}

\noindent
\begin{longtable}{c>{\raggedright\arraybackslash}p{5cm}rc>{\raggedright\arraybackslash}p{2cm}}\lsptoprule
{\small wnSense}&WordNet description&types&class&example\\\midrule
1&a tract of land used for burials&1&n&graveyard shift\\\lspbottomrule
\end{longtable}

\subsection{gravy        (4 compound types)}
Rated compound: \emph{gravy train}.

% \vspace*{1ex}

\noindent
\begin{longtable}{c>{\raggedright\arraybackslash}p{5cm}rc>{\raggedright\arraybackslash}p{2cm}}\lsptoprule
{\small wnSense}&WordNet description&types&class&example\\\midrule
1&a sauce made by adding stock, flour, or other ingredients to the juice and fat that drips from cooking meats&4&n&gravy bowl\\\lspbottomrule
\end{longtable}
% \vspace*{1em}

\noindent
Notes:\\
% \enlargethispage{1\baselineskip}
Used WordNet sense 1 for \emph{gravy} in \emph{gravy train}, with the
relation \textsc{idiom}. Note, though, that WordNet gives WordNet sense 3, `a
sudden happening that brings good fortune (as a sudden opportunity to
make money)', which might actually be related to its usage in the
idiom. The OED establishes an explicit link between a somehow related
sense and \emph{gravy train}: `d. Money easily acquired; an unearned or
unexpected bonus; a tip. Hence to ride (board) the gravy train (or
boat), to obtain easy financial success. slang (orig. U.S.).'

% ** gravy-n1
% - gravy train coded as \textsc{idiom}, because from riding the gravy train and
%   alternative etymologies unclear BUT: according to WordNet, there is
%   a sense of gravy "S: (n) boom, bonanza, gold rush, gravy, godsend,
%   manna from heaven, windfall, bunce (a sudden happening that brings
%   good fortune (as a sudden opportunity to make money)) "the demand
%   for testing has created a boom for those unregulated laboratories
%   where boxes of specimen jars are processed like an assembly line"

\subsection{ground       (84 compound types)}
Rated compound: \emph{ground floor}.

% 1    1   71
% 2    4   11
% 3    5    1
% 4    9    1
\vspace*{1ex}

\noindent
\begin{longtable}{c>{\raggedright\arraybackslash}p{5cm}rc>{\raggedright\arraybackslash}p{2cm}}\lsptoprule
{\small wnSense}&WordNet description&types&class&example\\\midrule
1&the solid part of the earth's surface&71&n&ground game\\\tablevspace
4&a relation that provides the foundation for something&11&n&ground rule\\\tablevspace
5&a position to be won or defended in battle (or as if in
battle)&1&n&ground threat\\\tablevspace
9&a connection between an electrical device and a large conducting
body, such as the earth (which is taken to be at zero
voltage)&1&n&ground plane\\\lspbottomrule
\end{longtable}
% ** ground-n1 
% - somehow quite time consuming to do
% - full alignment wnSense 4 and \textsc{be}, 
% \vspace*{1em}

\noindent
Notes:\\
Full alignment of WordNet sense 4 and \textsc{be}.
\subsection{guilt        (5 compound types)}
Rated compound: \emph{guilt trip}.

\vspace*{1ex}

\noindent
\begin{longtable}{c>{\raggedright\arraybackslash}p{5cm}rc>{\raggedright\arraybackslash}p{2cm}}\lsptoprule
{\small wnSense}&WordNet description&types&class&example\\\midrule
{2}&{remorse caused by feeling responsible for some offense}&5&n&guilt complex\\\lspbottomrule
\end{longtable}
\newpage
\subsection{head        ( compound types)}
Rated compound: \emph{head teacher}.

%      x freq
% 1    1   36
% 2   21    1
% 3   24    1
% 4   27    1
% 5   34    2
% 6    4   24
% 7    7   19
% 8    9    2
% 9 <NA>   11
% \vspace*{1ex}

\noindent
\begin{longtable}{c>{\raggedright\arraybackslash}p{5cm}rc>{\raggedright\arraybackslash}p{2cm}}\lsptoprule
{\small wnSense}&WordNet description&types&class&example\\\midrule
1&the upper part of the human body or the front part of the body in animals; contains the face and brains&36&n&head gear\\\tablevspace
{4}&{a person who is in charge}&24&n&head servant\\\tablevspace
7&the top of something&19&n&head stone\\\tablevspace
9&(grammar) the word in a grammatical constituent that plays the same grammatical role as the whole constituent&2&n&head word\\\tablevspace
21&forward movement&1&n&head way\\\tablevspace
24&a line of text serving to indicate what the passage below it is about)&1&n&head line\\\tablevspace
27&(computer science) a tiny electromagnetic coil and metal pole used to write and read magnetic patterns on a disk&1&n&head arm\\\tablevspace
34&{}&2&n&head start\\\lspbottomrule
\end{longtable}
% \vspace*{1em}

\noindent
Notes:\\
% \enlargethispage{1\baselineskip}
WordNet sense 21 contains coded compound as synonym. WordNet sense 34
added for \emph{head start} and \emph{head wind}. This group was cumbersome to do, WordNet sense 4 was also used for non-humans (e.g. \emph{head quarter}). WordNet sense 7 for consistency always linked to \textsc{in} (except for \emph{head string}, coded as \textsc{be}). As the numbers show, some very small groups with very specific WordNet senses are included.
% % ** -head-n1 added wnSense34 ahead for head sea headstart
% -cumbersome to do, wnSense4 also for non-humans, wnSense7 always
%  linked to \textsc{in}, not \textsc{be}
% - includes some very peculier wnSenses with small groups [use proportional cut off point in analysis??]
% - head teacher not included because classified as adjective in BNC

\subsection{health       (112 compound types)}
Rated compound: \emph{health insurance}

% \vspace*{1ex}

\noindent
\begin{longtable}{c>{\raggedright\arraybackslash}p{5cm}rc>{\raggedright\arraybackslash}p{2cm}}\lsptoprule
{\small wnSense}&WordNet description&types&class&example\\\midrule
1&a healthy state of wellbeing free from disease&112&n&health sector\\\lspbottomrule
\end{longtable}
% \vspace*{1em}

\noindent
Notes:\\
No attempt was made to distinguish between the 2 available WordNet
senses (\emph{a healthy state of wellbeing free from disease}
vs. \emph{the general condition of body and mind}). This group
contained many ABOUT/\textsc{for} relations (e.g. \emph{health column}
and \emph{health authority}). While mainly unproblematic to code, some
could be coded either way, e.g. \emph{health education}.

% - many ABOUT/\textsc{for}, mainly unproblematic, but some overlap

\subsection{human   (1 compound types)}
Rated compound: \emph{human being}

% \vspace*{1ex}

\noindent
\begin{longtable}{c>{\raggedright\arraybackslash}p{5cm}rc>{\raggedright\arraybackslash}p{2cm}}\lsptoprule
{\small wnSense}&WordNet description&types&class&example\\\midrule
3&relating to a person&1&adj&human being\\\lspbottomrule
\end{longtable}
% \vspace*{1em}

\noindent
Notes:\\
WordNet sense 3 was chosen because the illustrating example fits
(\emph{``the experiment was conducted on 6 monkeys and 2 human
  subjects"}). The best fitting WordNet pointer is the one for the
noun-sense. As the whole constituent family contains just one member,
this decision does not matter here, anyways.
% ** human-n1
% - used only adjective WordNet senses

\pagebreak[4]
\subsection{interest     (25 compound types)}
Rated compound: \emph{interest rate}

% \vspace*{1ex}

\noindent
\begin{longtable}{c>{\raggedright\arraybackslash}p{5cm}rc>{\raggedright\arraybackslash}p{2cm}}\lsptoprule
{\small wnSense}&WordNet description&types&class&example\\\midrule
1&a sense of concern with and curiosity about someone or something&9&n&interest span\\\tablevspace
4&a fixed charge for borrowing money; usually a percentage of the amount borrowed&16&n&interest rate\\\lspbottomrule
\end{longtable}

\subsection{ivory        (11 compound types)}
Rated compound: \emph{ivory tower}

% \vspace*{1ex}

\noindent
\begin{longtable}{c>{\raggedright\arraybackslash}p{5cm}rc>{\raggedright\arraybackslash}p{2cm}}\lsptoprule
{\small wnSense}&WordNet description&types&class&example\\\midrule
1&a hard smooth ivory colored dentine that makes up most of the tusks of elephants and walruses&9&n&ivory carver\\\tablevspace
2&a shade of white the color of bleached bones&2&n&ivory wall\\\lspbottomrule
\end{longtable}

\subsection{kangaroo     (3 compound types)}
Rated compound: \emph{kangaroo court}

% \vspace*{1ex}

\noindent
\begin{longtable}{c>{\raggedright\arraybackslash}p{5cm}rc>{\raggedright\arraybackslash}p{2cm}}\lsptoprule
{\small wnSense}&WordNet description&types&class&example\\\midrule
1&any of several herbivorous leaping marsupials of Australia and New Guinea having large powerful hind legs and a long thick tail)&3&n&kangaroo skin\\\lspbottomrule
\end{longtable}

\pagebreak[3]
\subsection{law         (54 compound types)}
Rated compound: \emph{law firm}

% \vspace*{1ex}

\noindent
\begin{longtable}{c>{\raggedright\arraybackslash}p{5cm}rc>{\raggedright\arraybackslash}p{2cm}}\lsptoprule
{\small wnSense}&WordNet description&types&class&example\\\midrule
 1& the collection of rules imposed by
authority& 53& n& law student\\\tablevspace
4&a generalization that describes recurring facts or events in
nature&1&n&law table\\\lspbottomrule
\end{longtable}
% \vspace*{1em}

\noindent
Notes:\\
In the BNC, \emph{law table} refers to the biblical tables of
law. While the WordNet sense 4 does not match this exactly, it is
better than any alternative lest covering it with WordNet sense 1, too. WordNet sense 1 has \emph{jurisprudence}
as a collocate, and the coded compounds all refer to worldly law. 
% in the non-godgiven sense.   
No attempt was made to distinguish between WordNet sense 1 and other,
closely related senses, e.g. WordNet sense 6 (`the learned profession that is mastered by graduate study in a law school and that is responsible for the judicial system').
% ** law-n1 
% - hard to do, used just wnSense1 [so no distinction between e.g. wnSense1 law, jurisprudence (the collection of rules imposed by authority) and wnSense6 S: (n) law, practice of law ()] and once wnSense4, many discarded, because shortened from longer NPs
% - plurals: 4 laws X forms, merged from consolidated on, laws family
%   only in plural
% laws,change,470,0.078707118
% laws,committee,36,0.0060286303
% laws,family,49,0.0082056357
% laws,regulations,1449,0.24265237

\subsection{lip          (6 compound types)}
Rated compound: \emph{lip service}

% \vspace*{1ex}

\noindent
\begin{longtable}{c>{\raggedright\arraybackslash}p{5cm}rc>{\raggedright\arraybackslash}p{2cm}}\lsptoprule
{\small wnSense}&WordNet description&types&class&example\\\midrule
1&either of two fleshy folds of tissue that surround the mouth and play a role in speaking&6&n&lip mike\\\lspbottomrule
\end{longtable}

\pagebreak[4]
\subsection{lotus        (8 compound types)}
Rated compound: \emph{lotus position}

% \vspace*{1ex}

\noindent
\begin{longtable}{c>{\raggedright\arraybackslash}p{5cm}rc>{\raggedright\arraybackslash}p{2cm}}\lsptoprule
{\small wnSense}&WordNet description&types&class&example\\\midrule
1&native to eastern Asia; widely cultivated for its large pink or white flowers&6&n&lotus pond\\\tablevspace
4&{}[brand name]&2&n&lotus product\\\lspbottomrule
\end{longtable}
% \vspace*{1em}

\noindent
Notes:\\
Introduced WordNet sense 4 to cover the brand name usage in the BNC
(either for the racing team or for the software company). The car name
\emph{Lotus Elan} and the construction \emph{Lotus name} where both
classified as non-compounds. For \emph{Lotus Elan}, this parallels the
discussion of \emph{the opera `Carmen'} in \citet[447, section 14.2]{HuddlestonandPullum:2002}.
% ** lotus-n1
% - Lotus in bnc as name for software and name for car brand
% - introduced new wnSense 4 for software brand and car brand
% - treated car names lotus elan/esprit/cortina as appositions,
%   cf. appositive modifiers in HP 447ff
%   constructions; 

\subsection{mailing      (3 compound types)}
Rated compound: \emph{mailing list}

\vspace*{1ex}

\noindent
\begin{longtable}{c>{\raggedright\arraybackslash}p{5cm}rc>{\raggedright\arraybackslash}p{2cm}}\lsptoprule
{\small wnSense}&WordNet description&types&class&example\\\midrule
2&the transmission of a letter&3&n&mailing label\\\lspbottomrule
\end{longtable}

\subsection{melting      (10 compound types)}
Rated compound: \emph{melting pot}

\vspace*{1ex}

\noindent
\begin{longtable}{c>{\raggedright\arraybackslash}p{5cm}rc>{\raggedright\arraybackslash}p{2cm}}\lsptoprule
{\small wnSense}&WordNet description&types&class&example\\\midrule
1&the process whereby heat changes something from a solid to a liquid&10&n&melting point\\\lspbottomrule
\end{longtable}
% \vspace*{1em}

\noindent
Notes:\\
% \enlargethispage{1\baselineskip}
Some members of this group could have been classified with the
adjectival WordNet sense (\emph{melting
  frost/glacier/ice/wax}). This was not done, among other things
because the relational coding already singles out this group.
% ** melting-n1 
% - melting range/temperature/point as ABOUT vs. melting pot \textsc{for} melting season \textsc{have1}
\subsection{memory       (48 compound types)}
Rated compound: \emph{memory lane}

% 1    1    7
% 2    2   19
% 3    3    2
% 4    4   19
% 5    5    1
\vspace*{1ex}

\noindent
\begin{longtable}{c>{\raggedright\arraybackslash}p{5cm}rc>{\raggedright\arraybackslash}p{2cm}}\lsptoprule
{\small wnSense}&WordNet description&types&class&example\\\midrule
1&something that is remembered&7&n&memory trace\\\tablevspace
2&the cognitive processes whereby past experience is
remembered&19&n&memory span\\\tablevspace
3&the power of retaining and recalling past experience&2&n&memory drum\\\tablevspace
4&an electronic memory device&19&n&memory kernel\\\tablevspace
5&the area of cognitive psychology that studies memory
processes&1&n&memory department\\\lspbottomrule
\end{longtable}
% ** memory-n1 
% - computer sense: \textsc{have2} vs. \textsc{for} memory address vs. memory manager
% - rather subtle distinction between wnSense1 (sth. remembered) and wnSense2 (cognitive proccesses)

\subsection{monkey      (10 compound types)}
Rated compound: \emph{monkey business}

\vspace*{1ex}

\noindent
\begin{longtable}{c>{\raggedright\arraybackslash}p{5cm}rc>{\raggedright\arraybackslash}p{2cm}}\lsptoprule
{\small wnSense}&WordNet description&types&class&example\\\midrule
1&any of various long-tailed primates (excluding the prosimians)&10&n&monkey wrench\\\lspbottomrule
\end{longtable}
% ** monkey-n1
% - monkey wrench ?? new relation: \textsc{idiom}, monkey suit

\subsection{nest         (6 compound types)}
Rated compound: \emph{nest egg}

\vspace*{-1ex}

\noindent
\begin{longtable}{c>{\raggedright\arraybackslash}p{5cm}rc>{\raggedright\arraybackslash}p{2cm}}\lsptoprule
{\small wnSense}&WordNet description&types&class&example\\\midrule
1&a structure in which animals lay eggs or give birth to their young&6&n&nest area\\\lspbottomrule
\end{longtable}
\vspace*{-.75cm}
\subsection{night        (116 compound types)}
Rated compound: \emph{night owl}

\vspace*{-1ex}

\noindent
\begin{longtable}{c>{\raggedright\arraybackslash}p{5cm}rc>{\raggedright\arraybackslash}p{2cm}}\lsptoprule
{\small wnSense}&WordNet description&types&class&example\\\midrule
1&the time after sunset and before sunrise while it is dark outside&116&n&night visitor\\\lspbottomrule
\end{longtable}
% vspace*{1em}

\noindent
Notes:\\
Decided to only use WordNet sense 1, though WordNet provides a number
of further senses. However, these are semantically very close to
WordNet sense one and did not allow one any principled decisions
between the senses. Noted a number of combinations that serve as or a
contained in
titles (e.g., all 3 \emph{night kitchen} occurrences in the BNC are either from \emph{In the night
  kitchen}, the title of a book by Maurice Sendak, or the name of a
theater group he has).
% ** night-n1
% - used mainly only wnSense1, others hard to keep apart or here maybe
% really irrelevant, like he watched television every night
% - night soil as \textsc{in}, but really too lexicalized. 
% - night classes > night class
% - everything mainly \textsc{in} or \textsc{for}, but some exceptions:
% period\textsc{be},blindnessCAUSE2,spririt/goblin/creatureFROM,fall\textsc{have2} 
% night fall maybe too lexicalized? Again, typical example that truly
% obviously lexicalized ones are really very rare in the families
% - many combinations serve as titles, at first discarded, now
%   classified as compounds because titles are not proper names;
%   examples: night kitchen, part of In the night kitchen by Sendak

\vspace*{-.4cm}
\subsection{number       (24 compound types)}
Rated compound: \emph{number crunching}

\vspace*{-.5ex}

\noindent
\begin{longtable}{c>{\raggedright\arraybackslash}p{5cm}rc>{\raggedright\arraybackslash}p{2cm}}\lsptoprule
{\small wnSense}&WordNet description&types&class&example\\\midrule
2&the property possessed by a sum or total or indefinite quantity of units or individuals&20&n&number theory\\\tablevspace
5&a symbol used to represent a number&3&n&number pad\\\tablevspace
13&{}[(Bible) Book of Numbers]&1&n&numbers story\\\lspbottomrule
\end{longtable}
% vspace*{1em}

\noindent
Notes:\\
Added WordNet sense 13.
% ** number-n1
% - "numbers stories" [for stories in the book of numbers in the bible]: new  wnSense 13

\subsection{panda        (2 compound types)}
Rated compound: \emph{panda car}

\vspace*{1ex}

\noindent
\begin{longtable}{c>{\raggedright\arraybackslash}p{5cm}rc>{\raggedright\arraybackslash}p{2cm}}\lsptoprule
{\small wnSense}&WordNet description&types&class&example\\\midrule
1&large black-and-white herbivorous mammal of bamboo forests of China and Tibet; in some classifications considered a member of the bear family or of a separate family Ailuropodidae)&2&n&panda population\\\lspbottomrule
\end{longtable}

\subsection{parking      (23 compound types)}
Rated compound: \emph{parking lot}


\vspace*{1ex}

\noindent
\begin{longtable}{c>{\raggedright\arraybackslash}p{5cm}rc>{\raggedright\arraybackslash}p{2cm}}\lsptoprule
{\small wnSense}&WordNet description&types&class&example\\\tablevspace
1&{}[The placing or leaving of a vehicle or vehicles in a car park or
other designated area, at the side of a road, etc. Also: space reserved or used for the parking of motor vehicles (freq. with modifying word)]&23&n&parking bay\\\lspbottomrule
\end{longtable}
% vspace*{1em}

\noindent
Notes:\\
The 2 noun WordNet sense are both not sufficient (\emph{space in
  which vehicles can be parked} and \emph{the act of maneuvering a
  vehicle into a location where it can be left temporarily}). The
single sense for parking used instead is a verbatim copy of the OED entry parking 4.a.

 %  4.
% Thesaurus »
% Categories »
 
%  a. The placing or leaving of a vehicle or vehicles in a car park or other designated area, at the side of a road, etc. Also: space reserved or used for the parking of motor vehicles (freq. with modifying word). Also in extended use.
\subsection{pecking     (1 compound types)}
Rated compound: \emph{pecking order}

% \vspace*{1ex}

\noindent
\begin{longtable}{c>{\raggedright\arraybackslash}p{5cm}rc>{\raggedright\arraybackslash}p{2cm}}\lsptoprule
{\small wnSense}&WordNet description&types&class&example\\\midrule
1&{}[The action of striking or picking up with the beak; an instance
of this.]&1&n&pecking order\\\lspbottomrule
\end{longtable}
% vspace*{1em}

\noindent
Notes:\\
There is no WordNet entry for the noun \emph{pecking}. The description here
corresponds to OED pecking, n.$^1$, 1.

\subsection{polo         (10 compound types)}
Rated compound: \emph{polo shirt}

% \vspace*{1ex}

\noindent
\begin{longtable}{c>{\raggedright\arraybackslash}p{5cm}rc>{\raggedright\arraybackslash}p{2cm}}\lsptoprule
{\small wnSense}&WordNet description&types&class&example\\\midrule
2&a game similar to field hockey but played on horseback using long-handled mallets and a wooden ball&10&n&polo match\\\lspbottomrule
\end{longtable}
% vspace*{1em}

\noindent
Notes:\\
WordNet only contains one other sense for \emph{polo}, designating Marco Polo. \emph{Polo shirt} and \emph{polo neck} were both classified as \textsc{for} and no new sense was added. Note that the OED has a separate entry `polo n.$^1$ 3. b. A polo neck sweater or shirt; (also) a polo shirt.'
% ** polo-n1
% -wnSenses only Marco Polo and the game
% -polo mint included with new WordNetSense3 and \textsc{be}
% -polo shirt/neck as \textsc{for}


\subsection{public       (7 compound types)}
Rated compound: \emph{public service}

% \vspace*{1ex}

\noindent
\begin{longtable}{c>{\raggedright\arraybackslash}p{5cm}rc>{\raggedright\arraybackslash}p{2cm}}\lsptoprule
{\small wnSense}&WordNet description&types&class&example\\\midrule
1&people in general considered as a whole&7&n&public access\\\lspbottomrule
\end{longtable}
% ** public-n1
% - originally not so easy, many as wnSense 3 adjective + \textsc{be}, other not clear
% (public access, public advertising)
% - better bnc selection reduced this problem


\subsection{radio        (106 compound types)}
Rated compound: \emph{radio station}

\vspace*{1ex}

\noindent
\begin{longtable}{c>{\raggedright\arraybackslash}p{5cm}rc>{\raggedright\arraybackslash}p{2cm}}\lsptoprule
{\small wnSense}&WordNet description&types&class&example\\\midrule
1&medium for communication&27&n&radio telescope\\\tablevspace
2&an electronic receiver that detects and demodulates and amplifies transmitted signals&9&n&radio set\\\tablevspace
3&a communication system based on broadcasting electromagnetic waves&70&n&radio transmission\\\lspbottomrule
\end{longtable}
% vspace*{1em}

\noindent
Notes:\\
This group contained many instances that could in principle be
classified with a number of relations, e.g. \emph{radio news}:
news for the radio, news the radio has, news in the radio, news from
the radio, or \emph{radio preacher}, which allows all of these but
also \textsc{use} (preacher who uses the radio). Went by plausibility and
group consistency.
% ** radio-n1
% here quite often: \textsc{for}/\textsc{have2}/\textsc{in}/FROM, e.g. radio news news for the radio,
% news on the radio, or news the radio has, or news from the radio?? ->
% used intuition; radio preacher \textsc{for}/FROM/\textsc{use}/\textsc{in}/\textsc{have2} ?? (used \textsc{use}) 


\subsection{rat          (22 compound types)}
Rated compound: \emph{rat race}, \emph{rat run}

\vspace*{1ex}

\noindent
\begin{longtable}{c>{\raggedright\arraybackslash}p{5cm}rc>{\raggedright\arraybackslash}p{2cm}}\lsptoprule
{\small wnSense}&WordNet description&types&class&example\\\midrule
1&any of various long-tailed rodents similar to but larger than a
mouse&21&n&rat poison\\\tablevspace
6&{}[name of a nucleotide/amino-acid sequence]&1&n&RAT part\\\lspbottomrule
\end{longtable}
% vspace*{1em}

\pagebreak[4]
\noindent
Notes:\\
WordNet sense 6 added. Both \emph{rat race} and \emph{rat run} classed as \textsc{have2}, while in \citet{BellandSchaefer:2013} \emph{rat race} was NONE.
% % is this an acronym?
% ** rat-n1
% - rat-race and rat run both classed as \textsc{have2}, while rat race was NONE
% in our original classification
% - rat running as \textsc{verb}
% - rat bag as \textsc{make2}, although not even clear whether etymologically
% related to current meaning
% - rat in RAT part as \textsc{be} with new wnSense6

\subsection{research     (160 compound types)}
Rated compound: \emph{research project}

\vspace*{1ex}

\noindent
\begin{longtable}{c>{\raggedright\arraybackslash}p{5cm}rc>{\raggedright\arraybackslash}p{2cm}}\lsptoprule
{\small wnSense}&WordNet description&types&class&example\\\midrule
1&systematic investigation to establish facts&160&n&research training\\\lspbottomrule
\end{longtable}
% vspace*{1em}

\noindent
Notes:\\
The 2 noun WordNet senses in WordNet were deemed identical
(\emph{systematic investigation to establish facts} and \emph{a search
for knowledge}).
% ** research-n1
% - merged program/programme;organization/organisation
% - not able to see any difference between the 2 WordNet senses

\subsection{rocket      (14 compound types)}
Rated compound: \emph{rocket science}

\vspace*{1ex}

\noindent
\begin{longtable}{c>{\raggedright\arraybackslash}p{5cm}rc>{\raggedright\arraybackslash}p{2cm}}\lsptoprule
{\small wnSense}&WordNet description&types&class&example\\\midrule
1&any vehicle self-propelled by a rocket engine&14&n&rocket launcher\\\lspbottomrule
\end{longtable}
% vspace*{1em}

\noindent
Notes:\\
No attempt was made to distinguish the first 2 WordNet senses,
e.g. the one given above and the second one, \emph{a jet engine containing its own propellant and driven by reaction propulsion}.
% ** rocket-n1
% - only used wnSense1, did not distinguish 1 and 2 because distinction
%   not relevant (e.g., a rocket missile is not a vehicle and for the
%   other the \textsc{use} relation applied)
% S: (n) rocket, projectile (any vehicle self-propelled by a rocket engine)
% S: (n) rocket, rocket engine (a jet engine containing its own propellant and driven by reaction propulsion) 

\pagebreak[4]
\subsection{rock         (99 compound types)}
Rated compound: \emph{rock bottom}

% 1    1   11
% 2    2   45
% 3    6   43
\vspace*{1ex}

\noindent
\begin{longtable}{c>{\raggedright\arraybackslash}p{5cm}rc>{\raggedright\arraybackslash}p{2cm}}\lsptoprule
{\small wnSense}&WordNet description&types&class&example\\\midrule
1&a lump or mass of hard consolidated mineral matter&11&n&rock field\\\tablevspace
 2& material consisting of the aggregate of minerals like those making
up the Earth's crust& 45& n& rock arch\\\tablevspace
6&a genre of popular music originating in the 1950s; a blend of black rhythm-and-blues with white country-and-western&43&n&rock tour\\\lspbottomrule
\end{longtable}
% vspace*{1em}

\noindent
Notes:\\
\emph{Rock bottom} is coded as \textsc{have2}, while in
\citet{BellandSchaefer:2013} it was coded as \textsc{be}.
% - rock-bottom
% - coded has \textsc{have2},wnSense 1, but in M/S13: \textsc{be}

\subsection{role         (8 compound types)}
Rated compound: \emph{role model}
\vspace*{1ex}

\noindent
\begin{longtable}{c>{\raggedright\arraybackslash}p{5cm}rc>{\raggedright\arraybackslash}p{2cm}}\lsptoprule
{\small wnSense}&WordNet description&types&class&example\\\midrule
1&the actions and activities assigned to or required or expected of a
person or group&7&n&role reversal\\\tablevspace
2&an actor's portrayal of someone in a play&1&n&role play\\\lspbottomrule
\end{longtable}
% vspace*{1em}

\noindent
Notes:\\
This group contained many false hits, e.g. reduced relative clauses like CRS 1842 \emph{the role parents are allowed to play}.
% ** role-n1
% - very many false hits role parents play in bla

\subsection{rush         (5 compound types)}
Rated compound: \emph{rush hour}

\vspace*{1ex}

\noindent
\begin{longtable}{c>{\raggedright\arraybackslash}p{5cm}rc>{\raggedright\arraybackslash}p{2cm}}\lsptoprule
{\small wnSense}&WordNet description&types&class&example\\\midrule
{1}&{the act of moving hurriedly and in a careless manner}&{2}&{n}&{rush job}\\\tablevspace
3&grasslike plants growing in wet places and having cylindrical often
hollow stems&2&n&rush seat\\\tablevspace
4&{}[proper name]&1&n&Rush concert\\\lspbottomrule
\end{longtable}
% vspace*{1em}

\noindent
WordNet sense 4 used generically for the proper name \emph{Rush}
(instead of the WordNet proper name use for Benjamin Rush, 1745-1813).
% ** rush-n1
% - rush hour 6000 tokens against 300 against neglegible in a small family
% - Rush show/concert used wnSense4 for Rush the band instead of the physician

\subsection{sacred       (1 compound types)}
Rated compound: \emph{sacred cow}

\vspace*{1ex}

\noindent
\begin{longtable}{c>{\raggedright\arraybackslash}p{5cm}rc>{\raggedright\arraybackslash}p{2cm}}\lsptoprule
{\small wnSense}&WordNet description&types&class&example\\\midrule
1&made or declared or believed to be holy; devoted to a deity or some religious ceremony or use&1&adj&sacred cow\\\lspbottomrule
\end{longtable}

\pagebreak[4]
\subsection{search       (56 compound types)}
Rated compound: \emph{search engine}

\vspace*{1ex}

\noindent
\begin{longtable}{c>{\raggedright\arraybackslash}p{5cm}rc>{\raggedright\arraybackslash}p{2cm}}\lsptoprule
{\small wnSense}&WordNet description&types&class&example\\\midrule
1&the activity of looking thoroughly in order to find something or
someone&26&n&search consultant\\\tablevspace
3&an operation that determines whether one or more of a set of items
has a specified property&30&n&search window\\\lspbottomrule
\end{longtable}
% ** search-n1
% - only used wnSense 1 and 3: 
% S1: (n) search, hunt, hunting (the activity of looking thoroughly in order to find something or someone)
% S3: (n) search, lookup (an operation that determines whether one or
% more of a set of items has a specified property) "they wrote a program
% to do a table lookup"

\subsection{shrinking   (1 compound types)}
Rated compound: \emph{shrinking violet}

\vspace*{1ex}

\noindent
\begin{longtable}{c>{\raggedright\arraybackslash}p{5cm}rc>{\raggedright\arraybackslash}p{2cm}}\lsptoprule
{\small wnSense}&WordNet description&types&class&example\\\midrule
1&process or result of becoming less or smaller&1&n&shrinking violet\\\lspbottomrule
\end{longtable}


% \newpage 
\subsection{silver       (138 compound types)}
Rated compound: \emph{silver spoon}, \emph{silver screen},
\emph{silver bullet}

% 1 1   43
% 2 2    5
% 3 3   77
% 4 4   12
% 5 5    1

\vspace*{1ex}

\noindent
\begin{longtable}{c>{\raggedright\arraybackslash}p{5cm}rc>{\raggedright\arraybackslash}p{2cm}}\lsptoprule
{\small wnSense}&WordNet description&types&class&example\\\midrule
1&a soft white precious univalent metallic element having the highest
electrical and thermal conductivity of any metal; occurs in argentite
and in free form; used in coins and jewelry and tableware and
photography&43&n&silver bar\\\tablevspace
2&coins made of silver&5&n&silver dollar\\\tablevspace
3&a light shade of grey&77&n&silver mist\\\tablevspace
4&silverware eating utensils&12&n&silver plate\\\tablevspace
5&a trophy made of silver (or having the appearance of silver) that is
usually awarded for winning second place in a competition&1&n&silver medal\\\lspbottomrule
\end{longtable}
% vspace*{1em}

\noindent
Note:\\ 
WordNet sense 5 contains the compound type in its description:
\emph{silver medal, silver (a trophy made of silver (or having the
appearance of silver) that is usually awarded for winning second place
in a competition) 
}. Compounds with \emph{silver} with WordNet senses 4 and 2 also allow
WordNet sense 1.  WordNet sense 3 used for everything that was not
real silver.

% ** silver-n1-FamAna
%  - used word net senses, but S3 for everything non-real-silver,
%    esp. only colour related
%  - for silverware: changed coding to \textsc{be} based on S4, although \textsc{make2}
%    would be appropriate for S1, and also possible
% *** silver-n1-final-singBrUnderscore-2015
% - added wnSense10 for silver tone "They paused as they heard the silver tones of a bell" 

\subsection{sitting      (3 compound types)}
Rated compound: \emph{sitting duck}

\vspace*{1ex}

\noindent
\begin{longtable}{c>{\raggedright\arraybackslash}p{5cm}rc>{\raggedright\arraybackslash}p{2cm}}\lsptoprule
{\small wnSense}&WordNet description&types&class&example\\\midrule
3&the act of assuming or maintaining a seated position&3&n&sitting room\\\lspbottomrule
\end{longtable}

\subsection{smoking      (6 compound types)}
Rated compounds: \emph{smoking gun}, \emph{smoking jacket}

\vspace*{1ex}

\noindent
\begin{longtable}{c>{\raggedright\arraybackslash}p{5cm}rc>{\raggedright\arraybackslash}p{2cm}}\lsptoprule
{\small wnSense}&WordNet description&types&class&example\\\midrule
1&the act of smoking tobacco or other substances&5&n&smoking car\\\tablevspace
5&emitting smoke in great volume&1&adj&smoking gun\\\lspbottomrule
\end{longtable}

\subsection{snail        (3 compound types)}
Rated compound: \emph{snail mail}

% \vspace*{1ex}

\noindent
\begin{longtable}{c>{\raggedright\arraybackslash}p{5cm}rc>{\raggedright\arraybackslash}p{2cm}}\lsptoprule
{\small wnSense}&WordNet description&types&class&example\\\midrule
1&freshwater or marine or terrestrial gastropod mollusk usually having an external enclosing spiral shell&3&n&snail body\\\lspbottomrule
\end{longtable}

\subsection{snake        (13 compound types)}
Rated compound: \emph{snake oil}

% \vspace*{1ex}

\noindent
\begin{longtable}{c>{\raggedright\arraybackslash}p{5cm}rc>{\raggedright\arraybackslash}p{2cm}}\lsptoprule
{\small wnSense}&WordNet description&types&class&example\\\midrule
1&limbless scaly elongate reptile; some are venomous&8&n&snake charmer\\\tablevspace
5&something long, thin, and flexible that resembles a snake&5&n&snake cable\\\lspbottomrule
\end{longtable}
% vspace*{1em}

\noindent
Notes:\\
Perfect example of different WordNet senses capturing a metaphoric shift.
\subsection{speed (36 compound types)}
Rated compound: \emph{speed limit}

% \vspace*{1ex}

\noindent
\begin{longtable}{c>{\raggedright\arraybackslash}p{5cm}rc>{\raggedright\arraybackslash}p{2cm}}\lsptoprule
{\small wnSense}&WordNet description&types&class&example\\\midrule
1&distance travelled per unit time&36&n&speed indicator\\\lspbottomrule
\end{longtable}
% vspace*{1em}

\noindent
Notes:\\
No attempt was made to differentiate the coded WordNet sense from the 
following 2 (\emph{a rate (usually rapid) at which something
  happens} and \emph{changing location rapidly}).
% ** speed-n1
% - did not differentiate wnSense 1-3 (differences/readings unclear)

\subsection{spelling     (11 compound types)}
Rated compound: \emph{spelling bee}

\vspace*{1ex}

\noindent
\begin{longtable}{c>{\raggedright\arraybackslash}p{5cm}rc>{\raggedright\arraybackslash}p{2cm}}\lsptoprule
{\small wnSense}&WordNet description&types&class&example\\\midrule
1&forming words with letters according to the principles underlying
accepted usage&11&n&spelling mistake\\\lspbottomrule
\end{longtable}
\subsection{spinning     (1 compound types)}
Rated compound: \emph{spinning jenny}

\vspace*{1ex}

\noindent
\begin{longtable}{c>{\raggedright\arraybackslash}p{5cm}rc>{\raggedright\arraybackslash}p{2cm}}\lsptoprule
{\small wnSense}&WordNet description&types&class&example\\\midrule
1&creating thread&1&n&spinning jenny\\\lspbottomrule
\end{longtable}

\subsection{swan         (10 compound types)}
Rated compound: \emph{swan song}

\vspace*{1ex}

\noindent
\begin{longtable}{c>{\raggedright\arraybackslash}p{5cm}rc>{\raggedright\arraybackslash}p{2cm}}\lsptoprule
{\small wnSense}&WordNet description&types&class&example\\\midrule
1&stately heavy-bodied aquatic bird with very long neck and usually white plumage as adult&10&n&swan population\\\lspbottomrule
\end{longtable}

\subsection{swimming     (19 compound types)}
Rated compound: \emph{swimming pool}


\vspace*{1ex}

\noindent
\begin{longtable}{c>{\raggedright\arraybackslash}p{5cm}rc>{\raggedright\arraybackslash}p{2cm}}\lsptoprule
{\small wnSense}&WordNet description&types&class&example\\\midrule
1&the act of swimming&19&n&swimming cap\\\lspbottomrule
\end{longtable}
\subsection{think        (2 compound types)}
Rated compound: \emph{think tank}

\vspace*{1ex}

\noindent
\begin{longtable}{c>{\raggedright\arraybackslash}p{5cm}rc>{\raggedright\arraybackslash}p{2cm}}\lsptoprule
{\small wnSense}&WordNet description&types&class&example\\\midrule
1&an instance of deliberate thinking&2&n&think sign\\\lspbottomrule
\end{longtable}
% vspace*{1em}

\noindent
Notes:\\
All other WordNet senses of \emph{think} are verb-senses. Both
compounds (\emph{think tank} and \emph{think sign}) can also plausibly
be analyzed as VN compounds. 

\subsection{video       (99 compound types)}
Rated compound: \emph{video game}

% S: (n) video, picture () "they could still receive the sound but the picture was gone"
% S: (n) video recording, video ()
% S: (n) video ((computer science) the appearance of text and graphics on a video display)
% S: (n) television, telecasting, TV, video (broadcasting visual images of stationary or moving objects) "she is a star of screen and video"; "Television is a medium because it is neither rare nor well done" - Ernie Kovacs

\vspace*{1ex}

\noindent
\begin{longtable}{c>{\raggedright\arraybackslash}p{5cm}rc>{\raggedright\arraybackslash}p{2cm}}\lsptoprule
{\small wnSense}&WordNet description&types&class&example\\\midrule
1&the visible part of a television transmission&20&n&video cable\\\tablevspace
2&a recording of both the visual and audible components (especially one containing a recording of a movie or television program)&69&n&video department\\\tablevspace
3&(computer science) the appearance of text and graphics on a video display&10&n&video chip\\\lspbottomrule
\end{longtable}
% vspace*{1em}

\noindent
Notes:\\ 
WordNet sense 4 was not used, unclear in how far it could be
applied. Often, it was difficult to decide on the senses or multiple
usages were possible. In the latter case, the most plausible one was chosen.
% ** video-n1
% - did not use wnSense 4, quite unclear
% - often difficult to decide which sense to use OR multiple usages
% possible, went for most plausible

\subsection{web          (3 compound types)}
Rated compound: \emph{web site}

\vspace*{1ex}

\noindent
\begin{longtable}{c>{\raggedright\arraybackslash}p{5cm}rc>{\raggedright\arraybackslash}p{2cm}}\lsptoprule
{\small wnSense}&WordNet description&types&class&example\\\midrule
5&computer network consisting of a collection of internet sites that offer text and graphics and sound and animation resources through the hypertext transfer protocol&1&n&web site\\\tablevspace
6&a fabric (especially a fabric in the process of being woven&2&n&web width\\\lspbottomrule
\end{longtable}

\subsection{zebra        (3 compound types)}
Rated compound: \emph{zebra crossing}

\vspace*{1ex}

\noindent
\begin{longtable}{c>{\raggedright\arraybackslash}p{5cm}rc>{\raggedright\arraybackslash}p{2cm}}\lsptoprule
{\small wnSense}&WordNet description&types&class&example\\\midrule
1&any of several fleet black-and-white striped African equines&3&n&zebra tarantula\\\lspbottomrule
\end{longtable}


\pagebreak[4]
\section{N2 families}
\label{sec:n2-WordNet-senses}
%  [1] account   aunt      bee       being     bottom    bullet    business 
%  [8] candy     card      car       centre    change    clay      clock    
% [15] course    court     cow       crossing  crunching dress     duck     
% [22] edge      egg       engine    firm      floor     form      game     
% [29] gun       hour      insurance jacket    jenny     lane      limit    
% [36] line      list      lot       mail      market    mine      model    
% [43] nine      oil       order     owl       park      plan      plate    
% [50] pool      position  potato    pot       project   race      rate     
% [57] reaction  ring      room      run       runner    science   screen   
% [64] service   sheet     shift     shirt     site      song      spoon    
% [71] station   student   study     tank      teacher   tear      test     
% [78] tower     train     trip      user      value     violet    wall     
% [85] wedding  

% \vspace*{1ex}

% \noindent
% \begin{longtable}{c>{\raggedright\arraybackslash}p{5cm}rc>{\raggedright\arraybackslash}p{2cm}}\lsptoprule
% {\small wnSense}&WordNet description&types&class&example\\\midrule
% 1&&&n&\\\midrule
% 2&&&n&\\\lspbottomrule
% \end{longtable}

\subsection{account       (92 compound types)}
Rated compound: \emph{bank account}

\vspace*{1ex}

\noindent
\begin{longtable}{c>{\raggedright\arraybackslash}p{5cm}rc>{\raggedright\arraybackslash}p{2cm}}\lsptoprule
{\small wnSense}&WordNet description&types&class&example\\\midrule
1&a record or narrative description of past events&10&n&insider account\\\tablevspace
{3}&{a formal contractual relationship established to provide for regular banking or brokerage or business services}&{61}&{n}&{bank account}\\\tablevspace
4&a statement that makes something comprehensible by describing the relevant structure or operation or circumstances etc.&6&n&materialist account\\\tablevspace
7&a statement of recent transactions and the resulting balance&14&n&parish account\\\tablevspace
9&an itemized statement of money owed for goods shipped or services rendered&1&n&farm account\\\lspbottomrule
\end{longtable}
% vspace*{1em}

\noindent
Notes:\\
No differentiation between the first 2 WordNet senses (`history, account, chronicle, story (a record or narrative description of past events)' vs. `report, news report, story, account, write up (a short account of the news)'). Both were coded as WordNet sense 1.
% S: (n) history, account, chronicle, story (a record or narrative description of past events) "a history of France"; "he gave an inaccurate account of the plot to kill the president"; "the story of exposure to lead"
% S: (n) report, news report, story, account, write up (a short account of the news) "the report of his speech"; "the story was on the 11 o'clock news"; "the account of his speech that was given on the evening news made the governor furious"
WordNet sense 3 extended to cover computer related usages such as \emph{mail account} and \emph{vms account} (vms = the proprietary Virtual Memory System operating system).
% - wnSense 1/2 treated as 1, difference not clear to me
% - wnSense3 business relationship used for quite a range: bank account, mail
%   account, vms account
WordNet sense 3 comes with clear cases of \textsc{for} vs. \textsc{have2}, e.g. \emph{savings account} and \emph{Virgin account} (where virgin= the record company), but many cases could be either, e.g. \emph{government account} or \emph{client account}. Went with plausibility: \textsc{have2} if it is not per se or not likely per se a specific kind of account. Highly specialized types of accounts/specialist banking terms classified with \textsc{idiom} (e.g. \emph{trust account} or \emph{nostro account}).  
 
% - wnSense3 has at least \textsc{for} savings account and \textsc{have2} Virgin account [], but many cases a bit unclear, e.g. government
%   account treated as \textsc{have2} but could be \textsc{for}, customer account treated as \textsc{for}
%   (Logic: \textsc{have2} if it is not per se or not likely to be a specific kind of account)
% - wnSense3: \textsc{idiom} for specialized types of account/specialist banking terms


\subsection{aunt      (1 compound types)}
Rated compound: \emph{agony aunt}

\vspace*{1ex}

\noindent
\begin{longtable}{c>{\raggedright\arraybackslash}p{5cm}rc>{\raggedright\arraybackslash}p{2cm}}\lsptoprule
{\small wnSense}&WordNet description&types&class&example\\\midrule
1&intense feelings of suffering; acute mental or physical pain&1&n&agony aunt\\\lspbottomrule
\end{longtable}
\subsection{bee       (7 compound types)}
Rated compound: \emph{spelling bee}

% vspace*{1ex}

\noindent
\begin{longtable}{c>{\raggedright\arraybackslash}p{5cm}rc>{\raggedright\arraybackslash}p{2cm}}\lsptoprule
{\small wnSense}&WordNet description&types&class&example\\\midrule
1&any of numerous hairy-bodied insects including social and solitary species&6&n&bumble bee\\\tablevspace
2&a social gathering to carry out some communal task or to hold competitions&1&n&spelling bee\\\lspbottomrule
\end{longtable}
\subsection{being     (3 compound types)}
Rated compound: \emph{human being}

\vspace*{1ex}

\noindent
\begin{longtable}{c>{\raggedright\arraybackslash}p{5cm}rc>{\raggedright\arraybackslash}p{2cm}}\lsptoprule
{\small wnSense}&WordNet description&types&class&example\\\midrule
1&the state or fact of existing&1&n&well being\\\tablevspace
2&a living thing that has (or can develop) the ability to act or function independently&2&n&half being\\\lspbottomrule
\end{longtable}
\subsection{bottom    (19 compound types)}
Rated compound: \emph{rock bottom}
\vspace*{-.3cm}

\noindent
\begin{longtable}{c>{\raggedright\arraybackslash}p{5cm}rc>{\raggedright\arraybackslash}p{2cm}}\lsptoprule
{\small wnSense}&WordNet description&types&class&example\\\midrule
1&the lower side of anything&5&n&box bottom\\\tablevspace
2&the lowest part of anything&14&n&sea bottom\\\lspbottomrule
\end{longtable}
% vspace*{1em}
\vspace*{-0.3cm}

\noindent
Notes:\\
\emph{Rock bottom} coded as \textsc{be} in \citet{BellandSchaefer:2013}, here coded as \textsc{have2}.
% ** TODO bottom-n2-check
% - "rock bottom" in Bell/Schäfer 2013: \textsc{be}!!

\vspace*{-0.3cm}
\subsection{bullet    (5 compound types)}
Rated compound: \emph{silver bullet}

% \vspace*{1ex}

\noindent
\begin{longtable}{c>{\raggedright\arraybackslash}p{5cm}rc>{\raggedright\arraybackslash}p{2cm}}\lsptoprule
{\small wnSense}&WordNet description&types&class&example\\\midrule
1&a projectile that is fired from a gun&5&n&dumdum bullet\\\lspbottomrule
\end{longtable}
\vspace*{-0.8cm}
\subsection{business (262 compound types)}
Rated compound: \emph{monkey business}

% \vspace*{1ex}

\noindent
\begin{longtable}{c>{\raggedright\arraybackslash}p{5cm}rc>{\raggedright\arraybackslash}p{2cm}}\lsptoprule
{\small wnSense}&WordNet description&types&class&example\\\midrule
1&a commercial or industrial enterprise and the people who constitute it&255&n&textile business\\\tablevspace
2&the activity of providing goods and services involving financial and commercial and industrial aspects&2&n&repeat business\\\tablevspace
5&an immediate objective&3&n&pound business\\\tablevspace
7&business concerns collectively&2&n&winter business\\\lspbottomrule
\end{longtable}
% vspace*{1em}

\noindent
Notes:\\
WordNet sense 5 is illustrated with ``gossip was the main business of the evening". Assigning this sense relied in this case more on this quote, a more appropriate paraphrase of \emph{business} here would simply be `activity/matter'.
% ** business-n2
% - should probably be ABOUT, seeing Levi's examples, I probably should
%  have used it more [ABOUT = dealing with, pertaining to, be concerned with]
% - vast majority wnSense1 and ABOUT, but clearly defined subgroups

\subsection{candy     (4 compound types)}
Rated compound: \emph{eye candy}

\vspace*{1ex}

\noindent
\begin{longtable}{c>{\raggedright\arraybackslash}p{5cm}rc>{\raggedright\arraybackslash}p{2cm}}\lsptoprule
{\small wnSense}&WordNet description&types&class&example\\\midrule
1&a rich sweet made of flavored sugar and often combined with fruit or nuts&4&n&cotton candy\\\lspbottomrule
\end{longtable}


\subsection{card      (97 compound types)}
Rated compound: \emph{credit card}

\vspace*{1ex}

\noindent
\begin{longtable}{c>{\raggedright\arraybackslash}p{5cm}rc>{\raggedright\arraybackslash}p{2cm}}\lsptoprule
{\small wnSense}&WordNet description&types&class&example\\\midrule
1&one of a set of small pieces of stiff paper marked in various ways and used for playing games or for telling fortunes&16&n&tarot card\\\tablevspace
2&a card certifying the identity of the bearer&24&n&security card\\\tablevspace
3&a rectangular piece of stiff paper used to send messages (may have printed greetings or pictures)&9&n&valentine card\\\tablevspace
4&thin cardboard, usually rectangular&31&n&vaccination card\\\tablevspace
7&a printed or written greeting that is left to indicate that you have visited)&4&n&business card\\\tablevspace
8&(golf) a record of scores (as in golf)&2&n&golf card\\\tablevspace
9&a list of dishes available at a restaurant&1&n&menu card\\\tablevspace
11&a printed circuit that can be inserted into expansion slots in a computer to increase the computer's capabilities&10&n&video card\\\lspbottomrule
\end{longtable}
% - FROM for Barclay card etc.
% - wnSense7 for calendar/business cards etc.
% - wnSense10 also for fight card DOES NOT OCCUR \textsc{in} FINAL DATASET!
% -wnSense4 sometimes both \textsc{have1} and \textsc{for} plausible
% - some metaphoric usages: race, conspiracy card etc.

\subsection{car       (100 compound types)}
Rated compound: \emph{panda car}

\vspace*{1ex}

\noindent
\begin{longtable}{c>{\raggedright\arraybackslash}p{5cm}rc>{\raggedright\arraybackslash}p{2cm}}\lsptoprule
{\small wnSense}&WordNet description&types&class&example\\\midrule
1&a motor vehicle with four wheels; usually propelled by an internal combustion engine&80&n&dream car\\\tablevspace
2&a wheeled vehicle adapted to the rails of railroad&19&n&freight car\\\tablevspace
5&a conveyance for passengers or freight on a cable railway&1&n&cable car\\\lspbottomrule
\end{longtable}
% vspace*{1em}

\noindent
WordNet sense 5 has \emph{cable car} as collocate.
\subsection{centre    (268 compound types)}
Rated compound: \emph{call centre}

\vspace*{1ex}

\noindent
\begin{longtable}{c>{\raggedright\arraybackslash}p{5cm}rc>{\raggedright\arraybackslash}p{2cm}}\lsptoprule
{\small wnSense}&WordNet description&types&class&example\\\midrule
2&an area that is approximately central within some larger region&11&n&village centre\\\tablevspace
3&a point equidistant from the ends of a line or the extremities of a figure&10&n&wheel centre\\\tablevspace
4&a place where some particular activity is concentrated&19&n&growth centre\\\tablevspace
8&a cluster of nerve cells governing a specific bodily process)&3&n&pleasure centre\\\tablevspace
9&a building dedicated to a particular activity&225&n&trade centre\\\lspbottomrule
\end{longtable}
% vspace*{1em}

\noindent
Notes:\\
WordNet has 2 different entries for the UK/US spelling variants. Here, the \emph{centre} version is used. Note the distinction between WordNet sense 4 and 9, where without context often both senses are OK. That is, \emph{Africa centre} refers to a building (in the BNC, it is used for the Africa centre in London), whereas \emph{banking centre} usually refers to larger places like towns (e.g. London). However, depending on context, a construal with the respective other WordNet sense should be possible, too.
% ** TODO centre-n2
% - just noticed: different entries for center and centre, I used centre!
% - call centre manually added to centre-n2-final file
% *** FamAna
% - arts/art centre to art centre [arts centre far more frequent]
% - visitor/s centre to sg [most frequent]
% - communication/s to sg [sg more frequent]
% - sport/s to sg [pl far more frequent]
% - resource/s centre to sg [most frequent]
% - subtle distinction, e.g. UK centre wnSense 4 (a place where some part. activity is concentrated) \textsc{have2} (HTD 1545 	Edinburgh is the UK centre for animal breeding research.) vs. Africa centre as wnSense 9 (a building) and \textsc{for}
% - new wnSense 10 for class centre : rugby player in centre position
% - wnSense 9 used for buildings and sometimes smaller/more extended single
%   properties: fitness/gliding centre
% - field centre: centre IN field better than centre \textsc{for} field studies, which do
%   not have to be done somewhere in the countryside
% - decided not to use wnSense6 (instead 4)

\subsection{change    (126 compound types)}
Rated compound: \emph{climate change}

\vspace*{1ex}

\noindent
\begin{longtable}{c>{\raggedright\arraybackslash}p{5cm}rc>{\raggedright\arraybackslash}p{2cm}}\lsptoprule
{\small wnSense}&WordNet description&types&class&example\\\midrule
1&an event that occurs when something passes from one state or phase to another&99&n&proportion change\\\tablevspace
3&the action of changing something&26&n&name change\\\tablevspace
11&{}[gear shift]&&n&column change\\\lspbottomrule
\end{longtable}
% vspace*{1em}

\noindent
Notes:\\
Added sense 11. Coding of WordNet senses 1 and 3 was linked to whether something changed or was exchanged. In the former case, the coded relation is IN (\emph{script changes}: relation IN, WordNet sense 1), in the latter case, the relation is \textsc{verb} (\emph{tyre change}, WordNet sense 3 and relation \textsc{verb}). Because of the many non-compounds, this group required manual look-up of almost every combination.

% *** change-n2-FamAna
% - consolidated benefit/s changes to sg, the most frequent
% - very many wnSenses
% - word change,\textsc{verb},3, (no Levi relation fits, or does it?) "bnc one real compound: AC2 187 	Klepner played the usual corporate cat and mouse game of making at least one or 2 word changes on each and every page so that his own secretary could re-type each page under her initials and the initials ‘FK’. "
% - interesting: the US [change of policy]
% - \textsc{verb},3 for things that where exchanged [tyre change], IN,1 for changes in a thing that
%   stayed [script changes]; much like German verändern vs. austauschen [but:
%   Zahlen verändern vs ?austauschen] similar problematic: harmony change and a
%   few others, direction change, chord change; actually, quite a few work with
%   IN and with austauschen/wechseln; went with gut feeling in cases of doubt
% - linked \textsc{verb},3 and IN,1
% - pretty much looked up every combination
% - new wnSense 11 for column change: change = gear shift
% - very many non-compounds, especially N V in bnc
% - climate change by way most frequent

\subsection{clay      (4 compound types)}
Rated compound: \emph{china clay}

\vspace*{1ex}

\noindent
\begin{longtable}{c>{\raggedright\arraybackslash}p{5cm}rc>{\raggedright\arraybackslash}p{2cm}}\lsptoprule
{\small wnSense}&WordNet description&types&class&example\\\midrule
1&a very fine-grained soil that is plastic when moist but hard when fired&4&n&pipe clay\\\lspbottomrule
\end{longtable}
% ** TODO clay-n2
% - china clay as \textsc{for}

\subsection{clock    (37 compound types)}
Rated compound: \emph{grandfather clock}

\vspace*{1ex}

\noindent
\begin{longtable}{c>{\raggedright\arraybackslash}p{5cm}rc>{\raggedright\arraybackslash}p{2cm}}\lsptoprule
{\small wnSense}&WordNet description&types&class&example\\\midrule
1&a timepiece that shows the time of day&36&n&cuckoo clock\\\tablevspace
2&{}[A trivial name for the pappus of the dandelion or similar composite flower.]&1&n&dandelion clock\\\lspbottomrule
\end{longtable}
% vspace*{1em}

\noindent
Notes:\\
Added sense 2. Description taken from the OED entry clock, n$^1$, 8.: ``A trivial name for the pappus of the dandelion or similar composite flower.  [So called from the child's play of blowing away the feathered seeds to find ‘what o'clock it is’.]" Some clocks could be either \textsc{have2} or \textsc{for}; went with intuition (based on whether it results in a specific type or not, cf. \emph{town clock} vs. \emph{car clock}) and BNC contexts.
% - has only one wnSense as a noun
% - added wn-sense2 for dandelion clock:  OED 8. A trivial name for the pappus of the dandelion or similar composite flower.  [So called from the child's play of blowing away the feathered seeds to find ‘what o'clock it is’.] 
% - period clock as \textsc{be}, frog and cuckoo clock likewise
% - some clocks could be either \textsc{have2} or \textsc{for}; went with intuition and sometimes bnc contexts

\subsection{course    (101 compound types)}
Rated compound: \emph{crash course}

\vspace*{1ex}

\noindent
\begin{longtable}{c>{\raggedright\arraybackslash}p{5cm}rc>{\raggedright\arraybackslash}p{2cm}}\lsptoprule
{\small wnSense}&WordNet description&types&class&example\\\midrule
1&education imparted in a series of lessons or meetings&68&n&zoology\\\tablevspace
3&general line of orientation&1&n&compass course\\\tablevspace
5&a line or route along which something travels or moves&8&n&zigzag course\\\tablevspace
7&part of a meal served at one time&9&n&vegetable course\\\tablevspace
8&(construction) a layer of masonry&1&n&damp course\\\tablevspace
9&facility consisting of a circumscribed area of land or water laid out for a sport&14&n&mountain course\\\lspbottomrule
\end{longtable}
% vspace*{1em}

\noindent
Notes:\\
Some combinations with WordNet sense 1 which were coded as ABOUT could also be classified as \textsc{for} (e.g. \emph{fitness course}), but generally easily distinguishable. Did not attempt to distinguish sense 1 from sense 6, `a body of students who are taught together', similarly for sense 5 and 4, `a mode of action'.

% - consolidated postgrad/uate course to postgraduate [most frequent]
% - wn1 and wn6 in practice indistinguishable, used only wnSense1
% - similarly did not distinguish wnSense4 and wnSense5, but kept wnSense for compass course
% - country/mountain course as \textsc{use}, together with obstacles
% - some wnSense 1 ABOUT could also be classified as 1 \textsc{for} (fitness course), but generally easily distinguishable

\pagebreak[4]
\subsection{court (83 compound types)}
Rated compound: \emph{kangaroo court}

\vspace*{1ex}

\noindent
\begin{longtable}{c>{\raggedright\arraybackslash}p{5cm}rc>{\raggedright\arraybackslash}p{2cm}}\lsptoprule
{\small wnSense}&WordNet description&types&class&example\\\midrule
1&an assembly (including one or more judges) to conduct judicial business&15&n&Singapore court\\\tablevspace
4&a specially marked horizontal area within which a game is played&6&n&squash court\\\tablevspace
6&the family and retinue of a sovereign or prince&6&n&renaissance court\\\tablevspace
7&a tribunal that is presided over by a magistrate or by one or more judges who administer justice according to the laws&43&n&orphan court\\\tablevspace
8&the residence of a sovereign or nobleman&1&n&Fire court\\\tablevspace
9&an area wholly or partly surrounded by walls or buildings&12&n&prison court\\\lspbottomrule
\end{longtable}
% ** TODO court-n2
% *** court-n2-FamAna
% - consolidated appeal/s court to sg [but pl more frequent; meaning seems to be exactly the same]
% - consolidated companies/y court to sg [most frequent]
% - used wnSense 6 for christmas court in the sense of a royal court holding court

\subsection{cow      (10 compound types)}
Rated compound: \emph{cash cow}

\vspace*{1ex}

\noindent
\begin{longtable}{c>{\raggedright\arraybackslash}p{5cm}rc>{\raggedright\arraybackslash}p{2cm}}\lsptoprule
{\small wnSense}&WordNet description&types&class&example\\\midrule
2&mature female of mammals of which the male is called `bull'&10&n&suckler cow\\\lspbottomrule
\end{longtable}
% ** TODO cow-n2
% - already done up to FamAna in June, now finalized
% - Reddy: cash cow, sacred cow [but sacred cow excluded due to A N structure]

\pagebreak[4]
\subsection{crossing  (21 compound types)}
Rated compound: \emph{zebra crossing}
\vspace*{-.3cm}

% \vspace*{1ex}

\noindent
\begin{longtable}{c>{\raggedright\arraybackslash}p{5cm}rc>{\raggedright\arraybackslash}p{2cm}}\lsptoprule
{\small wnSense}&WordNet description&types&class&example\\\midrule
1&traveling across&8&n&ferry crossing\\\tablevspace
3&a point where two lines (paths or arcs etc.) intersect&1&n&zero crossing\\\tablevspace
5&a path (often marked) where something (as a street or railroad) can be crossed to get from one side to the other&12&n&pedestrian crossing\\\lspbottomrule
\end{longtable}
\vspace*{-.3cm}

% - cyclists crossing not in bnc
% KWA
% ** crossing-n2
% not big, but very difficult, deverbal!, quite a few unsatisfying
% decision, introduced \textsc{verb} as relation, \textsc{for} used in pedestrian crossing
% as well as traffic crossing, river crossing as object, sea crossing as
% voyage

\subsection{crunching (1 compound types)}
Rated compound: \emph{number crunching}
\vspace*{-.3cm}

% \vspace*{1ex}

\noindent
\begin{longtable}{c>{\raggedright\arraybackslash}p{5cm}rc>{\raggedright\arraybackslash}p{2cm}}\lsptoprule
{\small wnSense}&WordNet description&types&class&example\\\midrule
1&{}[the action of crunching]&1&n&number crunching\\\lspbottomrule
\end{longtable}
\vspace*{-.3cm}

\noindent
Notes:\\Added sense 1. There is no WordNet entry for the noun \emph{crunching}.

\subsection{dress     (25 compound types)}
Rated compound: \emph{cocktail dress}
\vspace*{-.3cm}

% \vspace*{1ex}

\noindent
\begin{longtable}{c>{\raggedright\arraybackslash}p{5cm}rc>{\raggedright\arraybackslash}p{2cm}}\lsptoprule
{\small wnSense}&WordNet description&types&class&example\\\midrule
1&a one-piece garment for a woman; has skirt and bodice&21&n&cotton dress\\\tablevspace
2&clothing of a distinctive style or for a particular occasion&3&n&camouflage dress\\\tablevspace
3&clothing in general&1&n&head dress\\\lspbottomrule
\end{longtable}
% - cinderella dress as \textsc{idiom}

\pagebreak[4]
\subsection{duck     (7 compound types)}
Rated compound: \emph{sitting duck}

\vspace*{1ex}

\noindent
\begin{longtable}{c>{\raggedright\arraybackslash}p{5cm}rc>{\raggedright\arraybackslash}p{2cm}}\lsptoprule
{\small wnSense}&WordNet description&types&class&example\\\midrule
1&small wild or domesticated web-footed broad-billed swimming bird usually having a depressed body and short legs&6&n&plastic duck\\\tablevspace
3&flesh of a duck (domestic or wild)&1&n&peking duck\\\lspbottomrule
\end{longtable}
\subsection{edge      (57 compound types)}
Rated compound: \emph{cutting edge}

\vspace*{1ex}

\noindent
\begin{longtable}{c>{\raggedright\arraybackslash}p{5cm}rc>{\raggedright\arraybackslash}p{2cm}}\lsptoprule
{\small wnSense}&WordNet description&types&class&example\\\midrule
1&the boundary of a surface&46&n&mirror edge\\\tablevspace
2&a line determining the limits of an area&2&n&city edge\\\tablevspace
3&a sharp side formed by the intersection of two surfaces of an object&9&n&sword edge\\\lspbottomrule
\end{longtable}
% vspace*{1em}

\noindent
Notes:\\
WordNet sense 6 (`the outside limit of an object or area or surface; a place farthest away from the center of something') was not used, \emph{leaf edge} classified with WordNet sense 1.
% - did not use wnSense6; instead leaf edge also \textsc{have2},1,

\pagebreak[4]
\subsection{egg       (14 compound types)}
Rated compound: \emph{nest egg}

\vspace*{1ex}

\noindent
\begin{longtable}{c>{\raggedright\arraybackslash}p{5cm}rc>{\raggedright\arraybackslash}p{2cm}}\lsptoprule
{\small wnSense}&WordNet description&types&class&example\\\midrule
1&animal reproductive body consisting of an ovum or embryo together with nutritive and protective envelopes; especially the thin-shelled reproductive body laid by e.g. female birds&14&n&ostrich egg\\\lspbottomrule
\end{longtable}
\subsection{engine    (43 compound types)}
Rated compound: \emph{search engine}

\vspace*{1ex}

\noindent
\begin{longtable}{c>{\raggedright\arraybackslash}p{5cm}rc>{\raggedright\arraybackslash}p{2cm}}\lsptoprule
{\small wnSense}&WordNet description&types&class&example\\\midrule
1&motor that converts thermal energy to mechanical work&35&n&combustion engine\\\tablevspace
2&something used to achieve a purpose&5&n&search engine\\\tablevspace
3&a wheeled vehicle consisting of a self-propelled engine that is used to draw trains along railway tracks&2&n&express engine\\\tablevspace
4&an instrument or machine that is used in warfare, such as a battering ram, catapult, artillery piece, etc.&1&n&siege engine\\\lspbottomrule
\end{longtable}
% - added lots of names with \textsc{make2}
% - added search-engine to final
% ** engine-n2
% - did not use wnSense 3 railway engine for railway engine itself but
% for e.g. pilot engine

\subsection{firm      (69 compound types)}
Rated compound: \emph{law firm}

\vspace*{1ex}

\noindent
\begin{longtable}{c>{\raggedright\arraybackslash}p{5cm}rc>{\raggedright\arraybackslash}p{2cm}}\lsptoprule
{\small wnSense}&WordNet description&types&class&example\\\midrule
1&the members of a business organization that owns or operates one or more establishments&69&n&mystery firm\\\lspbottomrule
\end{longtable}
% vspace*{1em}

\noindent
Notes:\\
\emph{Security firm} in the BNC is consistently used to refer to businesses providing private security, whereas \emph{securities firm} is used to refer to banking businesses dealing in securities, that is, \emph{securities} occurs here in its certificate of ownership sense.
% - security and securities firm: not consolidated: 2 distinct meanings: protection vs. banking

\subsection{floor     (110 compound types)}
Rated compound: \emph{ground floor}

\vspace*{1ex}

\noindent
\begin{longtable}{c>{\raggedright\arraybackslash}p{5cm}rc>{\raggedright\arraybackslash}p{2cm}}\lsptoprule
{\small wnSense}&WordNet description&types&class&example\\\midrule
1&the inside lower horizontal surface (as of a room, hallway, tent, or other structure)&79&n&wagon floor\\\tablevspace
2&a structure consisting of a room or set of rooms at a single position along a vertical scale&8&n&executive floor\\\tablevspace
4&the ground on which people and animals move about&4&n&jungle floor\\\tablevspace
5&the bottom surface of any lake or other body of water&4&n&ocean floor\\\tablevspace
6&the lower inside surface of any hollow structure&5&n&cavern floor\\\tablevspace
9&the legislative hall where members debate and vote and conduct other business&4&n&senate floor\\\tablevspace
10&a large room in a exchange where the trading is done&6&n&trading floor\\\lspbottomrule
\end{longtable}
% vspace*{1em}

\noindent
Notes:\\
WordNet sense 10 has \emph{trading floor} as collocate.
% *** floor-n2-FamAna
% - dance floor: at first at wnSense 4: the ground on which people and animals move about, as dance floors can be outside, but finally with wnSense 1 and \textsc{for}
% - used \textsc{make2} for all materials from pebble/mirror to marble/wood

\subsection{form      (163 compound types)}
Rated compound: \emph{application form}

\vspace*{1ex}

\noindent
\begin{longtable}{c>{\raggedright\arraybackslash}p{5cm}rc>{\raggedright\arraybackslash}p{2cm}}\lsptoprule
{\small wnSense}&WordNet description&types&class&example\\\midrule
1&the phonological or orthographic sound or appearance of a word that can be used to describe or identify something&11&n&root form\\\tablevspace
2&a category of things distinguished by some common characteristic or quality&6&n&life form\\\tablevspace
3&a perceptual structure&10&n&paper form\\\tablevspace
4&any spatial attributes (especially as defined by outline)&30&n&leaf form\\\tablevspace
5&alternative names for the body of a human being&7&n&dwarf form\\\tablevspace
6&the spatial arrangement of something as distinct from its substance&5&n&tensor form\\\tablevspace
8&a printed document with spaces in which to write&76&n&transfer form\\\tablevspace
10&an arrangement of the elements in a composition or discourse&14&n&verse form\\\tablevspace
12&(physical chemistry) a distinct state of matter in a system; matter that is identical in chemical composition and physical state and separated from other material by the phase boundary&3&n&compound form\\\tablevspace
14&an ability to perform well&1&n&fighting form\\\lspbottomrule
\end{longtable}
\subsection{game     (155 compound types)}
Rated compound: \emph{blame game}, \emph{video game}

\vspace*{1ex}

\noindent
\begin{longtable}{c>{\raggedright\arraybackslash}p{5cm}rc>{\raggedright\arraybackslash}p{2cm}}\lsptoprule
{\small wnSense}&WordNet description&types&class&example\\\midrule
2&a single play of a sport or other contest&67&n&relegation game\\\tablevspace
3&an amusement or pastime&60&n&war game\\\tablevspace
9&the game equipment needed in order to play a particular game&1&n&Simon game\\\tablevspace
10&your occupation or line of work&27&n&subsidy game\\\lspbottomrule
\end{longtable}
% vspace*{1em}

\noindent
Notes:\\
\emph{Republic game}, \emph{Newcastle game} etc.: construal of these depends on the point of view: if Republic is your opponent, \textsc{have1} seems natural, if it refers to a game by your team, \textsc{have2} seems more appropriate. In the BNC, both usages occur. Decided to go with \textsc{have2} consistently. Note that the distribution of word senses used is interestingly different from the ones in the N1 family.
% \ex. \a. K2D 2516 	Big business interests have been quietly buying up Latvia match tickets to ensure seats for their clients at the Republic game. [BNC contexts: Northern Ireland's upcoming games]
% \b. 
% Republic game
% \textbf{TODO: below direct cop}
% - what to do with Republic game etc.? game that has Republic \textsc{have1} or \textsc{have2}? Both occur in BNC: either the Newcastle game against sb. where it is \textsc{have2}, or the Republic game from the point of view of another team, where it would rather be \textsc{have1}
%  Decided to go with \textsc{have2} consistently
% - several occurences of olympic games stuff: occur only in plural and are then IN: Seoul/Munich/Barcelona/Calgary games
% - did not use wnSense 1 anymore (instead either 2 (specific football games etc.) or 3); contrast to game-n1 use!!
% ** game-n2 

% - many metaphorical senses differentiated as wnSenses -> we definitely
% need to rework our shift/no-shift coding in relation to the word net
% senses very carefully -> still not clear to me whether one can say
% that a sense is so frequent so as not to be perceived as shifted,
% perhaps has more to do with the frequency ratio between original and
% shifted sense, and perhaps also with the distance of the shift
% - wnSense8: used for shifted meaning of game, but not in the sense of
%   illegal etc. [note: in september 2014 version changed to wnSense 10]

\subsection{gun       (40 compound types)}
Rated compound: \emph{smoking gun}

\vspace*{1ex}

\noindent
\begin{longtable}{c>{\raggedright\arraybackslash}p{5cm}rc>{\raggedright\arraybackslash}p{2cm}}\lsptoprule
{\small wnSense}&WordNet description&types&class&example\\\midrule
1&a weapon that discharges a missile at high velocity (especially from a metal tube or barrel)&36&n&machine gun\\\tablevspace
3&a person who shoots a gun (as regards their ability)&1&n&advertising gun\\\tablevspace
7&the discharge of a firearm as signal or as a salute in military ceremonies&3&n&start gun\\\lspbottomrule
\end{longtable}
% vspace*{1em}

\noindent
Notes:\\
WordNet senses 2, `large but transportable armament', and 5, `a hand-operated pump that resembles a pistol; forces grease into parts of a machine', where not distinguished, instead 1 was used.
% ** gun-n2
% - collapsed a number of uses (use steam/nails etc.)

\subsection{hour      (32 compound types)}
Rated compound: \emph{rush hour}

\vspace*{1ex}

\noindent
\begin{longtable}{c>{\raggedright\arraybackslash}p{5cm}rc>{\raggedright\arraybackslash}p{2cm}}\lsptoprule
{\small wnSense}&WordNet description&types&class&example\\\midrule
1&a period of time equal to 1/24th of a day&31&n&peak hour\\\tablevspace
4&{}[for one hour]&1&n&kilowatt hour\\\lspbottomrule
\end{longtable}
% vspace*{1em}

\noindent
Notes:\\
Added sense 4 for \emph{kilowatt hour}.
% ** TODO hour-n2-check
% - used wnSense 4 for kilowatt hour
% ** hour-n2
% - actually more plurals than singulars
% - wnSenses not very helpful

\subsection{insurance (9 compound types)}
Rated compound: \emph{health insurance}

% \vspace*{1ex}

\noindent
\begin{longtable}{c>{\raggedright\arraybackslash}p{5cm}rc>{\raggedright\arraybackslash}p{2cm}}\lsptoprule
{\small wnSense}&WordNet description&types&class&example\\\midrule
1&promise of reimbursement in the case of loss; paid to people or companies so concerned about hazards that they have made prepayments to an insurance company&9&n&fire insurance\\\lspbottomrule
\end{longtable}

\subsection{jacket    (25 compound types)}
Rated compound: \emph{smoking jacket}

% \vspace*{1ex}

\noindent
\begin{longtable}{c>{\raggedright\arraybackslash}p{5cm}rc>{\raggedright\arraybackslash}p{2cm}}\lsptoprule
{\small wnSense}&WordNet description&types&class&example\\\midrule
1&a short coat&22&n&cashmere jacket\\\tablevspace
2&an outer wrapping or casing&2&n&book jacket\\\tablevspace
5&the tough metal shell casing for certain kinds of ammunition&1&n&steam jacket\\\lspbottomrule
\end{longtable}
% vspace*{1em}

\noindent
Notes:\\
A \emph{steam jacket} is not a casing for certain kinds of ammunition, but it is usually a tough metal shell casing, so WordNet sense 5 was used.
\subsection{jenny     (1 compound types)}
Rated compound: \emph{spinning jenny}

\vspace*{1ex}

\noindent
\begin{longtable}{c>{\raggedright\arraybackslash}p{5cm}rc>{\raggedright\arraybackslash}p{2cm}}\lsptoprule
{\small wnSense}&WordNet description&types&class&example\\\midrule
1&{}[personal name]&1&n&spinning jenny\\\lspbottomrule
\end{longtable}
% vspace*{1em}

\noindent
Notes:\\
WordNet sense 1 refers to a specific person (Jenny, William Le Baron
Jenny), here used for Jenny as a proper name.
\subsection{lane      (22 compound types)}
Rated compound: \emph{memory lane}

\vspace*{1ex}

\noindent
\begin{longtable}{c>{\raggedright\arraybackslash}p{5cm}rc>{\raggedright\arraybackslash}p{2cm}}\lsptoprule
{\small wnSense}&WordNet description&types&class&example\\\midrule
1&a narrow way or road&11&n&country lane\\\tablevspace
2&a well-defined track or path; for e.g. swimmers or lines of traffic&11&n&emergency lane\\\lspbottomrule
\end{longtable}
\subsection{limit    (52 compound types)}
Rated compound: \emph{speed limit}

\vspace*{1ex}

\noindent
\begin{longtable}{c>{\raggedright\arraybackslash}p{5cm}rc>{\raggedright\arraybackslash}p{2cm}}\lsptoprule
{\small wnSense}&WordNet description&types&class&example\\\midrule
1&the greatest possible degree of something&49&n&confidence limit\\\tablevspace
4&the boundary of a specific area&3&n&city limit\\\lspbottomrule
\end{longtable}
% vspace*{1em}

\noindent
Notes:\\
Only the above 2 WordNet senses were used; all the other noun senses are very close to sense 1.
% ** limit-n2
% - no differentiation of wnSenses (differences unclear), except wnSense4

\pagebreak[4]
\subsection{line      (363 compound types)}
\vspace*{-.2cm}
Rated compounds: \emph{fine line}, \emph{firing line}

\vspace*{-.4cm}

\noindent
\begin{longtable}{c>{\raggedright\arraybackslash}p{5cm}rc>{\raggedright\arraybackslash}p{2cm}}\lsptoprule
{\small wnSense}&WordNet description&types&class&example\\\midrule
1&a formation of people or things one beside another&3&n&picket line\\\\[-.7em]
2&a mark that is long relative to its width&14&n&chalk line\\\\[-.7em]
3&a formation of people or things one behind another&9&n&coffee line\\\\[-.7em]
4&a length (straight or curved) without breadth or thickness; the trace of a moving point&7&n&regression line\\\\[-.7em]
5&text consisting of a row of words written across a page or computer screen&22&n&solo line\\\\[-.7em]
7&a fortified position (especially one marking the most forward position of troops)&12&n&enemy line\\\\[-.7em]
8&a course of reasoning aimed at demonstrating a truth or falsehood; the methodical process of logical reasoning&19&n&ap\-pease\-ment line\\\\[-.7em]
9&a conductor for transmitting electrical or optical signals or electric power&13&n&telegraph line\\\\[-.7em]
10&a connected series of events or actions or developments&3&n&plot line\\\\[-.7em]
11&a spatial location defined by a real or imaginary unidimensional extent&87&n&glacier line\\\\[-.7em]
12&a slight depression or fold in the smoothness of a surface&7&n&worry line\\\\[-.7em]
13&a pipe used to transport liquids or gases&8&n&fuel line\\\\[-.7em]
14&the road consisting of railroad track and roadbed&28&n&intercity line\\\\[-.7em]
15&a telephone connection&19&n&reception line\\\\[-.7em]
16&acting in conformity&1&n&policy line\\\\[-.7em]
17&the descendants of one individual&9&n&primate line\\\\[-.7em]
18&something (as a cord or rope) that is long and thin and flexible&24&n&trap line\\\\[-.7em]
20&in games or sports; a mark indicating positions or bounds of the playing area&10&n&goal line\\\\[-.7em]
21&(often plural) a means of communication or access&4&n&distribution line\\\\[-.7em]
22&a particular kind of product or merchandise&14&n&profit line\\\\[-.7em]
23&a commercial organization serving as a common carrier&13&n&tram line\\\\[-.7em]
25&the maximum credit that a customer is allowed&2&n&withdrawal line\\\\[-.7em]
26&a succession of notes forming a distinctive sequence&5&n&chorus line\\\\[-.7em]
29&a conceptual separation or distinction&23&n&wage line\\\\[-.7em]
30&mechanical system in a factory whereby an article is conveyed through sites at which successive operations are performed on it&7&n&canning line\\\lspbottomrule
\end{longtable}
% vspace*{1em}

\noindent
Mistakes:\\
 \emph{Birmingham line} is misclassified with WordNet sense 10, it ought to be WordNet sense 14. In addition, in its occurrence in the BNC it is part of a complex construction (\emph{London and Birmingham line}) and thus should be n1n2NotCompound:yes.
\subsection{list      (145 compound types)}
Rated compound: \emph{mailing list}

\vspace*{1ex}

\noindent
\begin{longtable}{c>{\raggedright\arraybackslash}p{5cm}rc>{\raggedright\arraybackslash}p{2cm}}\lsptoprule
{\small wnSense}&WordNet description&types&class&example\\\midrule
1&a database containing an ordered array of items (names or topics)&145&n&waiting list\\\lspbottomrule
\end{longtable}
% vspace*{1em}

\noindent
Notes:\\
Large numbers of \textsc{for} and \textsc{have1} compounds, e.g. \emph{wedding list} vs. \emph{witness list}. Often, both classifications are possible: \emph{staff list} as list for the staff or as list that has the staff (e.g. list that lists the staff). Heuristic: went for \textsc{have1} if that interpretation is possible. In cases of doubt, checked against BNC usage.

% ** list-n2 
% - large number of \textsc{for} and \textsc{have1}: wedding list, witness list. Often, both possible: staff list as list for the staff or as list that has the staff (e.g. lists the staff) 
% - heuristic: went for \textsc{have1} if that interpretation is possible, and, in cases of doubt, actually used in the bnc entry

\subsection{lot       (10 compound types)}
Rated compound: \emph{parking lot}

\vspace*{1ex}

\noindent
\begin{longtable}{c>{\raggedright\arraybackslash}p{5cm}rc>{\raggedright\arraybackslash}p{2cm}}\lsptoprule
{\small wnSense}&WordNet description&types&class&example\\\midrule
1&(often followed by `of') a large number or amount or extent&2&n&job lot\\\tablevspace
2&a parcel of land having fixed boundaries&8&n&studio lot\\\lspbottomrule
\end{longtable}
% ** lot-n2
% - job lot as FROM due to obscure OED etymology lot from a criminal job

\pagebreak[4]
\subsection{mail      (13 compound types)}
Rated compound: \emph{snail mail}

\vspace*{-.4cm}

\noindent
\begin{longtable}{c>{\raggedright\arraybackslash}p{5cm}rc>{\raggedright\arraybackslash}p{2cm}}\lsptoprule
{\small wnSense}&WordNet description&types&class&example\\\midrule
2&the bags of letters and packages that are transported by the postal service&3&n&air mail\\\tablevspace
4&any particular collection of letters or packages that is delivered&7&n&junk mail\\\tablevspace
5&(Middle Ages) flexible armor made of interlinked metal rings&2&n&chain mail\\\tablevspace
6&{}[newspaper]&1&n&Birmingham mail\\\lspbottomrule
\end{longtable}
% vspace*{1em}

\noindent
Notes:\\
Added sense 6. Note that \emph{snail mail} is coded here with WordNet sense 2 and the relation \textsc{use}, in parallel to \emph{air mail}. In the N1 family, it is coded as \textsc{be}.

\subsection{market    (245 compound types)}
Rated compound: \emph{flea market}

\vspace*{-.4cm}

\noindent
\begin{longtable}{c>{\raggedright\arraybackslash}p{5cm}rc>{\raggedright\arraybackslash}p{2cm}}\lsptoprule
{\small wnSense}&WordNet description&types&class&example\\\midrule
1&the world of commercial activity where goods and services are bought and sold&223&n&whisky market\\\tablevspace
3&a marketplace where groceries are sold&1&n&Com market\\\tablevspace
4&the securities markets in the aggregate&5&n&bull market\\\tablevspace
5&an area in a town where a public mercantile establishment is set up&16&n&Monday market\\\lspbottomrule
\end{longtable}
% vspace*{1em}

% \pagebreak[4]
\noindent
Notes:\\
No distinction made between WordNet sense 1 and 2 (`the customers for a particular product or service'), all classified as 1.
% ** TODO market-n2
% *** market-n2-FamAna
% - consolidated commodity/s market to sg [but plural more frequent]
% - school/s market to sg (most frequent)
% - job/s to job market (most frequent)
% - street/s to street market (most frequent)
% - growth market with \textsc{have1},1, Lloyds market as \textsc{have2},3
% - no distinction between wnSense1 ans wnSense2, all grouped as 1
% **** ABOUT 
% - wondering whether an analysis of e.g. oil market as ABOUT makes
%   sense? Levi 103ff as oil war as ABOUT (war over oil), but one can
%   also go to war for oil; however, market and \textsc{for} seem to be the
%   natural connection: this is a market for oil, not for vegetables
\subsection{mine      (22 compound types)}
Rated compound: \emph{gold mine}

% \vspace*{1ex}

\noindent
\begin{longtable}{c>{\raggedright\arraybackslash}p{5cm}rc>{\raggedright\arraybackslash}p{2cm}}\lsptoprule
{\small wnSense}&WordNet description&types&class&example\\\midrule
1&excavation in the earth from which ores and minerals are extracted&19&n&diamond mine\\\tablevspace
2&explosive device that explodes on contact; designed to destroy vehicles or ships or to kill or maim personnel&3&n&land mine\\\lspbottomrule
\end{longtable}
\subsection{model    (143 compound types)}
Rated compound: \emph{role model}

% \vspace*{1ex}

\noindent
\begin{longtable}{c>{\raggedright\arraybackslash}p{5cm}rc>{\raggedright\arraybackslash}p{2cm}}\lsptoprule
{\small wnSense}&WordNet description&types&class&example\\\midrule
1&a hypothetical description of a complex entity or process&16&n&regression model\\\tablevspace
2&a type of product&25&n&signature model\\\tablevspace
3&a person who poses for a photographer or painter or sculptor&1&n&life model\\\tablevspace
4&representation of something (sometimes on a smaller scale)&67&n&wax model\\\tablevspace
5&something to be imitated&1&n&artist model\\\tablevspace
6&someone worthy of imitation&1&n&role model\\\tablevspace
7&a representative form or pattern&25&n&ownership model\\\tablevspace
8&a woman who wears clothes to display fashions&7&n&agency model\\\lspbottomrule
\end{longtable}
% vspace*{1em}

\noindent
Notes:\\
WordNet sense 6 has \emph{role model} as collocate. 
% - I used wnSenses 1 and 4, ideally distinguishing between the
%   theoretical account idea (TRACE model) vs. the simulation idea
%   (regression model). However, this distinction is often not that
%   clear. As this distinction does not concern the Reddy et al
%   example, it probably does not matter to much
% - wnSenses 2: classified \textsc{be} for e.g. printer model, FROM for Toyota model heuristic: explicit name of a producer
% *** TODO Reddy example: "role model"
%  dominant token frequency, and its own
%   entry, wnSense 6; considered alternative wnSense7, but a) for consistency and b) becasue role models are typically single persons stuck with 6
% ** model-n2
% - did not use all wnSense
% - role model extro wnSense6, high token frequency,
% - life model single token wnSense 3

\subsection{nine      (1 compound types)}
Rated compound: \emph{cloud nine}

\vspace*{-.2cm}

\noindent
\begin{longtable}{c>{\raggedright\arraybackslash}p{5cm}rc>{\raggedright\arraybackslash}p{2cm}}\lsptoprule
{\small wnSense}&WordNet description&types&class&example\\\midrule
2&the cardinal number that is the sum of eight and one&1&n&cloud nine\\\lspbottomrule
\end{longtable}
% vspace*{1em}
\vspace*{-.2cm}

\noindent
Notes:\\
First sense in WordNet.
% - cloud nine manually added, no other nine compounds, cloud nine also not in celex

\subsection{oil       (30 compound types)}
Rated compound: \emph{snake oil}

% \vspace*{1ex}

\noindent
\begin{longtable}{c>{\raggedright\arraybackslash}p{5cm}rc>{\raggedright\arraybackslash}p{2cm}}\lsptoprule
{\small wnSense}&WordNet description&types&class&example\\\midrule
1&a slippery or viscous liquid or liquefiable substance not miscible with water&11&n&massage oil\\\tablevspace
2&oil paint containing pigment that is used by an artist&1&n&landscape oil\\\tablevspace
3&a dark oil consisting mainly of hydrocarbons&8&n&mineral oil\\\tablevspace
4&any of a group of liquid edible fats that are obtained from plants&10&n&sesame oil\\\lspbottomrule
\end{longtable}
% vspace*{1em}

\noindent
Notes:\\
Classified \emph{whale oil} with the vegetable oils.
% ** oil-n2
% whale/fish oil with vegetable oils

\subsection{order     (100 compound types)}
Rated compound: \emph{pecking order}

\vspace*{1ex}

\noindent
\begin{longtable}{c>{\raggedright\arraybackslash}p{5cm}rc>{\raggedright\arraybackslash}p{2cm}}\lsptoprule
{\small wnSense}&WordNet description&types&class&example\\\midrule
1&(often plural) a command given by a superior (e.g., a military or law enforcement officer) that must be obeyed&8&n&draft order\\\tablevspace
3&established customary state (especially of society)&5&n&gender order\\\tablevspace
4&logical or comprehensible arrangement of separate elements&18&n&seating order\\\tablevspace
6&a legally binding command or decision entered on the court record (as if issued by a court or judge)&44&n&restriction order\\\tablevspace
11&a group of person living under a religious rule&3&n&dervish order\\\tablevspace
12&(biology) taxonomic group containing one or more families&2&n&primate order\\\tablevspace
13&a request for something to be made, supplied, or served&20&n&telephone order\\\lspbottomrule
\end{longtable}
% vspace*{1em}

\noindent
Notes:\\
This group contains very many legal or half-legal terms. Many combinations can in principle be either \textsc{for} or \textsc{have2}. For example, \emph{question order} can refer to the ordering the questions have (in one's work for example), or the order for the questions (e.g. at a talk).
% ** order-n2
% - very many legal or half-legal terms
% - order \textsc{for}/\textsc{have2} question order could be order the questions have (in your
% work) or order for the question (at a talk); see especially wnSense4, where
% the decision are not so clearcut

\subsection{owl       (4 compound types)}
Rated compound: \emph{night owl}

\vspace*{1ex}

\noindent
\begin{longtable}{c>{\raggedright\arraybackslash}p{5cm}rc>{\raggedright\arraybackslash}p{2cm}}\lsptoprule
{\small wnSense}&WordNet description&types&class&example\\\midrule
1&nocturnal bird of prey with hawk-like beak and claws and large head with front-facing eyes&4&n&barn owl\\\lspbottomrule
\end{longtable}
% ** owl-n2
% - screech owl \textsc{verb}2; but it is a name so doesn't really matter

\subsection{park      (49 compound types)}
Rated compound: \emph{car park}

\vspace*{1ex}

\noindent
\begin{longtable}{c>{\raggedright\arraybackslash}p{5cm}rc>{\raggedright\arraybackslash}p{2cm}}\lsptoprule
{\small wnSense}&WordNet description&types&class&example\\\midrule
1&a large area of land preserved in its natural state as public property&4&n&nature park\\\tablevspace
2&a piece of open land for recreational use in an urban area&25&n&council park\\\tablevspace
3&a facility in which ball games are played (especially baseball games)&4&n&ball park\\\tablevspace
5&a lot where cars are parked&6&n&caravan park\\\tablevspace
7&{}[an area of land, often on the
  outskirts of a town, devoted to a particular activity or set of related
  pursuits]&4&n&science park\\\tablevspace
8&{}[used in district names formerly belonging to large estates]&6&n&grove park\\\lspbottomrule
\end{longtable}
% vspace*{1em}

\noindent
Notes:\\
Added sense 7 and sense 8. Sense 7 corresponds to OED park n. 3.f (``With modifying word: an area of land, often on the
  outskirts of a town, devoted to a particular activity or set of related
  pursuits."), sense 8 to OED park 1.d (``Used in the names of suburban districts built on land
  formerly belonging to large estates, as Holland Park, Tufnell Park, and
  later in the names of other urban areas, housing estates, etc.").
% - wnSense6 extended to cover call park
% - new wnSense8:OED 1.d.
% - Nene park as wnSense 3/ballpark and IN, although at/near
\emph{Ball park} is collocate of WordNet sense 3.
% ** TODO park-n2
% - new wnSense7: "With modifying word: an area of land, often on the
%   outskirts of a town, devoted to a particular activity or set of related
%   pursuits." OED park n. 3.f
% - wnSense6 extended to cover call park
% - new wnSense8:OED 1.d." Used in the names of suburban districts built on land
%   formerly belonging to large estates, as Holland Park, Tufnell Park, and
%   later in the names of other urban areas, housing estates, etc."
% - Nene park as wnSense 3/ballpark and IN, although at/near

\subsection{plan      (177 compound types)}
Rated compound: \emph{game plan}

\vspace*{1ex}

\noindent
\begin{longtable}{c>{\raggedright\arraybackslash}p{5cm}rc>{\raggedright\arraybackslash}p{2cm}}\lsptoprule
{\small wnSense}&WordNet description&types&class&example\\\midrule
1&a series of steps to be carried out or goals to be accomplished&142&n&privatisation plan\\\tablevspace
2&an arrangement scheme&4&n&seating plan\\\tablevspace
3&scale drawing of a structure&31&n&factory plan\\\lspbottomrule
\end{longtable}
% vspace*{1em}

\noindent
Notes:\\
Sometimes difficult to differentiate between \textsc{have2}/\textsc{make2}/\textsc{for}: an \emph{army plan} is a plan the army has, the \emph{Allon plan} was made by Allon, the \emph{abolition plan} is for abolition; but especially for army etc., all 3 are possible. Person names usually linked with \textsc{make2}, organizations with \textsc{have2}. \textsc{for} was used for aims as well as target groups (cf. \emph{recovery plan} vs. \emph{staff plan}).
% *** plan-n2-FamAna
% - sometimes difficult to differentiate between \textsc{have2}/\textsc{make2}/\textsc{for}: army plan is a plan the army has, the allon plan was made by Allon, the abolition plan is for abolition; but especially for army etc. all 3 would be possible
%  - started to link person name/\textsc{make2} organization/\textsc{have2}
% - consolidated reorganisation/reorganization to z, way more frequent
% - consolidated privatization/privatisation to z, more frequent  
% - consolidated defense/defence to s, more frequent
% - consolidated modernization/sation to z more frequent
% - consolidated Marshal to Marshall plan [in bnc in this usage]
% - \textsc{for} occurs either with the aim  (resettlement plan) or the target group (staff plan)
% - many proper names
% *** plan-n2-final-BR-clean
% - Marshal/Marshall reoccurs, this time, no consolidation [keeping in line with the general policy of only adding information at this stage AND with the fact that we did not run a general spelling normalization on the pl/sg data anyways]

\pagebreak[4]
\subsection{plate    (82 compound types)}
Rated compound: \emph{fashion plate}

\vspace*{-.2cm}

\noindent
\begin{longtable}{c>{\raggedright\arraybackslash}p{5cm}rc>{\raggedright\arraybackslash}p{2cm}}\lsptoprule
{\small wnSense}&WordNet description&types&class&example\\\midrule
1&(baseball) base consisting of a rubber slab where the batter stands; it must be touched by a base runner in order to score&1&n&home plate\\\\[-.6em]
2&a sheet of metal or wood or glass or plastic&48&n&aluminium plate\\\\[-.6em]
3&a full-page illustration (usually on slick paper)&2&n&colour plate\\\\[-.6em]
4&dish on which food is served or from which food is eaten&10&n&pie plate\\\\[-.6em]
6&a rigid layer of the Earth's crust that is believed to drift slowly&4&n&nazca plate\\\\[-.6em]
9&any flat platelike body structure or part&6&n&jaw plate\\\\[-.6em]
11&a flat sheet of metal or glass on which a photographic image can be recorded&3&n&printing plate\\\\[-.6em]
13&a shallow receptacle for collection in church&2&n&church plate\\\\[-.6em]
16&{}small label&1&n&book plate\\\\[-.6em]
17&{}[River Plate]&1&n&River Plate\\\\[-.6em]
18&{}[licence plate]&4&n&Texas plate\\\lspbottomrule
\end{longtable}
% vspace*{1em}
\vspace*{-.2cm}

\noindent
Notes:\\
Added senses 16, 17, and 18. Sense 18 used for licence plates (\emph{licence plate} itself is contained in this group). Sense 16 and 17 both occur only for one single type.
% - did not distinguish wnSense 14 and 2
% - new wnSense 16 for bookplate A bookplate, also known as ex-librīs [Latin, "from the books of..."], is usually a small print or decorative label pasted into a book, often on the inside front cover
% - put the material ones consistently under \textsc{make2},2,
% - wnSense6/tectonic plates: used \textsc{idiom} as the relation, as all other option
%   were not clearly fitting
% - wn13 distinguished, although wondering whether that is a good idea
% - did not use wnSense 12, again went for 2
% - new wn18 for licence plates
% - new wn17 River Plate
% - fashion plate: in bnc either for pages in fashion magazine or for people
WordNet sense 1 has \emph{home plate} as a collocate, WordNet sense 13 has \emph{collection plate} as collocate.

\pagebreak[4]
\subsection{pool      (43 compound types)}
Rated compound: \emph{swimming pool}
\vspace*{-.2cm}

\noindent
\begin{longtable}{c>{\raggedright\arraybackslash}p{5cm}rc>{\raggedright\arraybackslash}p{2cm}}\lsptoprule
{\small wnSense}&WordNet description&types&class&example\\\midrule
1&an excavation that is (usually) filled with water&14&n&hotel pool\\\tablevspace
2&a small lake&8&n& crocodile pool\\\tablevspace
3&an organization of people or resources that can be shared&20&n&player pool\\\tablevspace
8&something resembling a pool of liquid&1&n&moon pool\\\lspbottomrule
\end{longtable}
% vspace*{1em}
\vspace*{-.2cm}

\noindent
Notes:\\
WordSense 3 contains senses 4, `an association of companies for some definite purpose', and 5, `any communal combination of funds', and also 7, `the combined stakes of the betters'. The latter 3 were therefore not used.
\emph{Moon pool} (OED: n. a shaft open to the sea in the centre of an (esp. oil-drilling) ship, through which equipment can be hoisted.) is coded with WordNet sense 8 and \textsc{make2} according to the etymology suggested in one of the OED quotations:  ``1981   ‘D. Rutherford’ Porcupine Basin ii. 30   It was named moon-pool because on calm nights the water under a rig could reflect the moonlight and give the impression of a calm swimming pool.'' 

% ** pool-n2
% used only wnSense 1-3,8, collapsed several senses into 3 (hard to distinguish) 

\subsection{position  (161 compound types)}
Rated compound: \emph{lotus position}
\vspace*{-.2cm}

\noindent
\begin{longtable}{c>{\raggedright\arraybackslash}p{5cm}rc>{\raggedright\arraybackslash}p{2cm}}\lsptoprule
{\small wnSense}&WordNet description&types&class&example\\\midrule
1&the particular portion of space occupied by something&55&n&word position\\\tablevspace
2&a point occupied by troops for tactical reasons&9&n&artillery position\\\tablevspace
4&the arrangement of the body and its limbs&20&n&lotus position\\\tablevspace
6&a job in an organization&8&n&manage\-ment position\\\tablevspace
7&the spatial property of a place where or way in which something is situated&7&n&rotor position\\\tablevspace
9&(in team sports) the role assigned to an individual player&3&n&centre position\\\tablevspace
10&the act of putting something in a certain place&14&n&stock position\\\tablevspace
11&a condition or position in which you find yourself&19&n&monopoly position\\\tablevspace
12&a rationalized mental attitude&21&n&universalist position\\\tablevspace
14&an item on a list or in a sequence&5&n&pole position\\\lspbottomrule
\end{longtable}
% vspace*{1em}
\vspace*{-.2cm}

\noindent
Notes:\\
Used WordNet sense 10 for financial positions. Did not distinguish between sense 12 and 13 (`an opinion that is held in opposition to another in an argument or dispute'). 
% - used only a reduced set of wnSenses, large group, went with intuition on wnSense1 \textsc{for}/\textsc{have2} 
% - wnSense 10 for financial positions
% - no wnSense13, used wn12 instead

\subsection{potato    (3 compound types)}
Rated compound: \emph{couch potato}

\vspace*{-.2cm}

\noindent
\begin{longtable}{c>{\raggedright\arraybackslash}p{5cm}rc>{\raggedright\arraybackslash}p{2cm}}\lsptoprule
{\small wnSense}&WordNet description&types&class&example\\\midrule
1&an edible tuber native to South America; a staple food of Ireland&2&n&seed potato\\\tablevspace
3&{}[a person or character]&1&n&couch potato\\\lspbottomrule
\end{longtable}
% vspace*{1em}

\noindent
Notes:\\
Added sense 3, cf. OED potato, n. 4. b. colloq. (chiefly humorous). A person or character, esp. of a specified sort (usually with negative or derogatory connotations).
\subsection{pot       (51 compound types)}
Rated compound: \emph{melting pot}

\vspace*{-.2cm}

\noindent
\begin{longtable}{c>{\raggedright\arraybackslash}p{5cm}rc>{\raggedright\arraybackslash}p{2cm}}\lsptoprule
{\small wnSense}&WordNet description&types&class&example\\\midrule
1&metal or earthenware cooking vessel that is usually round and deep; often has a handle and lid&46&n&enamel pot\\\tablevspace
2&a plumbing fixture for defecation and urination&1&n&chamber pot\\\tablevspace
5&(often followed by `of') a large number or amount or extent&2&n&place plot\\\tablevspace
8&a resistor with three terminals, the third being an adjustable center terminal; used to adjust voltages in radios and TV sets&2&n&volume pot\\\lspbottomrule
\end{longtable}
% vspace*{1em}
\vspace*{-.2cm}

\noindent
Notes:\\
Sense 1 contains 2 clearly distinguishable main relations: \textsc{for} and \textsc{make2}.
% ** pot-n2 
% - only wnSense 1,2,5,8, BUT 2 only once and simple metaphorical ext? chamber pot
% - well-defined 2 classes for 1: \textsc{for} and \textsc{make2}
\subsection{project   (118 compound types)}
Rated compound: \emph{research project}


\vspace*{-.2cm}

\noindent
\begin{longtable}{c>{\raggedright\arraybackslash}p{5cm}rc>{\raggedright\arraybackslash}p{2cm}}\lsptoprule
{\small wnSense}&WordNet description&types&class&example\\\midrule
1&any piece of work that is undertaken or attempted&118&n&pilot project\\\lspbottomrule
\end{longtable}
% vspace*{1em}
\vspace*{-.2cm}

\noindent
Notes:\\
Used sense 1 without distinguishing it from sense 2, `a planned undertaking'.
% ** project-n2
% - again, wnSenses 1-2 not really helpful (actual vs. plan?),used only one
%  S: (n) undertaking, project, task, labor (any piece of work that is undertaken or attempted) "he prepared for great undertakings"
%  S: (n) project, projection () 
% - distinguished between ABOUT in domesday project against \textsc{for} in
%   breeding project and also volunteer project

\subsection{race      (70 compound types)}
Rated compound: \emph{rat race}
\vspace*{-.2cm}



\noindent
\begin{longtable}{c>{\raggedright\arraybackslash}p{5cm}rc>{\raggedright\arraybackslash}p{2cm}}\lsptoprule
{\small wnSense}&WordNet description&types&class&example\\\midrule
1&any competition&15&n&armament race\\\tablevspace
2&a contest of speed&41&n&marathon race\\\tablevspace
3&people who are believed to belong to the same genetic stock&10&n&elf race\\\tablevspace
6&a canal for a current of water&4&n&mill race\\\lspbottomrule
\end{longtable}
\subsection{rate     (249 compound types)}
Rated compound: \emph{interest rate}
\vspace*{-.2cm}

\begin{longtable}{c>{\raggedright\arraybackslash}p{5cm}rc>{\raggedright\arraybackslash}p{2cm}}\lsptoprule
{\small wnSense}&WordNet description&types&class&example\\\midrule
1&a magnitude or frequency relative to a time unit&82&n&acceleration rate\\\tablevspace
2&amount of a charge or payment relative to some basis&70&n&tax rate\\\tablevspace
4&a quantity or amount or measure considered as a proportion of another quantity or amount or measure&97&n&suicide rate\\\lspbottomrule
\end{longtable}
% vspace*{1em}

\noindent
Notes:\\
\emph{Savings rate} is the standard realization of the 2 lemmas, \emph{saving rate} only occurs once in the BNC. WordNet sense 3, ('the relative speed of progress or change') deemed to close to sense 1, no attempt at a differentiation was made.
% - saving/s rate: note consolidated, because plural also with different usages
% - wnSense 3 and 1 not differentiated
% - did not specifically check for \textsc{be} usages except in really unclear cases like headline/plus rate
% - consistently used \textsc{for}/1 for rates as processes over time, wnSense2 for rate as actual or calculated amount of money, and wnSense4 for all rates that are based on a stable proportion relative to a whole

\subsection{reaction  (30 compound types)}
Rated compound: \emph{chain reaction}

\vspace*{1ex}

\noindent
\begin{longtable}{c>{\raggedright\arraybackslash}p{5cm}rc>{\raggedright\arraybackslash}p{2cm}}\lsptoprule
{\small wnSense}&WordNet description&types&class&example\\\midrule
1&(chemistry) a process in which one or more substances are changed into others&10&n&fusion reaction\\\tablevspace
3&a bodily process occurring due to the effect of some antecedent stimulus or agent&8&n&anger reaction\\\tablevspace
4&(mechanics) the equal and opposite force that is produced when any force is applied to a body&1&n&torque reaction\\\tablevspace
5&a response that reveals a person's feelings or attitude&11&n&staff reaction\\\lspbottomrule
\end{longtable}

% ** TODO reaction-n2
% *** reaction-n2-FamAna
% - much chemical stuff which I needed to look up, classification still difficult
% - lightning reaction as \textsc{be} and wnSense 3

\subsection{ring      (56 compound types)}
Rated compound: \emph{brass ring}

\vspace*{1ex}

\noindent
\begin{longtable}{c>{\raggedright\arraybackslash}p{5cm}rc>{\raggedright\arraybackslash}p{2cm}}\lsptoprule
{\small wnSense}&WordNet description&types&class&example\\\midrule
2&a toroidal shape&5&n&tree ring\\\tablevspace
3&a rigid circular band of metal or wood or other material used for holding or fastening or hanging or pulling&19&n&brass ring\\\tablevspace
4&(chemistry) a chain of atoms in a molecule that forms a closed loop&2&n&benzene ring\\\tablevspace
5&an association of criminals&2&n&spy ring\\\tablevspace
7&a platform usually marked off by ropes in which contestants box or wrestle&13&n&wrestling ring\\\tablevspace
8&jewelry consisting of a circlet of precious metal (often set with jewels) worn on the finger&14&n&engagement ring\\\tablevspace
10&{}[Wagner's Ring]&1&n&Decca ring\\\lspbottomrule
\end{longtable}
% vspace*{1em}

\noindent
Notes:\\
Added sense 10 for Richard Wagner's \emph{Der Ring des Nibelungen}. WordNet sense 7 also used for terms like \emph{sale ring}, that is, a (circular) enclosure where sales take place etc.
% - wnSense 7 extended for practice ring for horses, and sale ring (which seems to be an auction term for the ring (= hall?) where the sales take place)
% - wnSense 4 used for number ring
% - new wnSense 10 for Wagner's ring
% . wnSense 2 also for machinery ring

\subsection{room      (292 compound types)}
Rated compound: \emph{engine room}

\vspace*{1ex}

\noindent
\begin{longtable}{c>{\raggedright\arraybackslash}p{5cm}rc>{\raggedright\arraybackslash}p{2cm}}\lsptoprule
{\small wnSense}&WordNet description&types&class&example\\\midrule
1&an area within a building enclosed by walls and floor and ceiling&289&n&utility room\\\tablevspace
2&space for movement&3&n&leg room\\\lspbottomrule
\end{longtable}
% vspace*{1em}

\noindent
Notes:\\
WordNet sense 2 has \emph{elbow room} as collocate. 
% ** TODO room-n2
% *** clean-files
% - now with dining and dinning room, left both in (see Marshal plan)
% *** room-n2-FamAna
% - n1 sg/pl to sg: barack/s room, billard/s room, weight/s room, twin/s room [sg more frequent]
% - n1 sg/pl to pl BUT pl more frequent: sale/s room
% - consolidated dinning/dining room to the later, way more frequent, dinning apparently mispelled in spoken portion of the BNC
% - labor/labour room consolidated to British spelling, because more frequent in BNC [labor room more frequent in Cyrus corpus]
% - downstairs/upstairs room coded as IN, answering the question where is the room

\pagebreak[4]
\subsection{run       (56 compound types)}
Rated compound: \emph{rat run}

\vspace*{1ex}

\noindent
\begin{longtable}{c>{\raggedright\arraybackslash}p{5cm}rc>{\raggedright\arraybackslash}p{2cm}}\lsptoprule
{\small wnSense}&WordNet description&types&class&example\\\midrule
1&a score in baseball made by a runner touching all four bases safely&1&n&home run\\\tablevspace
2&the act of testing something&4&n&measure\-ment run\\\tablevspace
3&a race run on foot&24&n&charity run\\\tablevspace
6&a regular trip&9&n&milk run\\\tablevspace
8&the continuous period of time during which something (a machine or a factory) operates or continues in operation&7&n&computer run\\\tablevspace
10&the production achieved during a continuous period of operation (of a machine or factory etc.)&2&n&print run\\\tablevspace
11&a small stream&3&n&gutter run\\\tablevspace
14&{}the pouring forth of a fluid&1&n&pot run\\\tablevspace
15&an unbroken chronological sequence&3&n&stage run\\\tablevspace
17&{}[An (often roofless) enclosure in which a (small) domestic animal may range freely.]&2&n&chicken run\\\lspbottomrule
\end{longtable}
% vspace*{1em}

\pagebreak[4]
\noindent
Notes:\\
Added sense 17, pointer is a verbatim copy of OED run n.2, 15.b. \emph{Pot run} as WordNet Sense 14 is a misclassification, but its sense remains a solitaire (probably OED pot n.2 35. a. gen. An extent in length; a continuous stretch of something).
% ** run-n2
% - added wnSense17 for chicken run etc.
% - some wnSense only with few types
% - not all senses used, some difference difficult

 


\subsection{runner    (10 compound types)}
Rated compound: \emph{front runner}

\vspace*{1ex}

\noindent
\begin{longtable}{c>{\raggedright\arraybackslash}p{5cm}rc>{\raggedright\arraybackslash}p{2cm}}\lsptoprule
{\small wnSense}&WordNet description&types&class&example\\\midrule
1&someone who imports or exports without paying duties&4&n&drug runner\\\tablevspace
6&a trained athlete who competes in foot races&4&n&marathon runner\\\tablevspace
9&device consisting of the parts on which something can slide along&2&n&window runner\\\lspbottomrule
\end{longtable}
\subsection{science   (14 compound types)}
Rated compound: \emph{rocket science}

\vspace*{1ex}

\noindent
\begin{longtable}{c>{\raggedright\arraybackslash}p{5cm}rc>{\raggedright\arraybackslash}p{2cm}}\lsptoprule
{\small wnSense}&WordNet description&types&class&example\\\midrule
1&a particular branch of scientific knowledge&14&n&defence science\\\lspbottomrule
\end{longtable}
% - enlightenment science as \textsc{have2}, parallel to UK science

\pagebreak[4]

\subsection{screen   (54 compound types)}
Rated compound: \emph{silver screen}

\vspace*{0ex}

\noindent
\begin{longtable}{c>{\raggedright\arraybackslash}p{5cm}rc>{\raggedright\arraybackslash}p{2cm}}\lsptoprule
{\small wnSense}&WordNet description&types&class&example\\\midrule
1&a white or silvered surface where pictures can be projected for viewing&4&n&cinema screen\\\tablevspace
2&a protective covering that keeps things out or hinders sight&1&n&wind screen\\\tablevspace
3&the display that is electronically created on the surface of the large end of a cathode-ray tube&37&n&radar screen\\\tablevspace
4&a covering that serves to conceal or shelter something&2&n&Stevenson screen\\\tablevspace
5&a protective covering consisting of netting; can be mounted in a frame&2&n&insect screen\\\tablevspace
7&a strainer for separating lumps from powdered material or grading particles&2&n&security screen\\\tablevspace
8&partition consisting of a decorative frame or panel that serves to divide a space&6&n&silk screen\\\lspbottomrule
\end{longtable}
\pagebreak[4]
\subsection{service   (240 compound types)}
Rated compound: \emph{public service}

\vspace*{1ex}

\noindent
\begin{longtable}{c>{\raggedright\arraybackslash}p{5cm}rc>{\raggedright\arraybackslash}p{2cm}}\lsptoprule
{\small wnSense}&WordNet description&types&class&example\\\midrule
1&work done by one person or group that benefits another&199&n&reservation service\\\tablevspace
3&the act of public worship following prescribed rules&29&n&funeral service\\\tablevspace
4&a company or agency that performs a public service; subject to government regulation&7&n&London service\\\tablevspace
6&a force that is a branch of the armed forces&2&n&field service\\\tablevspace
9&tableware consisting of a complete set of articles (silver or dishware) for use at table&1&n&tea service\\\tablevspace
12&(sports) a stroke that puts the ball in play&1&n&opening service\\\tablevspace
13&the performance of duties by a waiter or servant&1&n&court service\\\lspbottomrule
\end{longtable}
% vspace*{1em}

\noindent
Notes:\\
This group often contained combinations that in principle could be used and hence classified in a number of ways, e.g. \emph{student service} could be \textsc{for/from/}\\\textsc{have2/use}. Went with plausibility, and hence mostly \textsc{for}.
% ** TODO service-n2-check
% - used service wnsense4 + \textsc{for} for railway services
% ** service-n2
% - lumped together wnSense1 and 4 -> 1 SUPERSEEDED
% - also: wn14/15 -> 1 SUPERSEEDED
% - also often: student service wn1\textsc{for}/FROM/\textsc{have2}/\textsc{use} or even wn3\textsc{for} ->
% went mostly for \textsc{for}, but for e.g. cadet service for \textsc{use} 
% - or car service could be one for cars, or one that uses cars, even if
% we mean the same thing
% - in general: vast majority for wnSense 1, but quite a few church
% services, and tea and military
% - why is public service not in the data?

\pagebreak[4]
\subsection{sheet     (79 compound types)}
Rated compound: \emph{cheat sheet}

\vspace*{1ex}

\noindent
\begin{longtable}{c>{\raggedright\arraybackslash}p{5cm}rc>{\raggedright\arraybackslash}p{2cm}}\lsptoprule
{\small wnSense}&WordNet description&types&class&example\\\midrule
1&any broad thin expanse or surface&3&n&ice sheet\\\tablevspace
2&paper used for writing or printing&55&n&score sheet\\\tablevspace
3&bed linen consisting of a large rectangular piece of cotton or linen cloth; used in pairs&3&n&summer sheet\\\tablevspace
5&newspaper with half-size pages&2&n&scandal sheet\\\tablevspace
6&a flat artifact that is thin relative to its length and width&9&n&glass sheet\\\tablevspace
7&(nautical) a line (rope or chain) that regulates the angle at which a sail is set in relation to the wind&2&n&jib sheet\\\tablevspace
8&a large piece of fabric (usually canvas fabric) by means of which wind is used to propel a sailing vessel&5&n&fly sheet\\\lspbottomrule
\end{longtable}
% vspace*{1em}

\noindent
Notes:\\
WordNet sense 3 has \emph{bed sheet} as collocate. WordNet sense 1/\textsc{have2} used for \emph{cell sheet}, which occurs in the context of a biological text (ASL). WordNet sense 8 not only used for sailing related sheets. WordNet sense 2: this sense is almost fully linked to \textsc{for} (e.g. \emph{cheat/drawing sheet}). In some cases, \textsc{have1} is also possible or even more plausible then \textsc{for}, cf. e.g. \emph{erratum sheet}. For consistency, the coding went always with \textsc{for}. 
% - wnSense3 sheet as in beed sheet: also used for bath sheet
% bath sheet does not occur in final selection!
% - wnSense1 + \textsc{have2} for cell sheet: comes from biological usage of cell
% - wnSense8 not only used for sails
% [HIER WEITER: final check!]
% *** TODO final
% - added cheat sheet, which is only in Reddy, not in Cyrus
% - consistently coded wnSense2 as \textsc{for}, also for some \textsc{have1} seems also fine if not more plausible (e.g. erratum sheet)

\subsection{shift     (17 compound types)}
Rated compound: \emph{graveyard shift}

\vspace*{1ex}

\noindent
\begin{longtable}{c>{\raggedright\arraybackslash}p{5cm}rc>{\raggedright\arraybackslash}p{2cm}}\lsptoprule
{\small wnSense}&WordNet description&types&class&example\\\midrule
1&an event in which something is displaced without rotation&2&n&stick shift\\\tablevspace
2&a qualitative change&6&n&climate shift\\\tablevspace
3&the time period during which you are at work&7&n&evening shift\\\tablevspace
4&the act of changing one thing or position for another&2&n&paradigm shift\\\lspbottomrule
\end{longtable}
\subsection{shirt     (16 compound types)}
Rated compound: \emph{polo shirt}

\vspace*{1ex}

\noindent
\begin{longtable}{c>{\raggedright\arraybackslash}p{5cm}rc>{\raggedright\arraybackslash}p{2cm}}\lsptoprule
{\small wnSense}&WordNet description&types&class&example\\\midrule
1&a garment worn on the upper half of the body&16&n&cotton shirt\\\lspbottomrule
\end{longtable}
% vspace*{1em}

\noindent
Notes:\\
Small group with 3 clearly distinguishable relations (\textsc{be}, \textsc{for}, \textsc{make2}).
% ** shirt-n2
% - small group with 3 strong relations -> NICE!
\pagebreak[4]

\subsection{site      (148 compound types)}
Rated compound: \emph{web site}

\vspace*{-.2cm}

\noindent
\begin{longtable}{c>{\raggedright\arraybackslash}p{5cm}rc>{\raggedright\arraybackslash}p{2cm}}\lsptoprule
{\small wnSense}&WordNet description&types&class&example\\\midrule
1&the piece of land on which something is located (or is to be located)&138&n&stadium site\\\tablevspace
2&physical position in relation to the surroundings&5&n&attachment site\\\tablevspace
3&a computer connected to the internet that maintains a series of web pages on the World Wide Web&5&n&start site\\\lspbottomrule
\end{longtable}
% vspace*{1em}
\vspace*{-.2cm}

\noindent
Notes:\\
WordNet sense 3 has \emph{web site} as collocate. When site is used a location, it occurs with either \textsc{for} (\emph{golf site}), \textsc{have1} (\emph{accident site}), or \textsc{have2} (\emph{county site}). Tried to
use \textsc{for} for less solid/stable
 things, but this is not fully consistent because a spill site is
 presumably the site of the spill, so \textsc{have1}; other criterion was clear
 designation by somebody (again, quite soft, but see e.g. \emph{colony site});
 apart from the general difficulty, many things could be both
 (take e.g. \emph{explosion site}), or \textsc{for} changed to \textsc{have1} over time. Sites the county
 has are also sites for the county.
% - added web-site to final; added it as wnSense3, which already stands
%   for website
% KWA
% ** site-n2
% - site as location with either \textsc{for} (golf site) or \textsc{have1} (abbey/spill
%  site) or \textsc{have2} (county site); tried to use \textsc{for} for less solid/stable
%  things, but this is not fully consistent because spill site is
%  presumbly the site of the spill, so a have1; other criterium was clear
%  designation by somebody (again, quite soft, but see e.g. colony site);
%  apart from the general difficulty, many things could be both
%  (explosion site), or \textsc{for} changed to \textsc{have1} over time. Sites the county
%  has are also sites for the county.
\vspace*{-.3cm}

\subsection{song      (46 compound types)}
Rated compound: \emph{swan song}
\vspace*{-.4cm}


\noindent
\begin{longtable}{c>{\raggedright\arraybackslash}p{5cm}rc>{\raggedright\arraybackslash}p{2cm}}\lsptoprule
{\small wnSense}&WordNet description&types&class&example\\\midrule
1&a short musical composition with words&41&n&protest song\\\tablevspace
2&a distinctive or characteristic sound&2&n&siren song\\\tablevspace
4&the characteristic sound produced by a bird&3&n&bird song\\\lspbottomrule
\end{longtable}
% vspace*{1em}
\vspace*{-.2cm}

\noindent
WordNet sense 4 has \emph{bird song} as collocate.
% cc song is name for a bird song in BNC
\subsection{spoon    (8 compound types)}
Rated compound: \emph{silver spoon}


\vspace*{-.2cm}

\noindent
\begin{longtable}{c>{\raggedright\arraybackslash}p{5cm}rc>{\raggedright\arraybackslash}p{2cm}}\lsptoprule
{\small wnSense}&WordNet description&types&class&example\\\midrule
1&a piece of cutlery with a shallow bowl-shaped container and a handle; used to stir or serve or take up food&8&n&serving spoon\\\lspbottomrule
\end{longtable}
% ** spoon-n2
% - desert spoon is the name of a plant!
% - so here: silver-spoon is \textsc{make2}, parallel to plastic spoon etc. so
%  actually one would need the wnSenses of n1 and n2

\subsection{station   (98 compound types)}
Rated compound: \emph{radio station}

\vspace*{-.2cm}

\noindent
\begin{longtable}{c>{\raggedright\arraybackslash}p{5cm}rc>{\raggedright\arraybackslash}p{2cm}}\lsptoprule
{\small wnSense}&WordNet description&types&class&example\\\midrule
1&a facility equipped with special equipment and personnel for a particular purpose&93&n&petrol station\\\tablevspace
4&the position where someone (as a guard or sentry) stands or is assigned to stand&5&n&valley station\\\lspbottomrule
\end{longtable}
\subsection{student   (39 compound types)}
Rated compound: \emph{graduate student}
\vspace*{-.2cm}


\noindent
\begin{longtable}{c>{\raggedright\arraybackslash}p{5cm}rc>{\raggedright\arraybackslash}p{2cm}}\lsptoprule
{\small wnSense}&WordNet description&types&class&example\\\midrule
1&a learner who is enrolled in an educational institution&39&n&divinity student\\\lspbottomrule
\end{longtable}
% vspace*{1em}
\vspace*{-.2cm}

\noindent
Notes:\\
If first part is a proper name, \textsc{have2} was used for places of learning (\emph{Cornell student}), IN for general locations (\emph{Beijing student}).
% - proper names/place names: cornell student etc as \textsc{have2}, beijing student etc. as IN (criterion specific place of learning whether general location)
% *** -final-2015-clean
% - doublette post-graduate/postgraduate: again, no normalization at this point!

\pagebreak[4]
\subsection{study  (77 compound types)}
Rated compound: \emph{case study}

\vspace*{-.4cm}

\noindent
\begin{longtable}{c>{\raggedright\arraybackslash}p{5cm}rc>{\raggedright\arraybackslash}p{2cm}}\lsptoprule
{\small wnSense}&WordNet description&types&class&example\\\midrule
1&a detailed critical inspection&42&n&usability study\\\tablevspace
2&applying the mind to learning and understanding a subject (especially by reading)&4&n&bible study\\\tablevspace
3&a written document describing the findings of some individual or group&17&n&research study\\\tablevspace
6&a branch of knowledge&12&n&computer studies\\\tablevspace
7&preliminary drawing for later elaboration&2&n&period study\\\lspbottomrule
\end{longtable}
% vspace*{1em}

\noindent
Notes:\\
Many combinations allow several classifications; \emph{computer studies} is a subject, but \emph{computer study} is not. The distinction of WordNet senses 1 and 3 was done either via plausibility or with the help of the BNC context. In almost all cases, both readings should in principle be possible. The combination of proper name and study was always checked in its BNC context if it was recognized as a geographical location. As a result, either \textsc{in}, \textsc{from}, or \textsc{about} was used.
% have 
% - wnSense1 (enter a drug study) vs wnSense 3 (Duke studies): in cases of doubt
%   went with plausibility, sometimes with bnc context; in almost all
%   cases, both readings should in principle be possible
% - character study with many different senses, classified into wnSense7 (orig for drawing scetches)
% - Proper name + study always checked in BNC if recognizable a geographical location; either IN,1 Framingham study as IN,1 (heart study done there), or FROM,1 Bangladesh Study, in CONTRAST to city/London study ABOUT 1
% - landscape study as ABOUT,7, also more tokens of ABOUT,1, [but both in only 2 texts] in order to include a clear example of this WordNet sense
% - West study etc: used \textsc{make2} for the person who conducted the study
\vspace*{-.3cm}
\subsection{tank      (51 compound types)}
Rated compound: \emph{think tank}
\vspace*{-.3cm}

\noindent
\begin{longtable}{c>{\raggedright\arraybackslash}p{5cm}rc>{\raggedright\arraybackslash}p{2cm}}\lsptoprule
{\small wnSense}&WordNet description&types&class&example\\\midrule
1&an enclosed armored military vehicle; has a cannon and moves on caterpillar treads&7&n&battle tank\\\tablevspace
2&a large (usually metallic) vessel for holding gases or liquids&44&n&water tank\\\lspbottomrule
\end{longtable}
% vspace*{1em}

\noindent
Notes:\\
WordNet sense 1 has \emph{army tank} as collocate.
% - added in the names 
%  - tank (as warhorse) names added as \textsc{idiom}
% - Reddy: think tank used wnSense storage tank; shift not reflected

\subsection{teacher   (50 compound types)}
Rated compound: \emph{head teacher}

\vspace*{1ex}

\noindent
\begin{longtable}{c>{\raggedright\arraybackslash}p{5cm}rc>{\raggedright\arraybackslash}p{2cm}}\lsptoprule
{\small wnSense}&WordNet description&types&class&example\\\midrule
1&a person whose occupation is teaching&50&n&language teacher\\\lspbottomrule
\end{longtable}
\subsection{tear      (2 compound types)}
Rated compound: \emph{crocodile tears}

\vspace*{1ex}

\noindent
\begin{longtable}{c>{\raggedright\arraybackslash}p{5cm}rc>{\raggedright\arraybackslash}p{2cm}}\lsptoprule
{\small wnSense}&WordNet description&types&class&example\\\midrule
1&a drop of the clear salty saline solution secreted by the lacrimal glands&2&n&salt tear\\\lspbottomrule
\end{longtable}
\subsection{test     (125 compound types)}
Rated compound: \emph{acid test}

\vspace*{1ex}

\noindent
\begin{longtable}{c>{\raggedright\arraybackslash}p{5cm}rc>{\raggedright\arraybackslash}p{2cm}}\lsptoprule
{\small wnSense}&WordNet description&types&class&example\\\midrule
3&a set of questions or exercises evaluating skill or knowledge&15&n&language test\\\tablevspace
4&the act of undergoing testing&18&n&league test\\\tablevspace
5&the act of testing something&91&n&pregnancy test\\\tablevspace
7&{}[proper name]&1&n&River Test\\\lspbottomrule
\end{longtable}
% vspace*{1em}

\noindent
Notes:\\
Added sense 7 for the River Test. WordNet sense 4 used for all rugby/cricket tests.
% - river test: added wnSense 7
% - used wnSense 4 for rugby/cricket test sense
WordNet sense 5: difficult decision between \textsc{for} or \textsc{verb} 5: \emph{merit test} as `test for merit'
or as `test that tests the merit'? Decided by comparison to clear cases, with \emph{drug test} and \emph{cancer test} clearly \textsc{for}, \emph{endurance test} and \emph{connection test} clearly \textsc{verb}; thus, \emph{market test}
is more `a test for a market' and a \emph{performance test} tests the
performance. \textsc{use} seemed sometimes more appropriate, e.g. for \emph{dna
test}, where the dna is determined and it is
also clear that it does not concern the amount of dna or whether
there is any at all.
% - \textsc{for} clear cases:  etc.
% - \textsc{verb} clear cases: endurance test,connection test etc.
% - river test: added wnSense 7
% - used wnSense 4 for rugby/cricket test sense
% - awkward decision between \textsc{for} or \textsc{verb} 5: merit test as test for merit
%   or as test that tests the merit? decided by intuition/feel: market test
%   is more test for a market and performance test tests the
%   performance; other times, \textsc{use} seemed more appropriate, e.g. for dna
%   test, where the dna is determined and not really tested, and it is
%   also clear that it does not concern the amount of dna or whether
%   there is any at all
% - \textsc{for} clear cases: drug test, cancer test etc.
% - \textsc{verb} clear cases: endurance test,connection test etc.
\vspace*{-.3cm}

\subsection{tower     (48 compound types)}
Rated compound: \emph{ivory tower}
\vspace*{-.3cm}


\noindent
\begin{longtable}{c>{\raggedright\arraybackslash}p{5cm}rc>{\raggedright\arraybackslash}p{2cm}}\lsptoprule
{\small wnSense}&WordNet description&types&class&example\\\midrule
1&a structure taller than its diameter; can stand alone or be attached to a larger building&48&n&prison tower\\\lspbottomrule
\end{longtable}
% *** tower-n2-FamAna
% - nothing special, coded all material tower combinations as make2
% - some proper names
\vspace*{-.3cm}

\subsection{train     (54 compound types)}
Rated compound: \emph{gravy train}
\vspace*{-.3cm}

\noindent
\begin{longtable}{c>{\raggedright\arraybackslash}p{5cm}rc>{\raggedright\arraybackslash}p{2cm}}\lsptoprule
{\small wnSense}&WordNet description&types&class&example\\\midrule
1&public transport provided by a line of railway cars coupled together and drawn by a locomotive&49&n&passenger train\\\tablevspace
2&a sequentially ordered set of things or events or ideas in which each successive member is related to the preceding&2&n&pulse train\\\tablevspace
3&a procession (of wagons or mules or camels) traveling together in single file&3&n&wagon train\\\lspbottomrule
\end{longtable}
\vspace*{-.3cm}
\subsection{trip      (41 compound types)}
Rated compound: \emph{guilt trip}

\vspace*{-.3cm}

\noindent
\begin{longtable}{c>{\raggedright\arraybackslash}p{5cm}rc>{\raggedright\arraybackslash}p{2cm}}\lsptoprule
{\small wnSense}&WordNet description&types&class&example\\\midrule
1&a journey for some purpose (usually including the return)&39&n&canoe trip\\\tablevspace
4&an exciting or stimulating experience&2&n&ego trip\\\lspbottomrule
\end{longtable}
% \vspace*{1em}

% \noindent
% - \textsc{use} vs. In: bus trip analyzed as \textsc{use}, and so also canoe trip and likewise caving trip etc.
%  - IN for temporal, e.g. august trip, but also field trip
%  - note also: fishing trip: here \textsc{for}
% - ego trip as \textsc{for},4 (an exciting or stimulating experience) vs. power trip CAUSE2, same for guilt; not ideal wnSense, but seems to be the same for all three in any case
% - locations: US/Mars/Indonesia trip first IN, but then changed to \textsc{for} as the more general term
% - lightning trip as \textsc{be}; falls under Levi's exclusion of LIKE examples

\subsection{user      (39 compound types)}
Rated compound: \emph{end user}

\vspace*{1ex}

\noindent
\begin{longtable}{c>{\raggedright\arraybackslash}p{5cm}rc>{\raggedright\arraybackslash}p{2cm}}\lsptoprule
{\small wnSense}&WordNet description&types&class&example\\\midrule
1&a person who makes use of a thing; someone who uses or employs something&36&n&computer user\\\tablevspace
3&a person who takes drugs&3&n&heroin user\\\lspbottomrule
\end{longtable}
% vspace*{1em}

\noindent
Notes:\\
WordNet sense 3 has \emph{drug user} as a collocate.
\pagebreak[4]
\subsection{value     (152 compound types)}
Rated compound: \emph{face value}

\vspace*{1ex}

\noindent
\begin{longtable}{c>{\raggedright\arraybackslash}p{5cm}rc>{\raggedright\arraybackslash}p{2cm}}\lsptoprule
{\small wnSense}&WordNet description&types&class&example\\\midrule
1&a numerical quantity measured or assigned or computed&68&n&percentage value\\\tablevspace
2&the quality (positive or negative) that renders something desirable or valuable&20&n&entertain\-ment value\\\tablevspace
3&the amount (of money or goods or services) that is considered to be a fair equivalent for something else&59&n&property value\\\tablevspace
5&(music) the relative duration of a musical note&1&n&sound value\\\tablevspace
6&an ideal accepted by some individual or group&4&n&school values\\\lspbottomrule
\end{longtable}
% vspace*{1em}

\noindent
Mistakes:\\ lemma \emph{sale value} occurs with 2 different relational codings, deriving from an earlier distinction between \emph{sale value} as \textsc{for} and \emph{sales value} as \textsc{have2}.
% ** TODO value-n2
% *** value-n2-FamAna
% - stockmarket/surface value etc. as IN,3,
% - kept sale and sales value as \textsc{for}/\textsc{have2}

\subsection{violet    (1 compound types)}
Rated compound: \emph{shrinking violet}

\vspace*{1ex}

\noindent
\begin{longtable}{c>{\raggedright\arraybackslash}p{5cm}rc>{\raggedright\arraybackslash}p{2cm}}\lsptoprule
{\small wnSense}&WordNet description&types&class&example\\\midrule
1&any of numerous low-growing violas with small flowers&1&n&shrinking value\\\lspbottomrule
\end{longtable}
\subsection{wall     (179 compound types)}
Rated compound: \emph{brick wall}

\vspace*{1ex}

\noindent
\begin{longtable}{c>{\raggedright\arraybackslash}p{5cm}rc>{\raggedright\arraybackslash}p{2cm}}\lsptoprule
{\small wnSense}&WordNet description&types&class&example\\\midrule
1&an architectural partition with a height and length greater than its thickness; used to divide or enclose an area or to support another structure&125&n&lavatory wall\\\tablevspace
2&anything that suggests a wall in structure or function or effect&1&n&wave wall\\\tablevspace
3&(anatomy) a layer (a lining or membrane) that encloses a structure&14&n&stomach wall\\\tablevspace
5&a vertical (or almost vertical) smooth rock face (as of a cave or mountain)&10&n&cliff wall\\\tablevspace
6&a layer of material that encloses space&7&n&tyre wall\\\midrule
7&a masonry fence (as around an estate or garden)&10&n&garden wall\\\tablevspace
8&an embankment built around a space for defensive purposes&12&n&compound wall\\\lspbottomrule
\end{longtable}
% vspace*{1em}

\noindent
Notes:\\
Walls that plausibly form types coded as \textsc{for} (\emph{balcony wall}), walls that don't as \textsc{have2} (\emph{house wall}).
% *** wall-n2-FamAna
% - consolidated windows/window wall and foundation/s wall to sg [most frequent]
% - used wnSense 8 also for flood wall (going with defensive purposes)
% - walls that plausibly form types coded as \textsc{for} (bathroom wall/balcony wall), walls that do not, as \textsc{have2} (house wall, bridge wall)

\pagebreak[4]
\subsection{wedding  (8 compound types)}
Rated compound: \emph{diamond wedding}

\vspace*{1ex}

\noindent
\begin{longtable}{c>{\raggedright\arraybackslash}p{5cm}rc>{\raggedright\arraybackslash}p{2cm}}\lsptoprule
{\small wnSense}&WordNet description&types&class&example\\\midrule
1&the social event at which the ceremony of marriage is performed&8&n&church wedding\\\lspbottomrule 
\end{longtable}
% vspace*{1em}

\noindent
Notes:\\
Deemed it impossible to distinguish between WordNet sense 1 and 2 (`the act of marrying; the nuptial ceremony'); used only sense 1.
% ** TODO wedding-n2
% - added Reddy diamond wedding in final
% - did not distinguish the wnSenses (1,2 too close, 3 irrelevant)



%%% Local Variables: 
%%% mode: latex
%%% TeX-master: "notes-on-the-semantic-annotations"
%%% End: 
