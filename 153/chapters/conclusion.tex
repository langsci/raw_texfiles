\chapter{Summary and outlook}
\label{cha:conclusion}

The topic of this work has been the notion of semantic transparency
and its relation to the semantics of compound nouns. The first part
gave an overview of the place of semantic transparency in the analysis of compound
nouns, discussing its role in models of morphological
processing and differentiating it from related notions. After a chapter on the semantic analysis of
complex nominals, this first part closed with a chapter on previous attempts to model semantic transparency. The second part introduced new models
of semantic transparency.
% \noindent
In the following, I first summarize the most important points
of this work, and secondly, point to
some remaining questions and discuss some avenues for further research.

\section{Summary}
\label{sec:summary}

The first 4 chapters established the backdrop for the 2 empirical
chapters to follow. Chapter \ref{cha:semTranPsycho} focused on the
role of semantic transparency in psycholinguistics. First, it explained
the role of semantic transparency in different models of morphological
processing. Secondly, it provided an overview of the measures used in
psycholinguistics to assess the semantic transparency of compound
nouns. Thirdly, the results and findings of studies involving semantic
transparency as an independent variable were presented. Chapter
\ref{cha:theo} summarized the discussion of semantic
transparency in works on anaphor resolution and compound stress,
introduced a number of related terms and discussed transparency in
other domains. Chapter \ref{cha:semantics} gave an overview of
approaches to compound semantics, including discussion of work on the
semantics of phrasal constructions in formal semantics. Chapter
\ref{cha:modPrevious} provided an introduction to distributional
semantics and discussed 3 studies which include distributional
semantics measures in their statistical models of semantic
transparency.

In the 2 chapters of the empirical part, I presented new empirical
work on semantic transparency. Chapter \ref{cha:empirical-1}, building
on \citet{BellandSchaefer:2013}, first introduces models for semantic
transparency that include the meaning shifts of the compounds and
their constituents as well as the semantic relation between
constituents as predictors. I show that after switching from ordinary
least square regression models to mixed effects models, initially
observed effects
for semantic relations disappear. Furthermore, I
argue that the coding of the meaning shifts was missing a
principled basis. Chapter \ref{cha:empirical-2}, building
on \citet{BellandSchaefer:2016}, can be seen as a direct response to
the conclusions drawn from the earlier set of statistical
models. This time, all semantic-based predictors reflected
expectancies drawn from the distribution of the respective features
across the compounds' positional constituent families. In order to
assess these distributions, Melanie Bell and I created a large
compound database which I annotated for semantic relations and WordNet
meanings of the compound constituents. The resulting models show that
the semantic predictors representing the N1 and N2 families do not
behave similarly. Furthermore, the distribution of semantic relations
across the N1 families emerges as a stable, positive correlate of N1,
N2, and whole compound transparency. In contrast, the only effects
associated with the synset distribution were negative correlations
which, in the case of compound transparency, led to extensive
discussion of the nature of this variable. 

\section{Outlook}
\label{sec:outlook}

Semantic transparency is a rich and fascinating topic, and the
research presented in this work opens up many new avenues of
investigation. Here, I want to highlight 4 pathways which look
particularly promising to follow.

\begin{enumerate}
\item {Using the annotated compound database} \label{using-annotations}\\
  For the models presented in Chapter \ref{cha:empirical-2}, I have
  annotated a large compound database. This database contains compound
  families of very different sizes, ranging from 1 member to 363
  members. Compound families of comparable size sometimes differ
  considerably in the number and types of semantic relations and
  synsets used. A further variable that shows massive variation but
  was not considered in the studies presented here is the distribution of
  token frequencies across the members of a constituent family. In
  future studies, this database can be used to investigate the
  behavior of different compounds drawn from the same compound family
  against each other but also against the behavior of compound types
  drawn from groups that differ along the dimensions just
  mentioned. This will ideally lead to a much better understanding of
  the role of the distribution of the semantic features across the
  groups, but will also allow one to compare whether there are certain
  cut-off points, e.g. driven by family size or compound token
  frequency, that make certain compound types more important for the
  compositional processes involved in language use. In addition, drawing balanced
  samples from this database allows one to compare between
  frequent and less frequent compound types. Recall that the current
  investigation focused on transparency measures for high frequency compounds only, since one
  of the original selection criteria was high frequency.
\item Comparing the different measures of semantic transparency
  experimentally\\
  In the discussion of different ways to establish semantic
  transparency in Chapter \ref{cha:semTranPsycho}, I pointed out that
  the differences in establishing semantic transparency for
  experimental items make a comparison of the experimental results
  difficult. One simply cannot know whether the same properties were
  measured in every case. An experimental comparison of responses to
  different ways of asking for semantic transparency would allow one to
  establish which measurements yield similar results, and
  can therefore be assumed to establish similar variables for
  experimental purposes.
\item Synchronizing the measures across tasks and approaches\\
%  \begin{sloppypar}
While one can find some measures on compounds from the
Reddy et al. dataset in other works on compounds,
e.g. \citet{Juhasz:2015}, or in databases of
psycholinguistic measures like the English Lexicon Pro\-ject
\citep{Balotaetal:2007}, these items are too few in number to allow
systematic comparison. In order to gain better insight into the nature
of the semantic transparency judgments, getting psycholinguistic
measures with better understood features (e.g. lexical decision
ratings) on the items in the Reddy et al. dataset would be very
helpful. Besides further psycholinguistic measures, models including
distributional semantics and information theoretic measures together
with the semantic predictors introduced in this work can lead to a
deeper understanding of the ways in which semantic transparency and
semantic aspects of meaning are reflected in the former measures.
%  \end{sloppypar}
\item Obtaining online measures\\
When less frequent compound types are used, it is likely that an
off-line method like Likert scale ratings will not suffice to draw out
nuances between compounds at the high end of transparency. To
understand the influence of the distribution of semantic factors on
the processing of less frequent, and therefore presumably not
lexicalized compounds, data obtained by using an online task like
eye-movement measurements allows one a more fine-grained look at the underlying processes. \is{eye tracking}
Such data would have
maximal traction if obtained for a set of compounds that is already selected by
using the annotated compound database as a guide in choosing
the experimental items.
\end{enumerate}




. 

%%% Local Variables: 
%%% mode: latex
%%% TeX-master: "habil-master"
%%% End: 
