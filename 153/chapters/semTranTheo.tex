\chapter[Related phenomena and notions]{Semantic transparency: related phenomena and notions}
\label{cha:theo}

Outside of psycholinguistics, semantic transparency has only played a negligible role in the
discussion of specific linguistic phenomena, e.g. derivational morphology. Two areas where semantic
transparency has received some attention as a possible explanation are
the phenomenon of outbound anaphora and the factors determining stress
assignment in English noun noun compounds. Both are discussed in
the first part of this chapter. In the second part of this chapter, I briefly
discuss a number of other notions that describe phenomena that are closely related to semantic transparency or even partially
or fully overlapping with semantic transparency. At the end of the chapter, I briefly
discuss the notions of phonological and orthographical transparency.

\section{Semantic transparency reflected in other linguistic phenomena}

Semantic transparency is traditionally mentioned in introductions to
morphology, but how it is assessed or what sort of linguistic patterns it is
connected with is rarely discussed. Whether coincidental or not, the
2 areas where semantic transparency has been discussed in more
detail both involve English compounds: the possibility of anaphoric
reference to parts of a compound, and the factors driving
stress assignment.
Both will be discussed in turn.

\subsection{Semantic transparency and outbound anaphora}
\label{sec:outbound_anaphora}

\is{semantic transparency!{anaphora}|(}
Semantic transparency has been discussed as a factor influencing whether parts
of words are accessible as targets for anaphoric reference, or whether
they constitute so-called anaphoric
islands.
\is{anaphoric island|(}
The term \emph{anaphoric island} was introduced in \citet{Postal:1969}
in the discussion of the behavior of the constituents of complex words and of
entities contained in the meaning of words with respect to anaphora,
whether as targets of anaphoric references or as anaphorically referring
expressions. 
\citet{Postal:1969} argued that words,
whether monomorphemic or derived, are anaphoric islands. Being an
anaphoric island means that neither internal constituents of morphologically complex words nor
entities contained in the meaning of a word can serve as antecedents to a following anaphoric element nor can
they themselves refer anaphorically to other elements. \is{anaphoric island|)}
\is{outbound anaphora|(}
The property of serving as an
antecedent for an anaphoric element is discussed under the term outbound anaphora, the
 property of refering anaphorically
under the term of inbound anaphora. Outbound anaphora has been linked
to semantic transparency in \citet{Coulmas:1988}, \citet{Wardetal:1991} and
\citet{Schaefer:2011}. 

% Inbound anaphora means that no part
% of a word can anaphorically refer \textbf{CHECK}. 
A classic set of data that led Postal to introduce the notion of
anaphoric islands is \Next, reproducing (53) from \citet{Postal:1969}.

% \ex.	\a.	Harry was looking for a rack for books$_i$ but he only found racks for very small ones$_i$.
% \b.	* Harry was looking for a book$_i$rack but he only found racks for very
% small ones$_i$.

\ex. \a. Harry was looking for a rack for \textbf{books}$_\text{\textbf{i}}$ but he only found racks for very small \textbf{ones}$_\text{\textbf{i}}$.
\b. *Harry was looking for a \textbf{book}$_\text{\textbf{i}}$rack but he only found racks for very
small \textbf{ones}$_\text{\textbf{i}}$.

While anaphora from \emph{ones} to \emph{books} is easily possible in
\Last[a], it is not possible to refer back to \emph{book} via
\emph{ones} in \Last[b]. In Postal's terminology, \emph{books} within the
phrase \emph{a rack for books} allows
outbound anaphora, but \emph{book} within the compound \emph{bookrack}
does not allow outbound anaphora.

Soon after the publication of Postal's paper, it was observed that his
claim does not hold for all morphologically complex words. Rather, it
was pointed out that there is data that shows different degrees of
acceptability. \citet{LakoffandRoss:1972} illustrated this cline in
acceptability with the data and judgments reproduced in \Next, their
(2b) and (3a-b), where \emph{one} in \Next[a] and \emph{it} in
\Next[b] and \Next[c] are intended to refer to the guitar, which is
contained in the derivation \emph{guitarist}.
% \ex.  
% \a. *A guitarist bought one yesterday
% \b. ?*The guitarist thought that it was a beautiful instrument.
% \c. ?John became a guitarist because he thought that it was a beautiful
% instrument.

\ex.
\a. *A \textbf{guitar}$_{\mathbf{i}}$ist bought \textbf{one}$_{\mathbf{i}}$ yesterday
\b. ?*The \textbf{guitar}$_{\mathbf{i}}$ist thought that \textbf{it}$_{\mathbf{i}}$ was a beautiful instrument.
\c. ?John became a \textbf{guitar}$_{\mathbf{i}}$ist because he thought that \textbf{it}$_{\mathbf{i}}$ was a beautiful
instrument.

\enlargethispage{1\baselineskip}
This cline cannot be explained by Postal’s original proposal, which makes a
categoric difference between islands and non-islands. Other authors offering
counterexamples to Postal’s strong claim include \citet{Douloureux:1971}, \citet{Corum:1973}, \citet{Browne:1974} and \citet{Watt:1975}, whose main claims and accounts are
discussed in \citet{Wardetal:1991}, as well as \citet{Levi:1977}. A representative set of
counterexamples involving English compounds is presented below, first with
anaphoric references to the first element of the compound, cf. \Next, secondly with
anaphoric reference to the second part of the compound, cf. \NNext.

% \ex. \a. Although casual cocaine$_i$ use is down, the number of people using it$_i$
% routinely has increased.
% \b. Patty is a definite Kal Kan$_i$ cat. Every day she waits for it$_i$.
% \c. I was an IRS$_i$-agent for about 24 years. \dots I stopped working for them$_i$.

\ex. \a. Although casual \textbf{cocaine}$_{\mathbf{i}}$ use is down, the number of people using \textbf{it}$_{\mathbf{i}}$
routinely has increased.
\b. Patty is a definite \textbf{Kal Kan}$_{\mathbf{i}}$ cat. Every day she waits for \textbf{it}$_{\mathbf{i}}$.
\c. I was an \textbf{IRS}$_{\mathbf{i}}$-agent for about 24 years. \dots { }I stopped working for \textbf{them}$_{\mathbf{i}}$.

The examples in \Last are from the appendix of \citet{Wardetal:1991}. \emph{Cocaine use} in \Last[a] is a synthetic compound which should disallow anaphoric reference to its 2
constituents. However, the following \emph{it} refers back to the denotation of \emph{cocaine} and not to the denotation of \emph{cocaine use}. Similarly, \emph{it}
in \Last[b] refers back to the denotation of \emph{Kal Kan}, that is, to a specific brand of
catfood, where \emph{Kal Kan} is embedded in a standard endocentric compound, and
finally, in \Last[c], \emph{them} refers back to the IRS, the US
Internal Revenue Service. Another
notable feature of these 3 examples is that in 2 of the 3
compounds the actual anchors for the anaphors are proper names (\emph{Kal Kan} and \emph{IRS}). \is{proper name}\is{outbound anaphora!{proper names and}}
This corresponds to the
distribution of cases involving pronominal
reference to non-heads in the corpus investigated by
\citet{Wardetal:1991}: two-thirds of them are proper names (cf. \citealt[76]{TenHacken:1994}).
 
The data discussed by \citet{Wardetal:1991} is restricted to anaphoric reference to
the first element of the compound, whereas \citet{Levi:1977} presents data showing
that reference to the second element is also possible, cf. \Next, her (17b), (18b), and
(19a).

\ex. \a. State \textbf{taxes}$_{\mathbf{i}}$ were higher than municipal \textbf{ones}$_{\mathbf{i}}$.
\b. Steam \textbf{irons}$_{\mathbf{i}}$ need more maintenance than \textbf{those}$_{\mathbf{i}}$ that iron dry.
\c. Student \textbf{power}$_{\mathbf{i}}$ is insignificant compared to \textbf{that}$_{\mathbf{i}}$ of the Dean.

In \Last[a], \emph{ones} refers back to the denotation of \emph{taxes} and not to the denotation of \emph{state taxes}.
\emph{Those} in \Last[b] refers to the denotation of \emph{irons} and not to
the denotation of \emph{steam irons}. Finally, \emph{that} in \Last[c] refers back to the denotation of \emph{power} and not to the denotation
of \emph{student power}.

\citet{Coulmas:1988} discusses sets of German data that either allow or do not
allow anaphoric reference, cf. \Next and \NNext, his (3--4) and (5--6). \il{German!{anaphoric islands}}

\ex. \a. \gll *\textbf{Atomwaffen}$_{\mathbf{i}}$gegner haben gegen \textbf{ihre}$_{\mathbf{i}}$ Lagerung in
Europa protestiert.\\
nuclear:weapons:opponents have against their storage in Europe
protested\\
Intended: Opponents of nuclear weapons protested against their storage
in Europe.
\b. \gll *Der \textbf{Fuß}$_{\mathbf{i}}$gänger hat sich in \textbf{ihn}$_{\mathbf{i}}$ geschossen.\\
the pedestrian has himself in it shot\\
Intended: The pedestrian has shot himself in his own foot.

\ex. \a. \gll \textbf{Atom}$_{\mathbf{i}}$waffengegner haben immer wieder dagegen protestiert, daß \textbf{solche}$_{\mathbf{i}}$
Waffen in Europa gelagert werden.\\
nuclear:weapons:opponents have always again against protested, that
such weapons in Europe stored will\\
`Time and again, opponents of nuclear weapons have protested\\ against
storing such weapons in Europe.'
\b. \gll Die \textbf{Diamanten}$_{\mathbf{i}}$suche war noch nicht lange unterwegs, da hatten
sie \textbf{ihn}$_{\mathbf{i}}$ schon.\\
the diamond:hunt was yet not long underway, then have they it
already\\
`The diamond hunt had not been on for long when they already found it.'

\citeauthor{Coulmas:1988} also gives some
acceptable English examples, for example \ref{ex:coulmas_12_and_15}, his (12) and
(15b). 

\ex. \label{ex:coulmas_12_and_15}
\a. The \textbf{rocket}$_{\mathbf{i}}$ launch had to be delayed because of some unexpected problems
with \textbf{its}$_{\mathbf{i}}$ fuel tanks.
 \b. The \textbf{river}$_{\mathbf{i}}$bank was damaged when \textbf{it}$_{\mathbf{i}}$ overflowed after three days of
heavy rain.
% \\ =  (15b) in \citet{Coulmas:1988}

It is not the rocket launch but the rocket that has problems,
likewise, it is not the riverbank but the river that overflows. \citeauthor{Coulmas:1988} 
hypothesizes that the ability of outbound anaphora is proportional to the
compositionality of words and to the correspondence between formal and
semantic compositionality \citeyearpar[321]{Coulmas:1988}.\is{compositionality!{correspondence of formal and semantic}}

\citet{Wardetal:1991} take semantic transparency to be a key factor in
the facilitation of outbound anaphora. They argue that anaphoric
reference to parts of a word is only possible if the individual
constituents invoke individual discourse entities. Whether or not the
individual constituents of a compound invoke individual discourse
entities does, in turn, depend on whether they are semantically
transparent or not. In case of semantic opacity, they assume that ``[morphologically complex words]
% ``As a result, some morphologically complex words have
% become semantically opaque in that they 
can no longer be straightforwardly interpreted on the basis of their
component parts''  \citet[454]{Wardetal:1991}.\is{semantic transparency!{in Ward et al.}} 
Once a word is
semantically opaque, so they argue, outbound anaphora is inhibited, as
their example in \Next shows, where \# marks pragmatic deviance.

\ex. Fritz is a \textbf{cow}$_{\mathbf{i}}$boy. \#He says \textbf{they}$_{\mathbf{i}}$ can be difficult to look after.
% = (23) a. in \citet[454]{Wardetal:1991}
% b.
% Roberta is an ordained Lutheran minister. #She's currently
% studying the early years of his life.
% c. #Ironically, Paula had a Caesarean whil

While \citet[455]{Wardetal:1991} assume that the distinction
between transparent and opaque words is gradient, they provide no
measure for this gradience. 

Considering the above-mentioned preponderance of proper names as
anchors for anaphora in the
corpus used by \citet{Wardetal:1991} and their own analysis that
anaphoric reference requires the constituents to invoke individual
discourse entities, one could also argue that the very fact that
reference to individual entities is the core function of proper
names makes them more likely to actually lead to the activation of
the corresponding referents, even when embedded in compounds. \is{proper name}\is{outbound anaphora!{proper names and}}

In \citet{Schaefer:2011}, I discuss German adjective noun constructions and attempt to
provide a measure for different degrees of semantic
transparency.\il{German!{anaphoric islands}} 
Adapting paraphrase tests proposed in
\citet{Fahim:1977}, I distinguish between 5 different
classes of AN compounds that I hold to be semantically transparent
to a decreasing degree. The 5 classes are briefly illustrated in
\Next, ranked from most to least transparent. \is{semantic transparency!{paraphrase-based categorization}}

\ex. \a. Endocentric pattern A: [AN$_{\text{N}}$] = [AN]$_{\text{NP}}$:\\
\emph{Rotwein} = \emph{roter Wein} `red wine'
\b. Endocentric pattern B: [AN$_{\text{N}}$] $\approx$ [AN]$_{\text{NP}}$:\\
\emph{Großstadt} $\approx$  \emph{große Stadt} `big city'
\c. Endocentric pattern C : [AN$_{\text{N}}$] $\not=$ [AN]$_{\text{NP}}$:\\
\emph{Grünspecht} $\not=$ \emph{grüner Specht} `Green woodpecker'
\d. Exocentric pattern A:  [AN$_{\text{N}}$](x) $\rightarrow$ [A](x):\\
\emph{Ein Dummkopf ist dumm.} `A stupid.head is stupid.'
\e. Exocentric pattern B: [AN$_{\text{N}}$](x) $\not\rightarrow$ [A](x):\\
\emph{Ein Rotkelchen ist nicht rot.} `A red.throat (a robin) is not red.'

While the patterns as presented here are dependent on the semantics of the
corresponding phrases, the proposed 5-fold distinction is more or
less a mixture of criteria involving institutionalization, internal
semantic structure and metonymic
shifts. % (\textbf{ADD REF})

In \citet{Schaefer:2011}, I also tried to provide empirical support for my
classification by doing a small corpus study. However, I only found
attested examples of patterns involving outbound anaphora
for the first 2 endocentric classes.
\is{outbound anaphora|)}
\is{semantic transparency!{anaphora}|)}

% \textbf{Check \citet{Levi:1977}}% and \citet{Levi:1978} on inbound anaphora!} 

\subsection{Semantic transparency and compound stress}
\label{sec:semTran-Stress}
\is{stress pattern!{semantic transparency and}|(} 
\is{semantic transparency!{stress pattern and}|(} 
Semantic properties of compounds correlate with the stress patterns
found in English compounds. \citet{Plagetal:2008} show that the
categories of compound constituents as well as the semantic relations
between compound constituents are
highly predictive of compound stress. 
\citet{Bell:2012} also finds certain semantic relations to be highly
predictive of noun noun stress patterns, in particular, of right prominence. \citet[49--51]{Bell:2012} hypothesizes that the factor semantic transparency
might be a higher order feature that unites this group of
relations (cf. also the generalization in \citealt[6]{Giegerich:2009}
that ``end-stress favours transparent over non-transparent semantics"). How does \citet{Bell:2012} operationalize semantic
transparency? In a first step, \citet{Bell:2012} equates semantic transparency with
semantic compositionality.\is{semantic transparency!{as semantic compositionality}} In the context of her work, which
focuses exclusively on noun noun constructions, she also refers to noun noun constructions
with compositional meanings as constructions with \isi{phrase-like
semantics}, referencing an old tradition within the compound noun
community starting with Sweet's \citeyearpar[288]{Sweet:1891} observation
on nouns with ``even stress'' that ``the logical relation between the elements of the compound
resembles that between the elements of a free group, especially when
the first element is felt to be equivalent to an adjective". Semantic compositionality is then
operationalized as shown in \Next, her (3.10):
\is{compositionality!{via entailments}}

\ex. A NN is semantically compositional when its meaning entails one of a
small number of relations between N1 and N2, which can usually also
be expressed phrasally, and can be described schematically. \is{semantic transparency!{paraphrase-based categorization}}
% \\ (3.10) in \citet[86]{Bell:2012} % Quote is correct

The entailed relations that were considered were the following 4
(cf. Table 3.4, \citealt[65]{Bell:2012}): (1) N2 \textsc{is (made of)} N1, (2) N2 \textsc{is at/on/in} N1,
(3) N1 \textsc{has} N2, and (4) NN \textsc{is name}.
The entailment criterion can best be illustrated by a concrete example
from the N2 \textsc{is at/on/in} N1 group. A(n) NN was classified as belonging to
this dataset if the statement in \Next, applied to the NN of interest,
resulted in a true statement.\is{semantic relations!entailment criterion}

\ex. X is (an) NN entails X is (an) N2 and X is at/in/on N1\\
\citet[69]{Bell:2012}

\citet[69]{Bell:2012} gives \emph{London school} and \emph{Monday
  morning} as examples that fulfill this condition. While this so far
does not look different from other relational classifications, Bell
shows that the
entailment criterion can be used for further, non-trivial
distinctions. For the at/in/on group, it is used for the distinction between ``NNs where N1 simply gives the
location of NN, and those where N1 defines a type of NN, irrespective
of its location" \citet[69]{Bell:2012}. Thus, \citet[69]{Bell:2012}
excludes an item like \emph{door bell} from this group, since, according to her
argumentation, a door bell remains a door bell independent of its
actual location, whereas \emph{office ceiling} was judged to entail
that the ceiling is located in an office. Note that while this is an
important difference, discussing this difference under the term of
entailment is somewhat unfortunate, as what seems to have been judged
here are typical relations between the referents of the compound and
the referents of its constituents that need not always hold. On top of
that, speakers might vary in their judgments on this typicality. \citet[447]{Baueretal:2013}, discussing
the issue of individual variation in relation to the terminological
distinction between ascriptive and associative interpretations made in
\citet{Giegerich:2009}, remark:
``linguists may disagree as to whether \emph{door} in \emph{doorknob}
is associative (the knob is associated with a door) or ascriptive (the
knob has the property of being on a door) in nature".\is{compound!{ascriptive vs. associative interpretation}}

\citet{Bell:2012} tested her hypothesis by building models for NN
prominence, and all 4 relations emerged as significant predictors of
rightward stress in a logistic regression model that also included
other predictors usually associated with rightward stress. Interestingly,
\citet{Bell:2012} also compared her regression model with several
rule-based models, and discovered that the best rule-based model was
one that only made
use of the 4 semantic relations discussed above; its results were only
slightly worse than those of the regression model
(cf. \citealt{Bell:2012}, Table 3.11).
\is{stress pattern!{semantic transparency and}|)} 
\is{semantic transparency!{stress pattern and}|)} 

\subsection{Conclusion: semantic transparency and other phenomena}
\label{sec:ST-and-other}

As the extent of the discussion above has shown, semantic transparency
has rarely been used in concrete attempts to explain observed language
patterns. While the relationship between semantic transparency and
outbound anaphora sounds very plausible, the value and the extent to
which the observations are generalizable is unclear, as none of the studies used
empirical measures for semantic transparency. The approach by
\citet{Bell:2012} used a clear operationalization of semantic
transparency; however, it is not clear to what extent this categorization
actually captures the same notion of semantic transparency that was
used in the psycholinguistic operationalizations described in Chapter \ref{cha:semTranPsycho}.

\section[Other measures and notions]{Other measures and notions relating to semantic transparency}
\label{sec:other_measures_and_notions}

\subsection{Quantitative measures}
\label{sec:quantitative_measures}

There are a number of quantitative measures that, to varying degrees,
target semantic aspects of complex nominals. I will introduce these
in detail in Chapter \ref{cha:modPrevious}, which contains sections on
informativity related measures and on measures based on the word space
model (such as Latent Semantic Analysis). 
% One measure not explicitly
% discussed there is semantic overlap:


% \subsubsection{Family size}
% \label{sec:family_size}

% Family size is a concept introduced in
% \citet[121]{SchreuderandBaayen:1997}. The morphological family of a word denotes
% the set of all words that are either derivations from that word or that are
% compounds containing that word. The family size of a word is the number of
% different words in the morphological family, excluding the word itself. The
% cumulative family frequency are the summed token frequencies of all words in a
% words family, again exluding the frequency of the word itself. 





\subsection{Semantic overlap}
\label{sec:sem_over}

  \citet{Odegardetal:2005}, studying effects on memory and recollection, consider the \isi{semantic overlap} between compound
  triplets consisting of 2 parents, e.g. \emph{handball} and \emph{shotgun},
  and a recombined child, e.g. \emph{handgun}. For this particular example,
  they see a high semantic similarity between \emph{shotgun} and
  \emph{handgun}, and considerable semantic overlap between the meanings of
  \emph{hand} in both compounds. As an example with little similarity and
  overlap they give the 2 parents \emph{blackmail} and \emph{jailbird}
  and the recombined child
  \emph{blackbird}. For their experiments, they manually constructed a set of
  40 compound word triplets which was then rated by 24 participants for ``the
  level of similarity shared between the meaning of a parent word and its
  conjunction (e.g., blackmail to blackbird)'' \citep[419]{Odegardetal:2005} on
  a 5-point Likert-type scale (ranging from 1, ``not similar
  whatsoever", to 5, ``highly similar"). % blackmail to blackbird not in italics in the original

\citet{Ledingetal:2007} used a large-scale questionnaire study (185
participants) to establish, besides familiarity and memorability, semantic overlap
measures for 96 \isi{compound triplets}.



\subsection{Compositionality and literality}

There are 2 notions that are also often discussed together with the
notion of semantic transparency, namely the notion of compositionality and
the notion of literality. I will discuss these 2 notions in
turn.

\subsubsection{Compositionality}
\label{sec:compositionality}
\is{compositionality|(}
In lieu of the term semantic transparency, some psycholinguistic and linguistic studies
use the term
`semantic compositionality' to refer to similar phenomena.
This usage of the term also occurs in
some studies within distributional semantics,
e.g. \citet{Reddyetal:2011}, whose approach to establishing
compositionality of compound nouns was already
described in Chapter \ref{cha:semTranPsycho}, Section
\ref{sec:direct_measures}. In formal semantics, compositionality is usually
discussed in connection with the compositionality principle, cf. \Next for
the formulation of this principle in \citet[281]{Partee:1984}. 

\ex. The meaning of an expression is a function of the meanings of its parts
and of the way they are syntactically combined.
% \citet[281]{Partee:1984}. 

\is{underspecification!{compositionality and}|(}
\is{compositionality!{underspecification and}|(}
An expression is compositional if its meaning can be computed in accordance
with this principle. The problem is that it is very unclear which formalisms
do and which do not fit under this principle. This in turn is related to
questions pertaining to the exact meaning of `meaning' in \Last.
Thus, if we accept underspecified semantic
representations, and if we distinguish between a proper semantic and a proper
pragmatic level of interpretation, then almost all meanings are compositional. For example,
taking \emph{milkman} again, one can argue that  its semantic meaning is composed by combining
the 2 predicates MILK(x) and MAN(x) with the help of the
underspecified template in \Next, where R represents an underspecified relation (note that it is not relevant to the point illustrated here when and how this relation is eventually existentially bound).

\ex. $\lambda$B $\lambda$A $\lambda$y $\lambda$x [A(x) \& R(x,y)
\& B(y)] \label{ex:underspec}

This yields \Next, which, up to this point, is technically semantically
fully compositional.

\ex. $\lambda$y $\lambda$x [MILK(x) \& R(x,y)
\& MAN(y)] \label{ex:underspecConcrete}

In order to arrive at the final, correct interpretation of
\emph{milkman}, the relational parameter needs to
be specified. For the appropriate specification, access
to pragmatic information is needed, but this could be argued to lie outside of the realm
of semantics proper.
On this view, semantic transparency could easily be linked to 
compositionality. One approach would be to argue that semantic transparency correlates with the amount of
additional pragmatic input that is involved in arriving at the pragmatic meaning of a
complex expression whose semantic meaning has been calculated via the
principle of compositionality. 
\is{underspecification!{compositionality and}|)}
\is{compositionality!{underspecification and}|)}

In contrast to such a view, some authors have argued for a clear distinction
between transparency and compositionality. \citet[550]{Sandra:1990}, for
example, argues that transparency ``refers to the relationship between compound and constituent meanings, the latter [compositionality] refers to the possibility of determining the whole-word
meaning from the constituent meanings." This view is echoed in the final
paragraph in \citet{Zwitserlood:1994}:
\begin{quotation}
[S]emantic
transparency is not the same as compositionality. Although
the semantic relation between transparent compounds and their constituents
might be easy to establish, the meaning of the compound as a whole is often
more than the meaning of its component words.\\ \citep[366]{Zwitserlood:1994}  
\end{quotation}
\is{compositionality|)}

% \begin{quote}
% ``This might be related to a difference between
% the notions `transparency' and `compositionality'. Whereas the former
% notion refers to the relationship between compound and constituent mean-
% ings, the latter refers to the possibility of determining the whole-word
% meaning from the constituent meanings. The example of the word blackbird
% in the introduction illustrated that transparent words are not necessarily
% compositional. Perhaps only compositional words need no representation in
% the lexicon.'' \citet[550]{Sandra:1990}
 % \end{quote}


\subsubsection{Literality}
\label{sec:literality}
\is{literal meaning|(}
Within computational linguistics, literality is often linked to compositionality.
In Chapter \ref{cha:semTranPsycho}, Section \ref{sec:direct_measures}, I included the study of
\citet{Reddyetal:2011}, who argued for viewing compositionality as
literality. \is{literal meaning!{compositionality and}}
As pointed out there, their way of operationalizing
literality corresponds to the methods others have used to establish
semantic transparency. Others working in computational linguistics who also explicitly link literality and compositionality are for example
\citet{Lin:1999}, \citet{KatzandGiesbrecht:2006}, and
\citet{BiemannandGiesbrecht:2011}
(\citealt{BiemannandGiesbrecht:2011} are very similar to 
\citealt{Reddyetal:2011} in that they annotated phrases for compositionality by asking `How literal is this phrase'). \citet{Lin:1999} presents a method to detect
non-compositional phrases which is based on the assumption ``that
non-compositional phrases have a significantly different mutual information
value than the phrases that are similar to their literal
meanings" \citep[321]{Lin:1999}. However, the exact understanding of
`literality' is often not made very clear. Thus, \citet{Lin:1999} gives
\emph{red tape} vs. the ``compositional phrase'' \emph{economic impact} as a
starting example. Indeed, when one considers the collocation \emph{red
  tape} with its meaning `obstructive official routine or procedure;
time-consuming bureaucracy', one would intuitively judge it to be
less literal than \emph{economic impact}. Note, though, that the
impact in
economic impact can also be argued to be not a literal impact but only
a metaphorical impact, as no physical contact takes place. \citet{Lin:1999} uses the operationalization of
 non-compositionality given in \Next, cf. his (3).
% A0M 246 	Similarly, roundhouse kicks to the face may land with a slightly heavier impact because they are inherently more difficult to control and yet are to be encouraged. 

\ex. A collocation $\alpha$ is non-compositional if there does not exist another
collocation $\beta$ such that (a) $\beta$ is obtained by substituting the head or the
modifier in $\alpha$ with a similar word and (b) there is an overlap between the 95\%
confidence interval of the mutual information values of $\alpha$ and $\beta$.

Thus, the actual criterion is exclusively based on frequencies, and no
independent definition of literal or non-compositional meaning is
given. Considering the contrast between \emph{red tape} and
\emph{economic impact} in the light of the condition in \Last,
one might hypothesize that the decisive difference between the 2
combinations lies in the fact that \emph{impact}, but not \emph{tape},
already occurs often (if not mostly) in a non-concrete usage when
it occurs on its own.

This is problematic, because standard dictionary definitions of
\emph{literal} as applied to meanings are clearly based not on frequencies,
but on quite different concepts, cf. \Next, a definition taken from the OED.

% In turn, compositional meaning is simply (in Lin's case, quite implicitly)
% equated with literality without any further discussion. This procedure seems
% to be quite common, compare also \citet{KatzandGiesbrecht:2006}, who likewise
% equate compositional meaning with literal sense, or
% \citet{BiemannandGiesbrecht:2011}, who annotated phrases for
% compositionality by asking `How literal is this phrase'.


\ex. literal\\
II.c
\begin{sloppypar}
Of, relating to, or designating the primary, original, or etymological sense
of a word, or the exact sense expressed by the actual wording of a phrase or
passage, as distinguished from any extended sense, metaphorical meaning, or
underlying significance. 
\end{sloppypar}
OED
% "literal, adj. and n.". OED Online. September 2013. Oxford University Press. http://www.oed.com/view/Entry/109055?rskey=KNikjc&result=6&isAdvanced=false (accessed November 27, 2013).

Importantly, a purely distribution-based approach like the one by
\citet{Lin:1999} and the traditional understanding of literal meaning
as illustrated in the quote from the OED might sometimes yield the
same result, but this need not be the case. Take an example like \emph{sacred
  cow} from the dataset of
\citet{Reddyetal:2011}. In the BNC, only one of 15 uses clearly refers
to a real cow. In contrast, if looking at the word \emph{cow} on its
own, we find many uses referring to the real animal. Here, we would
expect Lin's distributional approach to coincide with the notion of
literality as described in the OED quote. However, it would be
interesting to compare the intuitive literality of examples with rare animals like lion (rare at least from a broadly
western point of view), e.g. in \emph{stone lion}, with actual corpus
occurrence of \emph{lion} on its own, many of which do seem to refer to
pictures, statues, or toy versions of lions. I will
return to this issue when discussing the annotation of constituent meanings in the 2 empirical studies presented in Chapters \ref{cha:empirical-1} and \ref{cha:empirical-2}.

Focusing on the state of the traditional idea of literality as
illustrated by the OED quote, I cannot possibly do justice to all the literature written on this
topic. However, I will illustrate the debates surrounding this notion
by considering 2 viewpoints on the notion of literal meaning from psychology
and formal semantics respectively.

\enlargethispage{1\baselineskip}
In psychology, \citet[249]{Gibbs:1989}, while agreeing that 
``[p]eople can sometimes judge some statements as literal and other as
metaphorical", points out that this does not
mean that literal meanings necessarily play a role in understanding
non-literal meaning, and, perhaps more importantly, that there is no
evidence to show that different cognitive processes are involved in processing
these meanings. 

For formal semantics, I will use \citet{Jaszczolt:2016} to illustrate
a possible point of view. In general, \citet{Jaszczolt:2016} discusses
the term \textit{literal meaning} at several places, but, and that is most
important for the discussion here, in her own model, called Default
Semantics, this term does not occur anymore. In doing so, she does not
abandon the idea of word meanings:
\begin{quotation}
\dots: if we want a semantic theory that
  allows for the freedom of context-dependence and at the same time
  recognizes the fact that there \emph{are} word meanings, that, to
  put it crudely, the word `dog' is much more likely to refer to dogs
  than cats or food processors, we have to start with the assumption
  that words stand for concepts but that these concepts are
  situation-specific \emph{not} because they shift according to some
  clear rules or that they are constrained by the possibilities of the
  grammar; neither are they situation-specific because they are built
  in the process of language use. Rather, they are dynamic simply
  because they are susceptible to new uses in virtue of past uses; the
  generalization over past uses does not produce an abstract concept
  but instead paves the way towards new uses.\\ \citep[133--134]{Jaszczolt:2016}  
\end{quotation}
However, these word meanings are not
literal meanings as traditionally understood. Rather, she argues
``to retain the concept of word meanings as sufficiently to subsume such
influences of context-driven inferences as well as automatic
interpretations of different provenance" % checked the quote
\citep[136]{Jaszczolt:2016}. Jaszczolt recognizes that some sentence
meanings, and for that matter word meanings, are more easily arrived
at when the sentences occur out of context. However, this is not
because there is a literal meaning, but rather because she adapts a
view she labels cognitive minimalism:

\ex. Cognitive minimalism\\
Sentences issued out of context come with different degrees of
plausibility and these degrees correlate with different intuitions
concerning context-free evaluability with respect to truth and
falsity. The plausibility and the intuitions all depend on the
accessibility of a default, `made-up' context that can be used as a
tool for such a `neutral', apparently context-free, evaluation. \citep[58]{Jaszczolt:2016}

Importantly, she points out that the standard meaning one assigns to
a sentence need not be the literal one. Consider \Next, her (48)
\citep[59]{Jaszczolt:2016}:

\ex. A star has died.

According to Jaszczolt, the default situation for \Last could be one that
refers to the death of a movie star rather than to the death of a star
in the astronomical sense (note that this point still seems to hold
even if the predicate \emph{die} is exchanged with something more
neutral, e.g. \emph{We saw a star}).
\is{literal meaning|)}

\subsection{Semantic transparency as one dimension of idiomaticity}
\label{sec:sem-trans-as-one}
\is{idioms|(}\is{semantic transparency!{idioms and}|(}
\citet{Nunbergetal:1994}, working on idioms, point out that existing attempts
at defining idioms often fail to keep key semantic concepts apart.  In
particular, they argue that 3 semantic dimensions should be distinguished:
an idiom's relative conventionality, an idiom's opacity/transparency, and an
idiom's compositionality. The relative conventionality is ``determined by the
discrepancy between the idiomatic phrasal meaning and the meaning we would
predict for the collocation if we were to consult only the rules that
determine the meanings of the constituents in isolation, and the relevant
operations of semantic composition" \citep[498]{Nunbergetal:1994}. The
opacity/transparency dimension stands for ``the ease with which the motivation for the
use (or some plausible motivation -- it needn't be etymologically correct) can
be recovered" \citep[498]{Nunbergetal:1994}. And finally, compositionality
stands for ``the degree to which the phrasal meaning, once known, can be
analyzed in terms of the contributions of the idiom
parts" \citep[498]{Nunbergetal:1994}.\is{compositionality!{of idioms}}
 They introduce the term
\emph{idiomatically combining expressions} to refer to idioms ``whose parts
carry identifiable parts of their idiomatic
meanings" \citep[496]{Nunbergetal:1994}, in contrast to
\emph{idiomatic phrases}, where this is not the case. In this context, their discussion of
the phrase \emph{to pull strings} is particularly helpful. Clearly, the
idiomatic meaning ``exert a hidden influence" cannot be predicted on the basis
of the meanings of its constituents and the relevant semantic construction
rules for verb object combinations, there is therefore a large amount of
\isi{conventionality} involved. On the other hand, as
\citet[496]{Nunbergetal:1994} point out,
on hearing a sentence like \emph{John was able to pull strings to get the job,
  since he had a lot of contacts in the industry}, the hearer might be able to
deduce the correct meaning of the phrase. Thus, the expression is not completely opaque, and more importantly, the hearer can now map parts of the idiom
to parts of the meaning. \citeauthor{Nunbergetal:1994}, using the interpretation
\emph{exploit personal connections}, argue that \emph{pull} can be mapped to
\emph{exploit}, and \emph{strings} can be mapped to the exploited connections.

\is{compositionality!{conventionality vs.}|(} 
That \isi{conventionality} should be kept apart from compositionality is illustrated
by \citet{Nunbergetal:1994} with the help of the contrast between American \emph{thumb tack} and
British \emph{drawing pin}, which both denote the same types of objects: Both
are compositional and do not involve any figuration. Their double
existence is solely due to different ways
of conventionalization (cf. \citealt[495]{Nunbergetal:1994}). \is{compositionality!{conventionality vs.}|)} 


\citet{TitoneandConnine:1999} provide a balanced overview of previous studies of
idiomaticity which either argue for a non-compositional or a compositional
approach. They explore the distinction between idiomatically combining expressions
and idiomatic phrases from \citet{Nunbergetal:1994} in an eye tracking study
working with preceding and following contexts favoring either the literal or
the non-literal interpretation. \is{eye tracking!{literal/non-literal idiom interpretation}}
They interpret their results as supporting a
hybrid model of idiom processing, according to which the idiomatic meanings
are directly retrieved but a literal analysis of the respective phrase is also
carried out. 
\is{idioms|)}\is{semantic transparency!{idioms and}|)}

Note that the research on idioms described above presupposes that a)
there is a literal meaning and b) we 
know what that literal meaning is. \is{literal meaning!{idioms and}}
As the discussion in the section on
literality has shown, though, literal meaning by itself is not in any
way well-understood. I suspect that one reason why the departure from
literal meaning is taken as a given in the discussion of idioms lies
in the fact that the expressions usually allow 2 interpretations,
that is, we can use \emph{kick the bucket} to refer to the action of
striking the corresponding vessel, as well as using it to refer to the act of
passing away. This is reminiscent of the contrast between \emph{red
  tape} and \emph{economic impact} discussed above: \emph{Red tape}
allows 2 interpretations, and the one that just refers to a narrow
strip with the color red is used as a foil for the second
interpretation. In contrast, \emph{economic impact} only comes with
one interpretation, which, since it is the only interpretation, is
intuitively judged to be literal.


\subsection{Semantic transparency and productivity}
\label{sec:st-productivity}

Just as one can hypothesize that there is a correlation between increased
lexicalization and less semantic transparency, it seems intuitively
plausible that \isi{productivity} and semantic transparency might likewise be
correlated, albeit with the effects going in the same direction: the
more productive, the more transparent and vice
versa. \is{semantic transparency!{productivity and}}
\citet[199]{Baayen:1993} points out that ``semantic
transparency, like phonological transparency, is a necessary but not a
sufficient condition for productivity." He gives some examples from
Dutch: the Dutch plural suffix \emph{-eren} is fully semantically
transparent, yet unproductive. Another example of phonologically
and semantically fully transparent constructions are female personal
nouns in \emph{-ster}, which are less productive than constructions
with an unmarked \emph{-er} or a de-adjectival
\emph{-heid}. ``Differences in the usefulness of items in
\emph{-ster}, \emph{-er} and \emph{-heid} to the language community,
differences in markedness, the effects of paradigmatic rivalry, but
also social convention as such -- Dutch \emph{-ster} is much less
productive than its German counterpart \emph{-in} -- should not be
neglected" \citep[199--200]{Baayen:1993}.  
% TODO: longer section!
\il{German!{transparency and productivity}}
\il{Dutch!{transparency and productivity}}

\section{Transparency in other domains} % Transparency in other dimensions}
\label{sec:transparency-other-dimensions}

In all of the examples for semantic transparency discussed so far, the 2 constituents making up the
compound are still recognizable. If the individual constituents can no longer
be recognized, considerations of semantic transparency become
moot. Consider \emph{lord}: Etymologically, it is a compound, according to the OED, derived from Old English
\emph{hláford}, in turn derived from the precursors of today's \emph{loaf} and
\emph{ward} respectively. However, as pointed out by \citet[40]{Dressler:2006}, it is not
recognizable as a compound anymore, being the end product of
fossilization. As witnessed by \emph{lord}, fossilization can affect a
construction's meaning as well as
a word's phonology and orthography, with the
latter usually trailing the latter. Both areas by themselves can also be described
in terms of transparency. 

\subsection{Phonological transparency}
\is{phonological transparency|(}
\is{transparency!phonological|(}
In general, phonological transparency
involves the relationship between the phonetic forms of a
construction in isolation vs. the phonetic form of that construction
when it is part of a larger, complex construction (this is in the spirit of
\citealt[5]{Marslen-Wilsonetal:1994}, although they only discuss cases of affixation).
Thus, the base \emph{friend} in \emph{friendly} is phonologically transparent, because
the string \textipa{[frend]} occurs unchanged in \textipa{[frendlI]}. In
contrast, the base \emph{conclude} in \emph{conclusive} is not phonologically
transparent, because the \textipa{[d]} in \textipa{[k@nklu:d]} is changed to
\textipa{[s]} in \textipa{[k@nklu:sIv]}. In the case of compounds, changes with
respect to the phonetic shape of the constituents in isolation can be
as extensive
as to make it doubtful whether, orthography aside, the compound status is still
perceivable, consider e.g. \emph{blackguard},  \emph{boatswain}, and \emph{shepherd},
pronounced \textipa{/"bl\ae g@rd/}) and \textipa{/"b@Usn/}, and
\textipa{/"Sep@d/} respectively. 
% The
% constituents are only recognisable in the written form and not
% phonologically. In addition, even in the written form, the degree of
% recognizability differs. First of all, the forms to be recognized differ in
% frequency. \emph{Black}, \emph{guard} and \emph{boat} are common English
% words. In contrast, \emph{swain} is not a common word, and is in fact obsolete in the sense
% of \emph{serving-man}. \emph{Sheep} is a common word, but occurs in the older orthographic
% form \emph{shep}. Finally,
% \emph{herd} in the sense of \emph{keeper of a herd} is now more common in
% combinations than as a stand-alone word. %Source; OED herd N2
% This then leads to the most basic question in terms of
% semantic transparency: namely, can two lexemes be recognised? Saussure argues
% that although many linguistic signs are arbitrary in the sense that there is
% no natural reason why a particular sequence of sounds should represent a
% particular idea, others are relatively motivated, in the sense that they are
% combinations of signs to which meanings have already been attributed. So the
% meaning of \emph{weekend}, for example is not completely arbitrary because the
% meaning of \emph{week} and \emph{end} can be recognised within it, in the case of \emph{boatswain},
% however, the elements boat and swain can no longer be recognised in the spoken
% form. \textbf{[Check: at which level does Saussure's point apply? What would he say
% with regard to compounds that are clearly recognizable as compounds but are
% not motivated at all? Are there such cases?] Maybe \emph{cloud nine}???} 
Phonological reduction is also a matter of degree. Thus, while \emph{man} in
\emph{postman} \textipa{[poUstm@n]} contrasts with the free form man
\textipa{[m\ae n]} and is therefore not
phonologically transparent, the pronunciation of the free form can be
retrieved in situations calling for contrastive stress, e.g. \emph{a post\textipa{[m\ae n]} not a postwoman}.  

\enlargethispage{1\baselineskip}
Phonologically opaque compounds bear some  similarity to (and might in practice be
indistinguishable from) pseudocompounds like
\emph{boycott}, an example used in \citet{Zwitserlood:1994}. \is{pseudocompound} \is{compound!{pseudocompound}} 
\is{phonological transparency|)}
\is{transparency!phonological|)}

% \is{semantic transparency}


\subsection{Orthographic transparency}
\is{orthographic transparency|(}
\is{transparency!orthographic|(}
Orthographic transparency \is{orthographic transparency}is, at least to a certain degree, unrelated to
semantic and/or phonological transparency. Thus, 2 of the examples for
phonologically opaque compounds of the previous section, \emph{blackguard} and
\emph{boatswain}, are orthographically
fully transparent. These 2 examples are also semantically opaque. In
contrast, \emph{shepherd} is not only phonologically opaque, but also
orthographically opaque. However, the first element $<$shep$>$ is not
semantically opaque.

If, as in the case of \emph{lord}, a construction is opaque with
regard to its meaning, phonology and orthography, then it is typically
impossible to synchronically recognize it as a compound.
\is{orthographic transparency|)}
\is{transparency!orthographic|)}

% \section{Conclusion: semantic transparency: related phenomena and notions}
\section{Conclusion}
\label{sec:con-related}

The aim of this chapter was threefold. First, I gave an overview of 2
linguistic phenomena, anaphora resolution and stress placement, where
semantic transparency is hypothesized to play a role.
Second, I gave a short overview of other terms that are related to
semantic transparency. Finally, I briefly discussed transparency in
phonology and orthography. 

As the first section has shown, while it seems plausible that semantic
transparency plays a role with regard to whether internal constituents
of complex words are accessible or not, the research so far has not
used a clear criterion to identify transparency in the first place,
or, in the case of my own research, the criterion was clear, but only
insufficient empirical evidence could be found. With regard to the role of
semantic transparency in stress assignment, \citet{Bell:2012} used
very clear criteria, but these were very different in nature from the
methods used in psycholinguistics to establish semantic transparency.

\enlargethispage{1\baselineskip}
The second section started by pointing to work on semantic overlap,
a notion that very likely at least partially taps into the same
features that semantic transparency is after. However, given the very
specific targets of that line of research (memory and recollection
effects), and the overall very small number of compounds thus
classified, it is hard to compare it to measures directly targeting
semantic transparency. The section on compositionality and literality
showed 2 points: (1) For many, transparency, compositionality and
literality are one and the same thing. For those that distinguish
between transparency and compositionality, compositionality refers to
meaning predictability whereas transparency is already fulfilled when
the constituent meanings can be recognized in the meaning of the
complex expression. (2) Literality is a difficult concept. 

The section
on idiomaticity and semantic transparency showed that a distinction should be made
between transparency, conventionality, and compositionality. While
conventionality here is closely related to the difficult notion of
literality, the combination of the transparency and the
compositionality dimension is very close to the conception of semantic
transparency as introduced in Chapter \ref{cha:intro}, namely a
gradual notion with meaning predictability at one end and
recoverability of constituent meanings at the other end. Finally,
productivity can be argued to result in transparency. In contrast,
semantic transparency does not automatically lead to or entail productivity.

The third section discussed the notion of phonological and
orthographical trans\-par\-en\-cy. These notions will not play a role in
this work, but it is important to realize that a sufficient degree of
transparency in a given construction in either of these 2 domains is a prerequisite for the
question of semantic transparency to arise. 

% \subsection{Transparency in other dimensions}
% \label{sec:transparency-other-dimensions}

% The notion of transparency is not only relevant for the semantics of
% complex constructions, but also for other domains, notably for
% phonology and orthography. 
% \paragraph{Phonological transparency}
% In general, phonological transparency
% involves the relationship between the phonetic forms of a
% construction in isolation vs. the phonetic form of that construction when
% being part of a larger, complex construction (this is in the spirit of
% \citet[5]{Marslen-Wilsonetal:1994}, although they only discuss cases of affixation).
% Thus, the base \emph{friend} in \emph{friendly} is phonologically transparent, because
% the string \textipa{[frend]} occurs unchanged in \textipa{[frendlI]}. In
% contrast, the base \emph{conclude} in \emph{conclusive} is not phonologically
% transparent, because the \textipa{[d]} in \textipa{[k@nklu:d]} is changed to
% \textipa{[s]} in \textipa{[k@nklu:sIv]}. In the case of compounds, changes with
% respect to the phonetic shape of the constituents in isolation can go so far
% as to make it doubtful whether, orthography aside, the compound status is still
% even perceivable, consider e.g. \emph{blackguard},  \emph{boatswain}, and \emph{shepherd},
% pronounced \textipa{/"bl\ae g@rd/}) and \textipa{/"b@Usn/}, and
% \textipa{/"Sep@d/} respectively. 
% % The
% % constituents are only recognisable in the written form and not
% % phonologically. In addition, even in the written form, the degree of
% % recognizability differs. First of all, the forms to be recognized differ in
% % frequency. \emph{Black}, \emph{guard} and \emph{boat} are common English
% % words. In contrast, \emph{swain} is not a common word, and is in fact obsolete in the sense
% % of \emph{serving-man}. \emph{Sheep} is a common word, but occurs in the older orthographic
% % form \emph{shep}. Finally,
% % \emph{herd} in the sense of \emph{keeper of a herd} is now more common in
% % combinations than as a stand-alone word. %Source; OED herd N2
% % This then leads to the most basic question in terms of
% % semantic transparency: namely, can two lexemes be recognised? Saussure argues
% % that although many linguistic signs are arbitrary in the sense that there is
% % no natural reason why a particular sequence of sounds should represent a
% % particular idea, others are relatively motivated, in the sense that they are
% % combinations of signs to which meanings have already been attributed. So the
% % meaning of \emph{weekend}, for example is not completely arbitrary because the
% % meaning of \emph{week} and \emph{end} can be recognised within it, in the case of \emph{boatswain},
% % however, the elements boat and swain can no longer be recognised in the spoken
% % form. \textbf{[Check: at which level does Saussure's point apply? What would he say
% % with regard to compounds that are clearly recognizable as compounds but are
% % not motivated at all? Are there such cases?] Maybe \emph{cloud nine}???} 
% Phonological reduction is also a matter of degree. Thus, while \emph{man} in
% \emph{postman} \textipa{[poUstm@n]} contrasts with the free form man
% \textipa{[m\ae n]} and is therefore not
% phonologically transparent, the pronunciation of the free form can be
% retrieved in contrastive situations, e.g. \emph{a post\textipa{[m\ae n]} not a postwoman}.  

% Phonologically opaque ABs bear some  similarity to (and might in practice be
% indistinguishable from) pseudo ABs like
% \emph{boycott}, an example used in \citet{Zwitserlood:1994}.

% \paragraph{Orthographical transparency}

% Orthographic transparency is, at least to a certain degree, unrelated to
% semantic and/or phonological transparency. Thus, two of the examples for
% phonologically opaque compounds of the previous section, \emph{blackguard} and
% \emph{boatswain}, are orthographically
% fully transparent. These two examples are also semantically opaque. In
% contrast, \emph{shepherd} is again phonologically opaque, but also
% orthographically opaque. However, the first element $<$shep$>$ is not
% semantically opaque.
 

%%% Local Variables: 
%%% mode: latex
%%% TeX-master: "habil-master_rev-1"
%%% End: 
