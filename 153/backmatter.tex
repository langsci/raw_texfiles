\backmatter
\phantomsection%this allows hyperlink in ToC to work
\addtocontents{toc}{\protect\setcounter{tocdepth}{1}} % to surpress lower level sections in the toc
\chapter{Appendix A: Semantic coding for \citet{BellandSchaefer:2013}}
\label{sec:appendix_semantic-coding-2013}
\input{chapters/appendix-semantic-coding-Bell-Schaefer-2013.tex}

\chapter{Appendix B: Semantic coding for \citet{BellandSchaefer:2016}}
\label{sec:appendix_semantic-coding-2016}

% Overview of the synset coding decisions for the final dataset
This appendix contains additional notes on the semantic coding of the dataset used in \citet{BellandSchaefer:2016}. The dataset is available at \url{http://martinschaefer.info/publications/download/Bell_and_Schaefer_2016_semantic-coding.zip}. The notes are also available as a standalone document at \url{http://martinschaefer.info/publications/download/Schaefer_2016_Notes-on-the-Semantic-Coding.pdf}. Updated versions of this document will be made available there.

The presentation here follows the lemmas in the N1 and N2 families in alphabetically order. The entry for every lemma starts with the compound containing this lemma in the respective position (N1 or N2) from the compound set rated in \citet{Reddyetal:2011}. The following table gives an overview of the synset coding for the final
dataset, using the following column headers:

\noindent
\begin{tabularx}{1\textwidth}{lQ}\lsptoprule 
column header&explanation\\\midrule
wnSense&value of the variable wnSense in the coded data, usually corresponding to the sense number in WordNet\\\tablevspace
example& example from the coded data\\\tablevspace
types&number of types in the data\\\tablevspace
class&word class according to WordNet\\\tablevspace
WordNet description&gloss (this is taken
verbatim from WordNet unless otherwise indicated)   \\\lspbottomrule
\end{tabularx}
\vspace*{2ex}

\noindent
% If there are multiple wordNet senses, the row containing the wordNet sense of the target item in the
% data occurs in boldface.
Following the table are any further comments on the WordNet senses as well as on the relational coding. In addition, known mistakes in the coding are listed here.
\input{chapters/appendix-wordnet-senses}

\chapter[Appendix C: Multiple readings and the 2016 coding]{Appendix C: Reddy et al. items with multiple readings and the 2016 semantic coding}
\addchap{Appendix}
\hypertarget{Toc63021257}{}
\label{App}
Regional distribution of newspapers, issues and articles published in 51 states and territories of the United States in the nineteenth century and contained in the databases AHN and NCNP (ranked according to the number of articles).\medskip

{\small
\noindent
\begin{tabularx}{\textwidth}{llYYY}
\lsptoprule
 & \textbf{State/Territory} & \textbf{ newspapers} & \textbf{ issues} & \textbf{ articles}\\
 \midrule
 1 & Massachusetts &  116 &  144,008 &  13,111,797\\
 2 & New York &  150 &  113,053 &  12,137,617\\
 3 & Pennsylvania &  76 &  85,017 &  6,210,241\\
 4 & Maryland &  28 &  33,654 &  3,369,921\\
 5 & Connecticut &  69 &  33,683 &  3,053,743\\
 6 & New Hampshire &  32 &  30,425 &  2,655,250\\
 7 & Missouri &  11 &  16,896 &  2,543,445\\
 8 & Georgia &  23 &  29,812 &  2,301,709\\
 9 & Texas &  37 &  20,637 &  2,284,520\\
 10 & Virginia &  51 &  16,273 &  2,124,670\\
 11 & Wisconsin &  17 &  27,306 &  2,086,384\\
 12 & District of Columbia &  34 &  27,716 &  1,940,043\\
 13 & South Carolina &  26 &  29,057 &  1,896,609\\
 14 & Ohio &  28 &  32,263 &  1,820,941\\
 15 & California &  18 &  20,585 &  1,810,992\\
 16 & Nebraska &  5 &  9,320 &  1,609,843\\
 17 & Rhode Island &  30 &  15,145 &  1,369,960\\
 18 & Maine &  25 &  26,586 &  1,298,668\\
 19 & Illinois &  17 &  7,386 &  1,235,044\\
 20 & Colorado &  9 &  16,782 &  1,167,769\\
 21 & North Carolina &  17 &  20,249 &  1,141,303\\
 22 & Vermont &  50 &  20,360 &  1,139,752\\
 23 & Louisiana &  19 &  12,399 &  1,006,593\\
 24 & Kansas &  38 &  13,217 &  945,314\\
 25 & West Virginia &  3 &  9,386 &  891,105\\
 26 & New Jersey &  23 &  15,225 &  858,120\\
 27 & Minnesota &  10 &  9,085 &  845,677\\
 28 & North Dakota &  4 &  10,306 &  585,770\\
 29 & Idaho &  20 &  10,667 &  577,660\\
 30 & Oregon &  10 &  22,825 &  500,698\\
 \midrule
 \end{tabularx}

 \noindent
\begin{tabularx}{\textwidth}{lQYYY}
 \midrule
 & \textbf{State/Territory} & \textbf{ newspapers} & \textbf{ issues} & \textbf{ articles}\\
 \midrule
 31 & New Mexico &  6 &  7,291 &  488,663\\
 32 & Arizona &  12 &  8,091 &  455,432\\
 33 & Arkansas &  8 &  9,416 &  388,149\\
 34 & Washington &  2 &  4,993 &  375,759\\
 35 & South Dakota &  6 &  5,012 &  336,131\\
 36 & Mississippi &  20 &  7,655 &  330,852\\
 37 & Utah &  16 &  3,048 &  296,505\\
 38 & Iowa &  6 &  3,116 &  262,443\\
 39 & Kentucky &  18 &  4,601 &  261,841\\
 40 & Alabama &  34 &  2,343 &  224,455\\
 41 & Indiana &  41 &  3,350 &  182,731\\
 42 & Montana &  5 &  1,312 &  176,296\\
 43 & Delaware &  7 &  2,026 &  106,742\\
 44 & Tennessee &  18 &  1,833 &  78,775\\
 45 & Oklahoma &  6 &  2,221 &  65,201\\
 46 & Hawaii &  13 &  1,330 &  62,574\\
 47 & Florida &  7 &  2,348 &  62,041\\
 48 & Michigan &  4 &  430 &  22,373\\
 49 & Wyoming &  3 &  476 &  21,146\\
 50 & Nevada &  5 &  497 &  9,465\\
 51 & Alaska &  8 &  56 &  2,539\\
 \midrule
\multicolumn{2}{c}{} &  \textbf{1,241} &  \textbf{950,768} &  \textbf{78,731,271}\\
\lspbottomrule
\end{tabularx}

}


\chapter[Appendix D: Corpus and dictionary sources]{Appendix D: Corpus identifiers and material from online dictionaries}
\input{chapters/appendix_sources.tex}


\sloppy
\printbibliography[heading=references] 
\cleardoublepage

\phantomsection 
\addcontentsline{toc}{chapter}{Index} 
\addcontentsline{toc}{section}{Name index}
\ohead{Name index} 
\printindex 
\cleardoublepage
  
\phantomsection 
\addcontentsline{toc}{section}{Language index}
\ohead{Language index} 
\printindex[lan] 
\cleardoublepage
  
\phantomsection 
\addcontentsline{toc}{section}{Subject index}
\ohead{Subject index} 
\printindex[sbj]
\ohead{} 
