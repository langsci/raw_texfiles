\documentclass[output=paper,hidelinks]{langscibook}
\ChapterDOI{10.5281/zenodo.13347654}

\author{Dankmar W. Enke\affiliation{LMU Munich} and Larry M. Hyman\affiliation{University of California, Berkeley} and Johanna Nichols\affiliation{University of California, Berkeley; University of Helsinki; \\ Higher School of Economics, Moscow} and Guido Seiler\affiliation{University of Zurich} and  Thilo Weber\affiliation{Leibniz Institute for the German Language (IDS), Mannheim}}
\title{Language change for the worse}

\abstract{\noabstract}

%% add all extra packages you need to load to this file

\usepackage{tabularx,multicol}
\usepackage{url}
\urlstyle{same}

\usepackage{listings}
\lstset{basicstyle=\ttfamily,tabsize=2,breaklines=true}

\usepackage{langsci-basic}
\usepackage{langsci-optional}
\usepackage{langsci-lgr}
\usepackage{langsci-osl}
% \usepackage{./langsci/styles/langsci-lgr}
% \usepackage{./langsci/styles/langsci-osl}
% \usepackage{langsci-gb4e}

\usepackage{tikz}
\usetikzlibrary{patterns,calc}
\pgfdeclarepatternformonly{south east lines}{\pgfqpoint{-0pt}{-0pt}}{\pgfqpoint{3pt}{3pt}}{\pgfqpoint{3pt}{3pt}}{
    \pgfsetlinewidth{0.6pt}
    \pgfpathmoveto{\pgfqpoint{0pt}{3pt}}
    \pgfpathlineto{\pgfqpoint{3pt}{0pt}}
    \pgfpathmoveto{\pgfqpoint{.2pt}{-.2pt}}
    \pgfpathlineto{\pgfqpoint{-.2pt}{.2pt}}
    \pgfpathmoveto{\pgfqpoint{3.2pt}{2.8pt}}
    \pgfpathlineto{\pgfqpoint{2.8pt}{3.2pt}}
    \pgfusepath{stroke}}
    
\usepackage{stmaryrd}
\usepackage{wasysym}
\usepackage{multirow}
\usepackage{caption}
\usepackage{subcaption}
\usepackage{mathrsfs}
\usepackage{qtree}

\usepackage{linguex}


%%pminos do not split footnotes
% \interfootnotelinepenalty=10000 %Footnote in Laporte chapters has to be split SN


%\DeclareIndexNameFormat{default}{%
%\nameparts{#1}%
%\usebibmacro{index:name}%
%{\index[names]}%
%{\namepartfamily}%
%{\namepartgiveni}%
% {}% L1
% {}% L2
%{\namepartprefix}% generates spurious space L3
%{\namepartsuffix}% generates spurious space L4
%}

%  {\DeclareIndexNameFormat{default}{%
%     \usebibmacro{index:name}{\index[names]}{#1}{#3}{#5}{#7}}}

%\DeclareIndexNameFormat{default}{%
%  \usebibmacro{index:name}{\sindex[nom]}{#1}{#3}{#5}{#7}}

%\DeclareIndexNameFormat{default}{%
%  \usebibmacro{index:name}{\sindex[person]}{#1}{#3}{#5}{#7}}
%\DeclareIndexNameFormat{default}{%
%\nameparts{#1} \usebibmacro{index:name}{\sindex[person]]}{\namepartfamily}{‌​\namepartgiven}{\nam‌​epartprefix}{\namepa‌​rtsuffix}}

%\newcommand{\smiley}{:)}

%\renewbibmacro*{index:name}[5]{%
%\usebibmacro{index:entry}{#1}%
%{\iffieldundef{usera}{}{\thefield{usera}\actualoperator}\mkbibindexname{#2}{#3}{#4}{#5}}}

% \newcommand{\noop}[1]{}

%remove for final
%\overfullrule=1mm

\newcommand{\tobi}[2]}}
\renewcommand{\S}[1]{\tobi{#1}{\textsc{*}}}

% this volume references
% puts: [this volume]
% already defined: \citetv
%\newcommand{\citepv}[1]{(\citeauthor{#1} \citeyear*{#1} [this volume])}
\newcommand{\citealtv}[1]{\citeauthor{#1} \citeyear*{#1} [this volume]}

%parentheses around example number
\newcommand{\pref}[1]{(\ref{#1})}

% in-text examples

\newcommand{\lnex}[1]{\textit{#1}} %target lang word
\newcommand{\lnlit}[1]{(lit.: `#1')} %literal reading
\newcommand{\lnlat}[1]{(#1)} % latinization
\newcommand{\lntrans}[1]{`#1'} %translation
\newcommand{\lnexl}[2]%
{\lnex{#1}{} \lnlat{#2}} % ex with latinization
\newcommand{\lnexlat}[3]{\lnex{#1}{} \lnlat{#2}{} \lntrans{#3}} % ex with latinization and tranl.

%ch01
\newcommand{\co}[1]{\mbox{\textbf{#1}}}

%ch09

\newcommand{\cyrbulg}[1]{\begin{otherlanguage*}{bulgarian}#1\end{otherlanguage*}}


%ch10
\newcommand{\nlp}{{\small NLP}}
\newcommand{\mwe}{{\small MWE}}
\newcommand{\rae}{{\small RAE}}
\newcommand{\lvc}{{\small LVC}}
\newcommand{\pos}{{\small P}o{\small S}}
%\newcommand{\todo}[1]{ \textcolor{red}{#1} }

%\renewcommand{\labelenumi}{\theenumi}
%\ainamefmt{{vv}{ll}{, ff}{, jj}} % fullname

\newcommand{\biberror}[1]{{\color{red}#1}}

\newcommand{\osenovaitem}{--~}

%\usepackage{tipa}



\IfFileExists{../localcommands.tex}{
   \addbibresource{../localbibliography.bib}
%    \addbibresource{thisvolume.bib}
   % add all extra packages you need to load to this file

\usepackage{tabularx,multicol}
\usepackage{url}
\urlstyle{same}

\usepackage{listings}
\lstset{basicstyle=\ttfamily,tabsize=2,breaklines=true}

\usepackage{langsci-basic}
\usepackage{langsci-optional}
\usepackage{langsci-lgr}
\usepackage{langsci-osl}
% \usepackage{./langsci/styles/langsci-lgr}
% \usepackage{./langsci/styles/langsci-osl}
% \usepackage{langsci-gb4e}

\usepackage{tikz}
\usetikzlibrary{patterns,calc}
\pgfdeclarepatternformonly{south east lines}{\pgfqpoint{-0pt}{-0pt}}{\pgfqpoint{3pt}{3pt}}{\pgfqpoint{3pt}{3pt}}{
    \pgfsetlinewidth{0.6pt}
    \pgfpathmoveto{\pgfqpoint{0pt}{3pt}}
    \pgfpathlineto{\pgfqpoint{3pt}{0pt}}
    \pgfpathmoveto{\pgfqpoint{.2pt}{-.2pt}}
    \pgfpathlineto{\pgfqpoint{-.2pt}{.2pt}}
    \pgfpathmoveto{\pgfqpoint{3.2pt}{2.8pt}}
    \pgfpathlineto{\pgfqpoint{2.8pt}{3.2pt}}
    \pgfusepath{stroke}}
    
\usepackage{stmaryrd}
\usepackage{wasysym}
\usepackage{multirow}
\usepackage{caption}
\usepackage{subcaption}
\usepackage{mathrsfs}
\usepackage{qtree}

\usepackage{linguex}


   %pminos do not split footnotes
% \interfootnotelinepenalty=10000 %Footnote in Laporte chapters has to be split SN


%\DeclareIndexNameFormat{default}{%
%\nameparts{#1}%
%\usebibmacro{index:name}%
%{\index[names]}%
%{\namepartfamily}%
%{\namepartgiveni}%
% {}% L1
% {}% L2
%{\namepartprefix}% generates spurious space L3
%{\namepartsuffix}% generates spurious space L4
%}

%  {\DeclareIndexNameFormat{default}{%
%     \usebibmacro{index:name}{\index[names]}{#1}{#3}{#5}{#7}}}

%\DeclareIndexNameFormat{default}{%
%  \usebibmacro{index:name}{\sindex[nom]}{#1}{#3}{#5}{#7}}

%\DeclareIndexNameFormat{default}{%
%  \usebibmacro{index:name}{\sindex[person]}{#1}{#3}{#5}{#7}}
%\DeclareIndexNameFormat{default}{%
%\nameparts{#1} \usebibmacro{index:name}{\sindex[person]]}{\namepartfamily}{‌​\namepartgiven}{\nam‌​epartprefix}{\namepa‌​rtsuffix}}

%\newcommand{\smiley}{:)}

%\renewbibmacro*{index:name}[5]{%
%\usebibmacro{index:entry}{#1}%
%{\iffieldundef{usera}{}{\thefield{usera}\actualoperator}\mkbibindexname{#2}{#3}{#4}{#5}}}

% \newcommand{\noop}[1]{}

%remove for final
%\overfullrule=1mm

\newcommand{\tobi}[2]}}
\renewcommand{\S}[1]{\tobi{#1}{\textsc{*}}}

% this volume references
% puts: [this volume]
% already defined: \citetv
%\newcommand{\citepv}[1]{(\citeauthor{#1} \citeyear*{#1} [this volume])}
\newcommand{\citealtv}[1]{\citeauthor{#1} \citeyear*{#1} [this volume]}

%parentheses around example number
\newcommand{\pref}[1]{(\ref{#1})}

% in-text examples

\newcommand{\lnex}[1]{\textit{#1}} %target lang word
\newcommand{\lnlit}[1]{(lit.: `#1')} %literal reading
\newcommand{\lnlat}[1]{(#1)} % latinization
\newcommand{\lntrans}[1]{`#1'} %translation
\newcommand{\lnexl}[2]%
{\lnex{#1}{} \lnlat{#2}} % ex with latinization
\newcommand{\lnexlat}[3]{\lnex{#1}{} \lnlat{#2}{} \lntrans{#3}} % ex with latinization and tranl.

%ch01
\newcommand{\co}[1]{\mbox{\textbf{#1}}}

%ch09

\newcommand{\cyrbulg}[1]{\begin{otherlanguage*}{bulgarian}#1\end{otherlanguage*}}


%ch10
\newcommand{\nlp}{{\small NLP}}
\newcommand{\mwe}{{\small MWE}}
\newcommand{\rae}{{\small RAE}}
\newcommand{\lvc}{{\small LVC}}
\newcommand{\pos}{{\small P}o{\small S}}
%\newcommand{\todo}[1]{ \textcolor{red}{#1} }

%\renewcommand{\labelenumi}{\theenumi}
%\ainamefmt{{vv}{ll}{, ff}{, jj}} % fullname

\newcommand{\biberror}[1]{{\color{red}#1}}

\newcommand{\osenovaitem}{--~}
   %% hyphenation points for line breaks
%% Normally, automatic hyphenation in LaTeX is very good
%% If a word is mis-hyphenated, add it to this file
%%
%% add information to TeX file before \begin{document} with:
%% %% hyphenation points for line breaks
%% Normally, automatic hyphenation in LaTeX is very good
%% If a word is mis-hyphenated, add it to this file
%%
%% add information to TeX file before \begin{document} with:
%% %% hyphenation points for line breaks
%% Normally, automatic hyphenation in LaTeX is very good
%% If a word is mis-hyphenated, add it to this file
%%
%% add information to TeX file before \begin{document} with:
%% \include{localhyphenation}
\hyphenation{
    Beck-man
    Ngu-yen
    back-chan-nel
    back-chan-nels
    mo-not-o-nous
    ste-reo-typ-i-cal
}

\hyphenation{
    Beck-man
    Ngu-yen
    back-chan-nel
    back-chan-nels
    mo-not-o-nous
    ste-reo-typ-i-cal
}

\hyphenation{
    Beck-man
    Ngu-yen
    back-chan-nel
    back-chan-nels
    mo-not-o-nous
    ste-reo-typ-i-cal
}

   \togglepaper[1]%%chapternumber
}{}



\lehead{Dankmar W. Enke et al.}
\begin{document}
\maketitle

%\section{Language change for the worse?}
\noindent Human languages are in a constant state of change. While this observation itself is uncontroversial, the same is not true for many of the questions raised by it. This includes, for example, the question of why a language undergoes change, the question of how and when a change begins, how it spreads across the linguistic system as well as the speech community and how and when it comes to a stop. It also includes the question of whether change can be evaluated in terms of its quality, and, if so, under what circumstances a particular change may be said to lead to an improvement or, conversely, have a worsening effect. The present volume focuses on this last set of questions.

\begin{sloppypar}
Conceptions of language change have varied considerably over time, and still today we find numerous competing views (see e.g. \citealt{Aitchison_2001} for an overview). \citet[251]{Aitchison_2001} gives a quotation by Curtius (1877) (quoted by \citealt[35]{Kiparsky_1972}) as an example of many nineteenth century scholars' conception of language change as decay:
\end{sloppypar}

\begin{quote}
A principal goal of this science [i.e. comparative historical linguistics] is to reconstruct the full, pure forms of an original stage from the variously disfigured and mutilated forms which are attested in the individual languages. (Curtius 1877; quoted by \citealt[251]{Aitchison_2001})
\end{quote}

A more optimistic view is advocated by \citet{Jespersen_1922}. Looking at the loss of inflectional endings in \ili{English} and \ili{Danish} and their replacement by word order as a means of expressing grammatical relations, Jespersen views this as an increase in efficiency, and thus as a manifestation of progress: 

\begin{quote}
In the evolution of languages the discarding of old flexions goes hand in hand with the development of simpler and more regular expedients that are rather less liable than the old ones to produce misunderstanding.  (\citealt{Jespersen_1922}; quoted by \citealt[7]{Aitchison_2001})
\end{quote}

\Citet{Saussure2011}, by contrast, refrains from viewing language change as either progress or decay. Instead, he emphasizes its non-teleological nature. According to him, the linguistic system is, in fact, not even directly changeable at all. If it changes, it does so only as an indirect -- and unintended -- consequence of a change in one of its elements:

\begin{quote}
A diachronic fact is an independent event; the particular synchronic consequences that may stem from it are wholly unrelated to it. [...] never is the system modified directly. In itself it is unchangeable; only certain elements are altered without regard for the solidarity that binds them to the whole. \citep[84]{Saussure2011}
\end{quote}

Other non-teleological conceptions of language change include, for example, that by \citet{Hockett_1958}, who views change as random drift, or that by \citet{Postal_1968}, who compares language change to stylistic changes in fashion:

\begin{quote}
It is just this sort of slow drifting of expectation distributions, shared by people who are in constant communication, that we mean to subsume under the term ‘sound change’ […]. The drift might well not be in any determinate direction (\citealt[443--445]{Hockett_1958}; quoted by \citealt[135--136]{Aitchison_2001})
\end{quote}

\begin{quote}
There is no more reason for language to change than there is for automobiles to add fins one year and remove them the next, for jackets to have three buttons one year and two the next […]. The ‘courses’ of sound change […] lie in the general tendency of human cultural products to undergo ‘non-functional’ stylistic change (\citealt[283]{Postal_1968}; quoted by \citealt[135]{Aitchison_2001})
\end{quote}

As a final example, we may look at the perspective taken by generative grammar. Here, the central locus of change is seen in the transition from one generation of speakers to the next, i.e. a key role is played by language acquisition. Language change can thus be conceived of as a learning error:

\begin{quote}
The child has to analyze and interpret the linguistic phenomena in her language-acquisition environment in order to be able to acquire the grammar of the previous generation […]. Now, if the surface is analyzed incorrectly, the child's goal is also incorrect. In other words, if the child's task is to match her input data, she is bound to fail as she sets out with wrong conclusions. She has misinterpreted the final state. \citep[119]{Hróarsdóttir_2009}
\end{quote}

\citet{Aitchison_2001} herself arrives at the conclusion that language change has no \textit{global} improving or worsening effect:

\begin{quote}
We must conclude therefore that language change is ebbing and flowing like the tide, but neither progressing nor decaying, as far as we can tell […]. As the famous Russian linguist Roman Jakobson said over fifty years ago: ``The spirit of equilibrium and the simultaneous tendency towards its rupture constitute the indispensable properties of that whole that is language.'' \citep[254--255]{Aitchison_2001}
\end{quote}

Nonetheless, many theories of language change hold that at least on a \textit{local} level, changes are motivated by improvement. \citet{Langacker_1977} uses the term ``optimality'', and he not only expresses the idea that there are different categories or types of optimality, but he also assumes that these different types may be in conflict with one another:

\begin{quote}
I believe we can isolate a number of broad categories of linguistic optimality. Languages will tend to change so as to maximize optimality in each of these categories […]. The tendencies toward these various types of optimality will often conflict with one another. (\citealt[102]{Langacker_1977}; quoted by \citealt[181]{Haspelmath1999optimality})
\end{quote}

\begin{sloppypar}
Naturalness Theory \citep{Wurzel_1984,Mayerthaler_1980}, too, assumes that structural features of language can be evaluated in terms of their ``naturalness''\slash``markedness'', and it predicts that over time, natural\slash unmarked structures will win out over unnatural\slash marked ones. Another example is Vennemann's (\citeyear{Vennemann_1988,Vennemann_1993}) preference theory, which holds that every syllable structure change will lead to an improvement of syllable structure. 
\end{sloppypar}

% Andi: I changed the reference to Haspelmath to the correct one throughout the chapter

\citet{Haspelmath1999optimality}, finally, relates functionalist/usage-based approaches to language change with an evolutionary perspective, interpreting local changes as functional adaptations to the needs of language users:

\begin{quote}\sloppy
In language change, variants are created from which speakers may choose. Being subject to various constraints on language use, speakers tend to choose those variants that suit them best. These variants then become increasingly frequent and entrenched in speakers’ minds, and at some point they may become obligatory parts of grammar. \citep[203]{Haspelmath1999optimality}
\end{quote}

Changes for the worse, or maladaptive changes, on the other hand, are often considered a mere side effect of a change for the better in some other area. This idea is expressed, for example, by \citet{Vennemann_1988}:

\begin{quote}
Every change in a language system is a local improvement relative to a certain parameter. For instance, every syllable structure change is an improvement of syllable structure as defined by some preference law for syllable structure. If a change worsens syllable structure, it is not a syllable structure change, by which I mean change motivated by syllable structure, but a change on some other parameter which merely happens also to affect syllable structure. 
\citep[1--2]{Vennemann_1988}
\end{quote}

In other words, pejorative or maladaptive changes are due to the fact that there are different types of optimality (see \citealt[102]{Langacker_1977} above), and the fact that what may be an optimization with respect to one type, may well amount to a pejoration with respect to some other type. The idea that the criteria defining optimality may conflict with each other is also present in naturalness, and, perhaps most prominently, in optimality theory (OT, \citealt{Prince_1993}). As regards naturalness, something that may be natural in terms of phonology (e.g. loss of a word-final unstressed vowel), for instance, may well be unnatural in terms of morphology (e.g. if, along with the vowel, an entire affix is lost, which may lead to syncretism). As regards OT, grammar is explicitly conceived of as a set of competing, violable and hierarchically ordered constraints. If an output violates a given constraint A, this violation will always have to be justified by the fulfilment of a higher-ranking constraint B. From a diachronic perspective, then, language change is nothing but a re-ranking of constraints. Constraint reranking, too, expresses the idea that ``worsening'' outputs are a side effect of some other local improvement, i.e. promotion of some other output constraint. 

The present volume aims to explore to what extent there are phenomena of language change that seem to run counter to the hypothesis outlined above: Are there changes for the worse that do not readily follow from an improvement in some other area of the language system? And if so, how could this type of change be explained and what would it mean for our models of change? 

In order to speak of ``improvement'' or ``worsening'', it is, first of all, necessary to establish the relevant criteria: What exactly is meant by ``meliorative'' and ``pejorative'' changes, respectively? Does it refer to constraints on construction and planning (speaker's perspective)? Or to constraints on perception and parsing (hearer's perspective)? Does it refer to economy and efficiency (such as inventory size, complexity (see below) and idiosyncrasy)? Does it refer to an increase or decrease in communicative value? Or does it refer to an increase or decrease in transparent symbolization (such that formal contrasts reflect functional contrasts and vice-versa)?

\begin{sloppypar}
The question of meliorative vs. pejorative changes is closely related to the debate on complexification and, conversely, simplification: This idea surely has more than dubious roots, namely the nineteenth century notion that some (mostly European) languages\il{European languages} are more advanced than others. It is a small wonder, then, that twentieth century linguistics more or less reached a consensus that human languages are constant and very much alike in their overall degree of complexity. However, this basic tenet has recently been called into question and has received considerable attention ever since (e.g. \citealt{Dahl_2004, MiestamoEtAl2008, Garrett_2008, Albright_2008, Sampson_2009, Trudgill_2011, Newmeyer_2014, SeilerBaechler_2016}). The leading idea is that complexity is not a global property of grammatical systems, but instead stems from the interaction of its different parts. Crucially, an increase in complexity seems to violate the markedness principles postulated by Naturalness or Optimality Theory. Typical questions that have been raised in this context and that are also immediately relevant to the topic of the present volume include the question of how complexity (and the processes associated with it) can be measured, the question of whether complexification in one sub-system (e.g. morphology) is always balanced out by simplification in another sub-system (e.g. syntax) and the question of what determines (de-)complexification.
\end{sloppypar}
    
The question of pejorative changes is interesting also from an evolutionary approach to language change (see e.g. \citealt{Eckardt_2008}), in particular, perhaps, to the question of whether language change is more similar to a ``Darwinian" or to a ``Lamarckian" conception of evolution \citep{deVogelaer_2007}. Under an evolutionary perspective, the guiding mechanisms in language change are variation and selection. While this is true for both the Darwinian and the Lamarckian view, the two differ with respect to the role played by functional factors. Under the Darwinian view, the emergence of new variants is essentially random, i.e. variants are being produced irrespective of any potential (dis)advantages in selection; only the selection process itself is guided by functional factors, i.e. by the extent to which a variant is adapted to its environment  (cf. \citealt[318]{McMahon_1994}). Under the Lamarckian view, by contrast, it is the very emergence of variants that is driven by functional factors. This latter view is advocated by \citet[38]{Croft_2000}: ``Functional factors [...] are responsible only for innovation, and social factors provide a selection mechanism for propagation." \citet[192--193]{Haspelmath1999optimality}, too, argues that ``the source of linguistic variation is often nonrandom […]. In this sense, the evolution of linguistic structures is in part ``Lamarckian"". The Lamarckian view thus predicts that there should be no such thing as dysfunctional change, as dysfunctional variants do not emerge in the first place. Under the Darwinian view, by contrast, the emergence of dysfunctional variants is fully expected -- even though they are predicted not to last.

To set the scene, we would like to give a few examples of diachronic developments (taken from the history of \ili{German}) that might be candidates for pejorative change. 

A well-known change (and one that is unproblematic in this context) is open syllable lengthening, which took place from \ili{Middle High German} to \ili{Early New High German} \citep{Lahiri_Dresher_1999}. Whereas \ili{Middle High German} allowed for light stressed syllables, viz. [ˈfo.ɡəl] `bird', later stages of \ili{German} eliminated light stressed syllables by lengthening the vowel (thus making the syllable bimoraic), resulting in [ˈfoː.ɡəl]. \citet{Vennemann_1988} motivates this change as an optimization of syllable structure, stating that:

\begin{quote}
Weight law: In stress accent languages an accented syllable is the more preferred, the closer its syllable weight is to two moras, and an unaccentuated syllable is the more preferred the closer its weight is to one mora. \citep[30]{Vennemann_1988}
\end{quote}


Thus, open syllable lengthening can be seen as a classical example of language change for the better (local improvement of syllable structure), and consequently anti-open syllable lengthening (if it exists) would be a ``worsening'' change. \ili{Bernese} \ili{Swiss German} did not undergo open syllable lengthening \citep{Seiler_2005}. Short stressed vowels in open syllables retained their original quantity. We might think at first glance that this dialect is simply conservative, but there are quite a few examples where \ili{Bernese} shortened previously (i.e., \ili{Middle High German}) long vowels in stressed open syllables, viz. 
[ˈhyː.sər] > [ˈhʏ.z̥ər] `houses', 
[ˈbliː.bən] > [ˈb̥lɪ.b̥ə] `stay',
[ˈɪ̯æː.rek] > [ˈɪ̯æ.rɪɡ̥] `a year old'
\citep[477]{Seiler_2005}. This change, open syllable shortening, is not only unexpected in the light of Vennemann's Weight Law, it even runs counter to it. If it is correct that open syllable lengthening must be interpreted as a local optimization of syllable structure, then we must conclude that \ili{Bernese} open syllable shortening worsens syllable structure.

To take an example from morphology, \ili{Old High German} had accusative case marking on proper nouns, e.g. \textit{Hartmuot} (nom) vs.  \textit{Hartmuotan} (acc) (cf. \citealt[186--187]{Braune_2004}). This feature can most likely be considered ``user-friendly" (to use a term by \citealt[191]{Haspelmath1999optimality}): After all, the default-function associated with proper nouns is that of subject rather than direct object (cf. their high degree of animacy and definiteness). Accusative marking will thus help the hearer identify those cases that deviate from the expected. Nonetheless, \ili{Modern German} has completely given up accusative case marking on proper nouns.

Finally, to take an example from syntax, another potential case of dysfunctional change might be the emergence of the well-known verb-second constraint of most modern \ili{Germanic} languages: In \ili{Modern German}, independent declarative clauses are characterized by the requirement that the finite (part of the) verb be preceded by exactly one XP. This XP \textit{may} be the topic, but it need not be. In fact, according to \citet[9]{Frey_2004}, one way of filling the first position is through a mechanism labelled ``formal movement", a mechanism that ``does not seem to be related to any semantic or pragmatic property". In \ili{Old High German}, by contrast, verb-second was only one of several options. Alternatively, declaratives could show verb first or ``verb late'' (i.e. third or even later position) order, and the choice was governed by information structure (cf. \citealt{HinterhölzlPetrova_2010, HinterhölzlPetrova_2011}). One way of looking at modern verb second, then, is that it is a fossilized information-structural pattern, explainable on the basis of functional factors only with respect to its history, but not with respect to its synchronic functioning.

\section*{About this volume}
%\section{About this volume}

The present volume aims to approach the topic of change for the worse (and for the better) from a wide range of perspectives: It addresses phenomena from different domains of grammar (phonology, morphosyntax, semantics), and it explicitly aims to avoid a bias towards particularly well-studied languages by considering a large variety of (often underresearched) languages. Indeed, the languages dealt with include, among others, \ili{Albanian}, \ili{English}, \ili{German},  \ili{Marathi}, \ili{Panãra}, \ili{Tungusic} and \ili{Wú Chinese}. In addition, the volume is not committed to any one particular theoretical orientation. It does, however, intend  to contribute to ongoing theoretical debates and discussions between linguists with a different theoretical background. The book is thought to appeal equally to anyone interested in diachronic and historical linguistics, typology, and theoretical modelling. Likewise, it will be of interest to phonologists, morphologists, syntacticians and semanticists.

The first two contributions discuss (potentially) pejorative change in the field of phonology.

\textsc{\name{Matthew}{Faytak}} embeds the question of potentially pejorative language change in an explicitly evolutionary framework by comparing the claim of language change always being optimizing with the strict adaptationist stance in biology, which has been challenged e.g. by \citet{gould-lewontin}. Whereas non-adaptive change is not uncommon in biology (cf. the development of ``spandrels''), and can certainly be found in language, too, \textsc{Faytak} goes one step further and raises the question whether there are examples of maladaptive language change. He discusses high vowel fricativization in \ili{Sūzhōu Chinese}, a dialect of \ili{Northern Wú Chinese}, as a possible example of the latter. The production of fricativized vowels requires a high degree of articulatory precision without any obvious functional gain in the specific case of Northern Wú. After presenting phonetic evidence from an ultrasound tongue imaging study, \textsc{Faytak} turns to the diachrony of high vowel fricativization, which is to be analyzed as an example of transphonologization (one acoustic signal of a phonological category is used in place of another), but crucially, the process in question cannot be functionally motivated as a strategy for contrast maintenance. Therefore, the change seems to have occurred and spread across a relatively small area by chance. However, it is very likely that high vowel fricativization has had a chance to spread due to a social function attached to it. If this is right, social factors favoring a change may well overrule factors of communicative efficiency (disfavoring the change), an antagonism that cannot easily be replicated in the field of biological evolution. (We might even speculate that ``unnatural'' forms, i.e. forms that are maladaptive in terms of communicative efficiency, are particularly prone to developing social meanings in terms of prestige \textit{because} they are difficult to produce?)

\textsc{\name{Myriam}{Lapierre}} discusses post-oralized and devoiced nasals in \ili{Panãra} (\ili{Jê}), where ND sequences turned into NT. Typological and diachronic crosslinguistic evidence is very robust that this change is unnatural insofar as it is the opposite of the much more common (and for aerodynamic reasons phonetically well-grounded) process NT > ND. Examples of ND > NT from other languages can be explained as the result of a telescoping sequence of individual but more natural sound changes. However, \textsc{Lapierre} convincingly shows that from \ili{Proto-Northern-Jê} to \ili{Panãra} ND directly turned into NT (i.e., as a single sound change). \textsc{Lapierre}'s solution to the puzzle is based on two arguments. First, instead of looking at ND > NT in isolation she analyzes it in the context of the \ili{Panãra} phonological system as a whole. \ili{Panãra} has a phonemic contrast between oral and nasal vowels. Postoralization of nasal consonants is analyzed as an allophonic realization of nasal stops when they are followed by an oral vowel. Second, she argues that phonetic naturalness may be grounded either in articulation or in perception. In the case at hand, devoicing of denasalized nasal stops enhances the perceptive salience of the phonemic contrast between oral and nasal vowels following them.

Possible examples of language change for the worse in the domain of morphosyntax are discussed in the next three contributions.

\textsc{\name{Christine}{Elsweiler}} and \textsc{\name{Judith}{Huber}} examine the loss of the number contrast in the \ili{English} second person pronoun. As is well-known, \ili{English} used to distinguish a singular form \textit{thou} and a plural form \textit{you}, but it lost the distinction through the extension of \textit{you} to the singular and the subsequent loss of \textit{thou}, a development completed by the eighteenth century. \textsc{Elsweiler} and \textsc{Huber} first establish that the loss of the number contrast may legitimately be labelled a change for worse; evidence is seen in the fact that most spoken varieties have developed repair strategies, i.e. new plural forms such as \textit{you guys} or \textit{you all}. However, the authors also hypothesize that the pejoration in question may be viewed as a by-product of two changes for the better: First, at least initially (viz. until the eventual demise of \textit{thou})  the extension of \textit{you} to the singular led to a two-term address system allowing for nuanced pragmatic distinctions with respect to e.g. politeness, intimacy and distance -- an improvement from the perspective of sociopragmatics. Second, once \textit{you} was available as a form for singular address, its increasing use may have been driven by deflexion: It allowed speakers to avoid the verbal suffix \textit{-st} triggered by \textit{thou}, resulting in a simplified inflectional paradigm --  a change for the better from a structural perspective, particularly against the background of the language and dialect contact situation in early modern London.

\textsc{\name{Veton}{Matoshi}} investigates the use of clitic doubling in \ili{Albanian} dialects from the perspective of functional transparency. In \ili{Albanian}, as in other Balkan languages, an object may additionally be marked by a depronominal clitic. Crucially, however, not all objects trigger clitic doubling; thus a key question is to determine the conditions under which it does or does not occur. \textsc{Matoshi} first provides the relevant theoretical and typological background on agreement and transitivity, pointing out that \ili{Albanian} clitic doubling constitutes a manifestation of the more general phenomenon of differential object marking. It is expected (i) that object marking is prone to occur where the object shows characteristics typical of subjects (such as +animate/human, +definite, +specific, +given, +topic, \textminus{}foc) and (ii) that in the course of a grammaticalization process, object marking may be generalized to all objects, thus loosing its original functional motivation. In an empirical section, \textsc{Matoshi} analyzes a newly-compiled \ili{Albanian} dialect corpus comprising data not only from the Republic of Albania but also from the \ili{Albanian}-speaking regions of neighboring countries. He shows that there are areal differences in the usage frequency of clitic doubling, which are taken to reflect different degrees of grammaticalization. In particular, the varieties spoken outside Albania use clitic doubling more frequently than most varieties spoken in Albania. In some varieties (specifically Montenegro\il{Montenegro Albanian} and Kosova\il{Kosova Albanian}) doubling tends towards a loss of any pragmatic, morphological or semantic restrictions, amounting to an increasing degree of functional opaqueness.

\textsc{\name{Tabea}{Reiner}} investigates a verb construction in \ili{German} whose very existence is disputed: a pattern whereby the infinitive-selecting auxiliary \textit{werden}, which itself usually only occurs in finite form, occurs in the infinitive. Given that \textit{werden} is often (though not undisputedly) analyzed as a future auxiliary, the construction in question is viewed as a (potential) posterior infinitive. After conducting a typological survey, \textsc{Reiner} concludes that this temporal category seems to be rare in the languages of the world, which raises the questions of \textit{why} it is rare and what functions it serves in those languages where it does occur. The latter question is then addressed on the basis of the \ili{German} construction. \textsc{Reiner} first shows that non-finite \textit{werden} can indeed be attested in corpora (if only rarely), and after an analysis of its structural, semantic and distributional properties, \textsc{Reiner} defends the view that it constitutes a posterior infinitive. However, she also concludes that it does not appear to offer any merits compared to the ``simple'' (viz. \textit{werden}-less) infinitive. Consequently, \textsc{Reiner} argues that the emergence of the construction is tough to motivate functionally (be it as a means of disambiguation, on the basis of Haspelmath's (\citeyear{Haspelmath1999}) notion of ``extravagance'' or as an example of hypercorrection). Instead, she proposes that it is motivated by analogy, with two constructions serving as models: (i) the (much more common) anterior infinitive, i.e. another temporally marked infinitive, and (ii) the passive infinitive, which, involving the same auxiliary, provides a model for non-finite \textit{werden}. According to \textsc{Reiner}, the analogy is ``functionally blind'', but, crucially, this blindness is only local. From a global perspective, analogical extensions of the type in question are considered to increase systematicity and processability.

The following contribution deals with language change on the semantic level exploring the morphosyntactic marking of a particular–characterizing meaning contrast.

\textsc{\name{Ashwini}{Deo}} considers a previously uninvestigated semantic contrast conveyed by copulas/auxiliaries that is common in several \ili{New Indo-Aryan} languages. She shows that the contrast in question can be best analyzed as lexicalizing particular vs. characterizing meanings. By investigating the origin of this contrast, based on historical data mostly drawn from \ili{Marathi}, \textsc{Deo} provides evidence indicating that the expression of this contrast is found to be grammatically categorical through the interpretational possibilities for the \textit{bh\={u}} copula and its cognates in the Modern \ili{New Indo-Aryan} languages.  While Middle\il{Middle Indo-Aryan} and \ili{Old Indo-Aryan} appear to show sensitivity to the semantic distinction between particular and characterizing claims, there is no specialized morphosyntactic device for conveying particular claims in these systems. Thus, \textsc{Deo} proposes that the \ili{New Indo-Aryan} languages may have transitioned into a strategy in which the contrast is categorically expressed as a secondary consequence of a change in their broader tense marking systems. \textsc{Deo} uses this observed categoricalization to reflect on whether the morphosyntactic marking of the particular–characterizing contrast represents a change for the better or for the worse.

Issues of complexification as a potential change for the worse are discussed in the next two contributions.

\textsc{\name{Andreas}{Hölzl}} investigates the reconstruction of the proto-\ili{Tungusic} pho\-neme \textit{*K} and its implications for \ili{Tungusic} interrogative systems. \textit{*K-} served the function of an interrogative submorpheme (or ``resonance''), similar to \ili{English} \textit{wh-}, but in most \ili{Tungusic} subbranches (with the exception of \ili{Nanaic}) it is lost, a change that led to very incoherent and opaque interrogative systems. Based on rediscovered data from \ili{Alchuka}, \textsc{Hölzl} proposes a new reconstruction of the proto-Tungusic\il{Proto-Tungusic} phoneme \textit{*K-} (and its loss). The consequences of the loss of \textit{*K-} for interrogative systems are discussed in the context of a general framework for complexity, for which \textsc{Hölzl} proposes seven distinct dimensions (regularity, redundancy, analyzability, amount, organization, coherence, delineation), whereby the dimension of coherence turns out to be the most relevant one for the issues discussed in the paper. \textsc{Hölzl} concludes with a critical discussion of the evaluative flavor of classifying a change as ``for the better'' or ``for the worse''.

\textsc{\name{Johanna}{Nichols}}, in her programmatic paper, discusses the complexity of paradigms, more specifically inflectional marking of argument roles in verbal paradigms. These paradigms are notorious for their great degree of complexity especially in polysynthetic languages. \textsc{Nichols} argues that while recent years have seen much progress in measuring the complexity of paradigms understood as outputs (e.g. in the light of canonical typology), little is known yet about the ``blueprint'' the complex output is based on. The paper pursues the hypothesis that whereas most existing work on morphological complexity has focused on the paradigmatic axis only, the blueprint underlying participant marking in verbal inflection is best approached from the perspective of the relational (or, in Saussure's terms, syntagmatic) axis. \textsc{Nichols} then applies complexity measures originally developed for the paradigmatic axis (such as biuniqueness) to the relational axis. What is canonical and what is complex turns out to be quite different depending on a paradigmatic or relational perspective. Relational canonicality is analyzed as the result of two general principles, namely the reduction of informational complexity (by drawing on universal patterns such as person hierarchies) on the one hand and (from a processing perspective) the salience of relational markers on the other.

Modelling language change: This section deals with how change for the worse may be accounted for within different theoretical models.

\textsc{\name{Gerhard}{Jäger}} investigates the overall question of this volume whether languages can change for the worse from an evolutionary theoretical perspective, arguing that evolving systems can be compared with regard to their fitness, and that “worse” can be translated as “less fit”. From this point of view, the paper pursues the hypothesis that the initial question has an analogy to the issue of whether biological Darwinian evolution can lead to the reduction of fitness. Following much recent work in historical and evolutionary linguistics, \textsc{Jäger} assumes that biological evolution and language change are two instances of an overarching principle of evolution via replication and selection. He then applies George Price’s mathematical model of Darwinian evolution to conceptual questions such as the one discussed here and spells out why Price’s approach is useful for the study of language change in general; \citet{Price_1995} sees selection as a very general mechanism that has been studied intensely in biology but is also at work in other domains. \textsc{Jäger} shows that parts of the language system can become worse in the sense that they are changed towards or replaced by alternatives that would be less fit than the original version under similar circumstances.

\textsc{\name{Roland}{Mühlenbernd}} presents an introduction to and a tutorial on game theoretic approaches within an evolutionary framework in the study of language variation and change. He discusses \citegen{Jäger_2007} Case game and \citegen{Deo_2015} Imperfective game as two pioneering and influential applications of evolutionary game theory (EGT) to phenomena in diverse grammatical domains and to problems of language change within the context of diachrony and stability aspects of grammar: Following \citet{Jäger_2007}, he focuses on case-marking patterns that help disambiguate syntactic core roles (such as nominative and accusative) in transitive sentences. Following \citet{Deo_2015} he addresses the diachrony of the Imperfective aspect and its cross-linguistically attested distinct sub-readings, the progressive and the habitual. After presenting the grammatical and empirical background to these two games, he exemplifies how individual grammars, namely grammatical systems, can be modeled within a model-theoretic approach by proposing a step-by-step application of important key notions and concepts of EGT in the process, such as \textit{evolutionary stability} or the \textit{replicator dynamics}. \textsc{Mühlenbernd} concludes by situating the framework of EGT within the broader context of language change for the worse and extends this perspective to the explanatory potential of how language use might drive grammatical change.  

\textsc{\name{Gerhard}{Schaden}} makes and investigates three major claims in his study: First, linguistic resources such as phonemes and constructions can be analyzed as Public Goods; second, under some circumstances and for some linguistic entities, the socially differentiating use of these expressions makes them Common Pool Resources; and third, communication is seen as a cooperative endeavor where systematic and intrinsic conflicts between speaker and hearer will lead to the Tragedy of the Commons. \textsc{Schaden} shows how this optimization problem can be modeled within a game-theoretic framework as a signaling game. He discusses the consequences of his approach in the context of two case studies of the Tragedies of Commons: the loss of syllable-final -\textit{s} in \ili{Western Romance} Languages and the aoristic drift of the present perfect. As for the present perfect, \textsc{Schaden}’s model predicts that the more frequent the present perfect is, the less it denotes current relevance. Consequently, he argues that simple past tenses do not trigger any inference that the event will have current relevance. \textsc{Schaden} concludes within the broader context of language change for the worse by arguing that such Tragedies of the Commons constitute cases where short-term advantages of the speaker lead to a long-term complication for successful communication.  


%\section*{Abbreviations}
\section*{Acknowledgments}


The present volume is the result of a highly collaborative endeavor in which not only the editors and authors of the chapters were involved. We would like to thank in particular our external reviewers for their critical and constructive assessment, whose comments have contributed to considerable improvement of all chapters: Artemis Alexiadou, Raffaela Bächler, Rajesh Bhatt, Walter Bisang, Ellen Brandner, Miriam Butt, Dan Dediu, Elvira Glaser, Dalina Kallulli, Henri Kauhanen, Götz Keydana, Felicitas Kleber, Christian Langstrof, Matti Miestamo, Tatiana Nikitina, Andrey Nikulin, Svetlana Petrova, Gareth Roberts, Björn Rothstein, Peter Trudgill, Søren Wichmann, and Alan Yu. Moreover, we thank Florian Fleischmann and Stefan Pfister for their invaluable help with typesetting. We would also like to thank the editorial board of the series Studies in Diversity Linguistics, in particular, series editor, Martin Haspelmath, whose words of encouragement have been very important from the outset to the final stages of the volumes’ production. This book project originated from a trans-Atlantic workshop on ``Language change for the worse'', jointly held by the LMU Germanic Linguistics Section and the UC Berkeley Linguistics Department at LMU Munich on 27--28 May 2017. We thank the IBZ Munich (The International Encounter Center for Science in Munich) for hosting the workshop, as well as the participants and the audience for contributing to an inspiring and stimulating discussion venue. The workshop was ultimately made possible thanks to the financial support through the LMU-UCB Research in the Humanities scheme (LMUexcellent), a funding program for collaborative research in the humanities between LMU Munich and the University of California, Berkeley.

{\sloppy\printbibliography[heading=subbibliography,notkeyword=this]}

\end{document}
