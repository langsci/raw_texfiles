\documentclass[output=paper,hidelinks]{langscibook}
\ChapterDOI{10.5281/zenodo.13347668}

\title{The particular--characterizing contrast in Marathi and its historical basis}
\author{Ashwini Deo\affiliation{The Ohio State University; The University of Texas at Austin}}

\abstract{The two-copula and two-auxiliary systems instantiated in a number of Indo-Aryan languages can be best analyzed as lexicalizing particular vs. characterizing meanings. The data is drawn from Marathi [mar, 71,700,000 speakers] but the pattern is observed across different sub-groups of Indo-Aryan, and is possibly inherited from Sanskrit. I provide historical data that indicates that while this contrast is not grammatically categorical in Middle and Old Indo-Aryan, it appears to be present through the interpretational possibilities for the \emph{bh\={u}} copula and its cognates. I use this observed categoricalization of a copular contrast to reflect on whether the overt marking of the particular--characterizing contrast represents a change for the better or for the worse.}

\IfFileExists{../localcommands.tex}{
 \addbibresource{../localbibliography.bib}
 \usepackage{langsci-optional}
\usepackage{langsci-gb4e}
\usepackage{langsci-lgr}

\usepackage{listings}
\lstset{basicstyle=\ttfamily,tabsize=2,breaklines=true}

%added by author
% \usepackage{tipa}
\usepackage{multirow}
\graphicspath{{figures/}}
\usepackage{langsci-branding}

 
\newcommand{\sent}{\enumsentence}
\newcommand{\sents}{\eenumsentence}
\let\citeasnoun\citet

\renewcommand{\lsCoverTitleFont}[1]{\sffamily\addfontfeatures{Scale=MatchUppercase}\fontsize{44pt}{16mm}\selectfont #1}
  
 %% hyphenation points for line breaks
%% Normally, automatic hyphenation in LaTeX is very good
%% If a word is mis-hyphenated, add it to this file
%%
%% add information to TeX file before \begin{document} with:
%% %% hyphenation points for line breaks
%% Normally, automatic hyphenation in LaTeX is very good
%% If a word is mis-hyphenated, add it to this file
%%
%% add information to TeX file before \begin{document} with:
%% %% hyphenation points for line breaks
%% Normally, automatic hyphenation in LaTeX is very good
%% If a word is mis-hyphenated, add it to this file
%%
%% add information to TeX file before \begin{document} with:
%% \include{localhyphenation}
\hyphenation{
affri-ca-te
affri-ca-tes
an-no-tated
com-ple-ments
com-po-si-tio-na-li-ty
non-com-po-si-tio-na-li-ty
Gon-zá-lez
out-side
Ri-chárd
se-man-tics
STREU-SLE
Tie-de-mann
}
\hyphenation{
affri-ca-te
affri-ca-tes
an-no-tated
com-ple-ments
com-po-si-tio-na-li-ty
non-com-po-si-tio-na-li-ty
Gon-zá-lez
out-side
Ri-chárd
se-man-tics
STREU-SLE
Tie-de-mann
}
\hyphenation{
affri-ca-te
affri-ca-tes
an-no-tated
com-ple-ments
com-po-si-tio-na-li-ty
non-com-po-si-tio-na-li-ty
Gon-zá-lez
out-side
Ri-chárd
se-man-tics
STREU-SLE
Tie-de-mann
}
 \togglepaper[7]%%chapternumber
}{}

\renewcommand{\lsSeries}{sidl}
\begin{document}
\maketitle
\renewcommand{\exfont}{\normalfont\itshape}
\section{Introduction}

 The goal of uncovering systematic principles governing cross-linguistic variation in (the realization of) meaning has been strongly pursued in semantics over the past two decades. One result of this research has been the identification of recurring similarities in the meaning contrasts that get reliably encoded across diverse grammatical systems.
 Yet another issue that emerges from the same pursuit is that of variation with respect to how a universal inventory of model theoretic components is mapped onto lexical/functional items. Specifically, we find that certain elements of functional meaning that are covert in some languages may find overt realization in others.
For instance, languages differ with respect to whether ``event-in-progress" and ``habitual/generic" meanings are expressed by distinct aspectual markers (progressive and imperfective) or by a single imperfective marker (see \citealt{bybee-gram1994, comrie76, deo2009c, Deo2015sp}, among others). Languages may also choose to lexicalize or keep covert the semantic contrast between ``alienable possession" and ``inalienable possession" (\citealt{clark1978, aristar1996, Stassen09}). This variation in languages has to do with whether salient semantic contrasts are individually packaged and lexicalized, or whether they are subject to contextual disambiguation. Given typological variation in this respect, one might ask the question: Is the ``individualized packaging" strategy more complex than a ``contextual disambiguation" strategy?

In this paper, I will consider a previously undescribed phenomenon -- a morphosyntactic contrast in copulas/auxiliaries that is pervasive in several \ili{New Indo-Aryan} languages. As I will show, restricting myself to the \ili{Marathi} facts (with a brief nod to \ili{Hindi}), the morphosyntactic contrast reflects a semantic distinction between particular and characterizing claims. When one considers the origin of this contrast, one finds that categoricality in the expression of this contrast is only to be found in the Modern \ili{New Indo-Aryan} languages. While earlier stages of \ili{Indo-Aryan} (Middle\il{Middle Indo-Aryan} and \ili{Old Indo-Aryan}) appear to show sensitivity to the semantic distinction between particular and characterizing claims, there is no specialized device for conveying particular claims in these systems. I suggest that the \ili{New Indo-Aryan} languages may have transitioned into a strategy in which this contrast is categorically expressed as a secondary consequence of a change in their broader tense marking systems.

\section{The phenomenon}\largerpage
Although they contain the same tensed form of the copula \emph{be}, the (a) and (b) sentences of \ili{English} in (\ref{bath}--\ref{asleep}) are understood very differently with regard to their temporal reference.

\ea \label{bath}
\ea The baby \emph{is} tired. Let's get the bath ready. \label{batha}
\ex The baby \emph{is} tired by the time we pick him up from daycare. So let us start picking him up earlier. \label{bathb}
\z

\ex \label{people}
\ea People \emph{are} unhappy because they just raised the taxes. \label{peoplea}
\ex People \emph{are} unhappy when they are on diets.\footnote{\url {http://www.brainyquote.com/quotes/quotes/m/mireillegu530744.html}} \label{peopleb}
\z

\ex \label{asleep}
\ea Sam \emph{was} asleep. He had had a long day.
\ex Whenever Mary telephoned on a Friday, Sam \emph{was} asleep. \normalfont{\citep[246]{partee84}}
\z
\z

The (a) sentences intuitively seem to be about a particular salient time -- either the time of utterance or a salient past time. The (b) sentences, on the other hand, do not seem to make reference to any particular time, but rather describe a larger situation, extending over an indefinite interval, characterized by a predictable recurrence of relevant episodes under certain circumstances.
For instance, while \REF{batha} conveys that the baby is tired at the time that the sentence is uttered, \REF{bathb} conveys that (almost) every time (within some larger contextually understood interval) at which the baby is picked up from daycare is a time when the baby is tired. Similarly, \REF{peoplea} conveys that people are unhappy at utterance time because of a rise in taxes, while \REF{peopleb} conveys that there is a tendency for individuals to be unhappy during the times that they are on a diet.\largerpage

\ili{Marathi} morphosyntactically distinguishes between these two uses of the \ili{English} tensed copula. In both the past and the present tenses, the language uses distinct copular paradigms to express the senses corresponding to (a) and (b). These are glossed as \textsc{cop1} and \textsc{cop2} respectively.
To compare with \ili{English}, note that \REF{mar1a}, which describes a single episode of the baby being tired, contains a form of \textsc{cop1} \emph{āhe}, contrasting with \REF{mar1b}, where the presence of the \textsc{cop2} form \emph{asta} signals that a recurring generalization over episodes is being described. Similar contrasts hold between the \textsc{cop1} and \textsc{cop2} sentences in \REF{mar2} and \REF{mar3}, with \REF{mar3} exemplifying the past referring copular forms.

\ea \label{mar1}
\ea
\gll \emph{bāḷ} \emph{thaklel-a} \textbf{āhe}\\
baby.\textsc{nom.n.sg} tired-\textsc{n.sg} \textsc{cop1.pres.3sg}\\
\glt `The baby is tired.' \label{mar1a}
\ex
\gll \emph{bāḷ} \emph{sandhyākāḷī} \emph{thaklel-a} \textbf{ as-ta}\\
baby.\textsc{nom.n.sg} evening.\textsc{loc} tired-\textsc{n.sg} \textsc{cop2-pres.3n.sg}\\
\glt `The baby is tired in the evenings.' \label{mar1b}
\z
\ex \label{mar2}
\ea
\gll \emph{karvāḍh-i} \emph{muḷe} \emph{lok-a} \emph{dukkhi} \textbf{āhe-t}\\
tax.increase-\textsc{obl} because people-\textsc{nom.n.pl} unhappy \textsc{cop1.pres-3pl}\\
\glt `The people are unhappy because of the tax increase.'
\ex
\gll \emph{ḍāyeṭ} \emph{kar-ṇāri} \emph{lok-a} \emph{dukkhi} \textbf{as-tāt}\\
diet do-\textsc{part.n.pl} people-\textsc{nom.n.pl} unhappy \textsc{cop2-pres.3pl}\\
\glt `Dieters (lit. diet-doing people) are unhappy.'
\z
\ex \label{mar3}
\ea
\gll \emph{rām} \emph{sandhyākāḷī} \emph{dukkhi} \textbf{hotā}\\
Rām.\textsc{nom.m.sg} evening.\textsc{loc} unhappy \textsc{cop1.past.3m.sg}\\
\glt `Rām was unhappy in the evening.'
\ex 
\gll \emph{rām} \emph{sandhyākāḷī} \emph{dukkhi} \textbf{as-āycā}\\
Rām.\textsc{nom.m.sg} evening.\textsc{loc} unhappy \textsc{cop2-past.3m.sg}\\
\glt `Rām used to be unhappy in the evenings.'
\z
\z

\ili{Hindi} exhibits the same semantic contrast but the morphosyntactic devices used to convey the contrast are slightly different. \ili{Hindi} has a single copular element and expresses characterizing claims by periphrastically combining the imperfective participial form of this copula with the tensed form. For example, \REF{hin1a}, which describes a single episode of the baby being happy, contains the simple tensed copular form \emph{hɛ}, contrasting with \REF{hin1b}, where the periphrastic imperfective construction \emph{hotā hɛ} signals that a recurring generalization over episodes is being described.

\ea \label{hin1}
\ea
\gll \emph{bacchā} \emph{khush} \textbf{hɛ}\\
baby.\textsc{nom.m.sg} happy \textsc{cop.pres.3sg}\\
\glt `The baby is happy.' \label{hin1a}

\ex
\gll \emph{bacchā} \emph{shām=ko} \emph{khush} \textbf{hotā} \textbf{hɛ}\\
baby.\textsc{nom.m.sg} evening=\textsc{dat} happy \textsc{cop.impf.m.sg} \textsc{cop.pres.3sg}\\
\glt `The baby is happy in the evenings.' \label{hin1b}
\z
\z

\begin{sloppypar}
Copulas are commonly taken to be the carriers of tense/aspect/modality distinctions without any additional lexical semantic contribution. In many languages (including \ili{English} and \ili{Marathi}), copulas in non-verbal clauses are identical to auxiliaries in verbal clauses, enabling further articulation of TAM distinctions in the linguistic system. For instance, in its auxiliary function \emph{be} is used in the realization of the progressive aspect (\emph{is/was/will be V-ing}) as well as the prospective aspect (\emph{is/was/will be going to V}) in \ili{English}. The two \ili{Marathi} copular paradigms behave similarly in that they are used as auxiliaries in marking the progressive and perfect aspects in the linguistic system, yielding contrasts in interpretation that remain covertly expressed in \ili{English}.\footnote{It is interesting that the distinction between the two copulas is neutralized in the non-finite part of the system. There are no non-finite forms of \textsc{cop1} and only \textsc{cop2} forms are available in non-finite contexts regardless of whether the claim is particular or characterizing.}
\end{sloppypar}

\begin{sloppypar}
Consider the contrast in (\ref{cont-a}--\ref{cont-b}) and (\ref{cont-c}--\ref{cont-d}). While the aspectual morphology remains the same in each pair, the interpretation of each member is clearly distinct. Example \REF{cont-a} asserts that the reference time is contained in an event of John smoking, while \REF{cont-b} conveys that all/most contextually relevant times (within some larger stretch of time) are contained in an event of John smoking. In \REF{cont-c}, the reference time is understood to be located after the time of an event of John's making dinner and setting the table. The example in \REF{cont-d}, in contrast conveys that in general, the time of my return is located after the time of an event of John's making dinner and setting the table. 
\end{sloppypar}

\ea \label{cont}
\ea John \emph{is smoking} in the common room (right now). \hfill{(episodic)} \label{cont-a}
\ex John \emph{is} always/often \emph{smoking} in the common room.\hfill{(characterizing)} \label{cont-b}
\ex John \emph{has made} dinner and \emph{set} the table (right now). \hfill{(episodic)} \label{cont-c}
\ex By the time I return, John \emph{has made} dinner and \emph{set} the table. \hfill{(characterizing)} \label{cont-d}
\z
\z

In \ili{English}, the contrast between episodic/particular and characterizing interpretations of verbal periphrases is facilitated by the presence of quantificational adverbial material -- lexical expressions in \REF{cont-b} and clausal material in \REF{cont-d}. In \ili{Marathi}, while such disambiguating material may be present, the episodic vs.\ characterizing readings are clearly and obligatorily disambiguated by the choice of auxiliary -- glossed \textsc{aux1} and \textsc{aux2}.
The sentence in \REF{marcontrasta} obligatorily conveys that a smoking event is ongoing at reference time and can never be used (even with overt quantificational adverbs) to express something like \REF{cont-b}.\footnote{The imperfective participle+\textsc{aux1} periphrasis is the general exponent of the progressive aspect in \ili{Marathi}. The imperfective participle optionally inflects for gender and number in this periphrasis.} In fact, \textsc{aux1} is unacceptable in sentences that contain overt quantificational adverbs. Example \REF{marcontrastb} correspondingly has only a characterizing interpretation, even in the absence of quantificational adverbs.
 
\ea \label{marcontrast}
\ea
\gll \emph{John} \emph{sigreṭ} \textbf{pī-t} \textbf{āhe}\\
John.\textsc{nom.m.sg} cigarette.\textsc{nom.f.sg} drink-\textsc{impf.part} \textsc{aux1.pres.3sg}\\
\glt
`John is smoking a cigarette.' \hfill{(episodic)} \label{marcontrasta}
\ex
\gll \emph{John} (\emph{nehmi/kadhi-kadhi}) \emph{sigreṭ} \textbf{pī-t} \textbf{as-to}\\
John.\textsc{nom.m.sg.} always/sometimes cigarette.\textsc{nom.f.sg} drink-\textsc{impf.part} \textsc{aux2-pres.3m.sg}\\
\glt `John is always/sometimes smoking cigarettes.' \hfill{(characterizing)} \label{marcontrastb}
\z
\z

The same distinction is made with the perfect aspect, where the disambiguation between episodic and characterizing readings is effected by the choice of the tense auxiliary.\largerpage

 \ea \label{marcontrast2}
\ea
\gll \emph{John=ne} \emph{svayampāk} \textbf{banav-lelā} \textbf{āhe}\\
 John=\textsc{erg} meal.\textsc{nom.m.sg} make-\textsc{perf.m.sg} \textsc{aux1.pres.3sg}\\
\glt `John has made dinner.' \hfill{(episodic)}\label{marcontrast-c}
\ex
\gll \emph{John=ne} \emph{svayampāk} \textbf{banav-lelā} \textbf{as-to}\\
 John=\textsc{erg} meal.\textsc{nom.m.sg} make-\textsc{perf.m.sg} \textsc{aux2-pres.3m.sg}\\
\glt `(By the time I return), John has made dinner.'\hfill{(characterizing)} \label{marcontrast-d}
\z
\z

This paper investigates this particular type of split copula/auxiliary system found in \ili{Marathi}. In fact, this appears to be a genetic feature, since several \ili{Indo-Aryan} languages, including \ili{Hindi}, \ili{Gujarati}, and \ili{Ahirani}, also exhibit this abstract pattern differing only with respect to the exponents that realize it (as briefly shown for \ili{Hindi}).\footnote{\ili{Bangla} and \ili{Oriya} \citep{mahapatra2009} also exhibit multi-copula/auxiliary systems, but the presence of more than two such elements (as well as the possibility of zero-copula constructions) in these languages yields a different pattern than the one present in the more commonly attested two-copula systems. A thorough investigation of these patterns must be left for later investigation.} The observed pattern demonstrates that semantic distinctions in the interpretations of tensed sentences that are only covertly made in some linguistic systems (e.g. the \ili{English} copula/auxiliary \emph{be}) can be teased apart systematically due to how markers of temporal reference are lexicalized in the systems of other languages. The division of labor effected by this two-copula/auxiliary system in \ili{Indo-Aryan} has not yet been described as a possible pattern in either the typological or the semantic literature, making its study particularly interesting from the typological perspective as well. The question for cross-linguistic semantic variation presented by the observed system can be framed as follows: why do some linguistic systems employ the same grammatical device to convey both particular and characterizing claims while other linguistic systems obligatorily signal this difference with distinct devices? Related to the question of semantic variation is a diachronic question: how does a contrast such as the one found in \ili{Marathi} and other \ili{Indo-Aryan} languages morphosyntactically emerge in languages? Focusing on this latter question here, I will suggest that the categorical nature of this contrast in \ili{Marathi} arises from changes in the tense marking system in the transition from \ili{Middle Indo-Aryan} to the \ili{Early New Indo-Aryan} languages. In a nutshell, the \ili{Middle Indo-Aryan} system is aspectually based and lacks the morphosyntactic means to mark the past-present distinction. In \ili{Late Middle Indo-Aryan}, a new tense auxiliary emerges with specific properties: it presupposes contextually salient intervals and anchors the proposition to the utterance world. This innovation effects a contrast in the copula/auxiliary system in which the contrast between particular and characterizing claims gets obligatorily expressed.

The rest of this paper is organized as follows. \S\ref{data} describes in detail the interpretations associated with the two paradigms in Modern \ili{Marathi}. I will only consider the effect of the contrast in copular clauses without any quantificational adverbs since this is enough to show that there are clear differences in interpretation that are obtained with individual-denoting vs.\ kind-denoting subjects and stage-level vs.\ individual-level predicates in combination with the relevant forms. In \S\ref{historical}, I present data from Epic \ili{Sanskrit} and \ili{Middle Indo-Aryan} to show that this contrast, while not identically manifested at these stages, is already partly realized by the presence of a copula that overwhelmingly occurs with characterizing readings. In \S\ref{discconc}, I discuss the changes with respect to their effect on the overall complexity of the system and conclude.


 \section{The Marathi facts}\label{data}
The relevant paradigms of the present and past tense copulas/auxiliaries are given in Tables~\ref{tab:present} and~\ref{tab:present2}. The forms exhibit agreement along the morphological categories of person and number, and also in many cases, gender. Gender-based contrast has been noted within each person-number cell in which it occurs, in the order masculine/feminine/neuter. The data presented henceforth contains examples only in the present tense since the facts are largely comparable for the past tense cases.

\begin{table}
\caption{Present tense copula/auxiliary paradigms of Marathi\label{tab:present}}
\begin{tabular}{lllll}
\lsptoprule
& \multicolumn{2}{c}{\textsc{cop1}} & \multicolumn{2}{c}{\textsc{cop2}}\\
 \cmidrule(lr){2-3}\cmidrule(lr){4-5}
 & \textsc{sg (m/f/n)} & \textsc{pl (m/f/n)} & \textsc{sg (m/f/n)} & \textsc{pl (m/f/n)}\\
 \midrule
1 & \emph{āhe} & \emph{āhot} & \emph{asto/aste} & \emph{asto}\\
2 & \emph{āhes} & \emph{āhāt} & \emph{astos/astes} & \emph{astā}\\
3 & \emph{āhe} & \emph{āhet} & \emph{asto/aste/asta} & \emph{astāt}\\
\lspbottomrule
\end{tabular} 
\end{table}

\begin{table}
\caption{Past tense copula/auxiliary paradigms of Marathi\label{tab:present2}}
\begin{tabularx}{\textwidth}{lllQQ}
\lsptoprule
& \multicolumn{2}{c}{\textsc{cop1}} & \multicolumn{2}{c}{\textsc{cop2}}\\
\cmidrule(lr){2-3}\cmidrule(lr){4-5}
 & \textsc{sg (m/f/n)} & \textsc{pl (m/f/n)} & \textsc{sg (m/f/n)} & \textsc{pl (m/f/n)}\\
\midrule
1 & \emph{hoto/hote} & \emph{hoto} & \emph{asāyco/asāyce} & \emph{asāyco}\\
2 & \emph{hotās/hotis} & \emph{hotā(t)} & \emph{asāycās/asāycis} & \emph{asāycā(t)}\\
3 & \emph{hotā/hoti/hota} & \emph{hote/hotyā/hoti} & \emph{asāycā/asāyci}\slash\emph{asāyca} &\emph{asāyce/asāycyā}\slash \emph{asāyci}\\
\lspbottomrule
\end{tabularx} 
\end{table}

\subsection{Copular clauses}
We now examine non-verbal predicational copular clauses, which consist of a subject, a non-verbal adjectival or nominal element or a postpositional phrase, and the relevant copula. The organization of the data is by the syntactic type of the subject -- names and bare nominals, and within each category, by the episodicity of the predicate -- i.e. whether it is most naturally construable as a stage-level or individual-level property.

\subsubsection{Names}
With stage-level predicates, the two copulas contrast particular temporally delimited claims (\textsc{cop1}) with habitual generalizations (\textsc{cop2}). In \REF{stage1a} the use of \textsc{cop1} conveys that the property of being busy or angry holds of Anu at the utterance time. In \REF{stage1bb}, the use of \textsc{cop2} obligatorily conveys that over some indefinite interval of time, there are recurring, regularly instantiated episodes of Anu being busy or angry.\footnote{The use of the latter adjective does not convey that Anu is an angry person but rather that Anu is often/regularly found in a state of anger.}

 \ea \label{stage1}
\ea
\gll \emph{anu} \emph{vyasta}/\emph{cidleli} \textbf{āhe}\\
Anu.\textsc{nom.f.sg} busy/angry \textsc{cop1.pres.3sg}\\
\glt `Anu is busy/angry (right now).' \label{stage1a}
\ex
\gll \emph{anu} \emph{vyasta}/\emph{cidleli} \textbf{as-te}\\
Anu.\textsc{nom.f.sg} busy/angry \textsc{cop2-pres.3.f.sg}\\
\glt `Anu is generally busy/angry.' \label{stage1bb}
\z
\z
\il{Marathi}

Sentences \REF{loca} and \REF{locb} provide examples in which the main predicate is a locative prepositional phrase, another instance of stage-level predication. The observation is identical: the two copulas contrast in whether the assertion pertains to the utterance time or conveys some generalization that holds at some larger interval including the utterance time.

\ea \label{loc}
\ea
\gll
 \emph{anu} \emph{gharā-t} \textbf{āhe}\\
Anu.\textsc{nom.f.sg} house.\textsc{obl}-in \textsc{cop1.pres.3sg}\\
\glt `Anu is in the house (right now).' \label{loca}
 
\ex
\gll \emph{anu} \emph{gharā-t} \textbf{as-te}\\
Anu.\textsc{nom.f.sg} house.\textsc{obl}-in \textsc{cop2-pres.3f.sg}\\
\glt `Anu is generally in the house (e.g. when the postman comes by.)' \label{locb}
\z
\z
\il{Marathi}

When the main predicate is individual-level and denotes a relatively permanent, intrinsic property of an individual, only \textsc{cop1} is acceptable as shown in the contrast between \REF{indiva} and \REF{indivb}. The use of \textsc{cop2} introduces the sort of oddity that is associated with the use of quantificational adverbs with individual-level predicates (\citealt{kratzer95stage, chierchia1995, magri2009} among others).\footnote{This is the observation that a sentence like \emph{John is generally/always/often/sometimes intelligent} is understood as deviant or unacceptable without context.} It conveys that Anu habitually or generally has the property of being cowardly, tall, or intelligent, which is infelicitous because it tends to give rise to a scalar inference that this property only holds discontinuously in time, i.e. that it is possible that there are times when Anu is not cowardly, tall, or intelligent.

\begin{exe}
\judgewidth{\#}
\ex \label{indiv}
\begin{xlist}
\ex[]{
\gll \emph{anu} \emph{ghābraṭ/unca/hu\'{s}ār} \textbf{āh-e}\\
Anu.\textsc{nom.f.sg} cowardly/tall/intelligent \textsc{cop1.pres.3sg}\\
\glt `Anu is cowardly/tall/intelligent.'\label{indiva}}
\ex[\#]{
\gll \emph{anu} \emph{ghābraṭ/unca/hu\'{s}ār} \textbf{as-te}\\
Anu.\textsc{nom.f.sg} cowardly/tall/intelligent \textsc{cop2-pres.3f.sg}\\
\glt `Anu is (habitually) cowardly/tall/intelligent.' \label{indivb}}
\z
\z
\il{Marathi}
%Rightaway we see this is not the ser-estar temporary--permanent contrast.

 \subsubsection{Bare nominal subjects}
 Like many languages without determiners, \ili{Marathi} allows both bare singular and bare plural arguments, and, in subject position, these may be understood either as making reference to unique, contextually salient entities or as making reference to kinds. In \citet{dayal1999,dayal2004}, Veneeta Dayal makes a convincing case for \ili{Hindi}, using arguments from scopal (non-)interaction that bare singulars are not ambiguous between indefinite and definite interpretations in \ili{Hindi} (and other determiner-less languages). The \ili{Marathi} facts closely parallel the \ili{Hindi} facts and I will investigate the range of readings of bare nominals in \ili{Marathi} only in the context of copular clauses here.\\
 
\subsubsubsection{Stage-level predicates}
Consider the examples in \REF{barenomstagea} and \REF{barenomstageb}, which contain bare singular subjects and stage-level predicates. These sentences are most naturally interpreted as describing the properties of the contextually most salient dog in the utterance context -- i.e. the bare nominal has a directly referential use -- like an NP with the definite article in \ili{English}.

\ea \label{barenomstage}
\ea
\gll \emph{kutrā} \emph{thaklelā/bhukelā} \textbf{āhe}\\
dog.\textsc{nom.m.sg} tired/hungry \textsc{cop1.pres.3sg}\\
\glt `The dog is tired/hungry.'\label{barenomstagea}

\ex
\gll \emph{kutrā} \emph{thaklelā/bhukelā} \textbf{as-to}\\
dog.\textsc{nom.m.sg} tired/hungry \textsc{cop2-pres.3m.sg}\\
\glt `The dog is generally tired/hungry.'\label{barenomstageb}
\z
\z
\il{Marathi}

With locative predicates, the pattern remains the same: the contrast lies in whether the property of being in the house is said to hold of the most salient dog in the utterance context, at the utterance time (\textsc{cop1}) or more generally over an indefinite interval that contains the utterance time (\textsc{cop2}).

 \ea \label{barenomloc}
\ea
\gll \emph{kutrā} \emph{gharā-t} \textbf{āhe}\\
dog.\textsc{nom.m.sg} house.\textsc{obi}-in \textsc{cop1.pres.3sg}\\
\glt `The dog is in the house (right now).'\label{barenomloca}
\ex
\gll \emph{kutrā} \emph{gharā-t} \textbf{as-to}\\
dog.\textsc{nom.m.sg} house.\textsc{obl}-in \textsc{cop2-pres.3m.sg}\\
\glt `The dog is generally in the house.' \label{barenomlocb}
\z
\z
\il{Marathi}

In both \REF{barenomstage} and \REF{barenomloc}, the bare nominal subject has most naturally a directly referential reading -- its referent is understood to be an entity that is most salient in the utterance context (in the actual world at utterance time).

However there is another non-referential reading of bare nominals that arises with the use of \textsc{cop2}. For illustration, consider \REF{rangolib}, which contains \textsc{cop2}. Here, the bare singular \emph{rā\.{n}goḷī} does not refer to any particular contextually salient entity at utterance time in the actual world, but rather to the rangoli that gets drawn every day by Anu in front of her door.\footnote{Rangoli (\ili{Marathi} \emph{rā\.{n}goḷī}) is a traditional art form in which decorative patterns are created on the floor using materials such as colored rice, dry flour, colored sand, or flower petals.}

\ea \label{rangoli}
\textit{Context:} My friend is telling me about her sister Anu, who draws elaborate rangoli motifs in front of her house every day. She says:

\ea
\gll \emph{anu} \emph{roj} \emph{rā\.{n}goḷī} \emph{kāḍh-te}\\
Anu.\textsc{nom.f.sg} everyday rangoli.\textsc{nom.f.sg} draw-\textsc{impf.pres.3f.sg}\\
\glt `Anu draws a rangoli motif every day.' \label{rangolia}

\ex
\gll \emph{rā\.{n}goḷī} \emph{dārā-samor} \textbf{as-te}\\
rangoli.\textsc{nom.f.sg} door.\textsc{obl}-in.front.of \textsc{cop2-pres.3f.sg}\\
\glt `The rangoli (that she draws) is in front of the (main) door.'\label{rangolib}
\z
\z
\il{Marathi}

In the given context, \REF{rangolib} conveys that for each day $d$ within some indefinite interval overlapping with the utterance time, the unique rangoli $r_d$ that Anu draws on $d$, is located in front of the main door. It is infelicitous to follow up \REF{rangolia} with \REF{rangolic}, which contains \textsc{cop1}, since the bare nominal \emph{rā\.{n}goḷī}, in this case, can only be taken to refer to the contextually salient rangoli \emph{at utterance time}.

 \ea [\#]{
\gll \emph{rā\.{n}goḷī} \emph{dārā-samor} \textbf{āhe}\\
rangoli.\textsc{nom.f.sg} door.\textsc{obl}-in.front.of \textsc{cop1.pres.3sg}\\
\glt `The rangoli is (right now) in front of the (main) door.'} \label{rangolic}
\z
\il{Marathi}

In addition to the directly referential and non-referential readings described above, bare singular subjects may also be understood as kind-denoting. For instance, \REF{clock} has two salient readings: on the definite referential reading of the nominal, it may describe the general coordinates of a specific clock salient in the utterance context (for instance, the one my uncle gave me for my birthday). On the other reading, the sentence describes a generalization about
where clocks in general tend to be located (i.e. as a claim about the kind clocks).

 \ea
\gll \emph{ghaḍyāḷ} \emph{bhinti-var} \textbf{as-ta}\\
clock.\textsc{nom.n.sg} wall.\textsc{obl}-on \textsc{cop2-pres.3n.sg}\\
\glt \emph{definite referential:} `The clock [my uncle gave me] is (generally) on the wall.' \label{clock}\\
\emph{kind:} `A clock (in general) is on a wall (rather than on the floor).'
\z
\il{Marathi}

In a slightly different context, the bare nominal in \REF{clock} can also be interpreted non-referentially. For example, in a context such as the one below, the bare nominal refers to the unique clock in each room in John's hotel that lacks a mantelpiece, and not to a unique entity in the utterance context.

\ea \textit{Context:} John is describing the organization of the rooms in his hotel to his manager. In each room, the time-piece is placed on the mantelpiece above the fireplace, if there is one. If there is none, the time-piece is hung on the wall above the bed. John says to his manager: ``When there is no mantelpiece..."
 
\gll \emph{ghaḍyāḷ} \emph{bhinti-var} \textbf{as-ta}\\
clock.\textsc{nom.n.sg} wall.\textsc{obl}-on \textsc{cop2-pres.3n.sg}\\
\glt \emph{definite non-referential:} `The clock (in a room without a mantelpiece) is (generally) on the wall.'
\z
\il{Marathi}

Bare plural subjects differ from bare singulars in that the kind interpretation is much more easily available for clauses in which they occur regardless of copula or predicate type. The sentence in \REF{workers1} contains a stage-level predicate, \textsc{cop1}, and a bare plural subject, \emph{kāmgār} `workers'. On one reading, it is a claim about the worker-kind at utterance time; we might be talking about workers all over the world (or in the US) working in exploitative conditions without job security, on the verge of a world-wide revolution.\footnote{This kind reading is also available when the subject is a bare singular but it is a little more difficult to access. There is no number morphology on the subject, but singular/plural reference is inferred through agreement marking on the copula.} But it can also be read as a claim about a contextually salient plural entity -- for instance, the group of workers that works at an air-conditioning plant that is planning to close shop and declare bankruptcy. This is the directly referential reading of the bare nominal.

\ea
\gll \emph{kāmgār} \emph{asantu\d{s}ṭa} \textbf{āhe-t}\\
worker.\textsc{nom.m.pl} discontented \textsc{cop1.pres-3pl}\\
\glt \emph{kind:} `Workers (in general) are discontented (right now).'\\
\emph{definite referential:} `The workers (working at the air-conditioning plant right now) are discontented (right now).' \label{workers1}
\z
\il{Marathi}

With \textsc{cop2} and a stage-level predicate, as in \REF{workers2}, the sentence is understood to report a generalization obtaining over an indefinite interval containing the utterance time. However, the content of the generalization depends on how the bare plural is interpreted. It may refer to the kind, it may refer to the contextually salient plural entity in the utterance context, e.g. the workers that work at the air-conditioning plant right now, or it may pick out (possibly different) groups of workers across different times -- this is the definite non-referential reading.\largerpage[-2]

\ea
\gll \emph{kāmgār} \emph{asantu\d{s}ṭa} \textbf{as-tāt}\\
 worker.\textsc{nom.m.pl} discontented \textsc{cop2-pres.3pl}\\
\glt \emph{kind:} `Workers (in general) are (generally) discontented.'\\
\emph{definite referential:} `The workers (who are working at the air-conditioning plant right now) are (generally) discontented.'\\
\emph{definite non-referential:} `The workers (whoever happen to work at the air-conditioning plant at a given time) are (generally) discontented.' \label{workers2}
\z
\il{Marathi}

\subsubsubsection{Individual-level predicates}
With individual-level predicates and \textsc{cop1}, both bare singulars and bare plurals, are preferentially interpreted as referring to singular or plural entities that are salient in the utterance context rather than to the kind.

\ea \label{indcop2}
\begin{xlist} 
\ex
\gll \emph{kāmgār} \emph{ghābraṭ/unca/hu\'{s}ār} \textbf{{ā}he}\\
worker.\textsc{nom.m.sg} cowardly/tall/intelligent \textsc{cop1.pres.3sg}\\
\glt
\emph{definite referential:} `The worker (at that air-conditioning plant) is cowardly/tall/intelligent.'\\
\emph{kind:} `?A worker (in general) is cowardly/tall/intelligent.'
\ex
\gll \emph{kāmgār} \emph{ghābraṭ/unca/hu\'{s}ār} \textbf{{ā}he-t}\\
worker.\textsc{nom.m.pl} cowardly/tall/intelligent \textsc{cop1.pres-3pl}\\
\glt \emph{definite referential:} `The workers (at that air-conditioning plant) are cowardly/tall/intelligent.'\\
\emph{kind:} `?Workers (in general) are cowardly/tall/intelligent.'
\z
\z
\il{Marathi}

However, the kind reading of bare nominals becomes available with individual-level predicates and \textsc{cop1} given suitable context and supporting linguistic information. For instance, in a context in which one is contrasting workers in this age with workers of previous eras, one may use \textsc{cop1} to describe ``the worker of today" (in contrast to that of yesteryear) as being intelligent -- \REF{indworker1a}. Similarly, \REF{indworker1b}, which contains a bare plural and \textsc{cop1}, is fully acceptable in a context where the evolutionary potential of donkeys is under consideration and one considers the possibility of intelligence determining genes mutating to make donkeys stupid.\footnote{Of course, the definite referential reading is available for both examples in \REF{indcop2}. In a context in which we are talking about the particular worker that has come in today to help with cleaning the machines, the singular expression, \emph{ājcā kāmgār} can refer to this specific worker. Similarly, the bare plural \emph{gāḍhav-e} can refer, in the right context, to my pet donkeys. It is the kind reading that is somewhat difficult to access with \textsc{cop1}, but can be made available given contexts such as those above.}

\ea \label{indworker1}
\ea
\gll \emph{āj-cā} \emph{kāmgār} \emph{jāsti} \emph{hu\'{s}ār} \textbf{{ā}he}\\
today-\textsc{gen.m.sg} worker.\textsc{nom.m.sg} more intelligent \textsc{cop1.pres.3sg}\\
\glt \emph{kind:} `The worker of this age (lit. today) is more intelligent.' \label{indworker1a}

\ex
\gll \emph{gāḍhav-e} \emph{hu\'{s}ār} \textbf{āhet}\\
donkey-\textsc{nom.n.pl} intelligent \textsc{cop1.pres-3pl}\\
\glt \emph{kind:} `Donkeys, as a kind, are intelligent (right now).' \label{indworker1b}
\z
\z
\il{Marathi}

With \textsc{cop2} and individual-level predicates like \emph{hu\'{s}ār} `intelligent,' definite referential readings are unavailable with bare nominals (both singular and plural). This can be illustrated with the example in \REF{inst}.

\begin{exe}
 \ex \textit{Context:} One/two of the workers at the air-conditioning plant fix(es) a problem with the cooling mechanism in an ingenious way. I praise his/their ingenuity, remarking to the manager:

\gll \# \emph{kāmgār} \emph{hu\'{s}ār} \textbf{as-to}/\textbf{as-tāt}\\
{} worker.\textsc{nom.m.sg}/\textsc{pl} intelligent \textsc{cop2-pres.3m.sg}/\textsc{pl}\\
\glt \emph{definite referential:} `The worker(s) (who fixed the problem) is/are intelligent.' \label{inst}
 \z
 \il{Marathi}

In such contexts, where it is clear that the bare nominal must refer to a singular/plural entity that is salient in the utterance context, speakers always choose \textsc{cop1} and reject \textsc{cop2}. In attempting to construe bare nominals in \textsc{cop2} sentences as definite referential expressions, speakers encounter the same oddity observed with names in \REF{indivb} -- that the property holds of a contextually salient singular or plural entity discontinuously in time.

Both definite non-referential and kind readings are possible with bare nominals when combined with \textsc{cop2} and individual level predicates.
\REF{indworker2} and \REF{indworker3} contain examples of the contexts in which the definite non-referential and kind readings of bare nominals arise respectively.

\begin{exe}

\ex \textit{Context:} One/two of the smarter worker(s) is/are assigned to work over-time each month to keep the machinery in working order and repair malfunctions. Because the workers are smart and already very familiar with the machinery, this system proves more efficient than calling outside expertise to service the machines. I explain this system to the manager saying:

\gll \emph{kāmgār} \emph{hu\'{s}ār} \textbf{as-to}/\textbf{as-tāt} \emph{mhaṇun} \emph{kām} \emph{lavkar} \emph{āṭap-t-a}\\
worker.\textsc{nom.m.sg}/\textsc{pl} intelligent \textsc{cop2-pres.3m.sg}/\textsc{pl} therefore work.\textsc{nom.n.sg} fast finish-\textsc{impf.pres.3n.sg}\\
\glt \emph{definite non-referential:} `The worker(s) (that get assigned to the job) is/are intelligent (smart) (and) so the work gets done faster.' \label{indworker2}
\z
\il{Marathi}

\pagebreak

\begin{exe}

\ex \textit{Context:}
 I am explaining to my students in a class on Labor Dynamics that they should always be transparent in their interactions with labor unions and try to understand their point of view.\footnote{The sentence has a kind reading with the adjective \emph{intelligent} as well, but it was difficult to construct a context in which such a sentence could be uttered without there also being some bias in the context that workers are not intelligent. The change from \emph{intelligent} to \emph{honest} is in order to avoid invoking such a bias.}

\gll \emph{kāmgār} \emph{prāmāṇik} \textbf{as-tāt} \emph{mhaṇun} \emph{tumhi} \emph{suddhā} \emph{prāmāṇik} \emph{as-āva}\\
worker.\textsc{nom.m.pl} honest \textsc{cop2-pres-3m.pl} therefore you also honest be-\textsc{pot.n.sg}\\
\glt \emph{kind:} `Workers (in general) are honest, therefore you should also be honest (in your interactions with them).' \label{indworker3}
\z
\il{Marathi}

%To summarize the facts, bare nominals may either be interpreted as individual-denoting or kind-denoting. On the former reading, they pattern exactly like names in copular clauses. On the latter reading, they help convey generalizations about a kind that are either episodic (e.g. \REF{ajca}) or that hold generally of kind-instantiations across time (e.g. \REF{inst}).
%

\subsection{The generalization}
The distribution of the two (present tense) copulas detailed in the previous section is summarized below in \tabref{presenttable}. The terms D-ref., D-non-ref., and kind-ref. stand for the definite referential, definite non-referential, and kind-referring readings of bare nominals.

\begin{sidewaystable}\small
\caption{Readings associated with copular clauses in Marathi\label{presenttable}}
\begin{tabular}{lllll}
\lsptoprule
                & \multicolumn{2}{c}{\textsc{cop1}} & \multicolumn{2}{c}{\textsc{cop2}}\\
\cmidrule(lr){2-3}\cmidrule(lr){4-5}
Subject NP      & \textsc{slp\,+\,cop1} & \textsc{ilp\,+\,cop1} & \textsc{slp\,+\,cop2} & \textsc{ilp\,+\,cop2}\\
\midrule
\emph{names}    & episode in & property in actual & generalization & \textit{induces oddity}\\
                & actual world & world overlapping & over episodes &\\
                & at UT & with UT & across time (invariant &\\
                & & & NP referent) &\\
\midrule
\emph{bare nominals} & episode in & property in actual & generalization & \textit{induces oddity}\\
D-ref.          & actual world & world overlapping & over episodes &\\
                & at UT & with UT & across time (invariant &\\
                & & & NP referent) &\\
\midrule
\emph{bare nominals} & \textit{unavailable} & \textit{unavailable} & generalization & generalization\\
D-non-ref.      & & & over episodes & over properties of\\
                & & & across time (variable & individuals across time\\
                & & & NP referent) & (variable NP referent)\\
\midrule
\emph{bare nominals} & episode involving & property of the kind & generalization over & generalization\\
kind-ref.       & the kind in actual & in the actual world & episodes involving & over properties of\\
                & world at UT & overlapping & kind-instances & the kind\\
                & & with UT & across time & across time\\
\lspbottomrule
\end{tabular}
\end{sidewaystable}
\il{Marathi}

What is immediately apparent through the table is that the type of copula influences the range of readings available to bare nominal subjects. Specifically, \textsc{cop1} forces a referential interpretation of bare nominals (i.e. the nominal must pick out an individual (singular, plural, or kind) in the actual world at utterance time). \textsc{cop2}, on the other hand, is unacceptable (oddity inducing) when neither the subject denotation nor the predication of the property of the subject denotation may be construed as variable across time. Intuitively, the meaning of \textsc{cop1} sentences seems to depend on the valuation of the embedded predication at utterance time in the actual world, while the meaning of \textsc{cop2} sentences seems to require consideration of the valuation of the embedded predication at times beyond the utterance time. In other words, the morphosyntactic devices \textsc{cop1} and \textsc{cop2} allow \ili{Marathi} to distinguish between descriptions whose interpretation is anchored to the utterance time $i_0$ and the utterance world $w_0$ on the one hand and those that lack such anchoring on the other.

\begin{sloppypar}
This contrast is typologically interesting since, as far as I know, it has not been described as being the basis of a multiple-copula system in any language (family). As reported in the introduction, the distinction is wide-spread in the \ili{New Indo-Aryan} languages. While the choice of devices may differ, all languages systematically disambiguate interpretations that are anchored to the utterance world and time from those that are not. The question we turn to next is: How/when does this semantic contrast become morphosyntactically expressed in \ili{Indo-Aryan} languages in a categorical way? To answer this question, I will consider facts from Epic \ili{Sanskrit} (\ili{Old Indo-Aryan}) and \ili{Prakrit} and Apabhra\d{m}\'{s}a\il{Apabhraṃśa} (\ili{Middle Indo-Aryan}). 
\end{sloppypar}

\section{Historical basis of the contrast} \label{historical}\largerpage

\subsection{Old Indo-Aryan (Epic Sanskrit)}

\ili{Old Indo-Aryan}, like most Ancient \ili{Indo-European} languages, inherits two PIE\il{Proto-Indo-European} ``be'' verbs -- \emph{as} (PIE\il{Proto-Indo-European} *\emph{h$_1$es}) and \emph{bh\={u}} (PIE\il{Proto-Indo-European} *\emph{b$^{h}$u̯eh$_2$}). The precise distribution of the two forms in Epic \ili{Sanskrit} is not well-established, but there are some observed environments in which each copula occurs.\footnote{A reviewer observes that using Epic \ili{Sanskrit} as the only source for investigating the distribution of the two copular forms in \ili{Old Indo-Aryan} is an unfortunate choice. The reasoning is that ``the Epic
\ili{Sanskrit} corpus is not coherent and in quintessence it is not even \ili{Old Indo-Aryan}. It varies
massively diachronically and diatopically." It is indeed conceivable that the writers of the \ili{Sanskrit} Epics, are, in fact, native speakers of a language with a \ili{Middle Indo-Aryan} type system, leading to peculiarities of \ili{Middle Indo-Aryan} entering the Epic corpus. We know that the \ili{Middle Indo-Aryan} Prakrits\il{Prakrit} were the vernacular languages in the region at least since 300 BCE (based on A\'{s}okan inscriptions). However, the limited goal of this paper is to establish that the semantic contrast observed in \ili{Marathi} (and other new \ili{Indo-Aryan} languages) is not present in the same categorical way in the diachronically prior systems although there is a tendency to associate the \emph{bh\={u}} copula with characterizing claims. I therefore take showing the existence of such tendential data in the Epic corpus to be sufficient for this goal.} For instance, \textit{as} is the tensed element of choice in existential clauses but it can also be used in predicational clauses. In both constructions, it can be used to make both particular and characterizing claims. \emph{bh\={u}}, in contrast, as a copular expression, only appears in predicational clauses and in those structures, appears to be compatible only with non-referential, characterizing readings.\footnote{\emph{bh\={u}} also has an inchoative use where it corresponds to a verb like `become' or `happen'. This is an eventive use of the verb, which I will be ignoring for the purposes of this paper.} Notice that this distribution, in which one copular element tends to have non-referential characterizing readings in its stative uses, yields a partially articulated contrast between particular and characterizing claims. To the best of my knowledge, this interpretive contrast between the two copular expressions has not been explicitly described for \ili{Sanskrit}, although, in \S\ref{discconc}, I point out that this appears to be a contrast instantiated in \ili{Old English} as well, potentially pointing to a tendency inherited from the PIE\il{Proto-Indo-European} system.
Consider the examples below from the Mahabharata, one of the two Epic \ili{Sanskrit} texts, that illustrate this distribution.\footnote{The Mahabharata text is attributed to a single author Vyāsa but is usually understood to be a compiled text (dateable to $\sim$100BCE) with interpolations from multiple authors.}

\begin{exe}
\label{as1}
\ex 
\begin{xlist} 
\ex
\gll \emph{madadhīn-o} \textbf{'-si} \emph{pārthiva}\\
me.dependent-\textsc{nom.m.sg} \textbf{as}-\textsc{pres.2.sg} king.\textsc{voc.sg}\\
\glt `O King, you are dependent on me.' (Mbh. 1.78.35b) \label{dependent}

\ex
\gll \emph{abhijāt-o} \textbf{'s-mi} \emph{siddh-o} \textbf{'s-mi} \emph{na} \textbf{as-mi} \emph{kevalamānu\d{s}-a\d{h}}\\
high.born-\textsc{nom.m.sg} \textbf{as}-\textsc{pres.1.sg} accomplished-\textsc{nom.m.sg} \textbf{as}-\textsc{pres.1.sg} \textsc{neg} \textbf{as}-\textsc{pres.1.sg} ordinary.man-\textsc{nom.m.sg}\\
\glt `I \emph{am} of high birth, I \emph{am} accomplished, I \emph{am} not an ordinary man.' (Mbh. 12.28.7a) \label{highbirth}

\ex
\gll \emph{y-e} \textbf{s-anti} \emph{vidyātapasopapann-ās} \emph{te-\d{s}ā\d{m}} \emph{vinā\'{s}a-\d{h}} \emph{prathama\d{m}} \emph{tu} \emph{kār-ya-\d{h}}\\
Those.\textsc{rel-nom.pl} \textbf{as}-\textsc{pres.3.sg} knowledge.ascetism.possessed-\textsc{nom.m.pl} those.\textsc{correl.gen.m.pl} destruction-\textsc{nom.m.sg} first \textsc{ptcl} do-\textsc{poten-nom.m.sg}\\
\glt `Those who \emph{are} possessed of knowledge and ascetic virtue, their destruction should be undertaken first.' (Mbh.3.99.19c) \label{destruction}
%course of time, however, the dreadful resolution arrived at by those conspiring sons of \ili{Diti}, was that they should, first of all, compass the destruction of all persons possessed of knowledge and ascetic virtue. The worlds are all supported by asceticism. They said: 
\z
\z

In \REF{dependent} and \REF{highbirth}, both predicational copular clauses, the copula \textit{as} is used to convey that the referent (the addressee and the speaker of the utterance context, respectively) has the relevant property at utterance time in the utterance world. In \REF{destruction}, a more involved sentence, describes a resolution arrived at in terms of a course of action to be undertaken at utterance time. The subject referent, in this case, is the set of all individuals in the utterance world at utterance time, who are possessed of knowledge and ascetic virtue. Crucially, the sentence does not express a general claim about how such people are to be treated across all situations.

In \REF{rain} and \REF{brave}, both existential clauses, we see that the \textit{as} copula is used to make characterizing claims. Example \REF{rain} makes the generalization that kingless king\-doms have neither rain nor Gods while \REF{brave} asserts the existence of cowardly and brave men across different indices of evaluation -- not only at utterance time.

\begin{exe}
\label{as2}
\ex

\begin{xlist}
\ex
\gll
\emph{arājak-e\d{s}u} \emph{rā\d{s}ṭr-e\d{s}u} \emph{na} \textbf{as-ti} \emph{v\d{r}\d{s}ṭi-r} \emph{na} \emph{devatā-\d{h}}\\
king.less-\textsc{loc.pl} kingdom-\textsc{loc.pl} \textsc{neg} \textbf{as}-\textsc{pres.3.sg} rain-\textsc{nom.sg} \textsc{neg} God-\textsc{nom.pl}\\
\glt `In kingdoms without a king, there \emph{is} no rain and no Gods.' (Mbh. 1.99.41a) \label{rain}

\ex
\gll \textbf{s-anti} \emph{vai} \emph{puru\d{s}-ā\d{h}} \emph{\'{s}\={u}r-ā\d{h}} \textbf{s-anti} \emph{kāpuru\d{s}-ās} \emph{tathā}\\
\textbf{as}-\textsc{pres.3.pl} \textsc{ptcl} man-\textsc{nom.m.pl} brave-\textsc{nom.m.pl} \textbf{as}-\textsc{pres.3.pl} coward-\textsc{nom.m.pl} likewise\\
\glt `There \emph{are} brave men, and likewise those that \emph{are} cowards.' (Mbh. 5.3.2a) \label{brave}
\z
\z

When we study the distribution of the \emph{bh\={u}} copula, it appears that its stative uses involve only characterizing readings, as in the examples in \REF{bhu1}.\footnote{I searched through John Smith's electronic text version of of the Bhandarkar Oriental Research Institute's critical edition of the Mahabharata (1933--1966) for instances of the 3\textsuperscript{rd} singular and plural present indicative forms of \emph{bh\={u}} -- \emph{bhavati} (362 occurences) and \emph{bhavanti} (114 occurences). Examining the first fifty occurrences among the results from each set did not yield a clear instance where \emph{bhu} appeared as the main predicate in a copular clause, was interpreted statively with present reference, and gave rise to a particular claim. The electronic edition used is available at \url{https://bombay.indology.info/mahabharata/statement.html}.} The sentence in \REF{husband} describes what is said in the code -- a guideline to be followed not just at utterance time but more generally. The example in \REF{demon} describes the defining properties of the Rāk\'{s}asa women kind and the clause containing the \emph{bh\={u}} copula predicates the property of being many-formed to the kind.

\ea
\label{bhu1}
\ea
\gll \emph{sakhībhartā} \emph{hi} \emph{dharm-eṇa} \emph{bhartā} \textbf{bhava-ti} {\'{s}obhane}\\
 friend.husband.\textsc{nom.m.sg} \textsc{ptcl} code-\textsc{ins.m.sg} husband.\textsc{nom.m.sg} \textbf{bhu}-\textsc{pres.3.sg} beautiful.\textsc{voc sg}\\
\glt `O beautiful one, the husband of a friend, according to the code, \emph{is} also one's husband.' (Mbh. 1.78.20c) \label{husband}

\ex
\gll \emph{sadyo} \emph{hi} \emph{garbha-\d{m}} \emph{rāk\d{s}as-yo} \emph{labh-ante} \emph{prasav-anti} \emph{ca}
 \emph{kāmar\={u}padhar-ā\'{s}} \emph{ca} \emph{eva} \textbf{bhav-anti} \emph{bahur\={u}piṇ-a\d{h}}\\
immediately \textsc{ptcl} embryo-\textsc{acc.sg} R-\textsc{nom.f.pl} receive-\textsc{pres.3.pl} give.birth-\textsc{pres.3.pl} and desire.form.holding-\textsc{nom.f.pl} and \textsc{ptcl} \textbf{bhu}-\textsc{pres.3.pl} many.formed-\textsc{nom.f.pl}\\
\glt `The Rāk\'{s}asa women give birth the very day they conceive, and being able to assume any form at will, they \emph{are} many-formed.' (Mbh.1.143.32a-c) \label{demon}
\z
\z

What is crucial is that the verbs \emph{as} and \emph{bh\={u}} do not seem to stand in free variation in the Epic \ili{Sanskrit} system. While \emph{as} is used to make both particular and characterizing claims, there is a strong tendency for the \emph{bh\={u}} copula to \textsc{not} be used in clauses that describe a state determined by the valuation of the embedded predication at utterance time in the actual world -- in other words, the uses that I examined occur in clauses that convey generalizations. I take these facts to suggest that there is some evidence in \ili{Old Indo-Aryan} for a dedicated device for expressing non-referential characterizing claims but there is no clear-cut division of labor between the two copulas of the kind one sees in \ili{Marathi} and other \ili{New Indo-Aryan} languages.

\subsection{Middle Indo-Aryan}

\subsubsection{Maharashtri Prakrit}

The changes from the inflectional system of verbal contrasts in \ili{Old Indo-Aryan} to the relatively morphologically impoverished inflectional system of \ili{Middle Indo-Aryan} have been described in terms of ``erosion'' or ``simplification'', primarily because many of the rich conjugational paradigms and the semantic categories
 expressed were lost in \ili{Middle Indo-Aryan} (\citealt{bloch65, beames66, bubenik98, bubenik96, pischel00, vale48, masica91} and
others). Of the several changes in the expression of tense-aspect-modality distinctions,\footnote{The \ili{Middle Indo-Aryan} tense/aspect system inherits only the Present, the Perfective Participle, and the Sigmatic Future
paradigms from \ili{Old Indo-Aryan}. The rich system of past tense markers is lost. \citet{pischel00}, on the
basis of careful textual study, reports that the Imperfect, the Aorist, and the Perfect occur in \ili{Middle} Indo-
\ili{Aryan} texts only as a few scattered forms for a few verbs. From among the past-referring forms of Epic
\ili{Sanskrit}, only the perfective participial paradigm remains and it is used regularly to refer to past time events in \ili{Middle Indo-Aryan}.} critical is the loss of a morphosyntactic distinction between the past and present tenses. This subsection describes the resulting contrasts in the re-organized aspectually based system in which temporal reference is established through contextual cues \citep{deo2012}.\footnote{In brief, the reorganization is as follows: The \ili{Old Indo-Aryan} Present tense realizes a tenseless imperfective and is compatible with both present and past imperfective reference. The \ili{Old Indo-Aryan} Present Participle is also starting to be used in this function. The \ili{Old Indo-Aryan} Past Participial form in \emph{-ta} realizes the perfective aspect and is used to refer to completed events. Therefore, by default, the use of this form leads to past temporal reference. However, this form may also be used systematically to describe future eventualities.} The examples in \REF{tree} and \REF{monkey} illustrate the basic \ili{Middle Indo-Aryan} pattern, which forms the backdrop for an innovation in \ili{Late Middle Indo-Aryan}. In \REF{tree1}, the \ili{Old Indo-Aryan} Present paradigm, glossed \textsc{impf}, has present reference, while in \REF{yoga}, the same paradigm has past reference.

%In \REF{passion}, we see the Present Participle paradigm used as the main verb in a clause with past imperfective reference.

\begin{exe}
\ex
\label{tree} 
\begin{xlist} 
\ex
\gll
\emph{nipphala-\d{m}} \emph{duma-\d{m}} \emph{pakkhiṇ-o} \emph{vi} \textbf{paricchaya-nti}\\
fruitless-\textsc{acc.n.sg} tree-\textsc{acc.n.sg} bird-\textsc{nom.m.pl} also abandon-\textsc{impf.}\textsc{3.pl}\\
\glt `Even birds \emph{abandon} a fruitless tree.' (VH.DH 31.24-25) \hfill{\emph{imperfective present reference}} \label{tree1}\\

\ex
\gll tato aham aṇṇayā kayāi
 āyariya-giha-rukkha-vāḍiyā-e joga-m \textbf{kare-mi}\\
 then i.\textsc{nom.sg} other {some time} teacher-house-tree-garden-\textsc{loc.f.sg} yoga-\textsc{acc.m.sg} do-\textsc{impf.}\textsc{1.sg}\\
\glt `Then, sometimes, I \emph{would perform} Yoga in the orchard at my teacher's house.' (VH:DH 37.1) \hfill{\emph{imperfective past reference}} \label{yoga}

%\ex
% \gll \emph{vaidehi-viraha} \emph{veyaṇa} \textbf{sah-antu} \emph{dasa} \emph{kāmavattha} \textbf{dakkhav-antu}\\
% vaidehi-separation pain bear-\textsc{pres.part.m.sg} ten passion-condition exhibit-\textsc{pres.part.m.sg}\\
% \glt (He) \emph{bore} the pain of separation from Vaidehi; he \emph{exhibited} the ten conditions (aspects) of passion. (PC. 2.22.5) \hfill{\emph{imperfective past reference}} \label{passion}
\z
\z

% Andi: perhaps check with Deo whether this example and the accompanying text should be in the paper or not

In \REF{monkey}, a set of consecutive sentences reports part of a past episode about a monkey who entered a mountain cave and
mistook some sticky liquid tar to be water. The example in \REF{notice} describes a past event using perfective marking while \REF{python} and \REF{snakes} also with past reference, describe past activities using the temporally unmarked imperfective.\largerpage

\ea
\label{monkey}
\ea
\gll \emph{te-ṇa} \emph{palāyamāṇ-eṇa} \emph{purāṇakuv-o} \emph{taṇadabbhaparichinn-o} \textbf{diṭ-ṭho}\\
that-\textsc{erg.m.sg} running-\textsc{erg.m.sg} old.well-\textsc{nom.m.sg} weed.grass.covered-\textsc{nom.m.sg} notice-\textsc{perf.m.sg}\\
\glt `That running one \emph{noticed} an old well\\
 covered with weed and grass.' (VH.KH. 8.6) \hfill{\emph{perfective past reference}} \label{notice}

\ex
\gll \emph{tattha} \emph{ayagar-o} \emph{mahākā-o} \emph{vidāriyamuh-o} \emph{gāsiukām-o} \emph{tam} \emph{purisam} \textbf{avaloe-i}\\
there python-\textsc{nom.m.sg} gigantic-\textsc{nom.m.sg} open.mouthed-\textsc{nom.m.sg} swallow.desiring-\textsc{nom.m.sg} that-\textsc{acc.m.sg} man-\textsc{acc.m.sg} observe-\textsc{impf.}\textsc{3.sg}\\
\glt `There a giant python, baring its mouth, eager to eat, \emph{observed} the man.' (VH.KH. 8.9) \hfill{\emph{imperfective past reference}}\label{python}

\ex
\gll \emph{sapp-ā} \emph{bhīsaṇ-ā} \emph{a\d{s}iukām-ā} \textbf{ciṭṭha-nti}\\
snake-\textsc{nom.m.pl} fearsome-\textsc{nom.m.pl} eat.desiring-\textsc{nom.m.pl} stand-\textsc{impf.}\textsc{3.pl}\\
\glt `Fearsome snakes, eager to bite, \emph{stood} (in the well).' (VH.KH. 8.9) \hfill{\emph{imperfective past reference}} \label{snakes}
\z
\z

To summarize, the core distinctions made in the \ili{Middle Indo-Aryan} aspecto-temporal system are as follows:

\ea
\begin{forest}
[
    [\emph{non-future} [\emph{imperfective} [\textsc{impf} or \textsc{pres.part}]] [\emph{perfective} [\textsc{perf}]]]
    [\emph{future} [\emph{neutral} [sigmatic future]] [\emph{perfective} [\textsc{perf}]]]
]
\end{forest}
\z
The \ili{Early New Indo-Aryan} aspecto-temporal system is systematically built up from this aspectual core through periphrastic constructions based on tense auxiliaries, already visible in \ili{Middle Indo-Aryan} (\citealt{bubenik96, bubenik98, kellogg93, beames66, chatterji70}, and others).

\subsubsection{Apabhra\d{m}\'{s}a}

A key change in the \ili{Late Middle Indo-Aryan} verbal system, specifically Apa\-bhra\d{m}\-\'{s}a, involves the loss of the \ili{Old Indo-Aryan} \textit{as} copula.\footnote{It has been proposed that the loss of \emph{as} is purely morphological; it is an athematic verb from the second conjugation, which is lost as part of the simplification of the verbal system \citep[293]{hinuber2001}.} Additionally, we see the introduction of a new tense auxiliary, based on the verb \emph{acch} `sit' that establishes temporal reference with respect to speech time \citep{turner1936}. This auxiliary presupposes contextually salient reference intervals and anchors the embedded property to the utterance world. The present tense auxiliary paradigms of several \ili{Indo-Aryan} languages are cognate forms of this original auxiliary, as seen below.\footnote{I note here that in its early uses, the \textit{acch} auxiliary is not always associated with present tense reference. It is compatible with both past and present temporal reference and uniformly conveys that a given state holds at reference time. In \ili{Early New Indo-Aryan} languages, however, this form is exclusively used to convey present temporal reference.}
 
\begin{itemize}\itemsep=0mm
\item \emph{acchai} $>$ \emph{āhe} \hfill{Marathi}
\item \emph{acchai} $>$ \emph{ahai} $>$ \emph{hai} \hfill{Hindi}
\item \emph{acchai} $>$ \emph{chai} $>$ \emph{che} \hfill{Gujarati}
\item \emph{acchai} $>$ \emph{āchi} \hfill{Bangla}
\end{itemize}

In \ili{Late Middle Indo-Aryan}, this auxiliary appears in copular clauses as illustrated below in \REF{cop1exmia}. All examples here come from the \emph{Paumacariu} of Svayambhudeva, a key Apabhra\d{m}\'{s}a verse text from $\sim$800CE.


\ea \label{cop1exmia}
 \ea
\gll {\emph{deva deva}} \emph{ki-u} \emph{j-eṇa} \emph{mahāra} \textbf{acch-ai} \emph{mattahatthi} \emph{airāva}\\
 Lord.\textsc{voc.sg} do-\textsc{perf.m.sg} that.\textsc{rel-ins.m.sg} great.sound.\textsc{nom.m.sg} \textbf{acch}-\textsc{pres.3.sg} musth.elephant.\textsc{nom.sg} Airāvata.\textsc{nom.sg}\\
\glt `Lord, the one who made a great sound, he is the elephant Airavata in musth season.' (PC 1.11.3.4)
\ex
\gll \textbf{acch-ahi} \emph{suha.dukkha.karamviya}\\
\textbf{acch}-\textsc{pres.2.sg} pleasure.pain.engrossed.\textsc{perf.m.sg}\\
\glt `(You) are engrossed with pleasure and pain.' (PC 2.33.5.2)
\ex
\gll \textbf{acch-ai} \emph{kailāsa-ho} \emph{uvari} \emph{sāhu}\\
\textbf{acch}-\textsc{pres.3.sg} Kailash-\textsc{gen.sg} on sage-\textsc{nom.m.sg}\\
\glt `There is a sage on the Kailasa mountain.' (PC 1.13.2.6)
\z
\z

The cognate of the \emph{bh\={u}} copula retains its properties observed in \ili{Old Indo-Aryan} -- in its stative uses, it gives rise to non-referential characterizing readings, as the examples in \REF{cop2exmia} illustrate.

\ea \label{cop2exmia}
\ea
\gll \emph{sappurisa} \emph{vi} \emph{ca\~{n}calacitta} \textbf{ho-nti}\\
good.man.\textsc{nom.m.pl} even unsteady.mind.\textsc{nom.m.pl} \textbf{bhu}\textsc{-pres.3.pl}\\
\glt `Even good men are fickle-minded.' (PC 2.22.10.7)\\
\ex
\gll \emph{sāsu-a} \textbf{ho-nti} \emph{viruāriya}\\
mother-in-law-\textsc{nom.f.pl} \textbf{bhu-}\textsc{pres.3.pl} cruel.\textsc{nom.f.pl}\\
\glt `Mothers-in-law are cruel' (PC 1.19.4.8)\\
\ex
\gll \emph{has-iu} \emph{purandar-eṇa} \emph{are} \emph{māṇava} \emph{devasamāṇa} \textbf{ho-nti} \emph{ki\d{m}} \emph{dāṇava}\\
 laugh-\textsc{perf.m.sg} Purandara-\textsc{ins.m.sg} O human.\textsc{voc.sg} God.equal \textbf{bhu-}\textsc{pres.3.pl} \textsc{inter} demon.\textsc{nom.m.pl}\\
\glt `Purandara (Indra) laughed: ``O human, are the Gods equatable with the demons?{''}' (PC. 1.8.8.8)
\z
\z

I speculate here that the \ili{Late Middle-Aryan} innovation of a tense auxiliary built on \textit{acch} `sit' that anchors the embedded predication to the utterance world and contextually salient reference intervals, in fact leads to a reorganization of the copular/auxiliary system. The idea is as follows: the original \ili{Old Indo-Aryan} system had a dedicated device for making non-referential characterizing claims (the \emph{bh\={u}} copula) but the \textit{as} copula was underspecified and could be used in both particular and characterizing senses. This underspecified copular element was independently lost for morphological reasons in \ili{Middle Indo-Aryan}. The innovation of a new tense auxiliary, anchored to the utterance world and time, in the \ili{Late Middle Indo-Aryan} system facilitated the ``hardening" of the soft contrast between particular and characterizing claims found in the older systems. Specifically, I hypothesize that the innovated copular element was incompatible with characterizing claims, which led to a complementary distribution between the two copular forms in the domain of copular clauses.\footnote{This contrast is still realized in some languages (e.g. \ili{Hindi} and \ili{Bangla}) by the cognates of \textit{acch} and \textit{bhu}, while in other languages, other lexical items and paradigms get recruited to realize the same semantic contrast.} There is much that is unknown about the precise development of tense marking in \ili{Indo-Aryan} and a fuller understanding of the pathway associated with the copular contrast in \ili{New Indo-Aryan} rests on that. I leave this investigation to future research, noting only that it is not the innovation of morphosyntactic tense marking per se that leads to the categorical realization of the contrast, but rather the particular temporal reference features associated with the auxiliary forms.

% that anchor the embedded predication to the utterance world and contextually salient reference intervals must be left to future research.
%%

\section{Concluding thoughts}\label{discconc}
Towards the beginning of this paper, I distinguished between morphosyntactic strategies in which salient contrasts within a semantic domain are overtly expressed or individually packaged vs. those in which such contrasts remain morphosyntactically unexpressed with the distinct meanings disambiguated in context. From the perspective of the questions that underpin this volume, one might ask if an individualized packaging strategy is more or less complex than a contextual disambiguation strategy. That is, are semantic contrasts retrievable by means of form-dependent strategies ``better'' than contrasts retrievable only in context (e.g. \ili{Modern English})?
Depending on the answer, the emergence of the categorical contrast between particular and characterizing claims in a linguistic system may be seen as a change for the worse. There are at least two considerations. On the one hand, successful use of the two-copula strategy observed in \ili{Marathi} (and mirrored in other \ili{Indo-Aryan} languages) involves the acquisition of two semantically distinct paradigms for temporal reference. While the two-form system guarantees communicative success, the acquisition process is rendered more complex since the distribution of two forms relative to a semantic domain must be learned. On the other hand, in a language that does not lexicalize the particular-characterizing contrast and has only a single form to convey distinct meanings (such as \ili{English}), hearers must be contextually attuned so that the intended temporal interpretation is indeed retrieved reliably in a given context.

In the \ili{Indo-Aryan} case, we observe a transition from a partially context-de\-pen\-dent strategy of meaning recovery to a form-dependent strategy of meaning recovery. While evaluating whether this transition is a worsening or complexification of the system, it is necessary to keep in mind that the categorical or ``hardened'' contrast between particular and characterizing meanings does not just emerge \textit{spontaneously} in \ili{Indo-Aryan} from the original soft contrast. Rather, it is situated within changes in the larger landscape of \ili{Middle Indo-Aryan} temporal reference. The \ili{Middle Indo-Aryan} system lacked dedicated devices corresponding to the present and the past tenses -- a distinct morphosyntactic impoverishment in comparison to the \ili{Old Indo-Aryan} system with three past-referring categories. In such a system, imperfective and perfective clauses are temporally underspecified and contextual cues are critical to the retrieval of information regarding their temporal reference. The development of the copular contrast described here was concomitant with the emergence of overt marking of tense distinctions via innovated tense markers. These tense markers (as illustrated here by the present tense forms) had the right anchoring properties -- they anchor the embedded predication to the utterance time and the utterance world. The categorical marking of particular vs. characterizing claims became possible only after such devices were available in the linguistic system.\footnote{To be clear, our earliest records of \ili{Marathi} already show these tense auxiliaries with established function. In contrast, they are rather infrequent the \ili{Late Middle Indo-Aryan} record. So it is likely that there might not be textual material that allows us to track the gradual development of these devices in the \ili{Indo-Aryan} languages.} Thus, the transition to what might appear, on the formal metric, a more complex strategy of meaning packaging and recovery in the copular domain, turns out to be a consequence of a far more general change in the language -- the development of overt morphosyntactic realization for the present and past tenses. If one argues that the development of basic temporal distinctions in the tense-aspect system does not constitute complexification, then the status of concomitant effects of this development becomes somewhat less clear.

To close the paper, I will use a counterpoint to underscore that the categorical copular pattern seen in synchronic \ili{Marathi} (and other \ili{Indo-Aryan} languages) crucially relies on particular historical facts about the \ili{Late Middle Indo-Aryan} tense-aspect system. The counterpoint is \ili{Old English}, which also inherited the two PIE\il{Proto-Indo-European} ``be'' verbs -- \emph{is} (PIE\il{Proto-Indo-European} *\emph{h$_1$es}) and \emph{bið} (PIE\il{Proto-Indo-European} *\emph{b$^h$u̯eh$_2$}). \citet{petre2013} (also citing prior research)
shows convincingly that in \ili{Old English}, \textit{is} was mainly used for predicating present states of specific subjects, and in identifying clauses, while \textit{bið} was used to encode future situations and generic statements, which are connected to future situations through their implication of future validity. The pattern, while not identical to that of Epic \ili{Sanskrit}, is similar in that the language morphosyntactically distinguishes between particular claims and characterizing claims (Petr\'{e}'s) ``generic statements". The two examples in \REF{oldeng} illustrate the functional distribution of \textit{bið} in \ili{Old English}.\footnote{Petr\'{e} refers to a quantitative analysis by \citet{kilpio1993}, which suggests that the pattern is tendential rather than categorical -- there is a soft rather than categorical contrast. According to Kilpi\"{o}, \textit{is} is found with its typical semantic characteristics in no less than 86.4\% of all instances in HC. For \textit{bið}, its presence in the typical semantic domain associated with it, amounts to only about 56.7\%.}

\ea \label{oldeng}
\ea
\gll Hit \textbf{byð} dysig \textthorn æt man speca ær \textthorn one he \textthorn ænce\\
it is foolish that man speak.\textsc{subj} ere then he think.\textsc{subj}\\
\glt `It \textit{be} foolish that a man speaks before he thinks.' {(c1100. Prov 1 [Cox]: 2.2, via \citealt{petre2013})}

\ex
\gll Wið stede \& for gebinde, heortes hær \textbf{beoð} swiðe gode mid to smeocanne wifmannum\\
against strangury and for constipation hart.\textsc{gen} hair.\textsc{pl} are.\textsc{ind.3.pl} very good with to smoke women.\textsc{dat}\\
\glt `Against strangury and constipation, hairs of the hart \textit{be} very good for women to fumigate with.' {(c1025. Med 1.1 [de Vriend]: 3.16, via \citealt{petre2013})}
\z
\z

If Petr\'{e} is right, what is striking about the evolution from \ili{Old English} to \ili{Middle English} is that this functional contrast, which is soft, but remarkably stable in \ili{Old English}, is eroded by the the grammaticalization of a future construction \textit{sceal beon} ``shall be.'' One result of this innovation is a drastic redistribution of the two ``be'' verbs, and finally their merger (the situation in \ili{Modern English}). The hardening of soft semantic contrasts of this sort thus appears to be entirely dependent on patterns of innovation and loss in the larger system of temporal/aspectual contrasts.\newpage

\section*{Abbreviations}

\begin{multicols}{2}
\begin{tabbing}
\textsc{pres.part} \= Imperfective Participle\kill
1 \> First person\\
2 \> Second person\\
3 \> Third person\\
\textsc{acc} \> accusative\\
\textsc{act} \> active voice marker\\
\textsc{aux}1 \> auxiliary (episodic)\\
\textsc{aux}2 \> auxiliary (characterizing)\\
\textsc{cop}1 \> copula (particular)\\
\textsc{cop}2 \> copula (characterizing)\\
\textsc{correl} \> correlative pronoun\\
\textsc{dat} \> dative\\
\textsc{excl} \> exclusive clitic\\
\textsc{f} \> feminine\\
\textsc{gen} \> genitive\\
\textsc{ger} \> gerund\\
\textsc{impf} \> imperfective aspect\\
\textsc{inf} \> infinitive\\
\textsc{ins} \> instrumental\\
\textsc{inter} \> interrogative particle\\
\textsc{loc} \> locative\\
\textsc{m} \> masculine\\
\textsc{neg} \> negation marker\\
\textsc{n} \> neuter\\
\textsc{nom} \> nominative\\
\textsc{obl} \> oblique\\
\textsc{part} \> participle\\
\textsc{pass} \> passive voice\\
\textsc{past} \> past tense\\
\textsc{pl} \> plural\\
\textsc{poten} \> potential mood\\
\textsc{pres.part} \> imperfective participle\\
\textsc{pres} \> present tense\\
\textsc{ptcl} \> particle\\
\textsc{rel} \> relative pronoun\\
\textsc{sg} \> singular\\
\textsc{voc} \> vocative
\end{tabbing}
\end{multicols}

{\sloppy\printbibliography[heading=subbibliography,notkeyword=this]}
\end{document}
