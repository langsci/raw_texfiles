\documentclass[output=paper,hidelinks]{langscibook}
\ChapterDOI{10.5281/zenodo.13347670}

\title{The complexification of Tungusic interrogative systems}

\author{Andreas Hölzl\affiliation{University of Potsdam}}

\abstract{The present study discusses the nature and development of a Tungusic phoneme *\textit{K-} that has proven difficult to reconstruct. It is only fully preserved in one subbranch of the Tungusic language family and today is usually considered a velar fricative *\textit{x-} \citep{Benzing1956}. However, there is evidence from a Tungusic language called Alchuka hitherto almost unknown outside China \citep{MuYejun1985,MuYejun1986,MuYejun1987,MuYejun1988}. In this language the phoneme is, somewhat irregularly, preserved as unaspirated \textit{k-}, which may corroborate its reconstruction as a plosive \citep{Rozycki1993}. The main focus of the paper is the role of the phoneme in the interrogative system of Proto-Tungusic as well as the detrimental implications of its loss in most Tungusic languages. In Proto-Tungusic the phoneme had the function of a \textit{submorpheme} or \textit{resonance} similar to English <wh> \citep{BickelNichols2007,Mackenzie2009}. Its loss led to incoherent interrogative systems with a large number of individual forms that are synchronically opaque, i.e. to complexification. Finally, the question is addressed whether this can be considered a ``change for the worse", as indicated by the title of this volume.}


\IfFileExists{../localcommands.tex}{
  \addbibresource{../localbibliography.bib}
  % add all extra packages you need to load to this file

\usepackage{tabularx,multicol}
\usepackage{url}
\urlstyle{same}

\usepackage{listings}
\lstset{basicstyle=\ttfamily,tabsize=2,breaklines=true}

\usepackage{langsci-basic}
\usepackage{langsci-optional}
\usepackage{langsci-lgr}
\usepackage{langsci-osl}
% \usepackage{./langsci/styles/langsci-lgr}
% \usepackage{./langsci/styles/langsci-osl}
% \usepackage{langsci-gb4e}

\usepackage{tikz}
\usetikzlibrary{patterns,calc}
\pgfdeclarepatternformonly{south east lines}{\pgfqpoint{-0pt}{-0pt}}{\pgfqpoint{3pt}{3pt}}{\pgfqpoint{3pt}{3pt}}{
    \pgfsetlinewidth{0.6pt}
    \pgfpathmoveto{\pgfqpoint{0pt}{3pt}}
    \pgfpathlineto{\pgfqpoint{3pt}{0pt}}
    \pgfpathmoveto{\pgfqpoint{.2pt}{-.2pt}}
    \pgfpathlineto{\pgfqpoint{-.2pt}{.2pt}}
    \pgfpathmoveto{\pgfqpoint{3.2pt}{2.8pt}}
    \pgfpathlineto{\pgfqpoint{2.8pt}{3.2pt}}
    \pgfusepath{stroke}}
    
\usepackage{stmaryrd}
\usepackage{wasysym}
\usepackage{multirow}
\usepackage{caption}
\usepackage{subcaption}
\usepackage{mathrsfs}
\usepackage{qtree}

\usepackage{linguex}


  %pminos do not split footnotes
% \interfootnotelinepenalty=10000 %Footnote in Laporte chapters has to be split SN


%\DeclareIndexNameFormat{default}{%
%\nameparts{#1}%
%\usebibmacro{index:name}%
%{\index[names]}%
%{\namepartfamily}%
%{\namepartgiveni}%
% {}% L1
% {}% L2
%{\namepartprefix}% generates spurious space L3
%{\namepartsuffix}% generates spurious space L4
%}

%  {\DeclareIndexNameFormat{default}{%
%     \usebibmacro{index:name}{\index[names]}{#1}{#3}{#5}{#7}}}

%\DeclareIndexNameFormat{default}{%
%  \usebibmacro{index:name}{\sindex[nom]}{#1}{#3}{#5}{#7}}

%\DeclareIndexNameFormat{default}{%
%  \usebibmacro{index:name}{\sindex[person]}{#1}{#3}{#5}{#7}}
%\DeclareIndexNameFormat{default}{%
%\nameparts{#1} \usebibmacro{index:name}{\sindex[person]]}{\namepartfamily}{‌​\namepartgiven}{\nam‌​epartprefix}{\namepa‌​rtsuffix}}

%\newcommand{\smiley}{:)}

%\renewbibmacro*{index:name}[5]{%
%\usebibmacro{index:entry}{#1}%
%{\iffieldundef{usera}{}{\thefield{usera}\actualoperator}\mkbibindexname{#2}{#3}{#4}{#5}}}

% \newcommand{\noop}[1]{}

%remove for final
%\overfullrule=1mm

\newcommand{\tobi}[2]}}
\renewcommand{\S}[1]{\tobi{#1}{\textsc{*}}}

% this volume references
% puts: [this volume]
% already defined: \citetv
%\newcommand{\citepv}[1]{(\citeauthor{#1} \citeyear*{#1} [this volume])}
\newcommand{\citealtv}[1]{\citeauthor{#1} \citeyear*{#1} [this volume]}

%parentheses around example number
\newcommand{\pref}[1]{(\ref{#1})}

% in-text examples

\newcommand{\lnex}[1]{\textit{#1}} %target lang word
\newcommand{\lnlit}[1]{(lit.: `#1')} %literal reading
\newcommand{\lnlat}[1]{(#1)} % latinization
\newcommand{\lntrans}[1]{`#1'} %translation
\newcommand{\lnexl}[2]%
{\lnex{#1}{} \lnlat{#2}} % ex with latinization
\newcommand{\lnexlat}[3]{\lnex{#1}{} \lnlat{#2}{} \lntrans{#3}} % ex with latinization and tranl.

%ch01
\newcommand{\co}[1]{\mbox{\textbf{#1}}}

%ch09

\newcommand{\cyrbulg}[1]{\begin{otherlanguage*}{bulgarian}#1\end{otherlanguage*}}


%ch10
\newcommand{\nlp}{{\small NLP}}
\newcommand{\mwe}{{\small MWE}}
\newcommand{\rae}{{\small RAE}}
\newcommand{\lvc}{{\small LVC}}
\newcommand{\pos}{{\small P}o{\small S}}
%\newcommand{\todo}[1]{ \textcolor{red}{#1} }

%\renewcommand{\labelenumi}{\theenumi}
%\ainamefmt{{vv}{ll}{, ff}{, jj}} % fullname

\newcommand{\biberror}[1]{{\color{red}#1}}

\newcommand{\osenovaitem}{--~}
  %% hyphenation points for line breaks
%% Normally, automatic hyphenation in LaTeX is very good
%% If a word is mis-hyphenated, add it to this file
%%
%% add information to TeX file before \begin{document} with:
%% %% hyphenation points for line breaks
%% Normally, automatic hyphenation in LaTeX is very good
%% If a word is mis-hyphenated, add it to this file
%%
%% add information to TeX file before \begin{document} with:
%% %% hyphenation points for line breaks
%% Normally, automatic hyphenation in LaTeX is very good
%% If a word is mis-hyphenated, add it to this file
%%
%% add information to TeX file before \begin{document} with:
%% \include{localhyphenation}
\hyphenation{
    Beck-man
    Ngu-yen
    back-chan-nel
    back-chan-nels
    mo-not-o-nous
    ste-reo-typ-i-cal
}

\hyphenation{
    Beck-man
    Ngu-yen
    back-chan-nel
    back-chan-nels
    mo-not-o-nous
    ste-reo-typ-i-cal
}

\hyphenation{
    Beck-man
    Ngu-yen
    back-chan-nel
    back-chan-nels
    mo-not-o-nous
    ste-reo-typ-i-cal
}

  \togglepaper[8]%%chapternumber
}{}




\begin{document}
\maketitle

\section{Introduction}\label{sec1}

\ili{Tungusic} is a small and highly endangered language family scattered across Northeast Asia. According to \citet{Janhunen2012}, \ili{Tungusic} can be divided into the \ili{Jurchenic} (IV), \ili{Nanaic} (III), \ili{Udegheic} (II), and \ili{Ewenic} (I) subbranches. The former two form the southern and the latter two the northern branch of \ili{Tungusic} (see also \citealt{Georg2004}). The enumeration of the individual subbranches with Roman numerals follows \citet{Ikegami1974}. Altogether, \ili{Tungusic} encompasses some twenty different languages.

This chapter investigates \ili{Tungusic} interrogatives, also known as interrogative words, question words etc. Most \ili{Tungusic} interrogatives started with an initial consonant for which several different reconstructions were proposed (see \citealt{Rozycki1993}). For the time being, the uncertainty of this consonant will be indicated using the label *\textit{K-}. The only subbranch that is known to preserve the feature in question is \ili{Nanaic}, which consists of the languages \ili{Hezhen}, \ili{Kilen}, \ili{Kili}, \ili{Nanai}, \ili{Samar} (Northern Nanai), \ili{Uilta}, \ili{Ulcha}, and \ili{Ussuri Nanai} (e.g., \citealt{Schmidt1923, Schmidt1928, Alonso2011, Janhunen2012}). For a first impression, consider \tabref{tab:1}, which lists four interrogatives from five different languages that represent all four \ili{Tungusic} subbranches.
% Roman numerals refer to the subbranches of \ili{Tungusic}.
In \ili{Nanai}, all four interrogatives start with an initial \textit{x-}, while in \ili{Manchu}, \ili{Udihe}, or \ili{Oroqen} no such phoneme is present. An underscore indicates the lack of the initial that was regularly lost in these languages. Unexpectedly, however, there is an initial \textit{k-} in a language called \ili{Alchuka}. This language is not very well known, but offers some crucial information for our understanding of \ili{Jurchenic} and \ili{Tungusic} in general \citep{Hölzl2017}.

\begin{table}
\caption{Selected cognates of interrogatives in Manchu \citep{Norman2013}, Alchuka \citep{MuYejun1986}, Nanai \citep{KoYurn2011}, Udihe \citep{NikolaevaTolskaya2001}, and Nanmu Oroqen \citep{Chaoke2007}. Not all variants are listed.}
\label{tab:1}
 \begin{tabular}{llllll} 
  \lsptoprule
  Meaning & \ili{Manchu} IV & \ili{Alchuka} IV & \ili{Nanai} III & \ili{Udihe} II & \ili{Oroqen} I\\
  \midrule
  how & - & - & \textbf{x}o:ni & \_ono & \_ooni\\
  how many & \_udu & \textbf{k}utu & \textbf{x}ado & \_adi & \_adi\\
  what & \_ai & \textbf{k}ai- & \textbf{x}aɪ & \_i:- & \_i-\\
  when & - & - & \textbf{x}a:li & \_aali & \_aala\\
  \lspbottomrule
 \end{tabular}
\end{table}

\noindent Two of the interrogatives were lost in the highly innovative \ili{Jurchenic} subbranch. Apart from regular phonological changes, some progressive (\ili{Udihe} \textit{on\textbf{o}}, \ili{Oroqen} \textit{aal\textbf{a}}) and regressive vowel assimilations (\ili{Manchu} \textit{\textbf{u}du}, \ili{Alchuka} \textit{k\textbf{u}tu}) have slightly distorted the picture. But there can be no doubt that these forms represent a valid set of cognates (e.g., \citealt[114]{Benzing1956}). The interrogatives form one coherent system with an initial \textit{x-} in \ili{Nanai}. This can be called a formal \textit{resonance}, which will be indicated using a tilde, i.e. \textit{x\textasciitilde}{ }\citep{Hölzl2018b}. This symbol is meant to indicate its partial analyzability as opposed to clearly analyzable morphemes that are indicated with a hyphen, e.g. \ili{Nanai} \textit{xaɪ-do} `what-\textsc{dat}'. As \citet[209]{BickelNichols2007} put it: ``parts of words resonate with each other and can therefore be extracted as meaningful formatives or morphemes." They give the example of \ili{English} demonstratives that share an initial /ð/. In \ili{English}, a similar phenomenon is known from certain lexical items such as \textit{\textbf{sn}ore}, \textit{\textbf{sn}eeze}, \textit{\textbf{sn}iff}, \textit{\textbf{sn}uff} etc., all of which start with \textit{sn\textasciitilde}{ }and have a vague similarity in meaning. The concept of a resonance can also be usefully applied to interrogatives \citep{Mackenzie2009}. The loss of the initial in \ili{Tungusic} would be comparable to the loss of the <wh> in \ili{English}, e.g. \textit{(wh)at}, \textit{(wh)en} etc. Usually, a resonance in an interrogative system is an indication of an old etymological connection. For instance, \ili{German} \textit{\textbf{w}o} `where' and \textit{\textbf{w}arum} `why' are synchronically unrelated, but share a resonance in \textit{w\textasciitilde}. Historically, however, the first part of \textit{war-um} (literally `where-around') is a cognate of \textit{wo}, which lost the final \textit{-r}, unless it was followed by a vowel. In this case, \textit{warum} constitutes a \textit{fused} form that is no longer analyzable \citep{MuyskenSmith1990}. In \ili{German}, there are only few interrogatives without an initial \textit{w\textasciitilde}, which can be due to prepositions or prefixes, e.g. \textit{in-\textbf{w}ie-fern}, \textit{in-\textbf{w}ie-weit} `how (far), to what extent' (literally `in how far'). \ili{Tungusic} languages only have suffixes and postpositions and exhibit the word order Interrogative Noun (IntN). There is thus a natural tendency to build up resonances over the course of time. Rephrasing \citet[413]{Givón1971}, one might say that, in these languages, today's resonance is yesterday's morphosyntax.

The resonance in \ili{Tungusic} is not as clear-cut as \ili{English} initial /ð/ in demonstratives such as \textit{\textbf{th}is}, because unlike in \ili{English} there are many other words with the same onset. However, the loss of the phoneme in lexical items such as `wind' had no effect on their meaning, e.g. \ili{Manchu} \textit{\textbf{\_}edun}, \ili{Nanai} \textit{\textbf{x}ədun}, \ili{Udihe} \textit{\textbf{\_}edi}, \ili{Nanmu Oroqen} \textit{\textbf{\_}ədin}. There simply is no set of weather-related words with an initial *\textit{K-}. On the contrary, the \ili{Tungusic} phoneme had a certain functional load in the interrogative system (cf. \citealt{Langacker2001}). This is why the loss of the phoneme led to a very incoherent interrogative system with a high number of individual forms that are synchronically unanalyzable. It changed the interrogatives in \tabref{tab:1} from a so-called \textit{fused} to an \textit{opaque} system that no longer exhibits any signs of a former etymological connection \citep{MuyskenSmith1990}. Of course, derivational or inflectional suffixes remained unaffected by this. For example, \ili{Manchu} (IV) \textit{ai-\textbf{de}}, \ili{Nanai} (III) \textit{xaɪ-\textbf{do}}, \ili{Udihe} (II) \textit{i:-\textbf{du}}, \ili{Chaoyang Oroqen} (I) \textit{i-\textbf{du}} `where' \citep[261]{HuZengyi2001} are all still analyzable as `what-\textsc{dat}'. Nevertheless, it will be argued in this chapter that the loss of the phoneme led to a \textit{complexification} of the \ili{Tungusic} interrogative systems. In \ili{Manchu}, for instance, there is no indication that \textit{ai} `what' and \textit{udu} `how many' could once have been related etymologically. \sectref{sec2} addresses the question how the complexity of interrogative systems can be described cross-linguistically.

One of the main problems for the comparison with \ili{Alchuka} is the irregularity of the occurrence of the initial \textit{k-}. For instance, it would be expected to be present in the word for `wind' as well, but this word has been recorded as \textit{\textbf{\_}ɔduŋ} \citep[6]{MuYejun1985}. At least in some cases, the irregular correspondences can be explained as cases of borrowing from \ili{Manchu} dialects that had a strong impact on \ili{Alchuka} (e.g., \ili{Aihui Manchu} \textit{\textbf{\_}odon/ŋ}, \citealt[162]{WangQingfeng2005}). For instance, the interrogative \textit{\textbf{k}ai-} has also been recorded as \textit{\textbf{\_}ei} by \citet[10]{MuYejun1986}. In this case, it is plausible to assume a certain amount of variation among what were probably the last speakers of the language when the data were collected during the 1960s \citep[5]{MuYejun1985}. Again, the variant \textit{\textbf{\_}ei} could represent a loan from a \ili{Manchu} dialect (e.g., \ili{Yibuqi Manchu} \textit{\textbf{\_}ɛi}, \citealt[127]{ZhaoJie1989}). Furthermore, the recording of the \ili{Alchuka} data by Mu Yejun is not always reliable. In specific cases, it is unclear whether the differences found in the recordings are true variation in the language itself or simple spelling mistakes. Nevertheless, independent evidence from other \ili{Jurchenic} varieties confirms that the initial consonant as such is a real phenomenon and cannot be due to mere spelling mistakes (e.g., \citealt{Kiyose2000}). \sectref{sec3} considers these problems relating to the \ili{Tungusic} interrogative systems further.

The following research questions will be addressed in this study: (1) Is \ili{Alchuka} \textit{k-} a reflex of \ili{Proto-Tungusic} *\textit{K-}? (2) What is the nature of the \ili{Proto-Tungusic} phoneme *\textit{K-}? (3) What were the consequences of its loss for the complexity of the \ili{Tungusic} interrogative system? (4) How can the complexity of interrogative systems be defined cross-linguistically? (5) And finally, can this development be considered a ``change for the worse" as indicated by the title of this volume?

The chapter has five subsections, including this introduction (\sectref{sec1}). \sectref{sec2} briefly introduces seven different dimensions of complexity and defines the complexification of interrogative systems. Based on this general outline, \sectref{sec3} analyses the \ili{Tungusic} interrogative system as well as its development through time. Given the rediscovered data from \ili{Alchuka}, a tentative reconstruction for the \ili{Proto-Tungusic} phoneme *\textit{K-} is proposed. \sectref{sec4} evaluates the loss of the phoneme as well as its consequences and inquires whether it can be considered a change for the worse. \sectref{sec5} presents some conclusions.

\section{The complexity of interrogative systems}\label{sec2}

Language is a very complex phenomenon. Every language requires a certain degree of complexity to be useful and engaging (cf. \citealt{Norman2010}). In comparison, differences between languages are relatively minor, but nevertheless clearly observable. There are different approaches to complexity. For instance, \citet{Miestamo2008grammatical} draws a distinction between absolute (i.e., objective) vs. relative (i.e., subjective) complexity on the one hand and between local vs. global complexity on the other. Following \citet{Miestamo2008grammatical}, this study is only concerned with \textit{absolute} and \textit{local complexity}, i.e. it tries to objectively describe the complexity of one domain of the \ili{Tungusic} languages. It will not evaluate the overall complexity of entire languages and will not refer to the relative difficulty in acquiring \ili{Tungusic} interrogatives or the like.

Complexity can tentatively be defined ``as the number of parts in a system or the length of its description'' \citep[27]{Miestamo2008grammatical}. However, complexity also depends on the quantity and quality of the interconnections of those parts (e.g., \citealt[viii]{Karlsson2008}) and on the status of the system in its linguistic environment (e.g., \citealt{Turvey2009}). The degree of complexity of a certain phenomenon can be described along different dimensions. For the purposes of this chapter, seven such dimensions will be differentiated (\tabref{tab:dimensions}).

\begin{table}
\caption{Dimensions of complexity considered in this study (loosely based on \citealt{McWhorter2007, Karlsson2008, Turvey2009, Trudgill2011}).}
\label{tab:dimensions}
 \begin{tabular}{llll} 
  \lsptoprule
  Number & Dimension & Simple & Complex\\
  \midrule
  1 & regularity & regular & irregular\\
  2 & redundancy & underspecified & overspecified\\
  3 & analyzability & analyzable & unanalyzable\\
  4 & amount & few forms & many forms\\
  5 & organization & organized & unorganized\\
  6 & coherence & coherent & incoherent\\
  7 & delineation & bounded & unbounded\\
  \lspbottomrule
 \end{tabular}
\end{table}

\noindent Depending on the domains of language under investigation, a different set of dimensions will be more adequate. The dimensions of organization, coherence, and delineation have been inspired by \citegen{Turvey2009} discussion of the notion of a nonsystem.\largerpage

\begin{quote}
What can be meant by nonsystem? A set of isolated pieces that don’t interact, or interact so weakly that their influences upon each other are negligible, seems to fit the bill. Even better, perhaps, is the notion of a collection of related pieces where the relations have no implications for the properties or behaviors of the pieces. Certainly lacking in the image of a nonsystem is the sense of shared influences or mutual dependencies; intuitively, a nonsystem exhibits no coherence or functional unity. Also lacking is the sense of a boundary, a separation of the pieces into “ground” (pieces that surround) and “figure” (pieces that are surrounded). \citep[98f.]{Turvey2009}
\end{quote}

\noindent Interrogatives cross-cut several different word classes (e.g., \citealt[409]{Dixon2012}). For this reason, an interrogative system is usually difficult to present in one table. Depending on the language, some of the interrogatives can often be inflected according to their word class, while others cannot. In \ili{Manchu}, for instance, \textit{we} `who' and \textit{ai} `what' can be inflected for case, while \textit{udu} `how many' can take morphology specific to numerals, e.g. \textit{udu-ci} `how\_many-\textsc{ord}', \textit{udu-te} `how\_many-\textsc{distr}'. Nevertheless, despite their inherent heterogeneity, interrogatives can still be said to form one functional domain. If we take the concept of an interrogative \textit{system} seriously (e.g., \citealt{MuyskenSmith1990}), it should exhibit all the hallmarks of a system described by Turvey. The more it resembles a \textit{nonsystem}, the more complex and unsystematic it is.

Diachronic changes in complexity can be called \textit{simplification} and \textit{complexification} \citep{Trudgill2011}. This paper is a case study in complexification, i.e. the emergence of more complex structures out of simpler ones. In the following, the seven dimensions presented in \tabref{tab:dimensions} will be briefly defined and exemplified by \ili{Tungusic} data. For a better understanding of the different dimensions, some cases of simplification will be mentioned as well.

\subsection{Regularity}
% (1) 
The first dimension of complexity is \textit{regularity} (e.g., \citealt[33--35]{McWhorter2007}; \citealt[85ff.]{Trudgill2011}). Exceptions to rules or irregularities in language structure, such as suppletion, increase the number of elements the system has. An example of irregularity in the interrogative system can be found in several northern \ili{Tungusic} languages. Most languages of this branch have a suffix that is usually exclusively encountered on one interrogative. In \ili{Udihe}, for example, only the nominative or unmarked form of the interrogative \textit{j'e-} exhibits the suffix \textit{-u}. It is lacking on other interrogatives and is replaced if the interrogative is inflected for case, e.g. \textit{j'e-\textbf{u}} `what', \textit{j'e-\textbf{du}} `what-\textsc{dat}' \citep[348]{NikolaevaTolskaya2001}. The suffix is also present in the closely related language \ili{Oroch}. However, case markers regularly attach to the suffix in \ili{Oroch}, which in this case could be analyzed as an augmentation of the nominal stem instead, e.g. \textit{jaa.\textbf{ʊ}} `what', \textit{jaa.\textbf{ʊ}-\textbf{du}} `what-\textsc{dat}' \citep[197]{AvrorinBoldyrev2001}. This is an example of a slight simplification of an interrogative system. In \ili{Manchu}, there is no trace of the suffix left, e.g., \textit{ya} `which', \textit{ya-de} `which-\textsc{dat}', i.e. the irregularity has been entirely lost.

Similar to cases of suppletion, exceptions from a resonance can also be considered an irregularity. The more exceptions from a resonance there are, the more irregular the interrogative system is. In \ili{Tungusic} the two interrogatives *\textit{ŋüi} `who' and *\textit{ja-} `(to do) what, which' were the only exceptions from the resonance in *\textit{K}\textasciitilde. The interrogative *\textit{ja-} has been entirely lost in \ili{Nanaic}, which is also the only subbranch to have generally preserved the resonance. This can be considered a decrease in irregularity in \ili{Nanaic} languages.

\subsection{Redundancy} 
% (2)
\textit{Redundancy} has also been called overspecification (e.g., \citealt[21--28]{McWhorter2007}). It is here understood as the number of different expressions for the same semantic category in a certain language has. Underspecification can lead to the creation of new forms and overspecification (or redundancy) can lead to competition and the loss of certain forms. A certain amount of redundancy must have already been present in the \ili{Proto-Tungusic} interrogative system. There seems to have been competition between two semantically very similar interrogatives that can roughly be reconstructed as *\textit{Kai-} and *\textit{ja-} `(to do) what, which'. These show an intriguing distribution among modern \ili{Tungusic} languages \citep[315f.]{Hölzl2018b}. The meaning `to do what' is usually expressed by *\textit{Kai-} in southern, but by *\textit{ja-} in northern \ili{Tungusic}. \ili{Ewenic} and \ili{Jurchenic} preserve both interrogatives, but \ili{Nanaic} has completely lost *\textit{ja-} and \ili{Udegheic} has almost entirely lost *\textit{Kai-}. It is only preserved in a few derived forms such as \ili{Udihe} \textit{i:-du} `where'. In other words, both \ili{Nanaic} and \ili{Udegheic} have simplified their interrogative system by reducing the amount of redundancy. To this day, several languages preserve some of this redundancy in sometimes allowing both stems for the same derivations and inflections, e.g. \ili{Manchu} (IV) \textit{ai-de} vs. \textit{ya-de}, \ili{Udihe} (II) \textit{i:-du} vs. \textit{j’e-du}, \ili{Even} (I) \textit{i-du} vs. \textit{ja-du} `where' etc. In some \ili{Ewenic} languages the stems have partly merged phonologically, e.g. Khamnigan \ili{Evenki} \textit{i(i)-} vs. \textit{i(e/i)-} \citep{Janhunen1991}, which is another example of simplification.

The presence of several resonances, such as \ili{English} /h/ (\textit{who}, \textit{how}, \textit{how many}/ \textit{much}) and /w/ (in all remaining interrogatives), can also be considered a form of overspecification. If a resonance, such as \ili{English} \textit{w\textasciitilde}, is in fact a submorpheme that carries a certain functional load (e.g., \citealt{Langacker2001, Mackenzie2009}), the existence of a second submorpheme \textit{h\textasciitilde} with the same function must be considered redundant.\footnote{This should not be confused with the first dimension that includes exceptions from any resonance.}

% (3) 
\subsection{Analyzability}
The dimension of \textit{analyzability} is understood here in the following sense: The more analyzable a form is with the help of the other elements in a system, the less complex it is.\footnote{Notice that this dimension does not take into account the number of elements an analyzable form exhibits.} But analyzability is not always clear-cut. Instead, there is usually a scale of more and less analyzable forms. Partly analyzable forms, cranberry morphs, or resonances complicate matters considerably. Interrogatives in creole languages usually tend to be analyzable (\citealt[65]{Bickerton2016}, \citealt[884]{MuyskenSmith1990}). However, an increase in analyzability is not necessarily restricted to creole languages. The \ili{Jurchenic} branch of \ili{Tungusic}, for example, is not a true creole but nevertheless exhibits some simplification due to non-native acquisition in its past (e.g., \citealt{McWhorter2007,Trudgill2011,Hölzl2018c}). Possibly, this can be observed in the interrogative system that exhibits a large amount of analyzable forms that are based on the interrogatives \textit{ai} and \textit{ya}, e.g. \ili{Manchu} \textit{ai-ba-}, \textit{ya-ba-} ‘where’ (\textit{ba} ‘place’). If no new forms are created, analyzability tends to decrease over the course of time. For instance, \ili{Manchu} \textit{ai-ba-de} `what-place-\textsc{dat}' has a variant \textit{ai-bi-de} `what-?place-\textsc{dat}'. The second element is a cranberry morph that most likely derives from \textit{ba} `place'. There is even a less analyzable variant \textit{abi-de} that must be the result of an additional contraction (e.g., \citealt{Norman2013}). Both \textit{-bi-} and \textit{a-} are no longer clearly analyzable within the system and therefore increase the number of elements. The suffix \textit{-u} in \ili{Udihe} only occurs on one interrogative and therefore is not analyzable with the help of other elements within the system.

From the perspective of analyzability, a resonance can be viewed as making a system more and less complex at the same time. On the one hand, a resonance by definition is only partly analyzable and therefore makes individual forms with the resonance more complex. On the other hand, it allows at least a partial analysis of all forms that exhibit the similarity. The loss of a resonance leads to the loss of an etymological connection between the individual interrogatives and thus to a decrease in analyzability. Due to this loss of analyzability, the number of individual interrogatives rises considerably.

\subsection{Amount}
% (4) 
The number or \textit{amount} of interrogatives appears relatively straightforward: the more interrogatives a language has, the more complex it is. However, the number depends on the analysis. One possibility would be to include ``basic" interrogatives, exclusively \citep{Hengeveld2012}. However, this approach suffers from the problems of analyzability mentioned above. Among \ili{Tungusic} languages, the lowest number of basic interrogatives seems to be present in \ili{Udegheic}. In \ili{Udihe}, for instance, there are only four unanalyzable forms: \textit{ni(:)} `who', \textit{ali} `when', \textit{adi} `how many', and \textit{ono} `how' \citep{NikolaevaTolskaya2001}. As we have seen above, even \textit{j’e-u} `what' contains a suffix. Of course, \ili{Udihe} has a wealth of additional interrogatives, but all of them are analyzable to different degrees. Most of them are derivations and inflected forms of \textit{j’e-} `(to do) what'. Some of those have a parallel based on the stem \textit{i:-} that is only preserved in derivations (\ref{ex:Udihe}).

\ea\upshape
    \label{ex:Udihe}
    Some variants of \ili{Udihe} interrogatives \citep{NikolaevaTolskaya2001}\\
    \textit{j'e-du}, \textit{i:-du} `what/which-\textsc{dat} > why, where'\\
    \textit{j'e-le}, \textit{i:-le} `what/which-\textsc{loc} > where'\\
    \textit{j'e-mi}, \textit{i:-mi} `what/which-\textsc{cvb} > why'
    \z

\noindent The forms make the impression of mere variants and are perhaps perceived as such by the speakers, but they are etymologically distinct (*\textit{ja-} vs. *\textit{Kai-}). In other words, the forms based on \textit{i:-} are synchronically only partly analyzable, because they contain some form of cranberry morph. It is an open question whether such partly analyzable forms should be treated as ``basic question words" or not.

The same problem applies to interrogative systems with a resonance. If, for example, a resonance is counted as one interrogative stem, \ili{Tungusic} most likely had only three interrogatives, i.e. *\textit{ŋüi} `who', *\textit{ja-} `(to do) what, which', and *\textit{K\textasciitilde} (e.g., \citealt{Benzing1956}, \citealt[312--330]{Hölzl2018b}). If, on the other hand, forms with a resonance that are otherwise unanalyzable are counted as well, their number increases substantially. If the loss of a resonance is a decrease in analyzability, theoretically the number of ``basic" interrogatives should rise, too.

\subsection{Organization}
% (5) 
Languages differ in their overall \textit{organization} of the interrogative system. In \ili{Tungusic}, for example, there is no dedicated interrogative meaning `where'. Instead, all languages employ case-marked interrogatives meaning `what' or `which' to express that notion. In these languages the interrogative meaning `where' is simply part of a paradigm. The inclusion into a paradigm could be interpreted as a form of regularity. However, this kind of organization leads to anomalous case forms that differ semantically from the rest of the paradigm, e.g. \ili{Udihe} \textit{j'e-we} `what (\textsc{acc})', but \textit{j'e-du} `where'. From this perspective, an interrogative system with a special locative interrogative, such as \ili{English} \textit{where}, could be considered more organized.

The dimension of organization can perhaps be applied to the semantic scope of a resonance. Ideally, the semantic scope of an interrogative covers a coherent region in semantic space (on which see \citealt{Cysouw2005,Cysouw2007}, \citealt[82f.]{Hölzl2018b} and references therein). More research is necessary on whether exactly the same principles can be applied to resonances. But because resonances usually emerge through the spread of one interrogative over several semantic categories and its subsequent decrease in analyzability, this is a plausible scenario. For instance, cross-linguistic research seems to indicate that the two categories \textsc{thing} (`what') and \textsc{quantity} (`how many/much') can only be expressed by the same form if \textsc{manner} (`how') is also expressed in the same way (e.g., \citealt{Cysouw2005}). However, in \ili{Kilen} the resonance \textit{χ\textasciitilde} can only be found on \textit{χai} `what' and \textit{χadu} `how many', but not on \textit{\_oni} `how' \citep{AnJun1986}. This unorganized system is the result of language contact in which the form \textit{\_oni} was borrowed from a northern \ili{Tungusic} language. \ili{Nanai} still has a more organized system with the form \textit{xo:ni} instead (\sectref{sec3}).

% (6) 
\subsection{Coherence}
Without doubt, \textit{coherence} (what holds the system together) is the most important dimension for this paper. There are several possibilities of analysis, but this study describes the coherence of an interrogative system in terms of resonances (i.e., formal coherence). One of the most striking examples of complexification in \ili{Tungusic} interrogatives was triggered by the phonological change pointed out in \sectref{sec1}. Consider Tables~\ref{tab:Uilta} and~\ref{tab:Evenki}, which list \ili{Uilta} and \ili{Evenki} interrogatives as examples of coherent and incoherent systems, respectively. \ili{Uilta} \textit{ŋui} and \ili{Evenki} \textit{niː}/\textit{nɪː} both go back to *\textit{ŋüi}. \ili{Tungusic} *\textit{ja-} has disappeared without a trace in \ili{Uilta}, but is still present in \ili{Evenki} as \textit{æː-}. Finally, the resonance *\textit{K\textasciitilde} is preserved in \ili{Uilta} as \textit{x\textasciitilde} but has been lost in \ili{Evenki}. Interestingly, \ili{Evenki} has many forms starting with \textit{i(ː)}/\textit{ɪ(ː)\textasciitilde}, while no such vowel follows the initial in \ili{Uilta} (see \sectref{sec3} for implications). This could be considered a secondary resonance in \textit{I\textasciitilde} that has been built up following the loss of the original resonance. However, there are several other forms such as \textit{ɔːqin}/\textit{ɔʁːin} `when' that do not conform to this pattern.

\begin{table}
\begin{floatrow}
\captionsetup{margin=.05\linewidth}
\ttabbox{\begin{tabular}{ll} 
  \lsptoprule
  Form & Meaning\\
  \midrule
  \textbf{ŋui} & who\\
  \textbf{x}aali & when\\
  \textbf{x}aawu, \textbf{x}auwu & which one\\
  \textbf{x}ai(-) & (to do) what\\
  \textbf{x}aidu & where\\
  \textbf{x}aimi & why\\
  \textbf{x}amaččuu & whence\\
  \textbf{x}amačiga & what kind of\\
  \textbf{x}asu & how many/much\\
  \textbf{x}awasai & whither\\
  \textbf{x}awwee & where, what place\\
  \textbf{x}ooni & how\\
  \\
  \lspbottomrule
 \end{tabular}}
 {\caption{The formally coherent interrogative system of Uilta (III) \citep{Ikegami1997}.}
\label{tab:Uilta}}
\ttabbox{\begin{tabular}{ll} 
  \lsptoprule
  Form & Meaning\\
  \midrule
  niː, nɪː & who\\
  æːqɷn, æːʁɷn & what\\
  æːχa & how\\
  adɪ, addi & how many\\
  ɔːqɪn, ɔʁːɪn & when\\
  irəgeɕin, irgəːtʃin & what kind\\
  iːdu & where\\
  iːli & where\\
  iːrba & how much\\
  iːʂ & which\\
  ɪraː & which one\\
  ɪrɢaː & how much\\
  ɪːla, ɪːra & where\\
  \lspbottomrule
 \end{tabular}}
 {\caption{The formally incoherent interrogative system of Aoluguya Evenki (I) \citep[171, 238]{Hasibateer2016}.}
\label{tab:Evenki}}
\end{floatrow}
\end{table}
\il{Uilta}\il{Evenki}

An important problem for this study is the question whether the presence of a \textit{resonance} is a complicating or simplifying factor. Perhaps, from the point of view of analyzability alone, a resonance makes things more complex by being only partly analyzable. From the point of view of coherence, however, one homogenous resonance as in \ili{Uilta} (\tabref{tab:Uilta}) could be said to be a simplifying factor instead because it holds the system together.

% (7) 
\subsection{Delineation}
\begin{sloppypar}
The dimension of \textit{delineation} (what differentiates the system from other elements) refers to the status of the interrogative system in a given language. A bounded system might exhibit certain phonological or morphological properties that are not found outside of the system. In \ili{English}, for instance, an initial /ð/ is almost exclusively encountered in the demonstrative system \citep[209]{BickelNichols2007}. Perhaps an analogy from visual perception can help make this point even clearer (cf. \citealt{Turvey2009}). A monochromatic piece of paper (a coherent interrogative system with one resonance) can be perceived much better than a multi-coloured one if they are held up before a heterogenous background (the linguistic system). Of course, it can be perceived even more clearly if the colour is not found in the background at all (if the resonance is restricted to the interrogative system). An example can be found in the nearby \ili{Turkic} language family. It has long been noted that an initial \textit{n-} in \ili{Proto-Turkic} was restricted to the interrogatives (see \citealt[354]{Hölzl2018b} and references therein). To my knowledge, no comparable phenomenon is known from \ili{Tungusic} interrogatives. Similar to \ili{English} /w/ or /h/ in the interrogatives, the \ili{Tungusic} initial *\textit{K-} also occurred on several other words.
\end{sloppypar}

However, there are some examples of the inflectional delineation of an interrogative system. In the \ili{Ewenic} language \ili{Even} (or Ewen), for instance, there is a suffix \textit{-k}, which is cognate with \ili{Udihe} \textit{-u}, \ili{Oroch} \textit{-ʊ}, and \ili{Evenki} \textit{-qɷn}/\textit{-ʁɷn} encountered before. Unlike these languages, \ili{Even} \textit{-k} can also be found on another interrogative and two demonstratives \citep[77, 79]{Benzing1955}. In this case, there is coherence in a subset of the interrogatives. Often, the interrogative system is only weakly delineated from the demonstrative system \citep{Diessel2003}. The two systems tend to have a certain amount of parallels and overlap in inflection or derivation. There can also be a formal resonance between interrogatives and demonstratives as a result of this, e.g. \ili{English} \textit{wh\textbf{ither}}, \textit{h\textbf{ither}}, and \textit{th\textbf{ither}}.

The different dimensions of complexity tentatively proposed in this section show complex patterns of interaction. For instance, the lack of analyzability is not only correlated with a higher number of individual forms, but also with incoherence. In \citegen[99]{Turvey2009} terms, the lack of analyzability with the help of elements within the system leads to a lack of ``mutual dependencies'' and therefore, to less coherence. Another example of such an interaction exists between coherence and delineation. A formally coherent interrogative system with one resonance in all interrogatives is more easily delineated from the rest of the language than one without any coherence.

\section{Loss of the resonance in Tungusic interrogatives}\label{sec3}

The phenomenon investigated in this section is the loss of the \ili{Proto-Tungusic} phoneme *\textit{K-} in word-initial position.\footnote{It may also have existed in word-internal position, see \citet{Janhunen2017} and references therein for some discussion. For reasons of space, this question cannot be addressed here.} As pointed out in \sectref{sec1}, this regular phonological process was extremely detrimental to the interrogative system in most languages where it fulfilled the role of a \textit{submorpheme} similar to \ili{English} <wh> \citep{BickelNichols2007,Mackenzie2009}. The implications of the loss will be pointed out in \sectref{sec4}. \ili{Tungusic} is one of several language families in Northeast Asia (NEA) and surrounding regions to exhibit what has been called \textit{K-interrogatives}: more than two interrogatives in a given language start with the same velar or uvular plosive or fricative \citep[6, 405f., 432]{Hölzl2018b}. Other language families with this feature include, for example, \ili{Mongolic} and \ili{Turkic}. As pointed out in \sectref{sec1}, a resonance usually indicates an etymological connection. A similar resonance across different language families is first and foremost a typological similarity, but could also indicate a certain connection in terms of language contact and/or a genetic relatedness. More research is necessary on their global distribution and origin, but K-interrogatives appear to be a relatively stable phenomenon in NEA and their loss in \ili{Tungusic} is rare, if not unique.

Traditionally, it was believed that the phoneme *\textit{K-} was lost everywhere but in \ili{Nanaic} (e.g., \citealt{Benzing1956}: 41f.). There is, however, also some evidence that the phoneme may have been present in \ili{Ewenic} at some point in time, where very few isolated relics with an initial \textit{h-} were preserved in peripheral varieties (e.g., \citealt{Vasilevich1958}, \citealt[581]{Doerfer1973}), e.g. Sakhalin \ili{Evenki} \textit{\_ure} `mountain' (\ili{Nanai} \textit{\textbf{x}ur\=ən}), but \textit{\textbf{h}erekī} `frog' (\ili{Nanai} \textit{\textbf{x}ərə}) \citep[106f.]{BulatovaCotrozzi2004}. It has also been speculated that a form of \ili{Jurchen} that can be called \ili{Jurchen} A \citep{Grube1896,Kiyose1977} may have had a few forms with an initial \textit{*h-} as well  \citep{Kiyose1996,Kiyose2000,Hölzl2017}. Certain modern \ili{Jurchenic} varieties potentially also preserve an initial \textit{h-} in some relics, e.g. Written \ili{Manchu} \textit{\_amaha}, but \textit{\textbf{h}amuha} `afterwards, later, future' as recorded in Qitamuzhen\il{Qitamuzhen Manchu} (\ili{Nanai} \textit{\textbf{x}ama-}, dial. \ili{Evenki} \textit{\textbf{h}ama-}) \citep{Hölzltoappear}. However, it was previously not widely known that the initial may also have been preserved as \textit{k-} in yet another language from the \ili{Jurchenic} branch called \ili{Alchuka} \citep{Hölzl2017,Hölzl2018b}. If correct, this clearly demonstrates that the phoneme was present in \ili{Proto-Tungusic} and must have been lost at a later stage. Additionally, this could give further evidence for the primary split of \ili{Tungusic} into northern and southern \ili{Tungusic}, as proposed by \citet{Georg2004} or \citet{Janhunen2012}. Most likely, *\textit{K-} was generally lost in northern \ili{Tungusic} -- there are only a few relics in \ili{Ewenic} and none in \ili{Udegheic} --, but was preserved in southern \ili{Tungusic}. It could have been lost at a relatively late stage in the majority of \ili{Jurchenic}.\footnote{\ili{Udegheic} seems to have a few cases that were borrowed from \ili{Nanaic}, e.g. \ili{Oroch} \textit{\textbf{x}uju(n)} `nine' (\ili{Ulcha} \textit{\textbf{x}uju(n)}, cf. \ili{Udihe} \textit{je(j)i}).}

\begin{sloppypar}
There have been several different more specific reconstructions of the phoneme *\textit{K-}, the most important of which are collected in \tabref{tab:reconstructions}. The difficulty of the reconstruction is due to the fact that the phoneme is only fully preserved in \ili{Nanaic}. The newly found data from a \ili{Jurchenic} language can potentially contribute much needed information for its reconstruction.
\end{sloppypar}

\begin{table}
\caption{A summary of the most important previous reconstructions of Tungusic *\textit{K-}.}
\label{tab:reconstructions}
 \begin{tabular}{lll} 
  \lsptoprule
  Source  &   Reconstr. &  Description\\
  \midrule
  \citealt{Schmidt1923}: 232 & *x- &  voiceless velar fricative\\
  \citealt{Shirokogoroff1931}: 244f.  &   *∅ &    later prothetic development\\
  \citealt{Cincius1949}: 250  &  *kxh-  &   aspirated voiceless velar affricate\\
  \citealt{Benzing1956}: 41ff.  &  *x-  &   voiceless velar fricative\\
  \citealt{Doerfer1973}: 579--591  &  *h-  &   voiceless glottal fricative\\
  \citealt{Cincius1975}: 300  &  *k’-  &   voiceless palatal plosive\\
  \citealt{Rozycki1993}: 211  &  *k’-  &   (un)aspirated voiceless (velar) plosive\\
  \lspbottomrule
 \end{tabular}
\end{table}

Most of the reconstructions are rather problematic and contradict what is known from cross-linguistic research on language change such as the cline in (\ref{ex:cline}).

\ea\label{ex:cline}
    \textit{k} > (\textit{kx} >) \textit{x} > \textit{h} > ∅
\z

\begin{sloppypar}
\noindent As \citet[29]{Bybee2015} points out: “These paths are unidirectional; that is, the changes always proceed from stop to affricate to fricative to /h/ to zero, and not in the other direction.” There is no evidence for an affricate in any \ili{Tungusic} language (cf. \citealt[250]{Cincius1949}), which has been put into parentheses. This general tendency also contradicts \citegen{Doerfer1973} assumption of a change of \ili{Tungusic} *\textit{h-} to \ili{Nanaic} \textit{x-}. If \ili{Alchuka} \textit{k-} can be shown to be an actual reflex of \ili{Proto-Tungusic} *\textit{K-}, the reconstruction would have to be changed to a plosive as well. \citet{Rozycki1993}, based on external comparisons, has also quite convincingly argued for the reconstruction as a plosive.\footnote{Please note that this study is mostly based on data from \ili{Tungusic} languages. For reasons of space, external comparisons, such as with \ili{Mongolic} languages, will be mentioned only briefly (e.g., \citealt{Doerfer1985, Rozycki1993, Janhunen2017} and references therein).} In addition, there are several areal parallels for a change from a velar plosive to a fricative in the interrogative system of, for example, \ili{Turkic} and \ili{Mongolic} languages (\tabref{tab:parallels}).
\end{sloppypar}

\begin{table}
\caption{Areal parallels for the lenition from plosive to fricative in the interrogative system \citep{HuImart1987,Anderson1998,Yamakoshi2007,Yamakoshi2011}.}
\label{tab:parallels}
 \begin{tabular}{lll} 
  \lsptoprule
  \ili{Turkic} & \ili{Fuyu Kirghiz} & \ili{Khakas}\\
  \midrule
  what kind of & \textbf{ɢ}adah, \textbf{ɢ}adǐh & \textbf{x}aydaɣ\\
  when & \textbf{ɢ}ajan & \textbf{x}aǯan\\
  where & \textbf{ɢ}ayda & \textbf{x}ayda\\
  which & \textbf{ɢ}ayzǐ & \textbf{x}ayzɨ\\
  \midrule
  \ili{Mongolic}  & \ili{Khamnigan Mongol} & \ili{Shineken Buryat}\\
  \midrule
  how many & \textbf{k}ədui & \textbf{x}edii\\
  when & \textbf{k}əzie & \textbf{x}ezee\\
  where & \textbf{k}aa- & \textbf{x}aa-\\
  who & \textbf{k}ən & \textbf{x}en\\
  \lspbottomrule
 \end{tabular}
\end{table}

In order to better decide which, if any, of the reconstructions is the most adequate, the actual reflexes observed among \ili{Tungusic} languages have to be consulted. \tabref{tab:IPA} represents a part of the consonant inventory of the \textit{International Phonetic Alphabet} (IPA). All attested reflexes in modern \ili{Tungusic} languages are printed in boldface. As can be seen, there is a wide variety of different reflexes that include one plosive (i.e., [k]) and six different fricatives differentiated by their place of articulation (i.e., [s], [ʃ], [ɕ], [x], [χ], and [h]). All fricatives and perhaps also the plosive are voiceless, which must be a feature inherited from the \ili{Proto-Tungusic} phoneme.

\begin{table}
\caption{IPA symbols for the phonetic space in question (voiceless / voiced). Attested reflexes of *\textit{K}, including allophones but not ∅, are in boldface.}
\label{tab:IPA}
\resizebox{\textwidth}{!}{\begin{tabular}{llllllllll} 
  \lsptoprule
  & Alv. & Postalv. & Alv.-pal. & Retr. & Pal. & Vel. & Uvul. & Phar. & Glot.\\
  \midrule
  Plosive & t / d & -- & -- & ʈ / ɖ & c / ɟ & \textbf{k} / g & q / ɢ & -- & ʔ / --\\
  Fricative & \textbf{s} / z & \textbf{ʃ} / ʒ & \textbf{ɕ} / ʑ & ʂ / ʐ & ç / ʝ & \textbf{x} / ɣ & \textbf{χ} / ʁ & ħ / ʕ & \textbf{h} / ɦ\\
  \lspbottomrule
 \end{tabular}}
\end{table}

The sounds mentioned in \tabref{tab:IPA} are a summary of the entire language family and cannot all be found in one single language. The plosive is only sufficiently attested in \ili{Alchuka}. However, a \ili{Jurchenic} variety that I call \ili{Chinese Kyakala} potentially also has one example of an initial \textit{k-}, i.e. \ili{Manchu} \textit{urun}, \ili{Chinese Kyakala} \textit{\textbf{k}ulun} (or perhaps \textit{kurun}) `wife, bride' (see \citealt{Hölzl2018a, Hölzltoappear}). Some peripheral \ili{Ewenic} and perhaps \ili{Jurchenic} languages exhibit an \textit{h-} \citep{Doerfer1973,Kiyose1996,Kiyose2000,Hölzl2017}. All fricatives are otherwise only attested in \ili{Nanaic}. For example, there are three different reflexes in \ili{Hezhen}, which seems to be the maximum among \ili{Tungusic} languages. Apart from some exceptions, the nature of the sound can be predicted by the following vowel.

\ea\upshape
    \label{ex:2} Reflexes of *\textit{K-} in \ili{Hezhen} \citep[79f.]{AnJun1986}\\
    \textit{*K-} > \textit{ɕ-} | \_\textit{i}\\
    \textit{*K-} > \textit{x-} | \_\textit{ə}, \_\textit{u}, (\textit{\_i})\\
    \textit{*K-} > \textit{χ-} | \_\textit{a}, \textit{\_o}\\
    \z

\noindent The set of reflexes in \tabref{tab:IPA} differs significantly from that proposed in \citet[41]{Benzing1956}, who, apart from ∅, only mentions \textit{s-}, \textit{x-}, \textit{h-}, and, problematically, \mbox{\textit{n-}.} The nasal appears to be a mistake that resulted from a misunderstanding of a  secondary innovation in \ili{Manchu}. \citet[43]{Benzing1956} mentions the two \ili{Manchu} examples \textit{(\textbf{n})imenggi} `oil' and \textit{\textbf{n}imanggi} `snow' that, apart from the differences in derivational suffixes, correspond to, for example, \ili{Uilta} \textit{\textbf{s}imuksə} and \textit{\textbf{s}imana}, respectively \citep{Ikegami1997}. However, the correspondence of \textit{n-} and \textit{s-} is only valid at a first glance. Consider the comparison in \tabref{tab:nasal}. There is a relatively clear correspondence between \textit{n-} in \ili{Manchu} and ∅ in \ili{Alchuka}, especially with a following \textit{m}.\footnote{There are, however, several irregularities regarding the initial (palatal) nasal in \ili{Jurchenic} and \ili{Kilen} that deserve a treatment of their own.} Crucially, this is a later phenomenon that can also be found in loanwords such as \textit{niman} `goat' that do not have a \ili{Tungusic} background, but derive from surrounding languages such as \ili{Khitan} (see \citealt{Tang2011}). In other words, the initial \textit{n-} in \ili{Manchu} cannot be a reflex of \ili{Tungusic} *\textit{K-}. The presence of the initial \textit{n-} in \textit{(\textbf{n})imenggi} `oil' and \textit{\textbf{n}imanggi} `snow' must be considered an epenthetic element (cf. \ili{Hezhen} \textit{\textbf{\_}imaχa} `fish' etc.).

\begin{table}
\caption{A comparison of Manchu and Alchuka (\citealt{MuYejun1985,MuYejun1987,Norman2013}). JA = Jurchen A \citep{Kiyose1977}.}
\label{tab:nasal}
 \begin{tabular}{lll} 
  \lsptoprule
   & \ili{Manchu} & \ili{Alchuka}\\
  \midrule
  fish & \textbf{n}imaha & \textbf{\_}imaha\\
  goat & \textbf{n}iman & \textbf{\_}iman\\
  mulberry tree & (\textbf{n})imala(n) & \textbf{\_}imala\\
  oil & (\textbf{n})imenggi & JA *\textbf{\_}imengi\\
  snow & \textbf{n}imanggi & \textbf{\_}imaŋi\\
  \lspbottomrule
 \end{tabular}
\end{table}

Concerning the reflexes of *\textit{K-} in \ili{Nanaic}, consider \tabref{tab:Nanaic}. Only a selection of examples and sources available for \ili{Nanaic} languages was chosen. The primary split of the phoneme in \ili{Nanaic} appears to have been triggered by the following vowel. As seen for \ili{Hezhen} above (\ref{ex:2}), the reflex usually is an \textit{s}-like sound in front of \textit{i} (or \textit{ɪ}) and an \textit{x}-like sound elsewhere (e.g., \citealt[41f.]{Benzing1956}). There are some language-specific problems that cannot all be addressed here. For instance, \citet[2]{Tsumagari2009} notes that an /s/ in \ili{Uilta} is only realized as [s] before the vowels \textit{a} and \textit{o} [ɔ]. Before all other vowels, including \textit{i}, it is pronounced as a [ʃ] or [s\uppercase{j}].

\begin{table}
\caption{Reflexes of *\textit{K-} in Nanaic according to different authors in alphabetical and chronological order. Inner-Tungusic loanwords are given in parentheses. Not all variants are mentioned. Accents removed.\label{tab:Nanaic}}
\resizebox{\textwidth}{!}{\begin{tabular}{lllll} 
  \lsptoprule
  Language & back(wards) & wind & snow & Source\\
  \midrule
  \ili{Hezhen} & \textbf{h}ami(kə) & \textbf{h}ət\~{ɔ} & \textbf{h}imana, \textbf{ʃ}imana &  Ling \citeyear{Ling1934}\\
   & \textbf{χ}amilə & \textbf{x}ədun & \textbf{x}imanə &  \citealt{AnJun1986}\\
  \ili{Kilen}  & (\textbf{\_}amidʒikə) & \textbf{h}ət\~{ɔ} & (\textbf{\_}imana) &  Ling \citeyear{Ling1934}\\
   & ? & ? & \textbf{ch}emana &  \citealt{Jettmar1937}\\
   & (\textbf{\_}amidʑgə) & (\textbf{\_}ədin) & (\textbf{\_}imanə) &  \citealt{AnJun1986}\\
  \ili{Kili} & (\textbf{\_}amaski) & (\textbf{\_}ədi\textsuperscript{n}) & (\textbf{\_}emana) & \citealt{Sunik1958}\\
  \ili{Nanai} & \textbf{x}amasi & \textbf{\'{x}}edun & \textbf{x}imana, \textbf{s}imota &  \citealt{Grube1900}\\
   & \textbf{x}amasi & \textbf{x}ödun [-ə-] & \textbf{x}imana &  \citealt{Schmidt1923}\\
   & \textbf{x}amasi & \textbf{x}ədun & \textbf{s}ɪmana, \textbf{s}ɪmata &  \citealt{KoYurn2011}\\
  \ili{Samar} & ? & \textbf{x}ödu(n) [-ə-] & \textbf{s}imana & \citealt{Schmidt1928}\\
  \ili{Ulcha} & \textbf{x}amasi & \textbf{x}ödu [-ə-] & \textbf{s}imata &  \citealt{Schmidt1923}\\
   & \textbf{x}amasi & \textbf{x}ydu & \textbf{x}emana, \textbf{s}imata &  \citealt{Majewicz2011}\\
  \ili{Uilta} & \textbf{h}amasai & \textbf{h}uidö [-ə] & \textbf{s}imana, \textbf{s}imatta &  \citealt{Nakanome1928}\\
   & \textbf{x}amaśa & \textbf{x}ydu & \textbf{s}imani, \textbf{s}imat(t)a &  \citealt{Majewicz2011}\\
   & \textbf{x}amasai & \textbf{x}ədu & \textbf{s}imana, \textbf{s}imatta &  \citealt{Ikegami1997}\\
  U. Nanai\il{Ussuri Nanai} & \textbf{h}amela  & \textbf{h}edou [-u] & ? & \citealt{Venjukov1862}\\
   & \textbf{χ}amas'ɪ & \textbf{x}ədu(n-) & \textbf{s'}ɪm(a)na, \textbf{s'}ɪm(a)ta &  \citealt{Sem1976}\\
  \lspbottomrule
 \end{tabular}}
\end{table}

Especially older descriptions suffer from an unclear and inexact notation of phonemes. It is not entirely clear, for instance, what sound the initial <ch> in \ili{Kilen} mentioned by \citet{Jettmar1937} in his \ili{German} description represents. In \ili{German}, a <ch> would normally be pronounced as [ç] before an \textit{e}, but it is doubtful that this rule applies here. Most likely, it represents a [x] instead, which is another allophone of <ch> in \ili{German}. This is one of several examples where a velar-like fricative is preserved in the word for snow. This, as well as the complementary distribution of the \textit{s}-like and \textit{x}-like phonemes, are the main arguments for the assumption that the same phoneme *\textit{K-} was present in this word and in similar cases.

\tabref{tab:interrogatives} lists the \ili{Nanaic} cognates of three interrogatives according to the same sources as in \tabref{tab:Nanaic}. Given that in no \ili{Nanaic} interrogative the resonance was followed by an \textit{i} or \textit{ɪ}, the velar-like phoneme is preserved everywhere (cf. \sectref{sec2}).

\begin{table}
\caption{Reflexes of *\textit{K-} in Nanaic interrogatives according to several different authors in alphabetical and chronological order. Likely inner-Tungusic loanwords are given in parentheses. Accents removed.}
\label{tab:interrogatives}
 \begin{tabular}{lllll} 
  \lsptoprule
  Language & what & how many & how & Source\\
  \midrule
  \ili{Hezhen} & \textbf{h}ai & ?\textbf{h}adu & ? &  Ling \citeyear{Ling1934}\\
   & ? & ? & ? & \citealt{AnJun1986}\\
  \ili{Kilen} & \textbf{h}ai & \textbf{h}adu & \textbf{h}ɔni- & Ling \citeyear{Ling1934}\\
   & ? & (\textbf{\_}adi) & ? &  \citealt{Jettmar1937}\\
   & \textbf{χ}ai & \textbf{χ}adu & (\textbf{\_}oni) &  \citealt{AnJun1986}\\
  \ili{Kili} & (\textbf{\_}ii-) & (\textbf{\_}adi) & (\textbf{\_}ōni) &  \citealt{Sunik1958}\\
  \ili{Nanai} & \textbf{x}ai, \textbf{h}ai- & \textbf{x}adu, \textbf{h}adu & \textbf{x}oń(e) &  \citealt{Grube1900}\\
   & \textbf{x}ai & \textbf{x}adu & \textbf{x}oņe & \citealt{Schmidt1923}\\
   & \textbf{x}aɪ & \textbf{x}ado & \textbf{x}o:ni & \citealt{KoYurn2011}\\
  \ili{Samar} & \textbf{x}ai & ? & ? & \citealt{Schmidt1928}\\
  \ili{Ulcha} & \textbf{x}ai & \textbf{x}adu & \textbf{x}ōni & \citealt{Schmidt1923}\\
   & \textbf{x}aj & ?\textbf{x}adum & \textbf{x}on(i)  & \citealt{Majewicz2011}\\
  \ili{Uilta} & \textbf{h}ai & - & \textbf{h}ôni &  \citealt{Nakanome1928}\\
   & \textbf{x}aj & - & \textbf{x}ōni &  \citealt{Majewicz2011}\\
   & \textbf{x}ai & - & \textbf{x}ooni &  \citealt{Ikegami1997}\\
  U. Nanai\il{Ussuri Nanai} & \textbf{h}aï & ? & \textbf{h}oni & \citealt{Venjukov1862}\\
   & \textbf{χ}aɪ & \textbf{χ}ado, \textbf{χ}adʊ & \textbf{χ}on'(i) &  \citealt{Sem1976}\\
  \lspbottomrule
 \end{tabular}
\end{table}

The question whether \ili{Alchuka} \textit{k-} is a reflex of \ili{Tungusic} *\textit{K-} is extremely complex and difficult to answer. Not all problems can be solved or even addressed in this chapter. As mentioned in \sectref{sec1}, there are certain irregularities. \tabref{tab:Alchuka} lists all attested interrogatives in \ili{Alchuka}. Apparently, *\textit{ja-} has not been recorded. Most likely, \textit{p`ə} `who' derives from *\textit{ŋüi}, but this cannot be a regular continuation \citep[314]{Hölzl2018b}. The resonance is only present in five out of the ten remaining recorded interrogatives. Those without the initial might represent borrowings from \ili{Manchu} dialects. However, only \ili{Bala} has an \textit{n} in the word for `when' \citep[330]{Hölzl2018b}.

\begin{table}
\caption{Interrogatives in Alchuka \citep{MuYejun1986,MuYejun1987,MuYejun1988} in comparison with Manchu \citep{Norman2013}. Likely inner-Tungusic loanwords are given in parentheses \citep[317]{Hölzl2018b}.}
\label{tab:Alchuka}
 \begin{tabular}{lll} 
  \lsptoprule
   & \ili{Alchuka} & \ili{Manchu}\\
  \midrule
   who & ?p`ə & we\\
   for what reason & (\textbf{\_}ei) t`uku & \textbf{\_}ai turgun\\
   how & \textbf{k}atiram & \textbf{\_}adarame\\
   how many & \textbf{k}utu & \textbf{\_}udu\\
   to do what & \textbf{k}ai-na-mei & \textbf{\_}ai-na-mbi\\
   what & (\textbf{\_}ei) & \textbf{\_}ai\\
      what has happened & \textbf{g}ai-na-hanbie & \textbf{\_}ai-na-habi\\
   what (is it) & \textbf{k}ent`aka & \textbf{\_}antaka\\
   when & (\textbf{\_}ant`aŋgi) & \textbf{\_}atanggi\\
   where & (\textbf{\_}ai-və-t) & \textbf{\_}ai-ba-de\\
   why & (\textbf{\_}einu) & \textbf{\_}ainu\\
  \lspbottomrule
 \end{tabular}
\end{table}

In general, it is possible to identify several different categories. First, there are words with an initial \textit{k-} that have a clear correspondence in \ili{Nanaic}. Second, there are words with an initial \textit{k-} that do not have a correspondence in \ili{Nanaic}. Third, there are many words that would be expected to exhibit the initial \textit{k-} based on \ili{Nanaic} data, but do not. Fourth, in a few cases there is a potential external comparison outside of \ili{Tungusic} (on which see also \citealt{Rozycki1993}).\footnote{Potentially, some of the interrogatives in \ili{Tungusic} could have a \ili{Mongolic} origin, too, but this requires further research.} \tabref{tab:categories} mentions three examples of each category. Finally, there are at least two cases in which the initial \textit{k-} has comparisons in \ili{Jurchenic} \citep{Hölzl2017}. The list is not exhaustive, but sufficient for the purposes of this paper.

\begin{table}
\caption{A comparison of Manchu \citep{Norman2013}, Alchuka \citep{MuYejun1985,MuYejun1986}, and Uilta \citep{Ikegami1997}. JA = Jurchen A, JM = Jing Manchu, Kh. = Khalaj, MK = Middle Korean, PM = Proto-Mongolic. Not all variants attested for Alchuka are shown.}
\label{tab:categories}
 \begin{tabular}{lllll} 
  \lsptoprule
  Category & Meaning & \ili{Manchu} IV & \ili{Alchuka} IV & \ili{Uilta} III\\
  \midrule
  1 & how many & \textbf{\_}udu & \textbf{k}utu & \textbf{x}adu\\
  & twenty & \textbf{\_}orin & (\textbf{k})ɔrin & \textbf{x}ori\\
  & what & \textbf{\_}ai- & (\textbf{k})ai- & \textbf{x}ai-\\
  \midrule
  2 & this & \textbf{\_}e-re & \textbf{k}ə-r(ə) & \textbf{\_}ǝ-ri\\
  & this way, here & \textbf{\_}ebsi & \textbf{k}e’uʐï & \textbf{\_}ǝwǝsǝi\\
  & to become & \textbf{\_}o- & (\textbf{k})ɔ- & \textbf{\_}o-\\
  \midrule
  3 & nine & \textbf{\_}uyun & \textbf{\_}ujen & \textbf{x}uju\\
  & what & \textbf{\_}ai & \textbf{\_}ei & \textbf{x}ai\\
  & wind & \textbf{\_}edun & \textbf{\_}ɔduŋ & \textbf{x}ǝdu\\
  \midrule
  4 & twenty & \textbf{\_}orin & (\textbf{k})ɔrin & PM *\textbf{k}ori/n\\
  & virtue & \textbf{\_}erdemu & \textbf{k}ǝrdem & Kh. \textbf{h}är `man'\\
  & \textsc{-q} & \textbf{\_}o & (\textbf{k})ɔ & MK (\textbf{k})o\\
  \midrule
  5 & nineteen & JA *\textbf{\_}onioxon & (\textbf{k})uniku & JM \textbf{k}uniu\\
  & to meet & \textbf{\_}aca- & \textbf{k}atʃ’a- & Bala \textbf{h}atʃ’a-\\
  \lspbottomrule
 \end{tabular}
\end{table}
\il{Bala}\il{Jing Manchu}\il{Jurchen}\il{Middle Korean}\il{Khalaj}\il{Proto-Mongolic}\il{Manchu}\il{Alchuka}\il{Uilta}

Theoretically, the initial \textit{k-} in \ili{Alchuka} could be a later prothetic development that is specific to this language (cf. \citealt{Shirokogoroff1931}). Given the strongly suffixing character of all of \ili{Tungusic}, it is implausible to assume an otherwise unknown prefix \textit{k-}. One should not exclude the possibility of a prothetic development for some cases, especially those of category two that have no correspondence in \ili{Nanaic}. However, there is evidence that at least in some cases the \textit{k-} cannot be a secondary innovation. Given the fact that there are \ili{Nanaic} correspondences in category one, the problem is unlikely to be due to chance. These examples cannot be explained by borrowing from \ili{Nanaic} either. For example, \ili{Alchuka} \textit{\textbf{k}utu} contains a vowel assimilation specific to \ili{Jurchenic} and \textit{\textbf{k}ai-na-} has a verbalizer that does not occur in this form in \ili{Nanaic}. As indicated in \sectref{sec1}, at least some examples of the third category can be readily explained with borrowing from \ili{Manchu} dialects, which appear to have had a strong influence on \ili{Alchuka}. In many cases, this might explain the absence of the initial \textit{k-} that would otherwise be expected on the basis of a comparison with \ili{Nanaic}. This is especially plausible if there are doublets such as \textit{\textbf{k}ai-} vs. \textit{\_ei} `what'. These must reflect an autochthonous and a borrowed form, respectively. It should be noted that the same problem exists for the three \ili{Nanaic} languages \ili{Kili}, \ili{Kilen}, and \ili{Ussuri Nanai}, which have many loanwords from \ili{Ewenic}, \ili{Udegheic}, and \ili{Jurchenic} without the initial. This explanation is especially convincing if a given loanword exhibits additional features that are only attested in another language. For instance, \ili{Kili} \textit{\textbf{\_}ǝdi\textsuperscript{n}} `wind' not only lacks the initial consonant that is present, for example, in \ili{Uilta} \textit{\textbf{x}ǝdu}, but the vowel \textit{i} in the second syllable is a feature specific to northern \ili{Tungusic} \citep{Benzing1956}. In the case of \ili{Alchuka}, such identifying features are often difficult to find because all languages involved are relatively closely related. More research on \ili{Manchu} dialects is necessary in order to identify the exact source of the borrowings.

Potentially, some of the words with an initial could also represent spelling mistakes that are not uncommon in Mu Yejun's data. However, one should not jump to the conclusion that all of the examples can be explained in this way. For example, the initial \textit{k-} in numeral nineteen has been independently confirmed by Aixinjueluo Yingsheng, who remembered to have heard the form \textit{\textbf{k}uniu} in his youth (see \citealt{Aixinjueluo2014,Hölzl2017}).

Apart from the comparison with \ili{Nanaic}, there are additional indications that the initial \textit{k-} in \ili{Alchuka} is neither due to chance, nor a spelling mistake. Crucially, there are a few potential comparisons outside of \ili{Tungusic} that deserve further discussion. For instance, the question marker \textit{=o} in \ili{Manchu} that lacks a \ili{Tungusic} background is most likely a loan from \ili{Middle Korean} \textit{-(k)o} \citep[213]{Hölzl2018b}. Furthermore, there is one example (\textit{\textbf{k}atʃ’a-} `to meet') that allows a comparison with an initial \textit{h-} in \ili{Bala} \textit{\textbf{h}atʃ’a-} and \ili{Jurchen} A *\textit{\textbf{h}ača-}. Problematically, this initial \textit{h-} is similarly irregular (e.g., \ili{Jurchen} A *\textbf{\_}\textit{onioxon} `19') and is only attested in a few words \citep{Kiyose1996,Kiyose2000,Hölzl2017,Hölzltoappear}. But it represents additional evidence that the \textit{k-} in \ili{Alchuka} is neither an isolated phenomenon, nor a spelling mistake. The initial \textit{h-} in \ili{Evenki} dialects is similarly problematic but is still accepted as a valid correspondence by \citet{Doerfer1973}.

Yet another problem concerns the nature of the phoneme in \ili{Alchuka}. It is usually written as <k> in \citet{MuYejun1986}, but as <g> in \citet{MuYejun1987,MuYejun1988} (see \tabref{tab:Alchuka}). Descriptions of \ili{Jurchenic} varieties disagree on the nature of the plosives. More research on the phonology of \ili{Jurchenic} is necessary to determine the exact phonetic value of the plosives. It is possible that, at least in some varieties and similar to \ili{Mandarin} (e.g., \citealt{ZhaoJie1989}), the distinction between <g> and <k> is only one of aspiration ([k], [kʰ]) and not of voice as well ([g], [kʰ]). But \citet[27]{Norman2004/5} argues that, in \ili{Manchu}, a <g> is only pronounced as a voiceless unaspirated [k] in initial position. In any case, \ili{Alchuka} <k> \citep{MuYejun1986} also corresponds to what is usually considered a voiced velar plosive <g> in \ili{Manchu}. If \ili{Alchuka} \textit{k-} is indeed a reflex of \ili{Tungusic} *\textit{K-}, it must have historically merged with the reflex of the original *\textit{g-}. Interestingly, the irregularity in \ili{Alchuka} seems to include both the reflexes of *\textit{K-} and *\textit{g-}. For instance, the interrogative \textit{\_ei} is attested in the complex expression \textit{ei əl'un ə'ɔ} \citep[10]{MuYejun1986}, a cognate of \ili{Manchu} \textit{ai gelhun akū} `how dare ...' \citep{Norman2013}. The lack of several word-internal consonants is a different problem. But the cognate in \ili{Alchuka} also lacks an initial \textit{k-} that would be expected in \textit{\_əl'un} `timid'. Thus, it seems that the question of the initial \textit{k-} in \ili{Alchuka} is a more general problem. Future research will have to explain the sporadic loss of the initial *\textit{g-} and some other consonants, which goes beyond the possibilities of this study.

Based on the evidence in this section, a more detailed reconstruction of *\textit{K-} might be possible. Apart from the \ili{Alchuka} data, the reconstruction as *\textit{x-} is, of course, very convincing, because it fits very well into the \ili{Proto-Tungusic} consonant system and also has a potential areal parallel in \ili{Mongolic} \citep{Janhunen2017}. However, as seen above, the phoneme could well have been a plosive rather than a fricative. A crucial question is the general structure of the \ili{Proto-Tungusic} obstruent system \citep{Rozycki1993,Janhunen2017}. According to the traditional reconstruction \citep[27]{Benzing1956}, \ili{Tungusic} had the velar consonants *\textit{g} and *\textit{k} (i.e., [g], [kʰ]). In most languages, *\textit{K} shows a different set of reflexes than *\textit{g} and *\textit{k}. Consequently, it must have differed in some respect from the other velar plosives. \citet[211]{Rozycki1993} assumes that there might have been a distinction in aspiration (i.e., [k], [g], [kʰ]), and indeed the \ili{Alchuka} data potentially give additional evidence for this point of view. However, there are several additional possibilities such as a difference in the place of articulation instead of the manner of articulation. For example, several languages have an alveolar-palatal or uvular reflex of the phoneme *\textit{K-}, which suggests that it could theoretically also have been a [c] or [q], with or without aspiration, instead of a [k] (cf. \citealt{Cincius1975}).


\section{Complexification or a change for the worse?}\label{sec4}

To sum up the discussion thus far, there are arguments for the existence of a phoneme *\textit{K-} in \ili{Proto-Tungusic} that was lost in the majority of the daughter languages. Its possible existence in a \ili{Jurchenic} language provides additional evidence against a later innovation (i.e., a prothetic development) and for its potential reconstruction as a plosive. Given that it used to have the function of a submorpheme in the interrogative system, its loss was more than a mere phonological change but also had functional implications.

More specifically, it had the consequence of making the interrogative system more complex on most or all of the seven dimensions mentioned in \sectref{sec2}. Arguably, the interrogative system in \ili{Nanaic} is more regular, less redundant, more analyzable, more organized, more coherent, and better delineated than that of most other \ili{Tungusic} languages.

First, the interrogative system became \textit{irregular} due to many exceptions from newly created resonances, such as \textit{I\textasciitilde} in \ili{Aoluguya Evenki} (see \sectref{sec3}). Some languages, such as \ili{Udihe}, lack a resonance entirely, i.e. there is no formal regularity in the first place.

Second, the new interrogative systems are \textit{redundant} in sometimes having more than one resonance, e.g. \textit{a\textasciitilde} and \textit{y\textasciitilde} in \ili{Sibe} (see below), although the spread of the resonance \textit{y\textasciitilde} might have been independent of the phonological change observed in this paper.

Third, forms that used to be at least partly analyzable (e.g., \ili{Nanai} \textit{xado}, \textit{xaɪ}) became entirely \textit{unanalyzable} and etymologically opaque (e.g., \ili{Manchu} \textit{udu}, \textit{ai}).

Fourth, in some languages this loss of analyzability led to an \textit{increase in the number} of interrogatives, especially if the resonance in *\textit{K\textasciitilde} is considered some form of partially analyzable interrogative stem in its own right. If, on the other hand, the resonance is not granted such a position, its loss did not necessarily affect the number of interrogatives.

Fifth, at a first glance, the overall organization of the interrogatives appears to be unaffected. Because \ili{Tungusic} languages have suffixes exclusively, inflectional paradigms and derivations generally remained intact. However, from the point of view of organization, the special position of \ili{Tungusic} *\textit{ŋüi} `who', which is even more pronounced in \ili{Nanaic}, could also have its merit if this mirrors a special and salient position of the category \textsc{person} in human cognition. In fact, there is empirical evidence for this assumption. In many languages, the personal interrogative stands apart phonologically or morphosyntactically from the rest of the interrogative system \citep[406]{Hölzl2018b}. In addition, few languages have one category for both \textsc{person} and \textsc{thing} and innovative interrogative systems such as in \ili{Manchu} are usually based on `what' or `which', but rarely on other categories (e.g., \citealt{Cysouw2007}). An interrogative system as in \ili{Uilta} with a special position of the category \textsc{person} and a larger set of forms with a shared origin is thus a very organized and natural outcome of general processes and tendencies. It indicates that in \ili{pre-Proto-Tungusic} times there may have been an innovative interrogative system with a large set of analyzable forms that resulted in the later resonance. Given that the resonance covers a historically grown (and ideally coherent) region in the semantic space of interrogatives (e.g., \citealt{Cysouw2005,Hölzl2018b}), its loss entirely \textit{disrupted the organization} (i.e., the form-function mapping) of that system (but see below).

Sixth, and most importantly, the formal coherence of the interrogative system was lost. The new interrogative systems simply have no phonological or morphological marker in common but consist of a loose set of synchronically unrelated and \textit{incoherent} forms that only share some semantic similarities.

At a first glance, the seventh dimension appears to be similarly unaffected as the fifth. The phoneme *\textit{K-} in \ili{Tungusic} was not restricted to its function as resonance, but also occurred in many other lexical items (e.g., \citealt[227--250]{Ikegami1997}). Although the same is true for the new systems, they are much less homogenous and therefore \textit{less delineated} if taken as a whole. For example, there are chance resemblances to the demonstrative systems (e.g., \ili{Manchu} \textit{\textbf{u}ttu} `thus, like this', \textit{\textbf{u}du} `how many') and many lexical items.

The title of this volume is \textit{Language change for the worse}. In the description of the workshop it is based on, ``changes for the worse'' were defined as those changes ``that do not readily follow from an improvement in some other area of the language system''. The complexification of the \ili{Tungusic} interrogative system is an epiphenomenon of a phonological change. Even if the change in \ili{Tungusic} had the consequence of a complexification -- whether this is a change for the better or the worse is another question --, it was triggered by another change that may have been somehow beneficial \citep[195]{Dixon2016}. By definition, the change in the interrogative system in \ili{Tungusic} can only be considered a change for the worse if this phonological change was not an improvement in itself. However, the \textit{evaluation} of the phonological change depends on the perspective taken.

\newpage
Consider, for example, the so-called \textit{preference laws} by \citet{Vennemann1988}, e.g.

\begin{quote}
A syllable head is the more preferred: (a) the closer the number of speech sounds in the head is to one, (b) the greater the Consonantal Strength value of its onset, and (c) the more sharply the Consonantal
Strength drops from the onset toward the Consonantal Strength of the following syllable nucleus. \citep[13f.]{Vennemann1988}
\end{quote}

\noindent \sectref{sec4} has shown that the onset in \ili{Tungusic} was most likely a plosive that changed to a fricative and then disappeared in most languages. In other words, there was a loss of the consonantal strength of the onset, a decrease in difference between onset and nucleus, and finally a loss of the head altogether. Notably, some parts of this change must have occurred not once but several times in the different branches and subbranches of \ili{Tungusic}. From this perspective, both the phonological change and its implications were a change for the worse.\footnote{Some languages such as \ili{Manchu} dialects potentially have an initial glottal stop instead of the resonance, but this problem requires additional research.}

From a different perspective, however, the lenition of the initial consonant can also be conceptualized ``as a successive decrease and loss of muscular activity" \citep[950]{Bybee2007}, i.e. a change for the better, because the articulation requires less effort. From this perspective, the changes in the interrogative system cannot be considered a change for the worse. Depending on which of the two perspectives we prefer, the change in \ili{Tungusic} can be said to be either for the ``better'' or for the ``worse''. This example nicely illustrates that an evaluation is always based on specific purposes and perspectives. Note that this discussion only shifts the evaluation of the development in the interrogatives to another level. The evaluation would also require a cost-benefit analysis that is almost impossible to achieve. Which is more important, the potential benefit of the phonological change or the functional implications in the interrogative system?

The qualitative evaluation of a language is both a problematic and dangerous endeavor (e.g., \citealt{Lehmann2006}). In the following, this will be illustrated through a criticism of \citet[213]{Dixon2016}, who mentions “some of the features [...] which should be ideally present in every language, to ensure that it is an effective vehicle for identification, cooperation, communication, argumentation, and so on.” For example: “An ideal language will have a separate form for each of the standard interrogative words: ‘who’, ‘what’, ‘which’, ‘where’, ‘when’, ‘why’, ‘how’, ‘how much’, and ‘how many’” \citep[227]{Dixon2016}. This appears to be a derivation of the ``One-Meaning-One-Form principle'' (e.g., \citealt[34]{Miestamo2008grammatical}). However, \citegen{Dixon2016} argument is highly problematic. (1) The list of interrogatives is rather arbitrary. \citet{Dixon2016} excludes interrogative words such as `to do what' from the list because of their cross-linguistic rarity. Consequently, the other categories must have been chosen on this criterion as well. However, even \citegen[407]{Dixon2012} more extensive and otherwise very good discussion fails to give any cross-linguistic data on the frequency of these forms. The list would also require a clearly specified threshold of when any given category is included or not. (2) It is by no means clear what ``separate" forms, also called ``basic question words" \citep[46]{Hengeveld2012}, are, given that the analysis of interrogatives is often not clear-cut (\sectref{sec2}). (3) Languages that lack a ``separate" form for any of the categories mentioned above can still be an effective means of communication. For instance, there is no ``basic question word" for the locative meaning `where' in \ili{Tungusic}. Nevertheless, all \ili{Tungusic} languages have means of expressing the notion. Even if any of the categories were entirely absent from a given language, it would presumably not have been required by the speech community.

\tabref{tab:separate} lists all nine categories mentioned by \citet{Dixon2016}, illustrated with examples from the \ili{Jurchenic} language \ili{Sibe}. The \ili{Sibe} interrogative system is as incoherent as that of \ili{Evenki}. *\textit{ŋüi} is preserved as \textit{və} and *\textit{ja-} as \textit{ya} and its derivations. Similar to \ili{Manchu}, the resonance in *\textit{K\textasciitilde} was lost, which made interrogatives such as \textit{\_afš} (\ili{Manchu} \textit{\_absi}), \textit{\_ai} (\ili{Manchu} \textit{\_ai}), or \textit{\_ut} (\ili{Manchu} \textit{\_udu}) unanalyzable.

\begin{table}
\caption{Some interrogatives in Sibe \citep{Zikmundová2013} with Nanai cognates \citep{KoYurn2011}. Not all forms and variants listed.}
\label{tab:separate}
 \begin{tabular}{lll} 
  \lsptoprule
  Category & \ili{Sibe} & \ili{Nanai}\\
  \midrule
  who & və & ui\\
  which & ya & -\\
  how much & yask(ə) & -\\
  where & yet & -\\
  how & \textbf{\_}afš & \textbf{x}aosi `wither'\\
  what & \textbf{\_}ai & \textbf{x}aɪ\\
  when & \textbf{\_}aitiɴ & -\\
  why & \textbf{\_}a\textsuperscript{n} & -\\
  how many & \textbf{\_}ut & \textbf{x}ado\\
  \lspbottomrule
 \end{tabular}
\end{table}

Following \citet{Dixon2016}, the \ili{Sibe} interrogative system would most likely be considered ``ideal" because no form is synchronically analyzable (one form, one meaning). However, why should only these categories be considered and not, say, `which one'. In \ili{Sibe}, this category is expressed with the form \textit{yam(kə\textsuperscript{n})}, which can be partly analyzed as \textit{ya} + \textit{əm(kə\textsuperscript{n})} `which + one'. In fact, the form \textit{ya əmkə\textsuperscript{n}} is also attested. Thus, depending on the choice of the categories, the system can be said to be more or less ideal. Most certainly, a language-specific approach that takes into account the whole interrogative system would be more beneficial. For example, \ili{Sibe} also has additional interrogatives that should be considered (e.g., \textit{ailiaɴ} `what kind of').

The loss of the resonance in \ili{Tungusic} led to separate forms that
are synchronically unrelated. Following \citet{Dixon2016}, this should be considered a change for the better. However, as pointed out in \sectref{sec2}, \textit{analyzability} is only one of several dimensions. If the dimension of \textit{organization} is taken into account, for instance, the analyzability of certain forms could well be a desirable factor. For instance, \ili{Sibe} \textit{yet} is still partly analyzable as \textit{ya} + \textsc{dat} and corresponds to \ili{Manchu} \textit{ya-de}. Given that this form is marked for case, it is part of a paradigm, e.g. \ili{Manchu} accusative \textit{ya-be} etc. A decrease in analyzability would certainly make the paradigm less organized, irregular, and thus more complex.

Instead of an evaluation, this chapter tried to \textit{objectively} describe the complexity of the \ili{Tungusic} interrogatives. Both evaluation and complexity can be applied locally or globally and both are graded categories that can be shown on a scale. However, evaluation is necessarily relative to a certain perspective \citep{Lehmann2006}, while complexity is perhaps best described in absolute terms \citep{Miestamo2008grammatical}. \citet[10]{Bybee2015} is certainly correct in her assessment that changes as such ``are natural to language and they are neither good nor bad.” A language can only be better or worse for a specific purpose, e.g. expressibility, acquirability, processing, articulation etc. Whether the change in \ili{Tungusic} is for the better or for the worse can be answered either way, depending on the perspective taken.

\section{Conclusion}\label{sec5}

This chapter is a case study of changes in absolute local complexity that was described along seven dimensions (regularity, redundancy, analyzability, amount, organization, coherence, and delineation). It was mostly concerned with changes of an initial phoneme *\textit{K-} in \ili{Tungusic} languages spoken in Northeast Asia and its functional implications in a subsystem of these languages. Given that the phoneme had the role of a submorpheme in the interrogative system, its loss in some \ili{Tungusic} languages led to a decrease in systematicity (i.e., complexification). Based on new evidence from a language called \ili{Alchuka}, it has been shown that the initial consonant *\textit{K-} perhaps was not a fricative but an unaspirated and unvoiced velar-like plosive (e.g., [c], [k], or [q]), but its exact place and manner of articulation have yet to be identified. Several issues were left open and require additional research. Nevertheless, some problems such as the putative reflex \textit{n-} in \ili{Manchu}, an epenthetic element unrelated to *\textit{K-}, could be solved. It remains an open question what the original reason for the sound change was and whether some sort of language contact may have been involved.


\section*{Acknowledgements}

I want to thank Bill Rozycki, the editors, and two anonymous reviewers for reading an earlier version of this paper.

{\sloppy\printbibliography[heading=subbibliography,notkeyword=this]}

\end{document}
