\documentclass[output=paper,hidelinks]{langscibook}
\ChapterDOI{10.5281/zenodo.13347672}
\author{Johanna Nichols\affiliation{University of California, Berkeley;University of Helsinki\\;Higher School of Economics, Moscow}}
\title{For better and/or for worse: Complexity and person hierarchies}  
\abstract{Implementation of person hierarchies in verb inflection can produce indexation paradigms that are very complex by the usual standards for measuring complexity in paradigms.  Criteria are proposed here for measuring hierarchical patterns in their own terms and more generally for describing relational, linear, and blueprint-driven properties of grammar.}

\IfFileExists{../localcommands.tex}{
   \addbibresource{../localbibliography.bib}
%    \addbibresource{thisvolume.bib}
   \usepackage{langsci-optional}
\usepackage{langsci-gb4e}
\usepackage{langsci-lgr}

\usepackage{listings}
\lstset{basicstyle=\ttfamily,tabsize=2,breaklines=true}

%added by author
% \usepackage{tipa}
\usepackage{multirow}
\graphicspath{{figures/}}
\usepackage{langsci-branding}

   
\newcommand{\sent}{\enumsentence}
\newcommand{\sents}{\eenumsentence}
\let\citeasnoun\citet

\renewcommand{\lsCoverTitleFont}[1]{\sffamily\addfontfeatures{Scale=MatchUppercase}\fontsize{44pt}{16mm}\selectfont #1}
  
   %% hyphenation points for line breaks
%% Normally, automatic hyphenation in LaTeX is very good
%% If a word is mis-hyphenated, add it to this file
%%
%% add information to TeX file before \begin{document} with:
%% %% hyphenation points for line breaks
%% Normally, automatic hyphenation in LaTeX is very good
%% If a word is mis-hyphenated, add it to this file
%%
%% add information to TeX file before \begin{document} with:
%% %% hyphenation points for line breaks
%% Normally, automatic hyphenation in LaTeX is very good
%% If a word is mis-hyphenated, add it to this file
%%
%% add information to TeX file before \begin{document} with:
%% \include{localhyphenation}
\hyphenation{
affri-ca-te
affri-ca-tes
an-no-tated
com-ple-ments
com-po-si-tio-na-li-ty
non-com-po-si-tio-na-li-ty
Gon-zá-lez
out-side
Ri-chárd
se-man-tics
STREU-SLE
Tie-de-mann
}
\hyphenation{
affri-ca-te
affri-ca-tes
an-no-tated
com-ple-ments
com-po-si-tio-na-li-ty
non-com-po-si-tio-na-li-ty
Gon-zá-lez
out-side
Ri-chárd
se-man-tics
STREU-SLE
Tie-de-mann
}
\hyphenation{
affri-ca-te
affri-ca-tes
an-no-tated
com-ple-ments
com-po-si-tio-na-li-ty
non-com-po-si-tio-na-li-ty
Gon-zá-lez
out-side
Ri-chárd
se-man-tics
STREU-SLE
Tie-de-mann
}
   \togglepaper[9]%%chapternumber
}{}



\begin{document}
\SetupAffiliations{output in groups = true, mark style=alphabetic}
\multicolsep=.25\baselineskip
\maketitle

\section{Introduction}
Consider the twisted willow (\textit{Salix matsudana} ‘tortuosa’) or the corkscrew hazel (\textit{Corylus avellana} ‘contorta’) or similar plants.\footnote{Photographs:\\
\url{https://commons.wikimedia.org/wiki/Category:Salix_babylonica_\%27Tortuosa\%27},
\url{https://commons.wikimedia.org/wiki/Category:Corylus_avellana_\%27Contorta\%27}
}
The living organism is formed by a simple blueprint applying repeatedly to produce a one-dimensional object of the same type as a syntactic tree: an open graph consisting of a linear extension and occasional nodes defining a branching structure.  But as we perceive the plant in space it is a tree diagram gone haywire in three dimensions: unpredictably convoluted and too complex to permit description in terms of constituents and dependency structure.  Its two-dimensional projection is even worse, as it then has numerous intersecting branches.

The blueprint is the basic metaphor behind this paper.  A paradigm is the linguist's description of the outputs of a series of decisions, processing steps, or applications of rules that occur in the production and interpretation of sentences -- i.e. the output of a blueprint. To properly assess language complexity we need to measure the complexity of both the output and the blueprint, and while recent years have seen much progress in measuring the complexity of outputs, little is known of blueprints and how to measure their complexity.  My thesis here is that simple blueprints can produce highly complex outputs, polysynthetic grammars and their components being a particularly clear case.  If complexity is assessed based only on paradigms and outputs, polysynthetic structures and hierarchical patterns are parade examples of complexity and as such must surely represent worse outcomes of whatever diachronic processes produce them.  But if measured on their own terms they are models of economy, simplicity, and consistency.  This chapter is a programmatic one, intended to propose some first steps toward measuring the complexity of blueprints and blueprint-driven structures in their own terms, beginning with the inflectional marking of argument roles and participant reference.

\section{Measuring paradigm complexity as non-biuniqueness}\label{sec:nichols:2}

Transparency can be measured as the number of departures from the one-form-one-function ideal of biuniqueness \citep{Nicholsa,Nicholsb}, and the theory of canonical morphology and syntax (\citealt{Corbett2016,Corbett2015,Corbett2013,Corbett2007,Bond2019}, and many others) has identified a number of non-biunique, or non-canonical, patterns such as syncretism, allomorphy, zero morphs, paradigmatic gaps, and the like, which can be counted up to give a measure of complexity \citep{Nichols2015,Nicholsb,Audring2017}.  Importantly, the count of non-canonicalities correlates with what is known as \textit{descriptive complexity} or \textit{Kolmogorov complexity}, the amount of information required to describe a system: more information is required  to describe a non-biunique (or less biunique) system than a biunique (or more biunique) one \citep[for types of complexity see][]{Dahl2004,MiestamoEtAl2008,Sinnemaeki2011}. Examples are the \ili{Mongolian} and \ili{Russian} partial case paradigms in Tables~\ref{tab1} and~\ref{tab2}.

\begin{table}
\caption{Partial Mongolian noun case paradigms (\citealt[163]{Svantesson2003}; \citealt[297--298, 106--112, 66--68]{Janhunen2012a}; Janhunen's transcription). Case suffixes are hyphenated off, extensions underlined.\label{tab1}}

\begin{tabularx}{\textwidth}{XXl}
\lsptoprule
Case 		 & ‘book’   & ‘year’ \\
 \midrule
Nominative	 & nom 		& or \\
Genitive 	 & nom-ÿn 	& or-\uline{n}-ÿ \\
Accusative   & nom-ÿg 	& or-ÿg \\
Dative       & nom-d 	& oro-\uline{n}-d \\
 \lspbottomrule
\end{tabularx}
\end{table}
\il{Mongolian}

\begin{table}
\caption{Partial Russian noun case paradigms (forms transliterated).  Under each noun gloss is the declension class, then the gender, of the noun shown.  Case suffixes are hyphenated off, extensions underlined.\label{tab2}}
\begin{tabularx}{\textwidth}{Xllllll}
\lsptoprule
			 & ‘brother’ & ‘house’ & ‘window’ & ‘book’ & ‘net’ & ‘time’ \\
Declension   & 1		 & 1	   & 1		  & 2	   & 3	   & 3	 \\
Gender		 & Masc.     & Masc.   & Neut.	  & Fem.   & Fem.  & Neut. \\
\midrule
Nominative 	 & brat-$\varnothing$  & dom-$\varnothing$ & okn-o & knig-a & set'-$\varnothing$ & vremja-$\varnothing$ \\
Genitive 	 & brat-a & dom-a & okn-a & knig-i & set-i & vrem-\uline{en}-i \\
Accusative 	 & brat-a & dom-$\varnothing$ & okn-o & knig-u & set'-$\varnothing$ & vremja-$\varnothing$ \\
Dative		 & brat-u & dom-u & okn-u & knig-e & set-i & vrem-\uline{en}-i \\
 \lspbottomrule
\end{tabularx}
\end{table}
\il{Russian}
 
In the \ili{Mongolian} paradigms there is one instance of allomorphy (the genitive endings, which are phonologically predictable) and an extension -\textit{n}- in two of the cases.  Describing the endings requires annotating nouns of the non-default type as requiring the extension, and stating the basis for the case allomorphy (referring to the morphophonology section of the grammar).  In the \ili{Russian} paradigms there are three patterns of syncretism:  genitive-accusative (in ‘brother’), nom\-i\-na\-tive-ac\-cusative (‘house’, ‘window’, ‘net’, ‘time’), and genitive-dative (‘net’, ‘time’).  For every case there is allomorphy based on declension class, gender, and animacy: two different genitive allomorphs and three for the other cases.  There is an extension -\textit{en}- in ‘time’.  Describing this requires annotating each noun for gender, declension class, animacy (grammatical animacy is largely but not entirely predictable from real-world animacy), and whether or not it takes an extension.  Describing the \ili{Russian} system takes more information than describing the \ili{Mongolian} system (contrast the lines of print describing the two in this paragraph) and requires displaying more paradigms. 
 
For both languages there are additional alternations that are phonologically predictable:  for \ili{Mongolian} these are mostly alternations in the affixal vowels, predictable from the noun stem structure \citep[106--108]{Janhunen2012a}; for \ili{Russian} they mostly involve the consequences of stress shifts and consonant palatalization.  These phonological and morphophonological alternations are not considered in this paper, which deals only with morphological complexity. 
 
Similar comparisons can be made in verb conjugation. For example, both \ili{Russian} and \ili{Tatar} (\ili{Turkic}) have verbal indexation of the S/A, but with considerable differences in complexity.  \ili{Russian} has coexponence of person-number with TAM; \ili{Tatar} does not.  More precisely, \ili{Tatar}, like most \ili{Turkic} languages, has two series of person-number markers, one originally verbal and one originally nominal but extended to verb conjugation as former participles have been reanalyzed as tense forms.  \ili{Russian} too has allomorphy of its person-number marking, which define two conjugation classes; in addition, it has a category discrepancy in indexation: person-number in the nonpast tenses vs. gender/number in the past tense.  \ili{Tatar} has a single verb stem class; \ili{Russian}, depending on how they are counted, has at least five (defined by conjugation suffixes, a.k.a. thematic suffixes, such as -\textit{i}-, -\textit{ej/e}-, -\textit{aj/a}, -\textit{i/a}-, -$\varnothing$, to which the TAM-person-number suffixes are attached).  Thus, in terms of affix conjugation classes, stem conjugation classes, and conjugation categories, \ili{Russian} is considerably more complex.   
 
These are all departures from biuniqueness, involving one-many and many-one and many-many relations of form to function.  There are also differences in sheer numbers of categories; notably, \ili{Russian} has only three tenses while \ili{Turkic} languages generally have more; \ili{Turkic} languages generally have evidentiality oppositions while \ili{Russian} does not; \ili{Russian} has an aspect opposition which \ili{Turkic} languages lack.  But these involve the number of elements in subsystems, i.e. inventory or taxonomic or enumerative complexity, which is not at issue here.

Canonicality theory has covered paradigms, especially inflectional paradigms, extensively and this enables us to make decisions like those above on what is and is not canonical in the case paradigms of \ili{Russian} and \ili{Mongolian} or the person-number paradigms of \ili{Russian} and \ili{Tatar}.  What has not had due attention is the extent of non-canonicality in subject-object indexation patterns like those of \ili{Yimas}, shown in \tabref{tab3}.

\begin{table}
\caption{Subject and object prefixes in Yimas (Lower Sepik-Ramu, New Guinea) \citep[200 ff.]{Foley1991}\label{tab3}}
\begin{tabularx}{\textwidth}{>{\scshape}XXXXl}
\lsptoprule
 & A & O & S & Alignment \\
 \midrule
1du & ngkra- & ngkra- & kapa- & A=O; S  \\
1pl & kay- & kra- & ipa- & 3-way  \\
1sg & ka- & nga- & ama- & 3-way \\
 \midrule
2du & ngkran- & ngkul- & kapwa- & 3-way \\
2pl & nan- & kul- & ipwa- & 3-way \\
2sg & n- & nan- & ma- & 3-way \\
 \midrule
3sg & n- & na- & na- & Ergative \\
3pl & mpu- & pu- & pu- & Ergative \\
3du & mpi- & impa- & impa- & Ergative \\
 \lspbottomrule
\end{tabularx}
\end{table}
\il{Yimas}\il{Lower Sepik-Ramu}

There is an alignment discrepancy, namely the presence of three different alignment patterns:  an anomalous one for 1du; three-way for the other first and second person forms; and ergative for third person.  These could also be defined as syncretism patterns:  A and O in 1\textsc{du}, S and O in the third person, none in the others.  There are two additional syncretisms:  2\textsc{pl} A and 2\textsc{sg} O \textit{nan}-; 2\textsc{sg} and 3\textsc{sg} A \textit{n}-.  There are also discrepancies in the linear order of the A and O prefixes:  the higher-ranked prefix in a hierarchy of 1 > 2 > 3 is adjacent to the verb, shown in (\ref{bel1}).\largerpage[-1]

\ea \label{bel1}\langinfo{Yimas}{}{\citealt[205]{Foley1991}}\\
    \begin{multicols}{2}\raggedcolumns
	\begin{xlist} 
	\ex 
		\gll pu- nga- tay\\ 
		3\textsc{pl}.A- \textsc{1sg}.O- see\\
		\glt ‘They saw me.’ 
	\columnbreak\ex 
		\gll pu- ka- tay\\
		3\textsc{pl}.O- 1.\textsc{sg}.A- see\\
		\glt ‘I saw them.’ 
	\end{xlist}
	\end{multicols}
 \z
In addition, there is a suppletive portmanteau prefix for 1>2\textsc{sg}, shown in (\ref{bel2}).

\ea \label{bel2} \langinfo{Yimas}{}{\citealt[207]{Foley1991}}\\
	\gll kampan- tay\\
	1.A>2\textsc{sg}.O- see\\
	\glt ‘I saw you.’ 
\z
Several of these patterns are non-canonical: the alignment discrepancy (or, alternatively, the syncretisms), the additional syncretisms, the discrepant ordering, and the portmanteau morpheme.

\ili{Jingulu} (\ili{Mirndi}, northern Australia; \citealt{Pensalfini2003}) has three different types of subject and object indexation: an explicit sequence of overt A and O markers, available for any two non-coreferential arguments except that 1\textsc{sg}>2 is impossible; monomorphemic fused forms for 3 > 1, 3> 2, 2> 1, 1 <> 2; and inverse \textit{ni} plus A marker.  Non-canonicalities are the existence of the three different types (which amounts to two-way or three-way allomorphy), the portmanteau morphemes, and the inverse marking, which can be described as creating allomorphy of object markers (since the A marker plus the inverse affix marks the O).  The inverse marking and the portmanteau forms implement person hierarchies, disallowing sequences of lower-ranked A and higher-ranked O.\largerpage[-1]

(\ref{bel3}) shows direct and inverse indexation combined with case on an argument and role-marking agreement affixes in \ili{Southern Tiwa}. 

\ea \label{bel3} \langinfo{Southern Tiwa}{Kiowa-Tanoan, southwestern U.S.}{\citet[180]{Zuniga2006} after \citet[219--220]{Klaiman1991}}\emph{. Inverse marker bold.}
	\begin{xlist}
		\ex \gll \ili{Seuan-ide} ti-mų-ban.\\
		man-\textsc{sg} \textsc{1sg}.IIA-see-\textsc{past}\\
		\glt ‘I saw the man.’ (1\textsc{sg}>3\textsc{sg}) 
		\ex \gll \ili{Seuan-ide}-ba  	 te-mų-\textbf{che}-ban.\\
		man-\textsc{sg-obl} 1\textsc{sg}.I-see-\textsc{\textbf{inverse}-past}\\
		\glt ‘The man saw me.’ (3\textsc{sg}>1\textsc{sg}) 		
	\end{xlist}
 \z
\il{Southern Tiwa}\il{Kiowa-Tanoan}

\noindent (3b), in which a lower-ranked subject acts on a higher-ranked object, has an inverse suffix, which, however, is redundant as the A in (b) is case-marked and the 1\textsc{sg} affixes \textit{ti} vs. \textit{te} mark A vs. O functions.  The inverse contributes to inventory complexity but also to non-canonicality, in that the choice of which argument to index varies not according to syntactic role but to person ranking.

Hierarchical indexation in which two arguments compete for a single slot often entails non-canonicality.  \tabref{tab4} gives a paradigm and (\ref{bel4}) gives some examples from a \ili{Laz} variety.  The paradigm is very economical, using minimal overt marking to convey almost entirely unambiguous meaning.  However, it has a number of non-canonical patterns.  These include zeroes in the prefix paradigm for second person S/A and third person S/A and O; a position discrepancy, as third person A is marked suffixally while other persons are prefixal; and discrepancies as to which person prefixes overtly distinguish the argument roles (first person does; second and third do not, but in different ways:  second person has only overt O, third person only A).  There is either syncretism or a category discrepancy in the S/A suffix column in the plural:  3\textsc{pl} is unambiguously S/A person indexation, while the -\textit{t} in 1--2\textsc{pl} is arguably either an S/A person index (thus, syncretic for person) or a plural marker (then there is a category discrepancy in this slot).  All of these depart from the biuniqueness ideal.

\begin{table}
\caption{Subject and object indexation in Arhavi Laz (Kartvelian, Turkey; \citealt[283]{Lacroix2009}).  \dots = verb root + thematic suffix.\label{tab4}}
\begin{tabularx}{.6\textwidth}{>{\scshape}Xllll}
\lsptoprule
    & S/A-  & O- & \dots    & -S/A \\
\midrule
1sg & b-    & m- &          &  \\
2sg &       & g- &          &  \\
3sg &       &    &          & -s/n/u \\
\midrule
1pl & b-    & m- &          & -t \\
2pl &       & g- &          & -t \\
3pl &       &    &          & -an/nan/es/n \\
\lspbottomrule
\end{tabularx}
\end{table}
\il{Arhavi Laz}\il{Kartvelian}

\ea \label{bel4}\langinfo{Arhavi Laz}{}{\citealt{Lacroix2009}}\\
\begin{multicols}{2}
\begin{tabbing}
m-dzir-on-\emph{an\hspace{2\tabcolsep}}\=‘you\textsubscript{sg} see me’\kill
\emph{b-dzir-om}    \> ‘I see him’\\
\emph{m-dzir-om-s}  \> ‘he sees me’\\
\emph{m-dzir-om}    \> ‘you\textsubscript{sg} see me’\\
\emph{g-dzir-om}    \> ‘I see you’\\
\emph{dzir-om}      \> ‘you\textsubscript{sg} see him’\\
\emph{dzir-om-s}    \> ‘he sees him’\\
\emph{b-dzir-om-t}  \> ‘we see him’\\
\emph{m-dzir-on-an} \> ‘he sees us’\\
\emph{dzir-om-t}    \> ‘you\textsubscript{pl} see him’\\
\emph{dzir-om-an}   \> ‘they see him’ \\
\end{tabbing}
\end{multicols}
 \z
\il{Arhavi Laz}

Another kind of departure from biuniqueness is illustrated by \ili{West Caucasian} argument indexation.  These languages have complex polysynthetic verbs, for which the main inflectional elements can be reduced to the family-wide structure shown in (\ref{bel5}); there are additional slots in each language, including slots between the argument slots. Tables \ref{tab5} and \ref{tab6} (page \pageref{tab5}) give argument indexation paradigms for two languages.

\ea \label{bel5}Shared \ili{West Caucasian} verb template.  \{G+\} is an open slot containing G (a.k.a. R, the more goal-like or recipient-like object of a ditransitive verb) and other G-like indexes (beneficiary, causee, applicative object, others), in the form of both bare person-number markers and incorporated PP's.\medskip\\ 
	S/O \quad  \{G+\} \quad  A \quad  Neg \quad Caus \quad ROOT \quad Aktionsart \quad \{TAM Neg\}
\z

\begin{table}[p]
\caption{Adyghe (West Caucasian: Circassian) argument indexation \citep{Arkadev2009}.  OP = object of postposition (the whole PP is incorporated to index the argument).\label{tab5}}
\begin{tabularx}{\textwidth}{>{\scshape}XXXXl}
\lsptoprule
 & S/O & OP & G & A \\
 \midrule
1sg & sə- & s- & se- & s- \\
2sg & wə- & p- & we- & p- \\
3sg & $\varnothing$/me- & $\varnothing$ & je- & jë- \\
\midrule
1pl & tə- & t- & te- & t- \\
2pl & ŝʷə- & ŝʷ- & ŝʷe- & ŝʷ- \\
3pl & $\varnothing$/me- & (j)a- & (j)a- & (j)a- \\
\lspbottomrule
\end{tabularx}
\end{table}
\il{Adyghe}\il{West Caucasian}\il{Circassian}

\begin{table}[p]
\caption{Abkhaz (West Caucasian) argument indexation \citep{Chirikba2003}.  \textsc{nh} = nonhuman.  \textsc{m}, \textsc{f} = natural gender of referent (there is no noun gender).  All consonant-final forms have an epenthetic schwa in certain phonological contexts. Schwa = /ə/, except /a/ after pharyngeal.\label{tab6}}
\begin{tabularx}{\textwidth}{>{\scshape}X>{\scshape}XXXXl}
\lsptoprule
 &  		& S/O & OP & G & A \\
 \midrule
1sg  &  	& s- & s-  & s- & s-/~z- \\
2sg	 & m	& w- & w- & w- & w- \\
	 & f 	& b- & b- & b- & b- \\
	 & nh	& w- & w- & w- & w- \\
3sg	 & m 	& d- & j- & j- & j- \\
	 & f 	& d- & l- & l- & l- \\
	 & nh 	& j- & a- & a- & (n)a- \\
1pl	 &		& h\textsuperscript{ʕ}-	& h\textsuperscript{ʕ}- & h\textsuperscript{ʕ} & h\textsuperscript{ʕ}-/~aa-  \\
2pl	 &		& šʷ-	& šʷ- & šʷ- & šʷ-/~žʷ- \\
3pl	 &		& j- & r-/~d- 	  & r-/~d-	 & r-/~d- \\
\lspbottomrule
\end{tabularx}
\end{table}
\il{Abkhaz}\il{West Caucasian}\il{Circassian}

The markers themselves, then, are nearly identical; only their occurrence in different slots distinguishes the argument roles.  Is such a system more or less canonical, or more or less complex, than one with different marker forms for different roles?  The next sections propose an approach to answering these questions and more generally to assessing the complexity of hierarchical and inverse patterns.

\section{The relational axis}

\citet{Saussure1916} famously distinguished the paradigmatic and syntagmatic axes of language.  The paradigmatic axis comprises sets of equivalent elements that are in complementary distribution or in opposition (as the members of a case paradigm are).  It is this axis that canonical typology and much of morphological theory more generally have focused on in recent years \citep[e.g.][]{Bond2019}.  The syntagmatic axis is simultaneously present elements, those that cooccur in a single phrase or clause in speech.  The gist of that distinction is still essential in linguistics, but the understanding of syntax available to Saussure was so incomparable to what we now know that it is probably best to leave the term \textit{syntagmatic} in Saussure's sense and devise new terminology.  I tentatively use the term \textit{relational} in this new sense.  The relational axis of language includes the relation of head to dependent, word to its inflectional index, and index to independent token (e.g. pronoun or noun argument).  

For a sentence to function in communication, its relational structure must be recovered.  The next sections offer proposals on what is optimal for recovery of argument relations in the two salient arenas of inflectional morphology, verbs and nominals.  For verbs, what needs to be recovered includes (a) relationality, i.e. the fact of a relation or dependency; (b) the type of relation (primarily argument roles: A, S, O, etc.); (c) properties of the argument (person, number, gender, case, etc.); (d) referentiality.  For nouns they include (a) the fact of relationality in the abstract; (b) properties of the adnominal possessor (person, number, etc.); (c) referentiality.


\section{Measuring complexity on the relational axis}

Let us assume that the biuniqueness criterion of one form, one function also applies to the relational axis of grammar.  Then indexation of argument properties such as person and gender together with marking of relationality amounts to coexponence, or cumulative exponence:  one form has the two functions of indexing categories and marking relationality.  This also pertains to marking of the type of relation (clause syntatic roles such as A and O, NP roles such as possessor) together with the abstract fact of relationality.  From this perspective, the canonical ideal is whatever is closest to pure relationality, since that lacks the indexation categories that create the non-canonicality.  By that criterion, canonical behavior is exhibited by the \ili{Jingulu} or \ili{Yimas} opaque 1 <> 2 portmanteau (which marks relationality but does not indicate which argument has which role) or \ili{Kabardian} or \ili{Abkhaz} argument indexation (where the forms mark person but not role). 

Further support for the optimality of minimal coexponence is Siewierska's accessibility marking scale (\citeyear[176]{Siewierska2004}), in which accessibility is greatest at the left end, where overt independent marking is least:     zero < reflexives < person affixes < person clitics and unstressed pronouns < stressed pronouns.  	

However, in any morphosyntactic approach the processor has to be able to recover who does what to whom or what; otherwise no message is communicated, and without a message we do not have language.\footnote{Note that this point is expressed in terms of processing, i.e. information, and not learning, learnability, etc.  The complexity or non-complexity of the job of the speaker, hearer, or language learner is a different matter from the complexity of the language system itself, an important one but not addressed here.}   One way to do this is to make the argument structure and type of argument relation recoverable from the verbal lexeme itself, as when middle, causative, factitive, applicative, etc. morphology narrows down the valence and argument structure.  Suppletive valence pairs such as \textit{fear} and \textit{scare} or \textit{see} and \textit{show} also make the argument structure clear.  In the \ili{Yimas} and \ili{West Caucasian} examples (\ref{bel1}), (\ref{bel2}), and (\ref{bel5}) above, the position of the indexation marker provides most of the needed information about its structure.

Another strategy is to make the argument markers distinguishable from other morphemes.  They can occupy salient positions such as word edges, the sole prefixal slot, the first pretonic slot, etc.  Or they can have an easy-to-distinguish canon form such as western American first person \textit{n} and second person \textit{m} in pronouns and indexes \citep{Nichols1996,Nichols2013}.  There can be strategic neutralization in other parts of the verb template, so that argument markers preserve full formal distinctiveness and are more informative.

Most important, which argument is which can follow from person and/or role hierarchies.  In opaque 1 <> 2 morphemes (which are cross-linguistically common: see \citealt{Heath1991,Heath1998} for surveys and analysis) both the roles and the referents follow from the speech act itself and have no additional information involved in their identity; none is needed as both speaker and hearer know who is who.  Much the same holds for egophoric (conjunct/disjunct) person marking, where the identities of the participants follow from the type of speech act.

Using this reasoning as springboard, \tabref{tab7} presents some thought experiments aimed at identifying the ideals or extremes on the relational axis.

\begin{table}
\caption{Some parameters and ideals of relationality.  < means ‘less ideal than’; boldface marks the ideal.\label{tab7}}
\begin{tabularx}{\textwidth}{Q}
\lsptoprule
Paradigm-specific < whole-language < \textbf{universal}\footnote{Here and below I use \textit{universal} loosely to include defaults and biases applying to more than the individual language, e.g. family biases, areal biases, cross-linguistically favored defaults, and universals if they exist.} \\ \tablevspace
Overt marking of roles on argument indexes < \textbf{hierarchical}  \\ \tablevspace
In templatic strings:    All morphemes/syllables/phonemes equally salient and essential  < a few \textbf{landmarks} or key items to anchor the string and guide processing  \\ \tablevspace
No neutralization < \textbf{neutralization} of non-landmark positions \\ \tablevspace
Individuals < \textbf{sets}  (where sets are based on some principle) \\ \tablevspace
Undemarcated < \textbf{demarcated} sets  (demarcated e.g. by rhyme, alliteration, or other shared phonological shape) \\
\lspbottomrule
\end{tabularx}
\end{table}

The first point is probably the most important, reflecting the fact that universals do not need to be spelled out in full detail for every language or every paradigm but are stated once and for all outside of the specific grammar or part of grammar.  Hierarchical patterns (including inverse marking) illustrate this principle: specific overt marking is not necessary where identities of arguments can be recovered from general principles.  The general principles include the very common person ranking of 1, 2 > 3, the common 1 > 2 > 3, and the somewhat less common 2 > 1 > 3; for many examples of these and others see \citet{Zuniga2006}.  No single pattern dominates exclusively, but nonetheless a choice of one of them requires less information on the particular paradigm than a full explicit specification would.  

Internally structured elements require processing, i.e. information, so less internal structure is relationally more canonical.   Landmark positions or landmark items defocus other kinds of information to allow processing to target the relational ones, and likewise for neutralization; both reduce the information involved in processing relationality.  The higher canonicality of sets compared to individuals involves the same principle as less internal structure: a set can be described with less information than the individual pieces one by one.  Importantly, the set needs to be based on some principle; there is little economy to be gained by generalizing (or attempting to generalize) over a random collection rather than over a group sharing some common basis.  Sets that are demarcated somehow, for instance by shared phonological properties, can more easily be targeted for relational processing.  In general there is an inverse correlation between frequency of elements and their distinctiveness \citep[100--129]{Meylan2018}, so that a shared phonotactic shape for (e.g.) person indexes can involve minimally distinctive segments (such as \textit{n} and \textit{m}, mentioned below) or identical elements such as rhyme or alliteration, reducing descriptive information without undermining the message.

All of these rankings boil down to two general principles: (1) reduce informational complexity by drawing on general or universal patterns; (2) increase the efficiency and economy of processing by making the relational markers more salient or identifiable.

The examples in \sectref{sec:nichols:2} above, illustrating the paradigmatic complexity and non-canonicality of inflection based on person and role hierarchies, are all quite ca-nonical in relational terms.  The person agreement of \ili{Yimas} (\ref{bel1}, \ref{bel2} and \tabref{tab3}) implements hierarchies in at least three ways:  the linear order of A and O prefixes uses a person hierarchy of 1 > 2 > 3; alignment of prefixes uses 1, 2 > 3; and linear order uses role (A > O or O > A depending on person).  Opaque 1 <> 2 markers implement a person hierarchy, and for \ili{Arhavi} \ili{Laz} access to the sole prefix slot follows the hierarchies 1, 2 > 3 and A > O.  All draw on language-wide or universal hierarchies instead of specifying information fully for each paradigm.

Minimization of internal structure within sets, and use of landmark positions or items to identify relational elements, are illustrated by \ili{West Caucasian} indexation (Tables \ref{tab5} and \ref{tab6}), which marks person but not role for each argument series, distinguishing role by relative position, and placing argument markers in the salient positions of word edge (initial) and pre-root or pre-stem.  In addition, sandhi effects such as manner-of-articulation assimilation and schwa-zero alternations respond to a following morpheme and help narrow down the position slot occupied by the argument index, but as they are automatic phonological alternations they need to be specified only once for the language, in the morphophonology component, and not for each slot or filler.

Opaque markers, as noted, are common for 1 <> 2 person combinations where the identities of the participants are clear, but they occur with other combinations as well. \ili{Kiowa} has a daunting set of 58 more or less portmanteau argument indexes for the combinations of three persons, three numbers, and two roles \citep[109--137]{Watkins1984}.  Though amenable to some comparative and internal reconstruction, they are synchronically best treated as unanalyzable, i.e. as a list for lookup by a processor (and memorized by speakers).\footnote{To the extent that they are analyzable into A and O morphemes, they utilize the following hierarchies for ordering: O > A; animate > inanimate; 1\textsc{sg} > other; nonsingular > singular \citep{Watkins1984}.}   The existence of such systems suggests strongly that it is more efficient to scan or consult or memorize a closed set of high-frequency, high-saliency short items than to separately monitor A and O morphemes.

A possibly relevant example is consonant gradation in \ili{Finnish} vs. its close sisters, the \ili{Saami} languages.  Consonant gradation in \ili{Finnish} is largely phonologically predictable: a consonant is in weak grade if it precedes a closed syllable, in strong grade before an open syllable.  In \ili{Saami}, especially the eastern varieties \ili{Skolt} and \ili{Kildin}, similar alternations were once phonologically conditioned as in \ili{Finnic}, but now, due to vowel loss and syllabic restructuring, they form a set of opaque morphomic patterns.  Thus, while \ili{Finnish} has a single declension class with phonologically predictable alternations of stem consonants, eastern \ili{Saami} has several declension classes based on morphomic alternations.  The \ili{Finnish} system requires attention to the internal structure of words; the \ili{Saami} one does not but requires memorization.  In the terms laid out here, the \ili{Finnish} system is paradigmatically canonical in having a single underlying form per stem and relationally canonical in drawing on the grammar-wide principle of consonant gradation; while the \ili{Saami} one is relationally canonical in requiring no monitoring of the internal structure of paradigm elements, but non-canonical in utilizing different morphomic patterns rather than drawing on a single principle.\footnote{For the \ili{Saami} languages mentioned see \citet{Sammallahti1998}, \citet{Nickel2011}, \citet{Feist2015}, \citet{Kert1971}, and chapters in \citet{Bakroinpress}. }  

Canonicality of sets and especially demarcated sets is illustrated by the numerous systems of pronouns and/or inflectional person markers structured by rhyme, alliteration and/or a common tone canon, including the widespread Eurasian\il{Eurasian languages} person system with first person \textit{m} and second person \textit{T} (\textit{T} = anterior obstruent, typically \textit{t}, \textit{č}, or \textit{s}) or the western American one with \textit{n} and \textit{m} (\citealt{Nichols1996}, \citealt{NicholsPeterson2013}, \citealt{NicholsPeterson2013a}; other examples are in \citealt{Nichols2001,Nichols2012,Nichols2013}).  Where these phonological onomatopoetic principles apply the members of the set are not maximally distinct; but as high-frequency items they do not have to be \citep[100--129]{Meylan2018}.

A last relevant example may be the distinction of indexation vs. registration (as the terms \textit{index} and \textit{register} are defined in \citealt[48--49]{Nichols1992}). In indexation, markers indicate relationality and relation types as well as copying categories of triggers; in registration, no categories are copied.  Probably the best-known example of registration is \ili{Semitic} construct state, where an affix or alternation on a noun indicates that it has a dependent (typically a possessor), i.e. it registers the possessor but does not agree with it in any category.  Contrast possessive marking of nouns in \ili{Turkic} and \ili{Uralic} languages, where the possessive suffix on a head noun agrees in person and number with the possessor, i.e. indexes it.  

\ili{Uralic} languages variously illustrate indexation, registration, and no marking of objects on verbs.  In all branches verbs index person and number of subjects.  In the eastern branches, in what is usually called the object conjugation verbs index the number but not the person of objects; in \ili{Hungarian} the object number indexation is lost and the object conjugation registers an object but does not index it; in the \ili{Mordvin} branch verbs index both person and number of objects; in the other western branches there is no object marking on verbs.  The ancestral state is the eastern one, with partial indexation, and it may be of evolutionary interest that it has remained stable in four branches (Samoyed, Ugric, Permic, \ili{Mari}) with one innovative change of indexation to registration and one change of partial to full indexation, as well as one or two losses of indexation (in \ili{Finnic} and \ili{Saami}, or once in \ili{Proto-Finno-Saami}). 

\section{Discussion and conclusions}

If ideal types, complexity, and what is canonical differ between the paradigmatic and relational axes of grammar, it is worth asking why only person, and not gender, seems to enter into hierarchical patterns.  Given that the evolution of a hierarchical pattern can replace paradigmatic complexity with relative relational non-complexity, one would expect to see any and all indexation categories develop hierarchical patterns like those discussed here for person.  The answer seems to be that person hierarchies are potential emergent patterns everywhere and readily grammaticalized and exploited where this is useful, while gender has no corresponding potential emergent pattern.  This seems to be because only person is referential (in the sense of referential defined by \citealt{Kibrik2011}:  a person index refers, while a gender index can agree or copy features but cannot refer (for the full argument on this point see \citealt[\S5]{Nicholsa}). 

Furthermore, referentiality varies with person categories.  Third person items (nouns, pronouns, and arguably also indexing morphemes on verbs) carry referential indices and make reference to time-stable entities, and their reference is established by the speaker.  First and second persons (SAP's), in contrast, are shifters; they have no fixed reference and no referential index (or at least not a fixed one).  They could be described as occupying not verb slots but speech-act slots; their reference is given or inherent in the speech act. \citet{Heath1991,Heath1998} notes that opaque 1<> 2 morphemes are opaque for good reason:  social-pragmatic considerations such as taboo, avoidance, politeness, and the awkwardness of asserting things about one's interlocutor to that interlocutor, make it expedient to suppress overt reference to an individual in these person combinations.  I have noted in addition that the identity of the first and second persons is automatic and inherent in speech-act context and does not require specification; it suffices to have a morpheme indicating that the two interlocutors are involved.  For these reasons, emergent hierarchical patterns can easily form at points where SAP's are involved.  The same considerations explain why the cross-linguistically favored cutoffs in hierarchies occur between SAP's and third persons (i.e. 1, 2 > 3). 

It is also worth considering whether the paradigmatic or the relational axis is the favored target of elaboration, simplification, etc. in language evolution.  There are a number of examples of large, spreading inter-ethnic languages with complex verb morphology and person-rich indexation, many of them with hierarchical patterns and all with relational simplicity and paradigmatic complexity: several \ili{Mayan} languages; \ili{Nahuatl} (\ili{Uto-Aztecan}, Aztec branch; contrast the more paradigmatically oriented \ili{Numic} branch and \ili{Hopi}), \ili{Ojibwe} and \ili{Cree} (\ili{Algonquian}), \ili{Navajo} and \ili{Hupa} (\ili{Athabaskan}), \ili{Ainu} (isolate),\footnote{Ainu, a moribund remnant of the pre-Japonic linguistic population of Japan, is not usually thought of as a large spreading inter-ethnic language; but see \cite{Janhunen2002}.} \ili{Circassian} and \ili{Abkhaz} (\ili{West Caucasian}). An emergent example comes from the Amdo sprachbund (eastern Tibet), where in an intensive contact situation direct person marking is lost and replaced by egophoric marking under \ili{Bodic} influence; the reverse does not occur \citep{Janhunen2012b}.  Since spreads usually involve absorption of adult L2 speakers, these languages can be regarded as having undergone decomplexification \citep{Trudgill2011}.  Such systems are also quite stable in families.  The opposite picture is also true, however, especially in northern Eurasia, where we find large spreading inter-ethnic languages with extensive but regular paradigmatic structure, case-rich inflection, massively recursive morphology, and corresponding configurational morphosyntax: this is the type of the \ili{Turkic}, \ili{Mongolic}, \ili{Tungusic}, and \ili{Uralic} families.  These systems are also very stable.  It must be that the two axes are equally favored, and spread and contact situations select for consistency rather than for a particular axis.

This paper has identified and illustrated some principles of optimal relationality, but has not attempted to cover their interaction with each other or with other parts of grammar.  The full picture is of course much more complex. Even within just the realm of hierarchical rankings, in \ili{Algonquian} languages one finds all three of 1 > 2, 2 > 1, and 1 = 2, usually more than one within a single language \citep{Macaulay2009}, and \ili{Yimas}, as noted above, uses both A > O and O > A, depending on person, to determine prefix ordering.  How such rankings work together is a question left open here (Macaulay makes a start for \ili{Algonquian}).

The canonical complexity measures presented in \sectref{sec:nichols:2} are the analog to describing the branches of a corkscrew tree or its two-dimensional projection.  Such structures are real and are products of selection in the biological and linguistic worlds, but they are not the whole story.  This paper has attempted some first steps toward describing linguistic relational phenomena in their own terms and discerning the outlines of the blueprint that produces them.

{\sloppy\printbibliography[heading=subbibliography,notkeyword=this]}

\end{document}
